%\iffalse
%<+paper>\def\filename{paper}
%<+journal>\def\filename{journal}
%\fi
\def\fileversion{1.0l}
\def\filedate{2008/05/30}
\def\docdate{1996/11/02}
% \CheckSum{2732}
%
% \iffalse    This is a META-COMMENT
%
% Copyright (C) 1992, 1998 by wenzel_matiaske@notes.uni-paderborn.de
%
% This file is to be used with the LaTeX2e system.
% ------------------------------------------------
%
% This macro is free software; you can redistribute it and/or modify it
% under the terms of the GNU General Public License as published by the
% Free Software Foundation; either version 1, or (at your option) any
% later version.
%
% The macros and the documentation are distributed in the hope that they
% will be useful, but WITHOUT ANY WARRANTY; without even the implied
% warranty of MERCHANTABILITY or FITNESS FOR A PARTICULAR PURPOSE.  See
% the GNU General Public License for more details.
%
% You should have received a copy of the GNU General Public License
% along with this program; if not, write to the Free Software
% Foundation, Inc., 675 Mass Ave, Cambridge, MA 02139, USA.
%
% Updates are available via ctan.org. This package lives at:
% http://tug.ctan.org/tex-archive/macros/latex/contrib/paper/
%
% \fi
% 
% \MakeShortVerb{\|}
%
%    \ifsolodoc
%      \title{Die \LaTeX-Stile \texttt{paper} und \texttt{journal}}
%      \author{Wenzel Matiaske}
%      \date{\docdate}
%      \maketitle
%      \selectlanguage{\english}
%      \def\localin{\par}
%      \begin{small}
%      \begin{center}\small\textbf{Abstract}\end{center}
%      \MakePercentComment % \iffalse
% local.tex -- updated for LaTeX2e 16 May 1996
%              first released 1 Sept 1993
%
% Copyright (C) 1993, 1998 by wenzel_matiaske@notes.uni-paderborn.de
%
% As input to the local LaTeX-guide "`local.tex"'.
%
% For distribution of this document see the copyright notice in the
% original sources mentioned below.
% \fi

\makeatletter
\newif\iflocalin
\newif\ifappendixin
\@ifundefined{localin}{\localintrue}{\localinfalse}
\@ifundefined{appendixin}{\appendixinfalse}{\appendixintrue}
\@ifundefined{docdir}{\def\docdir{\dots /emtex/doc/}}{}
\makeatother

\iflocalin
   \subsubsection{The style files \texttt{paper} and \texttt{journal}}
\fi
\ifappendixin
  \subsection{The style files \texttt{paper} and \texttt{journal}}
\fi

The style files \verb|paper| and \verb|journal| are derived from the
standard class \verb|article|. In difference to the standard document
class the layout can be changed via layout options (\verb|slanted|,
\verb|bold|, \verb|sfbold|) and font commands 
(\verb|\partfont{|{\em font\/}\verb|}|, 
\verb|\sectionfont{|{\em font\/}\verb|}| etc.).

The class \texttt{paper} defines a new environment called
\texttt{keywords} and the commands
\verb|\subtitle{|\emph{text}\verb|}| and
\verb|\institution{|\emph{text}\verb|}| for the title section. Three
commands allow a small table of contents
(\verb|\smalltableofcontents|), a small lists of figures
(\verb|\smalllistoffigures|) or a small lists of tables
(\verb|\smalllistoftables|). These commands are obsolete when using the
\texttt{journal} style file.

The format \texttt{journal} typically uses a master file which \verb|\include|
the articles. The command \verb|\journalofcontents| produces a table of
articles, revisions and parts of a journal. The new commands
\verb|\shortauthor{|\emph{text}\verb|}| and
\verb|\shorttitle{|\emph{text}\verb|}| are defined for head titles
containing authors and titles. Head titles for the whole journal may be
produced with the commands
\verb|\oddrunhead{|\emph{text}\verb|}| and
\verb|\evenrunhead{|\emph{text}\verb|}|.

If you want to declare parts between the papers, you may use the command
\verb|\journalpart[|\emph{option}\verb|]{|\emph{text}\verb|}| or
\verb|\journalpart*{|\emph{text}\verb|}|.

Two new commands are especially designed for revisions. The command
\verb|\revison[|\emph{option}\verb|]{|\emph{author}\verb|}{|\emph{title}\verb|}|
takes the author and the title of the revisited book.  It produces a
subsection like headline and an entry for the table of contents. The
optional argument is used to put also the author of the revision into
the table of contents. This command is also defined in the form
\verb|\revision*|. The command \verb|\revauthor{|\emph{text}\verb|}|
may be useful to sign a revision. It allows the commands \verb|\and|
and \verb|\thanks|.


\iflocalin
For more details see the German documentation
\file{\docdir thesis} and \file{\docdir paper}.
\fi

 \MakePercentIgnore
%      \end{small}
%      \newpage
%      \section{Einleitung}
%      \selectlanguage{\german}
%    \else 
%      \section{Artikel und Zeitschriften}
%    \fi
% 
%
%
%    Die \LaTeX-Classes\footnote{Version \fileversion{} vom
%    \filedate. Dokumentation vom \docdate.} \texttt{paper} und
%    \texttt{journal} basieren auf der Standard Class \texttt{article}
%    und eignen sich zur Erstellung von Artikeln sowie einfacher
%    Zeitschriften. Die wesentlichste Ver\"anderung gegen\"uber dem
%    Grundstil ist die M\"oglichkeit, das Layout mittels verschiedener
%    Optionen variieren zu k\"onnen. Diese Optionen entsprechen denen
%    der Stilarten \texttt{thesis} bzw. \texttt{thema}
%    \ifsolodoc\else\ (Vgl. Tab. \ref{theoptions})\fi. Im Unterschied
%    zu diesen Formaten sind hier die Optionen \texttt{sfbold},
%    \texttt{noupper} und \texttt{nocenter} voreingestellt. Die
%    Stilarten \texttt{paper} und \texttt{journal} weisen auch
%    identische Funktionen bez\"uglich der Einstellungen von
%    Schriftarten auf\ifsolodoc\else\ (Vgl. Tab. \ref{schriftwahl})\fi.
%
%
%    \ifsolodoc 
%       \section{Zus\"atzliche Befehle der Stilart \texttt{paper}} 
%    \else
%       \subsection{Zus\"atzliche Befehle der Stilart \texttt{paper}} 
%     \fi
%
%    Das Format \texttt{paper} stellt neben den oben erw\"ahnten
%    Stiloptionen einige weitere Befehle zur Verf\"ugung. Sie dienen der
%    Gestaltung des Titels, der Verzeichnisse und der Angabe von
%    Deskriptoren. 
%
% \DescribeMacro{\subtitle}
% \DescribeMacro{\institution}
%    Die Kommandos
%    |\subtitle{|\emph{text}|}|
%    und 
%    |\institution{|\emph{text}|}| 
%    erweitern die Titelei. Der Untertitel wird unterhalb der
%    eigentlichen Titels in kleineren Typen gesetzt. Die Institution
%    wird ggf. als letzte Zeile der Titelei im Anschlu\ss{} an den Autor
%    ausgegeben. Mehrere Institutionen werden durch \verb+\and+ getrennt.
%    In diesem Zusammenhang ist daraufhinzuweisen, da\ss{} die
%    Ausgabe des Datums entf\"allt. 
%
% \DescribeMacro{\smalltableofcontents}
% \DescribeMacro{\smalllistoftables}
% \DescribeMacro{\smalllistoffigures}
%   Die Stilart \texttt{paper} beinhaltet drei zus\"atzliche Kommandos,
%   die der Erzeugung von Verzeichnissen dienen. Der Befehl
%   |\smalltableofcontents| entspricht dem \"ublichen
%   |\tableofcontents| mit der Ausnahme, da\ss{} das
%   Inhaltsverzeichnis in kleineren Typen gesetzt wird. Entsprechende
%   Befehle sind f\"ur das Tabellen- (|\smalltableofcontents|) und
%   das Abbildungsverzeichnis (|\smalllistoffigures|)
%   definiert. Es ist darauf hinzuweisen, da\ss{} diese Befehle ebenso wie
%   die regul\"aren Verzeichnisbefehle in der Stilart \texttt{journal}
%   nicht verf\"ugbar sind.
%
% \DescribeEnv{keywords}
%    Die neue Umgebung \texttt{keywords} dient der Aufnahme von
%    Deskriptoren. Die Schlagworte sind in die Struktur
%    |\begin{keywords}| \dots |\end{keywords}|
%    einzuschlie\ss{}en.  
%
%    \ifsolodoc 
%       \section{Die Stilart \texttt{journal}}   
%    \else  
%       \subsection{Die Stilart \texttt{journal}}  
%    \fi
%    
%
%    Die Stilart \texttt{journal} stellt dar\"uber hinaus einige Befehle
%    zur Verf\"ugung, die bei der Erstellung von Zeitschriften
%    zweckdienlich sind. Die grundlegende Idee ist, mehrere Artikel zu
%    einem Journal zu b\"undeln. Die Artikel werden in einer
%    Formatierdatei mittels des Befehls
%    |\include{|\emph{datei}|}| eingelesen, wie das
%    Anwendungsbeispiel in Abbildung \ref{jourtest} zeigt. Es ist
%    m\"oglich, mittels \BibTeX{} f\"ur  jeden Artikel ein gesondertes
%    Literaturverzeichnis zu erstellen. Dies entspricht der Option
%    \texttt{cbib} der Stilart  \texttt{thesis}\ifsolodoc\else\
%    (Vgl. \ref{cbib})\fi. 
%
%    Dieses Format stellt dar\"uberhinaus einige Kommandos zu Verf\"ugung, 
%    welche die Gestaltung bestimmter Teile einer Zeitschrift, des
%    Inhaltsverzeichnisses, der Kopfzeilen und spezieller Beitr\"age wie
%    Rezensionen erleichtert. 
%
%\begin{figure}[ht]
%\begin{verbatim}
%\documentclass[12pt]{journal}
%\usepackage{jourbib}
%
%\begin{document}
%\oddrunhead{G-Animal's Journal, Vol. 15}
%
%\begin{titlepage} \begin{center}
% {\LARGE\sf G-Animal's Journal} \vfill
% {\Large 1983, Vol. 15, No. 4}  \vfill
% {\large\bf Fanstord University}
%\end{center} \end{titlepage}
%
%\journalcontents
%
%\include{article1}
%\include{article2}
%
%\newpage\journalpart{Revisions}
%\review[Masterly]{Larry Manmaker}{The Definitive Computer Manual}
%Golden edition, 1993, Chips-R-Us: Silicon Valley. 
%
%\bigskip
%The golden oldies \dots
%
%\revauthor{\'{E}douard Masterly\\Stanford University}
%
%\end{document}
%\end{verbatim}
%\caption{\label{jourtest}Anwendungsbeispiel der Stilart \texttt{journal}}
%\end{figure}
%
% \DescribeMacro{\journalcontents}
%    Zur Erstellung eines Inhaltsverzeichnisses ist ein neuer Befehl
%    implementiert. Das Kommando |\journalcontents| produziert
%    das Inhaltsverzeichnis einer Zeitschrift, welches die Autoren und
%    Titel der Beitr\"age und ggf. bestimmte Teil\"uberschriften sowie
%    Rezensionen enth\"alt. 
% 
% \DescribeMacro{\journalpart}
%    Das Kommando
%    |\journalpart[|\emph{option}|]{|\emph{text}|}|
%    dient dem Zweck, bestimmte Teile einer Zeitschrift voneinander
%    abzugrenzen. Im Unterschied zum \"ublichen Befehl |\part| wird
%    diese Teil\"uberschrift in das Inhaltsverzeichnis der Zeitschrift
%    \"ubernommen und initialisiert die Kopfzeilenmarkierung neu. Ist
%    die Aufnahme in das Inhaltsverzeichnis unerw\"unscht, so kann der
%    Eintrag durch Spezifizierung der Sternform
%    |\journalpart*{|\emph{text}|}| unterdr\"uckt werden.
%
% \DescribeMacro{\shorttitle}
% \DescribeMacro{\shortauthor}
% \DescribeMacro{\oddrunhead}
% \DescribeMacro{\evenrunhead}
%    In der Kopfzeile von Zeitschriften werden h\"aufig die Autoren und
%    der Titel des Beitrages aufgef\"uhrt. Dies erm\"oglichen die Befehle
%    |\shorttitle{|\emph{text}|}| und  
%    |\shortauthor{|\emph{text}|}|. Bei einseitigem Druck
%    werden beide Angaben in die Kopfzeile der Seite \"ubernommen, bei 
%    zweiseitigem Druck werden die Autoren auf den geraden, die Titel
%    auf den ungeraden Seiten mitgef\"uhrt. Die Kommandos sind f\"ur jeden
%    Beitrag erneut zu spezifizieren. 
%
%    Soll in der Kopfzeile durchg\"angig ein anderer Text erscheinen, 
%    beispielsweise der Name der Zeitschrift, wird dies durch die 
%    Kommandos |\oddrunhead{|\emph{text}|}| und 
%    |\evenrunhead{|\emph{text}|}| erm\"oglicht, wie
%    das Anwendungsbeispiel \ref{jourtest} zeigt. Die Kommandos
%    \"uberschreiben  
%    alle anderweitig eingestellten Markierungen. Wird lediglich 
%    eines der Kommandos benutzt, erscheinen im Zusammenhang
%    mit den Befehlen 
%    |\shorttitle{|\emph{text}|}| und
%    |\shortauthor{|\emph{text}|}| sowie zweiseitigem 
%    Druck Kurztitel und Autoren jeweils in der Kopfzeile der
%    gegen\"uberliegenden Seite.
%
% \DescribeMacro{\review}
% \DescribeMacro{\revauthor}
%    Der neue Befehl 
%    |\review[|\emph{optional}|]{|\emph{autor}|}{|\emph{titel}|}|  
%   ist zur Abfassung von Rezensionen konzipiert. Das Kommando
%   \"ubernimmt den Autor und den Titel der zu besprechenden
%   Arbeit. Diese werden wie ein |\subsubsection| gesetzt und in
%   das Inhaltsverzeichnis aufgenommen. Optional kann ferner der Autor
%   der Rezension spezifiziert werden, der dann ebenfalls im
%   Inhaltsverzeichnis erscheint. Die Sternform |\review*|
%   unterdr\"uckt den Verzeichniseintrag vollst\"andig. Das Kommando
%   |\revauthor| ist n\"utzlich, um eine Rezension zu zeichnen. Es
%   erm\"oglicht wie der \"ubliche Befehl |\author| mehrere,  durch
%   |\and| getrennte Autoren und das Kommando |\thanks|. 
%
% \StopEventually{}
%
%   \ifsolodoc 
%      \section{Implementation} 
%   \else
%       \subsection{Implementation} 
%   \fi
%
% \changes{theart~0.9a}{1992/09/01}{
%          Dokumentation in \texttt{doc} Format}
% \changes{theart~0.9b}{1993/07/01}{
%          Draft Headings, Schriftwahl als Optionen}
% \changes{journal~0.9c}{1993/09/15}{
%          \texttt{journal.sty} integriert}
% \changes{journal~1.0a}{1994/12/11}{
%          \texttt{journal.cls} Update auf \LaTeX3e}
% \changes{paper~1.0a}{1994/12/13}{
%          \texttt{paper.cls} Neue Befehle zur Schriftwahl}
% \changes{paper~1.0b}{1994/12/16}{
%         Optionen \texttt{fleqno} und \texttt{leqno}lesen 
%         externe Stildateien}
% \changes{paper~1.0c}{1994/12/30}{
%         Geringf\"ugige Verbesserungen, Dokumentation \"uberarbeitet}
% \changes{paper~1.0d}{1995/01/09}{
%         \"Uberarbeitung einiger Umgebungen}
% \changes{paper~1.0e}{1995/02/08}{
%         Doppelte Parameter entfernt. Einige Tippfehler
%         beseitigt. Bug in \texttt{journal} (\"Ubergabe der Optionen
%         an \texttt{paper}) beseitigt.}
% \changes{paper~1.0f}{1995/10/25}{
%         Vorschub in caption}
% \changes{paper~1.0g}{1996/05/16}{%
%          Abstand in Kopfzeilen korrigiert}
% \changes{paper~1.0h}{1996/06/10}{%
%          \texttt{titlepage}-Umgebung korrigiert, \texttt{abstract}-
%          und \texttt{keywords}-Umgebungen funktionieren nun.}
% \changes{paper~1.0i}{1996/07/25}{%
%          \texttt{shortitle} wird in das Inhaltsverzeichnis
%          aufgenommen (Michael P Urban).}
% \changes{paper~1.0j}{1996/06/08}{%
%          Kopfzeile bei doppelseitigem \texttt{journal} gekl\"art
%          (Luiz Henrique de Figueiredo).}
% \changes{paper~1.0k}{1998/10/30}{%
%          \texttt{thanks} verbessert
%          (Benjamin Bayart), \texttt{institution} erweitert,
%          \texttt{equatation}-Z\"uhler zur\"uckgesetzt (Daniel H. Rossi).}
% \changes{paper~1.0l}{2008/05/30}{%
%          Fix bug in abstract definition. (Richard G Heck).}
%
%    Die Implementation enth\"alt den Code f\"ur die Klassen \texttt{paper}
%    und \texttt{journal} und die kompatibles Style-Files.
%
%    \begin{macrocode}
%<*paper|paper.sty|journal|journal.sty>
\NeedsTeXFormat{LaTeX2e}
%</paper|paper.sty|journal|journal.sty>
%    \end{macrocode}
%
%    Die kompatiblen Styles laden die zugeh\"orige Klasse.
%
%    \begin{macrocode}
%<*paper.sty>
\@obsoletefile{paper.cls}{paper.sty}
\LoadClass{paper}
%</paper.sty>
%<*journal.sty>
\@obsoletefile{journal.cls}{journal.sty}
\LoadClass{journal}
%</journal.sty>
%
%    Im Fall der Klasse \texttt{journal} wird eine
%    Startmeldung ausgegeben, die Optionen werden der zugeh\"origen 
%    Klasse |paper| \"ubergeben und diese wird eingelesen.
%
%   \begin{macrocode}
%<*journal>
\ProvidesClass{journal}[\filedate\space\fileversion\space%
       LaTeX document class (wm).]
\DeclareOption*{\PassOptionsToClass{\CurrentOption}{paper}}
\ProcessOptions
\LoadClass[journal]{paper}
%</journal>
%    \end{macrocode}
%
%    Die Implementation des Haupstils \texttt{paper.cls} beginnt 
%    mit der Startmeldung f\"ur die Klasse \texttt{paper.cls}.
%
%    \begin{macrocode}
%<*paper>
\ProvidesClass{paper}[\filedate\space\fileversion\space%
       LaTeX document class (wm).]
%    \end{macrocode}
%
%    Es folgen weitere Definitionen und Initialisierungen, die den 
%    \LaTeXe{} Standard Klassen entnommen sind.
%
%    Kontrolliert die Schriftgr\"o\ss{}e.
%    \begin{macrocode}
\newcommand\@ptsize{}
%    \end{macrocode}
%
%    Schalter um zwischen zwei- und einspaltigen Satz zu wechseln.
%    \begin{macrocode}
\newif\if@restonecol
%    \end{macrocode}
%
%    Schalter, um die Erzeugung einer Titelseite anzuzeigen.
%    \begin{macrocode}
\newif\if@titlepage 
\@titlepagefalse
%    \end{macrocode}
%
%    Schalter f\"ur "`offenes"' oder "`geschlossenes"' Format der Bibliographie.
%    \begin{macrocode}
\newif\if@openbib 
\@openbibfalse
%    \end{macrocode}
%
%    Definitionen der Papierformate
%    \begin{macrocode}
\if@compatibility\else
\DeclareOption{a4paper}
   {\setlength\paperheight {297mm}%
    \setlength\paperwidth  {210mm}}
\DeclareOption{a5paper}
   {\setlength\paperheight {210mm}%
    \setlength\paperwidth  {148mm}}
\DeclareOption{b5paper}
   {\setlength\paperheight {250mm}%
    \setlength\paperwidth  {176mm}}
\DeclareOption{letterpaper}
   {\setlength\paperheight {11in}%
    \setlength\paperwidth  {8.5in}}
\DeclareOption{legalpaper}
   {\setlength\paperheight {14in}%
    \setlength\paperwidth  {8.5in}}
\DeclareOption{executivepaper}
   {\setlength\paperheight {10.5in}%
    \setlength\paperwidth  {7.25in}}
%    \end{macrocode}
%
%    Die Option \texttt{landscape} tauscht die Werte f\"ur Seitenh\"ohe 
%    und Seitenbreite.
%    \begin{macrocode}
\DeclareOption{landscape}
   {\setlength\@tempdima   {\paperheight}%
    \setlength\paperheight {\paperwidth}%
    \setlength\paperwidth  {\@tempdima}}
\fi
%    \end{macrocode}
%
%    Optionen f\"ur Schriftgr\"o\ss{}en.
%
%    \begin{macrocode}
\if@compatibility
  \renewcommand\@ptsize{0}
\else
   \DeclareOption{10pt}{\renewcommand\@ptsize{0}}
   \DeclareOption{11pt}{\renewcommand\@ptsize{1}}
   \DeclareOption{12pt}{\renewcommand\@ptsize{2}}
\fi
%    \end{macrocode}
%
%    Zwei oder einseitiger Druck.
%    \begin{macrocode}
\if@compatibility\else
\DeclareOption{oneside}{\@twosidefalse \@mparswitchfalse}
\fi
\DeclareOption{twoside}{\@twosidetrue  \@mparswitchtrue}
%    \end{macrocode}
%
%    Definitionen f\"ur die \texttt{draft} Option. Die Makros 
%    \verb+\SetTime+ und \verb+\now+ sind aus \texttt{tugboat.com}
%    \"ubernommen. Diese werden in der \texttt{draft} Option zur 
%    Gestaltung der Kopfzeile benutzt. Ferner wird ein unmaskierter 
%    Schalter \verb+\iffinal+ definiert, der standardm\"a\ss{}ig wahr, im 
%    Fall der Option \texttt{draft} dagegen falsch ist. 
%
%    \begin{macrocode}
\newcount\hours \newcount\minutes
\def\SetTime{\hours=\time
        \global\divide\hours by 60
        \minutes=\hours
        \multiply\minutes by 60
        \advance\minutes by-\time
        \global\multiply\minutes by-1 }
\def\now{\number\hours:\ifnum\minutes<10 0\fi\number\minutes}
\newif\iffinal \finaltrue
\DeclareOption{draft}{\setlength\overfullrule{5pt}\finalfalse \SetTime}
\if@compatibility\else
\DeclareOption{final}{\setlength\overfullrule{0pt}\finaltrue}
\fi
%    \end{macrocode}
%
%    Definition des Schalters |\if@journal|. Der Zustand dieses
%    Schalter ist von der Option der |journal| 
%    abh\"angig und steuert im folgenden die Auswahl der Makros.
%
%    \begin{macrocode}
\newif\if@journal       
\@journalfalse
\DeclareOption{journal}{\@journaltrue}
%    \end{macrocode}
%
%    Die Optionen initialisiert die Variable |option@crosshair|
%    die abgefragt wird, um auf leeren Seiten ein Markierung des 
%    Seitenkopfs einzuf\"ugen.
%
%    \begin{macrocode}
\newif\if@crosshair \@crosshairfalse
\DeclareOption{crosshair}{\@crosshairtrue}
%    \end{macrocode}
%
%    Die Optionen setzten den Schalter \verb+\@itemization+ der
%    abgefragt wird, um die Staffelung der \texttt{itemize}
%    Umgebung zu kontrollieren. In \texttt{paper.cls} standardm\"a\ss{}ig 
%    true.
%
%    \begin{macrocode}
\newif\if@itemization \@itemizationtrue
\DeclareOption{itemize}{\@itemationtrue}
\DeclareOption{noitemize}{\@itemizationfalse}
%    \end{macrocode}
%
%    Die Optionen setzten den Schalter \verb+\@enumeration+ der
%    abgefragt wird, um die Numerierung der \texttt{enumerate} Umgebung
%    Umgebung zu kontrollieren. In \texttt{paper.cls} standardm\"a\ss{}ig 
%    alphanumerische Z\"ahlung.
%
%    \begin{macrocode}
\newif\if@enumeration \@enumerationtrue           
\DeclareOption{enumerate}{\@enumerationtrue}
\DeclareOption{noenumerate}{\@enumerationfalse}
%    \end{macrocode}
%
%    Die Optionen setzten den Schalter \verb+\@noind+ der im folgenden
%    abgefragt wird, um Absatzabst\"ande und Fu\ss{}notenstil zu modifizieren.
%    Voreingestellt ist der Satz von Abschnitten und Fu\ss{}noten bei denen
%    die erste Zeile einger\"uckt gesetzt wird.
%
%    \begin{macrocode}
\newif\if@noind \@noindfalse           
\DeclareOption{indent}{\@noindfalse}
\DeclareOption{noindent}{\@noindtrue}
%    \end{macrocode}
%
%    Die Optionen setzen den Schalter \verb+\@center+ der im folgenden
%    abgefragt wird, um \"Uberschriften, Kopfzeilen und bestimmte Eintr\"age 
%    ins Inhaltsverzeichnis zu zentrieren. Standardm\"a\ss{}ig
%    werden diese in \texttt{paper.cls} rechtsb\"undig gesetzt.
%
%    \begin{macrocode}
\newif\if@center \@centerfalse
\DeclareOption{center}{\@centertrue}
\DeclareOption{nocenter}{\@centerfalse}
%    \end{macrocode}
%
%    Die Optionen setzen den Schalter \verb+\@upper+ der im folgenden
%    abgefragt wird, um Teil\"uberschriften in Gro\ss{}buchstaben setzen.
%    Standardm\"a\ss{}ig benutzt \texttt{paper.cls} keine Gro\ss{}buchstaben.
%
%    \begin{macrocode}
\newif\if@upper      \@upperfalse
\DeclareOption{upper}{\@uppertrue}
\DeclareOption{noupper}{\@upperfalse}
%    \end{macrocode}
%
%    Die Optionen setzen den Schalter \verb+\@headline+ der 
%    abgefragt wird, um Kopfzeilen zu unterstreichen. Vereinstellung 
%    in \texttt{paper.cls} sind unterstrichene Kopfzeilen.
%
%    \begin{macrocode}
\newif\if@headline  \@headlinetrue
\DeclareOption{headline}{\@headlinetrue}
\DeclareOption{noheadline}{\@headlinefalse}
%    \end{macrocode}
%
%    Die Optionen setzen den Schalter \verb+\@headcount+ der 
%    abgefragt wird, um ggf. Abschnittsz\"ahler in Kopfzeilen
%    auszugeben. Standardm\"a\ss{}ig wird in \texttt{paper.cls} der
%    Z\"ahler in der Kopfzeile ausgegeben.
%
%    \begin{macrocode}
\newif\if@headcount \@headcounttrue
\DeclareOption{headcount}{\@headcounttrue}
\DeclareOption{noheadcount}{\@headcountfalse}
%    \end{macrocode}
%
%    Die Benutzerschnittstellen zur Definition der Schriftarten in 
%    \"Uberschriften, Titelei, Kopfzeilen, Abbildungen etc. Die 
%    Voreinstellungen erfolgen in Optionen.
%
%    \begin{macrocode}
\def\partfont#1{\def\p@font{#1}}             \def\p@font{}
\def\sectionfont#1{\def\s@font{#1}}          \def\s@font{}
\def\subsectionfont#1{\def\ss@font{#1}}      \def\ss@font{}
\def\subsubsectionfont#1{\def\sss@font{#1}}  \def\sss@font{}
\def\paragraphfont#1{\def\pg@font{#1}}       \def\pg@font{}
\def\subparagraphfont#1{\def\spg@font{#1}}   \def\spg@font{}
\def\titlefont#1{\def\t@font{#1}}            \def\t@font{}
\def\subtitlefont#1{\def\st@font{#1}}        \def\st@font{}
\def\authorfont#1{\def\a@font{#1}}           \def\a@font{}
\def\institutionfont#1{\def\in@font{#1}}     \def\in@font{}
\def\theoremheaderfont#1{\def\thh@font{#1}}  \def\thh@font{}
\def\theorembodyfont#1{\def\thb@font{#1}}    \def\thb@font{}
\def\itemfont#1{\def\item@font{#1}}          \def\item@font{}
\def\examplefont#1{\def\ex@font{#1}}         \def\ex@font{}
\def\headingstextfont#1{\def\h@font{#1}}     \def\h@font{}
\def\pagenumberfont#1{\def\pn@font{#1}}      \def\pn@font{}
\def\captionheaderfont#1{\def\cph@font{#1}}  \def\cph@font{}
\def\captionbodyfont#1{\def\cpb@font{#1}}    \def\cpb@font{}
\def\figurefont#1{\def\fig@font{#1}}         \def\fig@font{}
\def\tablefont#1{\def\tab@font{#1}}          \def\tab@font{}
%    \end{macrocode}
%
%    Die Option definiert die Schriften der \texttt{slanted} Variante.
%    Gleichzeitig werden Gro\ss{}buchstaben im Titel voreingestellt.
%
%    \begin{macrocode}
\DeclareOption{slanted}{
    \partfont{\Large}
    \sectionfont{\large\slshape}
    \subsectionfont{\slshape}
    \subsubsectionfont{\slshape}
    \paragraphfont{\slshape}
    \subparagraphfont{\slshape}
    \titlefont{\Large}
    \subtitlefont{\large}
    \authorfont{\large}
    \institutionfont{\slshape}
    \theoremheaderfont{\upshape}
    \theorembodyfont{}
    \itemfont{\slshape}
    \examplefont{}
    \headingstextfont{\small\slshape}
    \pagenumberfont{\small}
    \captionheaderfont{\slshape\small}
    \captionbodyfont{\small}
    \figurefont{}
    \tablefont{}
\@uppertrue}
%    \end{macrocode}
% 
%    Die Option f\"ur die Schriften der \texttt{bold} Variante.
%    Diese ist im Fall des \texttt{paper.cls} voreingestellt.
%
%    \begin{macrocode}
\DeclareOption{bold}{%
    \partfont{\LARGE\bfseries}
    \sectionfont{\Large\bfseries}
    \subsectionfont{\large\bfseries}
    \subsubsectionfont{\bfseries}
    \paragraphfont{\bfseries}
    \subparagraphfont{\bfseries}
    \titlefont{\LARGE\bfseries}
    \subtitlefont{\large}
    \authorfont{}
    \institutionfont{\slshape}
    \theoremheaderfont{\bfseries}
    \theorembodyfont{\itshape}
    \itemfont{\bfseries}
    \examplefont{}
    \headingstextfont{\small\bfseries}
    \pagenumberfont{\small}
    \captionheaderfont{\bfseries}
    \captionbodyfont{}
    \figurefont{}
    \tablefont{}
}
%    \end{macrocode}
%
% Die Option f\"ur die Schriften der \texttt{sfbold} Variante.
%
%    \begin{macrocode}
\DeclareOption{sfbold}{%
    \partfont{\LARGE\sffamily\bfseries}
    \sectionfont{\large\sffamily\bfseries}
    \subsectionfont{\large\sffamily\bfseries}
    \subsubsectionfont{\sffamily\bfseries}
    \paragraphfont{\sffamily\bfseries}
    \subparagraphfont{\sffamily}
    \titlefont{\LARGE\sffamily\bfseries}
    \subtitlefont{\large\sffamily}
    \authorfont{\Large\sffamily\slshape}
    \institutionfont{\sffamily}
    \theoremheaderfont{\sffamily}
    \theorembodyfont{}
    \itemfont{\sffamily}
    \examplefont{}
    \headingstextfont{\small\sffamily}
    \pagenumberfont{\small\rmfamily}
    \captionheaderfont{\sffamily}
    \captionbodyfont{}
    \figurefont{}
    \tablefont{}
}
%    \end{macrocode}
%
%   Option zur Erzeugung einer Titelseite.
%   
%    \begin{macrocode}
\DeclareOption{titlepage}{\@titlepagetrue}
\if@compatibility\else
\DeclareOption{notitlepage}{\@titlepagefalse}
\fi
%    \end{macrocode}
%
%  
%
%     Option f\"ur zweispaltigen Satz.
%    \begin{macrocode}
\if@compatibility\else
\DeclareOption{onecolumn}{\@twocolumnfalse}
\fi
\DeclareOption{twocolumn}{\@twocolumntrue}
%    \end{macrocode}
%
%    Numerierung der Formeln auf der linken Seite.
%    \begin{macrocode}
\DeclareOption{leqno}{%%
%% This is file `leqno.sty',
%% generated with the docstrip utility.
%%
%% The original source files were:
%%
%% latex209.dtx  (with options: `leqno')
%% 
%% This is a generated file.
%% 
%% Copyright 1993-2014
%% The LaTeX3 Project and any individual authors listed elsewhere
%% in this file.
%% 
%% This file was generated from file(s) of the LaTeX base system.
%% --------------------------------------------------------------
%% 
%% It may be distributed and/or modified under the
%% conditions of the LaTeX Project Public License, either version 1.3c
%% of this license or (at your option) any later version.
%% The latest version of this license is in
%%    http://www.latex-project.org/lppl.txt
%% and version 1.3c or later is part of all distributions of LaTeX
%% version 2005/12/01 or later.
%% 
%% This file has the LPPL maintenance status "maintained".
%% 
%% This file may only be distributed together with a copy of the LaTeX
%% base system. You may however distribute the LaTeX base system without
%% such generated files.
%% 
%% The list of all files belonging to the LaTeX base distribution is
%% given in the file `manifest.txt'. See also `legal.txt' for additional
%% information.
%% 
%% The list of derived (unpacked) files belonging to the distribution
%% and covered by LPPL is defined by the unpacking scripts (with
%% extension .ins) which are part of the distribution.
%% \CharacterTable
%%  {Upper-case    \A\B\C\D\E\F\G\H\I\J\K\L\M\N\O\P\Q\R\S\T\U\V\W\X\Y\Z
%%   Lower-case    \a\b\c\d\e\f\g\h\i\j\k\l\m\n\o\p\q\r\s\t\u\v\w\x\y\z
%%   Digits        \0\1\2\3\4\5\6\7\8\9
%%   Exclamation   \!     Double quote  \"     Hash (number) \#
%%   Dollar        \$     Percent       \%     Ampersand     \&
%%   Acute accent  \'     Left paren    \(     Right paren   \)
%%   Asterisk      \*     Plus          \+     Comma         \,
%%   Minus         \-     Point         \.     Solidus       \/
%%   Colon         \:     Semicolon     \;     Less than     \<
%%   Equals        \=     Greater than  \>     Question mark \?
%%   Commercial at \@     Left bracket  \[     Backslash     \\
%%   Right bracket \]     Circumflex    \^     Underscore    \_
%%   Grave accent  \`     Left brace    \{     Vertical bar  \|
%%   Right brace   \}     Tilde         \~}
\@obsoletefile{leqno.clo}{leqno.sty}
%%
%% This is file `leqno.sty',
%% generated with the docstrip utility.
%%
%% The original source files were:
%%
%% latex209.dtx  (with options: `leqno')
%% 
%% This is a generated file.
%% 
%% Copyright 1993-2014
%% The LaTeX3 Project and any individual authors listed elsewhere
%% in this file.
%% 
%% This file was generated from file(s) of the LaTeX base system.
%% --------------------------------------------------------------
%% 
%% It may be distributed and/or modified under the
%% conditions of the LaTeX Project Public License, either version 1.3c
%% of this license or (at your option) any later version.
%% The latest version of this license is in
%%    http://www.latex-project.org/lppl.txt
%% and version 1.3c or later is part of all distributions of LaTeX
%% version 2005/12/01 or later.
%% 
%% This file has the LPPL maintenance status "maintained".
%% 
%% This file may only be distributed together with a copy of the LaTeX
%% base system. You may however distribute the LaTeX base system without
%% such generated files.
%% 
%% The list of all files belonging to the LaTeX base distribution is
%% given in the file `manifest.txt'. See also `legal.txt' for additional
%% information.
%% 
%% The list of derived (unpacked) files belonging to the distribution
%% and covered by LPPL is defined by the unpacking scripts (with
%% extension .ins) which are part of the distribution.
%% \CharacterTable
%%  {Upper-case    \A\B\C\D\E\F\G\H\I\J\K\L\M\N\O\P\Q\R\S\T\U\V\W\X\Y\Z
%%   Lower-case    \a\b\c\d\e\f\g\h\i\j\k\l\m\n\o\p\q\r\s\t\u\v\w\x\y\z
%%   Digits        \0\1\2\3\4\5\6\7\8\9
%%   Exclamation   \!     Double quote  \"     Hash (number) \#
%%   Dollar        \$     Percent       \%     Ampersand     \&
%%   Acute accent  \'     Left paren    \(     Right paren   \)
%%   Asterisk      \*     Plus          \+     Comma         \,
%%   Minus         \-     Point         \.     Solidus       \/
%%   Colon         \:     Semicolon     \;     Less than     \<
%%   Equals        \=     Greater than  \>     Question mark \?
%%   Commercial at \@     Left bracket  \[     Backslash     \\
%%   Right bracket \]     Circumflex    \^     Underscore    \_
%%   Grave accent  \`     Left brace    \{     Vertical bar  \|
%%   Right brace   \}     Tilde         \~}
\@obsoletefile{leqno.clo}{leqno.sty}
%%
%% This is file `leqno.sty',
%% generated with the docstrip utility.
%%
%% The original source files were:
%%
%% latex209.dtx  (with options: `leqno')
%% 
%% This is a generated file.
%% 
%% Copyright 1993-2014
%% The LaTeX3 Project and any individual authors listed elsewhere
%% in this file.
%% 
%% This file was generated from file(s) of the LaTeX base system.
%% --------------------------------------------------------------
%% 
%% It may be distributed and/or modified under the
%% conditions of the LaTeX Project Public License, either version 1.3c
%% of this license or (at your option) any later version.
%% The latest version of this license is in
%%    http://www.latex-project.org/lppl.txt
%% and version 1.3c or later is part of all distributions of LaTeX
%% version 2005/12/01 or later.
%% 
%% This file has the LPPL maintenance status "maintained".
%% 
%% This file may only be distributed together with a copy of the LaTeX
%% base system. You may however distribute the LaTeX base system without
%% such generated files.
%% 
%% The list of all files belonging to the LaTeX base distribution is
%% given in the file `manifest.txt'. See also `legal.txt' for additional
%% information.
%% 
%% The list of derived (unpacked) files belonging to the distribution
%% and covered by LPPL is defined by the unpacking scripts (with
%% extension .ins) which are part of the distribution.
%% \CharacterTable
%%  {Upper-case    \A\B\C\D\E\F\G\H\I\J\K\L\M\N\O\P\Q\R\S\T\U\V\W\X\Y\Z
%%   Lower-case    \a\b\c\d\e\f\g\h\i\j\k\l\m\n\o\p\q\r\s\t\u\v\w\x\y\z
%%   Digits        \0\1\2\3\4\5\6\7\8\9
%%   Exclamation   \!     Double quote  \"     Hash (number) \#
%%   Dollar        \$     Percent       \%     Ampersand     \&
%%   Acute accent  \'     Left paren    \(     Right paren   \)
%%   Asterisk      \*     Plus          \+     Comma         \,
%%   Minus         \-     Point         \.     Solidus       \/
%%   Colon         \:     Semicolon     \;     Less than     \<
%%   Equals        \=     Greater than  \>     Question mark \?
%%   Commercial at \@     Left bracket  \[     Backslash     \\
%%   Right bracket \]     Circumflex    \^     Underscore    \_
%%   Grave accent  \`     Left brace    \{     Vertical bar  \|
%%   Right brace   \}     Tilde         \~}
\@obsoletefile{leqno.clo}{leqno.sty}
\input{leqno.clo}
\endinput
%%
%% End of file `leqno.sty'.

\endinput
%%
%% End of file `leqno.sty'.

\endinput
%%
%% End of file `leqno.sty'.
}
%    \end{macrocode}
%
%    Links ausgerichtete Mathematische Umgebungen.
%    \begin{macrocode}
\DeclareOption{fleqn}{%%
%% This is file `fleqn.sty',
%% generated with the docstrip utility.
%%
%% The original source files were:
%%
%% latex209.dtx  (with options: `fleqn')
%% 
%% This is a generated file.
%% 
%% Copyright 1993-2014
%% The LaTeX3 Project and any individual authors listed elsewhere
%% in this file.
%% 
%% This file was generated from file(s) of the LaTeX base system.
%% --------------------------------------------------------------
%% 
%% It may be distributed and/or modified under the
%% conditions of the LaTeX Project Public License, either version 1.3c
%% of this license or (at your option) any later version.
%% The latest version of this license is in
%%    http://www.latex-project.org/lppl.txt
%% and version 1.3c or later is part of all distributions of LaTeX
%% version 2005/12/01 or later.
%% 
%% This file has the LPPL maintenance status "maintained".
%% 
%% This file may only be distributed together with a copy of the LaTeX
%% base system. You may however distribute the LaTeX base system without
%% such generated files.
%% 
%% The list of all files belonging to the LaTeX base distribution is
%% given in the file `manifest.txt'. See also `legal.txt' for additional
%% information.
%% 
%% The list of derived (unpacked) files belonging to the distribution
%% and covered by LPPL is defined by the unpacking scripts (with
%% extension .ins) which are part of the distribution.
%% \CharacterTable
%%  {Upper-case    \A\B\C\D\E\F\G\H\I\J\K\L\M\N\O\P\Q\R\S\T\U\V\W\X\Y\Z
%%   Lower-case    \a\b\c\d\e\f\g\h\i\j\k\l\m\n\o\p\q\r\s\t\u\v\w\x\y\z
%%   Digits        \0\1\2\3\4\5\6\7\8\9
%%   Exclamation   \!     Double quote  \"     Hash (number) \#
%%   Dollar        \$     Percent       \%     Ampersand     \&
%%   Acute accent  \'     Left paren    \(     Right paren   \)
%%   Asterisk      \*     Plus          \+     Comma         \,
%%   Minus         \-     Point         \.     Solidus       \/
%%   Colon         \:     Semicolon     \;     Less than     \<
%%   Equals        \=     Greater than  \>     Question mark \?
%%   Commercial at \@     Left bracket  \[     Backslash     \\
%%   Right bracket \]     Circumflex    \^     Underscore    \_
%%   Grave accent  \`     Left brace    \{     Vertical bar  \|
%%   Right brace   \}     Tilde         \~}
\@obsoletefile{fleqn.clo}{fleqn.sty}
%%
%% This is file `fleqn.sty',
%% generated with the docstrip utility.
%%
%% The original source files were:
%%
%% latex209.dtx  (with options: `fleqn')
%% 
%% This is a generated file.
%% 
%% Copyright 1993-2014
%% The LaTeX3 Project and any individual authors listed elsewhere
%% in this file.
%% 
%% This file was generated from file(s) of the LaTeX base system.
%% --------------------------------------------------------------
%% 
%% It may be distributed and/or modified under the
%% conditions of the LaTeX Project Public License, either version 1.3c
%% of this license or (at your option) any later version.
%% The latest version of this license is in
%%    http://www.latex-project.org/lppl.txt
%% and version 1.3c or later is part of all distributions of LaTeX
%% version 2005/12/01 or later.
%% 
%% This file has the LPPL maintenance status "maintained".
%% 
%% This file may only be distributed together with a copy of the LaTeX
%% base system. You may however distribute the LaTeX base system without
%% such generated files.
%% 
%% The list of all files belonging to the LaTeX base distribution is
%% given in the file `manifest.txt'. See also `legal.txt' for additional
%% information.
%% 
%% The list of derived (unpacked) files belonging to the distribution
%% and covered by LPPL is defined by the unpacking scripts (with
%% extension .ins) which are part of the distribution.
%% \CharacterTable
%%  {Upper-case    \A\B\C\D\E\F\G\H\I\J\K\L\M\N\O\P\Q\R\S\T\U\V\W\X\Y\Z
%%   Lower-case    \a\b\c\d\e\f\g\h\i\j\k\l\m\n\o\p\q\r\s\t\u\v\w\x\y\z
%%   Digits        \0\1\2\3\4\5\6\7\8\9
%%   Exclamation   \!     Double quote  \"     Hash (number) \#
%%   Dollar        \$     Percent       \%     Ampersand     \&
%%   Acute accent  \'     Left paren    \(     Right paren   \)
%%   Asterisk      \*     Plus          \+     Comma         \,
%%   Minus         \-     Point         \.     Solidus       \/
%%   Colon         \:     Semicolon     \;     Less than     \<
%%   Equals        \=     Greater than  \>     Question mark \?
%%   Commercial at \@     Left bracket  \[     Backslash     \\
%%   Right bracket \]     Circumflex    \^     Underscore    \_
%%   Grave accent  \`     Left brace    \{     Vertical bar  \|
%%   Right brace   \}     Tilde         \~}
\@obsoletefile{fleqn.clo}{fleqn.sty}
%%
%% This is file `fleqn.sty',
%% generated with the docstrip utility.
%%
%% The original source files were:
%%
%% latex209.dtx  (with options: `fleqn')
%% 
%% This is a generated file.
%% 
%% Copyright 1993-2014
%% The LaTeX3 Project and any individual authors listed elsewhere
%% in this file.
%% 
%% This file was generated from file(s) of the LaTeX base system.
%% --------------------------------------------------------------
%% 
%% It may be distributed and/or modified under the
%% conditions of the LaTeX Project Public License, either version 1.3c
%% of this license or (at your option) any later version.
%% The latest version of this license is in
%%    http://www.latex-project.org/lppl.txt
%% and version 1.3c or later is part of all distributions of LaTeX
%% version 2005/12/01 or later.
%% 
%% This file has the LPPL maintenance status "maintained".
%% 
%% This file may only be distributed together with a copy of the LaTeX
%% base system. You may however distribute the LaTeX base system without
%% such generated files.
%% 
%% The list of all files belonging to the LaTeX base distribution is
%% given in the file `manifest.txt'. See also `legal.txt' for additional
%% information.
%% 
%% The list of derived (unpacked) files belonging to the distribution
%% and covered by LPPL is defined by the unpacking scripts (with
%% extension .ins) which are part of the distribution.
%% \CharacterTable
%%  {Upper-case    \A\B\C\D\E\F\G\H\I\J\K\L\M\N\O\P\Q\R\S\T\U\V\W\X\Y\Z
%%   Lower-case    \a\b\c\d\e\f\g\h\i\j\k\l\m\n\o\p\q\r\s\t\u\v\w\x\y\z
%%   Digits        \0\1\2\3\4\5\6\7\8\9
%%   Exclamation   \!     Double quote  \"     Hash (number) \#
%%   Dollar        \$     Percent       \%     Ampersand     \&
%%   Acute accent  \'     Left paren    \(     Right paren   \)
%%   Asterisk      \*     Plus          \+     Comma         \,
%%   Minus         \-     Point         \.     Solidus       \/
%%   Colon         \:     Semicolon     \;     Less than     \<
%%   Equals        \=     Greater than  \>     Question mark \?
%%   Commercial at \@     Left bracket  \[     Backslash     \\
%%   Right bracket \]     Circumflex    \^     Underscore    \_
%%   Grave accent  \`     Left brace    \{     Vertical bar  \|
%%   Right brace   \}     Tilde         \~}
\@obsoletefile{fleqn.clo}{fleqn.sty}
\input{fleqn.clo}
\endinput
%%
%% End of file `fleqn.sty'.

\endinput
%%
%% End of file `fleqn.sty'.

\endinput
%%
%% End of file `fleqn.sty'.
}
%    \end{macrocode}
%
%    Offenes Bibliographie Format.
%    \begin{macrocode}
\DeclareOption{openbib}{%
  \AtEndOfPackage{%
   \renewcommand\@openbib@code{%
      \advance\leftmargin\bibindent
      \itemindent -\bibindent
      \listparindent \itemindent
      \parsep \z@
      }%
   \renewcommand\newblock{\par}}%
}
%    \end{macrocode}
%
%    Ausf\"uhren der voreinstellten Optionen.
%    \begin{macrocode}
\ExecuteOptions{letterpaper,10pt,oneside,onecolumn,final,sfbold}
%    \end{macrocode}
%
%    Ausf\"uhren der benutzerspezifischen Optionen.
%    \begin{macrocode}
\ProcessOptions
%    \end{macrocode}
%
%    Einlesen der Schriftgr\"o\ss{}en.
%    \begin{macrocode}
\input{size1\@ptsize.clo}
%    \end{macrocode}
%
%    Standardwerte beim Satz von Paragraphen (Zeilenabst\"ande,
%    Zeileneinzug Trennungen, etc.). Standardvorgaben der
%    Document-Classes.
%
%    \begin{macrocode}
\setlength\lineskip{1\p@}
\setlength\normallineskip{1\p@}
\renewcommand\baselinestretch{}
\if@noind
  \setlength\parskip{0.5\baselineskip 
         \@plus.1\baselineskip \@minus.1\baselineskip}
  \setlength\parindent{\z@} 
  \def\noparskip{\par\vspace{-\parskip}}
\else
  \setlength\parskip{0\p@ \@plus \p@}
  \let\noparskip\relax
\fi
\@lowpenalty   51
\@medpenalty  151
\@highpenalty 301
%    \end{macrocode}
%
%    Unver\"anderte Standardvorgaben zur Behandlung von Floats.
%    \begin{macrocode}
\setcounter{topnumber}{2}
\renewcommand\topfraction{.7}
\setcounter{bottomnumber}{1}
\renewcommand\bottomfraction{.3}
\setcounter{totalnumber}{3}
\renewcommand\textfraction{.2}
\renewcommand\floatpagefraction{.5}
\setcounter{dbltopnumber}{2}
\renewcommand\dbltopfraction{.7}
\renewcommand\dblfloatpagefraction{.5}
%    \end{macrocode}
%
%    Variable Kopfzeilen, ggf. zentriert und unterstrichen. Im Fall
%    eines \texttt{journal} sind die Befehle |\sectionmark| bzw.
%    |\subsectionmark| leer.
%
%    \begin{macrocode}
\def\e@skip{\h@font{\phantom{y}}}
\if@twoside                                
  \def\ps@headings{%
    \let\@oddfoot\@empty\let\@evenfoot\@empty
    \def\@evenhead{\vbox{\hsize=\textwidth
      \hbox to \textwidth{%
        {\pn@font\thepage}\hfill{\h@font\leftmark}\e@skip\if@center\hfill\fi}
        \if@headline \vskip 1.5pt \hrule \fi}}%
    \def\@oddhead{\vbox{\hsize=\textwidth
      \hbox to \textwidth{%
        \if@center\hfill\fi{\h@font\rightmark}\e@skip\hfill{\pn@font\thepage}}
        \if@headline \vskip 1.5pt \hrule \fi}}%
    \let\@mkboth\markboth
    \def\sectionmark##1{\markboth{\ifnum \c@secnumdepth
       >\z@ \if@headcount \thesection \ \fi \fi ##1}{}}
    \def\subsectionmark##1{\markright{\ifnum \c@secnumdepth >\@ne
      \if@headcount \thesubsection \ \fi \fi ##1}}}
\else 
  \def\ps@headings{%
    \let\@oddfoot\@empty\let\@evenfoot\@empty
    \def\@oddhead{\vbox{\hsize=\textwidth
       \hbox to \textwidth{%
         \if@center\hfill\fi{\h@font\rightmark}\e@skip\hfill{\pn@font\thepage}}
         \if@headline \vskip 1.5pt \hrule \fi}}%
    \def\sectionmark##1{\markright {\if@headcount
      \ifnum \c@secnumdepth >\z@ \thesection \ \fi \fi ##1}}}
\fi
%    \end{macrocode}
%
%    Variable Kopfzeilen, ggf. zentriert und unterstrichen.
%
%    \begin{macrocode}
\def\ps@myheadings{%
    \let\@oddfoot\@empty\let\@evenfoot\@empty
    \def\@oddhead{\vbox{\hsize=\textwidth
      \hbox to \textwidth{%
      \if@center\hfill\fi{\h@font\rightmark}\e@skip\hfill{\pn@font\thepage}}
      \if@headline \vskip 1.5pt \hrule \fi}}%
    \def\@evenhead{\vbox{\hsize=\textwidth
      \hbox to \textwidth{%
      {\pn@font\thepage}\hfill{\h@font\leftmark}\e@skip\if@center\hfill\fi}
      \if@headline \vskip 1.5pt \hrule \fi}}%
    \let\@mkboth\@gobbletwo
    \let\sectionmark\@gobble
    \let\subsectionmark\@gobble
}
%    \end{macrocode}
%
%    Sind die Kopfzeilendefinition |\@oddrunhead|
%    bzw. |\@evenrunhead| leer, werden  den anderweitig
%    vorgebebenen |\markboth| Kommandos in der Kopfzeile
%    ausgegeben. 
%
%    \begin{macrocode}
\if@journal
  \def\evenrunhead#1{\gdef\@evenrunhead{#1}}    \def\@evenrunhead{}
  \def\oddrunhead#1{\gdef\@oddrunhead{#1}}      \def\@oddrunhead{}
  \def\ps@journal{%
      \let\@oddfoot\@empty\let\@evenfoot\@empty
      \def\@oddhead{\vbox{\hsize=\textwidth
        \hbox to \textwidth{%
        \if@center\hfill\fi{\h@font%
        \ifx\@oddrunhead\@empty\rightmark\else\@oddrunhead\fi}
        \e@skip\hfill{\pn@font\thepage}}
        \if@headline \vskip 1.5pt \hrule \fi}}%
      \def\@evenhead{\vbox{\hsize=\textwidth
        \hbox to \textwidth{%
        {\pn@font\thepage}\e@skip\hfill{\h@font%
         \ifx\@evenrunhead\@empty\leftmark\else\@evenrunhead\fi}
        \if@center\hfill\fi}
        \if@headline \vskip 1.5pt \hrule \fi}}%
      \let\@mkboth\@gobbletwo
      \let\sectionmark\@gobble
      \let\subsectionmark\@gobble
  }
\fi
%    \end{macrocode}
%
%     Kopfzeile f\"ur vorl\"aufige Formatierungen im Zusammenhang mit der
%     Option \texttt{draft}. Die Kopfzeile enth\"alt Datum, Uhrzeit und
%     Seitenzahl. 
%
%    \begin{macrocode}
\def\ps@draft{%
    \let\@oddfoot\@empty\let\@evenfoot\@empty
    \def\@oddhead{\vbox{\hsize=\textwidth
      \hbox to \textwidth{%
      {\pn@font\today\ \now\ --- {\h@font \draftname: ``\jobname''}
        \hfil\e@skip \thepage}} \if@headline \vskip 1.5pt \hrule \fi}}%
    \def\@evenhead{\vbox{\hsize=\textwidth
   \hbox to \textwidth{%
       \pn@font\thepage\e@skip\hfil {\h@font \draftname: ``\jobname''} ---
       \today\ \now\ } \if@headline \vskip 1.5pt \hrule \fi}}%
    \let\@mkboth\@gobbletwo
    \let\sectionmark\@gobble
    \let\subsectionmark\@gobble
}
%    \end{macrocode}
%
%    Leere Seiten erhalten eine Markierung am oberen Seitenrand. Dem
%    Springer Makro `svma' entliehen. 
%
%    \begin{macrocode}
\@ifundefined{option@crosshair}{}{%
  \def\clap#1{\hbox to 0pt{\hss#1\hss}} \newdimen\@crosshairrule
  \@crosshairrule=.24pt \def\@crosshairs{\vbox to
    0pt{\hsize=0pt\baselineskip=0pt\lineskip=0pt \vss \clap{\vrule
        height .125in width \@crosshairrule depth 0pt} \clap{\vrule
        width .25in height \@crosshairrule depth 0pt} \clap{\vrule
        height .125in width \@crosshairrule depth 0pt} \vss}}
 \def\ps@empty{%
    \let\@oddfoot\@empty\let\@evenfoot\@empty
    \def\@oddhead{\hfill\raise\headheight\@crosshairs}
    \let\@evenhead\@oddhead}}
%    \end{macrocode}
%
%   Es folgen die Definitionen der erweiterten Titelei.
%   Der Titel enth\"alt zwei zus\"atzliche Kommandos zur
%   Aufnahme des Untertitels und der Institution. Diese werden
%   ggf. unterhalb des Haupttitels bzw. des Autors in den
%   eingestellten Schriftarten ausgeben. Da im |\@maketitle|
%   Kommando zentrierter und linksb\"undiger Satz m\"oglich ist, wird das
%   Kommando |\and| neu definiert. Der Seitenstil ist
%   \texttt{empty}, da Kopfzeilen voreingestellt sind.
%
%   Im Fall eines \texttt{journal} werden Kurzautoren und Kurztitel
%   ins Inhaltsverzeichnis desselben aufgenommen. Ferner werden in
%   Abh\"angigkeit von Seitenstil die Kurznamen von Autoren und Titel in
%   die Kopfzeile \"ubernommen. 
%
%    \begin{macrocode}
\def\@subtitle{}    \def\subtitle#1{\gdef\@subtitle{#1}}
\def\@institution{} \def\institution#1{\gdef\@institution{#1}}
\def\@shortauthor{} \def\shortauthor#1{\gdef\@shortauthor{#1}}
\def\@shorttitle{}  \def\shorttitle#1{\gdef\@shorttitle{#1}}
\def\and{\end{tabular}\hskip 1em plus.17fil
  \if@center\begin{tabular}[t]{c}\else\begin{tabular}[t]{@{}l@{}}\fi}
%    \end{macrocode}
%
%    Definition einer Titelseite mit den neuen Schriftvorgaben.
%
%    \begin{macrocode}
\if@titlepage
  \newcommand\maketitle{\begin{titlepage}%
  \let\footnotesize\small
  \let\footnoterule\relax
  \let\real@thanks\thanks
  \DeclareRobustCommand\thanks{\real@thanks}
  \let \footnote \thanks
  \null\vfil
  \vskip 60\p@
  \if@center \begin{center} \else \begin{raggedright} \fi
     {\t@font \if@upper \uppercase\expandafter{\@title} \else 
                 \@title \fi \par}% 
    \vskip 3em%
     {\ifx\@subtitle\@empty\else
                  \vskip.5em \st@font \@subtitle \par \fi}
      \vskip 1.5em             
     {\a@font \lineskip .75em
        \if@center\begin{tabular}[t]{c}\else\begin{tabular}[t]{@{}l@{}}\fi
         \@institution \end{tabular} \par }
     {\ifx\@institution\@empty\else\vskip.5em  
        \in@font\bf \lineskip .75em
        \if@center\begin{tabular}[t]{c}\else\begin{tabular}[t]{@{}l@{}}\fi
         \@institution \end{tabular} \par\fi}
      \vskip 1.5em%
    {\large \@date \par}%       % Set date in \large size.
  \if@center \end{center} \else \end{raggedright} \fi
  \vfil\null
  \@thanks
  \end{titlepage}%
  \setcounter{footnote}{0}%
  \let\thanks\relax\let\maketitle\relax
  \gdef\@thanks{}\gdef\@author{}\gdef\@title{}\gdef\@institution{}
  \gdef\@subtitle{}}
%    \end{macrocode}
%
%    Definition der Danksagung und der Kopfzeilen im Fall eines
%    Journals, wenn keine separate Titelseite angefordert wird.
%
%    \begin{macrocode}
\else
  \newcommand\maketitle{%
  \setcounter{footnote}{0}\par
  \begingroup
    \renewcommand\thefootnote{\fnsymbol{footnote}}%
    \def\@makefnmark{\hbox to\z@{$\m@th^{\@thefnmark}$\hss}}%
    \long\def\@makefntext##1{\parindent 1em\noindent
            \hbox to1.8em{\hss$\m@th^{\@thefnmark}$}##1}%
    \if@twocolumn
      \ifnum \col@number=\@ne
        \@maketitle
      \else
        \twocolumn[\@maketitle]%
      \fi
    \else
      \newpage
      \global\@topnum\z@   
      \@maketitle
    \fi
    \thispagestyle{plain}\@thanks
  \endgroup
  \setcounter{footnote}{0}%
  \if@journal
    \typeout{Article: \@shortauthor}
    \setcounter{section}{0}%
    \setcounter{subsection}{0}%
    \setcounter{subsubsection}{0}%
    \setcounter{paragraph}{0}%
    \setcounter{subparagraph}{0}%
    \setcounter{figure}{0}%
    \setcounter{table}{0}%
    \setcounter{equation}{0}%
    \addcontentsline{jou}{titles}{%
      {\sss@font\@shortauthor}\hfill\mbox{}\vskip\normallineskip%
       \ifx\@shorttitle\@empty\@title\else\@shorttitle\fi}
      \if@twoside
        \ifx\@oddrunhead\@empty\ifx\@evenrunhead\@empty
        \markboth{\@shortauthor}{\@shorttitle}\else
        \markboth{}{\@shortauthor, \@shorttitle}\fi \else
        \markboth{\@shortauthor, \@shorttitle}{}\fi
      \else
        \markright{\@shortauthor, \@shorttitle}
      \fi
  \else
    \let\maketitle\relax \let\@maketitle\relax
  \fi
    \gdef\@thanks{}\gdef\@author{}\gdef\@title{}\gdef\@institution{}
    \gdef\@subtitle{}}
%    \end{macrocode}
%
%    Wird keine seperate Titleseite angeforderte, \"ubernimmt dieses
%    Makro den Satz der Titelei. Der vertikale Vorschub vor dem Titel
%    ist variable.
%
%    \begin{macrocode}
\newlength{\beforetitlespace} \setlength{\beforetitlespace}{2em}
\def\@maketitle{%
  \cleardoublepage
  \null
  \vskip \beforetitlespace%
  \if@center \begin{center} \else \begin{raggedright} \fi
     {\t@font \if@upper \uppercase\expandafter{\@title} \else 
                 \@title \fi \par}% 
     {\ifx\@subtitle\@empty\else
                  \vskip.5em \st@font \@subtitle \par \fi}
      \vskip 1.5em             
     {\a@font \lineskip .5em
        \if@center\begin{tabular}[t]{c}\else\begin{tabular}[t]{@{}l@{}}\fi
         \@author \end{tabular} \par}
     {\ifx\@institution\@empty\else\vskip.5em  
        \in@font\bf \lineskip .75em
        \if@center\begin{tabular}[t]{c}\else\begin{tabular}[t]{@{}l@{}}\fi
         \@institution \end{tabular} \par\fi}
  \if@center \end{center} \else \end{raggedright} \fi
  \par \vskip 1.5em
} 
\fi
%    \end{macrocode}
%
%    Es folgt die Definition der Gliederungs\"uberschriften. 
%
%    Zun\"achst wird die Gliederungstiefe festgelegt, bis zu welcher
%    einer Numerierung erfolgt. 
%
%    \begin{macrocode}
\setcounter{secnumdepth}{3}
%    \end{macrocode}
%
%    Definition der Z\"ahler f\"ur die Gliederungsnumerierung.
%
%    \begin{macrocode}
\newcounter {part}
\newcounter {section}
\newcounter {subsection}[section]
\newcounter {subsubsection}[subsection]
\newcounter {paragraph}[subsubsection]
\newcounter {subparagraph}[paragraph]
%    \end{macrocode}
%
%    Definition der Ausgabeform dieser Z\"ahler.
%
%    \begin{macrocode}
\renewcommand\thepart          {\Roman{part}}
\renewcommand\thesection       {\arabic{section}}
\renewcommand\thesubsection    {\thesection.\arabic{subsection}}
\renewcommand\thesubsubsection {\thesubsection.\arabic{subsubsection}}
\renewcommand\theparagraph     {\thesubsubsection.\arabic{paragraph}}
\renewcommand\thesubparagraph  {\theparagraph.\arabic{subparagraph}}
%    \end{macrocode}
%
%
%    Der Satz einer Teil\"uberschrift erfolgt ggf. zentriert, in
%    Gro\ss{}buchstaben und der voreingestellten Schriftart. Im Fall eines
%    \texttt{journal} wird die Kopfzeilenmarkierung nicht neu
%    initialisiert. 
%
%    \begin{macrocode}
\newcommand\part{\par
   \addvspace{4ex}%
   \@afterindentfalse
   \secdef\@part\@spart}
\def\@part[#1]#2{%
    \ifnum \c@secnumdepth >\m@ne
      \refstepcounter{part}%
      \addcontentsline{toc}{part}{\thepart\hspace{1em}#1}%
    \else
      \addcontentsline{toc}{part}{#1}%
    \fi
    {\parindent \z@ \if@center\centering\else\raggedright\fi
     \interlinepenalty \@M
     \reset@font
     \ifnum \c@secnumdepth >\m@ne
       \p@font \partname~\thepart.\ 
     \fi
     \if@upper\uppercase{#2}\else#2\fi% 
     \if@journal\else\markboth{}{}\fi\par}%
    \nobreak
    \vskip 3ex
    \@afterheading}
\def\@spart#1{%
    {\parindent \z@ 
     \if@center\centering\else\raggedright\fi
     \interlinepenalty \@M
     \reset@font
     \p@font\if@upper\uppercase{#1}\else#1\fi\par}
     \nobreak
     \vskip 3ex
     \@afterheading}
\def\@endpart{\vfil\newpage
              \if@twoside
                \hbox{}%
                \thispagestyle{empty}%
                \newpage
              \fi
              \if@tempswa
                \twocolumn
              \fi}
%    \end{macrocode}
%
%    \"Aquivalente Definitionen eines Teil f\"ur die Klasse
%    \texttt{journal}. Die Teil\"uberschrift wird in gleicher H\"ohe wie
%    eine Titel\"uberschrift gesetzt, ins  Inhaltsverzeichnis der
%    Zeitschrift aufgenommen und der Titel in die Kopfzeile \"ubernommen. 
%
%    \begin{macrocode}
\if@journal 
\newcommand\journalpart{\par 
   \addvspace{4ex}%
   \@afterindentfalse 
   \secdef\@journalpart\@sjournalpart}
\def\@journalpart[#1]#2{\addcontentsline{jou}{part}{#1}
   {\parindent \z@ \if@center\centering\else\raggedright\fi
    \interlinepenalty \@M 
    \reset@font
    \t@font
    \if@upper\uppercase{#2}\else#2\fi%
    \markboth{#1}{#1}\par}%
    \nobreak 
    \vskip 3ex
    \@afterheading}
\def\@sjournalpart#1{{\parindent \z@ 
     \if@center\centering\else\raggedright\fi
     \interlinepenalty \@M
     \reset@font
     \t@font\if@upper\uppercase{#1}\else#1\fi
     \markboth{#1}{#1}\par}
     \nobreak
     \vskip 3ex
     \@afterheading}
\def\@endjournalpart{\vfil\newpage
              \if@twoside
                \hbox{}%
                \thispagestyle{empty}%
                \newpage
              \fi
              \if@tempswa
                \twocolumn
              \fi}
\fi
%    \end{macrocode}
%
%    Die folgenden Kommandos definieren \"Uberschriften auf tieferer
%    Gliederungsebene. Gegen\"uber den Standardklassen wurde jeweils die
%    M\"oglichkeit zentrierter \"Uberschriften eingef\"uhrt und es wurden
%    die Schriftarten ver\"andert.
%
%    \begin{macrocode}
\newcommand\section{\@startsection {section}{1}{\z@}%
                                   {-3.5ex \@plus -1ex \@minus -.2ex}%
                                   {2.3ex \@plus.2ex}%
                                   {\if@center\centering\else\raggedright\fi 
                                    \reset@font\s@font}}
\newcommand\subsection{\@startsection{subsection}{2}{\z@}%
                                     {-3.25ex\@plus -1ex \@minus -.2ex}%
                                     {1.5ex \@plus .2ex}%
                                     {\if@center\centering\else\raggedright\fi 
                                      \reset@font\ss@font}}
\newcommand\subsubsection{\@startsection{subsubsection}{3}{\z@}%
                                     {-3.25ex\@plus -1ex \@minus -.2ex}%
                                     {1.5ex \@plus .2ex}%
                                     {\if@center\centering\else\raggedright\fi 
                                      \reset@font\sss@font}}
\newcommand\paragraph{\@startsection{paragraph}{4}{\z@}%
                                    {3.25ex \@plus1ex \@minus.2ex}%
                                    {-1em}%
                                    {\reset@font\pg@font}}
\newcommand\subparagraph{\@startsection{subparagraph}{5}{\parindent}%
                                       {3.25ex \@plus1ex \@minus .2ex}%
                                       {-1em}%
                                      {\reset@font\spg@font}}
%    \end{macrocode}
%
%    Es folgt die Definition verschiedener Listenumgebungen. Zun\"achst
%    werden eine Reihe globaler Definitionen und Einstellungen
%    \"ubernommen. 
% 
%    \begin{macrocode}
\if@twocolumn
  \setlength\leftmargini  {2em}
\else
  \setlength\leftmargini  {2.5em}
\fi
\setlength\leftmarginii  {2.2em}
\setlength\leftmarginiii {1.87em}
\setlength\leftmarginiv  {1.7em}
\if@twocolumn
  \setlength\leftmarginv  {.5em}
  \setlength\leftmarginvi {.5em}
\else
  \setlength\leftmarginv  {1em}
  \setlength\leftmarginvi {1em}
\fi
\setlength\leftmargin    {\leftmargini}
\setlength  \labelsep  {.5em}
\setlength  \labelwidth{\leftmargini}
\addtolength\labelwidth{-\labelsep}
\@beginparpenalty -\@lowpenalty
\@endparpenalty   -\@lowpenalty
\@itempenalty     -\@lowpenalty
%    \end{macrocode}
%
%    Definitionen der Listenumgebungen. Zun\"achst die Voreinstellungen
%    f\"ur die Umgebung \verb+enumerate+ f\"ur Standard bzw. dekadische
%    Numerierung. 
%
%    \begin{macrocode}
\if@enumeration
  \renewcommand\theenumi{\arabic{enumi}}
  \renewcommand\theenumii{\alph{enumii}}
  \renewcommand\theenumiii{\roman{enumiii}}
  \renewcommand\theenumiv{\Alph{enumiv}}
  \newcommand\labelenumi{\theenumi.}
  \newcommand\labelenumii{(\theenumii)}
  \newcommand\labelenumiii{\theenumiii.}
  \newcommand\labelenumiv{\theenumiv.}
  \renewcommand\p@enumii{\theenumi}
  \renewcommand\p@enumiii{\theenumi(\theenumii)}
  \renewcommand\p@enumiv{\p@enumiii\theenumiii}
\else
  \renewcommand\theenumi{\arabic{enumi}}
  \renewcommand\theenumii{\arabic{enumii}}
  \renewcommand\theenumiii{\arabic{enumiii}}
  \renewcommand\theenumiv{\arabic{enumiv}}
  \newcommand\labelenumi{\theenumi.}
  \newcommand\labelenumii{\theenumi.\theenumii.}
  \newcommand\labelenumiii{\theenumi.\theenumii.\theenumiii.}
  \newcommand\labelenumiv{\theenumi.\theenumii.\theenumiii.\theenumiv.}
  \renewcommand\p@enumii{\theenumi}
  \renewcommand\p@enumiii{\theenumi(\theenumii)}
  \renewcommand\p@enumiv{\p@enumiii\theenumiii}
\fi
%    \end{macrocode}
%
%    Die Definition neuer Umgebungen beginnt mit dem \texttt{abstract}
%    und den \texttt{keywords}. 
%
%    \begin{macrocode}
\if@titlepage
  \newenvironment{keywords}{%
      \titlepage
      \null\vfil
      \@beginparpenalty\@lowpenalty
      \if@center\begin{center}\else\begin{raggedright}\fi%
        {\sss@font  \keywordname}
        \@endparpenalty\@M
      \if@center\end{center}\else\end{raggedright}\fi}%
     {\par\vfil\null\endtitlepage}
  \newenvironment{abstract}{%
      \titlepage
      \null\vfil
      \@beginparpenalty\@lowpenalty
      \if@center\begin{center}\else\begin{raggedright}\fi%
        {\sss@font  \abstractname}
        \@endparpenalty\@M
      \if@center\end{center}\else\end{raggedright}\fi}%
     {\par\vfil\null\endtitlepage}
\else
   \if@center
    \def\abstract{%
       \if@twocolumn
         \small\subsubsection*{\abstractname}%
      \else 
        \small 
        \begin{center} 
        {\sss@font \abstractname\vspace{-.5em}\vspace{\z@}}%
        \end{center} \quotation
      \fi}
    \def\endabstract{\if@twocolumn\else\endquotation\fi}
    \def\keywords{\small\paragraph*{\keywordname:}}
    \def\endkeywords{\par\bigskip}
   \else
     \def\abstract{\small\subsubsection*{\abstractname}}
     \def\endabstract{\par\bigskip}
     \def\keywords{\small\paragraph*{\keywordname: }}
     \let\endkeywords\endabstract
  \fi
\fi
%    \end{macrocode}
%
%    Definition \texttt{itemize} der Item-Markierungen. Entweder
%    \"ubliche Staffelung der Markierungen oder keine Hervorhebung der
%    "`itemization"'. 
%
%    \begin{macrocode}
\if@itemization
  \newcommand\labelitemi{$\m@th\bullet$}
  \newcommand\labelitemii{\normalfont\bfseries --}
  \newcommand\labelitemiii{$\m@th\ast$}
  \newcommand\labelitemiv{$\m@th\cdot$}
\else
  \newcommand\labelitemi{\bfseries --}
  \newcommand\labelitemii{\bfseries --}
  \newcommand\labelitemiii{\bfseries --}
  \newcommand\labelitemiv{\bfseries --}
\fi
%    \end{macrocode}
%
%    Die Labels der \texttt{description} Umgebung werden in der
%    Schriftart \verb+\item@font+ gesetzt. 
%
%    \begin{macrocode}
\newenvironment{description}
               {\list{}{\labelwidth\z@ \itemindent-\leftmargin
                        \let\makelabel\descriptionlabel}}
               {\endlist}
\newcommand\descriptionlabel[1]{\hspace\labelsep
                                \item@font #1}
%    \end{macrocode}
%
%    Unver\"anderte Definition der \texttt{verse} Umgebung.
%
%    \begin{macrocode}
\newenvironment{verse}
               {\let\\=\@centercr
                \list{}{\itemsep      \z@
                        \itemindent   -1.5em%
                        \listparindent\itemindent
                        \rightmargin  \leftmargin
                        \advance\leftmargin 1.5em}%
                \item[]}
               {\endlist}
%    \end{macrocode}
%
%     Unver\"anderte Definition der \texttt{quotation} Umgebung.
%
%    \begin{macrocode}
\newenvironment{quotation}
               {\list{}{\listparindent 1.5em%
                        \itemindent    \listparindent
                        \rightmargin   \leftmargin
                        \parsep        \z@ \@plus\p@}%
                \item[]}
               {\endlist}
%    \end{macrocode}
%
%     Unver\"anderte Definitionen der\texttt{quote} Umgebung.
%
%    \begin{macrocode}
\newenvironment{quote}
               {\list{}{\rightmargin\leftmargin}%
                \item[]}
               {\endlist}
%    \end{macrocode}
%
%    Die Bezeichnung einer \texttt{theorem} Umgebung wird in der
%    Schriftart \verb+\thh@font+ gesetzt, der Text selbst in
%    \verb+\thb@font+. 
%
%    \begin{macrocode}
\def\@begintheorem#1#2{\reset@font\thb@font\trivlist
      \item[\hskip \labelsep{\reset@font\thh@font #1\ #2:}]}
\def\@opargbegintheorem#1#2#3{\reset@font\thb@font\trivlist
      \item[\hskip \labelsep{\reset@font\thh@font #1\ #2\ (#3):}]}
\def\@endtheorem{\endtrivlist}
%    \end{macrocode}
%
%    Die Umgebung zur Beschreibung von Beispielen.
%
%    \begin{macrocode}
\newlength{\exampleindent}    \setlength{\exampleindent}{\parindent}
\newenvironment{example}%
   {\begin{list}{}{%
    \setlength{\leftmargin}{\exampleindent}}
    \ex@font \item[]}
   {\end{list}}
%    \end{macrocode}
%
%    Die Umgebung \texttt{describe} Umgebung. Das \"ubergebene Argument
%    dient zur Berechnung des breitesten Labels. 
%
%    \begin{macrocode}
\newenvironment{describe}[1][\quad]%
  {\begin{list}{}{%
    \renewcommand{\makelabel}[1]{{\item@font ##1}\hfil}%
    \settowidth{\labelwidth}{{\item@font #1}}%
    \setlength{\leftmargin}{\labelwidth}%
    \addtolength{\leftmargin}{\labelsep}}}%
  {\end{list}}
%    \end{macrocode}
%
%    Die Titelseite.
%
%    \begin{macrocode}
\newenvironment{titlepage}
    {%
      \if@twocolumn
        \@restonecoltrue\onecolumn
      \else
        \@restonecolfalse\newpage
      \fi
      \thispagestyle{empty}%
      \if@compatibility
        \setcounter{page}{0}
      \else
        \setcounter{page}{1}%
      \fi}%
    {\if@restonecol\twocolumn \else \newpage \fi
      \setcounter{page}{1}%
    }
%    \end{macrocode}
%
%
%    Das Makro |\review| ruft die Sternform oder die normale Form
%    ggf. mit optionalem Parameter auf. Der vertikale Abstand
%    entspricht einer \verb+\subsubsection+, ein Eintrag ins
%    Inhaltsverzeichnis wird vorgenommen. 
%
%    \begin{macrocode}
  \def\review{\@ifstar{\@sreview[]}{%
    \@ifnextchar [{\@tempswatrue\@review}{\@tempswafalse\@review[]}}}
  \def\@review[#1]#2#3{\setcounter{footnote}{0}
    \vskip 3.25ex plus1ex minus.2ex \noindent
    {\sss@font #2}\\\emph{#3.}\@afterheading
    \if@journal
      \addcontentsline{jou}{titles}{%
        {\sss@font#2}\hfill\mbox{}\vskip\normallineskip#3
        \if@tempswa (#1)\fi}
    \fi}
  \def\@sreview[#1]#2#3{\setcounter{footnote}{0}
     \vskip 3.25ex plus1ex minus.2ex \noindent
     {\sss@font #2}\\\emph{#3.} \@afterheading}
%    \end{macrocode}
%
%    Das Makro |\revauthor| \"ubernimmt den Autor und ruft
%    |\@makerevauthor|. Entspricht dem Satz von |\@author|
%    in der Titellei.  
%
%    \begin{macrocode}
  \def\revauthor#1{\setcounter{footnote}{0}
    \def\thefootnote{\fnsymbol{footnote}}
    \gdef\@revauthor{#1}\@makerevauthor}
  \def\@makerevauthor{\hfill{\lineskip .5em
    \if@center \begin{tabular}[t]{c} \else \begin{tabular}[t]{@{}l@{}} \fi
    \@revauthor \end{tabular} \par}\@thanks\@afterheading
    \setcounter{footnote}{0}\def\thefootnote{\arabic{footnote}}
    \gdef\@thanks{}\gdef\@revauthor{}}
%    \end{macrocode}
%
%    Unver\"anderte Definition eines Anhanges.
%
%    \begin{macrocode}
\newcommand\appendix{\par
  \setcounter{section}{0}%
  \setcounter{subsection}{0}%
  \renewcommand\thesection{\Alph{section}}}
%    \end{macrocode}
%
%    Unver\"anderte Voreinstellungen der \texttt{array} Umgebung.
%
%    \begin{macrocode}
\setlength\arraycolsep{5\p@}
\setlength\tabcolsep{6\p@}
\setlength\arrayrulewidth{.4\p@}
\setlength\doublerulesep{2\p@}
\setlength\tabbingsep{\labelsep}
%    \end{macrocode}
%
%    Unver\"anderte Voreinstellungen der \texttt{minipage} Umgebung.
%
%    \begin{macrocode}
\skip\@mpfootins = \skip\footins
%    \end{macrocode}
%
%    Unver\"anderte Voreinstellungen der \texttt{fbox}.
%
%    \begin{macrocode}
\setlength\fboxsep{3\p@}
\setlength\fboxrule{.4\p@}
%    \end{macrocode}
%
%    Z\"ahler zur Erzeugung von Gleichungsnummern.
%
%    \begin{macrocode}
\renewcommand{\theequation}{\arabic{equation}}
%    \end{macrocode}
%
%    Die Definition einiger Kommandos zur Erzeugung von Randnotizen in
%    Anlehnung an H. Partls \verb+\refman.sty+. Im Unterschied zur
%    dortigen Definition erscheinen die Notizen nicht nur am linken
%    Seitenrand. 
%
%    \begin{macrocode}
\def\marginlabel#1{\marginpar%
   {\if@twoside 
       \ifodd\c@page 
          \raggedright 
       \else 
          \raggedleft 
       \fi
     \else 
        \raggedright 
     \fi #1}}
\def\attention{\mbox{}%
    \marginpar[\raggedleft\large\bf! $\rightarrow$]%
        {\raggedright\large\bf $\leftarrow$ !}}
\def\seealso#1{\mbox{}%
    \marginpar[\raggedleft$\rightarrow$ \small #1]%
        {\raggedright\small  #1 $\leftarrow$}\ignorespaces}
%    \end{macrocode}
%
%
%     Definition des Abbildungsz\"ahlers.
%
%    \begin{macrocode}
\newcounter{figure}
\renewcommand\thefigure{\@arabic\c@figure}
%    \end{macrocode}
%
%     Voreinstellungen der \texttt{figure}  Umgebung. Im Makro
%     \verb+\fnum@figure+ wird der Kurzname benutzt. Die "`Floats"'
%     werden in der voreingestellten Schrift \verb+\fig@font+ gesetzt.
%
%    \begin{macrocode}
\def\fps@figure{tbp}
\def\ftype@figure{1}
\def\ext@figure{lof}
\def\fnum@figure{\figureshortname~\thefigure}
\newenvironment{figure}
               {\fig@font\@float{figure}}
               {\end@float}
\newenvironment{figure*}
               {\fig@font\@dblfloat{figure}}
               {\end@dblfloat}
%    \end{macrocode}
%
% Definition des Tabellenz\"ahlers.
%
%    \begin{macrocode}
\newcounter{table}
\renewcommand\thetable{\@arabic\c@table}
%    \end{macrocode}
%
%     Voreinstellungen der \texttt{table}  Umgebung. Im Makro
%     \verb+\fnum@table+ wird der Kurzname benutzt. Die "`Floats"'
%     werden in der voreingestellten Schrift \verb+\tab@font+ gesetzt.
%
%
%    \begin{macrocode}
\def\fps@table{tbp}
\def\ftype@table{2}
\def\ext@table{lot}
\def\fnum@table{\tableshortname~\thetable}
\newenvironment{table}
               {\tab@font\@float{table}}
               {\end@float}
\newenvironment{table*}
               {\tab@font\@dblfloat{table}}
               {\end@dblfloat}
%    \end{macrocode}
%
%    Unver\"anderte Abst\"ande.
%
%    \begin{macrocode}
\newlength\abovecaptionskip
\newlength\belowcaptionskip
\setlength\abovecaptionskip{10\p@}
\setlength\belowcaptionskip{0\p@}
%    \end{macrocode}
%
%    Lange Figuren oder Tabellenbeschriftungen werden um Bezeichnung
%    und Nummer einger\"uckt. Beschriftungen werden in den Fonts
%    |cph@font| und |\cpb@font| gesetzt.
%
%    \begin{macrocode}
\long\def\@makecaption#1#2{%
 \vskip\abovecaptionskip
 \setbox\@tempboxa\hbox{{\cph@font #1:} {\cpb@font #2}}%
 \ifdim \wd\@tempboxa >\hsize 
    \@hangfrom{\cph@font #1: }{\cpb@font #2\par}%
 \else 
    \hbox to\hsize{\hfil\box\@tempboxa\hfil}%
 \fi
 \vskip\belowcaptionskip}
%    \end{macrocode}
%
%    Einfache St\"utze zur Tabellenkonstruktion.
%
%    \begin{macrocode}
\def\rb#1{\raisebox{1.5ex}[-1.5ex]{#1}}
%    \end{macrocode}
%
%    Einfache horizontale Linien in Tabellen. Die Definition der
%    Makros entspricht der Konstruktion von |\hline| in
%    \verb+latex.tex+. Das von den Befehlen aufgerufene Makro
%    |\@xhline| f\"ugt einen zus\"atzlichen Vorschub ein,falls der
%    Befehl wiederholt gegeben wird.  
%
%    \begin{macrocode}
\def\tablerule{\noalign{\ifnum0=`}\fi
   \hrule \@height \arrayrulewidth \vskip\doublerulesep
   \futurelet \@tempa\@xhline}
\def\thicktablerule{\noalign{\ifnum0=`}\fi
   \hrule \@height 2\arrayrulewidth \vskip\doublerulesep
   \futurelet \@tempa\@xhline}
\def\doubletablerule{\noalign{\ifnum0=`}\fi
   \hrule \@height \arrayrulewidth \vskip2\arrayrulewidth
   \hrule \@height \arrayrulewidth \vskip\doublerulesep
   \futurelet \@tempa\@xhline}
%    \end{macrocode}
%
%    Definition st\"arkerer und doppelter |\hline| Varianten.
%
%    \begin{macrocode}
\def\thickhline{\noalign{\ifnum0=`}\fi
   \hrule \@height 2\arrayrulewidth\futurelet \@tempa\@xhline}
\def\doublehline{\noalign{\ifnum0=`}\fi
   \hrule \@height \arrayrulewidth\vskip2\arrayrulewidth 
   \hrule \@height \arrayrulewidth \futurelet \@tempa\@xhline}
%    \end{macrocode}
%
%    Unver\"anderte Definition der \LaTeX{} 2.09 Schriftartenkommandos
%    und der Kommandos f\"ur mathematische Zeichens\"atze.
%
%    \begin{macrocode}
\DeclareOldFontCommand{\rm}{\normalfont\rmfamily}{\mathrm}
\DeclareOldFontCommand{\sf}{\normalfont\sffamily}{\mathsf}
\DeclareOldFontCommand{\tt}{\normalfont\ttfamily}{\mathtt}
\DeclareOldFontCommand{\bf}{\normalfont\bfseries}{\mathbf}
\DeclareOldFontCommand{\it}{\normalfont\itshape}{\mathit}
\DeclareOldFontCommand{\sl}{\normalfont\slshape}{\@nomath\sl}
\DeclareOldFontCommand{\sc}{\normalfont\scshape}{\@nomath\sc}
\DeclareRobustCommand*{\cal}{\@fontswitch{\relax}{\mathcal}}
\DeclareRobustCommand*{\mit}{\@fontswitch{\relax}{\mathnormal}}
%    \end{macrocode}
%
%    Definition von Abst\"anden und Gliederungstiefe f\"ur das
%    Inhaltsverzeichnis. 
%
%    \begin{macrocode}
\newcommand\@pnumwidth{1.55em}
\newcommand\@tocrmarg {2.55em}
\newcommand\@dotsep{4.5}
\setcounter{tocdepth}{3}
%    \end{macrocode}
%
%    Definitionen eines regul\"aren und eines "`kleine"'
%    Inhaltsverzeichnisses. Diese Umgebungen stehen im Fall des
%    \texttt{journal} nicht zur Verf\"ugung.
%
%    \begin{macrocode}
\if@journal
   \let\tableofcontents\relax
   \let\smalltableofcontents\relax
   \def\journalcontents{\journalpart*{\contentsname}
          \@starttoc{jou}}
\else
   \newcommand\tableofcontents{%
       \let\smalltableofcontents\relax
       \section*{\contentsname
           \@mkboth{\contentsname}{\contentsname}}%
       \@starttoc{toc}%
       }
   \newcommand\smalltableofcontents{%
      \let\tableofcontents\relax
       \subsubsection*{\contentsname
           \@mkboth{\contentsname}{\contentsname}}%
       \begin{small}
       \@starttoc{toc}%
       \end{small}}
\fi
%    \end{macrocode}
%
%     Nur der Titeleintrag, nicht aber die Seitenzahl wird
%     hervorgehoben. 
%
%    \begin{macrocode}
\newcommand\l@part[2]{%
  \ifnum \c@tocdepth >-2\relax
    \addpenalty{\@secpenalty}%
    \addvspace{2.25em \@plus\p@}%
    \begingroup
      \setlength\@tempdima{3em}%
      \parindent \z@ \rightskip \@pnumwidth
      \parfillskip -\@pnumwidth
      {\leavevmode
       {\sss@font#1}\hfil \hbox to\@pnumwidth{\hss #2}}\par
       \nobreak
       \if@compatibility
         \global\@nobreaktrue
         \everypar{\global\@nobreakfalse\everypar{}}
      \fi
    \endgroup
  \fi}
%    \end{macrocode}
%
%    Titeleintrag in das Inhaltsverzeichnis eines \texttt{journal}. 
%
%    \begin{macrocode}
\if@journal
  \newcommand\l@titles[2]{%
    \addpenalty{-\@highpenalty}%
    \vskip 1.0em \@plus\p@
    \begingroup
      \parindent \z@ \rightskip \@pnumwidth
      \parfillskip -\@pnumwidth
      \leavevmode #1%
      \nobreak\leaders\hbox{%
          $\m@th \mkern \@dotsep mu.\mkern \@dotsep mu$}\hfill
      \nobreak \hbox to\@pnumwidth{\hfil\rm #2}\par
      \penalty\@highpenalty
    \endgroup}
\fi
%    \end{macrocode}
%
%    Auch der Eintrag f\"ur die |\l@section| erfolgt als
%    |\@dottedtocline|.
%
%    \begin{macrocode}
\newcommand\l@section{\@dottedtocline{1}{1.5em}{2.3em}}
\newcommand\l@subsection   {\@dottedtocline{2}{1.5em}{2.3em}}
\newcommand\l@subsubsection{\@dottedtocline{3}{3.8em}{3.2em}}
\newcommand\l@paragraph    {\@dottedtocline{4}{7.0em}{4.1em}}
\newcommand\l@subparagraph {\@dottedtocline{5}{10em}{5em}}
%    \end{macrocode}
% 
%    Die Definition von Abbildungs- und Tabellenverzeichnis. Die
%    Kopfzeilenmarkierungen werden nicht in Gro\ss{}buchstaben
%    umgewandelt. 
%
%    \begin{macrocode}
\if@journal
  \let\listoffigures\relax
  \let\smalllistoffigures\relax
  \let\listoftables\relax
  \let\smalllistoftables\relax
\else
  \newcommand\listoffigures{%
      \let\smalllistoffigures\relax
      \section*{\listfigurename
        \@mkboth{\listfigurename}{\listfigurename}}%
      \@starttoc{lof}%
      }
  \newcommand\smalllistoffigures{%
      \let\listoffigures\relax
      \subsubsection*{\listfigurename
        \@mkboth{\listfigurename}{\listfigurename}}%
      \begin{small}
      \@starttoc{lof}%  
      \end{small}     
      }
  \newcommand\l@figure{\@dottedtocline{1}{1.5em}{2.3em}}
  \newcommand\listoftables{%
      \let\smalllistoftables\relax
      \section*{\listtablename
        \@mkboth{\listtablename}{\listtablename}}%
    \@starttoc{lot}%
    }
  \newcommand\smalllistoftables{%
      \let\listoftables\relax
      \subsubsection*{\listtablename
        \@mkboth{\listtablename}{\listtablename}}%
      \begin{small}
      \@starttoc{lot}%        
      \end{small}
    }
  \let\l@table\l@figure
\fi
%    \end{macrocode}
%
%    Die Definition des Literaturverzeichnisses. Die
%    Kopfzeilenmarkierung wird nicht in Gro\ss{}buchstaben umgewandelt.
%
%    \begin{macrocode}
\newdimen\bibindent
\setlength\bibindent{1.5em}
\newenvironment{thebibliography}[1]
     {\section*{\refname
        \@mkboth{\refname}{\refname}}%
      \list{\@biblabel{\@arabic\c@enumiv}}%
           {\settowidth\labelwidth{\@biblabel{#1}}%
            \leftmargin\labelwidth
            \advance\leftmargin\labelsep
            \@openbib@code
            \usecounter{enumiv}%
            \let\p@enumiv\@empty
            \renewcommand\theenumiv{\@arabic\c@enumiv}}%
      \sloppy\clubpenalty4000\widowpenalty4000%
      \sfcode`\.\@m}
     {\def\@noitemerr
       {\@latex@warning{Empty `thebibliography' environment}}%
      \endlist}
\newcommand\newblock{\hskip .11em\@plus.33em\@minus.07em}
\let\@openbib@code\@empty
%    \end{macrocode}
%
%    Die folgenden Makros werden nur bei Verwendung der Variante
%    \texttt{journal} aktiviert. Das Kommando |\thebibliograpy|
%    liest nicht mehr |\jobname.bbl|, sondern
%    |@\bblfile|. Das Makro |\@include| wird dahingehend
%    ge\"andert, da\ss{} |\@bblfile| auf den Namen der entsprechenden
%    Include Datei initialisiert. Die Makros stammen von Joachim
%    Schrodt \texttt{`bibperinclude.sty'} .
%   
%    \begin{macrocode}
\if@journal
  \def\@mainbblfile{\jobname.bbl}
  \let\@bblfile=\@mainbblfile
  \def\bibliography#1{%
    \if@filesw\immediate\write\@auxout{\string\bibdata{#1}}\fi
    \@input{\@bblfile}}
  \def\@include#1 {\clearpage
    \if@filesw \immediate\write\@mainaux{\string\@input{#1.aux}}\fi
    \@tempswatrue
    \if@partsw \@tempswafalse\edef\@tempb{#1}
       \@for\@tempa:=\@partlist\do{\ifx\@tempa\@tempb\@tempswatrue\fi}
    \fi
    \if@tempswa 
       \if@filesw \let\@auxout=\@partaux 
           \immediate\openout\@partaux #1.aux
           \immediate\write\@partaux{\relax}
       \fi
       \def\@bblfile{#1.bbl}\@input{#1.tex}
       \let\@bblfile\@mainbblfile\clearpage
       \@writeckpt{#1}
       \if@filesw 
          \immediate\closeout\@partaux 
       \fi
       \let\@auxout=\@mainaux\else\@nameuse{cp@#1}
   \fi}
\fi
%    \end{macrocode}
%
%    Definition des Index. Die Kopfzeilenmarkierung wird nicht in
%    Gro\ss{}buchstaben umgewandelt. Der Seitenstil ist standardm\"a\ss{}ig
%    \texttt{empty}, da Kopfzeilen voreingestellt sind.
% 
%    \begin{macrocode}
\newenvironment{theindex}
               {\if@twocolumn
                  \@restonecolfalse
                \else
                  \@restonecoltrue
                \fi
                \columnseprule \z@
                \columnsep 35\p@
                \twocolumn[\section*{\indexname}]%
                \@mkboth{\indexname}%
                        {\indexname}%
                \thispagestyle{empty}\parindent\z@
                \parskip\z@ \@plus .3\p@\relax
                \let\item\@idxitem}
               {\if@restonecol\onecolumn\else\clearpage\fi}
\newcommand\@idxitem  {\par\hangindent 40\p@}
\newcommand\subitem   {\par\hangindent 40\p@ \hspace*{20\p@}}
\newcommand\subsubitem{\par\hangindent 40\p@ \hspace*{30\p@}}
\newcommand\indexspace{\par \vskip 10\p@ \@plus5\p@ \@minus3\p@\relax}
%    \end{macrocode}
%
%   Unver\"anderte Definition der \verb+\footnoterule+.
%
%    \begin{macrocode}
\renewcommand\footnoterule{%
  \kern-3\p@
  \hrule width .4\columnwidth
  \kern 2.6\p@}
%    \end{macrocode}
%
%    Der Satz von Fu\ss{}notenabs\"atzen variiert in Abh\"angigkeit von der
%    Option \texttt{par}; ggf. werden Fu\ss{}noten ohne Einzug
%    geblockt. In jedem Fall wird nach dem Fu\ss{}notenzeichen ein
%    Zwischenraum von \texttt{.25em} eingef\"ugt.
%
%    \begin{macrocode}
\if@noind
  \long\def\@makefntext#1{%
        \leftskip 2.0em%
        \noindent
        \hbox to 0em{\hss\@makefnmark\kern 0.25em}#1}
\else
  \long\def\@makefntext#1{%
      \parindent 1em%
      \noindent
      \hbox to 1.8em{\hss\@makefnmark\kern 0.25em}#1}
\fi           
%    \end{macrocode}
%
%    Definition der "`Captions"'. Standardm\"a\ss{}ig werden Kurznamen f\"ur
%    Abbildungen und Tabellen benutzt, die im entsprechenden
%    \verb+captions+\emph{language} Makro des \texttt{german.sty}
%    eingef\"ugt werden sollten. Die Definitionen der deutsch und
%    franz\"osischen Namen folgen im Anschlu\ss{} an die Liste als
%    Metakommentar.  
%
%    \begin{macrocode}
\newcommand\contentsname{Contents}
\newcommand\listfigurename{List of Figures}
\newcommand\listtablename{List of Tables}
\newcommand\refname{References}
\newcommand\indexname{Index}
\newcommand\figurename{Figure}
\newcommand\tablename{Table}
\newcommand\partname{Part}
\newcommand\appendixname{Appendix}
\newcommand\abstractname{Abstract}
\newcommand\figureshortname{Fig.}                      % <-- paper
\newcommand\tableshortname{Tab.}                       % <-- paper
\newcommand\keywordname{Keywords}                      % <-- paper
\newcommand\draftname{preliminary draft}               % <-- paper
%    \end{macrocode}
%\iffalse
% in: \def\captionsenglish{% 
% \def\figureshortname{Fig.}%
% \def\tableshortname{Tab.}%...}
% \def\keywordname{Keywords}
% in: \def\captionsfrench{% 
% \def\figureshortname{Fig.}%
% \def\tableshortname{Tab.}}%
% \def\keywordname{Motcl\'es}
% in: \def\captionsgerman{%  
% \def\figureshortname{Abb.}
% \def\tableshortname{Tab.}
% \def\keywordname{Deskriptoren}           
% \fi
%
%    Definition des Datums und verschiedene Initialisierungen. 
%    Im Unterschied zum Standardstil \texttt{article} ist die
%    Benutzung von Kopfzeilen voreingestellt.
%
%    \begin{macrocode}
\newcommand\today{\ifcase\month\or
  January\or February\or March\or April\or May\or June\or
  July\or August\or September\or October\or November\or December\fi
  \space\number\day, \number\year}
\setlength\columnsep{10\p@}
\setlength\columnseprule{0\p@}
\iffinal
  \if@journal
    \ps@journal
  \else
    \ps@headings
  \fi
\else
  \ps@draft
\fi
\pagenumbering{arabic}
\if@twoside
\else
  \raggedbottom
\fi
\if@twocolumn
  \twocolumn
  \sloppy
  \flushbottom
\else
  \onecolumn
\fi
%    \end{macrocode}
%    \begin{macrocode}
%</paper>
%    \end{macrocode}
%    \section{Treiber-Datei}
%
%    Der letzte Abschnitt enth\"alt die Treiberdatei zur Erstellung der
%    Dokumentation.
%    \begin{macrocode}
%<*driver>
\typeout{*******************************************************}
\typeout{* Documentation for LaTeX styles `paper' & `journal'  *}
\typeout{*******************************************************}

\documentclass[11pt]{ltxdoc}
\usepackage{german}

\makeatletter
\newif\ifsolodoc
 \@ifundefined{solo}{\solodoctrue}{\solodocfalse}
\IndexPrologue{\section*{Index}%
               \markboth{Index}{Index}%
               Die kursiv gesetzten Seitenzahlen
               verweisen auf Beschreibungen der Makros,
               unterstrichene Programmzeilennummern
               auf deren Definitionen.}
\GlossaryPrologue{\section*{Neuerungen}%
                 \markboth{Neuerungen}{Neuerungen}}
\def\saved@macroname{Neuerung}
\renewenvironment{theglossary}{%
    \glossary@prologue%
    \GlossaryParms \let\item\@idxitem \ignorespaces}%
   {}
\makeatother
\setcounter{StandardModuleDepth}{1}
%   \OnlyDescription
%   \CodelineIndex
\CodelineNumbered 
\RecordChanges
\setlength{\parindent}{0pt}
\begin{document}
\DocInput{paper.dtx} \newpage \PrintChanges % \newpage \PrintIndex
\end{document}
\endinput
%</driver>
%    \end{macrocode}
%% \CharacterTable
%%  {Upper-case    \A\B\C\D\E\F\G\H\I\J\K\L\M\N\O\P\Q\R\S\T\U\V\W\X\Y\Z
%%   Lower-case    \a\b\c\d\e\f\g\h\i\j\k\l\m\n\o\p\q\r\s\t\u\v\w\x\y\z
%%   Digits        \0\1\2\3\4\5\6\7\8\9
%%   Exclamation   \!     Double quote  \"     Hash (number) \#
%%   Dollar        \$     Percent       \%     Ampersand     \&
%%   Acute accent  \'     Left paren    \(     Right paren   \)
%%   Asterisk      \*     Plus          \+     Comma         \,
%%   Minus         \-     Point         \.     Solidus       \/
%%   Colon         \:     Semicolon     \;     Less than     \<
%%   Equals        \=     Greater than  \>     Question mark \?
%%   Commercial at \@     Left bracket  \[     Backslash     \\
%%   Right bracket \]     Circumflex    \^     Underscore    \_
%%   Grave accent  \`     Left brace    \{     Vertical bar  \|
%%   Right brace   \}     Tilde         \~}
%%
% \Finale
% \endinput
# Local Variables:
# mode: latex
# End:

