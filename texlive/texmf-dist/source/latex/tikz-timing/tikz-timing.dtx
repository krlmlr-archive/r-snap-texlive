% \iffalse
%<*copyright>
%
% Copyright (C) 2009 by Martin Scharrer <martin@scharrer-online.de>
% -----------------------------------------------------------------
%
% This work may be distributed and/or modified under the
% conditions of the LaTeX Project Public License, either version 1.3c
% of this license or (at your option) any later version.
% The latest version of this license is in
%
%   http://www.latex-project.org/lppl.txt
%
% and version 1.3c or later is part of all distributions of LaTeX
% version 2008/05/04 or later.
%
% This work has the LPPL maintenance status `maintained'.
%
% The Current Maintainer of this work is Martin Scharrer.
%
% This work consists of the files tikz-timing.dtx, tikz-timing.ins
% and the derived file tikz-timing.sty.
%
% $Id: tikz-timing.dtx 2038 2011-01-09 20:44:26Z martin $
%</copyright>
% \fi
%
% \iffalse
%<*driver>
\makeatletter
%</driver>
\RequirePackage{svn-prov}[2009/05/03]
%<package>\ProvidesPackageSVN
%<examplecode>\ProvidesPackageSVN[tikz-timing-examplecode]
%<ydoc>\ProvidesPackageSVN[ydoc]
%<lib-arrows>\ProvidesPackageSVN[tikz-timing-arrows]
%<lib-either>\ProvidesPackageSVN[tikz-timing-either]
%<lib-overlays>\ProvidesPackageSVN[tikz-timing-overlays]
%<lib-clockarrows>\ProvidesPackageSVN[tikz-timing-clockarrows]
%<lib-columntype>\ProvidesPackageSVN[tikz-timing-columntype]
%<lib-nicetabs>\ProvidesPackageSVN[tikz-timing-nicetabs]
%<lib-counters>\ProvidesPackageSVN[tikz-timing-counters]
%<lib-advnodes>\ProvidesPackageSVN[tikz-timing-advnodes]
%<lib-interval>\ProvidesPackageSVN[tikz-timing-interval]
%<lib-beamer>\ProvidesPackageSVN[tikz-timing-beamer]
%<lib-ifsym>\ProvidesPackageSVN[tikz-timing-ifsym]
%<*driver>
\ProvidesFileSVN
%</driver>
  {$Id: tikz-timing.dtx 2038 2011-01-09 20:44:26Z martin $}
  [v0.7d
%<package>   Digital Timing Diagrams using TikZ]
%<examplecode>   Macros to produce examples for the tikz-timing package]
%<ydoc>   Documentation Macros for tikz-timing package]
%<lib>   TikZ-Timing Library for
%<lib-arrows|      >   timing arrows]
%<lib-either|      >   uncertain transitions]
%<lib-overlays|    >   timing overlays]
%<lib-clockarrows| >   timing clock arrows]
%<lib-columntype|  >   timing cell types]
%<lib-nicetabs|    >   nice timing tables]
%<lib-counters|    >   counter values]
%<lib-advnodes|    >   advanced nodes]
%<lib-interval|    >   alternative Z transitions]
%<lib-beamer|      >   beamer overlay support]
%<lib-ifsym|       >   ifsym compatible macros]
%<*driver>
  DTX-File of 'tikz-timing' package.]
%</driver>
%<lib>\GetFileInfoSVN*%
%<lib>\RequirePackage{tikz-timing}[\filedate]%
%<*package|driver>
\DefineFileInfoSVN[tikztiming]
%</package|driver>
%<!package>\DefineFileInfoSVN
%<*driver>
\newif\ifprintversion % \fi
\ifx\printversion\@undefined
\else
  \printversiontrue
\fi

\ifprintversion
  \documentclass{ltxdoc}
  \let\chapter\section
  \let\section\subsection
  \let\subsection\subsubsection
\else
  \documentclass[landscape,12pt]{scrreprt}
  \usepackage{ydoc}
\fi

\usepackage{hyperref}
\usepackage{atbegshi}
\GetFileInfoSVN{tikz-timing.dtx}
\usepackage{tikz-timing}[\tikztiming@date]
\usetikztiminglibrary{arrows,either,overlays,clockarrows,columntype}
\usetikztiminglibrary{nicetabs,counters,advnodes,interval,ifsym}
\tikzset{timing/no nice tabs}
\tikzset{timing/interval/normal}
\tikzset{timing/nodes/old center}
\usetikzlibrary{plotmarks}
\makeatother
%%%^^A\usepackage[electronic]{ifsym}
\usepackage{calc}
\usepackage{graphicx}
\usepackage{xcolor}
\usepackage{tabularx}
\usepackage{array}
\usepackage{flafter,fnpos}
\usepackage{booktabs}
\usepackage{supertabular}
%\usepackage{xtab}
\usepackage{amsmath}
\usepackage{flafter}
\usepackage{placeins}
\makeFNbottom
\makeFNbelow
\usepackage{microtype}[2005/10/28]
\DisableLigatures{encoding = T1, family = tt* }%
\newsavebox{\mysb}
\newsavebox{\sba}
\newsavebox{\sbb}

\def\topfraction{0.91}
\def\bottomfraction{0.95}
\def\floatpagefraction{0.98}
\def\textfraction{0.05}
\setcounter{bottomnumber}{2}

\lstset{basicstyle=\ttfamily}

\usepackage{shortvrb}
\usepackage{newverbs}
\MakeSpecialShortVerb{\qverb}{\"}

\def\NewIn#1{
  \marginpar{\strut\raggedleft\raisebox{.125ex}{\textcolor{black}{\emph{New in #1}}}}%
}

\def\ExtIn#1{
  \marginpar{\strut\raggedleft\textcolor{black}{\emph{Extended in #1}}}%
}

\iffalse
\def\thesection{\texorpdfstring{\LARGE\texttiming[X]{D{\arabic{section}}.1X[black]}}{\arabic{section}}}
\def\thesubsection{\texorpdfstring{\Large\texttiming[X]{D{\arabic{section}}D{\arabic{subsection}}.1X[black]}}{\arabic{section}.\arabic{subsection}}}
\def\thesubsubsection{\texorpdfstring{\large\texttiming[X]{D{\arabic{section}}D{\arabic{subsection}}D{\arabic{subsubsection}}.1X[black]}}
{\arabic{section}.\arabic{subsection}.\arabic{subsubsection}}}

 %\def\timingsectionformat#1{%
 %  \sbox{\mysb}{\tikz \draw node [timing/d/text] {#1};}%
 %  \texttiming{ $ \wd\mysb / \tikztiming@xunit $ DD{#1} }%
 %}

\renewcommand\section{\@startsection {section}{1}{\z@}%
                                   {-3.5ex \@plus -1ex \@minus -.2ex}%
                                   {2.3ex \@plus.2ex}%
                                   {\normalfont\Large\bfseries\sffamily}}
                                   %[#1]{\texorpdfstring{\larger\timingsectionformat{#2}}{#2}}}
\renewcommand\subsection{\@startsection{subsection}{2}{\z@}%
                                     {-3.25ex\@plus -1ex \@minus -.2ex}%
                                     {1.5ex \@plus .2ex}%
                                     {\normalfont\large\bfseries\sffamily}}
\renewcommand\subsubsection{\@startsection{subsubsection}{3}{\z@}%
                                     {-3.25ex\@plus -1ex \@minus -.2ex}%
                                     {1.5ex \@plus .2ex}%
                                     {\normalfont\normalsize\bfseries\sffamily}}
\fi

\let\envv\env
\let\css\cs
\def\lib{\texttt}
\def\library{\section}

\usepackage{xspace}
\def\ie{i.e.\xspace}
\def\eg{e.g.\xspace}

\makeatletter
% The following code is only needed to produce package examples and therefor not 
% included in the style file but might be written to an additional file.
% \iffalse
%</driver>
%<*examplecode>
% \fi

% \begin{environment}{tikztimingexampletable}
%    \begin{macrocode}
\newenvironment{tikztimingexampletable}{%
  \begingroup
  \let\tikztimingtable@row\tikztimingexampletable@row
  \tikzset{timing/nice tabs}
  \tikztimingtable
}{%
  \extracode
    \tableheader{Characters}{Resulting Diagram}%
    \tablerules
\endtikztimingtable
\endgroup
}%
%    \end{macrocode}
% \end{environment}
%
% \begin{macro}{\tikztimingexampletable@row}[1]{Row content}
%    \begin{macrocode}
\long\def\tikztimingexampletable@row#1\\{%
  \def\tikztiming@text{#1}%
  \@onelevel@sanitize\tikztiming@text
  \tikztimingtable@row@@{\ttfamily\tikztiming@text}{#1}%
}
%    \end{macrocode}
% \end{macro}
%
% \begin{macro}{\tikztimingfullexampletable}
%    \begin{macrocode}
\def\tikztimingfullexampletable{%^^A
  \begin{tikzpicture}[timing/picture,x=1em,y=1em,font=\sffamily]
    \tikzset{timing/d/background/.style={fill={gray!25},fill opacity=0.5}}%
    \let\chars\tikztiming@chars@default
    \node (charnode) at (0,0) {%
       \scalebox{0.4}%
         {\rotatebox{-45}{$\frac{\mbox{\rotatebox{45}{to}}}%
         {\mbox{\rotatebox{45}{from}}}$}}%
    };
    \coordinate (charnodex) at (0.25,0);%
    \coordinate (charnodey) at (0,0);%
    \expandafter\foreach
    \expandafter\tchar
    \expandafter i\expandafter n\expandafter{\chars} {%
      \path (charnodex) ++(+2,0) node (charnodex) {\strut\tchar};
      \path (charnodey) ++(0,-2) node (charnodey) {\strut\tchar};
    }%
    \draw [line width=\heavyrulewidth] (charnodex) +(+1,+1) -- (-1,+1);
    \draw [line width=\lightrulewidth]  (charnodex) +(+1,-1) -- (-1,-1);
    \draw [line width=\lightrulewidth]  (charnodey) +(+1,-0.6) -- (+1,-1.4) 
    (+1,-0.6) -- (+1,+0.6);
    \draw [line width=\heavyrulewidth] (charnodey) ++(-1,-1) -- +($ (2,0) + 
    (charnodex) $);
    %
    \path (1.5,-2) node (charnodex) {\strut};
    \coordinate (charnodex) at (charnodex.base);
    \coordinate (charnodey) at ($ (charnodex.base) + (0,2) $);
    \def\@tempa{\timing at (charnodey)}
    \foreach\xchar in \chars {
      \foreach\ychar in \chars {
        \path (charnodey) +(0,-2) node (charnodey) {};
        \draw [xstep={\timingwidth/2.},ystep={\timingheight/2.},timing/grid,
          shift={(charnodey)}]
          (0,0) grid +(2,1);
        \expandafter\@tempa\expandafter{\ychar\xchar};
      }
      \path (charnodex) ++(+2,0) node (charnodex) {};
      \path (charnodex) ++(0,2)  node (charnodey) {};
    }
  \end{tikzpicture}%^^A
}
%    \end{macrocode}
% \end{macro}
%
% \iffalse
%</examplecode>
%<*driver>
% \fi

\makeatother
\EnableCrossrefs
%\CodelineIndex
\RecordChanges
\OnlyDescription
\listfiles

\DeleteShortVerb{\|}
\def\pipe{|}
\def\Pipe{{\normalfont\normalcolor\pipe}}
\MakeShortVerb{\|}

\begin{document}
  \DocInput{tikz-timing.dtx}
  \PrintChanges
  %\newpage\PrintIndex
\end{document}
%</driver>
%<*doc>
% \fi
%
% \CheckSum{5720}
%
% \CharacterTable
%  {Upper-case    \A\B\C\D\E\F\G\H\I\J\K\L\M\N\O\P\Q\R\S\T\U\V\W\X\Y\Z
%   Lower-case    \a\b\c\d\e\f\g\h\i\j\k\l\m\n\o\p\q\r\s\t\u\v\w\x\y\z
%   Digits        \0\1\2\3\4\5\6\7\8\9
%   Exclamation   \!     Double quote  \"     Hash (number) \#
%   Dollar        \$     Percent       \%     Ampersand     \&
%   Acute accent  \'     Left paren    \(     Right paren   \)
%   Asterisk      \*     Plus          \+     Comma         \,
%   Minus         \-     Point         \.     Solidus       \/
%   Colon         \:     Semicolon     \;     Less than     \<
%   Equals        \=     Greater than  \>     Question mark \?
%   Commercial at \@     Left bracket  \[     Backslash     \\
%   Right bracket \]     Circumflex    \^     Underscore    \_
%   Grave accent  \`     Left brace    \{     Vertical bar  \|
%   Right brace   \}     Tilde         \~}
%
%
%
% \GetFileInfo{tikz-timing.dtx}
%
% \DoNotIndex{\newcommand,\newenvironment,\def,\edef,\xdef,\DeclareRobustCommand}
% \DoNotIndex{\expandafter,\if,\else,\fi,\ifnum,\ifx,\let,\global,\long}
% \DoNotIndex{\newcounter,\newcount,\message,\meaning,\noexpand,\relax,\value}
% \DoNotIndex{\setcounter,\addtocounter,\advance,\afterassignment,\AtEndOfPackage}
% \DoNotIndex{\ProvidesPackage,\providecommand,\RequirePackage,\empty,\begin,\end}
% \DoNotIndex{\begingroup,\bgroup,\egroup,\endgroup,\csname,\endcsname,\@tempa}
% \DoNotIndex{\ignorespaces,\lccode,\sffamily,\@gobble,\@ifundefined,\@for,\or}
% \DoNotIndex{\@firstoftwo,\@ifnextchar,\@namedef,\@nameuse,\@secondoftwo}
% \DoNotIndex{\@temptokena,\toks@,\BODY,\do,\g@addto@macro,\lowercase,\uppercase,\the}
% \DoNotIndex{\height,\width,\slope,\style,\draw,\path,\newdraw,\newdrawns}
% \DoNotIndex{\dslope,\zslope}
%
% \ifpdf
% \hypersetup{%
%   pdfauthor   = {Martin Scharrer <martin@scharrer-online.de>},
%   pdftitle    = {The tikz-timing package, \fileversion\ from \filedate},
%   pdfsubject  = {Documentation of LaTeX package tikz-timing which allows the 
%   easy creation of timing diagrams inside tikz pictures or text.},
%   pdfkeywords = {tikz-timing, timing diagram, LaTeX}
% }%
% \fi
%
% \unless\ifprintversion
% \begingroup
% \makeatletter
% \title{{\tikz[baseline=(N.base),timing/picture,timing/z/.style={},%
%   timing/unit=1.44*\f@size pt,timing/d/text/.style={},timing/d/background/.style={fill=black!10},timing/inline node/.append style={rectangle,inner sep=0pt,outer sep=0pt}]
%   \timing at (0,0) {1L N[shift={(0,0.5)}]{\strut The} .9L 3d{[alias=N]\strut tikz} $.5em/\xunit$Z 5d{\strut timing} 1.75H N[shift={(0,-0.5)}]{Package} 1.75H } ;}}
% \author{Martin Scharrer}
% \date{Version \fileversion\ -- \filedate}
% \thispagestyle{empty}
% \begin{titlepage}
%   \sffamily\bfseries
%   \parindent=0pt
%   \centering
%   \vspace*{2cm}
%   \resizebox{.75\textwidth}{!}{\Huge\@title}
%   \par\vspace*{.75cm}
%   \Large A \LaTeX\ Package for Timing Diagrams
%   \par\vspace*{1.5cm}
%   {\LARGE\@date}
%   \par\vspace*{2cm}
%   {\LARGE\@author
%   \par\smallskip}
%   {\Large\href{mailto:martin@scharrer-online.de}{martin@scharrer-online.de}
%   \par\vspace*{1.25cm}}
%   {\Large
%   WWW: \url{http://latex.scharrer-online.de/tikz-timing}\\[.5ex]
%   CTAN: \url{http://www.ctan.org/pkg/tikz-timing}\\[1.2ex]
%   \par\smallskip}
% \end{titlepage}
% \clearpage
% \endgroup
%
% \else
%
% \null
% \vspace*{-2em}
% \begin{center}
%   \sffamily
%   \tikzset{timing/z/.style={black},
%   timing/intext/.style={timing,line width=0.1ex}}
%   {\LARGE\sffamily The \raisebox{-0.66ex}{\Huge\textsf
%   {\texttiming[z]{3d{\strut tikz}z5d{\strut timing}0.2z}}} package}\\[3ex]
%   {\large Version \fileversion\ -- \filedate}\\[2ex]
%   {\large Martin Scharrer \\\normalsize 
%   \url{martin@scharrer-online.de}\\[.8ex]
%   WWW: \url{http://latex.scharrer-online.de/tikz-timing}\\[.5ex]
%   CTAN: \url{http://www.ctan.org/pkg/tikz-timing}\\[1.2ex]}
% \end{center}
% \vspace{1.2em}%
% \fi
%
% ^^A\vfill \noindent
% ^^A\textbf{Note to Advanced Users:} This is a new package which internal
% ^^Amacros might still change.  Please only rely on the user macros for now.
%
% \clearpage
% \enlargethispage{\baselineskip}
% {\let\csd
%   \texttt\tableofcontents
%   \listof{example}{List of Examples}
% }
% \addtocontents{toc}{\vspace*{-1.5em}}
%
% \chapter{Introduction}
% This package uses the \pkg[pgf]{tikz} package to produce timing diagrams 
% inside text or \envv{tikzpicture} environments.  Also a tabular-like 
% environment is provided to produce a larger timing diagram with multiple 
% labeled signals and the possibility to add own drawing material.
% Additional optional functionality is provided by libraries.
%
% The signal levels of the timing diagram can be given by corresponding 
% characters/letters like "H" for \emph{Logical High} or "L" for 
% \emph{Logical Low}. So e.g.\ "{HLZXD}" gives `\texttiming{HLZXD}'.
% In order to fit (in)to normal text the diagram size (\ie its height, width and line width)
% is defined relatively to the currently active font size. The diagram height is about the height of an
% uppercase "X" (+$2\times\frac12$ line width). This way the diagrams
% can also be scaled using font size commands like \cs{small}.
% (Example: {\sffamily
%  \large        X\texttiming{LH}X
%  \normalsize   X\texttiming{LH}X
%  \small        X\texttiming{LH}X
%  \footnotesize X\texttiming{LH}X
%  \scriptsize   X\texttiming{LH}X
%  \tiny         X\texttiming{LH}X
% })
% \makeatletter
% \def\iex#1{^^A
%  \def\@tempa{#1}\@onelevel@sanitize\@tempa
%  `\texttt{\@tempa}'~$\rightarrow$~`\texttiming{#1}'}
% \makeatother
% A single timing character produces a diagram with a width identical to its height
% (\iex{H}). Longer diagrams can be produces by either 
% using the same character multiple times (\iex{HHH})
% or writing the width as number in front of the character (\iex{3.2H}).
% For (partial) compatibility with similar packages lowercase characters only produce
% a signal with half the width (\iex{h}, \iex{3.2h}).
%
% Recurring character combinations can be repeated using character groups
% (\iex{3{hlz}}) or be defined as so called \emph{meta-characters}
% ({\tikztimingmetachar{Y}{hlz} "Y"="hlz", \iex{3Y}}),
% which are similar to a \TeX~macro.
% Since v0.7 meta-characters can also include macro arguments.
%
% Additional features are the inclusion of in-line TikZ styles (\iex{H ;[orange] L}),
% in-line nodes (\iex{2H N[rectangle](Name){Text} 2L}), automatic width calculations
% ("H $ 1+\mymacro-\slope $ L "~$\rightarrow$~`\texttiming{H2.5L}')
% and much more.
%
% \iffalse
% \medskip
% \DeleteShortVerb{\|}
% \centerline{\texttiming[D]{
%   2{12{[yscale=.9,shift={(0,.05)}]D{}}
%   13{[yscale=1.1,shift={(0,-.05)}]D{}};}
%   $\dslope$ D{}
% }}
% \MakeShortVerb{\|}
% \medskip
% \fi
%
% \iffalse
% The package is build to make it possible to define new characters from scratch 
% or as modified copy of other characters. However, no user macros nor 
% documentation for this are provided at the moment. Interested \LaTeX\ users 
% should look at the default definitions at the end of the source code.
% \fi
%
% \iffalse
% \section{Similar Packages}
% There a some packages which target the same application like the package 
% presented by this document.
%
% \begin{description}
%   \itempkg{ifsym} This package (using the |electronic| option) provides a 
%   special font which contains graphical representation of the logical levels 
%   high and low at the corresponding letter "H" and "L".  The lower case 
%   versions have only half the width of the uppercase ones. Also a transition 
%   can be added using the "|" character which will (sometimes) be added 
%   automatically between |HL| and |LH|.
%   The diagrams are created using the command |\textifsym|\marg{characters}, 
%   e.g.\ \verb+\textifsym{H|L|h|l|H|L}+ results in
%   \DeleteShortVerb{\|} `\textifsym{H|L|h|l|H|L}'.
%   \MakeShortVerb{\|}
%
%   There is no support for transition slopes and no support for new 
%   user-defined logical levels.
%
%   \itempkg[(CTAN)]{timing} This package also provides a font for the logical 
%   levels but supports transition slopes and larger timing diagrams.  This 
%   package seems not been updated for a while.
%
%   \itempkgnoctan[(TikZ Example Page)]{timing}%
%   {http://www.texample.net/tikz/examples/timing-diagram/} This package is 
%   accidentally also called "timing.sty" and is not published on CTAN but on 
%   the TikZ example website.  It is a small package which is meant as an 
%   example for the graphics package \texttt{tikz} which is used to draw the 
%   diagram.  The logical levels must be provided using macros like 
%   `|\bit|\marg{0 or 1}\marg{length}'.
% \end{description}
% \fi
%
% \clearpage
% \section{Changelog}
% \begingroup
% \renewcommand{\changes}[3]{\paragraph{#1 from #2}\begin{itemize}\item #3\end{itemize}}
% \newenvironment{Changes}[2]{\paragraph{#1 from #2}\begin{itemize}\let\change\item}{\end{itemize}}
%
% \changes{v0.3}{2009/04/24}{First released version}
% \changes{v0.4}{2009/05/03}{Added output routine which combines successive 
% occurrences of the same character. This improves screen display quality and 
% reduces rendering time and file size.
% \item Removed own macros for lowercase characters. They 
% are now handled by the uppercase macros which receive half of the width. 
% Exceptions are possible like for the `m' character.
% \item Added parser for rows in \env{tikztimingtable}. 
% This makes the syntax much more stable. Also replaced row counter with TikZ 
% coordinates which is more user-friendly.
% \item User macros to draw grids and lines inside table.
% \item In-line Nodes, \eg to mark positions inside the diagram.}
% \changes{v0.4a}{2009/05/05}{Added \cs{tablerules} macro. Changed default style 
% of inline nodes to \texttt{coordinate}.}
% \changes{v0.5}{2009/05/15}{Added PGF shape for timing diagrams. Added 
% meta-characters. Changed `M' character to use PGF decorations. Added special 
% `B' character to reduce width of next character. Changed \cs{timing} syntax to 
% include an `at' before the coordinate. Bug fix for use with the `calc' 
% package.}
% \changes{v0.6}{2009/07/27}{Added ``forward'' modifier `\texttt{F}' as reverse 
% version of the ``backward'' modifier `\texttt{B}'.
% \item Added support for 
% lower-case modifiers ``\texttt{b}', `\texttt{f}' and \texttt{n}'.
% \item Added libaries for characters `\texttt{A}'/`\texttt{W}' for arrows and 
% '\texttt{E}' for uncertain low-to-high and high-to-low transitions.}
% \changes{v0.6a}{2009/07/28}{Added library for overlay modifier `\texttt{O}'.}
%
% \begin{Changes}{v0.7}{2009/12/05}
%   \change New libraries:
%     \begin{description}
%       \item[\lib{clockarrows}]  Library for clock arrows.
%       \item[\lib{columntype}]   Library providing a timing column type for |tabular|.
%       \item[\lib{nicetabs}]     Library to format \cs{tikztimingtable} like a |booktab| |tabular|.
%       \item[\lib{counters}]     Library to defined counter characters which display an incrementing 
%                             counter value every time there are used.
%       \item[\lib{advnodes}]     Library for advanced nodes with multiple anchor points.
%       \item[\lib{ifsym}]        Library providing the same timing symbols and characters as the |ifsym| package
%                             when loaded with the |electronic| option.
%     \end{description}
%   \change Additional experimental libraries:
%     \begin{description}
%       \item[\lib{interval}]     Library to change color of "ZL", "ZH" etc. transitions to 
%                             indicate borders of an interval.
%       \item[\lib{beamer}]       Library providing some marginal beamer overlay support.
%     \end{description}
%   \change \lib{overlays} library:
%     \begin{itemize}
%       \item Overlays can now be cascaded, \ie an overlay can be inside another one.
%       \item The second braces around the second part are now optional.
%       \item Fixed issues with "T" and "C" characters inside overlays.
%     \end{itemize}
%   \change Meta-characters can now have arguments.
%   \change Added more variety for in-line options: "[[ ]]", "[+ +]" and "[| |]".
%   \change Handling of in-line options and nodes got modified. Options are now placed directly 
%     where placed and are valid until the next ";". Please note that |[/utils/exec={..}]| now 
%     needs to be written as \verb+[|/utils/exec={..}|]+. Otherwise it is re-executed every time
%     the drawing path is renewed.
%   \change Added star version of \cs{tablegrid}.
%   \change Added background to "E" character (\lib{either} library).
%   \change Some fixes for placing of "D{}" texts.
%   \change Fixed wrong slopes (\eg |lslope| instead of |zslope|) for some transitions.
%   \change Major changes on internal character definition macros, parser and output routine.
%   \change Fixed problems with expanding code content in user input.
%   \change The \cs{texttiming} macro now uses a \cs{timing} macro internally.
%   \change The \cs{timing} macro is now only defined inside |tikzpictures|.
%           This includes |tikztimingtable|.
%   \change Added TikZ style |timing/draw grid| for grids behind \cs{timing} macros.
%   \change Replaced macros \cs{texttimingbefore}, \cs{texttimingafter} and \cs{texttiminggrid} with 
%           TikZ settings "timing/before text", "timing/after text" and "timing/draw grid".
%   \change Added separators "timing/outer sep" around \cs{texttiming}.
%   \change Graphical improvements for `double line' characters like "D", "U" and "E".
%     The whole character including both edges is drawn in a single process.
%   \change Character width can now be scaled using |wscale|.
%   \change Character width can now be calculated by placing code inside "$ $".
%   \change Fixed issue with \cs{horlines} macro.
%   \change The \env{tikztimingtable} environment and associated macros got enhanced:
%     \begin{itemize}
%       \item The content is no longer read as macro argument and can now include paragraphs.
%       \item Multiple |extracode| sections can be now included between rows, not only a single 
%             section at the very end.
%       \item A \env{extracode} environment has been added. Both macro and environment have now 
%             an optional argument for TikZ settings.
%       \item Added \cs{tableheader} macro to label both columns. The \cs{tablerules} macro got
%             adjusted to detect the header line and draw also a middle line.
%       \item Added \env{background} environment to draw things in the background.
%       \item Fixed broken optional argument of \cs{tablegrid}.
%       \item Added macro \cs{marknodes} and associated |debug/nodes| style to mark in-line nodes
%             for debug purposes/orientation during the diagram creation.
%     \end{itemize}
% \end{Changes}
% \changes{v0.7d}{2011/01/09}{Fix for end macro of extracode environment to support etoolbox's environment hooks.}
%
%
% \endgroup
%
% \section{Dependencies}
% \dots
%
% \clearpage
% \chapter{Usage}
%
% \section{Timing Characters}
% The logic levels are described by so called \emph{timing characters}.  Actually
% all of them are letters, but the general term \emph{character} is used here.
% Table~\ref{tab:chars} shows all by default defined logic characters and
% Table~\ref{tab:full} all possible two-character transitions. Additional
% functionality is provided by the \emph{modifiers} shown in
% Table~\ref{tab:modifiers}.
%
% \sbox{\sba}{%^^A
% \sffamily
% \tikzset{timing/draw grid}
% \begin{tabular}{clccc}
%  \toprule
%   Character   & Description            & Diagram                           & Transition            \\
%               &                        &                                   & Example               \\
%  \midrule
%    \texttt{H} & High                   & \texttiming{H}                    & \texttiming[L]{H}     \\
%    \texttt{L} & Low                    & \texttiming{L}                    & \texttiming[H]{L}     \\
%    \texttt{Z} & High Impedance         & \texttiming{Z}                    & \texttiming[L]{Z}     \\
%    \texttt{X} & Undefined / Don't Care & \texttiming{X}                    & \texttiming[L]{X}     \\
%    \texttt{D} & Data / Double          & \texttiming{D}                    & \texttiming[L]{D{A}D} \\
%    \texttt{U} & Unknown Data           & \texttiming{U}                    & \texttiming[D]{U}     \\
%    \texttt{T} & Toggle                 & \texttiming{L} or \texttiming{H}  & \texttiming{TTTT}     \\
%    \texttt{C} & Clock (no slope)       & \texttiming{L} or \texttiming{H}  & \texttiming{CCCC}     \\
%    \texttt{M} & Metastable Condition   & \texttiming{M}                    & \texttiming[H]{Ml}    \\
%   \midrule
%    \texttt{G} & Glitch (zero width)    & \texttiming{G}                    & \texttiming{HGH}      \\
%    \texttt{S} & Space (nothing)        & \texttiming{S}                    & \texttiming{HSL}      \\
%   \bottomrule
% \end{tabular}
% }%^^A
%
% \sbox{\sbb}{%^^A
%   \tikzset{timing/draw grid}
%   \sffamily\small
%   \tikztimingfullexampletable
% }%^^A
%
% \ifprintversion
%
% \begin{table}
%   \centering
%   \caption{Timing Characters}\label{tab:chars}
%   \usebox{\sba}
% \end{table}
%
% \begin{table}
%   \caption{Overview over all transitions.}
%   \label{tab:full}
%   \usebox{\sbb}
% \end{table}
%
% \else
%
% \begin{table}
%   \sffamily
%   \centering
%   \pgfmathparse{max(\wd\sba+\wd\sbb+1em,\textwidth)}%^^A
%   \makebox[\textwidth][c]{%^^A
%   \begin{minipage}{\wd\sba}
%     \caption{Timing Characters}
%     \label{tab:chars}
%     \usebox{\sba}
%   \end{minipage}%^^A
%   \hspace{1em}\hfill
%   \begin{minipage}{\wd\sbb}
%     \caption{Overview over all transitions.}
%     \label{tab:full}
%     \usebox{\sbb}
%   \end{minipage}%^^A
%   }%^^A
% \end{table}
%
% \fi
%
% \clearpage
% \DeleteShortVerb{\|}
% \begingroup^^A{table}[b]
%   \tikzset{timing/draw grid}
%   \let\normalfont\sffamily
%   \sffamily\centering
%   \makeatletter
%
%   \tablecaption{Modifiers for Timing Characters.}
%   \label{tab:modifiers}
%   \def\en{\,\textsf{--}\,}
%   \let\origtabular\tabular
%   \let\endorigtabular\endtabular
%   \def\tabular{\begingroup
%     \setbox\mysb\hbox\bgroup
%     \origtabular}%
%   \def\endtabular{\endorigtabular\egroup\makebox[\textwidth][c]{\usebox\mysb}
%   \endgroup}%
%   \tablefirsthead{%
%     \toprule
%     \normalfont
%       Modifier Syntax  & Description \\
%     \midrule
%   }%
%   \tablehead{%
%     \multicolumn{2}{c}%
%     {\raisebox{\belowcaptionskip}{\tablename\ \thetable{} --
%     continued from previous page}} \\
%     \toprule
%     \normalfont
%       Modifier Syntax  & Description \\
%     \midrule
%   }%
%   \tabletail{%
%     \bottomrule
%     ^^A\multicolumn{2}{r}{Continued on next page}\\
%   }
%   \tablelasttail{%
%     \bottomrule
%   }
% \def\Example#1{\def\@tempa{#1}\@onelevel@sanitize\@tempa
% \hbox{\emph{E.g.:} `\hbox{\texttt{\@tempa}}' $\to$ \texttiming{#1}}}
%   \begin{supertabular}{>{\ttfamily}l>{\raggedright\arraybackslash}p{.85\textwidth}}
% D\{\}D & Produces transition between two data values. \Example{D{}D} \\
% D\{\meta{Text}\} & Adds \meta{text} into a data signal using a node.  
% \Example{D{A}D{B}} \\
% D\{[\meta{\scriptsize 
% \raisebox{-1ex}{\shortstack{TikZ\\Settings}}}]\meta{Text}\} & Adds \meta{text} 
% using the given node \meta{settings}. \Example{D{[blue]A}} \\
% \meta{\small number}\meta{\small character} & Sets width of next signal to 
% given number.  Half of it if character is in lower case.
% \Example{2.6H5.2l}\\
% \meta{\small integer}\{\meta{\small characters}\} &  Repeats the given 
% characters \texttt{\meta{int}} times.  \Example{5{hl}}\\
% \{ \meta{\small characters} \} &  Encloses characters in a local scope. Options inside are only
% local to the scope. This also applies to the effect of `\texttt{;}' and similar modifiers.
% \Example{H {[blue] LH} L}\\
% \midrule
% \meta{\small number}B & Su\underline{b}tracts the given number from the width 
% of the next character. ``\textit{\texttt{B}ackwards}''
% \Example{H.5BL}\\
% \meta{\small number}F & Adds the given number to the width of the next 
% character.  ``\textit{\texttt{F}orwards}'' \Example{H.5FL}\\
% N\oarg{\tiny Settings}\parg{\tiny Name}\marg{\tiny Content} & Adds node at 
% current position.  All three arguments are optional.  \Example{H N(a1) L}\\
%   \midrule
% {}[\meta{TikZ Keys}]  & Executes given TikZ settings during the drawing 
% process. This settings will be re-executed when the internal drawing path is renewed which 
% can cause side-effects. \Example{H[blue]LH} \\
% {}[|\meta{TikZ Keys}|]  & Executes given TikZ settings during the drawing 
% process like \texttt{[ ]} but does not re-executes them.
% {\Example{D{.} [|/utils/exec={\def\m{...}}|] D{.} D{.}}} \\
% {}[!\meta{TikZ Keys}!]  & Executes given TikZ settings during the 
% \emph{parsing} process. Because this makes only sense for internal settings 
% the default path is `\texttt{/tikz/timing}', not `\texttt{/tikz}' like in all  
% other settings macros. \Example{H[!wscale=2.5!]LH} \\
% {}[[\meta{TikZ Keys}]]  & Executes given TikZ settings first during the 
% parsing process and again during the drawing process. This is for settings 
% which are needed for width calculations and again for the drawing code, \eg 
% the slope values. \Example{H[[timing/slope=.5]]L $\slope$H} \\
%   \midrule
% !\{\meta{code}\}  & Places given code into the internal \envv{tikzpicture}.
% See Example~\ref{exa:adv}. \\
% @\{\meta{code}\}  & Executes the given code immediately during the parsing 
% process. This can be used to change parsing parameters. To execute code during 
% the drawing process use \verb+[|/utils/exec=+\meta{code}\verb+|]+ instead. 
% \Example{L @{\setwscale{2}} H} \\
% \texttt{\$}\meta{math expression}\texttt{\$} & Takes a valid \texttt{pgfmath} 
% expression (See \pkg{pgf} manual), evaluates it and places the result back in 
% the input string so it can be used as width for the next character. The macros  
% \cs{slope}=\cs{lslope}, \cs{dslope}, \cs{zslope} and \cs{wscale} can be used 
% to access the corresponding values.  \Example{D{} $ \dslope $ D{} D} \\
%   \midrule
% ;  & Renews the internal drawing path which ends the scope of all options 
% given by \texttt{[\,]}. \Example{H;[blue]L;H} \\
% ,  & Same as `\texttt{;}', but timing specific options (atm.: slopes and line 
% width) are restored for the new path. \Example{[line width=1pt]L,H;L} \\
% \end{supertabular}
% \endgroup
% \MakeShortVerb{\|}
%
% \clearpage
% \section{Macro for use in Text Mode}
% \vspace{-\bigskipamount}%
% \DescribeMacro\texttiming[<initial~character/TikZ Settings>]{<characters>}
% This macro places a single timing diagram line into the current text. The 
% signals have the same height as a uppercase letter (like `X') of the current 
% font, \ie they scale with the font size.
% The macro argument must contain only valid logic characters and modifiers 
% which define the logical levels of the diagram line.
%
% An initial character can be given as an optional argument. No logic level will 
% be drawn for this character. Instead it will be used to define the initial 
% position of the signal so that the diagram line will start with a transition 
% from the initial to the first character. However, if the optional argument 
% holds more than a single character it is taken as TikZ settings for the 
% diagram. The initial character can then be given using the key 
% `|timing/initchar=|\meta{char}'.
%
% \par\medskip\noindent
% \begin{minipage}{\textwidth}\noindent
% \textit{Examples:}\\*[\smallskipamount]\hspace*{\parindent}%
% \begin{minipage}{\textwidth-4\parindent}
% \small |\texttiming{HLZDZLH}|^^A
% \phantom{\texttt{[L]}} gives `\texttiming{HLZDZLH}', with grid: 
% `{\let\texttimingbefore\texttiminggrid\texttiming{HLZDZLH}}'.\\
% |\texttiming[L]{HLZDZLH}| ^^A
% gives `\texttiming[L]{HLZDZLH}', with grid: 
% `{\let\texttimingbefore\texttiminggrid\texttiming[L]{HLZDZLH}}'.\\
% |\texttiming[green]{HLZDZLH}| ^^A
% gives `\texttiming[green]{HLZDZLH}'\\
% |\texttiming[green,timing/initchar=L]{HLZDZLH}| ^^A
% gives `\texttiming[green,timing/initchar=L]{HLZDZLH}'
% \end{minipage}
% \end{minipage}
%
% \DescribeMacro\texttimingbefore!{\hspace{2em}\textbf{Deprecated!}\hspace{8em}(\emph{defaults to}: \meta{empty})}!
% \DescribeMacro\texttimingafter!{\hspace{2.5em}\textbf{Deprecated!}\hspace{8em}(\emph{defaults to}: \meta{empty})}!
% This two macros are executed before and after every timing diagram line 
% created by \cs{texttiming} macro inside the same \envv{tikzpicture} 
% environment and can be used to add drawing macros. The argument of the 
% \cs{texttiming} macro is already processed before any of these macros are 
% expanded, therefore this macros can access the width of the diagram.
%
% These macros should not be used directly in newer code but instead the new 
% TikZ styles "timing/before text" and "timing/after text". For backward 
% compatibility these styles default to the two macros.
%
% {\small
% (Deprecated) Example: |\let\texttimingbefore\texttiminggrid| adds a grid into the 
% background of the \cs{texttiming} diagram.}
%
% \DescribeMacro\texttiminggrid!\hspace{2.5em}\textbf{Deprecated!}!
% This macro should only be used inside \cs{texttimingbefore} or 
% \cs{texttimingafter} and draws a grid of the full size of the
% \cs{texttiming} diagram.
% For newer code the TikZ styles "timing/draw grid" and "timing/no grid" 
% should be used instead, \eg
% |\tikzset{timing/intext/.append style={timing/draw grid}}| or simply enable the grid globally
% for all in-text and other timing diagrams with |\tikzset{timing/draw grid}|.
%
%
% \newpage
% \section{Macro for use inside TikZ-Pictures}
%
% \DescribeMacro\timing[<TikZ Keys>]~'at'~(<TikZ Coordinate>)~{[<initial character>]<characters>}';'
% This macro does the same as \cs{texttiming} but is designed to be used inside 
% a \envv{tikzpicture} environment and only there. Like normal TikZ macros 
% (\css{path}, \css{drawn}, \css{node}) it allows an optional argument with TikZ 
% settings and an optional TikZ-coordinate.  However, a own argument parser, not 
% the one used by TikZ, is used to detect and read these optional arguments.  
% Therefore the order of the arguments is mandatory and must not be reversed.  
% This small limitation might be overcome in future versions of this package.
%
% Please note that the optional initial character may be given \emph{inside} and 
% at the very start of the mandatory argument, not before it. This is necessary 
% because of several technical reasons.
%
% Example: |\tikz \timing [green] at (1,2) {HLZDZLH};|\phantom{\texttt{[L]}} 
% gives `\tikz \timing [green] at (1,2) {HLZDZLH};'.
% 
% Example: |\tikz \timing [green] at (1,2) {[L]HLZDZLH};| gives `\tikz \timing 
% [green] at (1,2) {[L]HLZDZLH};'.
%
% \ifprintversion\else
%   \clearpage
% \fi
% \subsection*{Timing Shape Anchors}\label{sec:timingshape}
% Every timing diagram line produced by \cs{timing}, which includes the rows in 
% \env{tikztimingtable}, is also a PGF shape (node) with several anchors. These 
% are shown in Figure~\ref{fig:shape}.
% The shape is very similar to the standard |rectangle| shape but does not 
% provide a |text| anchor. In addition to the standard points of the compass 
% anchors of TikZ the three logic levels |low|, |mid| and |high| can be used in 
% combination with |start|, |mid| and |end|.  An extra |origin| anchor is 
% located at the lower left, also called |south west| corner where the diagram 
% originates.  The two anchors called |start| and |end| are provided to mark the 
% start and end of the timing signal.  There are either located at the low, 
% middle or high logic level dependent on the used first (or initial) and last 
% timing character.
% 
% In order to use the timing node it has to be named which can be done using the 
% `\texttt{name=\meta{name}}' option inside the optional argument. The rows of a 
% \env{tikztimingtable} are automatically named as `|row|\meta{row number}' 
% where the first row has the number 1.
%
% ^^AFurther details how to use a shape inside a TikZ or PGF picture can be 
% ^^Afound in the PGF manual Section~15 and~59, respectively.
%
% \begin{figure}
% \centering
% \resizebox{\textwidth}{!}{%^^A
% \begin{tikzpicture}[timing,y=1.5cm,x=1.5cm,timing/slope=0]
%   \timing [y=1.5cm,x=1.5cm,gray,thick] at (0,0) {HLHLHL};
%   \node[name=s,shape=tikztiming@shape,anchor=origin] at (0,0) {};
%   \foreach \anchor/\placement in
%     {north west/above left, north/above,           north east/above right,
%     west/left,              center/right,          east/right,
%     mid west/right,         mid/left,              mid east/left,
%     base west/left,         base/left,             base east/above right,
%     south west/below left,  south/below,           south east/below right,
%     low start/below right,  mid start/below right, high start/above right,
%     low mid/above right,    high mid/right,
%     low end/below left,     mid end/below left,    high end/above left,
%     start/left,             end/right,             origin/right}
%   \draw[shift=(s.\anchor)] plot[mark=x] coordinates{(0,0)}
%     node[\placement] {\tiny\texttt{(s.\anchor)}};
% \end{tikzpicture}%^^A
% }
% \caption[Timing Shape Anchors]{Timing Shape Anchors. The \texttt{start} and 
% \texttt{end} anchors mark the start and end of the timing 
% signal.}\label{fig:shape}
% \end{figure}
%
%
% \clearpage
% \section{Table for Timing Diagrams}
% \begin{DescribeEnv}{tikztimingtable}[<TikZ settings for whole table>]\relax
%   \MacroArgs {<Signal Name>}~~'&'~~[<{\small Init.~Char./TikZ Keys for Row}>] <Characters> ~~'\\' \\
%   \dots\\
%   \hspace*{-1em}\Macro\extracode~~'% Optional' \\
%   <additional code>
% \end{DescribeEnv}
% This environment can be used to typeset multi-line timing diagrams.  The 
% syntax is like the one for a \envv{tabular} environment with two columns.  The 
% first column is for the signal name and the second one are the logic 
% characters which would be placed inside the argument of a \cs{texttiming} or 
% \cs{timing} macro.
% If the second column starts with an optional argument it is either taken as 
% initial character if it holds only a single character or as row wide settings 
% otherwise.
% The whole table will be drawn inside a \envv{tikzpicture} environment using 
% multiple \cs{timing} and \cs{node} macros for the timing signals and their 
% names, respectively.  Additional \pkg{tikz} drawing code can be insert at the 
% end of the table using \cs{extracode}.
%
% \DescribeMacro\extracode[TikZ Keys]
% This macro is only defined inside a \env{tikztimingtable} environment. In earlier
% versions of this package it could only be used after the last table line (\ie after a |\\|).
% If used there all code between it and the |\end{tikztimingtable}| will be placed inside the same 
% \envv{tikzpicture}. This allows to add some drawing lines or a grid to the 
% picture. The macro does \emph{not} start a TikZ |scope| or a \TeX\ group by itself.
% The optional \meta{settings} therefore affect all following code until the end of the picture.
%
% It is also possible to draw something behind the timing diagram by 
% using using the \env{background} environment or the PGF background layer:\\[\smallskipamount]
% |\begin{pgfonlayer}{background}|\ldots|\end{pgfonlayer}|
%
% \DescribeMacro\endextracode
% From\NewIn{v0.7} version 0.7 on it is possible to add further timing rows after an |extracode| section by using
% \cs{endextracode}. Everything after this macro is taken as part of a new row. It is allowed to use
% this macro direct before \cs{end}\env{tikztimingtable}. This makes it possible to use \cs{extracode} 
% anywhere inside the table, including at the very start before any rows are processed. Early
% insertion of extra code is necessary for \eg limiting the bounding box or setting up a clipping path.
%
% \DescribeEnv[<extra drawing code>]{extracode}[<TikZ settings>]
% Instead\NewIn{v0.7} of using \cs{extracode} \dots \cs{endextracode}, which is actual plain\TeX\ syntax, this 
% \LaTeX\ style environment can be used. Like any environment it creates a \TeX\ group around its 
% content, but no TikZ |scope| is created to allow drawing code (\eg clipping paths) to affect the
% rest of the table. The optional \meta{settings}, however, only affect the environment content.
%
% Please note that while \Macro\endextracode is optional if \Macro\extracode is used at the end of the table,
% a \Macro\begin{extracode} must always be closed by \Macro\end{extracode}.
%
% \subsection*{Macros for \cs{extracode} Section}
% The following macros are only defined inside a \env{tikztimingtable} after the 
% macro \cs{extracode}. They are useful for drawing additional material.
%
% \DescribeMacro\tablegrid*[<TikZ Keys>]
% After \css{extracode} this macro draws a grid in the background of the table.
% A separate grid is drawn for each row. The normal version draws all grids with 
% the width of the widest row while the star version draws them with the width 
% of the corresponding row. Because this macro draws material into the |background|
% layer it must not be placed inside a |pgfonlayer| environment itself.
%
% \DescribeMacro\fulltablegrid[<TikZ Keys>]
% After \css{extracode} this macro draws a big grid over all rows in the 
% background of the table.
%
% \DescribeMacro\nrows
% Returns the number of rows in the current table. Useful for use in 
% \cs{horlines}.
%
% \DescribeMacro\rowdist
% \DescribeMacro\coldist
% This macros return the row and column distance.  There are useful for drawing 
% additional material relative to the rows and columns.  This values can be set 
% (\eg in the optional argument of the table) using the \texttt{timing/rowdist} 
% and \texttt{timing/coldist} settings which are explained in 
% Section~\ref{sec:styles}.
%
% \DescribeMacro\twidth
% Returns the width (as multiple of the `period width') of the longest timing 
% diagram line in the table. Example: If the longest line would be
% "H 2.3L z" than \cs{twidth} would be $1+2.3+0.5=3.8$.
%
% \DescribeMacro\horlines[<TikZ Keys>]{<list>}
% Draws horizontal lines, optionally with the given \meta{Settings}, at the base 
% line of the rows given by \meta{list}. The PGF macro 
% \css{foreach}\footnote{See the \texttt{pgf} manual for more details.} is 
% internally used so the list can include not only row numbers as integer but 
% also fractional numbers and the "..." operator to auto-increment the 
% numbers. Please note that all numbers in the list are multiplied by 
% \cs{rowdist}.
% If the list is empty the default "1,2,...,\nrows" is used which draws lines 
% for all rows.
%
% \DescribeMacro\vertlines[<TikZ Keys>]{<list>}
% Like \cs{horlines} but draws vertical lines and the listed numbers a relative 
% to the basic width.
% If the list is empty the default "0,1,...,\twidth" is used which draws lines 
% after every period width.
%
% \DescribeMacro\tableheader[<TikZ Keys>]{<Description Title>}{<Signal Title>}
% This macro adds a table head row on top of the table. The two mandatory 
% arguments provide the header text for the description and signal columns.
%
% \DescribeMacro\tablerules[<TikZ Keys>]
% This macro adds top and bottom rules to the table in the same (or at least 
% very similar) way as the \texttt{booktabs} package is doing it for normal 
% \texttt{tabular}s. The current bounding box is used to calculate the needed 
% rule length, which makes this macro position dependent if further code
% changes the bounding box. If the \cs{tableheader} macro was used beforehand it 
% also draws a thinner horizontal line (like \texttt{booktabs} \cs{midrule}) 
% between the table head and body.
%
% \DescribeEnv[<drawing commands>]{background}[<TikZ Keys>]
% This environment can be used to draw material on the |background| layer and is 
% an abbreviation for:\\[\medskipamount]
% |    \begin{pgfonlayer}{background}|\\
% |      \begin{scope}|\oarg{TikZ Keys}\\
% |        |\meta{drawing commands}\\
% |      \end{scope}|\\
% |    \end{pgfonlayer}|
%
% \subsection*{Scaling the Table}
% The standard "scale" setting of TikZ will not scale all parts of the table correctly,
% \eg the line width and nodes will keep their original size. However there are scaled
% relative to the font size. Alternatively the table can be scaled using the
% \cs{scalebox}\marg{factor}\marg{content} macro from the \pkg{graphicx} package or 
% be placed inside a scaled \cs{node} inside another \env{tikzpicture} environment.
%
% \subsection*{Positions \& Nodes inside the Table}
%
% \subsubsection*{Coordinates}
% The first row starts at $y=0$ and the next rows are each
% |-1*\rowdist| lower than the previous one.
% The vertical unit is 1 \textsf{signal height} and the default row distance is 
% `2' (=$2\times$\textsf{signal height}).  This means that a normal table with 
% three rows goes from $y=+1$ (base line at 0 + 1 \textsf{signal height}) to 
% $y=-4$ (first row:~+0, second row:~-2, third row:~-4).  This are relative to 
% the middle of the drawn lines, \ie the bounding box is 
% $2\times\frac{\text{\textsf{line width}}}{2}=1\times $\textsf{line width} 
% higher.
%
% The timing column starts at $x=0$ and goes into the positive range while 
% scaled using the period width. Example: |HHHh| has a width of 3.5.
% The label column starts at $x=-$|\coldist| and the text is right align with 
% the right border at this position.
% See Figure~\ref{fig:distnodestable} for an illustration.
%
% \subsubsection*{Nodes}
% Each timing line is a timing node (see section~\ref{sec:timingshape}) 
% labeled (not fully correctly) as `|row|\meta{number}', where the first row has the number 1 and the last 
% one the number provided in \cs{nrows}, but can also accessed using the alias "last row".
% The corresponding labels are normal rectangle nodes named "label0", "label1", \dots,
% `|label|\cs{nrows}'/"last label".
%
% Both groups of `rows' and `labels' are enclosed in a rectangle node called "all rows" and
% "all labels", respectively. These nodes can be used to draw material relative to the rows,
% \eg the macros \cs{tableheader} and \cs{tablerules} are making use of them.
% The headers added by \cs{tableheader} are rectangle nodes names "label header" and "row header"
% and are placed between the x-coordinates of the inner and outer border of "all labels" and "all rows"
% respectively. By default the TikZ settings "pos=0" and "anchor=base east"/"anchor=base west", respectively,
% are applied to place them in the inner border, but this
% can be changed using the styles "timing/table/header" and/or "timing/table/label header"/"timing/table/row header".
% All nodes are shown in Figure~\ref{fig:distnodestable}.
%
%
% \begin{figure}
%   \centering
%   \scalebox{1.15}{\Huge
%   \begin{tikztimingtable}[semitransparent]
%     First Row   &  H L H L H L \\
%     Second Row  &  H L H L H L \\
%     $\vdots$\hspace*{2em} &  3S N[rectangle,above,inner sep=0pt]{$\vdots$} 3S \\
%     Last Row    &  H L H L H L \\
%    \begin{extracode}
%     \tableheader[draw=black,densely dashed,thin]{Label}{Timing}
%     \tablerules
%     \useasboundingbox (current bounding box.south west) rectangle (current bounding box.north east);
%     \begin{background}[black,densely dashed,thin,opacity=1]
%       \normalsize
%       \draw (row1.north west)     rectangle (row1.south east);
%       \draw (row2.north west)     rectangle (row2.south east);
%       \draw (last row.north west) rectangle (last row.south east);
%       \node [right] at (row1.east) {row1};
%       \node [right] at (row2.east) {row2};
%       \node [right] at (last row.east) {row\cs{nrows}};
%       \node [below right] at (last row.east) {last row\large\strut};
%
%       \path let \p1 = (label header.west), \p2 = (row header.east), \p3 = (all labels.west), \p4 = (all rows.east) in
%           node [left]  at (\x3,\y1) {label header}
%           node [right] at (\x4,\y2)   {row header}
%         ;
%       \begin{scope}[solid,semithick,black!80]\tiny
%       \draw [|->] let \p1 = (label header.south east), \p2 = (all labels.west) in
%           (\x1+.5\pgflinewidth,\y1-2pt) node [below left=-.4ex] {0}
%           -- node [below=-.4ex,midway] {pos} (\x2,\y1-2pt) node [below right=-.4ex] {~\rlap{1}}
%         ;
%       \draw [|->] let \p1 = (row header.south west), \p2 = (all rows.east) in
%           (\x1-.5\pgflinewidth,\y1-2pt) node [below right=-.4ex] {0}
%           -- node [below=-.4ex,midway] {pos} (\x2,\y1-2pt) node [below left=-.4ex] {\llap{1}\,~}
%         ;
%       \end{scope}
%
%       \draw [dotted,color=green] (all rows.south west) rectangle (all rows.north east);
%       \node [below right,green] at (all rows.east) {all rows};
%
%       \draw (label1.north west)   rectangle (label1.south east);
%       \draw (label2.north west)   rectangle (label2.south east);
%       \draw (last label.north west) rectangle (last label.south east);
%       \path let \p1 = (label1.west), \p2 = (label2.west), \p3 = (last label.west), \p4 = (all labels.west) in
%           node [left]  at (\x4,\y1)      {label1}
%           node [left]  at (\x4,\y2)      {label2}
%           node [left]  at (\x4,\y3)      {label\cs{nrows}}
%           node [below left] at (\x4,\y3) {last label\large\strut}
%         ;
%
%       \draw [dotted,color=green] (all labels.south west) rectangle (all labels.north east);
%       \node [below left,green] at (all labels.west) {all labels};
%
%       \begin{scope}[help lines]
%         \draw (all rows.north west)   -- +(0,3.25);
%         \draw (all rows.north east)   -- +(0,3.25);
%         \draw (all labels.north west) -- +(0,3);
%         \draw (all labels.north east) -- +(0,3);
%       \end{scope}
%
%       \begin{scope}[solid,thick]
%         \draw [fill] (0,0) circle (1pt) node [below right=-2pt] {\tiny origin};
%         \draw [>=stealth,->] (-.25,0) -- (.4,0);
%         \draw [>=stealth,->] (0,-.25) -- (0,.4);
%       \end{scope}
%
%       \begin{scope}[dotted,gray]
%         \draw (row1.south west) -- (label1.base west);
%       \end{scope}
%
%       \begin{scope}[solid]
%         \draw (row1.north east) -- ++(5em,0) coordinate (O) -- +(5pt,0);
%         \draw (row1.south east) -- ++(5em,0) coordinate (A) -- +(5pt,0);
%         \draw (row2.south east) -- ++(5em,0) coordinate (B) -- +(5pt,0);
%         \draw [<->] (A) -- node[midway,right] {\cs{rowdist}} (B);
%         \draw [<->] (O) -- node[midway,right] {\cs{yunit}} (A);
%
%         \draw (row1.south east)          -- ++(0,1.5) coordinate (A) -- +(0,5pt);
%         \draw (row1.south east) ++(-1,1) -- ++(0,0.5) coordinate (B) -- +(0,5pt);
%         \draw [<->] (B) -- (A) node[right] {\cs{xunit}};
%
%         \draw (row1.north west) -- ++(0,7em) coordinate (A) -- +(0,5pt);
%         \draw let \p1 = (row1.north west), \p2 = (label1.north east) in
%           (\x2,\y2) -- (\x2,\y1+7em) coordinate (B) -- +(0,5pt);
%         \draw [<->] (B) -- (A) node[right] {\cs{coldist}};
%       \end{scope}
%     \end{background}
%    \end{extracode}
%   \end{tikztimingtable}%^^A
%   }%^^A
%   \caption{Distances and Nodes inside a \env{tikztimingtable}}\label{fig:distnodestable}
% \end{figure}
%
% \clearpage
% \section{Macros for use inside the Character String}
% The modifiers "@" and `\texttt{\$}' allow the user to include macros.  These 
% macros are evaluated when the |tikz-timing| parser encounters them in the 
% input character string, \ie before any diagram element is drawn or any single 
% bracket "[ ]" options are processed. Therefore their values should be set 
% either outside the |tikz-timing| diagram or with the "[! .. !]" or
% "[[ .. ]]" option blocks.
%
% The following macros are provided for the user.
%
% \DescribeMacro\tikztimingsetwscale{<math expression>}
% \DescribeMacro\setwscale{<math expression>}
% This macro\NewIn{v0.7}, which can be called |\setwscale| for short inside modifier code, 
% sets the |wscale| value. This value is used during the parsing process to 
% scale the width of the characters, \eg |wscale=3.2| makes |1H| as long as 
% |3.2H| normally would be. Slopes are not affected, but the `width' values of 
% meta-characters are. It can also be set with the |timing/wscale| TikZ setting.  
% The current value can be accessed using \cs{wscale}.
%
% \DescribeMacro\wscale
% Returns the current width scaling "wscale" value. 
%
% \DescribeMacro\xunit
% \DescribeMacro\yunit
% This dimension registers can be used to access the x- and y-unitlength of the timing diagram.
% Assignments to these registers do not change the scaling!
%
% \DescribeMacro\slope
% \DescribeMacro\lslope!\hspace{.6em}\small(alias)!
% Returns the current logic slope, \ie the slope between |L| and |H| 
% levels.
% Set by the |timing/lslope| or indirectly by the |timing/slope| TikZ setting.  
% See Table~\ref{tab:styles} for more information.
%
% \DescribeMacro\zslope
% Returns the current Z slope. Set by the |timing/zslope| or indirectly by the 
% |timing/slope| TikZ setting.
%
% \DescribeMacro\dslope
% Returns the current Z slope. Set by the |timing/dslope| or indirectly by the 
% |timing/slope| TikZ setting.
%
% \subsection*{Examples:}
% Changing the slope and using its value to calculate the width of a 
% character:\\
% |\texttiming{ HLHLHL [[timing/slope=.5]] H $\slope$L }|\\
% gives:
% {\texttiming{ HLHLHL [[timing/slope=.5]] H $\slope$L }}
% \\*[\smallskipamount]
% Changing the width scaling for a curtain group of characters:\\*
% |\texttiming|\ignorespaces
% |{ HL [!wscale=\wscale/3!] 3D{a} Z D Z [!wscale=3*\wscale!] HL  }|\\*
% gives:
% {\texttiming{ HL [!wscale=\wscale/3!] 3D{a} Z D Z [!wscale=3*\wscale!] HL }}
%
% \clearpage
% \section{Meta-Characters}
% It is possible to define recurring groups of characters and modifiers as so 
% called meta-characters. These characters are than expanded to the group 
% whenever they appear inside the character list. Please note that like for 
% groups a numeric factor before such a meta-character is taken as a repetition 
% factor not as a width. The meta-character is case sensitive and the other case 
% is not affected by the definition, \ie the lower- and uppercase versions of 
% one character can have complete different meanings. It is possible to redefine 
% normal characters (only one or both cases) as meta-characters, which 
% suppresses its normal meaning. Using the meta-character in its own definition 
% group causes a infinite loop which will lead to an \TeX\ error.
% 
% \DescribeMacro\tikztimingmetachar{<Meta-Character>}[<Number of arguments>]{<Character Group>}
% This macro defines the given \meta{meta-character} to be identical to the 
% given \meta{character group}. Alternatively this can also be done using the 
% TikZ style `|timing/metachar=|\marg{Meta-Character}\oarg{Number of 
% arguments}\marg{Character~Group}'. 
%
% An empty group deletes the meta-character, which  might be necessary in cases 
% when normal characters are temporary redefined as meta-characters. However, if 
% the group only contains spaces the meta-character is practically ignored.
%
% \subsection*{Meta-Characters with Arguments}
% The\NewIn{v0.7}
% replacement text of meta-character can now include macro arguments.
% This allows the creation of more complex and flexible meta-characters.  The 
% optional argument \meta{Number of arguments} selects the number of macro 
% arguments in the same way it does for \cs{newcommand}. However, the first 
% argument |#1| is always set to the given `width' of the meta-character, \ie 
% the number value preceding it. All further arguments are read as normal for 
% macros from the text after the meta-character. It is recommended to enclose 
% them in braces.
%
% The default behaviour of meta-character without arguments is, as mentioned 
% above, to repeat the replacement group by the preceding number (`width'). This 
% is now archived by defining them internally as `|#1|\marg{Character Group}',
% which creates a repetition group. Users which want to include an own argument 
% but still want to repeat the group need to define a meta-character with at 
% least two arguments and define it as "#1{ .. #2 .. }". If the repetition is 
% not wanted the |#1| argument can be used as a real width for one or more group 
% internal characters: "{Y}[1]{Z #1D Z}", so "4Y" will give "Z 4D Z" 
% instead of "4{Z D Z}".
% 
% Also the modifier "@" (see Table~\ref{tab:modifiers}) together with the 
% \cs{setwscale} macro can be used to scale the whole group dependent on the 
% first argument:\\
% "{Y}[1]{ @{\setwscale{#1*\wscale}} Z 2D Z @{\setwscale{\wscale/#1}} }", so
% "4Y" is equivalent to "4Z 8D 4Z".
%
% The new `\texttt{\$}' modifier can be used to calculate the width of the group 
% characters: |{Y}[1]{$#1/3$D{A} $#1/3D${B} $#1/3$D{C}}|, so "4Y" results in 
% "1.333D{A} 1.333D{B} 1.333D{C}".
%
%
% \bigskip
% \noindent
% \textit{Examples:}\\*[\medskipamount]
% |\tikztimingmetachar{Y}{2D 0.5U 2D{}}  \texttiming{ZZ Y Z 3Y ZZ}|\\
% gives:
% {\tikztimingmetachar{Y}{2D 0.5U 2D{}}  \texttiming{ZZ Y Z 3Y ZZ}}
% \\[\bigskipamount]
% |\tikztimingmetachar{Y}{2D{Text}}  \tikztimingmetachar{y}{1D{\tiny Text}}|
% |\texttiming{ZZ Y Z 3y ZZ}|\\
% gives:
% {\tikztimingmetachar{Y}{2D{Text}}  \tikztimingmetachar{y}{1D{\tiny Text}}
% \texttiming{ZZ Y Z 3y ZZ}}
% \\[\bigskipamount]
% \DeleteShortVerb{\|}
% \MakeShortVerb{\`}
% `\newcounter{mycount}`\\
% `\tikztimingmetachar{Q}{2D{\stepcounter{mycount}\arabic{mycount}}}`\\
% `\tikztimingmetachar{R}{[| /utils/exec=\setcounter{mycount}{0} |]}`\\
% `\texttiming{ 5Q R 3Q R 10Q }`\\
% gives:
% {\newcounter{mycount}%
% \tikztimingmetachar{Q}{2D{\stepcounter{mycount}\arabic{mycount}}}%
% \tikztimingmetachar{R}{[| /utils/exec=\setcounter{mycount}{0} |]}%
% \texttiming{ 5Q R 3Q R 10Q }}
% \DeleteShortVerb{\`}
% \MakeShortVerb{\|}
% \\[\bigskipamount]
% Redefining the glitch "G" character:\\
% |\tikztimingmetachar{G}{.1T.1T .2B} \tikztimingmetachar{g}{.1T.1T}|\\
% |\texttiming{ 10{H G L G} H } % With correction of width `.2B'|\\
% |\texttiming{ 10{H g L g} H } % Without correction|\\
% |\texttiming{ 10{H   L  } H } % For comparison|\\
% gives:\\
% {\let\texttimingbefore\texttiminggrid
% \tikztimingmetachar{G}{.1T.1T .2B} \tikztimingmetachar{g}{.1T.1T}%
% \texttiming{ 10{H G L G} H }\\
% \texttiming{ 10{H g L g} H }\\
% \texttiming{ 10{H   L  } H }}
% \\[\bigskipamount]
% |\tikztimingmetachar{J}[2]{ 1.8D{#2} .2D{} }|\\
% |\texttiming{ D{} J{A} J{B} J{C} D }|\\
% gives:
% {\tikztimingmetachar{J}[2]{ 1.8D{\strut #2} .2D{} }%
%  \texttiming{ D{} J{A} J{B} J{C} D }}
% \\[\bigskipamount]
% |\tikztimingmetachar{Y}[3]{#1{ D{\strut #2} D{#3} }}|\\
% |\texttiming{ Z 2Y{a}{b} Z 4Y{1}{2} Z}|\\
% gives:
% {\tikztimingmetachar{Y}[3]{#1{ D{#2} D{#3} }}%
%  \texttiming{ Z 2Y{a}{b} Z 4Y{1}{2} Z}}
% \\[\bigskipamount]
% |\tikztimingmetachar{Y}[3]{ .2D .2B #1d{\strut #2} .2D .2B #1d{\strut #3} }|\\
% |\texttiming{ Z 2Y{a}{b} Z 4Y{1}{2} Z}|\\
% gives:
% {\tikztimingmetachar{Y}[3]{ .2D .2B #1d{\strut #2} .2D .2B #1d{\strut #3} }%
%  \texttiming{ Z 2Y{a}{b} Z 4Y{1}{2} Z}}
% \\[\bigskipamount]
% Mata-chars to set the width scaling. Because the scaling also affects the 
% meta-char width (|#1|) argument a compensation is needed to achieve absolute 
% values (|W|) instead of relative ones (|w|).\\
% |\tikztimingmetachar{w}[1]{ [! wscale=#1 !] }         % relative|\\
% |\tikztimingmetachar{W}[1]{ [! wscale=#1/\wscale !] } % absolute|\\
% |\texttiming{ HL .2w HLHLH 3w LH 1W LH }|\\
% gives:
% {\tikztimingmetachar{w}[1]{ [! wscale=#1 !] }%
%  \tikztimingmetachar{W}[1]{ [! wscale=#1/\wscale !] }\\
%  \texttiming{ HL .2w HLHLH 3w LH 1W LH }}
%
% \clearpage
% \chapter{TikZ Keys for Styles, Settings and Actions}\label{sec:styles}
% TikZ itself uses the |pgfkeys| package to define and apply drawing styles and settings.
% The same method is also used for |tikz-timing| which 
% places all of the keys under the ``subdirectory'' "timing" in 
% the main ``directory'' `tikz', which is the default when \css{tikzset} is 
% used.
% This keys are simply called \emph{TikZ Keys} throughout this manual and
% can be used in all places where \meta{TikZ Keys} is mentioned, while some only make sense at specific
% places.
% Three types of keys 
% are used by this package: styles, settings and actions.
%
% Styles simply define the style in which a certain element is drawn, \eg in which color or line width.
% This styles are defined and can be redefined using 
% \css{tikzset}|{|\meta{style name}|/.style=|\meta{value}|}|. However, while some styles
% are initial empty, some hold internal settings and therefore user styles should only be added
% using `|.append style=|\meta{value}'.
%
% Settings are TikZ keys which await an argument and set an internal macro or length. They are
% like \cs{setlength} and \cs{renewcommand}. They should not be redefined by the user.
%
% Action are TikZ keys which perform a drawing or other action on the current element, either directly or
% by en-/disable an internal setting which then in turn triggers the drawing process. Therefore some
% actions can be globally and/or locally but others only make sense if used locally on a single |tikz-timing|
% macro or environment or even a scope inside a \env{tikztimingtable}. 
% Action can be very similar to settings but they always execute code instead of only setting/redefining it.
%
% \begingroup
% \def\C{Action}
% \def\S{Style}
% \def\s{Setting}
% \def\en{\,\textsf{--}\,}
%
% \section*{General}
%   \begin{tabularx}{\textwidth}{>{\ttfamily}ll>{\raggedright\arraybackslash}X}
%     \toprule
%     \normalfont TikZ Key      & Type & Description \\
%     \midrule
%     timing                                     & \S & Base settings like signal height and period width.\\
%     timing/font=\meta{font commands}           & \s & Sets the normal \texttt{font} key and sets \texttt{x}/\texttt{y} keys to 1.6ex.\\
%     timing/intext                              & \S & Used for \cs{texttiming}. Depends on \texttt{timing}.\\
%     timing/picture                             & \S & Usable for own \texttt{tikzpictures} to set timing settings.\\
%     timing/inline node                         & \S & Used for nodes created by the \texttt{N} character. Defaults to \texttt{coordinate}.\\
%     \midrule
%     timing/initchar=\meta{char}                & \s & Sets initial character.  Only valid as first optional argument in table rows or in \cs{texttiming}.\\
%     \midrule
%     timing/metachar=\marg{C}\oarg{\#arg}\marg{G}& \C& Sets meta-character \meta{C} to character group \meta{G}.\\
%     \bottomrule
%   \end{tabularx}
%
% \section*{Slopes}
%   \begin{tabularx}{\textwidth}{>{\ttfamily}ll>{\raggedright\arraybackslash}X}
%     \toprule
%     \normalfont TikZ Key      & Type & Description \\
%     \midrule
%     timing/slope=\meta{0.0\en1.0}              & \s & Sets slope for logic transitions. \newline This also sets \texttt{dslope=2*slope}, \texttt{zslope=slope/2}.\\
%     timing/lslope=\meta{0.0\en1.0}             & \s & Sets slope for logic transitions only. Default: 0.1\\
%     timing/dslope=\meta{0.0\en1.0}             & \s & Sets slope for data transitions. Default: 0.2\\
%     timing/zslope=\meta{0.0\en1.0}             & \s & Sets slope for Z transitions. Default: 0.05\\
%     \bottomrule
%   \end{tabularx}
%
% \section*{Texttiming}
%   \begin{tabularx}{\textwidth}{>{\ttfamily}ll>{\raggedright\arraybackslash}X}
%     \toprule
%     \normalfont TikZ Key      & Type & Description \\
%     \midrule
%     timing/outer sep=\meta{dim}                & \s & Sets outer separation around \cs{texttiming} macros. \\
%     timing/outer xsep=\meta{dim}               & \s & See above. Only X-Coordinate. (Default:~0pt) \\
%     timing/outer ysep=\meta{dim}               & \s & See above. Only Y-Coordinate. (Default:~0pt) \\
%     \bottomrule
%   \end{tabularx}
%
% \section*{Grid}
%   \begin{tabularx}{\textwidth}{>{\ttfamily}ll>{\raggedright\arraybackslash}X}
%     \toprule
%     \normalfont TikZ Key      & Type & Description \\
%     \midrule
%     timing/grid                                & \S & Style used for drawing grids. Depends on \texttt{help lines} and \texttt{timing}.  \\
%     timing/draw grid                           & \C & Enables background grids for \cs{timing} macros. \\
%     timing/no grid                             & \C & Disabled background grids for \cs{timing} macros. \\
%     \bottomrule
%   \end{tabularx}
%
% \section*{Table}
%   \begin{tabularx}{\textwidth}{>{\ttfamily}ll>{\raggedright\arraybackslash}X}
%     \toprule
%     \normalfont TikZ Key      & Type & Description \\
%     \midrule
%     timing/table                               & \S & Used for \env{tikztimingtable}. Depends on \texttt{timing}.\\
%     timing/table/grid                          & \S & Used for table grid. Depends on \texttt{timing/grid}.\\
%     timing/table/lines                         & \S & Used for \cs{horlines} and \cs{vertlines}. \\
%     timing/table/rules                         & \S & Used for \cs{tablerules} for top and bottom lines. \\
%     timing/table/midrules                      & \S & Used for \cs{tablerules} between table head and body.  \\
%     timing/table/header                        & \S & Used for \cs{tableheader}. \\
%     timing/table/title header                  & \S & Used for label header in \cs{tableheader}. \\
%     timing/table/row header                    & \S & Used for timing row header in \cs{tableheader}. \\
%     \midrule
%     timing/rowdist=\meta{distance}             & \s & Sets (baseline) distance between rows in a tikztimingtable. Default: 2 (=$2\times$signal height)\\
%     timing/coldist=\meta{distance}             & \s & Sets distance between columns in a tikztimingtable. Default: 1 (=$1\times$period width)\\
%     \bottomrule
%   \end{tabularx}
%
% \section*{Character Styles}
%   \begin{tabularx}{\textwidth}{>{\ttfamily}ll>{\raggedright\arraybackslash}X}
%     \toprule
%     \normalfont TikZ Key      & Type & Description \\
%     \midrule
%     timing/\meta{lowercase char}               & \s & Style for character \meta{char}. Not used for `\texttt{H}' and `\texttt{L}'. \\
%     timing/\meta{lc char}/background           & \s & Background style for characters `\texttt{D}' and `\texttt{U}'. \\
%     timing/\meta{lc char}/text                 & \s & Text style for character \meta{char}.  Only defined for `\texttt{D}'. \\
%     \midrule
%     timing/text format=\meta{macros}           & \s & Define macros which are placed before the text of `\texttt{D\{text\}}'. The text is enclosed in braces to 
%                                                    allow the last macro to take it as an argument. A \cs{strut} is a good choice to ensure a common baseline.\\
%     \bottomrule
%   \end{tabularx}
%
% \section*{Debug}
% Some debug settings for users (first group) and the package developer (last group).
%
% \medskip\noindent
%   \begin{tabularx}{\textwidth}{>{\ttfamily}ll>{\raggedright\arraybackslash}X}
%     \toprule
%     \normalfont TikZ Key      & Type & Description \\
%     \midrule
%     timing/debug/nodes                         & \C & Enables marking of named inline nodes. \\
%     timing/debug/node                          & \S & Format style for inline node marker label, which itself is a TikZ node. \\
%     timing/debug/markcmd=\meta{code}           & \s & TikZ drawing code to draw marker (except label). The code can access the node name as \cs{N}. \\
%     timing/debug/scope                         & \S & Format for scope of node markers. \\
%     \midrule
%     timing/debug/level=\meta{integer}          & \s & Sets debug output level. This is only important for developers. \\
%     \bottomrule
%   \end{tabularx}
%
% \endgroup
%
% \clearpage
% \chapter{Libraries for Further Characters}
% All default timing characters described in Table~\ref{tab:chars} are always 
% made available by this package. Further, less-common characters are provided 
% by libraries which are loaded with the macro 
% \cs{usetikztiminglibrary}\marg{library}.  This is done to hold the memory 
% usage of this package small and reduce the risk of collisions with 
% user-defined (meta-)characters.
% The full syntax for the above macro is
% \cs{usetikztiminglibrary}\oarg{options}\marg{library,\dots}\oarg{date}, like 
% the one for \cs{usepackage}. 
% The date is used as a version number like for packages and is identical to the date 
% of the |tikztiming| package.
%
% \clearpage
% \library{arrows}{Arrows}
% The library "arrows" enables two characters "A" and "W" which draw 
% vertical up and down \textit{A}rro\!\textit{W}s. Such arrows are used in 
% timing diagrams to mark special polarized events, like clock edges of another 
% signal.
%
% The width provided with these character is added as whitespace after the 
% ``zero-width'' arrow: "A2AA" results in `\texttiming{A2AA}'. This space 
% can be avoided by specifying the width to zero: "0A". Like the "C" and 
% "T" characters the subsequent arrow characters are not combined into one.
%
% The arrow tips can be changed using the TikZ styles for this characters. See 
% section~\ref{sec:styles} for more information. The "A" and "W" character should 
% only be used which each each other, but not together with any other characters 
% except with "S" (space).
%
% \begin{table}[H]
% \centering
% \caption{Examples for Arrow Characters.}\label{tab:arrows}
% \begin{tikztimingexampletable}
%   0A           \\
%   AAA          \\
%   3A           \\
%   3{A}         \\
%   3A 3A        \\
%   3a 3a        \\
%   AW AW        \\
%   3{AW}        \\
%   3{aw}        \\
%   2S 2A 3W A W \\
%   \extracode
%   \tablegrid*
% \end{tikztimingexampletable}
% \end{table}
%
% \clearpage
% \library{either}{Either High or Low}
% The library "either" enables the "E" character which stands for 
% `\textit{either high or low}'. This character is designed to be used with the 
% "H" and "L" characters to display a uncertainty of a transition.  
% Sometimes a, e.g.\ low signal can go either to high or stay at low for a 
% certain time before it definitely goes to high. In this case both (or more) 
% possible transitions can be drawn using this character.
% Like the "C" and "T" characters subsequent "E" characters are not 
% combined into one.
%
% The drawing style can be changed using the |timing/e| and 
% |timing/e/background| TikZ style.
%
% \begin{table}[H]
% \centering
% \caption{Examples for the `\texttt{E}' Character.}\label{tab:either}
% \begin{tikztimingexampletable}[timing/e/background/.style={fill=black!20}]
%   L E H          \\
%   L D H          \\
%   H E L          \\
%   H D L          \\
%   L E E H        \\
%   L 3{.25E} H    \\
%   H E E L        \\
%   L EEE HH EEE L \\
%   l e e h        \\
%   h e e l        \\
%   H 2E  L        \\
%   H 2{E} L       \\
%   H 5{e} L       \\
%   H E E H        \\
%   L E E L        \\
% \end{tikztimingexampletable}
% \par\medskip
% \makebox[\textwidth][c]{%
%   \begin{tabular}{lc}
%     \toprule
%       Settings (\texttt{timing/e/.cd})         & Resulting Diagram (\texttt{LL EE HH})                         \\
%     \midrule
%       \texttt{.style=\{dotted,gray\}}          & \texttiming[timing/e/.style={dotted,gray}]{LL EE HH}          \\
%       \texttt{background/.style=\{fill=gray\}} & \texttiming[timing/e/background/.style={fill=gray}]{LL EE HH} \\
%     \bottomrule
%   \end{tabular}
% }%
% \end{table}
%
% \clearpage
% \library{overlays}{Overlays}
% The library "overlays" enables the "O" character which allows the 
% overlaying of timing characters, i.e.~different groups of timing characters
% are drawn on top of each other. This is not to be confused with `dynamic' overlay
% provided by the presentation class |beamer|. The |tikz-timing| library \lib{beamer} provides
% some support for such overlays.
%
% The "O" character awaits a set of character 
% enclosed by braces which are drawn as normal. The position before the "O" 
% character is then restored and the following characters are drawn over them.  
% Older versions of this character awaited a second set of characters in braces 
% but this braces are now optional. The exact syntax is:\\[\smallskipamount]
% \hspace*{2em} \meta{chars before} |O|\marg{background chars} \marg{foreground 
% chars} \meta{chars after}\\[\smallskipamount]
% or, without second set of braces, but equal:\\[\smallskipamount]
% \hspace*{2em} \meta{chars before} |O|\marg{background chars} \meta{foreground 
% chars, \dots}\\[\smallskipamount]
%
% It is the responsibility of the user to make sure that the lines drawn by the 
% first set reconnect to the main lines or do something else useful.  The 
% modifier ";" can be used to restart the drawn line, e.g.\ to change to a 
% different color. This is not done automatically to give the user the freedom 
% if and where this should happen. It is recommended to start and end the set 
% with characters identical with the main line to avoid ugly connection points.
%
% Please note that the width of the first argument is ignored and does not count to the 
% total width of the diagram line. The characters following the overlay should therefore
% be as wide or wider as the one of the overlay, otherwise the bounding box and background grid
% will be incorrect. 
%
% Overlays can be cascaded\NewIn{v0.7}, \ie an overlay can be included in the first argument 
% of another overlay. 
%
% \begin{table}[H]
% \centering
% \caption{Examples for the `\texttt{O}' Overlay Character.}\label{tab:overlays}
% \ifprintversion
% \begin{tikztimingexampletable}[timing/draw grid]
%   LLL O{HH}{LL} HHH              \\
%   LLL O{HHH}{LL} HHH             \\
%   LLL O{;[gray]HH.1H;}{LLH} HH   \\
%   LL  O{L;[gray]HH.1H;}{LLLH} HH \\
%   DD{} O{zd}{D}d 2D              \\
%   ZZ O{Z D Z}{Z 1.1M .9Z} ZZ     \\
%   ZZ O{d Z}O{DZ}{dD} ZZ          \\
%   ZZ O{dDZ}O{DZ}{dZ} ZZ          \\
%   ZZ 3D O{dDZ}{DZ} ZZ            \\
%   ZZ 3D O{dDZ}O{DZ}{dZ} ZZ       \\
%   ZZ 3D O{3D} DZZ                \\
%   Z O{DD} ZDDD O{DDZZ} DZ 2S     \\
%   Z O{6D Z}{Z 4D Z} Z            \\
%   Z O{8D Z}O{Z 6D Z}{2Z 4D 2Z} Z \\
% \end{tikztimingexampletable}
% \else
% \begin{tikztimingexampletable}[timing/draw grid]
%   LLL O{HH}{LL} HHH              \\
%   LLL O{HHH}{LL} HHH             \\
%   LLL O{;[gray]HH.1H;}{LLH} HH   \\
%   LL  O{L;[gray]HH.1H;}{LLLH} HH \\
%   DD{} O{zd}{D}d 2D              \\
%   ZZ O{Z D Z}{Z 1.1M .9Z} ZZ     \\
%   ZZ O{d Z}O{DZ}{dD} ZZ          \\
% \end{tikztimingexampletable}
% \begin{tikztimingexampletable}[timing/draw grid]
%   ZZ O{dDZ}O{DZ}{dZ} ZZ          \\
%   ZZ 3D O{dDZ}{DZ} ZZ            \\
%   ZZ 3D O{dDZ}O{DZ}{dZ} ZZ       \\
%   ZZ 3D O{3D} DZZ                \\
%   Z O{DD} ZDDD O{DDZZ} DZ 2S     \\
%   Z O{6D Z}{Z 4D Z} Z            \\
%   Z O{8D Z}O{Z 6D Z}{2Z 4D 2Z} Z \\
% \end{tikztimingexampletable}
% \fi
% \end{table}
%
% \clearpage
% \library{clockarrows}{Clock Arrows}
% The library "clockarrows"\NewIn{v0.7} is changing the "C" clock character to contain 
% arrows which mark the rising and/or falling clock edge. By default the rising 
% edges are marked. To simplify the implementation only the transition from a 
% "C" to another "C" character contains the arrows but not transitions from 
% or to different characters, like "HCH" or "LCL".
%
% The arrows can be controlled using the TikZ styles shown in 
% Table~\ref{tab:clockarrowstyles} below. This styles can also be used as 
% library options. The key ``directory'' "timing/c" must be dropped for 
% options, e.g.\\\hspace*{2em}
% \cs{usetikztiminglibrary}|[rising arrows]{clockarrows}|.

% \begin{table}[H]
%   \centering
%   \caption{TikZ Styles for Clock Arrows.}
%   \label{tab:clockarrowstyles}
%   \makebox[\textwidth][c]{%
%   \begin{tabularx}{1\textwidth}{>{\ttfamily}l>{\raggedright\arraybackslash}X}
%     \toprule
%       \normalfont TikZ Style                 & Description \\
%     \midrule
%       timing/c/rising arrows                 & Mark (only) rising edges with arrows.\\
%       timing/c/falling arrows                & Mark (only) falling edges with arrows.\\
%       timing/c/dual arrows                   & Mark both rising and falling edges with arrows.\\
%       timing/c/no arrows                     & Do not mark any edges with arrows. (Default)\\
%     \midrule
%       timing/c/arrow                         & Style for arrows. Can be modified to change arrow tip etc.  (Default: \{\})\\
%       timing/c/arrow pos=\meta{0.\!-\!\!-1.} & Position of arrows, i.e.\ its tip, on the edge.
%                                                May needs adjustment if different arrow tip shapes are selected.  (Default: 0.95)\\
%       timing/c/arrow tip=\meta{name}         & Tip shape of arrows. See the PGF manual for the list of arrow tips. (Default: `\texttt{to}')\\
%     \bottomrule
%   \end{tabularx}}
% \end{table}
%
% \begin{table}[H]
%   \caption{Examples for the Clock Arrows.}\label{tab:clockarrows}
%   \tikzset{timing/c/dual arrows}%
%   \centering
%   \ifprintversion
%   \begin{tabular}{lc}
%     \toprule
%       Settings (\texttt{timing/c/.cd}) & Resulting Diagram (\texttt{11\{C\}})           \\
%     \midrule
%       \texttt{rising arrows}           & \texttiming[timing/c/rising arrows]{10{C}}     \\
%       \texttt{falling arrows}          & \texttiming[timing/c/falling arrows]{10{C}}    \\
%       \texttt{no arrows}               & \texttiming[timing/c/no arrows]{10{C}}         \\
%       \texttt{dual arrows}             & \texttiming[timing/c/dual arrows]{10{C}}       \\
%     \midrule
%       \texttt{arrow pos=.5}            & \texttiming[timing/c/arrow pos=.5]{10{C}}      \\
%       \texttt{arrow tip=latex}         & \texttiming[timing/c/arrow tip=latex]{10{C}}   \\
%       \texttt{arrow tip=stealth}       & \texttiming[timing/c/arrow tip=stealth]{10{C}} \\
%     \bottomrule
%   \end{tabular}
%   \else
%   \begin{minipage}[t]{.49\textwidth}
%    \vspace{0pt}\centering
%    \begin{tabular}{lc}
%      \toprule
%        Settings (\texttt{timing/c/.cd}) & Resulting Diagram (\texttt{11\{C\}})           \\
%      \midrule
%        \texttt{rising arrows}           & \texttiming[timing/c/rising arrows]{10{C}}     \\
%        \texttt{falling arrows}          & \texttiming[timing/c/falling arrows]{10{C}}    \\
%        \texttt{no arrows}               & \texttiming[timing/c/no arrows]{10{C}}         \\
%        \texttt{dual arrows}             & \texttiming[timing/c/dual arrows]{10{C}}       \\
%      \bottomrule
%    \end{tabular}
%   \end{minipage}
%   \hfill
%   \begin{minipage}[t]{.49\textwidth}
%    \vspace{0pt}\centering
%    \begin{tabular}{lc}
%      \toprule
%        Settings (\texttt{timing/c/.cd}) & Resulting Diagram (\texttt{11\{C\}})           \\
%      \midrule
%        \texttt{arrow pos=.7}            & \texttiming[timing/c/arrow pos=.7]{10{C}}      \\
%        \texttt{arrow pos=.4}            & \texttiming[timing/c/arrow pos=.4]{10{C}}      \\
%        \texttt{arrow tip=latex}         & \texttiming[timing/c/arrow tip=latex]{10{C}}   \\
%        \texttt{arrow tip=stealth}       & \texttiming[timing/c/arrow tip=stealth]{10{C}} \\
%      \bottomrule
%    \end{tabular}
%   \end{minipage}
% \end{table}
%
% \clearpage
% \library{columntype}{Column Type}
% The library "columntype"\NewIn{v0.7} uses the \pkg{array} package to define a new 
% |tabular| column type for timing characters. The tabular column can then hold 
% timing characters like the \env{tikztimingtable}. An initial option block 
% "[...]" is taken as initial character or diagram line wide settings. The 
% main difference between these two table types is that \env{tikztimingtable} 
% creates a big common |tikzpicture| with one coordinate system with potential 
% extra drawing code and the column type creates single pictures for each 
% diagram line without common coordinate system.
%
% By default the letter "T" and left alignment are used for the timing column 
% type. The TikZ style |timing/columntype=|\marg{letter}\marg{alignment} can be 
% used to select different column letters and alignments. The \meta{alignment} 
% represents here the \emph{real} column letter like "l", "c" or "r".  
% Additional column code can be added using the `|>|\marg{code}' and 
% `|<|\marg{code}' argument of the \pkg{array} package.
% \iffalse
% \noindent\textbf{Examples:}\\
% \hspace*{1em}|\tikzset{timing/columntype={T}{r}}| \hspace{2em} Change to right 
% alignment.\\
% \hspace*{1em}|\tikzset{timing/columntype={I}{>{code}c<{code}}| \hspace{2em} 
% Defines letter "I" to create centered timing diagrams with some code before   
% and after it.
% \fi
%
% More complex column types (\eg one holding |@{..}| or |!{..}| arguments, 
% multiple columns, etc.) must be defined manually using \pkg{array}'s 
% \cs{newcolumntype} macro.  The default definition is equal 
% to\\[\smallskipamount]
% \hspace*{2em}|\newcolumntype{T}{>{\celltiming}l<{\endcelltiming}}|.
%
% \smallskip
% The default "T" definition can be suppressed by using either the library 
% option |notype| or a different |type| (equal to |timing/columntype|):
% \\[\smallskipamount]
% \hspace*{2em}|\usetikzlibrary[type={U}{l}]{columntype}|\\[\smallskipamount]
%
% \bigskip
% \begin{figure}
% \makeatletter
% \let\tikztiming@embedexample\empty
% \makeatother
% \centering
% \begin{examplecode}
%   \begin{tabular}{lcT}
%     \toprule
%       Description & Num & \multicolumn{1}{l}{Signal} \\
%     \midrule
%       Example  & A  &           HLHZLHL \\
%       Example  & B  & [][green] HLHZLHL \\
%       Example  & C  &   [green] HLHZLHL \\
%     \bottomrule
%   \end{tabular}
% \end{examplecode}
% \caption{Example for use of Timing Column Type.}\label{exa:columntype}
% \end{figure}
%
% \clearpage
% \library{nicetabs}{Nice Timing Tables}
% The library "nicetabs"\NewIn{v0.7} uses the settings of the \pkg{array} and 
% \pkg{booktabs} packages to improve the layout of \env{tikztimingtables}. The 
% resulting table matches a \env{tabular}|{rT}| table which uses the above 
% packages and the |columntype| library.  The table macros \cs{tabcolsep}, 
% \cs{arraystretch} and \cs{extrarowheight} are obeyed.
% 
% The original table layout is designed to produce integer coordinates between 
% the rows to simplify the drawing of extra drawings (see \cs{extracode}). The 
% improved layout will cause non-integer coordinates, but in-line nodes and the 
% \cs{rowdist} and \cs{coldist} macros can be used to draw extra material 
% relatively to the rows.
% 
% The TikZ styles "timing/nice tabs" (default) and "timing/no nice tabs" can 
% be used to activate and deactivate the layout, \eg for each table. Both 
% settings can be given as a library option, but without the "timing/" path.
% 
% \bigskip
% \begin{table}[H]
%   \caption{Timing tables using `nice' (left) and normal (right) Layout. For 
%   comparison a \texttt{\{tabular\}\{rT\}} table is placed in grey below the 
%   left table.}^^A
%   \hbox{}\hfill
%   \begin{tikzpicture}
%     \node [gray] {\sffamily\ignorespaces
%   \begin{tabular}{rT}
%    \toprule
%     Name    & \multicolumn{1}{l}{Timing} \\
%    \midrule
%     Example & HLHLLHHH \\
%     Example & HLHLLHHH \\
%     Example & HLHLLHHH \\
%    \bottomrule
%   \end{tabular}^^A
%     };
%     \node {\ignorespaces
%   \begin{tikztimingtable}[timing/nice tabs]
%     Example & HLHLLHHH \\
%     Example & HLHLLHHH \\
%     Example & HLHLLHHH \\
%    \extracode
%     \tableheader{Name}{Timing}
%     \tablerules
%   \end{tikztimingtable}^^A
%     };
%  \end{tikzpicture}
%  \hfill
%   \begin{tikztimingtable}[timing/no nice tabs]
%     Example & HLHLLHHH \\
%     Example & HLHLLHHH \\
%     Example & HLHLLHHH \\
%    \extracode
%     \tableheader{Name}{Timing}
%     \tablerules
%   \end{tikztimingtable}
%   \hfill\hbox{}^^A
% \end{table}
%
% \clearpage
% \library{counters}{Counter Character}
% The library "counters"\NewIn{v0.7}\ExtIn{v0.7a} allows the easy definition 
% of meta-characters which 
% implement up-counters. These characters show the counter value using the 
% "D{}" character and increment it by one every time they are used. A second 
% character can be defined to (re-)set the counter to a specific value. The 
% counter values can be decimal (base-10, default), hexadecimal (base-16) or any 
% other base from 2 to 36. By default the lower case version of the counter 
% character is defined to produce the same output only with half the width.
%
% Counter characters are defined using the TikZ key
% `\texttt{timing/counter/new=}\linebreak[3]\ignorespaces
% \texttt{\{char=\meta{char},\meta{settings}\}}' which can also be written as 
% "timing/new counter". The \meta{settings} are TikZ keys themselves and are 
% shown by Table~\ref{tab:countersettings}.  One or more "new" keys (path 
% "timing/counter" removed) can be given as library options.
% The counter values are global like normal \LaTeX\ counters. They should be reset 
% in every timing diagram line before they are used.
%
% \subsubsection*{Counter Style}
% The styles `\texttt{timing/counter/\meta{char}}' and `\texttt{timing/counter/\meta{char}/text}'
% (both initially empty) are used to format the graphic and text style of this counter, respectively.
% Because the "D{}" character is used internally the above styles need to change the corresponding
% "D" styles. This changes are only local and do not affect further `real' "D" characters.
%
% The settings "fg style", "bg style" and "text style" can be used to quickly define the
% foreground (\ie line), background and text style of the counter. While the "text style" setting
% simple sets the `\texttt{timing/counter/\meta{char}/text}' style, the other two are a shortcut for
% \\*[\smallskipamount]
% |   \tikzset{timing/counter/|\meta{char}|/.style={%|\\
% |     timing/d/.style=|\marg{fg style}|,|\\
% |     timing/d/background/.style=|\marg{bg style}|,|\\
% |   }}|\\
% \par\medskip
%
% \subsubsection*{Additional Macros}
% \DescribeMacro\tikztimingcounter{<char>}
% \DescribeMacro\tikztimingsetcounter{<char>}{<pgfmath expression>}
% The value of counter \meta{char} can be read or set using this macros.
%
% \begin{table}
%   \caption{Settings for Counter Meta-Characters}
%   \label{tab:countersettings}
%   \makebox[\textwidth][c]{%
% \begin{tabularx}{1\textwidth}{>{\ttfamily}lX}
%  \toprule
%   \multicolumn{1}{l}{Key name} & Description \\
%  \midrule
%   char=\meta{char} & Defines the given \meta{char} to be a counter 
%     meta-character. If it is a upper case character the lower case character 
%     will produce the same output but with the half width, as long this is not 
%     overwritten with the \texttt{half with char} key. \\
%   half width char=\meta{char} & Defines the given \meta{char} to be the half 
%     width version of the counter value. By default this is the lower case 
%     version of the counter character given with \texttt{char}. An empty value 
%     for \meta{char} deactivates the definition of a half width character.\\
%   reset char=\meta{char}
%    & Defines the given \meta{char} to (re-)set the counter value to the `width'
%      of the character, \ie the number preceding it. The lower case version of
%      the reset \meta{char} is not defined.\\
%   reset type=\small\meta{width\Pipe arg\Pipe both\Pipe Both}
%    & Defines the type of the reset character, \ie how the reset value is obtained.
%       {\begin{tabular}{>{\ttfamily}ll}
%         width & Width is reset value: `\meta{value}\meta{char}', \eg `\texttt{0R}'.
%           Value can not be negative.\\
%         arg & Reset value is provided as argument: `\meta{char}\marg{value}', \eg
%           `\texttt{R\{-1\}}'.\\
%         both & Uppercase \meta{char} is \texttt{width}-type, lowercase \meta{char}
%           is \texttt{arg}-type reset char.\\
%         Both & Lowercase \meta{char} is \texttt{width}-type, uppercase \meta{char}
%           is \texttt{arg}-type reset char.\\
%       \end{tabular}} \\
%   base=\meta{Num 2-36} & Defines the numeric base of the counter. If not used the
%     base 10 is used.\\
%  \midrule
%   increment=\small\meta{pgfmath expression} &
%                              Sets the increment which is added every time the counter
%                              character is used. This can be a formula which result is truncated 
%                              to a integer. The current counter value can be referenced as \cs{N}.
%                              The increment can be negative which causes
%                              the counter to count down. Default: 1\\
%   max value=\small\meta{pgfmath expression} & Sets the maximum counter value. Default: \emph{not set}\\
%   min value=\small\meta{pgfmath expression} & Sets the minimum counter value. Default: \emph{not set}\\
%   wraps=\meta{true\Pipe false} &
%                              If set to \texttt{true} the counter wraps around, \ie it counts to the
%                              minimum value when counting over the maximum value or the other
%                              way around if \texttt{increment} is negative.
%                              Initial value: \texttt{false}. Default value: \texttt{true}\\
%  \midrule
%   bg style=\meta{TikZ style(s)}   & Sets the background style of the counter.\\
%   fg style=\meta{TikZ style(s)}   & Sets the foreground (line etc.) style of the counter.\\
%   text style=\meta{TikZ style(s)} & Sets the text style of the counter.\\
%   text format=\meta{\TeX code}    & Sets the format code of the counter value. This should be a macro which receives
%                                     the counter value as first argument.\\
%  \bottomrule
% \end{tabularx}
% }%
% \end{table}
%
% \subsubsection*{Examples:}
% \noindent
% Counter character "Q" with base 16 (hexadecimal). "R" resets the 
% counter. The counter value should be in blue text typer font.\\*\noindent
% |\tikzset{timing/new counter={char=Q,base=16,reset char=R}}|\\*\noindent
% |\tikzset{timing/counter/Q/text/.style={font=\ttfamily,blue}}|\\*\noindent
% |\texttiming{ 3Q 3{Q} R 12{Q} 2R Q qq 0R 3{Q} }|\\*\noindent
% gives: {%
% \tikzset{timing/counter/new={char=Q,base=16,reset char=R}}%
% \tikzset{timing/counter/Q/text/.style={font=\ttfamily,blue}}%
% \texttiming{ 3Q 3{Q} R 12{Q} 2R Q qq 0R 3{Q} }%
% }%
%
% \par\bigskip
% \noindent
% \begingroup
% \lstset{basicstyle=\ttfamily\scriptsize}
% \begin{examplecode}
%    \scalebox{2}{%
%    \begin{tikztimingtable}
%      Dec & [timing/counter/new={char=Q,max value=15,                 wraps,text style={font=\scriptsize}}]    33{Q}d\\
%      Bin & [timing/counter/new={char=Q,max value=15,base=2, digits=4,wraps,text style={font=\tiny,scale=.8}}] 33{Q}d\\
%      Oct & [timing/counter/new={char=Q,max value=15,base=8, digits=3,wraps,text style={font=\tiny}}]          33{Q}d\\
%      Hex & [timing/counter/new={char=Q,max value=15,base=16,         wraps,text style={font=\scriptsize}}]    33{Q}d\\
%     \extracode
%      \begin{background}[shift={(0.1,0)},help lines]
%        \vertlines{}
%      \end{background}
%    \end{tikztimingtable}
%    }%
% \end{examplecode}
% \endgroup
%
% \clearpage
% \library{advnodes}{Advanced Nodes}
% The library "advnodes"\NewIn{v0.7} changes the in-line nodes, \ie the "N" character, to provide multiple 
% transition dependent and independent node anchors shown in Table~\ref{tab:advanchor}.
%
% Most transitions provide the three logic level anchors "low", "mid" and "high" which lie
% on the drawn timing line. Transitions of `double line' characters like "D" can have
% two low and/or high level anchors which are called "low2" and "high2".
%
% To align marker lines over multiple rows inside a \env{tikztimingtable} a set of transition
% independent node anchors are provided: "LOW", "MID", "HIGH". This anchors lie all at the begin of the 
% transition at the appropriate logic levels.
% With the normal coordinate-like in-line nodes the vertical node (center) position has to be taken into 
% account, while this advanced anchors do this automatically.
%
% Often marker lines should start and end with a little vertical offset from the timing diagram to be more distinguishable.
% For this the two anchors "TOP" and "BOTTOM" exist. They lie above and below of "HIGH" and "LOW", respectively.
% The vertical distance can be set using `|timing/nodes/offset=|\meta{dim.~or number}'.
%
% \textbf{Please note} that the node \emph{center} of advanced nodes will be different for some transitions.
% The "center" anchor (used by default if no node anchor is provided) will be placed at the logical center of the 
% transition, \ie mostly in the middle of it. In order not to break existing diagrams which use nodes as references 
% to draw additional material an TikZ styles is provided to select between the old and new node centers. This styles
% can be used globally or locally as required. The two explicit anchors "new center" and "old center" are also 
% provided. For existing documents with diagrams using normal nodes it is recommended
% to switch to simple nodes or to the old node centers globally or select such a style for each old diagram.
%
% The following TikZ settings can be used with the library. The node style settings affect all new nodes after them.
% The center settings affect all following \emph{references} (\eg "(NodeName)" or "(NodeName.center)") of advanced nodes.
% The settings can be used as library options with the "timing/nodes/" part removed.
%
% \vfill\noindent
% \begin{tabularx}{\textwidth}{lX}
%  \toprule
%   TikZ Setting              &  Description \\
%  \midrule
%   |timing/nodes/advanced|   &  Selects advanced in-line nodes. (library default) \\
%   |timing/nodes/simple|     &  Selects simple coordinate-style in-line nodes. (package default) \\
%  \midrule
%   |timing/nodes/new center| &  Center of in-line nodes is in the new position. (default for |advanced|) \\
%   |timing/nodes/old center| &  Center of in-line nodes is in the old position. (always on for |simple|) \\
%  \midrule
%   |timing/nodes/offset|     &  Sets offset for |TOP| and |BOTTOM| anchors.
%                                Can be a dimension with unit or a factor to the current |y| unit. (default: 0.25) \\
%  \bottomrule
% \end{tabularx}
%
% \subsection*{Examples:}
% "\usetikztiminglibrary[simple]{advnodes}"\ \ loads the library with nodes default to the old "simple" style.\\\noindent
% "\usetikztiminglibrary[old center]{advnodes}"\ \ \ loads the library with advanced nodes with have the center at the same
% place as the normal simple nodes. This is a good ``compatibility mode'' for existing pre-v0.7 diagrams.\\\noindent
% "\begin{tikztimingtable}[timing/nodes/simple]"\ \ starts a timing table which uses simple nodes. \\\noindent
% "\begin{tikztimingtable}[timing/nodes/.cd,advanced,old center]"\\\noindent starts a timing table which uses advanced nodes with old node centers.
%
%
% \begin{table}
%   \caption{Transition Dependent Anchor Points of Advanced Nodes}^^A
%   \label{tab:advanchor}^^A
%   \makeatletter
%   \let\mid\relax
%^^A
%   \tikzset{timing/draw grid}^^A
%   \tikzset{timing/.cd,
%     lslope=.5,
%     dslope=.5,
%     zslope=.5,
%   }^^A
%^^A
%   \tikzset{^^A
%     marker/.style={opacity=.8,fill opacity=.8},
%     low/.style={marker,draw=red},
%     mid/.style={marker,draw=yellow},
%     high/.style={marker,draw=green},
%     low2/.style={marker,draw=red!50!black},
%     high2/.style={marker,draw=green!50!black},
%     LOW/.style={marker,CAPITAL,draw=red},
%     MID/.style={marker,CAPITAL,draw=yellow},
%     HIGH/.style={marker,CAPITAL,draw=green},
%     CAPITAL/.style={marker,mark=x,mark size=2\pgflinewidth},
%     origin/.style={marker,mark=x,draw=gray},
%     center/.style={marker,mark=+,draw=black},
%     old center/.style={marker,mark=+,mark size=2\pgflinewidth,draw=blue},
%     new center/.style={marker,mark size=\pgflinewidth,mark=*,fill=blue,draw=blue},
%   }^^A
%^^A
%   \newcommand\namark[2][]{^^A 
%     \draw[mark=o,mark size=3\pgflinewidth,#1] plot coordinates {(#2)}
%   }^^A
%^^A
%   \def\NodeAnchors#1#2{^^A 
%     \Huge
%     {\raisebox{-.5ex}{\scalebox{1}{^^A 
%     \begin{tikzpicture}[timing/picture,line width=.5\pgflinewidth]
%       \let\pgf@sh@ma@A\@undefined
%       \timing [name=T] at (0,0) {#1 N(A) #2};
%       \useasboundingbox (T.south west) rectangle (T.north east);
%       ^^A 
%       \pgf@sh@ma@A
%       \anchorpoints
%       \foreach \a in {LOW,MID,HIGH,old center,new center} {^^A 
%         \namark[gray,\a]{A.\a};^^A 
%       }^^A 
%       \foreach \a in {low,mid,high,low2,high2} {^^A 
%         \@ifundefined{\a}{}^^A 
%           {\namark[semithick,\a]{A.\a};}^^A 
%       }^^A 
%       \namark[old center,blue]{A.old center};^^A 
%     \end{tikzpicture}^^A 
%     }}}^^A 
%     \Huge\strut
%   }^^A
%^^A
%   \def\bchars{,L,H,X,D,E}^^A 
%   \def\achars{,L,H,X,D,D{A},E,0LE,0HE}^^A 
%^^A
%   \def\DA{D{A}}^^A
%   \def\tablecontent{^^A 
%     \begin{tabular}{lrrrrrrrrrrrrrrrrrr}^^A
%       \toprule
%         \tikzpicture
%           \useasboundingbox (-1ex,-.5ex) rectangle (1ex,.5ex);
%           \node {^^A 
%         \scalebox{0.4}^^A 
%           {\rotatebox{-45}{$\frac{\mbox{\rotatebox{45}{to}}}^^A 
%           {\mbox{\rotatebox{45}{from}}}$}}^^A 
%         };^^A 
%       \endtikzpicture
%   }
%^^A
%   \def\atc#1{^^A 
%     \begingroup
%       \edef\@tempa{#1}^^A 
%       \expandafter\g@addto@macro\expandafter\tablecontent
%       \expandafter{\@tempa}^^A 
%     \endgroup
%   }^^A 
%^^A
%   \foreach \b in \bchars {^^A 
%     \atc{& \noexpand\multicolumn{1}{c}{\b}}^^A 
%   }^^A
%^^A
%   \atc{\noexpand\\ \noexpand\midrule}^^A
%^^A
%   \foreach \a in \achars {^^A 
%     \ifx\a\DA
%       \atc{D\noexpand\{A\noexpand\}}^^A 
%     \else
%       \atc{\a}^^A 
%     \fi
%     \foreach \b in \bchars {^^A 
%       \atc{& \noexpand\NodeAnchors{\a}{\b}}^^A 
%       ^^A \hbox to 5em{\llap{\a\b:} \NodeAnchors{\a}{\b}}^^A 
%     }^^A
%     \atc{\noexpand\\}^^A
%   }^^A
%^^A
%   \atc{^^A 
%     \noexpand\bottomrule
%     \noexpand\end{tabular}^^A 
%   }^^A
%^^A
%   \sffamily\centering
%   \tablecontent
%^^A
%   \def\legend#1#2{^^A
%     \namark[#1]{[shift={(0,.125ex)}]#2}; \node [right] at (#2) {\normalsize #1\strut}
%   }^^A
%   \par
%   \Huge
%   \begin{tikzpicture}[timing/picture,line width=.5\pgflinewidth]
%     \node [left] at (-1,0) {\large Legend:};
%     \legend{low}{0,0};
%     \legend{mid}{3,0};
%     \legend{high}{6,0};
%     \legend{low2}{0,-.8};
%     \legend{high2}{6,-.8};
%     \legend{LOW}{0,-1.6};
%     \legend{MID}{3,-1.6};
%     \legend{HIGH}{6,-1.6};
%     \legend{new center}{0,-2.4};
%     \legend{old center}{6,-2.4};
%   \end{tikzpicture}^^A
% \end{table}
%
% \clearpage
% \library{ifsym}{Compatibility Macros for \texttt{ifsym} package}
% The library "ifsym"\NewIn{v0.7} provides macros and timing styles to emulate the behaviour of the \pkg{ifsym} package
% when loaded with the |electronic| option. The \pkg{ifsym} package was an early inspiration to this package
% and this library allows the usage of |ifsym| style timing symbol macros and characters 
% (`|\textifsym|\marg{characters}' which uses `|\texttiming[timing/ifsym]|\marg{characters}')
% which are described in Table~\ref{tab:ifsymmacros} and Table~\ref{tab:ifsym}, respectively.
% This is useful if old |ifsym| timing diagrams should be reused. The |tikz-timing| replacements are a very close match
% and do not need a special font to be installed. The graphic quality should be equal or better than the original.
% The intermixing of |ifsym| and |tikz-timing| style characters in a \cs{textifsym} macro
% (the one provided by this library, not the one from the \pkg{ifsym} package) is supported but it is not guaranteed to work
% 100\% properly. Please note that the "M" character is defined to use "X" in black.
%
% The library can be loaded with one of the options "provide" (default), "new", "renew" or "off", respectively. 
% These select if the macros should be defined using \cs{providecommand}, \cs{newcommand}, \cs{renewcommand} 
% or not at all. This can be useful if the \pkg{ifsym} package is loaded beforehand.
%
%
%   \sbox{\sba}{%^^A
%    \begin{tabular}{lll}
%      \toprule
%        Macro               & Symbol          & Description (trivial) \\
%      \midrule
%        \cs{RaisingEdge}    & \RaisingEdge    & Raising Edge          \\
%        \cs{FallingEdge}    & \FallingEdge    & Falling Edge          \\
%        \cs{ShortPulseHigh} & \ShortPulseHigh & Short Pulse High      \\
%        \cs{ShortPulseLow}  & \ShortPulseLow  & Short Pulse Low       \\
%        \cs{PulseHigh}      & \PulseHigh      & Normal Pulse High     \\
%        \cs{PulseLow}       & \PulseLow       & Normal Pulse Low      \\
%        \cs{LongPulseHigh}  & \LongPulseHigh  & Long Pulse High       \\
%        \cs{LongPulseLow}   & \LongPulseLow   & Long Pulse Low        \\
%      \bottomrule
%    \end{tabular}%^^A
%   }%^^A
%   ^^A
%   \sbox{\sbb}{%^^A
%     \begin{tabular}{>{\ttfamily}lll}
%       \toprule
%         \normalfont
%         Character & Symbol                        & Description                                          \\
%       \midrule
%         l, h      & \textifsym{l}, \textifsym{h}  & Short low or high level signal.                      \\
%         L, H      & \textifsym{L}, \textifsym{H}  & Long low or high level signal.                       \\
%         \pipe     & \textifsym{|}                 & Transition/glitch between L/H or H/L levels.         \\
%         m, d      & \textifsym{m}, \textifsym{d}  & Short middle or double level signal.                 \\
%         M, D      & \textifsym{M}, \textifsym{D}  & Long middle or double level signal.                  \\
%         <, <<     & \textifsym{<}, \textifsym{<<} & Short or long slope between middle and double level. \\
%         >, >>     & \textifsym{>}, \textifsym{>>} & Short or long slope between double and middle level. \\
%       \bottomrule
%     \end{tabular}
%   }%^^A
% 
% \ifprintversion
%
% \begin{table}
%   \centering
%   \caption[ifsym style Timing Symbol Macros]{\texttt{ifsym} style Timing Symbol Macros}
%   \label{tab:ifsymmacros}
%   \usebox{\sba}%^^A
% \end{table}
%
% \begin{table}
%   \centering
%   \caption[ifsym style Timing Characters]{\texttt{ifsym} style Timing Characters (from \pkg{ifsym} manual)}
%   \label{tab:ifsym}
%   \usebox{\sba}%^^A
% \end{table}
%
% \else
%
% \begingroup
%   \expandafter\def\csname @captype\endcsname{table}%
%   \makebox[\textwidth][c]{%^^A
%     \let\tablecaption\caption
%     \begin{minipage}{\wd\sba}
%       \tablecaption[ifsym style Timing Symbol Macros]{\texttt{ifsym} style Timing Symbol Macros}%^^A
%       \label{tab:ifsymmacros}%^^A
%       \usebox{\sba}%^^A
%     \end{minipage}
%     \hspace{1em}\hfill
%     \begin{minipage}{\wd\sbb}
%       \tablecaption[ifsym style Timing Characters]{\texttt{ifsym} style Timing Characters (from \texttt{ifsym} manual)}%^^A
%       \label{tab:ifsym}%^^A
%       \usebox{\sbb}%^^A
%     \end{minipage}
%   }%^^A
% \endgroup
%
% \fi
%
% \clearpage
% \library{interval}{Intervals (experimental)}
% \textbf{This library is under development and might change in the future.}
%
% The library "interval"\NewIn{v0.7} allows the drawing of intervals using the "H", "L", "Z" and "X" logic levels.
% It provides modified definitions of X and Z transitions (e.g.~"LX", "XH") where the transition edges can 
% be coloured either way to indicate interval borders.
%
% The interval borders can be set using the `|timing/interval=|\marg{settings}' TikZ style. The 
% \meta{settings} are "lo" (left-open), "lc" (left-closed) and "ro" (right-open), "rc" (right-closed), which build the
% following combinations: "{lo,ro}", "{lc,ro}", "{lo,rc}" and "{lc,rc}". However, every of them can also be set on its own, 
% e.g.\ "timing/interval={lc}" simply sets the interval to `left-closed' without changing the right border.
%
% The key "timing/interval/normal" (alternative: "timing/interval={normal}") sets the transitions back to their default.
%
% \par\medskip
% \noindent
% Examples:
% \begingroup
% \tikzset{timing/draw grid,timing/slope=0}\sffamily
% \newcommand*\ttt[2][]{\texttiming[timing/interval={#1}]{#2}}
% \newcommand*\extab[1]{%
%     \begin{tabular}{rcc}
%          & ro              & rc              \\
%       lo & \ttt[lo,ro]{#1} & \ttt[lo,rc]{#1} \\
%       lc & \ttt[lc,ro]{#1} & \ttt[lc,rc]{#1} \\
%     \end{tabular}
% }
% \extab{zlzlz}
% \endgroup
% \par\bigskip
%
% \noindent
% A meta-character can be defined for quick changes if necessary:\\[\smallskipamount]
% \begingroup
% |\tikztimingmetachar{Y}{[timing/interval={lo,ro}]}|\\
% |\tikztimingmetachar{y}{[timing/interval/normal]}|\\
% |\texttiming[timing/draw grid]{ LZH Y HZH y LZH Y LZL } |\\
% |\texttiming[timing/draw grid]{ LXH Y HXH y LXH Y LXL } |\\
% gives:
% \tikztimingmetachar{Y}{[timing/interval={lo,ro}]}
% \tikztimingmetachar{y}{[timing/interval/normal]}
% \texttiming[timing/draw grid]{ LZH Y HZH y LZH Y LZL }
% \texttiming[timing/draw grid]{ LXH Y HXH y LXH Y LXL } \\
% \endgroup
%
% \par\bigskip
% \noindent
% A further alternative is to use a meta-character with an argument.
% Note that "#2" must be used, because "#1" holds the width, which is irrelevant here.
% This definition is also provided by the `|timing/interval/metachar|=\meta{Character}' key.
% \\[\smallskipamount]
% \begingroup
% |\tikztimingmetachar{I}[2]{[timing/interval={#2}]}|\\
% or
% |\tikzset{timing/interval/metachar=I}|\\
% |\texttiming[timing/draw grid]{ LZH I{lo,rc} HZH I{ro} LZH I{normal} LZL } |\\
% gives:
% \tikztimingmetachar{I}[2]{[timing/interval={#2}]}
% \texttiming[timing/draw grid]{ LZH I{lo,rc} HZH I{ro} LZH I{normal} LZL }\\
% \endgroup
%
% \clearpage
% \library{beamer}{Beamer Overlay Support (experimental)}
% \textbf{This library is under development and might change in the future.}
%
% The library "beamer"\NewIn{v0.7} provides (at the moment) marginal support for overlays of the \pkg{beamer} class.
% It allows the usage of beamer \emph{overlay specifications} (|<|\meta{spec}|>|) inside the timing string.
% However, the current simple implementation might not work properly and cause strange results.
% The support is designed for use inside \env{tikztimingtable}.
% See the \pkg{beamer} manual for more informations about overlays specifications.
%
% \subsection*{Usage}
% Insert an overlay specification, e.g.~|<|\meta{number}|>|, inside the timing string. It will affect the 
% rest of the timing characters of the current diagram line.
% Unfortunate due to the global nature of overlays it also affects the rest of the table. Therefore all 
% diagram lines should end with a specification which turns overlays off, i.e. |<*>| or |<0->|. Otherwise
% strange results can occur (e.g.~wrong/missing background graphics).
%
% \paragraph*{Example}
% | Timing Table Row  &  H L <2> Z L H <3> D{last} Z <*>  \\ |
%
% \subsection*{Display Rows Stepwise}
% The rows of a \env{tikztimingtable} can be uncovered row-by-row using the way shown below. The signal names
% must be enclosed into a \cs{mbox} because \cs{uncover} needs to be inside horizontal mode. Instead
% of \cs{uncover} another beamer overlay command like \cs{only} or \cs{invisible} can be used. To highlight
% the signal name use \cs{alert}|<|\meta{\dots}|>|\marg{signal name} inside \cs{uncover}.
% At the moment there is no simple way to highlight the timing lines.
%
% \begin{lstlisting}[gobble=4]
%   \begin{tikztimingtable}
%     \mbox{\uncover<+->{Signal Name 1}} & <.-> HL <*> \\
%     \mbox{\uncover<+->{Signal Name 2}} & <.-> HL <*> \\
%       % ...
%     \mbox{\uncover<+->{Signal Name n}} & <.-> HL <*> \\
%   \end{tikztimingtable}
% \end{lstlisting}
%
% \subsection*{Display Columns Stepwise}
% Different sections (`columns') of timing diagrams in a \env{tikztimingtable} can be uncovered stepwise using the way shown below.
% In this example the second section/column will be uncovered in the second slide. The first is always visible. Further sections/columns
% can be uncovered in further slides. 
%
% Please note that the total width of the table is constant and e.g.~\cs{tablerules} will always cover the full width independent of overlays.
%
% \begin{lstlisting}[gobble=4]
%   \begin{tikztimingtable}
%     Signal Name 1  &  10D{Sec. 1} <2> 10D{Sec. 2} <*> \\
%     Signal Name 2  &  10D{Sec. 1} <2> 10D{Sec. 2} <*> \\
%       % ...
%     Signal Name n  &  10D{Sec. 1} <2> 10D{Sec. 2} <*> \\
%   \end{tikztimingtable}
% \end{lstlisting}
%
% \subsection*{Overlay Extra Code}
% The |beamer| overlay specifications can be used inside the \cs{extracode} section like in a normal \env{tikzpicture} environment.
% However, in both cases strange results will occur if the end of the environment is hidden by an overlay specification. Due to this
% reason it is recommended to only use overlay commands which affect only an argument, like \cs{only}|<|\meta{\dots}|>|\marg{code},
% or to place a plain \cs{onlayer} before the end of the environment.
%
% \begin{lstlisting}[gobble=4]
%   \begin{tikztimingtable}
%     Signal Name 1  &  10D{Sec. 1} <2> 10D{Sec. 2} <*> \\
%       % ...
%    \extracode
%      % either
%       \draw<2> (0,0) -- (2,0);  \only<3> { ... }
%      % or
%       \onlayer<2>
%      % and then at the very end:
%       \onlayer   % or \onlayer<*>
%   \end{tikztimingtable}
% \end{lstlisting}
%
% \clearpage
% \chapter{Examples}
% This section shows some examples by putting either the full source code or 
% only the needed characters beside the graphical result. Please note that the 
% displayed syntax is the one of \cs{timing} where the initial character is 
% declared as optional argument (\oarg{char}) \emph{inside}/\emph{together} with 
% the logic characters. The syntax of \cs{textttiming} is identical except the 
% initial character is given as a normal optional argument before the characters 
% argument.
% All examples except Example~\ref{exa:adv} are attached in compilable form
% to this PDF.
% ^^A\\[\smallskipamount]
% ^^A\emph{Example notation example:} "[c]cccc" means either 
% ^^A|\timing{[c]cccc};| or |\texttiming[c]{cccc}|.
%
% \DeleteShortVerb{\|}
% \begin{exampletable}
% \centering
% \makeatletter
% \let\tikztiming@embedexample\empty
% \makeatother
% \pgfmathsetbasenumberlength{2}
% \newcounter{hexcount}
% \addtocounter{hexcount}{+8}%
% \newcommand{\hexcountmacro}{%
%   \pgfmathdectoBase\thehexcount{\value{hexcount}}{16}%
%   \hbox{\thehexcount}%
%   \addtocounter{hexcount}{+1}%
% }
% \caption{Initial Characters, Modifiers, TikZ Keys}\label{exa:adv}
% \def\somemacro{}%
% \def\code{}%
% \makebox[\textwidth][c]{%
% \begin{tikztimingexampletable}
%   HLZXDUTC \\
%   cccc \\
%   tttt \\
%   [c]cccc \\
%   4{c} \\
%   4c4c \\
%   4{1.8c} \\
%   [d] 4{5D{Text}} 0.2D \\
%   3.4H 0.6L \\
%   DDDUUUDDD \\
%   DDD{}DUUDD \\
%   8{2D{\hexcountmacro}} \\
%   3{2{0.25X 2.5D .25Z}} \\
%   DDD{} 3{0.2D{}} DDD \\
%   DDD{} 3{0.2D{}} 0.4D{} 0.6D{} DDD \\
%   HHHLLH SSSS HLLHHL \\
%   HHGHHGGHHLLGLLGH \\
%   ZZ G ZZ G XX G X \\
%   LLL 2{0.1H 0.1L} 0.6H HH \\
%   LLL [timing/slope=0.05] 4{.05H .05L} 0.6H HH\\
%   LLL 0.4U 0.6H HH \\
%   [L][timing/slope=1.0] HL HL HL HL HL \\
%   LLLLL !{-- +(.5,.5) -- ++(1,0)} HHHHHH \\
%   LLLLL [| /utils/exec={\somemacro\code} |] HHHHHH \\
%   LL [green] HH [brown] XX LL ZZ [orange] HH \\
%   [][line width=1pt] HLXZDU  [line width=0.1pt] HLXZDU \\
%   [][line width=1pt] HLXZDU ,[line width=0.1pt] HLXZDU \\
%   [][line width=1pt] HLXZDU ;[line width=0.1pt] HLXZDU \\
% \end{tikztimingexampletable}%
% }
% \\\scriptsize
% \emph{Note:} Optional argument must be placed before macro argument if 
% \cs{texttiming} is used.
% 
% \end{exampletable}
% \MakeShortVerb{\|}
%
% \begin{example}
% \centering
% \begin{examplecode}
%   \begin{tikztimingtable}
%     Name     & hLLLLh     \\
%     Clock    & 10{c}      \\
%     Signal   & z4D{Text}z \\
%   \end{tikztimingtable}
% \end{examplecode}
% \caption[tikztimingtable without extracode]{\env{tikztimingtable} without \cs{extracode}.}
% \end{example}
%
% \begin{example}
% \centering
% \begin{examplecode}
%   \begin{tikztimingtable}
%     Name     & hLLLLh     \\
%     Clock    & 10{c}      \\
%     Signal   & z4D{Text}z \\
%    \extracode
%     \draw (0,0) circle (0.2pt); % Origin
%     \begin{pgfonlayer}{background}
%       \vertlines[help lines]{0.5,4.5}
%     \end{pgfonlayer}
%   \end{tikztimingtable}
% \end{examplecode}
% \caption[tikztimingtable with extracode]{\env{tikztimingtable} with \cs{extracode}.}
% \end{example}
%
% \begin{example}
% \centering
% \begin{examplecode}
%   \begin{tikzpicture}[x=4cm,y=4cm]
%     \draw (0,0) rectangle (1,1);
%     \draw (0.2,0.7) circle (10pt);
%     \begin{scope}[green]
%       \draw (0.1,0.1) -- +(0.8,0.2);
%       \timing at (0.3,0.4) {hlzhhlhhl};
%     \end{scope}
%     \timing [rotate=-30]
%         at (0.4,0.7) {HLZHHLHHL};
%   \end{tikzpicture}
% \end{examplecode}
% \caption[timing inside general tikzpicture]{\cs{timing} inside general \envv{tikzpicture}.}
% \end{example}
%
% \begin{example}
% \centering
% \def\exampleextraheader{\noexpand\usetikztiminglibrary{advnodes}}
% \begin{examplecode}
%   \Huge
%   \begin{tikzpicture}[timing/picture,thick,
%       timing/nodes/advanced]
%     \timing at (0,2) {hH N(A) LHLHL};
%     \timing[timing/slope=.25] at (0,0)
%       {HLL N(B) HHLl};
%     \draw [orange,semithick]
%       (A.mid) ellipse (.2 and .6)
%       (B.mid) ellipse (.2 and .6);
%     \draw [orange,semithick,->]
%       ($ (A.mid) - (0,.6) $)
%       parabola [bend pos=0.5]
%       ($ (B.mid) + (0,.6) $);
%   \end{tikzpicture}
% \end{examplecode}
% \caption{Using In-Line Nodes to draw Relationships.}\label{exa:inlinenodes}
% \end{example}
%
% \begin{example}
% \centering
% \lstset{basicstyle=\scriptsize\ttfamily}
% \begin{examplecode}
%   \def\degr{${}^\circ$}
%   \begin{tikztimingtable}
%     Clock 128\,MHz 0\degr    &  H    2C  N(A1) 8{2C} N(A5) 3{2C} G \\
%     Clock 128\,MHz 90\degr   & [C] 2{2C} N(A2) 8{2C} N(A6) 2{2C} C \\
%     Clock 128\,MHz 180\degr  &  C  2{2C} N(A3) 8{2C} N(A7) 2{2C} G \\
%     Clock 128\,MHz 270\degr  &     3{2C} N(A4) 8{2C} N(A8)   2C  C \\
%     Coarse Pulse             & 3L 16H 6L                           \\
%     Coarse Pulse - Delayed 1 & 4L N(B2) 16H N(B6) 5L               \\
%     Coarse Pulse - Delayed 2 & 5L N(B3) 16H N(B7) 4L               \\
%     Coarse Pulse - Delayed 3 & 6L 16H 3L                           \\
%                                                                    \\
%     Final Pulse Set          & 3L 16H N(B5) 6L                     \\
%     Final Pulse $\overline{\mbox{Reset}}$ & 6L N(B4) 16H 3L        \\
%     Final Pulse              & 3L N(B1) 19H N(B8) 3L               \\
%   \extracode
%     \tablerules
%     \begin{pgfonlayer}{background}
%       \foreach \n in {1,...,8}
%         \draw [help lines] (A\n) -- (B\n);
%     \end{pgfonlayer}
%   \end{tikztimingtable}
% \end{examplecode}
% \caption{Using In-Line Nodes to draw Marker Lines.}
% \end{example}
%
% \begin{example}
% \centering
% \lstset{basicstyle=\scriptsize\ttfamily}
% \def\exampleextraheader{\noexpand\usetikztiminglibrary{advnodes,either}^^J
% \noexpand\usepackage{amsmath}}
% \begin{examplecode}
%   \def\degr#1{\makebox[2em][r]{#1}\ensuremath{{}^{\circ}}}%
%   
%   \begin{tikztimingtable}[%
%       timing/dslope=0.1, timing/.style={x=2ex,y=2ex}, x=2ex,
%       timing/rowdist=3ex,
%       timing/name/.style={font=\sffamily\scriptsize},
%       timing/nodes/advanced,
%     ]
%     Clock \degr{90} &  l  2{C} N(A1) 5{C}  \\
%     Clock \degr{180}&  C  2{C} N(A2) 4{C} c\\
%     Clock \degr{270}&  h  3{C} N(A3) 4{C}  \\
%     Clock \degr{0}  & [C] 2{C} N(A0) 2{C} N(A4) 3{C}c ;[dotted] 
%                       2L; 2{C} N(A5) 3{C} \\
%     Start of T$_{\text{sw}}$ & 4S G N(start) \\
%     Input Pulse & 2.0L 2H 3.5L ;[dotted] 2L; 5L \\
%     Set 3       & 2.5L 2H 3.0L ;[dotted] 2L; 5L \\
%     Set 2       & 3.0L 2H 2.5L ;[dotted] 2L; 5L \\
%     Set 1       & 3.5L 2H 2.0L ;[dotted] 2L; 5L \\
%     Set 0       & 4.0L 2H 1.5L ;[dotted] 2L; 5L \\
%     Reset       & 7.5L ;[dotted] 2L; 2L N(reset) 2H 1L \\
%     Final Pulse & 2.5L N(B1) e N(B2) e N(B3) e 3.5H; [dotted]
%                   2H; 2H 3L \\
%     Counter     & N(x) 2D{Full} N(B0) 2D{0} N(B4) 2D{1} 1.5D;[dotted]
%                   .25D{2} 1.75D{};
%     2D{~D$_\text{M}$} N(B5) 2D{D$_\text{M}$+1} D N(y)\\
%   \extracode
%    %\tablegrid
%    \node[anchor=north] at ($ (x) ! .5 ! (y) - (0,.75) $)
%       {\scriptsize D$_\text{M}$ =  MSBs of Duty Cycle};
%    \begin{background}[timing/picture,line width=.06ex,color=black!60]
%      \foreach \n in {0,...,5}
%        \draw (A\n.TOP) -- (B\n.BOTTOM);
%    \end{background}
%   \end{tikztimingtable}%
% \end{examplecode}
% \caption{Adjusting Diagram Parameters and using Advanced In-Line Nodes to draw Marker Lines.}
% \end{example}
%
% \begin{example}
% \centering
% \lstset{basicstyle=\ttfamily\footnotesize}
% \def\texta{}\def\textb{}^^A
% \makeatletter
% \ifpdf
%   \IfFileExists{SR_FF_timing_diagram.png}{^^A
%   \def\texta{ (left top)}\def\textb{ (left bottom)}^^A
%   \def\PrintExample{%
%     \begingroup
%     \BoxExample
%     \par\noindent
%       \centerline{%^^A
%       \minipage[c]{\wd\exampleresultbox}%^^A
%         \hbox to \textwidth{\hss\fbox{\usebox{\exampleresultbox}}}%^^A
%         \vspace*{1cm}%^^A
%         \hbox to \textwidth{\hss\fbox{\includegraphics[width=27ex]{SR_FF_timing_diagram.png}}}%^^A
%       \endminipage%
%       \hfill
%       \minipage[c]{\wd\examplecodebox}
%         \hbox{\usebox{\examplecodebox}}
%       \endminipage%
%       }
%     \endgroup
%     \par
%   }%^^A
%   }{}%^^A
% \fi
% \begin{examplecode}
%   \definecolor{bgblue}{rgb}{0.41961,0.80784,0.80784}%
%   \definecolor{bgred}{rgb}{1,0.61569,0.61569}%
%   \definecolor{fgblue}{rgb}{0,0,0.6}%
%   \definecolor{fgred}{rgb}{0.6,0,0}%
%   \begin{tikztimingtable}[timing/slope=0,
%     timing/coldist=2pt,xscale=2.05,yscale=1.1,semithick]
%     \scriptsize clock & 7{C}\\
%     S & .75L h 2.25L H LLl [fgblue]\\
%     R & 1.8L .8H 2.2L 1.4H 0.8L [fgblue]\\
%     Q & L .8H 1.7L 1.5H LL\\
%     $\overline{\mbox{Q}}$ & H .8L 1.7H 1.5L HH\\
%     Q & LHHHHLL[fgred]\\
%     $\overline{\mbox{Q}}$ & HLLLLHH[fgred]\\
%   \extracode
%    \makeatletter
%    \begin{pgfonlayer}{background}
%     \shade [right color=bgblue,left color=white]
%        (7,-8.45) rectangle (-1,-4.6);
%     \shade [right color=bgred,left color=white]
%        (7,-12.8) rectangle (-1,-8.6);
%     \begin{scope}[gray,semitransparent,semithick]
%       \horlines{}
%       \foreach \x in {1,...,6}
%         \draw (\x,1) -- (\x,-12.8);
%       % similar: \vertlines{1,...,6}
%     \end{scope}
%     \node [anchor=south east,inner sep=0pt]
%       at (7,-8.45) {\tiny clocked};
%     \node [anchor=south east,inner sep=0pt,fgred]
%       at (7,-12.8) {\tiny positive edge triggered};
%    \end{pgfonlayer}
%   \end{tikztimingtable}%
% \end{examplecode}
% \caption[SR flip-flop timing diagram]{SR flip-flop timing diagram\texta.  
% Redrawn from image\textb\\ {\small
% \url{http://commons.wikimedia.org/wiki/File:SR_FF_timing_diagram.png}
% }}
% \end{example}
%
% \begin{example}
% \centering
% \lstset{basicstyle=\ttfamily\scriptsize}
% \begin{examplecode}
%   \begin{tikztimingtable}[
%      timing/d/background/.style={fill=white},
%      timing/lslope=0.2,
%      timing/counter/new={char=Q,reset char=R},
%     ]
%             CPOL=0 & LL 15{T} LL \\
%             CPOL=1 & HH 15{T} HH \\
%                    & H 17L H     \\
%     \\
%           Cycle \# & U     R 8{2Q} 2U    \\
%               MISO & D{z}  R 8{2Q} 2D{z} \\
%               MOSI & D{z}  R 8{2Q} 2D{z} \\
%     \\
%           Cycle \# & UU    R 8{2Q} U    \\
%               MISO & D{z}U R 8{2Q} D{z} \\
%               MOSI & D{z}U R 8{2Q} D{z} \\
%   \extracode
%     % Add vertical lines in two colors
%     \begin{pgfonlayer}{background}
%       \begin{scope}[semitransparent,semithick]
%         \vertlines[red]{2.1,4.1,...,17.1}
%         \vertlines[blue]{3.1,5.1,...,17.1}
%       \end{scope}
%     \end{pgfonlayer}
%     % Add big group labels
%     \begin{scope}
%       [font=\sffamily\Large,shift={(-6em,-0.5)},anchor=east]
%       \node at (  0, 0) {SCK};    \node at (  0,-3 ) {SS};
%       \node at (1ex,-9) {CPHA=0}; \node at (1ex,-17) {CPHA=1};
%     \end{scope}
%   \end{tikztimingtable}%
% \end{examplecode}
% \caption[SPI Interface Timing]{SPI Interface Timing.  Redrawn from 
% image {\small
% \url{http://en.wikipedia.org/wiki/File:SPI_timing_diagram.svg}}}
% \end{example}
%
% \clearpage
% \StopEventually{}
% \clearpage
% \iffalse
%</doc>
% \fi
%
% \chapter{Implementation}
% \section{Package Header}
% \iffalse
%<*package>
% \fi
%    \begin{macrocode}
\RequirePackage{tikz}

\usetikzlibrary{calc}
\usetikzlibrary{backgrounds}
\usetikzlibrary{decorations.pathmorphing}
\ifx\collect@body\@undefined
  \IfFileExists{environ.sty}
    {\RequirePackage{environ}}
    {\RequirePackage{amsmath}}
\fi
%    \end{macrocode}
%
% \section{General Definitions}
%
%    \begin{macrocode}
\def\tikztimingwidth{0.0}
\newcount\tikztiming@numint
\newcount\tikztiming@numfrac
\def\tikztiming@wscale{1.0}%
\def\tikztimingsetwscale#1{%
  \pgfmathparse{#1}%
  \let\tikztiming@wscale\pgfmathresult
}
\def\tikztiming@num{0.0}%
\let\tikztiming@back\empty
\newlength\tikztiming@xunit
\newlength\tikztiming@yunit
\def\tikztiming@@xunit{1.6ex}%
\def\tikztiming@@yunit{1.6ex}%
\setlength{\tikztiming@xunit}{\tikztiming@@xunit}%
\setlength{\tikztiming@yunit}{\tikztiming@@yunit}%

\def\tikztiming@setnum#1{%
  \pgfmathparse{#1}%
  \global\let\tikztiming@num\pgfmathresult
}

\newcount\tikztiming@debug

\def\tikztiming@xsep{0pt}
\def\tikztiming@ysep{0pt}

\newcounter{tikztiming@nrows}%
\def\tikztiming@rowdist{2}%
\def\tikztiming@coldist{1}%

\def\tikztiming@prefix{tikztiming@trans@}%
\def\tikztiming@ttt#1#2#3{#1#2#3}%
\def\tikztiming@pfxtr{%
  \tikztiming@prefix\expandafter\tikztiming@ttt
}

\def\timingwidth{1}%
\def\timingheight{1}%
%    \end{macrocode}

% \section{TikZ Style Settings}
%    \begin{macrocode}
\tikzset{%
  timing/.style={/tikz,
    line cap=round, line join=round,
    timing/adjust@xy,
    every path/.append code={\tikz@expandcount=\tikztiming@expandcount\relax},
  },%
}
\tikzset{%
  timing/.cd,
  adjust@xy/.style={%
    /utils/exec={%
      \tikz@textfont
      \let\@tempa\fsize
      \let\fsize\@tempdima
      \fsize=\f@size pt\relax
      \pgfmathsetlength{\tikztiming@xunit}{\tikztiming@@xunit}%
      \pgfmathsetlength{\tikztiming@yunit}{\tikztiming@@yunit}%
      \let\fsize\@tempa
    },%
    /tikz/x=\tikztiming@xunit, /tikz/y=\tikztiming@yunit,
  },
  expand count/.store in={\tikztiming@expandcount},
  expand count=1000,
  unit/.style={%
    /utils/exec={%
      \def\tikztiming@@xunit{#1}%
      \let\tikztiming@@yunit\tikztiming@@xunit
    },%
    /tikz/timing/adjust@xy,%
  },
  xunit/.style={%
    /utils/exec={%
      \def\tikztiming@@xunit{#1}%
    },%
    /tikz/timing/adjust@xy,%
  },
  yunit/.style={%
    /utils/exec={%
      \def\tikztiming@@yunit{#1}%
    },%
    /tikz/timing/adjust@xy,%
  },
  font/.style={%
    /tikz/font={#1},
    /tikz/timing/adjust@xy
  },%
  initchar/.value required,
  initchar/.code={\uppercase{\gdef\tikztiming@lastchar{#1}}},
  metachar/.code={\tikztimingmetachar#1},
  grid/.style={timing,help lines},
  @grid/.code={\texttiminggrid},
  draw grid/.style={%
    timing/before text/.append style={timing/@grid},%
    timing/before/.append style={timing/@grid}%
  },
  no grid/.style={timing/@grid/.code={}},
  before/.code={},
  after/.code={},
  before text/.code={\texttimingbefore},
  after text/.code={\texttimingafter},
  picture/.style={timing,line width=0.15ex},
  intext/.style={timing,line width=0.15ex},
  inline node/.style={shape=coordinate},
  inline node/from char/.store in=\tikztiming@node@fromchar,
  inline node/to char/.store in=\tikztiming@node@tochar,
  table/.style={font=\sffamily,timing,line width=0.15ex},
  before table/.code={},
  after table/.code={},
  coord/.style={coordinate},
  save/.style={coordinate},
  restore/.style={/utils/exec={\expandafter\tikzset\expandafter{\tikztiming@saved@options}}},
  option/.style={/utils/exec={\g@addto@macro\tikztiming@saved@options{#1,}},#1},
  nooptions/.style={/utils/exec={\gdef\tikztiming@saved@options{}}},
  popoptions/.style={/utils/exec={\tikztiming@popoptions}},
  pushoptions/.style={/utils/exec={\tikztiming@pushoptions}},
  name/.style={inner sep=0pt,outer sep=0pt,minimum size=0pt},
  text format/.store in=\tikztiming@textformat,
  @/.style={},
  D/.style={timing/d},
  D/text/.style={timing/d/text},
  D/background/.style={timing/d/background},
  D@edge@/.style={timing/d@edge@},
  D@edge@/text/.style={timing/d@edge@/text},
  D@edge@/background/.style={timing/d@edge@/background},
  U/.style={timing/u},
  U/text/.style={timing/u/text},
  U/background/.style={timing/u/background},
  U@edge@/.style={timing/u@edge@},
  U@edge@/text/.style={timing/u@edge@/text},
  U@edge@/background/.style={timing/u@edge@/background},
  H/.style={timing/h},
  L/.style={timing/l},
  E/.style={timing/e},
  E/background/.style={timing/e/background},
  M/decoration/.style={timing/m/decoration},
  M/base/.style={timing/m/base},
  M/.style={timing/m},
  G/.style={timing/g},
  Z/.style={timing/z},
  T/.style={timing/t},
  C/.style={timing/c},
  X/.style={timing/x},
  d/.style={},
  d/text/.style={timing,scale=0.6,font=\sffamily},
  d/background/.style={draw=none},
  d@edge@/.style={timing/d},
  d@edge@/text/.style={timing/d/text},
  d@edge@/background/.style={timing/d/background},
  u/.style={},
  u/background/.style={draw=none,fill=gray},
  u@edge@/.style={timing/u},
  u@edge@/text/.style={timing/u/text},
  u@edge@/background/.style={timing/u/background},
  h/.style={},
  l/.style={},
  e/.style={},
  e/background/.style={draw=none},
  m/decoration/.style={decorate,%
    decoration={zigzag,segment length=.25\tikztiming@xunit,
    amplitude=.225\tikztiming@yunit}},
  m/base/.style={black!40!brown},
  m/.style={timing/m/base,timing/m/decoration},
  g/.style={decorate=false},
  z/.style={blue},
  t/.style={},
  c/.style={timing/lslope=0.0},
  x/.style={red},
  table/grid/.style={timing/grid},
  table/lines/.style={},
  table/rules/.style={line width=0.08em,line cap=butt},
  table/midrules/.style={line width=0.05em,line cap=butt},
  table/header/.style={timing/name},
  table/label header/.style={},
  table/row header/.style={},
  slope/.code={%
    \tikztimingsetslope{#1}%
    \tikztimingsetdslope{2*#1}%
    \tikztimingsetzslope{#1/2}%
  },
  lslope/.code={\tikztimingsetslope{#1}},
  dslope/.code={\tikztimingsetdslope{#1}},
  zslope/.code={\tikztimingsetzslope{#1}},
  wscale/.code={\tikztimingsetwscale{#1}},
  coldist/.store in=\tikztiming@coldist,
  rowdist/.store in=\tikztiming@rowdist,
  outer xsep/.store in=\tikztiming@xsep,
  outer ysep/.store in=\tikztiming@ysep,
  outer sep/.style={/tikz/timing/outer xsep=#1,/tikz/timing/outer ysep=#1},
  %
  debug/level/.code={\tikztiming@debug=#1\relax},
  debug/level/.value required,
  debug/scope/.style={font=\sffamily\tiny,gray},
  debug/node/.style={above,inner sep=2pt},
  debug/nodes/.style={%
    /tikz/timing/after table/.append code={\tikztiming@marknodes{\tikztiming@namednodes}},
    /tikz/timing/after text/.append code={\tikztiming@marknodes{\tikztiming@namednodes}}
  },
  debug/markcmd/.store in={\tikztiming@markcmd},
}
\tikzset{timing/inline node/.cd,
  from char={}, to char={}
}
%    \end{macrocode}
%
% \section{Text-Mode Macros}
% \begin{macro}{\texttimingbefore}
% This macro is executed inside the |tikzpicture| environment of \cs{texttiming} 
% before the timing diagram is drawn.
%    \begin{macrocode}
\def\texttimingbefore{}
%    \end{macrocode}
% \end{macro}
%
% \begin{macro}{\texttimingafter}
% This macro is executed inside the |tikzpicture| environment of \cs{texttiming} 
% after the timing diagram is drawn.
%    \begin{macrocode}
\def\texttimingafter{}
%    \end{macrocode}
% \end{macro}
%
% \begin{macro}{\texttiminggrid}
% Draws a background grid with the "timing/grid" setting. Should be used inside 
% \cs{texttimingbefore}.
%    \begin{macrocode}
\def\texttiminggrid{%
  \draw[xstep={\timingwidth/2.},ystep={\timingheight/2.},timing/grid] (0,0) grid 
  (\timingwidth*\tikztimingwidth,\timingheight);
}
%    \end{macrocode}
% \end{macro}
%
% \section{User Level Macros}
%
% \begin{macro}{\texttiming}[2]{Optional initial character}{Timing characters}
% This macro now only parses it optional argument to check if it is a initial 
% character or some style settings and then calls \cs{timing} appropriately.
%    \begin{macrocode}
\DeclareRobustCommand*\texttiming[2][]{%
  \begingroup
    \let\tikztiming@lastchar\empty
    \let\tikztiming@settings\empty
    \tikzset{timing/before/.append style={timing/before text}}%
    \tikzset{timing/after/.prefix style={timing/after text}}%
    \ifx\relax#1\relax\else
      \tikztiming@testoptarg#1\relax\relax%
    \fi
    \begin{tikzpicture}[timing/intext,baseline={(0,0)}]
      \tikztiming@picinit
      \def\@tempa{\expandafter\timing\expandafter[\tikztiming@settings] at (0,0)}%
      \expandafter\@tempa\expandafter{\expandafter[\tikztiming@lastchar]#2};%
    \end{tikzpicture}%
  \endgroup
}
%    \end{macrocode}
% \end{macro}
%
% \begin{macro}{\tikztiming@testoptarg}
%    \begin{macrocode}
\def\tikztiming@testoptarg#1#2\relax{%
  \ifx\relax#2\relax
    \uppercase{\def\tikztiming@lastchar{#1}}%
  \else
    \def\tikztiming@settings{#1#2}%
  \fi
}
%    \end{macrocode}
% \end{macro}
%
% \begin{macro}{\tikztiming@picinit}
% For things which should only by initialised once per |tikzpicture|.
%    \begin{macrocode}
\def\tikztiming@picinit{%
    \global\let\tikztiming@namednodes\empty
}
\let\tikztiming@namednodes\empty
%    \end{macrocode}
% \end{macro}
%
% \begin{macro}{\tikztiming@init}
%    \begin{macrocode}
\def\tikztiming@init{%
    \def\tikztiming@lastchar{@}%
    \def\tikztiming@num{0}%
    \let\tikztiming@currentchar\empty
    \let\tikztiming@settings\empty
    \let\tikztiming@fromchar\empty
    \let\tikztiming@tochar\empty
    \let\setwscale\tikztimingsetwscale
    \def\wscale{\tikztiming@wscale}%
    \let\xunit\tikztiming@xunit
    \let\yunit\tikztiming@yunit
    \def\slope{\timingslope}%
    \def\lslope{\timingslope}%
    \def\zslope{\timingzslope}%
    \def\dslope{\timingdslope}%
    \gdef\tikztiming@saved@settings{}%
    \gdef\tikztiming@saved@options{}%
    \gdef\tikztiming@oldnodecenter@{}%
    \def\tikztimingwidth{0.0}%
    \setcounter{tikztimingtrans}{-1}%
    \setcounter{tikztimingtranspos}{0}%
    \def\tikztiming@str{%
      \useasboundingbox (0,0) rectangle (\tikztimingwidth,\timingheight);
      \draw (0,0)
        coordinate (timing@refa)
        coordinate (timing@ref)  at (0,\timingheight/2)
        coordinate (timing@refb) at (0,\timingheight)
        coordinate (timing/start)
        coordinate (timing/start base)
    }%
}
%    \end{macrocode}
% \end{macro}
%
% \begin{macro}{\tikz@timing}
%    \begin{macrocode}
\def\tikz@timing{%
  \@ifnextchar{[}%
    {\tikz@timing@}%
    {\tikz@timing@[]}%
}
%    \end{macrocode}
% \end{macro}
%
% \begin{macro}{\tikz@installcommands}
% Add \cs{timing} to the TikZ internal |\tikz@installcommands| macro.
% The |\timing| macro will be installed at the begin of the |tikzpicture|.
%    \begin{macrocode}
\g@addto@macro\tikz@installcommands{%
  \let\tikz@origtiming=\timing%
  \let\timing=\tikz@timing%
}
%    \end{macrocode}
% \end{macro}
%
% \begin{macro}{\tikz@uninstallcommands}
% Add \cs{timing} to the TikZ internal |\tikz@uninstallcommands| macro.
% The |\timing| macro will be uninstalled at the begin of nodes.
%    \begin{macrocode}
\g@addto@macro\tikz@uninstallcommands{%
  \let\timing=\tikz@origtiming%
}
%    \end{macrocode}
% \end{macro}
%
% \begin{macro}{\tikz@timing@}[1]{Optional TikZ Keys}
%    \begin{macrocode}
\def\tikz@timing@[#1]{%
  \@ifnextchar{a}%
    {\tikz@timing@at{#1}}%
    {\PackageWarning{tikz-timing}{The \string\timing\space command awaits now an 
    `at' before the coordinate, like the \string\node\space command is doing.  
    Please update your source code to ensure compatibility with newer 
    versions}{}{}{}%
     \@ifnextchar{+}%
      {\tikz@timing@@{#1}}%
      {\@ifnextchar(%)
        {\tikz@timing@@{#1}}%
        {\tikz@timing@@{#1}++(0,0)}%
      }%
    }%
}
%    \end{macrocode}
% \end{macro}
%
% \begin{macro}{\tikz@timing@at}[1]{Optional TikZ Keys}
%    \begin{macrocode}
\def\tikz@timing@at#1at#2(#3){%
  \tikz@timing@@@{#1}{#2(#3)}%
}
%    \end{macrocode}
% \end{macro}
%
% \begin{macro}{\tikz@timing@@}[3]{Optional TikZ Keys}{Potential `+' or 
% empty}{Coordinate}
%    \begin{macrocode}
\def\tikz@timing@@#1#2(#3){%
  \tikz@timing@@@{#1}{#2(#3)}%
}
%    \end{macrocode}
% \end{macro}
%
% \begin{macro}{\tikz@timing@@@}[3]{Optional TikZ Keys}{Coordinate}{Timing 
% Characters}
%    \begin{macrocode}
\def\tikz@timing@@@#1#2#3{%
  \begingroup
    \tikztiming@init
    \let\tikz@alias=\pgfutil@empty%
    \begin{scope}[shift={#2},timing,#1]%
      \@ifnextchar{[}%
        {\tikz@timing@@@init}%
        {\tikz@timing@@@init[]}%
      #3\relax
      \ifnum\tikztiming@debug>1
      \message{^^J\meaning\tikztiming@str^^J}%
      \ifnum\tikztiming@debug>2
      {%
      \def\@protect##1{\def##1{\noexpand##1}}%
      \@protect\useasboundingbox
      \@protect\draw
      \@protect\p
      \@protect\n
      \@protect\x
      \@protect\y
      \@protect\tikztiming@overlay@save
      \edef\@tempa{\tikztiming@str}%
      \message{^^J\meaning\@tempa^^J}%
      }%
      \fi
      \fi
      \tikzset{timing/before}%
      \tikztiming@str;%
      \path let \p1 = (timing/end top), \p2 = (timing/end base),
                \p3 = (timing/end), \p4 = (timing/start) in
          [/utils/exec={%
            \xdef\tikztiming@northeast{\noexpand\pgfpoint{.5*\pgf@pt@aa*\x1}{.5*\pgf@pt@bb*\y1}}%
            \xdef\tikztiming@startpoint{\noexpand\pgfpoint{\pgf@pt@aa*(\x4-.5*\x1)}{\pgf@pt@bb*(\y4-.5*\y1)}}%
            \xdef\tikztiming@endpoint{\noexpand\pgfpoint{\pgf@pt@aa*(\x3-.5*\x1)}{\pgf@pt@bb*(\y3-.5*\y1)}}%
          }];
      \node [shape=tikztiming@shape,anchor=origin,#1] at (0,0) {};
      \tikzset{timing/after}%
    \end{scope}%
  \endgroup
  \tikz@timing@@@end
}
%    \end{macrocode}
% \end{macro}
%
% \begin{macro}{\tikz@timing@@@end}
%    \begin{macrocode}
\def\tikz@timing@@@end{%
  \@ifnextchar{;}%
    {\@gobble}%
    {\PackageError{tikz-package}{Can not parse timing path}{}}%
}
%    \end{macrocode}
% \end{macro}
%
% \begin{macro}{\tikz@timing@@@init}[1]{Initial character.}
% Opens a group which will end in \cs{tikztiming@parser@end}.
%    \begin{macrocode}
\def\tikz@timing@@@init[#1]{%
  \begingroup
  \ifx\relax#1\relax\else
    \uppercase{\def\tikztiming@lastchar{#1}}%
  \fi
  \@ifundefined{tikztiming@initcode@\tikztiming@lastchar}%
    {}%
    {\@nameuse{tikztiming@initcode@\tikztiming@lastchar}}%
  \if @\tikztiming@lastchar\relax\else
  \tikztiming@iftrans{@\tikztiming@lastchar}%
    {\tikztiming@nameaddtostr{@}{\tikztiming@lastchar}{}}%
    {\PackageWarning{tikz-timing}{Start value for timing character 
    '\tikztiming@lastchar' is not defined and will be ignored!}{}{}{}}%
  \fi
  \tikztiming@parser
}
%    \end{macrocode}
% \end{macro}
%
% \section{Internal Macros}
%
% \begin{macro}{\tikztiming@trans@}
%    \begin{macrocode}
\def\tikztiming@trans@#1#2#3#4{%
  \csname\tikztiming@prefix #2#3#4\endcsname{#1}{#2}{#3}{#4}%
}
%    \end{macrocode}
% \end{macro}
%
% \begin{macro}{\tikztiming@beforenextcode@D@edge@}
%    \begin{macrocode}
\def\tikztiming@beforenextcode@D@edge@{%
  \if D\tikztiming@currentchar\else
    \gdef\tikztiming@lastchar{D}%
  \fi
}
%    \end{macrocode}
% \end{macro}
%
% \begin{macro}{\tikztiming@beforenextcode@U@edge@}
%    \begin{macrocode}
\def\tikztiming@beforenextcode@U@edge@{%
  \if U\tikztiming@currentchar\else
    \gdef\tikztiming@lastchar{U}%
  \fi
}
%    \end{macrocode}
% \end{macro}
%
% \begin{macro}{\tikztiming@initcode@}
% If empty init char was given, exchange it with "@".
%    \begin{macrocode}
\def\tikztiming@initcode@{%
  \gdef\tikztiming@lastchar{@}%
}
%    \end{macrocode}
% \end{macro}
%
% \begin{macro}{\tikztiming@initcode@D}
%    \begin{macrocode}
\def\tikztiming@initcode@D{%
  \gdef\tikztiming@lastchar{D@edge@}%
}
%    \end{macrocode}
% \end{macro}
%
% \begin{macro}{\tikztiming@initcode@U}
%    \begin{macrocode}
\def\tikztiming@initcode@U{%
  \gdef\tikztiming@lastchar{U@edge@}%
}
%    \end{macrocode}
% \end{macro}
%
% \begin{macro}{\tikztiming@parser}
%    \begin{macrocode}
\def\tikztiming@parser{%
  \@ifnextchar\relax
    {\tikztiming@parser@end}%
    {\tikztiming@testfornum}%
}
%    \end{macrocode}
% \end{macro}
%
% \section{Add-to-Macro Macros}
% \begin{macro}{\tikztiming@eaddto}[2]{Macro name}{Tokens to add to macro}
%    \begin{macrocode}
\def\tikztiming@eaddto#1#2{%
  \begingroup
    \tikztiming@internaldefs
    \@temptokena\expandafter{#1}%
    \xdef#1{%
      \the\@temptokena
      #2%
    }%
  \endgroup
}
%    \end{macrocode}
% \end{macro}
%
%
% \begin{macro}{\tikztiming@eaddtostr}
%    \begin{macrocode}
\def\tikztiming@eaddtostr{%
  \tikztiming@eaddto\tikztiming@str
}
%    \end{macrocode}
% \end{macro}
%
% \begin{macro}{\tikztiming@addtostr}
%    \begin{macrocode}
\def\tikztiming@addtostr{%
  \g@addto@macro\tikztiming@str
}
%    \end{macrocode}
% \end{macro}
%
% \begin{macro}{\tikztiming@makeglobal}
%    \begin{macrocode}
\def\tikztiming@makeglobal#1{%
  \global\let#1#1\relax\relax
}
%    \end{macrocode}
% \end{macro}
%
% \begin{macro}{\tikztiming@iftrans}[1]{Characters}
%    \begin{macrocode}
\def\tikztiming@trans{}%
\def\tikztiming@iftrans#1{%
  \tikztiming@iftrans@#1\empty\empty\empty\relax
}
%    \end{macrocode}
% \end{macro}
%
% \begin{macro}{\tikztiming@iftrans}[4]
% {First character}{Second character}
% {Next character or empty}{Eats token till end marker}
%    \begin{macrocode}
\def\tikztiming@iftrans@#1#2#3#4\relax{%
  \begingroup
  \xdef\tikztiming@trans{#1#2}%
  \ifx\tikztiming@trans\empty
    \gdef\tikztiming@trans{{@}{}{@}}%
  \else
    \global\let\tikztiming@trans\relax
    % Use triple macros before double macros
    \@ifundefined{tikztiming@triple@#2}%
      {\@ifundefined{\tikztiming@prefix#1#2}%
        {}%
        {\xdef\tikztiming@trans{{#1}{#2}{}}}%
      }%
      {%
      \@ifundefined{\tikztiming@prefix#1#2#3}%
        {}%
        {\xdef\tikztiming@trans{{#1}{#2}{#3}}}%
      }%
    % If both do not exist, search for aliases:
    \ifx\tikztiming@trans\relax
      \@ifundefined{tikztiming@alias@#1}%
        {\edef\chara{#1}}%
        {\edef\chara{#1,\csname tikztiming@alias@#1\endcsname}}%
      \@ifundefined{tikztiming@alias@#2}%
        {\edef\charb{#2}}%
        {\edef\charb{#2,\csname tikztiming@alias@#2\endcsname}}%
      \@ifundefined{tikztiming@triple@#2}%
        {\def\charc{\empty}}%
        {%
         \@ifundefined{tikztiming@alias@#3}%
           {\edef\charc{#3,}}%
           {\edef\charc{#3,\csname tikztiming@alias@#3\endcsname,}}%
        }%
      \ifx\charb\empty
        \def\charb{\empty}%
      \fi
      \let\tikztiming@next\relax
      \foreach \b in \charb {%
        \foreach \a in \chara {%
          \foreach \c in \charc {%
            %\message{ ^^J Testing: \a\b\c ^^J}%
            \@ifundefined{\tikztiming@prefix\a\b\c}%
              {}{%
                \xdef\tikztiming@trans{{#1}{#2}{#3}}%
                \expandafter\tikztimingcpy\expandafter
                  {\tikztiming@trans}{\a\b\c}%
                \ifnum\tikztiming@debug>0
                  \message{^^J DEBUG: '#1#2#3' = '\a\b\c' ^^J}%
                \fi
                \expandafter\tikztiming@makeglobal
                \csname\tikztiming@pfxtr\tikztiming@trans\endcsname
                \ifnum\tikztiming@debug>1
                  \ifnum\tikztiming@debug>2
                    \expandafter\show\csname\tikztiming@pfxtr\tikztiming@trans\endcsname
                  \else
                    \message{^^J\tikztiming@trans: 
                    \expandafter\meaning\csname\tikztiming@pfxtr\tikztiming@trans\endcsname^^J}%
                    \message{^^J\a\b\c: 
                    \expandafter\meaning\csname\tikztiming@prefix\a\b\c\endcsname^^J}%
                  \fi
                \fi
                \global\let\tikztiming@next\breakforeach
              }%
            \tikztiming@next
          }%
          \tikztiming@next
        }%
        \tikztiming@next
      }%
    \fi
  \fi
  \endgroup
  \ifx\tikztiming@trans\relax
    \expandafter\@secondoftwo
  \else
    \expandafter\@firstoftwo
  \fi
}
%    \end{macrocode}
% \end{macro}
%
% \begin{macro}{\tikztiming@nameaddtostr}
% {First character}{Second character}
% {Optional next character}
%    \begin{macrocode}
\def\tikztiming@nameaddtostr#1#2#3{%
  \tikztiming@iftrans{{#1}{#2}{#3}}%
   {%
     \edef\@tempa{\noexpand\tikztiming@trans@{\tikztiming@num}\tikztiming@trans}%
     \expandafter\g@addto@macro
     \expandafter\tikztiming@str
     \expandafter{\@tempa}%
   }%
   {\PackageWarning{tikz-timing}{No transitions '#1#2' (or '#1#2#3') defined!}}%
}
%    \end{macrocode}
% \end{macro}
%
% \begin{macro}{\tikztiming@output@addcode}
% Adds code to the output code buffer. The code will not be expanded, i.e.\ will 
% be protected from the expansion when the buffer is flushed.
%    \begin{macrocode}
\def\tikztiming@output@addcode#1{%
  \g@addto@macro\tikztiming@output@bufcode{\unexpanded{ #1 }}%
}
%    \end{macrocode}
% \end{macro}
%
% \begin{macro}{\tikztiming@output@eaddcode}
% Adds code to the output code buffer. The code will be expanded when the buffer 
% is flushed.
%    \begin{macrocode}
\def\tikztiming@output@eaddcode#1{%
  \g@addto@macro\tikztiming@output@bufcode{ #1 }%
}
%    \end{macrocode}
% \end{macro}
%
% \begin{macro}{\tikztiming@output@Eaddcode}
% Adds code to the output code buffer. The code will be expanded immediately and 
% again when the buffer is flushed.
%    \begin{macrocode}
\def\tikztiming@output@Eaddcode{%
  \tikztiming@eaddto\tikztiming@output@bufcode%
}
%    \end{macrocode}
% \end{macro}
%
%
% \section{Output Routine}
%
% Init buffer macros:
%    \begin{macrocode}
\def\tikztiming@output@bufchara{}%
\def\tikztiming@output@bufcharb{}%
\def\tikztiming@output@bufnum{0}%
\def\tikztiming@output@bufcode{}%
\def\tikztiming@output@nextchar{}%
%    \end{macrocode}
%
% \begin{macro}{\tikztiming@output}[2]{Character 1}{Character 2}
%    \begin{macrocode}
\def\tikztiming@output#1#2{%
  \edef\tikztiming@fromchar{#1}%
  \edef\tikztiming@tochar{#2}%
  \ifx\relax#2\relax
    %\tikztiming@output@codeflush
    \tikztiming@nameaddtostr{}{#1}{}%
  \else
    \ifcase0%
      \ifx\tikztiming@output@bufchara\empty
        \ifx\tikztiming@output@bufcharb\empty
          1%
        \fi
      \fi\relax
      % not empty
      \edef\tikztiming@output@currentchar{#2}%
      \ifcase0%
       \ifx\tikztiming@output@bufcode\empty% Stop combining if code was inserted
        \expandafter\ifx\csname tikztiming@nocombine@#2\endcsname\relax
         \ifx\tikztiming@output@currentchar\tikztiming@output@bufcharb
           2%
         \fi
        \fi
       \else
         1%
       \fi
      \relax
        \edef\tikztiming@output@nextchar{#2}%
        \tikztiming@output@flush
        \xdef\tikztiming@output@bufchara{#1}%
        \xdef\tikztiming@output@bufcharb{#2}%
      \or
        \edef\tikztiming@output@nextchar{#2}%
        \tikztiming@output@flush
        \xdef\tikztiming@output@bufchara{#1}%
        \xdef\tikztiming@output@bufcharb{#2}%
      \or
        \pgfmathparse{\tikztiming@output@bufnum + \tikztiming@num}%
        \global\let\tikztiming@output@bufnum\pgfmathresult
      \fi
    \else % empty
      \tikztiming@output@codeflush
      \xdef\tikztiming@output@bufchara{#1}%
      \xdef\tikztiming@output@bufcharb{#2}%
      \global\let\tikztiming@output@bufnum\tikztiming@num
    \fi
  \fi
}
%    \end{macrocode}
% \end{macro}
%
% \begin{macro}{\tikztiming@output@flush}
%    \begin{macrocode}
\def\tikztiming@output@flush{%
  \tikztiming@output@charflush
  \tikztiming@output@codeflush
}
%    \end{macrocode}
% \end{macro}
%
% \begin{macro}{\tikztiming@output@codeflush}
%    \begin{macrocode}
\def\tikztiming@output@codeflush{%
  \tikztiming@eaddtostr{\tikztiming@output@bufcode}%
  \global\let\tikztiming@output@bufcode\empty%
}
%    \end{macrocode}
% \end{macro}
%
% \begin{macro}{\tikztiming@output@charflush}
% A group is used to protect the current value of |num| which is the length of 
% the new character after the flushed old character. This value will be assigned 
% to the new buffer length |bufnum|.
%    \begin{macrocode}
\def\tikztiming@output@charflush{%
  \begingroup
    \let\tikztiming@num\tikztiming@output@bufnum
    \tikztiming@nameaddtostr
      {\tikztiming@output@bufchara}%
      {\tikztiming@output@bufcharb}%
      {\tikztiming@output@nextchar}%
  \endgroup%
  \gdef\tikztiming@output@nextchar{}%
  \gdef\tikztiming@output@bufchara{}%
  \gdef\tikztiming@output@bufcharb{}%
  \global\let\tikztiming@output@bufnum\tikztiming@num
}
%    \end{macrocode}
% \end{macro}
%
%
% \section{Macros for Character Definitions}
%
% \begin{macro}{\tikztiming@nameedef}[4]{Macro name to be defined (without 
% prefix)}{Content of macro}
% Defines internal tikztiming macro with name |\|\meta{prefix}\meta{name (\#2)}.  
% The macro definition (\#3) is expanded while the internal drawing definitions 
% are active.
%    \begin{macrocode}
\newcommand\tikztiming@nameedef[4]{%
  \def\@gtempa##1{#4}%
  \begingroup
    \tikztiming@internaldefs
    \xdef\@gtempa##1##2##3##4{\@gtempa{\width}}%
  \endgroup
  \expandafter\let\csname\tikztiming@prefix#1#2#3\endcsname\@gtempa
  \let\@gtempa\empty
}
%    \end{macrocode}
% \end{macro}
%
% \begin{macro}{\tikztiming@namelet}[2]{Original Characters}{New Characters}
% Only execute |\let| if the original macro is defined or the destination macro 
% is defined and would now set to undefined.
%    \begin{macrocode}
\newcommand\tikztiming@namelet[2]{%
  \ifcase0%
    \@ifundefined{\tikztiming@prefix#2}%
      {\@ifundefined{\tikztiming@prefix#1}%
        {0}{1}%
      }%
      {1}%
    \relax
  \else
    \expandafter\let
    \csname\tikztiming@prefix#1\expandafter\endcsname
    \csname\tikztiming@prefix#2\endcsname
  \fi
}
%    \end{macrocode}
% \end{macro}
%
%
% \section{Parser}
%
% \begin{macro}{\tikztiming@parser@end}
%    \begin{macrocode}
\def\tikztiming@parser@end{%
  \def\tikztiming@output@nextchar{@}%
  \let\tikztiming@fromchar\tikztiming@tochar
  \edef\tikztiming@tochar{@}%
  \tikztiming@output@flush
  \global\let\tikztimingwidth\tikztimingwidth
  \tikztiming@addtostr{ coordinate (timing/end)
    let \p1 = (timing/start base), \p2 = (timing/end), \p3 = (0,1) in
      coordinate (timing/end base) at (\x2,\y1)
      coordinate (timing/end top)  at (\x2,\y1+\y3)
  }%
  \endgroup
  \tikztiming@endhook%
}
\let\tikztiming@endhook\relax
%    \end{macrocode}
% \end{macro}
%
% \begin{macro}{\tikztiming@parser@}[1]{Next Character}
%    \begin{macrocode}
\def\tikztiming@parser@#1{%
  \ifx\relax#1\empty
    \expandafter\tikztiming@parser@end
  \else
    \tikztiming@iflower{#1}%
      {\tikztiming@setnum{\tikztiming@num/2}}%
      {}%
    \ifx\tikztiming@back\empty\else
      \tikztiming@setnum{\tikztiming@num - \tikztiming@back}%
      \global\let\tikztiming@back\empty
    \fi
    \pgfmathparse{\tikztimingwidth + \tikztiming@num}%
    \global\let\tikztimingwidth\pgfmathresult
    \uppercase{\gdef\tikztiming@currentchar{#1}}%
    \@nameuse{tikztiming@beforenextcode@\tikztiming@lastchar}%
    \@nameuse{tikztiming@beforecode@\tikztiming@currentchar}%
    \tikztiming@output{\tikztiming@lastchar}{\tikztiming@currentchar}%
    \global\let\tikztiming@lastchar\tikztiming@currentchar
    \@nameuse{tikztiming@aftercode@\tikztiming@currentchar}%
    \expandafter\lowercase\expandafter{%
      \expandafter\def\expandafter\@tempa
      \expandafter{\tikztiming@currentchar}%
    }%
    \pgfkeysifdefined{/tikz/timing/\@tempa/text/.@cmd}%
      {\expandafter\expandafter\expandafter\tikztiming@testfortext}%
      {\expandafter\expandafter\expandafter\tikztiming@parser}%
  \fi
}
%    \end{macrocode}
% \end{macro}
%
% \begin{macro}{\tikztiming@testfortext}
%    \begin{macrocode}
\def\tikztiming@testfortext{%
  \@ifnextchar\bgroup
    {\tikztiming@handletext}%
    {\tikztiming@parser}%
}
%    \end{macrocode}
% \end{macro}
%
% \begin{macro}{\tikztiming@handletext}[1]{Text}
% Looks ahead if the current character is the last one and sets the end-of-line 
% marker if so. 
%
% This seems outdated because |handletext@| does not flush the output any more.
%    \begin{macrocode}
\def\tikztiming@handletext#1{%
  \@ifnextchar\relax
    {\def\tikztiming@output@nextchar{@}\tikztiming@handletext@{#1}}%
    {\def\tikztiming@output@nextchar{Z}\tikztiming@handletext@{#1}}%
}
%    \end{macrocode}
% \end{macro}
%
% \begin{macro}{\tikztiming@handletext@}[1]{Text}
%    \begin{macrocode}
\def\tikztiming@handletext@#1{%
  \@ifnextchar{[}%
    {\tikztiming@handletext@@}%
    {\tikztiming@handletext@@[]}%
  #1\tikztiming@endoftext
}
%    \end{macrocode}
% \end{macro}
%
% \begin{macro}{\tikztiming@handletext@@}[2]{Optional Settings}{Text}
%    \begin{macrocode}
\def\tikztiming@handletext@@[#1]#2\tikztiming@endoftext{%
  \@ifnextchar{\relax}%
    {\def\tikztiming@output@nextchar{@}%
     \tikztiming@handletext@@@{#1}{#2}}%
    {\tikztiming@handletext@@@{#1}{#2}}%
}
%    \end{macrocode}
% \end{macro}
%
% \begin{macro}{\tikztiming@handletext@@@}[2]{Optional Settings}
% {Text}
%    \begin{macrocode}
\def\tikztiming@handletext@@@#1#2{%
  \expandafter\lowercase\expandafter{%
    \expandafter\def\expandafter\@tempa
    \expandafter{\tikztiming@currentchar}%
  }%
  \edef\@tempa{node [timing/\@tempa/text,}%
  \expandafter\tikztiming@output@addcode\expandafter{%
    \@tempa
    #1] at ($ (timing@dstart) !.5! (timing@dend) $)
    {\tikztiming@textformat{#2}}
  }%
  \ifcase0%
    \if @\tikztiming@output@bufchara
      1%
    \else
      \ifx\tikztiming@output@bufchara\tikztiming@currentchar
        1%
      \fi
    \fi
  \relax
  \else
  \fi
  \xdef\tikztiming@output@bufcharb{\tikztiming@currentchar @edge@}%
  \xdef\tikztiming@lastchar{\tikztiming@currentchar @edge@}%
  \tikztiming@parser
}
\let\tikztiming@textformat\empty
%    \end{macrocode}
% \end{macro}
%
% \section{Definition of Modifiers}
%
% \begin{macro}{\tikztiming@defcode}[1]{Code Character}
%    \begin{macrocode}
\def\tikztiming@defcode#1{%
  \@namedef{tikztiming@code@\meaning#1}%
}
%    \end{macrocode}
% \end{macro}
%
% \begin{macro}{\tikztiming@defcode{@}}
% Simply executes given code. The |\relax| is added to have some protection 
% against user code manipulating the parser macro. The "@" character can be both 
% a letter (in style files) and a character (in user documents), so it is
% defined twice.
%    \begin{macrocode}
\tikztiming@defcode{@}#1{%
  #1\relax
  \tikztiming@parser
}
\begingroup
  \expandafter\let\expandafter\temp
    \csname tikztiming@code@\meaning @\endcsname
  \makeatother
  \expandafter\global\expandafter\let
    \csname tikztiming@code@\meaning @\endcsname\temp
\endgroup
%    \end{macrocode}
% \end{macro}
%
% \begin{macro}{\tikztiming@defcode{DOLLAR}}
% Takes the code between two dollar signs and parses it as math expression. The 
% result is put back on the input stream. This allows the calculation of 
% character width.
%    \begin{macrocode}
\tikztiming@defcode$#1${%
  \pgfmathparse{#1}%
  \expandafter
  \tikztiming@parser\pgfmathresult
}
%    \end{macrocode}
% \end{macro}
%
% \begin{macro}{\tikztiming@defcode{,}}
%    \begin{macrocode}
\tikztiming@defcode{,}{%
  \tikztiming@output@eaddcode{%
    \newdraw
  }%
  \tikztiming@parser
}
%    \end{macrocode}
% \end{macro}
%
% \begin{macro}{\tikztiming@defcode{;}}
%    \begin{macrocode}
\tikztiming@defcode{;}{%
  \endgroup\begingroup
  \tikztiming@output@eaddcode{%
    \newdrawns
  }%
  \tikztiming@parser
}
%    \end{macrocode}
% \end{macro}
%
% \begin{macro}{\tikztiming@defcode{N}}
%    \begin{macrocode}
\tikztiming@defcode{N}{%
  \@ifnextchar[%]
    {\tikztiming@addnode@getoptions}%
    {\tikztiming@addnode@getoptions[]}%
}
%    \end{macrocode}
% \end{macro}
%
% \begin{macro}{\tikztiming@defcode{n}}
%    \begin{macrocode}
\tikztiming@defcode{n}{%
  \@ifnextchar[%]
    {\tikztiming@addnode@getoptions}%
    {\tikztiming@addnode@getoptions[]}%
}
%    \end{macrocode}
% \end{macro}
%
% \begin{macro}{\tikztiming@defcode{B}}
%    \begin{macrocode}
\tikztiming@defcode{B}{%
  \pgfmathparse{\tikztiming@back+\tikztiming@num}%
  \global\let\tikztiming@back\pgfmathresult
  \tikztiming@parser
}
%    \end{macrocode}
% \end{macro}
%
% \begin{macro}{\tikztiming@defcode{b}}
%    \begin{macrocode}
\tikztiming@defcode{b}{%
  \pgfmathparse{\tikztiming@back+\tikztiming@num/2}%
  \global\let\tikztiming@back\pgfmathresult
  \tikztiming@parser
}
%    \end{macrocode}
% \end{macro}
%
% \begin{macro}{\tikztiming@defcode{F}}
%    \begin{macrocode}
\tikztiming@defcode{F}{%
  \pgfmathparse{\tikztiming@back-\tikztiming@num}%
  \global\let\tikztiming@back\pgfmathresult
  \tikztiming@parser
}
%    \end{macrocode}
% \end{macro}
%
% \begin{macro}{\tikztiming@defcode{f}}
%    \begin{macrocode}
\tikztiming@defcode{f}{%
  \pgfmathparse{\tikztiming@back-\tikztiming@num/2}%
  \global\let\tikztiming@back\pgfmathresult
  \tikztiming@parser
}
%    \end{macrocode}
% \end{macro}
%
%
% \begin{macro}{\tikztiming@defcode []}[1]{Options}
% Single brackets are in-picture options.
% Brackets and a "!" sign "[! !]" are parser options.
% Double brackets are both.
%
% Brackets and "| |" signs "[| |]" are only-once in-picture options.
%    \begin{macrocode}
\tikztiming@defcode[{%}]{
  \@ifnextchar[%]
    {\tikztiming@bothoption}%
    {\@ifnextchar{!}%
      {\tikztiming@parseroption}%
      {\@ifnextchar{|}%
        {\tikztiming@onceoption}%
        {\tikztiming@addoption}%
      }%
    }%
}
%    \end{macrocode}
% \end{macro}
%
% \begin{macro}{\tikztiming@parseroption}[1]{Options}
% Sets the given options immediately, so that they are in affect for the rest of 
% the parser run.
%    \begin{macrocode}
\def\tikztiming@parseroption!#1!]{%
  \pgfqkeys{/tikz/timing}{#1}%
  \tikztiming@parser
}
%    \end{macrocode}
% \end{macro}
%
% \begin{macro}{\tikztiming@addoption}[1]{Options}
% Adds the options to the internal list of timing options.
%    \begin{macrocode}
\def\tikztiming@addoption#1]{%
  \tikztiming@output@addcode{ [timing/option={#1}] }%
  \tikztiming@parser
}
%    \end{macrocode}
% \end{macro}
%
% \begin{macro}{\tikztiming@bothoption}[1]{Options}
% Does executes and adds the options to the output code buffer. This is 
% necessary for slopes if they are used in width calculations.
%    \begin{macrocode}
\def\tikztiming@bothoption[#1]]{%
  \tikzset{#1}%[
  \tikztiming@addoption{#1}]%
}
%    \end{macrocode}
% \end{macro}
%
% \begin{macro}{\tikztiming@onceoption}[1]{Options}
% Simply adds the options to the output code buffer.
%    \begin{macrocode}
\def\tikztiming@onceoption|#1|]{%
  \tikztiming@output@addcode{ [#1] }%
  \tikztiming@parser
}
%    \end{macrocode}
% \end{macro}
%
%
% \subsection{Local Scopes, Push/Pop Options}
%
% \begin{macro}{\tikztiming@defcode bgroup}
%    \begin{macrocode}
\tikztiming@defcode{\bgroup}{%
  \tikztiming@output@eaddcode{%
    [timing/pushoptions]
    \newdraw
  }%
  \tikztiming@parser
}
%    \end{macrocode}
% \end{macro}
%
% \begin{macro}{\tikztiming@defcode egroup}
%    \begin{macrocode}
\tikztiming@defcode{\egroup}{%
  \tikztiming@output@eaddcode{%
    [timing/popoptions]
    \newdraw
  }%
  \tikztiming@parser
}
%    \end{macrocode}
% \end{macro}
%
% \begin{macro}{\tikztiming@popoptions}
%    \begin{macrocode}
\def\tikztiming@popoptions{%
  \expandafter\tikztiming@popoptions@\tikztiming@optionstack\relax
}
\def\tikztiming@optionstack{{}}
%    \end{macrocode}
% \end{macro}
%
% \begin{macro}{\tikztiming@popoptions@}
%    \begin{macrocode}
\def\tikztiming@popoptions@#1#2\relax{%
  \gdef\tikztiming@saved@options{#1}%
  \gdef\tikztiming@optionstack{#2}%
  \ifx\tikztiming@optionstack\empty
    \gdef\tikztiming@optionstack{{}}%
  \fi
}
%    \end{macrocode}
% \end{macro}
%
%
% \begin{macro}{\tikztiming@pushoptions}
%    \begin{macrocode}
\def\tikztiming@pushoptions{%
  \expandafter\expandafter\expandafter\gdef
  \expandafter\expandafter\expandafter\tikztiming@optionstack
  \expandafter\expandafter\expandafter{%
  \expandafter\expandafter\expandafter{\expandafter\tikztiming@saved@options\expandafter}\tikztiming@optionstack}%
}
%    \end{macrocode}
% \end{macro}
%
%
% \subsection{Add-Node Macros}
%
% \begin{macro}{\tikztiming@addnode}[1]{Options}
%    \begin{macrocode}
\def\tikztiming@addnode@getoptions[#1]{%
  \@ifnextchar(%)
    {\tikztiming@addnode@getname{#1}}%
    {\tikztiming@addnode@getname{#1}()}%
}
%    \end{macrocode}
% \end{macro}
%
% \begin{macro}{\tikztiming@addnode@getname}[2]{Previous read options}{Node name}
%    \begin{macrocode}
\def\tikztiming@addnode@getname#1(#2){%
  \@ifnextchar\bgroup
    {\tikztiming@addnode@{#1}{#2}}%
    {\tikztiming@addnode@{#1}{#2}{}}%
}
%    \end{macrocode}
% \end{macro}
%
% \begin{macro}{\tikztiming@addnode@}[3]{Options}{Node name}{Node text}
% The |fromchar| and |tochar| macros are to be expanded in the output routine, 
% not inside this macro. They do not hold the correct values yet.
%    \begin{macrocode}
\def\tikztiming@addnode@#1#2#3{%
  \def\@tempa{#2}%
  \ifx\@tempa\empty
    \def\@tempa{ node [timing/inline node,%
      timing/inline node/from char={\tikztiming@fromchar},%
      timing/inline node/to char={\tikztiming@tochar},%
      \unexpanded{#1}] }%
    \def\@tempb{}%
  \else
    \def\@tempa{ node [timing/inline node,%
      timing/inline node/from char={\tikztiming@fromchar},%
      timing/inline node/to char={\tikztiming@tochar},%
      \unexpanded{#1}] (\unexpanded{#2}) }%
    \xdef\tikztiming@namednodes{\tikztiming@namednodes,#2}%
  \fi
  \expandafter\tikztiming@output@eaddcode\expandafter{\@tempa \unexpanded{{#3}}}%
  \tikztiming@parser
}
%    \end{macrocode}
% \end{macro}
%
% \begin{macro}{\tikztiming@marknodes}[1]{Comma-separated list of node names}
%    \begin{macrocode}
\newcommand*\tikztiming@marknodes[2][]{%
  \scope[timing/debug/scope,#1]
    \edef\@tempa{floor((\noexpand\y1-\tikztiming@xunit) / \tikztiming@rowdist\tikztiming@xunit) * \tikztiming@rowdist\tikztiming@xunit}%
    \foreach \N in #2 {%
      \ifx\N\empty\else
        \tikztiming@markcmd
        \path let \p1=(\N) in 
          node [timing/debug/node] at ($ (\x1,\noexpand\@tempa) + (0,1) $) {\N};
      \fi
    }%
  \endscope
}
%    \end{macrocode}
% \end{macro}
%
% \begin{macro}{\tikztiming@markcmd}
%    \begin{macrocode}
\def\tikztiming@markcmd{ \draw (\N) circle (1pt); }%
%    \end{macrocode}
% \end{macro}
%
% \begin{macro}{\tikztiming@mark@nodes}[1]{Comma-separated list of node names}
%    \begin{macrocode}
\newcommand*\tikztiming@mark@nodes[2][]{%
  \begingroup
    \def\@tempa{#2}%
    \ifx\@tempa\empty \let\@tempa\tikztiming@namednodes \fi
    \ifx\@tempa\space \let\@tempa\tikztiming@namednodes \fi
    \def\@tempb{\tikztiming@marknodes[#1]}%
    \expandafter\@tempb\expandafter{\expandafter{\@tempa}}%
  \endgroup
}
%    \end{macrocode}
% \end{macro}
%
% \subsection{Test for Code (`!' modifier)}
%
% \begin{macro}{\tikztiming@testforcode}
%    \begin{macrocode}
\def\tikztiming@testforcode{%
  \@ifnextchar{!}%
    {\tikztiming@testforcode@}%
    {\@ifundefined{tikztiming@code@\meaning\@let@token}%
      {\tikztiming@parser@}%
      {\expandafter\afterassignment
       \csname tikztiming@code@\meaning\@let@token\endcsname
       \let\@let@token=%
      }%
    }%
}
%    \end{macrocode}
% \end{macro}
%
% \begin{macro}{\tikztiming@testforcode@}[1]{Character to gobble}
%    \begin{macrocode}
\def\tikztiming@testforcode@#1{%
  \@ifnextchar\bgroup
    {\tikztiming@handlecode}%
    {%
      \PackageWarning{tikz-timing}{Missing braces after '!' character. Ignoring 
      this character}{}{}{}%
      \tikztiming@parser
    }%
}
%    \end{macrocode}
% \end{macro}
%
% \begin{macro}{\tikztiming@handlecode}[1]{Code to add to string}
%    \begin{macrocode}
\def\tikztiming@handlecode#1{%
  \tikztiming@output@addcode{ #1 }%
  \tikztiming@parser
}
%    \end{macrocode}
% \end{macro}
%
%
% \section{Parser Macros: Numbers}
%
% \begin{macro}{\tikztiming@testfornum}
%    \begin{macrocode}
\def\tikztiming@testfornum{%
  \global\let\tikztiming@num\tikztiming@wscale
  \let\tikztiming@numchars\empty
  \tikztiming@numfrac0\relax
  \afterassignment
  \tikztiming@testfornum@
  \tikztiming@numint0%
}
%    \end{macrocode}
% \end{macro}
%
% \begin{macro}{\tikztiming@testfornumfrac}
%    \begin{macrocode}
\def\tikztiming@testfornumfrac{%
  \afterassignment
  \tikztiming@testfornum@@@
  \tikztiming@numfrac1%
}
%    \end{macrocode}
% \end{macro}
%
% \begin{macro}{\tikztiming@numloop}
%    \begin{macrocode}
\def\tikztiming@numloop{%
  \ifnum\tikztiming@numint>0%
    \toks@\expandafter{\tikztiming@numchars}%
    \xdef\tikztiming@numchars{%
      \the\toks@
      \the\@temptokena
    }%
    \advance\tikztiming@numint by -1\relax
    \expandafter\tikztiming@numloop
  \fi
}
%    \end{macrocode}
% \end{macro}
%
% \begin{macro}{\tikztiming@testfornum@}
% The \cs{@let@token} set in \cs{tikztiming@parser} is used to handle "0" 
% which can not be detected using the \cs{tikztiming@numint} value.
%    \begin{macrocode}
\def\tikztiming@testfornum@{%
  \ifx0\@let@token
    \let\tikztiming@next\tikztiming@testfornum@@
  \else
  \ifnum0<\tikztiming@numint
    \let\tikztiming@next\tikztiming@testfornum@@
  \else
    \def\tikztiming@next{%
      \@ifnextchar{.}%
        {\expandafter\tikztiming@testfornumfrac\@gobble}%
        {%
          \tikztiming@numint1\relax
          \tikztiming@numfrac0\relax
          \global\let\tikztiming@num\tikztiming@wscale
          \if@tikztiming@metachar
            {\def\@tempa{\expandafter\expandafter\expandafter
             \tikztiming@parser
             \csname tikztiming@metachar@\meaning\@let@token\endcsname{1}}%
             \expandafter\@tempa\@gobble
            }%
            {\tikztiming@testforcode}%
        }%
    }%
  \fi\fi
  \tikztiming@next
}
%    \end{macrocode}
% \end{macro}
%
% \begin{macro}{\tikztiming@testfornum@@}
%    \begin{macrocode}
\def\tikztiming@testfornum@@{%
  \@ifnextchar{.}%
    {\expandafter\tikztiming@testfornumfrac\@gobble}%
    {\tikztiming@testfornum@@@}%
}
%    \end{macrocode}
% \end{macro}
%
% \begin{macro}{\tikztiming@testfornum@@@}
%    \begin{macrocode}
\def\tikztiming@testfornum@@@{%
  \edef\tikztiming@num{\the\tikztiming@numint.\expandafter\@gobble\the\tikztiming@numfrac}%
  \tikztiming@setnum{\tikztiming@wscale * \tikztiming@num}%
  \@ifnextchar\bgroup
    {%
      \expandafter\tikztiming@numfrac\expandafter0\expandafter
      \@gobble\the\tikztiming@numfrac\relax
      \ifnum0=\tikztiming@numfrac\else
        \pgfmathparse{round(\tikztiming@num)}%
        \PackageWarning{tikz-timing}%
          {Can not repeat group by a non-integer factor!^^J%
           Rounding '\tikztiming@num' to '\pgfmathresult'.}{}{}{}%
        \global\let\tikztiming@num\pgfmathresult
      \fi
      \tikztiming@testfornum@@@@
    }%
    {%
      \if@tikztiming@metachar
        {\def\@tempa{\expandafter\expandafter\expandafter
        \tikztiming@parser
        \csname tikztiming@metachar@\meaning\@let@token\expandafter\endcsname
        \expandafter{\tikztiming@num}}%
        \expandafter\@tempa\@gobble
        }%
        {\tikztiming@testforcode}%
    }%
}
%    \end{macrocode}
% \end{macro}
%
% \begin{macro}{\tikztiming@testfornum@@@@}[1]{Characters and other tokens to 
% repeat}
%    \begin{macrocode}
\def\tikztiming@testfornum@@@@#1{%
  \begingroup
    \@temptokena{#1}%
    \tikztiming@numloop%
  \endgroup
  \tikztiming@numint1\relax
  \tikztiming@numfrac0\relax
  \expandafter\tikztiming@parser\tikztiming@numchars
}
%    \end{macrocode}
% \end{macro}
%
% \section{Meta-Characters}
%
% \begin{macro}{\tikztimingmetachar}[1]{Meta-character}
% Checks if there is an optional argument and calls \cs{tikztiming@metachar} 
% appropriately.
%    \begin{macrocode}
\def\tikztimingmetachar#1{%
  \@ifnextchar{[}%]
    {\tikztiming@metachar{#1}}%
    {\tikztiming@metachar{#1}[]}%
}
%    \end{macrocode}
% \end{macro}
%
% \begin{macro}{\tikztiming@metachar}[3]{Macro character}{Number of 
% Arguments}{Substituted Characters}
%    \begin{macrocode}
\def\tikztiming@metachar#1[#2]#3{%
  \ifx\relax#3\relax
    \expandafter\let\csname tikztiming@metachar@\meaning#1\endcsname\@undefined
  \else
    \@namedef{tikztiming@metachar@\meaning#1}{}%
    \ifx\relax#2\relax
      \expandafter\renewcommand\expandafter*%
      \csname tikztiming@metachar@\meaning#1\expandafter
      \endcsname\expandafter[1]{##1{#3}}%
    \else
      \expandafter\renewcommand\expandafter*%
      \csname tikztiming@metachar@\meaning#1\expandafter
      \endcsname\expandafter[#2]{#3}%
    \fi
  \fi
  \ifnum\tikztiming@debug=-1
    \message{^^J Defined meta-char '#1' as: 
      \expandafter\meaning\csname tikztiming@metachar@\meaning#1\endcsname ^^J}%
  \fi
}
%    \end{macrocode}
% \end{macro}
%
% \begin{macro}{\if@tikztiming@metachar}[2]{True clause}{False clause}
% Only works in an argument of |\@ifnextchar|. This macro sets the needed 
% |\@let@token| macro.
%    \begin{macrocode}
\def\if@tikztiming@metachar#1#2{%
  \@ifundefined{tikztiming@metachar@\meaning\@let@token}{#2}{#1}%
}
%    \end{macrocode}
% \end{macro}
%
% \section{Table environment}
%    \begin{macrocode}
%\usetikzlibrary{backgrounds}
\newcounter{tikztimingrows}
%    \end{macrocode}
%
% \begin{environment}{tikztimingtable}[1]{Optional settings}
%    \begin{macrocode}
\def\abc{\tracingassigns=1}
\newenvironment{tikztimingtable}[1][]{%
  \begingroup
  \setcounter{tikztiming@nrows}{0}%
  \def\tikztiming@maxwidth{0.0}%
  \let\extracode\tikztiming@extracode
  \let\endextracode\tikztiming@endextracode
  \let\tablegrid\tikztiming@tablegrid
  \let\fulltablegrid\tikztiming@fulltablegrid
  \let\horlines\tikztiming@horlines
  \let\vertlines\tikztiming@vertlines
  \let\marknodes\tikztiming@mark@nodes
  \def\background{\pgfonlayer{background}\scope}%
  \def\endbackground{\endscope\endpgfonlayer}%
  \def\tablerules{\tikztiming@tablerules}%
  \def\tableheader{\tikztiming@tableheader}%
  \def\rowdist{\tikztiming@rowdist}%
  \def\coldist{\tikztiming@coldist}%
  \def\nrows{\the\c@tikztiming@nrows}%
  \def\twidth{\tikztiming@maxwidth}%
  \tikzpicture[timing/table,#1]%
    \coordinate (@last row)  at (0,\rowdist);
    \coordinate (label@header)   at ($ -1*(\tikztiming@coldist,0) + 1.5*(0,\tikztiming@rowdist) $);
    \coordinate (diagram@header) at ($                              1.5*(0,\tikztiming@rowdist) $);
    \coordinate (timing@table@bottom right) at (0,0);
    \coordinate (timing@table@bottom left)  at ($ -1*(\tikztiming@coldist,0) $);
    \tikztiming@picinit
    \tikzset{timing/before table}%
    \tikztimingtable@checkrow
}{%
  \tikzset{timing/after table}%
  \endtikzpicture
  \endgroup
}
%    \end{macrocode}
% \end{environment}
%
% \begin{macro}{\tikztiming@extracode}
% Set "all rows" rectangle node using the auxiliary coordinates.
%    \begin{macrocode}
\def\tikztiming@extracode{%
  \path
    let
      \p1 = (row1.north west),
      \p2 = (timing@table@bottom right),
      \p3 = (\x2-\x1,\y1-\y2),
      \p4 = (\pgf@pt@aa * \x3, \pgf@pt@bb * \y3)
    in
      node [shape=rectangle,anchor=north west,
            minimum width=\x4,minimum height=\y4,
            inner sep=0pt,outer sep=0pt]
        at (\x1,\y1) (all rows) {}
    ;
  \path
    let
      \p0 = ($ -1*(\tikztiming@coldist,0) + (0,1) $),
      \p1 = (label1.north east),
      \p2 = (timing@table@bottom left),
      \p3 = (\x0-\x2,\y1-\y2),
      \p4 = (\pgf@pt@aa * \x3, \pgf@pt@bb * \y3)
    in
      node [shape=rectangle,anchor=north east,
            minimum width=\x4,minimum height=\y4,
            inner sep=0pt,outer sep=0pt]
        at (\x1,\y1) (all labels) {}
    ;
   \@ifnextchar[{\tikztiming@oset}{}%
}
%    \end{macrocode}
% \end{macro}
%
% \begin{macro}{\tikztiming@endextracode}
% Now empty because \cs{tikztimingtable@chkendec} handles the
% continuation of the row scanning process.
%    \begin{macrocode}
\def\tikztiming@endextracode{%
}%
%    \end{macrocode}
% \end{macro}
%
% \begin{macro}{\tikztimingtable@checkrow}
% Checks the next token to be either \cs{end}, \cs{extracode} or
% \cs{begin}. In the case of the latter \cs{tikztimingtable@chkbegec}
% is called to test if it is \cs{begin}|{extracode}|.
%    \begin{macrocode}
\def\tikztimingtable@checkrow{%
  \@ifnextchar\end
    {\tikztimingtable@chkendec}%
    {\ifx\@let@token\extracode
       \let\tikztiming@next\empty
     \else
       \ifx\@let@token\begin
         \def\tikztiming@next{\tikztimingtable@chkbegec}%
       \else
         \def\tikztiming@next{\tikztimingtable@row}%
       \fi
     \fi
     \tikztiming@next
    }%
}
%    \end{macrocode}
% \end{macro}
%
% \begin{macro}{\tikztimingtable@chkbegec}
% Checks if value of \cs{begin} is |{extracode}|. If so
% it is executed, otherwise it is taken as normal row and so
% \cs{tikztimingtable@row} is called.
%    \begin{macrocode}
\def\tikztimingtable@chkbegec\begin#1{
  \begingroup
  \def\@tempa{#1}%
  \def\@tempb{extracode}%
  \ifx\@tempa\@tempb
    \endgroup
    \def\tikztiming@next{\begin{extracode}}%
  \else
    \endgroup
    \def\tikztiming@next{\tikztimingtable@row\begin{#1}}%
  \fi
  \tikztiming@next
}
%    \end{macrocode}
% \end{macro}
%
% \begin{macro}{\tikztimingtable@chkendec}
% Checks if value of \cs{end} is |{extracode}|. If so
% it is executed, otherwise it is taken as normal row and so
% \cs{tikztimingtable@row} is called.
%    \begin{macrocode}
\def\tikztimingtable@chkendec\end#1{
  \begingroup
  \def\@tempa{#1}%
  \def\@tempb{extracode}%
  \ifx\@tempa\@tempb
    \endgroup
    \def\tikztiming@next{\end{extracode}\tikztimingtable@row}%
  \else
    \endgroup
    \def\tikztiming@next{\end{#1}}%
  \fi
  \tikztiming@next
}
%    \end{macrocode}
% \end{macro}
%
% \begin{macro}{\tikztiming@oset}
% [1]{TikZ Keys}
% Handle optional TikZ settings. The \cs{tikz@options} must be used
% to ensure that the settings apply for the following graphic paths.
%    \begin{macrocode}
\def\tikztiming@oset[#1]{%
  \let\tikz@options\pgfutil@empty
  \tikzset{#1}%
  \tikz@options
}
%    \end{macrocode}
% \end{macro}
%
% \begin{macro}{\tikztiming@emptycell}
% Just used as marker. Needs unique definition.
%    \begin{macrocode}
\def\tikztiming@emptycell{%
  \@gobble{tikztiming@emptycell}%
}
%    \end{macrocode}
% \end{macro}
%
% \begin{macro}{\tikztimingtable@row}[1]{Row content}
%    \begin{macrocode}
\long\def\tikztimingtable@row#1\\{%
  \tikztimingtable@row@#1&\tikztiming@emptycell&\\
}
%    \end{macrocode}
% \end{macro}
%
% \begin{macro}{\tikztimingtable@row@}[3]{Cell 1}{Cell 2 or empty}{Potential 
% further cells}
%    \begin{macrocode}
\long\def\tikztimingtable@row@#1&#2&#3\\{%
  \ifx\\#3\\\else
    \begingroup
      \def\@tempa{\tikztiming@emptycell&}%
      \def\@tempb{#3}%
      \ifx\@tempa\@tempb\else
        \PackageWarning{tikz-timing}{%
          To many columns in tikztimingtable row! Only two are allowed%
        }{}{}{}%
      \fi
    \endgroup
  \fi
  \ifx\tikztiming@emptycell#2%
    \def\tikztiming@next{\tikztimingtable@row@@{#1}{}}%
  \else
    \def\tikztiming@next{\tikztimingtable@row@@{#1}{#2}}%
  \fi
  \tikztiming@next
}%
%    \end{macrocode}
% \end{macro}
%
% \begin{macro}{\tikztimingtable@row@@}[2]{Name/Label}{Timing characters}
%    \begin{macrocode}
\long\def\tikztimingtable@row@@#1#2{%
  \addtocounter{tikztiming@nrows}{1}%
  \coordinate (@last row) at ($ (@last row) - (0,\tikztiming@rowdist) $);
  \node [anchor=base east,timing/name,alias=last label] (label\the\c@tikztiming@nrows)
    at ($ (@last row) - (\tikztiming@coldist,0) $) {\ignorespaces #1\unskip\strut};
  \path let \p1 = (timing@table@bottom left), \p2 = (last label.south west) in
    coordinate (timing@table@bottom left) at ({min(\x1,\x2)},\y2);
  %
  \@ifnextchar{[}%
    {\tikztiming@tabletiming}%
    {\tikztiming@tabletiming[]}%
  #2\relax
  \path let \p1 = (timing@table@bottom right), \p2 = (timing/end base) in
    coordinate (timing@table@bottom right) at ({max(\x1,\x2)},\y2);
  %
  \pgfmathparse{max(\tikztiming@maxwidth,\tikztimingwidth)}%
  \let\tikztiming@maxwidth\pgfmathresult
  \tikztimingtable@checkrow
}
%    \end{macrocode}
% \end{macro}

% \begin{macro}{\tikztiming@tabletiming}
%    \begin{macrocode}
\def\tikztiming@tabletiming[#1]#2\relax{%
  \let\tikztiming@lastchar\empty
  \let\tikztiming@settings\empty
  \ifx\relax#1\relax\else
    \tikztiming@testoptarg#1\relax\relax
  \fi
  \ifx\tikztiming@lastchar\empty
    \def\tikztiming@lastchar{@}%
  \fi
  \edef\@tempa{\noexpand\timing
    [name=row\the\c@tikztiming@nrows,alias=last row,%
     \expandafter\unexpanded\expandafter{\tikztiming@settings}] at (@last row)}%
  \expandafter\@tempa\expandafter{\expandafter[\tikztiming@lastchar]#2};
}
%    \end{macrocode}
% \end{macro}

% \begin{macro}{\tikztiming@fulltablegrid}[1]{Optional settings}
%    \begin{macrocode}
\newcommand*\tikztiming@fulltablegrid[1][]{%
  \begin{pgfonlayer}{background}
    \scope[xstep={\timingwidth/2.},ystep={\timingheight/2.},
      shift={(all rows.south west)},timing/table/grid,#1]
      \draw (0,0) grid ($ (all rows.north east) - (all rows.south west) $);
   \endscope
  \end{pgfonlayer}
}
%    \end{macrocode}
% \end{macro}
%
% \begin{macro}{\tikztiming@tablegrid}[1]{Optional settings}
%    \begin{macrocode}
\def\tikztiming@tablegrid{%
  \@ifnextchar{*}%
    {\tikztiming@tablegrid@{row\row.high end}}%
    {\tikztiming@tablegrid@{all rows.north east}}%
}
%    \end{macrocode}
% \end{macro}
%
% \begin{macro}{\tikztiming@tablegrid@@}[2]{Optional settings}{End point}
%    \begin{macrocode}
\def\tikztiming@tablegrid@#1{%
  \@ifnextchar{[}%]
    {\tikztiming@tablegrid@@{#1}}%
    {\tikztiming@tablegrid@@{#1}[]}%
}
%    \end{macrocode}
% \end{macro}
%
% \begin{macro}{\tikztiming@tablegrid@@}[2]{Optional settings}{End point}
%    \begin{macrocode}
\def\tikztiming@tablegrid@@#1[#2]{%
  \begin{pgfonlayer}{background}
    \scope[xstep={\timingwidth/2.},ystep={\timingheight/2.},timing/table/grid,#2]
      \foreach \row in {1,...,\nrows} {%
        \draw {[shift={(row\row.low start)}]
          let \p1 = (#1) in (0,0) grid (\x1,1)};
      }%
    \endscope
  \end{pgfonlayer}
}
%    \end{macrocode}
% \end{macro}
%
% \begin{macro}{\tikztiming@tableheader}
% [3]{Optional Settings}{Text title}{Diagram title}
%    \begin{macrocode}
\def\tikztiming@tableheader{%
  \@ifnextchar[%]
    {\tikztiming@tableheader@}%
    {\tikztiming@tableheader@@{}[]}%
}
\def\tikztiming@tableheader@[#1]{%
  \@ifnextchar[%]
    {\tikztiming@tableheader@@{#1}}%
    {\tikztiming@tableheader@@{#1}[]}%
}
\def\tikztiming@tableheader@@#1[#2]#3{%
  \@ifnextchar[%]
    {\tikztiming@tableheader@@@{#1}{#2}{#3}}%
    {\tikztiming@tableheader@@@{#1}{#2}{#3}[]}%
}
%    \end{macrocode}
% \end{macro}
%
% \begin{macro}{\tikztiming@tableheader}
% [5]{Optional settings}{Optional label@header Settings}{Label title}{Optional Diagram title settings}{Diagram title}
%    \begin{macrocode}
\def\tikztiming@tableheader@@@#1#2#3[#4]#5{%
  \path
    let
        \p1 = (label@header),
        \p2 = (all labels.east),
        \p3 = (all labels.west)
     in
        (\x2,\y1) -- node [pos=0,anchor=base east,timing/table/header,timing/table/label header,#1,#2]
          (label header)   {\ignorespaces #3\unskip\strut} (\x3,\y1)
     ;
  \path
    let
        \p4 = (diagram@header),
        \p5 = (all rows.west),
        \p6 = (all rows.east)
     in
        (\x5,\y4) -- node [pos=0,anchor=base west,timing/table/header,timing/table/row header,#1,#4]
          (row header) {\ignorespaces #5\unskip\strut} (\x6,\y4)
     ;
  \tikztiming@tableheadtrue
}
\newif\iftikztiming@tablehead
%    \end{macrocode}
% \end{macro}
%
% \begin{macro}{\tikztiming@tablerules}[1]{Optional Settings}
%    \begin{macrocode}
\newcommand*\tikztiming@tablerules[1][]{%
  \begin{scope}[#1]
    \coordinate (NW) at (current bounding box.north west);
    \coordinate (SE) at (current bounding box.south east);
    %
    \draw [timing/table/rules]
    let
      \p1 = (NW),
      \p2 = (SE),
      \p3 = (@last row),
      \p4 = (0,\rowdist)
    in
      ($ (\x1-\tabcolsep,\y3) - (0,.5*\y4) $) --
      ($ (\x2+\tabcolsep,\y3) - (0,.5*\y4) $)
     \iftikztiming@tablehead
      (\x1-\tabcolsep,2.5*\y4) -- (\x2+\tabcolsep,2.5*\y4)
     \fi
    ;
    \draw
     \iftikztiming@tablehead [timing/table/midrules] \else [timing/table/rules] \fi
    let
      \p1 = (NW),
      \p2 = (SE)
    in
      (\x1-\tabcolsep,\rowdist) -- (\x2+\tabcolsep,\rowdist)
    ;
  \end{scope}
}
%    \end{macrocode}
% \end{macro}
%
% \begin{macro}{\tikztiming@horlines}[2]{Optional Settings}{Row numbers}
%    \begin{macrocode}
\newcommand*\tikztiming@horlines[2][]{%
  \begingroup
    \def\list{#2}%
    \ifx\list\empty
      \def\list{1,2,...,\nrows}%
    \fi
    \foreach \row in \list {%
      \draw [timing/table/lines,#1] let
        \p1 = (all rows.south east),
        \p2 = (0,\rowdist)
      in
        ($ (0,\y2) -  (0,\row * \y2) $) -- +(\x1,0);
    }%
  \endgroup
}
%    \end{macrocode}
% \end{macro}
%
% \begin{macro}{\tikztiming@vertlines}[2]{Optional Settings}{Column numbers}
%    \begin{macrocode}
\newcommand*\tikztiming@vertlines[2][]{%
  \begingroup
    \def\list{#2}%
    \ifx\list\empty
      \def\list{0,1,...,\twidth}%
    \fi
    \draw [timing/table/lines,#1] let
        \p1 = ($ (all rows.south east) - (0,2) $)
      in
        \foreach \clk in \list {
          (\clk,+1.5) -- +(0,\y1)
        }
    ;
  \endgroup
}
%    \end{macrocode}
% \end{macro}
%
% \section{Shapes}
%    \begin{macrocode}
\pgfdeclareshape{tikztiming@shape}{%
  \savedanchor\northeast{\tikztiming@northeast}%
  \savedanchor\startpoint{\tikztiming@startpoint}%
  \savedanchor\endpoint{\tikztiming@endpoint}%
  \anchor{center}{\pgfpointorigin}%
  \anchor{start}{\startpoint}%
  \anchor{end}{\endpoint}%
  %
  \anchor{origin}{\northeast \pgf@x=-\pgf@x \pgf@y=-\pgf@y }%
  \anchor{east}{\northeast \pgf@y=0pt }%
  \anchor{west}{\northeast \pgf@y=0pt \pgf@x=-\pgf@x }%
  \anchor{north}{\northeast \pgf@x=0pt }%
  \anchor{north west}{\northeast \pgf@x=-\pgf@x }%
  \anchor{north east}{\northeast}
  \anchor{high mid}{\northeast \pgf@x=0pt }%
  \anchor{high start}{\northeast \pgf@x=-\pgf@x }%
  \anchor{high end}{\northeast}
  \anchor{south}{\northeast \pgf@x=0pt \pgf@y=-\pgf@y }%
  \anchor{south west}{\northeast \pgf@x=-\pgf@x \pgf@y=-\pgf@y }%
  \anchor{south east}{\northeast \pgf@y=-\pgf@y }%
  \anchor{low mid}{\northeast \pgf@x=0pt \pgf@y=-\pgf@y }%
  \anchor{low start}{\northeast \pgf@x=-\pgf@x \pgf@y=-\pgf@y }%
  \anchor{low end}{\northeast \pgf@y=-\pgf@y }%
  \anchor{mid}{\pgfpointorigin}%
  \anchor{mid east}{\northeast \pgf@y=0pt }%
  \anchor{mid west}{\northeast \pgf@y=0pt \pgf@x=-\pgf@x }%
  \anchor{mid end}{\northeast \pgf@y=0pt }%
  \anchor{mid start}{\northeast \pgf@y=0pt \pgf@x=-\pgf@x }%
  \anchor{base}{\northeast \pgf@x=0pt \pgf@y=-\pgf@y }%
  \anchor{base west}{\northeast \pgf@x=-\pgf@x \pgf@y=-\pgf@y }%
  \anchor{base east}{\northeast \pgf@y=-\pgf@y }%
  \anchorborder{%
    \@tempdima=\pgf@x
    \@tempdimb=\pgf@y
    \pgfpointborderrectangle{\pgfpoint{\@tempdima}{\@tempdimb}}{\northeast}%
  }%
}
%    \end{macrocode}
%
% By default use the old node centers for backwards compatibility.
%    \begin{macrocode}
\gdef\tikztiming@oldnodecenter@{}%
\tikzset{timing/old@node@center/.code={\xdef\tikztiming@oldnodecenter@{%
  ++($ (0,0) - #1 $)
  [timing/no@old@node@center]}}%
}%
\tikzset{timing/no@old@node@center/.code={\gdef\tikztiming@oldnodecenter@{}}}%

\def\tikztiming@oldnodecenter#1{%
  [timing/old@node@center={#1}]
  ++#1
}

%    \end{macrocode}
%
% \section{Other Macros}
%
% \begin{macro}{\tikztiming@iflower}[1]{Character}
%    \begin{macrocode}
\def\tikztiming@iflower#1{%
  \begingroup
  \edef\@tempa{`#1}%
  \ifnum\@tempa=\lccode\@tempa
    \endgroup
    \expandafter
    \@firstoftwo
  \else
    \endgroup
    \expandafter
    \@secondoftwo
  \fi
}
%    \end{macrocode}
% \end{macro}
%
% \begin{macro}{\tikztiming@internaldefs}[3]{Character}{Character}{Character}
%    \begin{macrocode}
\def\tikztiming@internaldefs{%
  \let\draw\relax
  \let\path\relax
  \let\p\relax
  \let\x\relax
  \let\y\relax
  \let\n\relax
  \let\tikztiming@oldnodecenter\relax
  \let\tikztiming@oldnodecenter@\relax
  \def\width{####1*\noexpand\timingwidth}%
  \def\fwidth{\noexpand\timingwidth}%
  \def\height{\noexpand\timingheight}%
  \def\slope{\noexpand\timingslope}%
  \def\zslope{\noexpand\timingzslope}%
  \def\dslope{\noexpand\timingdslope}%
  \def\gslope{0}%
  \def\style{timing/####3}%
  \def\bgstyle{draw=none,timing/####3/background}%
  \def\nstyle{timing/####4}%
  \def\newdraw{\tikztiming@newdraw}%
  \def\newdrawns{\tikztiming@newdraw@nosave}%
  \def\code##1{ [/utils/exec={\unexpanded{##1}}] }%
}
%    \end{macrocode}
% \end{macro}

% \begin{macro}{\tikztimingsetslope}[1]{Slope value}
%    \begin{macrocode}
\def\tikztimingsetslope#1{%
  \pgfmathparse{min(1.0,{max(0.0,#1)})}%
  \let\tikztiming@slope\pgfmathresult
  \edef\timingslope{\tikztiming@slope*\noexpand\timingwidth}%
}
%    \end{macrocode}
% \end{macro}
%
% \begin{macro}{\tikztimingsetdslope}[1]{Slope value}
%    \begin{macrocode}
\def\tikztimingsetdslope#1{%
  \pgfmathparse{min(1.0,{max(0.0,#1)})}%
  \let\tikztiming@dslope\pgfmathresult
  \edef\timingdslope{\tikztiming@dslope*\noexpand\timingwidth}%
}
%    \end{macrocode}
% \end{macro}
%
% \begin{macro}{\tikztimingsetzslope}[1]{Slope value}
%    \begin{macrocode}
\def\tikztimingsetzslope#1{%
  \pgfmathparse{min(1.0,{max(0.0,#1)})}%
  \let\tikztiming@zslope\pgfmathresult
  \edef\timingzslope{\tikztiming@zslope*\noexpand\timingwidth}%
}
%    \end{macrocode}
% \end{macro}
%
%    \begin{macrocode}
\tikztimingsetslope{0.10}%
\tikztimingsetdslope{0.20}%
\tikztimingsetzslope{0.05}%
%    \end{macrocode}
%
% \section{Timing Character Definition Macros}
%
% \begin{macro}{\tikztiminguse}[1]{Character(s)}
%    \begin{macrocode}
\def\tikztiminguse#1#2{%
  \@nameuse{\tikztiming@prefix#1}{#2}{##2}{##3}{##4}%
}
%    \end{macrocode}
% \end{macro}
%
% \begin{macro}{\tikztimingdef}[1]{One till three characters}
%    \begin{macrocode}
\def\tikztimingdef#1{%
  \tikztimingdef@#1\empty\empty\relax
}
%    \end{macrocode}
% \end{macro}
%
% \begin{macro}{\tikztimingdef@}[3]{First character}{Second character or 
% empty}{Definition code}
%    \begin{macrocode}
\def\tikztimingdef@#1#2#3\relax#4{%
  \ifx\empty#2\empty
    \tikztiming@nameedef{}{#1}{}{#4}%
  \else
    \tikztiming@nameedef{#1}{#2}{#3}{#4}%
  \fi
}
%    \end{macrocode}
% \end{macro}
%
% \begin{macro}{\tikztimingcpy}[2]{New Character(s)}{Source Character(s)}
%    \begin{macrocode}
\def\tikztimingcpy#1#2{%
  \begingroup
  \edef\tikztimingcpy@a{#1}%
  \edef\tikztimingcpy@b{#2}%
  \ifx\tikztimingcpy@a\tikztimingcpy@b
    \endgroup
    %\message{^^J Copy macro '#2' to '#1' skipped.^^J}%
  \else
    \endgroup
    \@ifundefined{\tikztiming@pfxtr#2\empty\empty\empty}%
      {\ifnum\tikztiming@debug>0\relax
        \message{^^J Copy macro '#2' to '#1' failed.^^J}\fi}%
      {\ifnum\tikztiming@debug>0\relax
        \message{^^J Copy macro '#2' to '#1'.^^J}\fi
        \expandafter\let
        \csname\tikztiming@pfxtr#1\empty\empty\empty\expandafter\endcsname
        \csname\tikztiming@pfxtr#2\empty\empty\empty\endcsname
      }%
  \fi
}
%    \end{macrocode}
% \end{macro}
%
%
% \begin{macro}{\tikztiminglet}[2]{Original characters}{New characters}
%    \begin{macrocode}
\def\tikztiminglet#1#2{%
  \tikztiming@namelet{#1}{#2}%
}
%    \end{macrocode}
% \end{macro}
%
%
% \begin{macro}{\tikztiming@chars}[1]{Comma to gobble}
% Initial definition of character list. Will gobble the separation comma in 
% front of the first character which is added to the list.
%    \begin{macrocode}
\def\tikztiming@chars#1{}
%    \end{macrocode}
% \end{macro}
%
% \begin{macro}{\tikztiming@ifcharexists}[1]{Character to check if in list}
%    \begin{macrocode}
\def\tikztiming@ifcharexists#1{%
  \def\tikztiming@ifcharexists@##1,#1,##2\relax{%
    \ifx\relax##2\relax%
      \expandafter\@firstoftwo
    \else
      \expandafter\@secondoftwo
    \fi
  }%
  \expandafter\tikztiming@ifcharexists@
  \expandafter,\tikztiming@chars,#1,\relax%
}
%    \end{macrocode}
% \end{macro}
%
% \begin{macro}{\tikztiming@addchar}[1]{Character}
%    \begin{macrocode}
\def\tikztiming@addchar#1{%
  \tikztiming@ifcharexists{#1}{%
    \edef\tikztiming@chars{\tikztiming@chars,#1}%
  }{}%
}
%    \end{macrocode}
% \end{macro}
%
% \begin{macro}{\tikztimingchar}[1]{New character}
%    \begin{macrocode}
\def\tikztimingchar#1{%
  \uppercase{%
  \tikztiming@addchar{#1}%
  \tikztimingchar@{#1}}%
}
%    \end{macrocode}
% \end{macro}
%
%    \begin{macrocode}
\@namedef{\tikztiming@prefix @@}#1#2#3#4{ coordinate (timing/start)}%
%    \end{macrocode}
%
% \begin{macro}{\tikztimingchar@}[3]{Character}{Start position}{Character 
% drawing code}
%    \begin{macrocode}
\def\tikztimingchar@#1#2#3{%
  \tikztiming@nameedef{@}{#1}{}{#2 coordinate (timing/start) #3}%
  \tikztimingdef{{#1}{#1}}{#3}%
}
%    \end{macrocode}
% \end{macro}
%
% \begin{macro}{\tikztimingecopy}[2]{New character}{Original character}
%    \begin{macrocode}
\def\tikztimingecopy#1#2{%
  \@namedef{tikztiming@alias@#1}{#2}%
  \tikztiming@addchar{#1}%
  \expandafter\let
  \csname tikztiming@triple@#1\expandafter\endcsname
  \csname tikztiming@triple@#2\endcsname
  \expandafter\let
  \csname tikztiming@nocombine@#1\expandafter\endcsname
  \csname tikztiming@nocombine@#2\endcsname
}
%    \end{macrocode}
% \end{macro}
%
% \begin{macro}{\tikztiming@setref}[2]{Reference Name}{Shift Offset}
%    \begin{macrocode}
\def\tikztiming@setref#1#2{%
  coordinate (#1) coordinate (#1) at ([shift={(#2)}]#1)
}
%    \end{macrocode}
% \end{macro}
%
% \section{NewDraw Macros}
%
% \begin{macro}{\tikztiming@newdraw}
%    \begin{macrocode}
\def\tikztiming@newdraw{%
  node [timing/save] (timing@save) {};%
  \draw [timing/restore] (timing@save) ++(0,0)
}
%    \end{macrocode}
% \end{macro}

% \begin{macro}{\tikztiming@newdraw@nosave}
%    \begin{macrocode}
\def\tikztiming@newdraw@nosave{%
  node [timing/coord] (timing@save) {};%
  \draw [timing/nooptions] (timing@save) ++(0,0)
}
%    \end{macrocode}
% \end{macro}
%
% \begin{macro}{\tikztiming@baseline}
%    \begin{macrocode}
\def\tikztiming@baseline{%
  node [timing/save] (timing@save) {};%
  \draw [timing/restore] let \p1 = (timing@save) in (\x1,0) ++(0,0)
}
%    \end{macrocode}
% \end{macro}
%
% \section{Definition of Timing Characters}
%
% Pre-define list of timing characters to have it in a sorted fashion.
%    \begin{macrocode}
\def\tikztiming@chars{H,L,Z,X,M,D,U,T,C}%
%    \end{macrocode}
%
% \subsection{Character `S'}
% This character is now implemented as modifier to avoid hassle with triple 
% character transitions.
%    \begin{macrocode}
\tikztiming@defcode{S}{%
  \pgfmathparse{\tikztimingwidth + \tikztiming@num}%
  \global\let\tikztimingwidth\pgfmathresult
  \tikztiming@output@Eaddcode{%
    coordinate (timing@ref)  at ($ (timing@ref)  + (\tikztiming@num,0) $)
    coordinate (timing@refa) at ($ (timing@refa) + (\tikztiming@num,0) $)
    coordinate (timing@refb) at ($ (timing@refb) + (\tikztiming@num,0) $)
    ++(\tikztiming@num,0)
  }%
  \tikztiming@parser
}

\tikztiming@defcode{s}{%
  \pgfmathparse{\tikztimingwidth + \tikztiming@num/2}%
  \global\let\tikztimingwidth\pgfmathresult
  \tikztiming@output@Eaddcode{%
    coordinate (timing@ref)  at ($ (timing@ref)  + (\tikztiming@num/2,0) $)
    coordinate (timing@refa) at ($ (timing@refa) + (\tikztiming@num/2,0) $)
    coordinate (timing@refb) at ($ (timing@refb) + (\tikztiming@num/2,0) $)
    ++(\tikztiming@num/2,0)
  }%
  \tikztiming@parser
}
%    \end{macrocode}
%
% \subsection{Character `G'}
%    \begin{macrocode}
\tikztiming@defcode{G}{%
  \tikztiming@output@addcode{%
    \tikztiming@glitch
  }%
  \tikztiming@parser
}

\tikztiming@defcode{g}{%
  \tikztiming@output@addcode{%
    \tikztiming@glitch
  }%
  \tikztiming@parser
}

\def\tikztiming@glitch{%
  {[timing/g] let \p1 = +(0,0) in (\x1,0) -- (\x1,\timingheight)
   coordinate (timing@dstart) at (\x1,\timingheight/2) (\x1,\y1) 
  }%
}
%    \end{macrocode}
%
% \subsection{Character `H'}
%    \begin{macrocode}
\tikztimingchar{H}{++(0,\height)}{-- ++(#1,0)}

\tikztimingdef{HH}{
  -- ++(#1,0)
}

\tikztimingdef{LH}{
  --  +(\slope, \height)
  -- ++(#1, \height)
}

\tikztimingdef{ZH}{
    \newdraw
  -- ++(\zslope,+\height/2.)
  -- ++($ (#1,0) - (\zslope,0) $)
}

\tikztimingdef{DH}{
  \tikztiming@oldnodecenter@%
  \newdraw [\style]
  --  +(\dslope/2.,\height/2.)
  --   ($ (timing@ref) + (#1,\height/2) $)
}

%    \end{macrocode}
%
% \subsection{Character `L'}
%    \begin{macrocode}
\tikztimingdef{DL}{
  \tikztiming@oldnodecenter@
  \newdraw [\style]
  --  +(\dslope/2.,-\height/2.)
  --   ($ (timing@ref) + (#1,-\height/2) $)
}

%    \end{macrocode}
%
% \subsection{Character `'}
%    \begin{macrocode}

\tikztimingchar{L}{++(0,0)}{-- ++(#1,0)}

\tikztimingchar{Z}{++(0,\height/2.)}{%
  \newdraw [\style]
  -- ++(#1,0)
}

\tikztimingchar{D}{++(0,\height/2.)}{
  node [timing/save] (timing@save) {}; \path [\bgstyle] (timing@save) ++(0,0)
  -- +(#1,0)
  -- +($ (#1,0) + 0.5*(\dslope,\height) $)
  -- +(#1,\height)
  -- +(0,\height)
  -- cycle;
  \draw [timing/restore,\style] (timing@save) ++(0,0)
  node [timing/save] (timing@dstart) at +(-\dslope/2.,\height/2.0) {}
  --  +(#1,0) ++(0,+\height)
  -- ++(#1,0) ++(0,-\height)
}

\def\tikztiming@triple@D{}%
\def\tikztiming@triple@D@edge@{}%
\def\tikztiming@nocombine@D@edge@{}%
\def\tikztiming@alias@D@edge@{D}%
%    \end{macrocode}
%
% \begin{macro}{\tikztiming@trans@D@fill}[2]{Original width}{Width to subtract}
%    \begin{macrocode}
\def\tikztiming@trans@D@fill#1#2{%
  node [timing/save] (timing@save) {}; \path [\bgstyle] (timing@save) ++(0,0)
  -- +(0.5*\dslope,-0.5*\height)
  -- ++($ (#1,-0.5*\height) - (#2,0) $)
  -- +(0.5*\dslope,0.5*\height)
  -- +(0,\height)
  -- ++($ (#2,\height) - (#1,0) + (0.5*\dslope,0) $)
  -- cycle;
  \draw [timing/restore,\style] (timing@save) ++(0,0)
  node [timing/save] (timing@dstart) {}
}
%    \end{macrocode}
% \end{macro}
%
%    \begin{macrocode}
\tikztimingdef{LL}{
  -- ++(#1,0)
}

\tikztimingdef{HL}{
  -- ++(\slope,-\height)
  \tikztiminguse{HH}{#1-\slope}
}


\tikztimingdef{LZ}{
  \newdraw [\style]
  -- ++(\zslope,+\height/2.)
  -- ++($ (#1,0) - (\zslope,0) $)
}
\tikztimingdef{HZ}{%
  \newdraw [\style]
  -- ++(\zslope,-\height/2.)
  -- ++($ (#1,0) - (\zslope,0) $)
}
\tikztimingdef{ZL}{%
  \newdraw
  -- ++(\zslope,-\height/2.)
  -- ++($ (#1,0) - (\zslope,0) $)
}

\tikztimingdef{DZ}{
  \tikztiming@oldnodecenter@
  \newdraw [\style]
  -- ($ (timing@ref) + (#1,0) $)
}

\def\tikztiming@chard#1{%
  \newdraw [\style,preaction={\bgstyle}]
  coordinate (timing@dstart) 
  -- +(\dslope/2,\height/2)
  --  ($ (timing@ref) + (#1,\height/2) $)
  -- +(\dslope/2,-\height/2)
  -- +(0,-\height)
  --  ($ (timing@dstart) + (\dslope/2,-\height/2) $)
  --  (timing@dstart)
      ($ (timing@ref) + (#1,0) $)
  coordinate (timing@ref)
    ++(\dslope/2,0)
  coordinate (timing@dend)
  \tikztiming@oldnodecenter{%
    (-\dslope/2,-\height/2)
  }%
  \newdraw
}

\def\tikztiming@chard@#1{%
  \newdraw [\bgstyle]
  -- +(\dslope/2,\height/2)
  --  ($ (timing@ref) + (#1,\height/2) $)
  -- +(0,-\height)%
  --  ($ (timing@save) + (\dslope/2,-\height/2) $)
  --  (timing@save)
  \newdraw [\style]
     coordinate (timing@dstart)
  -- +(\dslope/2,\height/2)
  --  ($ (timing@ref) + (#1,\height/2) $)
     +(0,-\height)%
  --  ($ (timing@save) + (\dslope/2,-\height/2) $)
  --  (timing@save)
      ($ (timing@ref) + (#1,0) $)
     coordinate (timing@ref)
     \tikztiming@setref{timing@dend}{\dslope/2,0}
  \tikztiming@oldnodecenter{%
    (0,-\height/2)
  }%
  \newdraw
}

\def\tikztiming@@chard#1{%
    ++(0,\height/2)
  \newdraw [\bgstyle]
  -- +(0,\height/2)
  --  ($ (timing@ref) + (#1,\height/2) $)
  -- +(\dslope/2,-\height/2)
  -- +(0,-\height)
  --  ($ (timing@save) + (0,-\height/2) $)
  --  (timing@save)
  \newdraw [\style]
     +(0,0)
     coordinate (timing/start)
     +(-\dslope/2,0)
     coordinate (timing@dstart)
     +(0,\height/2)
  --  ($ (timing@ref) + (#1,\height/2) $)
  -- +(\dslope/2,-\height/2)
  -- +(0,-\height)
  --  ($ (timing@save) + (0,-\height/2) $)
      ($ (timing@ref) + (#1,0) $)
     coordinate (timing@ref)
    ++(\dslope/2,0)
     coordinate (timing@dend)
  \tikztiming@oldnodecenter{%
    (-\dslope/2,-\height/2)
  }%
  \newdraw
}

\def\tikztiming@@chard@#1{%
    ++(0,\height/2)
  \newdraw [\bgstyle]
  -- +(0,\height/2)
  --  ($ (timing@ref) + (#1,\height/2) $)
  -- +(0,-\height)
  --  ($ (timing@save) + (0,-\height/2) $)
  --  (timing@save)
  \newdraw [\style]
     +(0,0)
     coordinate (timing/start)
     coordinate (timing@ref)
     \tikztiming@setref{timing@dstart}{-\dslope/2,0}
     +(0,\height/2)
  -- +(#1,\height/2)
     +(#1,-\height/2)
  --  ($ (timing@save) + (0,-\height/2) $)
    ++(#1,\height/2)
     coordinate (timing@ref)
     \tikztiming@setref{timing@dend}{\dslope/2,0}
  \tikztiming@oldnodecenter{%
    (0,-\height/2)
  }%
  \newdraw
}

\def\tikztiming@dchardu#1{%
  \tikztiming@oldnodecenter@
  \newdraw [\bgstyle]
  -- +(0,\height/2)
  --  ($ (timing@ref) + (#1,\height/2) $)
  -- +(\dslope/2,-\height/2)
  -- +(0,-\height)
  --  ($ (timing@save) + (0,-\height/2) $)
  --  (timing@save)
  \newdraw [\style]
     +(0,\height/2)
  --  ($ (timing@ref) + (#1,\height/2) $)
  -- +(\dslope/2,-\height/2)
  -- +(0,-\height)
  --  ($ (timing@save) + (0,-\height/2) $)
      ($ (timing@ref) + (#1,0) $)
     coordinate (timing@ref)
    ++(\dslope/2,0)
     coordinate (timing@dend)
  \tikztiming@oldnodecenter{%
    (-\dslope/2,-\height/2)
  }%
  \newdraw
}

\def\tikztiming@uchardd#1{%
  \tikztiming@oldnodecenter@
  \newdraw [\bgstyle]
  -- +(\dslope/2,\height/2)
  --  ($ (timing@ref) + (#1,\height/2) $)
  -- +(0,-\height)
  --  ($ (timing@save) + (\dslope/2,-\height/2) $)
  --  (timing@save)
  \newdraw [\style]
     coordinate (timing@dstart)
  -- +(\dslope/2,\height/2)
  --  ($ (timing@ref) + (#1,\height/2) $)
     +(0,-\height)
  --  ($ (timing@save) + (\dslope/2,-\height/2) $)
  --  (timing@save)
      ($ (timing@ref) + (#1,0) $)
     coordinate (timing@ref)
     coordinate [shift={(\dslope/2,0)}](timing@dend)
  \tikztiming@oldnodecenter{%
      ($ (0,0) - (0,\height/2) $)
  }%
  \newdraw
}

\tikztimingdef{DD}{%
  \tikztiming@dchardu{#1}%
}

\tikztimingdef{UU}{%
  \tikztiming@dchardu{#1}%
}

\def\tikztiming@dchard@#1{%
  \tikztiming@oldnodecenter@
  \newdraw [\bgstyle]
  -- +(0,\height/2)
  --  ($ (timing@ref) + (#1,\height/2) $)
  -- +(0,-\height)
  --  ($ (timing@save) + (0,-\height/2) $)
  --  (timing@save)
  \newdraw [\style]
     +(0,\height/2)
  --  ($ (timing@ref) + (#1,\height/2) $)
     +(0,-\height)
  --  ($ (timing@save) + (0,-\height/2) $)
      ($ (timing@ref) + (#1,0) $)
    coordinate (timing@ref)
    \tikztiming@setref{timing@dend}{\dslope/2,0}
  \tikztiming@oldnodecenter{%
    (0,-\height/2)
  }%
  \newdraw
}

\tikztimingdef{DD@}{
  \tikztiming@dchard@{#1}%
}

\tikztimingcpy{DDD}{DD@}
\tikztimingcpy{UUU}{DDD}

\tikztimingdef{{D@edge@}{D@edge@}}{%
  \tikztiming@oldnodecenter@
  \tikztiming@chard{#1}%
}%

\tikztimingdef{@{D@edge@}}{%
  \tikztiming@@chard@{#1}%
}%

\tikztimingdef{@{D@edge@}@}{%
  \tikztiming@@chard@{#1}%
}%

\tikztimingdef{@{D@edge@}D}{%
  \tikztiming@@chard{#1}%
}

\tikztimingdef{D{D@edge@}}{%
  \tikztiming@dchardu{#1}%
}

\tikztimingdef{U{U@edge@}}{%
  \tikztiming@dchardu{#1}%
}

\tikztimingdef{U{D@edge@}}{%
  \tikztiming@oldnodecenter@
  \tikztiming@chard{#1}%
}

\tikztimingdef{U{D@edge@}@}{%
  \tikztiming@oldnodecenter@%
  \tikztiming@chard@{#1}%
}

\tikztimingdef{D{U@edge@}}{%
  \tikztiming@oldnodecenter@%
  \tikztiming@chard{#1}%
}

\tikztimingdef{D{U@edge@}@}{%
  \tikztiming@oldnodecenter@%
  \tikztiming@chard@{#1}%
}

\tikztimingdef{D{D@edge@}@}{%
  \tikztiming@dchard@{#1}%
}%

\tikztimingdef{U{U@edge@}@}{%
  \tikztiming@dchard@{#1}%
}%

\tikztimingdef{{D@edge@}{D@edge@}@}{%
  \tikztiming@oldnodecenter@%
  \tikztiming@chard@{#1}%
}%

\tikztimingdef{@{D@edge@}}{%
  \tikztiming@@chard{#1}%
}%

\tikztimingdef{{D@edge@}D}{%
  \tikztiming@oldnodecenter@%
  \tikztiming@chard{#1}%
}%

\tikztimingdef{{D@edge@}D@}{%
  \tikztiming@oldnodecenter@%
  \tikztiming@chard@{#1}%
}%

\tikztimingdef{ZDZ}{
  coordinate (timing@ref)
  \tikztiming@chard{#1}
}
\tikztimingcpy{ZDX}{ZDZ}
\tikztimingcpy{XDX}{ZDZ}
\tikztimingcpy{XDZ}{ZDZ}

\tikztimingdef{ZDD}{
  coordinate (timing@ref)
  \tikztiming@chard@{#1}
}

\tikztimingdef{ZDU}{
  coordinate (timing@ref)
  \tikztiming@chard{#1}
}

\tikztimingcpy{ZUD}{ZDU}
\tikztimingcpy{ZUU}{ZDD}

\tikztimingdef{ZD}{
  coordinate (timing@ref)
  \tikztiming@chard{#1}
}

\tikztimingdef{@DD}{%
  \tikztiming@@chard@{#1}%
}

\tikztimingdef{ZD@}{%
  coordinate (timing@ref)
  \tikztiming@chard@{#1}%
}

\tikztimingdef{HD@}{%
  \tikztiming@setref{timing@ref}{0,-\height/2}
  -- ++(\dslope/2, -\height/2)
  \tikztiming@chard@{#1}%
}

\tikztimingdef{TD@}{%
  \tikztiming@setref{timing@ref}{0,-\height/2 * \value{tikztimingtrans}}
  -- ++(\dslope/2, -\height/2 * \value{tikztimingtrans})
  \tikztiming@chard@{#1}%
}

\tikztimingdef{TD}{%
  \code{\setcounter{tikztimingtrans}{-\value{tikztimingtrans}}}
  \tikztiming@setref{timing@ref}{0,\height/2 * \value{tikztimingtrans}}
  -- ++(\dslope/2, \height/2 * \value{tikztimingtrans})
  \tikztiming@chard{#1}%
}

\tikztimingdef{TDD}{%
  \code{\setcounter{tikztimingtrans}{-\value{tikztimingtrans}}}
  \tikztiming@setref{timing@ref}{0,\height/2 * \value{tikztimingtrans}}
  -- ++(\dslope/2, \height/2 * \value{tikztimingtrans})
  \tikztiming@chard@{#1}%
}

\tikztimingdef{TUD}{%
  \code{\setcounter{tikztimingtrans}{-\value{tikztimingtrans}}}
  \tikztiming@setref{timing@ref}{0,\height/2 * \value{tikztimingtrans}}
  -- ++(\dslope/2, \height/2 * \value{tikztimingtrans})
  \tikztiming@chard{#1}%
}

\tikztimingcpy{TDU}{TUD}
\tikztimingcpy{TUU}{TDD}

\tikztimingdef{LD@}{%
  \tikztiming@setref{timing@ref}{0,\height/2}
  -- ++(\dslope/2, \height/2)
  \tikztiming@chard@{#1}%
}

\tikztimingdef{@D}{%
  \tikztiming@@chard@{#1}%
}

\tikztimingdef{@D@}{%
  \tikztiming@@chard@{#1}%
}

\tikztimingcpy{H{D@edge@}@}{HD@}
\tikztimingcpy{L{D@edge@}@}{LD@}
\tikztimingcpy{Z{D@edge@}@}{ZD@}
\tikztimingcpy{T{D@edge@}@}{TD@}

\tikztimingdef{@DZ}{%
  \tikztiming@@chard{#1}%
}
\tikztimingdef{@DH}{%
  \tikztiming@@chard{#1}%
}
\tikztimingdef{@DL}{%
  \tikztiming@@chard{#1}%
}
\tikztimingdef{@DT}{%
  \tikztiming@@chard{#1}%
}

\tikztimingdef{LD}{
  \tikztiming@setref{timing@ref}{0,\height/2}
  -- ++(\dslope/2, \height/2)
  \tikztiming@chard{#1}%
}

\tikztimingdef{LDD}{
  \tikztiming@setref{timing@ref}{0,\height/2}
  -- ++(\dslope/2, \height/2)
  \tikztiming@chard@{#1}
}

\tikztimingdef{LDU}{
  \tikztiming@setref{timing@ref}{0,\height/2}
  -- ++(\dslope/2, \height/2)
  \tikztiming@chard{#1}
}

\tikztimingcpy{LUD}{LDU}
\tikztimingcpy{LUU}{LDD}

\tikztimingdef{HD}{
  \tikztiming@setref{timing@ref}{0,-\height/2}
  -- ++(\dslope/2, -\height/2)
  \tikztiming@chard{#1}
}

\tikztimingdef{HDD}{
  \tikztiming@setref{timing@ref}{0,-\height/2}
  -- ++(\dslope/2, -\height/2)
  \tikztiming@chard@{#1}
}

\tikztimingdef{HDU}{
  \tikztiming@setref{timing@ref}{0,-\height/2}
  -- ++(\dslope/2, -\height/2)
  \tikztiming@chard{#1}
}
\tikztimingcpy{HUD}{HDU}
\tikztimingcpy{HUU}{HDD}

%    \end{macrocode}
%
% \subsection{Character `M'}
%    \begin{macrocode}
\tikztimingecopy{M}{Z}
\tikztimingchar{M}{++(0,\height/2.)}{
  \newdraw [\style]
  -- ++(#1,0)
  \newdraw [\style/base]
}

\tikztimingdef{ZM}{
  \newdraw [\style]
  -- ++(#1,0)
  \newdraw [\style/base]
}

\tikztimingdef{LM}{
  \newdraw [\style/base]
  -- ++($ (1/8,0) + (0,\height/2) $)
  \newdraw [\style]
  -- ++($ (-1/8,0) + (#1,0) $)
  \newdraw [\style/base]
}

\tikztimingdef{HM}{
  \newdraw [\style/base]
  -- ++($ (1/8,0) + (0,-1*\height/2) $)
  \newdraw [\style]
  -- ++($ (-1/8,0) + (#1,0) $)
  \newdraw [\style/base]
}

\tikztimingdef{DM}{
  \tikztiming@oldnodecenter@%
  \newdraw [\style]
  -- ($ (timing@ref) + (#1,0) $)
  \newdraw [\style/base]
}

%    \end{macrocode}
%
% \subsection{Character `T'}
%    \begin{macrocode}
\newcounter{tikztimingtrans}
\newcounter{tikztimingtranspos}

\tikztimingchar{T}{++(0,0)}{
  -- ++(#1,0)
}
\def\tikztiming@nocombine@T{}%

\tikztimingdef{HT}{%
  {[\style]
  \code{\setcounter{tikztimingtrans}{-1}}
  -- ++(\slope,\value{tikztimingtrans}*\height) -- ++($ (#1,0) - (\slope,0) $)
  }
}

\tikztimingdef{LT}{%
  {[\style]
  \code{\setcounter{tikztimingtrans}{+1}}
  -- ++(\slope,\value{tikztimingtrans}*\height) -- ++($ (#1,0) - (\slope,0) $)
  }
}

\tikztimingdef{TL}{%
  \code{\setcounter{tikztimingtranspos}{\value{tikztimingtrans}}%
  \addtocounter{tikztimingtranspos}{+1}}
  -- ++(\slope, -0.5*\value{tikztimingtranspos}*\height) -- ++($ (#1,0) - (\slope,0) $)
}

\tikztimingdef{TH}{%
  \code{\setcounter{tikztimingtranspos}{\value{tikztimingtrans}}%
  \addtocounter{tikztimingtranspos}{-1}}
  -- ++(\slope, -0.5*\value{tikztimingtranspos}*\height) -- ++($ (#1,0) - (\slope,0) $)
}

\tikztimingdef{TZ}{%
  \newdraw [\style]
  \code{\setcounter{tikztimingtrans}{-\value{tikztimingtrans}}}
  -- ++(\zslope,\value{tikztimingtrans}*\height/2.)
  -- ++($ (#1,0) - (\zslope,0) $)
}

\tikztimingdef{ZT}{%
  \newdraw {[\style]
  \code{\setcounter{tikztimingtrans}{-\value{tikztimingtrans}}}
  -- ++(\zslope,\value{tikztimingtrans}*\height/2.)
  -- ++($ (#1,0) - (\zslope,0) $)
  }
}

\tikztimingdef{TT}{%
  {[\style]
  \code{\setcounter{tikztimingtrans}{-\value{tikztimingtrans}}}
  -- ++(\slope,\value{tikztimingtrans}*\height)
  -- ++($ (#1,0) - (\slope,0) $)
  }
}

\tikztimingdef{DT}{
  \code{\setcounter{tikztimingtrans}{-1}}
  \tikztiming@oldnodecenter@%
  \newdraw {[\style]
  --  +(\dslope/2.,-\height/2)
  --   ($ (timing@ref) + (#1,-\height/2) $)
  }
}

\tikztimingdef{TM}{%
  \newdraw [\style/base]
  \code{\setcounter{tikztimingtrans}{-\value{tikztimingtrans}}}
  -- ++($ (1/8,0) + (0,\value{tikztimingtrans}*\height/2) $)
  \newdraw [\style]
  -- ++($ (-1/8,0) + (#1,0) $)
  \newdraw [\style/base]
}
%    \end{macrocode}
%
% \subsection{Character `C'}
%    \begin{macrocode}
\tikztimingecopy{C}{T}
%    \end{macrocode}
%
% \subsection{Character `U'}
%    \begin{macrocode}
\tikztimingecopy{U}{D}

\def\tikztiming@triple@U@edge@{}%
\def\tikztiming@nocombine@U@edge@{}%
\def\tikztiming@alias@U@edge@{D@edge@}%

\tikztimingcpy{UU@}{DD@}%

\tikztimingdef{UDU}{%
  \tikztiming@oldnodecenter@%
  \tikztiming@chard{#1}%
}
\tikztimingcpy{DUD}{UDU}%

\tikztimingdef{UDD}{%
  \tikztiming@uchardd{#1}%
}

\tikztimingcpy{{D@edge@}DD}{UDD}
\tikztimingcpy{{U@edge@}UU}{{D@edge@}DD}

\tikztimingdef{H{D@edge@}}{
  \tikztiming@setref{timing@ref}{0,-\height/2}
  -- ++(\dslope/2, -\height/2)
  \tikztiming@chard{#1}%
}

\tikztimingdef{L{D@edge@}}{
  \tikztiming@setref{timing@ref}{0,\height/2}
  -- ++(\dslope/2, \height/2)
  \tikztiming@chard{#1}%
}

\tikztimingdef{Z{D@edge@}}{
  coordinate (timing@ref)
  \tikztiming@chard{#1}%
}

\tikztimingdef{T{D@edge@}}{
  \code{\setcounter{tikztimingtrans}{-\value{tikztimingtrans}}}
  \tikztiming@setref{timing@ref}{0,\height/2 * \value{tikztimingtrans}}
  -- ++(\dslope/2, \height/2 * \value{tikztimingtrans})
  \tikztiming@chard{#1}%
}

\tikztimingdef{{D@edge@}DU}{
  \tikztiming@oldnodecenter@
  \tikztiming@chard{#1}
}
\tikztimingcpy{{U@edge@}UD}{{D@edge@}DU}

\tikztimingdef{DDU}{%
  \tikztiming@dchardu{#1}
}

\tikztimingcpy{DUU}{UDD}%
\tikztimingcpy{UUD}{DDU}%

\tikztimingdef{@UU}{%
  \tikztiming@@chard@{#1}%
}

\tikztimingdef{@UD}{%
  \tikztiming@@chard{#1}%
}

\tikztimingdef{@DU}{%
  \tikztiming@@chard{#1}%
}

\tikztimingdef{UD}{%
  \tikztiming@oldnodecenter@%
  \tikztiming@chard{#1}%
}

\tikztimingdef{DU}{%
  \tikztiming@oldnodecenter@%
  \tikztiming@chard{#1}%
}

\tikztimingdef{UD@}{%
  \tikztiming@oldnodecenter@%
  \tikztiming@chard@{#1}%
}

\tikztimingdef{DU@}{%
  \tikztiming@oldnodecenter@%
  \tikztiming@chard@{#1}%
}

%    \end{macrocode}
%
% \subsection{Character `X'}
%    \begin{macrocode}
\tikztimingecopy{X}{Z}
%    \end{macrocode}
%
% Save current list of characters as list of default characters.
%    \begin{macrocode}
\let\tikztiming@chars@default\tikztiming@chars
%    \end{macrocode}
%
% \clearpage
% \section{Libraries}
%
%    \begin{macrocode}
\newcommand*\usetikztiminglibrary[2][]{%
  \edef\tikztiming@library@options{#1}%
  \@ifnextchar{[}%]
    {\tikztiming@library{#2}}%
    {\tikztiming@library{#2}[]}%
}
\def\tikztiming@library#1[#2]{%
  \edef\tikztiming@library@date{#2}%
  \tikztiming@library@#1,\relax
}
%    \end{macrocode}
% Load the library as package. The \LaTeX\ core macro \cs{zap@space} is patched 
% to a no-op to keep spaces in the option text. Otherwise the option keys may 
% not contain spaces.
%    \begin{macrocode}
\def\tikztiming@zap@space#1 \@empty{#1}

\def\tikztiming@library@#1,{%
  \IfFileExists{tikz-timing-#1.sty}%
    { \let\tikztiming@orig@zap@space\zap@space
      \let\zap@space\tikztiming@zap@space
      \RequirePackage
        [\tikztiming@library@options]%
        {tikz-timing-#1}%
        [\tikztiming@library@date]%
      \let\zap@space\tikztiming@orig@zap@space
    }%
    {\PackageError{tikz-timing}%
      {No tikz-timing library `#1' found!}{}%
    }%
  \@ifnextchar\relax{}{\tikztiming@library@}%
}
%    \end{macrocode}
%
% \iffalse
%</package>
% \fi
%
% \section{Libraries}
%
% \subsection{Either High or Low}
% \iffalse
%<*lib-either>
% \fi
% Provides character "E".
%    \begin{macrocode}
\tikztimingchar{E}{
  ++(0,0)
  \code{\setcounter{tikztimingtranspos}{0}}%
  \code{\setcounter{tikztimingtrans}{1}}%
}{
  \code{\setcounter{tikztimingtranspos}{0}}%
  \code{\setcounter{tikztimingtrans}{1}}%
  \tikztiminguse{D}{#1}%
  \newdraw
}
\def\tikztiming@triple@E{}%

\tikztimingdef{LE@}{%
  \newdraw [\bgstyle]
  -- ++(\slope,\height)%
  -- ++($ (#1,0) - (\slope,0) $)%
  -- ++(0,-\height)%
  -- ++($ -1*(#1,0) $)%
  \newdraw [\style]
  -- ++(\slope,\height)%
  -- ++($ (#1,0) - (\slope,0) $)%
     ++($ -1*(#1,\height) $)%
  -- ++($  1*(#1,0) $)%
     ++(0,.5\height)%
  \newdraw
  \code{\setcounter{tikztimingtranspos}{0}}%
  \code{\setcounter{tikztimingtrans}{1}}%
}%

\tikztimingdef{HE@}{%
  \newdraw [\bgstyle]
  -- ++(\slope,-\height)%
  -- ++($ (#1,0) - (\slope,0) $)%
  -- ++(0,\height)%
  -- ++($ -1*(#1,0) $)%
  \newdraw [\style]
  -- ++(\slope,-\height)%
  -- ++($ (#1,0) - (\slope,0) $)%
     ++($ -1*(#1,-\height) $)%
  -- ++($  1*(#1,0) $)%
     ++(0,-.5\height)%
  \newdraw
  \code{\setcounter{tikztimingtranspos}{1}}%
  \code{\setcounter{tikztimingtrans}{-1}}%
}%

\tikztimingdef{TE@}{%
  \newdraw [\bgstyle]
  -- ++(\slope,-\height * \value{tikztimingtrans})%
  -- ++($ (#1,0) - (\slope,0) $)%
  -- ++(0,\height * \value{tikztimingtrans})%
  -- ++($ -1*(#1,0) $)%
  \newdraw [\style]
  -- ++(\slope,-\height * \value{tikztimingtrans})%
  -- ++($ (#1,0) - (\slope,0) $)%
     ++($ -1*(#1,-\height * \value{tikztimingtrans}) $)%
  -- ++($  1*(#1,0) $)%
     ++(0,-1*\value{tikztimingtrans}*.5*\height)%
  \newdraw
}%

\tikztimingdef{LEL}{%
  \newdraw [\style,preaction={\bgstyle}]
  -- +(\slope,\height)%
  -- +(#1,\height)%
  -- +($ (#1,0) + (\slope,0) $)%
  -- +(0,0)
    ++(#1,\height/2) coordinate (timing@ref)
    ++(\slope,-\height/2)%
  \newdraw
}%

\tikztimingdef{LEH}{%
  \newdraw [\style,preaction={\bgstyle}]
  -- +(\slope,\height)%
  -- +($ (#1,\height) + (\slope,0) $)%
  -- +(#1,0)%
  -- +(0,0)
    ++(#1,\height/2) coordinate (timing@ref)
    ++(\slope,\height/2)%
  \newdraw
}%

\tikztimingdef{HEH}{%
  \newdraw [\style,preaction={\bgstyle}]
  -- +($ (#1,0) + (\slope,0) $)%
  -- +(#1,-\height)%
  -- +(\slope,-\height)
  -- +(0,0)
    ++(#1,-\height/2) coordinate (timing@ref)
    ++(\slope,\height/2)%
  \newdraw
}%

\tikztimingdef{HEL}{%
  \newdraw [\style,preaction={\bgstyle}]
  -- +(#1,0)
  -- +($ (#1,-\height) + (\slope,0) $)%
  -- +(\slope,-\height)
  -- +(0,0)
    ++(#1,-\height/2) coordinate (timing@ref)
    ++(\slope,-\height/2)%
  \newdraw
}%

\tikztimingdef{HET}{
  \newdraw [\style,preaction={\bgstyle}]
  -- +(#1,0)
  {[\nstyle]
  -- +($ (#1,-\height) + (\slope,0) $)%
  }
  -- +(\slope,-\height)
  -- +(0,0)
  {[\nstyle]
    ++(#1,-\height/2) coordinate (timing@ref)
    ++(\slope,-\height/2)%
  }
  \newdraw
  \code{\setcounter{tikztimingtrans}{-1}}%
}

\tikztimingdef{LET}{
  \newdraw [\style,preaction={\bgstyle}]
  -- +(\slope,\height)%
  {[\nstyle]
  -- +($ (#1,\height) + (\slope,0) $)%
  }
  -- +(#1,0)%
  -- +(0,0)
  {[\nstyle]
    ++(#1,\height/2)%
      coordinate (timing@ref)
    ++(\slope,\height/2)%
  }
  \newdraw
  \code{\setcounter{tikztimingtrans}{1}}%
}

\tikztimingdef{HEE}{%
  \newdraw [\style,preaction={\bgstyle}]
  -- +(#1,0) coordinate (timing@refb) 
      \tikztiming@setref{timing@ref}{0,-\height/2}
  -- +($ (#1,-\height) + (\slope,0) $) coordinate (timing@refa)
  -- +(\slope,-\height)
  -- +(0,0)
      \tikztiming@setref{timing@ref}{#1,-\height/2}
    ++($ (#1,-\height) + (\slope,0) $)%
  \code{\setcounter{tikztimingtranspos}{1}}%
  \code{\setcounter{tikztimingtrans}{-1}}%
  \newdraw
}%

\tikztimingdef{LEE}{%
  \newdraw [\style,preaction={\bgstyle}]
  -- +(#1,0) coordinate (timing@refa)
      \tikztiming@setref{timing@ref}{0,\height/2}
  -- +($ (#1,\height) + (\slope,0) $) coordinate (timing@refb)
  -- +(\slope,\height)
  -- +(0,0)
    ++(#1,0) \tikztiming@setref{timing@ref}{0,\height/2}
  \code{\setcounter{tikztimingtranspos}{0}}%
  \code{\setcounter{tikztimingtrans}{1}}%
  \newdraw
}%

\tikztimingdef{TEE}{%
  \newdraw [\style,preaction={\bgstyle}]
   -- +(\slope, -\height * \value{tikztimingtrans})%
   -- +($ (#1,0) + (\slope, -\height * \value{tikztimingtrans}) $)
   -- +(#1,0)
   -- +(0,0)%
      \code{\setcounter{tikztimingtranspos}{\value{tikztimingtrans}}%
      \addtocounter{tikztimingtranspos}{+1}}
      +($ (#1,0) + (\slope/2 + \slope/2 * \value{tikztimingtrans}, -\height/2 * \value{tikztimingtranspos}) $)
        coordinate (timing@refa)
      +($ (#1,0) + (\slope/2 - \slope/2 * \value{tikztimingtrans}, \height - \height/2 * \value{tikztimingtranspos}) $)
        coordinate (timing@refb)
     ++(#1,0) 
      \tikztiming@setref{timing@ref}{0,-\height/2 * \value{tikztimingtrans}}
     ++(
          .5 * \slope  + .5 * \value{tikztimingtrans} * \slope,
         -.5 * \height - .5 * \value{tikztimingtrans} * \height
       )
  \code{\setcounter{tikztimingtrans}{-\value{tikztimingtrans}}}
  \newdraw
}%


\tikztimingdef{TEH}{%
  \newdraw [\style,preaction={\bgstyle}]
  -- ++(\slope,-\height * \value{tikztimingtrans})
  -- ++($ (#1,0) - (\slope/2 + \slope/2 * \value{tikztimingtrans},0) $)
  -- ++(\slope * \value{tikztimingtrans},
        \height * \value{tikztimingtrans})
  -- ++($ -1*(#1,0) - (\slope/2 + \slope/2 * \value{tikztimingtrans},0) $)
     ++($ (#1,-\height/2 * \value{tikztimingtrans}) $) coordinate (timing@ref)
     ++(\slope,\height/2)
  \newdraw
}

\tikztimingdef{TEL}{%
  \newdraw [\style,preaction={\bgstyle}]
  -- ++(\slope,-\height * \value{tikztimingtrans})
  -- ++($ (#1,0) - (\slope/2 - \slope/2 * \value{tikztimingtrans},0) $)
  -- ++(-\slope * \value{tikztimingtrans},
        \height * \value{tikztimingtrans})
  -- ++($ -1*(#1,0) - (\slope/2 + -\slope/2 * \value{tikztimingtrans},0) $)
     ++($ (#1,-\height/2 * \value{tikztimingtrans}) $) coordinate (timing@ref)
     ++(\slope,-\height/2)
  \newdraw
}

\tikztimingdef{TEZ}{%
  \newdraw [\style,preaction={\bgstyle}]
  -- ++(\slope,-\height * \value{tikztimingtrans})
  -- ++($ (#1,0) - (\slope,0) $)
  -- ++(\zslope,
        \height/2 * \value{tikztimingtrans})
  -- ++(-\zslope,
        \height/2 * \value{tikztimingtrans})
  -- ++($ -1*(#1,0) $)
     ++($ (#1,0) + (0,-\height/2 * \value{tikztimingtrans}) $)
       coordinate (timing@ref)
     ++(\zslope,0)
  \newdraw
}

\tikztimingdef{TED}{%
  \newdraw [\style,preaction={\bgstyle}]
  -- ++(\slope,-\height * \value{tikztimingtrans})
  -- ++($ (#1,0) - (\slope,0) $)
  -- ++(\dslope/2,
        \height/2 * \value{tikztimingtrans})
  -- ++(-\dslope/2,
        \height/2 * \value{tikztimingtrans})
  -- ++($ -1*(#1,0) $)
     ++($ (#1,0) + (0,-\height/2 * \value{tikztimingtrans}) $)
       coordinate (timing@ref)
     ++(\dslope/2,0)
  \newdraw
}

\tikztimingdef{TET}{%
  \newdraw [\style,preaction={\bgstyle}]
   -- +(\slope, -\height * \value{tikztimingtrans})%
   {[\nstyle]
   -- +($ (#1,0) + (\slope, -\height * \value{tikztimingtrans}) $)%
   }
   -- +(#1,0)%
   -- +(0,0)%
   {[\nstyle]
     ++($ (#1,0) + (0, -\height/2 * \value{tikztimingtrans}) $)%
       coordinate (timing@ref)
     ++(\slope, -\height/2 * \value{tikztimingtrans})%
   }
  \code{\setcounter{tikztimingtrans}{-\value{tikztimingtrans}}}
  \newdraw
}%

\tikztimingdef{EEZ}{%
  \newdraw [\style,preaction={\bgstyle}]
   --   (timing@refb)
   --   ($ (timing@ref)  + (#1,+\height/2) $)
   --   ($ (timing@ref)  + (#1,0) + (\zslope,0) $)
   --   ($ (timing@ref)  + (#1,-\height/2) $)
   --   (timing@refa)
        ($ (timing@ref)  + (#1,0) + (\zslope,0) $)
      coordinate (timing@ref) at ($ (timing@ref) + (#1,0) $)
  \newdraw
}%

\tikztimingdef{EED}{%
  \newdraw [\style,preaction={\bgstyle}]
   --   (timing@refb)
   --   ($ (timing@ref)  + (#1,+\height/2) $)
   --   ($ (timing@ref)  + (#1,0) + (\dslope/2,0) $)
   --   ($ (timing@ref)  + (#1,-\height/2) $)
   --   (timing@refa)
        ($ (timing@ref)  + (#1,0) + (\dslope/2,0) $)
      coordinate (timing@ref) at ($ (timing@ref) + (#1,0) $)
  \newdraw
}%


\tikztimingdef{HEZ}{%
  \newdraw [\style,preaction={\bgstyle}]
  -- +(#1,0)
  -- +($ (#1,-\height/2) + (\zslope,0) $)%
  -- +(#1,-\height)
  -- +(\slope,-\height)
  -- +(0,0)
    ++(#1,-\height/2) coordinate (timing@ref)
    ++(\zslope,0)%
  \newdraw
}%


\tikztimingdef{LEZ}{%
  \newdraw [\style,preaction={\bgstyle}]
  -- +(\slope,\height)
  -- +(#1,\height)
  -- +($ (#1,\height/2) + (\zslope,0) $)%
  -- +(#1,0)
  -- +(0,0)
    ++(#1,\height/2) coordinate (timing@ref)
    ++(\zslope,0)%
  \newdraw
}%


\tikztimingdef{LED}{%
  \newdraw [\style,preaction={\bgstyle}]
  -- +(\slope,\height)
  -- +(#1,\height)
  -- +($ (#1,\height/2) + (\dslope/2,0) $)%
  -- +(#1,0)
  -- +(0,0)
    ++(#1,\height/2) coordinate (timing@ref)
    ++(\dslope/2,0)%
  \newdraw
}%

\tikztimingdef{HED}{%
  \newdraw [\style,preaction={\bgstyle}]
  -- +(#1,0)
  -- +($ (#1,-\height/2) + (\dslope/2,0) $)%
  -- +(#1,-\height)
  -- +(\slope,-\height)
  -- +(0,0)
    ++(#1,-\height/2) coordinate (timing@ref)
    ++(\dslope/2,0)%
  \newdraw
}%


\tikztimingdef{EZ}{%
  \newdraw [\style]
  --   ($ (timing@ref) + (#1,0) $)%
}

\tikztimingdef{EM}{%
  \newdraw [\style]
  --   ($ (timing@ref) + (#1,0) $)%
  \newdraw [\style/base]
}

\tikztimingdef{EL}{%
  --   ($ (timing@ref) + (#1,-\height/2) $)%
}%
\tikztimingdef{EH}{%
  --   ($ (timing@ref) + (#1,+\height/2) $)%
}%

\tikztimingdef{ED@}{%
  \tikztiming@chard@{#1}%
}
\tikztimingcpy{E{D@edge@}@}{ED@}

\tikztimingdef{ED}{%
  \tikztiming@chard{#1}%
}
\tikztimingcpy{E{D@edge@}}{ED}
\tikztimingdef{EDD}{%
  \tikztiming@chard@{#1}%
}
\tikztimingdef{EUU}{%
  \tikztiming@chard@{#1}%
}
\tikztimingdef{EDU}{%
  \tikztiming@chard{#1}%
}
\tikztimingdef{EUD}{%
  \tikztiming@chard{#1}%
}


\tikztimingdef{DE@}{%
  \tikztiming@oldnodecenter@%
  \newdraw [\bgstyle]
  -- +(\dslope/2,\height/2)
  --  ($ (timing@ref) + (#1,\height/2) $)%
  --  ($ (timing@ref) + (#1,-\height/2) $)%
  --  ($ (timing@save) + (\dslope/2,-\height/2) $)
  --  (timing@save)
  \newdraw [\style]
  -- +(\dslope/2,\height/2)
  --  ($ (timing@ref) + (#1,\height/2) $)%
      ($ (timing@ref) + (#1,-\height/2) $)%
  --  ($ (timing@save) + (\dslope/2,-\height/2) $)
  --  (timing@save)
      ($ (timing@ref) + (#1,0) $)%
      coordinate (timing@ref)
}

\tikztimingdef{EEE}{%
  \newdraw [\style,preaction={\bgstyle}]
   --   (timing@refb)
   --   ($ (timing@refb)  + (#1,0) $)
   --   ($ (timing@refa)  + (#1,0) $)
   --   (timing@refa)
      ++(#1,0) coordinate (timing@refa)
        coordinate (timing@refb) at ($ (timing@refb) + (#1,0) $)
        coordinate (timing@ref)  at ($ (timing@ref)  + (#1,0) $)
  \newdraw
}%

\tikztimingdef{EET}{%
  \newdraw [\style,preaction={\bgstyle}]
   --   (timing@refb)
   {[\nstyle]
   --   ($ (timing@ref)  + (#1,+\height/2) + (\slope/2 + \slope/2 * \value{tikztimingtrans},0) $)
   --   ($ (timing@ref)  + (#1,-\height/2) + (\slope/2 - \slope/2 * \value{tikztimingtrans},0) $)
   }
   --   (timing@refa)
   {[\nstyle]
      ($ (timing@ref)  + (#1,\height/2 * \value{tikztimingtrans}) + (\slope,0) $)
   }
      coordinate (timing@ref) at ($ (timing@ref) + (#1,0) $)
  \newdraw
}%

\tikztimingdef{EEL}{%
  \newdraw [\style,preaction={\bgstyle}]
   --   (timing@refb)
   --   ($ (timing@ref)  + (#1,+\height/2) $)
   --   ($ (timing@ref)  + (#1,-\height/2) + (\slope,0) $)
   --   (timing@refa)
        ($ (timing@ref)  + (#1,-\height/2) + (\slope,0) $)
      coordinate (timing@ref) at ($ (timing@ref) + (#1,0) $)
  \newdraw
}%

\tikztimingdef{EEH}{%
  \newdraw [\style,preaction={\bgstyle}]
   --   (timing@refb)
   --   ($ (timing@ref)  + (#1,+\height/2) + (\slope,0) $)
   --   ($ (timing@ref)  + (#1,-\height/2) $)
   --   (timing@refa)
        ($ (timing@ref)  + (#1,+\height/2) + (\slope,0) $)
      coordinate (timing@ref) at ($ (timing@ref) + (#1,0) $)
  \newdraw
}%


\tikztimingdef{EE@}{%
  \newdraw [\bgstyle]
      let \p0 = (timing@ref), \p1 = (timing@refa), \p2 = (timing@refb) in
  --  (\p2)
  --  ($ (\x0,\y2) + (#1,0) $)
  -- +(0,\y1-\y2)
  --  (\p1)
  \newdraw [\style]
      let \p0 = (timing@ref), \p1 = (timing@refa), \p2 = (timing@refb) in
  --  (\p2)
  --  ($ (\x0,\y2) + (#1,0) $)
     +(0,\y1-\y2)
  --  (\p1)
      ($ (timing@ref) + (#1,0) $) coordinate (timing@ref)
  \newdraw
}%

\tikztimingdef{ZE@}{
  \newdraw [\bgstyle]
  -- +(\zslope,\height/2)
  -- +(#1,\height/2)%
  -- +(#1,-\height/2)%
  -- +(\zslope,-\height/2)
  -- +(0,0)
  \newdraw [\style]
  -- +(\zslope,\height/2)
  -- +(#1,\height/2)%
     +(#1,-\height/2)%
  -- +(\zslope,-\height/2)
  -- +(0,0)
    ++(#1,0) coordinate (timing@ref)
}

\tikztimingdef{ZEH}{%
  \newdraw [\style,preaction={\bgstyle}]
  -- +(\zslope, \height/2)
  -- +($ (#1,0) + (\slope, \height/2) $)
  -- +(#1,-\height/2)
  -- +(\zslope, -\height/2)
  -- +(0,0)
    ++(#1,0) coordinate (timing@ref)
    ++(\slope, \height/2)
  \newdraw
}

\tikztimingdef{ZEL}{%
  \newdraw [\style,preaction={\bgstyle}]
  -- +(\zslope, \height/2)
  -- +(#1,\height/2)
  -- +($ (#1,0) + (\slope, -\height/2) $)
  -- +(\zslope, -\height/2)
  -- +(0,0)
    ++(#1,0) coordinate (timing@ref)
    ++(\slope, -\height/2)
  \newdraw
}

\tikztimingdef{ZET}{%
  \newdraw [\style,preaction={\bgstyle}]
  -- +(\zslope, \height/2)
  {[\nstyle]
  -- +($ (#1, \height/2) + (\slope/2 - \slope/2 * \value{tikztimingtrans}, 0) $)
  -- +($ (#1,-\height/2) + (\slope/2 + \slope/2 * \value{tikztimingtrans}, 0) $)
  }
  -- +(\zslope, -\height/2)
  -- +(0,0)
  {[\nstyle]
     ++(#1,0) coordinate (timing@ref)
     ++(\slope,-\height/2 * \value{tikztimingtrans})
  }
  \code{\setcounter{tikztimingtrans}{-\value{tikztimingtrans}}}
  \newdraw
}

\tikztimingdef{DET}{%
  \tikztiming@oldnodecenter@%
  \newdraw [\style,preaction={\bgstyle}]
  -- +(\dslope/2, \height/2)
  {[\nstyle]
  --  ($ (timing@ref) + (#1, \height/2) + (\slope/2 - \slope/2 * \value{tikztimingtrans}, 0) $)
  --  ($ (timing@ref) + (#1,-\height/2) + (\slope/2 + \slope/2 * \value{tikztimingtrans}, 0) $)
  }
  --  ($ (timing@save) + (\dslope/2, -\height/2) $)
  --  (timing@save)
  {[\nstyle]
       ($ (timing@ref) + (#1,0) $) coordinate (timing@ref)
     ++(\slope,-\height/2 * \value{tikztimingtrans})
  }
  \code{\setcounter{tikztimingtrans}{-\value{tikztimingtrans}}}
  \newdraw
}

\tikztimingdef{ZEZ}{%
  \newdraw [\style,preaction={\bgstyle}]
  -- +(\zslope, \height/2)
  -- +(#1,\height/2)
  -- +($ (#1,0) + (\zslope, 0) $)
  -- +(#1,-\height/2)
  -- +(\zslope, -\height/2)
  -- +(0,0)
    ++(#1,0) coordinate (timing@ref)
    ++(\zslope, 0)
  \newdraw
}

\tikztimingdef{ZEE}{%
  \newdraw [\style,preaction={\bgstyle}]
  -- +(\zslope, \height/2)
  -- +($ (#1,0) + (\slope, \height/2) $) coordinate (timing@refb)
  -- +(#1,-\height/2) coordinate (timing@refa)
  -- +(\zslope, -\height/2)
  -- +(0,0)
     +(#1,0) coordinate (timing@ref)
    ++(#1,-\height/2)
  \code{\setcounter{tikztimingtranspos}{0}}%
  \code{\setcounter{tikztimingtrans}{1}}%
}

\tikztimingdef{DEE}{%
  \tikztiming@oldnodecenter@
  \newdraw [\style,preaction={\bgstyle}]
  -- +(\dslope/2, \height/2)
  --  ($ (timing@ref) + (#1,0) + (\slope, \height/2) $) coordinate (timing@refb)
  --  ($ (timing@ref) + (#1,-\height/2) $) coordinate (timing@refa)
  --  ($ (timing@save) + (\dslope/2, -\height/2) $)
  --  (timing@save)
      coordinate (timing@ref) at ($ (timing@ref) + (#1,0) $)
      (timing@refa)
  \code{\setcounter{tikztimingtranspos}{0}}%
  \code{\setcounter{tikztimingtrans}{1}}%
}


\tikztimingdef{DED}{%
  \tikztiming@oldnodecenter@%
  \newdraw [\style,preaction={\bgstyle}]
  -- +(\dslope/2, \height/2)
  --  ($ (timing@ref) + (#1,\height/2) $)
  --  ($ (timing@ref) + (#1,0) + (\dslope/2,0) $)
  --  ($ (timing@ref) + (#1,-\height/2) $)
  --  ($ (timing@save) + (\dslope/2, -\height/2) $)
  --  (timing@save)
      ($ (timing@ref) + (#1,0) $)
      coordinate (timing@ref)
    ++(\dslope/2,0)
  \newdraw
}

\tikztimingdef{ZED}{%
  \newdraw [\style,preaction={\bgstyle}]
  -- +(\zslope, \height/2)
  -- +(#1,\height/2)
  -- +($ (#1,0) + (\dslope/2, 0) $)
  -- +(#1,-\height/2)
  -- +(\zslope, -\height/2)
  -- +(0,0)
    ++(#1,0) coordinate (timing@ref)
    ++(\dslope/2, 0)
  \newdraw
}

\tikztimingdef{DEZ}{%
  \tikztiming@oldnodecenter@
  \newdraw [\style,preaction={\bgstyle}]
  -- +(\dslope/2, \height/2)
  --  ($ (timing@ref) + (#1,\height/2) $)
  --  ($ (timing@ref) + (#1,0) + (\zslope, 0) $)
  --  ($ (timing@ref) + (#1,-\height/2) $)
  --  ($ (timing@save) + (\dslope/2, -\height/2) $)
  --  (timing@save)
      ($ (timing@ref) + (#1,0) $)
      coordinate (timing@ref)
    ++(\zslope, 0)
  \newdraw
}


\tikztimingdef{DEH}{%
  \tikztiming@oldnodecenter@%
  \newdraw [\style,preaction={\bgstyle}]
  -- +(\dslope/2, \height/2)
  --  ($ (timing@ref) + (#1,0) + (\slope, \height/2) $) 
  --  ($ (timing@ref) + (#1,-\height/2) $)
  --  ($ (timing@save) + (\dslope/2, -\height/2) $)
  --  (timing@save)
      ($ (timing@ref) + (#1,0) $) coordinate (timing@ref)
    ++(\slope, \height/2)
  \newdraw
}

\tikztimingdef{DEL}{%
  \tikztiming@oldnodecenter@%
  \newdraw [\style,preaction={\bgstyle}]
  -- +(\dslope/2, \height/2)
  --  ($ (timing@ref) + (#1,\height/2) $)
  --  ($ (timing@ref) + (#1,0) + (\slope, -\height/2) $) 
  --  ($ (timing@save) + (\dslope/2, -\height/2) $)
  --  (timing@save)
      ($ (timing@ref) + (#1,0) $) coordinate (timing@ref)
    ++(\slope, -\height/2)
  \newdraw
}


\tikztimingdef{@E}{%
  \code{\setcounter{tikztimingtranspos}{0}}%
  \code{\setcounter{tikztimingtrans}{1}}%
  \tikztiminguse{@D}{#1}
}

\tikztimingdef{@DE}{%
  \tikztiming@@chard{#1}%
}
\tikztimingdef{DDE}{
  \tikztiming@@chard{#1}%
}

\tikztimingdef{@EH}{%
  \tikztiming@setref{timing/start}{0,\height/2}
  \newdraw [\bgstyle]
  -- +(0,\height)
  -- +($ (#1,\height) + (\slope,0) $)
  -- +(#1,0)
  -- +(0,0)
  \newdraw [\style]
      +(0,\height)
   -- +($ (#1,\height) + (\slope,0) $)
   -- +(#1,0)
   -- +(0,0)
     ++(#1,\height/2) coordinate (timing@ref)
     ++(\slope,\height/2)
  \newdraw
}

\tikztimingdef{@ET}{%
  \tikztiming@setref{timing/start}{0,\height/2}
  \newdraw [\bgstyle]
  -- +(0,\height)
  {[\nstyle]
  --  ($ (timing@ref) + (#1,\height/2) + (\slope,0) $)
  }
  --  ($ (timing@ref) + (#1,-\height/2) $)
  --  (timing@save)
  \newdraw [\style]
     +(0,\height)
  {[\nstyle]
  --  ($ (timing@ref) + (#1,\height/2) + (\slope,0) $)
  }
  --  ($ (timing@ref) + (#1,-\height/2) $)
  --  (timing@save)
  {[\nstyle]
      ($ (timing@ref) + (#1,0) $) coordinate (timing@ref)
    ++(\slope,\height/2)
  }
  \newdraw
  \code{\setcounter{tikztimingtrans}{-\value{tikztimingtrans}}}
}

\tikztimingdef{ET}{%
  {[\style]
  --   ($ (timing@ref) + (#1,\height/2 * \value{tikztimingtrans}) $)
  }
}

\tikztimingdef{@EL}{%
  \tikztiming@setref{timing/start}{0,\height/2}
  \newdraw [\bgstyle]
  -- +(0,\height)
  -- +(#1,\height)
  -- +($ (#1,0) + (\slope,0) $)
  -- +(0,0)
  \newdraw [\style]
      +(0,\height)
   -- +(#1,\height)
   -- +($ (#1,0) + (\slope,0) $)
   -- +(0,0)
     ++(#1,\height/2) coordinate (timing@ref)
     ++(\slope,-\height/2)
  \newdraw
}

\tikztimingdef{@ED}{%
  \tikztiming@setref{timing/start}{0,\height/2}
  \newdraw [\bgstyle]
  -- +(0,\height)
  -- +(#1,\height)
  -- +($ (#1,\height/2) + (\dslope/2,0) $)
  -- +(#1,0)
  -- +(0,0)
  \newdraw [\style]
      +(0,\height)
   -- +(#1,\height)
   -- +($ (#1,\height/2) + (\dslope/2,0) $)
   -- +(#1,0)
   -- +(0,0)
     ++(#1,\height/2) coordinate (timing@ref)
     ++(\dslope/2,0)
}

\tikztimingdef{@EZ}{%
  \tikztiming@setref{timing/start}{0,\height/2}
  \newdraw [\bgstyle]
  -- +(0,\height)
  -- +(#1,\height)
  -- +($ (#1,\height/2) + (\zslope,0) $)
  -- +(#1,0)
  -- +(0,0)
  \newdraw [\style]
      +(0,\height)
   -- +(#1,\height)
   -- +($ (#1,\height/2) + (\zslope,0) $)
   -- +(#1,0)
   -- +(0,0)
     ++(#1,\height/2) coordinate (timing@ref)
     ++(\zslope,0)
}


\tikztimingdef{@EE}{%
  \tikztiming@setref{timing/start}{0,\height/2}
  \newdraw [\bgstyle]
  -- +(0,\height)
  -- +($ (#1,\height) + (\slope,0) $)
  -- +(#1,0)
  -- +(0,0)
  \newdraw [\style]
      +(0,\height)
   -- +($ (#1,\height) + (\slope,0) $) coordinate (timing@refb)
   -- +(#1,0) coordinate (timing@refa)
   -- +(0,0)
     ++(#1,0)
     \tikztiming@setref{timing@ref}{0,\height/2}
  \code{\setcounter{tikztimingtranspos}{0}}%
  \code{\setcounter{tikztimingtrans}{1}}%
}

\def\tikztiming@nocombine@E{}%%
%    \end{macrocode}
%
%  \iffalse
%</lib-either>
% \fi
%
% \subsection{Clock Arrows}
% \iffalse
%<*lib-clockarrows>
%  \fi
%    \begin{macrocode}
\tikztiming@iftrans{CC}%
  {\let\tikztiming@trans@C@normal@C\tikztiming@trans@CC}%
  {\PackageError{\@currname}{No 'CC' transition defined by the main tikz-timing 
  package.}{}}%

\tikztimingdef{CC}{%
  \code{\setcounter{tikztimingtrans}{-\value{tikztimingtrans}}}
  [timing/c/save]
  \newdraw 
  [\style,timing/c/restore,>=\noexpand\tikztiming@c@arrow@tip,timing/c/arrow@style]
  -- ++(\slope*\noexpand\tikztiming@c@arrow@pos,\value{tikztimingtrans}*\height*\noexpand\tikztiming@c@arrow@pos)
  ; \draw [timing/restore,timing/c/restore] (timing@save) ++(0,0) [\style]
  -- ++(\slope,\value{tikztimingtrans}*\height)
  -- ++($ (#1,0) - (\slope,0) $)
}
\let\tikztiming@trans@C@arrow@C\tikztiming@trans@CC

\tikzset{timing/c/.cd,
  rising arrows/.code={%
    \def\tikztiming@trans@CC{\tikztiming@trans@C@arrow@C}%
    \def\tikztiming@clockarrow@style{\ifnum1=\c@tikztimingtrans\tikzset{->,timing/c/arrow}\fi}%
  },%
  falling arrows/.code={%
    \def\tikztiming@trans@CC{\tikztiming@trans@C@arrow@C}%
    \def\tikztiming@clockarrow@style{\ifnum-1=\c@tikztimingtrans\tikzset{->,timing/c/arrow}\fi}%
  },%
  dual arrows/.code={%
    \def\tikztiming@trans@CC{\tikztiming@trans@C@arrow@C}%
    \def\tikztiming@clockarrow@style{\tikzset{->,timing/c/arrow}}%
  },%
  no arrows/.code={%
    \def\tikztiming@trans@CC{\tikztiming@trans@C@normal@C}%
    \def\tikztiming@clockarrow@style{}%
  },%
  arrow@style/.code={\tikztiming@clockarrow@style},%
  arrow/.style={},%
  arrow pos/.store in=\tikztiming@c@arrow@pos,
  arrow tip/.store in=\tikztiming@c@arrow@tip,
  save/.code={%
    \global\let\tikztiming@clockarrow@savedstyle\tikztiming@clockarrow@style
    \global\let\tikztiming@trans@saved@CC\tikztiming@trans@CC
    \global\let\tikztiming@c@arrow@savedpos\tikztiming@c@arrow@pos
    \global\let\tikztiming@c@arrow@savedtip\tikztiming@c@arrow@tip
  },
  restore/.code={%
    \let\tikztiming@clockarrow@style\tikztiming@clockarrow@savedstyle
    \let\tikztiming@trans@CC\tikztiming@trans@saved@CC
    \let\tikztiming@c@arrow@pos\tikztiming@c@arrow@savedpos
    \let\tikztiming@c@arrow@tip\tikztiming@c@arrow@savedtip
  },
}
\tikzset{timing/c/.cd,
  no arrows,
  arrow pos=0.95,
  arrow tip=to,
}

%    \end{macrocode}
% Use the \pkg{pgfopts} to handle the package options.
%    \begin{macrocode}
\RequirePackage{pgfopts}
\ProcessPgfOptions{/tikz/timing/c}
%    \end{macrocode}
% \iffalse
%</lib-clockarrows>
% \fi
%
% \subsection{Arrows}
% \iffalse
%<*lib-arrows>
% \fi
% This library is only defining the "A" and "W" basic transitions ("AA", "AW", "WA", "WW").
% While "W" is simply a copy of "A" with a different style. Both simple draw a line from |low|
% to |high| and than move the position to the right according to the |width| value and back to 
% the |low| position. The different arrow directions are done by different styles.
%    \begin{macrocode}
\tikzset{timing/.cd,
  A/.style={timing/a},
  W/.style={timing/w},
  a/.style={->},
  w/.style={<-},
}

\tikztimingchar{A}{++(0,0)}{%
  \newdraw [\style] -- ++(0,\height)
  \newdraw ++(#1,-\height)
}
\def\tikztiming@nocombine@A{}%

\tikztimingecopy{W}{A}
%    \end{macrocode}
%
% \iffalse
%</lib-arrows>
% \fi
%
% \subsection{Overlays}
% \iffalse
%<*lib-overlays>
% \fi
% \begin{macro}{\tikztiming@defcode{O}}
% The last character before the "O" is saved and restored. Also the drawing 
% state is saved and restored by the \cs{tikztiming@overlay@begin} and 
% \cs{tikztiming@overlay@end} macros. The parser end macro 
% \cs{tikztiming@parser@end} which is triggered at the end of the overlay string 
% is redefined to avoid problems. The double \TeX\ group is necessary because 
% the overlay string can close and reopen the inner \TeX\ 
% group.
%    \begin{macrocode}
\tikztiming@defcode{O}#1{%
  \ifx\relax#1\empty
    \def\tikztiming@next{\tikztiming@parser#1}%
  \else
    \let\tikztiming@overlay@lastchar\tikztiming@lastchar
    \let\tikztiming@overlay@width\tikztimingwidth
    \tikztiming@output@addcode{\tikztiming@overlay@begin}%
    \begingroup
      \def\tikztiming@parser@end{%
        \def\tikztiming@output@nextchar{@}%
        \tikztiming@output@flush
      }
      \begingroup
        \tikztiming@parser #1\relax
      \endgroup
    \endgroup
    \tikztiming@output@addcode{\tikztiming@overlay@end}%
    \global\let\tikztiming@lastchar\tikztiming@overlay@lastchar
    \global\let\tikztimingwidth\tikztiming@overlay@width
    \def\tikztiming@next{\expandafter\tikztiming@parser\@firstofone}%
  \fi
  \tikztiming@next
}
%    \end{macrocode}
% \end{macro}
%
% Set the later expanded macros to some sane initial value.
%    \begin{macrocode}
\let\tikztiming@overlay@save\empty
\let\pgf@sh@nt@tikztiming@overlay@node\empty
%    \end{macrocode}
%
% \begin{macro}{\tikztiming@overlay@begin}
% The drawing transition state (for "T", etc.) is saved and restored.
% In order to be able to include an "O" inside the argument of another one the 
% overlay save node coordinates (\cs{pfg@sh@nt@tikztiming@overlay@node}) is also 
% saved and restored. The \cs{tikztiming@overlay@save} macro also includes an 
% self-assignment to its old version, which implements a stack. The macros must 
% be global because the overlay string can reopen a new TikZ path.
%    \begin{macrocode}
\def\tikztiming@overlay@begin{
    [/utils/exec={%
    \xdef\tikztiming@overlay@save{%
      \noexpand\global\noexpand\c@tikztimingtrans\the\c@tikztimingtrans\relax%
      \noexpand\global\noexpand\c@tikztimingtranspos\the\c@tikztimingtranspos\relax%
      \noexpand\gdef\noexpand\pgf@sh@nt@tikztiming@overlay@node
        {\csname pgf@sh@nt@tikztiming@overlay@node\endcsname}%
      \noexpand\gdef\noexpand\pgf@sh@nt@timing@refa
        {\csname pgf@sh@nt@timing@refa\endcsname}%
      \noexpand\gdef\noexpand\pgf@sh@nt@timing@refb
        {\csname pgf@sh@nt@timing@refb\endcsname}%
      \noexpand\gdef\noexpand\pgf@sh@nt@timing@ref
        {\csname pgf@sh@nt@timing@ref\endcsname}%
      \noexpand\gdef\noexpand\tikztiming@overlay@save{%
        \expandafter\unexpanded\expandafter{\tikztiming@overlay@save}%
      }%
    }%
    }]%
    node [timing/save] (tikztiming@overlay@node) {}
  }%
%    \end{macrocode}
% \end{macro}
%
% \begin{macro}{\tikztiming@overlay@end}
% Start new drawing path, restore position and saved drawing state.
%    \begin{macrocode}
\def\tikztiming@overlay@end{%
  \tikztiming@newdraw (tikztiming@overlay@node)
  [/utils/exec={\tikztiming@overlay@save}]
}%
%    \end{macrocode}
% \end{macro}
%
%
% \iffalse
%</lib-overlays>
% \fi
%
% \subsection{Column Type}
% \iffalse
%<*lib-columntype>
% \fi
% The \pkg{array} package is needed to define the column type.
%    \begin{macrocode}
\RequirePackage{array}

\tikzset{timing/columntype/.code 2 args =
  \newcolumntype{#1}{>{\celltiming}#2<{\endcelltiming}}%
}%
%    \end{macrocode}
% 
% The options |type| (calls |timing/columntype|) and |notype| are defined.
%    \begin{macrocode}
\RequirePackage{pgfopts}
\tikzset{timing/columntype/.cd,
  type/.style 2 args={/tikz/timing/columntype={#1}{#2}},%
  notype/.code={},
  notype/.value forbidden={},
}

\ProcessPgfOptions{/tikz/timing/columntype}
%    \end{macrocode}
% Use default settings if no options where given. This works because all options 
% so far will change the default type.
%    \begin{macrocode}
\expandafter\ifx\csname opt@\@currname.\@currext\endcsname\empty
  \tikzset{timing/columntype={T}{l}}%
\fi
%    \end{macrocode}
%
% \begin{macro}{\celltiming}[1]{Some ignored tokens}
% The macro first reads everything to the |\ignorespaces| placed by the 
% |tabular|. After placing the tokens except |\ignorespaces| back a 
% |tikzpicture| is started. The \cs{tikztiming@endhook} is defined to call 
% \cs{end@celltiming}.
% After the usual init process the input is tested for an optional argument and 
% \cs{tikztiming@celltiming} is called accordantly.
%    \begin{macrocode}
\def\celltiming#1\ignorespaces{%
  #1%
  \begin{tikzpicture}[timing/picture,baseline={(0,0)}]
    \def\tikztiming@endhook{\end@celltiming}
    \tikztiming@init
    \let\tikztiming@settings\empty
    \@ifnextchar{[}%
      {\tikztiming@celltiming}%
      {\tikztiming@celltiming[]}%
}
\let\endcelltiming\relax
%    \end{macrocode}
% \end{macro}
%
% \begin{macro}{\tikztiming@celltiming}[1]{Optional Initial Character or 
% Settings}
% First the optional argument is tested to see if it is a single initial 
% character or some settings. The \cs{tikztiming@testoptarg} macro sets the 
% \cs{lastchar} and \cs{settings} macros to the appropriate values.
%
% If the \cs{lastchar} is empty it is set to the normal start value.
%
% Then a \pkg{pgf} scope with the potential user settings is started. The macros  
% |\unskip| and |\\| which are used internally by the |tabular| environment are 
% |\let| to |\relax| to work as end marker for the \cs{tikztiming@parser}.  This 
% will not influence the table because they will revert to their original 
% definition at the end of the |tikzpicture| environment.
%
% Finally the \cs{tikztiming@parser} is called.
%    \begin{macrocode}
\def\tikztiming@celltiming[#1]{%
  \ifx\relax#1\relax\else
    \tikztiming@testoptarg#1\relax\relax
  \fi
  \ifx\tikztiming@lastchar\empty
    \def\tikztiming@lastchar{@}%
  \fi
  \expandafter\scope\expandafter[\tikztiming@settings]%
    \begingroup
    \let\unskip\relax
    \let\\\relax
    \tikztiming@parser
}
%    \end{macrocode}
% \end{macro}
%
% \begin{macro}{\end@celltiming}
% This macro is called from the \cs{tikztiming@endhook} which is located in 
% \cs{tikztiming@parser@end} when the end marker (|\relax| or alias) is 
% encountered. It executes the accumulated \cs{tikztiming@str} and places the 
% timing shape. Finally the |scope| and |tikzpicture| is closed.
%    \begin{macrocode}
\def\end@celltiming{%
    \tikzset{timing/before}%
    \tikztiming@str;
    \expandafter\node 
    \expandafter[\tikztiming@settings,shape=tikztiming@shape,anchor=origin] at (0,0) {};
    \tikzset{timing/after}%
  \endscope
  \end{tikzpicture}%
}
%    \end{macrocode}
% \end{macro}
%
% \iffalse
%</lib-columntype>
% \fi
%
% \subsection{Nice Timing Tables}
% \iffalse
%<*lib-nicetabs>
% \fi
% The dimensions of the following packages are used.
%    \begin{macrocode}
\RequirePackage{array}
\RequirePackage{booktabs}
%    \end{macrocode}

%
% \begin{macro}{\tikztiming@nicetabs@tableheader}[1]{Optional Settings}
% Calculate separator heights and set nodes appropriately, then call
% the normal table header macro.
%    \begin{macrocode}
\newcommand*\tikztiming@nicetabs@tableheader[1][]{%
  \path [timing/table/midrules] [/utils/exec=\xdef\@gtempa{\the\pgflinewidth}];
  \path [timing/table/header,#1]
    let \p1 = (0,\belowrulesep + \aboverulesep + \@gtempa 
            + \arraystretch\ht\strutbox + \arraystretch*\extrarowheight 
            + \arraystretch\dp\strutbox)
    in
      coordinate (diagram@header) at (0,\y1)
      coordinate (label@header)   at ($ -1*(\tikztiming@coldist,0) + (0,\y1) $)
    ;
  \tikztiming@normal@tableheader[#1]%
}
%    \end{macrocode}
% \end{macro}
%
% \begin{macro}{\tikztiming@nicetabs@tablerules}[1]{Optional Settings}
%    \begin{macrocode}
\newcommand*\tikztiming@nicetabs@tablerules[1][]{%
  \begin{scope}[#1]
    \coordinate (NW) at (current bounding box.north west);
    \coordinate (SE) at (current bounding box.south east);
    %
    \draw [timing/table/rules]
    let
      \p1 = (NW),
      \p2 = (SE),
      \p3 = (last label.base),
      \p4 = ($ (0,\y3) - (0,\arraystretch\dp\strutbox + \aboverulesep +
        .5\pgflinewidth) $)
    in
      (\x1-\tabcolsep,\y4) -- (\x2+\tabcolsep,\y4)
     \iftikztiming@tablehead
      let
        \p5 = (diagram@header.base),
        \p6 = ($ (0,\y5) + (0,\arraystretch\ht\strutbox + \arraystretch*\extrarowheight
        + \belowrulesep + .5\pgflinewidth) $)
      in
        (\x1-\tabcolsep,\y6) -- (\x2+\tabcolsep,\y6)
     \fi
    ;
    \draw
     \iftikztiming@tablehead [timing/table/midrules] \else [timing/table/rules] \fi
    let
      \p1 = (NW),
      \p2 = (SE),
      \p3 = (0, \arraystretch\ht\strutbox + \arraystretch*\extrarowheight +
        \belowrulesep + .5\pgflinewidth)
    in
      (\x1-\tabcolsep, \y3)
       --
      (\x2+\tabcolsep, \y3)
    ;
  \end{scope}
}
%    \end{macrocode}
% \end{macro}
%
%    \begin{macrocode}
\let\tikztiming@normal@tableheader\tikztiming@tableheader
\let\tikztiming@normal@tablerules\tikztiming@tablerules
\let\tikztiming@normal@coldist\tikztiming@coldist
\let\tikztiming@normal@rowdist\tikztiming@rowdist

\tikzset{%
  timing/no nice tabs/.code={%
    \let\tikztiming@coldist\tikztiming@normal@coldist
    \let\tikztiming@rowdist\tikztiming@normal@rowdist
    \let\tikztiming@tableheader\tikztiming@normal@tableheader
    \let\tikztiming@tablerules\tikztiming@normal@tablerules
  },
  timing/nice tabs/.code={%
    \ifx\tikztiming@tableheader\tikztiming@nicetabs@tableheader\else
      \let\tikztiming@normal@coldist\tikztiming@coldist
      \let\tikztiming@normal@rowdist\tikztiming@rowdist
    \fi
    \def\tikztiming@rowdist{%
       \arraystretch\ht\strutbox
      +\arraystretch*\extrarowheight
      +\arraystretch\dp\strutbox
    }%
    \def\tikztiming@coldist{2\tabcolsep}%
    \let\tikztiming@tableheader\tikztiming@nicetabs@tableheader
    \let\tikztiming@tablerules\tikztiming@nicetabs@tablerules
  },
  timing/nicetabs/no nice tabs/.style={/tikz/timing/no nice tabs},
  timing/nicetabs/off/.style={/tikz/timing/no nice tabs},
  timing/nicetabs/nice tabs/.style={/tikz/timing/nice tabs},
  timing/nicetabs/on/.style={/tikz/timing/nice tabs},
}
%
\tikzset{timing/nice tabs}%
%    \end{macrocode}
%
% Use the \pkg{pgfopts} to handle the package options.
%    \begin{macrocode}
\RequirePackage{pgfopts}
\ProcessPgfOptions{/tikz/timing/nicetabs}
%    \end{macrocode}
% \iffalse
%</lib-nicetabs>
% \fi
%
% \subsection{Counters}
% \iffalse
%<*lib-counters>
% \fi
%
%    \begin{macrocode}
\newif\if@tikztiming@counter@wraps

\tikzset{%
  timing/new counter/.style={timing/counter/new={#1}},
  timing/counter/.cd,
  new/.code={\tikztiming@counter@new{#1}},
  new/.value required,
  opt/.cd,
  char/.initial=,
  char/.value required,
  reset char/.initial=,
  reset char/.value required,
  reset type/.is choice,
  reset type/width/.code={\def\tikztiming@resettype{0}},
  reset type/arg/.code={\def\tikztiming@resettype{1}},
  reset type/both/.code={\def\tikztiming@resettype{2}},
  reset type/Both/.code={\def\tikztiming@resettype{3}},
  half width char/.initial=,
  half width char/.value required,
  base/.initial=10,
  base/.value required,
  increment/.initial=1,
  increment/.value required,
  max value/.initial=,
  max value/.value required,
  min value/.initial=,
  min value/.value required,
  wraps/.is if=@tikztiming@counter@wraps,
  digits/.initial=,
  digits/.value required,
  fg style/.initial=,
  bg style/.initial=,
  text style/.initial=,
  text format/.initial=,
  fg style/.value required,
  bg style/.value required,
  text style/.value required,
  text format/.value required,
}
\def\tikztiming@counter@textformat{}%
%    \end{macrocode}
%
% \begin{macro}{\tikztimingcounter}[1]{Counter Character}
%    \begin{macrocode}
\def\tikztimingcounter#1{%
  \csname tikztiming@counter@#1\endcsname
}
%    \end{macrocode}
% \end{macro}
%
% \begin{macro}{\tikztimingsetcounter}[2]{Counter Character}{pgfmath expression}
%    \begin{macrocode}
\def\tikztimingsetcounter#1#2{%
  \begingroup
    \pgfmathtruncatemacro\@tempa{#2}%
    \expandafter\global\expandafter\let\csname tikztiming@counter@#1\endcsname\@tempa
  \endgroup
}
%    \end{macrocode}
% \end{macro}
%
% \begin{macro}{\tikztiming@counter@new}[1]{Styles}
% Most parts of this macro is enclosed in a group to keep most changes locally.
%    \begin{macrocode}
\def\tikztiming@counter@new#1{%
 \begingroup
  \def\tikztiming@resettype{0}%
%    \end{macrocode}
% The argument is taken as pgf keys relative to the "/tikz/timing/counter/opt" path.
% The counter character is mandatory and an error is raised if it is not defined.
%    \begin{macrocode}
  \pgfqkeys{/tikz/timing/counter/opt}{#1}%
  \tikzset{timing/counter/opt/char/.get=\counterchar}%
  \ifx\counterchar\empty
    \PackageError{tikz-timing-counters}{Counter character must be given using 
    the 'char' key!}{}%
    \endgroup
  \else
%    \end{macrocode}
% The counter boundary test code is now assembled in |\@tempb|.
% If the "wraps" option was used the \cs{tikztiming@counter@wrap} macro is used with the
% "min value" and "max value" as (expanded) arguments.
%    \begin{macrocode}
    \pgfkeys{/tikz/timing/counter/opt/min value/.get=\min}%
    \pgfkeys{/tikz/timing/counter/opt/max value/.get=\max}%
    \if@tikztiming@counter@wraps
      \ifx\min\empty
        \def\min{0}%
      \fi
      \ifx\max\empty
        \PackageError{tikz-timing-counters}{The option 'wraps' needs the option 'max value' to be set!}{}%
      \fi
      \def\@tempb{%
        \noexpand\tikztiming@counter@wrap%
          {\expandafter\unexpanded\expandafter{\min}}%
          {\expandafter\unexpanded\expandafter{\max}}%
      }%
    \else
%    \end{macrocode}
% Otherwise either the \cs{tikztiming@counter@maxsat}, \cs{tikztiming@counter@minsat} or \cs{tikztiming@counter@minmaxsat}
% macro is used if "max value", "min value" or both are set, respectively.
%    \begin{macrocode}
      \ifx\min\empty
        \ifx\max\empty
          \let\@tempb\empty
        \else
          \def\@tempb{%
            \noexpand\tikztiming@counter@maxsat%
              {\expandafter\unexpanded\expandafter{\max}}%
          }%
        \fi
      \else
        \ifx\max\empty
          \def\@tempb{%
            \noexpand\tikztiming@counter@minsat%
              {\expandafter\unexpanded\expandafter{\min}}%
          }%
        \else
          \def\@tempb{%
            \noexpand\tikztiming@counter@minmaxsat%
              {\expandafter\unexpanded\expandafter{\min}}%
              {\expandafter\unexpanded\expandafter{\max}}%
          }%
        \fi
      \fi
    \fi
%    \end{macrocode}
% The counter styles are set up.
%    \begin{macrocode}
    \pgfkeys{/tikz/timing/counter/opt/fg style/.get=\fgstyle}%
    \pgfkeys{/tikz/timing/counter/opt/bg style/.get=\bgstyle}%
    \pgfkeys{/tikz/timing/counter/opt/text style/.get=\textstyle}%
    \pgfkeys{/tikz/timing/counter/opt/text format/.get=\textformat}%
%    \end{macrocode}
% Finally the \cs{tikztiming@counter@new@} macro is called with most of the 
% settings as expanded arguments.
%    \begin{macrocode}
    \pgfkeys{/tikz/timing/counter/opt/increment/.get=\increment}%
    \edef\@tempa{%
      \noexpand\tikztiming@counter@new@
        {\pgfkeysvalueof{/tikz/timing/counter/opt/char}}%
        {\pgfkeysvalueof{/tikz/timing/counter/opt/base}}%
        {\pgfkeysvalueof{/tikz/timing/counter/opt/half width char}}%
        {\pgfkeysvalueof{/tikz/timing/counter/opt/reset char}}%
        {\pgfkeysvalueof{/tikz/timing/counter/opt/digits}}%
        {\expandafter\unexpanded\expandafter{\increment}}%
        {\@tempb}%
        {\min}%
        {\tikztiming@resettype}%
        {\expandafter\unexpanded\expandafter{\fgstyle}}%
        {\expandafter\unexpanded\expandafter{\bgstyle}}%
        {\expandafter\unexpanded\expandafter{\textstyle}}%
        {\expandafter\unexpanded\expandafter{\textformat}}%
    }%
    \expandafter\expandafter\expandafter\endgroup
    \expandafter\@tempa
  \fi
}
%    \end{macrocode}
% \end{macro}
%
% \begin{macro}{\tikztiming@counter@new@}[9]{Character}{Base}{Half Width 
% Character}{Reset Character}{Number of Digits}{Increment}{Boundary Code}{Min Value}{Reset Type}
%    \begin{macrocode}
\def\tikztiming@counter@new@#1#2#3#4#5#6#7#8#9{%
  \ifx\empty#8\empty
    \@namedef{tikztiming@counter@#1}{0}%
  \else
    \expandafter
    \pgfmathtruncatemacro\csname tikztiming@counter@#1\endcsname{#8}%
  \fi
  \tikztimingmetachar{#1}[1]{ { [timing/counter/#1] ##1D{[timing/counter/#1/text]\tikztiming@counter@vs{#1}{#2}{#5}{#6}{#7}} } }%
  \ifx\empty#3\empty
      \tikztiming@iflower{#1}{}%
        {\lowercase{\tikztimingmetachar{#1}}[1]%
          { { [timing/counter/#1] ##1d{[timing/counter/#1/text]\tikztiming@counter@vs{#1}{#2}{#5}{#6}{#7}}} }%
        }%
  \else
    \tikztimingmetachar{#3}[1]{%
      { [timing/counter/#1] ##1d{[timing/counter/#1/text]\tikztiming@counter@vs{#1}{#2}{#5}{#6}{#7}} }%
    }%
  \fi
  \ifx\empty#4\empty\else
    \ifcase0#9\relax
      \tikztimingmetachar{#4}[1]{[|/utils/exec=\tikztiming@counter@reset{#1}{##1}|]}%
    \or
      \tikztimingmetachar{#4}[2]{[|/utils/exec=\tikztimingsetcounter{#1}{##2}|]}%
    \or
      \uppercase{\tikztimingmetachar{#4}}[1]{[|/utils/exec=\tikztiming@counter@reset{#1}{##1}|]}%
      \lowercase{\tikztimingmetachar{#4}}[2]{[|/utils/exec=\tikztimingsetcounter{#1}{##2}|]}%
    \or
      \lowercase{\tikztimingmetachar{#4}}[1]{[|/utils/exec=\tikztiming@counter@reset{#1}{##1}|]}%
      \uppercase{\tikztimingmetachar{#4}}[2]{[|/utils/exec=\tikztimingsetcounter{#1}{##2}|]}%
    \fi
  \fi
  \tikztiming@counter@new@@{#1}%
}
%    \end{macrocode}
% \end{macro}
%
% \begin{macro}{\tikztiming@counter@new@@}[5]{Character}{Foreground Style}{Background Style}{Text Style}
% {Text format}
% The counter styles are set up.
%    \begin{macrocode}
\def\tikztiming@counter@new@@#1#2#3#4#5{%
  \tikzset{timing/counter/#1/.style={}}%
  \ifx\empty#2\empty\else
    \tikzset{timing/counter/#1/.append style={%
      /tikz/timing/d/.style={#2}%
    }}%
  \fi
  \ifx\empty#3\empty\else
    \tikzset{timing/counter/#1/.append style={%
      /tikz/timing/d/background/.style={#3}%
    }}%
  \fi
  \tikzset{timing/counter/#1/text/.style={#4}}%
  \ifx\empty#5\empty\else
    \tikzset{timing/counter/#1/.append style={%
      /utils/exec={\def\tikztiming@counter@textformat{#5}}%
    }}%
  \fi
}
%    \end{macrocode}
% \end{macro}
%
% \begin{macro}{\tikztiming@counter@reset}[2]{Character to Reset}{Reset value}
% The counter is set first using pgf to round the value correctly and then again 
% with \cs{setcounter} to make it global.
%    \begin{macrocode}
\def\tikztiming@counter@reset#1#2{%
  \begingroup
    \pgfmathtruncatemacro\@tempa{round((#2)/\tikztiming@wscale)}%
    \expandafter\global\expandafter\let\csname tikztiming@counter@#1\endcsname\@tempa
  \endgroup
}
%    \end{macrocode}
% \end{macro}
%
% \begin{macro}{\tikztiming@counter@vs}[5]{Character}{Base}{Number of Digits}{Increment}{Boundary Code}
% Note: \cs{pgfmathdectoBase} is only used for a base different to 10 to allow
% negative number for decimal base.
%    \begin{macrocode}
\def\tikztiming@counter@vs#1#2#3#4#5{%
  \begingroup
    \ifx\empty#3\empty\else
      \pgfmathsetbasenumberlength{#3}%
    \fi
    \ifnum#2=10
      \expandafter\expandafter\expandafter\tikztiming@counter@textformat
      \expandafter\expandafter\expandafter{\csname tikztiming@counter@#1\endcsname}%
    \else
      \pgfmathdectoBase\pgfmathresult{\@nameuse{tikztiming@counter@#1}}{#2}%
      \expandafter\tikztiming@counter@textformat\expandafter{\pgfmathresult}%
    \fi
    \edef\N{\@nameuse{tikztiming@counter@#1}}%
    \pgfmathtruncatemacro\N{\N + #4}%
    #5%
    \expandafter\global\expandafter\let\csname tikztiming@counter@#1\endcsname\N
  \endgroup
}
%    \end{macrocode}
% \end{macro}
%
% The following macros checks the boundaries of the counter value. The value is stored in the
% macro \cs{N}, the boundaries are provided as arguments. If necessary the value can be
% adjusted by changing \cs{N}. All values are taken as \emph{integers}.
%
% \begin{macro}{\tikztiming@counter@minsat}[1]{Minimum Value}
%    \begin{macrocode}
\def\tikztiming@counter@minsat#1{%
  \pgfmathtruncatemacro\MIN{#1}%
  \ifnum\N<\MIN \let\N\MIN \fi
}
%    \end{macrocode}
% \end{macro}
%
% \begin{macro}{\tikztiming@counter@maxsat}[1]{Maximum Value}
%    \begin{macrocode}
\def\tikztiming@counter@maxsat#1{%
  \pgfmathtruncatemacro\MAX{#1}%
  \ifnum\N>\MAX \let\N\MAX \fi
}
%    \end{macrocode}
% \end{macro}
%
% \begin{macro}{\tikztiming@counter@minmaxsat}[2]{Minimum Value}{Maximum Value}
%    \begin{macrocode}
\def\tikztiming@counter@minmaxsat#1#2{%
  \tikztiming@counter@minsat{#1}%
  \tikztiming@counter@maxsat{#2}%
}
%    \end{macrocode}
% \end{macro}
%
% \begin{macro}{\tikztiming@counter@wrap}[2]{Minimum Value}{Maximum Value}
%    \begin{macrocode}
\def\tikztiming@counter@wrap#1#2{%
  \pgfmathtruncatemacro\MIN{#1}%
  \pgfmathtruncatemacro\MAX{#2}%
  \ifnum\N>\MAX
    \pgfmathtruncatemacro\N{\MIN + (\N-\MAX-1)}%
  \fi
  \ifnum\N<\MIN
    \pgfmathtruncatemacro\N{\MAX - (\MIN-\N+1)}%
  \fi
}
%    \end{macrocode}
% \end{macro}
%
% Use the \pkg{pgfopts} to handle the package options.
%    \begin{macrocode}
\RequirePackage{pgfopts}
\ProcessPgfOptions{/tikz/timing/counter}
%    \end{macrocode}
% \iffalse
%</lib-counters>
% \fi
%
% \subsection{Advanced Nodes}
% \iffalse
%<*lib-advnodes>
% \fi
%    \begin{macrocode}

\let\tikztiming@orig@oldnodecenter\tikztiming@oldnodecenter

\tikzset{timing/nodes/.cd,
  simple/.code={%
    \let\tikztiming@oldnodecenter\tikztiming@orig@oldnodecenter
    \tikzset{%
      timing/inline node/.style={shape=coordinate},
    }%
  },
  advanced/.code={%
    \let\tikztiming@oldnodecenter\@gobble
    \tikzset{%
      timing/inline node/.style={shape=tikztiming coordinate},
      timing/nodes/new center,
    }%
  },
  old center/.code={%
    \let\tikztiming@nodecenteranchor\tikztiming@old@nodecenteranchor
  },%
  new center/.code={%
    \let\tikztiming@nodecenteranchor\empty
  },%
}
\tikzset{timing/nodes/.cd,advanced,new center}

\def\anchoralias#1#2{%
  \expandafter\def
  \csname pgf@anchor@\shape@name @#1\expandafter\endcsname
  \expandafter{\csname pgf@anchor@\shape@name @#2\endcsname}%
}

\def\tikztiming@anchorpoint#1#2#3{%
  \pgf@xa=#2\relax
  \pgf@ya=#3\relax
  \expandafter\edef\csname#1\endcsname{%
    \pgf@x=\the\pgf@xa\relax
    \pgf@y=\the\pgf@ya\relax
  }%
  \expandafter\addtosavedmacro\csname#1\endcsname
}
%    \end{macrocode}
%
% \begin{macro}{\tikztiming@tryanchor@}[2]{From}{To}
%    \begin{macrocode}
\def\tikztiming@tryanchor#1#2{%
  \let\tikztiming@next\relax
  \ifnum\tikztiming@debug>2
    \message{^^J Try Anchor: '#1' '#2' ^^J }%
    \def\tikztiming@next{\message{^^J TERM ^^J}}%
  \fi
  \ifx\relax#1\empty\else
  \ifx\relax#2\empty\else
    \def\tikztiming@next{%
      \@ifundefined{tikztiming@anchors@#1@#2}%
        {%
          \edef\@tempa{%
            {\csname tikztiming@alias@#1\endcsname}%
            {\csname tikztiming@alias@#2\endcsname}%
          }%
          \expandafter\tikztiming@tryanchor@\@tempa{#1}{#2}%
        }%
        {\let\tikztiming@tryanchor\@gobbletwo
         \ifnum\tikztiming@debug>2
           \message{^^J Anchor: '#1' '#2' ^^J }%
         \fi
         \@nameuse{tikztiming@anchors@#1@#2}%
        }%
    }%
  \fi\fi
  \tikztiming@next
}
%    \end{macrocode}
% \end{macro}
%
% \begin{macro}{\tikztiming@tryanchor@}[4]{Alias From}{Alias To}{From}{To}
%    \begin{macrocode}
\def\tikztiming@tryanchor@#1#2#3#4{%
  \tikztiming@tryanchor{#3}{#2}%
  \tikztiming@tryanchor{#1}{#4}%
  \tikztiming@tryanchor{#1}{#2}%
  \let\tikztiming@tryanchor@\@gobblefour
  \tikztiming@tryanchor{#3}{}%
  \tikztiming@tryanchor{}{#4}%
}
%    \end{macrocode}
% \end{macro}
%
%    \begin{macrocode}
\pgfdeclareshape{tikztiming coordinate}{%
  \savedmacro\anchorpoints{%
    \let\anchorpoint\tikztiming@anchorpoint
    \let\slope\@tempdima
    \let\dslope\@tempdimb
    \let\zslope\@tempdimc
    \let\height\pgf@yb
    \let\fromchar\tikztiming@node@fromchar
    \let\tochar\tikztiming@node@tochar
    %
    \ifnum\tikztiming@debug>2
      \message{^^J Node: \fromchar \tochar ^^J}%
    \fi
    %
    \edef\@tempa{{timing/\tochar}}%
    \expandafter\lowercase\expandafter
      {\expandafter\tikzset\@tempa}%
    \pgfpointxy{\timingslope}{0}%
    \slope=\pgf@x\relax
    \pgfpointxy{\timingdslope}{0}%
    \dslope=\pgf@x\relax
    \pgfpointxy{\timingzslope}{0}%
    \zslope=\pgf@x\relax
    \pgfpointxy{0}{\timingheight}%
    \height=\pgf@y\relax
    \edef\anchorpoints{%
      \noexpand\pgf@yb=\the\height\relax
      \let\noexpand\height\noexpand\pgf@yb
    }%
    %
    \tikztiming@tryanchor{\fromchar}{\tochar}%
  }
  \savedanchor\lowpoint{%
    \pgfpointanchor{timing/start base}{center}%
    \pgf@x=0pt
  }%
  \anchor{origin}{%
    \pgfpointorigin
    \anchorpoints
    \@nameuse{originoffset}
    \pgf@x=-\pgf@x
    \pgf@y=-\pgf@y
  }%
  \anchor{old center}{%
    \pgfpointorigin
    \tikztiming@old@nodecenteranchor
  }%
  \anchor{new center}{\pgfpointorigin}%
  \anchor{center}{%
    \pgfpointorigin
    \tikztiming@nodecenteranchor
  }%
  \anchor{MID}{\tikztiming@useanchor{HIGH}\advance\pgf@y by -.5\height\relax }%
  \anchor{LOW}{\tikztiming@useanchor{HIGH}\advance\pgf@y by -\height\relax   }%
  \anchor{HIGH}{\tikztiming@useanchor{HIGH}}%
  \anchor{TOP}{%
    \tikztiming@nodes@getoffset
    \tikztiming@useanchor{HIGH}
    \advance\pgf@y by  \pgf@ya\relax
  }%
  \anchor{BOTTOM}{%
    \tikztiming@nodes@getoffset
    \tikztiming@useanchor{HIGH}
    \advance\pgf@y by -\pgf@ya\relax
    \advance\pgf@y by -\height\relax
  }%
  \anchor{mid}{\tikztiming@useanchor{mid}}%
  \anchor{low}{\tikztiming@useanchor{low}}%
  \anchor{high}{\tikztiming@useanchor{high}}%
  \anchor{low2}{\tikztiming@useanchor{low2}}%
  \anchor{high2}{\tikztiming@useanchor{high2}}%
  %
  \expandafter\anchorborder\expandafter{\csname pgf@anchor@\shape@name @center\endcsname}%
  %
  \nodeparts{}%
}

\def\tikztiming@nodes@getoffset{%
  \begingroup
    \pgfpointxy{0}{1}%
    \pgfmathsetresultunitscale{\pgf@y}%
    \pgfpoint{0}{\tikztiming@nodes@offset}%
    \global\pgf@ya=\pgf@y
  \endgroup
}

\tikzset{%
  timing/nodes/.cd,
  offset/.store in=\tikztiming@nodes@offset,
  offset=0.25,
}

\def\tikztiming@nodecenteranchor{}

\def\tikztiming@old@nodecenteranchor{%
    \anchorpoints
    \@ifundefined{oldcenter}%
      {\@nameuse{originoffset}%
       \pgf@x=-\pgf@x\relax
       \pgf@y=-\pgf@y\relax
      }%
      {\@nameuse{oldcenter}}%
}

\def\tikztiming@useanchor#1{%
  \anchorpoints
  \@ifundefined{#1}%
   {\PackageError{tikz-timing}%
      {No such anchor '#1' defined for this transition!}{}%
    \pgfpointorigin
   }%
   {\@nameuse{#1}}%
}

\tikzset{%
  every tikztiming coordinate node/.style={anchor=origin},
}
%    \end{macrocode}
%    \begin{macrocode}
\def\tikztiming@anchors#1#2{%
  \@namedef{tikztiming@anchors@#1@#2}%
}
\def\tikztiming@anchorslet#1#2#3#4{%
  \expandafter\let
  \csname tikztiming@anchors@#1@#2\expandafter\endcsname
  \csname tikztiming@anchors@#3@#4\endcsname
}

\tikztiming@anchors{@}{D}{%
  \anchorpoint{originoffset}{0pt}{.5\height}%
  \anchorpoint{mid}{0pt}{0pt}%
  \anchorpoint{high}{0pt}{.5\height}%
  \anchorpoint{low}{0pt}{-.5\height}%
  \anchorpoint{HIGH}{0pt}{.5\height}%
}

\tikztiming@anchors{@}{H}{%
  \anchorpoint{originoffset}{0pt}{\height}%
  \anchorpoint{high}{0pt}{0pt}%
  \anchorpoint{HIGH}{0pt}{0pt}%
}

\tikztiming@anchors{@}{L}{%
  \anchorpoint{originoffset}{0pt}{0pt}%
  \anchorpoint{low}{0pt}{0pt}%
  \anchorpoint{HIGH}{0pt}{\height}%
}

\tikztiming@anchors{@}{T}{%
  \anchorpoint{originoffset}{0pt}{0pt}%
  \anchorpoint{low}{0pt}{0pt}%
  \anchorpoint{HIGH}{0pt}{\height}%
}

\tikztiming@anchors{@}{Z}{%
  \anchorpoint{originoffset}{0pt}{.5\height}%
  \anchorpoint{mid}{0pt}{0pt}%
  \anchorpoint{HIGH}{0pt}{.5\height}%
}

\tikztiming@anchors{D}{D}{%
  \anchorpoint{originoffset}{0pt}{0pt}%
  \anchorpoint{mid}{0pt}{0pt}%
  \anchorpoint{HIGH}{0pt}{.5\height}%
  \anchorpoint{high}{0pt}{.5\height}%
  \anchorpoint{low}{0pt}{-.5\height}%
}

\tikztiming@anchors{L}{H}{%
  \anchorpoint{originoffset}{.5\slope}{.5\height}%
  \anchorpoint{mid}{0pt}{0pt}%
  \anchorpoint{high}{.5\slope}{.5\height}%
  \anchorpoint{low}{-.5\slope}{-.5\height}%
  \anchorpoint{HIGH}{-.5\slope}{.5\height}%
}

\tikztiming@anchors{E}{H}{%
  \anchorpoint{originoffset}{-.5\slope}{-.5\height}%
  \anchorpoint{oldcenter}{-.5\slope}{-.5\height}%
  \anchorpoint{mid}{0pt}{0pt}%
  \anchorpoint{high}{.5\slope}{.5\height}%
  \anchorpoint{low}{-.5\slope}{-.5\height}%
  \anchorpoint{HIGH}{-.5\slope}{.5\height}%
}

\tikztiming@anchors{T}{T}{%
  \anchorpoint{mid}{0pt}{0pt}%
  \anchorpoint{HIGH}{-.5\slope}{.5\height}%
  \ifnum 1=\c@tikztimingtrans
    \anchorpoint{originoffset}{.5\slope}{-.5\height}%
    \anchorpoint{high}{-.5\slope}{.5\height}%
    \anchorpoint{low}{.5\slope}{-.5\height}%
  \else
    \anchorpoint{originoffset}{.5\slope}{.5\height}%
    \anchorpoint{high}{.5\slope}{.5\height}%
    \anchorpoint{low}{-.5\slope}{-.5\height}%
  \fi
}
\tikztiming@anchorslet{T}{E}{T}{T}%

\tikztiming@anchors{T}{@}{%
  \ifnum 1=\c@tikztimingtrans
    \anchorpoint{high}{0pt}{0pt}%
    \anchorpoint{HIGH}{0pt}{0pt}%
  \else
    \anchorpoint{low}{0pt}{0pt}%
    \anchorpoint{HIGH}{0pt}{\height}%
  \fi
}

\tikztiming@anchors{H}{H}{%
  \anchorpoint{originoffset}{0pt}{0pt}%
  \anchorpoint{high}{0pt}{0pt}%
  \anchorpoint{HIGH}{0pt}{0pt}%
}

\tikztiming@anchors{L}{L}{%
  \anchorpoint{originoffset}{0pt}{0pt}%
  \anchorpoint{low}{0pt}{0pt}%
  \anchorpoint{HIGH}{0pt}{\height}%
}

\tikztiming@anchors{Z}{Z}{%
  \anchorpoint{originoffset}{0pt}{0pt}%
  \anchorpoint{mid}{0pt}{0pt}%
  \anchorpoint{HIGH}{0pt}{.5\height}%
}

\tikztiming@anchors{E}{E}{%
  \anchorpoint{mid}{0pt}{0pt}%
  \ifnum 1=\c@tikztimingtrans
    \anchorpoint{originoffset}{.5\slope}{.5\height}%
    \anchorpoint{HIGH}{-.5\slope}{.5\height}%
    \anchorpoint{high}{.5\slope}{.5\height}%
    \anchorpoint{low}{-.5\slope}{-.5\height}%
  \else
    \anchorpoint{originoffset}{-.5\slope}{.5\height}%
    \anchorpoint{oldcenter}{-.5\slope}{-.5\height}%
    \anchorpoint{HIGH}{-.5\slope}{.5\height}%
    \anchorpoint{high}{-.5\slope}{.5\height}%
    \anchorpoint{low}{.5\slope}{-.5\height}%
  \fi
}

\tikztiming@anchors{H}{L}{%
  \anchorpoint{originoffset}{.5\slope}{-.5\height}%
  \anchorpoint{mid}{0pt}{0pt}%
  \anchorpoint{high}{-.5\slope}{.5\height}%
  \anchorpoint{HIGH}{-.5\slope}{.5\height}%
  \anchorpoint{low}{.5\slope}{-.5\height}%
}
\tikztiming@anchorslet{H}{E}{H}{L}%

\tikztiming@anchors{E}{L}{%
  \anchorpoint{originoffset}{-.5\slope}{.5\height}%
  \anchorpoint{oldcenter}{-.5\slope}{-.5\height}%
  \anchorpoint{mid}{0pt}{0pt}%
  \anchorpoint{high}{-.5\slope}{.5\height}%
  \anchorpoint{HIGH}{-.5\slope}{.5\height}%
  \anchorpoint{low}{.5\slope}{-.5\height}%
}

\tikztiming@anchors{E}{T}{%
  \anchorpoint{mid}{0pt}{0pt}%
  \anchorpoint{HIGH}{-.5\slope}{.5\height}%
  \ifnum 1=\c@tikztimingtrans
    \anchorpoint{originoffset}{-.5\slope}{-.5\height}%
    \anchorpoint{oldcenter}{-.5\slope}{-.5\height}%
    \anchorpoint{high}{.5\slope}{.5\height}%
    \anchorpoint{low}{-.5\slope}{-.5\height}%
  \else
    \anchorpoint{originoffset}{-.5\slope}{.5\height}%
    \anchorpoint{oldcenter}{-.5\slope}{-.5\height}%
    \anchorpoint{high}{-.5\slope}{.5\height}%
    \anchorpoint{low}{.5\slope}{-.5\height}%
  \fi
}

\tikztiming@anchors{E}{Z}{%
  \anchorpoint{mid}{0pt}{0pt}%
  \anchorpoint{oldcenter}{-\zslope}{-.5\height}%
  \anchorpoint{HIGH}{-\zslope}{.5\height}%
  \anchorpoint{high}{-\zslope}{.5\height}%
  \anchorpoint{low}{-\zslope}{-.5\height}%
}

\tikztiming@anchors{Z}{D}{%
  \anchorpoint{originoffset}{0pt}{0pt}%
  \anchorpoint{mid}{0pt}{0pt}%
  \anchorpoint{HIGH}{0pt}{.5\height}%
  \anchorpoint{high}{.5\dslope}{.5\height}%
  \anchorpoint{low}{.5\dslope}{-.5\height}%
}

\tikztiming@anchors{D}{Z}{%
  \anchorpoint{originoffset}{0pt}{0pt}%
  \anchorpoint{oldcenter}{-.5\dslope}{-.5\height}%
  \anchorpoint{mid}{0pt}{0pt}%
  \anchorpoint{HIGH}{-.5\dslope}{.5\height}%
  \anchorpoint{high}{-.5\dslope}{.5\height}%
  \anchorpoint{low}{-.5\dslope}{-.5\height}%
}

\tikztiming@anchors{D}{H}{%
  \anchorpoint{originoffset}{0pt}{0pt}%
  \anchorpoint{oldcenter}{-.5\dslope}{-.5\height}%
  \anchorpoint{mid}{0pt}{0pt}%
  \anchorpoint{HIGH}{-.5\dslope}{.5\height}%
  \anchorpoint{high}{.5\dslope}{.5\height}%
  \anchorpoint{high2}{-.5\dslope}{.5\height}%
  \anchorpoint{low}{-.5\dslope}{-.5\height}%
}

\tikztiming@anchors{D}{L}{%
  \anchorpoint{originoffset}{0pt}{0pt}%
  \anchorpoint{oldcenter}{-.5\dslope}{-.5\height}%
  \anchorpoint{mid}{0pt}{0pt}%
  \anchorpoint{HIGH}{-.5\dslope}{.5\height}%
  \anchorpoint{high}{-.5\dslope}{.5\height}%
  \anchorpoint{low2}{-.5\dslope}{-.5\height}%
  \anchorpoint{low}{.5\dslope}{-.5\height}%
}
\tikztiming@anchorslet{D}{T}{D}{L}%

\tikztiming@anchors{D@edge@}{D}{%
  \anchorpoint{originoffset}{0pt}{0pt}%
  \anchorpoint{oldcenter}{-.5\dslope}{-.5\height}%
  \anchorpoint{mid}{0pt}{0pt}%
  \anchorpoint{HIGH}{-.5\dslope}{.5\height}%
  \anchorpoint{high}{-.5\dslope}{.5\height}%
  \anchorpoint{high2}{.5\dslope}{.5\height}%
  \anchorpoint{low}{-.5\dslope}{-.5\height}%
  \anchorpoint{low2}{.5\dslope}{-.5\height}%
}
\tikztiming@anchorslet{D}{U}{D@edge@}{D}
\tikztiming@anchorslet{U}{D}{D@edge@}{D}
\tikztiming@anchorslet{D}{D@edge@}{D@edge@}{D}
\tikztiming@anchorslet{D@edge@}{D@edge@}{D@edge@}{D}
\tikztiming@anchorslet{D}{E}{D@edge@}{D}
\tikztiming@anchorslet{E}{D}{D@edge@}{D}

\tikztiming@anchorslet{L}{E}{L}{H}%
\tikztiming@anchorslet{L}{T}{L}{H}%
\tikztiming@anchorslet{H}{T}{H}{L}%
\tikztiming@anchorslet{T}{H}{H}{L}%

\tikztiming@anchors{D}{D}{%
  \anchorpoint{originoffset}{0pt}{0pt}%
  \anchorpoint{oldcenter}{0pt}{-.5\height}%
  \anchorpoint{mid}{0pt}{0pt}%
  \anchorpoint{high}{0pt}{.5\height}%
  \anchorpoint{low}{0pt}{-.5\height}%
  \anchorpoint{HIGH}{0pt}{.5\height}%
}
\tikztiming@anchorslet{D}{@}{D}{D}
\tikztiming@anchorslet{E}{@}{D}{D}
\tikztiming@anchorslet{U}{U}{D}{D}

\tikztiming@anchorslet{@}{E}{@}{D}

\tikztiming@anchors{L}{D}{%
  \anchorpoint{originoffset}{.5\dslope}{.5\height}%
  \anchorpoint{mid}{0pt}{0pt}%
  \anchorpoint{high}{.5\dslope}{.5\height}%
  \anchorpoint{low}{-.5\dslope}{-.5\height}%
  \anchorpoint{low2}{.5\dslope}{-.5\height}%
  \anchorpoint{HIGH}{-.5\dslope}{.5\height}%
}

\tikztiming@anchors{T}{D}{%
  \anchorpoint{mid}{0pt}{0pt}%
  \anchorpoint{HIGH}{-.5\dslope}{.5\height}%
  \ifnum 1=\c@tikztimingtrans
    \anchorpoint{originoffset}{.5\dslope}{-.5\height}%
    \anchorpoint{high}{-.5\dslope}{.5\height}%
    \anchorpoint{high2}{.5\dslope}{.5\height}%
    \anchorpoint{low}{.5\dslope}{-.5\height}%
  \else
    \anchorpoint{originoffset}{.5\dslope}{.5\height}%
    \anchorpoint{high}{.5\dslope}{.5\height}%
    \anchorpoint{low}{-.5\dslope}{-.5\height}%
    \anchorpoint{low2}{.5\dslope}{-.5\height}%
  \fi
}

\tikztiming@anchors{H}{D}{%
  \anchorpoint{originoffset}{.5\dslope}{-.5\height}%
  \anchorpoint{mid}{0pt}{0pt}%
  \anchorpoint{HIGH}{-.5\dslope}{.5\height}%
  \anchorpoint{high}{-.5\dslope}{.5\height}%
  \anchorpoint{high2}{.5\dslope}{.5\height}%
  \anchorpoint{low}{.5\dslope}{-.5\height}%
}

\tikztiming@anchors{Z}{E}{%
  \anchorpoint{mid}{0pt}{0pt}%
  \anchorpoint{high}{\zslope}{.5\height}%
  \anchorpoint{low}{\zslope}{-.5\height}%
  \anchorpoint{HIGH}{0pt}{.5\height}%
}

\tikztiming@anchors{Z}{H}{%
  \anchorpoint{mid}{0pt}{0pt}%
  \anchorpoint{high}{\zslope}{.5\height}%
  \anchorpoint{HIGH}{0pt}{.5\height}%
}

\tikztiming@anchors{Z}{T}{%
  \anchorpoint{mid}{0pt}{0pt}%
  \anchorpoint{HIGH}{0pt}{.5\height}%
  \ifnum1=\c@tikztimingtrans\relax
    \anchorpoint{low}{\zslope}{-.5\height}%
  \else
    \anchorpoint{high}{\zslope}{.5\height}%
  \fi
}

\tikztiming@anchors{T}{H}{%
  \ifnum1=\c@tikztimingtrans\relax
    \anchorpoint{high}{0pt}{0pt}%
    \anchorpoint{HIGH}{0pt}{0pt}%
  \else
    \anchorpoint{originoffset}{.5\slope}{.5\height}%
    \anchorpoint{low}{-.5\slope}{-.5\height}%
    \anchorpoint{mid}{0pt}{0pt}%
    \anchorpoint{high}{.5\slope}{.5\height}%
    \anchorpoint{HIGH}{-.5\slope}{.5\height}%
  \fi
}

\tikztiming@anchors{T}{L}{%
  \ifnum1=\c@tikztimingtrans\relax
    \anchorpoint{originoffset}{.5\slope}{-.5\height}%
    \anchorpoint{low}{.5\slope}{-.5\height}%
    \anchorpoint{mid}{0pt}{0pt}%
    \anchorpoint{high}{-.5\slope}{.5\height}%
    \anchorpoint{HIGH}{-.5\slope}{.5\height}%
  \else
    \anchorpoint{low}{0pt}{0pt}%
    \anchorpoint{HIGH}{0pt}{\height}%
  \fi
}

\tikztiming@anchors{T}{Z}{%
  \ifnum1=\c@tikztimingtrans\relax
    \anchorpoint{high}{0pt}{0pt}%
    \anchorpoint{mid}{\zslope}{-.5\height}%
    \anchorpoint{HIGH}{0pt}{0pt}%
  \else
    \anchorpoint{low}{0pt}{0pt}%
    \anchorpoint{mid}{\zslope}{.5\height}%
    \anchorpoint{HIGH}{0pt}{\height}%
  \fi
}

\tikztiming@anchors{Z}{L}{%
  \anchorpoint{mid}{0pt}{0pt}%
  \anchorpoint{low}{\zslope}{-.5\height}%
  \anchorpoint{HIGH}{0pt}{.5\height}%
}

\tikztiming@anchors{H}{Z}{%
  \anchorpoint{high}{0pt}{0pt}%
  \anchorpoint{mid}{\zslope}{-.5\height}%
  \anchorpoint{HIGH}{0pt}{0pt}%
}

\tikztiming@anchors{L}{Z}{%
  \anchorpoint{low}{0pt}{0pt}%
  \anchorpoint{mid}{\zslope}{.5\height}%
  \anchorpoint{HIGH}{0pt}{\height}%
}

\tikztiming@anchors{Z}{Z}{%
  \anchorpoint{mid}{0pt}{0pt}%
  \anchorpoint{HIGH}{0pt}{.5\height}%
}

\tikztiming@anchors{H}{@}{%
  \anchorpoint{high}{0pt}{0pt}%
  \anchorpoint{HIGH}{0pt}{0pt}%
}

\tikztiming@anchors{L}{@}{%
  \anchorpoint{low}{0pt}{0pt}%
  \anchorpoint{HIGH}{0pt}{\height}%
}

\tikztiming@anchors{Z}{@}{%
  \anchorpoint{mid}{0pt}{0pt}%
  \anchorpoint{HIGH}{0pt}{.5\height}%
}

\tikztiming@anchors{}{@}{%
  \anchorpoint{originoffset}{0pt}{.5\height}%
  \anchorpoint{mid}{0pt}{0pt}%
  \anchorpoint{high}{0pt}{.5\height}%
  \anchorpoint{low}{0pt}{-.5\height}%
  \anchorpoint{HIGH}{0pt}{.5\height}%
}

%    \end{macrocode}
%
% Use the \pkg{pgfopts} to handle the package options.
%    \begin{macrocode}
\RequirePackage{pgfopts}
\ProcessPgfOptions{/tikz/timing/node}
%    \end{macrocode}
% \iffalse
%</lib-advnodes>
% \fi
%
% \subsection{Intervals}
% \iffalse
%<*lib-interval>
% \fi
%    \begin{macrocode}

\let\tikztiming@linterval\relax
\let\tikztiming@rinterval\relax

\tikztimingdef{LZ}{
  \tikztiming@rinterval
  {\newdraw [\style]}
  {-- ++(\zslope,+\height/2.)}
  -- ++($ (#1,0) - (\zslope,0) $)
}

\tikztimingdef{HZ}{%
  \tikztiming@rinterval
  {\newdraw [\style]}
  {-- ++(\zslope,-\height/2.)}
  -- ++($ (#1,0) - (\zslope,0) $)
}

\tikztimingdef{TZ}{%
  \code{\setcounter{tikztimingtrans}{-\value{tikztimingtrans}}}
  \tikztiming@rinterval
  {\newdraw [\style]}
  {-- ++(\zslope,\value{tikztimingtrans}*\height/2.)}
  -- ++($ (#1,0) - (\zslope,0) $)
}

\tikztimingdef{ZL}{%
  \tikztiming@linterval
  {\newdraw}
  {-- ++(\zslope,-\height/2.)}
  -- ++($ (#1,0) - (\zslope,0) $)
}

\tikztimingdef{ZH}{%
  \tikztiming@linterval
  {\newdraw}
  {-- ++(\zslope,\height/2.)}
  -- ++($ (#1,0) - (\zslope,0) $)
}

\tikztimingdef{ZT}{%
  \code{\setcounter{tikztimingtrans}{-\value{tikztimingtrans}}}
  \tikztiming@linterval
  {\newdraw [\style]}
  {-- ++(\zslope,\value{tikztimingtrans}*\height/2.)}
  -- ++($ (#1,0) - (\zslope,0) $)
}

\def\tikztiming@onetwo#1#2{#1#2}
\def\tikztiming@twoone#1#2{#2#1}

\let\tikztiming@linterval\tikztiming@onetwo
\let\tikztiming@rinterval\tikztiming@onetwo

\tikzset{timing/interval/.code={\pgfqkeys{/tikz/timing/interval}{#1}}}
\tikzset{timing/interval/.cd,
   lo/.style={/tikz/timing/interval/left open},
   lc/.style={/tikz/timing/interval/left close},
   ro/.style={/tikz/timing/interval/right open},
   rc/.style={/tikz/timing/interval/right close},
   left close/.code={\let\tikztiming@linterval\tikztiming@onetwo},
   left open/.code={\let\tikztiming@linterval\tikztiming@twoone},
   right open/.code={\let\tikztiming@rinterval\tikztiming@onetwo},
   right close/.code={\let\tikztiming@rinterval\tikztiming@twoone},
}
\tikzset{
  timing/interval/normal/.style={/tikz/timing/interval/left close,/tikz/timing/interval/right open},
  timing/interval/verilog/.style={/tikz/timing/interval/normal},
  timing/interval/metachar/.style={/tikz/timing/metachar={#1}[2]{[timing/interval={##2}]}},
}

%    \end{macrocode}
% \iffalse
%</lib-interval>
% \fi
%
% \subsection{Beamer Overlay Support}
% \iffalse
%<*lib-beamer>
% \fi
% Experimental library for marginal beamer overlay support. It will change in future versions. Use at own risk!
%    \begin{macrocode}

\PackageWarning{\@currname}{This library is experimental and will change in future versions!}{}{}

\tikztiming@defcode<#1>{
  \tikztiming@output@addcode{
    node [timing/save] (timing@save) {};%
    \onslide<#1>
    \draw [timing/restore] (timing@save) ++(0,0)
  }
  \tikztiming@parser
}
%    \end{macrocode}
% \iffalse
%</lib-beamer>
% \fi
%
% \subsection{Compatibility with \protect\pkg{ifsym} Macros}
% \iffalse
%<*lib-ifsym>
% \fi
%
% Set up styles to match behaviour of \pkg{ifsym}. Slope and size is is adjusted.
% The macros only have 60\% of the height.
%    \begin{macrocode}
\tikzset{%
  timing/ifsym/macro/.style={/tikz/timing/ifsym,/utils/exec={\pgfmathparse{.6*\f@size}\let\ysize\pgfmathresult},/tikz/y=\ysize},
  timing/ifsym/.style={%
    /tikz,timing/slope=0,x=\f@size,y=\f@size,line width=0.1ex,
%    \end{macrocode}
% Define "|" to create a glitch. While it can be ignored in "H|L" or "L|H" it can
% also be used like "L|L" and "H|H".
%    \begin{macrocode}
    timing/metachar={|}{G},
%    \end{macrocode}
% The "X" is used for "M" and is set to black. 
%    \begin{macrocode}
    timing/x/.style={},
%    \end{macrocode}
% Define "M"/"m" to be equal to "X"/"x":
%    \begin{macrocode}
    timing/metachar={M}[1]{#1X},
    timing/metachar={m}[1]{#1x},
%    \end{macrocode}
%
% Code changes:
%    \begin{macrocode}
    /utils/exec={%
%    \end{macrocode}
%
% The |setdslope| macro is redefined to allow value higher than 1.
%    \begin{macrocode}
\def\tikztimingsetdslope##1{%
  \pgfmathparse{max(0.0,##1)}%
  \let\tikztiming@dslope\pgfmathresult
  \edef\timingdslope{\tikztiming@dslope*\noexpand\timingwidth}%
}
%    \end{macrocode}
%
% The "<" and ">" characters are defined as code characters. They
% check if there appear twice which doubles their width.
% A macro is used to remove the second appearance from the input.
%    \begin{macrocode}
\tikztiming@defcode{<}{%
  \@ifnextchar{<}%
    {\tikztiming@ifsym@longslope{X}{D}}%
    {\tikztiming@parser [timing/dslope=1] 0Xd;}%
}%
\tikztiming@defcode{>}{%
  \@ifnextchar{>}%
    {\tikztiming@ifsym@longslope{D}{X}}%
    {\tikztiming@parser [timing/dslope=1] 0Dx;}%
}%
\def\tikztiming@ifsym@longslope##1##2##3{
    \tikztiming@parser [timing/dslope=2.] 0##1 ##2;%
}%
%    \end{macrocode}
%    \begin{macrocode}
  }},
}
%    \end{macrocode}
%
% Define library options.
%    \begin{macrocode}
\tikzset{%
  timing/ifsym/.cd,
  provide/.code={\let\tikztiming@ifsym@def\providecommand},
  renew/.code={\let\tikztiming@ifsym@def\renewcommand},
  new/.code={\let\tikztiming@ifsym@def\newcommand},
  off/.code={\let\tikztiming@ifsym@def\@gobbletwo},
}
%    \end{macrocode}
%
% Execute library options.
%    \begin{macrocode}
\tikzset{timing/ifsym/provide}%
\RequirePackage{pgfopts}
\ProcessPgfOptions{/tikz/timing/ifsym}
%    \end{macrocode}
%
% Provide all timing symbol macros from |[electronic]{ifsym}|:
%    \begin{macrocode}
\tikztiming@ifsym@def{\RaisingEdge}{\hbox{\texttiming[timing/ifsym/macro]{.4L .4H}}}
\tikztiming@ifsym@def{\FallingEdge}{\hbox{\texttiming[timing/ifsym/macro]{.4H .4L}}}
\tikztiming@ifsym@def{\ShortPulseHigh}{\hbox{\texttiming[timing/ifsym/macro]{.4L .2H .4L}}}
\tikztiming@ifsym@def{\ShortPulseLow}{\hbox{\texttiming[timing/ifsym/macro]{.4H .2L .4H}}}
\tikztiming@ifsym@def{\PulseHigh}{\hbox{\texttiming[timing/ifsym/macro]{.4L .6H .4L}}}
\tikztiming@ifsym@def{\PulseLow}{\hbox{\texttiming[timing/ifsym/macro]{.4H .6L .4H}}}
\tikztiming@ifsym@def{\LongPulseHigh}{\hbox{\texttiming[timing/ifsym/macro]{.4L H .4L}}}
\tikztiming@ifsym@def{\LongPulseLow}{\hbox{\texttiming[timing/ifsym/macro]{.4H L .4H}}}
%    \end{macrocode}
%
% Provide \cs{textifsym} macro:
% \begin{macro}{\textifsym}
%    \begin{macrocode}
\tikztiming@ifsym@def{\textifsym}{\texttiming[timing/ifsym]}
%    \end{macrocode}
% \end{macro}
%
% Clean-up:
%    \begin{macrocode}
\let\tikztiming@ifsym@def\@undefined
%    \end{macrocode}
% \iffalse
%</lib-ifsym>
% \fi
%
% \Finale
\endinput
% \iffalse
% vim: foldmarker=%<*,%</ foldmethod=marker foldenable foldtext=foldtext()
% \fi
