%% \CharacterTable
%%  {Upper-case    \A\B\C\D\E\F\G\H\I\J\K\L\M\N\O\P\Q\R\S\T\U\V\W\X\Y\Z
%%   Lower-case    \a\b\c\d\e\f\g\h\i\j\k\l\m\n\o\p\q\r\s\t\u\v\w\x\y\z
%%   Digits        \0\1\2\3\4\5\6\7\8\9
%%   Exclamation   \!     Double quote  \"     Hash (number) \#
%%   Dollar        \$     Percent       \%     Ampersand     \&
%%   Acute accent  \'     Left paren    \(     Right paren   \)
%%   Asterisk      \*     Plus          \+     Comma         \,
%%   Minus         \-     Point         \.     Solidus       \/
%%   Colon         \:     Semicolon     \;     Less than     \<
%%   Equals        \=     Greater than  \>     Question mark \?
%%   Commercial at \@     Left bracket  \[     Backslash     \\
%%   Right bracket \]     Circumflex    \^     Underscore    \_
%%   Grave accent  \`     Left brace    \{     Vertical bar  \|
%%   Right brace   \}     Tilde         \~}
%\iffalse
%
% (c) copyright 2004 Yannis Potamianos and Apostolos Syropoulos
% This program can be redistributed and/or modified under the terms
% of the LaTeX Project Public License Distributed from CTAN
% archives in directory macros/latex/base/lppl.txt; either
% version 1 of the License, or any later version.
%
% Please report errors or suggestions for improvement to
%
%    Yannis Potamianos     and/or    Apostolos Syropoulos
%    potamianos@sch.gr               apostolo@obelix.ee.duth.gr
%
%\fi
% \CheckSum{1030}
% \iffalse This is a Metacomment
%
%<elmath, >\ProvidesFile{elmath.sty}
%
%<elmath, >[2004/11/20 v1.2 Package `elmath.sty']
%
%    \begin{macrocode}
%<*driver>
\documentclass{ltxdoc}
\GetFileInfo{elmath.drv}
\begin{document}
   \DocInput{elmath.dtx}
\end{document}
%</driver>
%    \end{macrocode}
% \fi
%\StopEventually{}
%\MakeShortVerb{\|}
%\title{The `elmath' package}
%\author{Yannis Potamianos and Apostolos Syropoulos}
% \date{2004/11/20}
%\maketitle
% \begin{abstract}
% This package has been designed to partially implement the elmath-new input
% encoding in both and text and math modes. As such it supersedes package
% \textsf{elmath}, which was designed by the same authors. Note that if package
% \textsf{elmath-new} is used, then one should not use the (unofficial)
% \texttt{elmath-new} option of the \textsf{inputenc} package.
%\end{abstract}
%
% \section{Introduction and Usage}
% The package has been designed so to allow users with a Greek keyboard to
% type both Greek text and Greek mathematical symbols without any need to
% use special commands.
%
% Users of the package need only a standard ISO8859-7 keyboard, but obviously
% it is not difficult to adapt the package for other encoding.
%
% \section{The Implementation}
%
% Initially, we need to make sure that all the Greek letters are actually
% active. We assume that |~| is actually an active character.
%    \begin{macrocode}
%<*elmath>
\ifcat\noexpand~\noexpand^^e1\relax\else\catcode`\^^e1=\active\fi
\ifcat\noexpand~\noexpand^^e2\relax\else\catcode`\^^e2=\active\fi
\ifcat\noexpand~\noexpand^^e3\relax\else\catcode`\^^e3=\active\fi
\ifcat\noexpand~\noexpand^^e4\relax\else\catcode`\^^e4=\active\fi
\ifcat\noexpand~\noexpand^^e5\relax\else\catcode`\^^e5=\active\fi
\ifcat\noexpand~\noexpand^^e6\relax\else\catcode`\^^e6=\active\fi
\ifcat\noexpand~\noexpand^^e7\relax\else\catcode`\^^e7=\active\fi
\ifcat\noexpand~\noexpand^^e8\relax\else\catcode`\^^e8=\active\fi
\ifcat\noexpand~\noexpand^^e9\relax\else\catcode`\^^e9=\active\fi
\ifcat\noexpand~\noexpand^^ea\relax\else\catcode`\^^ea=\active\fi
\ifcat\noexpand~\noexpand^^eb\relax\else\catcode`\^^eb=\active\fi
\ifcat\noexpand~\noexpand^^ec\relax\else\catcode`\^^ec=\active\fi
\ifcat\noexpand~\noexpand^^ed\relax\else\catcode`\^^ed=\active\fi
\ifcat\noexpand~\noexpand^^ee\relax\else\catcode`\^^ee=\active\fi
\ifcat\noexpand~\noexpand^^ef\relax\else\catcode`\^^ef=\active\fi
\ifcat\noexpand~\noexpand^^f0\relax\else\catcode`\^^f0=\active\fi
\ifcat\noexpand~\noexpand^^f1\relax\else\catcode`\^^f1=\active\fi
\ifcat\noexpand~\noexpand^^f3\relax\else\catcode`\^^f3=\active\fi
\ifcat\noexpand~\noexpand^^f2\relax\else\catcode`\^^f2=\active\fi
\ifcat\noexpand~\noexpand^^f4\relax\else\catcode`\^^f4=\active\fi
\ifcat\noexpand~\noexpand^^f5\relax\else\catcode`\^^f5=\active\fi
\ifcat\noexpand~\noexpand^^f6\relax\else\catcode`\^^f6=\active\fi
\ifcat\noexpand~\noexpand^^f7\relax\else\catcode`\^^f7=\active\fi
\ifcat\noexpand~\noexpand^^f8\relax\else\catcode`\^^f8=\active\fi
\ifcat\noexpand~\noexpand^^f9\relax\else\catcode`\^^f9=\active\fi
\ifcat\noexpand~\noexpand^^c1\relax\else\catcode`\^^c1=\active\fi
\ifcat\noexpand~\noexpand^^c2\relax\else\catcode`\^^c2=\active\fi
\ifcat\noexpand~\noexpand^^c3\relax\else\catcode`\^^c3=\active\fi
\ifcat\noexpand~\noexpand^^c4\relax\else\catcode`\^^c4=\active\fi
\ifcat\noexpand~\noexpand^^c5\relax\else\catcode`\^^c5=\active\fi
\ifcat\noexpand~\noexpand^^c6\relax\else\catcode`\^^c6=\active\fi
\ifcat\noexpand~\noexpand^^c7\relax\else\catcode`\^^c7=\active\fi
\ifcat\noexpand~\noexpand^^c8\relax\else\catcode`\^^c8=\active\fi
\ifcat\noexpand~\noexpand^^c9\relax\else\catcode`\^^c9=\active\fi
\ifcat\noexpand~\noexpand^^ca\relax\else\catcode`\^^ca=\active\fi
\ifcat\noexpand~\noexpand^^cb\relax\else\catcode`\^^cb=\active\fi
\ifcat\noexpand~\noexpand^^cc\relax\else\catcode`\^^cc=\active\fi
\ifcat\noexpand~\noexpand^^cd\relax\else\catcode`\^^cd=\active\fi
\ifcat\noexpand~\noexpand^^ce\relax\else\catcode`\^^ce=\active\fi
\ifcat\noexpand~\noexpand^^cf\relax\else\catcode`\^^cf=\active\fi
\ifcat\noexpand~\noexpand^^d0\relax\else\catcode`\^^d0=\active\fi
\ifcat\noexpand~\noexpand^^d1\relax\else\catcode`\^^d1=\active\fi
\ifcat\noexpand~\noexpand^^d3\relax\else\catcode`\^^d3=\active\fi
\ifcat\noexpand~\noexpand^^d4\relax\else\catcode`\^^d4=\active\fi
\ifcat\noexpand~\noexpand^^d5\relax\else\catcode`\^^d5=\active\fi
\ifcat\noexpand~\noexpand^^d6\relax\else\catcode`\^^d6=\active\fi
\ifcat\noexpand~\noexpand^^d7\relax\else\catcode`\^^d7=\active\fi
\ifcat\noexpand~\noexpand^^d8\relax\else\catcode`\^^d8=\active\fi
\ifcat\noexpand~\noexpand^^d9\relax\else\catcode`\^^d9=\active\fi
\ifcat\noexpand~\noexpand^^a2\relax\else\catcode`\^^a2=\active\fi
\ifcat\noexpand~\noexpand^^b6\relax\else\catcode`\^^b6=\active\fi
\ifcat\noexpand~\noexpand^^b8\relax\else\catcode`\^^b8=\active\fi
\ifcat\noexpand~\noexpand^^b9\relax\else\catcode`\^^b9=\active\fi
\ifcat\noexpand~\noexpand^^ba\relax\else\catcode`\^^ba=\active\fi
\ifcat\noexpand~\noexpand^^bc\relax\else\catcode`\^^bc=\active\fi
\ifcat\noexpand~\noexpand^^be\relax\else\catcode`\^^be=\active\fi
\ifcat\noexpand~\noexpand^^bf\relax\else\catcode`\^^bf=\active\fi
\ifcat\noexpand~\noexpand^^c0\relax\else\catcode`\^^c0=\active\fi
\ifcat\noexpand~\noexpand^^da\relax\else\catcode`\^^da=\active\fi
\ifcat\noexpand~\noexpand^^db\relax\else\catcode`\^^db=\active\fi
\ifcat\noexpand~\noexpand^^dc\relax\else\catcode`\^^dc=\active\fi
\ifcat\noexpand~\noexpand^^dd\relax\else\catcode`\^^dd=\active\fi
\ifcat\noexpand~\noexpand^^de\relax\else\catcode`\^^de=\active\fi
\ifcat\noexpand~\noexpand^^df\relax\else\catcode`\^^df=\active\fi
\ifcat\noexpand~\noexpand^^e0\relax\else\catcode`\^^e0=\active\fi
\ifcat\noexpand~\noexpand^^fa\relax\else\catcode`\^^fa=\active\fi
\ifcat\noexpand~\noexpand^^fb\relax\else\catcode`\^^fb=\active\fi
\ifcat\noexpand~\noexpand^^fc\relax\else\catcode`\^^fc=\active\fi
\ifcat\noexpand~\noexpand^^fd\relax\else\catcode`\^^fd=\active\fi
\ifcat\noexpand~\noexpand^^fe\relax\else\catcode`\^^fe=\active\fi
\ifcat\noexpand~\noexpand^^ab\relax\else\catcode`\^^ab=\active\fi
\ifcat\noexpand~\noexpand^^bb\relax\else\catcode`\^^bb=\active\fi
%    \end{macrocode}
% Now we define a command that defines all Greek characters as commands, which when
% expanded produce the expected command.
%    \begin{macrocode}
\def\el@math{%
    \def^^e1{\relax\ifmmode\alpha\else\textgreek{a}\fi}%
    \def^^c1{\relax\ifmmode{A}\else\textgreek{A}\fi}%
    \def^^dc{\textgreek{'a}}%
    \def^^a2{\textgreek{'A}}%
    \def^^b6{\textgreek{'A}}%
    \def^^e2{\relax\ifmmode\beta\else\textgreek{b}\fi}%
    \def^^c2{\relax\ifmmode{B}\else\textgreek{B}\fi}%
    \def^^e3{\relax\ifmmode\gamma\else\textgreek{g}\fi}%
    \def^^c3{\relax\ifmmode\Gamma\else\textgreek{G}\fi}%
    \def^^e4{\relax\ifmmode\delta\else\textgreek{d}\fi}%
    \def^^c4{\relax\ifmmode\Delta\else\textgreek{D}\fi}%
    \def^^e5{\relax\ifmmode\epsilon\else\textgreek{e}\fi}%
    \def^^c5{\relax\ifmmode{E}\else\textgreek{E}\fi}%
    \def^^dd{\textgreek{'e}}%
    \def^^b8{\textgreek{'E}}%
    \def^^e6{\relax\ifmmode\zeta\else\textgreek{z}\fi}%
    \def^^c6{\relax\ifmmode{Z}\else\textgreek{Z}\fi}%
    \def^^e7{\relax\ifmmode\eta\else\textgreek{h}\fi}%
    \def^^c7{\relax\ifmmode{H}\else\textgreek{H}\fi}%
    \def^^de{\textgreek{'h}}%
    \def^^b9{\textgreek{'H}}%
    \def^^e8{\relax\ifmmode\theta\else\textgreek{j}\fi}%
    \def^^c8{\relax\ifmmode{\Theta}\else\textgreek{J}\fi}%
    \def^^e9{\relax\ifmmode\iota\else\textgreek{i}\fi}%
    \def^^c9{\relax\ifmmode{I}\else\textgreek{I}\fi}%
    \def^^df{\textgreek{'i}}%
    \def^^fa{\textgreek{"i}} %
    \def^^c0{\textgreek{"'i}} %
    \def^^ba{\textgreek{'I}}%
    \def^^da{\textgreek{"I}}%
    \def^^ea{\relax\ifmmode\kappa\else\textgreek{k}\fi}%
    \def^^ca{\relax\ifmmode{K}\else\textgreek{K}\fi}%
    \def^^eb{\relax\ifmmode\lambda\else\textgreek{l}\fi}%
    \def^^cb{\relax\ifmmode\Lambda\else\textgreek{L}\fi}%
    \def^^ec{\relax\ifmmode\mu\else\textgreek{m}\fi}%
    \def^^cc{\relax\ifmmode{M}\else\textgreek{M}\fi}%
    \def^^ed{\relax\ifmmode\nu\else\textgreek{n}\fi}%
    \def^^cd{\relax\ifmmode{N}\else\textgreek{N}\fi}%
    \def^^ee{\relax\ifmmode\xi\else\textgreek{x}\fi}%
    \def^^ce{\relax\ifmmode\Xi\else\textgreek{X}\fi}%
    \def^^ef{\relax\ifmmode{o}\else\textgreek{o}\fi}%
    \def^^cf{\relax\ifmmode{O}\else\textgreek{O}\fi}%
    \def^^fc{\textgreek{'o}}%
    \def^^bc{\textgreek{'O}}%
    \def^^f0{\relax\ifmmode\pi\else\textgreek{p}\fi}%
    \def^^d0{\relax\ifmmode\Pi\else\textgreek{P}\fi}%
    \def^^f1{\relax\ifmmode\rho\else\textgreek{r}\fi}%
    \def^^d1{\relax\ifmmode{P}\else\textgreek{R}\fi}%
    \def^^f2{\relax\ifmmode\varsigma\else\textgreek{c}\fi}%
    \def^^f3{\relax\ifmmode\sigma\else\textgreek{sv}\fi}%
    \def^^d3{\relax\ifmmode\Sigma\else\textgreek{S}\fi}%
    \def^^f4{\relax\ifmmode\tau\else\textgreek{t}\fi}%
    \def^^d4{\relax\ifmmode{T}\else\textgreek{T}\fi}%
    \def^^f5{\relax\ifmmode\upsilon\else\textgreek{u}\fi}%
    \def^^d5{\relax\ifmmode{Y}\else\textgreek{U}\fi}%
    \def^^fd{\textgreek{'u}}%
    \def^^fb{\textgreek{"u}} %
    \def^^e0{\textgreek{"'u}} %
    \def^^be{\textgreek{'U}}%
    \def^^f6{\relax\ifmmode\phi\else\textgreek{f}\fi}%
    \def^^d6{\relax\ifmmode{\Phi}\else\textgreek{F}\fi}%
    \def^^f7{\relax\ifmmode\chi\else\textgreek{q}\fi}%
    \def^^d7{\relax\ifmmode{X}\else\textgreek{Q}\fi}%
    \def^^f8{\relax\ifmmode\psi\else\textgreek{y}\fi}%
    \def^^d8{\relax\ifmmode{\Psi}\else\textgreek{Y}\fi}%
    \def^^f9{\relax\ifmmode\omega\else\textgreek{w}\fi}%
    \def^^d9{\relax\ifmmode{\Omega}\else\textgreek{W}\fi}%
    \def^^fe{\textgreek{'w}}%
    \def^^bf{\textgreek{'W}}%
}%
%    \end{macrocode}
% Now, we need to invoke the command |\el@math| every time we enter a math mode. For
% this reason we redefine the tokens |\everymath| and |\everydisplay| as follows:
%    \begin{macrocode}
\everymath\expandafter{\the\everymath
   \relax\el@math}
\everydisplay\expandafter{\the\everydisplay
   \relax\el@math}
%</elmath>
%    \end{macrocode}
% \Finale
% That's all!
