% -*- coding: iso-latin-1; time-stamp-format: "%02d-%02m-%:y at %02H:%02M:%02S %Z" -*-
%<*drv>
\def\bnedocdate {2014/09/22} % package bnumexpr documentation date
\def\bnepackdate{2014/09/22} % package bnumexpr date
\def\bneversion {1.1a}       % package bnumexpr version
%</drv>
%<*dtx>
\def\bnedtxtimestamp  {Time-stamp: <22-09-2014 at 23:21:33 CEST>}
\iffalse 
%</dtx>
%%----------------------------------------------------------------
%% The bnumexpr package: Expressions with big integers
%% Copyright (C) 2014 by Jean-Francois Burnol 
%%----------------------------------------------------------------
%<*readme>--------------------------------------------------------
Source:  bnumexpr.dtx
Version: v1.1a, 2014/09/22
Author:  Jean-Francois Burnol
Info:    Expressions with big integers
License: LPPL 1.3c or later

This README file: Usage, Installation, License

Usage
=====

\usepackage{bnumexpr}

Then \thebnumexpr <expression with +,-,*,/,(,)> \relax is like 
     \the\numexpr <expression with +,-,*,/,(,)> \relax
with the difference of accepting or producing arbitrarily big
integers.

Example:
     \thebnumexpr 30*(21-43*(512-67*(6133-812*2897)))\relax
outputs:
     -202785405180
which would create an arithmetic overflow in \numexpr.

\bnumexpr...\relax is a scaled down version of \xintiiexpr...\relax
from package xintexpr. 

By default, bnumexpr.sty loads xint.sty for its arithmetic macros
doing addition, subtraction, multiplication, division. 

- with option custom, xint.sty is not loaded and it is up to the
user to define \bnumexprAdd, \bnumexprSub, \bnumexprMul, \bnumexprDiv
 
- option bigintcalc loads the package of the same name and uses
its arithmetic macros,

- option l3bigint similarly with package l3bigint, which is
downloadable from the development repository of the on-going
LaTeX3 project.

Option allowpower enables ^ as power operator (only for xint and
bigintcalc currently).

Installation
============

Obtain bnumexpr.dtx (and possibly, bnumexpr.ins and the README) 
from CTAN: 
                 http://www.ctan.org/pkg/bnumexpr

To generate files from the source bnumexpr.dtx:

 - with bnumexpr.ins: "tex bnumexpr.ins" in the same repertory as
   bnumexpr.dtx will create (or overwrite) the files in this repertory.

 - without bnumexpr.ins: "tex bnumexpr.dtx" also extracts the files.

 * bnumexpr.sty is the style file

 * bnumexpr.readme reconstitutes this README.

 * bnumexpr.changes lists changes since last version.

 * bnumexpr.tex is used for generating the documentation: 

 - with latex+dvipdfmx: 
   "latex bnumexpr.tex" (thrice) then "dvipdfmx bnumexpr.dvi"
   Ignore dvipdfmx warnings, but if the pdf file has problems with
   fonts (possibly from an old dvipdfmx), use then rather pdflatex.

 - with pdflatex: 
   set the suitable toggle in bnumexpr.tex to disable dvipdfmx
   settings and compile it with pdflatex (thrice).

 * without bnumexpr.tex: 

 pdflatex bnumexpr.dtx (thrice) generates simultaneously the style
 file and the pdf documentation.

Finishing the installation:

           bnumexpr.sty   --> TDS:tex/latex/bnumexpr/ 

           bnumexpr.dtx   --> TDS:source/latex/bnumexpr/
           bnumexpr.ins   --> TDS:source/latex/bnumexpr/

           bnumexpr.pdf   --> TDS:doc/latex/bnumexpr/
                 README   --> TDS:doc/latex/bnumexpr/

Files bnumexpr.tex, bnumexpr.changes, bnumexpr.readme may be discarded.

License
=======

Copyright (C) 2014 by Jean-Francois Burnol (jfbu at free dot fr)

This Work may be distributed and/or modified under the
conditions of the LaTeX Project Public License, either
version 1.3c of this license or (at your option) any later
version. This version of this license is in
   http://www.latex-project.org/lppl/lppl-1-3c.txt
and the latest version of this license is in
   http://www.latex-project.org/lppl.txt
and version 1.3 or later is part of all distributions of
LaTeX version 2005/12/01 or later.

This Work has the LPPL maintenance status "maintained".

The Current Maintainer of this Work is Jean-Francois Burnol.

This Work consists of the main source file bnumexpr.dtx
and the derived files
   bnumexpr.sty, bnumexpr.pdf, bnumexpr.ins, bnumexpr.tex, 
   bnumexpr.changes, bnumexpr.readme

End of README file.
%</readme>--------------------------------------------------------
%<*changes>-------------------------------------------------------
%% This is a generated file.
%% 
%% For distribution see the terms in the source file bnumexpr.dtx.
%%
\item[1.1a (2014/09/22)] \begin{itemize}
  \item added |l3bigint| option to use experimental \LaTeX3
    package of the same name.

  \item added Changes and Readme sections to the documentation.

  \item better |\BNE_protect| mechanism for use of
  |\bnumexpr...\relax| inside an |\edef| (without |\bnethe|). Previous one,
  inherited from |xintexpr.sty 1.09n|, assumed that the |\.=<digits>| dummy
  control sequence encapsulating the computation result had |\relax|
  meaning. But removing this assumption was only a matter of letting
  |\BNE_protect| protect two, not one, tokens. This will be backported to
  next version of |xintexpr.sty|, naturally.
  \end{itemize}
\item[1.1 (2014/09/21)] First release. This is down-scaled from the
  (development version of) |xintexpr.sty|. Motivation came the previous day
  from a chat with \textsc{Joseph Wright} over big int status in \LaTeX3.
  The |\bnumexpr...\relax| parser can be used on top of big int macros of
  one's choice. Functionalities limited to the basic operations. I leave
  the power operator |^| as an option.
%</changes>-------------------------------------------------------
%<*drv>-----------------------------------------------------------
%% This is a generated file. 
%%
%% latex bnumexpr.tex (thrice) && dvipdfmx bnumexpr.dvi --> bnumexpr.dtx
%% for pdflatex, set the \Withdvipdfmx toggle below to 0
%%
%% For distribution see the terms in the source file bnumexpr.dtx.
%%
\NeedsTeXFormat{LaTeX2e}
\ProvidesFile{bnumexpr.tex}%
[\bnepackdate\space v\bneversion\space driver file for %
  bnumexpr documentation (jfB)]%
\PassOptionsToClass{a4paper,fontsize=11pt}{scrdoc}
\chardef\Withdvipdfmx 1 % replace 1 by 0 for using pdflatex
\chardef\NoSourceCode 0 % replace 0 by 1 for not including source code
\input bnumexpr.dtx
%%% Local Variables:
%%% mode: latex
%%% End:
%</drv>-----------------------------------------------------------
%<*ins>-----------------------------------------------------------
%% This is a generated file.
%% 
%% tex bnumexpr.ins will extract bnumexpr.sty from bnumexpr.dtx
%%
%% For distribution see the terms in the source file bnumexpr.dtx.
%%
\input docstrip.tex
\askforoverwritefalse
\generate{\nopreamble\nopostamble
\file{bnumexpr.changes}{\from{bnumexpr.dtx}{changes}}
\file{bnumexpr.readme}{\from{bnumexpr.dtx}{readme}}
\usepostamble\defaultpostamble
\file{bnumexpr.tex}{\from{bnumexpr.dtx}{drv}}
\usepreamble\defaultpreamble
\file{bnumexpr.sty}{\from{bnumexpr.dtx}{package}}}
\catcode32=13\relax% active space
\let =\space%
\Msg{************************************************************************}
\Msg{*}
\Msg{* To finish the installation you have to move the following}
\Msg{* files into a directory searched by TeX:}
\Msg{*}
\Msg{*     bnumexpr.sty} 
\Msg{*}
\Msg{* To produce the documentation run latex thrice on bnumexpr.tex}
\Msg{* then dvipdfmx on bnumexpr.dvi. (ignore the dvipdfmx warnings)}
\Msg{*}
\Msg{* Happy TeXing!}
\Msg{*}
\Msg{************************************************************************}
\endbatchfile
%</ins>-----------------------------------------------------------
%<*dtx>
\fi % end of \iffalse block around generated files
\chardef\noetex 0
\expandafter\ifx\csname numexpr\endcsname\relax \chardef\noetex 1 \fi
\ifnum\noetex=1 \chardef\extractfiles 0 % extract files, then stop
\else
    \expandafter\ifx\csname ProvidesFile\endcsname\relax
      \chardef\extractfiles 0 % etex etc.. on bnumexpr.dtx
    \else % latex/pdflatex on bnumexpr.tex or on bnumexpr.dtx
      \expandafter\ifx\csname Withdvipdfmx\endcsname\relax
        % latex run is on bnumexpr.dtx, we will extract all files
        \chardef\extractfiles 1 % 1 = extract and typeset, 2=only typeset
        \chardef\Withdvipdfmx 0 % 1 = use dvipdfmx, 0 = use pdflatex
        \chardef\NoSourceCode 0 % 1 = do not include source code
        \NeedsTeXFormat{LaTeX2e}%
        \PassOptionsToClass{a4paper,fontsize=11pt}{scrdoc}% 
      \else % latex run is on bnumexpr.tex, 
        \chardef\extractfiles 2 % no extractions
      \fi
      \ProvidesFile{bnumexpr.dtx}[bundle source (\bnedtxtimestamp)]%
    \fi
\fi
\ifnum\extractfiles<2 % extract files
\def\MessageDeFin{\newlinechar10 \let\Msg\message
\Msg{^^J}%
\Msg{********************************************************************^^J}%
\Msg{*^^J}%
\Msg{* To finish the installation you have to move the following^^J}%
\Msg{* files into a directory searched by TeX:^^J}%
\Msg{*^^J}%
\Msg{*\space\space\space\space bnumexpr.sty^^J}%
\Msg{*^^J}%
\Msg{* To produce the documentation run latex thrice on bnumexpr.tex^^J}%
\Msg{* then dvipdfmx on bnumexpr.dvi. (ignore the dvipdfmx warnings)^^J}%
\Msg{*^^J}%
\Msg{* Happy TeXing!^^J}%
\Msg{*^^J}%
\Msg{********************************************************************^^J}%
}%
\begingroup
    \input docstrip.tex
    \askforoverwritefalse
    \generate{\nopreamble\nopostamble
    \file{bnumexpr.changes}{\from{bnumexpr.dtx}{changes}}
    \file{bnumexpr.readme}{\from{bnumexpr.dtx}{readme}}
    \usepostamble\defaultpostamble
    \file{bnumexpr.ins}{\from{bnumexpr.dtx}{ins}}
    \file{bnumexpr.tex}{\from{bnumexpr.dtx}{drv}}
    \usepreamble\defaultpreamble
    \file{bnumexpr.sty}{\from{bnumexpr.dtx}{package}}}
\endgroup
\fi % end of file extraction
\ifnum\extractfiles=0 
% direct tex/etex/xetex on bnumexpr.dtx, files now extracted, stop
  \MessageDeFin\expandafter\end
\fi
% no use of docstrip to extract files if latex compilation was on bnumexpr.tex
\ifdefined\MessageDeFin\AtEndDocument{\MessageDeFin}\fi
%-------------------------------------------------------------------------------
\documentclass {scrdoc}
\ifnum\NoSourceCode=1 \OnlyDescription\fi
\makeatletter
\ifnum\Withdvipdfmx=1
   \@for\@tempa:=hyperref,bookmark,graphicx,xcolor,pict2e\do
            {\PassOptionsToPackage{dvipdfmx}\@tempa}
   %
   \PassOptionsToPackage{dvipdfm}{geometry}
   \PassOptionsToPackage{bookmarks=true}{hyperref}
   \PassOptionsToPackage{dvipdfmx-outline-open}{hyperref}
   \PassOptionsToPackage{dvipdfmx-outline-open}{bookmark}
   %
   \def\pgfsysdriver{pgfsys-dvipdfm.def}
\else
   \PassOptionsToPackage{bookmarks=true}{hyperref}
\fi
\makeatother

\pagestyle{headings}
\makeatletter
\def\buggysectionmark #1{% KOMA 3.12 as released to CTAN December 2013
    \if@twoside\expandafter\markboth\else\expandafter\markright\fi
    {\MakeMarkcase{\ifnumbered{section}{\sectionmarkformat\fi}{}#1}}{}}
\ifx\buggysectionmark\sectionmark
\def\sectionmark #1{%
    \if@twoside\expandafter\markboth\else\expandafter\markright\fi
    {\MakeMarkcase{\ifnumbered{section}{\sectionmarkformat}{}#1}}{}}
\fi
\makeatother

\usepackage[T1]{fontenc}
\usepackage[latin1]{inputenc}

\usepackage[hscale=0.66,vscale=0.75]{geometry}

\usepackage[zerostyle=a,scaled=0.95]{newtxtt}
\renewcommand\familydefault\ttdefault
\usepackage[noendash]{mathastext}
\renewcommand\familydefault\sfdefault

\usepackage{graphicx}
\usepackage[dvipsnames]{xcolor}
\definecolor{joli}{RGB}{225,95,0}
\definecolor{JOLI}{RGB}{225,95,0}
\definecolor{BLUE}{RGB}{0,0,255}
\definecolor{niceone}{RGB}{38,128,192}
\colorlet{jfverbcolor}{yellow!5}

\usepackage[english]{babel}

\usepackage[pdfencoding=pdfdoc]{hyperref}
\hypersetup{%
%linktoc=all,%
breaklinks=true,%
colorlinks=true,%
urlcolor=niceone,%
linkcolor=blue,%
pdfauthor={Jean-Fran\c cois Burnol},%
pdftitle={The bnumexpr package},%
pdfsubject={Arithmetic with TeX},%
pdfkeywords={Expansion, arithmetic, TeX},%
pdfstartview=FitH,%
pdfpagemode=UseOutlines}
\usepackage{bookmark} 

%---- \verb, and verbatim like `environments'. \MicroFont et \MacroFont
\def\MicroFont {\ttfamily }
\def\MacroFont {\ttfamily\baselineskip12pt\relax}
\makeatletter

\def\lowast{\raisebox{-.25\height}{*}}
\begingroup
   \catcode`* 13
   \gdef\makestarlowast {\let*\lowast\catcode`\*\active}%
\endgroup

% modif de \do@noligs: \char`#1} --> \char`#1 } 
\def\do@noligs #1%
{%
    \catcode `#1\active 
    \begingroup \lccode `\~=`#1\relax 
    \lowercase {\endgroup \def ~{\leavevmode \kern \z@ \char `#1 }}%
}% 
% Tentative, Mardi 09 septembre 2014 � 22:41:28
\def\verb 
% on pourrait mettre le #1 ici et �conomiser \@@jfverb
{%
  \relax\leavevmode\null
  \begingroup\MicroFont
  \let\do\do@noligs  \verbatim@nolig@list % voir plus tard si vraiment
                                % n�cessaire maintenant
  \let\do\@makeother \dospecials 
                \makestarlowast
  \@vobeyspaces \fboxsep0pt
  \@@jfverb 
}% 

\def\@@jfverb #1{\catcode`#1 3 \@@@jfverb }

\def\@@@jfverb #1{\ifcat\noexpand#1$% $
                      \endgroup\else 
      \penalty\z@
      \colorbox{jfverbcolor}{\strut #1}%
      \expandafter\@@@jfverb\fi }

\makeatother
\catcode`\_=11

\def\csa_aux #1{\ttfamily\hyphenchar\font45 \char`\\%
                \scantokens{#1}\endgroup }

\DeclareRobustCommand\csa {\begingroup\catcode`\_=11
                           \everyeof{\noexpand}\endlinechar -1
                           \makeatother
                           \makestarlowast
                           \csa_aux }
\newcommand\csh[1]{\texorpdfstring{\csa{#1}}{\textbackslash #1}}
\catcode`\_=8

\usepackage{xspace}

\def\bnename
   {\texorpdfstring
          {\hyperref[sec:bnumexpr]%
             {{\color{joli}\bfseries\ttfamily bnumexpr}}}
          {bnumexpr}%
     \xspace }%
\def\bnenameimp
   {\texorpdfstring
              {\hyperref[sec:bnumexprcode]%
                {{\color[named]{RoyalPurple}%
                 \bfseries\ttfamily bnumexpr}}}
              {bnumexpr}%
    \xspace }%


\frenchspacing
% possible options: custom, bigintcalc, l3bigint, nocsv, notacitmul, allowpower
\usepackage[allowpower]{bnumexpr}

\usepackage{etoc}

\begin{document}
\thispagestyle{empty}
\ttzfamily
\pdfbookmark[1]{Title page}{TOP}

{%
\normalfont\Large\parindent0pt \parfillskip 0pt\relax 
 \leftskip 2cm plus 1fil \rightskip 2cm plus 1fil
 The \bnename package\par
}

{\centering
  \textsc{Jean-Fran�ois Burnol}\par
  \footnotesize 
  jfbu (at) free (dot) fr\par
  Package version: \bneversion\ (\bnepackdate); 
            documentation date: \bnedocdate.\par
  {From source file \texttt{bnumexpr.dtx}. \bnedtxtimestamp.}\par
}

\etocsetnexttocdepth{section}
\tableofcontents

\section{Readme}
\begingroup
\makeatletter\def\x{\baselineskip10pt
                    \ttfamily\settowidth\dimen@{X}%
                    %\parindent \dimexpr.5\linewidth-33\dimen@\relax
                    \let\do\do@noligs\verbatim@nolig@list
                    \let\do\@makeother\dospecials
                    \def\par{\leavevmode \null\@@par\penalty\interlinepenalty}%
                    \makestarlowast
                    \@vobeyspaces\obeylines
                    \noindent\kern\parindent\input bnumexpr.readme
\endgroup }\x


\section{Introduction}
\label{sec:bnumexpr}

Package \bnename provides |\bnumexpr...\relax| which is analogous to
|\numexpr...\relax|, while allowing arbitrarily big
integers. Important items:
\begin{enumerate}
\item the |\relax| token ending the expression is mandatory,
\item one must use either |\thebnumexpr| or |\bnethe\bnumexpr| to get a
  printable result, as |\bnumexpr...\relax| expands to a private format;
  however one may embed directly one |\bnumexpr...\relax| in another
  |\bnumexpr...\relax|,
\item one may do |\edef\tmp{\bnumexpr 1+2\relax}|, and then either use |\tmp| in
  another |\bnumexpr...\relax|, or print it via |\bnethe\tmp|. The computation
  is done at the time of the |\edef| (and two expansion steps suffice),
\item tacit multiplication applies in front of parenthesized sub-expressions, or
  sub |\bnumexpr...\relax| (or |\numexpr...\relax|), or in front of a |\count|
  or |\dimen| register. This may be de-activated by option |notacitmul|,
\item expressions may be comma separated. On input, spaces are ignored,
  naturally, and on output the values are comma separated with a space after
  each comma. This functionality may be turned off via option |nocsv|,
\item even with options |notacitmul| and |nocsv| the syntax is more flexible
  than with |\numexpr|: things such as 
  |\bnumexpr -(1+1)\relax| are legal.
\end{enumerate}

The parser |\bnumexpr| is a scaled-down version of parser |\xintiiexpr| from
package \href{http://www.ctan.org/pkg/xint}{xintexpr}: support for boolean
operators, functions such as |abs|, |max|, |lcm|, the |!| as factorial,
handling of hexadecimal numbers, etc\dots has
been removed. The goal here is to extend |\numexpr| only to the extent of
accepting big integers. Thus by default, the syntax allows |+,-,*,/|,
parentheses, and also |\count| or |\dimen| registers or variables. Option
|allowpower| enables |^| as power operator.

Of course, one
needs some underlying big integer engine to provide the macros doing the
actual computations. By default, \bnename uses package |xint| and its
|\xintiiAdd|, |\xintiiSub|, and |\xintiiMul| macros (also |\xintiiPow| if
option |allowpower| is made use of). As we want here |/| to do
rounded division while |xint|'s |\xintiiQuo| does Euclidean division,
\bnename contains a few extra code lines on top of the underlying
division macros from |xint.sty|.

See the discussion of options |bigintcalc|, |l3bigint| and |custom| in
\autoref{sec:options} for alternatives.

The starting point for the |\bnumexpr| parser was not the |\xintiiexpr| version
|1.09n| as available (at the time of writing) on CTAN, but a development
version for future release |1.1|. This is why the version number of package
|bnumexpr| is |1.1|.
It may well be that the code of the parser is in some places quite sub-optimal
from the fact that it was derived from code handling much more stuff.

The |\xintNewExpr| construct has been left out.

I recall from documentation of |xintexpr| that there is a potential impact on
the memory of \TeX{} (the hash table) because each arithmetic operation is
done inside a dummy |\csname...\endcsname| used as single token to move
around in one-go the possibly hundreds of digits composing a number.

\section{Options}\label{sec:options}

The package does by default:
\begin{verbatim}
\RequirePackage{xint}
\let\bnumexprAdd\xintiiAdd
\let\bnumexprSub\xintiiSub
\let\bnumexprMul\xintiiMul
%% \bnumexprDiv has custom definition on top of macros from xint.sty
\let\bnumexprPow\xintiiPow % only if option allowpower
\end{verbatim}
% To let |/| do euclidean division (like currently in |\xintiiexpr|) it is
% thus sufficient to do |\let\bnumexprDiv\xintiiQuo| after loading the package.

Option |bigintcalc| says to not load \href{http://www.ctan.org/pkg/xint}{xint}
but to use rather the macros from package
\href{http://www.ctan.org/pkg/bigintcalc}{bigintcalc} 
by \textsc{Heiko Oberdiek}. Note though that |/| is mapped to |\bigintcalcDiv|
which does \emph{truncated} (not rounded) division.

Option |l3bigint| similarly says to use the macros which are provided with the
eponym package, a part of the development work of the
\href{http://latex-project.org/code.html}{\LaTeX3 project}. There is no power
operation available with this option.

Option |custom| does not load any package and leaves it up to the user to
specify the macros to be used, i.e. provide definitions for |\bnumexprAdd|,
|\bnumexprSub|, |\bnumexprMul|, |\bnumexprDiv| (and possibly |\bnumexprPow|).

If using option |custom|: the four arithmetic macros |\bnumexprAdd|,
|\bnumexprSub|, |\bnumexprMul|, |\bnumexprDiv| (and possibly |\bnumexprPow|)
must be expandable, and they must allow arguments in need to be first (`f'-)
expanded. They should produce on output (big) integers with no leading zeros, at
most one minus sign and no plus sign (else the \bnename macro used for
handling the |-| prefix operator may need to be modified). They will be
expanded inside |\csname...\endcsname|. The macros from |xint.sty| (as well as
those of |bigintcalc.sty|) are expandable in a stronger sense (only two
expansion steps suffice). Perhaps speed gains are achievable from dropping
these stronger requirements.

Option |nocsv| makes comma separated expressions illegal. 

Option |notacitmul| removes the possibility of tacit multiplication in front of
parentheses, \string\count\space registers, sub-expressions.

Option |allowpower| enables the |^| as power operator (left associative).

\section{Further commands}

The package provides |\bnumexprUsesxint|, |\bnumexprUsesbigintcalc| and
|\bnumexprUsesliiibigint| which allow to use in the same document more than one
of the available big integer mathematical engines. 

It is up to the user to have issued the necessary |\usepackage| or
|\RequirePackage| in the preamble, and this may be done after having loaded
|bnumexpr.sty|. Recall that |xint| is loaded by default, but is not loaded in
case of one of the options |custom|, |bigintcalc|, |l3bigint|.


\section{Examples}

\begin{flushleft}\parindent0pt \obeylines
|\thebnumexpr 128637867168*2187917891279\relax| %
\thebnumexpr 128637867168*2187917891279\relax

|\thebnumexpr 30*(21-43*(512-67*(6133-812*2897)))\relax| %
\thebnumexpr 30*(21-43*(512-67*(6133-812*2897)))\relax

\newcount\cnta\cnta 123
\newcount\cntb \cntb 188

|\newcount\cnta \cnta 123 \newcount\cntb \cntb 188|
% |\the\numexpr \cnta*\cnta*\cnta\relax|  %
% \the\numexpr \cnta*\cnta*\cnta\relax {}
% |\the\bnumexpr \cnta*\cnta*\cnta\relax| %
% \thebnumexpr \cnta*\cnta*\cnta\relax {}
|\the\numexpr \cnta*\cnta*\cnta/\cntb+\cntb*\cntb*\cntb/\cnta\relax| %
\the\numexpr \cnta*\cnta*\cnta/\cntb+\cntb*\cntb*\cntb/\cnta\relax {}
|\thebnumexpr \cnta*\cnta*\cnta/\cntb+\cntb*\cntb*\cntb/\cnta\relax|  %
\thebnumexpr \cnta*\cnta*\cnta/\cntb+\cntb*\cntb*\cntb/\cnta\relax   {}

% |\the\numexpr \cnta*\cnta*(\cnta/\cntb)+\cntb*\cntb*(\cntb/\cnta)\relax| 
% \the\numexpr \cnta*\cnta*(\cnta/\cntb)+\cntb*\cntb*(\cntb/\cnta)\relax  {}
% |\thebnumexpr \cnta*\cnta*(\cnta/\cntb)+\cntb*\cntb*(\cntb/\cnta)\relax| 
% \thebnumexpr \cnta*\cnta*(\cnta/\cntb)+\cntb*\cntb*(\cntb/\cnta)\relax {}
|\the\numexpr 123/188*188\relax|, |\the\numexpr 123/(188*188)\relax|, %
|\thebnumexpr 123/188*188\relax|, |\thebnumexpr 123/(188*188)\relax|. 
\the\numexpr 123/188*188\relax, \the\numexpr 123/(188*188)\relax, %
\thebnumexpr 123/188*188\relax, \thebnumexpr 123/(188*188)\relax.

\edef\tmp {\bnumexpr 121873197*123-218137917*188\relax}
|\edef\tmp {\bnumexpr 121873197*123-218137917*188\relax}\bnethe\tmp|
\bnethe\tmp {}
|\meaning\tmp| 
\meaning\tmp {} 
|\thebnumexpr \tmp*(173197129797-\tmp)*(2179171982-\tmp)\relax|
\thebnumexpr \tmp*(173197129797-\tmp)*(2179171982-\tmp)\relax

|\cnta \thebnumexpr 2152966419779999/987654321\relax\space|
|\the\cnta| %
\cnta \thebnumexpr 2152966419779999/987654321\relax\space \the\cnta

|\thebnumexpr 2179878*987654321-2152966419779999, 2179879*987654321-2152966419779999\relax| %
\thebnumexpr  2179878*987654321-2152966419779999,2179879*987654321-2152966419779999\relax\ \ \ (there was indeed rounding of the exact quotient.)

An example with the power operator |^| (option |allowpower|, not compatible with
|l3bigint|): %
|\thebnumexpr (1^10+2^10+3^10+4^10+5^10+6^10)^3\relax| %
\thebnumexpr (1^10+2^10+3^10+4^10+5^10+6^10)^3\relax

\end{flushleft}

\StopEventually{\end{document}\endinput}

\newgeometry{hscale=0.75,vscale=0.75}% ATTENTION \newgeometry fait
                                % un reset de vscale si on ne le
                                % pr�cise pas ici !!!

\MakePercentIgnore
% 
% \catcode`\<=0 \catcode`\>=11 \catcode`\*=11 \catcode`\/=11
% \let</dtx>\relax
% \def<*package>{\catcode`\<=12 \catcode`\>=12 \catcode`\*=12 \catcode`\/=12 }
%</dtx>
%<*package>
%
% \section{Package \bnenameimp implementation}
% \label{sec:bnumexprcode}
%
% \localtableofcontents \bigskip 
%
% Comments are sparse. Error handling by the parser is kept to a minimum; if
% something goes wrong, the offensive token gets discarded, and some undefined
% control sequence attempts to trigger writing to the log of some sort of
% informative message. It is recommended to set |\errorcontextlines| to at
% least |2| for more meaningful context.
% \subsection{Package identification and catcode setup}
%    \begin{macrocode}
\NeedsTeXFormat{LaTeX2e}%
\ProvidesPackage{bnumexpr}[2014/09/22 v1.1a Expressions with big integers (jfB)]%
\edef\BNErestorecatcodes {\catcode`\noexpand\!\the\catcode`\!
                   \catcode`\noexpand\?\the\catcode`\?
                   \catcode`\noexpand\_\the\catcode`\_
                   \catcode`\noexpand\:\the\catcode`\:\relax }%
\catcode`\! 11 % some other catcodes will be manipulated: comma, (, ),
\catcode`\? 11 % but we reset them to their standard values, thus 
\catcode`\_ 11 % \BNErestorecatcodes is a bit pedantic here.
\catcode`\: 11
%    \end{macrocode}
% \subsection{Package options}
%    \begin{macrocode}
\def\BNE_tmpa {0}%
\DeclareOption {custom}{\def\BNE_tmpa {1}%
    \PackageWarningNoLine{bnumexpr}{^^J
  Option custom: package xint not loaded. Definitions are needed for:^^J 
  \protect\bnumexprAdd, \protect\bnumexprSub,
  \protect\bnumexprMul\space and \protect\bnumexprDiv }%
}%
\DeclareOption {bigintcalc}{\def\BNE_tmpa {2}%
    \PackageWarningNoLine{bnumexpr}{^^J
    Option bigintcalc: the macros from package bigintcalc are used.^^J
    Notice that / is mapped to \protect\bigintcalcDiv\space which does truncated division}%
}%
\DeclareOption {l3bigint}{\def\BNE_tmpa {3}%
    \PackageWarningNoLine{bnumexpr}{^^J
    Option l3bigint: the macros from package l3bigint are used.^^J
    There is no power operation, currently}%
}%
\DeclareOption {nocsv}{%
    \PackageInfo{bnumexpr}{Comma separated expressions disabled}%
    \AtEndOfPackage{\expandafter\let\csname BNE_precedence_,\endcsname 
                                    \undefined }%
}%
\DeclareOption {notacitmul}{%
    \PackageInfo{bnumexpr}{Tacit multiplication disabled}%
    \AtEndOfPackage{\BNE_notacitmultiplication}%
}%
\def\BNE_allowpower {0}%
\DeclareOption {allowpower}{%
    \PackageInfo{bnumexpr}{Power operator ^ authorized}%
    \def\BNE_allowpower {1}%
}%
\ProcessOptions\relax
%    \end{macrocode}
% \subsection{Mapping to an underlying big integer engine.}
% In case option |bigintcalc| is used, notice that |/| is mapped to the macro
% |\bigintcalcDiv| which does truncated division. We did not add the extra code
% for rounded division in that case.
% 
% With option |l3bigint|, there is no power operation available currently.
% Furthermore the package is part of the experimental trunk of the \LaTeX3
% project hence the names of its macros could change.
%    \begin{macrocode}
\def\bnumexprUsesxint {%
    \let\bnumexprAdd\xintiiAdd
    \let\bnumexprSub\xintiiSub
    \let\bnumexprMul\xintiiMul
    \let\bnumexprDiv\BNE_xintiiDivRound 
    \let\bnumexprPow\xintiiPow
}%
\def\bnumexprUsesbigintcalc {%
    \let\bnumexprAdd\bigintcalcAdd
    \let\bnumexprSub\bigintcalcSub
    \let\bnumexprMul\bigintcalcMul
    \let\bnumexprDiv\bigintcalcDiv % NOTE: THIS DOES TRUNCATED DIVISION
    \let\bnumexprPow\bigintcalcPow
}%
\def\bnumexprUsesliiibigint {%
    \let\bnumexprAdd\bigint_add:nn
    \let\bnumexprSub\bigint_sub:nn
    \let\bnumexprMul\bigint_mul:nn
    \let\bnumexprDiv\bigint_div_round:nn
    \let\bnumexprPow\bigint_pow:nn % does not exist!
}%
%    \end{macrocode}
% The |\xintiiQuo| macro from |xint.sty| does Euclidean division. Rounded
% division is available from |xintfrac.sty|, but rather than loading it, we
% define directly here |\bnumexprDiv| as a suitable wrapper to the |xint.sty|
% division macros. This, or something similar, should be incorporated in next
% release of |xint|.
%
% Current CTAN version of |xint| (1.09n) has some sub-optimal code for dealing
% with the signs of the divisor and dividend, this has been improved in
% development version 1.1, which we follow here.
%    \begin{macrocode}
\def\BNE_xintiiDivRound      {\romannumeral0\BNE_xintiidivround }%
\def\BNE_xintiidivround    #1{\expandafter\BNE_div \romannumeral-`0#1\Z }%
\def\BNE_div #1#2\Z #3{\expandafter\BNE_div_a\expandafter #1%
                             \romannumeral-`0#3\Z #2\Z }%
\def\BNE_div_a #1#2% #1 de A, #2 de B.
{%
    \if0#2\xint_dothis\BNE_div_divbyzero\fi
    \if0#1\xint_dothis\BNE_div_aiszero\fi
    \if-#2\xint_dothis{\BNE_div_bneg #1}\fi
          \xint_orthat{\BNE_div_bpos #1#2}%
}%
\def\BNE_div_divbyzero #1\Z #2\Z {\BNE:DivisionByZero\space 0}%
\def\BNE_div_aiszero   #1\Z #2\Z { 0}%
\def\BNE_div_bpos #1%
{%
    \xint_UDsignfork
            #1{\xintiiopp\BNE_div_pos {}}%
             -{\BNE_div_pos #1}%
    \krof
}%
\def\BNE_div_bneg #1%
{%
    \xint_UDsignfork
            #1{\BNE_div_pos {}}%
             -{\xintiiopp\BNE_div_pos #1}%
    \krof
}%
\def\BNE_div_pos #1#2\Z #3\Z{\expandafter\BNE_div_pos_a
                          \romannumeral0\XINT_div_prepare {#2}{#1#30}}%
\def\BNE_div_pos_a #1#2{\xintReverseOrder {#1\BNE_div_pos_b}\Z }%
\def\BNE_div_pos_b #1#2{\xint_gob_til_Z #2\BNE_div_pos_small\Z 
                        \BNE_div_pos_c #1#2}%
\def\BNE_div_pos_c #1#2\Z {\ifnum #1>\xint_c_iv 
                              \expandafter\BNE_div_pos_up
                      \else   \expandafter\xintreverseorder
                      \fi {#2}}%
\def\BNE_div_pos_up #1{\xintinc {\xintReverseOrder{#1}}}% 
\def\BNE_div_pos_small\Z\BNE_div_pos_c #1#2{\ifnum #1>\xint_c_iv\expandafter
                                  \xint_secondoftwo\else\expandafter
                                  \xint_firstoftwo\fi { 0}{ 1}}%
\if0\BNE_tmpa % Toggle to load xint.sty (and also xinttools.sty)
    \RequirePackage{xint}%
    \bnumexprUsesxint
\fi
\if2\BNE_tmpa % Toggle to load bigintcalc.sty
    \RequirePackage{bigintcalc}%
    \bnumexprUsesbigintcalc
\fi
\if3\BNE_tmpa % Toggle to load l3bigint.sty
    \RequirePackage{l3bigint}%
    \bnumexprUsesliiibigint
\fi
%    \end{macrocode}
% \subsection{Some helper macros and constants from xint}
% These macros from xint should not change, hence overwriting them here should
% not be cause for alarm. I opted against renaming everything with |\BNE_|
% prefix rather than |\xint_|. The |\xint_dothis|/|\xint_orthat| thing is a new
% style I have adopted for expandably forking. The least probable branches
% should be specified first, for better efficiency. See examples of uses in the
% present code.
%    \begin{macrocode}
\chardef\xint_c_     0
\chardef\xint_c_i    1
\chardef\xint_c_ii   2
% \chardef\xint_c_iii  3
% \chardef\xint_c_iv   4
% \chardef\xint_c_v    5
\chardef\xint_c_vi   6
\chardef\xint_c_vii  7
\chardef\xint_c_viii 8
\chardef\xint_c_ix   9
% \chardef\xint_c_x     10
% \chardef\xint_c_xviii 18
\long\def\xint_gobble_i      #1{}%
\long\def\xint_gobble_iii    #1#2#3{}%
\long\def\xint_firstofone    #1{#1}%
\long\def\xint_firstoftwo    #1#2{#1}%
\long\def\xint_secondoftwo   #1#2{#2}%
\long\def\xint_firstofthree  #1#2#3{#1}%
\long\def\xint_secondofthree #1#2#3{#2}%
\long\def\xint_thirdofthree  #1#2#3{#3}%
\def\xint_gob_til_!   #1!{}% this ! has catcode 11
\def\xint_UDsignfork  #1-#2#3\krof {#2}%
\long\def\xint_afterfi       #1#2\fi {\fi #1}%
\long\def\xint_dothis        #1#2\xint_orthat #3{\fi #1}% new in v1.1
\let\xint_orthat             \xint_firstofone
%    \end{macrocode}
% \subsection{Encapsulation of numbers in pseudo cs names}
% We define here a |\BNE_num| to not have to invoke |\xintNum|; hence
% dependency on |xint.sty| is kept to the actual arithmetic operations. We
% only need to get rid of leading zeros as plus and minus signs have already
% been stripped off; generally speaking user input will have no leading zeros
% thus the macro is designed to go fast when it is not needed... and as
% everything happens inside a |\csname...\endcsname|, we can leave some
% trailing |\fi|'s.
%
% Note: the |1.09n| |\xintiiexpr| currently on CTAN has a
% bug related to leading zeros, |\xinttheiiexpr 001+1\relax| does not return
% |2|. This bug is absent from |\xintexpr|, |\xintfloatexpr|, |\xintiexpr| and
% only present in |\xintiiexpr|.
%    \begin{macrocode}
\edef\BNE_lock #1!{\noexpand\expandafter\space\noexpand
                            \csname .=\noexpand\BNE_num #1\endcsname }%
\def\BNE_num #1{\if #10\expandafter\BNE_num\else
                \ifcat #1\relax 0\expandafter\expandafter\expandafter #1\else
                #1\fi\fi }%
\def\BNE_unlock   {\expandafter\BNE_unlock_a\string }%
\def\BNE_unlock_a #1.={}%
%    \end{macrocode}
% \subsection{\csh{bnumexpr}, \csh{bnethe}, \csh{thebnumexpr}, \dots}
% In the full |\xintexpr|, the final unlocking may involve post-treatment of
% the comma separated values, hence there are |_print| macros to handle the
% possibly comma separated values. Here we may just identify |_print| with
% |_unlock|.
%    \begin{macrocode}
\def\bnumexpr {\romannumeral0\bnumeval }%
\def\bnumeval {\expandafter\BNE_wrap\romannumeral0\BNE_eval }%
\def\BNE_eval {\expandafter\BNE_until_end_a\romannumeral-`0\BNE_getnext }%
\def\BNE_wrap { !\BNE_usethe\BNE_protect\BNE_unlock }%
\protected\def\BNE_usethe\BNE_protect {\BNE:missing_bnethe!}%
\def\BNE_protect\BNE_unlock {\noexpand\BNE_protect\noexpand\BNE_unlock\noexpand }%
\let\BNE_done\space
\def\thebnumexpr 
              {\romannumeral-`0\expandafter\BNE_unlock\romannumeral0\BNE_eval }%
\def\bnethe #1{\romannumeral-`0\expandafter\xint_gobble_iii\romannumeral-`0#1}%
%    \end{macrocode}
% \subsection{\csh{BNE\_getnext}}
% The getnext scans forward to find a number: after expansion of what comes
% next, an opening parenthesis signals a parenthesized sub-expression, a |!|
% with catcode 11 signals there was there a sub |\bnumexpr...\relax| (now
% evaluated), a minus sign is treated as a prefix operator inheriting its
% precedence level from the previous operator, a plus sign is swallowed, a
% |\count| or |\dimen| will get fetched to |\number| (in case of a count
% variable, this provides a full locked number but |\count0 1| for example is
% like |1231| if |\count0|'s value is |123|); a digit triggers the
% number scanner. After the digit scanner finishes the integer is trimmed of
% leading zeros and locked as a single token into a |\csname .=...\endcsname|.
% The flow then proceeds with |\BNE_getop| which looks for the next operator
% or possibly the end of the expression. Note: |\bnumexpr\relax| is illegal.
%    \begin{macrocode}
\def\BNE_getnext #1%
{%
    \expandafter\BNE_getnext_a\romannumeral-`0#1%
}%
\def\BNE_getnext_a #1%
{%
    \xint_gob_til_! #1\BNE_gn_foundexpr !% this ! has catcode 11
    \ifcat\relax#1% \count or \numexpr etc... token or count, dimen, skip cs
       \expandafter\BNE_gn_countetc
    \else
       \expandafter\expandafter\expandafter\BNE_gn_fork\expandafter\string
    \fi
    #1%
}%
\def\BNE_gn_foundexpr !#1\fi !{\expandafter\BNE_getop\xint_gobble_iii }%
\def\BNE_gn_countetc #1%
{%
    \ifx\count#1\else\ifx#1\dimen\else\ifx#1\numexpr\else\ifx#1\dimexpr\else
    \ifx\skip#1\else\ifx\glueexpr#1\else\ifx\fontdimen#1\else
      \BNE_gn_unpackvar
    \fi\fi\fi\fi\fi\fi\fi
    \expandafter\BNE_getnext\number #1%
}%
\def\BNE_gn_unpackvar\fi\fi\fi\fi\fi\fi\fi\expandafter
                            \BNE_getnext\number #1%
{%
  \fi\fi\fi\fi\fi\fi\fi
  \expandafter\BNE_getop\csname .=\number#1\endcsname 
}%
%    \end{macrocode}
% This is quite simplified here compared to |\xintexpr|, for various reasons: we
% have dropped the |\xintNewExpr| thing, and we can treat the |(| directly as we
% don't have to get back to check if we are in an |\xintexpr|,
% |\xintfloatexpr|, etc.. 
%    \begin{macrocode}
\def\BNE_gn_fork #1{%
    \if#1+\xint_dothis \BNE_getnext\fi
    \if#1-\xint_dothis -\fi
    \if#1(\xint_dothis \BNE_oparen \fi
    \xint_orthat       {\BNE_scan_number #1}%
}%
%    \end{macrocode}
% \subsection{Parsing an integer}
% We gather a string of digits, plus and minus prefixes have already been
% swallowed. There might be some leading string of zeros which will have to be
% removed. In the full |\xintexpr| the situation is more involved as it has to
% recognize and accept decimal numbers, numbers in scientific notation, also
% hexadecimal numbers, function names, etc... and variable names in current
% development version |1.1| (not yet finished).
%    \begin{macrocode}
\def\BNE_scan_number #1% this #1 has necessarily here catcode 12
{%
    \ifnum \xint_c_ix<1#1 \expandafter \BNE_scan_nbr\else
                          \expandafter \BNE_notadigit\fi #1%
}%
\def\BNE_notadigit #1{\BNE:not_a_digit! \xint_gobble_i {#1}}%
%    \end{macrocode}
% Scanning for a number. Once gathered, lock it and do |_getop|. If we hit
% against some catcode eleven |!|, this means there was a sub |\bnumexpr..\relax|.
% We then apply tacit multiplication.
%    \begin{macrocode}
\def\BNE_scan_nbr
{%
    \expandafter\BNE_getop\romannumeral-`0\expandafter
    \BNE_lock\romannumeral-`0\BNE_scan_nbr_c
}%
\def\BNE_scan_nbr_a #1%
{% careful that ! has catcode letter here
    \ifcat \relax #1\xint_dothis{!#1}\fi % stops the scan
    \ifx         !#1\xint_dothis{!*!}\fi % tacit multiplication before subexpr
    \xint_orthat {\expandafter\BNE_scan_nbr_b\string #1}%
}%
\def\BNE_scan_nbr_b #1% #1 with catcode 12
{%
    \ifnum \xint_c_ix<1#1 \expandafter\BNE_scan_nbr_c
    \else\expandafter !\fi #1%
}%
\def\BNE_scan_nbr_c #1#2%
{%
    \expandafter #1\romannumeral-`0\expandafter
                   \BNE_scan_nbr_a\romannumeral-`0#2%
}%
%    \end{macrocode}
%    \begin{macrocode}
%    \end{macrocode}
% \subsection{\csh{BNE\_getop}}
% This finds the next infix operator or closing parenthesis or expression end.
% It then leaves in the token flow <precedence> <operator> <locked number>. The
% <precedence> stops expansion and ultimately gives back
% control to a |\BNE_until_<op>| command. The code here is derived from more
% involved context where the actual macro associated to the operator may vary,
% depending if we are in |\xintexpr|, |\xintfloatexpr| or |\xintiiexpr|. Here
% things are simpler but I have kept the general scheme, thus the actual macro
% to be used for the <operator> is not decided immediately (development version
% of |xintexpr.sty| has extra things to allow multi-characters operators like 
% |&&|).  
%    \begin{macrocode}
\def\BNE_getop #1#2% this #1 is the current locked computed value
{%
    \expandafter\BNE_getop_a\expandafter #1\romannumeral-`0#2%
}%
\catcode`* 11
\def\BNE_getop_a #1#2%
{%  if a control sequence is found, must be \relax, or possibly register or
 %  variable if tacit multiplication is allowed
    \ifx \relax #2\xint_dothis\xint_firstofthree\fi
    % tacit multiplications:
    \ifcat \relax #2\xint_dothis\xint_secondofthree\fi
    \if    (#2\xint_dothis      \xint_secondofthree\fi
    \ifx   !#2\xint_dothis      \xint_secondofthree\fi
    \xint_orthat \xint_thirdofthree
    {\BNE_foundend #1}%
    {\BNE_precedence_* *#1#2}% tacit multiplication
    {\BNE_foundop #2#1}%
}%
\catcode`* 12
\def\BNE_foundend {\xint_c_ \relax }% \relax is only a place-holder here.
\def\BNE_foundop #1% 
{%
    \ifcsname BNE_precedence_#1\endcsname
        \csname BNE_precedence_#1\expandafter\endcsname
        \expandafter #1%
    \else
        \BNE_notanoperator {#1}\expandafter\BNE_getop
    \fi
}%
\def\BNE_notanoperator #1{\BNE:not_an_operator! \xint_gobble_i {#1}}%
%    \end{macrocode}
% \subsection{Until macros for global expression and parenthesized sub-ones}
% The minus sign as prefix is treated here. 
%    \begin{macrocode}
\catcode`) 11
\def\BNE_tmpa #1{%  #1=\BNE_op_-vi token
    \def\BNE_until_end_a ##1%
    {%
        \xint_UDsignfork
            ##1{\expandafter\BNE_until_end_a\romannumeral-`0#1}%
              -{\BNE_until_end_b ##1}%
        \krof 
    }%
}\expandafter\BNE_tmpa\csname BNE_op_-vi\endcsname
\def\BNE_until_end_b #1#2%
    {%
        \ifcase #1\expandafter\BNE_done 
        \or
        \xint_afterfi{\BNE:extra_)_?\expandafter
                      \BNE_until_end_a\romannumeral-`0\BNE_getop }%
        \else 
        \xint_afterfi{\expandafter\BNE_until_end_a
                      \romannumeral-`0\csname BNE_op_#2\endcsname }%
        \fi
    }%
\catcode`( 11
\def\BNE_op_( {\expandafter\BNE_until_)_a\romannumeral-`0\BNE_getnext }%
\let\BNE_oparen\BNE_op_(
\catcode`( 12
\def\BNE_tmpa #1{% #1=\BNE_op_-vi
    \def\BNE_until_)_a ##1{\xint_UDsignfork
                           ##1{\expandafter \BNE_until_)_a\romannumeral-`0#1}%
                             -{\BNE_until_)_b ##1}%
                      \krof }%
}\expandafter\BNE_tmpa\csname BNE_op_-vi\endcsname
\def \BNE_until_)_b #1#2%
    {%
     \ifcase  #1\expandafter    \BNE_missing_)_? % missing ) ?
                \or\expandafter \BNE_getop       % found closing )
                \else \xint_afterfi 
      {\expandafter \BNE_until_)_a\romannumeral-`0\csname BNE_op_#2\endcsname }%
        \fi
    }%
\def\BNE_missing_)_? {\BNE:missing_)_inserted \xint_c_ \BNE_done }%
\let\BNE_precedence_) \xint_c_i
\let\BNE_op_)   \BNE_getop 
\catcode`) 12 
%    \end{macrocode}
% \subsection{The arithmetic operators.}
% This is where the infix operators are mapped to actual macros. These macros
% must ``f-expand'' their arguments, and know how to handle then big integers
% having no leading zeros and at most a minus sign.
%    \begin{macrocode}
\def\BNE_tmpc #1#2#3#4#5#6#7%
{%
  \def #1##1% \BNE_op_<op>
  {% keep value, get next number and operator, then do until
    \expandafter #2\expandafter ##1\romannumeral-`0\expandafter\BNE_getnext }%
  \def #2##1##2% \BNE_until_<op>_a
  {\xint_UDsignfork 
    ##2{\expandafter #2\expandafter ##1\romannumeral-`0#4}%
      -{#3##1##2}%
   \krof }%
  \def #3##1##2##3##4% \BNE_until_<op>_b 
  {% either execute next operation now, or first do next (possibly unary)
    \ifnum ##2>#5%
    \xint_afterfi {\expandafter #2\expandafter ##1\romannumeral-`0%
      \csname BNE_op_##3\endcsname {##4}}%
    \else \xint_afterfi {\expandafter ##2\expandafter ##3%
      \csname .=#6{\BNE_unlock ##1}{\BNE_unlock ##4}\endcsname }%
    \fi }%
  \let #7#5%
}%
\def\BNE_tmpb #1#2#3%
{%
  \expandafter\BNE_tmpc 
  \csname BNE_op_#1\expandafter\endcsname
  \csname BNE_until_#1_a\expandafter\endcsname 
  \csname BNE_until_#1_b\expandafter\endcsname 
  \csname BNE_op_-#2\expandafter\endcsname 
  \csname xint_c_#2\expandafter\endcsname 
  \csname #3\expandafter\endcsname
  \csname BNE_precedence_#1\endcsname
}%
\BNE_tmpb  +{vi}{bnumexprAdd}%
\BNE_tmpb  -{vi}{bnumexprSub}%
\BNE_tmpb  *{vii}{bnumexprMul}%
\BNE_tmpb  /{vii}{bnumexprDiv}%
\if1\BNE_allowpower\BNE_tmpb ^{viii}{bnumexprPow}\fi
%    \end{macrocode}
% \subsection{The minus as prefix operator of variable precedence level}
% We only need here two levels of precedence, |vi| and |vii|. If the power |^|
% operation is authorized, then one further level |viii| is needed.
%    \begin{macrocode}
\def\BNE_tmpa #1% #1=vi or vii
{%
\expandafter\BNE_tmpb
    \csname BNE_op_-#1\expandafter\endcsname
    \csname BNE_until_-#1_a\expandafter\endcsname
    \csname BNE_until_-#1_b\expandafter\endcsname
    \csname xint_c_#1\endcsname
}%
\def\BNE_tmpb #1#2#3#4%
{%
    \def #1% \BNE_op_-<level> 
    {%  get next number+operator then switch to _until macro
        \expandafter #2\romannumeral-`0\BNE_getnext 
    }%
    \def #2##1% \BNE_until_-<level>_a
    {\xint_UDsignfork
        ##1{\expandafter #2\romannumeral-`0#1}%
          -{#3##1}%
     \krof }%
    \def #3##1##2##3% \BNE_until_-<level>_b
    {%
        \ifnum ##1>#4%
         \xint_afterfi {\expandafter #2\romannumeral-`0%
                        \csname BNE_op_##2\endcsname {##3}}%
        \else
         \xint_afterfi {\expandafter ##1\expandafter ##2%
                        \csname .=\expandafter\BNE_Opp
                                  \romannumeral-`0\BNE_unlock ##3\endcsname }%
        \fi
    }%
}%
\BNE_tmpa {vi}%
\BNE_tmpa {vii}%
\if1\BNE_allowpower\BNE_tmpa {viii}\fi
\def\BNE_Opp #1{\if-#1\else\if0#10\else-#1\fi\fi }%
%    \end{macrocode}
% \subsection{The comma may separate expressions.}
% It suffices to treat the comma as a binary operator of precedence |ii|. We
% insert a space after the comma. The current code in |\xintexpr| does not do
% it at this stage, but only later during the final unlocking, as there is
% anyhow need for some processing for final formatting and was considered to
% be as well the opportunity to insert the space. Here, let's do it
% immediately. These spaces are not an issue when |\bnumexpr| is
% identified as a sub-expression in |\xintexpr|, for example in:
% |\xinttheiiexpr lcm(\bnumexpr 175-12,123+34,56*31\relax)\relax| (this
% example requires package |xintgcd|).
%    \begin{macrocode}
\catcode`, 11
\def\BNE_op_, #1%
{%
    \expandafter \BNE_until_,_a\expandafter #1\romannumeral-`0\BNE_getnext 
}%
\def\BNE_tmpa #1{% #1 = \BNE_op_-vi
  \def\BNE_until_,_a ##1##2%
  {%
    \xint_UDsignfork
        ##2{\expandafter \BNE_until_,_a\expandafter ##1\romannumeral-`0#1}%
          -{\BNE_until_,_b ##1##2}%
     \krof }%
}\expandafter\BNE_tmpa\csname BNE_op_-vi\endcsname
\def\BNE_until_,_b #1#2#3#4%
{%
    \ifnum #2>\xint_c_ii
        \xint_afterfi {\expandafter \BNE_until_,_a
                   \expandafter #1\romannumeral-`0%
                   \csname BNE_op_#3\endcsname {#4}}%
    \else
        \xint_afterfi {\expandafter #2\expandafter #3%
                       \csname .=\BNE_unlock #1, \BNE_unlock #4\endcsname }%
    \fi
}%
\let \BNE_precedence_, \xint_c_ii
\catcode`, 12
%    \end{macrocode}
% \subsection{Disabling tacit multiplication}
%    \begin{macrocode}
\def\BNE_notacitmultiplication{%
  \def\BNE_getop_a ##1##2{%
    \ifx \relax ##2\expandafter\xint_firstoftwo\else
                   \expandafter\xint_secondoftwo\fi
    {\BNE_foundend ##1}%
    {\BNE_foundop  ##2##1}%
  }%
  \def\BNE_scan_nbr_a ##1{%
    \ifcat \relax ##1\expandafter\xint_firstoftwo\else
                     \expandafter\xint_secondoftwo\fi
    {!##1}{\expandafter\BNE_scan_nbr_b\string ##1}%
  }%
}%
%    \end{macrocode}
% \subsection{Cleanup}
%    \begin{macrocode}
\let\BNE_tmpa\relax \let\BNE_tmpb\relax \let\BNE_tmpc\relax 
\BNErestorecatcodes
%    \end{macrocode}
% \MakePercentComment
%</package>
%<*dtx>
\section{Changes}
\small
\begin{description}
\input bnumexpr.changes
\end{description}

\DeleteShortVerb{\|}
\CharacterTable
 {Upper-case    \A\B\C\D\E\F\G\H\I\J\K\L\M\N\O\P\Q\R\S\T\U\V\W\X\Y\Z
  Lower-case    \a\b\c\d\e\f\g\h\i\j\k\l\m\n\o\p\q\r\s\t\u\v\w\x\y\z
  Digits        \0\1\2\3\4\5\6\7\8\9
  Exclamation   \!     Double quote  \"     Hash (number) \#
  Dollar        \$     Percent       \%     Ampersand     \&
  Acute accent  \'     Left paren    \(     Right paren   \)
  Asterisk      \*     Plus          \+     Comma         \,
  Minus         \-     Point         \.     Solidus       \/
  Colon         \:     Semicolon     \;     Less than     \<
  Equals        \=     Greater than  \>     Question mark \?
  Commercial at \@     Left bracket  \[     Backslash     \\
  Right bracket \]     Circumflex    \^     Underscore    \_
  Grave accent  \`     Left brace    \{     Vertical bar  \|
  Right brace   \}     Tilde         \~}
\CheckSum {915}
\makeatletter\check@checksum\makeatother
\Finale
%% End of file xint.dtx
