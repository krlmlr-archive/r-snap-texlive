\ProvidesFile{fifinddo.tex}[2011/11/19 documenting fifinddo.sty]
\title{\textsf{fifinddo}\\---\\Filtering \TeX(t) Files by 
       \TeX\thanks{This file describes 
       version \fileversion\ of \textsf{\filename} as of \filedate.}} 
% \listfiles
\documentclass[fleqn]{article}
\usepackage{makedoc}
% \usepackage{substr} %% for failure examples TODO 2010/03/17
\ProvidesFile{makedoc.cfg}[2011/06/27 documentation settings] 

\author{Uwe L\"uck\thanks{\url{http://contact-ednotes.sty.de.vu}}}
% \author{Uwe L\"uck---{\tt http://contact-ednotes.sty.de.vu}}

%% hyperref:
\RequirePackage{ifpdf}
\usepackage[%
  \ifpdf
%     bookmarks=false,          %% 2010/12/22
%     bookmarksnumbered,
    bookmarksopen,              %% 2011/01/24!?
    bookmarksopenlevel=2,       %% 2011/01/23
%     pdfpagemode=UseNone,
%     pdfstartpage=10,
%     pdfstartview=FitH,
    citebordercolor={ .6 1    .6},
    filebordercolor={1    .6 1},
    linkbordercolor={1    .9  .7},
     urlbordercolor={ .7 1   1},   %% playing 2011/01/24
  \else
    draft
  \fi
]{hyperref}

\RequirePackage{niceverb}[2011/01/24] 
\RequirePackage{readprov}               %% 2010/12/08
\RequirePackage{hypertoc}               %% 2011/01/23
\RequirePackage{texlinks}               %% 2011/01/24
\makeatletter
  \@ifundefined{strong} 
               {\let\strong\textbf}     %% 2011/01/24
               {} 
  \@ifundefined{file} 
               {\let\file\texttt}       %% 2011/05/23
               {} 
\makeatother

\errorcontextlines=4
\pagestyle{headings}

\endinput


\MDkeywords{text filtering, macro programming, 
            .txt to .tex enhancement}
\hypersetup{%
    pdftitle=fifinddo.sty filters text by TeX,
    pdfsubject=documenting fifinddo.sty
}
\ReadPackageInfos{fifinddo}
\begin{document}
\maketitle

\begin{flushright}\small\it FIDO, FIND!\\or:\\FIND FIDO!\\oder:\\FIFI, 
                            SUCH!\end{flushright} 
\begin{MDabstract}
'fifinddo' starts implementing parsing of plain text or \TeX\ files 
using \TeX, generalizing the philosophy behind 'docstrip', 
based on how \TeX\ reads macro arguments. 
Rather than typsetting the edited input stream immediately, 
results are written to another file, 
in the first instance as input for \TeX.
Rather than presenting a ``complete study" of a computer-scientific 
idea, it aims at practical applications. 
The main one at present is '\mbox{makedoc}' which removes certain comment 
marks from package files and inserts listing commands. 
Parsing macros are not defined anew at every input chunk, 
but once before a file is processed. 
This also allows for \emph{expandable} sequences of replacements, 
e.g., with `txt'\,$\to$\,\TeX\ functionality. 
The method of testing for substrings is carefully discussed,
revealing an earlier mistake (then) shared with 'substr.sty' 
and \LaTeX's internal `\in@'. 
\end{MDabstract}
\tableofcontents

\pagebreak                                      %% 2011/11/19
\section{Introduction: The Gnome of the Aim} 
\subsection{Parsing by \TeX---are you mad?}
The package name 'fifinddo' is a &\listfiles-compatible abbreviation 
of \lq file-find-do\rq\footnote{\lq file\rq\ possibly for ``searching 
  \TeX(t) files" (I don't remember my thoughts!), 
  while there were requests for doing replacements on 
  \LaTeX\ \emph{environments} on 'texhax'. However, the package might 
  be enhanced in this direction $\dots$ so the name may be wrong 
  $\dots$ but now I like it so much .\,.\,. Or the reason was that 
  results are written to a \emph{separate file}, not typeset 
  immediately.---Let me also mention that \emph{\lq Fifi\rq} (as the 
  package name starts) is a kind of German equivalent to the 
  ``English" \emph{\lq Fido\rq}, or may have been.}
(or think of \lq if found do\rq).
'fifinddo' implements (or aims at) general parsing 
(extracting, replacing [converting], expanding, \dots) 
using \TeX\ where 'texhax' posters strongly urge to use 'sed', 
'awk', or 'Perl'.
'fifinddo''s opposed rationales are: 
\begin{itemize}
\item It works instantly on any \TeX\ installation. 
      (\emph{Restrictions:} Some \TeX\ versions &\write certain hex 
      codes for certain characters, cf.\ \TeX book p.~45, 
      I have seen this with PC\TeX. However, some applications of 
      'fifinddo' are nothing but technical steps where you will read 
      the result files rarely anyway. 
      %% And I have not yet explored extended encodings.)
      %% <- done meanwhile 2010/03/22
\item You can apply and customize it like any \TeX\ macros, knowing 
      just \TeX\ (or even only the documentation of some user-friendly 
      extension of 'fifinddo'), without the need of learning any 
      additional script language. 
\item The syntax of usual utilities (e.g., ``wildcards") 
      is sometimes difficult with \emph{\TeX} files 
      with all their backslashes, square brackets, stars, 
      question marks \dots
\end{itemize}
%
At least the first item is just the philosophy of the 'docstrip' program, 
standard for installing \TeX\ packages; 
and while I am typing this, I find at least 14 other similar packages 
in J\"urgen Fenn's \emph{Topic Index} of the \emph{\TeX\ Catalogue:}
\begin{center}                      %% 2011/11/13
% \begin{quote}\small
% \[\texttt{%
%   \hbox\bgroup\url{%                %% \url impossible!? 2010/03/19 
  \url{%
    http://mirror.ctan.org/help/Catalogue/bytopic.html\string#parsingfiles}% 
%       \footnotemark
%        \egroup\] 
% \]
% \footnotetext{%
%   \href{http://mirror.ctan.org/help/Catalogue/bytopic.html\string#parsingfiles}%
%        {Click here!}}
% \end{quote}
\end{center}
(Some of them may have been \emph{reactance} to 'texhax' and other 
postings urging not to try something like this; some seem just to be 
celebrations of the power of \TeX---yes, celebrate!)

Actually, \TeX's mechanism of collecting macro arguments is hard-wired 
parsing at quite a high level. \LaTeX\ hides this from 
``simple-minded" users by a convention \emph{not} to use that full 
power of \TeX\ for \emph{end-user macros}. 
\emph{Internally}, \LaTeX\ \emph{does} use it in reading 
lists of options and file dates as well as to implement certain 
`FOR'- and `WHILE'-like loop programming structures. 
\LaTeX's `\in@'/`\ifin@' construction is an implementation of a 
``<string1> occurs in <string2>" test. More packages seem to use 
this idea for extracting file informations, like 
'texshade'.\footnote{\url{http://ctan.org/pkg/texshade}} 


However, such packages don't make much ado about parsing, 
there seems to be no general setup mechanism as are presented by 
'fifinddo'. Indeed, tayloring parsing macros to specific applications 
may often be more efficient than a general approach. 

\subsection{Useful for \dots}
My main application of 'fifinddo' at present is typesetting 
documentations of packages using 'makedoc' which removes certain 
percent marks and inserts listing commands, so you edit a package file 
with as little documentation markup as possible. 
This may be extended to other kinds of documents as an alternative 
to 'easylatex' or 'wiki' (the approach of which is dangerous and 
incompatible with certain other things).

I have used a similar own package 'txtproc' successfully, 
where more features were implemented for practical purposes 
than are here so far, yet I don't like its implementation, want to 
improve it here. This package also \emph{created batch files}, e.g., 
to remove temporary files.
This could be used for package handling: 
typset the documentation at the desired place in the tree, 
write the packages to another, write a batch file to remove 
files that are not needed any more after installation
(cf.\ 'make'). %%% !? cf. Wiki `Make (software) 2009/04/11

I used 'txtproc' also for \emph{large-scale substitutions} 
(it had been decided to change the orthography in a part of a book).
Other large-scale substitutions may be: 
\begin{itemize}
\item inserting &\index commands;
\item inserting (soft) hyphenation commands near accents; 
\item manual umlaut-conversion.\footnote{If you know the ``names" 
        of the encodings, Heiko Oberdiek's 
        \ctanpkgref{stringenc}
        may be preferable.}
\item typographical (or even orthographical) corrections (same mistake 
      many times on each of hundreds of pages). 
      You may turn `...' into `$\dots$' and `etc.' into `etc.\' 
      etc.\footnote{But what when a new sentence is starting indeed? 
        Well, `cf.' is an easier example.---`etc.' even showed 
        a problem in 'niceverb'. 'mdoccorr.cfg' replaces `etc. ' 
        only, so you can keep the extra space by a code line break.}
      This could replace packages like 
      'easylatex',\footnote{\url{http://ctan.org/pkg/easylatex}} 
      'txt2latex',\footnote{\url{http://ctan.org/pkg/txt2latex}} 
      'txt2tex'\footnote{\url{http://ctan.org/pkg/txt2tex}} 
      in a customizable way, using, e.g., the ``correct"
      hook from 'makedoc.sty' as exemplified in 'mdoccorr.cfg' 
      (see examples section of 'makedoc.pdf'). 
      You should find 'fdtxttex.tpl', a 'fifinddo' script 
      to try or apply `\MakeDocCorrectHook' from 'mdoccorr.cfg', 
      as well as 'fdtxttex.tex' that runs a dialogue for the 
      same purpose if you can manage to run it ('WinShell'?).
      You can then try to create your own `\MakeDocCorrectHook'. 
      Section~\ref{sec:replchain} provides setup for macros 
      of this kind. 
\item as to 'easylatex' again, \emph{lists} could be detected and 
      transformed into \LaTeX\ list commands. This could re-implement 
      the lists functionality of 'wiki.sty' that is somewhat 
      dangerous. %% TODO 2009/04/15
\item introduce your own \emph{shorthands} to be expanded not as \TeX\ macros, 
      but by text substitution;
\end{itemize}
%% % <- TODO present macros/"jobs" doing this. 
%%           %% <- TOOD single quote tests could be used for 
%%                      double quote character as well 
%%                      to turn it into left quotes if ... 2010/03/19
%
In certain cases, insertions deteriorate readability, hyphenation 
corrections even make text search difficult. 
It is therefore suggested to 
\begin{enumerate}
\item keep editing the file without the insertions, 
\item run the script (commands based on 'fifinddo') for insertions 
      in the preamble of the main file 
      (```\jobname.tex'", maybe &\input the script file) and 
\item &\input the result file within the `document' environment. 
\end{enumerate}

In general, differences to ``manual" replacing by the substitution 
function of your \emph{text editor} is that 
\begin{itemize}
\item you first keep the original version, 
\item you can check the resulting file before you replace the original 
      file by it, 
\item you can store the replacement script in order to check for 
      mistakes at a later stage of your work, 
\item you can do \emph{all} the replacements in \emph{one run} 
      (by \emph{one} script to check for mistakes), 
\item you can store replacement scripts for future applications, 
      so you needn't type the patterns and replacement strings anew. 
\end{itemize}


\subsubsection{Comparisons} %% was `Missing' 2010/03/18
% \subsection{To implement or not to implement \dots} %% TODO!? 2009/04/13
% Actually, 'fifinddo' doesn't implement very much, and this possibly 
% won't change, even if I am lucky. ``Practical applications" tend to 
% show (in my view) that there is a very complex space of possibilities 
% what parsing macros could do. If I try to ``attack" 'Perl' or so, 
% cf.\ 'CPAN'\@. If I am lucky and become famous, 'fifinddo' will get many 
% extensions in separate packages, cf.\ 'pstricks'. I also like it when 
% you can choose among packages for similar purposes, and this may apply 
% to extensions of 'fifinddo' as well. 
% 
% What \emph{really} is missing very badly at present is matching 
% \emph{sequences} of strings. However, 'makedoc' contains a practical 
% application of this for the case that both strings appear \emph{in the same 
% line}. A better algorithm would have to read more lines from the input 
% file etc. 
% 
% As it just comes to my frightened mind, there is even a problem with 
% ``strings"/``patterns" that are \emph{phrases} of words, containing 
% spaces \dots
% 
% Clearly much comfort concerning \emph{routine} jobs is missing now, 
% 'txtproc' has more. Please sponsor (or otherwise support) the work for 
% improving this. 
% 
It should be noted (perhaps here) that the present approach to parsing 
is a quite \emph{simple} one and in this respect much different to the 
string handling mechanisms of 
'stringstrings',\footnote{\url{http://ctan.org/pkg/stringstrings}} 
'ted',\footnote{\url{http://ctan.org/pkg/ted}} 
'xstrings'\footnote{\url{http://ctan.org/pkg/xstrings}} 
(as I understand them, perhaps also 
'coolstr'\footnote{\url{http://ctan.org/pkg/coolstr}}) 
which are \emph{much more powerful} than what is offered here---but 
perhaps slow and for practical applications possibly replaceable by 
the present approach.
\emph{Expandable replacement} seems not to exist outside 'fifinddo' 
(2009/04/13).

Much is missing, I know.\footnote{There 
  is more in my badly implemented 'txtproc.sty'.}
I am just implementing what I actually need 
and what could show that this approach is worth being pursued. 
  %% removed 2010/03/18:
% It may need being sponsored or otherwise supported. 

\subsection{For insiders}

\emph{Warning:} You may (at least at the present state of the work) 
have little success with this package, if you don't know about 
\TeX's category codes and how \TeX\ macros are defined. 
The package rather provides tools for package writers. 
You may, however, be able to run other packages which just load 
'fifinddo' as required background.

That 'fifinddo' acts on ``\TeX(t)" files or so means that 
(at present) I think of applications on ``plain text" files which 
will usually be \TeX\ input files. ``At present" they are read 
without ``special characters," so essentially category codes of input 
characters are either 11 (``letter") or 12 (``other"). 
This way some things are easier than with usual \TeX\ applications: 
\begin{enumerate}
\item You can ``look into" curly braces and ``behind" comment characters. 
\item There are exact or safe tests especially of \emph{empty macro arguments} 
      that are ``expandable," i.e., they are ``robust," 
      don't need assignments, can be executed in `\write'ing and in 
      `\edef' definitions. ``Usually," the safe way to test emptiness 
      is storing a macro argument as a macro, say `\tempo', in order 
      to test `\ifx\tempo\empty' where `\empty' has been defined 
      by `\def\empty{}' in the format. But this requires some 
      `\def\tempo{#<n>}' which breaks in ``mere expanding" 
      (\TeX\ \emph{evaluates} `\tempo' instead of defining it).
      An \emph{expandable} test on emptiness is, e.g. `\ifx$#<n>$', 
      where we hope that it becomes `\iftrue' just if macro argument 
      `#<n>' is empty indeed. However, ``usually" it may \emph{also} 
      become `\iftrue' when `#<n>' starts with `$'---if the latter 
      has category code 3 (``math shift"). But 'fifinddo' does 
      not assign category code 3 to any character from the input file! 
      Therefore `\ifx$#<n>$' is `\iftrue' \emph{exactly} if `#<n>' is 
      empty. 
\item You can avoid interference with packages that are needed for 
      typesetting. You can do the ``preprocessing" in one run with 
      typesetting, but you should do the preprocessing before you 
      load packages needed for typesetting. One may even try to keep 
      the macros and settings for preprocessing local to a group. 
\end{enumerate}
%% done, moving to .sty 2010/04/06:
% Once there may be an option to read input with some usual \TeX\ 
% category codes as well, it may be useful to (some of)
% \begin{itemize}
% \item avoid matching substrings of control words, 
% \item skip blank spaces as \TeX\ does it usually,
% \item catch \emph{balanced} input pieces,
% \item ignore comments, 
% \item ignore certain characters. 
% \end{itemize}

The essential approach of 'fifinddo' to looking for single strings is 
described in some detail in section~\ref{sec:theory}.

The implementation of 'fifinddo' is as follows.
User commands are specially highlighted (boxed\slash coloured), 
together with their syntax description. 

%   \end{document}

\section{Preliminaries}
\subsection{Head of file (Legalese)}
\sloppy 
\ProcessLineMessage{}
\MakeInputJobDoc{23}{\SectionLevelTwoParseInput}
\end{document}

TODO cf. Kastrup 'makematch' and others 2009/04/09, 
         'parselines' 2010/11/13

2009/04/10  title break, by -> with(!?); `boxed/coloured'
2009/04/12  useful: own shorthands
2009/04/13  substr.sty, \emph{expandable}, w/o `implement'!?
2009/04/15  same with \in@
2010/03/08  EPS application removed
2010/03/17  use \MakeInputJobDoc
2010/03/18  `Missing' -> `Comparisons' etc.; \Require... 
2010/03/19  Catalogue \url; '' -> "
2010/03/20   \ctanpkgref
2010/03/22  debugging; done ...
2010/03/23  more URLs; abstract "then shared"; 
            removed dates from 2009, fdtxttex.tex/tpl
2010/04/06  possible uses of reading TeX category codes moved to .sty 
2010/11/11  \thanks <- package version
2010/11/12  package date updated, \author -> makedoc.cfg 
2010/11/13  package date updated; parselines
2010/11/24  debugging
2010/11/25f. using readprov.sty
2010/11/27  reworked \ProvidesFile
2011/08/22  PDF metadata; adjusting to makedoc.cfg
2011/11/13  [fleqn]
2011/11/19  \pagebreak
