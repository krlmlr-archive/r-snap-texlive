\ProvidesFile{makedoc.tex}[2012/11/30 documenting makedoc.sty]
\RequirePackage{makedoc}
\documentclass[fleqn]{article}              %% fleqn 2011/10/12
\ProvidesFile{makedoc.cfg}[2011/06/27 documentation settings] 

\author{Uwe L\"uck\thanks{\url{http://contact-ednotes.sty.de.vu}}}
% \author{Uwe L\"uck---{\tt http://contact-ednotes.sty.de.vu}}

%% hyperref:
\RequirePackage{ifpdf}
\usepackage[%
  \ifpdf
%     bookmarks=false,          %% 2010/12/22
%     bookmarksnumbered,
    bookmarksopen,              %% 2011/01/24!?
    bookmarksopenlevel=2,       %% 2011/01/23
%     pdfpagemode=UseNone,
%     pdfstartpage=10,
%     pdfstartview=FitH,
    citebordercolor={ .6 1    .6},
    filebordercolor={1    .6 1},
    linkbordercolor={1    .9  .7},
     urlbordercolor={ .7 1   1},   %% playing 2011/01/24
  \else
    draft
  \fi
]{hyperref}

\RequirePackage{niceverb}[2011/01/24] 
\RequirePackage{readprov}               %% 2010/12/08
\RequirePackage{hypertoc}               %% 2011/01/23
\RequirePackage{texlinks}               %% 2011/01/24
\makeatletter
  \@ifundefined{strong} 
               {\let\strong\textbf}     %% 2011/01/24
               {} 
  \@ifundefined{file} 
               {\let\file\texttt}       %% 2011/05/23
               {} 
\makeatother

\errorcontextlines=4
\pagestyle{headings}

\endinput

 
%% 2011/08/22:
\MDkeywords{literate programming, .txt to .tex enhancement, 
            preprocessing documentation, macro programming}
\hypersetup{%
    pdftitle=makedoc.sty preprocesses .sty for documentation,
    pdfsubject=documenting makedoc.sty
}%% /2011/08/22
\sfcode`/=1001 %% TODO makedoc.cfg!? 2010/03/21 niceverb!? 2011/01/25 
\makeatletter %% TEST for hyperref compatibility 2010/03/11
%   \def\@testdef #1#2#3{%
%     \def\reserved@a{#3}%
%     \expandafter \ifx \csname #1@#2\endcsname
%    \reserved@a  \else \@tempswatrue \fi
%     \if@tempswa
%       \typeout{^^J*** Type `r' <input> to get around 
%                       \string\label\space issues! ***^^J}
%       \errorcontextlines=0
%       \show\reserved@a
%       \expandafter \show \csname #1@#2\endcsname
%     \fi
%    }
% \makeatother
\begin{document}
\title{'makedoc'---Preprocessing documentation by \TeX\thanks{%
    This document describes                         %% 2011/01/25
    version~\textcolor{blue}{\UseVersionOf{makedoc.sty}}
    of 'makedoc.sty' as of \UseDateOf{makedoc.sty}.}}
\maketitle
\begin{MDabstract}
'makedoc' provides commands for generating \LaTeX\ input from a 
package file in order to typeset the latter's documentation 
(somewhat similar and opposite to 'docstrip')---with 
v0.3 \emph{a single one usually suffices}. 
Certain comment marks are removed, listing commands are inserted, 
and some (configurable) typographical `txt'$\to$\TeX\ corrections 
are applied.---This 
continues the policy of 'niceverb' to minimize documentation markup in 
package files. 'makedoc' extends and exemplifies the parsing package 
'fifinddo'. After an edit (and test) of your package, you get the new 
documentation in one run (or the usual number of runs) of the 
documentation driver file.---The present approach is meant to be an 
\emph{alternative} to the standard 'doc' package and its `\DocInput'. 
It provides \emph{less} than 'doc' does, rather deliberately. It may 
be helpful at least for the development of small packages, or at least 
at early stages.
\end{MDabstract}
\tableofcontents
\section{Introduction}
\emph{The abstract will not be repeated in this section.} Let me add 
instead that I was in dire need of such a package, I got stuck with my 
packages because I lost orientation in them, and I was unhappy with 
the forms of documentations of my other packages, and documenting them 
with the standard \LaTeX\ 'doc' system was not attractive for me 
(neither considered helpful). %% clarified 2010/03/13
I also worked on \emph{Windows} until September 2008, and I 
find a system like the present one still more attractive then using 
(learning!\@) other filtering utilities (see below on 'awk'). And I 
may work on \emph{Windows} once again and don't want to depend on 
installing some $\dots$ there---\emph{I really would like to have 
powerful tools for everything depending on nothing but \TeX\slash 
\LaTeX!}

\section{Prior work and what is new}
It is, of course, not a new idea to get around comment marks `%' to 
typeset the documentation. 'doc''s `\DocInput' does this by making `%' 
an ``ignored" character. This way you cannot use `%' for commenting 
comments (so 'doc' offers a ``new comment mark" 
`^'`^'`A'). %% TODO `^^A' suddenly failed 2010/03/15 -- "ligature"!?
You also cannot use `%' for commenting out code (that you are 
pondering---or using for debugging---only). %% clarified 2010/03/13

Moreover, 'doc' requires enclosing package code explicitly by 
environment commands (behind comment marks). Stephan I. B\"ottcher 
with his '\href{http://ctan.org/pkg/lineno}{lineno.sty}' 
and Grzegorz Murzynowski in \ctanpkgref{gmdoc}
aimed at doing away with this requirement. 
'lineno.sty' contains 'awk' scripts 
to remove starting comment marks and to insert listing commands. 
A file 'lineno.tex' is generated that typesets the documentation. 
By the way, 'lineno.sty' is full of discussions, but it is not 
'docstrip'ped---the maintainers never have received a complaint 
that inputting 'lineno.sty' were too slow. 

'gmdoc' seems to get around comment marks and insert listing commands 
\emph{while typesetting} by a refined version of `\DocInput', 
through some careful detecting and analysing comment marks, 
the approach resembles detection of lists in 'wiki.sty'.\footnote{See 
  'gmdoc.pdf' on &\DocInput. You can learn a lot from this 220 pages 
  document! I also find 
  \ctanpkgref{pauldoc} and \ctanpkgref{xdoc} inspiring.}
And this is a matter of principles---comparing the approaches of 
\emph{preprocessing} ('lineno.sty') and \emph{``smart typesetting"} 
('gmdoc', 'wiki'). Sometimes preprocessing seems to be simpler, 
sometimes detecting while typesetting. 
(Another example is the preprocessor 
\ctanpkgref{easylatex}
of which 'wiki.sty' is a much reduced ``while typesetting" variant.) 
``While typesetting" may be easier when single characters or 
sequences of two or three encode markup
information---but such detection can badly interfere with other 
packages etc. ``Preprocessing" may be easier when entire ``strings" 
of characters decide, which may be anywhere in a file line. 

'makedoc' chooses \emph{preprocessing}, as 'lineno.sty', but by 
\emph{\TeX}. There is a general discussion of this choice in the 
documentation of 'fifinddo'. Preprocessing here can be done in the 
same \LaTeX\ run as typesetting, though you can avoid 
incompatabilities with packages needed for typesetting 
(by inputting them only \emph{after} preprocessing). 

'lineno.sty' exemplifies why preprocessing with \emph{\TeX} may be 
preferable to preprocessing with other utilities: 
When I took over maintenance of 'lineno.sty', 
I needed hard work to get the 'awk' script running. 
The \emph{Munich} 'awk' seemed \emph{not} to behave as the \emph{Kiel} 
'awk' (I chose a Munich 'nawk' and reworked the script a little). 
\TeX\ seems to have better fixed functionality than other utilities! 

A different alternative to \LaTeX's 'doc' system is 
Paul Isambert's '\href{http://ctan.org/pkg/codedoc}{CodeDoc}' 
where the code environments extract package code in typesetting the 
documentation. %% added 2010/03/10

\section{Styles supported (parsers provided)}\label{sec:styles}
% \section{Styles of commenting '.sty's}
We find different styles of documenting \LaTeX\ packages. 
As the main aspects I consider 
(i)~\emph{telling code from comments} 
and (ii)~\emph{markup in comments}. 
(You may find more details on the next matters in the 
 ``implementation" section.)

\subsection{Telling code from comments}
\emph{Comment marks} (usually \lq`%'\rq\ in the case of \TeX) 
probably were named so to mark \emph{``comments"} as opposed 
to code $\dots$ great, but actually, in ``daily practice," 
they are so handy---and used---for ``commenting out" \emph{code}, 
i.e., \emph{managing code versions} in a simple way: 
one does not actually want to \emph{delete} code, 
one might want to use it another time, maybe for debugging
$\dots$ or to remind of earlier attempts that should not be tried 
again $\dots$

This is a problem for \emph{high-quality typesetting} of 
documentation. \emph{Code} should be typeset about as you see it on 
the \emph{screen}---\emph{monospaced}, this allows structuring by 
indenting, it is common practice to use a typewriter typeface for 
this. Real \emph{comments} should be typeset in \emph{high quality} as 
usual with \LaTeX. Little dilemmas therefore occur with \emph{``hidden 
code"} (``commented-out"). A comment mark starts the line, but 
obviously it is not really a comment and rather should be typeset 
like code (and otherwise they may break).           %% 2010/03/22
Another problem are comments at the \emph{end} of a 
\emph{code} line. Sometimes they are ``real comments" ('gmdoc' 
supports this style). But sometimes 
this is only another version of ``version management," code 
``commented-out."

I like the style of writing packages described before and use it all 
the time. I mark ``real comments" with \emph{two} adjacent comment 
marks and an ensuing space to distinguish them clearly from code 
commented out.
%% Adapted to v0.4 2010/03/29:
\emph{This style is presently the one supported by \textup{'makedoc'} 
      as default.}
This way only a line starting with 
|%% | is considered a ``real" comment line. The first three 
characters are removed, and the rest is typeset in high quality. 
Any other lines are typeset verbatim. The 'makedoc' \emph{parser} 
doing this has an ``identifier" |PPScomment| (``percent, percent, 
space"). Another identifier |comment| is a placeholder for 
the comment parser to be used, by default it is an alias for 
`PPScomment'. Lines just containing |%%| (without the space) may be 
used to suppress empty code lines preceding section titles and for 
keeping some visual, relieving space between code and comment lines.

The style I described previously may be considered ``unprofessional." 
The many \LaTeX\ packages documented using the 'doc'\slash'.dtx' 
system don't use comment marks for \emph{``commenting-out"}. 
Or one may mark code commented out by putting no space between the 
percent mark and the code. 
With v0.4 of 'makedoc', this style is supported as |PScomment|. 
You can directly call this as <main-parser> as described below, 
or you can switch to it by 
\[`\CopyFDconditionFromTo{PScomment}{comment}'\]

\subsection{Markup in comments}
\label{sec:mdoccorr-tex}                                %% 2012/03/18
Packages using the 'doc'\slash '.dtx' system as well as alternative 
highly developed systems mentioned above use (enhanced) usual 
\emph{\LaTeX} syntax for markup of comments. Other packages just use 
an \emph{ASCII} style \emph{without} any markup. My idea was to 
support the latter style by some `txt'$\to$\LaTeX\ functionality. 
'makedoc' does this using a file 'mdoccorr.cfg' which is very small 
right now.

I also thought of introducing another sort of ``decent" markup not 
needing much more space than the ``ASCII kernel" of the comments. 
This is to some extent implemented in 'niceverb.sty'. I thought of the 
syntax of editing \textit{Wikipedia} pages; this is partially 
implemented in 'wiki.sty' which unfortunately is not yet compatible 
with 'niceverb'.

But 'makedoc' implements one \textit{Wikipedia} feature in a different 
way than 'wiki.sty' (cf.~'wikicheat.pdf') that looks about as follows:
\begin{eqnarray*}
  \endcell\endcell`%% == Section =='\\
  \endcell\endcell`%% === Subsection ==='\\
  \endcell\endcell`%% ==== Subsubsection ===='
\end{eqnarray*}
i.e., you type `== <title> ==' in place of `\section{<title>}' etc.
The parser must replace `====<title3>===' before `===<title2>===' and 
the latter before `==<title1>=='. In fact, 'makedoc' provides three 
parsers for these situations:
\begin{description}
\cmdboxitem|\SectionLevelThreeParseInput| is the most general parser 
    offered. If it does not find two strings \lq`===='\rq\ enclosing 
    \emph{something}, it passes to
\cmdboxitem|\SectionLevelTwoParseInput| which unless finding 
    two strings `===' enclosing something passes to
\cmdboxitem|\SectionLevelOneParseInput| $\dots$ passes to the comment 
    detector |comment|. 
\end{description}


\section{Requirements}
'makedoc' requires \LaTeXe\ (supporting star forms of `\newcommand' 
etc.)\ as \TeX-format, the package 'fifinddo.sty' from the same 
directory (on CTAN etc.)\ as where 'makedoc.sty' is, and the 
\LaTeX-package 'moreverb' by Robin Fairbairns (after others)---it 
should be installed anyway, or you can get its latest version 
(v2.3, 2008/06/03?) from CTAN. 

'makedoc''s `.txt'$\to$\TeX\ functionality moreover needs a file 
'mdoccorr.cfg' that should have come along with 'makedoc.sty' and 
'fifinddo.sty'. You may need to have a modified copy of it in the 
directory of your main `.tex' file `<jobname>.tex' fitting special 
needs of your project. 

\section{Using 'makedoc' the simplest way}
In the most simple case, you are preparing documentation for a package 
file `<jobname>.sty' only, and you prepare a file `<jobname>.tex' 
containing 
\[`\title{\textsf{<jobname>}---a \LaTeX\ Package for <whatever>}'\]
and `\maketitle' etc.\ about your package.\footnote{With 'niceverb' 
    and &\title\ after &\begin{document}, you may replace 
    \lq&\textsf{<jobname>}\rq\ by \lq&'<jobname>&'\rq.}
The documentation will be produced by running `<jobname>.tex' with 
\LaTeX\ (e.g., \texttt{latex <jobname>.tex}).

First, `<jobname>.tex' must have 
\[|\usepackage{makedoc}|\quad\mbox{or}\quad 
  |\RequirePackage{makedoc}|\]                          %% 2011/11/05
in its preamble. There are no package options. 

Second, to typeset the commented implementation from `<jobname>.sty', 
include in <jobname>.tex's `document' environment a line 
\[|\MakeInputJobDoc{<header-lines>}{\SectionLevelThreeParseInput}|\]
<header-lines> refers to a non-negative integer as follows: 
We think the most simple and useful way of typesetting the first lines 
of a package file including license and copyrights is ``depicting them 
as image," i.e., \textit{verbatim}. We could try to determine the 
number of these lines by parsing, but we won't do so soon. Please just 
count them and enter the number as <header-lines>---and change it 
until you can accept the outcome.

\section{Steps of advanced usage}
\subsection{Different main parsers (second mandatory argument)}

`\MakeInputJobDoc''s mandatory syntax actually is 
\[|\MakeInputJobDoc{<header-lines>}{<main-parser>}|\]
<main-parser> refers to the parsing macro that is applied to each 
input line whose number is greater than <header-lines>. 
Examples for <main-parser> are named in section~\ref{sec:styles} above. 
  %% TODO above/below macro 2010/03/15
`\SectionLevelThreeParseInput' is just the most general one. 
For \emph{efficiency} (!? or also to avoid problems?) you may 
replace `Three' by `Two' or by `One', if the `====' or the `===' 
feature is not used in `<jobname>.sty'. If the ``\textit{Wikipedia} 
sectioning" feature is not used at all, use 
\[|\MakeInputJobDoc{<header-lines>}{\ProcessInputWith{comment}}|\]
---provided you want to adopt the \lq`%% '\rq\ style of marking 
comments, cf.~section~\ref{sec:styles}. For the \lq`% '\rq style 
instead, use 
\[|\MakeInputJobDoc{<header-lines>}{\ProcessInputWith{PScomment}}|\]

\subsection{Different extensions (optional arguments)}
If your package to be documented is a \emph{class} `<jobname>.cls', 
a local configuration file `<jobname>.cfg' or something 
else---<jobname>.<ext-in>, e.g., <ext-in>=`cls' or <ext-in>=`cfg', 
use 
\[|\MakeInputJobDoc[<ext-in>]{<header>}{<parser>}|\]
Moreover, `\MakeInputJobDoc' writes an intermediate file 
`<jobname>.doc' and then `\input's it. If you do not like `doc' 
as extension for the written file name (maybe you use 
`<jobname>.doc' for something different already), preferring extension 
<ext-out>, use
\[|\MakeInputJobDoc[<ext-in>][<ext-out>]{<header>}{<parser>}|\]
Yes, you must state <ext-in> then as well, I can't help $\dots$

If even <jobname> is wrong in your view, see next step $\dots$

\subsection{Commands modifying &\MakeInputJobDoc's behaviour}
\label{sec:modimake}
Already <jobname> may not be what you want. E.g., you may want 
to collect documentations of some other files <job-1>, <job-2>, 
$\dots$ in a single <jobname>. Then precede `\MakeInputJobDoc'
with 
\[`\renewcommand*{\mdJobName}{<job-1>}'\]
etc.\ (please reason yourself about additional requirements \dots)
As a matter of fact, `\MakeInputJobDoc' reads 
\[`\mdJobName.<ext-in>' \mbox{\quad and writes\quad} 
  `\mdJobName.<ext-out>'\]
Stated another way, <jobname> above referred to |\mdJobName|.

`\MakeInputJobInput' moreover (by default) produces one dot 
per input line processed on screen to show progress. 
The reason is that `makedoc' issues the command 
|\ProcessLineMessage{\message{.}}|.
Already this trivial thing seems to slow down processing considerably 
(nowadays). `\MakeInputJobInput' will run faster if preceded by 
\[|\ProcessLineMessage{}|\]
which will suppress any message about processing.
However, the message may be helpful in trouble-shooting.

\subsection{Separating preprocessing from typesetting}
\label{sec:sep-pre-type}                                %% 2011/10/07
  %% extended 2010/03/16
To some surprise, I observe that `\MakeInputJobDoc' \emph{works.} 
This is quite a new discovery of mine (2010/03/13); 
before I thought that, for safety, preprocessing should happen 
inside a local group \emph{preceding} `\documentclass'.
|\MakeJobDoc| works like `\MakeInputJobDoc' described above, 
yet it just \emph{preprocesses} the package to be documented, 
waiting for an 
\[`\input{<jobname>.<ext-out>}'\] 
in the `document' environment to \emph{typeset} the documentation. 
So 'makedoc.tex' once had in its preamble
% \[`{\RequirePackage{makedoc} \MakeJobDoc{<header>}{<parser>}}'\]
% at the top of `<jobname>.tex' and `\input{<jobname>.<out-ext>}' later. 
\begin{eqnarray*}\endcell\endcell
  `{\RequirePackage{makedoc}'\cr    \endcell\endcell
  ` \ProcessLineMessage{}'\cr       \endcell\endcell
  ` \MakeJobDoc{22}{\ProcessInputWith{comment}}}'\cr
                                    \endcell\endcell
  `\documentclass{article}'
\end{eqnarray*} 
I did experience some truth in my earlier safety policy: 
With 'niceverb' ``running," `\MakeJobDoc' cannot (easily) be used 
in the `document' environment. `\MakeInputJobDoc' in fact does some 
'niceverb' switching (provided 'niceverb' has been loaded) 
when making use of `\MakeJobDoc'.
  %% <- verbose to improve line breaks 2010/03/16

Thinking of this ``safety" approach, especially grouping 
(`{<code>}'),   %%% (`{\code}'),                        %% 2011/10/07
I had not much cared for compatibility with other packages 
in choosing 'makedoc' macro names. 

\subsection{Other 'makedoc' (and 'fifindo') script commands}
The next script commands may be considered of a lower level than 
`\MakeJobDoc' and `\MakeInputJobDoc', they underlie the latter 
commands. We also list commands from 'fifinddo.sty' that may be useful 
or, indeed, are needed for preparing package documentations. 
This may result in ideas on how do use the script commands for 
different purposes than for preparing package documentations---e.g., 
apply `txt'$\to$\TeX\ preprocessing to arbitrary text files. 

\subsubsection{Choosing parameter values for next preprocessing run}

This actually continues section~\ref{sec:modimake}.

\begin{description}
\cmdboxitem|\ResultFile{<output>}| (from 'fifinddo') 
    determines (and opens through the \TeX\ primitive `\openout') 
    the file <output> which will contain the result of 
    preprocessing the package file.
\cmdboxitem|\LaTeXresultFile{<output>}|---see next section.
\cmdboxitem|\Headerlines{<number>}| determines the <number> of lines 
    starting the input file to be copied \emph{verbatim} 
    (the first mandatory argument of `\MakeJobDoc'). 
\cmdboxitem|\MainDocParser{<parser>}| determines the <parser> 
    as in the \emph{second} mandatory argument of `\MakeJobDoc'.
\end{description}

We are now describing some parameters which rather must be switched 
``manually" instead of being modifiable by comfortable 'makedoc' 
script commands.

With the \emph{``Wikipedia sectioning"} feature, you may change the outcome 
regarding levels. Assume, e.g., the package file has titles along the 
scheme \[`== <title> =='\] only, but these should become %% \[...\] 2012/11/12
\emph{subsections} of the ``implementation" section of the 
corresponding `.tex' file. Then 
\[`\renewcommand*{\mdSectionLevelOne}{\string\subsection}'\]
-- see the implementation of the sectioning feature for details. 

There is a command 
\[|\NoEmptyInputLines| \mbox{\quad and a parameter macro\quad}
  |\OnEmptyInputLine|\] 
which is modified by the former. However, I cannot say much about them 
right now, I think they just were a difficult matter that I soon 
decided no longer to think about for a while (cf.\ the 
implementation). About the same holds for the hook |\EveryComment|.

The `txt'$\to$\TeX\ functionality is implemented through a hook 
\[|\MakeDocCorrectHook{<characters>}|\] 
'makedoc' initializes it as an alias of \LaTeX's `\@firstofone', i.e., 
it won't change <characters>. 'mdoccorr.cfg' should redefine it so it 
really ``corrects" <characters>. You might try other definitions of 
`\MakeDocCorrectHook' for different ``correcting" functions.
It should be \emph{noted} that (currently) 
`\MakeDocCorrectHook' must be \emph{expandable}, 'fifinddo.sty' 
provides setup for (expandable) chains of expandable replacements. 
The ``Wikipedia" sectioning feature moreover uses expandable 
trimming (single) surrounding spaces, this might be provided in a more 
general way.\footnote{%% TODO 2010/03/16
    The \ctanpkgref{trimspaces} package 
    has been a \emph{model} for this feature here. It cannot be used 
    directly here because it reads blank spaces as \TeX\ characters with 
    category code 10 while 'makedoc' reads blank spaces as ``other" 
    characters (category code 12) in order to \emph{keep} all blank spaces.}

\subsubsection{``Manual" insertions to the output file}
\begin{description}
\cmdboxitem|\WriteResult{<balanced>}| (from 'fifinddo') writes 
    <balanced> to <output> according to the earlier command 
    `\ResultFile{<output>}'.
\cmdboxitem|\WriteProvides| (from 'fifindo') writes a 
    `\ProvidesFile' line into <output> that declares the file 
    to be generated by 'fifindo'.
\cmdboxitem|\LaTeXresultFile{<output>}| issues 
    `\ResultFile{<output>}' and then writes a 
    `\ProvidesFile' line into <output> that declares the file 
    to be generated by 'makedoc'.
\end{description}

\subsubsection{Processing input and closing output}
\begin{description}
\cmdboxitem|\MakeDoc{<input>}|\hskip 0pt plus 4em
    reads 'mdoccorr.cfg' 
    (for `\MakeDocCorrectHook', see above),
    %% removed \LaTeXresult... 2010/03/17
    copies <number> according to `\HeaderLines' (see above) 
    from <input> into <output> (according to `\ResultFile'), 
    then processes the remaining lines of <input> according 
    to `\MainDocParser' (writing several things to <output>). 
    `\MakeDoc' invokes{\sloppy\par}
\cmdboxitem|\ProcessFileWith{<input>}{<loop-body>}| 
    (from 'fifindo') reads <input> line by line---each one stored as 
    macro |\fdInputLine| and applies <loop-body> to it. 
    \TeX's ``special" character codes (of characters listed in 
    macro `\dospecials') are replaced by 12 (``other") by 
    default---see the 'fifinddo' documentation.
\cmdboxitem|\CloseResultFile| (from 'fifinddo') 
    \hskip 0pt plus 1em
    \emph{closes} 
    <output> (using \TeX's primitive `\closeout'). 
\cmdboxitem|\MakeCloseDoc{<input>}| issues 
    `\MakeDoc{<input>}\CloseResultFile'.
\end{description}
%
Using `\MakeDoc' \emph{instead} of `\MakeCloseDoc' allows processing 
additional <input> files writing into the same <output>. Or maybe you 
want to add some additional lines manually to <output> using 
`\WriteResult'.---By contrast, you may want to make a single output 
file from a single input file! Therefore:
%% <- 2011/11/05 ->
%
\begin{description}
\cmdboxitem|\MakeSingleDoc[<out-ext>]{<in-output>.<in-ext>}| \
    issues\\`\LaTeXresultFile{<in-output>.<out-ext>}'\quad 
    and\\`\MakeDocCloseDoc{<in-output>.<in-ext>}'.\\
    The default for <out-ext> is `doc'.
\end{description}

%% removed 2010/03/09:
% At least in the long run, using 'makedoc' should not imply commitment 
% to a certain design or to certain \LaTeX\ packages for typesetting 
% listings and documentations. Therefore, 'makedoc.cfg' (currently) 
% contains \emph{local} or \emph{personal choices}, but also 
% \emph{experiments} with future features of 'niceverb'. 
% Especially, (at present) the `packagecode' 
% environment that 'makedoc' `\write's must be chosen. 
% Currently this is the `listing' environment from 'moreverb' 
% with some modifications or extra settings. 
% It may be vital to `\MakeOther' the active characters from 'niceverb' 
% in the setup of `packagecode'. See the \emph{example} in 
% section~\ref{sec:fifinddo}.
% 
% Finally, 
% Each package file to be typeset needs its own little
% \emph{script} of 'makedoc' commands. 
% With v0.3, one or two of these should suffice. 

% It should fit into the preamble 
% of the main file for documenting the package (currently %% 2009/04/09
%   just 5 commands seem to suffice, see the \emph{example} in 
%   section~\ref{sec:fifinddo}). 
% The script commands are described 
% in the \dqtd{File handling} section of 'fifinddo.pdf' and in the 
% present section~\ref{sec:script} (and \ref{sec:emptylines}).
% As an alternative, you may prefer to have ``content only" (as much as 
% possible) in the main typesetting file and in its preamble only 
% `\input' a separate script file.
%% removed 2010/03/10:
% Yes, the idea of documenting a package \emph{here} is to have a 
% separate ``driver" file for typsetting the documentation. 
% It may contain an introduction and a guide for users. 
% The documentation of the package code that has been prepared by the 
% 'makedoc' script will be `\input'. Alternatively, the ``driver file" 
% could have title etc.\ only, or preamble and a minimal `document' 
% environment only. 
% 
% So there may be many files, which may look confusing, especially as 
% compared with the 'doc' procedure. However, 
% \begin{enumerate}
% \item ``One file distribution" still is possible thanks to the 
%       `filecontents' environment. 
% \item The 'makedoc' script can create a batch file (fitting the 
%       system, maybe using Will Robertson's 'ifplatform', or 
%       'texsys.cfg', or \dots) 
%       that removes certain auxiliary files or moves them to a 
%       certain directory. 
% \item I find it helpful to have rather little ``contentual" text 
%       in the package file. 
% \item The procedure now runs very smoothly, once the stumbling blocks 
%       have been overcome.\footnote{\hspace{1sp}%% TODO help in 'niceverb'!
%         'niceverb' v0.1 was too sloppy with 
%         some things, and self-documentation of 'makedoc.sty' was 
%         difficult---its parsing and that from 'verbatim' cannot 
%         distinguish between markup code and typeset code.}
% \end{enumerate}
\subsection{Star avoids &\input{mdoccorrhook}}
All the preprocessing commands described above---apart from the 
'fifinddo' command `\ProcessFileWith'---input the file 
'mdoccorr.cfg' with (typographical) replacement rules 
(Sec.~\ref{sec:mdoccorr-tex}, Sec.~\ref{sec:mdoccorr})
automatically. They have star variants that don't: 
|\MakeInputJobDoc*|, |\MakeJobDoc*|, |\MakeDoc*|, 
|\MakeCloseDoc*|, and |\MakeSingleDoc*|.
This is useful when rules very specific to a certain package 
must be applied instead of the usual ones.
The examples named in Sec.~\ref{sec:otherauth} use this feature 
for formatting other author's plain text documentation 
without modifying their files.

\section{Examples}%%% (documentation of 'mdoccorr.cfg')}
%% moved here 2010/03/23
\subsection{Documenting 'nicetext''s `mdoccorr.cfg'}
The documentations of 'fifinddo', 'makedoc', and 'niceverb' 
themselves are typeset using 'makedoc'.
'fifinddo.pdf' documents 'fifinddo.sty', typeset 
from 'fifinddo.tex', likewise 'makedoc.pdf' and 'niceverb.pdf'. 
% Section~\ref{sec:fifinddo} contains a listing of 
% 'makedoc.cfg' and 
% the 'makedoc' script file 'mkfddoc.tex' 
% especially made for 'fifinddo.pdf'. 
% 'fifinddo.doc', 'makedoc.doc', and 'niceverb.doc' are the \TeX\ input 
% files that were made with 'makedoc.sty'---I have only looked at them 
% when something was wrong (often syntax mistakes in typing). 
The Wikipedia syntax feature 
\begin{quote}
  `%% === subsection ===' 
\end{quote}
is used in 'fifinddo.sty' and 'niceverb.sty' only.

Along with 'makedoc' should come files `makedoc.tpl'---a template 
'makedoc' script, and a file `fdtxttex.tex' that should start a dialogue 
on trying `\MakeDocCorrectHook' if you can manage to run it ('WinShell'?). 
With other definitions of `\MakeDocCorrectHook'---see below---you can 
use this dialogue for arbitrary replacement jobs (as an application of 
'fifinddo' rather than 'makedoc').{\sloppy\par}

'fifinddo.pdf', 'makedoc.pdf', and 'niceverb.pdf' were typeset with the following 
typographical corrections in 'mdoccorr.cfg' that defines 
`\MakeDocCorrectHook':
% \strut\hrule              %% rm. those struts 2012/11/12
%   \listinginput[5]{1}{mdoccorr.cfg}
  %% <- 2012/11/12 >
\MDsamplecodeinput{mdoccorr.cfg}
\noindent
This code also exemplifies the syntax 'niceverb' provides for writing 
about \LaTeX\ macros. It is typeset here with 'makedoc.sty' 
and then looks thus:
\bigskip 
\hrule
\renewcommand*{\mdJobName}{mdoccorr}
\MakeInputJobDoc[cfg]{0}{\ProcessInputWith{comment}}
\hrule 
\bigskip
\noindent
And this is the content of the intermediate generated file:
%   \listinginput[5]{1}{mdoccorr.doc}
\MDsamplecodeinput[75pt]{mdoccorr.doc}
% \enlargethispage{8pt} %% 2011/09/14 keep rule on same page %% rm. 2011/10/25
% \hrule 
%  \fussy %% 2010/03/29

  \iffalse      %% 2012/11/30
%% 2012/11/12:
\pagebreak
\subsection{Documenting 'nicetext''s `makedoc.cfg'}
`makedoc.cfg' once was meant to be just ``configuration," 
but then I introduced some definitions there that may be 
more interesting and once become a package. 
Its typeset documentation first:
\pagebreak[3]
\strut\bigskip 
\hrule
\smallskip
\renewcommand*{\mdJobName}{makedoc}
\MakeInputJobDoc[cfg]{0}{\ProcessInputWith{comment}}
\hrule 
\bigskip
\noindent\strut
The bare code of `makedoc.cfg' is:
\MDsamplecodeinput{makedoc.cfg}

  \fi
\subsection{Packages from other authors}
\label{sec:otherauth}
`substr.tex' should typeset a nicely formatted documentation 
of Harald Har\-ders' 'substr.sty', see my own result `substr.pdf'. 
'substr.sty' is a rare case of the \lq`%% '\rq\ commenting style 
that 'nicetext' has used itself.

`arseneau.tex' should typeset nicely formatted documentations 
of a few packages by Donald Arseneau, see my own result `arseneau.pdf'. 
This demonstrates the usual \lq`% '\rq\ commenting style 
that 'makedoc' supports with v0.4.

% \pagebreak              %% 2010/03/29; rm. 2011/09/14
\ResetCodeLineNumbers   %% 2010/03/29
\section{Implementation}
\subsection{Preliminaries} 
Head of file (Legalese):
\sloppy
\renewcommand*{\mdJobName}{makedoc}
\ProcessLineMessage{}
\MakeInputJobDoc{22}{\ProcessInputWith{comment}}
\enlargethispage{2\baselineskip}                    %% 2011/11/05
The previous empty code line is the one \TeX\ insists to add at every 
end of a file it writes. %% todo TeXbook where? 2009/04/08

%% 2012/11/30:
\section{`makedoc.cfg' Documented}
`makedoc.cfg' once was meant to be just ``configuration," 
but then I introduced some definitions there that may be 
more interesting and once become a package. 
\renewcommand*{\mdJobName}{makedoc}
\MakeInputJobDoc[cfg]{0}{\ProcessInputWith{comment}}

%% removed (TODO) 2010/03/15:
% \section{Examples: documentation of 'fifinddo'}
% \label{sec:fifinddo} 
%% removed 2010/03/10:
% \subsection{'makedoc.cfg'} 
% 'fifinddo.pdf' and 'makedoc.pdf' were typeset with the following 
% configuration file 'makedoc.cfg':
% \begingroup \MakeOther\|\MakeOther\`\MakeOther\'\MakeOther\<
%   %% <- TODO should be 'niceverb' command 2009/04/08
%   \listinginput[5]{1}{makedoc.cfg}
% \endgroup
%
%% TODO 'niceverb' example---to 'niceverb.tex'!? 2010/03/15
% \subsection{'mkfddoc.tex'}
% 'fifinddo.pdf' was typeset with the following 'makedoc' script file 
% 'mkfddoc.tex':
% \begingroup 
%   \MakeOther\|\MakeOther\`\MakeOther\'\MakeOther\<
%   \listinginput[5]{1}{mkfddoc.tex}
% \endgroup
% 
%
\end{document}

2009/04/12  more on examples
2009/04/15  exemplification of niceverb.sty by mdcorr.cfg
2009/04/21  === subsubsection -> === subsection
2010/03/08  moved `only' for better line break
2010/03/09  removed something from "Basics"
2010/03/10  more changes in "Basics", pdf stuff to makedoc.cfg, 
            makedoc.cfg no longer example; CodeDoc
2010/03/11  use \MakeCloseDoc; \hfuzz with \listinginput;
            tracing spurious `Label(s) may have changed'
2010/03/12  tests for hyperref compatibility completed
2010/03/13  use \MakeInputJobDoc; clarified ...; ctan.org/pkg
2010/03/14  updated ``Examples" and abstract; \href...easylatex
2010/03/15  ``styles supported"; abstract: txt->TeX; usage
2010/03/16  more on usage; mdcorr -> mdoccorr
2010/03/17  corr. mistake with \MakeDoc
2010/03/19  '' -> " 
2010/03/20  for niceverb v0.31
2010/03/21  for niceverb v0.32
2010/03/22  "may break"
2010/03/23  \noindent in example, moved, added mdoccorr.doc, 
            makedoc.tpl
2010/03/29  \ResetCodeLineNumbers, 
            examples and explanations for v0.4
2010/03/30  \listfiles test
2010/11/27  \ProvidesFile for myfilist
2011/01/25  \thanks with readprov.sty 
2011/08/22  using new makedoc.cfg features
2011/09/14  rm. \pagebreak
2011/10/07  "grouping" fix(?) + \label in "Separating"
2011/10/12  fleqn
2011/10/25  without \enlargethispage
2011/11/05  or \RequirePackage{makedoc}; \MakeSingleDoc
2011/11/19  debugging for fifinddo v0.5
2012/03/18  * forms; Harald H. corrected
2012/11/12  reworked `mdoccorr' example (\MDsampleinput), 
            added `makedoc.cfg' documentation as sample
2012/11/13  using \MDsamplecodeinput from `makedoc.cfg' same day
