\ProvidesFile{nicefilelist.tex}[2012/12/12 documenting nicefilelist.sty]
\title{\pkg{\huge nicefilelist.sty
       }\\---\\\cs{listfiles} Alignment for Connoisseurs\thanks{This
       document describes version
       \textcolor{blue}{\UseVersionOf{\jobname.sty}}
       of \pkg{\jobname.sty} as of \UseDateOf{\jobname.sty}.}}
{ \RequirePackage{makedoc} \ProcessLineMessage{}
  \MakeJobDoc{18}%% 2012/10/30
  {\SectionLevelTwoParseInput}  }
\documentclass[fleqn]{article}%% TODO paper dimensions!?
\ProvidesFile{makedoc.cfg}[2011/06/27 documentation settings] 

\author{Uwe L\"uck\thanks{\url{http://contact-ednotes.sty.de.vu}}}
% \author{Uwe L\"uck---{\tt http://contact-ednotes.sty.de.vu}}

%% hyperref:
\RequirePackage{ifpdf}
\usepackage[%
  \ifpdf
%     bookmarks=false,          %% 2010/12/22
%     bookmarksnumbered,
    bookmarksopen,              %% 2011/01/24!?
    bookmarksopenlevel=2,       %% 2011/01/23
%     pdfpagemode=UseNone,
%     pdfstartpage=10,
%     pdfstartview=FitH,
    citebordercolor={ .6 1    .6},
    filebordercolor={1    .6 1},
    linkbordercolor={1    .9  .7},
     urlbordercolor={ .7 1   1},   %% playing 2011/01/24
  \else
    draft
  \fi
]{hyperref}

\RequirePackage{niceverb}[2011/01/24] 
\RequirePackage{readprov}               %% 2010/12/08
\RequirePackage{hypertoc}               %% 2011/01/23
\RequirePackage{texlinks}               %% 2011/01/24
\makeatletter
  \@ifundefined{strong} 
               {\let\strong\textbf}     %% 2011/01/24
               {} 
  \@ifundefined{file} 
               {\let\file\texttt}       %% 2011/05/23
               {} 
\makeatother

\errorcontextlines=4
\pagestyle{headings}

\endinput

 %% shared formatting settings
\usepackage[wrap]{nicefilelist} \listfiles 
%% <- 2012/10/30 try without 'myfilist'
\MDkeywords{Package management, document management, plain text output}
\sloppy
\begin{document}
\maketitle
\begin{MDabstract}
While \CtanPkgRef{longnamefilelist}{longnamefilelist.sty}
improves \LaTeX's \cs{listfiles} with respect to long base filenames only,
'nicefilelist.sty' can keep separate columns for 
(i)~date, (ii)~version, and (iii)~``caption"    %% 1st , 2012/10/30
(don't write caption text in date column), 
their alignment not being disturbed by short filename extensions such as `.fd'. 
This is achieved basing on the \ctanpkgref{monofill} package.

% Thus 'nicefilelist' is more ``powerful" than 'longnamefilelist', 
% the former however is an ``extension" of the latter 
% neither with respect to implementation nor with respect to user interface.

%% <- 2012/10/29 ->
v0.7 offers a package option `[wrap]' for automatic word wrapping within 
the caption column (using the \ctanpkgref{hardwrap} package), 
so filenames and captions can be quite long without disturbing alignment.

As opposed to the \ctanpkgref{dateiliste} package, this is about the 
\meta{plain text} output in the `.log' file or, with \ctanpkgref{myfilist}, 
as a stand-alone plain text file.

\MDaddtoabstract{Related packages:} Cf.~\ctanpkgref{latexfileinfo-pkgs}.
\end{MDabstract}
\tableofcontents
\section{Features and Usage}
% We are describing relations to, ahm, related packages---rather brief{}ly. 
% The \ctanpkgref{latexfileinfo-pkgs} package provides a more general overview.
% %% <- 2012/05/18
%% 2012/10/29:
Additionally or also ``complementarily" to the presentation given here, 
the functionality of the package is summarized in the file 
`latexfileinfo_pkgs.htm' from the \ctanpkgref{latexfileinfo-pkgs},
in a comparison with packages resembling 'nicefilelist' in 
certain respects.

\subsection{Relation to 'longnamefilelist.sty'}
\CtanPkgRef{longnamefilelist}{longnamefilelist.sty} equips 
\cs{listfiles} with an optional argument for the maximum number of characters 
in the base filename. By contrast, 'nicefilelist' does not provide 
arguments for \cs{listfiles}, rather column widths for basename, extension, 
and version number are determined by \emph{templates} using 
\CtanPkgRef{monofill}{monofill.sty}. As a ``template" for doing this, 
see the initial settings in Sec.~\ref{sec:templates}. 
(Such settings must precede the \cs{listfiles command})
So 'nicefilelist''s \emph{user interface} (at present) does not \emph{extend} 
'longnamefilelist''s user interface.

Using 'monofill' is a very different approach than the one of 'longnamefilelist'. 
'nicefilelist' is more powerful than 'longnamefilelist', but is not based on it 
in any way. It does not make sense to load both packages, they just overwrite 
each other's behaviour of \cs{listfiles}. 

'longnamefilelist' may become ``obsolete" by the present package, 
unless one finds that its version of \cs{listfiles} looks fine enough 
and it is easier to understand and to use than 'nicefilelist'.

\subsection{Installing}
The file 'nicefilelist.sty' is provided ready, installation only requires
putting it somewhere where \TeX\ finds it
(which may need updating the filename data
 base).\urlfoot{ukfaqref}{inst-wlcf} 

\subsection{Calling}
%% restructured for v0.7 2012/10/29:
Below the `\documentclass' line(s) and above `\begin{document}',
you load 'nicefilelist.sty' (as usually) by
\[`\usepackage{nicefilelist}'\]
or by 
\[`\usepackage[<options>]{nicefilelist}'\]
where <options> may be `r', `wrap', or `r,wrap'~\dots---see 
summaries in sections~\ref{sec:set} and \ref{sec:opt} 
on the package options and an example in Section~\ref{sec:short}.
Alternatively---e.g., for use with \ctanpkgref{myfilist} from the 
\ctanpkgref{fileinfo} bundle (in a ``\TeX~script"), see Section~\ref{sec:myfilist}, 
or in order to include the `.cls' file in the list---you may load it by 
\[`\RequirePackage{nicefilelist}'\]
or by
\[`\RequirePackage[<options>]{nicefilelist}'\]
before `\documentclass' or when you don't use `\documentclass'. 

\subsection{Choosing Settings}                      %% new 2012/10/29
\label{sec:set}

\subsubsection{The Columns, Their Widths, and Their ``Missing" Content}
\label{sec:cols}

\AddQuotes                          %% 2012/10/29

The 'nicefilelist' package considers the listing from `\listfiles' 
a five-column table, the columns being (reserved for) \
(i)~the base filename, \ (ii)~the filename extension, \ 
(iii)~the date, \ 
(iv)~the version (or with option `[r]': the release) number, and \
(v)~the caption \ of a \LaTeX\ source file. The filename base column 
is right-adjusted, the other ones are left-adjusted. 
Date, version, and caption are made up from the <f-info> argument in 
\[|\Provides<f-type>{<f-base>.<f-ext>}[<f-info>|]\]
where <f-type> is `Class', `Package', or `File'.

The fixed usual format `YYYY/MM/DD' for the date is assumed; 
in fact, when <f-info> doesn't start according to this format, 
it is assumed that no date is given, and some ``missing''
text will appear in the ``date" column, determined by a macro 
|\NFLnodate|. The version number 
(or ``string") must follow in format `v<digit>.<digits>', 
otherwise some ``missing" text appears in the ``version" column, 
determined by a macro |\NFLnoversion|. 
What remains is placed in the ``caption" column. 
|\NFLnotfound| determines an alternative filling 
in the case that <f-info> cannot be obtained. 
See the default settings for these ``failure" texts in 
Section~\ref{sec:fail}.

The column widths for filename base and extension 
and the column width for version or release 
are determined using the \ctanpkgref{monofill} package. 
They have ``field identifiers" |f-base|, |f-ext|, 
and |f-version| respectively. The respective widths are 
determined by templates <longest> in
\[|\MDfieldtemplate{<field-id>}{<longest>}|\]
See Section~\ref{sec:templates} for the default settings. 
Probably only adjusting the width for \emph{base} filenames
is required in real life, see the example in Section~\ref{sec:short}.

The spaces between the columns are determined by macros 
|\NFLspaceI|, |\NFLspaceII|, and |\NFLspaceIII|, 
see Section~\ref{sec:templates} for the defaults.

\DontAddQuotes              %% 2012/10/29

\subsubsection{The Caption Column}
The width of the caption column (unfortunately) is determined by 
the stuff enumerated above and the width of the console output window or screen. 
With long filenames and long captions, the result may look poor. 
the \emph{characters} that don't fit into the line may continue 
at left end of the window or screen, disturbing the appearance 
of a ``table"---unless you use package option |[wrap]|. 
The latter requires the \ctanpkgref{hardwrap} package by 
Will Robertson and Kevin Godby 
(``\Wikienref{not invented here}"). This package tries to determine 
the screen width by some subtle tests, and until it finds something 
better, it assumes a width of 80 characters (I suppose). 
'hardwrap' does \wikienref{Word wrap}{\emph{word wrapping}}, 
i.e., it doesn't just put \emph{characters} not fitting into the 
next line, but entire \emph{words}. Moreover, it allows inserting
some ``newline sequence" before the first word that is too much, 
and we use this feature here to put the next word into the 
\emph{caption column} rather than at the beginning of the next line.
(Details and implementation are in Section~\ref{sec:opt}.)

If you are not happy with the column width that 'hardwrap' 
chooses, but want to assume your own width <max-line-chars>
(e.g., your width, measured by your doctor, divided by the 
 width of one character),
compute its difference <max-line-chars-minus-one> to 1
(maybe by your electronic calculator, or an emulation, or 
 a Lua script, cf.~\ctanpkgref{lualatex-doc}, or by 
 \ctanpkgref{bigintcalc}), and enter 
the 'hardwrap' instruction
\[|\setmaxprintline{<max-line-chars-minus-one>}|\] 
when 'hardwrap' or 'nicefilelist' have been loaded 
\emph{and} before the internal macro `\@dofilelist' is run
(which happens at the end of the document or when 
 \ctanpkgref{myfilist}'s `\ListInfos' is issued, for instance).

\subsection{Usage and Samples with 'myfilist.sty'}
\label{sec:myfilist}
\subsubsection{Basically}
In order to get a reduced and/or rearranged list of file infos 
with the \ctanpkgref{myfilist} package, 
`nicefilelist.sty' must be loaded earlier than 
`myfilist.sty'. This is due to a kind of limitation of the latter, 
 it \emph{issues} `\listfiles' (\TODO). 
Therefore \cs{listfiles} must be modified earlier---or \emph{issued} earlier, 
in this case the \cs{listfiles} in `myfilist.sty' does nothing.
The file `SrcFILEs.txt' accompanying the---first---distribution of 
'nicefilelist' %% corr., was `longname'; `first' etc. 2012/10/29
was generated by running the following file `srcfiles.tex' with \LaTeX:
\begin{quotation}\tt\small
\expandafter\def\expandafter\{\expandafter{\string{}
\expandafter\def\expandafter\}\expandafter{\string}}
\obeyspaces\obeylines
\cs{ProvidesFile}\{srcfiles.tex\}[2012/03/23
 ~                           file infos -\empty> SrcFILEs.txt]
\cs{RequirePackage}\{nicefilelist\}
\%\% INSERT MODIFICATIONS OF INITIAL 
\%\% \verb+`nicefilelist'/`monofill'+ SETTINGS HERE!
\cs{RequirePackage}\{myfilist\}
\%\% documentation:
\cs{ReadFileInfos}\{nicefilelist\}
\%\% demonstration:
\cs{ReadFileInfos}\{provonly.fd,wrong.prv,empty.f\}
\% \cs{ReadFileInfos}\{utopia.xyz\}
\%\% present file:
\cs{ReadFileInfos}\{nicefilelist\}
\cs{ReadFileInfos}\{srcfiles\}
\cs{ListInfos}[SrcFILEs.txt]
\end{quotation}
Note the lines where to place \strong{custom} modifications of settings 
for alignment (Section~\ref{sec:templates}) or failure displays 
(Section~\ref{sec:fail}).

The previous code mentions the following files:
\begin{description}
  \item[`provonly.fd'] has a proper `\ProvidesFile' line without date, 
                       for seeing what happens in the date and version columns. 
                       It also was a test for the case that there are fewer 
                       characters than a date has, and there is no blank space.
  \item[`wrong.prv']   has a `\ProvidesFile' line with wrong file name.
  \item[`empty.f']     just is an empty file.
  \item[`utopia.xyz']  is not present at all, you get an error when you remove 
                       the comment mark.
\end{description}
Moreover, my `.tex' files have dates, but not version numbers, 
so you see what happens then: 
\vskip\topsep
\begin{small}\tt
\obeyspaces\obeylines
~    *File List*
nicefilelist.sty  2012/03/23  v0.1   \rlap{more file list alignment (UL)}
~   monofill.sty  2012/03/19  v0.1a  monospace alignment (UL)
~   myfilist.sty  2011/01/30  v0.3a  \rlap{\cs{listfiles} -- mine only (UL)}
~   readprov.sty  2010/11/27  v0.3   \rlap{file infos without loading (UL)}
nicefilelist.tex  2012/03/23   --    \rlap{documenting nicefilelist.sty}
~   provonly.fd    --  -- --   --    such 
~      wrong.prv   * NOT FOUND *
~      empty.f     * NOT FOUND *
~   srcfiles.tex  2012/03/23   --    file infos -> SrcFILEs.txt
~    ***********
~
~List made at 2012/03/23, 10:31
~from script file srcfiles.tex
\end{small}
%% TODO update example!?
%%
\subsubsection{More Generally and Shorthand}
\label{sec:short}

\AddQuotes                          %% 2012/10/11

In the above example, the 'myfilist' command `\EmptyFileList'
was missing---it was not intended there. Usually however,   %% is -> was 2012/10/29
it \emph{is} intended, i.e., the following sequence of 
lines is wanted:
\begin{quotation}\tt\small
\expandafter\def\expandafter\{\expandafter{\string{}
\expandafter\def\expandafter\}\expandafter{\string}}
\obeyspaces\obeylines
\cs{RequirePackage}[r]\{nicefilefilelist\}
\cs{MFfieldtemplate}\{f-base\}[<longest-name>]
\cs{RequirePackage}\{myfilist\}
\cs{EmptyFileList}[<read-again-files>]
\end{quotation}
Here you also see usage of package option |[r]| for 
release numbers and the adjustment 
   \[|\MFfieldtemplate{f-base}{<longest-name>}|\]
according to Section~\ref{sec:templates}. 

With v0.5, the last three code lines in the snippet above 
can be replaced by 
\[|\MaxBaseEmptyList{<longest-name>}[<read-again-files>]|\]
---``optionally" without `[<read-again-files>]'.
This may save the user from worrying about usage 
with 'myfilist'. 

\DontAddQuotes

'nicefilelist' formats file lists nicely 
even when base filenames have eight characters at most, 
what \LaTeX's original `\listfiles' was made for.
v0.6 simplifies this case by a star version of `\MaxBaseEmptyList':
\pagebreak[2]
\[|\MaxBaseEmptyList*|\]
works like `\MaxBaseEmptyList{nicefile}' (eight characters)---still, 
optional `[<read-again-files>]' may follow. This feature is demonstrated 
with \ctanpkgref{inputtrc} v/r0.3.

\subsubsection{Sample with Wrapped Caption Column}  %% 2012/10/30

\AddQuotes

The most recent version of the accompanying `SrcFILEs.txt' contains 
the following:
\bigskip
\begingroup \footnotesize
\hrule
\verbatiminput{SrcFILEs.txt}
\hrule
\endgroup
\bigskip

This exemplifies
\begin{enumerate}
  \item \strong{wrapping} of `provonly.fd''s and \ctanpkgref{kvsetkeys}\file{.sty}
        file info within the \strong{caption} column 
        using 'nicefilelist''s `[wrap]' option,
  \item inserted ``\strong{comments}" from \ctanpkgref{myfilist}'s `\FileListRemark', 
  \item a file `nicefilelist.RLS' for a \strong{release summary}. This is to track 
        what has happened most recently, whether the most recent release 
        has been installed (system-wide), or (for me) whether most recent versions 
        of package and documentation have been released. 
        When such an `.RLS' file is installed together with packages 
        in the `tex' subtree of a \acro{TDS}, the release summary can be 
        accessed quickly as a \strong{terminal display}
        by one of the packages \ctanpkgref{ltxfileinfo}, 
        \ctanpkgref{latexfileversion}, or \ctanpkgref{typeoutfileinfo}.
        One aim of the `[wrap]' option is allowing longer ``release captions" 
        (looking fine in the package file list) than fit into a small part 
        of a single line.
\end{enumerate} 
%
The above `SrcFILEs.txt' has been generated from the following version 
of the \TeX\ script `srcfiles.tex':
\bigskip
\hrule
\verbatiminput{srcfiles.tex}
\smallskip
\hrule

\DontAddQuotes

\section{Implementation}
\subsection{Package File Header (Legalese)}
\NeedsTeXFormat{LaTeX2e}[1994/12/01]
\ProvidesPackage{nicefilelist}[2012/12/12 v0.7a
                               more file list alignment (UL)]
%% Copyright (C) 2012 Uwe Lueck,
%% http://www.contact-ednotes.sty.de.vu
%% -- author-maintained in the sense of LPPL below --
%%
%% This file can be redistributed and/or modified under
%% the terms of the LaTeX Project Public License; either
%% version 1.3c of the License, or any later version.
%% The latest version of this license is in
%%     http://www.latex-project.org/lppl.txt
%% We did our best to help you, but there is NO WARRANTY.
%%
%% Please report bugs, problems, and suggestions via
%%
%%   http://www.contact-ednotes.sty.de.vu
%%
%% === Alignment Settings        ===
%% \label{sec:templates}
%% We use the \ctanpkgref{monofill} package for alignment of plain text:
\RequirePackage{monofill}[2012/10/29]
%% See its documentation for details.
%% %% <- added 2012/10/29 ->
%% The `[wrap]' option provided by 'nicefilelist' v0.7 requires 
%% 'monofill' v0.2 as of 2012-10-29.
%%
%% We support three alignment ``fields" according to the terminology of \pkg{monofill}. 
%% Their ids are |f-base| for base filenames, |f-ext| for filename extensions, 
%% % |f-date| for the revision date, 
%% and |f-version| for the revision version id of a file %%% , 
%% % the latter two 
%% as read from `\ProvidesFile', `\ProvidesPackage', 
%% or `\ProvidesClass' command in the file.
%% Initial settings for them are following.
%% For modifying them, load 'nicefilelist.sty', 
%% then type your own settings, then issue `\listfiles'
%% or load 'myfilist.sty'.
\MFfieldtemplate{f-base}{nicefilelist}
\MFfieldtemplate{f-ext}{tex}
\MFfieldtemplate{f-version}{v0.11a}
%% We are not supporting version numbers greater than 9 at present---sorry! (TODO)
%% 
%% |\NFLspaceI|, |\NFLspaceII|, and |\NFLspaceIII| 
%% determine the space between the four 
%% columns for names, dates, versions, and ``captions":
\newcommand*{\NFLspaceI}  { \space}
\newcommand*{\NFLspaceII} { \space}
\newcommand*{\NFLspaceIII}{ }
%% 
%% === Failure Displays          ===
%% \label{sec:fail}                 %% was failure 2012/10/29
%% |\NFLnodate| is displayed in place of a file date if it seems 
%% not to be given (configurable):
\newcommand*{\NFLnodate}{ -- \space-- --}
%% |\NFLnoversion| likewise---however, for alignment, 
%% each wanted space must be specified as `\space'
%% (not just a code blank space). 
%% It may need adjustment (by `\renewcommand') when
%% `\MFfieldtemplate{f-version}' is modified:
\newcommand*{\NFLnoversion}{\space--}
%% |\NFLnotfound| is for files with wrong or no 
%% \cs{Provides.\empty..} command:
\newcommand*{\NFLnotfound}{ * NOT FOUND *}
%% 
%% === Package Options           ===
%% \label{sec:opt}
%% v0.4 offers package option |[r]| that allows strings with `r' %% |...| 2012/10/30
%% in place of `v', for ``release." `\NFL@v@digit''s definition
%% therefore depends ... we use |\@listfiles| for a ``message" 
%% there. For the original restricted functionality, it expands 
%% to |\NFL@false|. 
\def\@listfiles{\noexpand\NFL@false}
%% Package option `[r]' carries out another test instead.
%% See the accompanying file `SrcFILEs.txt' to see the effect. 
%% TODO: update example!?
\DeclareOption{r}{%
  \def\@listfiles{%
    {\noexpand\NFL@ifx@kbl##1r%
        {\noexpand\NFL@digits##2\noexpand\@nnil}%
        \noexpand\NFL@false}%
  }%
}
%% v0.7 offers package option |[wrap]| for automatical wrapping within %% |...| 2012/10/30
%% the "captions" column, based on Will Robertson's and Kevin Godby's 
%% \ctanpkgref{hardwrap} package. The difference between this option 
%% and the functionality without is controlled by the macro 
%% `\NFL@filerow'. 
%% %% 2012/10/30:
%% \emph{Without} it expands to `\typeout'
\newcommand*{\NFL@filerow}{\typeout}
%% ---`\let' does'nt work with 'myfilist''s redefinition of 
%% `\typeout'.
%% \emph{With} `[wrap]', `\NFL@filerow' applies 'hardwrap''s 
%% `\HardWrap':
\DeclareOption{wrap}{%
  \renewcommand*{\NFL@filerow}[1]{%
    \HardWrap\typeout\hw@maxprintline\relax{^^J%
      \MFrightinfield\space{f-base} %
      \MFleftinfield \space{f-ext}%
      \NFLspaceI\@spaces\space\@spaces\space \NFLspaceII
      \MFrightinfield\space{f-version}\NFLspaceIII}{%
      #1}}%
%% Alignment of filenames with 'hardwrap' seems to need
  \renewcommand*{\MFfillelement}{\MFotherspace}
%% from \ctanpkgref{monofill} v0.2.             %% was `monospace' 2012/12/12
}
%% The display width is controlled by 'hardwrap''s counter 
%% `\hw@maxprintline'. Unless 'hardwrap' finds something special, 
%% its content is 79, corresponding to a display width of 
%% 80 characters (I believe---counting the leftmost character 
%%  as \qtd{0}, as editors like to do). You can choose a different 
%% content value <max-char-col> by 'hardwrap''s
%% \[|\setmaxprintline{<max-char-col>}|\]
\ProcessOptions
%% The next `\ifx' is to check whether `[wrap]' has been demanded 
%% and 'hardwrap' is needed:
\ifx\NFL@filerow\typeout \else
  \RequirePackage{hardwrap}
\fi
%% 
%% === Safe Tests                === %% below options 2012/10/29
%% \label{sec:safe}
%% For fairly safe tests, we brief{}ly use an exotic version of `Q'
%% (similarly to \ctanpkgref{ifmptarg} and \ctanpkgref{url}):
\catcode`\Q=7 \let\NFL@criterion=Q \catcode`\Q=11
%% %% 2012/03/27:
%% It appears to me that expandable tests like the ones employed here 
%% never are perfectly safe; you only can say that it is safe with a 
%% source meeting certain conditions. \ctanpkgref{fifinddo} originally 
%% was made for ``plain text," to be read from files without assigning 
%% \TeX's special category codes. \emph{Here} we assume that the source 
%% (text in \cs{Provides.\empty..} arguments) will never contain such a 
%% ``funny `Q'".
%%
%% === Redefining \cs{listfiles} ===
%% Similarly to original \LaTeX, |\listfiles| carries almost everything 
%% that is needed for the file list only. 2012-10-29: little point 
%% in this, perhaps, in that the package should be loaded when 
%% running `\listfiles' is intended---TODO. Or maybe it is loaded 
%% \emph{just in case?}
\renewcommand*{\listfiles}{%
  \let\listfiles\relax
%% ---this clears memory. Now \LaTeX\ doesn't collect file names for 
%% &\listfiles when &\@listfiles is undefined, therefore 
%   \let\@listfiles\relax
%% % although we don't use it.
%% % (TODO use it in place of some `\NFL@'...?)
%% ... postponed for v0.4 ...
%% \vskip\smallskipamount\noindent
%% |\@dofilelist| is executed by the standard \LaTeX\ `\enddocument'
%% macro or by `\ListInfos' from the \ctanpkgref{myfilist} package.
  \def\@dofilelist{%
%% ``Title:"
     \typeout{^^J           %% trick 2012/03/29 vv
              \MFrightinfield{*File Lis}{f-base}t*}%
     \@for\@currname:=\@filelist\do{%
%% This starts the loop through the list of files
       \filename@parse\@currname
       \edef\filename@ext{%
          \ifx\filename@ext\relax tex\else\filename@ext\fi}%
%% Like \LaTeX's &\reserved@b:
       \expandafter\let\expandafter\@tempb
            \csname ver@\filename@base.\filename@ext\endcsname
%% According to `source2e.pdf', &\filename@area may be a directory. 
%% Trying support of this is seems to be a new feature with v0.2---not 
%% tested, TODO!
       \edef\@tempa{\filename@area\filename@base}%
%% Actually I would like to be able to do even the filename parsing 
%% expandably---for all systems, `texsys.cfg'!?? TODO
       \NFL@filerow{%
%% Now all parsing and checking must be expandable.
         \NFL@make@macro@arg\MFrightinfield\@tempa      {f-base}.%
         \NFL@make@macro@arg\MFleftinfield \filename@ext{f-ext}%
         \NFLspaceI
         \NFL@ifx@kbl\@tempb\relax\NFLnotfound{%
           \NFL@make@macro@arg\NFL@space@split\@tempb
                                              \NFL@maybe@three
                                              \NFL@date@or@rest
         }%
       }%
     }%
%% The line of stars:
     \typeout{                        %% trick vvv 2012/03/29
              \MFrightinfield{********}{f-base}***^^J}%
%% TODO or more stars as with 'longnamefilelist'?
  }%
%% This finishes the definition of `\@dofilelist'. 
%% % The definitions of macros called from `\@dofilelist' are following.
%% \[|\NFL@make@macro@arg<cmd-1><cmd-2>|\] results in 
%% `<cmd-1>{<t-list>}' where <t-list> is the one-step expansion of 
%% <cmd-2>:
  \def\NFL@make@macro@arg##1##2{\expandafter##1\expandafter{##2}}%
%% |\NFL@space@split{<token-list>}{<spaced>}{<unspaced>}|
%% passes prefix and suffix as arguments to <spaced> 
%% if a space token is within <token-list>, otherwise 
%% <unspaced> gets the original <token-list> as single 
%% argument. The latter is useful here where <token-list> 
%% becomes visible only by an &\expandafter.
%% The following construction is discussed more generally 
%% in the \ctanpkgref{bitelist} package.
  \def\NFL@space@split##1{%
    \NFL@return@space@split##1\@nil: \NFL@criterion\@nil\@nil@{##1}}%
%% `\NFL@return@spaces@split' essentially has \emph{three} parameters 
%% delimited by ` ', `\@nil', and `\@nil' again.
  \def\NFL@return@space@split##1 ##2\@nil##3\@nil@##4##5##6{%
    \NFL@ifx@kbl\NFL@criterion{##2}%
%% If #2 is empty, `\NFL@ifx@kbl' (as of v0.3) compares 
%% `\NFL@criterion' (criterion indicating ``unspaced") with 
%% `\expandafter'. This only happens when the space is the last thing 
%% in <token-list>, and `<spaced>' is chosen correctly.
    {##6{##4}}{##5{##1}{##2}}}%
%% |\NFL@ifx@kbl{<token>}{<maybe-token>}{<ifx>}{<unlessx>}| 
%% as of v0.3  should save some tokens, in some longer run, 
%% especially if we want to add nestings---cf. \file{source2e.pdf}
%% for ``Kabelschacht."
  \def\NFL@ifx@kbl##1##2{%
    \ifx##1##2\expandafter  \@firstoftwo
      \else   \expandafter \@secondoftwo \fi}%
%% Dealing with |\NFL@date@or@rest{<token-list>}| before `\NFL@maybe@three':
  \def\NFL@date@or@rest##1{%
    \NFL@if@date{##1}{##1}{\NFL@no@date@version##1}}%
%% |\NFL@if@date{<token-list>}{<yes>}{<no>}| ...
  \def\NFL@if@date##1{\NFL@slashes##1\NFL@xi xyzxyzxyzx\@nil}% 
%% |\NFL@slashes| checks that there are slashes at the expected places:
  \def\NFL@slashes##1##2##3##4##5##6##7##8{%
    \NFL@ifx@kbl##5/%
      {\NFL@ifx@kbl##8/\NFL@ten@only\NFL@false}%
      \NFL@false
%% This especially happens when <token-list> is empty.
%% Digit candidates back:
    {##1##2##3##4##6##7}}%
%% If the word is a date, we now have taken 6 of the 8 digits. 
%% \[|\NFL@ten@only{<digits>}<digit><digit>Q|\]
%% takes the two remaining and then a thing that should be `Q' 
%% in the funny sense of Sec.~\ref{sec:safe}.
  \def\NFL@ten@only##1##2##3##4{%
    \NFL@ifx@kbl\NFL@xi##4\NFL@digits\NFL@false
%% Finally checking digits:
    ##1##2##3\@nnil}%
%% |\NFL@digits<token>| is a loop through single tokens:
  \def\NFL@digits##1{% 
    \NFL@ifx@kbl##1\@nnil\NFL@true{%
      \NFL@if@digit@code##1<0\NFL@false{%
        \NFL@if@digit@code##1>9\NFL@false\NFL@digits
      }%
    }%
  }%
%% |\NFL@if@digit@code<char-1><relation><char-2><fits><bad>|:
  \def\NFL@if@digit@code##1##2##3{%
    \ifnum`##1##2`##3 \expandafter  \@firstoftwo
      \else           \expandafter \@secondoftwo \fi}%
%% |\NFL@false| skips further candidates and dummies and chooses <no>:
  \def\NFL@false##1\@nil{\@secondoftwo}%
%% |\NFL@true| skips further candidates and dummies and chooses <yes>:
  \def\NFL@true##1\@nil{\@firstoftwo}%
%% We don't support version without date, therefore run 
%% |\NFL@no@date@version| as soon as we find that the file info 
%% does not start with a date:
  \def\NFL@no@date@version{%
    \NFLnodate\NFLspaceII\NFLnoversion@\NFLspaceIII}%
%% |\NFLnoversion@| adds filler to &\NFLnoversion:
  \def\NFLnoversion@{%
    \NFL@make@macro@arg\NFL@place@version\NFLnoversion}%
%% |\NFL@maybe@three{<word-1>}{<rest>}| looks whether <word-1> 
%% is a date. If it is, it is written to screen, and then we look
%% if <rest> contains a version id. Otherwise ```<word-1>_<rest>'"
%% is considered a ``caption" only.
  \def\NFL@maybe@three##1##2{%
    \NFL@if@date{##1}%
                {##1\NFLspaceII 
                 \NFL@space@split{##2}%
                                 \NFL@maybe@version@rest
                                 \NFL@version@or@rest}%
                {\NFL@no@date@version##1 ##2}}%
%% |\NFL@version@or@rest{<token-list>}|:
  \def\NFL@version@or@rest##1{%
    \NFL@if@version{##1}%
                   {\NFL@place@version{##1}}%
                   {\NFLnoversion@\NFLspaceIII##1}}%
%% |\NFL@if@version{<token-list>}{<yes>}{<no>}|:
  \def\NFL@if@version##1{\NFL@v@digit##1xy\@nil}%
%% TODO: At applications you see how some tokens could be saved.
%% On the other hand, the macros are more transparent in the 
%% present way.
%% \vskip\smallskipamount\noindent
%% |\NFL@v@digit{<t1>}{<t2>}{<rest>}| checks whether the first thing
%% is a `v' and the second a digit---unless package option `[r]' 
%% was chosen. v0.4 uses `\edef' for choosing:
  \edef\NFL@v@digit##1##2##3\@nil{%
    \noexpand\NFL@ifx@kbl##1v%
        {\noexpand\NFL@digits##2\noexpand\@nnil}%
%% |\@listfiles| will either expand to the original `\NFL@false'
%% or to a test on `r':
        \@listfiles
    \noexpand\@nil}%
  \let\@listfiles\relax
%% |\NFL@place@version{<token-list>}| adds filler to version id:
  \def\NFL@place@version##1{\MFleftinfield{##1}{f-version}}%
%% |\NFL@maybe@version@rest{<list-1>}{<list-2>}|:
  \def\NFL@maybe@version@rest##1##2{%
    \NFL@if@version{##1}%
                   {\NFL@place@version{##1}\NFLspaceIII##2}%
                   {\NFLnoversion@\NFLspaceIII##1 ##2}}%
}
%% === Shorthand for 'myfilist'  ===
%% \[|\MaxBaseEmptyList{<longest-name>}[<read-again-files>]|\]
%% (v0.5) or 
%% \[|\MaxBaseEmptyList*[<read-again-files>]|\]
%% (v0.6) as described in Section~\ref{sec:short}: 
\newcommand*{\MaxBaseEmptyList}{%
    \@ifstar{\maxBaseEmptyList{abcdabcd}}\maxBaseEmptyList}
\newcommand*{\maxBaseEmptyList}[1]{%
    \MFfieldtemplate{f-base}{#1}%
    \RequirePackage{myfilist}\EmptyFileList}
%% So |\maxBaseEmptyList| is like former `\MaxBaseEmptyList'
%% without expecting a star---available to users.
%%
%% === Leaving the Package File  ===
\endinput
%%
%% === VERSION HISTORY ===

v0.1   2012/03/20   started 
       2012/03/22   almost ready
       2012/03/23   debugging; \NFLspaceI etc.; 
                    documentation completed 

v0.2   2012/03/24   file info processed by \typeout - start
       2012/03/25   trying, debugging
       2012/03/26   continued; \NFL@place@version, \NFLnoversion@; 
                    works, reordered; another fix about Q -> \@empty
       2012/03/27   undone the latter, explained; improved remarks on
                    \@listfiles
       2012/03/29   alignment of title/stars with base<11

v0.30  2012/05/18f. \NFL@ifx@kbl in \NFL@return@space@split
       2012/05/20   all \ifx reimplemented, old code kept
       STORED INTERNALLY
v0.31  2012/05/20   removing old code - STORED INTERNALLY
v0.32  2012/05/20   removing \NFL@xpxpxp; replacing \NFL@after@false
                    by \NFL@ifnum@kbl, keeping old code
       STORED INTERNALLY
v0.33  2012/05/20   removing old code; added 3 %s
       STORED INTERNALLY
v0.4   2012/05/20   option [r]
v0.5   2012/09/30   \MaxBaseEmptyList
v0.6   2012/10/03   \MaxBaseEmptyLists first arg. only optional
       2012/10/11   ... bad with 2nd opt. arg., *
v0.7   2012/10/13   "updating" date in \Provides...!
       2012/10/28   \HardWrap first try 
       2012/10/29   \HardWrap newline material -> [wrap]
                    sec:test below sec:opt, mentioning `url'
       2012/10/30   correcting \NFL@filerow without wrapping, 
                    doc.: |...| in sec:opt
v0.7a  2012/12/12   doc. monospace -> monofill


% \pagebreak          %% 2012/09/30 rm. 2012/10/29
\section{Credits}
\begin{enumerate}   %% 2012/05/20
  \item It was \ctanpkgauref{muench-hm}{\textsc{Martin M\"unch}} 
        who pointed out the shortcomings 
        of 'longnamefilelist' that the present package addresses---thanks!
  %% 2012/05/20:
  \item For \textsc{Alois Kabelschacht}---whose idea in 
        TUGboat~\textbf{8}~\#2\footnote{\tugbartref{tb08-2/tb18kabel}{%
            ``&\expandafter\ vs.\               %% .\ 2012/12/12
            &\let\ and &\def\ in Conditionals 
            and a Generalization of PLAIN's &\loop,"}
            TUGboat Vol.~8 (1987), No.~2, pp.~184f. 
            (\urlhttpref{tug.org/TUGboat/tb08-2/tb18kabel.pdf})}
        is used for v0.3---cf.\ the \ctanpkgref{dowith} documentation.
\end{enumerate}

\section{Missing}
% \begin{enumerate} %% rm. 2012/10/29
%   \item 
        The package once might provide \ctanpkgref{keyval}-style 
        optional arguments for \cs{listfiles} or even call \cs{listfiles} 
        automatically with 'keyval' package options.
%   %% 2012/05/18:
%   \item Another idea from \textsc{Martin Muench}: wrapping inside caption column. 
%         Can \ctanpkgref{hardwrap} help?
% \end{enumerate}

\end{document}

VERSION HISTORY

2012/03/20  for v0.1    started
2012/03/22              trying something 
2012/03/23  for v0.1a   without "aligning the dots" etc.
2012/05/18  for v0.30   Legalize -> Legalese; add. TODO "wrapping"; 
                        ref. to `latexfileinfo-pkgs'
2012/05/20  for v0.31   mention Kabelschacht
2012/05/20  for v0.4    on new package option [r]
2012/09/30  for v0.5    \MaxBaseEmptyList, doc. \pagebreak
2012/10/11  for v0.6    \MaxBaseEmptyList*, \AddQuotes
2012/10/29  for v0.7    "M�nch", one item less missing, MDabstract, 
                        usage restructured and extended
2012/10/30              \listfiles
2012/12/12  for v0.7a   vs.\ 
