\ProvidesFile{blog.tex}[2013/01/04 documenting blog.sty]
\title{\textsf{blog.sty}\\---\\%
       Generating \HTML\ Quickly with \TeX\thanks{This 
       document describes version 
       \textcolor{blue}{\UseVersionOf{\jobname.sty}} 
       of \textsf{\jobname.sty} as of \UseDateOf{\jobname.sty}.}}
% \listfiles 
{ \RequirePackage{makedoc} 
  \ProcessLineMessage{} 
  %% three section levels 2012/08/07:
  \renewcommand\mdSectionLevelOne  {\string\subsection}
  \renewcommand\mdSectionLevelTwo  {\string\subsubsection}
  \renewcommand\mdSectionLevelThree{\string\paragraph}
  \MainDocParser{\SectionLevelThreeParseInput}
  \HeaderLines{16}              %% 2012/10/03
  \MakeSingleDoc{blog.sty}
  \HeaderLines{18}              %% 2012/11/28
  \MakeSingleDoc{blogligs.sty}
  \HeaderLines{18}              %% 2012/11/28
  \MakeSingleDoc{markblog.sty}
  \HeaderLines{18}              %% 2012/10/05
  \MakeSingleDoc{lnavicol.sty}
  \HeaderLines{17}              %% 2012/10/05
  \MakeSingleDoc{blogdot.sty}
}
\documentclass[fleqn]{article}%% TODO paper dimensions!?
\sfcode`-=1001      %% 2011/10/15
\ProvidesFile{makedoc.cfg}[2011/06/27 documentation settings] 

\author{Uwe L\"uck\thanks{\url{http://contact-ednotes.sty.de.vu}}}
% \author{Uwe L\"uck---{\tt http://contact-ednotes.sty.de.vu}}

%% hyperref:
\RequirePackage{ifpdf}
\usepackage[%
  \ifpdf
%     bookmarks=false,          %% 2010/12/22
%     bookmarksnumbered,
    bookmarksopen,              %% 2011/01/24!?
    bookmarksopenlevel=2,       %% 2011/01/23
%     pdfpagemode=UseNone,
%     pdfstartpage=10,
%     pdfstartview=FitH,
    citebordercolor={ .6 1    .6},
    filebordercolor={1    .6 1},
    linkbordercolor={1    .9  .7},
     urlbordercolor={ .7 1   1},   %% playing 2011/01/24
  \else
    draft
  \fi
]{hyperref}

\RequirePackage{niceverb}[2011/01/24] 
\RequirePackage{readprov}               %% 2010/12/08
\RequirePackage{hypertoc}               %% 2011/01/23
\RequirePackage{texlinks}               %% 2011/01/24
\makeatletter
  \@ifundefined{strong} 
               {\let\strong\textbf}     %% 2011/01/24
               {} 
  \@ifundefined{file} 
               {\let\file\texttt}       %% 2011/05/23
               {} 
\makeatother

\errorcontextlines=4
\pagestyle{headings}

\endinput

 %% shared formatting settings
\usepackage{filesdo} \MDfinaldatechecks             %% 2012/12/20
\ReadPackageInfos{blog}
  %% \tagcode seems to be a quite recent pdfTeX primitive, 
  %% cf. microtype.pdf ... %% 2010/11/06
\newcommand*{\xmltagcode}[1]{\texttt{<#1>}}
\newcommand*{\HTML}{\acro{HTML}}            %% 2011/09/08
\newcommand*{\CSS}{\acro{CSS}}              %% 2011/11/09
\newcommand*{\secref}[1]{Sec.~\ref{sec:#1}} %% 2011/11/23
\providecommand*{\LuaTeX}{Lua\TeX}          %% 2012/12/20
\sloppy
\begin{document}
\maketitle
\begin{abstract}\noindent
'blog.sty' provides \TeX\ macros for generating web pages, 
based on processing text files using the 'fifinddo' package. 
Some \LaTeX\ 
commands %%% command names 
are redefined to access their \HTML\ 
equivalents, other new macro names ``quote" the names of \HTML\ elements. 
The package has evolved in several little steps 
each aiming at getting pretty-looking ``hypertext" \textbf{notes}
with little effort, 
where ``little effort" also has meant avoiding studying 
documentation of similar packages already existing. 
[\textcolor{blue}{TODO:} list them!]
% Version v0.3 is the remainder of v0.2 after moving some stuff 
% to 'fifinddo.sty' (especially `\CopyFile'); 
% moreover, the new `\BlogCopyFile' replaces empty source lines 
% by \HTML's \xmltagcode{p} (starting a new paragraph).---Real 
% \emph{typesetting} from the same `.tex' source 
% (pretty printable output) has not been tried yet.
%% <- 2011/01/24 ->
The package %%% rather 
\emph{``misuses"} \TeX's macro language 
for generating \HTML\ code and entirely \emph{ignores} 
\TeX's typesetting capabilities.%%%---What about 
% such a ``small" \TeX\ with macros only and 
% \emph{no} typesetting capabilities ...!?
---'lnavicol.sty' adds a more \strong{professional} look 
(towards CMS?), and 'blogdot.sty' uses 'blog.sty' 
for \HTML\ \strong{beamer} presentations.
\end{abstract}
\tableofcontents

\section{Installing and Usage}
The file 'blog.sty' is provided ready, 
\strong{installation} only requires 
putting it somewhere where \TeX\ finds it 
(which may need updating the filename data 
 base).\urlfoot{ukfaqref}{inst-wlcf}

\strong{User commands} are described near their implementation below.

However, we must present an \strong{outline} of the procedure 
for generating \HTML\ files: 

At least one \strong{driver} file and one \strong{source} file are 
needed.

The \strong{driver} file's name is stored in `\jobname'. 
It loads 'blog.sty' by 
\begin{verbatim}
  \RequirePackage{blog}
\end{verbatim}
and uses file handling commands from 'blog.sty' and 
\CtanPkgRef{nicetext}{fifinddo}
(cf. `mdoccheat.pdf' from the \ctanpkgref{nicetext} bundle).\urlfoot{CtanPkgRef}{nicetext} 
%% <- \urlfoot 2012/11/30 
It chooses \strong{source} files and the name(s) for the resulting 
\HTML\ file(s). It may also need to load local settings, such as 
%% 2012/11/29: 
`\uselangcode' with the \ctanpkgdref{langcode} package  %% dref 2012/11/30
and settings for converting the editor's text encoding 
into the encoding that the head of the resulting \HTML\ file 
advertises---or into \HTML\ named entities 
(for me, `atari_ht.fdf' has done this).

The driver file could be run a terminal dialogue in order to choose source 
and target files and settings. So far, I rather have programmed a 
dialogue just for converting UTF-8 into an encoding that my 
Atari editor \textsc{xEDIT} can deal with. 
I do not present this now because it was conceptually mistaken, 
I must set up this conversion from scratch some time.
% [TODO: present in 'nicetext'].    %% 2011/01/24

The \strong{source} file(s) should contain user commands defined below 
to generate the necessary \xmltagcode{head} section and the 
\xmltagcode{body} tags. 

\section{Examples}
\subsection{Hello World!}
This is the \strong{source} code for a ``Hello World" example, 
in `hellowor.tex':
\MDsamplecodeinput{hellowor}
The \HTML\ file `hellowor.htm' is generated from `hellowor.tex'
by the following \strong{driver} file `mkhellow.tex':
\pagebreak[2]
\MDsamplecodeinput{mkhellow}

  \iffalse                                  %% 2012/11/29
\subsection{A Very Plain Style}
My ``\TeX-generated 
pages"{\foothttpurlref{www.webdesign-bu.de/%
                       uwe\string_lueck/texmap.htm}}
use a \strong{driver} file `makehtml.tex'. 
To choose a page to generate, I ``uncomment"ed just one 
of several lines that set the ``current conversion job" 
from a list (for some time). 
I choose the example of a simple ``site map:" 
`texmap.htm' is generated from \strong{source} file 
`texmap.tex'.---More recently however, I have started to 
read the job name and perhaps extra settings from a file 
`jobname.tex' that is created by a Bash script.

In order to make it easier for the reader to see what is essential, 
I~have moved many `.cfg'-like extra definitions into a file 
`texblog.fdf'. Some of these definitions may later move into 
`blog.sty'. You should find `makehtml.tex', `texmap.tex', and 
`texblog.fdf' in a directory `demo/texblog' 
(or `texblog.fdf' may be together with the `.sty' files), 
perhaps you can use them as templates.

\begingroup
  \MakeOther\|
%   \MakeOther\`\MakeOther\'  %% disables \tt! 2011/09/08
  \MakeOther\<
  \MakeActive\� \def�{\"o}                  %% 2011/10/10
  \MakeActive\� \def�{\"u}
  \hfuzz=\textwidth \advance \hfuzz by 28pt
\subsubsection{Driver File `makehtml.tex'}
 %% <- TODO \file needs protection for PDF 2011/09/08
 \enlargethispage{1\baselineskip}
  \listinginput[5]{1}{CTAN/morehype/demo/texblog/makehtml.tex}
\subsubsection{Source File \texttt{texmap.tex}}
  \listinginput[5]{1}{CTAN/morehype/demo/texblog/texmap.tex}
\endgroup

  \fi
\subsection{A Style with a Navigation Column}
\label{sec:example-lnavicol}
A style of web pages looking more professional 
% than                                %% rm. 2012/11/29
% `texmap.htm'                        %% was `texhax.hmt' 2011/09/02
(while perhaps becoming outdated) has a small navigation column 
on the left, side by side with a column for the main content. 
Both columns are spanned by a header section above and a footer 
section below. The package 'lnavicol.sty' provides commands 
`\PAGEHEAD', `\PAGENAVI', `\PAGEMAIN', `\PAGEFOOT', `\PAGEEND' 
(and some more) for structuring the source so that the code 
following `\PAGEHEAD' generates the header, the code following 
`\PAGENAVI' forms the content of the navigation column, etc.
Its code is presented in Sec.~\ref{sec:lnavicol}.
For real professionality, somebody must add some fine \acro{CSS}, 
and the macros mentioned may need to be redefined to use the `@class' 
attribute. Also, I am not sure about the table macros in 'blog.sty', 
so much may change later.

With things like these, can 'blog.sty' become a part of a 
``\Wikienref{content management system}" for \TeX\ addicts? 
This idea rather is based on the 
\wikideref{Content Management System}{\meta{German}} 
Wikipedia article.

As an example, I present parts of the source for my 
``home page"{\foothttpurlref{www.webdesign-bu.de/%
                             uwe\string_lueck/schreibt.html}}.
As the footer is the same on all pages of this style, 
it is added in the driver file `makehtml.tex'. 
`schreibt.tex' is the source file for generating `schreibt.html'.
You should find \emph{this} `makehtml.tex', a cut down version of 
`schreibt.tex', and `writings.fdf' with my extra macros for these pages 
in a directory 
`blogdemo/writings',                        %% blog 2012/11/30
hopefully useful as templates.

\begingroup
  \MakeActive\� \def�{\"a}                  %% 2011/10/05
  \MakeActive\� \def�{\"u}
  \hfuzz=\textwidth \advance \hfuzz by 10pt
  %% 2012/11/29 CTAN/morehype/demo -> blogdemo:
\subsubsection{Driver File `makehtml.tex'}
  \listinginput[5]{1}{blogdemo/writings/makehtml.tex}
\subsubsection{Source File `schreibt.tex'}
  \listinginput[5]{1}{blogdemo/writings/schreibt.tex}
\endgroup

\pagebreak                                  %% 2013/01/04
\section{The File \file{blog.sty}}          %% 2011/11/09
% \section{The File \texttt{blog.sty}}
% \section{The File {\tt blog.sty}} 
%% <- strange 2011/11/08 ->
% \section{The File `blog.sty'}             %% 2011/10/04 allow other files
\subsection{Preliminaries}                  %% 2012/10/03
\subsubsection{Package File Header (Legalese)} %% ize -> ese, subsub 2012/10/03
\ResetCodeLineNumbers
\ProvidesFile{blog.tex}[2013/01/04 documenting blog.sty]
\title{\textsf{blog.sty}\\---\\%
       Generating \HTML\ Quickly with \TeX\thanks{This 
       document describes version 
       \textcolor{blue}{\UseVersionOf{\jobname.sty}} 
       of \textsf{\jobname.sty} as of \UseDateOf{\jobname.sty}.}}
% \listfiles 
{ \RequirePackage{makedoc} 
  \ProcessLineMessage{} 
  %% three section levels 2012/08/07:
  \renewcommand\mdSectionLevelOne  {\string\subsection}
  \renewcommand\mdSectionLevelTwo  {\string\subsubsection}
  \renewcommand\mdSectionLevelThree{\string\paragraph}
  \MainDocParser{\SectionLevelThreeParseInput}
  \HeaderLines{16}              %% 2012/10/03
  \MakeSingleDoc{blog.sty}
  \HeaderLines{18}              %% 2012/11/28
  \MakeSingleDoc{blogligs.sty}
  \HeaderLines{18}              %% 2012/11/28
  \MakeSingleDoc{markblog.sty}
  \HeaderLines{18}              %% 2012/10/05
  \MakeSingleDoc{lnavicol.sty}
  \HeaderLines{17}              %% 2012/10/05
  \MakeSingleDoc{blogdot.sty}
}
\documentclass[fleqn]{article}%% TODO paper dimensions!?
\sfcode`-=1001      %% 2011/10/15
\ProvidesFile{makedoc.cfg}[2011/06/27 documentation settings] 

\author{Uwe L\"uck\thanks{\url{http://contact-ednotes.sty.de.vu}}}
% \author{Uwe L\"uck---{\tt http://contact-ednotes.sty.de.vu}}

%% hyperref:
\RequirePackage{ifpdf}
\usepackage[%
  \ifpdf
%     bookmarks=false,          %% 2010/12/22
%     bookmarksnumbered,
    bookmarksopen,              %% 2011/01/24!?
    bookmarksopenlevel=2,       %% 2011/01/23
%     pdfpagemode=UseNone,
%     pdfstartpage=10,
%     pdfstartview=FitH,
    citebordercolor={ .6 1    .6},
    filebordercolor={1    .6 1},
    linkbordercolor={1    .9  .7},
     urlbordercolor={ .7 1   1},   %% playing 2011/01/24
  \else
    draft
  \fi
]{hyperref}

\RequirePackage{niceverb}[2011/01/24] 
\RequirePackage{readprov}               %% 2010/12/08
\RequirePackage{hypertoc}               %% 2011/01/23
\RequirePackage{texlinks}               %% 2011/01/24
\makeatletter
  \@ifundefined{strong} 
               {\let\strong\textbf}     %% 2011/01/24
               {} 
  \@ifundefined{file} 
               {\let\file\texttt}       %% 2011/05/23
               {} 
\makeatother

\errorcontextlines=4
\pagestyle{headings}

\endinput

 %% shared formatting settings
\usepackage{filesdo} \MDfinaldatechecks             %% 2012/12/20
\ReadPackageInfos{blog}
  %% \tagcode seems to be a quite recent pdfTeX primitive, 
  %% cf. microtype.pdf ... %% 2010/11/06
\newcommand*{\xmltagcode}[1]{\texttt{<#1>}}
\newcommand*{\HTML}{\acro{HTML}}            %% 2011/09/08
\newcommand*{\CSS}{\acro{CSS}}              %% 2011/11/09
\newcommand*{\secref}[1]{Sec.~\ref{sec:#1}} %% 2011/11/23
\providecommand*{\LuaTeX}{Lua\TeX}          %% 2012/12/20
\sloppy
\begin{document}
\maketitle
\begin{abstract}\noindent
'blog.sty' provides \TeX\ macros for generating web pages, 
based on processing text files using the 'fifinddo' package. 
Some \LaTeX\ 
commands %%% command names 
are redefined to access their \HTML\ 
equivalents, other new macro names ``quote" the names of \HTML\ elements. 
The package has evolved in several little steps 
each aiming at getting pretty-looking ``hypertext" \textbf{notes}
with little effort, 
where ``little effort" also has meant avoiding studying 
documentation of similar packages already existing. 
[\textcolor{blue}{TODO:} list them!]
% Version v0.3 is the remainder of v0.2 after moving some stuff 
% to 'fifinddo.sty' (especially `\CopyFile'); 
% moreover, the new `\BlogCopyFile' replaces empty source lines 
% by \HTML's \xmltagcode{p} (starting a new paragraph).---Real 
% \emph{typesetting} from the same `.tex' source 
% (pretty printable output) has not been tried yet.
%% <- 2011/01/24 ->
The package %%% rather 
\emph{``misuses"} \TeX's macro language 
for generating \HTML\ code and entirely \emph{ignores} 
\TeX's typesetting capabilities.%%%---What about 
% such a ``small" \TeX\ with macros only and 
% \emph{no} typesetting capabilities ...!?
---'lnavicol.sty' adds a more \strong{professional} look 
(towards CMS?), and 'blogdot.sty' uses 'blog.sty' 
for \HTML\ \strong{beamer} presentations.
\end{abstract}
\tableofcontents

\section{Installing and Usage}
The file 'blog.sty' is provided ready, 
\strong{installation} only requires 
putting it somewhere where \TeX\ finds it 
(which may need updating the filename data 
 base).\urlfoot{ukfaqref}{inst-wlcf}

\strong{User commands} are described near their implementation below.

However, we must present an \strong{outline} of the procedure 
for generating \HTML\ files: 

At least one \strong{driver} file and one \strong{source} file are 
needed.

The \strong{driver} file's name is stored in `\jobname'. 
It loads 'blog.sty' by 
\begin{verbatim}
  \RequirePackage{blog}
\end{verbatim}
and uses file handling commands from 'blog.sty' and 
\CtanPkgRef{nicetext}{fifinddo}
(cf. `mdoccheat.pdf' from the \ctanpkgref{nicetext} bundle).\urlfoot{CtanPkgRef}{nicetext} 
%% <- \urlfoot 2012/11/30 
It chooses \strong{source} files and the name(s) for the resulting 
\HTML\ file(s). It may also need to load local settings, such as 
%% 2012/11/29: 
`\uselangcode' with the \ctanpkgdref{langcode} package  %% dref 2012/11/30
and settings for converting the editor's text encoding 
into the encoding that the head of the resulting \HTML\ file 
advertises---or into \HTML\ named entities 
(for me, `atari_ht.fdf' has done this).

The driver file could be run a terminal dialogue in order to choose source 
and target files and settings. So far, I rather have programmed a 
dialogue just for converting UTF-8 into an encoding that my 
Atari editor \textsc{xEDIT} can deal with. 
I do not present this now because it was conceptually mistaken, 
I must set up this conversion from scratch some time.
% [TODO: present in 'nicetext'].    %% 2011/01/24

The \strong{source} file(s) should contain user commands defined below 
to generate the necessary \xmltagcode{head} section and the 
\xmltagcode{body} tags. 

\section{Examples}
\subsection{Hello World!}
This is the \strong{source} code for a ``Hello World" example, 
in `hellowor.tex':
\MDsamplecodeinput{hellowor}
The \HTML\ file `hellowor.htm' is generated from `hellowor.tex'
by the following \strong{driver} file `mkhellow.tex':
\pagebreak[2]
\MDsamplecodeinput{mkhellow}

  \iffalse                                  %% 2012/11/29
\subsection{A Very Plain Style}
My ``\TeX-generated 
pages"{\foothttpurlref{www.webdesign-bu.de/%
                       uwe\string_lueck/texmap.htm}}
use a \strong{driver} file `makehtml.tex'. 
To choose a page to generate, I ``uncomment"ed just one 
of several lines that set the ``current conversion job" 
from a list (for some time). 
I choose the example of a simple ``site map:" 
`texmap.htm' is generated from \strong{source} file 
`texmap.tex'.---More recently however, I have started to 
read the job name and perhaps extra settings from a file 
`jobname.tex' that is created by a Bash script.

In order to make it easier for the reader to see what is essential, 
I~have moved many `.cfg'-like extra definitions into a file 
`texblog.fdf'. Some of these definitions may later move into 
`blog.sty'. You should find `makehtml.tex', `texmap.tex', and 
`texblog.fdf' in a directory `demo/texblog' 
(or `texblog.fdf' may be together with the `.sty' files), 
perhaps you can use them as templates.

\begingroup
  \MakeOther\|
%   \MakeOther\`\MakeOther\'  %% disables \tt! 2011/09/08
  \MakeOther\<
  \MakeActive\� \def�{\"o}                  %% 2011/10/10
  \MakeActive\� \def�{\"u}
  \hfuzz=\textwidth \advance \hfuzz by 28pt
\subsubsection{Driver File `makehtml.tex'}
 %% <- TODO \file needs protection for PDF 2011/09/08
 \enlargethispage{1\baselineskip}
  \listinginput[5]{1}{CTAN/morehype/demo/texblog/makehtml.tex}
\subsubsection{Source File \texttt{texmap.tex}}
  \listinginput[5]{1}{CTAN/morehype/demo/texblog/texmap.tex}
\endgroup

  \fi
\subsection{A Style with a Navigation Column}
\label{sec:example-lnavicol}
A style of web pages looking more professional 
% than                                %% rm. 2012/11/29
% `texmap.htm'                        %% was `texhax.hmt' 2011/09/02
(while perhaps becoming outdated) has a small navigation column 
on the left, side by side with a column for the main content. 
Both columns are spanned by a header section above and a footer 
section below. The package 'lnavicol.sty' provides commands 
`\PAGEHEAD', `\PAGENAVI', `\PAGEMAIN', `\PAGEFOOT', `\PAGEEND' 
(and some more) for structuring the source so that the code 
following `\PAGEHEAD' generates the header, the code following 
`\PAGENAVI' forms the content of the navigation column, etc.
Its code is presented in Sec.~\ref{sec:lnavicol}.
For real professionality, somebody must add some fine \acro{CSS}, 
and the macros mentioned may need to be redefined to use the `@class' 
attribute. Also, I am not sure about the table macros in 'blog.sty', 
so much may change later.

With things like these, can 'blog.sty' become a part of a 
``\Wikienref{content management system}" for \TeX\ addicts? 
This idea rather is based on the 
\wikideref{Content Management System}{\meta{German}} 
Wikipedia article.

As an example, I present parts of the source for my 
``home page"{\foothttpurlref{www.webdesign-bu.de/%
                             uwe\string_lueck/schreibt.html}}.
As the footer is the same on all pages of this style, 
it is added in the driver file `makehtml.tex'. 
`schreibt.tex' is the source file for generating `schreibt.html'.
You should find \emph{this} `makehtml.tex', a cut down version of 
`schreibt.tex', and `writings.fdf' with my extra macros for these pages 
in a directory 
`blogdemo/writings',                        %% blog 2012/11/30
hopefully useful as templates.

\begingroup
  \MakeActive\� \def�{\"a}                  %% 2011/10/05
  \MakeActive\� \def�{\"u}
  \hfuzz=\textwidth \advance \hfuzz by 10pt
  %% 2012/11/29 CTAN/morehype/demo -> blogdemo:
\subsubsection{Driver File `makehtml.tex'}
  \listinginput[5]{1}{blogdemo/writings/makehtml.tex}
\subsubsection{Source File `schreibt.tex'}
  \listinginput[5]{1}{blogdemo/writings/schreibt.tex}
\endgroup

\pagebreak                                  %% 2013/01/04
\section{The File \file{blog.sty}}          %% 2011/11/09
% \section{The File \texttt{blog.sty}}
% \section{The File {\tt blog.sty}} 
%% <- strange 2011/11/08 ->
% \section{The File `blog.sty'}             %% 2011/10/04 allow other files
\subsection{Preliminaries}                  %% 2012/10/03
\subsubsection{Package File Header (Legalese)} %% ize -> ese, subsub 2012/10/03
\ResetCodeLineNumbers
\ProvidesFile{blog.tex}[2013/01/04 documenting blog.sty]
\title{\textsf{blog.sty}\\---\\%
       Generating \HTML\ Quickly with \TeX\thanks{This 
       document describes version 
       \textcolor{blue}{\UseVersionOf{\jobname.sty}} 
       of \textsf{\jobname.sty} as of \UseDateOf{\jobname.sty}.}}
% \listfiles 
{ \RequirePackage{makedoc} 
  \ProcessLineMessage{} 
  %% three section levels 2012/08/07:
  \renewcommand\mdSectionLevelOne  {\string\subsection}
  \renewcommand\mdSectionLevelTwo  {\string\subsubsection}
  \renewcommand\mdSectionLevelThree{\string\paragraph}
  \MainDocParser{\SectionLevelThreeParseInput}
  \HeaderLines{16}              %% 2012/10/03
  \MakeSingleDoc{blog.sty}
  \HeaderLines{18}              %% 2012/11/28
  \MakeSingleDoc{blogligs.sty}
  \HeaderLines{18}              %% 2012/11/28
  \MakeSingleDoc{markblog.sty}
  \HeaderLines{18}              %% 2012/10/05
  \MakeSingleDoc{lnavicol.sty}
  \HeaderLines{17}              %% 2012/10/05
  \MakeSingleDoc{blogdot.sty}
}
\documentclass[fleqn]{article}%% TODO paper dimensions!?
\sfcode`-=1001      %% 2011/10/15
\ProvidesFile{makedoc.cfg}[2011/06/27 documentation settings] 

\author{Uwe L\"uck\thanks{\url{http://contact-ednotes.sty.de.vu}}}
% \author{Uwe L\"uck---{\tt http://contact-ednotes.sty.de.vu}}

%% hyperref:
\RequirePackage{ifpdf}
\usepackage[%
  \ifpdf
%     bookmarks=false,          %% 2010/12/22
%     bookmarksnumbered,
    bookmarksopen,              %% 2011/01/24!?
    bookmarksopenlevel=2,       %% 2011/01/23
%     pdfpagemode=UseNone,
%     pdfstartpage=10,
%     pdfstartview=FitH,
    citebordercolor={ .6 1    .6},
    filebordercolor={1    .6 1},
    linkbordercolor={1    .9  .7},
     urlbordercolor={ .7 1   1},   %% playing 2011/01/24
  \else
    draft
  \fi
]{hyperref}

\RequirePackage{niceverb}[2011/01/24] 
\RequirePackage{readprov}               %% 2010/12/08
\RequirePackage{hypertoc}               %% 2011/01/23
\RequirePackage{texlinks}               %% 2011/01/24
\makeatletter
  \@ifundefined{strong} 
               {\let\strong\textbf}     %% 2011/01/24
               {} 
  \@ifundefined{file} 
               {\let\file\texttt}       %% 2011/05/23
               {} 
\makeatother

\errorcontextlines=4
\pagestyle{headings}

\endinput

 %% shared formatting settings
\usepackage{filesdo} \MDfinaldatechecks             %% 2012/12/20
\ReadPackageInfos{blog}
  %% \tagcode seems to be a quite recent pdfTeX primitive, 
  %% cf. microtype.pdf ... %% 2010/11/06
\newcommand*{\xmltagcode}[1]{\texttt{<#1>}}
\newcommand*{\HTML}{\acro{HTML}}            %% 2011/09/08
\newcommand*{\CSS}{\acro{CSS}}              %% 2011/11/09
\newcommand*{\secref}[1]{Sec.~\ref{sec:#1}} %% 2011/11/23
\providecommand*{\LuaTeX}{Lua\TeX}          %% 2012/12/20
\sloppy
\begin{document}
\maketitle
\begin{abstract}\noindent
'blog.sty' provides \TeX\ macros for generating web pages, 
based on processing text files using the 'fifinddo' package. 
Some \LaTeX\ 
commands %%% command names 
are redefined to access their \HTML\ 
equivalents, other new macro names ``quote" the names of \HTML\ elements. 
The package has evolved in several little steps 
each aiming at getting pretty-looking ``hypertext" \textbf{notes}
with little effort, 
where ``little effort" also has meant avoiding studying 
documentation of similar packages already existing. 
[\textcolor{blue}{TODO:} list them!]
% Version v0.3 is the remainder of v0.2 after moving some stuff 
% to 'fifinddo.sty' (especially `\CopyFile'); 
% moreover, the new `\BlogCopyFile' replaces empty source lines 
% by \HTML's \xmltagcode{p} (starting a new paragraph).---Real 
% \emph{typesetting} from the same `.tex' source 
% (pretty printable output) has not been tried yet.
%% <- 2011/01/24 ->
The package %%% rather 
\emph{``misuses"} \TeX's macro language 
for generating \HTML\ code and entirely \emph{ignores} 
\TeX's typesetting capabilities.%%%---What about 
% such a ``small" \TeX\ with macros only and 
% \emph{no} typesetting capabilities ...!?
---'lnavicol.sty' adds a more \strong{professional} look 
(towards CMS?), and 'blogdot.sty' uses 'blog.sty' 
for \HTML\ \strong{beamer} presentations.
\end{abstract}
\tableofcontents

\section{Installing and Usage}
The file 'blog.sty' is provided ready, 
\strong{installation} only requires 
putting it somewhere where \TeX\ finds it 
(which may need updating the filename data 
 base).\urlfoot{ukfaqref}{inst-wlcf}

\strong{User commands} are described near their implementation below.

However, we must present an \strong{outline} of the procedure 
for generating \HTML\ files: 

At least one \strong{driver} file and one \strong{source} file are 
needed.

The \strong{driver} file's name is stored in `\jobname'. 
It loads 'blog.sty' by 
\begin{verbatim}
  \RequirePackage{blog}
\end{verbatim}
and uses file handling commands from 'blog.sty' and 
\CtanPkgRef{nicetext}{fifinddo}
(cf. `mdoccheat.pdf' from the \ctanpkgref{nicetext} bundle).\urlfoot{CtanPkgRef}{nicetext} 
%% <- \urlfoot 2012/11/30 
It chooses \strong{source} files and the name(s) for the resulting 
\HTML\ file(s). It may also need to load local settings, such as 
%% 2012/11/29: 
`\uselangcode' with the \ctanpkgdref{langcode} package  %% dref 2012/11/30
and settings for converting the editor's text encoding 
into the encoding that the head of the resulting \HTML\ file 
advertises---or into \HTML\ named entities 
(for me, `atari_ht.fdf' has done this).

The driver file could be run a terminal dialogue in order to choose source 
and target files and settings. So far, I rather have programmed a 
dialogue just for converting UTF-8 into an encoding that my 
Atari editor \textsc{xEDIT} can deal with. 
I do not present this now because it was conceptually mistaken, 
I must set up this conversion from scratch some time.
% [TODO: present in 'nicetext'].    %% 2011/01/24

The \strong{source} file(s) should contain user commands defined below 
to generate the necessary \xmltagcode{head} section and the 
\xmltagcode{body} tags. 

\section{Examples}
\subsection{Hello World!}
This is the \strong{source} code for a ``Hello World" example, 
in `hellowor.tex':
\MDsamplecodeinput{hellowor}
The \HTML\ file `hellowor.htm' is generated from `hellowor.tex'
by the following \strong{driver} file `mkhellow.tex':
\pagebreak[2]
\MDsamplecodeinput{mkhellow}

  \iffalse                                  %% 2012/11/29
\subsection{A Very Plain Style}
My ``\TeX-generated 
pages"{\foothttpurlref{www.webdesign-bu.de/%
                       uwe\string_lueck/texmap.htm}}
use a \strong{driver} file `makehtml.tex'. 
To choose a page to generate, I ``uncomment"ed just one 
of several lines that set the ``current conversion job" 
from a list (for some time). 
I choose the example of a simple ``site map:" 
`texmap.htm' is generated from \strong{source} file 
`texmap.tex'.---More recently however, I have started to 
read the job name and perhaps extra settings from a file 
`jobname.tex' that is created by a Bash script.

In order to make it easier for the reader to see what is essential, 
I~have moved many `.cfg'-like extra definitions into a file 
`texblog.fdf'. Some of these definitions may later move into 
`blog.sty'. You should find `makehtml.tex', `texmap.tex', and 
`texblog.fdf' in a directory `demo/texblog' 
(or `texblog.fdf' may be together with the `.sty' files), 
perhaps you can use them as templates.

\begingroup
  \MakeOther\|
%   \MakeOther\`\MakeOther\'  %% disables \tt! 2011/09/08
  \MakeOther\<
  \MakeActive\� \def�{\"o}                  %% 2011/10/10
  \MakeActive\� \def�{\"u}
  \hfuzz=\textwidth \advance \hfuzz by 28pt
\subsubsection{Driver File `makehtml.tex'}
 %% <- TODO \file needs protection for PDF 2011/09/08
 \enlargethispage{1\baselineskip}
  \listinginput[5]{1}{CTAN/morehype/demo/texblog/makehtml.tex}
\subsubsection{Source File \texttt{texmap.tex}}
  \listinginput[5]{1}{CTAN/morehype/demo/texblog/texmap.tex}
\endgroup

  \fi
\subsection{A Style with a Navigation Column}
\label{sec:example-lnavicol}
A style of web pages looking more professional 
% than                                %% rm. 2012/11/29
% `texmap.htm'                        %% was `texhax.hmt' 2011/09/02
(while perhaps becoming outdated) has a small navigation column 
on the left, side by side with a column for the main content. 
Both columns are spanned by a header section above and a footer 
section below. The package 'lnavicol.sty' provides commands 
`\PAGEHEAD', `\PAGENAVI', `\PAGEMAIN', `\PAGEFOOT', `\PAGEEND' 
(and some more) for structuring the source so that the code 
following `\PAGEHEAD' generates the header, the code following 
`\PAGENAVI' forms the content of the navigation column, etc.
Its code is presented in Sec.~\ref{sec:lnavicol}.
For real professionality, somebody must add some fine \acro{CSS}, 
and the macros mentioned may need to be redefined to use the `@class' 
attribute. Also, I am not sure about the table macros in 'blog.sty', 
so much may change later.

With things like these, can 'blog.sty' become a part of a 
``\Wikienref{content management system}" for \TeX\ addicts? 
This idea rather is based on the 
\wikideref{Content Management System}{\meta{German}} 
Wikipedia article.

As an example, I present parts of the source for my 
``home page"{\foothttpurlref{www.webdesign-bu.de/%
                             uwe\string_lueck/schreibt.html}}.
As the footer is the same on all pages of this style, 
it is added in the driver file `makehtml.tex'. 
`schreibt.tex' is the source file for generating `schreibt.html'.
You should find \emph{this} `makehtml.tex', a cut down version of 
`schreibt.tex', and `writings.fdf' with my extra macros for these pages 
in a directory 
`blogdemo/writings',                        %% blog 2012/11/30
hopefully useful as templates.

\begingroup
  \MakeActive\� \def�{\"a}                  %% 2011/10/05
  \MakeActive\� \def�{\"u}
  \hfuzz=\textwidth \advance \hfuzz by 10pt
  %% 2012/11/29 CTAN/morehype/demo -> blogdemo:
\subsubsection{Driver File `makehtml.tex'}
  \listinginput[5]{1}{blogdemo/writings/makehtml.tex}
\subsubsection{Source File `schreibt.tex'}
  \listinginput[5]{1}{blogdemo/writings/schreibt.tex}
\endgroup

\pagebreak                                  %% 2013/01/04
\section{The File \file{blog.sty}}          %% 2011/11/09
% \section{The File \texttt{blog.sty}}
% \section{The File {\tt blog.sty}} 
%% <- strange 2011/11/08 ->
% \section{The File `blog.sty'}             %% 2011/10/04 allow other files
\subsection{Preliminaries}                  %% 2012/10/03
\subsubsection{Package File Header (Legalese)} %% ize -> ese, subsub 2012/10/03
\ResetCodeLineNumbers
\ProvidesFile{blog.tex}[2013/01/04 documenting blog.sty]
\title{\textsf{blog.sty}\\---\\%
       Generating \HTML\ Quickly with \TeX\thanks{This 
       document describes version 
       \textcolor{blue}{\UseVersionOf{\jobname.sty}} 
       of \textsf{\jobname.sty} as of \UseDateOf{\jobname.sty}.}}
% \listfiles 
{ \RequirePackage{makedoc} 
  \ProcessLineMessage{} 
  %% three section levels 2012/08/07:
  \renewcommand\mdSectionLevelOne  {\string\subsection}
  \renewcommand\mdSectionLevelTwo  {\string\subsubsection}
  \renewcommand\mdSectionLevelThree{\string\paragraph}
  \MainDocParser{\SectionLevelThreeParseInput}
  \HeaderLines{16}              %% 2012/10/03
  \MakeSingleDoc{blog.sty}
  \HeaderLines{18}              %% 2012/11/28
  \MakeSingleDoc{blogligs.sty}
  \HeaderLines{18}              %% 2012/11/28
  \MakeSingleDoc{markblog.sty}
  \HeaderLines{18}              %% 2012/10/05
  \MakeSingleDoc{lnavicol.sty}
  \HeaderLines{17}              %% 2012/10/05
  \MakeSingleDoc{blogdot.sty}
}
\documentclass[fleqn]{article}%% TODO paper dimensions!?
\sfcode`-=1001      %% 2011/10/15
\input{makedoc.cfg} %% shared formatting settings
\usepackage{filesdo} \MDfinaldatechecks             %% 2012/12/20
\ReadPackageInfos{blog}
  %% \tagcode seems to be a quite recent pdfTeX primitive, 
  %% cf. microtype.pdf ... %% 2010/11/06
\newcommand*{\xmltagcode}[1]{\texttt{<#1>}}
\newcommand*{\HTML}{\acro{HTML}}            %% 2011/09/08
\newcommand*{\CSS}{\acro{CSS}}              %% 2011/11/09
\newcommand*{\secref}[1]{Sec.~\ref{sec:#1}} %% 2011/11/23
\providecommand*{\LuaTeX}{Lua\TeX}          %% 2012/12/20
\sloppy
\begin{document}
\maketitle
\begin{abstract}\noindent
'blog.sty' provides \TeX\ macros for generating web pages, 
based on processing text files using the 'fifinddo' package. 
Some \LaTeX\ 
commands %%% command names 
are redefined to access their \HTML\ 
equivalents, other new macro names ``quote" the names of \HTML\ elements. 
The package has evolved in several little steps 
each aiming at getting pretty-looking ``hypertext" \textbf{notes}
with little effort, 
where ``little effort" also has meant avoiding studying 
documentation of similar packages already existing. 
[\textcolor{blue}{TODO:} list them!]
% Version v0.3 is the remainder of v0.2 after moving some stuff 
% to 'fifinddo.sty' (especially `\CopyFile'); 
% moreover, the new `\BlogCopyFile' replaces empty source lines 
% by \HTML's \xmltagcode{p} (starting a new paragraph).---Real 
% \emph{typesetting} from the same `.tex' source 
% (pretty printable output) has not been tried yet.
%% <- 2011/01/24 ->
The package %%% rather 
\emph{``misuses"} \TeX's macro language 
for generating \HTML\ code and entirely \emph{ignores} 
\TeX's typesetting capabilities.%%%---What about 
% such a ``small" \TeX\ with macros only and 
% \emph{no} typesetting capabilities ...!?
---'lnavicol.sty' adds a more \strong{professional} look 
(towards CMS?), and 'blogdot.sty' uses 'blog.sty' 
for \HTML\ \strong{beamer} presentations.
\end{abstract}
\tableofcontents

\section{Installing and Usage}
The file 'blog.sty' is provided ready, 
\strong{installation} only requires 
putting it somewhere where \TeX\ finds it 
(which may need updating the filename data 
 base).\urlfoot{ukfaqref}{inst-wlcf}

\strong{User commands} are described near their implementation below.

However, we must present an \strong{outline} of the procedure 
for generating \HTML\ files: 

At least one \strong{driver} file and one \strong{source} file are 
needed.

The \strong{driver} file's name is stored in `\jobname'. 
It loads 'blog.sty' by 
\begin{verbatim}
  \RequirePackage{blog}
\end{verbatim}
and uses file handling commands from 'blog.sty' and 
\CtanPkgRef{nicetext}{fifinddo}
(cf. `mdoccheat.pdf' from the \ctanpkgref{nicetext} bundle).\urlfoot{CtanPkgRef}{nicetext} 
%% <- \urlfoot 2012/11/30 
It chooses \strong{source} files and the name(s) for the resulting 
\HTML\ file(s). It may also need to load local settings, such as 
%% 2012/11/29: 
`\uselangcode' with the \ctanpkgdref{langcode} package  %% dref 2012/11/30
and settings for converting the editor's text encoding 
into the encoding that the head of the resulting \HTML\ file 
advertises---or into \HTML\ named entities 
(for me, `atari_ht.fdf' has done this).

The driver file could be run a terminal dialogue in order to choose source 
and target files and settings. So far, I rather have programmed a 
dialogue just for converting UTF-8 into an encoding that my 
Atari editor \textsc{xEDIT} can deal with. 
I do not present this now because it was conceptually mistaken, 
I must set up this conversion from scratch some time.
% [TODO: present in 'nicetext'].    %% 2011/01/24

The \strong{source} file(s) should contain user commands defined below 
to generate the necessary \xmltagcode{head} section and the 
\xmltagcode{body} tags. 

\section{Examples}
\subsection{Hello World!}
This is the \strong{source} code for a ``Hello World" example, 
in `hellowor.tex':
\MDsamplecodeinput{hellowor}
The \HTML\ file `hellowor.htm' is generated from `hellowor.tex'
by the following \strong{driver} file `mkhellow.tex':
\pagebreak[2]
\MDsamplecodeinput{mkhellow}

  \iffalse                                  %% 2012/11/29
\subsection{A Very Plain Style}
My ``\TeX-generated 
pages"{\foothttpurlref{www.webdesign-bu.de/%
                       uwe\string_lueck/texmap.htm}}
use a \strong{driver} file `makehtml.tex'. 
To choose a page to generate, I ``uncomment"ed just one 
of several lines that set the ``current conversion job" 
from a list (for some time). 
I choose the example of a simple ``site map:" 
`texmap.htm' is generated from \strong{source} file 
`texmap.tex'.---More recently however, I have started to 
read the job name and perhaps extra settings from a file 
`jobname.tex' that is created by a Bash script.

In order to make it easier for the reader to see what is essential, 
I~have moved many `.cfg'-like extra definitions into a file 
`texblog.fdf'. Some of these definitions may later move into 
`blog.sty'. You should find `makehtml.tex', `texmap.tex', and 
`texblog.fdf' in a directory `demo/texblog' 
(or `texblog.fdf' may be together with the `.sty' files), 
perhaps you can use them as templates.

\begingroup
  \MakeOther\|
%   \MakeOther\`\MakeOther\'  %% disables \tt! 2011/09/08
  \MakeOther\<
  \MakeActive\� \def�{\"o}                  %% 2011/10/10
  \MakeActive\� \def�{\"u}
  \hfuzz=\textwidth \advance \hfuzz by 28pt
\subsubsection{Driver File `makehtml.tex'}
 %% <- TODO \file needs protection for PDF 2011/09/08
 \enlargethispage{1\baselineskip}
  \listinginput[5]{1}{CTAN/morehype/demo/texblog/makehtml.tex}
\subsubsection{Source File \texttt{texmap.tex}}
  \listinginput[5]{1}{CTAN/morehype/demo/texblog/texmap.tex}
\endgroup

  \fi
\subsection{A Style with a Navigation Column}
\label{sec:example-lnavicol}
A style of web pages looking more professional 
% than                                %% rm. 2012/11/29
% `texmap.htm'                        %% was `texhax.hmt' 2011/09/02
(while perhaps becoming outdated) has a small navigation column 
on the left, side by side with a column for the main content. 
Both columns are spanned by a header section above and a footer 
section below. The package 'lnavicol.sty' provides commands 
`\PAGEHEAD', `\PAGENAVI', `\PAGEMAIN', `\PAGEFOOT', `\PAGEEND' 
(and some more) for structuring the source so that the code 
following `\PAGEHEAD' generates the header, the code following 
`\PAGENAVI' forms the content of the navigation column, etc.
Its code is presented in Sec.~\ref{sec:lnavicol}.
For real professionality, somebody must add some fine \acro{CSS}, 
and the macros mentioned may need to be redefined to use the `@class' 
attribute. Also, I am not sure about the table macros in 'blog.sty', 
so much may change later.

With things like these, can 'blog.sty' become a part of a 
``\Wikienref{content management system}" for \TeX\ addicts? 
This idea rather is based on the 
\wikideref{Content Management System}{\meta{German}} 
Wikipedia article.

As an example, I present parts of the source for my 
``home page"{\foothttpurlref{www.webdesign-bu.de/%
                             uwe\string_lueck/schreibt.html}}.
As the footer is the same on all pages of this style, 
it is added in the driver file `makehtml.tex'. 
`schreibt.tex' is the source file for generating `schreibt.html'.
You should find \emph{this} `makehtml.tex', a cut down version of 
`schreibt.tex', and `writings.fdf' with my extra macros for these pages 
in a directory 
`blogdemo/writings',                        %% blog 2012/11/30
hopefully useful as templates.

\begingroup
  \MakeActive\� \def�{\"a}                  %% 2011/10/05
  \MakeActive\� \def�{\"u}
  \hfuzz=\textwidth \advance \hfuzz by 10pt
  %% 2012/11/29 CTAN/morehype/demo -> blogdemo:
\subsubsection{Driver File `makehtml.tex'}
  \listinginput[5]{1}{blogdemo/writings/makehtml.tex}
\subsubsection{Source File `schreibt.tex'}
  \listinginput[5]{1}{blogdemo/writings/schreibt.tex}
\endgroup

\pagebreak                                  %% 2013/01/04
\section{The File \file{blog.sty}}          %% 2011/11/09
% \section{The File \texttt{blog.sty}}
% \section{The File {\tt blog.sty}} 
%% <- strange 2011/11/08 ->
% \section{The File `blog.sty'}             %% 2011/10/04 allow other files
\subsection{Preliminaries}                  %% 2012/10/03
\subsubsection{Package File Header (Legalese)} %% ize -> ese, subsub 2012/10/03
\ResetCodeLineNumbers
\input{blog.doc}

\section{``Pervasive Ligatures" with 'blogligs.sty'} %% 2012/11/29
\label{sec:moreligs}
% \AddQuotes
This is the code and documentation of the package mentioned in 
Sec.~\ref{sec:ligs}, loadable by option |[ligs]|.
See below for what is offered. 
\ResetCodeLineNumbers
\input{blogligs.doc}
% \DontAddQuotes

\section{Wiki Markup by 'markblog.sty'}     %% 2012/11/29
\label{sec:mark}
\subsection{Introduction}                   %% 2012/12/20
\AddQuotes
This is the code and documentation of the package mentioned in 
Sec.~\ref{sec:ligs}, loadable by option |[mark]|.
See below for what is offered. You should also find a file 
`markblog.htm' that sketches it. Moreover, `texlinks.pdf'
describes in detail to what extent Wikipedia's 
``\wikiref{Help:Links#Piped_links}{piped links}"
with `[[<wikipedia-link>]]' is supported.       %% <...> 2012/12/20

\subsection{Similar Packages}               %% 2012/12/20
'wiki.sty' from the \ctanpkgdref{nicetext} bundle has offered 
some Wikipedia-like markup as a front-end for ordinary 
typesetting with \LaTeX\ (for \acro{DVI}/\acro{PDF}), 
implemented in a way very different from what is going on here, 
rather converting markup sequences \emph{during} typesetting.

More similar to the present approach is the way how 
Wikipedia section titles in package documentation 
is implemented by 'makedoc' from the 'nicetext' bundle, 
based on \strong{preprocessing} by 'fifinddo'.

In general, John MacFarlane's 
\httpref{johnmacfarlane.net/pandoc}{\pkg{pandoc}}
(cf.~\wikideref{pandoc}{German Wikipedia})
converts between wiki-like (simplified) markup and 
\LaTeX\ markup. (It deals with rather fixed 
markup rules, while we here process markup sequences 
independently of an entire markup \emph{language}.)

Another straightforward and well-documented way to 
\emph{preprocess} source files for converting simplified 
markup into \TeX\ markup is \ctanpkgauref{isambert}{Paul Isambert}'s 
\ctanpkgref{interpreter}. It relies on \wikiref{LuaTeX}{\LuaTeX}
where Lua does the preprocessing.

\subsection{Package File Header}            %% 2012/12/20
\ResetCodeLineNumbers
\input{markblog.doc}
\DontAddQuotes

%% rm. \pagebreak 2013/01/04
\section{Real Web Pages with 'lnavicol.sty'}
\label{sec:lnavicol}
This is the code and documentation of the package mentioned in 
Sec.~\ref{sec:example-lnavicol}.
\ResetCodeLineNumbers
\input{lnavicol.doc}

\section{Beamer Presentations with 'blogdot.sty'}
\subsection{Overview}
'blogdot.sty' extends 'blog.sty' in order to construct ``\HTML\ 
slides." One ``slide" is a 3$\times$3 table such that 
\begin{enumerate}
  \item it \strong{fills} the computer \strong{screen}, 
  \item the center cell is the \strong{``type area,"}
  \item the ``margin cell" below the center cell 
        is a \strong{link} to the \strong{next} ``slide,"
  \item the lower right-hand cell is a \strong{``restart"} link.
\end{enumerate}
Six \strong{size parameters} listed in Sec.~\ref{sec:dot-size} 
must be adjusted to the screen in `blogdot.cfg' 
(or in a file with project-specific definitions).

We deliver a file |blogdot.css| containing \strong{\acro{CSS}} font size 
declarations that have been used so far; you may find better ones 
or ones that work better with your screen size, or you may need to add 
style declarations for additional \HTML\ elements.

Another parameter that the user may want to modify is the 
\strong{``restart" anchor name} |\BlogDotRestart| 
(see Sec.~\ref{sec:dot-fin}). 
Its default value is |START| for the ``slide" opened by the command 
|\titlescreenpage| that is defined in Sec.~\ref{sec:dot-start}.

That slide is meant to be the ``\strong{title} slide" 
of the presentation. In order to \strong{display} it, 
I recommend to make and use a \strong{link} to |START| somewhere 
(such as with 'blog.sty''s `\ancref' command). 
The \emph{content} of the title slide is \emph{centered} horizontically, 
so certain commands mentioned \emph{below} 
(centering on other slides) may be useful.

\emph{After} `\titlescreenpage', the next main \strong{user commands} 
are
\begin{description}
  \cmdboxitem|\nextnormalscreenpage{<anchor-name>}| \
    starts a slide whose content is aligned flush left,
  \cmdboxitem|\nextcenterscreenpage{<anchor-name>}| \
    starts a slide whose content is centered horizontally.
\end{description}
---cf.~Sec.~\ref{sec:dot-next}. Right after these commands, 
as well as right after `\title'\-`screen`\-'page', code is used to 
generate the content of the \strong{type area} of the corresponding 
slide. Another `\next...' command closes that content and opens 
another slide. The presentation (the content of the very last slide) 
may be finished using |\screenbottom{<final>}| where <final> may be 
arbitrary, or `START' may be a fine choice for <final>.

Finally, there are user commands for \strong{centering} slide content 
horizontically (cf.~Sec.~\ref{sec:dot-type}): 
\begin{description}
  \cmdboxitem|\cheading{<digit>}{<title>}| \
    ``printing" a heading centered horizontically---even on slides 
    whose remaining content is aligned \emph{flush left} 
    (I have only used <digit>=2 so far), 
  \cmdboxitem|\begin{textblock}{<width>}| \     %% not metavar 2012/07/19
    ``printing" the content of a `{textblock}' environment with 
    maximum line width <width> flush left, 
    while that ``block" as a whole may be centered 
    horizontically on the slide due to choosing 
    `\nextcenterscreenpage'---especially for \strong{list} 
    environments with entry lines that are shorter than the 
    type area width and thus would not look centered 
    (below a centered heading from `\cheading'). 
\end{description}

The so far single \strong{example} of a presentation prepared using 'blogdot' 
is \ctanfileref{info/fifinddo-info}{dantev45.htm}
%% <- 2011/10/21 ->
(\ctanpkgref{fifinddo-info} bundle), 
a sketch of applying 'fifinddo' to package 
documentation and \HTML\ generation. A ``driver" file is needed 
for generating the \HTML\ code for the presentation from a `.tex' 
source by analogy to generating any \HTML\ file using 'blog.sty'. 
For the latter purpose, I have named my driver files `makehtml.tex'. 
For `dantev45.htm', I have called that file |makedot.tex|, 
the main difference to `makehtml.tex' is loading `blogdot.sty' 
in place of `blog.sty'.

This example also uses a file `dantev45.fdf' that defines some 
commands that may be more appropriate as user-level commands 
than the ones presented here (which may appear to be still too 
low-level-like): 
\begin{description}
  \cmdboxitem|\teilpage{<number>}{<title>}|
    making a ``cover slide" for announcing a new ``part" 
    of the presentation in German, 
  \cmdboxitem|\labelsection{<label>}{<title>}|
    starting a slide with heading <title> 
    and with anchor <label> 
    (that is displayed on clicking a \emph{link} to 
     <label>)---using 
     \[`\nextnormalscreenpage{<label>}'\mbox{ and } 
     `\cheading2{<title>}',\] 
  \cmdboxitem|\labelcentersection{<label>}{<title>}|
    like the previous command except that the slide content will be 
    \emph{centered} horizontally, using 
    \[`\nextcenterscreenpage{<title>}'.\]
\end{description}

%% 2011/10/10:
\strong{Reasons} to make \HTML\ presentations may be:\ \
(i)~As opposed to office software, this is a transparent light-weight 
approach.\ \
Considering \emph{typesetting} slides with \TeX,\ \
(ii)~\TeX's advanced typesetting abilities such as automatical 
page breaking are not very relevant for slides;\ \
(iii)~a typesetting run needs a second or a few seconds, 
while generating \HTML\ with 'blog.sty' needs a fraction of a second;\ \ 
(iv)~adjusting formatting parameters such as sizes and colours 
needed for slides is somewhat more straightforward with \HTML\ 
than with \TeX.

%% 2011/10/11, 2011/10/15:
\strong{Limitations:} \
First I was happy about how it worked on my netbook, 
but then I realized how difficult it is to present the ``slides" ``online."
Screen sizes (centering) are one problem. 
(Without the ``restart" idea, this might be much easier.)
Another problem is that the ``hidden links" don't work with 
\Wikienref{Internet Explorer} as they work with 
\Wikienref{Firefox}, \Wikienref{Google Chrome}, and 
\Wikiendisambref{Opera}{web browser}.
% I am now working at an easy choice of ``recompiling options." 
And finally, in internet shops some 
\HTML\ entities/symbols were not supported. 
In any case I (again) became aware of the fact 
that \HTML\ is not as \strong{``portable"} as \acro{PDF}.

Some \strong{workarounds} are described in Sec.~\ref{sec:cfgs}. 
|\FillBlogDotTypeArea| has two effects: \ (i)~providing an additional 
link to the \emph{next} slide for MSIE, \ (ii)~\emph{widening} 
and centering the \emph{type area} on larger screens 
than the one which the presentation originally was made for. \ 
An optional argument of |\TryBlogDotCFG| is offered for a `.cfg' file 
overriding the original settings for the presentation. 
Using it, I learnt that for ``portability," some manual line breaks 
(`\\', \xmltagcode{br}) should be replaced by ``ties" between the 
words \emph{after} the intended line break 
(when the line break is too ugly in a wider type area). 
For keeping the original type area width on wider screens 
(for certain ``slides", perhaps when line breaks really are wanted 
 to be preserved), the |{textblock}| environment may be used. 
Better \HTML\ and \acro{CSS} expertise may eventually 
lead to better solutions. 

The \strong{name} \qtd{blogdot} is a ``pun" on the name of the 
\ctanpkgref{powerdot} package (which in turn refers to 
``\Wikienref{PowerPoint}").

\subsection{File Header}
\ResetCodeLineNumbers
\input{blogdot.doc}

\end{document}

HISTORY

2010/11/05   for v0.2
2010/11/11   for v0.3
2011/01/23   using readprov and color
2011/01/27   using \urlfoot
2011/09/01   using new makedoc.cfg incl. \acro and \foothttp...; 
             extension for twocolpg.sty
with morehype RELEASE r0.4
2011/09/02   twocolpg.sty renamed into lnavicol.sty, typo fixes
2011/09/08   \HTML
2011/09/23   TODO in abstract blue again
2011/10/05   umlaut-a in schreibt.tex
2011/10/07f. blogdot
2011/10/09   lnavicol
2011/10/10   tuning; reasons for blogdot
2011/10/11   limitations of blogdot, corrected makedoc code
2011/10/15   more on limitations of blogdot; 
             abstract on lnavicol and blogdot
2011/10/21   links to fifinddo-info/dantev45.htm
2011/11/05   using \MakeSingleDoc from makedoc.sty v0.42
2011/11/08   playing with alternatives to defective `blog.sty' 
             in page head
2011/11/09   so use \file with new makedoc.cfg for hyperref; \CSS 
2011/11/23   \secref
2012/07/19   `textblock' not metavar
2012/08/07   three section levels
2012/10/03   ize -> ese, some restructuring, corr. \HeaderLines
2012/10/05   adjusted \HeaderLines (differ!)
2012/11/29   adding `blogligs' and `markblog'
2012/11/30   hello world, texblog sample removed, url foots
2012/12/20   filedate checks, doc. more about `markblog'
2013/01/04   \pagebreak +/-


\section{``Pervasive Ligatures" with 'blogligs.sty'} %% 2012/11/29
\label{sec:moreligs}
% \AddQuotes
This is the code and documentation of the package mentioned in 
Sec.~\ref{sec:ligs}, loadable by option |[ligs]|.
See below for what is offered. 
\ResetCodeLineNumbers
\NeedsTeXFormat{LaTeX2e}[1994/12/01] %% \newcommand* etc. 
\ProvidesPackage{blogligs}[2012/11/29 v0.2 
                           pervasive blog ligatures (UL)]
%% copyright (C) 2012 Uwe Lueck, 
%% http://www.contact-ednotes.sty.de.vu 
%% -- author-maintained in the sense of LPPL below.
%%
%% This file can be redistributed and/or modified under 
%% the terms of the LaTeX Project Public License; either 
%% version 1.3c of the License, or any later version.
%% The latest version of this license is in
%%     http://www.latex-project.org/lppl.txt
%% We did our best to help you, but there is NO WARRANTY. 
%%
%% Please report bugs, problems, and suggestions via 
%% 
%%   http://www.contact-ednotes.sty.de.vu 
%%
%% == 'blog' Required ==
%% 'blogdot' is an extension of 'blog', and must be loaded \emph{later}
%% (but what about options? TODO):
\RequirePackage{blog}
%% == Task and Idea   ==
%% |\UseBlogLigs| as offered by 'blog.sty' does not work 
%% inside macro arguments. You can use |\ParseLigs{<text>}|
%% at such locations to enable ``ligatures" again. 
%% 'blogligs.sty' saves you from this manual trick. 
%% Many macros have one ``text" argument only, 
%% others additionally have ``attribute" arguments. 
%% Most macros `<elt-cmd>{<text>}' of the first kind are defined 
%% to expand to `\SimpleTagSurr{<elt>}{<text>}'
%% or to `\TagSurr{<elt>}{<attrs>}{<text>}' for some 
%% \HTML\ element <elt> and some attribute assignments <attrs>. 
%% When a macro in addition to a ``text" element has 
%% ``attribute" parameters, `\TagSurr' is used as well.
%% %% 2012/01/08, eigentlich schon 2012/01/04, verloren ...:
% \let\blogtextcolor\textcolor
% \renewcommand*{\textcolor}[2]{\blogtextcolor{#1}{\ParseLigs{#2}}}
%%
%% \pagebreak[2]
%% == Quotation Marks ==
%% ``Inline quote" macros `<qtd>{<text>}' to surround <text>
%% by quotation marks do not follow this rule. We are just 
%% dealing with English and German double quotes 
%% that I have mostly treated by `catchdq.sty'. 
%% `"<text>"' then (eventually) expands to either 
%% `\deqtd{<text>}' or `\endqtd{<text>}', so we redefine these:
%% %% 2012/01/10:
\let\blogdedqtd\dedqtd 
\renewcommand*{\dedqtd}[1]{\blogdedqtd{\ParseLigs{#1}}}
%% %% 2012/08/20:
\let\blogendqtd\endqtd
\renewcommand*{\endqtd}[1]{\blogendqtd{\ParseLigs{#1}}}
%%
%% == \HTML\ Elements ==
%% When the above rule holds:
%% %% 2012/01/19:
\let\BlogTagSurr\TagSurr 
\renewcommand*{\TagSurr}[3]{%
    \BlogTagSurr{#1}{#2}{\ParseLigs{#3}}}
\let\BlogSimpleTagSurr\SimpleTagSurr 
\renewcommand*{\SimpleTagSurr}[2]{%
    \BlogSimpleTagSurr{#1}{\ParseLigs{#2}}}
%%
%% == Avoiding ``Ligatures" though ==
%% |\noligs{<text>}| saves <text> from ``ligature" replacements 
%% (except in arguments of macros inside <text> where 
%%  'blogligs' enables ligatures):
\newcommand*{\noligs}{}     \let\noligs\@firstofone     %% !!!
%% I have found it useful to disable replacements within
%% |\code{<text>}|: 
\renewcommand*{\code}[1]{\STS{code}{\noligs{#1}}}
%% TODO: kind of mistake, `\STS' has not been affected anyway so far, 
%% then defining `\code' as `\STS{code}' should suffice.
%%
%% |\NoBlogLigs| has been meant to disable ``ligatures" altogether again. 
%% I am not sure about everything ...
%% %% 2012/03/14, not optimal TODO:
\renewcommand*{\NoBlogLigs}{%
    \def\BlogOutputJob{LEAVE}%
%     \let\deqtd\blogdeqtd                       %% rm. 2012/06/03
    \let\TagSurr\BlogTagSurr
    \let\SimpleTagSurr\BlogSimpleTagSurr
    \FDnormalTilde 
    \MakeActiveDef\~{&nbsp;}%                    %% TODO new blog cmd
}
%% TODO: |\UseBlogLigs| might be redefined likewise 
%% (\textcolor{red}{in fact 'blogligs' activates ligatures 
%%                  inside text arguments unconditionally at present}, 
%%  I keep this for now since I have used it this way with `texblog.fdf' 
%%  over months, and changing it may be dangerous 
%%  where I have used tricky workarounds to overcome the 
%%  `texblog.fdf' mistake). 
%% But with \[`\BlogInteceptEnvironments'\] this is not needed 
%% when you use `\NoBlogLigs' for the contents of some \LaTeX\ 
%% environment.
%% 
%% == The End and \acro{HISTORY} ==
\endinput
%% VERSION HISTORY
v0.1    2012/01/08ff. developed in `texblog.fdf'
v0.2    2012/11/29    own file


% \DontAddQuotes

\section{Wiki Markup by 'markblog.sty'}     %% 2012/11/29
\label{sec:mark}
\subsection{Introduction}                   %% 2012/12/20
\AddQuotes
This is the code and documentation of the package mentioned in 
Sec.~\ref{sec:ligs}, loadable by option |[mark]|.
See below for what is offered. You should also find a file 
`markblog.htm' that sketches it. Moreover, `texlinks.pdf'
describes in detail to what extent Wikipedia's 
``\wikiref{Help:Links#Piped_links}{piped links}"
with `[[<wikipedia-link>]]' is supported.       %% <...> 2012/12/20

\subsection{Similar Packages}               %% 2012/12/20
'wiki.sty' from the \ctanpkgdref{nicetext} bundle has offered 
some Wikipedia-like markup as a front-end for ordinary 
typesetting with \LaTeX\ (for \acro{DVI}/\acro{PDF}), 
implemented in a way very different from what is going on here, 
rather converting markup sequences \emph{during} typesetting.

More similar to the present approach is the way how 
Wikipedia section titles in package documentation 
is implemented by 'makedoc' from the 'nicetext' bundle, 
based on \strong{preprocessing} by 'fifinddo'.

In general, John MacFarlane's 
\httpref{johnmacfarlane.net/pandoc}{\pkg{pandoc}}
(cf.~\wikideref{pandoc}{German Wikipedia})
converts between wiki-like (simplified) markup and 
\LaTeX\ markup. (It deals with rather fixed 
markup rules, while we here process markup sequences 
independently of an entire markup \emph{language}.)

Another straightforward and well-documented way to 
\emph{preprocess} source files for converting simplified 
markup into \TeX\ markup is \ctanpkgauref{isambert}{Paul Isambert}'s 
\ctanpkgref{interpreter}. It relies on \wikiref{LuaTeX}{\LuaTeX}
where Lua does the preprocessing.

\subsection{Package File Header}            %% 2012/12/20
\ResetCodeLineNumbers
\ProvidesFile{markblog.tex}[2012/11/29 extended blog markup]
\EXECUTE{\def\pkg{\code}\def\HTML{\abbr{HTML}}
         \def\mystrong{\textcolor{\#aa0000}}
         \def\myalert{\textcolor{red}}}
\head %%% \texmetadata 2012/09/23
  \metanamecontent{author}{Uwe L\"uck}
  \metanamecontent{date}{\isotoday}
  \robots{index,follow,noarchive}
\title{blog.sty \& wikis} %%% CMS}
\body 
\begin{center}
\heading2{\CtanPkgRef{morehype}{markblog.sty} %%%{{blog[exec].sty}}
          \& 
          [[Wiki]]s} %%% \wikiref{content management system}{\abbr{CMS}}}
\begin{stdallrulestable}
 \EXECUTE{\let\blogcode\code \def\code#1{\textcolor{\#003300}{\blogcode{#1}}}} %% vs. CSS 
                                                     %% ^^ war 66 2012/09/23
  \pkg{blog.sty} Syntax |    Output      | cf. \HTML | cf. \TeX                         | cf. ...      \cr
  \code{''italics''}    |  ''italics''   | \xmleltcode{i}{italics}
                                                     | \code{\{\cs{it} italics\}}       | Wikipedia    \cr
  \code{'''boldface'''} | '''boldface''' | \xmleltcode{b}{boldface}
                                                     | \code{\{\cs{bf} boldface\}}      | Wikipedia    \cr
  \code{[[Wikipedia]]}  | [[Wikipedia]]  | \xmleltattrcode{a}{href=...}{Wikipedia}}
                                                     | \code{\cs{href}\{...\}\{Wikipedia\}} 
                                                                                        | Wikipedia    \cr
  \code{**mystrong**}   | **mystrong**   | \xmleltattrcode{span}{style=...}{mystrong}}  
                                                     | \code{\cs{textcolor}\{...\}\{mystrong\}} 
                                                                                        | [[Markdown]] \cr
  \code{***myalert***}  | ***myalert***  | \xmleltattrcode{span}{style=...}{myalert}  
                                                     | \code{\cs{textcolor}{red}{myalert}}
                                                                                        | [[Markdown]] \cr
  \code{...}            | ...            | \code{\&hellip;}                 | \cs{dots} |              \cr
  \code{--}, \code{---} | --, ---        | \code{\&ndash;}, \code{\&mdash;} 
                                                     | \code{--}, \code{---} -- "ligs"  |              \cr
  \code{->}, \code{<-}  | ->, <-         | \code{\&rarr;}, \code{\&larr;} 
                                                     | \cs{to}, \cs{gets}               | 
\end{stdallrulestable}
\end{center}
\finish

\DontAddQuotes

%% rm. \pagebreak 2013/01/04
\section{Real Web Pages with 'lnavicol.sty'}
\label{sec:lnavicol}
This is the code and documentation of the package mentioned in 
Sec.~\ref{sec:example-lnavicol}.
\ResetCodeLineNumbers
\ProvidesPackage{lnavicol}[2011/10/13
                           left navigation column with blog.sty]
%%
%% Copyright (C) 2011 Uwe Lueck, 
%% http://www.contact-ednotes.sty.de.vu 
%% -- author-maintained in the sense of LPPL below -- 
%%
%% This file can be redistributed and/or modified under 
%% the terms of the LaTeX Project Public License; either 
%% version 1.3c of the License, or any later version.
%% The latest version of this license is in
%%     http://www.latex-project.org/lppl.txt
%% We did our best to help you, but there is NO WARRANTY. 
%%
%% Please report bugs, problems, and suggestions via 
%% 
%%   http://www.contact-ednotes.sty.de.vu 
%%
%% == 'blog.sty' Required ==
%% ---but what about options (TODO)?    %% 2011/10/09
\RequirePackage{blog} 
%%
%% == Switches ==
%% %% introduced 2011/04/29, it seems
%% There is a ``standard" page width and a ``tight one" 
%% (the latter for contact forms)---|\iftight|:
\newif\iftight 
%% In order to move an anchor to the \emph{top} of the screen 
%% when the anchor is near the page end, the page must get 
%% some extra length by adding empty space at its 
%% bottom---|\ifdeep|:
\newif\ifdeep 
%% 
%% == Page Style Settings (to be set locally) ==
% \newcommand*{\pagebgcolor}{\#f5f5f5}  %% CSS whitesmoke
% \newcommand*{\pagespacing}{\@cellpadding{4} \@cellspacing{7}} 
% \newcommand*{\pagenavicolwidth}{125}
% \newcommand*{\pagemaincolwidth}{584}
% \newcommand*{\pagewholewidth}  {792}
%% == Possible Additions to 'blog.sty' ==
%% === Tables ===
%% |\begin{spancolscell}{<number>}{<style>}|
%% opens an environment that contains a row and a single cell 
%% that will span <number> table cells and have style <style>:
\newenvironment{spancolscell}[2]{%
    \starttr\startTd{\@colspan{#1} #2 % 
                     \@width{100\%}}% %% TODO works? 
    }{\endTd\endtr}
%% The |{hiddencells}| einvironment contains cells that do not align 
%% with other cells in the surrounding table. The purpose is using
%% cells for horizontal spacing.
\newenvironment{hiddencells}
    {\startTable{}\starttr}
    {\endtr\endTable}
%% |{pagehiddencells}| is like `{hiddencells}' except that 
%% the \HTML\ code is indented:
\newenvironment{pagehiddencells}
    {\indentii\hiddencells}
    {\indentii\endhiddencells}
%% |\begin{FixedWidthCell}{<width>}{<style>}| \ opens the 
%% `{FixedWidthCell}' environment. The content will form a cell 
%% of width <width>. <style> are additional formatting parameters:
\newenvironment{FixedWidthCell}[2]
    {\startTd{#2}\startTable{\@width{#1}}%
     \starttr\startTd{}}
    {\endTd\endtr\endTable\endTd}
%% |\tablehspace{<width>}| is a variant of \LaTeX's `\hspace{<glue>}'. 
%% It may appear in a table row: 
\newcommand*{\tablehspace}[1]{\startTd{\@width{#1} /}}
%%
%% === Graphics ===
%% The command names in this section are inspired by the names 
%% in the standard \LaTeX\ \ctanpkgref{graphics} package.
%% (They may need some re-organization TODO.)
%% 
%% |\simpleinclgrf{<file>}| embeds a graphic file <file> without 
%% the tricks of the remaining commands.
\newcommand*{\simpleinclgrf}[1]{\IncludeGrf{alt="" \@border{0}}%
                                           {#1}}
%% |\IncludeGrf{<style>}{<file>}| embeds a graphic file <file> 
%% with style settings <style>:
\newcommand*{\IncludeGrf}[2]{<img #1 src="#2">}
%% |\includegraphic{<width>}{<height>}{<file>}{<border>}{<alt>}{<tooltip>}| 
%% ...:                             %% fine with mdoccorr 2011/10/13
\newcommand*{\includegraphic}[6]{% 
    \IncludeGrf{%
        \@width{#1} \@height{#2} %% data; presentation:
        \@border{#4} 
        alt="#5" \@title{#6}}%
        {#3}}
%% |\insertgraphic{<wd>}{<ht>}{<f>}{<b>}{<align>}{<hsp>}{<vsp>}{<alt>}{<t>}|
%% \\adds <hsp> for the `@hspace' and <vsp> for the `@vspace' 
%% attribute:
\newcommand*{\insertgraphic}[9]{%
    \IncludeGrf{%
        \@width{#1} \@height{#2} %% data; presentation:
        \@border{#4} 
        align="#5" hspace="#6" vspace="#8"
        alt="#8" \@title{#9}}%
        {#3}}
%% |\includegraphic{<wd>}{<ht>}{<file>}{<anchor>}{<border>}{<alt>}{<tooltip>}| 
%% \\uses an image with `\includegraphic' parameters as a link to 
%% <anchor>:
\newcommand*{\inclgrfref}[7]{%
    \fileref{#4}{\includegraphic{#1}{#2}{#3}%
                                {#5}{#6}{#7}}}
%%
%% === \acro{HTTP}/Wikipedia tooltips ===
%% |\httptipref{<tip>}{<www>}{<text>}| \ works like \
%% `\httpref{<www>}{<text>}' except that <tip> appears as ``tooltip":
\newcommand*{\httptipref}[2]{%
  \TagSurr a{\@title{#1}\@href{http://#2}\@target@blank}}
%% |\@target@blank| abbreviates the `@target' setting for 
%% opening the target in a new window or tab:
\newcommand*{\@target@blank}{target="_blank"}
%% % |\wikitipref{<language-code>}{<lemma>}{<text>}| \ 
%% |\wikitipref{<lc>}{<lem>}{<text>}| 
%% works like
%% % \\
%% % `\wikiref{<language-code>}{<lemma>}{<text>}' 
%% `\wikiref{<lc>}{<lem>}{<text>}' 
%% except that 
%% ``Wikipedia" appears as ``tooltip". 
%% |\wikideref| and |\wikienref| are redefined to use it:
\newcommand*{\wikitipref}[2]{%
    \httptipref{Wikipedia}{#1.wikipedia.org/wiki/#2}}
\renewcommand*{\wikideref}{\wikitipref{de}}
\renewcommand*{\wikienref}{\wikitipref{en}}
%%
%% == Page Structure ==
%% The body of the page is a table of three rows and two columns. 
%% === Page Head Row ===
%% |\PAGEHEAD| opens the head row and a single cell that will span 
%% the two columns of the second row.
\newcommand*{\PAGEHEAD}{%
  \startTable{%
    \@align@c\ 
    \@bgcolor{\pagebgcolor}%
    \@border{0}%%                       %% TODO local 
    \pagespacing
    \iftight \else \@width\pagewholewidth \fi 
  }\CLBrk
  %% omitting <tbody>
  \ \comment{ HEAD ROW }\CLBrk
  \indenti\spancolscell{2}{}%
}
% \newcommand*{\headgrf}  [1]{%                     %% rm. 2011/10/09
%     \indentiii\simplecell{\simpleinclgrf{#1}}}
% \newcommand*{\headgrfskiptitle}[3]{%
%   \pagehiddencells
%     \headgrf{#1}\CLBrk
%     \headskip{#2}\CLBrk
%     \headtitle1{#3}\CLBrk
%   \endpagehiddencells}
%% |\headuseskiptitle{<grf>}{<skip>}{<title>}|
%% first places <grf>, then skips horizontally by <skip>, 
%% and then prints the page title as \xmltagcode{h1}:
\newcommand*{\headuseskiptitle}[3]{%
  \pagehiddencells\CLBrk
    \indentiii\simplecell{#1}\CLBrk
    \headskip{#2}\CLBrk
    \headtitle1{#3}\CLBrk
  \endpagehiddencells}
%% |\headskip{<skip>}| is like `\tablehspace{<skip>}'
%% except that the \HTML\ code gets an indent.
\newcommand*{\headskip}    {\indentiii\tablehspace}
%% Similarly, |\headtitle{<digit>}{<text>}| is like 
%% `\heading<digit>{<text>}' apart from an indent and 
%% being put into a cell:
\newcommand*{\headtitle}[2]{\indentiii\simplecell{\heading#1{#2}}}
%%
%% === Navigation and Main Row ===
%% |\PAGENAVI| closes the head row and opens the ``navigation" 
%% column, actually including an `{itemize}' environment.
%% Accordingly, `writings.fdf' has a command `\fileitem'. 
%% But it seems that I have not been sure ...
\newcommand*{\PAGENAVI}{%
    \indenti\endspancolscell\CLBrk
    \indenti\starttr\CLBrk 
    \ \comment{NAVIGATION COL}\CLBrk 
    \indentii\FixedWidthCell\pagenavicolwidth
                           {\@class{paper} 
%% <- using `@class'=`paper' here is my brother's idea, 
%% not sure about it ...
                           \@valign@t}
    %% omitting `\@height{100\%}' 
    \itemize}
%% |\PAGEMAINvar{<width>}| closes the navigation column 
%% and opens the ``main content" column. The latter gets 
%% width <width>:
\newcommand*{\PAGEMAINvar}[1]{%
    \indentii\enditemize\ \endFixedWidthCell\CLBrk
    \ \comment{ MAIN COL }\CLBrk
    \indentii\FixedWidthCell{#1}{}} 
%% ... The width may be specified as |\pagemaincolwidth|, 
%% then |\PAGEMAIN| works like `\PAGEMAINvar{\pagemaincolwidth}':
\newcommand*{\PAGEMAIN}{\PAGEMAINvar\pagemaincolwidth}
%%
%% === Footer Row ===
%% |\PAGEFOOT| closes the ``main content" column as well as 
%% the second row, and opens the footer row:
\newcommand*{\PAGEFOOT}{%
    \indentii\endFixedWidthCell\CLBrk
%     \indentii\tablehspace{96}\CLBrk %% vs. \pagemaincolwidth
  %% <- TODO margin right of foot
    \indenti\endtr\CLBrk
    \ \comment{ FOOT ROW / }\CLBrk
    \indenti\spancolscell{2}{\@class{paper} \@align@c}%
%% <- again class ``paper"!?
}
%% |\PAGEEND| closes the footer row and provides all the rest 
%% ... needed?
\newcommand*{\PAGEEND}{\indenti\endspancolscell\endTable}
%%
%% == The End and HISTORY ==
\endinput

HISTORY 

2011/04/29   started (? \if...)
2011/09/01   to CTAN as `twocolpg.sty'
2011/09/02   renamed
2011/10/09f. documentation more serious 
2011/10/13   `...:' OK


\section{Beamer Presentations with 'blogdot.sty'}
\subsection{Overview}
'blogdot.sty' extends 'blog.sty' in order to construct ``\HTML\ 
slides." One ``slide" is a 3$\times$3 table such that 
\begin{enumerate}
  \item it \strong{fills} the computer \strong{screen}, 
  \item the center cell is the \strong{``type area,"}
  \item the ``margin cell" below the center cell 
        is a \strong{link} to the \strong{next} ``slide,"
  \item the lower right-hand cell is a \strong{``restart"} link.
\end{enumerate}
Six \strong{size parameters} listed in Sec.~\ref{sec:dot-size} 
must be adjusted to the screen in `blogdot.cfg' 
(or in a file with project-specific definitions).

We deliver a file |blogdot.css| containing \strong{\acro{CSS}} font size 
declarations that have been used so far; you may find better ones 
or ones that work better with your screen size, or you may need to add 
style declarations for additional \HTML\ elements.

Another parameter that the user may want to modify is the 
\strong{``restart" anchor name} |\BlogDotRestart| 
(see Sec.~\ref{sec:dot-fin}). 
Its default value is |START| for the ``slide" opened by the command 
|\titlescreenpage| that is defined in Sec.~\ref{sec:dot-start}.

That slide is meant to be the ``\strong{title} slide" 
of the presentation. In order to \strong{display} it, 
I recommend to make and use a \strong{link} to |START| somewhere 
(such as with 'blog.sty''s `\ancref' command). 
The \emph{content} of the title slide is \emph{centered} horizontically, 
so certain commands mentioned \emph{below} 
(centering on other slides) may be useful.

\emph{After} `\titlescreenpage', the next main \strong{user commands} 
are
\begin{description}
  \cmdboxitem|\nextnormalscreenpage{<anchor-name>}| \
    starts a slide whose content is aligned flush left,
  \cmdboxitem|\nextcenterscreenpage{<anchor-name>}| \
    starts a slide whose content is centered horizontally.
\end{description}
---cf.~Sec.~\ref{sec:dot-next}. Right after these commands, 
as well as right after `\title'\-`screen`\-'page', code is used to 
generate the content of the \strong{type area} of the corresponding 
slide. Another `\next...' command closes that content and opens 
another slide. The presentation (the content of the very last slide) 
may be finished using |\screenbottom{<final>}| where <final> may be 
arbitrary, or `START' may be a fine choice for <final>.

Finally, there are user commands for \strong{centering} slide content 
horizontically (cf.~Sec.~\ref{sec:dot-type}): 
\begin{description}
  \cmdboxitem|\cheading{<digit>}{<title>}| \
    ``printing" a heading centered horizontically---even on slides 
    whose remaining content is aligned \emph{flush left} 
    (I have only used <digit>=2 so far), 
  \cmdboxitem|\begin{textblock}{<width>}| \     %% not metavar 2012/07/19
    ``printing" the content of a `{textblock}' environment with 
    maximum line width <width> flush left, 
    while that ``block" as a whole may be centered 
    horizontically on the slide due to choosing 
    `\nextcenterscreenpage'---especially for \strong{list} 
    environments with entry lines that are shorter than the 
    type area width and thus would not look centered 
    (below a centered heading from `\cheading'). 
\end{description}

The so far single \strong{example} of a presentation prepared using 'blogdot' 
is \ctanfileref{info/fifinddo-info}{dantev45.htm}
%% <- 2011/10/21 ->
(\ctanpkgref{fifinddo-info} bundle), 
a sketch of applying 'fifinddo' to package 
documentation and \HTML\ generation. A ``driver" file is needed 
for generating the \HTML\ code for the presentation from a `.tex' 
source by analogy to generating any \HTML\ file using 'blog.sty'. 
For the latter purpose, I have named my driver files `makehtml.tex'. 
For `dantev45.htm', I have called that file |makedot.tex|, 
the main difference to `makehtml.tex' is loading `blogdot.sty' 
in place of `blog.sty'.

This example also uses a file `dantev45.fdf' that defines some 
commands that may be more appropriate as user-level commands 
than the ones presented here (which may appear to be still too 
low-level-like): 
\begin{description}
  \cmdboxitem|\teilpage{<number>}{<title>}|
    making a ``cover slide" for announcing a new ``part" 
    of the presentation in German, 
  \cmdboxitem|\labelsection{<label>}{<title>}|
    starting a slide with heading <title> 
    and with anchor <label> 
    (that is displayed on clicking a \emph{link} to 
     <label>)---using 
     \[`\nextnormalscreenpage{<label>}'\mbox{ and } 
     `\cheading2{<title>}',\] 
  \cmdboxitem|\labelcentersection{<label>}{<title>}|
    like the previous command except that the slide content will be 
    \emph{centered} horizontally, using 
    \[`\nextcenterscreenpage{<title>}'.\]
\end{description}

%% 2011/10/10:
\strong{Reasons} to make \HTML\ presentations may be:\ \
(i)~As opposed to office software, this is a transparent light-weight 
approach.\ \
Considering \emph{typesetting} slides with \TeX,\ \
(ii)~\TeX's advanced typesetting abilities such as automatical 
page breaking are not very relevant for slides;\ \
(iii)~a typesetting run needs a second or a few seconds, 
while generating \HTML\ with 'blog.sty' needs a fraction of a second;\ \ 
(iv)~adjusting formatting parameters such as sizes and colours 
needed for slides is somewhat more straightforward with \HTML\ 
than with \TeX.

%% 2011/10/11, 2011/10/15:
\strong{Limitations:} \
First I was happy about how it worked on my netbook, 
but then I realized how difficult it is to present the ``slides" ``online."
Screen sizes (centering) are one problem. 
(Without the ``restart" idea, this might be much easier.)
Another problem is that the ``hidden links" don't work with 
\Wikienref{Internet Explorer} as they work with 
\Wikienref{Firefox}, \Wikienref{Google Chrome}, and 
\Wikiendisambref{Opera}{web browser}.
% I am now working at an easy choice of ``recompiling options." 
And finally, in internet shops some 
\HTML\ entities/symbols were not supported. 
In any case I (again) became aware of the fact 
that \HTML\ is not as \strong{``portable"} as \acro{PDF}.

Some \strong{workarounds} are described in Sec.~\ref{sec:cfgs}. 
|\FillBlogDotTypeArea| has two effects: \ (i)~providing an additional 
link to the \emph{next} slide for MSIE, \ (ii)~\emph{widening} 
and centering the \emph{type area} on larger screens 
than the one which the presentation originally was made for. \ 
An optional argument of |\TryBlogDotCFG| is offered for a `.cfg' file 
overriding the original settings for the presentation. 
Using it, I learnt that for ``portability," some manual line breaks 
(`\\', \xmltagcode{br}) should be replaced by ``ties" between the 
words \emph{after} the intended line break 
(when the line break is too ugly in a wider type area). 
For keeping the original type area width on wider screens 
(for certain ``slides", perhaps when line breaks really are wanted 
 to be preserved), the |{textblock}| environment may be used. 
Better \HTML\ and \acro{CSS} expertise may eventually 
lead to better solutions. 

The \strong{name} \qtd{blogdot} is a ``pun" on the name of the 
\ctanpkgref{powerdot} package (which in turn refers to 
``\Wikienref{PowerPoint}").

\subsection{File Header}
\ResetCodeLineNumbers
\NeedsTeXFormat{LaTeX2e}[1994/12/01] %% \newcommand* etc. 
\ProvidesPackage{blogdot}[2013/01/22 v0.41b HTML presentations (UL)]
%% copyright (C) 2011 Uwe Lueck, 
%% http://www.contact-ednotes.sty.de.vu 
%% -- author-maintained in the sense of LPPL below.
%%
%% This file can be redistributed and/or modified under 
%% the terms of the LaTeX Project Public License; either 
%% version 1.3c of the License, or any later version.
%% The latest version of this license is in
%%     http://www.latex-project.org/lppl.txt
%% We did our best to help you, but there is NO WARRANTY. 
%%
%% Please report bugs, problems, and suggestions via 
%% 
%%   http://www.contact-ednotes.sty.de.vu 
%%
%% == 'blog' Required ==
%% 'blogdot' is an extension of 'blog' 
%% (but what about options? TODO):
\RequirePackage{blog}
%% == Size Parameters ==
%% \label{sec:dot-size}
%% I assume that it is clear what the following
%% six page dimension parameters 
%% \begin{quote}
%% |\leftpagemargin|, |\rightpagemargin|, 
%% |\upperpagemargin|,\\|\lowerpagemargin|, 
%% |\typeareawidth|, |\typeareaheight|
%% \end{quote}
%% mean. 
%% The choices are what I thought should work best 
%% on my 1024$\times$600 screen (in fullscreen mode); 
%% but I had to optimize the left and right margins experimentally
%% (with Mozilla Firefox~3.6.22 for Ubuntu canonical~-~1.0).
%% It seems to be best when the horizontal parameters 
%% together with what the brouswer adds 
%% (scroll bar, probably 32px with me) 
%% sum up to the screen width.
\newcommand*{\leftpagemargin}{176}
\newcommand*{\rightpagemargin}{\leftpagemargin}
%% So |\rightpagemargin| ultimately is the same as 
%% |\leftpagemargin| as long as you don't redefine it, 
%% and it suffices to `\renewcommand' `\leftpagemargin'
%% in order to get a horizontically centered type area 
%% with user-defined margin widths.---Something analogous
%% applies to |\upperpagemargin| and |\lowerpagemargin|:
\newcommand*{\upperpagemargin}{80}
\newcommand*{\lowerpagemargin}{\upperpagemargin}
%% A difference to the ``horizontal" parameters is 
%% (I expect) that the position of the type area on the 
%% screen is affected by |\upperpagemargin| only, 
%% and you may choose |\lowerpagemargin| just large enough 
%% that the next slide won't be visible on any computer screen 
%% you can think of.
\newcommand*{\typeareawidth}{640}
\newcommand*{\typeareaheight}{440}
%% Centering with respect to web page body may work better on 
%% different screens (2011/10/03), but it doesn't work here
%% (2011/10/04).
% \renewcommand*{\body}{%
%     </head>\CLBrk
%     <body \@bgcolor{\bodybgcolor} \@align@c>}
%% |\CommentBlogDotWholeWidth| procuces no \HTML\ code ...
\global\let\BlogDotWholeWidth\@empty
%% ... unless calculated with |\SumBlogDotWidth|: 
\newcommand*{\SumBlogDotWidth}{%
    \relax{%                        %% \relax 2011/10/22 magic ...
    \count@\typeareawidth
    \advance\count@ \leftpagemargin
    \advance\count@\rightpagemargin
    \typeout{ * blogdot slide width = \the\count@\space*}%
    \xdef\CommentBlogDotWholeWidth{%
        \comment{ slide width = \the\count@\ }}}}
%%
%% == (Backbone for) Starting a ``Slide" ==
%% \label{sec:dot-start}
%% |\startscreenpage{<style>}{<anchor-name>}|
\newcommand*{\startscreenpage}[2]{%% 0 2011/09/25!?:
    \\\CLBrk                                %% 2012/11/19
%% <- `\\' suddenly necessary, likewise in `texblog.fdf'
%%    with `\NextView' and `\nextruleview'. 
%%    Due to recent `firefox'?              %% 2012/11/21
    \startTable{%
        \@cellpadding{0} \@cellspacing{0}%
        \maybe@blogdot@borders              %% 2011/10/12
        \maybe@blogdot@frame                %% 2011/10/14
    }%
    \CLBrk                                  %% 2011/10/03
    \starttr
%% First cell determines both
%% height of upper page margin |\upperpagemargin|
%% and
%% width of left page margin |\leftpagemargin|:
      \startTd{\@width {\leftpagemargin }%
               \@height{\upperpagemargin}}%
%         \textcolor{\bodybgcolor}{XYZ}%
      \endTd
%% Using |\typeareawidth|:
%       \startTd{\@width{\typeareawidth}}\endTd
      \simplecell{%
        \CLBrk
        \hanc{#2}{\hvspace{\typeareawidth}% 
                          {\upperpagemargin}}%
        \CLBrk
      }%
%% Final cell of first row determines right margin width:
      \startTd{\@width{\leftpagemargin}}\endTd
    \endtr
    \starttr
    \emptycell\startTd{\@height{\typeareaheight}#1}%
}
%% |\titlescreenpage| \ (`\STARTscreenpage' TODO?) \ %% 2011/10/03 \ 2012/11/19
%% opens the title page (I thought). To get it to your screen, 
%% (make and) click a link like 
%% \[`\ancref{START}{start presentation}':\]
\newcommand*{\titlescreenpage}{%
    \startscreenpage{\@align@c}{START}}
%% 
%% == Finishing a ``Slide" and ``Restart" (Backbone) ==
%% \label{sec:dot-fin}
%% |\screenbottom{<next-anchor>}| finishes the current slide 
%% and links to the <next-anchor>, the anchor of a slide opened by 
%% \[`\startscreenpage{<style>}{<next-anchor>}'.\] 
%% More precisely, the margin below the type area is that link.
%% The corner at its right is a link to the anchor to whose name 
%% |\BlogDotRestart| expands. 
\newcommand*{\screenbottom}[1]{%
    \ifFillBlogDotTypeArea 
      <p>\ancref{#1}{\BlogDotFillText}%    %% not </p> 2011/10/22
    \fi
    \endTd\emptycell
    \endtr
    \CLBrk
    \tablerow{bottom margin}%                       %% 2011/10/13
      \emptycell
      \CLBrk
      \startTd{\@align@c}%
        \ancref{#1}{\HVspace{\BlogDotBottomFill}%
%% <- seems to be useless now (2011/10/15).
                            {\typeareawidth}%
                            {\lowerpagemargin}}%
      \endTd
      \CLBrk
      \simplecell{\ancref{\BlogDotRestart}% 
                         {\hvspace{\rightpagemargin}% 
                                  {\lowerpagemargin}}}%
    \endtablerow
    \CLBrk
    \endTable
}
%% The default for |\BlogDotRestart| is |START|---the title page. 
%% You can `\renew'\-`command' it so you get to a slide 
%% containing an overview of the presentation.
\newcommand*{\BlogDotRestart}{START}
%% 
%% == Moving to Next ``Slide" (User Level) == 
%% \label{sec:dot-next}
%% |\nextscreenpage{<style>}{<anchor-name>}|
%% puts closing the previous slide and opening the next 
%% one---having anchor name `<anchor-name>'---together.
%% <style> is for style settings for the next page, 
%% made here for choosing between centering the page/slide content 
%% and aligning it flush left.
\newcommand*{\nextscreenpage}[2]{%
    \screenbottom{#2}\CLBrk
    \hrule           \CLBrk 
    \startscreenpage{#1}{#2}}
%% |\nextcenterscreenpage{<anchor-name>}| chooses centering 
%% the slide content:
\newcommand*{\nextcenterscreenpage}{\nextscreenpage{\@align@c}}
%% |\nextnormalscreenpage{<anchor-name>}| chooses flush left
%% on the type area determined by |\typeareawidth|:
\newcommand*{\nextnormalscreenpage}{\nextscreenpage{}}
%% 
%% == Constructs for Type Area ==
%% \label{sec:dot-type}
%% If you want to get centered titles with \xmltagcode{h2} etc., 
%% you should declare this in `.css' files. But you may consider 
%% this way too difficult, and you may prefer to declare this 
%% right in the \HTML\ code. That's what I do! I use 
%% |\cheading{<digit>}{<text>}| for this purpose. 
\newcommand*{\cheading}[1]{\CLBrk\TagSurr{h#1}{\@align@c}}
%% |\begin{textblock}{<width>}| opens a |{textblock}| 
%% environment. The latter will contain text that will be flush left
%% in a narrower text area---of width <width>---than the one 
%% determined by |\typeareawidth|. It may be used on 
%% "centered" slides. It is made for lists whose entries are 
%% so short that the page would look unbalanced under a 
%% centered title with the list adjusted to the left 
%% of the entire type area. (Thinking of standard \LaTeX, 
%% it is almost the `{minipage}' environment, however lacking 
%% the footnote feature, in that respect it is rather 
%% similar to `\parbox' which however is not an environment.)
\newenvironment*{textblock}[1]
    {\startTable{\@width{#1}}\starttr\startTd{}}
    {\endTd\endtr\endTable}
%%
%% == Debugging and `.cfg's ==
%% \label{sec:cfgs}
%% |\ShowBlogDotBorders| shows borders of the page margins 
%% and may be undone by |\DontShowBlogDotBorders|:
\newcommand*{\ShowBlogDotBorders}{%
    \def\maybe@blogdot@borders{rules="all"}}
\newcommand*{\DontShowBlogDotBorders}{%
    \let\maybe@blogdot@borders\@empty}
\DontShowBlogDotBorders
%% %% 2011/10/14:
%% |\ShowBlogDotFrame| shows borders of the page margins 
%% and may be undone by |\DontShowBlogDotFrame|:
\newcommand*{\ShowBlogDotFrame}{%
    \def\maybe@blogdot@frame{\@frame@box}}
\newcommand*{\DontShowBlogDotFrame}{%
    \let\maybe@blogdot@frame\@empty}
\DontShowBlogDotFrame
%% However, the rules seem to affect horizontal positions ...
%%
%% |\BlogDotFillText| is a dirty trick ... seems to widen 
%% %% doc. extended 2011/10/13
%% the type area and this way centers the text on wider screens 
%% than the one used originally. Of course, this can corrupt 
%% intended line breaks. 
\newcommand*{\BlogDotFillText}{%            %% 2011/10/11
    \center
        \BlogDotFillTextColor{%             %% 2011/10/12
%                 X\\X                      %% insufficient
                 X X X X X X X X X X X X X X X X X X X X 
                 X X X X X X X X X X X X X X X X X X X X 
                 X X X X X X X X X X 
                 X X X X X X X X X X 
%                  X X X X X X X X X X X X X X X X X X X X 
        }
    \endcenter
}
%% |\FillBlogDotTypeArea| fills `\BlogDotFillText' into the 
%% type area, also as a link to the next slide. This may widen
%% the type area so that the text is centered on wider screens 
%% than the one the \HTML\ page was made for. The link may serve 
%% as an alternative to the bottom margin link 
%% (which sometimes fails). 
%% `\FillBlogDotTypeArea'                   %% 2011/10/22
%% can be undone 
%% by |\DontFillBlogDotTypeArea|:
\newcommand*{\FillBlogDotTypeArea}{%
    \let\ifFillBlogDotTypeArea\iftrue 
    \typeout{ * blogdot filling type area *}}       %% 2011/10/13
\newcommand*{\DontFillBlogDotTypeArea}{%
    \let\ifFillBlogDotTypeArea\iffalse}
\DontFillBlogDotTypeArea
%% |\FillBlogDotBottom| fills `\BlogDotFillText' into the 
%% center bottom cell. I tried it before `\FillBlogDotTypeArea'
%% and I am not sure ... 
%% It can be undone by 
%% |\DontFillBlogDotBottom|:
\newcommand*{\FillBlogDotBottom}{%
    \let\BlogDotBottomFill\BlogDotFillText}
%% ... actually, it doesn't seem to make a difference! 
%% (2011/10/13)
\newcommand*{\DontFillBlogDotBottom}{\let\BlogDotBottomFill\@empty} 
\DontFillBlogDotBottom
%% |\DontShowBlogDotFillText| makes `\BlogDotFillText' invisible,\\ 
%% |\ShowBlogDotFillText| makes it visible. 
%% Until 2011/10/22, `\textcolor' ('blog.sty') used the 
%% \xmltagcode{font} element that is deprecated. 
%% I still use it here because it seems to suppress the 
%% `hover' \acro{CSS} indication for the link. 
%% (I might offer a choice---TODO)
\newcommand*{\DontShowBlogDotFillText}{%
%     \def\BlogDotFillTextColor{\textcolor{\bodybgcolor}}}
    \def\BlogDotFillTextColor{%
        \TagSurr{font}{color="\bodybgcolor"}}}
\newcommand*{\ShowBlogDotFillText}{%
    \def\BlogDotFillTextColor{\textcolor{red}}}
\DontShowBlogDotFillText
%% As of 2013/01/22, 'texlinks.sty' provides    %% adjusted 2013/01/22
%% `\ctanfileref{<path>}{<file-name>}' that uses an online 
%% \TeX\ archive randomly chosen or determined by the user. 
%% This is preferable for an online version of the presentation. 
%% In `dantev45.htm', this is used for example files.
%% When, on the other hand, internet access during the presentation is 
%% bad, such example files may instead be loaded from the 
%% ``current directory." \ |\usecurrdirctan| \ modifies `\ctanfileref' 
%% for this purpose (i.e., it will ignore <path>):
\newcommand*{\usecurrdirctan}{%
    \renewcommand*{\ctanfileref}[2]{%
        \hnewref{}{##2}{\filenamefmt{##2}}}}
%% (Using a local \acro{TDS} tree would be funny, but I don't 
%%  have good idea for this right now. )
%%
%% |\TryBlogDotCFG| looks for `blogdot.cfg', 
%% \[|\TryBlogDotCFG[<file-name-base>]|\]       %% \[...\] 2011/10/21 
%% looks for `<file-name-base>.cfg' 
%% (for recompiling a certain file):
\newcommand*{\TryBlogDotCFG}[1][blogdot]{%
    \InputIfFileExists{#1.cfg}{%
        \typeout{
            * Using local settings from \string`#1.cfg\string' *}%
    }{}%
}
\TryBlogDotCFG
%%
%% %% rm. \pagebreak 2013/01/04
%% == The End and HISTORY ==
\endinput
%% VERSION HISTORY
v0.1    2011/09/21f.  started
        2011/09/25    spacing/padding off
        2011/09/27    \CLBrk
        2011/09/30    \BlogDotRestart
        used for DANTE meeting
v0.2    2011/10/03    four possibly independent page margin 
                      parameters; \hvspace moves to texblog.fdf
        2011/10/04    renewed \body commented out
        2011/10/07    documentation
        2011/10/08    added some labels
        2011/10/10    v etc. in \ProvidesPackage
        part of morehype RELEASE r0.5 
v0.3    2011/10/11    \HVspace, \BlogDotFillText
        2011/10/12    commands for \BlogDotFillText
        2011/10/13    more doc. on "debugging"; 
                      \ifFillBlogDotTypeArea, \tablerow, messages
        2011/10/14    \maybe@blogdot@frame
        2011/10/15    doc. note: \HVspace useless
        part of morehype RELEASE r0.51 
v0.4    2011/10/21    \usecurrdirctan
        2011/10/22    FillText with <p> instead of </p>, its color 
                      uses <font>; some more reworking of doc.
        part of morehype RELEASE r0.6 
v0.41   2012/11/19    \startscreenpage with \\; doc. \ 
        2012/11/21    updating version infos, doc. \pagebreak
v0.41a  2013/01/04    rm. \pagebreak 
        part of morehype RELEASE r0.81
v0.41b  2013/01/22    adjusted doc. on `texlinks'


\end{document}

HISTORY

2010/11/05   for v0.2
2010/11/11   for v0.3
2011/01/23   using readprov and color
2011/01/27   using \urlfoot
2011/09/01   using new makedoc.cfg incl. \acro and \foothttp...; 
             extension for twocolpg.sty
with morehype RELEASE r0.4
2011/09/02   twocolpg.sty renamed into lnavicol.sty, typo fixes
2011/09/08   \HTML
2011/09/23   TODO in abstract blue again
2011/10/05   umlaut-a in schreibt.tex
2011/10/07f. blogdot
2011/10/09   lnavicol
2011/10/10   tuning; reasons for blogdot
2011/10/11   limitations of blogdot, corrected makedoc code
2011/10/15   more on limitations of blogdot; 
             abstract on lnavicol and blogdot
2011/10/21   links to fifinddo-info/dantev45.htm
2011/11/05   using \MakeSingleDoc from makedoc.sty v0.42
2011/11/08   playing with alternatives to defective `blog.sty' 
             in page head
2011/11/09   so use \file with new makedoc.cfg for hyperref; \CSS 
2011/11/23   \secref
2012/07/19   `textblock' not metavar
2012/08/07   three section levels
2012/10/03   ize -> ese, some restructuring, corr. \HeaderLines
2012/10/05   adjusted \HeaderLines (differ!)
2012/11/29   adding `blogligs' and `markblog'
2012/11/30   hello world, texblog sample removed, url foots
2012/12/20   filedate checks, doc. more about `markblog'
2013/01/04   \pagebreak +/-


\section{``Pervasive Ligatures" with 'blogligs.sty'} %% 2012/11/29
\label{sec:moreligs}
% \AddQuotes
This is the code and documentation of the package mentioned in 
Sec.~\ref{sec:ligs}, loadable by option |[ligs]|.
See below for what is offered. 
\ResetCodeLineNumbers
\NeedsTeXFormat{LaTeX2e}[1994/12/01] %% \newcommand* etc. 
\ProvidesPackage{blogligs}[2012/11/29 v0.2 
                           pervasive blog ligatures (UL)]
%% copyright (C) 2012 Uwe Lueck, 
%% http://www.contact-ednotes.sty.de.vu 
%% -- author-maintained in the sense of LPPL below.
%%
%% This file can be redistributed and/or modified under 
%% the terms of the LaTeX Project Public License; either 
%% version 1.3c of the License, or any later version.
%% The latest version of this license is in
%%     http://www.latex-project.org/lppl.txt
%% We did our best to help you, but there is NO WARRANTY. 
%%
%% Please report bugs, problems, and suggestions via 
%% 
%%   http://www.contact-ednotes.sty.de.vu 
%%
%% == 'blog' Required ==
%% 'blogdot' is an extension of 'blog', and must be loaded \emph{later}
%% (but what about options? TODO):
\RequirePackage{blog}
%% == Task and Idea   ==
%% |\UseBlogLigs| as offered by 'blog.sty' does not work 
%% inside macro arguments. You can use |\ParseLigs{<text>}|
%% at such locations to enable ``ligatures" again. 
%% 'blogligs.sty' saves you from this manual trick. 
%% Many macros have one ``text" argument only, 
%% others additionally have ``attribute" arguments. 
%% Most macros `<elt-cmd>{<text>}' of the first kind are defined 
%% to expand to `\SimpleTagSurr{<elt>}{<text>}'
%% or to `\TagSurr{<elt>}{<attrs>}{<text>}' for some 
%% \HTML\ element <elt> and some attribute assignments <attrs>. 
%% When a macro in addition to a ``text" element has 
%% ``attribute" parameters, `\TagSurr' is used as well.
%% %% 2012/01/08, eigentlich schon 2012/01/04, verloren ...:
% \let\blogtextcolor\textcolor
% \renewcommand*{\textcolor}[2]{\blogtextcolor{#1}{\ParseLigs{#2}}}
%%
%% \pagebreak[2]
%% == Quotation Marks ==
%% ``Inline quote" macros `<qtd>{<text>}' to surround <text>
%% by quotation marks do not follow this rule. We are just 
%% dealing with English and German double quotes 
%% that I have mostly treated by `catchdq.sty'. 
%% `"<text>"' then (eventually) expands to either 
%% `\deqtd{<text>}' or `\endqtd{<text>}', so we redefine these:
%% %% 2012/01/10:
\let\blogdedqtd\dedqtd 
\renewcommand*{\dedqtd}[1]{\blogdedqtd{\ParseLigs{#1}}}
%% %% 2012/08/20:
\let\blogendqtd\endqtd
\renewcommand*{\endqtd}[1]{\blogendqtd{\ParseLigs{#1}}}
%%
%% == \HTML\ Elements ==
%% When the above rule holds:
%% %% 2012/01/19:
\let\BlogTagSurr\TagSurr 
\renewcommand*{\TagSurr}[3]{%
    \BlogTagSurr{#1}{#2}{\ParseLigs{#3}}}
\let\BlogSimpleTagSurr\SimpleTagSurr 
\renewcommand*{\SimpleTagSurr}[2]{%
    \BlogSimpleTagSurr{#1}{\ParseLigs{#2}}}
%%
%% == Avoiding ``Ligatures" though ==
%% |\noligs{<text>}| saves <text> from ``ligature" replacements 
%% (except in arguments of macros inside <text> where 
%%  'blogligs' enables ligatures):
\newcommand*{\noligs}{}     \let\noligs\@firstofone     %% !!!
%% I have found it useful to disable replacements within
%% |\code{<text>}|: 
\renewcommand*{\code}[1]{\STS{code}{\noligs{#1}}}
%% TODO: kind of mistake, `\STS' has not been affected anyway so far, 
%% then defining `\code' as `\STS{code}' should suffice.
%%
%% |\NoBlogLigs| has been meant to disable ``ligatures" altogether again. 
%% I am not sure about everything ...
%% %% 2012/03/14, not optimal TODO:
\renewcommand*{\NoBlogLigs}{%
    \def\BlogOutputJob{LEAVE}%
%     \let\deqtd\blogdeqtd                       %% rm. 2012/06/03
    \let\TagSurr\BlogTagSurr
    \let\SimpleTagSurr\BlogSimpleTagSurr
    \FDnormalTilde 
    \MakeActiveDef\~{&nbsp;}%                    %% TODO new blog cmd
}
%% TODO: |\UseBlogLigs| might be redefined likewise 
%% (\textcolor{red}{in fact 'blogligs' activates ligatures 
%%                  inside text arguments unconditionally at present}, 
%%  I keep this for now since I have used it this way with `texblog.fdf' 
%%  over months, and changing it may be dangerous 
%%  where I have used tricky workarounds to overcome the 
%%  `texblog.fdf' mistake). 
%% But with \[`\BlogInteceptEnvironments'\] this is not needed 
%% when you use `\NoBlogLigs' for the contents of some \LaTeX\ 
%% environment.
%% 
%% == The End and \acro{HISTORY} ==
\endinput
%% VERSION HISTORY
v0.1    2012/01/08ff. developed in `texblog.fdf'
v0.2    2012/11/29    own file


% \DontAddQuotes

\section{Wiki Markup by 'markblog.sty'}     %% 2012/11/29
\label{sec:mark}
\subsection{Introduction}                   %% 2012/12/20
\AddQuotes
This is the code and documentation of the package mentioned in 
Sec.~\ref{sec:ligs}, loadable by option |[mark]|.
See below for what is offered. You should also find a file 
`markblog.htm' that sketches it. Moreover, `texlinks.pdf'
describes in detail to what extent Wikipedia's 
``\wikiref{Help:Links#Piped_links}{piped links}"
with `[[<wikipedia-link>]]' is supported.       %% <...> 2012/12/20

\subsection{Similar Packages}               %% 2012/12/20
'wiki.sty' from the \ctanpkgdref{nicetext} bundle has offered 
some Wikipedia-like markup as a front-end for ordinary 
typesetting with \LaTeX\ (for \acro{DVI}/\acro{PDF}), 
implemented in a way very different from what is going on here, 
rather converting markup sequences \emph{during} typesetting.

More similar to the present approach is the way how 
Wikipedia section titles in package documentation 
is implemented by 'makedoc' from the 'nicetext' bundle, 
based on \strong{preprocessing} by 'fifinddo'.

In general, John MacFarlane's 
\httpref{johnmacfarlane.net/pandoc}{\pkg{pandoc}}
(cf.~\wikideref{pandoc}{German Wikipedia})
converts between wiki-like (simplified) markup and 
\LaTeX\ markup. (It deals with rather fixed 
markup rules, while we here process markup sequences 
independently of an entire markup \emph{language}.)

Another straightforward and well-documented way to 
\emph{preprocess} source files for converting simplified 
markup into \TeX\ markup is \ctanpkgauref{isambert}{Paul Isambert}'s 
\ctanpkgref{interpreter}. It relies on \wikiref{LuaTeX}{\LuaTeX}
where Lua does the preprocessing.

\subsection{Package File Header}            %% 2012/12/20
\ResetCodeLineNumbers
\ProvidesFile{markblog.tex}[2012/11/29 extended blog markup]
\EXECUTE{\def\pkg{\code}\def\HTML{\abbr{HTML}}
         \def\mystrong{\textcolor{\#aa0000}}
         \def\myalert{\textcolor{red}}}
\head %%% \texmetadata 2012/09/23
  \metanamecontent{author}{Uwe L\"uck}
  \metanamecontent{date}{\isotoday}
  \robots{index,follow,noarchive}
\title{blog.sty \& wikis} %%% CMS}
\body 
\begin{center}
\heading2{\CtanPkgRef{morehype}{markblog.sty} %%%{{blog[exec].sty}}
          \& 
          [[Wiki]]s} %%% \wikiref{content management system}{\abbr{CMS}}}
\begin{stdallrulestable}
 \EXECUTE{\let\blogcode\code \def\code#1{\textcolor{\#003300}{\blogcode{#1}}}} %% vs. CSS 
                                                     %% ^^ war 66 2012/09/23
  \pkg{blog.sty} Syntax |    Output      | cf. \HTML | cf. \TeX                         | cf. ...      \cr
  \code{''italics''}    |  ''italics''   | \xmleltcode{i}{italics}
                                                     | \code{\{\cs{it} italics\}}       | Wikipedia    \cr
  \code{'''boldface'''} | '''boldface''' | \xmleltcode{b}{boldface}
                                                     | \code{\{\cs{bf} boldface\}}      | Wikipedia    \cr
  \code{[[Wikipedia]]}  | [[Wikipedia]]  | \xmleltattrcode{a}{href=...}{Wikipedia}}
                                                     | \code{\cs{href}\{...\}\{Wikipedia\}} 
                                                                                        | Wikipedia    \cr
  \code{**mystrong**}   | **mystrong**   | \xmleltattrcode{span}{style=...}{mystrong}}  
                                                     | \code{\cs{textcolor}\{...\}\{mystrong\}} 
                                                                                        | [[Markdown]] \cr
  \code{***myalert***}  | ***myalert***  | \xmleltattrcode{span}{style=...}{myalert}  
                                                     | \code{\cs{textcolor}{red}{myalert}}
                                                                                        | [[Markdown]] \cr
  \code{...}            | ...            | \code{\&hellip;}                 | \cs{dots} |              \cr
  \code{--}, \code{---} | --, ---        | \code{\&ndash;}, \code{\&mdash;} 
                                                     | \code{--}, \code{---} -- "ligs"  |              \cr
  \code{->}, \code{<-}  | ->, <-         | \code{\&rarr;}, \code{\&larr;} 
                                                     | \cs{to}, \cs{gets}               | 
\end{stdallrulestable}
\end{center}
\finish

\DontAddQuotes

%% rm. \pagebreak 2013/01/04
\section{Real Web Pages with 'lnavicol.sty'}
\label{sec:lnavicol}
This is the code and documentation of the package mentioned in 
Sec.~\ref{sec:example-lnavicol}.
\ResetCodeLineNumbers
\ProvidesPackage{lnavicol}[2011/10/13
                           left navigation column with blog.sty]
%%
%% Copyright (C) 2011 Uwe Lueck, 
%% http://www.contact-ednotes.sty.de.vu 
%% -- author-maintained in the sense of LPPL below -- 
%%
%% This file can be redistributed and/or modified under 
%% the terms of the LaTeX Project Public License; either 
%% version 1.3c of the License, or any later version.
%% The latest version of this license is in
%%     http://www.latex-project.org/lppl.txt
%% We did our best to help you, but there is NO WARRANTY. 
%%
%% Please report bugs, problems, and suggestions via 
%% 
%%   http://www.contact-ednotes.sty.de.vu 
%%
%% == 'blog.sty' Required ==
%% ---but what about options (TODO)?    %% 2011/10/09
\RequirePackage{blog} 
%%
%% == Switches ==
%% %% introduced 2011/04/29, it seems
%% There is a ``standard" page width and a ``tight one" 
%% (the latter for contact forms)---|\iftight|:
\newif\iftight 
%% In order to move an anchor to the \emph{top} of the screen 
%% when the anchor is near the page end, the page must get 
%% some extra length by adding empty space at its 
%% bottom---|\ifdeep|:
\newif\ifdeep 
%% 
%% == Page Style Settings (to be set locally) ==
% \newcommand*{\pagebgcolor}{\#f5f5f5}  %% CSS whitesmoke
% \newcommand*{\pagespacing}{\@cellpadding{4} \@cellspacing{7}} 
% \newcommand*{\pagenavicolwidth}{125}
% \newcommand*{\pagemaincolwidth}{584}
% \newcommand*{\pagewholewidth}  {792}
%% == Possible Additions to 'blog.sty' ==
%% === Tables ===
%% |\begin{spancolscell}{<number>}{<style>}|
%% opens an environment that contains a row and a single cell 
%% that will span <number> table cells and have style <style>:
\newenvironment{spancolscell}[2]{%
    \starttr\startTd{\@colspan{#1} #2 % 
                     \@width{100\%}}% %% TODO works? 
    }{\endTd\endtr}
%% The |{hiddencells}| einvironment contains cells that do not align 
%% with other cells in the surrounding table. The purpose is using
%% cells for horizontal spacing.
\newenvironment{hiddencells}
    {\startTable{}\starttr}
    {\endtr\endTable}
%% |{pagehiddencells}| is like `{hiddencells}' except that 
%% the \HTML\ code is indented:
\newenvironment{pagehiddencells}
    {\indentii\hiddencells}
    {\indentii\endhiddencells}
%% |\begin{FixedWidthCell}{<width>}{<style>}| \ opens the 
%% `{FixedWidthCell}' environment. The content will form a cell 
%% of width <width>. <style> are additional formatting parameters:
\newenvironment{FixedWidthCell}[2]
    {\startTd{#2}\startTable{\@width{#1}}%
     \starttr\startTd{}}
    {\endTd\endtr\endTable\endTd}
%% |\tablehspace{<width>}| is a variant of \LaTeX's `\hspace{<glue>}'. 
%% It may appear in a table row: 
\newcommand*{\tablehspace}[1]{\startTd{\@width{#1} /}}
%%
%% === Graphics ===
%% The command names in this section are inspired by the names 
%% in the standard \LaTeX\ \ctanpkgref{graphics} package.
%% (They may need some re-organization TODO.)
%% 
%% |\simpleinclgrf{<file>}| embeds a graphic file <file> without 
%% the tricks of the remaining commands.
\newcommand*{\simpleinclgrf}[1]{\IncludeGrf{alt="" \@border{0}}%
                                           {#1}}
%% |\IncludeGrf{<style>}{<file>}| embeds a graphic file <file> 
%% with style settings <style>:
\newcommand*{\IncludeGrf}[2]{<img #1 src="#2">}
%% |\includegraphic{<width>}{<height>}{<file>}{<border>}{<alt>}{<tooltip>}| 
%% ...:                             %% fine with mdoccorr 2011/10/13
\newcommand*{\includegraphic}[6]{% 
    \IncludeGrf{%
        \@width{#1} \@height{#2} %% data; presentation:
        \@border{#4} 
        alt="#5" \@title{#6}}%
        {#3}}
%% |\insertgraphic{<wd>}{<ht>}{<f>}{<b>}{<align>}{<hsp>}{<vsp>}{<alt>}{<t>}|
%% \\adds <hsp> for the `@hspace' and <vsp> for the `@vspace' 
%% attribute:
\newcommand*{\insertgraphic}[9]{%
    \IncludeGrf{%
        \@width{#1} \@height{#2} %% data; presentation:
        \@border{#4} 
        align="#5" hspace="#6" vspace="#8"
        alt="#8" \@title{#9}}%
        {#3}}
%% |\includegraphic{<wd>}{<ht>}{<file>}{<anchor>}{<border>}{<alt>}{<tooltip>}| 
%% \\uses an image with `\includegraphic' parameters as a link to 
%% <anchor>:
\newcommand*{\inclgrfref}[7]{%
    \fileref{#4}{\includegraphic{#1}{#2}{#3}%
                                {#5}{#6}{#7}}}
%%
%% === \acro{HTTP}/Wikipedia tooltips ===
%% |\httptipref{<tip>}{<www>}{<text>}| \ works like \
%% `\httpref{<www>}{<text>}' except that <tip> appears as ``tooltip":
\newcommand*{\httptipref}[2]{%
  \TagSurr a{\@title{#1}\@href{http://#2}\@target@blank}}
%% |\@target@blank| abbreviates the `@target' setting for 
%% opening the target in a new window or tab:
\newcommand*{\@target@blank}{target="_blank"}
%% % |\wikitipref{<language-code>}{<lemma>}{<text>}| \ 
%% |\wikitipref{<lc>}{<lem>}{<text>}| 
%% works like
%% % \\
%% % `\wikiref{<language-code>}{<lemma>}{<text>}' 
%% `\wikiref{<lc>}{<lem>}{<text>}' 
%% except that 
%% ``Wikipedia" appears as ``tooltip". 
%% |\wikideref| and |\wikienref| are redefined to use it:
\newcommand*{\wikitipref}[2]{%
    \httptipref{Wikipedia}{#1.wikipedia.org/wiki/#2}}
\renewcommand*{\wikideref}{\wikitipref{de}}
\renewcommand*{\wikienref}{\wikitipref{en}}
%%
%% == Page Structure ==
%% The body of the page is a table of three rows and two columns. 
%% === Page Head Row ===
%% |\PAGEHEAD| opens the head row and a single cell that will span 
%% the two columns of the second row.
\newcommand*{\PAGEHEAD}{%
  \startTable{%
    \@align@c\ 
    \@bgcolor{\pagebgcolor}%
    \@border{0}%%                       %% TODO local 
    \pagespacing
    \iftight \else \@width\pagewholewidth \fi 
  }\CLBrk
  %% omitting <tbody>
  \ \comment{ HEAD ROW }\CLBrk
  \indenti\spancolscell{2}{}%
}
% \newcommand*{\headgrf}  [1]{%                     %% rm. 2011/10/09
%     \indentiii\simplecell{\simpleinclgrf{#1}}}
% \newcommand*{\headgrfskiptitle}[3]{%
%   \pagehiddencells
%     \headgrf{#1}\CLBrk
%     \headskip{#2}\CLBrk
%     \headtitle1{#3}\CLBrk
%   \endpagehiddencells}
%% |\headuseskiptitle{<grf>}{<skip>}{<title>}|
%% first places <grf>, then skips horizontally by <skip>, 
%% and then prints the page title as \xmltagcode{h1}:
\newcommand*{\headuseskiptitle}[3]{%
  \pagehiddencells\CLBrk
    \indentiii\simplecell{#1}\CLBrk
    \headskip{#2}\CLBrk
    \headtitle1{#3}\CLBrk
  \endpagehiddencells}
%% |\headskip{<skip>}| is like `\tablehspace{<skip>}'
%% except that the \HTML\ code gets an indent.
\newcommand*{\headskip}    {\indentiii\tablehspace}
%% Similarly, |\headtitle{<digit>}{<text>}| is like 
%% `\heading<digit>{<text>}' apart from an indent and 
%% being put into a cell:
\newcommand*{\headtitle}[2]{\indentiii\simplecell{\heading#1{#2}}}
%%
%% === Navigation and Main Row ===
%% |\PAGENAVI| closes the head row and opens the ``navigation" 
%% column, actually including an `{itemize}' environment.
%% Accordingly, `writings.fdf' has a command `\fileitem'. 
%% But it seems that I have not been sure ...
\newcommand*{\PAGENAVI}{%
    \indenti\endspancolscell\CLBrk
    \indenti\starttr\CLBrk 
    \ \comment{NAVIGATION COL}\CLBrk 
    \indentii\FixedWidthCell\pagenavicolwidth
                           {\@class{paper} 
%% <- using `@class'=`paper' here is my brother's idea, 
%% not sure about it ...
                           \@valign@t}
    %% omitting `\@height{100\%}' 
    \itemize}
%% |\PAGEMAINvar{<width>}| closes the navigation column 
%% and opens the ``main content" column. The latter gets 
%% width <width>:
\newcommand*{\PAGEMAINvar}[1]{%
    \indentii\enditemize\ \endFixedWidthCell\CLBrk
    \ \comment{ MAIN COL }\CLBrk
    \indentii\FixedWidthCell{#1}{}} 
%% ... The width may be specified as |\pagemaincolwidth|, 
%% then |\PAGEMAIN| works like `\PAGEMAINvar{\pagemaincolwidth}':
\newcommand*{\PAGEMAIN}{\PAGEMAINvar\pagemaincolwidth}
%%
%% === Footer Row ===
%% |\PAGEFOOT| closes the ``main content" column as well as 
%% the second row, and opens the footer row:
\newcommand*{\PAGEFOOT}{%
    \indentii\endFixedWidthCell\CLBrk
%     \indentii\tablehspace{96}\CLBrk %% vs. \pagemaincolwidth
  %% <- TODO margin right of foot
    \indenti\endtr\CLBrk
    \ \comment{ FOOT ROW / }\CLBrk
    \indenti\spancolscell{2}{\@class{paper} \@align@c}%
%% <- again class ``paper"!?
}
%% |\PAGEEND| closes the footer row and provides all the rest 
%% ... needed?
\newcommand*{\PAGEEND}{\indenti\endspancolscell\endTable}
%%
%% == The End and HISTORY ==
\endinput

HISTORY 

2011/04/29   started (? \if...)
2011/09/01   to CTAN as `twocolpg.sty'
2011/09/02   renamed
2011/10/09f. documentation more serious 
2011/10/13   `...:' OK


\section{Beamer Presentations with 'blogdot.sty'}
\subsection{Overview}
'blogdot.sty' extends 'blog.sty' in order to construct ``\HTML\ 
slides." One ``slide" is a 3$\times$3 table such that 
\begin{enumerate}
  \item it \strong{fills} the computer \strong{screen}, 
  \item the center cell is the \strong{``type area,"}
  \item the ``margin cell" below the center cell 
        is a \strong{link} to the \strong{next} ``slide,"
  \item the lower right-hand cell is a \strong{``restart"} link.
\end{enumerate}
Six \strong{size parameters} listed in Sec.~\ref{sec:dot-size} 
must be adjusted to the screen in `blogdot.cfg' 
(or in a file with project-specific definitions).

We deliver a file |blogdot.css| containing \strong{\acro{CSS}} font size 
declarations that have been used so far; you may find better ones 
or ones that work better with your screen size, or you may need to add 
style declarations for additional \HTML\ elements.

Another parameter that the user may want to modify is the 
\strong{``restart" anchor name} |\BlogDotRestart| 
(see Sec.~\ref{sec:dot-fin}). 
Its default value is |START| for the ``slide" opened by the command 
|\titlescreenpage| that is defined in Sec.~\ref{sec:dot-start}.

That slide is meant to be the ``\strong{title} slide" 
of the presentation. In order to \strong{display} it, 
I recommend to make and use a \strong{link} to |START| somewhere 
(such as with 'blog.sty''s `\ancref' command). 
The \emph{content} of the title slide is \emph{centered} horizontically, 
so certain commands mentioned \emph{below} 
(centering on other slides) may be useful.

\emph{After} `\titlescreenpage', the next main \strong{user commands} 
are
\begin{description}
  \cmdboxitem|\nextnormalscreenpage{<anchor-name>}| \
    starts a slide whose content is aligned flush left,
  \cmdboxitem|\nextcenterscreenpage{<anchor-name>}| \
    starts a slide whose content is centered horizontally.
\end{description}
---cf.~Sec.~\ref{sec:dot-next}. Right after these commands, 
as well as right after `\title'\-`screen`\-'page', code is used to 
generate the content of the \strong{type area} of the corresponding 
slide. Another `\next...' command closes that content and opens 
another slide. The presentation (the content of the very last slide) 
may be finished using |\screenbottom{<final>}| where <final> may be 
arbitrary, or `START' may be a fine choice for <final>.

Finally, there are user commands for \strong{centering} slide content 
horizontically (cf.~Sec.~\ref{sec:dot-type}): 
\begin{description}
  \cmdboxitem|\cheading{<digit>}{<title>}| \
    ``printing" a heading centered horizontically---even on slides 
    whose remaining content is aligned \emph{flush left} 
    (I have only used <digit>=2 so far), 
  \cmdboxitem|\begin{textblock}{<width>}| \     %% not metavar 2012/07/19
    ``printing" the content of a `{textblock}' environment with 
    maximum line width <width> flush left, 
    while that ``block" as a whole may be centered 
    horizontically on the slide due to choosing 
    `\nextcenterscreenpage'---especially for \strong{list} 
    environments with entry lines that are shorter than the 
    type area width and thus would not look centered 
    (below a centered heading from `\cheading'). 
\end{description}

The so far single \strong{example} of a presentation prepared using 'blogdot' 
is \ctanfileref{info/fifinddo-info}{dantev45.htm}
%% <- 2011/10/21 ->
(\ctanpkgref{fifinddo-info} bundle), 
a sketch of applying 'fifinddo' to package 
documentation and \HTML\ generation. A ``driver" file is needed 
for generating the \HTML\ code for the presentation from a `.tex' 
source by analogy to generating any \HTML\ file using 'blog.sty'. 
For the latter purpose, I have named my driver files `makehtml.tex'. 
For `dantev45.htm', I have called that file |makedot.tex|, 
the main difference to `makehtml.tex' is loading `blogdot.sty' 
in place of `blog.sty'.

This example also uses a file `dantev45.fdf' that defines some 
commands that may be more appropriate as user-level commands 
than the ones presented here (which may appear to be still too 
low-level-like): 
\begin{description}
  \cmdboxitem|\teilpage{<number>}{<title>}|
    making a ``cover slide" for announcing a new ``part" 
    of the presentation in German, 
  \cmdboxitem|\labelsection{<label>}{<title>}|
    starting a slide with heading <title> 
    and with anchor <label> 
    (that is displayed on clicking a \emph{link} to 
     <label>)---using 
     \[`\nextnormalscreenpage{<label>}'\mbox{ and } 
     `\cheading2{<title>}',\] 
  \cmdboxitem|\labelcentersection{<label>}{<title>}|
    like the previous command except that the slide content will be 
    \emph{centered} horizontally, using 
    \[`\nextcenterscreenpage{<title>}'.\]
\end{description}

%% 2011/10/10:
\strong{Reasons} to make \HTML\ presentations may be:\ \
(i)~As opposed to office software, this is a transparent light-weight 
approach.\ \
Considering \emph{typesetting} slides with \TeX,\ \
(ii)~\TeX's advanced typesetting abilities such as automatical 
page breaking are not very relevant for slides;\ \
(iii)~a typesetting run needs a second or a few seconds, 
while generating \HTML\ with 'blog.sty' needs a fraction of a second;\ \ 
(iv)~adjusting formatting parameters such as sizes and colours 
needed for slides is somewhat more straightforward with \HTML\ 
than with \TeX.

%% 2011/10/11, 2011/10/15:
\strong{Limitations:} \
First I was happy about how it worked on my netbook, 
but then I realized how difficult it is to present the ``slides" ``online."
Screen sizes (centering) are one problem. 
(Without the ``restart" idea, this might be much easier.)
Another problem is that the ``hidden links" don't work with 
\Wikienref{Internet Explorer} as they work with 
\Wikienref{Firefox}, \Wikienref{Google Chrome}, and 
\Wikiendisambref{Opera}{web browser}.
% I am now working at an easy choice of ``recompiling options." 
And finally, in internet shops some 
\HTML\ entities/symbols were not supported. 
In any case I (again) became aware of the fact 
that \HTML\ is not as \strong{``portable"} as \acro{PDF}.

Some \strong{workarounds} are described in Sec.~\ref{sec:cfgs}. 
|\FillBlogDotTypeArea| has two effects: \ (i)~providing an additional 
link to the \emph{next} slide for MSIE, \ (ii)~\emph{widening} 
and centering the \emph{type area} on larger screens 
than the one which the presentation originally was made for. \ 
An optional argument of |\TryBlogDotCFG| is offered for a `.cfg' file 
overriding the original settings for the presentation. 
Using it, I learnt that for ``portability," some manual line breaks 
(`\\', \xmltagcode{br}) should be replaced by ``ties" between the 
words \emph{after} the intended line break 
(when the line break is too ugly in a wider type area). 
For keeping the original type area width on wider screens 
(for certain ``slides", perhaps when line breaks really are wanted 
 to be preserved), the |{textblock}| environment may be used. 
Better \HTML\ and \acro{CSS} expertise may eventually 
lead to better solutions. 

The \strong{name} \qtd{blogdot} is a ``pun" on the name of the 
\ctanpkgref{powerdot} package (which in turn refers to 
``\Wikienref{PowerPoint}").

\subsection{File Header}
\ResetCodeLineNumbers
\NeedsTeXFormat{LaTeX2e}[1994/12/01] %% \newcommand* etc. 
\ProvidesPackage{blogdot}[2013/01/22 v0.41b HTML presentations (UL)]
%% copyright (C) 2011 Uwe Lueck, 
%% http://www.contact-ednotes.sty.de.vu 
%% -- author-maintained in the sense of LPPL below.
%%
%% This file can be redistributed and/or modified under 
%% the terms of the LaTeX Project Public License; either 
%% version 1.3c of the License, or any later version.
%% The latest version of this license is in
%%     http://www.latex-project.org/lppl.txt
%% We did our best to help you, but there is NO WARRANTY. 
%%
%% Please report bugs, problems, and suggestions via 
%% 
%%   http://www.contact-ednotes.sty.de.vu 
%%
%% == 'blog' Required ==
%% 'blogdot' is an extension of 'blog' 
%% (but what about options? TODO):
\RequirePackage{blog}
%% == Size Parameters ==
%% \label{sec:dot-size}
%% I assume that it is clear what the following
%% six page dimension parameters 
%% \begin{quote}
%% |\leftpagemargin|, |\rightpagemargin|, 
%% |\upperpagemargin|,\\|\lowerpagemargin|, 
%% |\typeareawidth|, |\typeareaheight|
%% \end{quote}
%% mean. 
%% The choices are what I thought should work best 
%% on my 1024$\times$600 screen (in fullscreen mode); 
%% but I had to optimize the left and right margins experimentally
%% (with Mozilla Firefox~3.6.22 for Ubuntu canonical~-~1.0).
%% It seems to be best when the horizontal parameters 
%% together with what the brouswer adds 
%% (scroll bar, probably 32px with me) 
%% sum up to the screen width.
\newcommand*{\leftpagemargin}{176}
\newcommand*{\rightpagemargin}{\leftpagemargin}
%% So |\rightpagemargin| ultimately is the same as 
%% |\leftpagemargin| as long as you don't redefine it, 
%% and it suffices to `\renewcommand' `\leftpagemargin'
%% in order to get a horizontically centered type area 
%% with user-defined margin widths.---Something analogous
%% applies to |\upperpagemargin| and |\lowerpagemargin|:
\newcommand*{\upperpagemargin}{80}
\newcommand*{\lowerpagemargin}{\upperpagemargin}
%% A difference to the ``horizontal" parameters is 
%% (I expect) that the position of the type area on the 
%% screen is affected by |\upperpagemargin| only, 
%% and you may choose |\lowerpagemargin| just large enough 
%% that the next slide won't be visible on any computer screen 
%% you can think of.
\newcommand*{\typeareawidth}{640}
\newcommand*{\typeareaheight}{440}
%% Centering with respect to web page body may work better on 
%% different screens (2011/10/03), but it doesn't work here
%% (2011/10/04).
% \renewcommand*{\body}{%
%     </head>\CLBrk
%     <body \@bgcolor{\bodybgcolor} \@align@c>}
%% |\CommentBlogDotWholeWidth| procuces no \HTML\ code ...
\global\let\BlogDotWholeWidth\@empty
%% ... unless calculated with |\SumBlogDotWidth|: 
\newcommand*{\SumBlogDotWidth}{%
    \relax{%                        %% \relax 2011/10/22 magic ...
    \count@\typeareawidth
    \advance\count@ \leftpagemargin
    \advance\count@\rightpagemargin
    \typeout{ * blogdot slide width = \the\count@\space*}%
    \xdef\CommentBlogDotWholeWidth{%
        \comment{ slide width = \the\count@\ }}}}
%%
%% == (Backbone for) Starting a ``Slide" ==
%% \label{sec:dot-start}
%% |\startscreenpage{<style>}{<anchor-name>}|
\newcommand*{\startscreenpage}[2]{%% 0 2011/09/25!?:
    \\\CLBrk                                %% 2012/11/19
%% <- `\\' suddenly necessary, likewise in `texblog.fdf'
%%    with `\NextView' and `\nextruleview'. 
%%    Due to recent `firefox'?              %% 2012/11/21
    \startTable{%
        \@cellpadding{0} \@cellspacing{0}%
        \maybe@blogdot@borders              %% 2011/10/12
        \maybe@blogdot@frame                %% 2011/10/14
    }%
    \CLBrk                                  %% 2011/10/03
    \starttr
%% First cell determines both
%% height of upper page margin |\upperpagemargin|
%% and
%% width of left page margin |\leftpagemargin|:
      \startTd{\@width {\leftpagemargin }%
               \@height{\upperpagemargin}}%
%         \textcolor{\bodybgcolor}{XYZ}%
      \endTd
%% Using |\typeareawidth|:
%       \startTd{\@width{\typeareawidth}}\endTd
      \simplecell{%
        \CLBrk
        \hanc{#2}{\hvspace{\typeareawidth}% 
                          {\upperpagemargin}}%
        \CLBrk
      }%
%% Final cell of first row determines right margin width:
      \startTd{\@width{\leftpagemargin}}\endTd
    \endtr
    \starttr
    \emptycell\startTd{\@height{\typeareaheight}#1}%
}
%% |\titlescreenpage| \ (`\STARTscreenpage' TODO?) \ %% 2011/10/03 \ 2012/11/19
%% opens the title page (I thought). To get it to your screen, 
%% (make and) click a link like 
%% \[`\ancref{START}{start presentation}':\]
\newcommand*{\titlescreenpage}{%
    \startscreenpage{\@align@c}{START}}
%% 
%% == Finishing a ``Slide" and ``Restart" (Backbone) ==
%% \label{sec:dot-fin}
%% |\screenbottom{<next-anchor>}| finishes the current slide 
%% and links to the <next-anchor>, the anchor of a slide opened by 
%% \[`\startscreenpage{<style>}{<next-anchor>}'.\] 
%% More precisely, the margin below the type area is that link.
%% The corner at its right is a link to the anchor to whose name 
%% |\BlogDotRestart| expands. 
\newcommand*{\screenbottom}[1]{%
    \ifFillBlogDotTypeArea 
      <p>\ancref{#1}{\BlogDotFillText}%    %% not </p> 2011/10/22
    \fi
    \endTd\emptycell
    \endtr
    \CLBrk
    \tablerow{bottom margin}%                       %% 2011/10/13
      \emptycell
      \CLBrk
      \startTd{\@align@c}%
        \ancref{#1}{\HVspace{\BlogDotBottomFill}%
%% <- seems to be useless now (2011/10/15).
                            {\typeareawidth}%
                            {\lowerpagemargin}}%
      \endTd
      \CLBrk
      \simplecell{\ancref{\BlogDotRestart}% 
                         {\hvspace{\rightpagemargin}% 
                                  {\lowerpagemargin}}}%
    \endtablerow
    \CLBrk
    \endTable
}
%% The default for |\BlogDotRestart| is |START|---the title page. 
%% You can `\renew'\-`command' it so you get to a slide 
%% containing an overview of the presentation.
\newcommand*{\BlogDotRestart}{START}
%% 
%% == Moving to Next ``Slide" (User Level) == 
%% \label{sec:dot-next}
%% |\nextscreenpage{<style>}{<anchor-name>}|
%% puts closing the previous slide and opening the next 
%% one---having anchor name `<anchor-name>'---together.
%% <style> is for style settings for the next page, 
%% made here for choosing between centering the page/slide content 
%% and aligning it flush left.
\newcommand*{\nextscreenpage}[2]{%
    \screenbottom{#2}\CLBrk
    \hrule           \CLBrk 
    \startscreenpage{#1}{#2}}
%% |\nextcenterscreenpage{<anchor-name>}| chooses centering 
%% the slide content:
\newcommand*{\nextcenterscreenpage}{\nextscreenpage{\@align@c}}
%% |\nextnormalscreenpage{<anchor-name>}| chooses flush left
%% on the type area determined by |\typeareawidth|:
\newcommand*{\nextnormalscreenpage}{\nextscreenpage{}}
%% 
%% == Constructs for Type Area ==
%% \label{sec:dot-type}
%% If you want to get centered titles with \xmltagcode{h2} etc., 
%% you should declare this in `.css' files. But you may consider 
%% this way too difficult, and you may prefer to declare this 
%% right in the \HTML\ code. That's what I do! I use 
%% |\cheading{<digit>}{<text>}| for this purpose. 
\newcommand*{\cheading}[1]{\CLBrk\TagSurr{h#1}{\@align@c}}
%% |\begin{textblock}{<width>}| opens a |{textblock}| 
%% environment. The latter will contain text that will be flush left
%% in a narrower text area---of width <width>---than the one 
%% determined by |\typeareawidth|. It may be used on 
%% "centered" slides. It is made for lists whose entries are 
%% so short that the page would look unbalanced under a 
%% centered title with the list adjusted to the left 
%% of the entire type area. (Thinking of standard \LaTeX, 
%% it is almost the `{minipage}' environment, however lacking 
%% the footnote feature, in that respect it is rather 
%% similar to `\parbox' which however is not an environment.)
\newenvironment*{textblock}[1]
    {\startTable{\@width{#1}}\starttr\startTd{}}
    {\endTd\endtr\endTable}
%%
%% == Debugging and `.cfg's ==
%% \label{sec:cfgs}
%% |\ShowBlogDotBorders| shows borders of the page margins 
%% and may be undone by |\DontShowBlogDotBorders|:
\newcommand*{\ShowBlogDotBorders}{%
    \def\maybe@blogdot@borders{rules="all"}}
\newcommand*{\DontShowBlogDotBorders}{%
    \let\maybe@blogdot@borders\@empty}
\DontShowBlogDotBorders
%% %% 2011/10/14:
%% |\ShowBlogDotFrame| shows borders of the page margins 
%% and may be undone by |\DontShowBlogDotFrame|:
\newcommand*{\ShowBlogDotFrame}{%
    \def\maybe@blogdot@frame{\@frame@box}}
\newcommand*{\DontShowBlogDotFrame}{%
    \let\maybe@blogdot@frame\@empty}
\DontShowBlogDotFrame
%% However, the rules seem to affect horizontal positions ...
%%
%% |\BlogDotFillText| is a dirty trick ... seems to widen 
%% %% doc. extended 2011/10/13
%% the type area and this way centers the text on wider screens 
%% than the one used originally. Of course, this can corrupt 
%% intended line breaks. 
\newcommand*{\BlogDotFillText}{%            %% 2011/10/11
    \center
        \BlogDotFillTextColor{%             %% 2011/10/12
%                 X\\X                      %% insufficient
                 X X X X X X X X X X X X X X X X X X X X 
                 X X X X X X X X X X X X X X X X X X X X 
                 X X X X X X X X X X 
                 X X X X X X X X X X 
%                  X X X X X X X X X X X X X X X X X X X X 
        }
    \endcenter
}
%% |\FillBlogDotTypeArea| fills `\BlogDotFillText' into the 
%% type area, also as a link to the next slide. This may widen
%% the type area so that the text is centered on wider screens 
%% than the one the \HTML\ page was made for. The link may serve 
%% as an alternative to the bottom margin link 
%% (which sometimes fails). 
%% `\FillBlogDotTypeArea'                   %% 2011/10/22
%% can be undone 
%% by |\DontFillBlogDotTypeArea|:
\newcommand*{\FillBlogDotTypeArea}{%
    \let\ifFillBlogDotTypeArea\iftrue 
    \typeout{ * blogdot filling type area *}}       %% 2011/10/13
\newcommand*{\DontFillBlogDotTypeArea}{%
    \let\ifFillBlogDotTypeArea\iffalse}
\DontFillBlogDotTypeArea
%% |\FillBlogDotBottom| fills `\BlogDotFillText' into the 
%% center bottom cell. I tried it before `\FillBlogDotTypeArea'
%% and I am not sure ... 
%% It can be undone by 
%% |\DontFillBlogDotBottom|:
\newcommand*{\FillBlogDotBottom}{%
    \let\BlogDotBottomFill\BlogDotFillText}
%% ... actually, it doesn't seem to make a difference! 
%% (2011/10/13)
\newcommand*{\DontFillBlogDotBottom}{\let\BlogDotBottomFill\@empty} 
\DontFillBlogDotBottom
%% |\DontShowBlogDotFillText| makes `\BlogDotFillText' invisible,\\ 
%% |\ShowBlogDotFillText| makes it visible. 
%% Until 2011/10/22, `\textcolor' ('blog.sty') used the 
%% \xmltagcode{font} element that is deprecated. 
%% I still use it here because it seems to suppress the 
%% `hover' \acro{CSS} indication for the link. 
%% (I might offer a choice---TODO)
\newcommand*{\DontShowBlogDotFillText}{%
%     \def\BlogDotFillTextColor{\textcolor{\bodybgcolor}}}
    \def\BlogDotFillTextColor{%
        \TagSurr{font}{color="\bodybgcolor"}}}
\newcommand*{\ShowBlogDotFillText}{%
    \def\BlogDotFillTextColor{\textcolor{red}}}
\DontShowBlogDotFillText
%% As of 2013/01/22, 'texlinks.sty' provides    %% adjusted 2013/01/22
%% `\ctanfileref{<path>}{<file-name>}' that uses an online 
%% \TeX\ archive randomly chosen or determined by the user. 
%% This is preferable for an online version of the presentation. 
%% In `dantev45.htm', this is used for example files.
%% When, on the other hand, internet access during the presentation is 
%% bad, such example files may instead be loaded from the 
%% ``current directory." \ |\usecurrdirctan| \ modifies `\ctanfileref' 
%% for this purpose (i.e., it will ignore <path>):
\newcommand*{\usecurrdirctan}{%
    \renewcommand*{\ctanfileref}[2]{%
        \hnewref{}{##2}{\filenamefmt{##2}}}}
%% (Using a local \acro{TDS} tree would be funny, but I don't 
%%  have good idea for this right now. )
%%
%% |\TryBlogDotCFG| looks for `blogdot.cfg', 
%% \[|\TryBlogDotCFG[<file-name-base>]|\]       %% \[...\] 2011/10/21 
%% looks for `<file-name-base>.cfg' 
%% (for recompiling a certain file):
\newcommand*{\TryBlogDotCFG}[1][blogdot]{%
    \InputIfFileExists{#1.cfg}{%
        \typeout{
            * Using local settings from \string`#1.cfg\string' *}%
    }{}%
}
\TryBlogDotCFG
%%
%% %% rm. \pagebreak 2013/01/04
%% == The End and HISTORY ==
\endinput
%% VERSION HISTORY
v0.1    2011/09/21f.  started
        2011/09/25    spacing/padding off
        2011/09/27    \CLBrk
        2011/09/30    \BlogDotRestart
        used for DANTE meeting
v0.2    2011/10/03    four possibly independent page margin 
                      parameters; \hvspace moves to texblog.fdf
        2011/10/04    renewed \body commented out
        2011/10/07    documentation
        2011/10/08    added some labels
        2011/10/10    v etc. in \ProvidesPackage
        part of morehype RELEASE r0.5 
v0.3    2011/10/11    \HVspace, \BlogDotFillText
        2011/10/12    commands for \BlogDotFillText
        2011/10/13    more doc. on "debugging"; 
                      \ifFillBlogDotTypeArea, \tablerow, messages
        2011/10/14    \maybe@blogdot@frame
        2011/10/15    doc. note: \HVspace useless
        part of morehype RELEASE r0.51 
v0.4    2011/10/21    \usecurrdirctan
        2011/10/22    FillText with <p> instead of </p>, its color 
                      uses <font>; some more reworking of doc.
        part of morehype RELEASE r0.6 
v0.41   2012/11/19    \startscreenpage with \\; doc. \ 
        2012/11/21    updating version infos, doc. \pagebreak
v0.41a  2013/01/04    rm. \pagebreak 
        part of morehype RELEASE r0.81
v0.41b  2013/01/22    adjusted doc. on `texlinks'


\end{document}

HISTORY

2010/11/05   for v0.2
2010/11/11   for v0.3
2011/01/23   using readprov and color
2011/01/27   using \urlfoot
2011/09/01   using new makedoc.cfg incl. \acro and \foothttp...; 
             extension for twocolpg.sty
with morehype RELEASE r0.4
2011/09/02   twocolpg.sty renamed into lnavicol.sty, typo fixes
2011/09/08   \HTML
2011/09/23   TODO in abstract blue again
2011/10/05   umlaut-a in schreibt.tex
2011/10/07f. blogdot
2011/10/09   lnavicol
2011/10/10   tuning; reasons for blogdot
2011/10/11   limitations of blogdot, corrected makedoc code
2011/10/15   more on limitations of blogdot; 
             abstract on lnavicol and blogdot
2011/10/21   links to fifinddo-info/dantev45.htm
2011/11/05   using \MakeSingleDoc from makedoc.sty v0.42
2011/11/08   playing with alternatives to defective `blog.sty' 
             in page head
2011/11/09   so use \file with new makedoc.cfg for hyperref; \CSS 
2011/11/23   \secref
2012/07/19   `textblock' not metavar
2012/08/07   three section levels
2012/10/03   ize -> ese, some restructuring, corr. \HeaderLines
2012/10/05   adjusted \HeaderLines (differ!)
2012/11/29   adding `blogligs' and `markblog'
2012/11/30   hello world, texblog sample removed, url foots
2012/12/20   filedate checks, doc. more about `markblog'
2013/01/04   \pagebreak +/-


\section{``Pervasive Ligatures" with 'blogligs.sty'} %% 2012/11/29
\label{sec:moreligs}
% \AddQuotes
This is the code and documentation of the package mentioned in 
Sec.~\ref{sec:ligs}, loadable by option |[ligs]|.
See below for what is offered. 
\ResetCodeLineNumbers
\NeedsTeXFormat{LaTeX2e}[1994/12/01] %% \newcommand* etc. 
\ProvidesPackage{blogligs}[2012/11/29 v0.2 
                           pervasive blog ligatures (UL)]
%% copyright (C) 2012 Uwe Lueck, 
%% http://www.contact-ednotes.sty.de.vu 
%% -- author-maintained in the sense of LPPL below.
%%
%% This file can be redistributed and/or modified under 
%% the terms of the LaTeX Project Public License; either 
%% version 1.3c of the License, or any later version.
%% The latest version of this license is in
%%     http://www.latex-project.org/lppl.txt
%% We did our best to help you, but there is NO WARRANTY. 
%%
%% Please report bugs, problems, and suggestions via 
%% 
%%   http://www.contact-ednotes.sty.de.vu 
%%
%% == 'blog' Required ==
%% 'blogdot' is an extension of 'blog', and must be loaded \emph{later}
%% (but what about options? TODO):
\RequirePackage{blog}
%% == Task and Idea   ==
%% |\UseBlogLigs| as offered by 'blog.sty' does not work 
%% inside macro arguments. You can use |\ParseLigs{<text>}|
%% at such locations to enable ``ligatures" again. 
%% 'blogligs.sty' saves you from this manual trick. 
%% Many macros have one ``text" argument only, 
%% others additionally have ``attribute" arguments. 
%% Most macros `<elt-cmd>{<text>}' of the first kind are defined 
%% to expand to `\SimpleTagSurr{<elt>}{<text>}'
%% or to `\TagSurr{<elt>}{<attrs>}{<text>}' for some 
%% \HTML\ element <elt> and some attribute assignments <attrs>. 
%% When a macro in addition to a ``text" element has 
%% ``attribute" parameters, `\TagSurr' is used as well.
%% %% 2012/01/08, eigentlich schon 2012/01/04, verloren ...:
% \let\blogtextcolor\textcolor
% \renewcommand*{\textcolor}[2]{\blogtextcolor{#1}{\ParseLigs{#2}}}
%%
%% \pagebreak[2]
%% == Quotation Marks ==
%% ``Inline quote" macros `<qtd>{<text>}' to surround <text>
%% by quotation marks do not follow this rule. We are just 
%% dealing with English and German double quotes 
%% that I have mostly treated by `catchdq.sty'. 
%% `"<text>"' then (eventually) expands to either 
%% `\deqtd{<text>}' or `\endqtd{<text>}', so we redefine these:
%% %% 2012/01/10:
\let\blogdedqtd\dedqtd 
\renewcommand*{\dedqtd}[1]{\blogdedqtd{\ParseLigs{#1}}}
%% %% 2012/08/20:
\let\blogendqtd\endqtd
\renewcommand*{\endqtd}[1]{\blogendqtd{\ParseLigs{#1}}}
%%
%% == \HTML\ Elements ==
%% When the above rule holds:
%% %% 2012/01/19:
\let\BlogTagSurr\TagSurr 
\renewcommand*{\TagSurr}[3]{%
    \BlogTagSurr{#1}{#2}{\ParseLigs{#3}}}
\let\BlogSimpleTagSurr\SimpleTagSurr 
\renewcommand*{\SimpleTagSurr}[2]{%
    \BlogSimpleTagSurr{#1}{\ParseLigs{#2}}}
%%
%% == Avoiding ``Ligatures" though ==
%% |\noligs{<text>}| saves <text> from ``ligature" replacements 
%% (except in arguments of macros inside <text> where 
%%  'blogligs' enables ligatures):
\newcommand*{\noligs}{}     \let\noligs\@firstofone     %% !!!
%% I have found it useful to disable replacements within
%% |\code{<text>}|: 
\renewcommand*{\code}[1]{\STS{code}{\noligs{#1}}}
%% TODO: kind of mistake, `\STS' has not been affected anyway so far, 
%% then defining `\code' as `\STS{code}' should suffice.
%%
%% |\NoBlogLigs| has been meant to disable ``ligatures" altogether again. 
%% I am not sure about everything ...
%% %% 2012/03/14, not optimal TODO:
\renewcommand*{\NoBlogLigs}{%
    \def\BlogOutputJob{LEAVE}%
%     \let\deqtd\blogdeqtd                       %% rm. 2012/06/03
    \let\TagSurr\BlogTagSurr
    \let\SimpleTagSurr\BlogSimpleTagSurr
    \FDnormalTilde 
    \MakeActiveDef\~{&nbsp;}%                    %% TODO new blog cmd
}
%% TODO: |\UseBlogLigs| might be redefined likewise 
%% (\textcolor{red}{in fact 'blogligs' activates ligatures 
%%                  inside text arguments unconditionally at present}, 
%%  I keep this for now since I have used it this way with `texblog.fdf' 
%%  over months, and changing it may be dangerous 
%%  where I have used tricky workarounds to overcome the 
%%  `texblog.fdf' mistake). 
%% But with \[`\BlogInteceptEnvironments'\] this is not needed 
%% when you use `\NoBlogLigs' for the contents of some \LaTeX\ 
%% environment.
%% 
%% == The End and \acro{HISTORY} ==
\endinput
%% VERSION HISTORY
v0.1    2012/01/08ff. developed in `texblog.fdf'
v0.2    2012/11/29    own file


% \DontAddQuotes

\section{Wiki Markup by 'markblog.sty'}     %% 2012/11/29
\label{sec:mark}
\subsection{Introduction}                   %% 2012/12/20
\AddQuotes
This is the code and documentation of the package mentioned in 
Sec.~\ref{sec:ligs}, loadable by option |[mark]|.
See below for what is offered. You should also find a file 
`markblog.htm' that sketches it. Moreover, `texlinks.pdf'
describes in detail to what extent Wikipedia's 
``\wikiref{Help:Links#Piped_links}{piped links}"
with `[[<wikipedia-link>]]' is supported.       %% <...> 2012/12/20

\subsection{Similar Packages}               %% 2012/12/20
'wiki.sty' from the \ctanpkgdref{nicetext} bundle has offered 
some Wikipedia-like markup as a front-end for ordinary 
typesetting with \LaTeX\ (for \acro{DVI}/\acro{PDF}), 
implemented in a way very different from what is going on here, 
rather converting markup sequences \emph{during} typesetting.

More similar to the present approach is the way how 
Wikipedia section titles in package documentation 
is implemented by 'makedoc' from the 'nicetext' bundle, 
based on \strong{preprocessing} by 'fifinddo'.

In general, John MacFarlane's 
\httpref{johnmacfarlane.net/pandoc}{\pkg{pandoc}}
(cf.~\wikideref{pandoc}{German Wikipedia})
converts between wiki-like (simplified) markup and 
\LaTeX\ markup. (It deals with rather fixed 
markup rules, while we here process markup sequences 
independently of an entire markup \emph{language}.)

Another straightforward and well-documented way to 
\emph{preprocess} source files for converting simplified 
markup into \TeX\ markup is \ctanpkgauref{isambert}{Paul Isambert}'s 
\ctanpkgref{interpreter}. It relies on \wikiref{LuaTeX}{\LuaTeX}
where Lua does the preprocessing.

\subsection{Package File Header}            %% 2012/12/20
\ResetCodeLineNumbers
\ProvidesFile{markblog.tex}[2012/11/29 extended blog markup]
\EXECUTE{\def\pkg{\code}\def\HTML{\abbr{HTML}}
         \def\mystrong{\textcolor{\#aa0000}}
         \def\myalert{\textcolor{red}}}
\head %%% \texmetadata 2012/09/23
  \metanamecontent{author}{Uwe L\"uck}
  \metanamecontent{date}{\isotoday}
  \robots{index,follow,noarchive}
\title{blog.sty \& wikis} %%% CMS}
\body 
\begin{center}
\heading2{\CtanPkgRef{morehype}{markblog.sty} %%%{{blog[exec].sty}}
          \& 
          [[Wiki]]s} %%% \wikiref{content management system}{\abbr{CMS}}}
\begin{stdallrulestable}
 \EXECUTE{\let\blogcode\code \def\code#1{\textcolor{\#003300}{\blogcode{#1}}}} %% vs. CSS 
                                                     %% ^^ war 66 2012/09/23
  \pkg{blog.sty} Syntax |    Output      | cf. \HTML | cf. \TeX                         | cf. ...      \cr
  \code{''italics''}    |  ''italics''   | \xmleltcode{i}{italics}
                                                     | \code{\{\cs{it} italics\}}       | Wikipedia    \cr
  \code{'''boldface'''} | '''boldface''' | \xmleltcode{b}{boldface}
                                                     | \code{\{\cs{bf} boldface\}}      | Wikipedia    \cr
  \code{[[Wikipedia]]}  | [[Wikipedia]]  | \xmleltattrcode{a}{href=...}{Wikipedia}}
                                                     | \code{\cs{href}\{...\}\{Wikipedia\}} 
                                                                                        | Wikipedia    \cr
  \code{**mystrong**}   | **mystrong**   | \xmleltattrcode{span}{style=...}{mystrong}}  
                                                     | \code{\cs{textcolor}\{...\}\{mystrong\}} 
                                                                                        | [[Markdown]] \cr
  \code{***myalert***}  | ***myalert***  | \xmleltattrcode{span}{style=...}{myalert}  
                                                     | \code{\cs{textcolor}{red}{myalert}}
                                                                                        | [[Markdown]] \cr
  \code{...}            | ...            | \code{\&hellip;}                 | \cs{dots} |              \cr
  \code{--}, \code{---} | --, ---        | \code{\&ndash;}, \code{\&mdash;} 
                                                     | \code{--}, \code{---} -- "ligs"  |              \cr
  \code{->}, \code{<-}  | ->, <-         | \code{\&rarr;}, \code{\&larr;} 
                                                     | \cs{to}, \cs{gets}               | 
\end{stdallrulestable}
\end{center}
\finish

\DontAddQuotes

%% rm. \pagebreak 2013/01/04
\section{Real Web Pages with 'lnavicol.sty'}
\label{sec:lnavicol}
This is the code and documentation of the package mentioned in 
Sec.~\ref{sec:example-lnavicol}.
\ResetCodeLineNumbers
\ProvidesPackage{lnavicol}[2011/10/13
                           left navigation column with blog.sty]
%%
%% Copyright (C) 2011 Uwe Lueck, 
%% http://www.contact-ednotes.sty.de.vu 
%% -- author-maintained in the sense of LPPL below -- 
%%
%% This file can be redistributed and/or modified under 
%% the terms of the LaTeX Project Public License; either 
%% version 1.3c of the License, or any later version.
%% The latest version of this license is in
%%     http://www.latex-project.org/lppl.txt
%% We did our best to help you, but there is NO WARRANTY. 
%%
%% Please report bugs, problems, and suggestions via 
%% 
%%   http://www.contact-ednotes.sty.de.vu 
%%
%% == 'blog.sty' Required ==
%% ---but what about options (TODO)?    %% 2011/10/09
\RequirePackage{blog} 
%%
%% == Switches ==
%% %% introduced 2011/04/29, it seems
%% There is a ``standard" page width and a ``tight one" 
%% (the latter for contact forms)---|\iftight|:
\newif\iftight 
%% In order to move an anchor to the \emph{top} of the screen 
%% when the anchor is near the page end, the page must get 
%% some extra length by adding empty space at its 
%% bottom---|\ifdeep|:
\newif\ifdeep 
%% 
%% == Page Style Settings (to be set locally) ==
% \newcommand*{\pagebgcolor}{\#f5f5f5}  %% CSS whitesmoke
% \newcommand*{\pagespacing}{\@cellpadding{4} \@cellspacing{7}} 
% \newcommand*{\pagenavicolwidth}{125}
% \newcommand*{\pagemaincolwidth}{584}
% \newcommand*{\pagewholewidth}  {792}
%% == Possible Additions to 'blog.sty' ==
%% === Tables ===
%% |\begin{spancolscell}{<number>}{<style>}|
%% opens an environment that contains a row and a single cell 
%% that will span <number> table cells and have style <style>:
\newenvironment{spancolscell}[2]{%
    \starttr\startTd{\@colspan{#1} #2 % 
                     \@width{100\%}}% %% TODO works? 
    }{\endTd\endtr}
%% The |{hiddencells}| einvironment contains cells that do not align 
%% with other cells in the surrounding table. The purpose is using
%% cells for horizontal spacing.
\newenvironment{hiddencells}
    {\startTable{}\starttr}
    {\endtr\endTable}
%% |{pagehiddencells}| is like `{hiddencells}' except that 
%% the \HTML\ code is indented:
\newenvironment{pagehiddencells}
    {\indentii\hiddencells}
    {\indentii\endhiddencells}
%% |\begin{FixedWidthCell}{<width>}{<style>}| \ opens the 
%% `{FixedWidthCell}' environment. The content will form a cell 
%% of width <width>. <style> are additional formatting parameters:
\newenvironment{FixedWidthCell}[2]
    {\startTd{#2}\startTable{\@width{#1}}%
     \starttr\startTd{}}
    {\endTd\endtr\endTable\endTd}
%% |\tablehspace{<width>}| is a variant of \LaTeX's `\hspace{<glue>}'. 
%% It may appear in a table row: 
\newcommand*{\tablehspace}[1]{\startTd{\@width{#1} /}}
%%
%% === Graphics ===
%% The command names in this section are inspired by the names 
%% in the standard \LaTeX\ \ctanpkgref{graphics} package.
%% (They may need some re-organization TODO.)
%% 
%% |\simpleinclgrf{<file>}| embeds a graphic file <file> without 
%% the tricks of the remaining commands.
\newcommand*{\simpleinclgrf}[1]{\IncludeGrf{alt="" \@border{0}}%
                                           {#1}}
%% |\IncludeGrf{<style>}{<file>}| embeds a graphic file <file> 
%% with style settings <style>:
\newcommand*{\IncludeGrf}[2]{<img #1 src="#2">}
%% |\includegraphic{<width>}{<height>}{<file>}{<border>}{<alt>}{<tooltip>}| 
%% ...:                             %% fine with mdoccorr 2011/10/13
\newcommand*{\includegraphic}[6]{% 
    \IncludeGrf{%
        \@width{#1} \@height{#2} %% data; presentation:
        \@border{#4} 
        alt="#5" \@title{#6}}%
        {#3}}
%% |\insertgraphic{<wd>}{<ht>}{<f>}{<b>}{<align>}{<hsp>}{<vsp>}{<alt>}{<t>}|
%% \\adds <hsp> for the `@hspace' and <vsp> for the `@vspace' 
%% attribute:
\newcommand*{\insertgraphic}[9]{%
    \IncludeGrf{%
        \@width{#1} \@height{#2} %% data; presentation:
        \@border{#4} 
        align="#5" hspace="#6" vspace="#8"
        alt="#8" \@title{#9}}%
        {#3}}
%% |\includegraphic{<wd>}{<ht>}{<file>}{<anchor>}{<border>}{<alt>}{<tooltip>}| 
%% \\uses an image with `\includegraphic' parameters as a link to 
%% <anchor>:
\newcommand*{\inclgrfref}[7]{%
    \fileref{#4}{\includegraphic{#1}{#2}{#3}%
                                {#5}{#6}{#7}}}
%%
%% === \acro{HTTP}/Wikipedia tooltips ===
%% |\httptipref{<tip>}{<www>}{<text>}| \ works like \
%% `\httpref{<www>}{<text>}' except that <tip> appears as ``tooltip":
\newcommand*{\httptipref}[2]{%
  \TagSurr a{\@title{#1}\@href{http://#2}\@target@blank}}
%% |\@target@blank| abbreviates the `@target' setting for 
%% opening the target in a new window or tab:
\newcommand*{\@target@blank}{target="_blank"}
%% % |\wikitipref{<language-code>}{<lemma>}{<text>}| \ 
%% |\wikitipref{<lc>}{<lem>}{<text>}| 
%% works like
%% % \\
%% % `\wikiref{<language-code>}{<lemma>}{<text>}' 
%% `\wikiref{<lc>}{<lem>}{<text>}' 
%% except that 
%% ``Wikipedia" appears as ``tooltip". 
%% |\wikideref| and |\wikienref| are redefined to use it:
\newcommand*{\wikitipref}[2]{%
    \httptipref{Wikipedia}{#1.wikipedia.org/wiki/#2}}
\renewcommand*{\wikideref}{\wikitipref{de}}
\renewcommand*{\wikienref}{\wikitipref{en}}
%%
%% == Page Structure ==
%% The body of the page is a table of three rows and two columns. 
%% === Page Head Row ===
%% |\PAGEHEAD| opens the head row and a single cell that will span 
%% the two columns of the second row.
\newcommand*{\PAGEHEAD}{%
  \startTable{%
    \@align@c\ 
    \@bgcolor{\pagebgcolor}%
    \@border{0}%%                       %% TODO local 
    \pagespacing
    \iftight \else \@width\pagewholewidth \fi 
  }\CLBrk
  %% omitting <tbody>
  \ \comment{ HEAD ROW }\CLBrk
  \indenti\spancolscell{2}{}%
}
% \newcommand*{\headgrf}  [1]{%                     %% rm. 2011/10/09
%     \indentiii\simplecell{\simpleinclgrf{#1}}}
% \newcommand*{\headgrfskiptitle}[3]{%
%   \pagehiddencells
%     \headgrf{#1}\CLBrk
%     \headskip{#2}\CLBrk
%     \headtitle1{#3}\CLBrk
%   \endpagehiddencells}
%% |\headuseskiptitle{<grf>}{<skip>}{<title>}|
%% first places <grf>, then skips horizontally by <skip>, 
%% and then prints the page title as \xmltagcode{h1}:
\newcommand*{\headuseskiptitle}[3]{%
  \pagehiddencells\CLBrk
    \indentiii\simplecell{#1}\CLBrk
    \headskip{#2}\CLBrk
    \headtitle1{#3}\CLBrk
  \endpagehiddencells}
%% |\headskip{<skip>}| is like `\tablehspace{<skip>}'
%% except that the \HTML\ code gets an indent.
\newcommand*{\headskip}    {\indentiii\tablehspace}
%% Similarly, |\headtitle{<digit>}{<text>}| is like 
%% `\heading<digit>{<text>}' apart from an indent and 
%% being put into a cell:
\newcommand*{\headtitle}[2]{\indentiii\simplecell{\heading#1{#2}}}
%%
%% === Navigation and Main Row ===
%% |\PAGENAVI| closes the head row and opens the ``navigation" 
%% column, actually including an `{itemize}' environment.
%% Accordingly, `writings.fdf' has a command `\fileitem'. 
%% But it seems that I have not been sure ...
\newcommand*{\PAGENAVI}{%
    \indenti\endspancolscell\CLBrk
    \indenti\starttr\CLBrk 
    \ \comment{NAVIGATION COL}\CLBrk 
    \indentii\FixedWidthCell\pagenavicolwidth
                           {\@class{paper} 
%% <- using `@class'=`paper' here is my brother's idea, 
%% not sure about it ...
                           \@valign@t}
    %% omitting `\@height{100\%}' 
    \itemize}
%% |\PAGEMAINvar{<width>}| closes the navigation column 
%% and opens the ``main content" column. The latter gets 
%% width <width>:
\newcommand*{\PAGEMAINvar}[1]{%
    \indentii\enditemize\ \endFixedWidthCell\CLBrk
    \ \comment{ MAIN COL }\CLBrk
    \indentii\FixedWidthCell{#1}{}} 
%% ... The width may be specified as |\pagemaincolwidth|, 
%% then |\PAGEMAIN| works like `\PAGEMAINvar{\pagemaincolwidth}':
\newcommand*{\PAGEMAIN}{\PAGEMAINvar\pagemaincolwidth}
%%
%% === Footer Row ===
%% |\PAGEFOOT| closes the ``main content" column as well as 
%% the second row, and opens the footer row:
\newcommand*{\PAGEFOOT}{%
    \indentii\endFixedWidthCell\CLBrk
%     \indentii\tablehspace{96}\CLBrk %% vs. \pagemaincolwidth
  %% <- TODO margin right of foot
    \indenti\endtr\CLBrk
    \ \comment{ FOOT ROW / }\CLBrk
    \indenti\spancolscell{2}{\@class{paper} \@align@c}%
%% <- again class ``paper"!?
}
%% |\PAGEEND| closes the footer row and provides all the rest 
%% ... needed?
\newcommand*{\PAGEEND}{\indenti\endspancolscell\endTable}
%%
%% == The End and HISTORY ==
\endinput

HISTORY 

2011/04/29   started (? \if...)
2011/09/01   to CTAN as `twocolpg.sty'
2011/09/02   renamed
2011/10/09f. documentation more serious 
2011/10/13   `...:' OK


\section{Beamer Presentations with 'blogdot.sty'}
\subsection{Overview}
'blogdot.sty' extends 'blog.sty' in order to construct ``\HTML\ 
slides." One ``slide" is a 3$\times$3 table such that 
\begin{enumerate}
  \item it \strong{fills} the computer \strong{screen}, 
  \item the center cell is the \strong{``type area,"}
  \item the ``margin cell" below the center cell 
        is a \strong{link} to the \strong{next} ``slide,"
  \item the lower right-hand cell is a \strong{``restart"} link.
\end{enumerate}
Six \strong{size parameters} listed in Sec.~\ref{sec:dot-size} 
must be adjusted to the screen in `blogdot.cfg' 
(or in a file with project-specific definitions).

We deliver a file |blogdot.css| containing \strong{\acro{CSS}} font size 
declarations that have been used so far; you may find better ones 
or ones that work better with your screen size, or you may need to add 
style declarations for additional \HTML\ elements.

Another parameter that the user may want to modify is the 
\strong{``restart" anchor name} |\BlogDotRestart| 
(see Sec.~\ref{sec:dot-fin}). 
Its default value is |START| for the ``slide" opened by the command 
|\titlescreenpage| that is defined in Sec.~\ref{sec:dot-start}.

That slide is meant to be the ``\strong{title} slide" 
of the presentation. In order to \strong{display} it, 
I recommend to make and use a \strong{link} to |START| somewhere 
(such as with 'blog.sty''s `\ancref' command). 
The \emph{content} of the title slide is \emph{centered} horizontically, 
so certain commands mentioned \emph{below} 
(centering on other slides) may be useful.

\emph{After} `\titlescreenpage', the next main \strong{user commands} 
are
\begin{description}
  \cmdboxitem|\nextnormalscreenpage{<anchor-name>}| \
    starts a slide whose content is aligned flush left,
  \cmdboxitem|\nextcenterscreenpage{<anchor-name>}| \
    starts a slide whose content is centered horizontally.
\end{description}
---cf.~Sec.~\ref{sec:dot-next}. Right after these commands, 
as well as right after `\title'\-`screen`\-'page', code is used to 
generate the content of the \strong{type area} of the corresponding 
slide. Another `\next...' command closes that content and opens 
another slide. The presentation (the content of the very last slide) 
may be finished using |\screenbottom{<final>}| where <final> may be 
arbitrary, or `START' may be a fine choice for <final>.

Finally, there are user commands for \strong{centering} slide content 
horizontically (cf.~Sec.~\ref{sec:dot-type}): 
\begin{description}
  \cmdboxitem|\cheading{<digit>}{<title>}| \
    ``printing" a heading centered horizontically---even on slides 
    whose remaining content is aligned \emph{flush left} 
    (I have only used <digit>=2 so far), 
  \cmdboxitem|\begin{textblock}{<width>}| \     %% not metavar 2012/07/19
    ``printing" the content of a `{textblock}' environment with 
    maximum line width <width> flush left, 
    while that ``block" as a whole may be centered 
    horizontically on the slide due to choosing 
    `\nextcenterscreenpage'---especially for \strong{list} 
    environments with entry lines that are shorter than the 
    type area width and thus would not look centered 
    (below a centered heading from `\cheading'). 
\end{description}

The so far single \strong{example} of a presentation prepared using 'blogdot' 
is \ctanfileref{info/fifinddo-info}{dantev45.htm}
%% <- 2011/10/21 ->
(\ctanpkgref{fifinddo-info} bundle), 
a sketch of applying 'fifinddo' to package 
documentation and \HTML\ generation. A ``driver" file is needed 
for generating the \HTML\ code for the presentation from a `.tex' 
source by analogy to generating any \HTML\ file using 'blog.sty'. 
For the latter purpose, I have named my driver files `makehtml.tex'. 
For `dantev45.htm', I have called that file |makedot.tex|, 
the main difference to `makehtml.tex' is loading `blogdot.sty' 
in place of `blog.sty'.

This example also uses a file `dantev45.fdf' that defines some 
commands that may be more appropriate as user-level commands 
than the ones presented here (which may appear to be still too 
low-level-like): 
\begin{description}
  \cmdboxitem|\teilpage{<number>}{<title>}|
    making a ``cover slide" for announcing a new ``part" 
    of the presentation in German, 
  \cmdboxitem|\labelsection{<label>}{<title>}|
    starting a slide with heading <title> 
    and with anchor <label> 
    (that is displayed on clicking a \emph{link} to 
     <label>)---using 
     \[`\nextnormalscreenpage{<label>}'\mbox{ and } 
     `\cheading2{<title>}',\] 
  \cmdboxitem|\labelcentersection{<label>}{<title>}|
    like the previous command except that the slide content will be 
    \emph{centered} horizontally, using 
    \[`\nextcenterscreenpage{<title>}'.\]
\end{description}

%% 2011/10/10:
\strong{Reasons} to make \HTML\ presentations may be:\ \
(i)~As opposed to office software, this is a transparent light-weight 
approach.\ \
Considering \emph{typesetting} slides with \TeX,\ \
(ii)~\TeX's advanced typesetting abilities such as automatical 
page breaking are not very relevant for slides;\ \
(iii)~a typesetting run needs a second or a few seconds, 
while generating \HTML\ with 'blog.sty' needs a fraction of a second;\ \ 
(iv)~adjusting formatting parameters such as sizes and colours 
needed for slides is somewhat more straightforward with \HTML\ 
than with \TeX.

%% 2011/10/11, 2011/10/15:
\strong{Limitations:} \
First I was happy about how it worked on my netbook, 
but then I realized how difficult it is to present the ``slides" ``online."
Screen sizes (centering) are one problem. 
(Without the ``restart" idea, this might be much easier.)
Another problem is that the ``hidden links" don't work with 
\Wikienref{Internet Explorer} as they work with 
\Wikienref{Firefox}, \Wikienref{Google Chrome}, and 
\Wikiendisambref{Opera}{web browser}.
% I am now working at an easy choice of ``recompiling options." 
And finally, in internet shops some 
\HTML\ entities/symbols were not supported. 
In any case I (again) became aware of the fact 
that \HTML\ is not as \strong{``portable"} as \acro{PDF}.

Some \strong{workarounds} are described in Sec.~\ref{sec:cfgs}. 
|\FillBlogDotTypeArea| has two effects: \ (i)~providing an additional 
link to the \emph{next} slide for MSIE, \ (ii)~\emph{widening} 
and centering the \emph{type area} on larger screens 
than the one which the presentation originally was made for. \ 
An optional argument of |\TryBlogDotCFG| is offered for a `.cfg' file 
overriding the original settings for the presentation. 
Using it, I learnt that for ``portability," some manual line breaks 
(`\\', \xmltagcode{br}) should be replaced by ``ties" between the 
words \emph{after} the intended line break 
(when the line break is too ugly in a wider type area). 
For keeping the original type area width on wider screens 
(for certain ``slides", perhaps when line breaks really are wanted 
 to be preserved), the |{textblock}| environment may be used. 
Better \HTML\ and \acro{CSS} expertise may eventually 
lead to better solutions. 

The \strong{name} \qtd{blogdot} is a ``pun" on the name of the 
\ctanpkgref{powerdot} package (which in turn refers to 
``\Wikienref{PowerPoint}").

\subsection{File Header}
\ResetCodeLineNumbers
\NeedsTeXFormat{LaTeX2e}[1994/12/01] %% \newcommand* etc. 
\ProvidesPackage{blogdot}[2013/01/22 v0.41b HTML presentations (UL)]
%% copyright (C) 2011 Uwe Lueck, 
%% http://www.contact-ednotes.sty.de.vu 
%% -- author-maintained in the sense of LPPL below.
%%
%% This file can be redistributed and/or modified under 
%% the terms of the LaTeX Project Public License; either 
%% version 1.3c of the License, or any later version.
%% The latest version of this license is in
%%     http://www.latex-project.org/lppl.txt
%% We did our best to help you, but there is NO WARRANTY. 
%%
%% Please report bugs, problems, and suggestions via 
%% 
%%   http://www.contact-ednotes.sty.de.vu 
%%
%% == 'blog' Required ==
%% 'blogdot' is an extension of 'blog' 
%% (but what about options? TODO):
\RequirePackage{blog}
%% == Size Parameters ==
%% \label{sec:dot-size}
%% I assume that it is clear what the following
%% six page dimension parameters 
%% \begin{quote}
%% |\leftpagemargin|, |\rightpagemargin|, 
%% |\upperpagemargin|,\\|\lowerpagemargin|, 
%% |\typeareawidth|, |\typeareaheight|
%% \end{quote}
%% mean. 
%% The choices are what I thought should work best 
%% on my 1024$\times$600 screen (in fullscreen mode); 
%% but I had to optimize the left and right margins experimentally
%% (with Mozilla Firefox~3.6.22 for Ubuntu canonical~-~1.0).
%% It seems to be best when the horizontal parameters 
%% together with what the brouswer adds 
%% (scroll bar, probably 32px with me) 
%% sum up to the screen width.
\newcommand*{\leftpagemargin}{176}
\newcommand*{\rightpagemargin}{\leftpagemargin}
%% So |\rightpagemargin| ultimately is the same as 
%% |\leftpagemargin| as long as you don't redefine it, 
%% and it suffices to `\renewcommand' `\leftpagemargin'
%% in order to get a horizontically centered type area 
%% with user-defined margin widths.---Something analogous
%% applies to |\upperpagemargin| and |\lowerpagemargin|:
\newcommand*{\upperpagemargin}{80}
\newcommand*{\lowerpagemargin}{\upperpagemargin}
%% A difference to the ``horizontal" parameters is 
%% (I expect) that the position of the type area on the 
%% screen is affected by |\upperpagemargin| only, 
%% and you may choose |\lowerpagemargin| just large enough 
%% that the next slide won't be visible on any computer screen 
%% you can think of.
\newcommand*{\typeareawidth}{640}
\newcommand*{\typeareaheight}{440}
%% Centering with respect to web page body may work better on 
%% different screens (2011/10/03), but it doesn't work here
%% (2011/10/04).
% \renewcommand*{\body}{%
%     </head>\CLBrk
%     <body \@bgcolor{\bodybgcolor} \@align@c>}
%% |\CommentBlogDotWholeWidth| procuces no \HTML\ code ...
\global\let\BlogDotWholeWidth\@empty
%% ... unless calculated with |\SumBlogDotWidth|: 
\newcommand*{\SumBlogDotWidth}{%
    \relax{%                        %% \relax 2011/10/22 magic ...
    \count@\typeareawidth
    \advance\count@ \leftpagemargin
    \advance\count@\rightpagemargin
    \typeout{ * blogdot slide width = \the\count@\space*}%
    \xdef\CommentBlogDotWholeWidth{%
        \comment{ slide width = \the\count@\ }}}}
%%
%% == (Backbone for) Starting a ``Slide" ==
%% \label{sec:dot-start}
%% |\startscreenpage{<style>}{<anchor-name>}|
\newcommand*{\startscreenpage}[2]{%% 0 2011/09/25!?:
    \\\CLBrk                                %% 2012/11/19
%% <- `\\' suddenly necessary, likewise in `texblog.fdf'
%%    with `\NextView' and `\nextruleview'. 
%%    Due to recent `firefox'?              %% 2012/11/21
    \startTable{%
        \@cellpadding{0} \@cellspacing{0}%
        \maybe@blogdot@borders              %% 2011/10/12
        \maybe@blogdot@frame                %% 2011/10/14
    }%
    \CLBrk                                  %% 2011/10/03
    \starttr
%% First cell determines both
%% height of upper page margin |\upperpagemargin|
%% and
%% width of left page margin |\leftpagemargin|:
      \startTd{\@width {\leftpagemargin }%
               \@height{\upperpagemargin}}%
%         \textcolor{\bodybgcolor}{XYZ}%
      \endTd
%% Using |\typeareawidth|:
%       \startTd{\@width{\typeareawidth}}\endTd
      \simplecell{%
        \CLBrk
        \hanc{#2}{\hvspace{\typeareawidth}% 
                          {\upperpagemargin}}%
        \CLBrk
      }%
%% Final cell of first row determines right margin width:
      \startTd{\@width{\leftpagemargin}}\endTd
    \endtr
    \starttr
    \emptycell\startTd{\@height{\typeareaheight}#1}%
}
%% |\titlescreenpage| \ (`\STARTscreenpage' TODO?) \ %% 2011/10/03 \ 2012/11/19
%% opens the title page (I thought). To get it to your screen, 
%% (make and) click a link like 
%% \[`\ancref{START}{start presentation}':\]
\newcommand*{\titlescreenpage}{%
    \startscreenpage{\@align@c}{START}}
%% 
%% == Finishing a ``Slide" and ``Restart" (Backbone) ==
%% \label{sec:dot-fin}
%% |\screenbottom{<next-anchor>}| finishes the current slide 
%% and links to the <next-anchor>, the anchor of a slide opened by 
%% \[`\startscreenpage{<style>}{<next-anchor>}'.\] 
%% More precisely, the margin below the type area is that link.
%% The corner at its right is a link to the anchor to whose name 
%% |\BlogDotRestart| expands. 
\newcommand*{\screenbottom}[1]{%
    \ifFillBlogDotTypeArea 
      <p>\ancref{#1}{\BlogDotFillText}%    %% not </p> 2011/10/22
    \fi
    \endTd\emptycell
    \endtr
    \CLBrk
    \tablerow{bottom margin}%                       %% 2011/10/13
      \emptycell
      \CLBrk
      \startTd{\@align@c}%
        \ancref{#1}{\HVspace{\BlogDotBottomFill}%
%% <- seems to be useless now (2011/10/15).
                            {\typeareawidth}%
                            {\lowerpagemargin}}%
      \endTd
      \CLBrk
      \simplecell{\ancref{\BlogDotRestart}% 
                         {\hvspace{\rightpagemargin}% 
                                  {\lowerpagemargin}}}%
    \endtablerow
    \CLBrk
    \endTable
}
%% The default for |\BlogDotRestart| is |START|---the title page. 
%% You can `\renew'\-`command' it so you get to a slide 
%% containing an overview of the presentation.
\newcommand*{\BlogDotRestart}{START}
%% 
%% == Moving to Next ``Slide" (User Level) == 
%% \label{sec:dot-next}
%% |\nextscreenpage{<style>}{<anchor-name>}|
%% puts closing the previous slide and opening the next 
%% one---having anchor name `<anchor-name>'---together.
%% <style> is for style settings for the next page, 
%% made here for choosing between centering the page/slide content 
%% and aligning it flush left.
\newcommand*{\nextscreenpage}[2]{%
    \screenbottom{#2}\CLBrk
    \hrule           \CLBrk 
    \startscreenpage{#1}{#2}}
%% |\nextcenterscreenpage{<anchor-name>}| chooses centering 
%% the slide content:
\newcommand*{\nextcenterscreenpage}{\nextscreenpage{\@align@c}}
%% |\nextnormalscreenpage{<anchor-name>}| chooses flush left
%% on the type area determined by |\typeareawidth|:
\newcommand*{\nextnormalscreenpage}{\nextscreenpage{}}
%% 
%% == Constructs for Type Area ==
%% \label{sec:dot-type}
%% If you want to get centered titles with \xmltagcode{h2} etc., 
%% you should declare this in `.css' files. But you may consider 
%% this way too difficult, and you may prefer to declare this 
%% right in the \HTML\ code. That's what I do! I use 
%% |\cheading{<digit>}{<text>}| for this purpose. 
\newcommand*{\cheading}[1]{\CLBrk\TagSurr{h#1}{\@align@c}}
%% |\begin{textblock}{<width>}| opens a |{textblock}| 
%% environment. The latter will contain text that will be flush left
%% in a narrower text area---of width <width>---than the one 
%% determined by |\typeareawidth|. It may be used on 
%% "centered" slides. It is made for lists whose entries are 
%% so short that the page would look unbalanced under a 
%% centered title with the list adjusted to the left 
%% of the entire type area. (Thinking of standard \LaTeX, 
%% it is almost the `{minipage}' environment, however lacking 
%% the footnote feature, in that respect it is rather 
%% similar to `\parbox' which however is not an environment.)
\newenvironment*{textblock}[1]
    {\startTable{\@width{#1}}\starttr\startTd{}}
    {\endTd\endtr\endTable}
%%
%% == Debugging and `.cfg's ==
%% \label{sec:cfgs}
%% |\ShowBlogDotBorders| shows borders of the page margins 
%% and may be undone by |\DontShowBlogDotBorders|:
\newcommand*{\ShowBlogDotBorders}{%
    \def\maybe@blogdot@borders{rules="all"}}
\newcommand*{\DontShowBlogDotBorders}{%
    \let\maybe@blogdot@borders\@empty}
\DontShowBlogDotBorders
%% %% 2011/10/14:
%% |\ShowBlogDotFrame| shows borders of the page margins 
%% and may be undone by |\DontShowBlogDotFrame|:
\newcommand*{\ShowBlogDotFrame}{%
    \def\maybe@blogdot@frame{\@frame@box}}
\newcommand*{\DontShowBlogDotFrame}{%
    \let\maybe@blogdot@frame\@empty}
\DontShowBlogDotFrame
%% However, the rules seem to affect horizontal positions ...
%%
%% |\BlogDotFillText| is a dirty trick ... seems to widen 
%% %% doc. extended 2011/10/13
%% the type area and this way centers the text on wider screens 
%% than the one used originally. Of course, this can corrupt 
%% intended line breaks. 
\newcommand*{\BlogDotFillText}{%            %% 2011/10/11
    \center
        \BlogDotFillTextColor{%             %% 2011/10/12
%                 X\\X                      %% insufficient
                 X X X X X X X X X X X X X X X X X X X X 
                 X X X X X X X X X X X X X X X X X X X X 
                 X X X X X X X X X X 
                 X X X X X X X X X X 
%                  X X X X X X X X X X X X X X X X X X X X 
        }
    \endcenter
}
%% |\FillBlogDotTypeArea| fills `\BlogDotFillText' into the 
%% type area, also as a link to the next slide. This may widen
%% the type area so that the text is centered on wider screens 
%% than the one the \HTML\ page was made for. The link may serve 
%% as an alternative to the bottom margin link 
%% (which sometimes fails). 
%% `\FillBlogDotTypeArea'                   %% 2011/10/22
%% can be undone 
%% by |\DontFillBlogDotTypeArea|:
\newcommand*{\FillBlogDotTypeArea}{%
    \let\ifFillBlogDotTypeArea\iftrue 
    \typeout{ * blogdot filling type area *}}       %% 2011/10/13
\newcommand*{\DontFillBlogDotTypeArea}{%
    \let\ifFillBlogDotTypeArea\iffalse}
\DontFillBlogDotTypeArea
%% |\FillBlogDotBottom| fills `\BlogDotFillText' into the 
%% center bottom cell. I tried it before `\FillBlogDotTypeArea'
%% and I am not sure ... 
%% It can be undone by 
%% |\DontFillBlogDotBottom|:
\newcommand*{\FillBlogDotBottom}{%
    \let\BlogDotBottomFill\BlogDotFillText}
%% ... actually, it doesn't seem to make a difference! 
%% (2011/10/13)
\newcommand*{\DontFillBlogDotBottom}{\let\BlogDotBottomFill\@empty} 
\DontFillBlogDotBottom
%% |\DontShowBlogDotFillText| makes `\BlogDotFillText' invisible,\\ 
%% |\ShowBlogDotFillText| makes it visible. 
%% Until 2011/10/22, `\textcolor' ('blog.sty') used the 
%% \xmltagcode{font} element that is deprecated. 
%% I still use it here because it seems to suppress the 
%% `hover' \acro{CSS} indication for the link. 
%% (I might offer a choice---TODO)
\newcommand*{\DontShowBlogDotFillText}{%
%     \def\BlogDotFillTextColor{\textcolor{\bodybgcolor}}}
    \def\BlogDotFillTextColor{%
        \TagSurr{font}{color="\bodybgcolor"}}}
\newcommand*{\ShowBlogDotFillText}{%
    \def\BlogDotFillTextColor{\textcolor{red}}}
\DontShowBlogDotFillText
%% As of 2013/01/22, 'texlinks.sty' provides    %% adjusted 2013/01/22
%% `\ctanfileref{<path>}{<file-name>}' that uses an online 
%% \TeX\ archive randomly chosen or determined by the user. 
%% This is preferable for an online version of the presentation. 
%% In `dantev45.htm', this is used for example files.
%% When, on the other hand, internet access during the presentation is 
%% bad, such example files may instead be loaded from the 
%% ``current directory." \ |\usecurrdirctan| \ modifies `\ctanfileref' 
%% for this purpose (i.e., it will ignore <path>):
\newcommand*{\usecurrdirctan}{%
    \renewcommand*{\ctanfileref}[2]{%
        \hnewref{}{##2}{\filenamefmt{##2}}}}
%% (Using a local \acro{TDS} tree would be funny, but I don't 
%%  have good idea for this right now. )
%%
%% |\TryBlogDotCFG| looks for `blogdot.cfg', 
%% \[|\TryBlogDotCFG[<file-name-base>]|\]       %% \[...\] 2011/10/21 
%% looks for `<file-name-base>.cfg' 
%% (for recompiling a certain file):
\newcommand*{\TryBlogDotCFG}[1][blogdot]{%
    \InputIfFileExists{#1.cfg}{%
        \typeout{
            * Using local settings from \string`#1.cfg\string' *}%
    }{}%
}
\TryBlogDotCFG
%%
%% %% rm. \pagebreak 2013/01/04
%% == The End and HISTORY ==
\endinput
%% VERSION HISTORY
v0.1    2011/09/21f.  started
        2011/09/25    spacing/padding off
        2011/09/27    \CLBrk
        2011/09/30    \BlogDotRestart
        used for DANTE meeting
v0.2    2011/10/03    four possibly independent page margin 
                      parameters; \hvspace moves to texblog.fdf
        2011/10/04    renewed \body commented out
        2011/10/07    documentation
        2011/10/08    added some labels
        2011/10/10    v etc. in \ProvidesPackage
        part of morehype RELEASE r0.5 
v0.3    2011/10/11    \HVspace, \BlogDotFillText
        2011/10/12    commands for \BlogDotFillText
        2011/10/13    more doc. on "debugging"; 
                      \ifFillBlogDotTypeArea, \tablerow, messages
        2011/10/14    \maybe@blogdot@frame
        2011/10/15    doc. note: \HVspace useless
        part of morehype RELEASE r0.51 
v0.4    2011/10/21    \usecurrdirctan
        2011/10/22    FillText with <p> instead of </p>, its color 
                      uses <font>; some more reworking of doc.
        part of morehype RELEASE r0.6 
v0.41   2012/11/19    \startscreenpage with \\; doc. \ 
        2012/11/21    updating version infos, doc. \pagebreak
v0.41a  2013/01/04    rm. \pagebreak 
        part of morehype RELEASE r0.81
v0.41b  2013/01/22    adjusted doc. on `texlinks'


\end{document}

HISTORY

2010/11/05   for v0.2
2010/11/11   for v0.3
2011/01/23   using readprov and color
2011/01/27   using \urlfoot
2011/09/01   using new makedoc.cfg incl. \acro and \foothttp...; 
             extension for twocolpg.sty
with morehype RELEASE r0.4
2011/09/02   twocolpg.sty renamed into lnavicol.sty, typo fixes
2011/09/08   \HTML
2011/09/23   TODO in abstract blue again
2011/10/05   umlaut-a in schreibt.tex
2011/10/07f. blogdot
2011/10/09   lnavicol
2011/10/10   tuning; reasons for blogdot
2011/10/11   limitations of blogdot, corrected makedoc code
2011/10/15   more on limitations of blogdot; 
             abstract on lnavicol and blogdot
2011/10/21   links to fifinddo-info/dantev45.htm
2011/11/05   using \MakeSingleDoc from makedoc.sty v0.42
2011/11/08   playing with alternatives to defective `blog.sty' 
             in page head
2011/11/09   so use \file with new makedoc.cfg for hyperref; \CSS 
2011/11/23   \secref
2012/07/19   `textblock' not metavar
2012/08/07   three section levels
2012/10/03   ize -> ese, some restructuring, corr. \HeaderLines
2012/10/05   adjusted \HeaderLines (differ!)
2012/11/29   adding `blogligs' and `markblog'
2012/11/30   hello world, texblog sample removed, url foots
2012/12/20   filedate checks, doc. more about `markblog'
2013/01/04   \pagebreak +/-
