\immediate\write17{^^J%
+----------------------------------------------------------------------^^J%
| WARNING (`scrlayer-notecolumn.tex'):^^J%
+----------------------------------------------------------------------^^J%
| YOU SHOULD NOTE, THAT THIS IS NOT AN OFFICIAL RELEASE!^^J%
| THIS VERSION IS INTENDED TO BE USED FOR TESTING ONLY!^^J%
| USAGE FOR PURPOSES OTHER THAN INTENDED IS DANGEROUS!^^J%
| DON'T EXPECT ANY SUPPORT FOR THIS VERSION!^^J%
+----------------------------------------------------------------------^^J%
}
%%
%% This is file `scrlayer-notecolumn.tex',
%% generated with the docstrip utility, extended by scrdocstrip.
%%
%% The original source files were:
%%
%% scrlayer-notecolumn.dtx  (with options: `doc')
%% 
%% Copyright (c) 1994-2013 Markus Kohm [komascript at gmx info]
%% 
%% This file was generated from file(s) of the KOMA-Script bundle.
%% ---------------------------------------------------------------
%% 
%% It may be distributed under the conditions of the
%% LaTeX Project Public License in the version distributed together
%% with KOMA-Script, see file `lppl.txt' or `lppl-de.txt'.
%% 
%% This file may only be distributed together with a copy of the
%% KOMA-Script bundle.  You may however distribute the
%% KOMA-Script bundle without all such generated files.  See also
%% `lppl.txt' or `lppl-de.txt' for additional information.
%% 
%% The list of files belonging to KOMA-Script distribution is given in
%% the file `manifest.txt'.  See also `lppl.txt' or `lppl-de.txt' for
%% additional information.
%% 
%% If this file is a beta version, you may not be allowed to distribute
%% it.  Look at the comments below to see if this is the case.
%% 
%% English and German manuals are part of KOMA-Script bundle.
%% ----------------------------------------------------------
%% 
%% See `README'.
%% 
%% The KOMA-Script bundle (but maybe not this file) was based upon the
%% LaTeX 2.09 Script family created by Frank Neukam 1993 and the LaTeX2e
%% standard classes created by The LaTeX3 Project 1994-1996.  You may
%% find a complete unmodified copy of LaTeX2e at
%% <http://www.ctan.org/pub/tex-archive/macros/latex/base/>.
%% 
\ProvidesFile{scrlayer-notecolumn.tex}[%
  15/11/2013 v0.1.1514
  (end user interface for scrlayer)]

\chapter[{End User Interface \Package{scrlayer-notecolumn}}]{End User
  Interface \Package{scrlayer-notecolumn}\footnote{This is not a translation
  of the corresponding chapter of the German \KOMAScript{} manual but the
  tentative and primary implementation manual of package
  \Package{scrlayer-notecolumn}. As long as \Package{scrlayer-notecolumn} is
  primary a proof of concept this doesn't matter. But nevertheless, a
  German to English translator would still be welcome.}}
\label{sec:scrlayer.notecolumn}
\labelbase{scrlayer-notecolumn}

Sometimes documents need an additional notes column beside the main text
column and maybe beside the marginal note column. Problematicly may be
having page breaks inside the notes column. This package provides a
solution based on the layer interface of \Package{scrlayer}. You may
define several notes columns, each of them with page breaks
inside. It supports automatic page break inside of notes and recognises
note collisions and solve them by moving notes.

The idea of this package raised while an e-mail contact with Heiner
Richter from FH~Stralsund, who told me of a book project, and asked
whether or not such a functionality may be added to another package of
mine easily. It couldn't, but design and implementation of a first test
alpha of \Package{scrlayer-notecolumn} has been done in about three
hours. So maybe it was a really good idea to ask me the question.

\begin{Declaration}
  \Macro{DeclareNoteColumn}%
  \OParameter{option list}\Parameter{note column name}%
  \\
  \Macro{DeclareNewNoteColumn}%
  \OParameter{option list}\Parameter{note column name}\\%^^A
  \Macro{ProvideNoteColumn}%
  \OParameter{option list}\Parameter{note column name}\\%^^A
  \Macro{RedeclareNoteColumn}%
  \OParameter{option list}\Parameter{note column name}%
\end{Declaration}
\BeginIndex{Cmd}{DeclareNoteColumn}%
\BeginIndex{Cmd}{DeclareNewNoteColumn}%
\BeginIndex{Cmd}{ProvideNoteColumn}%
\BeginIndex{Cmd}{RedeclareNoteColumn}%
Command \Macro{DeclareNoteColumn}
may be used to declare a note column. Each note column is referenced by its
name, defined by the obligatory argument \PName{note column name}. A note
column vertically span the text area.

By default its horizontal position is the current position of the marginal
note column. It also regards \Macro{reversemarginpar} and
\Macro{normalmarginpar}, while output of a note. Note that this is not the
same like making a margin note. The default width is
\Length{marginparwidth}. But you may use the optional argument \PName{option
  list} to change this behaviour.

The \PName{option list} is a comma-separated list of
\texttt{\PName{key}=\PName{value}} options. Currently the following
\PName{key}s are supported:
\begin{labeling}[\ --]{\KOption{position}\PName{horizontal offset}}
\item [\KOption{font}\PName{font declaration}] set the initial font attributes
  to be used for the note column. Everything, that is allowed to be set by
  \Macro{setkomafont} or \Macro{addtokomafont} can be used as
  \PName{font declaration}. Note, that
  \Macro{normalfont}\Macro{normalsize} will be used
  before. So you don't need one of them at your own initialisation.
\item [\Option{marginpar}] set up \Option{position} and \Option{width} to use
  the marginal note column of \Macro{marginpar}.
\item [\Option{normalmarginpar}] set up \Option{position} and \Option{width}
  to use the normal marginal note column and ignore \Macro{reversemarginpar}
  and \Macro{normalmarginpar}.
\item [\KOption{position}\PName{horizontal offset}] set up the horizontal
  offset of the note column from the left edge of the paper. The
  \PName{horizontal offset} can be either a \LaTeX{} length, a \TeX{}
  dimension, a \TeX{} skip, a length value, or a dimensional expression using
  \texttt{+}, \texttt{-}, \texttt{*}, \texttt{/} and parenthesis (see
  \cite[section~3.5]{manual:eTeX} for more information about dimensional
  expressions). The value will be calculated at usage time not at definition
  time. So you may even use more complex constructs like:
\begin{lstcode}
  \ifodd\value{page}
    \oddsidemargin+1in
    +\textwidth+\marginparsep
  \else
    \evensidemargin+1in
    -\marginparsep
    -\marginparwidth
  \fi
\end{lstcode}
\item [\Option{reversemarginpar}] set up \Option{position} and \Option{width}
  to use the reverse marginal note column and ignore \Macro{reversemarginpar}
  and \Macro{normalmarginpar}.
\item [\KOption{width}\PName{size}] set up the horizontal size of the note
  column. You can use the same values for \PName{size} like for
  \PName{horizontal offset} of option \Option{position}.
\end{labeling}

You may use as much note columns as \LaTeX{} provides free box registers,
because every note column needs one box register. You may use package
\Package{etex} to increase the normal amount of box registers from 256 to
32768. This should be enough to avoid the ``\texttt{no room for
  \Macro{newbox}}'' error.

The implementation of note columns uses foreground layers from package
\Package{scrlayer}. Every \Macro{DeclareNoteColumn} generates a layer and adds
this layer to the pseudo page style \Pagestyle{@everystyle@}. Theoretically
you may modify the generated layer, but it is recommended not to modify them.

Overlapping of different note columns is allowed. There's one predefined note
column named ``\texttt{marginpar}''.

\BeginIndex{FontElement}{notecolumn.\PName{note column name}}%
Note, that if you change the initial font using
\begin{quote}
  \Macro{setkomafont}\linebreak[1]\PParameter{marginnote.\PName{note column
      name}}\Parameter{font declaration}
\end{quote}
or
\begin{quote}
  \Macro{addtokomafont}\linebreak[1]\PParameter{marginnote.\PName{note column
      name}}\Parameter{font declaration}
\end{quote}
the font of all succeeding note column outputs would change. This could make it
hard to find out, which notes will be affected by the change.
See \autoref{sec:maincls.textmarkup},
\autopageref{desc:maincls.cmd.setkomafont} for more information about
\Macro{setkomafont} and \Macro{addtokomafont}.%
\EndIndex{FontElement}{notecolumn.\PName{note column name}}%

The difference between \Macro{DeclareNoteColumn} and the other commands is,
that \Macro{DeclareNoteColumn} defines the note column independent whether a
note column with the same name has already been defined or
not. \Macro{DeclareNewNoteColumn} defined only note columns which names are
not yet used for another note column. It's an error to use it with an already
used \PName{note column name}. \Macro{ProvideNoteColumn} is like
\Macro{DeclareNewNoteColumn}, but if the \PName{note column name} is already
used for a note column nothing happens. \Macro{RedeclareNoteColumn} may be
used to declare a note column with a name, that is already used. Using it with
a new name is an error.
\EndIndex{Cmd}{RedeclareNoteColumn}%^^A
\EndIndex{Cmd}{ProvideNoteColumn}%^^A
\EndIndex{Cmd}{DeclareNewNoteColumn}%^^A
\EndIndex{Cmd}{DeclareNoteColumn}%^^A

\begin{Declaration}
  \Macro{makenote}%
  \OParameter{note column name}\Parameter{note text}
\end{Declaration}
\BeginIndex{Cmd}{makenote}%
This command
may be used to make a new note. The note will be written to a helper file with
extension \File{.slnc} in this run. In the next run, it will be
typeset. Normally it will be typeset with the same vertical position, that
would be used for a \Macro{parbox[t]}, but horizontally moved to the note
column. Collisions at this position of this note column (but not of other note
columns) will be recognised. In such a case the note may be moved down and
even to another page. Automatic page breaks at the note columns are
supported. This is something different to \Macro{marginpar},
\Macro{marginline} (\KOMAScript{} classes), or \Macro{marginnote} (package
\Package{marginnote}).

Note, that after changing the document, at least two \LaTeX{} runs are needed
to move the note to its final position.

Note, that fragile commands should be protected to avoid error messages while
writing or reading the helper file.

Note, that \Package{scrlayer-notecolumn} needs only one helper file for all
note columns together. You should not edit or change that helper file, because
the syntax is fragile. It won't be read using \Macro{input} but
\Macro{read}. Because of this, line breaks, empty lines and \Macro{par} should
be avoided.

Currently it is unknown where you may use \Macro{makenote} and where not. But
you must not use \Macro{makenote} inside of \Macro{makenote}!

The default for the optional argument \PName{makenote} is \PValue{marginpar},
that is the \PName{note column name} of the pre-defined note column.
\EndIndex{Cmd}{makenote}

\begin{Declaration}
  \KOption{autoclearnotecolumns}\PName{simple switch}\\
  \Macro{clearnotecolumn}\OParameter{name of a note column}%
  \\
  \Macro{clearnotecolumns}\OParameter{list of note column names}
\end{Declaration}
\BeginIndex{Cmd}{clearnotcolumn}%
\BeginIndex{Cmd}{clearnotcolumns}%
The commands
may be used to output all hanging notes for either one or several note
columns. With \Macro{clearnotecolumn} only the notes of the note column with
the given \PName{name of a note column} will be output. If you omit the
optional argument (this is not the same like using an empty value!) note
column \PValue{marginnote} will be used.

With \Macro{clearnotecolumns} the notes of all note columns of the comma
separated \PName{list of note column names} will be output. If you omit the
optional Argument or if the optional Argument is empty (or start with
\Macro{relax}) the notes of all note columns will be output.

Macro \Macro{clearpage} will be used to output hanging notes and several
\LaTeX{} runs may be needed to terminate with the correct number of
\Macro{clearpage} for the whole document. You will get the message ``\LaTeX{}
Warning: Label(s) may have changed. Rerun to get cross-references right.'' as
long as the process hasn't been terminated.

But normally you don't need to us \Macro{clearnotecolum} or
\Macro{clearnotecolumns} yourself. Because by default option
\Option{autoclearnotecolumns} is switched on, which means, that
\Macro{clearnotecolumns} will be done at every \Macro{clearpage}. You may
change this, switching off \Macro{autoclearnotecolumns} using one of the
values for simple switches.

Note, that it is not allowed to use \Macro{clearnotecolumn} or
\Macro{clearnotecolumns} inside a note. You can use these commands only inside
the main text of the document.%
\EndIndex{Cmd}{clearnotcolumns}%
\EndIndex{Cmd}{clearnotcolumn}%

\begin{Declaration}
  \Macro{syncwithnotecolumn}\OParameter{name of a note column}%
  \\
  \Macro{syncwithnotecolumns}\OParameter{list of note column names}%
\end{Declaration}
\BeginIndex{Cmd}{syncwithnotecolumn}%
\BeginIndex{Cmd}{syncwithnotecolumns}%
These commands
may be used to synchronise the vertical position of the main text with one or
several note columns. If you use the optional argument of
\Macro{syncwithnotecolumn} it should be one \PName{name of a note column}. If
you omit it, \PValue{marginpar} will be used as the \PName{name of a note
  column}. Usage of \Macro{syncwithnotecolumn} is exactly the same like usage
of \Macro{syncwithnotecolumns} with exactly one \PName{name of a note column}.

The optional argument of \Macro{syncwithnotecolumns} is a comma-separated list
of names of note columns. If you omit this optional argument or if you use an
empty argument or if the argument starts with \Macro{relax} synchronisation
will be done with all declared note columns.

Synchronisation with a note column means, that the current position of the
main text will be changed to be the current vertical position of the note
column after cleaning the note column. This differs in several point from
cleaning a note column:
\begin{itemize}
\item After cleaning a note column the text position would change to be at a
  new page, even if the note column doesn't fill the whole page
  before. Synchronisation will change the vertical position just after the
  last vertical position of the note column.
\item Cleaning a note column needs less \LaTeX{} runs than synchronising a
  note column.
\end{itemize}
Both, cleaning an synchronising a note column, uses the label mechanism of
\LaTeX{} without using \Macro{label}, \Macro{ref}, or \Macro{pageref} itself
(but \Macro{newlabel}). So you should have a look at \LaTeX{} rerun messages,
because of undefined or changed labels.

Note, that it is not allowed to use \Macro{syncwithnotecolumn} or
\Macro{syncwithnotecolumns} inside a note. If you do so, you need endless
\LaTeX{} runs without reaching a final state. But synchronisation of several
note columns in the main text of a document will also synchronise the notes
columns with each others.%
\EndIndex{Cmd}{syncwithnotecolumns}
\EndIndex{Cmd}{syncwithnotecolumn}

\endinput
%%
%% End of file `scrlayer-notecolumn.tex'.
