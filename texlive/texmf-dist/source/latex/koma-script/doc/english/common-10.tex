% ======================================================================
% common-10.tex
% Copyright (c) Markus Kohm, 2001-2013
%
% This file is part of the LaTeX2e KOMA-Script bundle.
%
% This work may be distributed and/or modified under the conditions of
% the LaTeX Project Public License, version 1.3c of the license.
% The latest version of this license is in
%   http://www.latex-project.org/lppl.txt
% and version 1.3c or later is part of all distributions of LaTeX 
% version 2005/12/01 or later and of this work.
%
% This work has the LPPL maintenance status "author-maintained".
%
% The Current Maintainer and author of this work is Markus Kohm.
%
% This work consists of all files listed in manifest.txt.
% ----------------------------------------------------------------------
% common-10.tex
% Copyright (c) Markus Kohm, 2001-2013
%
% Dieses Werk darf nach den Bedingungen der LaTeX Project Public Lizenz,
% Version 1.3c, verteilt und/oder veraendert werden.
% Die neuste Version dieser Lizenz ist
%   http://www.latex-project.org/lppl.txt
% und Version 1.3c ist Teil aller Verteilungen von LaTeX
% Version 2005/12/01 oder spaeter und dieses Werks.
%
% Dieses Werk hat den LPPL-Verwaltungs-Status "author-maintained"
% (allein durch den Autor verwaltet).
%
% Der Aktuelle Verwalter und Autor dieses Werkes ist Markus Kohm.
% 
% Dieses Werk besteht aus den in manifest.txt aufgefuehrten Dateien.
% ======================================================================
%
% Paragraphs that are common for several chapters of the KOMA-Script guide
% Maintained by Markus Kohm
%
% ----------------------------------------------------------------------
%
% Absaetze, die mehreren Kapiteln der KOMA-Script-Anleitung gemeinsam sind
% Verwaltet von Markus Kohm
%
% ======================================================================

\KOMAProvidesFile{common-10.tex}
                 [$Date: 2013-12-13 12:11:01 +0100 (Fr, 13. Dez 2013) $
                  KOMA-Script guide (common paragraphs)]
\translator{Markus Kohm\and Krickette Murabayashi}

% Date of the translated German file: 2012/01/01

\makeatletter
\@ifundefined{ifCommonmaincls}{\newif\ifCommonmaincls}{}%
\@ifundefined{ifCommonscrextend}{\newif\ifCommonscrextend}{}%
\@ifundefined{ifCommonscrlttr}{\newif\ifCommonscrlttr}{}%
\@ifundefined{ifIgnoreThis}{\newif\ifIgnoreThis}{}%
\makeatother


\ifIgnoreThis %+++++++++++++++++++++++++++++++++++++++++++++ nicht maincls +
\else %------------------------------------------------------- nur maincls -
\begin{Declaration}
  \Macro{footref}\Parameter{reference}
\end{Declaration}
\BeginIndex{Cmd}{footref}%
Sometimes\ChangedAt{v3.00}{\Class{scrbook}\and \Class{scrreprt}\and
  \Class{scrartcl}\and \Class{scrlttr2}} there are single footnotes to
 multiple text passages. The least sensible way to typeset this would
be to repeatedly use \Macro{footnotemark} with the same manually set
number. The disadvantages of this method would be that you have to
know the number and manually fix all the \Macro{footnotemark}
commands, and if the number changes because of adding or removing a
footnote before, each \Macro{footnotemark} would have to be
changed. Because of this, \KOMAScript{} provides the use of the
\Macro{label}\IndexCmd{label}\important{\Macro{label}} mechanism in
such cases. After simply setting a \Macro{label} inside the footnote,
\Macro{footref} may be used to mark all the other text passages with
the same footnote mark.
\begin{Example}
  Maybe you have to mark each trade name with a footnote which states that it
  is a registered trade name. You may write, e.\,g.,
\begin{lstcode}
  Company SplishSplash\footnote{This is a registered trade name.
    All rights are reserved.\label{refnote}}
  produces not only SplishPlump\footref{refnote}
  but also SplishPlash\footref{refnote}.
\end{lstcode}
  This will produce the same footnote mark three times, but only one footnote
  text. The first footnote mark is produced by \Macro{footnote}
  itself, and the following two footnote marks are produced by
  the additional \Macro{footref} commands. The footnote text will be produced by
  \Macro{footnote}.  
\end{Example}
Because of setting the additional footnote marks using the \Macro{label}
mechanism, changes of the footnote numbers will need at least two \LaTeX{}
runs to ensure correct numbers for all \Macro{footref} marks.%
%
\EndIndex{Cmd}{footref}%
\IfCommon{scrlttr2}{\EndIndex{}{footnotes}}
\fi %**************************************************** Ende nur maincls *


%%% Local Variables:
%%% mode: latex
%%% coding: us-ascii
%%% TeX-master: "../guide"
%%% End:
