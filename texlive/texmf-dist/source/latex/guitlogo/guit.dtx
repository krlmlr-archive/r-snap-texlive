% \iffalse
% 
% guit.dtx
% Copyright 2003, 2004, 2005, 2006, 2009, 2012 Gruppo Utilizzatori Italiani
% di TeX
% 
% This work may be distributed and/or modified under the
% conditions of the LaTeX Project Public License, either
% version 1.3a of this license or (at your option) any
% later version.
% The latest version of the license is in
%    http://www.latex-project.org/lppl.txt
% 
% Author: Emanuele Vicentini
%         (emanuelevicentini at yahoo dot it)
% 
% This work has the LPPL maintenance status "maintained".
% 
% The Current Maintainer of this work is Emanuele Vicentini.
% 
% This work consists of the files: README, guit.dtx, guit.ins and the
% derived files guit.sty, guit.cfg and guittest.tex
% 
%<*driver>
\documentclass[10pt, a4paper]{ltxdoc}
\usepackage[italian]{babel}
\usepackage{booktabs, guit, subfig, tabularx, hyperref, footnote}
\makesavenoteenv{table}
\EnableCrossrefs
\CodelineIndex
\RecordChanges
\OnlyDescription
\begin{document}
  \DocInput{guit.dtx}
\end{document}
%</driver>
% 
%<*test>
\documentclass[12pt, a4paper]{article}
\usepackage[margin=1in, noheadfoot]{geometry}
\usepackage{booktabs, guit, rotating, hyperref}

\pagestyle{empty}
\setupguit[link]

\begin{document}
\section*{Piccolo test per \guit}
Vediamo un po' come si comporta in una footnote\footnote{\guit~\guit\ldots
direi che \`e passabile, no?}. Diamoci\footnote{Un'altra nota:
\fontfamily{pzc}\fontseries{mb}\fontshape{it}\selectfont\guittext} dentro
con qualche cosa\footnote{Un ultima nota: Copyright 2003, 2004, 2005, 2006,
2009, 2012 \guittext} di ``strambo'':

\begin{center}
\itshape\guit~agus \TeX~go br\'ach!\\
$==$\\
\rmfamily\bfseries\guit~and \TeX~forever!
\end{center}

Una piccola ``spirale'' colorata e divertente:

\begingroup
\setupGuIT[color=yes]
\GuITcolor[rgb]{1, 0, 0}
\newcount\wang
\newsavebox{\wangtext}
\newdimen\wangspace
\def\wheel#1{\savebox{\wangtext}{#1}%
  \wangspace\wd\wangtext
  \advance\wangspace by 1cm%
  \centerline{%
    \rule{0pt}{\wangspace}%
    \rule[-\wangspace]{0pt}{\wangspace}%
    \wang=-180
    \loop
      \ifnum\wang<180
        \rlap{\begin{rotate}{\the\wang}%
          \rule{0.75cm}{0pt}#1
          \end{rotate}}%
        \advance\wang by 20
        \space
        \guitcolor*{coloredelGuIT!90!green}%
      \repeat}}
\wheel{\guit}
\endgroup

Ed ora una tabella riepilogativa parziale dei font supportati dalla versione
attuale.

\begin{center}
\begin{tabular}{lc}
\toprule
Font & Logo \\
\midrule
Computer Modern Roman & \guit[family=cmr] \\
Times New Roman & \guit[family=ptm] \\
Palatino & \guit[family=ppl] \\
NewCentury Schoolbook & \guit[family=pnc] \\
Charter & \guit[family=bch] \\
Bookman & \guit[family=pbk] \\
\bottomrule
\end{tabular}
\end{center}

Per finire, un sfilza di \guit:

\begingroup
  \let\pippo\par
  \makeatletter
  \@tfor\famiglia:={cmr}{ptm}{ppl}{pnc}{bch}{pbk}\do{%
    \noindent
    \setupGuIT[family=\famiglia]
    \@tfor\dimensione:=\tiny\scriptsize\footnotesize\small\normalsize
      \large\Large\LARGE\huge\do{%
      \dimensione\guit~}%
    \Huge\guit\pippo}%
\endgroup

\begin{center}
\Huge Venite tutti al prossimo \guitmeeting[color]
\end{center}

\end{document}
%</test>
% \fi
% 
% 
% 
% \CheckSum{538}
% 
% \CharacterTable
%   {Upper-case    \A\B\C\D\E\F\G\H\I\J\K\L\M\N\O\P\Q\R\S\T\U\V\W\X\Y\Z
%    Lower-case    \a\b\c\d\e\f\g\h\i\j\k\l\m\n\o\p\q\r\s\t\u\v\w\x\y\z
%    Digits        \0\1\2\3\4\5\6\7\8\9
%    Exclamation   \!     Double quote  \"     Hash (number) \#
%    Dollar        \$     Percent       \%     Ampersand     \&
%    Acute accent  \'     Left paren    \(     Right paren   \)
%    Asterisk      \*     Plus          \+     Comma         \,
%    Minus         \-     Point         \.     Solidus       \/
%    Colon         \:     Semicolon     \;     Less than     \<
%    Equals        \=     Greater than  \>     Question mark \?
%    Commercial at \@     Left bracket  \[     Backslash     \\
%    Right bracket \]     Circumflex    \^     Underscore    \_
%    Grave accent  \`     Left brace    \{     Vertical bar  \|
%    Right brace   \}     Tilde         \~}
% 
% 
% 
% \GetFileInfo{guit.sty}
% 
% 
% 
% \newcommand*{\pacchetto}[1]{\textsf{#1}}
% \newcommand*{\altro}[1]{\texttt{#1}}
% 
% 
% 
% \changes{v0.1}{2003/02/15}{Prima release pubblica del pacchetto}
% \changes{v0.2}{2003/02/21}{Arricchito il file di test del pacchetto}
% \changes{v0.5}{2003/09/04}{Aggiunta tutta la gestione del colore nel logo}
% \changes{v0.6}{2003/09/16}{Effettuate un paio di modifiche al file di
%   test} 
% \changes{v0.7}{2003/10/10}{Rimossi dal file di test gli usi del font
%   Utopia}
% \changes{v0.7}{2003/10/13}{Integrato il pacchetto \pacchetto{hyperref} su
%   richiesta di Fabiano Busdraghi}
% \changes{v0.7.2}{2004/09/16}{Aggiornata la documentazione riferentesi a
%   \pacchetto{xcolor}}
% \changes{v0.7.2}{2004/09/16}{Personalizzati indice e change history}
% \changes{v0.7.3}{2004/10/11}{Alcune piccole modifiche al file di test}
% \changes{v0.8}{2004/10/28}{Riallineato il file di test all'ultima modifica
%   della sintassi di \cs{GuIT}}
% \changes{v0.8.2}{2005/04/08}{Aggiunto il file di configurazione esterno}
% \changes{v0.9}{2005/11/21}{Modificata l'interfaccia in modo da fare ampio
%   uso di chiavi definite tramite \pacchetto{xkeyval} (ora ricorda molto lo
%   ``stile'' di Con\TeX t); mantenuta la compatibilit\`a quasi totale con
%   le versioni precedenti, ma \cs{swapGuITcommands} \`e stato
%   definitivamente rimosso}
% \changes{v0.9}{2005/11/27}{Il logo ora pu\`o essere composto anche usando
%   la serie \altro{bx}}
% \changes{v0.9}{2005/12/03}{Nel file di test ora viene impostata
%   globalmente la chiave \altro{link}}
% 
% 
% 
% \title{Documentazione del pacchetto \pacchetto{guit}\thanks{Il numero di
% versione di questo file \`e \fileversion; l'ultima revisione \`e avvenuta
% in data \filedate.}}
% \author{Emanuele Vicentini\\(\texttt{emanuelevicentini at yahoo dot it})}
% \date{\filedate}
% 
% 
% 
% \maketitle
% \begin{abstract}
%     Questo piccolissimo pacchetto permette di riprodurre il logo del
%     \emph{\guittext} (\emph{\guittexten}) in maniera sufficientemente
%     indipendente dal font utilizzato, cercando di mantenerne l'aspetto
%     originale.
% \end{abstract}
% \tableofcontents
% \clearpage
% 
% 
% 
% \section{Introduzione}
% Questa documentazione \`e stata scritta seguendo le convenzioni
% dell'applicazione \LaTeX~\textsc{docstrip} che permette l'estrazione
% automatica del codice \LaTeX~dal file di documentazione~\cite{GOOSSENS94}.
% 
% 
% 
% \iffalse\section{I logo}\fi
% 
% 
% 
% \section{Il logo del Gruppo}
% Per riprodurre il pi\`u fedelmente possibile il logo del \guit~ho deciso
% che la soluzione migliore per ottenere gli effetti desiderati fosse quella
% di utilizzare solo font con Caps \& Small Caps. I font attualmente
% supportati\footnote{Il font Utopia non \`e presente in tutte le
% distribuzioni di \TeX/\LaTeX~per motivi legali e non \`e pi\`u considerato
% un componente fondamentale, quindi nel file di esempio che accompagna
% questo pacchetto il font Utopia non \`e stato utilizzato. Se qualche
% persona volenterosa potesse controllare i coefficienti utilizzati per
% questo font e volesse contribuire con i coefficienti per usare Utopia con
% la serie \altro{bx}, sarebbe sempre ben accetta.} sono elencati nella
% tabella~\ref{tab:font}.
% 
% \begin{table}[!h]
% \centering
% \begin{tabular}{l>{\ttfamily}cc>{\scshape}c}
% \toprule
% \itshape Font & \normalfont\itshape Famiglia & \itshape Bx & \itshape Provenienza \\
% \midrule
% Computer Modern Roman & cmr & No & \normalfont Standard \\
% Times & ptm & S\`\i & psnfss \\
% Palatino & ppl & S\`\i & psnfss \\
% New Century Schoolbook & pnc & S\`\i & psnfss \\
% Utopia & put & S\`\i\footnote{In realt\`a viene usata solo la serie \altro{m}.} & psnfss \\
% Charter & bch & S\`\i & psnfss \\
% Bookman & pbk & S\`\i & psnfss \\
% \bottomrule
% \end{tabular}
% \caption{Font attualmente supportati}
% \label{tab:font}
% \end{table}
% 
% Per evitare spiacevoli ed inaspettati effetti dovuti a combinazioni dei
% parametri \textsc{nfss2} indicanti font inesistenti nella famiglia
% Computer Modern originale (esempio: \altro{OT1/cmr/bx/sc} non esiste e
% viene sostituito automaticamente con \altro{OT1/cmr/bx/n}), tutte le
% occorrenze del logo composte con la famiglia Computer Roman usano il font
% \altro{OT1/cmr/m/sc}. Le altre famiglie, laddove esista il font
% appropriato, possono essere utilizzate anche con serie e codifiche
% diverse, ma questo pacchetto fornisce solo le istanze preconfezionate per
% le serie \altro{m} e \altro{bx}. Il pacchetto \`e stato testato solo con
% le codifiche \altro{OT1} e \altro{T1}; conferme riguardanti il corretto
% utilizzo del pacchetto con codifiche diverse saranno sicuramente
% benvenute.
% 
% L'idea di utilizzare la codifica \altro{T1} per la famiglia Computer
% Modern (perch\'e esiste il font \altro{T1/cmr/bx/sc}) \`e stata scartata a
% priori perch\'e i font in formato Type1 di tale famiglia non sono ancora,
% a mio parere, sufficientemente diffusi. Per quanto riguarda la famiglia
% Latin Modern, non vi sar\`a alcun supporto per il suo utilizzo fino a
% quando non avr\`a raggiunto la completa stabilit\`a, ma, se siete disposti
% a correre qualche rischio, potete aggiungere la dichiarazione appropriata
% per la famiglia \altro{lmr} nel file di configurazione del pacchetto
% (vedere \S\ref{sec:config}) utilizzando i parametri impiegati per i font
% Computer Modern.
% 
% 
% 
% \section{Il logo del convegno}
% Per cercare di alleviare parte del lavoro degli organizzatori, ho cercato
% di riprodurre, per quanto mi \`e possibile, il logo del convegno periodico
% del \guittext\ usato a partire dal meeting svoltosi a Pisa il 10 ottobre
% 2004.
% 
% Il logo producibile da questo pacchetto, esemplificato nella
% figura~\ref{fig:guitmeeting}, \`e il risultato di un gran numero di
% tentativi di riprodurre correttamente quello originariamente utilizzato
% per il poster-programma del meeting di quell'anno. Per questa ragione, il
% logo viene sempre composto usando la famiglia Computer Modern,
% indipendentemente dalla famiglia principale usata nel documento.
% 
% Come molti processi automatici, anche questo non \`e perfetto. Si
% accettano molto volentieri suggerimenti per migliorarlo (possibilmente in
% forma di codice gi\`a funzionante o patch per l'ultima versione di questo
% pacchetto).
% 
% \begin{figure}
%   \centering
%   \Huge\guitmeeting\qquad\guitmeeting[year=2010]\\[10pt]
%   \guitmeeting[style=inline]\quad\guitmeeting[year=2010, style=inline]
%   \caption{Visione d'insieme del logo del convegno periodico}
%   \label{fig:guitmeeting}
% \end{figure}
% 
% 
% 
% \section{Il logo della rivista}
% A partire dall'aprile del 2006, il \guittext realizza una rivista dedicata
% a \TeX, \LaTeX\ e la tipografia digitale; maggiori informazioni sono
% reperibili nelle pagine dedicate del sito del Gruppo:
% \guiturl[document=home/it/arstexnica]. Nella figura~\ref{fig:arstexnica} sono
% raccolte diverse istanze e varianti del logo di \Ars\ ed i comandi
% correlati, la cui ideazione e realizzazione sono dovute a Massimo
% Caschili.
% 
% \makeatletter
% \newcommand*{\loghetto}[2][1]{%
%   \subfloat[\cs{#2}]{\scalebox{#1}{\csname #2\endcsname}}}
% \makeatother
% \begin{figure}
% \centering\Huge
% \loghetto[2]{Ars}\qquad
% \loghetto[2]{Arsob}\\
% \loghetto[2]{ars}\qquad
% \loghetto[2]{tecnica}\\
% \loghetto[2]{arsta}\qquad
% \loghetto[2]{arstb}\\
% \loghetto[2]{arstv}\qquad
% \loghetto[2]{arsto}\\
% \loghetto[2]{Arsto}
% \caption{Il logo della rivista e le sue varianti}\label{fig:arstexnica}
% \end{figure}
% 
% 
% 
% \section{Comandi}
% 
% 
% 
% \subsection{Il logo del Gruppo}
% \changes{v0.8}{2004/10/28}{Ristrutturata l'implementazione di \cs{GuIT}
%   per l'aggiunta dell'argomento opzionale per indicare la famiglia di font
%   da utilizzare}
% \DescribeMacro{\GuIT}\oarg{famiglia, keywords}\newline
% \DescribeMacro{\GuIT*}\oarg{famiglia, keywords}\newline
% \DescribeMacro{\guit}\oarg{famiglia, keywords}\newline
% \DescribeMacro{\guit*}\oarg{famiglia, keywords}\newline
% Per riprodurre il logo di \guit~l'utente ha a disposizione il comando
% \cs{GuIT} che si prende cura di controllare la famiglia di font utilizzata
% e cerca di produrre il risultato migliore. Anche se \`e tecnicamente
% errato, per comodit\`a l'utente pu\`o usare anche l'equivalente forma
% \cs{guit}. Entrambi i comandi dispongono di una versione ``stellata'' che
% riproduce il logo con uno schema di colori simile a quello usato nel sito
% di \guit~(\guiturl).
% 
% Desidero richiamare l'attenzione sul fatto che in molti casi non \`e
% possibile giungere al risultato \emph{perfetto} in quanto le grazie della
% G e della U non sono necessariamente sovrapponibili senza
% ``sbavature''. Questi difetti non sono eliminabili, ma \`e stato fatto il
% possibile per minimizzarli.
% 
% Il font utilizzato per comporre il logo \`e, normalmente, il font attivo
% nel punto in cui compare l'occorrenza di \cs{GuIT}, ma pu\`o essere
% modificato utilizzando gli argomenti opzionali nei seguenti modi:
% 
% \begin{itemize}
% \item indicando direttamente il nome della famiglia di font da utilizzare
%   (esempio: \cs{GuIT[ppl]});
% \item indicando direttamente il nome della famiglia e della serie da
%   utilizzare separati dal carattere ``\altro{/}'' (esempio:
%   \cs{GuIT[ppl/bx]});
% \item assegnando alla chiave |family| (vedere \S\ref{sec:keywords}) il
%   nome della famiglia di font da utilizzare (esempio:
%   \cs{GuIT[family=ppl]});
% \item assegnando alla chiave |series| (vedere \S\ref{sec:keywords}) il
%   nome di una serie da utilizzare (esempio: \cs{GuIT[series=bx]}).
% \end{itemize}
% 
% Ovviamente, in tutti questi casi il font indicato viene utilizzato solo se
% \`e presente nella lista di quelli supportati dal pacchetto, altrimenti si
% ricade sul classico Computer Modern Roman. Gli effetti di queste
% alterazioni sono limitati alla singola occorrenza di \cs{GuIT} nella quale
% compaiono.
% 
% 
% 
% \subsection{Selezione del colore}
% Con entrambi i seguenti comandi \`e possibile usare le espressioni
% ``estese'' fornite dal pacchetto \pacchetto{xcolor}~\cite{Kern05}.
% 
% \medskip
% 
% \noindent\DescribeMacro{\GuITcolor}\oarg{modello}\marg{dichiarazione}\newline
% \DescribeMacro{\guitcolor}\oarg{modello}\marg{dichiarazione}\newline
% Il colore utilizzato da \cs{GuIT*} e da \cs{guit*} pu\`o essere modificato
% con questo comando. Se l'argomento opzionale viene omesso si assume l'uso
% del modello \emph{cmyk}. Il colore definito in questo modo \`e accessibile
% con il nome \altro{coloredelGuIT}.
% 
% \medskip
% 
% \noindent\DescribeMacro{\GuITcolor*}\oarg{modello}\marg{espressione xcolor}\newline
% \DescribeMacro{\guitcolor*}\oarg{modello}\marg{espressione xcolor}\newline
% Questo comando permette di operare sul colore utilizzato da \cs{GuIT*}
% utilizzando le espressioni supportate dal comando \cs{colorlet}; per una
% discussione pi\`u ampia e dettagliata si rimanda alla documentazione del
% pacchetto \pacchetto{xcolor}. \cs{GuITcolor*} agisce implicitamente su
% \altro{coloredelGuIT}, quindi \`e essenzialmente equivalente a
% \begin{quote}
% 	|\colorlet{coloredelGuIT}|\marg{espressione xcolor}
% \end{quote}
% L'argomento opzionale specifica il modello di colore in cui
% \meta{espressione xcolor} deve essere trasformata prima dell'applicazione
% a \altro{coloredelGuIT}.
% 
% In entrambi i casi, si sconsiglia l'uso del modello \emph{named}. Come per
% il comando \cs{GuIT} anche in questo caso si possono usare le equivalenti
% forme \cs{guitcolor} e \cs{guitcolor*}.
% 
% 
% 
% \subsection{Dicitura completa del Gruppo}
% \changes{v0.2}{2003/02/17}{Aggiunti due nuovi comandi per indicare la
%   denominazione completa del \guittext}
% \DescribeMacro{\GuITtext}\oarg{keywords}\newline
% \DescribeMacro{\guittext}\oarg{keywords}\newline
% Per evitare errori nel riportare la denominazione ufficiale e completa del
% Gruppo (\guittext) l'utente pu\`o utilizzare il comando \cs{GuITtext} che
% fornisce l'espansione corretta dell'acronimo \guit: \guittext. Notate che
% il comando non effettua alcuna modifica al font ed al colore attualmente
% in uso e non assume l'uso di alcun font particolare. Come per il comando
% \cs{GuIT} anche in questo caso si pu\`o usare l'equivalente forma
% \cs{guittext}.
% 
% \medskip
% 
% \changes{v0.9}{2005/10/29}{Aggiunti due nuovi comandi per indicare la
%   denominazione completa anglicizzata del \guittext\ (o \guittexten\ che
%   dir si voglia) su espressa richiesta di Gustavo Cevolani}
% \noindent\DescribeMacro{\GuITtextEn}\oarg{keywords}\newline
% \DescribeMacro{\guittexten}\oarg{keywords}\newline
% I comandi appena illustrati hanno la loro controparte in \cs{GuITtextEn} e
% nell'immancabile \cs{guittexten}, i quali si comportano esattamente nello
% stesso modo per quanto riguarda font e colori, ma producono l'equivalente
% inglese di \guittext: \guittexten.
% 
% Per quanto riguarda l'argomento opzionale \meta{keywords}, vedere
% \S\ref{sec:keywords}.
% 
% 
% 
% \subsection{Indirizzo del sito internet e del forum pubblico}
% \DescribeMacro{\GuITurl}\oarg{keywords}\newline
% \DescribeMacro{\guiturl}\oarg{keywords}\newline
% Per evitare errori nel riportare l'indirizzo completo del sito internet
% del Gruppo (\guiturl) l'utente pu\`o utilizzare il comando \cs{GuITurl}
% che scrive l'indirizzo corretto sfruttando il pacchetto \pacchetto{url}
% presente nella dotazione standard di qualunque distribuzione di
% \TeX\,/\LaTeX. Come per il comando \cs{GuIT} anche in questo caso si pu\`o
% utilizzare l'equivalente forma \cs{guiturl}.
% 
% \medskip
% 
% \noindent\DescribeMacro{\GuITforum}\oarg{keywords}\newline
% \DescribeMacro{\guitforum}\oarg{keywords}\newline
% Questo comando permette l'inserimento dell'indirizzo completo del forum
% pubblico (\guitforum), sfruttando nuovamente il pacchetto \pacchetto{url}.
% Come per il comando \cs{GuIT} anche in questo caso si pu\`o utilizzare
% l'equivalente forma \cs{guitforum}.
% 
% Desidero richiamare l'attenzione sul comportamento di alcuni programmi di
% visualizzazione di documenti in formato \textsc{pdf}, i quali analizzano
% il testo dei documenti alla ricerca di elementi che rispecchino la
% struttura di un \textsc{url} ed interpretano automaticamente tali elementi
% come link realmente funzionanti. Questo fatto non \`e alterabile tramite
% comandi od altri costrutti di \LaTeX, quindi, se doveste riscontrare
% ``anomalie'' negli effetti dei due comandi appena illustrati, esaminate la
% configurazione del programma che usate ed eventualmente alteratela di
% conseguenza.
% 
% Per quanto riguarda l'argomento opzionale \meta{keywords}, vedere
% \S\ref{sec:keywords}.
% 
% 
% 
% \subsection{Il logo del convegno}
% \DescribeMacro{\GuITmeeting}\oarg{keywords}\newline
% \DescribeMacro{\GuITmeeting*}\oarg{keywords}\newline
% \DescribeMacro{\guitmeeting}\oarg{keywords}\newline
% \DescribeMacro{\guitmeeting*}\oarg{keywords}\newline
% Per riprodurre il logo del periodico convegno l'utente ha a disposizione
% il comando \cs{GuITmeeting}; trattandosi di un'entit\`a grafica
% particolare, non \`e possibile alcuna personalizzazione del risultato ad
% eccezione dello schema di colori. Come per gli altri comandi, anche in
% questo caso si pu\`o utilizzare l'equivalente forma \cs{guitmeeting}.
% Entrambi i comandi dispongono di una versione ``stellata'' che
% riproduce il logo a colori con uno schema di colori simile a quello usato
% nel sito di \guit~(\guiturl).
% 
% Per quanto riguarda l'argomento opzionale \meta{keywords}, vedere
% \S\ref{sec:keywords}.
% 
% 
% 
% \subsection{Il logo della rivista}
% \changes{v0.9}{2006/05/17}{Inglobati i comandi per generare i logo della
%   rivista ideati da Massimo Caschili}
% I comandi per riprodurre le diverse istanze e varianti del logo della
% rivista sono illustrati direttamente nella
% figura~\ref{fig:arstexnica}; questi comandi non hanno alcun
% parametro. L'autore ed ideatore dei comandi, per conto del \guittext, \`e
% Massimo Caschili.
% 
% 
% 
% \subsection{Usare le chiavi}\label{sec:keywords}
% Il pacchetto fa ampio uso di chiavi, definite tramite
% \pacchetto{xkeyval}~\cite{Adriaens05}, in modo da rendere l'utilizzo dei
% vari comandi pi\`u flessibile. Applicare una chiave ad uno dei comandi
% visti in precedenza pu\`o potenzialmente modificare la relativa
% caratteristica \emph{solo ed esclusivamente per quell'istanza del
% comando}.
% 
% \begin{table}
% \centering
% \begin{tabularx}{0.91\textwidth}{ll>{\raggedright\arraybackslash}X}
% \toprule
% \itshape Chiave & \itshape Valori & \itshape Applicabilit\`a \\
% \midrule
% color & \textit{yes}, true, no, false & \cs{GuIT}, \cs{GuITmeeting} \\
% link & \textit{yes}, true, no, false & \cs{GuIT}, \cs{GuITmeeting}, \cs{GuITtext}, \cs{GuITtextEn}, \cs{GuITurl}, \cs{GuITforum} \\
% family & \meta{famiglia} & \cs{GuIT} \\
% series & \meta{serie} & \cs{GuIT} \\
% url & \meta{indirizzo} & \cs{GuIT}, \cs{GuITmeeting}, \cs{GuITtext}, \cs{GuITtextEn}, \cs{GuITurl}, \cs{GuITforum} \\
% document & \meta{documento} & \cs{GuITurl} \\
% year & \meta{anno} & \cs{GuITmeeting} \\
% style & display, inline & \cs{GuITmeeting} \\
% \bottomrule
% \end{tabularx}
% \caption{Chiavi definite dal pacchetto \pacchetto{guit}}
% \label{tab:keywords}
% \end{table}
% 
% Le chiavi attualmente definite sono riportate nella
% tabella~\ref{tab:keywords} assieme ai valori che possono assumere e ad
% un'indicazione dei comandi sui quali hanno un qualche effetto. I valori in
% corsivo sono i valori di default assegnati implicitamente alle chiavi
% corrispondenti se quest'ultime vengono usate senza indicare esplicitamente
% alcun valore.
% 
% Alcune annotazioni sulle chiavi:
% 
% \begin{description}
% \item[|color|] permette di controllare l'uso del colore per il logo del
%   Gruppo; da notare che sono possibili combinazioni ``strane'' come
%   |\GuIT*[color=no]|, la quale, nonostante l'utilizzo della versione
%   ``stellata'', produrr\`a un'istanza del logo non colorata;
% \item[|link|] permette di controllare la generazione di un link alla
%   pagina principale del sito del \guit\ in corrispondenza di ogni
%   occorrenza del logo e dell'espansione della dicitura completa del Gruppo
%   tramite l'uso di \pacchetto{hyperref}, \emph{che deve essere caricato
%   esplicitamente} nel preambolo del documento; come effetto collaterale, i
%   prodotti di \cs{GuITurl} e \cs{GuITforum} diventano automaticamente
%   \textsc{url} realmente funzionanti;
% \item[|family|] \`e un modo pi\`u esplicito e prolisso di impostare il
%   font con il quale viene composto il logo del Gruppo; assegnare a questa
%   chiave un valore nullo fa si che il font utilizzato per comporre il logo
%   di \guit\ sia quello attivo nel punto in cui compare il comando
%   \cs{GuIT};
% \item[|series|] permette di impostare la serie con la quale viene composto
%   il logo del Gruppo; assegnare a questa chiave un valore nullo fa si che
%   la serie utilizzata per comporre il logo sia quella attiva nel punto in
%   cui compare il comando \cs{GuIT};
% \item[|url|] permette di definire od alterare l'indirizzo del sito
%   internet del Gruppo usato dai vari comandi che producono testo o link al
%   sito stesso; assegnare a questa chiave un valore nullo sopprimer\`a
%   l'output di \cs{GuITurl} e disabiliter\`a la creazione di link da parte
%   di \pacchetto{hyperref};
% \item[|document|] permette di indicare pagine o sezioni del sito del
%   Gruppo facendo in modo che il risultato si comporti in maniera
%   automatica esattamente come \cs{GuITurl} per quanto riguarda
%   l'interazione con \pacchetto{hyperref} e la creazione di link; assegnare
%   a questa chiave un valore nullo disabilita la visualizzazione di
%   quest'estensione dell'indirizzo;
% \item[|year|] permette di impostare l'anno che comparir\`a nel logo del
%   convegno periodico in una posizione simile a quella in cui compare nei
%   poster, accanto all'illustrazione realizzata per il Gruppo da Duane
%   Bibby\footnote{Se gi\`a si poteva sconsigliare l'uso di \cs{GuITmeeting}
%   all'interno di un paragrafo o di un qualsiasi altro blocco di testo,
%   l'aggiunta dell'anno crea un'entit\`a decisamente ingombrante e porta a
%   rafforzare il consiglio: \emph{non usate \cs{GuITmeeting} all'interno di
%   un blocco di testo}.}; assegnare a questa chiave un valore nullo
%   disabilita la visualizzazione dell'anno;
% \item[|style|] permette di modificare l'aspetto del logo del convegno
%   alterando il modo nel quale vengono assemblati i sui componenti: i primi
%   due esempi che compaiono in figura~\ref{fig:guitmeeting} sono composti
%   con |style=display|, gli ultimi due con
%   |style=inline|\footnote{L'aspetto del logo prodotto da |inline| potrebbe
%   non essere molto stabile e subire drastiche variazioni nelle prossime
%   versioni del pacchetto.}.
% \end{description}
% 
% Non \`e strettamente necessario attenersi alle indicazioni d'uso delle
% chiavi riportate nella tabella~\ref{tab:keywords} ed \`e quindi possibile
% utilizzarle liberamente, con l'avvertenza che le chiavi i cui effetti non
% sono applicabili al comando di cui sono argomento non produrranno alcun
% risultato mentre tutte le altre avranno effetto solo su quella particolare
% istanza del comando a cui sono legate.
% 
% \medskip
% 
% \changes{v0.9}{2005/10/29}{Aggiunto un nuovo comando per gestire in
%   maniera ``globale'' le chiavi}
% \noindent\DescribeMacro{\setupGuIT}\oarg{keywords}\newline
% \DescribeMacro{\setupguit}\oarg{keywords}\newline
% Oltre ad essere utilizzabili direttamente nei comandi indicati in
% precedenza, tutte le chiavi appena illustrate possono essere usate come
% argomenti di \cs{setupGuIT}, che provveder\`a ad elaborarle e ad
% applicarne gli effetti \emph{a tutto ci\`o che segue, tenendo conto delle
% normali regole di scoping}.
% 
% Utilizzare questo comando senza alcun argomento equivale a disabilitare
% l'utilizzo del colore e la generazione dei link, ad impostare il font
% usato per il logo a quello attivo in quel punto del documento, ad
% impostare l'indirizzo del sito internet del Gruppo a quello attuale, a
% disabilitare la visualizzazione del documento che estente \cs{GuITurl} e
% dell'anno e ad impostare lo stile |display| per \cs{GuITmeeting}.
% 
% Giova rimarcare la differenza tra l'applicazione delle chiavi a
% \cs{setupGuIT} ed agli altri comandi:
% 
% \begin{itemize}
% \item l'esecuzione di un'istanza di \cs{setupGuIT} ha effetto \emph{su
%   tutti i comandi di questo pacchetto fino al termine del documento o del
%   pi\`u piccolo gruppo che lo contiene};
% \item applicare una chiave ad un qualsiasi altro comando di questo
%   pacchetto pu\`o potenzialmente produrre un qualche effetto
%   \emph{solamente su quella particolare istanza di quel particolare
%   comando}.
% \end{itemize}
% 
% 
% 
% \subsection{Creare nuove realizzazioni del logo del Gruppo}
% \DescribeMacro{\DeclareGuITLogoCommand}\marg{famiglia}\oarg{serie}\marg{u-h}\marg{u-v}\marg{i-h}\marg{t-h}\newline
% Il numero di font supportati da questa versione del pacchetto non \`e
% molto alto; per gli utenti desiderosi di usare font non previsti
% dall'autore, \`e disponibile il comando \cs{DeclareGuITLogoCommand},
% utilizzabile solo nel preambolo del documento o nel file di configurazione
% \altro{guit.cfg}.
% 
% Tenendo ben presente che sia la \meta{famiglia} tanto quanto la
% \meta{serie} sono da considerarsi nell'accezione tipica dei termini del
% sistema \emph{NFSS2}, i sei parametri rappresentano:
% 
% \begin{enumerate}
% \item il nome della famiglia di font;
% \item la serie (valore di default: \altro{m}, cio\`e il tradizionale
%   valore di \cs{mddefault});
% \item lo scartamento orizzontale della \emph{u};
% \item lo scartamento verticale della \emph{u};
% \item lo scartamento orizzontale della \emph{i};
% \item lo scartamento orizzontale della \emph{t}.
% \end{enumerate}
% 
% Valori negativi degli ultimi quattro parametri indicano spostamenti verso
% sinistra o, dove applicabile, verso l'alto; valori positivi indicano,
% ovviamente, spostamenti nelle direzioni opposte. Usando
% \cs{DeclareGuITLogoCommand} \`e possibile ridefinire le realizzazioni del
% logo fornite dalla versione attuale del pacchetto, con l'unica eccezione
% del logo composto con la famiglia Computer Modern\footnote{Un eventuale
% tentativo di ridefinizione avrebbe conseguenze deleterie sul risultato
% finale.}.
% 
% \medskip
% 
% \noindent\DescribeMacro{\AliasGuITLogoCommand}\marg{fam1}\marg{ser1}\marg{fam2}\marg{ser2}\newline
% Per evitare che il pacchetto utilizzi il font Computer Modern Roman in
% sostituzione di un font non supportato, \`e possibile creare un alias per
% la particolare combinazione inesistente di famiglia e serie. I primi due
% argomenti rappresentano la famiglia e la serie che devono essere
% rappresentate dal font identificato dagli ultimi due parametri e per il
% quale \emph{deve} essere gi\`a stata creata un'istanza del logo (tramite
% \cs{DeclareGuITLogoCommand} oppure tramite un altro alias).
% 
% A titolo d'esempio, la definizione per il font Utopia, serie \altro{bx},
% \`e attualmente un alias creato in questo modo:
% \begin{quote}
% 	|\AliasGuITLogoCommand{put}{bx}{put}{m}|
% \end{quote}
% 
% 
% 
% \section{Configurazione}\label{sec:config}
% Il pacchetto \`e corredato di un piccolo file di configurazione,
% \altro{guit.cfg}, nel quale l'utente pu\`o impostare un colore di default
% per il logo diverso dall'originale, definire nuove realizzazioni del logo
% o ridefinire quelle esistenti.
% 
% Il file cos\`{\i} come viene distribuito \`e praticamente vuoto, in quanto
% tutte le dichiarazioni utili sono state commentate perch\'e riprendono
% semplicemente quanto gi\`a definito nel codice del pacchetto e servono
% unicamente da esempio.
% 
% 
% 
% \section{Opzioni}
% Il pacchetto ha tre opzioni, ma il loro uso \`e sconsigliato a favore del
% comando \cs{setupGuIT}, che permette una maggiore flessibilit\`a:
% 
% \begin{description}
% \item[color] forza la colorazione di tutte le occorrenze del logo,
% 	indipendentemente dalla forma del comando usata; \emph{prestate bene
% 	attenzione al fatto che il significato di quest'opzione \`e
% 	radicalmente cambiato rispetto alle versioni precedenti alla 0.7 e
% 	che l'uso di \cs{setupGuIT} pu\`o annullarne l'effetto};
% \item[nocolor] disattiva il supporto del colore; quando viene specificata
% 	quest'opzione \cs{GuIT*} e \cs{guit*} producono gli stessi risultati
% 	di \cs{GuIT}, cio\`e il colore del logo non viene alterato in alcun
% 	modo; \emph{prestate bene attenzione al fatto che a partire dalla
% 	versione 0.9 l'uso di \cs{setupGuIT} pu\`o annullare l'effetto di
% 	quest'opzione};
% \item[link] abilita l'uso di \pacchetto{hyperref}, \emph{che deve essere
%   caricato esplicitamente} nel preambolo del documento) per trasformare ogni
% 	occorrenza del logo del Gruppo e dell'espansione della dicitura
% 	completa in un link al sito di \guit; prestate attenzione al fatto
%	che l'aspetto del logo \emph{non muta} minimamente; inoltre, come effetto
% 	collaterale dell'uso di \pacchetto{hyperref}, \cs{GuITurl} diventa
% 	automaticamente un link al sito internet di \guit;  \emph{prestate
% 	bene attenzione al fatto che a partire dalla versione 0.9 l'uso di
% 	\cs{setupGuIT} pu\`o annullare l'effetto di quest'opzione}.
% \end{description}
% 
% 
% 
% \section{Dipendenze da altri pacchetti}
% La lista dei pacchetti da cui questo codice dipende \`e, per ora, molto
% breve e si tratta di pacchetti fondamentali oppure estremamente comuni:
% 
% \begin{itemize}
% \item \pacchetto{hyperref};
% \item \pacchetto{graphics};
% \item \pacchetto{url};
% \item \pacchetto{xcolor}, versione 2.00 o successive;
% \item \pacchetto{xkeyval}, versione 2.5 o successive.
% \end{itemize}
% 
% \pacchetto{hyperref} non viene pi\`u caricato automaticamente a partire
% dalla versione 0.9.1.
% 
% Tutti gli altri pacchetti vengono caricati senza indicare esplicitamente
% alcuna opzione, ma questo non significa che ci\`o non possa causarvi
% problemi: laddove \`e perfettamente lecito tentare di caricare due o pi\`u
% volte lo stesso pacchetto \emph{con le stesse opzioni}, non \`e possibile
% farlo cambiando le opzioni, quindi il codice seguente \`e errato:
% 
% \begin{verbatim}
% % ...
% \usepackage{graphics}
% \usepackage[draft]{graphics}
% % ...
% \end{verbatim}
% 
% Per semplificare al massimo il problema, se doveste incontrare messaggi
% d'errore che menzionano i pacchetti elencati in precedenza alterate
% l'ordine di caricamento dei pacchetti nel vostro documento in modo che
% questo pacchetto si trovi dopo quello indicato dai messaggi.
% 
% 
% 
% \StopEventually{%
%   \bibliographystyle{alpha}
%   \begin{thebibliography}{GMS94}
%   \addcontentsline{toc}{section}{\refname}
%   \bibitem[Adr05]{Adriaens05} Hendri Adriaens, \emph{The
%   	\pacchetto{xkeyval} package}, v2.5e, 2005/11/25 (disponibile presso
%   	\textsc{ctan} in \url{macros/latex/contrib/xkeyval}).
%   \bibitem[GMS94]{GOOSSENS94} Michel Goossens, Frank Mittelbach e Alexander
%   	Samarin. \emph{The \LaTeX\ Companion}. Addison-Wesley Company, 1994.
%   \bibitem[Ker05]{Kern05} Dr.~Uwe Kern, \emph{Extending \LaTeX's color
%   	facilities: the \pacchetto{xcolor} package}, v2.09, 2005/12/21
%   	(disponibile presso \textsc{ctan} in
%   	\url{macros/latex/contrib/xcolor}).
%   \bibitem[Knu86]{CMT} Donal E. Knuth. \emph{Computer Modern Typefaces},
%   	volume E di \emph{Computer \& Typesetting}. Addison-Wesley
%   	Publishing, 1986.
%   \end{thebibliography}}
% 
% 
% 
% \section{Il codice}
% 
% 
% 
% \subsection{Il pacchetto}
% \changes{v0.2}{2003/02/17}{Corretti un paio di errori nel codice}
% \changes{v0.9}{2005/11/19}{I pacchetti \pacchetto{xcolor} e
%   \pacchetto{hyperref} vengono ora caricati di default; \pacchetto{color}
%   \`e stato definitivamente rimosso}
% \changes{v0.9.1}{2009/07/31}{\pacchetto{hyperref} non viene pi\`u caricato
%   automaticamente; suggerimento e codice di Enrico Gregorio}
% Richiediamo espressamente l'uso di \LaTeXe, ci annunciamo al mondo e
% carichiamo tutti i pacchetti necessari.
%    \begin{macrocode}
%<*style>
\NeedsTeXFormat{LaTeX2e}
\ProvidesPackage{guit}[2012/08/17 v0.9.2 Logo del GuIT]
\RequirePackage{graphics, url}
\RequirePackage{xcolor}[2004/07/04]
\RequirePackage{xkeyval}[2005/05/07]
%    \end{macrocode}
% 
% \changes{v0.7}{2003/10/17}{Modificata la semantica di \altro{color}: ora
%   forza la colorazione di tutte le occorrenze del logo}
% \changes{v0.9}{2005/11/21}{Semplificata la gestione delle opzioni}
% Le tre opzioni che il pacchetto presentava fino alla versione 0.8.2 sono
% da considerarsi obsolete, ma il loro effetto \`e, almeno parzialmente,
% ancora onorato.
%    \begin{macrocode}
\newif\if@guit@colorized
\newif\if@guit@link
\DeclareOption{color}{\AtEndOfPackage{\setupGuIT[color=yes]}}
\DeclareOption{nocolor}{\AtEndOfPackage{\setupGuIT[color=no]}}
\DeclareOption{link}{\AtEndOfPackage{\setupGuIT[link=yes]}}
\ProcessOptions\relax
%    \end{macrocode}
% 
% \begin{macro}{\setupGuIT}
% \changes{v0.9.2}{2012/08/17}{Aggiornato l'indirizzo del sito internet}
% \begin{macro}{\setupguit}
% Il comando per gestire le varie chiavi \`e un semplice wrapper alla macro
% \cs{setkeys*} (ho optato per usare questa macro in quanto non genera alcun
% messaggio d'errore se le chiavi che riceve in input non esistono nella
% famiglia sotto esame).
%    \begin{macrocode}
\DeclareRobustCommand{\setupGuIT}[1][color=no, link=no, family=, series=, url=http://www.guitex.org, document=, year=, style=display]{%
  \setkeys*{guit}{#1}%
  \ignorespaces}
\let\setupguit\setupGuIT
%    \end{macrocode}
% \end{macro}
% \end{macro}
% 
% Alcune delle chiavi usate nel codice vengono definite qui, altre sono
% definite durante la generazione delle macro interne che si occupano di
% comporre il logo (vedere pagina~\pageref{p:fntcmd}).
%    \begin{macrocode}
\define@choicekey*{guit}{color}[\val\nr]{yes,true,no,false}[yes]{%
  \ifcase\nr\relax
    \@guit@colorizedtrue
  \or
    \@guit@colorizedtrue
  \or
    \@guit@colorizedfalse
  \or
    \@guit@colorizedfalse
  \fi}
\define@choicekey*{guit}{link}[\val\nr]{yes,true,no,false}[yes]{%
  \ifcase\nr\relax
    \@guit@linktrue
  \or
    \@guit@linktrue
  \or
    \@guit@linkfalse
  \or
    \@guit@linkfalse
  \fi}
\define@key{guit}{family}{\def\@guit@family{#1}}
\define@key{guit}{series}{\def\@guit@series{#1}}
%    \end{macrocode}
% 
% Il caricamento automatico di \pacchetto{hyperref} ha causato qualche
% problema, quindi a partire dalla versione 0.9.1 \`e stato rimosso. Di
% seguito viene effettuato il controllo sul fatto che il pacchetto sia stato
% caricato esplicitamente dall'utente.
%    \begin{macrocode}
\def\@if@guit@link@or@hyperref#1#2{%
  \if@guit@link
    \@if@guit@hyperrefloaded{#1}{#2}%
  \else
    #2%
  \fi}
\AtBeginDocument{%
  \@ifpackageloaded{hyperref}
    {\let\@if@guit@hyperrefloaded\@firstoftwo}
    {\let\@if@guit@hyperrefloaded\@secondoftwo}}
%    \end{macrocode}
% 
% \begin{macro}{\GuITcolor}
% \changes{v0.6}{2003/09/16}{Esteso con una versione ``stellata'' che
%   beneficia delle funzionalit\`a di \pacchetto{xcolor}}
% \changes{v0.7}{2003/09/26}{Trasformato in un comando \emph{robusto} (mea
%   culpa)}
% \changes{v0.8.2}{2005/04/08}{Modificata la dichiarazione del colore di
%   default}
% \changes{v0.9}{2005/11/19}{Eliminate le definizioni riguardanti
%   esclusivamente \pacchetto{color}}
% \begin{macro}{\guitcolor}
% Procediamo alla definizione di un nuovo comando per alterare il colore
% utilizzato da \cs{GuIT*} e \cs{guit*}. Definiamo anche il colore
% utilizzato di default.
%    \begin{macrocode}
\DeclareRobustCommand*{\GuITcolor}{%
  \@ifstar\x@guit@color@imp\@guit@color@imp}
%    \end{macrocode}
% Teniamo conto della presenza delle estensioni fornite dal pacchetto
% \pacchetto{xcolor} e realizziamo le implementazioni del comando
% precedente\footnote{In realt\`a, quel \cs{definecolor} proviene da
% \pacchetto{xcolor}, versione 2.00 o successiva}.
%    \begin{macrocode}
\newcommand*{\@guit@color@imp}[2][cmyk]{%
  \definecolor{coloredelGuIT}{#1}{#2}}
\newcommand*{\x@guit@color@imp}[2][\@empty]{%
  \edef\@tempa{#1}%
  \ifx\@empty\@tempa
    \colorlet{coloredelGuIT}{#2}%
  \else
    \colorlet{coloredelGuIT}[#1]{#2}%
  \fi}
\GuITcolor{1, 0, 1, 0.6}
\let\guitcolor\GuITcolor
%    \end{macrocode}
% \end{macro}
% \end{macro}
% 
% \begin{macro}{\@guit@url}
% \changes{v0.9}{2005/12/26}{Trasformata in una ``chiave'' creata tramite
%   \pacchetto{xkeyval}}
% L'impostazione dell'indirizzo avviene tramite \cs{setupGuIT} prima
% dell'inizio del documento.
%    \begin{macrocode}
\define@cmdkey{guit}[@guit@]{url}{\relax}
%    \end{macrocode}
% \end{macro}
% 
% \begin{macro}{\@guitimp}
% \changes{v0.8}{2004/10/28}{Aggiunto un argomento}
% \changes{v0.9}{2005/11/21}{Spostato in questa macro il controllo sul
%   colore a causa delle modifiche imposte dall'uso delle chiavi}
% \changes{v0.9}{2006/05/03}{Sostituiti i due \cs{normalcolor} con il nulla
%   quando il logo non deve essere colorato}
% Questa macro fornisce un primo livello di astrazione dalla vera
% implementazione del logo, permettendo di attivare opzionalmente la
% colorazione ed il meccanismo che trasforma ogni logo in un link al sito di
% \guit. Volendo evitare intromissioni di \pacchetto{hyperref} nell'aspetto
% del logo si \`e disabilitata la colorazione dei link (anche se
% \pacchetto{hyperref} venisse caricato con l'opzione |colorlinks=true|) e
% la creazione del box che \altro{Acrobat Reader} e qualche altro previewer
% visualizzano attorno ai link.
% 
% Al livello inferiore dell'implementazione viene passato come unico
% argomento la giustapposizione di \cs{@guit@family}, che contiene il nome
% della famiglia di font attualmente in uso oppure di quella scelta
% dall'utente tramite la chiave |family| oppure tramite l'uso diretto della
% chiavi implicitamente associate ad ogni famiglia, e \cs{@guit@series},
% l'analogo di \cs{@guit@family} per ci\`o che riguarda le serie.
%    \begin{macrocode}
\def\@guitimp#1{%
  \begingroup
    \setkeys*{guit}{#1}%
    \if@guit@colorized
      \def\@colorize@guit{\color{coloredelGuIT}}%
      \def\@decolorize@guit{\normalcolor}%
    \else
      \def\@colorize@guit{}%
      \def\@decolorize@guit{}%
    \fi
    \edef\@tempa{\@guit@family}%
    \ifx\@empty\@tempa
      \let\@guit@family\f@family
    \fi
    \edef\@tempa{\@guit@series}%
    \ifx\@empty\@tempa
      \let\@guit@series\f@series
    \fi
    \@if@guit@link@or@hyperref
      {\Hy@colorlinksfalse
       \def\@pdfborder{0 0 0}%
       \href{\@guit@url}{\@@guitimp{\@guit@family\@guit@series}}}
      {\@@guitimp{\@guit@family\@guit@series}}%
  \endgroup}
%    \end{macrocode}
% \end{macro}
% 
% Iniziamo ora a preparare i comandi da utilizzarsi in combinazione con i
% font attualmente supportati. Il codice poteva probabilmente essere scritto
% in maniera diversa e maggiormente compatta, ma ho preferito adottare un
% approccio modulare e pi\`u semplice.
% 
% \begin{macro}{\DeclareGuITLogoCommand}
% \changes{v0.8.2}{2005/04/08}{Nuovo comando per l'utente finale per creare
%   nuove realizzazioni del logo}
% Il lavoro necessario alla creazione delle diverse realizzazioni del logo
% viene interamente demandato ad una macro interna. Per mantenere un minimo
% di coerenza, quanto meno all'interno del singolo documento, permettiamo
% l'uso di questo comando solo nel preambolo e, di conseguenza, anche nel
% file di configurazione esterno.
%    \begin{macrocode}
\DeclareRobustCommand*{\DeclareGuITLogoCommand}{%
  \@gen@guit@fntcmd}
\@onlypreamble\DeclareGuITLogoCommand
%    \end{macrocode}
% \end{macro}
% 
% \begin{macro}{\@gen@guit@fntcmd}\label{p:fntcmd}
% \changes{v0.7.1}{2004/02/28}{Aggiunta questa macro per generare le macro
%   interne per i vari font; semplificate le macro dei font (con l'eccezione
%   di quella per il Computer Modern Roman)}
% \changes{v0.8}{2004/10/28}{Alterata la selezione del font da usare}
% \changes{v0.9}{2005/11/21}{Aggiunta la definizione contestuale di una
%   chiave con lo stesso nome della famiglia di font}
% \changes{v0.9}{2005/11/27}{Riscritta per gestire come secondo argomento
%   opzionale la serie}
% \begin{macro}{\@gen@guit@fntcmd@imp}
% Nell'ottica di voler semplificare il codice del pacchetto viene fatto
% largo uso di questa macro per generare ``automaticamente'' le macro
% interne che si occupano di comporre il logo nella maniera pi\`u corretta
% possibile a seconda del font in uso.
% 
% Vengono anche create due chiavi: una con il nome uguale al nome della
% famiglia, che ignora qualunque valore eventualmente assegnatole ed imposta
% \cs{@guit@family} in modo che sia utilizzabile in seguito, e la seconda
% con il nome formato dalla giustapposizione del nome della famiglia, del
% carattere \altro{/} e della serie, che ignora qualunque valore
% eventualmente assegnatole ed imposta \cs{@guit@family} e \cs{@guit@series}
% in modo che siano utilizzabili in seguito.
%    \begin{macrocode}
\def\@gen@guit@fntcmd#1{%
  \@ifnextchar[%
    {\@gen@guit@fntcmd@imp#1}%
    {\@gen@guit@fntcmd@imp#1[m]}}
\def\@gen@guit@fntcmd@imp#1[#2]#3#4#5#6{%
  \define@key{guit}{#1}[#1]{\def\@guit@family{#1}}%
  \define@key{guit}{#1/#2}[#1/#2]{%
    \def\@guit@family{#1}%
    \def\@guit@series{#2}}%
  \@namedef{@guit#1#2}{%
    \fontfamily{#1}\fontseries{#2}\scshape
    \@colorize@guit g\kern #3\lower #4\hbox{u}%
    \@decolorize@guit\kern #5 I\@colorize@guit\kern #6 t}}
%    \end{macrocode}
% \end{macro}
% \end{macro}
% 
% \begin{macro}{\AliasGuITLogoCommand}
% \changes{v0.9}{2005/11/30}{Nuovo comando per l'utente finale per creare
%   alias di nuove realizzazioni del logo}
% La creazione di un alias non si discosta molto da quanto appena visto, se
% non fosse per il fatto che l'ultima riga di \cs{AliasGuITLogoCommand} non
% \`e di immediata lettura.
%    \begin{macrocode}
\DeclareRobustCommand*{\AliasGuITLogoCommand}[4]{%
  \define@key{guit}{#1/#2}[#1/#2]{%
    \def\@guit@family{#3}%
    \def\@guit@series{#4}}%
  \expandafter\let\csname @guit#1#2\expandafter\endcsname\csname @guit#3#4\endcsname}
\@onlypreamble\AliasGuITLogoCommand
%    \end{macrocode}
% \end{macro}
% 
% \begin{macro}{\@guitcmrm}
% \changes{v0.3}{2003/06/12}{Aggiunta l'indicazione esplicita della famiglia
%   \textsf{cmr}; in precedenza si usava la famiglia di font con grazie
%   predefinita, qualunque essa fosse}
% \changes{v0.7.3}{2004/10/11}{Un minimo aggiustamento al coefficiente di
%   traslazione orizzontale della ``u''}
% \changes{v0.9}{2005/11/27}{Modificato il nome aggiungendo la serie}
% Se il logo deve essere scritto utilizzando il Computer Modern Roman
% richiediamo espressamente l'uso del font originale di Knuth impostando per
% l'encoding il valore OT1: questo ci permette di produrre documenti in
% formato \textsc{pdf} utilizzando font PostScript\texttrademark~anche senza
% dover usare la collezione \altro{cm-super}. Questo comando viene usanto
% anche quando il font principale usato dall'utente non \`e tra quelli
% supportati dal pacchetto. \emph{Non \`e possibile creare questa macro con
% \cs{@gen@guit@fntcmd}}.
%    \begin{macrocode}
\def\@guitcmrm{%
  \fontencoding{OT1}\fontfamily{cmr}\fontseries{m}\scshape
  \@colorize@guit g\kern -0.26em\lower 0.714ex\hbox{u}%
  \@decolorize@guit\kern -0.125em I\@colorize@guit\kern -0.125em t}
\define@key{guit}{cmr}[cmr]{\def\@guit@family{cmr}}
\define@key{guit}{cmr/m}[cmr/m]{%
  \def\@guit@family{cmr}%
  \def\@guit@series{m}}
%    \end{macrocode}
% \end{macro}
% 
% \begin{macro}{\@guitptmm}
% \changes{v0.7.3}{2004/10/11}{Un minimo aggiustamento al coefficiente di
%   traslazione orizzontale della ``u''}
% \changes{v0.9}{2005/11/27}{Modificato il nome aggiungendo la serie}
% \begin{macro}{\@guitptmbx}
% Questa macro viene utilizzata se l'utente usa il font Times.
%    \begin{macrocode}
\@gen@guit@fntcmd{ptm}{-0.27em}{0.5475ex}{-0.125em}{-0.125em}
\@gen@guit@fntcmd{ptm}[bx]{-0.345em}{0.68ex}{-0.125em}{-0.125em}
%    \end{macrocode}
% \end{macro}
% \end{macro}
% 
% \begin{macro}{\@guitpplm}
% \changes{v0.7.3}{2004/10/11}{Minimi aggiustamenti ai coefficienti di
%   traslazione orizzontale e verticale della ``u''}
% \changes{v0.9}{2005/11/27}{Modificato il nome aggiungendo la serie}
% \begin{macro}{\@guitpplbx}
% Se viene usato Palatino \cs{GuIT} user\`a questo comando.
%    \begin{macrocode}
\@gen@guit@fntcmd{ppl}{-0.276em}{0.717ex}{-0.125em}{-0.125em}
\@gen@guit@fntcmd{ppl}[bx]{-0.33em}{0.689ex}{-0.125em}{-0.125em}
%    \end{macrocode}
% \end{macro}
% \end{macro}
% 
% \begin{macro}{\@guitpncm}
% \changes{v0.7.3}{2004/10/11}{Minimi aggiustamenti ai coefficienti di
%   traslazione orizzontale e verticale della ``u''}
% \changes{v0.9}{2005/11/27}{Modificato il nome aggiungendo la serie}
% \begin{macro}{\@guitpncbx}
% Se il font attualmente in uso \`e il New Century Schoolbook verr\`a usata
% questa macro.
%    \begin{macrocode}
\@gen@guit@fntcmd{pnc}{-0.322em}{0.678ex}{-0.125em}{-0.125em}
\@gen@guit@fntcmd{pnc}[bx]{-0.36em}{0.656ex}{-0.125em}{-0.125em}
%    \end{macrocode}
% \end{macro}
% \end{macro}
% 
% \begin{macro}{\@guitputm}
% \changes{v0.9}{2005/11/27}{Modificato il nome aggiungendo la serie}
% Per il font Utopia si usano i valori di questo comando.
%    \begin{macrocode}
\@gen@guit@fntcmd{put}{-0.285em}{0.61ex}{-0.125em}{-0.125em}
\AliasGuITLogoCommand{put}{bx}{put}{m}
%    \end{macrocode}
% \end{macro}
% 
% \begin{macro}{\@guitbchm}
% \changes{v0.9}{2005/11/27}{Modificato il nome aggiungendo la serie}
% \begin{macro}{\@guitbchbx}
% Nel caso si usi il font Charter questa \`e la macro utilizzata.
%    \begin{macrocode}
\@gen@guit@fntcmd{bch}{-0.28em}{0.58ex}{-0.125em}{-0.125em}
\@gen@guit@fntcmd{bch}[bx]{-0.31em}{0.56ex}{-0.125em}{-0.125em}
%    \end{macrocode}
% \end{macro}
% \end{macro}
% 
% \begin{macro}{\@guitpbkm}
% \changes{v0.9}{2005/11/27}{Modificato il nome aggiungendo la serie}
% \begin{macro}{\@guitpbkbx}
% L'ultimo comando ``interno'' viene usato per il font Bookman.
%    \begin{macrocode}
\@gen@guit@fntcmd{pbk}{-0.3em}{0.59ex}{-0.125em}{-0.125em}
\@gen@guit@fntcmd{pbk}[bx]{-0.34em}{0.52ex}{-0.125em}{-0.125em}
%    \end{macrocode}
% \end{macro}
% \end{macro}
% 
% \begin{macro}{\@@guitimp}
% \changes{v0.8}{2004/10/28}{Enormemente semplificata grazie ai preziosi
%   suggerimenti del Prof.\ Enrico Gregorio; aggiunto un argomento}
% L'implementazione del comando a disposizione dell'utente \`e molto
% semplice: se non esiste la macro interna associata alla famiglia di font
% selezionata si ricade sull'onnipresente Computer Modern Roman, altrimenti
% si utilizza la macro appropriata.
%    \begin{macrocode}
\def\@@guitimp#1{%
  \@ifundefined{@guit#1}{\@guitcmrm}{\@nameuse{@guit#1}}}
%    \end{macrocode}
% \end{macro}
% 
% \begin{macro}{\@guit@color@on}
% \changes{v0.7}{2003/10/17}{Estratto il controllo sul valore di
%   \cs{@guit@colorized} in modo da effettuare l'operazione una volta sola}
% \changes{v0.8}{2004/10/28}{Aggiunto un argomento opzionale}
% \changes{v0.9}{2005/11/21}{Eliminati i controlli sull'uso del colore a
%   causa delle modifiche imposte dall'uso delle chiavi}
% La porzione dell'implementazione di \cs{GuIT} che attiva la colorazione
% del logo in seguito alla presenza del carattere \altro{*}.
%    \begin{macrocode}
\newcommand*{\@guit@color@on}[1][]{%
  \@guitimp{color, #1}\ignorespaces}
%    \end{macrocode}
% \end{macro}
% 
% \begin{macro}{\@guit@color@off}
% \changes{v0.8}{2004/10/28}{Aggiunto un argomento opzionale}
% \changes{v0.9}{2005/11/21}{Eliminati i controlli sull'uso del colore a
%   causa delle modifiche imposte dall'uso delle chiavi}
% La porzione dell'implementazione di \cs{GuIT} che potenzialmente non
% altera la colorazione del logo.
%    \begin{macrocode}
\newcommand*{\@guit@color@off}[1][]{%
  \@guitimp{#1}\ignorespaces}
%    \end{macrocode}
% \end{macro}
% 
% \begin{macro}{\GuIT}
% Alla fine, ecco il comando a disposizione dell'utente finale per scrivere
% il logo del \GuIT. L'implementazione gestisce unicamente la presenza
% dell'asterisco, demandando il controllo dell'eventuale argomento opzionale
% allo strato sottostante.
%    \begin{macrocode}
\DeclareRobustCommand*{\GuIT}{%
  \@ifstar\@guit@color@on\@guit@color@off}
%    \end{macrocode}
% \end{macro}
% 
% \begin{macro}{\guit}
% Per semplificarci (o, piuttosto, semplificarmi) la vita ecco la versione
% del comando precedente con il nome completamente in lettere minuscole.
%    \begin{macrocode}
\let\guit\GuIT
%    \end{macrocode}
% \end{macro}
% 
% \begin{macro}{\GuITtext}
% \changes{v0.4}{2003/07/15}{Modificato il testo prodotto da questo comando
%   per aderire perfettamente alla dizione ufficiale del Gruppo}
% \changes{v0.7}{2003/10/15}{Sdoppiata la definizione per tenere conto
%   dell'uso di \pacchetto{hyperref}}
% \changes{v0.9}{2005/11/21}{Unificata la definizione per poter influenzare
%   la generazione del link tramite l'uso delle chiavi}
% Il comando per indicare la denominazione completa e corretta del
% Gruppo. L'unica chiave che ha effetto su questo comando \`e |link|.
%    \begin{macrocode}
\DeclareRobustCommand*{\GuITtext}[1][]{%
  \begingroup
    \setkeys*{guit}{#1}%
    \@if@guit@link@or@hyperref
      {\href{\@guit@url}{Gruppo Utilizzatori Italiani di \TeX}}
      {Gruppo Utilizzatori Italiani di \TeX}%
  \endgroup
  \ignorespaces}
%    \end{macrocode}
% \end{macro}
% 
% \begin{macro}{\guittext}
% Come per \cs{GuIT} forniamo la versione con il nome completamente in
% lettere minuscole.
%    \begin{macrocode}
\let\guittext\GuITtext
%    \end{macrocode}
% \end{macro}
% 
% \begin{macro}{\GuITtextEn}
% \begin{macro}{\guittexten}
% Per il comando che riporta la denominazione anglicizzata del \guit\ vale
% quanto riportato per \cs{GuITtext} e \cs{guittext}.
%    \begin{macrocode}
\DeclareRobustCommand*{\GuITtextEn}[1][]{%
  \begingroup
    \setkeys*{guit}{#1}%
    \@if@guit@link@or@hyperref
      {\href{\@guit@url}{Italian \TeX\ User Group}}
      {Italian \TeX\ User Group}%
  \endgroup
  \ignorespaces}
\let\guittexten\GuITtextEn
%    \end{macrocode}
% \end{macro}
% \end{macro}
% 
% \begin{macro}{\GuITurl}
% \changes{v0.7}{2003/10/15}{Aggiunto questo comando per scrivere
%   l'indirizzo del sito internet di \guit~senza troppi errori}
% \changes{v0.9}{2005/11/25}{Aggiunto un argomento opzionale per gestire le
%   chiavi}
% Il comando per scrivere automaticamente l'indirizzo del sito internet di
% \guit~``subisce'', implicitamente, l'effetto del caricamente del pacchetto
% \pacchetto{hyperref}, quindi il comando scriverebbe sempre l'indirizzo del
% sito come collegamento al sito stesso: per evitare questo effetto
% collaterale \`e stato realizzato un accrocchio ad hoc. L'argomento
% opzionale viene considerato un'estensione dell'indirizzo (una pagina, una
% sezione, etc.) e composto di conseguenza.
%    \begin{macrocode}
\DeclareRobustCommand*{\GuITurl}[1][]{%
  \begingroup
    \setkeys*{guit}{#1}%
    \@if@guit@link@or@hyperref
      {\let\@guit@url@imp\url}
      {\let\@guit@url@imp\@guit@url@nolink}%
    \edef\@tempa{\@guit@url}%
    \ifx\@empty\@tempa
      \relax
    \else
      \edef\@tempa{\@guit@document}%
      \ifx\@empty\@tempa
        \expandafter\@guit@url@imp\expandafter{\@guit@url}%
      \else
        \expandafter\expandafter\expandafter\@guit@url@imp%
          \expandafter\expandafter\expandafter{%
          \expandafter\@guit@url\expandafter/\@guit@document}%
      \fi
    \fi
  \endgroup
  \ignorespaces}
\DeclareUrlCommand\@guit@url@nolink{}
\define@cmdkey{guit}[@guit@]{document}{\relax}
%    \end{macrocode}
% \end{macro}
% 
% \begin{macro}{\guiturl}
% Come per \cs{GuIT} forniamo la versione con il nome completamente in
% lettere minuscole.
%    \begin{macrocode}
\let\guiturl\GuITurl
%    \end{macrocode}
% \end{macro}
% 
% \begin{macro}{\GuITforum}
% \changes{v0.7.2}{2004/09/16}{Aggiunto questo comando per scrivere
%   l'indirizzo del forum di \guit~senza troppi errori}
% \changes{v0.9}{2005/11/27}{Reimplementato usando \cs{GuITurl} ed il nuovo
%   argomento opzionale; modificato l'indirizzo del forum su suggerimento di
%   Emiliano Giovanni Vavassori e Maurizio W. Himmelman}
% Quanto detto per \cs{GuITurl} vale anche per questo comando.
%    \begin{macrocode}
\DeclareRobustCommand*{\GuITforum}[1][]{%
  \guiturl[#1, document=forum]}
%    \end{macrocode}
% \end{macro}
% 
% \begin{macro}{\guitforum}
% Come per \cs{GuIT} forniamo la versione con il nome completamente in
% lettere minuscole.
%    \begin{macrocode}
\let\guitforum\GuITforum
%    \end{macrocode}
% \end{macro}
% 
% \begin{macro}{\GuITmeeting}
% \changes{v0.7.3}{2004/10/11}{Aggiunto questo comando ed alcune macro
%   accessorie per realizzare il logo del convegno periodico}
% \begin{macro}{\@guit@meeting@color@on}
% \changes{v0.7.4}{2004/10/15}{Corretta una svista clamorosa nella selezione
%   del font da usare}
% \changes{v0.7.5}{2004/10/17}{Ri-corretta questa macro (si spera per
%   l'ultima volta)}
% \changes{v0.8.1}{2004/11/15}{Aggiunta l'indicazione dell'encoding da
%   usare}
% \changes{v0.9}{2005/11/21}{Isolato praticamente tutto il codice in
%   un'altra macro}
% \begin{macro}{\@guit@meeting@color@off}
% \changes{v0.7.4}{2004/10/15}{Corretta una svista clamorosa nella selezione
%   del font da usare}
% \changes{v0.7.5}{2004/10/17}{Ri-corretta questa macro (si spera per
%   l'ultima volta)}
% \changes{v0.8.1}{2004/11/15}{Aggiunta l'indicazione dell'encoding da
%   usare}
% \changes{v0.9}{2005/11/21}{Isolato praticamente tutto il codice in
%   un'altra macro}
% L'implementazione del logo del convegno \`e stata partizionata per
% realizzare una versione a colori duplicando poco codice.
%    \begin{macrocode}
\DeclareRobustCommand*{\GuITmeeting}{%
  \@ifstar\@guit@meeting@color@on\@guit@meeting@color@off}
\newcommand*{\@guit@meeting@color@on}[1][]{%
  \@guit@meeting@imp{*}{#1}}
\newcommand*{\@guit@meeting@color@off}[1][]{%
  \@guit@meeting@imp{}{#1}}
%    \end{macrocode}
% \end{macro}
% \end{macro}
% \end{macro}
% 
% Definiamo due chiavi per gestire l'anno e lo stile del logo.
%    \begin{macrocode}
\define@cmdkey{guit}[@guit@]{year}{\relax}
\newif\if@guit@inline
\define@choicekey*{guit}{style}[\val\nr]{inline,display}{%
  \ifcase\nr\relax
    \@guit@inlinetrue
  \or
    \@guit@inlinefalse
  \fi}
%    \end{macrocode}
% 
% \begin{macro}{\@guit@meeting@imp}
% In questa macro gestiamo le chiavi (|#2|) e l'asterisco (|#1|), impostiamo
% il font da usare ed ordiniamo la macro con \textit{meeting} e quella che
% si occupa dell'anno in base allo stile di visualizzazione attivo.
%    \begin{macrocode}
\def\@guit@meeting@imp#1#2{%
  \begingroup
    \setkeys*{guit}{#2}%
    \fontencoding{OT1}\fontfamily{cmr}\fontseries{m}\selectfont
    \guit#1
    \if@guit@inline
      \@guit@meeting@part\@guit@meeting@year
    \else
      \@guit@meeting@year\@guit@meeting@part
    \fi
  \endgroup
  \ignorespaces}
%    \end{macrocode}
% \end{macro}
% 
% \begin{macro}{\@guit@meeting@part}
% \changes{v0.9}{2005/11/21}{Modificata per tenere conto dei due stili del
%   logo}
% Se il valore di |style| \`e |display|, allora la parte
% ``\textit{meeting}'' del logo viene traslata usando per entrambe le
% direzioni, come unit\`a di misura, \emph{ex} per cercare di limitare al
% minimo le deformazioni dello spazio dovute alla definizione di \emph{em}
% (che nella famiglia Computer Modern varia in maniera non proporzionale
% alla dimensione nominale del font~\cite{CMT}).
%    \begin{macrocode}
\def\@guit@meeting@part{%
  \if@guit@inline
    \textit{meeting}%
  \else
    \kern -2.02ex\lower 1.25ex\hbox{\textit{meeting}}%
  \fi}
%    \end{macrocode}
% \end{macro}
% 
% \begin{macro}{\@guit@meeting@year}
% \changes{v0.9}{2005/11/21}{Aggiunto questo comando per gestire l'anno su
%   esplicita richiesta di Maurizio W.~Himmelmann}
% \begin{macro}{\@guit@meeting@year@imp}
% Se il valore di |style| \`e |display|, allora l'anno viene traslato e
% ridimensionato usando \cs{scalebox}.
%    \begin{macrocode}
\def\@guit@meeting@year{%
  \edef\@tempa{\@guit@year}%
  \ifx\@empty\@tempa
    \relax
  \else
    \@guit@meeting@year@imp{\@guit@year}%
  \fi}
\def\@guit@meeting@year@imp#1{%
  \if@guit@inline
    \fontfamily{pzc}\selectfont #1
  \else
    \rlap{%
      \hskip0.7em\fontfamily{pzc}\selectfont
      \raise 0.5ex\hbox{\scalebox{0.85}{#1}}}%
  \fi}
%    \end{macrocode}
% \end{macro}
% \end{macro}
% 
% \begin{macro}{\guitmeeting}
% Come per \cs{GuIT} forniamo la versione con il nome completamente in
% lettere minuscole.
%    \begin{macrocode}
\let\guitmeeting\GuITmeeting
%    \end{macrocode}
% \end{macro}
% 
% 
% \`E essenziale che vengano impostate alcune chiavi in modo da creare
% alcune macro d'uso interno prima di proseguire; per fare questo \`e
% sufficiente utilizzare \cs{setupGuIT} con le impostazioni di default.
%    \begin{macrocode}
\setupGuIT
%    \end{macrocode}
% 
% Il codice seguente \`e stato integrato da \pacchetto{arslogo}, v0.1.3,
% scritto da Massimo Caschili, dietro esplicita richiesta del \guittext,
% detentore del copyright.
%    \begin{macrocode}
\DeclareRobustCommand*{\Ars}{%
  \textsf{\lower -.48ex\hbox{\rotatebox{-20}{A}}\kern -.3em{rs}}%
  \kern -.05em\TeX\kern -.17em\lower -.357ex\hbox{nica}}
\DeclareRobustCommand*{\Arsob}{\rotatebox{20}{\Ars}}
\DeclareRobustCommand*{\ars}{%
  \textsf{\lower -.48ex\hbox{\rotatebox{-20}{A}}\kern -.3em{rs}}}
\DeclareRobustCommand*{\tecnica}{%
  \TeX\kern -.17em\lower -.357ex\hbox{nica}}
\DeclareRobustCommand*{\arsta}{%
  \ars \kern -0.65em\lower -1.3ex\hbox{\scalebox{0.18}{\hbox{%
    \TeX\kern -.17em\lower -.357ex\hbox{nica}}}}}
\DeclareRobustCommand*{\arstb}{%
  \ars \kern -0.65em\lower -1.25ex\hbox{\scalebox{0.34}{\hbox{%
    \TeX\kern -.17em\lower -.357ex\hbox{nica}}}}}
\DeclareRobustCommand*{\arstv}{%
  \ars \kern -0.05em\lower -1.818ex\hbox{\rotatebox{-90}{\hbox{%
    \scalebox{0.225}{\hbox{%
      \TeX\kern -.17em\lower -.357ex\hbox{nica}}}}}}}
\DeclareRobustCommand*{\arsto}{%
  \rotatebox{49}{\lower -.84ex\hbox{\scalebox{0.214}{\hbox{%
    \TeX\kern -.17em\lower -.357ex\hbox{nica}}}}}%
  \kern -.74em\hbox{\ars}}
\DeclareRobustCommand*{\Arsto}{%
  \rotatebox{49}{\lower -.84ex\hbox{\scalebox{0.214}{\hbox{%
    \TeX\kern -.17em\lower -.357ex\hbox{nica}}}}}%
  \kern -.74em\hbox{\Ars}}
%    \end{macrocode}
% 
% Terminiamo leggendo il file di configurazione. Nel caso non si riuscisse a
% trovarlo, ci limitiamo ad avvertire l'utente.
%    \begin{macrocode}
\InputIfFileExists{guit.cfg}%
  {\relax}%
  {\PackageInfo{guit}{Configuration file not found}}
%</style>
%    \end{macrocode}
% 
% 
% 
% \subsection{Il file di configurazione}
% Il file di configurazione \`e, in realt\`a, vuoto e fornisce solo qualche
% esempio. La definizione delle realizzazioni del logo e del colore
% predefinito del logo non \`e inserita veramente nel file di configurazione
% per diminuire il rischio che l'utente le modifichi o le rimuova
% accidentalmente.
%    \begin{macrocode}
%<*cfg>
\ProvidesFile{guit.cfg}[2009/07/31 v0.9.1 File di configurazione di guit.sty]
%% Esempi:
%% \GuITcolor{1, 0, 1, 0.6}
%% \DeclareGuITLogoCommand{ptm}{-0.27em}{0.5475ex}{-0.125em}{-0.125em}
%% \DeclareGuITLogoCommand{ptm}[bx]{-0.345em}{0.68ex}{-0.125em}{-0.125em}
%% \DeclareGuITLogoCommand{ppl}{-0.276em}{0.717ex}{-0.125em}{-0.125em}
%% \DeclareGuITLogoCommand{ppl}[bx]{-0.33em}{0.689ex}{-0.125em}{-0.125em}
%% \DeclareGuITLogoCommand{pnc}{-0.322em}{0.678ex}{-0.125em}{-0.125em}
%% \DeclareGuITLogoCommand{pnc}[bx]{-0.36em}{0.656ex}{-0.125em}{-0.125em}
%% \DeclareGuITLogoCommand{put}{-0.285em}{0.61ex}{-0.125em}{-0.125em}
%% \AliasGuITLogoCommand{put}{bx}{put}{m}
%% \DeclareGuITLogoCommand{bch}{-0.28em}{0.58ex}{-0.125em}{-0.125em}
%% \DeclareGuITLogoCommand{bch}[bx]{-0.31em}{0.56ex}{-0.125em}{-0.125em}
%% \DeclareGuITLogoCommand{pbk}{-0.3em}{0.59ex}{-0.125em}{-0.125em}
%% \DeclareGuITLogoCommand{pbk}[bx]{-0.34em}{0.52ex}{-0.125em}{-0.125em}
%</cfg>
%    \end{macrocode}
% 
% 
% 
% \makeatletter
% \c@IndexColumns=2
% \c@GlossaryColumns=2
% \def\index@prologue{\section*{Indice}%
%   \markboth{Indice}{Indice}%
%   I numeri scritti in corsivo si riferiscono alla pagina in cui la voce
%   corrispondente viene descritta; i numeri sottolineati si riferiscono
%   alla
%   \ifcodeline@index
%     linea di codice della
%   \fi
%   definizione; i numeri in carattere tondo si riferiscono alle
%   \ifcodeline@index
%     linee di codice
%   \else
%     pagine
%   \fi
%   in cui la voce viene usata.}%
% \def\glossary@prologue{\section*{{Storico dei cambiamenti}}%
%   \markboth{{Storico dei cambiamenti}}{{Storico dei cambiamenti}}}%
% \makeatother
% \clearpage
% \Finale
% \clearpage
% \PrintIndex
% \clearpage
% \PrintChanges
% 
% 
% 
\endinput
