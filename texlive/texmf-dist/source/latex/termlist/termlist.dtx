% \iffalse meta-comment
%
% The termlist package provides environments to typeset lists of
% numbered terms, where the "terms" may contain anything printable.
% The list of terms is formatted to look consistent to a list of
% mathematical equations inside an eqnarray environment.  Also for
% consistency, the same equation counter is used for numbering.  You
% may think of termlist as a generalisation of the eqnarray
% environment, if you want to typeset more than just mathematics.
%
% Copyright 2008, 2010 Bernd Rellermeyer
%
% This file may be distributed and/or modified under the conditions
% of the LaTeX Project Public License, either version 1.3b of this
% license or (at your option) any later version.  The latest version
% of this license is in:
%
% http://www.latex-project.org/lppl.txt
%
% and version 1.3b or later is part of all distributions of LaTeX
% version 2005/12/01 or later.
%
% Unpacking:
%    (a) If termlist.ins is present:
%           tex termlist.ins
%    (b) Without termlist.ins:
%           tex termlist.dtx
%    (c) If you insist on using LaTeX
%           latex \let\install=y% \iffalse meta-comment
%
% The termlist package provides environments to typeset lists of
% numbered terms, where the "terms" may contain anything printable.
% The list of terms is formatted to look consistent to a list of
% mathematical equations inside an eqnarray environment.  Also for
% consistency, the same equation counter is used for numbering.  You
% may think of termlist as a generalisation of the eqnarray
% environment, if you want to typeset more than just mathematics.
%
% Copyright 2008, 2010 Bernd Rellermeyer
%
% This file may be distributed and/or modified under the conditions
% of the LaTeX Project Public License, either version 1.3b of this
% license or (at your option) any later version.  The latest version
% of this license is in:
%
% http://www.latex-project.org/lppl.txt
%
% and version 1.3b or later is part of all distributions of LaTeX
% version 2005/12/01 or later.
%
% Unpacking:
%    (a) If termlist.ins is present:
%           tex termlist.ins
%    (b) Without termlist.ins:
%           tex termlist.dtx
%    (c) If you insist on using LaTeX
%           latex \let\install=y% \iffalse meta-comment
%
% The termlist package provides environments to typeset lists of
% numbered terms, where the "terms" may contain anything printable.
% The list of terms is formatted to look consistent to a list of
% mathematical equations inside an eqnarray environment.  Also for
% consistency, the same equation counter is used for numbering.  You
% may think of termlist as a generalisation of the eqnarray
% environment, if you want to typeset more than just mathematics.
%
% Copyright 2008, 2010 Bernd Rellermeyer
%
% This file may be distributed and/or modified under the conditions
% of the LaTeX Project Public License, either version 1.3b of this
% license or (at your option) any later version.  The latest version
% of this license is in:
%
% http://www.latex-project.org/lppl.txt
%
% and version 1.3b or later is part of all distributions of LaTeX
% version 2005/12/01 or later.
%
% Unpacking:
%    (a) If termlist.ins is present:
%           tex termlist.ins
%    (b) Without termlist.ins:
%           tex termlist.dtx
%    (c) If you insist on using LaTeX
%           latex \let\install=y% \iffalse meta-comment
%
% The termlist package provides environments to typeset lists of
% numbered terms, where the "terms" may contain anything printable.
% The list of terms is formatted to look consistent to a list of
% mathematical equations inside an eqnarray environment.  Also for
% consistency, the same equation counter is used for numbering.  You
% may think of termlist as a generalisation of the eqnarray
% environment, if you want to typeset more than just mathematics.
%
% Copyright 2008, 2010 Bernd Rellermeyer
%
% This file may be distributed and/or modified under the conditions
% of the LaTeX Project Public License, either version 1.3b of this
% license or (at your option) any later version.  The latest version
% of this license is in:
%
% http://www.latex-project.org/lppl.txt
%
% and version 1.3b or later is part of all distributions of LaTeX
% version 2005/12/01 or later.
%
% Unpacking:
%    (a) If termlist.ins is present:
%           tex termlist.ins
%    (b) Without termlist.ins:
%           tex termlist.dtx
%    (c) If you insist on using LaTeX
%           latex \let\install=y\input{termlist.dtx}
%        (quote the arguments according to the demands of your shell)
%
% Documentation:
%    (a) If termlist.drv is present:
%           latex termlist.drv
%    (b) Without termlist.drv:
%           latex termlist.dtx; ...
%    The class ltxdoc loads the configuration file ltxdoc.cfg if
%    available.  Here you can specify further options, e.g. use A4
%    as paper format:
%       \PassOptionsToClass{a4paper}{article}
%
%    Programm calls to get the documentation:
%       pdflatex termlist.dtx
%       pdflatex termlist.dtx
%
% Installation:
%    TDS:tex/latex/termlist/termlist.sty
%    TDS:doc/latex/termlist/termlist.pdf
%    TDS:source/latex/termlist/termlist.dtx
%
%<package>\NeedsTeXFormat{LaTeX2e}
%<package>\ProvidesPackage{termlist}[2010/06/12 v1.1 Minor layout corrections]
%<*ignore>
\begingroup
  \def\x{LaTeX2e}%
\expandafter\endgroup
\ifcase 0\ifx\install y1\fi\expandafter
         \ifx\csname processbatchFile\endcsname\relax\else1\fi
         \ifx\fmtname\x\else 1\fi\relax
\else\csname fi\endcsname
%</ignore>
%<*install>
\input docstrip.tex
\keepsilent

\usedir{tex/latex/termlist}

\preamble

Typeset lists of numbered terms.
 
Copyright 2008, 2010 Bernd Rellermeyer

This file may be distributed and/or modified under the conditions
of the LaTeX Project Public License, either version 1.3b of this
license or (at your option) any later version.  The latest version
of this license is in:

http://www.latex-project.org/lppl.txt

and version 1.3b or later is part of all distributions of LaTeX
version 2005/12/01 or later.

\endpreamble

\generate{\file{termlist.sty}{\from{termlist.dtx}{package}}}

\ifToplevel{
\obeyspaces
\Msg{***********************************************************}
\Msg{*}
\Msg{* To finish the installation you have to move the following}
\Msg{* file into a directory searched by TeX:}
\Msg{*}
\Msg{* \space\space termlist.sty}
\Msg{*}
\Msg{* To produce the documentation run the file `termlist.dtx'}
\Msg{* through LaTeX2e.}
\Msg{*}
\Msg{* Happy TeXing}
\Msg{***********************************************************}
}

\endbatchfile
%</install>
%<*ignore>
\fi
%</ignore>
%<*driver>
\documentclass{ltxdoc}
\usepackage{termlist}
\usepackage{sverb}
\begin{document}
\DocInput{termlist.dtx}
\end{document}
%</driver>
% \fi
%
% \CheckSum{73}
% \CharacterTable
%  {Upper-case    \A\B\C\D\E\F\G\H\I\J\K\L\M\N\O\P\Q\R\S\T\U\V\W\X\Y\Z
%   Lower-case    \a\b\c\d\e\f\g\h\i\j\k\l\m\n\o\p\q\r\s\t\u\v\w\x\y\z
%   Digits        \0\1\2\3\4\5\6\7\8\9
%   Exclamation   \!     Double quote  \"     Hash (number) \#
%   Dollar        \$     Percent       \%     Ampersand     \&
%   Acute accent  \'     Left paren    \(     Right paren   \)
%   Asterisk      \*     Plus          \+     Comma         \,
%   Minus         \-     Point         \.     Solidus       \/
%   Colon         \:     Semicolon     \;     Less than     \<
%   Equals        \=     Greater than  \>     Question mark \?
%   Commercial at \@     Left bracket  \[     Backslash     \\
%   Right bracket \]     Circumflex    \^     Underscore    \_
%   Grave accent  \`     Left brace    \{     Vertical bar  \|
%   Right brace   \}     Tilde         \~}
%
%
% \changes{v1.0}{2008/08/18}{First release}
% \changes{v1.1}{2010/06/12}{Minor layout corrections}
%
% \GetFileInfo{termlist.sty}
%
% \title{The \textsf{termlist} package}
% \author{Bernd Rellermeyer}
% \date{\fileversion, \filedate}
%
% \maketitle
%
% \begin{abstract}\noindent
%   The \textsf{termlist} package provides environments to typeset
%   lists of numbered terms, where the ``terms'' may contain anything
%   printable.  The list of terms is formatted to look consistent to a
%   list of mathematical equations inside an |eqnarray| environment.
%   Also for consistency, the same |equation| counter is used for
%   numbering.  You may think of termlist as a generalisation of the
%   |eqnarray| environment, if you want to typeset more than just
%   mathematics.
% \end{abstract}
%
% \section{Introduction}
%
% The \LaTeX\ typesetting system is widely used by mathematicians.  A
% number of constructs, like the |equation| or |eqnarray|
% environments, fit the specific needs of mathematical texts.  The
% math modes are intrinsic parts of \TeX\ and have left their marks on
% the whole system.
%
% The \textsf{termlist} package may be seen as a generalization of the
% |equation| and |eqnarray| environments for non-mathematical texts.
% It provides environments to typeset any kinds of mathematical or
% non-mathematical terms, like propositions, hypotheses, examples, or
% linguistic terms, in the same way mathematical terms are typesetted
% inside an |equation| or |eqnarray| environment.  I.\,e., the terms
% are indented and labelled with a continuous number---for consistency
% with the |equation| and |eqnarray| environments, the same |equation|
% counter is used for that---so that the look consistent to
% mathematical terms inside those environments.
%
% The \textsf{termlist} package is based on an idea of the
% \textsf{covington} package.
%
% \section{Usage}
%
% The \textsf{termlist} package provides the four new environments
% |termlist|, |term|, |termlist*|, and |term*|.
%
% \DescribeEnv{termlist} The |termlist| environment is the fundamental
% one of these environments.  It is a list of terms, each indented and
% labelled with the |equation| counter on the left side by default.
% The default label can be changed by the optional argument of the
% |\item| command.
%
% For consistency with the |equation| and |eqnarray| environments, the
% global |leqno| option or something similar should be used to print
% the labels of mathematical terms inside these environments also on
% the left side.
%
% \begin{demo}{termlist}
%   \begin{termlist}
%   \item Here is an example of
%     the \verb|termlist|
%     environment.
%   \item\label{second item}
%     Each term is indented
%     and labelled with the
%     \verb|equation| counter.
%   \item[(\ref{second item}a)]
%     The label of the term
%     can be changed by the
%     optional argument of the
%     \verb|\item| command.
%   \item Using the optional
%     argument of the
%     \verb|\item| command
%     does not increment the
%     \verb|equation| counter.
%   \end{termlist}
% \end{demo}
%
% \DescribeEnv{term} The |term| environment is a |termlist|
% environment with only one list item.
%
% \begin{demo}{term}
%   \begin{term}
%     The \verb|term| environment
%     labels the term with the
%     \verb|equation| counter.
%   \end{term}
% \end{demo}
%
% \DescribeEnv{termlist*} The |termlist*| environment is the same as
% the |termlist| environment, but with empty labels.
%
% \begin{demo}{termlist*}
%   \begin{termlist*}
%     \item The \verb|termlist*|
%       environment is a list of
%       terms.
%     \item The terms do not
%       have labels.
%   \end{termlist*}
% \end{demo}
%
% \DescribeEnv{term*} The |term*| environment is the same as the
% |term| environment, but with an empty label.
%
% \begin{demo}{term*}
%   \begin{term*}
%     The \verb|term*|
%     environment does not
%     label the term.
%   \end{term*}
%   \begin{term}
%     Using the
%     \verb|termlist*|
%     or \verb|term*|
%     environment does not
%     increment the
%     \verb|equation|
%     counter.
%   \end{term}
% \end{demo}
%
% The four environments provided by this package can be customized by
% the three macros |\termlabel|, |\termlabelfont|, and |\termindent|.
%
% \DescribeMacro\termlabel The label of each term is defined by the
% macro |\termlabel|.  For consistency with the |equation| and
% |eqnarray| environments, its default value is |(\theequation)|.  Use
% one of the commands
% \begin{termlist*}
% \item |\def\termlabel{|\meta{labeldescription}|}|, or
% \item |\renewcommand{\termlabel}{|\meta{labeldescription}|}|
% \end{termlist*}
% to change its value.  Note that even if the value of |\termlabel|
% does not contain something like |\theequation|, the |equation|
% counter is incremented with each list item.
%
% \DescribeMacro\termlabelfont The label font and label color is
% defined by the macro |\termlabelfont|.  For consistency with the
% |equation| and |eqnarray| environments, its default value is
% |\normalfont\normalcolor|.  Use one of the commands
% \begin{termlist*}
% \item |\def\termlabelfont{|\meta{fontdescription}|}|, or
% \item |\renewcommand{\termlabelfont}{|\meta{fontdescription}|}|
% \end{termlist*}
% to change its value.
%
% Note that for the |equation| and |eqnarray| environments, the whole
% label, including its font and color, is defined by the macro
% |\@eqnnum|.  For consistency, |\termlabel| and |\termlabelfont|
% should always have values that match the value of |\@eqnnum|.
%
% \DescribeMacro\termindent The amount of the left indentation of each
% term is defined by the macro |\termindent|.  Its default value is
% |\leftmargini|.  This is also the default value of the macro
% |\mathindent|, which defines the amount of the left indentation of
% mathematical terms inside an |equation| or |eqnarray| environment,
% if the \textsf{fleqn} package is used.  In this case, |\termindent|
% and |\mathindent| should always have the same values.  Use on of the
% commands
% \begin{termlist*}
% \item |\termindent|\meta{dimension}, or
% \item |\setlength{\termindent}{|\meta{dimension}|}|
% \end{termlist*}
% to change the value of |\termindent|.
%
% \StopEventually{}
%
% \iffalse
%<*package>
% \fi
%
% \section{Implementation}
%
% \begin{macro}{\termlabel}
%   The |termlabel| macro defines the label of each term inside the
%   |termlist| and |term| environments.
%    \begin{macrocode}
\def\termlabel{(\theequation)}
%    \end{macrocode}
% \end{macro}
%
% \begin{macro}{\termlabelfont}
% \begin{macro}{\@termlistmklab}
%   The |termlabelfont| macro defines the font and color of the label
%   of each term inside the |termlist| and |term| environments.  The
%   macro |\@termlistmklab| is used internally in the definition of
%   the |termlist| environment.
%    \begin{macrocode}
\def\termlabelfont{\normalfont\normalcolor}
\def\@termlistmklab#1{\termlabelfont #1\hfil}
%    \end{macrocode}
% \end{macro}
% \end{macro}
%
% \begin{macro}{\termindent}
%   The |termindent| macro defines the amount of the left indentation
%   of each term inside one of the four environments provided by this
%   package.
%    \begin{macrocode}
\newdimen\termindent
\termindent\leftmargini
%    \end{macrocode}
% \end{macro}
%
% \begin{environment}{termlist}
%   The |termlist| environment is implemented as a list.  Each list
%   item has a default label defined by |\termlabel| and
%   |\@termlistmklab| and left and right margins defined by
%   |\termindent| and |\leftmargini|, resp.  The indentation
%   |\listparindent| and the amount of space between two paragraphs
%   |\parsep| inside the list are set to their corresponding values
%   outside the list, |\parindent| and |\parskip|, resp.  The total
%   amount of space between two list items is the sum of |\parsep| and
%   |\itemsep|.  The extra amount of space |\itemsep| is set to such a
%   value, that the total amount of space between two list items is
%   the maximum of |\parsep| and the extra amount of space |\jot|
%   between two lines inside an |eqnarray| environment.  To be further
%   consistent to the |equation| and |eqnarray| environments, the
%   amount of space above and below the list |\topsep| is set to
%   |\abovedisplayskip|.
%    \begin{macrocode}
\newenvironment{termlist}{%
  \list{\termlabel}{%
    \@nmbrlisttrue
    \def\@listctr{equation}%
    \partopsep\z@\@plus\p@\@minus\p@
    \topsep\abovedisplayskip
    \leftmargin\termindent
    \labelwidth\leftmargin
    \advance\labelwidth-\labelsep
    \listparindent\parindent
    \parsep\parskip
    \ifdim\parsep<\jot
      \itemsep\jot
      \advance\itemsep-\parsep
    \else
      \itemsep\z@skip
    \fi
    \let\makelabel\@termlistmklab}}{%
  \endlist}
%    \end{macrocode}
% \end{environment}
%
% \begin{environment}{term}
%   The |term| environment is implemented as a |termlist| environment
%   with only one list item.
%    \begin{macrocode}
\newenvironment{term}{%
  \begin{termlist}\item\relax}{%
  \end{termlist}}
%    \end{macrocode}
% \end{environment}
%
% \begin{environment}{termlist*}
%   The |termlist*| environment is implemented as a |termlist|
%   environment with empty labels.
%    \begin{macrocode}
\newenvironment{termlist*}{%
  \termlist\@nmbrlistfalse\let\termlabel\relax}{%
  \endlist}
%    \end{macrocode}
% \end{environment}
%
% \begin{environment}{term*}
%   The |term*| environment is implemented as a |termlist*|
%   environment with only one list item.
%    \begin{macrocode}
\newenvironment{term*}{%
  \begin{termlist*}\item\relax}{%
  \end{termlist*}}
%    \end{macrocode}
% \end{environment}
%
% \iffalse
%</package>
% \fi
%
% \Finale
\endinput

%        (quote the arguments according to the demands of your shell)
%
% Documentation:
%    (a) If termlist.drv is present:
%           latex termlist.drv
%    (b) Without termlist.drv:
%           latex termlist.dtx; ...
%    The class ltxdoc loads the configuration file ltxdoc.cfg if
%    available.  Here you can specify further options, e.g. use A4
%    as paper format:
%       \PassOptionsToClass{a4paper}{article}
%
%    Programm calls to get the documentation:
%       pdflatex termlist.dtx
%       pdflatex termlist.dtx
%
% Installation:
%    TDS:tex/latex/termlist/termlist.sty
%    TDS:doc/latex/termlist/termlist.pdf
%    TDS:source/latex/termlist/termlist.dtx
%
%<package>\NeedsTeXFormat{LaTeX2e}
%<package>\ProvidesPackage{termlist}[2010/06/12 v1.1 Minor layout corrections]
%<*ignore>
\begingroup
  \def\x{LaTeX2e}%
\expandafter\endgroup
\ifcase 0\ifx\install y1\fi\expandafter
         \ifx\csname processbatchFile\endcsname\relax\else1\fi
         \ifx\fmtname\x\else 1\fi\relax
\else\csname fi\endcsname
%</ignore>
%<*install>
\input docstrip.tex
\keepsilent

\usedir{tex/latex/termlist}

\preamble

Typeset lists of numbered terms.
 
Copyright 2008, 2010 Bernd Rellermeyer

This file may be distributed and/or modified under the conditions
of the LaTeX Project Public License, either version 1.3b of this
license or (at your option) any later version.  The latest version
of this license is in:

http://www.latex-project.org/lppl.txt

and version 1.3b or later is part of all distributions of LaTeX
version 2005/12/01 or later.

\endpreamble

\generate{\file{termlist.sty}{\from{termlist.dtx}{package}}}

\ifToplevel{
\obeyspaces
\Msg{***********************************************************}
\Msg{*}
\Msg{* To finish the installation you have to move the following}
\Msg{* file into a directory searched by TeX:}
\Msg{*}
\Msg{* \space\space termlist.sty}
\Msg{*}
\Msg{* To produce the documentation run the file `termlist.dtx'}
\Msg{* through LaTeX2e.}
\Msg{*}
\Msg{* Happy TeXing}
\Msg{***********************************************************}
}

\endbatchfile
%</install>
%<*ignore>
\fi
%</ignore>
%<*driver>
\documentclass{ltxdoc}
\usepackage{termlist}
\usepackage{sverb}
\begin{document}
\DocInput{termlist.dtx}
\end{document}
%</driver>
% \fi
%
% \CheckSum{73}
% \CharacterTable
%  {Upper-case    \A\B\C\D\E\F\G\H\I\J\K\L\M\N\O\P\Q\R\S\T\U\V\W\X\Y\Z
%   Lower-case    \a\b\c\d\e\f\g\h\i\j\k\l\m\n\o\p\q\r\s\t\u\v\w\x\y\z
%   Digits        \0\1\2\3\4\5\6\7\8\9
%   Exclamation   \!     Double quote  \"     Hash (number) \#
%   Dollar        \$     Percent       \%     Ampersand     \&
%   Acute accent  \'     Left paren    \(     Right paren   \)
%   Asterisk      \*     Plus          \+     Comma         \,
%   Minus         \-     Point         \.     Solidus       \/
%   Colon         \:     Semicolon     \;     Less than     \<
%   Equals        \=     Greater than  \>     Question mark \?
%   Commercial at \@     Left bracket  \[     Backslash     \\
%   Right bracket \]     Circumflex    \^     Underscore    \_
%   Grave accent  \`     Left brace    \{     Vertical bar  \|
%   Right brace   \}     Tilde         \~}
%
%
% \changes{v1.0}{2008/08/18}{First release}
% \changes{v1.1}{2010/06/12}{Minor layout corrections}
%
% \GetFileInfo{termlist.sty}
%
% \title{The \textsf{termlist} package}
% \author{Bernd Rellermeyer}
% \date{\fileversion, \filedate}
%
% \maketitle
%
% \begin{abstract}\noindent
%   The \textsf{termlist} package provides environments to typeset
%   lists of numbered terms, where the ``terms'' may contain anything
%   printable.  The list of terms is formatted to look consistent to a
%   list of mathematical equations inside an |eqnarray| environment.
%   Also for consistency, the same |equation| counter is used for
%   numbering.  You may think of termlist as a generalisation of the
%   |eqnarray| environment, if you want to typeset more than just
%   mathematics.
% \end{abstract}
%
% \section{Introduction}
%
% The \LaTeX\ typesetting system is widely used by mathematicians.  A
% number of constructs, like the |equation| or |eqnarray|
% environments, fit the specific needs of mathematical texts.  The
% math modes are intrinsic parts of \TeX\ and have left their marks on
% the whole system.
%
% The \textsf{termlist} package may be seen as a generalization of the
% |equation| and |eqnarray| environments for non-mathematical texts.
% It provides environments to typeset any kinds of mathematical or
% non-mathematical terms, like propositions, hypotheses, examples, or
% linguistic terms, in the same way mathematical terms are typesetted
% inside an |equation| or |eqnarray| environment.  I.\,e., the terms
% are indented and labelled with a continuous number---for consistency
% with the |equation| and |eqnarray| environments, the same |equation|
% counter is used for that---so that the look consistent to
% mathematical terms inside those environments.
%
% The \textsf{termlist} package is based on an idea of the
% \textsf{covington} package.
%
% \section{Usage}
%
% The \textsf{termlist} package provides the four new environments
% |termlist|, |term|, |termlist*|, and |term*|.
%
% \DescribeEnv{termlist} The |termlist| environment is the fundamental
% one of these environments.  It is a list of terms, each indented and
% labelled with the |equation| counter on the left side by default.
% The default label can be changed by the optional argument of the
% |\item| command.
%
% For consistency with the |equation| and |eqnarray| environments, the
% global |leqno| option or something similar should be used to print
% the labels of mathematical terms inside these environments also on
% the left side.
%
% \begin{demo}{termlist}
%   \begin{termlist}
%   \item Here is an example of
%     the \verb|termlist|
%     environment.
%   \item\label{second item}
%     Each term is indented
%     and labelled with the
%     \verb|equation| counter.
%   \item[(\ref{second item}a)]
%     The label of the term
%     can be changed by the
%     optional argument of the
%     \verb|\item| command.
%   \item Using the optional
%     argument of the
%     \verb|\item| command
%     does not increment the
%     \verb|equation| counter.
%   \end{termlist}
% \end{demo}
%
% \DescribeEnv{term} The |term| environment is a |termlist|
% environment with only one list item.
%
% \begin{demo}{term}
%   \begin{term}
%     The \verb|term| environment
%     labels the term with the
%     \verb|equation| counter.
%   \end{term}
% \end{demo}
%
% \DescribeEnv{termlist*} The |termlist*| environment is the same as
% the |termlist| environment, but with empty labels.
%
% \begin{demo}{termlist*}
%   \begin{termlist*}
%     \item The \verb|termlist*|
%       environment is a list of
%       terms.
%     \item The terms do not
%       have labels.
%   \end{termlist*}
% \end{demo}
%
% \DescribeEnv{term*} The |term*| environment is the same as the
% |term| environment, but with an empty label.
%
% \begin{demo}{term*}
%   \begin{term*}
%     The \verb|term*|
%     environment does not
%     label the term.
%   \end{term*}
%   \begin{term}
%     Using the
%     \verb|termlist*|
%     or \verb|term*|
%     environment does not
%     increment the
%     \verb|equation|
%     counter.
%   \end{term}
% \end{demo}
%
% The four environments provided by this package can be customized by
% the three macros |\termlabel|, |\termlabelfont|, and |\termindent|.
%
% \DescribeMacro\termlabel The label of each term is defined by the
% macro |\termlabel|.  For consistency with the |equation| and
% |eqnarray| environments, its default value is |(\theequation)|.  Use
% one of the commands
% \begin{termlist*}
% \item |\def\termlabel{|\meta{labeldescription}|}|, or
% \item |\renewcommand{\termlabel}{|\meta{labeldescription}|}|
% \end{termlist*}
% to change its value.  Note that even if the value of |\termlabel|
% does not contain something like |\theequation|, the |equation|
% counter is incremented with each list item.
%
% \DescribeMacro\termlabelfont The label font and label color is
% defined by the macro |\termlabelfont|.  For consistency with the
% |equation| and |eqnarray| environments, its default value is
% |\normalfont\normalcolor|.  Use one of the commands
% \begin{termlist*}
% \item |\def\termlabelfont{|\meta{fontdescription}|}|, or
% \item |\renewcommand{\termlabelfont}{|\meta{fontdescription}|}|
% \end{termlist*}
% to change its value.
%
% Note that for the |equation| and |eqnarray| environments, the whole
% label, including its font and color, is defined by the macro
% |\@eqnnum|.  For consistency, |\termlabel| and |\termlabelfont|
% should always have values that match the value of |\@eqnnum|.
%
% \DescribeMacro\termindent The amount of the left indentation of each
% term is defined by the macro |\termindent|.  Its default value is
% |\leftmargini|.  This is also the default value of the macro
% |\mathindent|, which defines the amount of the left indentation of
% mathematical terms inside an |equation| or |eqnarray| environment,
% if the \textsf{fleqn} package is used.  In this case, |\termindent|
% and |\mathindent| should always have the same values.  Use on of the
% commands
% \begin{termlist*}
% \item |\termindent|\meta{dimension}, or
% \item |\setlength{\termindent}{|\meta{dimension}|}|
% \end{termlist*}
% to change the value of |\termindent|.
%
% \StopEventually{}
%
% \iffalse
%<*package>
% \fi
%
% \section{Implementation}
%
% \begin{macro}{\termlabel}
%   The |termlabel| macro defines the label of each term inside the
%   |termlist| and |term| environments.
%    \begin{macrocode}
\def\termlabel{(\theequation)}
%    \end{macrocode}
% \end{macro}
%
% \begin{macro}{\termlabelfont}
% \begin{macro}{\@termlistmklab}
%   The |termlabelfont| macro defines the font and color of the label
%   of each term inside the |termlist| and |term| environments.  The
%   macro |\@termlistmklab| is used internally in the definition of
%   the |termlist| environment.
%    \begin{macrocode}
\def\termlabelfont{\normalfont\normalcolor}
\def\@termlistmklab#1{\termlabelfont #1\hfil}
%    \end{macrocode}
% \end{macro}
% \end{macro}
%
% \begin{macro}{\termindent}
%   The |termindent| macro defines the amount of the left indentation
%   of each term inside one of the four environments provided by this
%   package.
%    \begin{macrocode}
\newdimen\termindent
\termindent\leftmargini
%    \end{macrocode}
% \end{macro}
%
% \begin{environment}{termlist}
%   The |termlist| environment is implemented as a list.  Each list
%   item has a default label defined by |\termlabel| and
%   |\@termlistmklab| and left and right margins defined by
%   |\termindent| and |\leftmargini|, resp.  The indentation
%   |\listparindent| and the amount of space between two paragraphs
%   |\parsep| inside the list are set to their corresponding values
%   outside the list, |\parindent| and |\parskip|, resp.  The total
%   amount of space between two list items is the sum of |\parsep| and
%   |\itemsep|.  The extra amount of space |\itemsep| is set to such a
%   value, that the total amount of space between two list items is
%   the maximum of |\parsep| and the extra amount of space |\jot|
%   between two lines inside an |eqnarray| environment.  To be further
%   consistent to the |equation| and |eqnarray| environments, the
%   amount of space above and below the list |\topsep| is set to
%   |\abovedisplayskip|.
%    \begin{macrocode}
\newenvironment{termlist}{%
  \list{\termlabel}{%
    \@nmbrlisttrue
    \def\@listctr{equation}%
    \partopsep\z@\@plus\p@\@minus\p@
    \topsep\abovedisplayskip
    \leftmargin\termindent
    \labelwidth\leftmargin
    \advance\labelwidth-\labelsep
    \listparindent\parindent
    \parsep\parskip
    \ifdim\parsep<\jot
      \itemsep\jot
      \advance\itemsep-\parsep
    \else
      \itemsep\z@skip
    \fi
    \let\makelabel\@termlistmklab}}{%
  \endlist}
%    \end{macrocode}
% \end{environment}
%
% \begin{environment}{term}
%   The |term| environment is implemented as a |termlist| environment
%   with only one list item.
%    \begin{macrocode}
\newenvironment{term}{%
  \begin{termlist}\item\relax}{%
  \end{termlist}}
%    \end{macrocode}
% \end{environment}
%
% \begin{environment}{termlist*}
%   The |termlist*| environment is implemented as a |termlist|
%   environment with empty labels.
%    \begin{macrocode}
\newenvironment{termlist*}{%
  \termlist\@nmbrlistfalse\let\termlabel\relax}{%
  \endlist}
%    \end{macrocode}
% \end{environment}
%
% \begin{environment}{term*}
%   The |term*| environment is implemented as a |termlist*|
%   environment with only one list item.
%    \begin{macrocode}
\newenvironment{term*}{%
  \begin{termlist*}\item\relax}{%
  \end{termlist*}}
%    \end{macrocode}
% \end{environment}
%
% \iffalse
%</package>
% \fi
%
% \Finale
\endinput

%        (quote the arguments according to the demands of your shell)
%
% Documentation:
%    (a) If termlist.drv is present:
%           latex termlist.drv
%    (b) Without termlist.drv:
%           latex termlist.dtx; ...
%    The class ltxdoc loads the configuration file ltxdoc.cfg if
%    available.  Here you can specify further options, e.g. use A4
%    as paper format:
%       \PassOptionsToClass{a4paper}{article}
%
%    Programm calls to get the documentation:
%       pdflatex termlist.dtx
%       pdflatex termlist.dtx
%
% Installation:
%    TDS:tex/latex/termlist/termlist.sty
%    TDS:doc/latex/termlist/termlist.pdf
%    TDS:source/latex/termlist/termlist.dtx
%
%<package>\NeedsTeXFormat{LaTeX2e}
%<package>\ProvidesPackage{termlist}[2010/06/12 v1.1 Minor layout corrections]
%<*ignore>
\begingroup
  \def\x{LaTeX2e}%
\expandafter\endgroup
\ifcase 0\ifx\install y1\fi\expandafter
         \ifx\csname processbatchFile\endcsname\relax\else1\fi
         \ifx\fmtname\x\else 1\fi\relax
\else\csname fi\endcsname
%</ignore>
%<*install>
\input docstrip.tex
\keepsilent

\usedir{tex/latex/termlist}

\preamble

Typeset lists of numbered terms.
 
Copyright 2008, 2010 Bernd Rellermeyer

This file may be distributed and/or modified under the conditions
of the LaTeX Project Public License, either version 1.3b of this
license or (at your option) any later version.  The latest version
of this license is in:

http://www.latex-project.org/lppl.txt

and version 1.3b or later is part of all distributions of LaTeX
version 2005/12/01 or later.

\endpreamble

\generate{\file{termlist.sty}{\from{termlist.dtx}{package}}}

\ifToplevel{
\obeyspaces
\Msg{***********************************************************}
\Msg{*}
\Msg{* To finish the installation you have to move the following}
\Msg{* file into a directory searched by TeX:}
\Msg{*}
\Msg{* \space\space termlist.sty}
\Msg{*}
\Msg{* To produce the documentation run the file `termlist.dtx'}
\Msg{* through LaTeX2e.}
\Msg{*}
\Msg{* Happy TeXing}
\Msg{***********************************************************}
}

\endbatchfile
%</install>
%<*ignore>
\fi
%</ignore>
%<*driver>
\documentclass{ltxdoc}
\usepackage{termlist}
\usepackage{sverb}
\begin{document}
\DocInput{termlist.dtx}
\end{document}
%</driver>
% \fi
%
% \CheckSum{73}
% \CharacterTable
%  {Upper-case    \A\B\C\D\E\F\G\H\I\J\K\L\M\N\O\P\Q\R\S\T\U\V\W\X\Y\Z
%   Lower-case    \a\b\c\d\e\f\g\h\i\j\k\l\m\n\o\p\q\r\s\t\u\v\w\x\y\z
%   Digits        \0\1\2\3\4\5\6\7\8\9
%   Exclamation   \!     Double quote  \"     Hash (number) \#
%   Dollar        \$     Percent       \%     Ampersand     \&
%   Acute accent  \'     Left paren    \(     Right paren   \)
%   Asterisk      \*     Plus          \+     Comma         \,
%   Minus         \-     Point         \.     Solidus       \/
%   Colon         \:     Semicolon     \;     Less than     \<
%   Equals        \=     Greater than  \>     Question mark \?
%   Commercial at \@     Left bracket  \[     Backslash     \\
%   Right bracket \]     Circumflex    \^     Underscore    \_
%   Grave accent  \`     Left brace    \{     Vertical bar  \|
%   Right brace   \}     Tilde         \~}
%
%
% \changes{v1.0}{2008/08/18}{First release}
% \changes{v1.1}{2010/06/12}{Minor layout corrections}
%
% \GetFileInfo{termlist.sty}
%
% \title{The \textsf{termlist} package}
% \author{Bernd Rellermeyer}
% \date{\fileversion, \filedate}
%
% \maketitle
%
% \begin{abstract}\noindent
%   The \textsf{termlist} package provides environments to typeset
%   lists of numbered terms, where the ``terms'' may contain anything
%   printable.  The list of terms is formatted to look consistent to a
%   list of mathematical equations inside an |eqnarray| environment.
%   Also for consistency, the same |equation| counter is used for
%   numbering.  You may think of termlist as a generalisation of the
%   |eqnarray| environment, if you want to typeset more than just
%   mathematics.
% \end{abstract}
%
% \section{Introduction}
%
% The \LaTeX\ typesetting system is widely used by mathematicians.  A
% number of constructs, like the |equation| or |eqnarray|
% environments, fit the specific needs of mathematical texts.  The
% math modes are intrinsic parts of \TeX\ and have left their marks on
% the whole system.
%
% The \textsf{termlist} package may be seen as a generalization of the
% |equation| and |eqnarray| environments for non-mathematical texts.
% It provides environments to typeset any kinds of mathematical or
% non-mathematical terms, like propositions, hypotheses, examples, or
% linguistic terms, in the same way mathematical terms are typesetted
% inside an |equation| or |eqnarray| environment.  I.\,e., the terms
% are indented and labelled with a continuous number---for consistency
% with the |equation| and |eqnarray| environments, the same |equation|
% counter is used for that---so that the look consistent to
% mathematical terms inside those environments.
%
% The \textsf{termlist} package is based on an idea of the
% \textsf{covington} package.
%
% \section{Usage}
%
% The \textsf{termlist} package provides the four new environments
% |termlist|, |term|, |termlist*|, and |term*|.
%
% \DescribeEnv{termlist} The |termlist| environment is the fundamental
% one of these environments.  It is a list of terms, each indented and
% labelled with the |equation| counter on the left side by default.
% The default label can be changed by the optional argument of the
% |\item| command.
%
% For consistency with the |equation| and |eqnarray| environments, the
% global |leqno| option or something similar should be used to print
% the labels of mathematical terms inside these environments also on
% the left side.
%
% \begin{demo}{termlist}
%   \begin{termlist}
%   \item Here is an example of
%     the \verb|termlist|
%     environment.
%   \item\label{second item}
%     Each term is indented
%     and labelled with the
%     \verb|equation| counter.
%   \item[(\ref{second item}a)]
%     The label of the term
%     can be changed by the
%     optional argument of the
%     \verb|\item| command.
%   \item Using the optional
%     argument of the
%     \verb|\item| command
%     does not increment the
%     \verb|equation| counter.
%   \end{termlist}
% \end{demo}
%
% \DescribeEnv{term} The |term| environment is a |termlist|
% environment with only one list item.
%
% \begin{demo}{term}
%   \begin{term}
%     The \verb|term| environment
%     labels the term with the
%     \verb|equation| counter.
%   \end{term}
% \end{demo}
%
% \DescribeEnv{termlist*} The |termlist*| environment is the same as
% the |termlist| environment, but with empty labels.
%
% \begin{demo}{termlist*}
%   \begin{termlist*}
%     \item The \verb|termlist*|
%       environment is a list of
%       terms.
%     \item The terms do not
%       have labels.
%   \end{termlist*}
% \end{demo}
%
% \DescribeEnv{term*} The |term*| environment is the same as the
% |term| environment, but with an empty label.
%
% \begin{demo}{term*}
%   \begin{term*}
%     The \verb|term*|
%     environment does not
%     label the term.
%   \end{term*}
%   \begin{term}
%     Using the
%     \verb|termlist*|
%     or \verb|term*|
%     environment does not
%     increment the
%     \verb|equation|
%     counter.
%   \end{term}
% \end{demo}
%
% The four environments provided by this package can be customized by
% the three macros |\termlabel|, |\termlabelfont|, and |\termindent|.
%
% \DescribeMacro\termlabel The label of each term is defined by the
% macro |\termlabel|.  For consistency with the |equation| and
% |eqnarray| environments, its default value is |(\theequation)|.  Use
% one of the commands
% \begin{termlist*}
% \item |\def\termlabel{|\meta{labeldescription}|}|, or
% \item |\renewcommand{\termlabel}{|\meta{labeldescription}|}|
% \end{termlist*}
% to change its value.  Note that even if the value of |\termlabel|
% does not contain something like |\theequation|, the |equation|
% counter is incremented with each list item.
%
% \DescribeMacro\termlabelfont The label font and label color is
% defined by the macro |\termlabelfont|.  For consistency with the
% |equation| and |eqnarray| environments, its default value is
% |\normalfont\normalcolor|.  Use one of the commands
% \begin{termlist*}
% \item |\def\termlabelfont{|\meta{fontdescription}|}|, or
% \item |\renewcommand{\termlabelfont}{|\meta{fontdescription}|}|
% \end{termlist*}
% to change its value.
%
% Note that for the |equation| and |eqnarray| environments, the whole
% label, including its font and color, is defined by the macro
% |\@eqnnum|.  For consistency, |\termlabel| and |\termlabelfont|
% should always have values that match the value of |\@eqnnum|.
%
% \DescribeMacro\termindent The amount of the left indentation of each
% term is defined by the macro |\termindent|.  Its default value is
% |\leftmargini|.  This is also the default value of the macro
% |\mathindent|, which defines the amount of the left indentation of
% mathematical terms inside an |equation| or |eqnarray| environment,
% if the \textsf{fleqn} package is used.  In this case, |\termindent|
% and |\mathindent| should always have the same values.  Use on of the
% commands
% \begin{termlist*}
% \item |\termindent|\meta{dimension}, or
% \item |\setlength{\termindent}{|\meta{dimension}|}|
% \end{termlist*}
% to change the value of |\termindent|.
%
% \StopEventually{}
%
% \iffalse
%<*package>
% \fi
%
% \section{Implementation}
%
% \begin{macro}{\termlabel}
%   The |termlabel| macro defines the label of each term inside the
%   |termlist| and |term| environments.
%    \begin{macrocode}
\def\termlabel{(\theequation)}
%    \end{macrocode}
% \end{macro}
%
% \begin{macro}{\termlabelfont}
% \begin{macro}{\@termlistmklab}
%   The |termlabelfont| macro defines the font and color of the label
%   of each term inside the |termlist| and |term| environments.  The
%   macro |\@termlistmklab| is used internally in the definition of
%   the |termlist| environment.
%    \begin{macrocode}
\def\termlabelfont{\normalfont\normalcolor}
\def\@termlistmklab#1{\termlabelfont #1\hfil}
%    \end{macrocode}
% \end{macro}
% \end{macro}
%
% \begin{macro}{\termindent}
%   The |termindent| macro defines the amount of the left indentation
%   of each term inside one of the four environments provided by this
%   package.
%    \begin{macrocode}
\newdimen\termindent
\termindent\leftmargini
%    \end{macrocode}
% \end{macro}
%
% \begin{environment}{termlist}
%   The |termlist| environment is implemented as a list.  Each list
%   item has a default label defined by |\termlabel| and
%   |\@termlistmklab| and left and right margins defined by
%   |\termindent| and |\leftmargini|, resp.  The indentation
%   |\listparindent| and the amount of space between two paragraphs
%   |\parsep| inside the list are set to their corresponding values
%   outside the list, |\parindent| and |\parskip|, resp.  The total
%   amount of space between two list items is the sum of |\parsep| and
%   |\itemsep|.  The extra amount of space |\itemsep| is set to such a
%   value, that the total amount of space between two list items is
%   the maximum of |\parsep| and the extra amount of space |\jot|
%   between two lines inside an |eqnarray| environment.  To be further
%   consistent to the |equation| and |eqnarray| environments, the
%   amount of space above and below the list |\topsep| is set to
%   |\abovedisplayskip|.
%    \begin{macrocode}
\newenvironment{termlist}{%
  \list{\termlabel}{%
    \@nmbrlisttrue
    \def\@listctr{equation}%
    \partopsep\z@\@plus\p@\@minus\p@
    \topsep\abovedisplayskip
    \leftmargin\termindent
    \labelwidth\leftmargin
    \advance\labelwidth-\labelsep
    \listparindent\parindent
    \parsep\parskip
    \ifdim\parsep<\jot
      \itemsep\jot
      \advance\itemsep-\parsep
    \else
      \itemsep\z@skip
    \fi
    \let\makelabel\@termlistmklab}}{%
  \endlist}
%    \end{macrocode}
% \end{environment}
%
% \begin{environment}{term}
%   The |term| environment is implemented as a |termlist| environment
%   with only one list item.
%    \begin{macrocode}
\newenvironment{term}{%
  \begin{termlist}\item\relax}{%
  \end{termlist}}
%    \end{macrocode}
% \end{environment}
%
% \begin{environment}{termlist*}
%   The |termlist*| environment is implemented as a |termlist|
%   environment with empty labels.
%    \begin{macrocode}
\newenvironment{termlist*}{%
  \termlist\@nmbrlistfalse\let\termlabel\relax}{%
  \endlist}
%    \end{macrocode}
% \end{environment}
%
% \begin{environment}{term*}
%   The |term*| environment is implemented as a |termlist*|
%   environment with only one list item.
%    \begin{macrocode}
\newenvironment{term*}{%
  \begin{termlist*}\item\relax}{%
  \end{termlist*}}
%    \end{macrocode}
% \end{environment}
%
% \iffalse
%</package>
% \fi
%
% \Finale
\endinput

%        (quote the arguments according to the demands of your shell)
%
% Documentation:
%    (a) If termlist.drv is present:
%           latex termlist.drv
%    (b) Without termlist.drv:
%           latex termlist.dtx; ...
%    The class ltxdoc loads the configuration file ltxdoc.cfg if
%    available.  Here you can specify further options, e.g. use A4
%    as paper format:
%       \PassOptionsToClass{a4paper}{article}
%
%    Programm calls to get the documentation:
%       pdflatex termlist.dtx
%       pdflatex termlist.dtx
%
% Installation:
%    TDS:tex/latex/termlist/termlist.sty
%    TDS:doc/latex/termlist/termlist.pdf
%    TDS:source/latex/termlist/termlist.dtx
%
%<package>\NeedsTeXFormat{LaTeX2e}
%<package>\ProvidesPackage{termlist}[2010/06/12 v1.1 Minor layout corrections]
%<*ignore>
\begingroup
  \def\x{LaTeX2e}%
\expandafter\endgroup
\ifcase 0\ifx\install y1\fi\expandafter
         \ifx\csname processbatchFile\endcsname\relax\else1\fi
         \ifx\fmtname\x\else 1\fi\relax
\else\csname fi\endcsname
%</ignore>
%<*install>
\input docstrip.tex
\keepsilent

\usedir{tex/latex/termlist}

\preamble

Typeset lists of numbered terms.
 
Copyright 2008, 2010 Bernd Rellermeyer

This file may be distributed and/or modified under the conditions
of the LaTeX Project Public License, either version 1.3b of this
license or (at your option) any later version.  The latest version
of this license is in:

http://www.latex-project.org/lppl.txt

and version 1.3b or later is part of all distributions of LaTeX
version 2005/12/01 or later.

\endpreamble

\generate{\file{termlist.sty}{\from{termlist.dtx}{package}}}

\ifToplevel{
\obeyspaces
\Msg{***********************************************************}
\Msg{*}
\Msg{* To finish the installation you have to move the following}
\Msg{* file into a directory searched by TeX:}
\Msg{*}
\Msg{* \space\space termlist.sty}
\Msg{*}
\Msg{* To produce the documentation run the file `termlist.dtx'}
\Msg{* through LaTeX2e.}
\Msg{*}
\Msg{* Happy TeXing}
\Msg{***********************************************************}
}

\endbatchfile
%</install>
%<*ignore>
\fi
%</ignore>
%<*driver>
\documentclass{ltxdoc}
\usepackage{termlist}
\usepackage{sverb}
\begin{document}
\DocInput{termlist.dtx}
\end{document}
%</driver>
% \fi
%
% \CheckSum{73}
% \CharacterTable
%  {Upper-case    \A\B\C\D\E\F\G\H\I\J\K\L\M\N\O\P\Q\R\S\T\U\V\W\X\Y\Z
%   Lower-case    \a\b\c\d\e\f\g\h\i\j\k\l\m\n\o\p\q\r\s\t\u\v\w\x\y\z
%   Digits        \0\1\2\3\4\5\6\7\8\9
%   Exclamation   \!     Double quote  \"     Hash (number) \#
%   Dollar        \$     Percent       \%     Ampersand     \&
%   Acute accent  \'     Left paren    \(     Right paren   \)
%   Asterisk      \*     Plus          \+     Comma         \,
%   Minus         \-     Point         \.     Solidus       \/
%   Colon         \:     Semicolon     \;     Less than     \<
%   Equals        \=     Greater than  \>     Question mark \?
%   Commercial at \@     Left bracket  \[     Backslash     \\
%   Right bracket \]     Circumflex    \^     Underscore    \_
%   Grave accent  \`     Left brace    \{     Vertical bar  \|
%   Right brace   \}     Tilde         \~}
%
%
% \changes{v1.0}{2008/08/18}{First release}
% \changes{v1.1}{2010/06/12}{Minor layout corrections}
%
% \GetFileInfo{termlist.sty}
%
% \title{The \textsf{termlist} package}
% \author{Bernd Rellermeyer}
% \date{\fileversion, \filedate}
%
% \maketitle
%
% \begin{abstract}\noindent
%   The \textsf{termlist} package provides environments to typeset
%   lists of numbered terms, where the ``terms'' may contain anything
%   printable.  The list of terms is formatted to look consistent to a
%   list of mathematical equations inside an |eqnarray| environment.
%   Also for consistency, the same |equation| counter is used for
%   numbering.  You may think of termlist as a generalisation of the
%   |eqnarray| environment, if you want to typeset more than just
%   mathematics.
% \end{abstract}
%
% \section{Introduction}
%
% The \LaTeX\ typesetting system is widely used by mathematicians.  A
% number of constructs, like the |equation| or |eqnarray|
% environments, fit the specific needs of mathematical texts.  The
% math modes are intrinsic parts of \TeX\ and have left their marks on
% the whole system.
%
% The \textsf{termlist} package may be seen as a generalization of the
% |equation| and |eqnarray| environments for non-mathematical texts.
% It provides environments to typeset any kinds of mathematical or
% non-mathematical terms, like propositions, hypotheses, examples, or
% linguistic terms, in the same way mathematical terms are typesetted
% inside an |equation| or |eqnarray| environment.  I.\,e., the terms
% are indented and labelled with a continuous number---for consistency
% with the |equation| and |eqnarray| environments, the same |equation|
% counter is used for that---so that the look consistent to
% mathematical terms inside those environments.
%
% The \textsf{termlist} package is based on an idea of the
% \textsf{covington} package.
%
% \section{Usage}
%
% The \textsf{termlist} package provides the four new environments
% |termlist|, |term|, |termlist*|, and |term*|.
%
% \DescribeEnv{termlist} The |termlist| environment is the fundamental
% one of these environments.  It is a list of terms, each indented and
% labelled with the |equation| counter on the left side by default.
% The default label can be changed by the optional argument of the
% |\item| command.
%
% For consistency with the |equation| and |eqnarray| environments, the
% global |leqno| option or something similar should be used to print
% the labels of mathematical terms inside these environments also on
% the left side.
%
% \begin{demo}{termlist}
%   \begin{termlist}
%   \item Here is an example of
%     the \verb|termlist|
%     environment.
%   \item\label{second item}
%     Each term is indented
%     and labelled with the
%     \verb|equation| counter.
%   \item[(\ref{second item}a)]
%     The label of the term
%     can be changed by the
%     optional argument of the
%     \verb|\item| command.
%   \item Using the optional
%     argument of the
%     \verb|\item| command
%     does not increment the
%     \verb|equation| counter.
%   \end{termlist}
% \end{demo}
%
% \DescribeEnv{term} The |term| environment is a |termlist|
% environment with only one list item.
%
% \begin{demo}{term}
%   \begin{term}
%     The \verb|term| environment
%     labels the term with the
%     \verb|equation| counter.
%   \end{term}
% \end{demo}
%
% \DescribeEnv{termlist*} The |termlist*| environment is the same as
% the |termlist| environment, but with empty labels.
%
% \begin{demo}{termlist*}
%   \begin{termlist*}
%     \item The \verb|termlist*|
%       environment is a list of
%       terms.
%     \item The terms do not
%       have labels.
%   \end{termlist*}
% \end{demo}
%
% \DescribeEnv{term*} The |term*| environment is the same as the
% |term| environment, but with an empty label.
%
% \begin{demo}{term*}
%   \begin{term*}
%     The \verb|term*|
%     environment does not
%     label the term.
%   \end{term*}
%   \begin{term}
%     Using the
%     \verb|termlist*|
%     or \verb|term*|
%     environment does not
%     increment the
%     \verb|equation|
%     counter.
%   \end{term}
% \end{demo}
%
% The four environments provided by this package can be customized by
% the three macros |\termlabel|, |\termlabelfont|, and |\termindent|.
%
% \DescribeMacro\termlabel The label of each term is defined by the
% macro |\termlabel|.  For consistency with the |equation| and
% |eqnarray| environments, its default value is |(\theequation)|.  Use
% one of the commands
% \begin{termlist*}
% \item |\def\termlabel{|\meta{labeldescription}|}|, or
% \item |\renewcommand{\termlabel}{|\meta{labeldescription}|}|
% \end{termlist*}
% to change its value.  Note that even if the value of |\termlabel|
% does not contain something like |\theequation|, the |equation|
% counter is incremented with each list item.
%
% \DescribeMacro\termlabelfont The label font and label color is
% defined by the macro |\termlabelfont|.  For consistency with the
% |equation| and |eqnarray| environments, its default value is
% |\normalfont\normalcolor|.  Use one of the commands
% \begin{termlist*}
% \item |\def\termlabelfont{|\meta{fontdescription}|}|, or
% \item |\renewcommand{\termlabelfont}{|\meta{fontdescription}|}|
% \end{termlist*}
% to change its value.
%
% Note that for the |equation| and |eqnarray| environments, the whole
% label, including its font and color, is defined by the macro
% |\@eqnnum|.  For consistency, |\termlabel| and |\termlabelfont|
% should always have values that match the value of |\@eqnnum|.
%
% \DescribeMacro\termindent The amount of the left indentation of each
% term is defined by the macro |\termindent|.  Its default value is
% |\leftmargini|.  This is also the default value of the macro
% |\mathindent|, which defines the amount of the left indentation of
% mathematical terms inside an |equation| or |eqnarray| environment,
% if the \textsf{fleqn} package is used.  In this case, |\termindent|
% and |\mathindent| should always have the same values.  Use on of the
% commands
% \begin{termlist*}
% \item |\termindent|\meta{dimension}, or
% \item |\setlength{\termindent}{|\meta{dimension}|}|
% \end{termlist*}
% to change the value of |\termindent|.
%
% \StopEventually{}
%
% \iffalse
%<*package>
% \fi
%
% \section{Implementation}
%
% \begin{macro}{\termlabel}
%   The |termlabel| macro defines the label of each term inside the
%   |termlist| and |term| environments.
%    \begin{macrocode}
\def\termlabel{(\theequation)}
%    \end{macrocode}
% \end{macro}
%
% \begin{macro}{\termlabelfont}
% \begin{macro}{\@termlistmklab}
%   The |termlabelfont| macro defines the font and color of the label
%   of each term inside the |termlist| and |term| environments.  The
%   macro |\@termlistmklab| is used internally in the definition of
%   the |termlist| environment.
%    \begin{macrocode}
\def\termlabelfont{\normalfont\normalcolor}
\def\@termlistmklab#1{\termlabelfont #1\hfil}
%    \end{macrocode}
% \end{macro}
% \end{macro}
%
% \begin{macro}{\termindent}
%   The |termindent| macro defines the amount of the left indentation
%   of each term inside one of the four environments provided by this
%   package.
%    \begin{macrocode}
\newdimen\termindent
\termindent\leftmargini
%    \end{macrocode}
% \end{macro}
%
% \begin{environment}{termlist}
%   The |termlist| environment is implemented as a list.  Each list
%   item has a default label defined by |\termlabel| and
%   |\@termlistmklab| and left and right margins defined by
%   |\termindent| and |\leftmargini|, resp.  The indentation
%   |\listparindent| and the amount of space between two paragraphs
%   |\parsep| inside the list are set to their corresponding values
%   outside the list, |\parindent| and |\parskip|, resp.  The total
%   amount of space between two list items is the sum of |\parsep| and
%   |\itemsep|.  The extra amount of space |\itemsep| is set to such a
%   value, that the total amount of space between two list items is
%   the maximum of |\parsep| and the extra amount of space |\jot|
%   between two lines inside an |eqnarray| environment.  To be further
%   consistent to the |equation| and |eqnarray| environments, the
%   amount of space above and below the list |\topsep| is set to
%   |\abovedisplayskip|.
%    \begin{macrocode}
\newenvironment{termlist}{%
  \list{\termlabel}{%
    \@nmbrlisttrue
    \def\@listctr{equation}%
    \partopsep\z@\@plus\p@\@minus\p@
    \topsep\abovedisplayskip
    \leftmargin\termindent
    \labelwidth\leftmargin
    \advance\labelwidth-\labelsep
    \listparindent\parindent
    \parsep\parskip
    \ifdim\parsep<\jot
      \itemsep\jot
      \advance\itemsep-\parsep
    \else
      \itemsep\z@skip
    \fi
    \let\makelabel\@termlistmklab}}{%
  \endlist}
%    \end{macrocode}
% \end{environment}
%
% \begin{environment}{term}
%   The |term| environment is implemented as a |termlist| environment
%   with only one list item.
%    \begin{macrocode}
\newenvironment{term}{%
  \begin{termlist}\item\relax}{%
  \end{termlist}}
%    \end{macrocode}
% \end{environment}
%
% \begin{environment}{termlist*}
%   The |termlist*| environment is implemented as a |termlist|
%   environment with empty labels.
%    \begin{macrocode}
\newenvironment{termlist*}{%
  \termlist\@nmbrlistfalse\let\termlabel\relax}{%
  \endlist}
%    \end{macrocode}
% \end{environment}
%
% \begin{environment}{term*}
%   The |term*| environment is implemented as a |termlist*|
%   environment with only one list item.
%    \begin{macrocode}
\newenvironment{term*}{%
  \begin{termlist*}\item\relax}{%
  \end{termlist*}}
%    \end{macrocode}
% \end{environment}
%
% \iffalse
%</package>
% \fi
%
% \Finale
\endinput
