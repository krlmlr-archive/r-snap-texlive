% \iffalse meta-comment 
% 
% Copyright (C) 2013 by Ben Vitecek
% 
% This file may be distributed and/or modified under the 
% conditions of the LaTeX Project Public License, either 
% version 1.3 of this license or (at your option) any later 
% version. The latest version of this license is in: 
% 
%	http://www.latex-project.org/lppl.txt 
% 
% and version 1.3 or later is part of all distributions of 
% LaTeX version 2008-05-04 or later. 
% 
% \fi
%
% \iffalse 
%<*driver> 
\documentclass{ltxdoc} 
\usepackage{iitem} 
\usepackage[ngerman,english]{babel}
\usepackage[T1]{fontenc}
\usepackage[applemac]{inputenc}
\usepackage{paralist, xcolor, listings,ragged2e} 
\usepackage{lmodern} 
\usepackage{microtype}
\usepackage[colorlinks=true, linkcolor=blue!70!black]{hyperref} 

\lstset{%
basicstyle=\small\sffamily,
language=[LaTeX]TeX,   
breaklines=true,
frame=single,
%framexleftmargin=8mm, 
framexrightmargin=0mm, 
%xleftmargin=0mm,
prebreak = \raisebox{0ex}[0ex][0ex]{\ensuremath{\hookleftarrow}},
backgroundcolor=\color{green!5},
texcsstyle=*\color{blue},
rulecolor=\color{red},
morekeywords={\iitem, \iiitem, \ivtem},
keywordstyle=\color[rgb]{0,0,1},                    % keywords
commentstyle=\color[rgb]{1,0,0},    % comments
stringstyle=\color[rgb]{0.627,0.926,0.941},  % strings
columns=fullflexible ,
literate={\{ }{\textcolor{green!70!black}{\{}}1
{\}}{\textcolor{green!70!black}{\}}}1
{\%}{\textcolor{green!5}{}}1     
{]}{\textcolor{black}{]}}1     
{\ }{\textcolor{blue}{\ }}1     
}%


\EnableCrossrefs 
\CodelineIndex 
\RecordChanges 

\makeatletter 
\g@addto@macro\@marginparreset{\Huge\RaggedLeft\vspace{-0.5ex}} \makeatother


\begin{document}
\DocInput{iitem.dtx} 
\end{document} 
%</driver> 
% \fi
%
%
% \CharacterTable
%  {Upper-case    \A\B\C\D\E\F\G\H\I\J\K\L\M\N\O\P\Q\R\S\T\U\V\W\X\Y\Z
%   Lower-case    \a\b\c\d\e\f\g\h\i\j\k\l\m\n\o\p\q\r\s\t\u\v\w\x\y\z
%   Digits        \0\1\2\3\4\5\6\7\8\9
%   Exclamation   \!     Double quote  \"     Hash (number) \#
%   Dollar        \$     Percent       \%     Ampersand     \&
%   Acute accent  \'     Left paren    \(     Right paren   \)
%   Asterisk      \*     Plus          \+     Comma         \,
%   Minus         \-     Point         \.     Solidus       \/
%   Colon         \:     Semicolon     \;     Less than     \<
%   Equals        \=     Greater than  \>     Question mark \?
%   Commercial at \@     Left bracket  \[     Backslash     \\
%   Right bracket \]     Circumflex    \^     Underscore    \_
%   Grave accent  \`     Left brace    \{     Vertical bar  \|
%   Right brace   \}     Tilde         \~}
%
%
%
%\changes {v1.0} {2013/01/19} {Initial version}
%
%
%
% \GetFileInfo{iitem.sty}
%
%\DoNotIndex {\begin{tikzpicture},\end{tikzpicture}}
%\DoNotIndex {\verb, \DeclareRobustCommandx}
%\DoNotIndex {=\verb}
%\DoNotIndex {\verb||}
%\DoNotIndex {\draw,\fill,\DeclareRobustCommand, \colorlet,\shade}
%\DoNotIndex {\xspace,\begin,\end}
%
%
%
%
%\title{Package \textsf{iitem}\thanks{This document corresponds to \textsf{iitem}~\fileversion, dated~\filedate.}}
%\author{Ben Vitecek \\ \texttt{b.vitecek@gmx.at}} 
%\maketitle
%
%
% \begin{abstract}
% It was always a bit annoying to write \lstinline|\begin{itemize}| inside \lstinline|\begin{itemize}| to go a step deeper. Using this package you can just write \lstinline|\iitem|, \lstinline|\iiitem| and \lstinline|\ivtem|. Of course you can also use \verb|description|, \verb|enumerate|, etc. {\color{red}But\marginpar{!!!} they are not usable in \verb|\footnote{}|, \verb|\marginpar{}| (I think they are named ``fragile'') etc. commands!}
% \end{abstract}
%
% \tableofcontents
%
% \section{Interesting Part}
% \lstset{deletekeywords={iitem}}
% First of all, include this package via: 
%\begin{lstlisting}
%\usepackage{iitem}\end{lstlisting}
%
% \lstset{morekeywords={iitem}}
%
% This package offers three new commands: \lstinline|\iitem|, \lstinline|\iiitem| and \lstinline|\ivtem|. \bigskip
%
%\noindent Instead of writing ~
%\begin{minipage}{0.24\textwidth}
% \begin{lstlisting}
%\begin{itemize}
% \item 1 
%  \begin{itemize} 
%  \item 2 
%  \end{itemize}
%\end{itemize}\end{lstlisting}
%\end{minipage}\quad
%you can just write ~
%\begin{minipage}{0.22\textwidth}
% \begin{lstlisting}
%\begin{itemize}
%\item 1 
%\iitem 2 
%\end{itemize}\end{lstlisting}
%\end{minipage}
%
% But there \emph{has to be} an \verb|\item| at the beginning of the list (it's the default behavior of  lists). And they \emph{don't} work in \verb|\footnote{}| etc.
%\bigskip
%
%Of course it also works with \verb|enumerate|, \verb|description|, and several others (e.g. compactitem, \dots{} from \verb|paralist|). \bigskip 
%
%\begin{minipage}{0.49\textwidth}
%\begin{lstlisting} 
%\begin{enumerate}
% \item First
% \iitem Second
% \iitem Second two  
% \iiitem Third
% \iiitem Third two
% \iitem Second three
% \iiitem Third
% \ivtem Fourth
% \ivtem Fourth four
% \iitem Second four
% \ivtem Fourth
% \end{enumerate}\end{lstlisting}

%\end{minipage}\hfill%
%\begin{minipage}{0.49\textwidth}
% \begin{enumerate}
% \item First
% \iitem Second
% \iitem Second two  
% \iiitem Third
% \iiitem Third two
% \iitem Second three
% \iiitem Third
% \ivtem Fourth
% \ivtem Fourth two
%\iitem Second four
% \ivtem Fourth
% \end{enumerate}
%\end{minipage}\bigskip\bigskip 
%
%

%
% \section{What doesn't work}
% I am bad at explaining so here is an example:
%
%\begin{minipage}{0.49\textwidth}
%\begin{lstlisting} 
%\begin{enumerate}
% \item Using the ``normal'' item, 
% you can make paragraphs and the text 
%is still caught by the item.
% \iitem Using the new commands, this doesn't work anymore.
%If you make a paragraph, the text isn't caught anymore.
%\iiitem So it's the best, if you write a long sentence after an iitem, iiitem and ivtem  that you write the sentence in a long line.
% \end{enumerate}\end{lstlisting}

%\end{minipage}\hfill%
%\begin{minipage}{0.49\textwidth}
%\begin{enumerate}
% \item Using the ``normal'' item, 
% you can make paragraphs and the text 
%is still caught by the item.
% \iitem Using the new commands, this doesn't work anymore.
%If you make a paragraph, the text isn't caught anymore.
%\iiitem So it's the best, if you write a long sentence after an iitem, iiitem and ivtem  that you write the sentence in a long line.
%\end{enumerate}
%\end{minipage}\bigskip 

%
% \section{Any ideas?}
% Like I said before you cannot use them in \verb|\footnote{}| etc. 
% If anyone has the knowledge 	of how this  behavior can be changed, he or she might write an email to me (I don't have the knowledge to do it).
%


% \section{Code (in german)}
% In der ersten Zeile steht immer dasselbe: Was brauche ich, wie hei�t das Paket.
%    \begin{macrocode} 
\NeedsTeXFormat{LaTeX2e}[2011/06/27] 
\ProvidesPackage{iitem} 
[2013/03/31 v1.0 Go a step deeper in lists 
using the new commands ``iitem'', ``iiitem'' and ``ivtem''.] 
%    \end{macrocode} 

%\begin{macro}{\Pseudo@item}
% Unsere Befehle werden dadurch definiert, dass Umgebungen um ein \verb|\item| in der Mitte gebaut werden.
% Nun muss aber in jeder Liste ein \verb|\item| vorhanden sein.
% Um dieses Problem zu l�sen, definiere ich dieses \verb|\Pseudo@item|.
% Es ist eigentlich nichts anderes, als ein unsichtbares \verb|\item|.
%    \begin{macrocode} 
\def\Pseudo@item{{{%
  \@inmatherr\item
  \@ifnextchar [\@item{\@item[]}
 }}}
%    \end{macrocode} 
%\end{macro}

%\begin{macro}{iitemcounter}
%\begin{macro}{iiitemcounter}
%\begin{macro}{ivtemcounter}
% Damit die Nummerierung bei \verb|enumerate| richtig funktioniert, brauchen wir diese drei Z�hler.
%    \begin{macrocode} 
\newcounter{iitemcounter}
\newcounter{iiitemcounter}
\newcounter{ivtemcounter}
%    \end{macrocode} 
%\end{macro}
%\end{macro}
%\end{macro}

%\begin{macro}{\item}
% Ich hoffe, ich mach keinen Bl�dsinn \dots{}
% Zuerst sage ich ihm, dass \verb|\AltesItem| \verb|\item| ist.
% Dann definiere ich \verb|\item| um: Wir brauchen diese Konstruktion, damit die Nummerierung  funktioniert.
% Die Z�hler werden auf Null gesetzt, wenn ein \verb|\item| auf der obersten Ebene geschrieben wird. 
%    \begin{macrocode} 
\let\AltesItem\item
\renewcommand{\item}{%
\if@nmbrlist%
 \ifnum%
  \@enumdepth=1%
  \setcounter{iitemcounter}{0}%
  \setcounter{iiitemcounter}{0}%
  \setcounter{ivtemcounter}{0}%
 \fi%
\fi%
\AltesItem}
%    \end{macrocode} 
%\end{macro}


% \begin{macro}{\iitem@line@command}
% Eigentlich habe ich ja keine Ahnung von \LaTeX{} Code und dem Programmieren damit.
%Aber zum Gl�ck fand ich folgenden Code auf dieser Internetseite: \href{http://tex.stackexchange.com/questions/10426/def-taking-rest-of-the-line-as-argument}{http://tex.stackexchange.com/questions/10426/def-taking-rest-of-the-line-as-argument}.
% Ich habe den Code genommen, den Namen ge�ndert und hier ist er:
%    \begin{macrocode} 
\newcommand*{\iitem@line@command}[2]{%
  \newcommand*{#1}{%
    \begingroup%
    \escapechar=`\\%
    \catcode\endlinechar=\active%
    \csname\string#1\endcsname%
  }%
  \begingroup%
  \escapechar=`\\%
  \lccode`\~=\endlinechar%
  \lowercase{%
    \expandafter\endgroup
    \expandafter\def\csname\string#1\endcsname##1~%
  }{\endgroup#2\space}%
}
%    \end{macrocode} 
%\end{macro}

% \begin{macro}{\iitem}
% Wir definieren \verb|\iitem|. 
% Als erstes setzen wir \verb|\partopsep0pt| damit die Abst�nde nicht zu gro� sind.
% Danach kommt die Umgebung: ich benutze das Makro \verb|\@currenvir|, dieses hat die Umgebung, die gerade aktiv, ist gespeichert (z.B.~itemize, enumerate, description, etc.).
% Wenn ich einfach \verb|\begin{\@currenvir}| hinschreiben w�rde, w�rde \LaTeX{} zuerst \verb|\begin{| ausf�hren, dann auf \verb|\@currenvir| sto�en und versuchen, dieses als Umgebung zu aktivieren.
% Da dies aber nicht funktioniert, brauchen wir \verb|\exapndafter|: Mit dem Ersten, springen wir zum Zweiten, und mit diesem, sagen wir, dass \LaTeX{} \verb|\@currenvir| aktivieren soll, sprich \verb|\@currenvir| wird zu der gerade aktiven Umgebung.
% Dann kommt \verb|\begin{|. \LaTeX{} liest also z.B.~\verb|\begin{itemize}| statt \verb|\begin{\@currenvir}|
%
% Noch da? Als n�chstes pr�fen wir, ob es eine Nummernliste ist, wenn ja, sagen wir das der merkw�rdige \verb|\@listctr| Befehl auf \verb|\theiitemcounter| gesetzt wird, damit kommt die richtige Nummerierung zustande (hoffentlich).
% Danach schreiben wir \verb|\item| hin, schlie�en die Umgebung und erh�hen \verb|iitemcounter| um eins, und setzen \verb|iiitemcounter| auf 0.
%    \begin{macrocode}  
\iitem@line@command{\iitem}{\partopsep0pt%
 \expandafter\begin\expandafter{\@currenvir}%
  \if@nmbrlist\setcounter{\@listctr}{\theiitemcounter}\fi%
  \item #1%
 \expandafter\end\expandafter{\@currenvir}%
\stepcounter{iitemcounter}%
\setcounter{iiitemcounter}{0}\setcounter{ivtemcounter}{0}}
%    \end{macrocode} 
%\end{macro}


% \begin{macro}{\iiitem}
% Eigentlich dasselbe wie vorher.
% Nur kommt hier \verb|\Pseudo@item| ins Spiel.
% Wenn wir diesen Befehl nicht einf�gen, meldet \LaTeX{} einen Fehler, da ein \verb|\item| fehlt (missing item). Der Rest ist mehr oder weniger derselbe.
%    \begin{macrocode} 
\iitem@line@command{\iiitem}{\partopsep0pt%
 \expandafter\begin\expandafter{\@currenvir}\Pseudo@item%
  \expandafter\begin\expandafter{\@currenvir}%
   \if@nmbrlist\setcounter{\@listctr}{\theiiitemcounter}\fi%
   \item #1 
  \expandafter\end\expandafter{\@currenvir}%
 \expandafter\end\expandafter{\@currenvir}%
\stepcounter{iiitemcounter}\setcounter{ivtemcounter}{0}}
%    \end{macrocode} 
%\end{macro}

% \begin{macro}{\ivtem}
% Selbe wie vorher.
%    \begin{macrocode} 
\iitem@line@command{\ivtem}{\partopsep0pt%
 \expandafter\begin\expandafter{\@currenvir}\Pseudo@item%
  \expandafter\begin\expandafter{\@currenvir}\Pseudo@item%
   \expandafter\begin\expandafter{\@currenvir}%
    \if@nmbrlist  \setcounter{\@listctr}{\theivtemcounter}\fi%
    \item #1%
   \expandafter\end\expandafter{\@currenvir}%
  \expandafter\end\expandafter{\@currenvir}%
 \expandafter\end\expandafter{\@currenvir}%
\stepcounter{ivtemcounter}}
%    \end{macrocode} 
%\end{macro}

% Well that's it. Happy \TeX ing. 




%\PrintChanges \PrintIndex