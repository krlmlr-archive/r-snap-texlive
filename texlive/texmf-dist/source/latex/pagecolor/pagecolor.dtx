% \iffalse meta-comment
%
% File: pagecolor.dtx
% Version: 2012/02/23 v1.0e
%
% Copyright (C) 2011 - 2012 by
%    H.-Martin M"unch <Martin dot Muench at Uni-Bonn dot de>
%
% This work may be distributed and/or modified under the
% conditions of the LaTeX Project Public License, either
% version 1.3c of this license or (at your option) any later
% version. This version of this license is in
%    http://www.latex-project.org/lppl/lppl-1-3c.txt
% and the latest version of this license is in
%    http://www.latex-project.org/lppl.txt
% and version 1.3c or later is part of all distributions of
% LaTeX version 2005/12/01 or later.
%
% This work has the LPPL maintenance status "maintained".
%
% The Current Maintainer of this work is H.-Martin Muench.
%
% This work consists of the main source file pagecolor.dtx,
% the README, and the derived files
%    pagecolor.sty, pagecolor.pdf,
%    pagecolor.ins, pagecolor.drv,
%    pagecolor-example.tex, pagecolor-example.pdf.
%
% Distribution:
%    CTAN:macros/latex/contrib/pagecolor/pagecolor.dtx
%    CTAN:macros/latex/contrib/pagecolor/pagecolor.pdf
%    CTAN:install/macros/latex/contrib/pagecolor.tds.zip
%
% Cf. http://ctan.org/pkg/pagecolor and
% http://www.tex.ac.uk/tex-archive/help/Catalogue/entries/pagecolor.html
%
% Unpacking:
%    (a) If pagecolor.ins is present:
%           tex pagecolor.ins
%    (b) Without pagecolor.ins:
%           tex pagecolor.dtx
%    (c) If you insist on using LaTeX
%           latex \let\install=y% \iffalse meta-comment
%
% File: pagecolor.dtx
% Version: 2012/02/23 v1.0e
%
% Copyright (C) 2011 - 2012 by
%    H.-Martin M"unch <Martin dot Muench at Uni-Bonn dot de>
%
% This work may be distributed and/or modified under the
% conditions of the LaTeX Project Public License, either
% version 1.3c of this license or (at your option) any later
% version. This version of this license is in
%    http://www.latex-project.org/lppl/lppl-1-3c.txt
% and the latest version of this license is in
%    http://www.latex-project.org/lppl.txt
% and version 1.3c or later is part of all distributions of
% LaTeX version 2005/12/01 or later.
%
% This work has the LPPL maintenance status "maintained".
%
% The Current Maintainer of this work is H.-Martin Muench.
%
% This work consists of the main source file pagecolor.dtx,
% the README, and the derived files
%    pagecolor.sty, pagecolor.pdf,
%    pagecolor.ins, pagecolor.drv,
%    pagecolor-example.tex, pagecolor-example.pdf.
%
% Distribution:
%    CTAN:macros/latex/contrib/pagecolor/pagecolor.dtx
%    CTAN:macros/latex/contrib/pagecolor/pagecolor.pdf
%    CTAN:install/macros/latex/contrib/pagecolor.tds.zip
%
% Cf. http://ctan.org/pkg/pagecolor and
% http://www.tex.ac.uk/tex-archive/help/Catalogue/entries/pagecolor.html
%
% Unpacking:
%    (a) If pagecolor.ins is present:
%           tex pagecolor.ins
%    (b) Without pagecolor.ins:
%           tex pagecolor.dtx
%    (c) If you insist on using LaTeX
%           latex \let\install=y% \iffalse meta-comment
%
% File: pagecolor.dtx
% Version: 2012/02/23 v1.0e
%
% Copyright (C) 2011 - 2012 by
%    H.-Martin M"unch <Martin dot Muench at Uni-Bonn dot de>
%
% This work may be distributed and/or modified under the
% conditions of the LaTeX Project Public License, either
% version 1.3c of this license or (at your option) any later
% version. This version of this license is in
%    http://www.latex-project.org/lppl/lppl-1-3c.txt
% and the latest version of this license is in
%    http://www.latex-project.org/lppl.txt
% and version 1.3c or later is part of all distributions of
% LaTeX version 2005/12/01 or later.
%
% This work has the LPPL maintenance status "maintained".
%
% The Current Maintainer of this work is H.-Martin Muench.
%
% This work consists of the main source file pagecolor.dtx,
% the README, and the derived files
%    pagecolor.sty, pagecolor.pdf,
%    pagecolor.ins, pagecolor.drv,
%    pagecolor-example.tex, pagecolor-example.pdf.
%
% Distribution:
%    CTAN:macros/latex/contrib/pagecolor/pagecolor.dtx
%    CTAN:macros/latex/contrib/pagecolor/pagecolor.pdf
%    CTAN:install/macros/latex/contrib/pagecolor.tds.zip
%
% Cf. http://ctan.org/pkg/pagecolor and
% http://www.tex.ac.uk/tex-archive/help/Catalogue/entries/pagecolor.html
%
% Unpacking:
%    (a) If pagecolor.ins is present:
%           tex pagecolor.ins
%    (b) Without pagecolor.ins:
%           tex pagecolor.dtx
%    (c) If you insist on using LaTeX
%           latex \let\install=y% \iffalse meta-comment
%
% File: pagecolor.dtx
% Version: 2012/02/23 v1.0e
%
% Copyright (C) 2011 - 2012 by
%    H.-Martin M"unch <Martin dot Muench at Uni-Bonn dot de>
%
% This work may be distributed and/or modified under the
% conditions of the LaTeX Project Public License, either
% version 1.3c of this license or (at your option) any later
% version. This version of this license is in
%    http://www.latex-project.org/lppl/lppl-1-3c.txt
% and the latest version of this license is in
%    http://www.latex-project.org/lppl.txt
% and version 1.3c or later is part of all distributions of
% LaTeX version 2005/12/01 or later.
%
% This work has the LPPL maintenance status "maintained".
%
% The Current Maintainer of this work is H.-Martin Muench.
%
% This work consists of the main source file pagecolor.dtx,
% the README, and the derived files
%    pagecolor.sty, pagecolor.pdf,
%    pagecolor.ins, pagecolor.drv,
%    pagecolor-example.tex, pagecolor-example.pdf.
%
% Distribution:
%    CTAN:macros/latex/contrib/pagecolor/pagecolor.dtx
%    CTAN:macros/latex/contrib/pagecolor/pagecolor.pdf
%    CTAN:install/macros/latex/contrib/pagecolor.tds.zip
%
% Cf. http://ctan.org/pkg/pagecolor and
% http://www.tex.ac.uk/tex-archive/help/Catalogue/entries/pagecolor.html
%
% Unpacking:
%    (a) If pagecolor.ins is present:
%           tex pagecolor.ins
%    (b) Without pagecolor.ins:
%           tex pagecolor.dtx
%    (c) If you insist on using LaTeX
%           latex \let\install=y\input{pagecolor.dtx}
%        (quote the arguments according to the demands of your shell)
%
% Documentation:
%    (a) If pagecolor.drv is present:
%           (pdf)latex pagecolor.drv
%           makeindex -s gind.ist pagecolor.idx
%           (pdf)latex pagecolor.drv
%           makeindex -s gind.ist pagecolor.idx
%           (pdf)latex pagecolor.drv
%    (b) Without pagecolor.drv:
%           (pdf)latex pagecolor.dtx
%           makeindex -s gind.ist pagecolor.idx
%           (pdf)latex pagecolor.dtx
%           makeindex -s gind.ist pagecolor.idx
%           (pdf)latex pagecolor.dtx
%
%    The class ltxdoc loads the configuration file ltxdoc.cfg
%    if available. Here you can specify further options, e.g.
%    use DIN A4 as paper format:
%       \PassOptionsToClass{a4paper}{article}
%
% Installation:
%    TDS:tex/latex/pagecolor/pagecolor.sty
%    TDS:doc/latex/pagecolor/pagecolor.pdf
%    TDS:doc/latex/pagecolor/pagecolor-example.tex
%    TDS:doc/latex/pagecolor/pagecolor-example.pdf
%    TDS:source/latex/pagecolor/pagecolor.dtx
%
%<*ignore>
\begingroup
  \catcode123=1 %
  \catcode125=2 %
  \def\x{LaTeX2e}%
\expandafter\endgroup
\ifcase 0\ifx\install y1\fi\expandafter
         \ifx\csname processbatchFile\endcsname\relax\else1\fi
         \ifx\fmtname\x\else 1\fi\relax
\else\csname fi\endcsname
%</ignore>
%<*install>
\input docstrip.tex
\Msg{*****************************************************************}
\Msg{* Installation                                                   }
\Msg{* Package: pagecolor 2012/02/23 v1.0e Provides thepagecolor (HMM)}
\Msg{*****************************************************************}

\keepsilent
\askforoverwritefalse

\let\MetaPrefix\relax
\preamble

This is a generated file.

Project: pagecolor
Version: 2012/02/23 v1.0e

Copyright (C) 2011 - 2012 by
    H.-Martin M"unch <Martin dot Muench at Uni-Bonn dot de>

The usual disclaimer applies:
If it doesn't work right that's your problem.
(Nevertheless, send an e-mail to the maintainer
 when you find an error in this package.)

This work may be distributed and/or modified under the
conditions of the LaTeX Project Public License, either
version 1.3c of this license or (at your option) any later
version. This version of this license is in
   http://www.latex-project.org/lppl/lppl-1-3c.txt
and the latest version of this license is in
   http://www.latex-project.org/lppl.txt
and version 1.3c or later is part of all distributions of
LaTeX version 2005/12/01 or later.

This work has the LPPL maintenance status "maintained".

The Current Maintainer of this work is H.-Martin Muench.

This work consists of the main source file pagecolor.dtx,
the README, and the derived files
   pagecolor.sty, pagecolor.pdf,
   pagecolor.ins, pagecolor.drv,
   pagecolor-example.tex, pagecolor-example.pdf.

\endpreamble
\let\MetaPrefix\DoubleperCent

\generate{%
  \file{pagecolor.ins}{\from{pagecolor.dtx}{install}}%
  \file{pagecolor.drv}{\from{pagecolor.dtx}{driver}}%
  \usedir{tex/latex/pagecolor}%
  \file{pagecolor.sty}{\from{pagecolor.dtx}{package}}%
  \usedir{doc/latex/pagecolor}%
  \file{pagecolor-example.tex}{\from{pagecolor.dtx}{example}}%
}

\catcode32=13\relax% active space
\let =\space%
\Msg{************************************************************************}
\Msg{*}
\Msg{* To finish the installation you have to move the following}
\Msg{* file into a directory searched by TeX:}
\Msg{*}
\Msg{*     pagecolor.sty}
\Msg{*}
\Msg{* To produce the documentation run the file `pagecolor.drv'}
\Msg{* through (pdf)LaTeX, e.g.}
\Msg{*  pdflatex pagecolor.drv}
\Msg{*  makeindex -s gind.ist pagecolor.idx}
\Msg{*  pdflatex pagecolor.drv}
\Msg{*  makeindex -s gind.ist pagecolor.idx}
\Msg{*  pdflatex pagecolor.drv}
\Msg{*}
\Msg{* At least three runs are necessary e.g. to get the}
\Msg{*  references right!}
\Msg{*}
\Msg{* Happy TeXing!}
\Msg{*}
\Msg{************************************************************************}

\endbatchfile
%</install>
%<*ignore>
\fi
%</ignore>
%
% \section{The documentation driver file}
%
% The next bit of code contains the documentation driver file for
% \TeX{}, i.\,e., the file that will produce the documentation you
% are currently reading. It will be extracted from this file by the
% \texttt{docstrip} programme. That is, run \LaTeX{} on \texttt{docstrip}
% and specify the \texttt{driver} option when \texttt{docstrip}
% asks for options.
%
%    \begin{macrocode}
%<*driver>
\NeedsTeXFormat{LaTeX2e}[2011/06/27]
\ProvidesFile{pagecolor.drv}%
  [2012/02/23 v1.0e Provides thepagecolor (HMM)]
\documentclass{ltxdoc}[2007/11/11]% v2.0u
\usepackage{holtxdoc}[2011/02/04]%  v0.21
%% pagecolor may work with earlier versions of LaTeX2e and those
%% class and package, but this was not tested.
%% Please consider updating your LaTeX, class, and package
%% to the most recent version (if they are not already the most
%% recent version).
\hypersetup{%
 pdfsubject={Providing the thepagecolor command (HMM)},%
 pdfkeywords={LaTeX, pagecolor, thepagecolor, H.-Martin Muench},%
 pdfencoding=auto,%
 pdflang={en},%
 breaklinks=true,%
 linktoc=all,%
 pdfstartview=FitH,%
 pdfpagelayout=OneColumn,%
 bookmarksnumbered=true,%
 bookmarksopen=true,%
 bookmarksopenlevel=3,%
 pdfmenubar=true,%
 pdftoolbar=true,%
 pdfwindowui=true,%
 pdfnewwindow=true%
}
\CodelineIndex
\hyphenation{printing docu-ment}
\gdef\unit#1{\mathord{\thinspace\mathrm{#1}}}%
\begin{document}
  \DocInput{pagecolor.dtx}%
\end{document}
%</driver>
%    \end{macrocode}
%
% \fi
%
% \CheckSum{137}
%
% \CharacterTable
%  {Upper-case    \A\B\C\D\E\F\G\H\I\J\K\L\M\N\O\P\Q\R\S\T\U\V\W\X\Y\Z
%   Lower-case    \a\b\c\d\e\f\g\h\i\j\k\l\m\n\o\p\q\r\s\t\u\v\w\x\y\z
%   Digits        \0\1\2\3\4\5\6\7\8\9
%   Exclamation   \!     Double quote  \"     Hash (number) \#
%   Dollar        \$     Percent       \%     Ampersand     \&
%   Acute accent  \'     Left paren    \(     Right paren   \)
%   Asterisk      \*     Plus          \+     Comma         \,
%   Minus         \-     Point         \.     Solidus       \/
%   Colon         \:     Semicolon     \;     Less than     \<
%   Equals        \=     Greater than  \>     Question mark \?
%   Commercial at \@     Left bracket  \[     Backslash     \\
%   Right bracket \]     Circumflex    \^     Underscore    \_
%   Grave accent  \`     Left brace    \{     Vertical bar  \|
%   Right brace   \}     Tilde         \~}
%
% \GetFileInfo{pagecolor.drv}
%
% \begingroup
%   \def\x{\#,\$,\^,\_,\~,\ ,\&,\{,\},\%}%
%   \makeatletter
%   \@onelevel@sanitize\x
% \expandafter\endgroup
% \expandafter\DoNotIndex\expandafter{\x}
% \expandafter\DoNotIndex\expandafter{\string\ }
% \begingroup
%   \makeatletter
%     \lccode`9=32\relax
%     \lowercase{%^^A
%       \edef\x{\noexpand\DoNotIndex{\@backslashchar9}}%^^A
%     }%^^A
%   \expandafter\endgroup\x
%
% \DoNotIndex{\,,\\}
% \DoNotIndex{\documentclass,\usepackage,\ProvidesPackage,\begin,\end}
% \DoNotIndex{\MessageBreak}
% \DoNotIndex{\NeedsTeXFormat,\DoNotIndex,\verb}
% \DoNotIndex{\def,\edef,\gdef,\global}
% \DoNotIndex{\ifx,\listfiles,\mathord,\mathrm}
% \DoNotIndex{\kvoptions,\SetupKeyvalOptions,\ProcessKeyvalOptions}
% \DoNotIndex{\bigskip,\space,\thinspace,\ldots}
% \DoNotIndex{\indent,\noindent,\newline,\linebreak,\pagebreak,\newpage}
% \DoNotIndex{\textbf,\textit,\textsf,\texttt,\textsc,\textquotedblleft,\textquotedblright}
% \DoNotIndex{\plainTeX,\TeX,\LaTeX,\pdfLaTeX}
% \DoNotIndex{\section,\color,\pagenumbering,\smallskip}
% \DoNotIndex{\@ifundefined,\@ifpackageloaded}
% \DoNotIndex{\DeclareStringOption}
%
% \title{The \xpackage{pagecolor} package}
% \date{2012/02/23 v1.0e}
% \author{H.-Martin M\"{u}nch\\\xemail{Martin.Muench at Uni-Bonn.de}}
%
% \maketitle
%
% \begin{abstract}
%  This \LaTeX{} package provides the command |\thepagecolor|,
% which gives the current page (background) colour,
% i.\,e.~the argument used with the most recent call of |\pagecolor{...}|.
% The command |\thepagecolornone| gives the same colour as |\thepagecolor|,
% except when the page background colour is
% \textquotedblleft|none|\textquotedblright. In that case |\thepagecolor| is
% |white| and |\thepagecolornone| is |none|.\\
% Similar to |\newgeometry| and |\restoregeometry| of the \xpackage{geometry}
% package\\
% |\newpagecolor{<|\textit{some colour} |>}| and |\restorepagecolor|
% are provided.
% \end{abstract}
%
% \bigskip
%
% \noindent Disclaimer for web links: The author is not responsible for any contents
% referred to in this work unless he has full knowledge of illegal contents.
% If any damage occurs by the use of information presented there, only the
% author of the respective pages might be liable, not the one who has referred
% to these pages.
%
% \bigskip
%
% \noindent {\color{green} Save per page about $200\unit{ml}$ water,
% $2\unit{g}$ CO$_{2}$ and $2\unit{g}$ wood:\\
% Therefore please print only if this is really necessary.}
%
% \pagebreak
%
% \tableofcontents
%
% \bigskip
%
% \section{Introduction\label{sec:Introduction}}
%
% \indent This \LaTeX{} package provides the command |\thepagecolor|,
% which gives the current page (background) colour,
% i.\,e.~the argument used with the most recent call of |\pagecolor{...}|.
% The package should be loaded before any package sets a page (background)
% colour, but after \xpackage{color} or \xpackage{xcolor} package.
% Its option |pagecolor={...}| is used to set the |\pagecolor{...}|.\\
% The command |\thepagecolornone| gives the same colour as |\thepagecolor|,
% except when the page background colour is
% \textquotedblleft|none|\textquotedblright (e.\,g. result of using the
% |\nopagecolor| command). In that case |\thepagecolor| is |white| and\linebreak
% |\thepagecolornone| is |none|.\\
% Similar to |\newgeometry| and |\restoregeometry| of the \xpackage{geometry}
% package\\
% |\newpagecolor{<|\textit{some colour}|>}| and |\restorepagecolor|
% are provided.
%
% \pagebreak
%
% \section{Usage}
%
% \indent Just load the package placing
% \begin{quote}
%   |\usepackage[<|\textit{option}|>]{pagecolor}|
% \end{quote}
% \noindent in the preamble of your \LaTeXe{} source file.
% This should be done before another package uses |\pagecolor|,
% but after |\nopagecolor| is defined (if that is defined in the document
% at all).\\
% Afterwards |\pagecolor{...}| can be used to change the page (background)
% colour as usual. Then |\thepagecolor| gives the current page (background)
% colour (in the same format as given with |\pagecolor{...}|).\\
% Similar to |\newgeometry| and |\restoregeometry| of the \xpackage{geometry}
% package\\
% |\newpagecolor{<|\textit{some colour} |>}| and |\restorepagecolor|
% are provided:\\
% |\newpagecolor{<|\textit{some colour} |>}| will execute
% |\pagecolor{<|\textit{some colour} |>}| and remember the page colour used before.
% |\restorecolor| (without argument) restores the page colour to the one used before
%  use of the |\newpagecolor{...}| command. When you want to change the colour for
%  just one page and do not want to (or cannot) manually determine where the page ends,\\
% |\newpagecolor{<|\textit{some colour}|>}\afterpage{\restorepagecolor}|\\
% does the trick (and requires a |\usepackage{afterpage}| in the document's
% preamble), or for short\\
% |\newcommand{\onepagecolor}[1]{%|\\
% |\newpagecolor{#1}\afterpage{\restorepagecolor}}|\\
% in the preamble and\\
% |\onepagecolor{<|\textit{some colour}|>}| in the document.
%
% \subsection{Option}
% \DescribeMacro{option}
% \indent The \xpackage{pagecolor} package takes the following option:
%
% \subsubsection{pagecolor\label{sss:pagecolor}}
% \DescribeMacro{pagecolor}
% \indent The option |pagecolor={...}| takes as value a colour. This
% could be as simple as |white| or |black|, but when e.\,g.~the
% \xpackage{xcolor} package is used (loaded before \xpackage{pagecolor}!),
% also colours like |red!50!green!20!blue| are possible.
% The default is |pagecolor={none}|.
%
% \section{Alternatives\label{sec:Alternatives}}
%
% As I neither know what exactly you want to accomplish
% when using this package (e.\,g.~hiding text), nor what resources you have
% (e.\,g.~\pdfTeX{} version), here is a list of some possible alternatives:\\
%
% \begin{description}
%  \item[-]
%  \DescribeMacro{transparent}
%  \xpackage{transparent} package: With it some object can be made
%  (fully or partially) transparent.
% \end{description}
%
% \begin{description}
%  \item[-]
%  \DescribeMacro{hrefhide}\xpackage{hrefhide} package: It allows to
%  \textquotedblleft hide\textquotedblright\ some (hyperlinked) text when
%  printing the document while keeping the layout.
% \end{description}
%
% \bigskip
%
% \noindent (You programmed or found another alternative,
%  which is available at \CTAN{}?\\
%  OK, send an e-mail to me with the name, location at \CTAN{},
%  and a short notice, and I will probably include it in
%  the list above.)\\
%
% \smallskip
%
% \noindent About how to get those packages, please see
% subsection~\ref{ss:Downloads}.
%
% \newpage
%
% \section{Example}
%
%    \begin{macrocode}
%<*example>
\documentclass[british]{article}[2007/10/19]% v1.4h
%%%%%%%%%%%%%%%%%%%%%%%%%%%%%%%%%%%%%%%%%%%%%%%%%%%%%%%%%%%%%%%%%%%%%
\usepackage[%
 extension=pdf,%
 plainpages=false,%
 pdfpagelabels=true,%
 hyperindex=false,%
 pdflang={en},%
 pdftitle={pagecolor package example},%
 pdfauthor={H.-Martin Muench},%
 pdfsubject={Example for the pagecolor package},%
 pdfkeywords={LaTeX, pagecolor, thepagecolor, page colour,%
  H.-Martin Muench},%
 pdfview=Fit,pdfstartview=Fit,%
 pdfpagelayout=SinglePage%
]{hyperref}[2012/02/06]% v6.82o
\usepackage[x11names]{xcolor}[2007/01/21]% v2.11
 % The xcolor package would not be needed for just using
 % the base colours. The color package would be sufficient for that.
\usepackage[pagecolor={LightGoldenrod1}]{pagecolor}[2012/02/23]% v1.0e

\usepackage{afterpage}[1995/10/27]% v1.08
 % The afterpage package is generally not needed,
 % but the |\newpagecolor{somecolour}\afterpage{\restorepagecolor}|
 % construct shall be demonstrated.

\usepackage{lipsum}[2011/04/14]% v1.2
 % The lipsum package is generally not needed,
 % but some blind text is needed for the example.

\gdef\unit#1{\mathord{\thinspace\mathrm{#1}}}%
\listfiles
\begin{document}
\pagenumbering{arabic}
\section*{Example for pagecolor}

This example demonstrates the use of package\newline
\textsf{pagecolor}, v1.0e as of 2012/02/23 (HMM).\newline
The used option was \verb|pagecolor={LightGoldenrod1}|.\newline
\verb|pagecolor={none}| would be the default.\newline

For more details please see the documentation!\newline

\noindent {\color{teal} Save per page about $200\unit{ml}$ water,
$2\unit{g}$ CO$_{2}$ and $2\unit{g}$ wood:\newline
Therefore please print only if this is really necessary.}\newline

The current page (background) colour is\newline
\verb|\thepagecolor|\ =\ \thepagecolor \newline
(and \verb|\thepagecolornone|\ =\ \thepagecolornone ,
which would only be different from \verb|\thepagecolor|,
when the page colour would be \verb|none|).

\pagebreak
\pagecolor{rgb:-green!40!yellow,3;green!40!yellow,2;red,1}

{\color{white} The current page (background) colour is\newline
\verb|\thepagecolor|\ =\ \thepagecolor . \newline}

{\color{\thepagecolor} And that makes this text practically invisible.
\newline}

{\color{white} Which made the preceding line of text practically
invisible.}

\pagebreak
\newpagecolor{red}

This page uses \verb|\newpagecolor{red}|.

\pagebreak
\restorepagecolor

{\color{white}And this page uses \verb|\restorepagecolor| to restore
the page colour to the value it had before the red page.}

\pagebreak
\pagecolor{none}

This page uses \verb|\pagecolor{none}|. If the \verb|\nopagecolor|
command is known (pdf\TeX and Lua\TeX; not yet for dvips, dvipdfm(x)
or Xe\TeX), the page colour is now \verb|none|, otherwise \verb|white|:
\verb|\thepagecolor|\ =\ \thepagecolor\ and
\verb|\thepagecolornone|\ =\ \thepagecolornone .

\pagebreak
\restorepagecolor

{\color{white}\verb|\restorepagecolor| restored the page colour again.}

\pagebreak
\pagecolor{green}

This page is green due to \verb|\pagecolor{green}|.

\pagebreak
\newpagecolor{blue}\afterpage{\restorepagecolor}

{\color{white}\verb|\newpagecolor{blue}\afterpage{\restorepagecolor}|%
\newline
was used here, i.\,e.~this page is blue, and the next one will
automatically have the same page colour before it was changed to blue
here (i.\,e. green).}

\smallskip
{\color{red}\textbf{\lipsum[1-11]}}
\bigskip

The page colour was changed back at the end of the page -
in mid-sentence!

\end{document}
%</example>
%    \end{macrocode}
%
% \StopEventually{}
%
% \newpage
%
% \section{The implementation}
%
% We start off by checking that we are loading into \LaTeXe{} and
% announcing the name and version of this package.
%
%    \begin{macrocode}
%<*package>
%    \end{macrocode}
%
%    \begin{macrocode}
\NeedsTeXFormat{LaTeX2e}[2011/06/27]
\ProvidesPackage{pagecolor}[2012/02/23 v1.0e
            Provides thepagecolor (HMM)]
%    \end{macrocode}
%
% A short description of the \xpackage{pagecolor} package:
%
%    \begin{macrocode}
%% Provides the \thepagecolor, \thepagecolornone, \newpagecolor{...},
%% and \restorepagecolor commands.
%    \end{macrocode}
%
% We need the \xpackage{kvoptions} package by \textsc{Heiko Oberdiek}:
%
%    \begin{macrocode}
\RequirePackage{kvoptions}[2010/12/23]% v3.10
%    \end{macrocode}
%
% and either the \xpackage{color} or the \xpackage{xcolor} package:
%
%    \begin{macrocode}
%% \RequirePackage{ either color or xcolor }:
\@ifpackageloaded{xcolor}{% xcolor loaded
   \@ifpackagelater{xcolor}{2007/01/21}{%
      % 2007/01/21, v2.11, or even more recent: OK
    }{% else: older package version
      \PackageWarning{pagecolor}{%
        It is requested version `2007/01/21' of package\MessageBreak%
        xcolor, but only an older version is available.\MessageBreak%
       }%
     }%
 }{% xcolor not loaded
   \@ifpackageloaded{color}{%
      \RequirePackage{color}[2005/11/14]% v1.0j
    }{% \else
      \PackageWarning{pagecolor}{%
        The pagecolor package must be loaded after either\MessageBreak%
        package color or after package xcolor (at your\MessageBreak%
        option). Neither package was loaded before package\MessageBreak%
        pagecolor. Loading of package xcolor will now be\MessageBreak%
        tried automatically.\MessageBreak%
        When the pagecolor package is used with option\MessageBreak%
        pagecolor using a colour requiring e. g. x11names\MessageBreak%
        option for xcolor package, this will not work!\MessageBreak%
       }
     }% \fi
   \RequirePackage{xcolor}[2007/01/21]% v2.11
  }% \fi
%    \end{macrocode}
%
% A last information for the user:
%
%    \begin{macrocode}
%% pagecolor may work with earlier versions of LaTeX and the
%% packages, but this was not tested. Please consider updating
%% your LaTeX and packages to the most recent version
%% (if they are not already the most recent version).

%    \end{macrocode}
%
% \indent See subsection~\ref{ss:Downloads} about how to get them.\\
%
% We process the options:
%
%    \begin{macrocode}
\SetupKeyvalOptions{family=pagecolor,prefix=pagecolor@}
\DeclareStringOption[none]{pagecolor}% \pagecolor@pagecolor
\ProcessKeyvalOptions*

%    \end{macrocode}
%
% We save the original |\pagecolor| command,
%
%    \begin{macrocode}
\let\origpagecolour\pagecolor

%    \end{macrocode}
%
%    \begin{macro}{\pagecolor}
% before we redefine it to include a definition of |\thepagecolor|
% and \linebreak
% |\thepagecolornone|:
%
%    \begin{macrocode}
\renewcommand{\pagecolor}[1]{\@bsphack%
  \edef\pagecolourtmpa{#1}%
  \def\pagecolourtmpb{none}%
  \ifx\pagecolourtmpa\pagecolourtmpb
    \@ifundefined{nopagecolor}{%
      \PackageError{pagecolor}{%
        pagecolor=none requested but \string\nopagecolor\space unknown%
       }{%
        \string\pagecolor{none} was used, but the command %
        \string\nopagecolor\space is undefined.\MessageBreak%
        Please use another colour. pagecolor=white will be used now.%
        \MessageBreak%
        }%
      \gdef\thepagecolor{white}%
      \gdef\thepagecolornone{white}% although it should be "none"
      \origpagecolour{white}%
     }{%
      \nopagecolor%
     }%
  \else%
    \xdef\thepagecolor{#1}%
    \xdef\thepagecolornone{#1}%
    \origpagecolour{\thepagecolornone}%
  \fi%
  \@esphack%
}

%    \end{macrocode}
%    \end{macro}
%
% |\nopagecolor| is only defined for \pdfTeX and \LuaTeX, but not yet for dvips,
% dvipdfm(x) or \XeTeX. Maybe
% \begin{verbatim}
% \@ifundefined{nopagecolor}{\newcommand{\nopagecolor}{\pagecolor{white}}}{}
% \end{verbatim}
% could be an alternative.\\
% When |\pagecolor{none}| is used and |\nopagecolor| is defined, then \\
% |\pagecolor{none}| is made into a synonym for |\nopagecolor|.\\
% If |\nopagecolor| has not been defined, nothing needs to be done.
% Otherwise we redefine |\nopagecolor|, thus that |\thepagecolor| is
% set to |white| and |\thepagecolornone| to |none| whenever
% |\nopagecolor| is used.
%
%    \begin{macrocode}
\@ifundefined{nopagecolor}{%
  \PackageWarning{pagecolor}{%
    \string\nopagecolor\space is undefined.\MessageBreak%
    (As long as it is neither used anywhere\MessageBreak%
    \space nor defined later, that will be no problem.)\MessageBreak%
   }
  \AtEndDocument{%
  \@ifundefined{nopagecolor}{% no problem
    }{%
    \PackageError{pagecolor}{%
      \string\nopagecolor\space defined after loading package%
      \MessageBreak%
      pagecolor%
      }{\string\nopagecolor\space was defined after loading the %
       pagecolor package.\MessageBreak%
       Either define \string\nopagecolor\space earlier or load the %
       pagecolor package later.\MessageBreak%
      }
    }%
  }%
  }{% \else
    \let\orignopagecolour\nopagecolor
    \renewcommand{\nopagecolor}{%
      \xdef\thepagecolor{white}
      \xdef\thepagecolornone{none}
      \orignopagecolour
      }
  }

%    \end{macrocode}
%
% If the page colour as given with option |pagecolor={...}| is |none|,
% but |\nopagecolor| is not known, |\pagecolor@pagecolor| is set to
% |white| and a warning is given.
%
%    \begin{macrocode}
\def\pagecolourtmpb{none}
\ifx\pagecolor@pagecolor\pagecolourtmpb
  \@ifundefined{nopagecolor}{%
    \PackageWarning{pagecolor}{%
      Option pagecolor=none (maybe by default) used,\MessageBreak%
      but \string\nopagecolor\space is unknown. Please use another%
      \MessageBreak%
      option value; white will be used now.\MessageBreak%
     }
    \setkeys{pagecolor}{pagecolor=white}%
  }{% ok
  }%
\fi

%    \end{macrocode}
%
% The (new) |\pagecolor| is now just carried out.
%
%    \begin{macrocode}
\pagecolor{\pagecolor@pagecolor}

%    \end{macrocode}
%
% Now the page (background) colour and |\thepagecolor| and \\
% |\thepagecolornone| are |\pagecolor@pagecolor| (or page (background) colour
% and |\thepagecolornone| are |none|=|\pagecolor@pagecolor| and
% |\thepagecolor| is |white|), and when the page (background) colour
% is changed, |\thepagecolor| and |\thepagecolornone| are changed
% accordingly.
%
% \newpage
%
%    \begin{macro}{\newpagecolor}
% There have been requests (via e-mail and at \\
% \url{http://tex.stackexchange.com/q/25137}) to change the colour of just one
% (or two) page(s) only, similar to |\newgeometry| and |\restoregeometry| of the
% \xpackage{geometry} package (\url{http://ctan.org/pkg/geometry}). Therefore
% |\newpagecolor| and |\restorepagecolor| are introduced (as suggested by
% \textsc{haoyun\_tex}):
%
%    \begin{macrocode}
\newcommand{\newpagecolor}[1]{%
\xdef\pagecolourtmpc{\thepagecolornone}%
\pagecolor{#1}%
}

%    \end{macrocode}
%    \end{macro}
%
% |\newpagecolor{<|\textit{some colour}|>}| will execute |\pagecolor{somecolour}|
% and remember the page colour used before.\\
%
%    \begin{macro}{\restorepagecolor}
%    \begin{macrocode}
\newcommand{\restorepagecolor}{\pagecolor{\pagecolourtmpc}}

%    \end{macrocode}
%    \end{macro}
%
% |\restorecolor| (without argument) restores the page colour to the one used before
% use of the |\newpagecolor{...}| command.
%
%    \begin{macrocode}
\gdef\pagecolourtmpc{\thepagecolor}

%    \end{macrocode}
% is just a precaution for |\restorecolor| being used when no |\newpagecolor{...}|
% was used before it.\\
%
% When you want to change the colour for just one page and do not want to
% (or cannot) manually determine where the page ends,\\
% |\newpagecolor{<|\textit{some colour}|>}\afterpage{\restorepagecolor}|\\
% does the trick (and requires an additional |\usepackage{afterpage}| in the
% document's preamble).
%    \begin{macrocode}
%</package>
%    \end{macrocode}
%
% \newpage
%
% \section{Installation}
%
% \subsection{Downloads\label{ss:Downloads}}
%
% Everything is available on \CTAN{}, \url{http://www.ctan.org/tex-archive/},
% but may need additional packages themselves.\\
%
% \DescribeMacro{pagecolor.dtx}
% For unpacking the |pagecolor.dtx| file and constructing the documentation
% it is required:
% \begin{description}
% \item[-] \TeX{} Format \LaTeXe{}, \url{http://www.CTAN.org/}
%
% \item[-] document class \xpackage{ltxdoc}, 2007/11/11, v2.0u,
%           \url{http://ctan.org/pkg/ltxdoc}
%
% \item[-] package \xpackage{holtxdoc}, 2011/02/04, v0.21,
%           \url{http://ctan.org/pkg/holtxdoc}
% \end{description}
%
% \DescribeMacro{pagecolor.sty}
% The |pagecolor.sty| for \LaTeXe{} (i.\,e. each document using
% the \xpackage{pagecolor} package) requires:
% \begin{description}
% \item[-] \TeX{} Format \LaTeXe{}, \url{http://www.CTAN.org/}
%
% \item[-] package \xpackage{kvoptions}, 2010/12/23, v3.10,
%           \url{http://ctan.org/pkg/kvoptions}
% \end{description}
% and either
% \begin{description}
% \item[-] package \xpackage{color}, 2005/11/14, v1.0j,
%           \url{http://ctan.org/pkg/color} (from the\\
%           \xpackage{graphics} package bundle)
% \end{description}
% or
% \begin{description}
% \item[-] package \xpackage{xcolor}, 2007/01/21, v2.11,
%           \url{http://ctan.org/pkg/xcolor}
% \end{description}
%
% \DescribeMacro{pagecolor-example.tex}
% The |pagecolor-example.tex| requires the same files as all
% documents using the \xpackage{pagecolor} package and additionally:
% \begin{description}
% \item[-] class \xpackage{article}, 2007/10/19, v1.4h, from \xpackage{classes.dtx}:\\
%           \CTAN{macros/latex/base/classes.dtx}
%
% \item[-] package \xpackage{pagecolor}, 2012/02/23, v1.0e,
%           \url{http://ctan.org/pkg/pagecolor}\\
%           (Well, it is the example file for this package, and because you are
%            reading the documentation for the \xpackage{pagecolor} package,
%            it can be assumed that you already have some version of it --
%            is it the current one?)
%
% \item[-] package \xpackage{xcolor}, 2007/01/21, v2.11,
%           \url{http://ctan.org/pkg/xcolor}\\
%           This package would not be needed for the use of just base colours only,
%           the \xpackage{color} package would be sufficient for that.
% \end{description}
%
% \DescribeMacro{transparent}
% \DescribeMacro{hrefhide}
% As possible alternatives in section \ref{sec:Alternatives} there are listed
% \begin{description}
% \item[-] package \xpackage{transparent}, 2007/01/08, v1.0, \\
%           \url{http://ctan.org/pkg/transparent}
%
% \item[-] package \xpackage{hrefhide}, 2011/04/29, v1.0f, \\
%           \url{http://ctan.org/pkg/hrefhide}
% \end{description}
%
% \DescribeMacro{Oberdiek}
% \DescribeMacro{holtxdoc}
% \DescribeMacro{kvoptions}
% All packages of \textsc{Heiko Oberdiek}'s bundle `oberdiek'
% (especially \xpackage{holtxdoc} and \xpackage{kvoptions})
% are also available in a TDS compliant ZIP archive:\\
% \CTAN{install/macros/latex/contrib/oberdiek.tds.zip}.\\
% It is probably best to download and use this, because the packages in there
% are quite probably both recent and compatible among themselves.\\
%
% \DescribeMacro{hyperref}
% \noindent \xpackage{hyperref} is not included in that bundle and needs to be
% downloaded separately,\\
% \url{http://mirror.ctan.org/install/macros/latex/contrib/hyperref.tds.zip}.\\
%
% \DescribeMacro{M\"{u}nch}
% A hyperlinked list of my (other) packages can be found at
% \url{http://www.Uni-Bonn.de/~uzs5pv/LaTeX.html}.\\
%
% \subsection{Package, unpacking TDS}
% \paragraph{Package.} This package is available on \CTAN{}
% \begin{description}
% \item[\CTAN{macros/latex/contrib/pagecolor/pagecolor.dtx}]\hspace*{0.1cm} \\
%       The source file.
% \item[\CTAN{macros/latex/contrib/pagecolor/pagecolor.pdf}]\hspace*{0.1cm} \\
%       The documentation.
% \item[\CTAN{macros/latex/contrib/pagecolor/pagecolor-example.pdf}]\hspace*{0.1cm} \\
%       The compiled example file, as it should look like.
% \item[\CTAN{macros/latex/contrib/pagecolor/README}]\hspace*{0.1cm} \\
%       The README file.
% \end{description}
% There is also a pagecolor.tds.zip available:
% \begin{description}
% \item[\CTAN{install/macros/latex/contrib/pagecolor.tds.zip}]\hspace*{0.1cm} \\
%       Everything in \xfile{TDS} compliant, compiled format.
% \end{description}
% which additionally contains\\
% \begin{tabular}{ll}
% pagecolor.ins & The installation file.\\
% pagecolor.drv & The driver to generate the documentation.\\
% pagecolor.sty & The \xext{sty}le file.\\
% pagecolor-example.tex & The example file.
% \end{tabular}
%
% \bigskip
%
% \noindent For required other packages, please see the preceding subsection.
%
% \paragraph{Unpacking.} The \xfile{.dtx} file is a self-extracting
% \docstrip{} archive. The files are extracted by running the
% \xext{.dtx} through \plainTeX{}:
% \begin{quote}
%   \verb|tex pagecolor.dtx|
% \end{quote}
%
% About generating the documentation see paragraph~\ref{GenDoc} below.\\
%
% \paragraph{TDS.} Now the different files must be moved into
% the different directories in your installation TDS tree
% (also known as \xfile{texmf} tree):
% \begin{quote}
% \def\t{^^A
% \begin{tabular}{@{}>{\ttfamily}l@{ $\rightarrow$ }>{\ttfamily}l@{}}
%   pagecolor.sty & tex/latex/pagecolor/pagecolor.sty\\
%   pagecolor.pdf & doc/latex/pagecolor/pagecolor.pdf\\
%   pagecolor-example.tex & doc/latex/pagecolor/pagecolor-example.tex\\
%   pagecolor-example.pdf & doc/latex/pagecolor/pagecolor-example.pdf\\
%   pagecolor.dtx & source/latex/pagecolor/pagecolor.dtx\\
% \end{tabular}^^A
% }^^A
% \sbox0{\t}^^A
% \ifdim\wd0>\linewidth
%   \begingroup
%     \advance\linewidth by\leftmargin
%     \advance\linewidth by\rightmargin
%   \edef\x{\endgroup
%     \def\noexpand\lw{\the\linewidth}^^A
%   }\x
%   \def\lwbox{^^A
%     \leavevmode
%     \hbox to \linewidth{^^A
%       \kern-\leftmargin\relax
%       \hss
%       \usebox0
%       \hss
%       \kern-\rightmargin\relax
%     }^^A
%   }^^A
%   \ifdim\wd0>\lw
%     \sbox0{\small\t}^^A
%     \ifdim\wd0>\linewidth
%       \ifdim\wd0>\lw
%         \sbox0{\footnotesize\t}^^A
%         \ifdim\wd0>\linewidth
%           \ifdim\wd0>\lw
%             \sbox0{\scriptsize\t}^^A
%             \ifdim\wd0>\linewidth
%               \ifdim\wd0>\lw
%                 \sbox0{\tiny\t}^^A
%                 \ifdim\wd0>\linewidth
%                   \lwbox
%                 \else
%                   \usebox0
%                 \fi
%               \else
%                 \lwbox
%               \fi
%             \else
%               \usebox0
%             \fi
%           \else
%             \lwbox
%           \fi
%         \else
%           \usebox0
%         \fi
%       \else
%         \lwbox
%       \fi
%     \else
%       \usebox0
%     \fi
%   \else
%     \lwbox
%   \fi
% \else
%   \usebox0
% \fi
% \end{quote}
% If you have a \xfile{docstrip.cfg} that configures and enables \docstrip{}'s
% \xfile{TDS} installing feature, then some files can already be in the right
% place, see the documentation of \docstrip{}.
%
% \subsection{Refresh file name databases}
%
% If your \TeX{}~distribution (\teTeX{}, \mikTeX{},\dots{}) relies on file name
% databases, you must refresh these. For example, \teTeX{} users run
% \verb|texhash| or \verb|mktexlsr|.
%
% \subsection{Some details for the interested}
%
% \paragraph{Unpacking with \LaTeX{}.}
% The \xfile{.dtx} chooses its action depending on the format:
% \begin{description}
% \item[\plainTeX:] Run \docstrip{} and extract the files.
% \item[\LaTeX:] Generate the documentation.
% \end{description}
% If you insist on using \LaTeX{} for \docstrip{} (really,
% \docstrip{} does not need \LaTeX{}), then inform the autodetect routine
% about your intention:
% \begin{quote}
%   \verb|latex \let\install=y\input{pagecolor.dtx}|
% \end{quote}
% Do not forget to quote the argument according to the demands
% of your shell.
%
% \paragraph{Generating the documentation.\label{GenDoc}}
% You can use both the \xfile{.dtx} or the \xfile{.drv} to generate
% the documentation. The process can be configured by a
% configuration file \xfile{ltxdoc.cfg}. For instance, put the following
% line into this file, if you want to have A4 as paper format:
% \begin{quote}
%   \verb|\PassOptionsToClass{a4paper}{article}|
% \end{quote}
%
% \noindent An example follows how to generate the
% documentation with \pdfLaTeX{}:
%
% \begin{quote}
%\begin{verbatim}
%pdflatex pagecolor.dtx
%makeindex -s gind.ist pagecolor.idx
%pdflatex pagecolor.dtx
%makeindex -s gind.ist pagecolor.idx
%pdflatex pagecolor.dtx
%\end{verbatim}
% \end{quote}
%
% \subsection{Compiling the example}
%
% The example file, \textsf{pagecolor-example.tex}, can be compiled via\\
% |(pdf)latex pagecolor-example.tex|.\\
%
% \section{Acknowledgements}
%
% I would like to thank \textsc{Heiko Oberdiek} for providing a~lot~(!) of
% useful packages (from which I also got everything I know about creating
% a file in \xext{dtx} format, ok, say it: copying), and the
% \Newsgroup{comp.text.tex} and \Newsgroup{de.comp.text.tex} newsgroups
% for their help in all things \TeX{}, especially all contributors to the
% discussion at
% \url{http://groups.google.com/group/comp.text.tex/browse_thread/thread/533576ebe44d60f6/e1654d70a19de63c?lnk=gst&q=Determination+of+the+current+page+colour#e1654d70a19de63c}
% \ (\textsc{H.~Oberdiek} \& \textsc{Gouailles}).\\
% I thank \textsc{haoyun\_tex} for suggesting the |\newpagecolor|/|\restorepagecolor|
% pair of commands and everyone at \url{http://tex.stackexchange.com/q/25137}
% for their contributions there.
%
% \bigskip
%
% \phantomsection
% \begin{History}\label{History}
%   \begin{Version}{2011/07/16 v1.0a}
%     \item First version discussed at \Newsgroup{comp.text.tex}.
%   \end{Version}
%   \begin{Version}{2011/08/06 v1.0b}
%     \item Changed version uploaded to \CTAN{}.
%   \end{Version}
%   \begin{Version}{2011/08/08 v1.0c}
%     \item Fixed a |\setkeys|.
%   \end{Version}
%   \begin{Version}{2012/02/01 v1.0d}
%     \item Bugfix: Obsolete installation path given in the documentation, updated.
%     \item New commands: |\newpagecolor{...}|, |\restorepagecolor|.
%     \item Update of documentation, README, and \xfile{dtx} internals.
%   \end{Version}
%   \begin{Version}{2012/02/23 v1.0e}
%     \item Fixed an error in the documentation.
%     \item Check for loading of \xpackage{color} or \xpackage{xcolor} package
%             and their versions has been changed, because \xpackage{xcolor} sets\\
%             |\@namedef{ver@color.sty}{1999/02/16}|\\
%             which gave a warning about old \xpackage{color} package even if a new
%             version was used.
%   \end{Version}
% \end{History}
%
% \bigskip
%
% When you find a mistake or have a suggestion for an improvement of this package,
% please send an e-mail to the maintainer, thanks! (Please see BUG REPORTS in the README.)
%
% \newpage
%
% \PrintIndex
%
% \Finale
\endinput
%        (quote the arguments according to the demands of your shell)
%
% Documentation:
%    (a) If pagecolor.drv is present:
%           (pdf)latex pagecolor.drv
%           makeindex -s gind.ist pagecolor.idx
%           (pdf)latex pagecolor.drv
%           makeindex -s gind.ist pagecolor.idx
%           (pdf)latex pagecolor.drv
%    (b) Without pagecolor.drv:
%           (pdf)latex pagecolor.dtx
%           makeindex -s gind.ist pagecolor.idx
%           (pdf)latex pagecolor.dtx
%           makeindex -s gind.ist pagecolor.idx
%           (pdf)latex pagecolor.dtx
%
%    The class ltxdoc loads the configuration file ltxdoc.cfg
%    if available. Here you can specify further options, e.g.
%    use DIN A4 as paper format:
%       \PassOptionsToClass{a4paper}{article}
%
% Installation:
%    TDS:tex/latex/pagecolor/pagecolor.sty
%    TDS:doc/latex/pagecolor/pagecolor.pdf
%    TDS:doc/latex/pagecolor/pagecolor-example.tex
%    TDS:doc/latex/pagecolor/pagecolor-example.pdf
%    TDS:source/latex/pagecolor/pagecolor.dtx
%
%<*ignore>
\begingroup
  \catcode123=1 %
  \catcode125=2 %
  \def\x{LaTeX2e}%
\expandafter\endgroup
\ifcase 0\ifx\install y1\fi\expandafter
         \ifx\csname processbatchFile\endcsname\relax\else1\fi
         \ifx\fmtname\x\else 1\fi\relax
\else\csname fi\endcsname
%</ignore>
%<*install>
\input docstrip.tex
\Msg{*****************************************************************}
\Msg{* Installation                                                   }
\Msg{* Package: pagecolor 2012/02/23 v1.0e Provides thepagecolor (HMM)}
\Msg{*****************************************************************}

\keepsilent
\askforoverwritefalse

\let\MetaPrefix\relax
\preamble

This is a generated file.

Project: pagecolor
Version: 2012/02/23 v1.0e

Copyright (C) 2011 - 2012 by
    H.-Martin M"unch <Martin dot Muench at Uni-Bonn dot de>

The usual disclaimer applies:
If it doesn't work right that's your problem.
(Nevertheless, send an e-mail to the maintainer
 when you find an error in this package.)

This work may be distributed and/or modified under the
conditions of the LaTeX Project Public License, either
version 1.3c of this license or (at your option) any later
version. This version of this license is in
   http://www.latex-project.org/lppl/lppl-1-3c.txt
and the latest version of this license is in
   http://www.latex-project.org/lppl.txt
and version 1.3c or later is part of all distributions of
LaTeX version 2005/12/01 or later.

This work has the LPPL maintenance status "maintained".

The Current Maintainer of this work is H.-Martin Muench.

This work consists of the main source file pagecolor.dtx,
the README, and the derived files
   pagecolor.sty, pagecolor.pdf,
   pagecolor.ins, pagecolor.drv,
   pagecolor-example.tex, pagecolor-example.pdf.

\endpreamble
\let\MetaPrefix\DoubleperCent

\generate{%
  \file{pagecolor.ins}{\from{pagecolor.dtx}{install}}%
  \file{pagecolor.drv}{\from{pagecolor.dtx}{driver}}%
  \usedir{tex/latex/pagecolor}%
  \file{pagecolor.sty}{\from{pagecolor.dtx}{package}}%
  \usedir{doc/latex/pagecolor}%
  \file{pagecolor-example.tex}{\from{pagecolor.dtx}{example}}%
}

\catcode32=13\relax% active space
\let =\space%
\Msg{************************************************************************}
\Msg{*}
\Msg{* To finish the installation you have to move the following}
\Msg{* file into a directory searched by TeX:}
\Msg{*}
\Msg{*     pagecolor.sty}
\Msg{*}
\Msg{* To produce the documentation run the file `pagecolor.drv'}
\Msg{* through (pdf)LaTeX, e.g.}
\Msg{*  pdflatex pagecolor.drv}
\Msg{*  makeindex -s gind.ist pagecolor.idx}
\Msg{*  pdflatex pagecolor.drv}
\Msg{*  makeindex -s gind.ist pagecolor.idx}
\Msg{*  pdflatex pagecolor.drv}
\Msg{*}
\Msg{* At least three runs are necessary e.g. to get the}
\Msg{*  references right!}
\Msg{*}
\Msg{* Happy TeXing!}
\Msg{*}
\Msg{************************************************************************}

\endbatchfile
%</install>
%<*ignore>
\fi
%</ignore>
%
% \section{The documentation driver file}
%
% The next bit of code contains the documentation driver file for
% \TeX{}, i.\,e., the file that will produce the documentation you
% are currently reading. It will be extracted from this file by the
% \texttt{docstrip} programme. That is, run \LaTeX{} on \texttt{docstrip}
% and specify the \texttt{driver} option when \texttt{docstrip}
% asks for options.
%
%    \begin{macrocode}
%<*driver>
\NeedsTeXFormat{LaTeX2e}[2011/06/27]
\ProvidesFile{pagecolor.drv}%
  [2012/02/23 v1.0e Provides thepagecolor (HMM)]
\documentclass{ltxdoc}[2007/11/11]% v2.0u
\usepackage{holtxdoc}[2011/02/04]%  v0.21
%% pagecolor may work with earlier versions of LaTeX2e and those
%% class and package, but this was not tested.
%% Please consider updating your LaTeX, class, and package
%% to the most recent version (if they are not already the most
%% recent version).
\hypersetup{%
 pdfsubject={Providing the thepagecolor command (HMM)},%
 pdfkeywords={LaTeX, pagecolor, thepagecolor, H.-Martin Muench},%
 pdfencoding=auto,%
 pdflang={en},%
 breaklinks=true,%
 linktoc=all,%
 pdfstartview=FitH,%
 pdfpagelayout=OneColumn,%
 bookmarksnumbered=true,%
 bookmarksopen=true,%
 bookmarksopenlevel=3,%
 pdfmenubar=true,%
 pdftoolbar=true,%
 pdfwindowui=true,%
 pdfnewwindow=true%
}
\CodelineIndex
\hyphenation{printing docu-ment}
\gdef\unit#1{\mathord{\thinspace\mathrm{#1}}}%
\begin{document}
  \DocInput{pagecolor.dtx}%
\end{document}
%</driver>
%    \end{macrocode}
%
% \fi
%
% \CheckSum{137}
%
% \CharacterTable
%  {Upper-case    \A\B\C\D\E\F\G\H\I\J\K\L\M\N\O\P\Q\R\S\T\U\V\W\X\Y\Z
%   Lower-case    \a\b\c\d\e\f\g\h\i\j\k\l\m\n\o\p\q\r\s\t\u\v\w\x\y\z
%   Digits        \0\1\2\3\4\5\6\7\8\9
%   Exclamation   \!     Double quote  \"     Hash (number) \#
%   Dollar        \$     Percent       \%     Ampersand     \&
%   Acute accent  \'     Left paren    \(     Right paren   \)
%   Asterisk      \*     Plus          \+     Comma         \,
%   Minus         \-     Point         \.     Solidus       \/
%   Colon         \:     Semicolon     \;     Less than     \<
%   Equals        \=     Greater than  \>     Question mark \?
%   Commercial at \@     Left bracket  \[     Backslash     \\
%   Right bracket \]     Circumflex    \^     Underscore    \_
%   Grave accent  \`     Left brace    \{     Vertical bar  \|
%   Right brace   \}     Tilde         \~}
%
% \GetFileInfo{pagecolor.drv}
%
% \begingroup
%   \def\x{\#,\$,\^,\_,\~,\ ,\&,\{,\},\%}%
%   \makeatletter
%   \@onelevel@sanitize\x
% \expandafter\endgroup
% \expandafter\DoNotIndex\expandafter{\x}
% \expandafter\DoNotIndex\expandafter{\string\ }
% \begingroup
%   \makeatletter
%     \lccode`9=32\relax
%     \lowercase{%^^A
%       \edef\x{\noexpand\DoNotIndex{\@backslashchar9}}%^^A
%     }%^^A
%   \expandafter\endgroup\x
%
% \DoNotIndex{\,,\\}
% \DoNotIndex{\documentclass,\usepackage,\ProvidesPackage,\begin,\end}
% \DoNotIndex{\MessageBreak}
% \DoNotIndex{\NeedsTeXFormat,\DoNotIndex,\verb}
% \DoNotIndex{\def,\edef,\gdef,\global}
% \DoNotIndex{\ifx,\listfiles,\mathord,\mathrm}
% \DoNotIndex{\kvoptions,\SetupKeyvalOptions,\ProcessKeyvalOptions}
% \DoNotIndex{\bigskip,\space,\thinspace,\ldots}
% \DoNotIndex{\indent,\noindent,\newline,\linebreak,\pagebreak,\newpage}
% \DoNotIndex{\textbf,\textit,\textsf,\texttt,\textsc,\textquotedblleft,\textquotedblright}
% \DoNotIndex{\plainTeX,\TeX,\LaTeX,\pdfLaTeX}
% \DoNotIndex{\section,\color,\pagenumbering,\smallskip}
% \DoNotIndex{\@ifundefined,\@ifpackageloaded}
% \DoNotIndex{\DeclareStringOption}
%
% \title{The \xpackage{pagecolor} package}
% \date{2012/02/23 v1.0e}
% \author{H.-Martin M\"{u}nch\\\xemail{Martin.Muench at Uni-Bonn.de}}
%
% \maketitle
%
% \begin{abstract}
%  This \LaTeX{} package provides the command |\thepagecolor|,
% which gives the current page (background) colour,
% i.\,e.~the argument used with the most recent call of |\pagecolor{...}|.
% The command |\thepagecolornone| gives the same colour as |\thepagecolor|,
% except when the page background colour is
% \textquotedblleft|none|\textquotedblright. In that case |\thepagecolor| is
% |white| and |\thepagecolornone| is |none|.\\
% Similar to |\newgeometry| and |\restoregeometry| of the \xpackage{geometry}
% package\\
% |\newpagecolor{<|\textit{some colour} |>}| and |\restorepagecolor|
% are provided.
% \end{abstract}
%
% \bigskip
%
% \noindent Disclaimer for web links: The author is not responsible for any contents
% referred to in this work unless he has full knowledge of illegal contents.
% If any damage occurs by the use of information presented there, only the
% author of the respective pages might be liable, not the one who has referred
% to these pages.
%
% \bigskip
%
% \noindent {\color{green} Save per page about $200\unit{ml}$ water,
% $2\unit{g}$ CO$_{2}$ and $2\unit{g}$ wood:\\
% Therefore please print only if this is really necessary.}
%
% \pagebreak
%
% \tableofcontents
%
% \bigskip
%
% \section{Introduction\label{sec:Introduction}}
%
% \indent This \LaTeX{} package provides the command |\thepagecolor|,
% which gives the current page (background) colour,
% i.\,e.~the argument used with the most recent call of |\pagecolor{...}|.
% The package should be loaded before any package sets a page (background)
% colour, but after \xpackage{color} or \xpackage{xcolor} package.
% Its option |pagecolor={...}| is used to set the |\pagecolor{...}|.\\
% The command |\thepagecolornone| gives the same colour as |\thepagecolor|,
% except when the page background colour is
% \textquotedblleft|none|\textquotedblright (e.\,g. result of using the
% |\nopagecolor| command). In that case |\thepagecolor| is |white| and\linebreak
% |\thepagecolornone| is |none|.\\
% Similar to |\newgeometry| and |\restoregeometry| of the \xpackage{geometry}
% package\\
% |\newpagecolor{<|\textit{some colour}|>}| and |\restorepagecolor|
% are provided.
%
% \pagebreak
%
% \section{Usage}
%
% \indent Just load the package placing
% \begin{quote}
%   |\usepackage[<|\textit{option}|>]{pagecolor}|
% \end{quote}
% \noindent in the preamble of your \LaTeXe{} source file.
% This should be done before another package uses |\pagecolor|,
% but after |\nopagecolor| is defined (if that is defined in the document
% at all).\\
% Afterwards |\pagecolor{...}| can be used to change the page (background)
% colour as usual. Then |\thepagecolor| gives the current page (background)
% colour (in the same format as given with |\pagecolor{...}|).\\
% Similar to |\newgeometry| and |\restoregeometry| of the \xpackage{geometry}
% package\\
% |\newpagecolor{<|\textit{some colour} |>}| and |\restorepagecolor|
% are provided:\\
% |\newpagecolor{<|\textit{some colour} |>}| will execute
% |\pagecolor{<|\textit{some colour} |>}| and remember the page colour used before.
% |\restorecolor| (without argument) restores the page colour to the one used before
%  use of the |\newpagecolor{...}| command. When you want to change the colour for
%  just one page and do not want to (or cannot) manually determine where the page ends,\\
% |\newpagecolor{<|\textit{some colour}|>}\afterpage{\restorepagecolor}|\\
% does the trick (and requires a |\usepackage{afterpage}| in the document's
% preamble), or for short\\
% |\newcommand{\onepagecolor}[1]{%|\\
% |\newpagecolor{#1}\afterpage{\restorepagecolor}}|\\
% in the preamble and\\
% |\onepagecolor{<|\textit{some colour}|>}| in the document.
%
% \subsection{Option}
% \DescribeMacro{option}
% \indent The \xpackage{pagecolor} package takes the following option:
%
% \subsubsection{pagecolor\label{sss:pagecolor}}
% \DescribeMacro{pagecolor}
% \indent The option |pagecolor={...}| takes as value a colour. This
% could be as simple as |white| or |black|, but when e.\,g.~the
% \xpackage{xcolor} package is used (loaded before \xpackage{pagecolor}!),
% also colours like |red!50!green!20!blue| are possible.
% The default is |pagecolor={none}|.
%
% \section{Alternatives\label{sec:Alternatives}}
%
% As I neither know what exactly you want to accomplish
% when using this package (e.\,g.~hiding text), nor what resources you have
% (e.\,g.~\pdfTeX{} version), here is a list of some possible alternatives:\\
%
% \begin{description}
%  \item[-]
%  \DescribeMacro{transparent}
%  \xpackage{transparent} package: With it some object can be made
%  (fully or partially) transparent.
% \end{description}
%
% \begin{description}
%  \item[-]
%  \DescribeMacro{hrefhide}\xpackage{hrefhide} package: It allows to
%  \textquotedblleft hide\textquotedblright\ some (hyperlinked) text when
%  printing the document while keeping the layout.
% \end{description}
%
% \bigskip
%
% \noindent (You programmed or found another alternative,
%  which is available at \CTAN{}?\\
%  OK, send an e-mail to me with the name, location at \CTAN{},
%  and a short notice, and I will probably include it in
%  the list above.)\\
%
% \smallskip
%
% \noindent About how to get those packages, please see
% subsection~\ref{ss:Downloads}.
%
% \newpage
%
% \section{Example}
%
%    \begin{macrocode}
%<*example>
\documentclass[british]{article}[2007/10/19]% v1.4h
%%%%%%%%%%%%%%%%%%%%%%%%%%%%%%%%%%%%%%%%%%%%%%%%%%%%%%%%%%%%%%%%%%%%%
\usepackage[%
 extension=pdf,%
 plainpages=false,%
 pdfpagelabels=true,%
 hyperindex=false,%
 pdflang={en},%
 pdftitle={pagecolor package example},%
 pdfauthor={H.-Martin Muench},%
 pdfsubject={Example for the pagecolor package},%
 pdfkeywords={LaTeX, pagecolor, thepagecolor, page colour,%
  H.-Martin Muench},%
 pdfview=Fit,pdfstartview=Fit,%
 pdfpagelayout=SinglePage%
]{hyperref}[2012/02/06]% v6.82o
\usepackage[x11names]{xcolor}[2007/01/21]% v2.11
 % The xcolor package would not be needed for just using
 % the base colours. The color package would be sufficient for that.
\usepackage[pagecolor={LightGoldenrod1}]{pagecolor}[2012/02/23]% v1.0e

\usepackage{afterpage}[1995/10/27]% v1.08
 % The afterpage package is generally not needed,
 % but the |\newpagecolor{somecolour}\afterpage{\restorepagecolor}|
 % construct shall be demonstrated.

\usepackage{lipsum}[2011/04/14]% v1.2
 % The lipsum package is generally not needed,
 % but some blind text is needed for the example.

\gdef\unit#1{\mathord{\thinspace\mathrm{#1}}}%
\listfiles
\begin{document}
\pagenumbering{arabic}
\section*{Example for pagecolor}

This example demonstrates the use of package\newline
\textsf{pagecolor}, v1.0e as of 2012/02/23 (HMM).\newline
The used option was \verb|pagecolor={LightGoldenrod1}|.\newline
\verb|pagecolor={none}| would be the default.\newline

For more details please see the documentation!\newline

\noindent {\color{teal} Save per page about $200\unit{ml}$ water,
$2\unit{g}$ CO$_{2}$ and $2\unit{g}$ wood:\newline
Therefore please print only if this is really necessary.}\newline

The current page (background) colour is\newline
\verb|\thepagecolor|\ =\ \thepagecolor \newline
(and \verb|\thepagecolornone|\ =\ \thepagecolornone ,
which would only be different from \verb|\thepagecolor|,
when the page colour would be \verb|none|).

\pagebreak
\pagecolor{rgb:-green!40!yellow,3;green!40!yellow,2;red,1}

{\color{white} The current page (background) colour is\newline
\verb|\thepagecolor|\ =\ \thepagecolor . \newline}

{\color{\thepagecolor} And that makes this text practically invisible.
\newline}

{\color{white} Which made the preceding line of text practically
invisible.}

\pagebreak
\newpagecolor{red}

This page uses \verb|\newpagecolor{red}|.

\pagebreak
\restorepagecolor

{\color{white}And this page uses \verb|\restorepagecolor| to restore
the page colour to the value it had before the red page.}

\pagebreak
\pagecolor{none}

This page uses \verb|\pagecolor{none}|. If the \verb|\nopagecolor|
command is known (pdf\TeX and Lua\TeX; not yet for dvips, dvipdfm(x)
or Xe\TeX), the page colour is now \verb|none|, otherwise \verb|white|:
\verb|\thepagecolor|\ =\ \thepagecolor\ and
\verb|\thepagecolornone|\ =\ \thepagecolornone .

\pagebreak
\restorepagecolor

{\color{white}\verb|\restorepagecolor| restored the page colour again.}

\pagebreak
\pagecolor{green}

This page is green due to \verb|\pagecolor{green}|.

\pagebreak
\newpagecolor{blue}\afterpage{\restorepagecolor}

{\color{white}\verb|\newpagecolor{blue}\afterpage{\restorepagecolor}|%
\newline
was used here, i.\,e.~this page is blue, and the next one will
automatically have the same page colour before it was changed to blue
here (i.\,e. green).}

\smallskip
{\color{red}\textbf{\lipsum[1-11]}}
\bigskip

The page colour was changed back at the end of the page -
in mid-sentence!

\end{document}
%</example>
%    \end{macrocode}
%
% \StopEventually{}
%
% \newpage
%
% \section{The implementation}
%
% We start off by checking that we are loading into \LaTeXe{} and
% announcing the name and version of this package.
%
%    \begin{macrocode}
%<*package>
%    \end{macrocode}
%
%    \begin{macrocode}
\NeedsTeXFormat{LaTeX2e}[2011/06/27]
\ProvidesPackage{pagecolor}[2012/02/23 v1.0e
            Provides thepagecolor (HMM)]
%    \end{macrocode}
%
% A short description of the \xpackage{pagecolor} package:
%
%    \begin{macrocode}
%% Provides the \thepagecolor, \thepagecolornone, \newpagecolor{...},
%% and \restorepagecolor commands.
%    \end{macrocode}
%
% We need the \xpackage{kvoptions} package by \textsc{Heiko Oberdiek}:
%
%    \begin{macrocode}
\RequirePackage{kvoptions}[2010/12/23]% v3.10
%    \end{macrocode}
%
% and either the \xpackage{color} or the \xpackage{xcolor} package:
%
%    \begin{macrocode}
%% \RequirePackage{ either color or xcolor }:
\@ifpackageloaded{xcolor}{% xcolor loaded
   \@ifpackagelater{xcolor}{2007/01/21}{%
      % 2007/01/21, v2.11, or even more recent: OK
    }{% else: older package version
      \PackageWarning{pagecolor}{%
        It is requested version `2007/01/21' of package\MessageBreak%
        xcolor, but only an older version is available.\MessageBreak%
       }%
     }%
 }{% xcolor not loaded
   \@ifpackageloaded{color}{%
      \RequirePackage{color}[2005/11/14]% v1.0j
    }{% \else
      \PackageWarning{pagecolor}{%
        The pagecolor package must be loaded after either\MessageBreak%
        package color or after package xcolor (at your\MessageBreak%
        option). Neither package was loaded before package\MessageBreak%
        pagecolor. Loading of package xcolor will now be\MessageBreak%
        tried automatically.\MessageBreak%
        When the pagecolor package is used with option\MessageBreak%
        pagecolor using a colour requiring e. g. x11names\MessageBreak%
        option for xcolor package, this will not work!\MessageBreak%
       }
     }% \fi
   \RequirePackage{xcolor}[2007/01/21]% v2.11
  }% \fi
%    \end{macrocode}
%
% A last information for the user:
%
%    \begin{macrocode}
%% pagecolor may work with earlier versions of LaTeX and the
%% packages, but this was not tested. Please consider updating
%% your LaTeX and packages to the most recent version
%% (if they are not already the most recent version).

%    \end{macrocode}
%
% \indent See subsection~\ref{ss:Downloads} about how to get them.\\
%
% We process the options:
%
%    \begin{macrocode}
\SetupKeyvalOptions{family=pagecolor,prefix=pagecolor@}
\DeclareStringOption[none]{pagecolor}% \pagecolor@pagecolor
\ProcessKeyvalOptions*

%    \end{macrocode}
%
% We save the original |\pagecolor| command,
%
%    \begin{macrocode}
\let\origpagecolour\pagecolor

%    \end{macrocode}
%
%    \begin{macro}{\pagecolor}
% before we redefine it to include a definition of |\thepagecolor|
% and \linebreak
% |\thepagecolornone|:
%
%    \begin{macrocode}
\renewcommand{\pagecolor}[1]{\@bsphack%
  \edef\pagecolourtmpa{#1}%
  \def\pagecolourtmpb{none}%
  \ifx\pagecolourtmpa\pagecolourtmpb
    \@ifundefined{nopagecolor}{%
      \PackageError{pagecolor}{%
        pagecolor=none requested but \string\nopagecolor\space unknown%
       }{%
        \string\pagecolor{none} was used, but the command %
        \string\nopagecolor\space is undefined.\MessageBreak%
        Please use another colour. pagecolor=white will be used now.%
        \MessageBreak%
        }%
      \gdef\thepagecolor{white}%
      \gdef\thepagecolornone{white}% although it should be "none"
      \origpagecolour{white}%
     }{%
      \nopagecolor%
     }%
  \else%
    \xdef\thepagecolor{#1}%
    \xdef\thepagecolornone{#1}%
    \origpagecolour{\thepagecolornone}%
  \fi%
  \@esphack%
}

%    \end{macrocode}
%    \end{macro}
%
% |\nopagecolor| is only defined for \pdfTeX and \LuaTeX, but not yet for dvips,
% dvipdfm(x) or \XeTeX. Maybe
% \begin{verbatim}
% \@ifundefined{nopagecolor}{\newcommand{\nopagecolor}{\pagecolor{white}}}{}
% \end{verbatim}
% could be an alternative.\\
% When |\pagecolor{none}| is used and |\nopagecolor| is defined, then \\
% |\pagecolor{none}| is made into a synonym for |\nopagecolor|.\\
% If |\nopagecolor| has not been defined, nothing needs to be done.
% Otherwise we redefine |\nopagecolor|, thus that |\thepagecolor| is
% set to |white| and |\thepagecolornone| to |none| whenever
% |\nopagecolor| is used.
%
%    \begin{macrocode}
\@ifundefined{nopagecolor}{%
  \PackageWarning{pagecolor}{%
    \string\nopagecolor\space is undefined.\MessageBreak%
    (As long as it is neither used anywhere\MessageBreak%
    \space nor defined later, that will be no problem.)\MessageBreak%
   }
  \AtEndDocument{%
  \@ifundefined{nopagecolor}{% no problem
    }{%
    \PackageError{pagecolor}{%
      \string\nopagecolor\space defined after loading package%
      \MessageBreak%
      pagecolor%
      }{\string\nopagecolor\space was defined after loading the %
       pagecolor package.\MessageBreak%
       Either define \string\nopagecolor\space earlier or load the %
       pagecolor package later.\MessageBreak%
      }
    }%
  }%
  }{% \else
    \let\orignopagecolour\nopagecolor
    \renewcommand{\nopagecolor}{%
      \xdef\thepagecolor{white}
      \xdef\thepagecolornone{none}
      \orignopagecolour
      }
  }

%    \end{macrocode}
%
% If the page colour as given with option |pagecolor={...}| is |none|,
% but |\nopagecolor| is not known, |\pagecolor@pagecolor| is set to
% |white| and a warning is given.
%
%    \begin{macrocode}
\def\pagecolourtmpb{none}
\ifx\pagecolor@pagecolor\pagecolourtmpb
  \@ifundefined{nopagecolor}{%
    \PackageWarning{pagecolor}{%
      Option pagecolor=none (maybe by default) used,\MessageBreak%
      but \string\nopagecolor\space is unknown. Please use another%
      \MessageBreak%
      option value; white will be used now.\MessageBreak%
     }
    \setkeys{pagecolor}{pagecolor=white}%
  }{% ok
  }%
\fi

%    \end{macrocode}
%
% The (new) |\pagecolor| is now just carried out.
%
%    \begin{macrocode}
\pagecolor{\pagecolor@pagecolor}

%    \end{macrocode}
%
% Now the page (background) colour and |\thepagecolor| and \\
% |\thepagecolornone| are |\pagecolor@pagecolor| (or page (background) colour
% and |\thepagecolornone| are |none|=|\pagecolor@pagecolor| and
% |\thepagecolor| is |white|), and when the page (background) colour
% is changed, |\thepagecolor| and |\thepagecolornone| are changed
% accordingly.
%
% \newpage
%
%    \begin{macro}{\newpagecolor}
% There have been requests (via e-mail and at \\
% \url{http://tex.stackexchange.com/q/25137}) to change the colour of just one
% (or two) page(s) only, similar to |\newgeometry| and |\restoregeometry| of the
% \xpackage{geometry} package (\url{http://ctan.org/pkg/geometry}). Therefore
% |\newpagecolor| and |\restorepagecolor| are introduced (as suggested by
% \textsc{haoyun\_tex}):
%
%    \begin{macrocode}
\newcommand{\newpagecolor}[1]{%
\xdef\pagecolourtmpc{\thepagecolornone}%
\pagecolor{#1}%
}

%    \end{macrocode}
%    \end{macro}
%
% |\newpagecolor{<|\textit{some colour}|>}| will execute |\pagecolor{somecolour}|
% and remember the page colour used before.\\
%
%    \begin{macro}{\restorepagecolor}
%    \begin{macrocode}
\newcommand{\restorepagecolor}{\pagecolor{\pagecolourtmpc}}

%    \end{macrocode}
%    \end{macro}
%
% |\restorecolor| (without argument) restores the page colour to the one used before
% use of the |\newpagecolor{...}| command.
%
%    \begin{macrocode}
\gdef\pagecolourtmpc{\thepagecolor}

%    \end{macrocode}
% is just a precaution for |\restorecolor| being used when no |\newpagecolor{...}|
% was used before it.\\
%
% When you want to change the colour for just one page and do not want to
% (or cannot) manually determine where the page ends,\\
% |\newpagecolor{<|\textit{some colour}|>}\afterpage{\restorepagecolor}|\\
% does the trick (and requires an additional |\usepackage{afterpage}| in the
% document's preamble).
%    \begin{macrocode}
%</package>
%    \end{macrocode}
%
% \newpage
%
% \section{Installation}
%
% \subsection{Downloads\label{ss:Downloads}}
%
% Everything is available on \CTAN{}, \url{http://www.ctan.org/tex-archive/},
% but may need additional packages themselves.\\
%
% \DescribeMacro{pagecolor.dtx}
% For unpacking the |pagecolor.dtx| file and constructing the documentation
% it is required:
% \begin{description}
% \item[-] \TeX{} Format \LaTeXe{}, \url{http://www.CTAN.org/}
%
% \item[-] document class \xpackage{ltxdoc}, 2007/11/11, v2.0u,
%           \url{http://ctan.org/pkg/ltxdoc}
%
% \item[-] package \xpackage{holtxdoc}, 2011/02/04, v0.21,
%           \url{http://ctan.org/pkg/holtxdoc}
% \end{description}
%
% \DescribeMacro{pagecolor.sty}
% The |pagecolor.sty| for \LaTeXe{} (i.\,e. each document using
% the \xpackage{pagecolor} package) requires:
% \begin{description}
% \item[-] \TeX{} Format \LaTeXe{}, \url{http://www.CTAN.org/}
%
% \item[-] package \xpackage{kvoptions}, 2010/12/23, v3.10,
%           \url{http://ctan.org/pkg/kvoptions}
% \end{description}
% and either
% \begin{description}
% \item[-] package \xpackage{color}, 2005/11/14, v1.0j,
%           \url{http://ctan.org/pkg/color} (from the\\
%           \xpackage{graphics} package bundle)
% \end{description}
% or
% \begin{description}
% \item[-] package \xpackage{xcolor}, 2007/01/21, v2.11,
%           \url{http://ctan.org/pkg/xcolor}
% \end{description}
%
% \DescribeMacro{pagecolor-example.tex}
% The |pagecolor-example.tex| requires the same files as all
% documents using the \xpackage{pagecolor} package and additionally:
% \begin{description}
% \item[-] class \xpackage{article}, 2007/10/19, v1.4h, from \xpackage{classes.dtx}:\\
%           \CTAN{macros/latex/base/classes.dtx}
%
% \item[-] package \xpackage{pagecolor}, 2012/02/23, v1.0e,
%           \url{http://ctan.org/pkg/pagecolor}\\
%           (Well, it is the example file for this package, and because you are
%            reading the documentation for the \xpackage{pagecolor} package,
%            it can be assumed that you already have some version of it --
%            is it the current one?)
%
% \item[-] package \xpackage{xcolor}, 2007/01/21, v2.11,
%           \url{http://ctan.org/pkg/xcolor}\\
%           This package would not be needed for the use of just base colours only,
%           the \xpackage{color} package would be sufficient for that.
% \end{description}
%
% \DescribeMacro{transparent}
% \DescribeMacro{hrefhide}
% As possible alternatives in section \ref{sec:Alternatives} there are listed
% \begin{description}
% \item[-] package \xpackage{transparent}, 2007/01/08, v1.0, \\
%           \url{http://ctan.org/pkg/transparent}
%
% \item[-] package \xpackage{hrefhide}, 2011/04/29, v1.0f, \\
%           \url{http://ctan.org/pkg/hrefhide}
% \end{description}
%
% \DescribeMacro{Oberdiek}
% \DescribeMacro{holtxdoc}
% \DescribeMacro{kvoptions}
% All packages of \textsc{Heiko Oberdiek}'s bundle `oberdiek'
% (especially \xpackage{holtxdoc} and \xpackage{kvoptions})
% are also available in a TDS compliant ZIP archive:\\
% \CTAN{install/macros/latex/contrib/oberdiek.tds.zip}.\\
% It is probably best to download and use this, because the packages in there
% are quite probably both recent and compatible among themselves.\\
%
% \DescribeMacro{hyperref}
% \noindent \xpackage{hyperref} is not included in that bundle and needs to be
% downloaded separately,\\
% \url{http://mirror.ctan.org/install/macros/latex/contrib/hyperref.tds.zip}.\\
%
% \DescribeMacro{M\"{u}nch}
% A hyperlinked list of my (other) packages can be found at
% \url{http://www.Uni-Bonn.de/~uzs5pv/LaTeX.html}.\\
%
% \subsection{Package, unpacking TDS}
% \paragraph{Package.} This package is available on \CTAN{}
% \begin{description}
% \item[\CTAN{macros/latex/contrib/pagecolor/pagecolor.dtx}]\hspace*{0.1cm} \\
%       The source file.
% \item[\CTAN{macros/latex/contrib/pagecolor/pagecolor.pdf}]\hspace*{0.1cm} \\
%       The documentation.
% \item[\CTAN{macros/latex/contrib/pagecolor/pagecolor-example.pdf}]\hspace*{0.1cm} \\
%       The compiled example file, as it should look like.
% \item[\CTAN{macros/latex/contrib/pagecolor/README}]\hspace*{0.1cm} \\
%       The README file.
% \end{description}
% There is also a pagecolor.tds.zip available:
% \begin{description}
% \item[\CTAN{install/macros/latex/contrib/pagecolor.tds.zip}]\hspace*{0.1cm} \\
%       Everything in \xfile{TDS} compliant, compiled format.
% \end{description}
% which additionally contains\\
% \begin{tabular}{ll}
% pagecolor.ins & The installation file.\\
% pagecolor.drv & The driver to generate the documentation.\\
% pagecolor.sty & The \xext{sty}le file.\\
% pagecolor-example.tex & The example file.
% \end{tabular}
%
% \bigskip
%
% \noindent For required other packages, please see the preceding subsection.
%
% \paragraph{Unpacking.} The \xfile{.dtx} file is a self-extracting
% \docstrip{} archive. The files are extracted by running the
% \xext{.dtx} through \plainTeX{}:
% \begin{quote}
%   \verb|tex pagecolor.dtx|
% \end{quote}
%
% About generating the documentation see paragraph~\ref{GenDoc} below.\\
%
% \paragraph{TDS.} Now the different files must be moved into
% the different directories in your installation TDS tree
% (also known as \xfile{texmf} tree):
% \begin{quote}
% \def\t{^^A
% \begin{tabular}{@{}>{\ttfamily}l@{ $\rightarrow$ }>{\ttfamily}l@{}}
%   pagecolor.sty & tex/latex/pagecolor/pagecolor.sty\\
%   pagecolor.pdf & doc/latex/pagecolor/pagecolor.pdf\\
%   pagecolor-example.tex & doc/latex/pagecolor/pagecolor-example.tex\\
%   pagecolor-example.pdf & doc/latex/pagecolor/pagecolor-example.pdf\\
%   pagecolor.dtx & source/latex/pagecolor/pagecolor.dtx\\
% \end{tabular}^^A
% }^^A
% \sbox0{\t}^^A
% \ifdim\wd0>\linewidth
%   \begingroup
%     \advance\linewidth by\leftmargin
%     \advance\linewidth by\rightmargin
%   \edef\x{\endgroup
%     \def\noexpand\lw{\the\linewidth}^^A
%   }\x
%   \def\lwbox{^^A
%     \leavevmode
%     \hbox to \linewidth{^^A
%       \kern-\leftmargin\relax
%       \hss
%       \usebox0
%       \hss
%       \kern-\rightmargin\relax
%     }^^A
%   }^^A
%   \ifdim\wd0>\lw
%     \sbox0{\small\t}^^A
%     \ifdim\wd0>\linewidth
%       \ifdim\wd0>\lw
%         \sbox0{\footnotesize\t}^^A
%         \ifdim\wd0>\linewidth
%           \ifdim\wd0>\lw
%             \sbox0{\scriptsize\t}^^A
%             \ifdim\wd0>\linewidth
%               \ifdim\wd0>\lw
%                 \sbox0{\tiny\t}^^A
%                 \ifdim\wd0>\linewidth
%                   \lwbox
%                 \else
%                   \usebox0
%                 \fi
%               \else
%                 \lwbox
%               \fi
%             \else
%               \usebox0
%             \fi
%           \else
%             \lwbox
%           \fi
%         \else
%           \usebox0
%         \fi
%       \else
%         \lwbox
%       \fi
%     \else
%       \usebox0
%     \fi
%   \else
%     \lwbox
%   \fi
% \else
%   \usebox0
% \fi
% \end{quote}
% If you have a \xfile{docstrip.cfg} that configures and enables \docstrip{}'s
% \xfile{TDS} installing feature, then some files can already be in the right
% place, see the documentation of \docstrip{}.
%
% \subsection{Refresh file name databases}
%
% If your \TeX{}~distribution (\teTeX{}, \mikTeX{},\dots{}) relies on file name
% databases, you must refresh these. For example, \teTeX{} users run
% \verb|texhash| or \verb|mktexlsr|.
%
% \subsection{Some details for the interested}
%
% \paragraph{Unpacking with \LaTeX{}.}
% The \xfile{.dtx} chooses its action depending on the format:
% \begin{description}
% \item[\plainTeX:] Run \docstrip{} and extract the files.
% \item[\LaTeX:] Generate the documentation.
% \end{description}
% If you insist on using \LaTeX{} for \docstrip{} (really,
% \docstrip{} does not need \LaTeX{}), then inform the autodetect routine
% about your intention:
% \begin{quote}
%   \verb|latex \let\install=y% \iffalse meta-comment
%
% File: pagecolor.dtx
% Version: 2012/02/23 v1.0e
%
% Copyright (C) 2011 - 2012 by
%    H.-Martin M"unch <Martin dot Muench at Uni-Bonn dot de>
%
% This work may be distributed and/or modified under the
% conditions of the LaTeX Project Public License, either
% version 1.3c of this license or (at your option) any later
% version. This version of this license is in
%    http://www.latex-project.org/lppl/lppl-1-3c.txt
% and the latest version of this license is in
%    http://www.latex-project.org/lppl.txt
% and version 1.3c or later is part of all distributions of
% LaTeX version 2005/12/01 or later.
%
% This work has the LPPL maintenance status "maintained".
%
% The Current Maintainer of this work is H.-Martin Muench.
%
% This work consists of the main source file pagecolor.dtx,
% the README, and the derived files
%    pagecolor.sty, pagecolor.pdf,
%    pagecolor.ins, pagecolor.drv,
%    pagecolor-example.tex, pagecolor-example.pdf.
%
% Distribution:
%    CTAN:macros/latex/contrib/pagecolor/pagecolor.dtx
%    CTAN:macros/latex/contrib/pagecolor/pagecolor.pdf
%    CTAN:install/macros/latex/contrib/pagecolor.tds.zip
%
% Cf. http://ctan.org/pkg/pagecolor and
% http://www.tex.ac.uk/tex-archive/help/Catalogue/entries/pagecolor.html
%
% Unpacking:
%    (a) If pagecolor.ins is present:
%           tex pagecolor.ins
%    (b) Without pagecolor.ins:
%           tex pagecolor.dtx
%    (c) If you insist on using LaTeX
%           latex \let\install=y\input{pagecolor.dtx}
%        (quote the arguments according to the demands of your shell)
%
% Documentation:
%    (a) If pagecolor.drv is present:
%           (pdf)latex pagecolor.drv
%           makeindex -s gind.ist pagecolor.idx
%           (pdf)latex pagecolor.drv
%           makeindex -s gind.ist pagecolor.idx
%           (pdf)latex pagecolor.drv
%    (b) Without pagecolor.drv:
%           (pdf)latex pagecolor.dtx
%           makeindex -s gind.ist pagecolor.idx
%           (pdf)latex pagecolor.dtx
%           makeindex -s gind.ist pagecolor.idx
%           (pdf)latex pagecolor.dtx
%
%    The class ltxdoc loads the configuration file ltxdoc.cfg
%    if available. Here you can specify further options, e.g.
%    use DIN A4 as paper format:
%       \PassOptionsToClass{a4paper}{article}
%
% Installation:
%    TDS:tex/latex/pagecolor/pagecolor.sty
%    TDS:doc/latex/pagecolor/pagecolor.pdf
%    TDS:doc/latex/pagecolor/pagecolor-example.tex
%    TDS:doc/latex/pagecolor/pagecolor-example.pdf
%    TDS:source/latex/pagecolor/pagecolor.dtx
%
%<*ignore>
\begingroup
  \catcode123=1 %
  \catcode125=2 %
  \def\x{LaTeX2e}%
\expandafter\endgroup
\ifcase 0\ifx\install y1\fi\expandafter
         \ifx\csname processbatchFile\endcsname\relax\else1\fi
         \ifx\fmtname\x\else 1\fi\relax
\else\csname fi\endcsname
%</ignore>
%<*install>
\input docstrip.tex
\Msg{*****************************************************************}
\Msg{* Installation                                                   }
\Msg{* Package: pagecolor 2012/02/23 v1.0e Provides thepagecolor (HMM)}
\Msg{*****************************************************************}

\keepsilent
\askforoverwritefalse

\let\MetaPrefix\relax
\preamble

This is a generated file.

Project: pagecolor
Version: 2012/02/23 v1.0e

Copyright (C) 2011 - 2012 by
    H.-Martin M"unch <Martin dot Muench at Uni-Bonn dot de>

The usual disclaimer applies:
If it doesn't work right that's your problem.
(Nevertheless, send an e-mail to the maintainer
 when you find an error in this package.)

This work may be distributed and/or modified under the
conditions of the LaTeX Project Public License, either
version 1.3c of this license or (at your option) any later
version. This version of this license is in
   http://www.latex-project.org/lppl/lppl-1-3c.txt
and the latest version of this license is in
   http://www.latex-project.org/lppl.txt
and version 1.3c or later is part of all distributions of
LaTeX version 2005/12/01 or later.

This work has the LPPL maintenance status "maintained".

The Current Maintainer of this work is H.-Martin Muench.

This work consists of the main source file pagecolor.dtx,
the README, and the derived files
   pagecolor.sty, pagecolor.pdf,
   pagecolor.ins, pagecolor.drv,
   pagecolor-example.tex, pagecolor-example.pdf.

\endpreamble
\let\MetaPrefix\DoubleperCent

\generate{%
  \file{pagecolor.ins}{\from{pagecolor.dtx}{install}}%
  \file{pagecolor.drv}{\from{pagecolor.dtx}{driver}}%
  \usedir{tex/latex/pagecolor}%
  \file{pagecolor.sty}{\from{pagecolor.dtx}{package}}%
  \usedir{doc/latex/pagecolor}%
  \file{pagecolor-example.tex}{\from{pagecolor.dtx}{example}}%
}

\catcode32=13\relax% active space
\let =\space%
\Msg{************************************************************************}
\Msg{*}
\Msg{* To finish the installation you have to move the following}
\Msg{* file into a directory searched by TeX:}
\Msg{*}
\Msg{*     pagecolor.sty}
\Msg{*}
\Msg{* To produce the documentation run the file `pagecolor.drv'}
\Msg{* through (pdf)LaTeX, e.g.}
\Msg{*  pdflatex pagecolor.drv}
\Msg{*  makeindex -s gind.ist pagecolor.idx}
\Msg{*  pdflatex pagecolor.drv}
\Msg{*  makeindex -s gind.ist pagecolor.idx}
\Msg{*  pdflatex pagecolor.drv}
\Msg{*}
\Msg{* At least three runs are necessary e.g. to get the}
\Msg{*  references right!}
\Msg{*}
\Msg{* Happy TeXing!}
\Msg{*}
\Msg{************************************************************************}

\endbatchfile
%</install>
%<*ignore>
\fi
%</ignore>
%
% \section{The documentation driver file}
%
% The next bit of code contains the documentation driver file for
% \TeX{}, i.\,e., the file that will produce the documentation you
% are currently reading. It will be extracted from this file by the
% \texttt{docstrip} programme. That is, run \LaTeX{} on \texttt{docstrip}
% and specify the \texttt{driver} option when \texttt{docstrip}
% asks for options.
%
%    \begin{macrocode}
%<*driver>
\NeedsTeXFormat{LaTeX2e}[2011/06/27]
\ProvidesFile{pagecolor.drv}%
  [2012/02/23 v1.0e Provides thepagecolor (HMM)]
\documentclass{ltxdoc}[2007/11/11]% v2.0u
\usepackage{holtxdoc}[2011/02/04]%  v0.21
%% pagecolor may work with earlier versions of LaTeX2e and those
%% class and package, but this was not tested.
%% Please consider updating your LaTeX, class, and package
%% to the most recent version (if they are not already the most
%% recent version).
\hypersetup{%
 pdfsubject={Providing the thepagecolor command (HMM)},%
 pdfkeywords={LaTeX, pagecolor, thepagecolor, H.-Martin Muench},%
 pdfencoding=auto,%
 pdflang={en},%
 breaklinks=true,%
 linktoc=all,%
 pdfstartview=FitH,%
 pdfpagelayout=OneColumn,%
 bookmarksnumbered=true,%
 bookmarksopen=true,%
 bookmarksopenlevel=3,%
 pdfmenubar=true,%
 pdftoolbar=true,%
 pdfwindowui=true,%
 pdfnewwindow=true%
}
\CodelineIndex
\hyphenation{printing docu-ment}
\gdef\unit#1{\mathord{\thinspace\mathrm{#1}}}%
\begin{document}
  \DocInput{pagecolor.dtx}%
\end{document}
%</driver>
%    \end{macrocode}
%
% \fi
%
% \CheckSum{137}
%
% \CharacterTable
%  {Upper-case    \A\B\C\D\E\F\G\H\I\J\K\L\M\N\O\P\Q\R\S\T\U\V\W\X\Y\Z
%   Lower-case    \a\b\c\d\e\f\g\h\i\j\k\l\m\n\o\p\q\r\s\t\u\v\w\x\y\z
%   Digits        \0\1\2\3\4\5\6\7\8\9
%   Exclamation   \!     Double quote  \"     Hash (number) \#
%   Dollar        \$     Percent       \%     Ampersand     \&
%   Acute accent  \'     Left paren    \(     Right paren   \)
%   Asterisk      \*     Plus          \+     Comma         \,
%   Minus         \-     Point         \.     Solidus       \/
%   Colon         \:     Semicolon     \;     Less than     \<
%   Equals        \=     Greater than  \>     Question mark \?
%   Commercial at \@     Left bracket  \[     Backslash     \\
%   Right bracket \]     Circumflex    \^     Underscore    \_
%   Grave accent  \`     Left brace    \{     Vertical bar  \|
%   Right brace   \}     Tilde         \~}
%
% \GetFileInfo{pagecolor.drv}
%
% \begingroup
%   \def\x{\#,\$,\^,\_,\~,\ ,\&,\{,\},\%}%
%   \makeatletter
%   \@onelevel@sanitize\x
% \expandafter\endgroup
% \expandafter\DoNotIndex\expandafter{\x}
% \expandafter\DoNotIndex\expandafter{\string\ }
% \begingroup
%   \makeatletter
%     \lccode`9=32\relax
%     \lowercase{%^^A
%       \edef\x{\noexpand\DoNotIndex{\@backslashchar9}}%^^A
%     }%^^A
%   \expandafter\endgroup\x
%
% \DoNotIndex{\,,\\}
% \DoNotIndex{\documentclass,\usepackage,\ProvidesPackage,\begin,\end}
% \DoNotIndex{\MessageBreak}
% \DoNotIndex{\NeedsTeXFormat,\DoNotIndex,\verb}
% \DoNotIndex{\def,\edef,\gdef,\global}
% \DoNotIndex{\ifx,\listfiles,\mathord,\mathrm}
% \DoNotIndex{\kvoptions,\SetupKeyvalOptions,\ProcessKeyvalOptions}
% \DoNotIndex{\bigskip,\space,\thinspace,\ldots}
% \DoNotIndex{\indent,\noindent,\newline,\linebreak,\pagebreak,\newpage}
% \DoNotIndex{\textbf,\textit,\textsf,\texttt,\textsc,\textquotedblleft,\textquotedblright}
% \DoNotIndex{\plainTeX,\TeX,\LaTeX,\pdfLaTeX}
% \DoNotIndex{\section,\color,\pagenumbering,\smallskip}
% \DoNotIndex{\@ifundefined,\@ifpackageloaded}
% \DoNotIndex{\DeclareStringOption}
%
% \title{The \xpackage{pagecolor} package}
% \date{2012/02/23 v1.0e}
% \author{H.-Martin M\"{u}nch\\\xemail{Martin.Muench at Uni-Bonn.de}}
%
% \maketitle
%
% \begin{abstract}
%  This \LaTeX{} package provides the command |\thepagecolor|,
% which gives the current page (background) colour,
% i.\,e.~the argument used with the most recent call of |\pagecolor{...}|.
% The command |\thepagecolornone| gives the same colour as |\thepagecolor|,
% except when the page background colour is
% \textquotedblleft|none|\textquotedblright. In that case |\thepagecolor| is
% |white| and |\thepagecolornone| is |none|.\\
% Similar to |\newgeometry| and |\restoregeometry| of the \xpackage{geometry}
% package\\
% |\newpagecolor{<|\textit{some colour} |>}| and |\restorepagecolor|
% are provided.
% \end{abstract}
%
% \bigskip
%
% \noindent Disclaimer for web links: The author is not responsible for any contents
% referred to in this work unless he has full knowledge of illegal contents.
% If any damage occurs by the use of information presented there, only the
% author of the respective pages might be liable, not the one who has referred
% to these pages.
%
% \bigskip
%
% \noindent {\color{green} Save per page about $200\unit{ml}$ water,
% $2\unit{g}$ CO$_{2}$ and $2\unit{g}$ wood:\\
% Therefore please print only if this is really necessary.}
%
% \pagebreak
%
% \tableofcontents
%
% \bigskip
%
% \section{Introduction\label{sec:Introduction}}
%
% \indent This \LaTeX{} package provides the command |\thepagecolor|,
% which gives the current page (background) colour,
% i.\,e.~the argument used with the most recent call of |\pagecolor{...}|.
% The package should be loaded before any package sets a page (background)
% colour, but after \xpackage{color} or \xpackage{xcolor} package.
% Its option |pagecolor={...}| is used to set the |\pagecolor{...}|.\\
% The command |\thepagecolornone| gives the same colour as |\thepagecolor|,
% except when the page background colour is
% \textquotedblleft|none|\textquotedblright (e.\,g. result of using the
% |\nopagecolor| command). In that case |\thepagecolor| is |white| and\linebreak
% |\thepagecolornone| is |none|.\\
% Similar to |\newgeometry| and |\restoregeometry| of the \xpackage{geometry}
% package\\
% |\newpagecolor{<|\textit{some colour}|>}| and |\restorepagecolor|
% are provided.
%
% \pagebreak
%
% \section{Usage}
%
% \indent Just load the package placing
% \begin{quote}
%   |\usepackage[<|\textit{option}|>]{pagecolor}|
% \end{quote}
% \noindent in the preamble of your \LaTeXe{} source file.
% This should be done before another package uses |\pagecolor|,
% but after |\nopagecolor| is defined (if that is defined in the document
% at all).\\
% Afterwards |\pagecolor{...}| can be used to change the page (background)
% colour as usual. Then |\thepagecolor| gives the current page (background)
% colour (in the same format as given with |\pagecolor{...}|).\\
% Similar to |\newgeometry| and |\restoregeometry| of the \xpackage{geometry}
% package\\
% |\newpagecolor{<|\textit{some colour} |>}| and |\restorepagecolor|
% are provided:\\
% |\newpagecolor{<|\textit{some colour} |>}| will execute
% |\pagecolor{<|\textit{some colour} |>}| and remember the page colour used before.
% |\restorecolor| (without argument) restores the page colour to the one used before
%  use of the |\newpagecolor{...}| command. When you want to change the colour for
%  just one page and do not want to (or cannot) manually determine where the page ends,\\
% |\newpagecolor{<|\textit{some colour}|>}\afterpage{\restorepagecolor}|\\
% does the trick (and requires a |\usepackage{afterpage}| in the document's
% preamble), or for short\\
% |\newcommand{\onepagecolor}[1]{%|\\
% |\newpagecolor{#1}\afterpage{\restorepagecolor}}|\\
% in the preamble and\\
% |\onepagecolor{<|\textit{some colour}|>}| in the document.
%
% \subsection{Option}
% \DescribeMacro{option}
% \indent The \xpackage{pagecolor} package takes the following option:
%
% \subsubsection{pagecolor\label{sss:pagecolor}}
% \DescribeMacro{pagecolor}
% \indent The option |pagecolor={...}| takes as value a colour. This
% could be as simple as |white| or |black|, but when e.\,g.~the
% \xpackage{xcolor} package is used (loaded before \xpackage{pagecolor}!),
% also colours like |red!50!green!20!blue| are possible.
% The default is |pagecolor={none}|.
%
% \section{Alternatives\label{sec:Alternatives}}
%
% As I neither know what exactly you want to accomplish
% when using this package (e.\,g.~hiding text), nor what resources you have
% (e.\,g.~\pdfTeX{} version), here is a list of some possible alternatives:\\
%
% \begin{description}
%  \item[-]
%  \DescribeMacro{transparent}
%  \xpackage{transparent} package: With it some object can be made
%  (fully or partially) transparent.
% \end{description}
%
% \begin{description}
%  \item[-]
%  \DescribeMacro{hrefhide}\xpackage{hrefhide} package: It allows to
%  \textquotedblleft hide\textquotedblright\ some (hyperlinked) text when
%  printing the document while keeping the layout.
% \end{description}
%
% \bigskip
%
% \noindent (You programmed or found another alternative,
%  which is available at \CTAN{}?\\
%  OK, send an e-mail to me with the name, location at \CTAN{},
%  and a short notice, and I will probably include it in
%  the list above.)\\
%
% \smallskip
%
% \noindent About how to get those packages, please see
% subsection~\ref{ss:Downloads}.
%
% \newpage
%
% \section{Example}
%
%    \begin{macrocode}
%<*example>
\documentclass[british]{article}[2007/10/19]% v1.4h
%%%%%%%%%%%%%%%%%%%%%%%%%%%%%%%%%%%%%%%%%%%%%%%%%%%%%%%%%%%%%%%%%%%%%
\usepackage[%
 extension=pdf,%
 plainpages=false,%
 pdfpagelabels=true,%
 hyperindex=false,%
 pdflang={en},%
 pdftitle={pagecolor package example},%
 pdfauthor={H.-Martin Muench},%
 pdfsubject={Example for the pagecolor package},%
 pdfkeywords={LaTeX, pagecolor, thepagecolor, page colour,%
  H.-Martin Muench},%
 pdfview=Fit,pdfstartview=Fit,%
 pdfpagelayout=SinglePage%
]{hyperref}[2012/02/06]% v6.82o
\usepackage[x11names]{xcolor}[2007/01/21]% v2.11
 % The xcolor package would not be needed for just using
 % the base colours. The color package would be sufficient for that.
\usepackage[pagecolor={LightGoldenrod1}]{pagecolor}[2012/02/23]% v1.0e

\usepackage{afterpage}[1995/10/27]% v1.08
 % The afterpage package is generally not needed,
 % but the |\newpagecolor{somecolour}\afterpage{\restorepagecolor}|
 % construct shall be demonstrated.

\usepackage{lipsum}[2011/04/14]% v1.2
 % The lipsum package is generally not needed,
 % but some blind text is needed for the example.

\gdef\unit#1{\mathord{\thinspace\mathrm{#1}}}%
\listfiles
\begin{document}
\pagenumbering{arabic}
\section*{Example for pagecolor}

This example demonstrates the use of package\newline
\textsf{pagecolor}, v1.0e as of 2012/02/23 (HMM).\newline
The used option was \verb|pagecolor={LightGoldenrod1}|.\newline
\verb|pagecolor={none}| would be the default.\newline

For more details please see the documentation!\newline

\noindent {\color{teal} Save per page about $200\unit{ml}$ water,
$2\unit{g}$ CO$_{2}$ and $2\unit{g}$ wood:\newline
Therefore please print only if this is really necessary.}\newline

The current page (background) colour is\newline
\verb|\thepagecolor|\ =\ \thepagecolor \newline
(and \verb|\thepagecolornone|\ =\ \thepagecolornone ,
which would only be different from \verb|\thepagecolor|,
when the page colour would be \verb|none|).

\pagebreak
\pagecolor{rgb:-green!40!yellow,3;green!40!yellow,2;red,1}

{\color{white} The current page (background) colour is\newline
\verb|\thepagecolor|\ =\ \thepagecolor . \newline}

{\color{\thepagecolor} And that makes this text practically invisible.
\newline}

{\color{white} Which made the preceding line of text practically
invisible.}

\pagebreak
\newpagecolor{red}

This page uses \verb|\newpagecolor{red}|.

\pagebreak
\restorepagecolor

{\color{white}And this page uses \verb|\restorepagecolor| to restore
the page colour to the value it had before the red page.}

\pagebreak
\pagecolor{none}

This page uses \verb|\pagecolor{none}|. If the \verb|\nopagecolor|
command is known (pdf\TeX and Lua\TeX; not yet for dvips, dvipdfm(x)
or Xe\TeX), the page colour is now \verb|none|, otherwise \verb|white|:
\verb|\thepagecolor|\ =\ \thepagecolor\ and
\verb|\thepagecolornone|\ =\ \thepagecolornone .

\pagebreak
\restorepagecolor

{\color{white}\verb|\restorepagecolor| restored the page colour again.}

\pagebreak
\pagecolor{green}

This page is green due to \verb|\pagecolor{green}|.

\pagebreak
\newpagecolor{blue}\afterpage{\restorepagecolor}

{\color{white}\verb|\newpagecolor{blue}\afterpage{\restorepagecolor}|%
\newline
was used here, i.\,e.~this page is blue, and the next one will
automatically have the same page colour before it was changed to blue
here (i.\,e. green).}

\smallskip
{\color{red}\textbf{\lipsum[1-11]}}
\bigskip

The page colour was changed back at the end of the page -
in mid-sentence!

\end{document}
%</example>
%    \end{macrocode}
%
% \StopEventually{}
%
% \newpage
%
% \section{The implementation}
%
% We start off by checking that we are loading into \LaTeXe{} and
% announcing the name and version of this package.
%
%    \begin{macrocode}
%<*package>
%    \end{macrocode}
%
%    \begin{macrocode}
\NeedsTeXFormat{LaTeX2e}[2011/06/27]
\ProvidesPackage{pagecolor}[2012/02/23 v1.0e
            Provides thepagecolor (HMM)]
%    \end{macrocode}
%
% A short description of the \xpackage{pagecolor} package:
%
%    \begin{macrocode}
%% Provides the \thepagecolor, \thepagecolornone, \newpagecolor{...},
%% and \restorepagecolor commands.
%    \end{macrocode}
%
% We need the \xpackage{kvoptions} package by \textsc{Heiko Oberdiek}:
%
%    \begin{macrocode}
\RequirePackage{kvoptions}[2010/12/23]% v3.10
%    \end{macrocode}
%
% and either the \xpackage{color} or the \xpackage{xcolor} package:
%
%    \begin{macrocode}
%% \RequirePackage{ either color or xcolor }:
\@ifpackageloaded{xcolor}{% xcolor loaded
   \@ifpackagelater{xcolor}{2007/01/21}{%
      % 2007/01/21, v2.11, or even more recent: OK
    }{% else: older package version
      \PackageWarning{pagecolor}{%
        It is requested version `2007/01/21' of package\MessageBreak%
        xcolor, but only an older version is available.\MessageBreak%
       }%
     }%
 }{% xcolor not loaded
   \@ifpackageloaded{color}{%
      \RequirePackage{color}[2005/11/14]% v1.0j
    }{% \else
      \PackageWarning{pagecolor}{%
        The pagecolor package must be loaded after either\MessageBreak%
        package color or after package xcolor (at your\MessageBreak%
        option). Neither package was loaded before package\MessageBreak%
        pagecolor. Loading of package xcolor will now be\MessageBreak%
        tried automatically.\MessageBreak%
        When the pagecolor package is used with option\MessageBreak%
        pagecolor using a colour requiring e. g. x11names\MessageBreak%
        option for xcolor package, this will not work!\MessageBreak%
       }
     }% \fi
   \RequirePackage{xcolor}[2007/01/21]% v2.11
  }% \fi
%    \end{macrocode}
%
% A last information for the user:
%
%    \begin{macrocode}
%% pagecolor may work with earlier versions of LaTeX and the
%% packages, but this was not tested. Please consider updating
%% your LaTeX and packages to the most recent version
%% (if they are not already the most recent version).

%    \end{macrocode}
%
% \indent See subsection~\ref{ss:Downloads} about how to get them.\\
%
% We process the options:
%
%    \begin{macrocode}
\SetupKeyvalOptions{family=pagecolor,prefix=pagecolor@}
\DeclareStringOption[none]{pagecolor}% \pagecolor@pagecolor
\ProcessKeyvalOptions*

%    \end{macrocode}
%
% We save the original |\pagecolor| command,
%
%    \begin{macrocode}
\let\origpagecolour\pagecolor

%    \end{macrocode}
%
%    \begin{macro}{\pagecolor}
% before we redefine it to include a definition of |\thepagecolor|
% and \linebreak
% |\thepagecolornone|:
%
%    \begin{macrocode}
\renewcommand{\pagecolor}[1]{\@bsphack%
  \edef\pagecolourtmpa{#1}%
  \def\pagecolourtmpb{none}%
  \ifx\pagecolourtmpa\pagecolourtmpb
    \@ifundefined{nopagecolor}{%
      \PackageError{pagecolor}{%
        pagecolor=none requested but \string\nopagecolor\space unknown%
       }{%
        \string\pagecolor{none} was used, but the command %
        \string\nopagecolor\space is undefined.\MessageBreak%
        Please use another colour. pagecolor=white will be used now.%
        \MessageBreak%
        }%
      \gdef\thepagecolor{white}%
      \gdef\thepagecolornone{white}% although it should be "none"
      \origpagecolour{white}%
     }{%
      \nopagecolor%
     }%
  \else%
    \xdef\thepagecolor{#1}%
    \xdef\thepagecolornone{#1}%
    \origpagecolour{\thepagecolornone}%
  \fi%
  \@esphack%
}

%    \end{macrocode}
%    \end{macro}
%
% |\nopagecolor| is only defined for \pdfTeX and \LuaTeX, but not yet for dvips,
% dvipdfm(x) or \XeTeX. Maybe
% \begin{verbatim}
% \@ifundefined{nopagecolor}{\newcommand{\nopagecolor}{\pagecolor{white}}}{}
% \end{verbatim}
% could be an alternative.\\
% When |\pagecolor{none}| is used and |\nopagecolor| is defined, then \\
% |\pagecolor{none}| is made into a synonym for |\nopagecolor|.\\
% If |\nopagecolor| has not been defined, nothing needs to be done.
% Otherwise we redefine |\nopagecolor|, thus that |\thepagecolor| is
% set to |white| and |\thepagecolornone| to |none| whenever
% |\nopagecolor| is used.
%
%    \begin{macrocode}
\@ifundefined{nopagecolor}{%
  \PackageWarning{pagecolor}{%
    \string\nopagecolor\space is undefined.\MessageBreak%
    (As long as it is neither used anywhere\MessageBreak%
    \space nor defined later, that will be no problem.)\MessageBreak%
   }
  \AtEndDocument{%
  \@ifundefined{nopagecolor}{% no problem
    }{%
    \PackageError{pagecolor}{%
      \string\nopagecolor\space defined after loading package%
      \MessageBreak%
      pagecolor%
      }{\string\nopagecolor\space was defined after loading the %
       pagecolor package.\MessageBreak%
       Either define \string\nopagecolor\space earlier or load the %
       pagecolor package later.\MessageBreak%
      }
    }%
  }%
  }{% \else
    \let\orignopagecolour\nopagecolor
    \renewcommand{\nopagecolor}{%
      \xdef\thepagecolor{white}
      \xdef\thepagecolornone{none}
      \orignopagecolour
      }
  }

%    \end{macrocode}
%
% If the page colour as given with option |pagecolor={...}| is |none|,
% but |\nopagecolor| is not known, |\pagecolor@pagecolor| is set to
% |white| and a warning is given.
%
%    \begin{macrocode}
\def\pagecolourtmpb{none}
\ifx\pagecolor@pagecolor\pagecolourtmpb
  \@ifundefined{nopagecolor}{%
    \PackageWarning{pagecolor}{%
      Option pagecolor=none (maybe by default) used,\MessageBreak%
      but \string\nopagecolor\space is unknown. Please use another%
      \MessageBreak%
      option value; white will be used now.\MessageBreak%
     }
    \setkeys{pagecolor}{pagecolor=white}%
  }{% ok
  }%
\fi

%    \end{macrocode}
%
% The (new) |\pagecolor| is now just carried out.
%
%    \begin{macrocode}
\pagecolor{\pagecolor@pagecolor}

%    \end{macrocode}
%
% Now the page (background) colour and |\thepagecolor| and \\
% |\thepagecolornone| are |\pagecolor@pagecolor| (or page (background) colour
% and |\thepagecolornone| are |none|=|\pagecolor@pagecolor| and
% |\thepagecolor| is |white|), and when the page (background) colour
% is changed, |\thepagecolor| and |\thepagecolornone| are changed
% accordingly.
%
% \newpage
%
%    \begin{macro}{\newpagecolor}
% There have been requests (via e-mail and at \\
% \url{http://tex.stackexchange.com/q/25137}) to change the colour of just one
% (or two) page(s) only, similar to |\newgeometry| and |\restoregeometry| of the
% \xpackage{geometry} package (\url{http://ctan.org/pkg/geometry}). Therefore
% |\newpagecolor| and |\restorepagecolor| are introduced (as suggested by
% \textsc{haoyun\_tex}):
%
%    \begin{macrocode}
\newcommand{\newpagecolor}[1]{%
\xdef\pagecolourtmpc{\thepagecolornone}%
\pagecolor{#1}%
}

%    \end{macrocode}
%    \end{macro}
%
% |\newpagecolor{<|\textit{some colour}|>}| will execute |\pagecolor{somecolour}|
% and remember the page colour used before.\\
%
%    \begin{macro}{\restorepagecolor}
%    \begin{macrocode}
\newcommand{\restorepagecolor}{\pagecolor{\pagecolourtmpc}}

%    \end{macrocode}
%    \end{macro}
%
% |\restorecolor| (without argument) restores the page colour to the one used before
% use of the |\newpagecolor{...}| command.
%
%    \begin{macrocode}
\gdef\pagecolourtmpc{\thepagecolor}

%    \end{macrocode}
% is just a precaution for |\restorecolor| being used when no |\newpagecolor{...}|
% was used before it.\\
%
% When you want to change the colour for just one page and do not want to
% (or cannot) manually determine where the page ends,\\
% |\newpagecolor{<|\textit{some colour}|>}\afterpage{\restorepagecolor}|\\
% does the trick (and requires an additional |\usepackage{afterpage}| in the
% document's preamble).
%    \begin{macrocode}
%</package>
%    \end{macrocode}
%
% \newpage
%
% \section{Installation}
%
% \subsection{Downloads\label{ss:Downloads}}
%
% Everything is available on \CTAN{}, \url{http://www.ctan.org/tex-archive/},
% but may need additional packages themselves.\\
%
% \DescribeMacro{pagecolor.dtx}
% For unpacking the |pagecolor.dtx| file and constructing the documentation
% it is required:
% \begin{description}
% \item[-] \TeX{} Format \LaTeXe{}, \url{http://www.CTAN.org/}
%
% \item[-] document class \xpackage{ltxdoc}, 2007/11/11, v2.0u,
%           \url{http://ctan.org/pkg/ltxdoc}
%
% \item[-] package \xpackage{holtxdoc}, 2011/02/04, v0.21,
%           \url{http://ctan.org/pkg/holtxdoc}
% \end{description}
%
% \DescribeMacro{pagecolor.sty}
% The |pagecolor.sty| for \LaTeXe{} (i.\,e. each document using
% the \xpackage{pagecolor} package) requires:
% \begin{description}
% \item[-] \TeX{} Format \LaTeXe{}, \url{http://www.CTAN.org/}
%
% \item[-] package \xpackage{kvoptions}, 2010/12/23, v3.10,
%           \url{http://ctan.org/pkg/kvoptions}
% \end{description}
% and either
% \begin{description}
% \item[-] package \xpackage{color}, 2005/11/14, v1.0j,
%           \url{http://ctan.org/pkg/color} (from the\\
%           \xpackage{graphics} package bundle)
% \end{description}
% or
% \begin{description}
% \item[-] package \xpackage{xcolor}, 2007/01/21, v2.11,
%           \url{http://ctan.org/pkg/xcolor}
% \end{description}
%
% \DescribeMacro{pagecolor-example.tex}
% The |pagecolor-example.tex| requires the same files as all
% documents using the \xpackage{pagecolor} package and additionally:
% \begin{description}
% \item[-] class \xpackage{article}, 2007/10/19, v1.4h, from \xpackage{classes.dtx}:\\
%           \CTAN{macros/latex/base/classes.dtx}
%
% \item[-] package \xpackage{pagecolor}, 2012/02/23, v1.0e,
%           \url{http://ctan.org/pkg/pagecolor}\\
%           (Well, it is the example file for this package, and because you are
%            reading the documentation for the \xpackage{pagecolor} package,
%            it can be assumed that you already have some version of it --
%            is it the current one?)
%
% \item[-] package \xpackage{xcolor}, 2007/01/21, v2.11,
%           \url{http://ctan.org/pkg/xcolor}\\
%           This package would not be needed for the use of just base colours only,
%           the \xpackage{color} package would be sufficient for that.
% \end{description}
%
% \DescribeMacro{transparent}
% \DescribeMacro{hrefhide}
% As possible alternatives in section \ref{sec:Alternatives} there are listed
% \begin{description}
% \item[-] package \xpackage{transparent}, 2007/01/08, v1.0, \\
%           \url{http://ctan.org/pkg/transparent}
%
% \item[-] package \xpackage{hrefhide}, 2011/04/29, v1.0f, \\
%           \url{http://ctan.org/pkg/hrefhide}
% \end{description}
%
% \DescribeMacro{Oberdiek}
% \DescribeMacro{holtxdoc}
% \DescribeMacro{kvoptions}
% All packages of \textsc{Heiko Oberdiek}'s bundle `oberdiek'
% (especially \xpackage{holtxdoc} and \xpackage{kvoptions})
% are also available in a TDS compliant ZIP archive:\\
% \CTAN{install/macros/latex/contrib/oberdiek.tds.zip}.\\
% It is probably best to download and use this, because the packages in there
% are quite probably both recent and compatible among themselves.\\
%
% \DescribeMacro{hyperref}
% \noindent \xpackage{hyperref} is not included in that bundle and needs to be
% downloaded separately,\\
% \url{http://mirror.ctan.org/install/macros/latex/contrib/hyperref.tds.zip}.\\
%
% \DescribeMacro{M\"{u}nch}
% A hyperlinked list of my (other) packages can be found at
% \url{http://www.Uni-Bonn.de/~uzs5pv/LaTeX.html}.\\
%
% \subsection{Package, unpacking TDS}
% \paragraph{Package.} This package is available on \CTAN{}
% \begin{description}
% \item[\CTAN{macros/latex/contrib/pagecolor/pagecolor.dtx}]\hspace*{0.1cm} \\
%       The source file.
% \item[\CTAN{macros/latex/contrib/pagecolor/pagecolor.pdf}]\hspace*{0.1cm} \\
%       The documentation.
% \item[\CTAN{macros/latex/contrib/pagecolor/pagecolor-example.pdf}]\hspace*{0.1cm} \\
%       The compiled example file, as it should look like.
% \item[\CTAN{macros/latex/contrib/pagecolor/README}]\hspace*{0.1cm} \\
%       The README file.
% \end{description}
% There is also a pagecolor.tds.zip available:
% \begin{description}
% \item[\CTAN{install/macros/latex/contrib/pagecolor.tds.zip}]\hspace*{0.1cm} \\
%       Everything in \xfile{TDS} compliant, compiled format.
% \end{description}
% which additionally contains\\
% \begin{tabular}{ll}
% pagecolor.ins & The installation file.\\
% pagecolor.drv & The driver to generate the documentation.\\
% pagecolor.sty & The \xext{sty}le file.\\
% pagecolor-example.tex & The example file.
% \end{tabular}
%
% \bigskip
%
% \noindent For required other packages, please see the preceding subsection.
%
% \paragraph{Unpacking.} The \xfile{.dtx} file is a self-extracting
% \docstrip{} archive. The files are extracted by running the
% \xext{.dtx} through \plainTeX{}:
% \begin{quote}
%   \verb|tex pagecolor.dtx|
% \end{quote}
%
% About generating the documentation see paragraph~\ref{GenDoc} below.\\
%
% \paragraph{TDS.} Now the different files must be moved into
% the different directories in your installation TDS tree
% (also known as \xfile{texmf} tree):
% \begin{quote}
% \def\t{^^A
% \begin{tabular}{@{}>{\ttfamily}l@{ $\rightarrow$ }>{\ttfamily}l@{}}
%   pagecolor.sty & tex/latex/pagecolor/pagecolor.sty\\
%   pagecolor.pdf & doc/latex/pagecolor/pagecolor.pdf\\
%   pagecolor-example.tex & doc/latex/pagecolor/pagecolor-example.tex\\
%   pagecolor-example.pdf & doc/latex/pagecolor/pagecolor-example.pdf\\
%   pagecolor.dtx & source/latex/pagecolor/pagecolor.dtx\\
% \end{tabular}^^A
% }^^A
% \sbox0{\t}^^A
% \ifdim\wd0>\linewidth
%   \begingroup
%     \advance\linewidth by\leftmargin
%     \advance\linewidth by\rightmargin
%   \edef\x{\endgroup
%     \def\noexpand\lw{\the\linewidth}^^A
%   }\x
%   \def\lwbox{^^A
%     \leavevmode
%     \hbox to \linewidth{^^A
%       \kern-\leftmargin\relax
%       \hss
%       \usebox0
%       \hss
%       \kern-\rightmargin\relax
%     }^^A
%   }^^A
%   \ifdim\wd0>\lw
%     \sbox0{\small\t}^^A
%     \ifdim\wd0>\linewidth
%       \ifdim\wd0>\lw
%         \sbox0{\footnotesize\t}^^A
%         \ifdim\wd0>\linewidth
%           \ifdim\wd0>\lw
%             \sbox0{\scriptsize\t}^^A
%             \ifdim\wd0>\linewidth
%               \ifdim\wd0>\lw
%                 \sbox0{\tiny\t}^^A
%                 \ifdim\wd0>\linewidth
%                   \lwbox
%                 \else
%                   \usebox0
%                 \fi
%               \else
%                 \lwbox
%               \fi
%             \else
%               \usebox0
%             \fi
%           \else
%             \lwbox
%           \fi
%         \else
%           \usebox0
%         \fi
%       \else
%         \lwbox
%       \fi
%     \else
%       \usebox0
%     \fi
%   \else
%     \lwbox
%   \fi
% \else
%   \usebox0
% \fi
% \end{quote}
% If you have a \xfile{docstrip.cfg} that configures and enables \docstrip{}'s
% \xfile{TDS} installing feature, then some files can already be in the right
% place, see the documentation of \docstrip{}.
%
% \subsection{Refresh file name databases}
%
% If your \TeX{}~distribution (\teTeX{}, \mikTeX{},\dots{}) relies on file name
% databases, you must refresh these. For example, \teTeX{} users run
% \verb|texhash| or \verb|mktexlsr|.
%
% \subsection{Some details for the interested}
%
% \paragraph{Unpacking with \LaTeX{}.}
% The \xfile{.dtx} chooses its action depending on the format:
% \begin{description}
% \item[\plainTeX:] Run \docstrip{} and extract the files.
% \item[\LaTeX:] Generate the documentation.
% \end{description}
% If you insist on using \LaTeX{} for \docstrip{} (really,
% \docstrip{} does not need \LaTeX{}), then inform the autodetect routine
% about your intention:
% \begin{quote}
%   \verb|latex \let\install=y\input{pagecolor.dtx}|
% \end{quote}
% Do not forget to quote the argument according to the demands
% of your shell.
%
% \paragraph{Generating the documentation.\label{GenDoc}}
% You can use both the \xfile{.dtx} or the \xfile{.drv} to generate
% the documentation. The process can be configured by a
% configuration file \xfile{ltxdoc.cfg}. For instance, put the following
% line into this file, if you want to have A4 as paper format:
% \begin{quote}
%   \verb|\PassOptionsToClass{a4paper}{article}|
% \end{quote}
%
% \noindent An example follows how to generate the
% documentation with \pdfLaTeX{}:
%
% \begin{quote}
%\begin{verbatim}
%pdflatex pagecolor.dtx
%makeindex -s gind.ist pagecolor.idx
%pdflatex pagecolor.dtx
%makeindex -s gind.ist pagecolor.idx
%pdflatex pagecolor.dtx
%\end{verbatim}
% \end{quote}
%
% \subsection{Compiling the example}
%
% The example file, \textsf{pagecolor-example.tex}, can be compiled via\\
% |(pdf)latex pagecolor-example.tex|.\\
%
% \section{Acknowledgements}
%
% I would like to thank \textsc{Heiko Oberdiek} for providing a~lot~(!) of
% useful packages (from which I also got everything I know about creating
% a file in \xext{dtx} format, ok, say it: copying), and the
% \Newsgroup{comp.text.tex} and \Newsgroup{de.comp.text.tex} newsgroups
% for their help in all things \TeX{}, especially all contributors to the
% discussion at
% \url{http://groups.google.com/group/comp.text.tex/browse_thread/thread/533576ebe44d60f6/e1654d70a19de63c?lnk=gst&q=Determination+of+the+current+page+colour#e1654d70a19de63c}
% \ (\textsc{H.~Oberdiek} \& \textsc{Gouailles}).\\
% I thank \textsc{haoyun\_tex} for suggesting the |\newpagecolor|/|\restorepagecolor|
% pair of commands and everyone at \url{http://tex.stackexchange.com/q/25137}
% for their contributions there.
%
% \bigskip
%
% \phantomsection
% \begin{History}\label{History}
%   \begin{Version}{2011/07/16 v1.0a}
%     \item First version discussed at \Newsgroup{comp.text.tex}.
%   \end{Version}
%   \begin{Version}{2011/08/06 v1.0b}
%     \item Changed version uploaded to \CTAN{}.
%   \end{Version}
%   \begin{Version}{2011/08/08 v1.0c}
%     \item Fixed a |\setkeys|.
%   \end{Version}
%   \begin{Version}{2012/02/01 v1.0d}
%     \item Bugfix: Obsolete installation path given in the documentation, updated.
%     \item New commands: |\newpagecolor{...}|, |\restorepagecolor|.
%     \item Update of documentation, README, and \xfile{dtx} internals.
%   \end{Version}
%   \begin{Version}{2012/02/23 v1.0e}
%     \item Fixed an error in the documentation.
%     \item Check for loading of \xpackage{color} or \xpackage{xcolor} package
%             and their versions has been changed, because \xpackage{xcolor} sets\\
%             |\@namedef{ver@color.sty}{1999/02/16}|\\
%             which gave a warning about old \xpackage{color} package even if a new
%             version was used.
%   \end{Version}
% \end{History}
%
% \bigskip
%
% When you find a mistake or have a suggestion for an improvement of this package,
% please send an e-mail to the maintainer, thanks! (Please see BUG REPORTS in the README.)
%
% \newpage
%
% \PrintIndex
%
% \Finale
\endinput|
% \end{quote}
% Do not forget to quote the argument according to the demands
% of your shell.
%
% \paragraph{Generating the documentation.\label{GenDoc}}
% You can use both the \xfile{.dtx} or the \xfile{.drv} to generate
% the documentation. The process can be configured by a
% configuration file \xfile{ltxdoc.cfg}. For instance, put the following
% line into this file, if you want to have A4 as paper format:
% \begin{quote}
%   \verb|\PassOptionsToClass{a4paper}{article}|
% \end{quote}
%
% \noindent An example follows how to generate the
% documentation with \pdfLaTeX{}:
%
% \begin{quote}
%\begin{verbatim}
%pdflatex pagecolor.dtx
%makeindex -s gind.ist pagecolor.idx
%pdflatex pagecolor.dtx
%makeindex -s gind.ist pagecolor.idx
%pdflatex pagecolor.dtx
%\end{verbatim}
% \end{quote}
%
% \subsection{Compiling the example}
%
% The example file, \textsf{pagecolor-example.tex}, can be compiled via\\
% |(pdf)latex pagecolor-example.tex|.\\
%
% \section{Acknowledgements}
%
% I would like to thank \textsc{Heiko Oberdiek} for providing a~lot~(!) of
% useful packages (from which I also got everything I know about creating
% a file in \xext{dtx} format, ok, say it: copying), and the
% \Newsgroup{comp.text.tex} and \Newsgroup{de.comp.text.tex} newsgroups
% for their help in all things \TeX{}, especially all contributors to the
% discussion at
% \url{http://groups.google.com/group/comp.text.tex/browse_thread/thread/533576ebe44d60f6/e1654d70a19de63c?lnk=gst&q=Determination+of+the+current+page+colour#e1654d70a19de63c}
% \ (\textsc{H.~Oberdiek} \& \textsc{Gouailles}).\\
% I thank \textsc{haoyun\_tex} for suggesting the |\newpagecolor|/|\restorepagecolor|
% pair of commands and everyone at \url{http://tex.stackexchange.com/q/25137}
% for their contributions there.
%
% \bigskip
%
% \phantomsection
% \begin{History}\label{History}
%   \begin{Version}{2011/07/16 v1.0a}
%     \item First version discussed at \Newsgroup{comp.text.tex}.
%   \end{Version}
%   \begin{Version}{2011/08/06 v1.0b}
%     \item Changed version uploaded to \CTAN{}.
%   \end{Version}
%   \begin{Version}{2011/08/08 v1.0c}
%     \item Fixed a |\setkeys|.
%   \end{Version}
%   \begin{Version}{2012/02/01 v1.0d}
%     \item Bugfix: Obsolete installation path given in the documentation, updated.
%     \item New commands: |\newpagecolor{...}|, |\restorepagecolor|.
%     \item Update of documentation, README, and \xfile{dtx} internals.
%   \end{Version}
%   \begin{Version}{2012/02/23 v1.0e}
%     \item Fixed an error in the documentation.
%     \item Check for loading of \xpackage{color} or \xpackage{xcolor} package
%             and their versions has been changed, because \xpackage{xcolor} sets\\
%             |\@namedef{ver@color.sty}{1999/02/16}|\\
%             which gave a warning about old \xpackage{color} package even if a new
%             version was used.
%   \end{Version}
% \end{History}
%
% \bigskip
%
% When you find a mistake or have a suggestion for an improvement of this package,
% please send an e-mail to the maintainer, thanks! (Please see BUG REPORTS in the README.)
%
% \newpage
%
% \PrintIndex
%
% \Finale
\endinput
%        (quote the arguments according to the demands of your shell)
%
% Documentation:
%    (a) If pagecolor.drv is present:
%           (pdf)latex pagecolor.drv
%           makeindex -s gind.ist pagecolor.idx
%           (pdf)latex pagecolor.drv
%           makeindex -s gind.ist pagecolor.idx
%           (pdf)latex pagecolor.drv
%    (b) Without pagecolor.drv:
%           (pdf)latex pagecolor.dtx
%           makeindex -s gind.ist pagecolor.idx
%           (pdf)latex pagecolor.dtx
%           makeindex -s gind.ist pagecolor.idx
%           (pdf)latex pagecolor.dtx
%
%    The class ltxdoc loads the configuration file ltxdoc.cfg
%    if available. Here you can specify further options, e.g.
%    use DIN A4 as paper format:
%       \PassOptionsToClass{a4paper}{article}
%
% Installation:
%    TDS:tex/latex/pagecolor/pagecolor.sty
%    TDS:doc/latex/pagecolor/pagecolor.pdf
%    TDS:doc/latex/pagecolor/pagecolor-example.tex
%    TDS:doc/latex/pagecolor/pagecolor-example.pdf
%    TDS:source/latex/pagecolor/pagecolor.dtx
%
%<*ignore>
\begingroup
  \catcode123=1 %
  \catcode125=2 %
  \def\x{LaTeX2e}%
\expandafter\endgroup
\ifcase 0\ifx\install y1\fi\expandafter
         \ifx\csname processbatchFile\endcsname\relax\else1\fi
         \ifx\fmtname\x\else 1\fi\relax
\else\csname fi\endcsname
%</ignore>
%<*install>
\input docstrip.tex
\Msg{*****************************************************************}
\Msg{* Installation                                                   }
\Msg{* Package: pagecolor 2012/02/23 v1.0e Provides thepagecolor (HMM)}
\Msg{*****************************************************************}

\keepsilent
\askforoverwritefalse

\let\MetaPrefix\relax
\preamble

This is a generated file.

Project: pagecolor
Version: 2012/02/23 v1.0e

Copyright (C) 2011 - 2012 by
    H.-Martin M"unch <Martin dot Muench at Uni-Bonn dot de>

The usual disclaimer applies:
If it doesn't work right that's your problem.
(Nevertheless, send an e-mail to the maintainer
 when you find an error in this package.)

This work may be distributed and/or modified under the
conditions of the LaTeX Project Public License, either
version 1.3c of this license or (at your option) any later
version. This version of this license is in
   http://www.latex-project.org/lppl/lppl-1-3c.txt
and the latest version of this license is in
   http://www.latex-project.org/lppl.txt
and version 1.3c or later is part of all distributions of
LaTeX version 2005/12/01 or later.

This work has the LPPL maintenance status "maintained".

The Current Maintainer of this work is H.-Martin Muench.

This work consists of the main source file pagecolor.dtx,
the README, and the derived files
   pagecolor.sty, pagecolor.pdf,
   pagecolor.ins, pagecolor.drv,
   pagecolor-example.tex, pagecolor-example.pdf.

\endpreamble
\let\MetaPrefix\DoubleperCent

\generate{%
  \file{pagecolor.ins}{\from{pagecolor.dtx}{install}}%
  \file{pagecolor.drv}{\from{pagecolor.dtx}{driver}}%
  \usedir{tex/latex/pagecolor}%
  \file{pagecolor.sty}{\from{pagecolor.dtx}{package}}%
  \usedir{doc/latex/pagecolor}%
  \file{pagecolor-example.tex}{\from{pagecolor.dtx}{example}}%
}

\catcode32=13\relax% active space
\let =\space%
\Msg{************************************************************************}
\Msg{*}
\Msg{* To finish the installation you have to move the following}
\Msg{* file into a directory searched by TeX:}
\Msg{*}
\Msg{*     pagecolor.sty}
\Msg{*}
\Msg{* To produce the documentation run the file `pagecolor.drv'}
\Msg{* through (pdf)LaTeX, e.g.}
\Msg{*  pdflatex pagecolor.drv}
\Msg{*  makeindex -s gind.ist pagecolor.idx}
\Msg{*  pdflatex pagecolor.drv}
\Msg{*  makeindex -s gind.ist pagecolor.idx}
\Msg{*  pdflatex pagecolor.drv}
\Msg{*}
\Msg{* At least three runs are necessary e.g. to get the}
\Msg{*  references right!}
\Msg{*}
\Msg{* Happy TeXing!}
\Msg{*}
\Msg{************************************************************************}

\endbatchfile
%</install>
%<*ignore>
\fi
%</ignore>
%
% \section{The documentation driver file}
%
% The next bit of code contains the documentation driver file for
% \TeX{}, i.\,e., the file that will produce the documentation you
% are currently reading. It will be extracted from this file by the
% \texttt{docstrip} programme. That is, run \LaTeX{} on \texttt{docstrip}
% and specify the \texttt{driver} option when \texttt{docstrip}
% asks for options.
%
%    \begin{macrocode}
%<*driver>
\NeedsTeXFormat{LaTeX2e}[2011/06/27]
\ProvidesFile{pagecolor.drv}%
  [2012/02/23 v1.0e Provides thepagecolor (HMM)]
\documentclass{ltxdoc}[2007/11/11]% v2.0u
\usepackage{holtxdoc}[2011/02/04]%  v0.21
%% pagecolor may work with earlier versions of LaTeX2e and those
%% class and package, but this was not tested.
%% Please consider updating your LaTeX, class, and package
%% to the most recent version (if they are not already the most
%% recent version).
\hypersetup{%
 pdfsubject={Providing the thepagecolor command (HMM)},%
 pdfkeywords={LaTeX, pagecolor, thepagecolor, H.-Martin Muench},%
 pdfencoding=auto,%
 pdflang={en},%
 breaklinks=true,%
 linktoc=all,%
 pdfstartview=FitH,%
 pdfpagelayout=OneColumn,%
 bookmarksnumbered=true,%
 bookmarksopen=true,%
 bookmarksopenlevel=3,%
 pdfmenubar=true,%
 pdftoolbar=true,%
 pdfwindowui=true,%
 pdfnewwindow=true%
}
\CodelineIndex
\hyphenation{printing docu-ment}
\gdef\unit#1{\mathord{\thinspace\mathrm{#1}}}%
\begin{document}
  \DocInput{pagecolor.dtx}%
\end{document}
%</driver>
%    \end{macrocode}
%
% \fi
%
% \CheckSum{137}
%
% \CharacterTable
%  {Upper-case    \A\B\C\D\E\F\G\H\I\J\K\L\M\N\O\P\Q\R\S\T\U\V\W\X\Y\Z
%   Lower-case    \a\b\c\d\e\f\g\h\i\j\k\l\m\n\o\p\q\r\s\t\u\v\w\x\y\z
%   Digits        \0\1\2\3\4\5\6\7\8\9
%   Exclamation   \!     Double quote  \"     Hash (number) \#
%   Dollar        \$     Percent       \%     Ampersand     \&
%   Acute accent  \'     Left paren    \(     Right paren   \)
%   Asterisk      \*     Plus          \+     Comma         \,
%   Minus         \-     Point         \.     Solidus       \/
%   Colon         \:     Semicolon     \;     Less than     \<
%   Equals        \=     Greater than  \>     Question mark \?
%   Commercial at \@     Left bracket  \[     Backslash     \\
%   Right bracket \]     Circumflex    \^     Underscore    \_
%   Grave accent  \`     Left brace    \{     Vertical bar  \|
%   Right brace   \}     Tilde         \~}
%
% \GetFileInfo{pagecolor.drv}
%
% \begingroup
%   \def\x{\#,\$,\^,\_,\~,\ ,\&,\{,\},\%}%
%   \makeatletter
%   \@onelevel@sanitize\x
% \expandafter\endgroup
% \expandafter\DoNotIndex\expandafter{\x}
% \expandafter\DoNotIndex\expandafter{\string\ }
% \begingroup
%   \makeatletter
%     \lccode`9=32\relax
%     \lowercase{%^^A
%       \edef\x{\noexpand\DoNotIndex{\@backslashchar9}}%^^A
%     }%^^A
%   \expandafter\endgroup\x
%
% \DoNotIndex{\,,\\}
% \DoNotIndex{\documentclass,\usepackage,\ProvidesPackage,\begin,\end}
% \DoNotIndex{\MessageBreak}
% \DoNotIndex{\NeedsTeXFormat,\DoNotIndex,\verb}
% \DoNotIndex{\def,\edef,\gdef,\global}
% \DoNotIndex{\ifx,\listfiles,\mathord,\mathrm}
% \DoNotIndex{\kvoptions,\SetupKeyvalOptions,\ProcessKeyvalOptions}
% \DoNotIndex{\bigskip,\space,\thinspace,\ldots}
% \DoNotIndex{\indent,\noindent,\newline,\linebreak,\pagebreak,\newpage}
% \DoNotIndex{\textbf,\textit,\textsf,\texttt,\textsc,\textquotedblleft,\textquotedblright}
% \DoNotIndex{\plainTeX,\TeX,\LaTeX,\pdfLaTeX}
% \DoNotIndex{\section,\color,\pagenumbering,\smallskip}
% \DoNotIndex{\@ifundefined,\@ifpackageloaded}
% \DoNotIndex{\DeclareStringOption}
%
% \title{The \xpackage{pagecolor} package}
% \date{2012/02/23 v1.0e}
% \author{H.-Martin M\"{u}nch\\\xemail{Martin.Muench at Uni-Bonn.de}}
%
% \maketitle
%
% \begin{abstract}
%  This \LaTeX{} package provides the command |\thepagecolor|,
% which gives the current page (background) colour,
% i.\,e.~the argument used with the most recent call of |\pagecolor{...}|.
% The command |\thepagecolornone| gives the same colour as |\thepagecolor|,
% except when the page background colour is
% \textquotedblleft|none|\textquotedblright. In that case |\thepagecolor| is
% |white| and |\thepagecolornone| is |none|.\\
% Similar to |\newgeometry| and |\restoregeometry| of the \xpackage{geometry}
% package\\
% |\newpagecolor{<|\textit{some colour} |>}| and |\restorepagecolor|
% are provided.
% \end{abstract}
%
% \bigskip
%
% \noindent Disclaimer for web links: The author is not responsible for any contents
% referred to in this work unless he has full knowledge of illegal contents.
% If any damage occurs by the use of information presented there, only the
% author of the respective pages might be liable, not the one who has referred
% to these pages.
%
% \bigskip
%
% \noindent {\color{green} Save per page about $200\unit{ml}$ water,
% $2\unit{g}$ CO$_{2}$ and $2\unit{g}$ wood:\\
% Therefore please print only if this is really necessary.}
%
% \pagebreak
%
% \tableofcontents
%
% \bigskip
%
% \section{Introduction\label{sec:Introduction}}
%
% \indent This \LaTeX{} package provides the command |\thepagecolor|,
% which gives the current page (background) colour,
% i.\,e.~the argument used with the most recent call of |\pagecolor{...}|.
% The package should be loaded before any package sets a page (background)
% colour, but after \xpackage{color} or \xpackage{xcolor} package.
% Its option |pagecolor={...}| is used to set the |\pagecolor{...}|.\\
% The command |\thepagecolornone| gives the same colour as |\thepagecolor|,
% except when the page background colour is
% \textquotedblleft|none|\textquotedblright (e.\,g. result of using the
% |\nopagecolor| command). In that case |\thepagecolor| is |white| and\linebreak
% |\thepagecolornone| is |none|.\\
% Similar to |\newgeometry| and |\restoregeometry| of the \xpackage{geometry}
% package\\
% |\newpagecolor{<|\textit{some colour}|>}| and |\restorepagecolor|
% are provided.
%
% \pagebreak
%
% \section{Usage}
%
% \indent Just load the package placing
% \begin{quote}
%   |\usepackage[<|\textit{option}|>]{pagecolor}|
% \end{quote}
% \noindent in the preamble of your \LaTeXe{} source file.
% This should be done before another package uses |\pagecolor|,
% but after |\nopagecolor| is defined (if that is defined in the document
% at all).\\
% Afterwards |\pagecolor{...}| can be used to change the page (background)
% colour as usual. Then |\thepagecolor| gives the current page (background)
% colour (in the same format as given with |\pagecolor{...}|).\\
% Similar to |\newgeometry| and |\restoregeometry| of the \xpackage{geometry}
% package\\
% |\newpagecolor{<|\textit{some colour} |>}| and |\restorepagecolor|
% are provided:\\
% |\newpagecolor{<|\textit{some colour} |>}| will execute
% |\pagecolor{<|\textit{some colour} |>}| and remember the page colour used before.
% |\restorecolor| (without argument) restores the page colour to the one used before
%  use of the |\newpagecolor{...}| command. When you want to change the colour for
%  just one page and do not want to (or cannot) manually determine where the page ends,\\
% |\newpagecolor{<|\textit{some colour}|>}\afterpage{\restorepagecolor}|\\
% does the trick (and requires a |\usepackage{afterpage}| in the document's
% preamble), or for short\\
% |\newcommand{\onepagecolor}[1]{%|\\
% |\newpagecolor{#1}\afterpage{\restorepagecolor}}|\\
% in the preamble and\\
% |\onepagecolor{<|\textit{some colour}|>}| in the document.
%
% \subsection{Option}
% \DescribeMacro{option}
% \indent The \xpackage{pagecolor} package takes the following option:
%
% \subsubsection{pagecolor\label{sss:pagecolor}}
% \DescribeMacro{pagecolor}
% \indent The option |pagecolor={...}| takes as value a colour. This
% could be as simple as |white| or |black|, but when e.\,g.~the
% \xpackage{xcolor} package is used (loaded before \xpackage{pagecolor}!),
% also colours like |red!50!green!20!blue| are possible.
% The default is |pagecolor={none}|.
%
% \section{Alternatives\label{sec:Alternatives}}
%
% As I neither know what exactly you want to accomplish
% when using this package (e.\,g.~hiding text), nor what resources you have
% (e.\,g.~\pdfTeX{} version), here is a list of some possible alternatives:\\
%
% \begin{description}
%  \item[-]
%  \DescribeMacro{transparent}
%  \xpackage{transparent} package: With it some object can be made
%  (fully or partially) transparent.
% \end{description}
%
% \begin{description}
%  \item[-]
%  \DescribeMacro{hrefhide}\xpackage{hrefhide} package: It allows to
%  \textquotedblleft hide\textquotedblright\ some (hyperlinked) text when
%  printing the document while keeping the layout.
% \end{description}
%
% \bigskip
%
% \noindent (You programmed or found another alternative,
%  which is available at \CTAN{}?\\
%  OK, send an e-mail to me with the name, location at \CTAN{},
%  and a short notice, and I will probably include it in
%  the list above.)\\
%
% \smallskip
%
% \noindent About how to get those packages, please see
% subsection~\ref{ss:Downloads}.
%
% \newpage
%
% \section{Example}
%
%    \begin{macrocode}
%<*example>
\documentclass[british]{article}[2007/10/19]% v1.4h
%%%%%%%%%%%%%%%%%%%%%%%%%%%%%%%%%%%%%%%%%%%%%%%%%%%%%%%%%%%%%%%%%%%%%
\usepackage[%
 extension=pdf,%
 plainpages=false,%
 pdfpagelabels=true,%
 hyperindex=false,%
 pdflang={en},%
 pdftitle={pagecolor package example},%
 pdfauthor={H.-Martin Muench},%
 pdfsubject={Example for the pagecolor package},%
 pdfkeywords={LaTeX, pagecolor, thepagecolor, page colour,%
  H.-Martin Muench},%
 pdfview=Fit,pdfstartview=Fit,%
 pdfpagelayout=SinglePage%
]{hyperref}[2012/02/06]% v6.82o
\usepackage[x11names]{xcolor}[2007/01/21]% v2.11
 % The xcolor package would not be needed for just using
 % the base colours. The color package would be sufficient for that.
\usepackage[pagecolor={LightGoldenrod1}]{pagecolor}[2012/02/23]% v1.0e

\usepackage{afterpage}[1995/10/27]% v1.08
 % The afterpage package is generally not needed,
 % but the |\newpagecolor{somecolour}\afterpage{\restorepagecolor}|
 % construct shall be demonstrated.

\usepackage{lipsum}[2011/04/14]% v1.2
 % The lipsum package is generally not needed,
 % but some blind text is needed for the example.

\gdef\unit#1{\mathord{\thinspace\mathrm{#1}}}%
\listfiles
\begin{document}
\pagenumbering{arabic}
\section*{Example for pagecolor}

This example demonstrates the use of package\newline
\textsf{pagecolor}, v1.0e as of 2012/02/23 (HMM).\newline
The used option was \verb|pagecolor={LightGoldenrod1}|.\newline
\verb|pagecolor={none}| would be the default.\newline

For more details please see the documentation!\newline

\noindent {\color{teal} Save per page about $200\unit{ml}$ water,
$2\unit{g}$ CO$_{2}$ and $2\unit{g}$ wood:\newline
Therefore please print only if this is really necessary.}\newline

The current page (background) colour is\newline
\verb|\thepagecolor|\ =\ \thepagecolor \newline
(and \verb|\thepagecolornone|\ =\ \thepagecolornone ,
which would only be different from \verb|\thepagecolor|,
when the page colour would be \verb|none|).

\pagebreak
\pagecolor{rgb:-green!40!yellow,3;green!40!yellow,2;red,1}

{\color{white} The current page (background) colour is\newline
\verb|\thepagecolor|\ =\ \thepagecolor . \newline}

{\color{\thepagecolor} And that makes this text practically invisible.
\newline}

{\color{white} Which made the preceding line of text practically
invisible.}

\pagebreak
\newpagecolor{red}

This page uses \verb|\newpagecolor{red}|.

\pagebreak
\restorepagecolor

{\color{white}And this page uses \verb|\restorepagecolor| to restore
the page colour to the value it had before the red page.}

\pagebreak
\pagecolor{none}

This page uses \verb|\pagecolor{none}|. If the \verb|\nopagecolor|
command is known (pdf\TeX and Lua\TeX; not yet for dvips, dvipdfm(x)
or Xe\TeX), the page colour is now \verb|none|, otherwise \verb|white|:
\verb|\thepagecolor|\ =\ \thepagecolor\ and
\verb|\thepagecolornone|\ =\ \thepagecolornone .

\pagebreak
\restorepagecolor

{\color{white}\verb|\restorepagecolor| restored the page colour again.}

\pagebreak
\pagecolor{green}

This page is green due to \verb|\pagecolor{green}|.

\pagebreak
\newpagecolor{blue}\afterpage{\restorepagecolor}

{\color{white}\verb|\newpagecolor{blue}\afterpage{\restorepagecolor}|%
\newline
was used here, i.\,e.~this page is blue, and the next one will
automatically have the same page colour before it was changed to blue
here (i.\,e. green).}

\smallskip
{\color{red}\textbf{\lipsum[1-11]}}
\bigskip

The page colour was changed back at the end of the page -
in mid-sentence!

\end{document}
%</example>
%    \end{macrocode}
%
% \StopEventually{}
%
% \newpage
%
% \section{The implementation}
%
% We start off by checking that we are loading into \LaTeXe{} and
% announcing the name and version of this package.
%
%    \begin{macrocode}
%<*package>
%    \end{macrocode}
%
%    \begin{macrocode}
\NeedsTeXFormat{LaTeX2e}[2011/06/27]
\ProvidesPackage{pagecolor}[2012/02/23 v1.0e
            Provides thepagecolor (HMM)]
%    \end{macrocode}
%
% A short description of the \xpackage{pagecolor} package:
%
%    \begin{macrocode}
%% Provides the \thepagecolor, \thepagecolornone, \newpagecolor{...},
%% and \restorepagecolor commands.
%    \end{macrocode}
%
% We need the \xpackage{kvoptions} package by \textsc{Heiko Oberdiek}:
%
%    \begin{macrocode}
\RequirePackage{kvoptions}[2010/12/23]% v3.10
%    \end{macrocode}
%
% and either the \xpackage{color} or the \xpackage{xcolor} package:
%
%    \begin{macrocode}
%% \RequirePackage{ either color or xcolor }:
\@ifpackageloaded{xcolor}{% xcolor loaded
   \@ifpackagelater{xcolor}{2007/01/21}{%
      % 2007/01/21, v2.11, or even more recent: OK
    }{% else: older package version
      \PackageWarning{pagecolor}{%
        It is requested version `2007/01/21' of package\MessageBreak%
        xcolor, but only an older version is available.\MessageBreak%
       }%
     }%
 }{% xcolor not loaded
   \@ifpackageloaded{color}{%
      \RequirePackage{color}[2005/11/14]% v1.0j
    }{% \else
      \PackageWarning{pagecolor}{%
        The pagecolor package must be loaded after either\MessageBreak%
        package color or after package xcolor (at your\MessageBreak%
        option). Neither package was loaded before package\MessageBreak%
        pagecolor. Loading of package xcolor will now be\MessageBreak%
        tried automatically.\MessageBreak%
        When the pagecolor package is used with option\MessageBreak%
        pagecolor using a colour requiring e. g. x11names\MessageBreak%
        option for xcolor package, this will not work!\MessageBreak%
       }
     }% \fi
   \RequirePackage{xcolor}[2007/01/21]% v2.11
  }% \fi
%    \end{macrocode}
%
% A last information for the user:
%
%    \begin{macrocode}
%% pagecolor may work with earlier versions of LaTeX and the
%% packages, but this was not tested. Please consider updating
%% your LaTeX and packages to the most recent version
%% (if they are not already the most recent version).

%    \end{macrocode}
%
% \indent See subsection~\ref{ss:Downloads} about how to get them.\\
%
% We process the options:
%
%    \begin{macrocode}
\SetupKeyvalOptions{family=pagecolor,prefix=pagecolor@}
\DeclareStringOption[none]{pagecolor}% \pagecolor@pagecolor
\ProcessKeyvalOptions*

%    \end{macrocode}
%
% We save the original |\pagecolor| command,
%
%    \begin{macrocode}
\let\origpagecolour\pagecolor

%    \end{macrocode}
%
%    \begin{macro}{\pagecolor}
% before we redefine it to include a definition of |\thepagecolor|
% and \linebreak
% |\thepagecolornone|:
%
%    \begin{macrocode}
\renewcommand{\pagecolor}[1]{\@bsphack%
  \edef\pagecolourtmpa{#1}%
  \def\pagecolourtmpb{none}%
  \ifx\pagecolourtmpa\pagecolourtmpb
    \@ifundefined{nopagecolor}{%
      \PackageError{pagecolor}{%
        pagecolor=none requested but \string\nopagecolor\space unknown%
       }{%
        \string\pagecolor{none} was used, but the command %
        \string\nopagecolor\space is undefined.\MessageBreak%
        Please use another colour. pagecolor=white will be used now.%
        \MessageBreak%
        }%
      \gdef\thepagecolor{white}%
      \gdef\thepagecolornone{white}% although it should be "none"
      \origpagecolour{white}%
     }{%
      \nopagecolor%
     }%
  \else%
    \xdef\thepagecolor{#1}%
    \xdef\thepagecolornone{#1}%
    \origpagecolour{\thepagecolornone}%
  \fi%
  \@esphack%
}

%    \end{macrocode}
%    \end{macro}
%
% |\nopagecolor| is only defined for \pdfTeX and \LuaTeX, but not yet for dvips,
% dvipdfm(x) or \XeTeX. Maybe
% \begin{verbatim}
% \@ifundefined{nopagecolor}{\newcommand{\nopagecolor}{\pagecolor{white}}}{}
% \end{verbatim}
% could be an alternative.\\
% When |\pagecolor{none}| is used and |\nopagecolor| is defined, then \\
% |\pagecolor{none}| is made into a synonym for |\nopagecolor|.\\
% If |\nopagecolor| has not been defined, nothing needs to be done.
% Otherwise we redefine |\nopagecolor|, thus that |\thepagecolor| is
% set to |white| and |\thepagecolornone| to |none| whenever
% |\nopagecolor| is used.
%
%    \begin{macrocode}
\@ifundefined{nopagecolor}{%
  \PackageWarning{pagecolor}{%
    \string\nopagecolor\space is undefined.\MessageBreak%
    (As long as it is neither used anywhere\MessageBreak%
    \space nor defined later, that will be no problem.)\MessageBreak%
   }
  \AtEndDocument{%
  \@ifundefined{nopagecolor}{% no problem
    }{%
    \PackageError{pagecolor}{%
      \string\nopagecolor\space defined after loading package%
      \MessageBreak%
      pagecolor%
      }{\string\nopagecolor\space was defined after loading the %
       pagecolor package.\MessageBreak%
       Either define \string\nopagecolor\space earlier or load the %
       pagecolor package later.\MessageBreak%
      }
    }%
  }%
  }{% \else
    \let\orignopagecolour\nopagecolor
    \renewcommand{\nopagecolor}{%
      \xdef\thepagecolor{white}
      \xdef\thepagecolornone{none}
      \orignopagecolour
      }
  }

%    \end{macrocode}
%
% If the page colour as given with option |pagecolor={...}| is |none|,
% but |\nopagecolor| is not known, |\pagecolor@pagecolor| is set to
% |white| and a warning is given.
%
%    \begin{macrocode}
\def\pagecolourtmpb{none}
\ifx\pagecolor@pagecolor\pagecolourtmpb
  \@ifundefined{nopagecolor}{%
    \PackageWarning{pagecolor}{%
      Option pagecolor=none (maybe by default) used,\MessageBreak%
      but \string\nopagecolor\space is unknown. Please use another%
      \MessageBreak%
      option value; white will be used now.\MessageBreak%
     }
    \setkeys{pagecolor}{pagecolor=white}%
  }{% ok
  }%
\fi

%    \end{macrocode}
%
% The (new) |\pagecolor| is now just carried out.
%
%    \begin{macrocode}
\pagecolor{\pagecolor@pagecolor}

%    \end{macrocode}
%
% Now the page (background) colour and |\thepagecolor| and \\
% |\thepagecolornone| are |\pagecolor@pagecolor| (or page (background) colour
% and |\thepagecolornone| are |none|=|\pagecolor@pagecolor| and
% |\thepagecolor| is |white|), and when the page (background) colour
% is changed, |\thepagecolor| and |\thepagecolornone| are changed
% accordingly.
%
% \newpage
%
%    \begin{macro}{\newpagecolor}
% There have been requests (via e-mail and at \\
% \url{http://tex.stackexchange.com/q/25137}) to change the colour of just one
% (or two) page(s) only, similar to |\newgeometry| and |\restoregeometry| of the
% \xpackage{geometry} package (\url{http://ctan.org/pkg/geometry}). Therefore
% |\newpagecolor| and |\restorepagecolor| are introduced (as suggested by
% \textsc{haoyun\_tex}):
%
%    \begin{macrocode}
\newcommand{\newpagecolor}[1]{%
\xdef\pagecolourtmpc{\thepagecolornone}%
\pagecolor{#1}%
}

%    \end{macrocode}
%    \end{macro}
%
% |\newpagecolor{<|\textit{some colour}|>}| will execute |\pagecolor{somecolour}|
% and remember the page colour used before.\\
%
%    \begin{macro}{\restorepagecolor}
%    \begin{macrocode}
\newcommand{\restorepagecolor}{\pagecolor{\pagecolourtmpc}}

%    \end{macrocode}
%    \end{macro}
%
% |\restorecolor| (without argument) restores the page colour to the one used before
% use of the |\newpagecolor{...}| command.
%
%    \begin{macrocode}
\gdef\pagecolourtmpc{\thepagecolor}

%    \end{macrocode}
% is just a precaution for |\restorecolor| being used when no |\newpagecolor{...}|
% was used before it.\\
%
% When you want to change the colour for just one page and do not want to
% (or cannot) manually determine where the page ends,\\
% |\newpagecolor{<|\textit{some colour}|>}\afterpage{\restorepagecolor}|\\
% does the trick (and requires an additional |\usepackage{afterpage}| in the
% document's preamble).
%    \begin{macrocode}
%</package>
%    \end{macrocode}
%
% \newpage
%
% \section{Installation}
%
% \subsection{Downloads\label{ss:Downloads}}
%
% Everything is available on \CTAN{}, \url{http://www.ctan.org/tex-archive/},
% but may need additional packages themselves.\\
%
% \DescribeMacro{pagecolor.dtx}
% For unpacking the |pagecolor.dtx| file and constructing the documentation
% it is required:
% \begin{description}
% \item[-] \TeX{} Format \LaTeXe{}, \url{http://www.CTAN.org/}
%
% \item[-] document class \xpackage{ltxdoc}, 2007/11/11, v2.0u,
%           \url{http://ctan.org/pkg/ltxdoc}
%
% \item[-] package \xpackage{holtxdoc}, 2011/02/04, v0.21,
%           \url{http://ctan.org/pkg/holtxdoc}
% \end{description}
%
% \DescribeMacro{pagecolor.sty}
% The |pagecolor.sty| for \LaTeXe{} (i.\,e. each document using
% the \xpackage{pagecolor} package) requires:
% \begin{description}
% \item[-] \TeX{} Format \LaTeXe{}, \url{http://www.CTAN.org/}
%
% \item[-] package \xpackage{kvoptions}, 2010/12/23, v3.10,
%           \url{http://ctan.org/pkg/kvoptions}
% \end{description}
% and either
% \begin{description}
% \item[-] package \xpackage{color}, 2005/11/14, v1.0j,
%           \url{http://ctan.org/pkg/color} (from the\\
%           \xpackage{graphics} package bundle)
% \end{description}
% or
% \begin{description}
% \item[-] package \xpackage{xcolor}, 2007/01/21, v2.11,
%           \url{http://ctan.org/pkg/xcolor}
% \end{description}
%
% \DescribeMacro{pagecolor-example.tex}
% The |pagecolor-example.tex| requires the same files as all
% documents using the \xpackage{pagecolor} package and additionally:
% \begin{description}
% \item[-] class \xpackage{article}, 2007/10/19, v1.4h, from \xpackage{classes.dtx}:\\
%           \CTAN{macros/latex/base/classes.dtx}
%
% \item[-] package \xpackage{pagecolor}, 2012/02/23, v1.0e,
%           \url{http://ctan.org/pkg/pagecolor}\\
%           (Well, it is the example file for this package, and because you are
%            reading the documentation for the \xpackage{pagecolor} package,
%            it can be assumed that you already have some version of it --
%            is it the current one?)
%
% \item[-] package \xpackage{xcolor}, 2007/01/21, v2.11,
%           \url{http://ctan.org/pkg/xcolor}\\
%           This package would not be needed for the use of just base colours only,
%           the \xpackage{color} package would be sufficient for that.
% \end{description}
%
% \DescribeMacro{transparent}
% \DescribeMacro{hrefhide}
% As possible alternatives in section \ref{sec:Alternatives} there are listed
% \begin{description}
% \item[-] package \xpackage{transparent}, 2007/01/08, v1.0, \\
%           \url{http://ctan.org/pkg/transparent}
%
% \item[-] package \xpackage{hrefhide}, 2011/04/29, v1.0f, \\
%           \url{http://ctan.org/pkg/hrefhide}
% \end{description}
%
% \DescribeMacro{Oberdiek}
% \DescribeMacro{holtxdoc}
% \DescribeMacro{kvoptions}
% All packages of \textsc{Heiko Oberdiek}'s bundle `oberdiek'
% (especially \xpackage{holtxdoc} and \xpackage{kvoptions})
% are also available in a TDS compliant ZIP archive:\\
% \CTAN{install/macros/latex/contrib/oberdiek.tds.zip}.\\
% It is probably best to download and use this, because the packages in there
% are quite probably both recent and compatible among themselves.\\
%
% \DescribeMacro{hyperref}
% \noindent \xpackage{hyperref} is not included in that bundle and needs to be
% downloaded separately,\\
% \url{http://mirror.ctan.org/install/macros/latex/contrib/hyperref.tds.zip}.\\
%
% \DescribeMacro{M\"{u}nch}
% A hyperlinked list of my (other) packages can be found at
% \url{http://www.Uni-Bonn.de/~uzs5pv/LaTeX.html}.\\
%
% \subsection{Package, unpacking TDS}
% \paragraph{Package.} This package is available on \CTAN{}
% \begin{description}
% \item[\CTAN{macros/latex/contrib/pagecolor/pagecolor.dtx}]\hspace*{0.1cm} \\
%       The source file.
% \item[\CTAN{macros/latex/contrib/pagecolor/pagecolor.pdf}]\hspace*{0.1cm} \\
%       The documentation.
% \item[\CTAN{macros/latex/contrib/pagecolor/pagecolor-example.pdf}]\hspace*{0.1cm} \\
%       The compiled example file, as it should look like.
% \item[\CTAN{macros/latex/contrib/pagecolor/README}]\hspace*{0.1cm} \\
%       The README file.
% \end{description}
% There is also a pagecolor.tds.zip available:
% \begin{description}
% \item[\CTAN{install/macros/latex/contrib/pagecolor.tds.zip}]\hspace*{0.1cm} \\
%       Everything in \xfile{TDS} compliant, compiled format.
% \end{description}
% which additionally contains\\
% \begin{tabular}{ll}
% pagecolor.ins & The installation file.\\
% pagecolor.drv & The driver to generate the documentation.\\
% pagecolor.sty & The \xext{sty}le file.\\
% pagecolor-example.tex & The example file.
% \end{tabular}
%
% \bigskip
%
% \noindent For required other packages, please see the preceding subsection.
%
% \paragraph{Unpacking.} The \xfile{.dtx} file is a self-extracting
% \docstrip{} archive. The files are extracted by running the
% \xext{.dtx} through \plainTeX{}:
% \begin{quote}
%   \verb|tex pagecolor.dtx|
% \end{quote}
%
% About generating the documentation see paragraph~\ref{GenDoc} below.\\
%
% \paragraph{TDS.} Now the different files must be moved into
% the different directories in your installation TDS tree
% (also known as \xfile{texmf} tree):
% \begin{quote}
% \def\t{^^A
% \begin{tabular}{@{}>{\ttfamily}l@{ $\rightarrow$ }>{\ttfamily}l@{}}
%   pagecolor.sty & tex/latex/pagecolor/pagecolor.sty\\
%   pagecolor.pdf & doc/latex/pagecolor/pagecolor.pdf\\
%   pagecolor-example.tex & doc/latex/pagecolor/pagecolor-example.tex\\
%   pagecolor-example.pdf & doc/latex/pagecolor/pagecolor-example.pdf\\
%   pagecolor.dtx & source/latex/pagecolor/pagecolor.dtx\\
% \end{tabular}^^A
% }^^A
% \sbox0{\t}^^A
% \ifdim\wd0>\linewidth
%   \begingroup
%     \advance\linewidth by\leftmargin
%     \advance\linewidth by\rightmargin
%   \edef\x{\endgroup
%     \def\noexpand\lw{\the\linewidth}^^A
%   }\x
%   \def\lwbox{^^A
%     \leavevmode
%     \hbox to \linewidth{^^A
%       \kern-\leftmargin\relax
%       \hss
%       \usebox0
%       \hss
%       \kern-\rightmargin\relax
%     }^^A
%   }^^A
%   \ifdim\wd0>\lw
%     \sbox0{\small\t}^^A
%     \ifdim\wd0>\linewidth
%       \ifdim\wd0>\lw
%         \sbox0{\footnotesize\t}^^A
%         \ifdim\wd0>\linewidth
%           \ifdim\wd0>\lw
%             \sbox0{\scriptsize\t}^^A
%             \ifdim\wd0>\linewidth
%               \ifdim\wd0>\lw
%                 \sbox0{\tiny\t}^^A
%                 \ifdim\wd0>\linewidth
%                   \lwbox
%                 \else
%                   \usebox0
%                 \fi
%               \else
%                 \lwbox
%               \fi
%             \else
%               \usebox0
%             \fi
%           \else
%             \lwbox
%           \fi
%         \else
%           \usebox0
%         \fi
%       \else
%         \lwbox
%       \fi
%     \else
%       \usebox0
%     \fi
%   \else
%     \lwbox
%   \fi
% \else
%   \usebox0
% \fi
% \end{quote}
% If you have a \xfile{docstrip.cfg} that configures and enables \docstrip{}'s
% \xfile{TDS} installing feature, then some files can already be in the right
% place, see the documentation of \docstrip{}.
%
% \subsection{Refresh file name databases}
%
% If your \TeX{}~distribution (\teTeX{}, \mikTeX{},\dots{}) relies on file name
% databases, you must refresh these. For example, \teTeX{} users run
% \verb|texhash| or \verb|mktexlsr|.
%
% \subsection{Some details for the interested}
%
% \paragraph{Unpacking with \LaTeX{}.}
% The \xfile{.dtx} chooses its action depending on the format:
% \begin{description}
% \item[\plainTeX:] Run \docstrip{} and extract the files.
% \item[\LaTeX:] Generate the documentation.
% \end{description}
% If you insist on using \LaTeX{} for \docstrip{} (really,
% \docstrip{} does not need \LaTeX{}), then inform the autodetect routine
% about your intention:
% \begin{quote}
%   \verb|latex \let\install=y% \iffalse meta-comment
%
% File: pagecolor.dtx
% Version: 2012/02/23 v1.0e
%
% Copyright (C) 2011 - 2012 by
%    H.-Martin M"unch <Martin dot Muench at Uni-Bonn dot de>
%
% This work may be distributed and/or modified under the
% conditions of the LaTeX Project Public License, either
% version 1.3c of this license or (at your option) any later
% version. This version of this license is in
%    http://www.latex-project.org/lppl/lppl-1-3c.txt
% and the latest version of this license is in
%    http://www.latex-project.org/lppl.txt
% and version 1.3c or later is part of all distributions of
% LaTeX version 2005/12/01 or later.
%
% This work has the LPPL maintenance status "maintained".
%
% The Current Maintainer of this work is H.-Martin Muench.
%
% This work consists of the main source file pagecolor.dtx,
% the README, and the derived files
%    pagecolor.sty, pagecolor.pdf,
%    pagecolor.ins, pagecolor.drv,
%    pagecolor-example.tex, pagecolor-example.pdf.
%
% Distribution:
%    CTAN:macros/latex/contrib/pagecolor/pagecolor.dtx
%    CTAN:macros/latex/contrib/pagecolor/pagecolor.pdf
%    CTAN:install/macros/latex/contrib/pagecolor.tds.zip
%
% Cf. http://ctan.org/pkg/pagecolor and
% http://www.tex.ac.uk/tex-archive/help/Catalogue/entries/pagecolor.html
%
% Unpacking:
%    (a) If pagecolor.ins is present:
%           tex pagecolor.ins
%    (b) Without pagecolor.ins:
%           tex pagecolor.dtx
%    (c) If you insist on using LaTeX
%           latex \let\install=y% \iffalse meta-comment
%
% File: pagecolor.dtx
% Version: 2012/02/23 v1.0e
%
% Copyright (C) 2011 - 2012 by
%    H.-Martin M"unch <Martin dot Muench at Uni-Bonn dot de>
%
% This work may be distributed and/or modified under the
% conditions of the LaTeX Project Public License, either
% version 1.3c of this license or (at your option) any later
% version. This version of this license is in
%    http://www.latex-project.org/lppl/lppl-1-3c.txt
% and the latest version of this license is in
%    http://www.latex-project.org/lppl.txt
% and version 1.3c or later is part of all distributions of
% LaTeX version 2005/12/01 or later.
%
% This work has the LPPL maintenance status "maintained".
%
% The Current Maintainer of this work is H.-Martin Muench.
%
% This work consists of the main source file pagecolor.dtx,
% the README, and the derived files
%    pagecolor.sty, pagecolor.pdf,
%    pagecolor.ins, pagecolor.drv,
%    pagecolor-example.tex, pagecolor-example.pdf.
%
% Distribution:
%    CTAN:macros/latex/contrib/pagecolor/pagecolor.dtx
%    CTAN:macros/latex/contrib/pagecolor/pagecolor.pdf
%    CTAN:install/macros/latex/contrib/pagecolor.tds.zip
%
% Cf. http://ctan.org/pkg/pagecolor and
% http://www.tex.ac.uk/tex-archive/help/Catalogue/entries/pagecolor.html
%
% Unpacking:
%    (a) If pagecolor.ins is present:
%           tex pagecolor.ins
%    (b) Without pagecolor.ins:
%           tex pagecolor.dtx
%    (c) If you insist on using LaTeX
%           latex \let\install=y\input{pagecolor.dtx}
%        (quote the arguments according to the demands of your shell)
%
% Documentation:
%    (a) If pagecolor.drv is present:
%           (pdf)latex pagecolor.drv
%           makeindex -s gind.ist pagecolor.idx
%           (pdf)latex pagecolor.drv
%           makeindex -s gind.ist pagecolor.idx
%           (pdf)latex pagecolor.drv
%    (b) Without pagecolor.drv:
%           (pdf)latex pagecolor.dtx
%           makeindex -s gind.ist pagecolor.idx
%           (pdf)latex pagecolor.dtx
%           makeindex -s gind.ist pagecolor.idx
%           (pdf)latex pagecolor.dtx
%
%    The class ltxdoc loads the configuration file ltxdoc.cfg
%    if available. Here you can specify further options, e.g.
%    use DIN A4 as paper format:
%       \PassOptionsToClass{a4paper}{article}
%
% Installation:
%    TDS:tex/latex/pagecolor/pagecolor.sty
%    TDS:doc/latex/pagecolor/pagecolor.pdf
%    TDS:doc/latex/pagecolor/pagecolor-example.tex
%    TDS:doc/latex/pagecolor/pagecolor-example.pdf
%    TDS:source/latex/pagecolor/pagecolor.dtx
%
%<*ignore>
\begingroup
  \catcode123=1 %
  \catcode125=2 %
  \def\x{LaTeX2e}%
\expandafter\endgroup
\ifcase 0\ifx\install y1\fi\expandafter
         \ifx\csname processbatchFile\endcsname\relax\else1\fi
         \ifx\fmtname\x\else 1\fi\relax
\else\csname fi\endcsname
%</ignore>
%<*install>
\input docstrip.tex
\Msg{*****************************************************************}
\Msg{* Installation                                                   }
\Msg{* Package: pagecolor 2012/02/23 v1.0e Provides thepagecolor (HMM)}
\Msg{*****************************************************************}

\keepsilent
\askforoverwritefalse

\let\MetaPrefix\relax
\preamble

This is a generated file.

Project: pagecolor
Version: 2012/02/23 v1.0e

Copyright (C) 2011 - 2012 by
    H.-Martin M"unch <Martin dot Muench at Uni-Bonn dot de>

The usual disclaimer applies:
If it doesn't work right that's your problem.
(Nevertheless, send an e-mail to the maintainer
 when you find an error in this package.)

This work may be distributed and/or modified under the
conditions of the LaTeX Project Public License, either
version 1.3c of this license or (at your option) any later
version. This version of this license is in
   http://www.latex-project.org/lppl/lppl-1-3c.txt
and the latest version of this license is in
   http://www.latex-project.org/lppl.txt
and version 1.3c or later is part of all distributions of
LaTeX version 2005/12/01 or later.

This work has the LPPL maintenance status "maintained".

The Current Maintainer of this work is H.-Martin Muench.

This work consists of the main source file pagecolor.dtx,
the README, and the derived files
   pagecolor.sty, pagecolor.pdf,
   pagecolor.ins, pagecolor.drv,
   pagecolor-example.tex, pagecolor-example.pdf.

\endpreamble
\let\MetaPrefix\DoubleperCent

\generate{%
  \file{pagecolor.ins}{\from{pagecolor.dtx}{install}}%
  \file{pagecolor.drv}{\from{pagecolor.dtx}{driver}}%
  \usedir{tex/latex/pagecolor}%
  \file{pagecolor.sty}{\from{pagecolor.dtx}{package}}%
  \usedir{doc/latex/pagecolor}%
  \file{pagecolor-example.tex}{\from{pagecolor.dtx}{example}}%
}

\catcode32=13\relax% active space
\let =\space%
\Msg{************************************************************************}
\Msg{*}
\Msg{* To finish the installation you have to move the following}
\Msg{* file into a directory searched by TeX:}
\Msg{*}
\Msg{*     pagecolor.sty}
\Msg{*}
\Msg{* To produce the documentation run the file `pagecolor.drv'}
\Msg{* through (pdf)LaTeX, e.g.}
\Msg{*  pdflatex pagecolor.drv}
\Msg{*  makeindex -s gind.ist pagecolor.idx}
\Msg{*  pdflatex pagecolor.drv}
\Msg{*  makeindex -s gind.ist pagecolor.idx}
\Msg{*  pdflatex pagecolor.drv}
\Msg{*}
\Msg{* At least three runs are necessary e.g. to get the}
\Msg{*  references right!}
\Msg{*}
\Msg{* Happy TeXing!}
\Msg{*}
\Msg{************************************************************************}

\endbatchfile
%</install>
%<*ignore>
\fi
%</ignore>
%
% \section{The documentation driver file}
%
% The next bit of code contains the documentation driver file for
% \TeX{}, i.\,e., the file that will produce the documentation you
% are currently reading. It will be extracted from this file by the
% \texttt{docstrip} programme. That is, run \LaTeX{} on \texttt{docstrip}
% and specify the \texttt{driver} option when \texttt{docstrip}
% asks for options.
%
%    \begin{macrocode}
%<*driver>
\NeedsTeXFormat{LaTeX2e}[2011/06/27]
\ProvidesFile{pagecolor.drv}%
  [2012/02/23 v1.0e Provides thepagecolor (HMM)]
\documentclass{ltxdoc}[2007/11/11]% v2.0u
\usepackage{holtxdoc}[2011/02/04]%  v0.21
%% pagecolor may work with earlier versions of LaTeX2e and those
%% class and package, but this was not tested.
%% Please consider updating your LaTeX, class, and package
%% to the most recent version (if they are not already the most
%% recent version).
\hypersetup{%
 pdfsubject={Providing the thepagecolor command (HMM)},%
 pdfkeywords={LaTeX, pagecolor, thepagecolor, H.-Martin Muench},%
 pdfencoding=auto,%
 pdflang={en},%
 breaklinks=true,%
 linktoc=all,%
 pdfstartview=FitH,%
 pdfpagelayout=OneColumn,%
 bookmarksnumbered=true,%
 bookmarksopen=true,%
 bookmarksopenlevel=3,%
 pdfmenubar=true,%
 pdftoolbar=true,%
 pdfwindowui=true,%
 pdfnewwindow=true%
}
\CodelineIndex
\hyphenation{printing docu-ment}
\gdef\unit#1{\mathord{\thinspace\mathrm{#1}}}%
\begin{document}
  \DocInput{pagecolor.dtx}%
\end{document}
%</driver>
%    \end{macrocode}
%
% \fi
%
% \CheckSum{137}
%
% \CharacterTable
%  {Upper-case    \A\B\C\D\E\F\G\H\I\J\K\L\M\N\O\P\Q\R\S\T\U\V\W\X\Y\Z
%   Lower-case    \a\b\c\d\e\f\g\h\i\j\k\l\m\n\o\p\q\r\s\t\u\v\w\x\y\z
%   Digits        \0\1\2\3\4\5\6\7\8\9
%   Exclamation   \!     Double quote  \"     Hash (number) \#
%   Dollar        \$     Percent       \%     Ampersand     \&
%   Acute accent  \'     Left paren    \(     Right paren   \)
%   Asterisk      \*     Plus          \+     Comma         \,
%   Minus         \-     Point         \.     Solidus       \/
%   Colon         \:     Semicolon     \;     Less than     \<
%   Equals        \=     Greater than  \>     Question mark \?
%   Commercial at \@     Left bracket  \[     Backslash     \\
%   Right bracket \]     Circumflex    \^     Underscore    \_
%   Grave accent  \`     Left brace    \{     Vertical bar  \|
%   Right brace   \}     Tilde         \~}
%
% \GetFileInfo{pagecolor.drv}
%
% \begingroup
%   \def\x{\#,\$,\^,\_,\~,\ ,\&,\{,\},\%}%
%   \makeatletter
%   \@onelevel@sanitize\x
% \expandafter\endgroup
% \expandafter\DoNotIndex\expandafter{\x}
% \expandafter\DoNotIndex\expandafter{\string\ }
% \begingroup
%   \makeatletter
%     \lccode`9=32\relax
%     \lowercase{%^^A
%       \edef\x{\noexpand\DoNotIndex{\@backslashchar9}}%^^A
%     }%^^A
%   \expandafter\endgroup\x
%
% \DoNotIndex{\,,\\}
% \DoNotIndex{\documentclass,\usepackage,\ProvidesPackage,\begin,\end}
% \DoNotIndex{\MessageBreak}
% \DoNotIndex{\NeedsTeXFormat,\DoNotIndex,\verb}
% \DoNotIndex{\def,\edef,\gdef,\global}
% \DoNotIndex{\ifx,\listfiles,\mathord,\mathrm}
% \DoNotIndex{\kvoptions,\SetupKeyvalOptions,\ProcessKeyvalOptions}
% \DoNotIndex{\bigskip,\space,\thinspace,\ldots}
% \DoNotIndex{\indent,\noindent,\newline,\linebreak,\pagebreak,\newpage}
% \DoNotIndex{\textbf,\textit,\textsf,\texttt,\textsc,\textquotedblleft,\textquotedblright}
% \DoNotIndex{\plainTeX,\TeX,\LaTeX,\pdfLaTeX}
% \DoNotIndex{\section,\color,\pagenumbering,\smallskip}
% \DoNotIndex{\@ifundefined,\@ifpackageloaded}
% \DoNotIndex{\DeclareStringOption}
%
% \title{The \xpackage{pagecolor} package}
% \date{2012/02/23 v1.0e}
% \author{H.-Martin M\"{u}nch\\\xemail{Martin.Muench at Uni-Bonn.de}}
%
% \maketitle
%
% \begin{abstract}
%  This \LaTeX{} package provides the command |\thepagecolor|,
% which gives the current page (background) colour,
% i.\,e.~the argument used with the most recent call of |\pagecolor{...}|.
% The command |\thepagecolornone| gives the same colour as |\thepagecolor|,
% except when the page background colour is
% \textquotedblleft|none|\textquotedblright. In that case |\thepagecolor| is
% |white| and |\thepagecolornone| is |none|.\\
% Similar to |\newgeometry| and |\restoregeometry| of the \xpackage{geometry}
% package\\
% |\newpagecolor{<|\textit{some colour} |>}| and |\restorepagecolor|
% are provided.
% \end{abstract}
%
% \bigskip
%
% \noindent Disclaimer for web links: The author is not responsible for any contents
% referred to in this work unless he has full knowledge of illegal contents.
% If any damage occurs by the use of information presented there, only the
% author of the respective pages might be liable, not the one who has referred
% to these pages.
%
% \bigskip
%
% \noindent {\color{green} Save per page about $200\unit{ml}$ water,
% $2\unit{g}$ CO$_{2}$ and $2\unit{g}$ wood:\\
% Therefore please print only if this is really necessary.}
%
% \pagebreak
%
% \tableofcontents
%
% \bigskip
%
% \section{Introduction\label{sec:Introduction}}
%
% \indent This \LaTeX{} package provides the command |\thepagecolor|,
% which gives the current page (background) colour,
% i.\,e.~the argument used with the most recent call of |\pagecolor{...}|.
% The package should be loaded before any package sets a page (background)
% colour, but after \xpackage{color} or \xpackage{xcolor} package.
% Its option |pagecolor={...}| is used to set the |\pagecolor{...}|.\\
% The command |\thepagecolornone| gives the same colour as |\thepagecolor|,
% except when the page background colour is
% \textquotedblleft|none|\textquotedblright (e.\,g. result of using the
% |\nopagecolor| command). In that case |\thepagecolor| is |white| and\linebreak
% |\thepagecolornone| is |none|.\\
% Similar to |\newgeometry| and |\restoregeometry| of the \xpackage{geometry}
% package\\
% |\newpagecolor{<|\textit{some colour}|>}| and |\restorepagecolor|
% are provided.
%
% \pagebreak
%
% \section{Usage}
%
% \indent Just load the package placing
% \begin{quote}
%   |\usepackage[<|\textit{option}|>]{pagecolor}|
% \end{quote}
% \noindent in the preamble of your \LaTeXe{} source file.
% This should be done before another package uses |\pagecolor|,
% but after |\nopagecolor| is defined (if that is defined in the document
% at all).\\
% Afterwards |\pagecolor{...}| can be used to change the page (background)
% colour as usual. Then |\thepagecolor| gives the current page (background)
% colour (in the same format as given with |\pagecolor{...}|).\\
% Similar to |\newgeometry| and |\restoregeometry| of the \xpackage{geometry}
% package\\
% |\newpagecolor{<|\textit{some colour} |>}| and |\restorepagecolor|
% are provided:\\
% |\newpagecolor{<|\textit{some colour} |>}| will execute
% |\pagecolor{<|\textit{some colour} |>}| and remember the page colour used before.
% |\restorecolor| (without argument) restores the page colour to the one used before
%  use of the |\newpagecolor{...}| command. When you want to change the colour for
%  just one page and do not want to (or cannot) manually determine where the page ends,\\
% |\newpagecolor{<|\textit{some colour}|>}\afterpage{\restorepagecolor}|\\
% does the trick (and requires a |\usepackage{afterpage}| in the document's
% preamble), or for short\\
% |\newcommand{\onepagecolor}[1]{%|\\
% |\newpagecolor{#1}\afterpage{\restorepagecolor}}|\\
% in the preamble and\\
% |\onepagecolor{<|\textit{some colour}|>}| in the document.
%
% \subsection{Option}
% \DescribeMacro{option}
% \indent The \xpackage{pagecolor} package takes the following option:
%
% \subsubsection{pagecolor\label{sss:pagecolor}}
% \DescribeMacro{pagecolor}
% \indent The option |pagecolor={...}| takes as value a colour. This
% could be as simple as |white| or |black|, but when e.\,g.~the
% \xpackage{xcolor} package is used (loaded before \xpackage{pagecolor}!),
% also colours like |red!50!green!20!blue| are possible.
% The default is |pagecolor={none}|.
%
% \section{Alternatives\label{sec:Alternatives}}
%
% As I neither know what exactly you want to accomplish
% when using this package (e.\,g.~hiding text), nor what resources you have
% (e.\,g.~\pdfTeX{} version), here is a list of some possible alternatives:\\
%
% \begin{description}
%  \item[-]
%  \DescribeMacro{transparent}
%  \xpackage{transparent} package: With it some object can be made
%  (fully or partially) transparent.
% \end{description}
%
% \begin{description}
%  \item[-]
%  \DescribeMacro{hrefhide}\xpackage{hrefhide} package: It allows to
%  \textquotedblleft hide\textquotedblright\ some (hyperlinked) text when
%  printing the document while keeping the layout.
% \end{description}
%
% \bigskip
%
% \noindent (You programmed or found another alternative,
%  which is available at \CTAN{}?\\
%  OK, send an e-mail to me with the name, location at \CTAN{},
%  and a short notice, and I will probably include it in
%  the list above.)\\
%
% \smallskip
%
% \noindent About how to get those packages, please see
% subsection~\ref{ss:Downloads}.
%
% \newpage
%
% \section{Example}
%
%    \begin{macrocode}
%<*example>
\documentclass[british]{article}[2007/10/19]% v1.4h
%%%%%%%%%%%%%%%%%%%%%%%%%%%%%%%%%%%%%%%%%%%%%%%%%%%%%%%%%%%%%%%%%%%%%
\usepackage[%
 extension=pdf,%
 plainpages=false,%
 pdfpagelabels=true,%
 hyperindex=false,%
 pdflang={en},%
 pdftitle={pagecolor package example},%
 pdfauthor={H.-Martin Muench},%
 pdfsubject={Example for the pagecolor package},%
 pdfkeywords={LaTeX, pagecolor, thepagecolor, page colour,%
  H.-Martin Muench},%
 pdfview=Fit,pdfstartview=Fit,%
 pdfpagelayout=SinglePage%
]{hyperref}[2012/02/06]% v6.82o
\usepackage[x11names]{xcolor}[2007/01/21]% v2.11
 % The xcolor package would not be needed for just using
 % the base colours. The color package would be sufficient for that.
\usepackage[pagecolor={LightGoldenrod1}]{pagecolor}[2012/02/23]% v1.0e

\usepackage{afterpage}[1995/10/27]% v1.08
 % The afterpage package is generally not needed,
 % but the |\newpagecolor{somecolour}\afterpage{\restorepagecolor}|
 % construct shall be demonstrated.

\usepackage{lipsum}[2011/04/14]% v1.2
 % The lipsum package is generally not needed,
 % but some blind text is needed for the example.

\gdef\unit#1{\mathord{\thinspace\mathrm{#1}}}%
\listfiles
\begin{document}
\pagenumbering{arabic}
\section*{Example for pagecolor}

This example demonstrates the use of package\newline
\textsf{pagecolor}, v1.0e as of 2012/02/23 (HMM).\newline
The used option was \verb|pagecolor={LightGoldenrod1}|.\newline
\verb|pagecolor={none}| would be the default.\newline

For more details please see the documentation!\newline

\noindent {\color{teal} Save per page about $200\unit{ml}$ water,
$2\unit{g}$ CO$_{2}$ and $2\unit{g}$ wood:\newline
Therefore please print only if this is really necessary.}\newline

The current page (background) colour is\newline
\verb|\thepagecolor|\ =\ \thepagecolor \newline
(and \verb|\thepagecolornone|\ =\ \thepagecolornone ,
which would only be different from \verb|\thepagecolor|,
when the page colour would be \verb|none|).

\pagebreak
\pagecolor{rgb:-green!40!yellow,3;green!40!yellow,2;red,1}

{\color{white} The current page (background) colour is\newline
\verb|\thepagecolor|\ =\ \thepagecolor . \newline}

{\color{\thepagecolor} And that makes this text practically invisible.
\newline}

{\color{white} Which made the preceding line of text practically
invisible.}

\pagebreak
\newpagecolor{red}

This page uses \verb|\newpagecolor{red}|.

\pagebreak
\restorepagecolor

{\color{white}And this page uses \verb|\restorepagecolor| to restore
the page colour to the value it had before the red page.}

\pagebreak
\pagecolor{none}

This page uses \verb|\pagecolor{none}|. If the \verb|\nopagecolor|
command is known (pdf\TeX and Lua\TeX; not yet for dvips, dvipdfm(x)
or Xe\TeX), the page colour is now \verb|none|, otherwise \verb|white|:
\verb|\thepagecolor|\ =\ \thepagecolor\ and
\verb|\thepagecolornone|\ =\ \thepagecolornone .

\pagebreak
\restorepagecolor

{\color{white}\verb|\restorepagecolor| restored the page colour again.}

\pagebreak
\pagecolor{green}

This page is green due to \verb|\pagecolor{green}|.

\pagebreak
\newpagecolor{blue}\afterpage{\restorepagecolor}

{\color{white}\verb|\newpagecolor{blue}\afterpage{\restorepagecolor}|%
\newline
was used here, i.\,e.~this page is blue, and the next one will
automatically have the same page colour before it was changed to blue
here (i.\,e. green).}

\smallskip
{\color{red}\textbf{\lipsum[1-11]}}
\bigskip

The page colour was changed back at the end of the page -
in mid-sentence!

\end{document}
%</example>
%    \end{macrocode}
%
% \StopEventually{}
%
% \newpage
%
% \section{The implementation}
%
% We start off by checking that we are loading into \LaTeXe{} and
% announcing the name and version of this package.
%
%    \begin{macrocode}
%<*package>
%    \end{macrocode}
%
%    \begin{macrocode}
\NeedsTeXFormat{LaTeX2e}[2011/06/27]
\ProvidesPackage{pagecolor}[2012/02/23 v1.0e
            Provides thepagecolor (HMM)]
%    \end{macrocode}
%
% A short description of the \xpackage{pagecolor} package:
%
%    \begin{macrocode}
%% Provides the \thepagecolor, \thepagecolornone, \newpagecolor{...},
%% and \restorepagecolor commands.
%    \end{macrocode}
%
% We need the \xpackage{kvoptions} package by \textsc{Heiko Oberdiek}:
%
%    \begin{macrocode}
\RequirePackage{kvoptions}[2010/12/23]% v3.10
%    \end{macrocode}
%
% and either the \xpackage{color} or the \xpackage{xcolor} package:
%
%    \begin{macrocode}
%% \RequirePackage{ either color or xcolor }:
\@ifpackageloaded{xcolor}{% xcolor loaded
   \@ifpackagelater{xcolor}{2007/01/21}{%
      % 2007/01/21, v2.11, or even more recent: OK
    }{% else: older package version
      \PackageWarning{pagecolor}{%
        It is requested version `2007/01/21' of package\MessageBreak%
        xcolor, but only an older version is available.\MessageBreak%
       }%
     }%
 }{% xcolor not loaded
   \@ifpackageloaded{color}{%
      \RequirePackage{color}[2005/11/14]% v1.0j
    }{% \else
      \PackageWarning{pagecolor}{%
        The pagecolor package must be loaded after either\MessageBreak%
        package color or after package xcolor (at your\MessageBreak%
        option). Neither package was loaded before package\MessageBreak%
        pagecolor. Loading of package xcolor will now be\MessageBreak%
        tried automatically.\MessageBreak%
        When the pagecolor package is used with option\MessageBreak%
        pagecolor using a colour requiring e. g. x11names\MessageBreak%
        option for xcolor package, this will not work!\MessageBreak%
       }
     }% \fi
   \RequirePackage{xcolor}[2007/01/21]% v2.11
  }% \fi
%    \end{macrocode}
%
% A last information for the user:
%
%    \begin{macrocode}
%% pagecolor may work with earlier versions of LaTeX and the
%% packages, but this was not tested. Please consider updating
%% your LaTeX and packages to the most recent version
%% (if they are not already the most recent version).

%    \end{macrocode}
%
% \indent See subsection~\ref{ss:Downloads} about how to get them.\\
%
% We process the options:
%
%    \begin{macrocode}
\SetupKeyvalOptions{family=pagecolor,prefix=pagecolor@}
\DeclareStringOption[none]{pagecolor}% \pagecolor@pagecolor
\ProcessKeyvalOptions*

%    \end{macrocode}
%
% We save the original |\pagecolor| command,
%
%    \begin{macrocode}
\let\origpagecolour\pagecolor

%    \end{macrocode}
%
%    \begin{macro}{\pagecolor}
% before we redefine it to include a definition of |\thepagecolor|
% and \linebreak
% |\thepagecolornone|:
%
%    \begin{macrocode}
\renewcommand{\pagecolor}[1]{\@bsphack%
  \edef\pagecolourtmpa{#1}%
  \def\pagecolourtmpb{none}%
  \ifx\pagecolourtmpa\pagecolourtmpb
    \@ifundefined{nopagecolor}{%
      \PackageError{pagecolor}{%
        pagecolor=none requested but \string\nopagecolor\space unknown%
       }{%
        \string\pagecolor{none} was used, but the command %
        \string\nopagecolor\space is undefined.\MessageBreak%
        Please use another colour. pagecolor=white will be used now.%
        \MessageBreak%
        }%
      \gdef\thepagecolor{white}%
      \gdef\thepagecolornone{white}% although it should be "none"
      \origpagecolour{white}%
     }{%
      \nopagecolor%
     }%
  \else%
    \xdef\thepagecolor{#1}%
    \xdef\thepagecolornone{#1}%
    \origpagecolour{\thepagecolornone}%
  \fi%
  \@esphack%
}

%    \end{macrocode}
%    \end{macro}
%
% |\nopagecolor| is only defined for \pdfTeX and \LuaTeX, but not yet for dvips,
% dvipdfm(x) or \XeTeX. Maybe
% \begin{verbatim}
% \@ifundefined{nopagecolor}{\newcommand{\nopagecolor}{\pagecolor{white}}}{}
% \end{verbatim}
% could be an alternative.\\
% When |\pagecolor{none}| is used and |\nopagecolor| is defined, then \\
% |\pagecolor{none}| is made into a synonym for |\nopagecolor|.\\
% If |\nopagecolor| has not been defined, nothing needs to be done.
% Otherwise we redefine |\nopagecolor|, thus that |\thepagecolor| is
% set to |white| and |\thepagecolornone| to |none| whenever
% |\nopagecolor| is used.
%
%    \begin{macrocode}
\@ifundefined{nopagecolor}{%
  \PackageWarning{pagecolor}{%
    \string\nopagecolor\space is undefined.\MessageBreak%
    (As long as it is neither used anywhere\MessageBreak%
    \space nor defined later, that will be no problem.)\MessageBreak%
   }
  \AtEndDocument{%
  \@ifundefined{nopagecolor}{% no problem
    }{%
    \PackageError{pagecolor}{%
      \string\nopagecolor\space defined after loading package%
      \MessageBreak%
      pagecolor%
      }{\string\nopagecolor\space was defined after loading the %
       pagecolor package.\MessageBreak%
       Either define \string\nopagecolor\space earlier or load the %
       pagecolor package later.\MessageBreak%
      }
    }%
  }%
  }{% \else
    \let\orignopagecolour\nopagecolor
    \renewcommand{\nopagecolor}{%
      \xdef\thepagecolor{white}
      \xdef\thepagecolornone{none}
      \orignopagecolour
      }
  }

%    \end{macrocode}
%
% If the page colour as given with option |pagecolor={...}| is |none|,
% but |\nopagecolor| is not known, |\pagecolor@pagecolor| is set to
% |white| and a warning is given.
%
%    \begin{macrocode}
\def\pagecolourtmpb{none}
\ifx\pagecolor@pagecolor\pagecolourtmpb
  \@ifundefined{nopagecolor}{%
    \PackageWarning{pagecolor}{%
      Option pagecolor=none (maybe by default) used,\MessageBreak%
      but \string\nopagecolor\space is unknown. Please use another%
      \MessageBreak%
      option value; white will be used now.\MessageBreak%
     }
    \setkeys{pagecolor}{pagecolor=white}%
  }{% ok
  }%
\fi

%    \end{macrocode}
%
% The (new) |\pagecolor| is now just carried out.
%
%    \begin{macrocode}
\pagecolor{\pagecolor@pagecolor}

%    \end{macrocode}
%
% Now the page (background) colour and |\thepagecolor| and \\
% |\thepagecolornone| are |\pagecolor@pagecolor| (or page (background) colour
% and |\thepagecolornone| are |none|=|\pagecolor@pagecolor| and
% |\thepagecolor| is |white|), and when the page (background) colour
% is changed, |\thepagecolor| and |\thepagecolornone| are changed
% accordingly.
%
% \newpage
%
%    \begin{macro}{\newpagecolor}
% There have been requests (via e-mail and at \\
% \url{http://tex.stackexchange.com/q/25137}) to change the colour of just one
% (or two) page(s) only, similar to |\newgeometry| and |\restoregeometry| of the
% \xpackage{geometry} package (\url{http://ctan.org/pkg/geometry}). Therefore
% |\newpagecolor| and |\restorepagecolor| are introduced (as suggested by
% \textsc{haoyun\_tex}):
%
%    \begin{macrocode}
\newcommand{\newpagecolor}[1]{%
\xdef\pagecolourtmpc{\thepagecolornone}%
\pagecolor{#1}%
}

%    \end{macrocode}
%    \end{macro}
%
% |\newpagecolor{<|\textit{some colour}|>}| will execute |\pagecolor{somecolour}|
% and remember the page colour used before.\\
%
%    \begin{macro}{\restorepagecolor}
%    \begin{macrocode}
\newcommand{\restorepagecolor}{\pagecolor{\pagecolourtmpc}}

%    \end{macrocode}
%    \end{macro}
%
% |\restorecolor| (without argument) restores the page colour to the one used before
% use of the |\newpagecolor{...}| command.
%
%    \begin{macrocode}
\gdef\pagecolourtmpc{\thepagecolor}

%    \end{macrocode}
% is just a precaution for |\restorecolor| being used when no |\newpagecolor{...}|
% was used before it.\\
%
% When you want to change the colour for just one page and do not want to
% (or cannot) manually determine where the page ends,\\
% |\newpagecolor{<|\textit{some colour}|>}\afterpage{\restorepagecolor}|\\
% does the trick (and requires an additional |\usepackage{afterpage}| in the
% document's preamble).
%    \begin{macrocode}
%</package>
%    \end{macrocode}
%
% \newpage
%
% \section{Installation}
%
% \subsection{Downloads\label{ss:Downloads}}
%
% Everything is available on \CTAN{}, \url{http://www.ctan.org/tex-archive/},
% but may need additional packages themselves.\\
%
% \DescribeMacro{pagecolor.dtx}
% For unpacking the |pagecolor.dtx| file and constructing the documentation
% it is required:
% \begin{description}
% \item[-] \TeX{} Format \LaTeXe{}, \url{http://www.CTAN.org/}
%
% \item[-] document class \xpackage{ltxdoc}, 2007/11/11, v2.0u,
%           \url{http://ctan.org/pkg/ltxdoc}
%
% \item[-] package \xpackage{holtxdoc}, 2011/02/04, v0.21,
%           \url{http://ctan.org/pkg/holtxdoc}
% \end{description}
%
% \DescribeMacro{pagecolor.sty}
% The |pagecolor.sty| for \LaTeXe{} (i.\,e. each document using
% the \xpackage{pagecolor} package) requires:
% \begin{description}
% \item[-] \TeX{} Format \LaTeXe{}, \url{http://www.CTAN.org/}
%
% \item[-] package \xpackage{kvoptions}, 2010/12/23, v3.10,
%           \url{http://ctan.org/pkg/kvoptions}
% \end{description}
% and either
% \begin{description}
% \item[-] package \xpackage{color}, 2005/11/14, v1.0j,
%           \url{http://ctan.org/pkg/color} (from the\\
%           \xpackage{graphics} package bundle)
% \end{description}
% or
% \begin{description}
% \item[-] package \xpackage{xcolor}, 2007/01/21, v2.11,
%           \url{http://ctan.org/pkg/xcolor}
% \end{description}
%
% \DescribeMacro{pagecolor-example.tex}
% The |pagecolor-example.tex| requires the same files as all
% documents using the \xpackage{pagecolor} package and additionally:
% \begin{description}
% \item[-] class \xpackage{article}, 2007/10/19, v1.4h, from \xpackage{classes.dtx}:\\
%           \CTAN{macros/latex/base/classes.dtx}
%
% \item[-] package \xpackage{pagecolor}, 2012/02/23, v1.0e,
%           \url{http://ctan.org/pkg/pagecolor}\\
%           (Well, it is the example file for this package, and because you are
%            reading the documentation for the \xpackage{pagecolor} package,
%            it can be assumed that you already have some version of it --
%            is it the current one?)
%
% \item[-] package \xpackage{xcolor}, 2007/01/21, v2.11,
%           \url{http://ctan.org/pkg/xcolor}\\
%           This package would not be needed for the use of just base colours only,
%           the \xpackage{color} package would be sufficient for that.
% \end{description}
%
% \DescribeMacro{transparent}
% \DescribeMacro{hrefhide}
% As possible alternatives in section \ref{sec:Alternatives} there are listed
% \begin{description}
% \item[-] package \xpackage{transparent}, 2007/01/08, v1.0, \\
%           \url{http://ctan.org/pkg/transparent}
%
% \item[-] package \xpackage{hrefhide}, 2011/04/29, v1.0f, \\
%           \url{http://ctan.org/pkg/hrefhide}
% \end{description}
%
% \DescribeMacro{Oberdiek}
% \DescribeMacro{holtxdoc}
% \DescribeMacro{kvoptions}
% All packages of \textsc{Heiko Oberdiek}'s bundle `oberdiek'
% (especially \xpackage{holtxdoc} and \xpackage{kvoptions})
% are also available in a TDS compliant ZIP archive:\\
% \CTAN{install/macros/latex/contrib/oberdiek.tds.zip}.\\
% It is probably best to download and use this, because the packages in there
% are quite probably both recent and compatible among themselves.\\
%
% \DescribeMacro{hyperref}
% \noindent \xpackage{hyperref} is not included in that bundle and needs to be
% downloaded separately,\\
% \url{http://mirror.ctan.org/install/macros/latex/contrib/hyperref.tds.zip}.\\
%
% \DescribeMacro{M\"{u}nch}
% A hyperlinked list of my (other) packages can be found at
% \url{http://www.Uni-Bonn.de/~uzs5pv/LaTeX.html}.\\
%
% \subsection{Package, unpacking TDS}
% \paragraph{Package.} This package is available on \CTAN{}
% \begin{description}
% \item[\CTAN{macros/latex/contrib/pagecolor/pagecolor.dtx}]\hspace*{0.1cm} \\
%       The source file.
% \item[\CTAN{macros/latex/contrib/pagecolor/pagecolor.pdf}]\hspace*{0.1cm} \\
%       The documentation.
% \item[\CTAN{macros/latex/contrib/pagecolor/pagecolor-example.pdf}]\hspace*{0.1cm} \\
%       The compiled example file, as it should look like.
% \item[\CTAN{macros/latex/contrib/pagecolor/README}]\hspace*{0.1cm} \\
%       The README file.
% \end{description}
% There is also a pagecolor.tds.zip available:
% \begin{description}
% \item[\CTAN{install/macros/latex/contrib/pagecolor.tds.zip}]\hspace*{0.1cm} \\
%       Everything in \xfile{TDS} compliant, compiled format.
% \end{description}
% which additionally contains\\
% \begin{tabular}{ll}
% pagecolor.ins & The installation file.\\
% pagecolor.drv & The driver to generate the documentation.\\
% pagecolor.sty & The \xext{sty}le file.\\
% pagecolor-example.tex & The example file.
% \end{tabular}
%
% \bigskip
%
% \noindent For required other packages, please see the preceding subsection.
%
% \paragraph{Unpacking.} The \xfile{.dtx} file is a self-extracting
% \docstrip{} archive. The files are extracted by running the
% \xext{.dtx} through \plainTeX{}:
% \begin{quote}
%   \verb|tex pagecolor.dtx|
% \end{quote}
%
% About generating the documentation see paragraph~\ref{GenDoc} below.\\
%
% \paragraph{TDS.} Now the different files must be moved into
% the different directories in your installation TDS tree
% (also known as \xfile{texmf} tree):
% \begin{quote}
% \def\t{^^A
% \begin{tabular}{@{}>{\ttfamily}l@{ $\rightarrow$ }>{\ttfamily}l@{}}
%   pagecolor.sty & tex/latex/pagecolor/pagecolor.sty\\
%   pagecolor.pdf & doc/latex/pagecolor/pagecolor.pdf\\
%   pagecolor-example.tex & doc/latex/pagecolor/pagecolor-example.tex\\
%   pagecolor-example.pdf & doc/latex/pagecolor/pagecolor-example.pdf\\
%   pagecolor.dtx & source/latex/pagecolor/pagecolor.dtx\\
% \end{tabular}^^A
% }^^A
% \sbox0{\t}^^A
% \ifdim\wd0>\linewidth
%   \begingroup
%     \advance\linewidth by\leftmargin
%     \advance\linewidth by\rightmargin
%   \edef\x{\endgroup
%     \def\noexpand\lw{\the\linewidth}^^A
%   }\x
%   \def\lwbox{^^A
%     \leavevmode
%     \hbox to \linewidth{^^A
%       \kern-\leftmargin\relax
%       \hss
%       \usebox0
%       \hss
%       \kern-\rightmargin\relax
%     }^^A
%   }^^A
%   \ifdim\wd0>\lw
%     \sbox0{\small\t}^^A
%     \ifdim\wd0>\linewidth
%       \ifdim\wd0>\lw
%         \sbox0{\footnotesize\t}^^A
%         \ifdim\wd0>\linewidth
%           \ifdim\wd0>\lw
%             \sbox0{\scriptsize\t}^^A
%             \ifdim\wd0>\linewidth
%               \ifdim\wd0>\lw
%                 \sbox0{\tiny\t}^^A
%                 \ifdim\wd0>\linewidth
%                   \lwbox
%                 \else
%                   \usebox0
%                 \fi
%               \else
%                 \lwbox
%               \fi
%             \else
%               \usebox0
%             \fi
%           \else
%             \lwbox
%           \fi
%         \else
%           \usebox0
%         \fi
%       \else
%         \lwbox
%       \fi
%     \else
%       \usebox0
%     \fi
%   \else
%     \lwbox
%   \fi
% \else
%   \usebox0
% \fi
% \end{quote}
% If you have a \xfile{docstrip.cfg} that configures and enables \docstrip{}'s
% \xfile{TDS} installing feature, then some files can already be in the right
% place, see the documentation of \docstrip{}.
%
% \subsection{Refresh file name databases}
%
% If your \TeX{}~distribution (\teTeX{}, \mikTeX{},\dots{}) relies on file name
% databases, you must refresh these. For example, \teTeX{} users run
% \verb|texhash| or \verb|mktexlsr|.
%
% \subsection{Some details for the interested}
%
% \paragraph{Unpacking with \LaTeX{}.}
% The \xfile{.dtx} chooses its action depending on the format:
% \begin{description}
% \item[\plainTeX:] Run \docstrip{} and extract the files.
% \item[\LaTeX:] Generate the documentation.
% \end{description}
% If you insist on using \LaTeX{} for \docstrip{} (really,
% \docstrip{} does not need \LaTeX{}), then inform the autodetect routine
% about your intention:
% \begin{quote}
%   \verb|latex \let\install=y\input{pagecolor.dtx}|
% \end{quote}
% Do not forget to quote the argument according to the demands
% of your shell.
%
% \paragraph{Generating the documentation.\label{GenDoc}}
% You can use both the \xfile{.dtx} or the \xfile{.drv} to generate
% the documentation. The process can be configured by a
% configuration file \xfile{ltxdoc.cfg}. For instance, put the following
% line into this file, if you want to have A4 as paper format:
% \begin{quote}
%   \verb|\PassOptionsToClass{a4paper}{article}|
% \end{quote}
%
% \noindent An example follows how to generate the
% documentation with \pdfLaTeX{}:
%
% \begin{quote}
%\begin{verbatim}
%pdflatex pagecolor.dtx
%makeindex -s gind.ist pagecolor.idx
%pdflatex pagecolor.dtx
%makeindex -s gind.ist pagecolor.idx
%pdflatex pagecolor.dtx
%\end{verbatim}
% \end{quote}
%
% \subsection{Compiling the example}
%
% The example file, \textsf{pagecolor-example.tex}, can be compiled via\\
% |(pdf)latex pagecolor-example.tex|.\\
%
% \section{Acknowledgements}
%
% I would like to thank \textsc{Heiko Oberdiek} for providing a~lot~(!) of
% useful packages (from which I also got everything I know about creating
% a file in \xext{dtx} format, ok, say it: copying), and the
% \Newsgroup{comp.text.tex} and \Newsgroup{de.comp.text.tex} newsgroups
% for their help in all things \TeX{}, especially all contributors to the
% discussion at
% \url{http://groups.google.com/group/comp.text.tex/browse_thread/thread/533576ebe44d60f6/e1654d70a19de63c?lnk=gst&q=Determination+of+the+current+page+colour#e1654d70a19de63c}
% \ (\textsc{H.~Oberdiek} \& \textsc{Gouailles}).\\
% I thank \textsc{haoyun\_tex} for suggesting the |\newpagecolor|/|\restorepagecolor|
% pair of commands and everyone at \url{http://tex.stackexchange.com/q/25137}
% for their contributions there.
%
% \bigskip
%
% \phantomsection
% \begin{History}\label{History}
%   \begin{Version}{2011/07/16 v1.0a}
%     \item First version discussed at \Newsgroup{comp.text.tex}.
%   \end{Version}
%   \begin{Version}{2011/08/06 v1.0b}
%     \item Changed version uploaded to \CTAN{}.
%   \end{Version}
%   \begin{Version}{2011/08/08 v1.0c}
%     \item Fixed a |\setkeys|.
%   \end{Version}
%   \begin{Version}{2012/02/01 v1.0d}
%     \item Bugfix: Obsolete installation path given in the documentation, updated.
%     \item New commands: |\newpagecolor{...}|, |\restorepagecolor|.
%     \item Update of documentation, README, and \xfile{dtx} internals.
%   \end{Version}
%   \begin{Version}{2012/02/23 v1.0e}
%     \item Fixed an error in the documentation.
%     \item Check for loading of \xpackage{color} or \xpackage{xcolor} package
%             and their versions has been changed, because \xpackage{xcolor} sets\\
%             |\@namedef{ver@color.sty}{1999/02/16}|\\
%             which gave a warning about old \xpackage{color} package even if a new
%             version was used.
%   \end{Version}
% \end{History}
%
% \bigskip
%
% When you find a mistake or have a suggestion for an improvement of this package,
% please send an e-mail to the maintainer, thanks! (Please see BUG REPORTS in the README.)
%
% \newpage
%
% \PrintIndex
%
% \Finale
\endinput
%        (quote the arguments according to the demands of your shell)
%
% Documentation:
%    (a) If pagecolor.drv is present:
%           (pdf)latex pagecolor.drv
%           makeindex -s gind.ist pagecolor.idx
%           (pdf)latex pagecolor.drv
%           makeindex -s gind.ist pagecolor.idx
%           (pdf)latex pagecolor.drv
%    (b) Without pagecolor.drv:
%           (pdf)latex pagecolor.dtx
%           makeindex -s gind.ist pagecolor.idx
%           (pdf)latex pagecolor.dtx
%           makeindex -s gind.ist pagecolor.idx
%           (pdf)latex pagecolor.dtx
%
%    The class ltxdoc loads the configuration file ltxdoc.cfg
%    if available. Here you can specify further options, e.g.
%    use DIN A4 as paper format:
%       \PassOptionsToClass{a4paper}{article}
%
% Installation:
%    TDS:tex/latex/pagecolor/pagecolor.sty
%    TDS:doc/latex/pagecolor/pagecolor.pdf
%    TDS:doc/latex/pagecolor/pagecolor-example.tex
%    TDS:doc/latex/pagecolor/pagecolor-example.pdf
%    TDS:source/latex/pagecolor/pagecolor.dtx
%
%<*ignore>
\begingroup
  \catcode123=1 %
  \catcode125=2 %
  \def\x{LaTeX2e}%
\expandafter\endgroup
\ifcase 0\ifx\install y1\fi\expandafter
         \ifx\csname processbatchFile\endcsname\relax\else1\fi
         \ifx\fmtname\x\else 1\fi\relax
\else\csname fi\endcsname
%</ignore>
%<*install>
\input docstrip.tex
\Msg{*****************************************************************}
\Msg{* Installation                                                   }
\Msg{* Package: pagecolor 2012/02/23 v1.0e Provides thepagecolor (HMM)}
\Msg{*****************************************************************}

\keepsilent
\askforoverwritefalse

\let\MetaPrefix\relax
\preamble

This is a generated file.

Project: pagecolor
Version: 2012/02/23 v1.0e

Copyright (C) 2011 - 2012 by
    H.-Martin M"unch <Martin dot Muench at Uni-Bonn dot de>

The usual disclaimer applies:
If it doesn't work right that's your problem.
(Nevertheless, send an e-mail to the maintainer
 when you find an error in this package.)

This work may be distributed and/or modified under the
conditions of the LaTeX Project Public License, either
version 1.3c of this license or (at your option) any later
version. This version of this license is in
   http://www.latex-project.org/lppl/lppl-1-3c.txt
and the latest version of this license is in
   http://www.latex-project.org/lppl.txt
and version 1.3c or later is part of all distributions of
LaTeX version 2005/12/01 or later.

This work has the LPPL maintenance status "maintained".

The Current Maintainer of this work is H.-Martin Muench.

This work consists of the main source file pagecolor.dtx,
the README, and the derived files
   pagecolor.sty, pagecolor.pdf,
   pagecolor.ins, pagecolor.drv,
   pagecolor-example.tex, pagecolor-example.pdf.

\endpreamble
\let\MetaPrefix\DoubleperCent

\generate{%
  \file{pagecolor.ins}{\from{pagecolor.dtx}{install}}%
  \file{pagecolor.drv}{\from{pagecolor.dtx}{driver}}%
  \usedir{tex/latex/pagecolor}%
  \file{pagecolor.sty}{\from{pagecolor.dtx}{package}}%
  \usedir{doc/latex/pagecolor}%
  \file{pagecolor-example.tex}{\from{pagecolor.dtx}{example}}%
}

\catcode32=13\relax% active space
\let =\space%
\Msg{************************************************************************}
\Msg{*}
\Msg{* To finish the installation you have to move the following}
\Msg{* file into a directory searched by TeX:}
\Msg{*}
\Msg{*     pagecolor.sty}
\Msg{*}
\Msg{* To produce the documentation run the file `pagecolor.drv'}
\Msg{* through (pdf)LaTeX, e.g.}
\Msg{*  pdflatex pagecolor.drv}
\Msg{*  makeindex -s gind.ist pagecolor.idx}
\Msg{*  pdflatex pagecolor.drv}
\Msg{*  makeindex -s gind.ist pagecolor.idx}
\Msg{*  pdflatex pagecolor.drv}
\Msg{*}
\Msg{* At least three runs are necessary e.g. to get the}
\Msg{*  references right!}
\Msg{*}
\Msg{* Happy TeXing!}
\Msg{*}
\Msg{************************************************************************}

\endbatchfile
%</install>
%<*ignore>
\fi
%</ignore>
%
% \section{The documentation driver file}
%
% The next bit of code contains the documentation driver file for
% \TeX{}, i.\,e., the file that will produce the documentation you
% are currently reading. It will be extracted from this file by the
% \texttt{docstrip} programme. That is, run \LaTeX{} on \texttt{docstrip}
% and specify the \texttt{driver} option when \texttt{docstrip}
% asks for options.
%
%    \begin{macrocode}
%<*driver>
\NeedsTeXFormat{LaTeX2e}[2011/06/27]
\ProvidesFile{pagecolor.drv}%
  [2012/02/23 v1.0e Provides thepagecolor (HMM)]
\documentclass{ltxdoc}[2007/11/11]% v2.0u
\usepackage{holtxdoc}[2011/02/04]%  v0.21
%% pagecolor may work with earlier versions of LaTeX2e and those
%% class and package, but this was not tested.
%% Please consider updating your LaTeX, class, and package
%% to the most recent version (if they are not already the most
%% recent version).
\hypersetup{%
 pdfsubject={Providing the thepagecolor command (HMM)},%
 pdfkeywords={LaTeX, pagecolor, thepagecolor, H.-Martin Muench},%
 pdfencoding=auto,%
 pdflang={en},%
 breaklinks=true,%
 linktoc=all,%
 pdfstartview=FitH,%
 pdfpagelayout=OneColumn,%
 bookmarksnumbered=true,%
 bookmarksopen=true,%
 bookmarksopenlevel=3,%
 pdfmenubar=true,%
 pdftoolbar=true,%
 pdfwindowui=true,%
 pdfnewwindow=true%
}
\CodelineIndex
\hyphenation{printing docu-ment}
\gdef\unit#1{\mathord{\thinspace\mathrm{#1}}}%
\begin{document}
  \DocInput{pagecolor.dtx}%
\end{document}
%</driver>
%    \end{macrocode}
%
% \fi
%
% \CheckSum{137}
%
% \CharacterTable
%  {Upper-case    \A\B\C\D\E\F\G\H\I\J\K\L\M\N\O\P\Q\R\S\T\U\V\W\X\Y\Z
%   Lower-case    \a\b\c\d\e\f\g\h\i\j\k\l\m\n\o\p\q\r\s\t\u\v\w\x\y\z
%   Digits        \0\1\2\3\4\5\6\7\8\9
%   Exclamation   \!     Double quote  \"     Hash (number) \#
%   Dollar        \$     Percent       \%     Ampersand     \&
%   Acute accent  \'     Left paren    \(     Right paren   \)
%   Asterisk      \*     Plus          \+     Comma         \,
%   Minus         \-     Point         \.     Solidus       \/
%   Colon         \:     Semicolon     \;     Less than     \<
%   Equals        \=     Greater than  \>     Question mark \?
%   Commercial at \@     Left bracket  \[     Backslash     \\
%   Right bracket \]     Circumflex    \^     Underscore    \_
%   Grave accent  \`     Left brace    \{     Vertical bar  \|
%   Right brace   \}     Tilde         \~}
%
% \GetFileInfo{pagecolor.drv}
%
% \begingroup
%   \def\x{\#,\$,\^,\_,\~,\ ,\&,\{,\},\%}%
%   \makeatletter
%   \@onelevel@sanitize\x
% \expandafter\endgroup
% \expandafter\DoNotIndex\expandafter{\x}
% \expandafter\DoNotIndex\expandafter{\string\ }
% \begingroup
%   \makeatletter
%     \lccode`9=32\relax
%     \lowercase{%^^A
%       \edef\x{\noexpand\DoNotIndex{\@backslashchar9}}%^^A
%     }%^^A
%   \expandafter\endgroup\x
%
% \DoNotIndex{\,,\\}
% \DoNotIndex{\documentclass,\usepackage,\ProvidesPackage,\begin,\end}
% \DoNotIndex{\MessageBreak}
% \DoNotIndex{\NeedsTeXFormat,\DoNotIndex,\verb}
% \DoNotIndex{\def,\edef,\gdef,\global}
% \DoNotIndex{\ifx,\listfiles,\mathord,\mathrm}
% \DoNotIndex{\kvoptions,\SetupKeyvalOptions,\ProcessKeyvalOptions}
% \DoNotIndex{\bigskip,\space,\thinspace,\ldots}
% \DoNotIndex{\indent,\noindent,\newline,\linebreak,\pagebreak,\newpage}
% \DoNotIndex{\textbf,\textit,\textsf,\texttt,\textsc,\textquotedblleft,\textquotedblright}
% \DoNotIndex{\plainTeX,\TeX,\LaTeX,\pdfLaTeX}
% \DoNotIndex{\section,\color,\pagenumbering,\smallskip}
% \DoNotIndex{\@ifundefined,\@ifpackageloaded}
% \DoNotIndex{\DeclareStringOption}
%
% \title{The \xpackage{pagecolor} package}
% \date{2012/02/23 v1.0e}
% \author{H.-Martin M\"{u}nch\\\xemail{Martin.Muench at Uni-Bonn.de}}
%
% \maketitle
%
% \begin{abstract}
%  This \LaTeX{} package provides the command |\thepagecolor|,
% which gives the current page (background) colour,
% i.\,e.~the argument used with the most recent call of |\pagecolor{...}|.
% The command |\thepagecolornone| gives the same colour as |\thepagecolor|,
% except when the page background colour is
% \textquotedblleft|none|\textquotedblright. In that case |\thepagecolor| is
% |white| and |\thepagecolornone| is |none|.\\
% Similar to |\newgeometry| and |\restoregeometry| of the \xpackage{geometry}
% package\\
% |\newpagecolor{<|\textit{some colour} |>}| and |\restorepagecolor|
% are provided.
% \end{abstract}
%
% \bigskip
%
% \noindent Disclaimer for web links: The author is not responsible for any contents
% referred to in this work unless he has full knowledge of illegal contents.
% If any damage occurs by the use of information presented there, only the
% author of the respective pages might be liable, not the one who has referred
% to these pages.
%
% \bigskip
%
% \noindent {\color{green} Save per page about $200\unit{ml}$ water,
% $2\unit{g}$ CO$_{2}$ and $2\unit{g}$ wood:\\
% Therefore please print only if this is really necessary.}
%
% \pagebreak
%
% \tableofcontents
%
% \bigskip
%
% \section{Introduction\label{sec:Introduction}}
%
% \indent This \LaTeX{} package provides the command |\thepagecolor|,
% which gives the current page (background) colour,
% i.\,e.~the argument used with the most recent call of |\pagecolor{...}|.
% The package should be loaded before any package sets a page (background)
% colour, but after \xpackage{color} or \xpackage{xcolor} package.
% Its option |pagecolor={...}| is used to set the |\pagecolor{...}|.\\
% The command |\thepagecolornone| gives the same colour as |\thepagecolor|,
% except when the page background colour is
% \textquotedblleft|none|\textquotedblright (e.\,g. result of using the
% |\nopagecolor| command). In that case |\thepagecolor| is |white| and\linebreak
% |\thepagecolornone| is |none|.\\
% Similar to |\newgeometry| and |\restoregeometry| of the \xpackage{geometry}
% package\\
% |\newpagecolor{<|\textit{some colour}|>}| and |\restorepagecolor|
% are provided.
%
% \pagebreak
%
% \section{Usage}
%
% \indent Just load the package placing
% \begin{quote}
%   |\usepackage[<|\textit{option}|>]{pagecolor}|
% \end{quote}
% \noindent in the preamble of your \LaTeXe{} source file.
% This should be done before another package uses |\pagecolor|,
% but after |\nopagecolor| is defined (if that is defined in the document
% at all).\\
% Afterwards |\pagecolor{...}| can be used to change the page (background)
% colour as usual. Then |\thepagecolor| gives the current page (background)
% colour (in the same format as given with |\pagecolor{...}|).\\
% Similar to |\newgeometry| and |\restoregeometry| of the \xpackage{geometry}
% package\\
% |\newpagecolor{<|\textit{some colour} |>}| and |\restorepagecolor|
% are provided:\\
% |\newpagecolor{<|\textit{some colour} |>}| will execute
% |\pagecolor{<|\textit{some colour} |>}| and remember the page colour used before.
% |\restorecolor| (without argument) restores the page colour to the one used before
%  use of the |\newpagecolor{...}| command. When you want to change the colour for
%  just one page and do not want to (or cannot) manually determine where the page ends,\\
% |\newpagecolor{<|\textit{some colour}|>}\afterpage{\restorepagecolor}|\\
% does the trick (and requires a |\usepackage{afterpage}| in the document's
% preamble), or for short\\
% |\newcommand{\onepagecolor}[1]{%|\\
% |\newpagecolor{#1}\afterpage{\restorepagecolor}}|\\
% in the preamble and\\
% |\onepagecolor{<|\textit{some colour}|>}| in the document.
%
% \subsection{Option}
% \DescribeMacro{option}
% \indent The \xpackage{pagecolor} package takes the following option:
%
% \subsubsection{pagecolor\label{sss:pagecolor}}
% \DescribeMacro{pagecolor}
% \indent The option |pagecolor={...}| takes as value a colour. This
% could be as simple as |white| or |black|, but when e.\,g.~the
% \xpackage{xcolor} package is used (loaded before \xpackage{pagecolor}!),
% also colours like |red!50!green!20!blue| are possible.
% The default is |pagecolor={none}|.
%
% \section{Alternatives\label{sec:Alternatives}}
%
% As I neither know what exactly you want to accomplish
% when using this package (e.\,g.~hiding text), nor what resources you have
% (e.\,g.~\pdfTeX{} version), here is a list of some possible alternatives:\\
%
% \begin{description}
%  \item[-]
%  \DescribeMacro{transparent}
%  \xpackage{transparent} package: With it some object can be made
%  (fully or partially) transparent.
% \end{description}
%
% \begin{description}
%  \item[-]
%  \DescribeMacro{hrefhide}\xpackage{hrefhide} package: It allows to
%  \textquotedblleft hide\textquotedblright\ some (hyperlinked) text when
%  printing the document while keeping the layout.
% \end{description}
%
% \bigskip
%
% \noindent (You programmed or found another alternative,
%  which is available at \CTAN{}?\\
%  OK, send an e-mail to me with the name, location at \CTAN{},
%  and a short notice, and I will probably include it in
%  the list above.)\\
%
% \smallskip
%
% \noindent About how to get those packages, please see
% subsection~\ref{ss:Downloads}.
%
% \newpage
%
% \section{Example}
%
%    \begin{macrocode}
%<*example>
\documentclass[british]{article}[2007/10/19]% v1.4h
%%%%%%%%%%%%%%%%%%%%%%%%%%%%%%%%%%%%%%%%%%%%%%%%%%%%%%%%%%%%%%%%%%%%%
\usepackage[%
 extension=pdf,%
 plainpages=false,%
 pdfpagelabels=true,%
 hyperindex=false,%
 pdflang={en},%
 pdftitle={pagecolor package example},%
 pdfauthor={H.-Martin Muench},%
 pdfsubject={Example for the pagecolor package},%
 pdfkeywords={LaTeX, pagecolor, thepagecolor, page colour,%
  H.-Martin Muench},%
 pdfview=Fit,pdfstartview=Fit,%
 pdfpagelayout=SinglePage%
]{hyperref}[2012/02/06]% v6.82o
\usepackage[x11names]{xcolor}[2007/01/21]% v2.11
 % The xcolor package would not be needed for just using
 % the base colours. The color package would be sufficient for that.
\usepackage[pagecolor={LightGoldenrod1}]{pagecolor}[2012/02/23]% v1.0e

\usepackage{afterpage}[1995/10/27]% v1.08
 % The afterpage package is generally not needed,
 % but the |\newpagecolor{somecolour}\afterpage{\restorepagecolor}|
 % construct shall be demonstrated.

\usepackage{lipsum}[2011/04/14]% v1.2
 % The lipsum package is generally not needed,
 % but some blind text is needed for the example.

\gdef\unit#1{\mathord{\thinspace\mathrm{#1}}}%
\listfiles
\begin{document}
\pagenumbering{arabic}
\section*{Example for pagecolor}

This example demonstrates the use of package\newline
\textsf{pagecolor}, v1.0e as of 2012/02/23 (HMM).\newline
The used option was \verb|pagecolor={LightGoldenrod1}|.\newline
\verb|pagecolor={none}| would be the default.\newline

For more details please see the documentation!\newline

\noindent {\color{teal} Save per page about $200\unit{ml}$ water,
$2\unit{g}$ CO$_{2}$ and $2\unit{g}$ wood:\newline
Therefore please print only if this is really necessary.}\newline

The current page (background) colour is\newline
\verb|\thepagecolor|\ =\ \thepagecolor \newline
(and \verb|\thepagecolornone|\ =\ \thepagecolornone ,
which would only be different from \verb|\thepagecolor|,
when the page colour would be \verb|none|).

\pagebreak
\pagecolor{rgb:-green!40!yellow,3;green!40!yellow,2;red,1}

{\color{white} The current page (background) colour is\newline
\verb|\thepagecolor|\ =\ \thepagecolor . \newline}

{\color{\thepagecolor} And that makes this text practically invisible.
\newline}

{\color{white} Which made the preceding line of text practically
invisible.}

\pagebreak
\newpagecolor{red}

This page uses \verb|\newpagecolor{red}|.

\pagebreak
\restorepagecolor

{\color{white}And this page uses \verb|\restorepagecolor| to restore
the page colour to the value it had before the red page.}

\pagebreak
\pagecolor{none}

This page uses \verb|\pagecolor{none}|. If the \verb|\nopagecolor|
command is known (pdf\TeX and Lua\TeX; not yet for dvips, dvipdfm(x)
or Xe\TeX), the page colour is now \verb|none|, otherwise \verb|white|:
\verb|\thepagecolor|\ =\ \thepagecolor\ and
\verb|\thepagecolornone|\ =\ \thepagecolornone .

\pagebreak
\restorepagecolor

{\color{white}\verb|\restorepagecolor| restored the page colour again.}

\pagebreak
\pagecolor{green}

This page is green due to \verb|\pagecolor{green}|.

\pagebreak
\newpagecolor{blue}\afterpage{\restorepagecolor}

{\color{white}\verb|\newpagecolor{blue}\afterpage{\restorepagecolor}|%
\newline
was used here, i.\,e.~this page is blue, and the next one will
automatically have the same page colour before it was changed to blue
here (i.\,e. green).}

\smallskip
{\color{red}\textbf{\lipsum[1-11]}}
\bigskip

The page colour was changed back at the end of the page -
in mid-sentence!

\end{document}
%</example>
%    \end{macrocode}
%
% \StopEventually{}
%
% \newpage
%
% \section{The implementation}
%
% We start off by checking that we are loading into \LaTeXe{} and
% announcing the name and version of this package.
%
%    \begin{macrocode}
%<*package>
%    \end{macrocode}
%
%    \begin{macrocode}
\NeedsTeXFormat{LaTeX2e}[2011/06/27]
\ProvidesPackage{pagecolor}[2012/02/23 v1.0e
            Provides thepagecolor (HMM)]
%    \end{macrocode}
%
% A short description of the \xpackage{pagecolor} package:
%
%    \begin{macrocode}
%% Provides the \thepagecolor, \thepagecolornone, \newpagecolor{...},
%% and \restorepagecolor commands.
%    \end{macrocode}
%
% We need the \xpackage{kvoptions} package by \textsc{Heiko Oberdiek}:
%
%    \begin{macrocode}
\RequirePackage{kvoptions}[2010/12/23]% v3.10
%    \end{macrocode}
%
% and either the \xpackage{color} or the \xpackage{xcolor} package:
%
%    \begin{macrocode}
%% \RequirePackage{ either color or xcolor }:
\@ifpackageloaded{xcolor}{% xcolor loaded
   \@ifpackagelater{xcolor}{2007/01/21}{%
      % 2007/01/21, v2.11, or even more recent: OK
    }{% else: older package version
      \PackageWarning{pagecolor}{%
        It is requested version `2007/01/21' of package\MessageBreak%
        xcolor, but only an older version is available.\MessageBreak%
       }%
     }%
 }{% xcolor not loaded
   \@ifpackageloaded{color}{%
      \RequirePackage{color}[2005/11/14]% v1.0j
    }{% \else
      \PackageWarning{pagecolor}{%
        The pagecolor package must be loaded after either\MessageBreak%
        package color or after package xcolor (at your\MessageBreak%
        option). Neither package was loaded before package\MessageBreak%
        pagecolor. Loading of package xcolor will now be\MessageBreak%
        tried automatically.\MessageBreak%
        When the pagecolor package is used with option\MessageBreak%
        pagecolor using a colour requiring e. g. x11names\MessageBreak%
        option for xcolor package, this will not work!\MessageBreak%
       }
     }% \fi
   \RequirePackage{xcolor}[2007/01/21]% v2.11
  }% \fi
%    \end{macrocode}
%
% A last information for the user:
%
%    \begin{macrocode}
%% pagecolor may work with earlier versions of LaTeX and the
%% packages, but this was not tested. Please consider updating
%% your LaTeX and packages to the most recent version
%% (if they are not already the most recent version).

%    \end{macrocode}
%
% \indent See subsection~\ref{ss:Downloads} about how to get them.\\
%
% We process the options:
%
%    \begin{macrocode}
\SetupKeyvalOptions{family=pagecolor,prefix=pagecolor@}
\DeclareStringOption[none]{pagecolor}% \pagecolor@pagecolor
\ProcessKeyvalOptions*

%    \end{macrocode}
%
% We save the original |\pagecolor| command,
%
%    \begin{macrocode}
\let\origpagecolour\pagecolor

%    \end{macrocode}
%
%    \begin{macro}{\pagecolor}
% before we redefine it to include a definition of |\thepagecolor|
% and \linebreak
% |\thepagecolornone|:
%
%    \begin{macrocode}
\renewcommand{\pagecolor}[1]{\@bsphack%
  \edef\pagecolourtmpa{#1}%
  \def\pagecolourtmpb{none}%
  \ifx\pagecolourtmpa\pagecolourtmpb
    \@ifundefined{nopagecolor}{%
      \PackageError{pagecolor}{%
        pagecolor=none requested but \string\nopagecolor\space unknown%
       }{%
        \string\pagecolor{none} was used, but the command %
        \string\nopagecolor\space is undefined.\MessageBreak%
        Please use another colour. pagecolor=white will be used now.%
        \MessageBreak%
        }%
      \gdef\thepagecolor{white}%
      \gdef\thepagecolornone{white}% although it should be "none"
      \origpagecolour{white}%
     }{%
      \nopagecolor%
     }%
  \else%
    \xdef\thepagecolor{#1}%
    \xdef\thepagecolornone{#1}%
    \origpagecolour{\thepagecolornone}%
  \fi%
  \@esphack%
}

%    \end{macrocode}
%    \end{macro}
%
% |\nopagecolor| is only defined for \pdfTeX and \LuaTeX, but not yet for dvips,
% dvipdfm(x) or \XeTeX. Maybe
% \begin{verbatim}
% \@ifundefined{nopagecolor}{\newcommand{\nopagecolor}{\pagecolor{white}}}{}
% \end{verbatim}
% could be an alternative.\\
% When |\pagecolor{none}| is used and |\nopagecolor| is defined, then \\
% |\pagecolor{none}| is made into a synonym for |\nopagecolor|.\\
% If |\nopagecolor| has not been defined, nothing needs to be done.
% Otherwise we redefine |\nopagecolor|, thus that |\thepagecolor| is
% set to |white| and |\thepagecolornone| to |none| whenever
% |\nopagecolor| is used.
%
%    \begin{macrocode}
\@ifundefined{nopagecolor}{%
  \PackageWarning{pagecolor}{%
    \string\nopagecolor\space is undefined.\MessageBreak%
    (As long as it is neither used anywhere\MessageBreak%
    \space nor defined later, that will be no problem.)\MessageBreak%
   }
  \AtEndDocument{%
  \@ifundefined{nopagecolor}{% no problem
    }{%
    \PackageError{pagecolor}{%
      \string\nopagecolor\space defined after loading package%
      \MessageBreak%
      pagecolor%
      }{\string\nopagecolor\space was defined after loading the %
       pagecolor package.\MessageBreak%
       Either define \string\nopagecolor\space earlier or load the %
       pagecolor package later.\MessageBreak%
      }
    }%
  }%
  }{% \else
    \let\orignopagecolour\nopagecolor
    \renewcommand{\nopagecolor}{%
      \xdef\thepagecolor{white}
      \xdef\thepagecolornone{none}
      \orignopagecolour
      }
  }

%    \end{macrocode}
%
% If the page colour as given with option |pagecolor={...}| is |none|,
% but |\nopagecolor| is not known, |\pagecolor@pagecolor| is set to
% |white| and a warning is given.
%
%    \begin{macrocode}
\def\pagecolourtmpb{none}
\ifx\pagecolor@pagecolor\pagecolourtmpb
  \@ifundefined{nopagecolor}{%
    \PackageWarning{pagecolor}{%
      Option pagecolor=none (maybe by default) used,\MessageBreak%
      but \string\nopagecolor\space is unknown. Please use another%
      \MessageBreak%
      option value; white will be used now.\MessageBreak%
     }
    \setkeys{pagecolor}{pagecolor=white}%
  }{% ok
  }%
\fi

%    \end{macrocode}
%
% The (new) |\pagecolor| is now just carried out.
%
%    \begin{macrocode}
\pagecolor{\pagecolor@pagecolor}

%    \end{macrocode}
%
% Now the page (background) colour and |\thepagecolor| and \\
% |\thepagecolornone| are |\pagecolor@pagecolor| (or page (background) colour
% and |\thepagecolornone| are |none|=|\pagecolor@pagecolor| and
% |\thepagecolor| is |white|), and when the page (background) colour
% is changed, |\thepagecolor| and |\thepagecolornone| are changed
% accordingly.
%
% \newpage
%
%    \begin{macro}{\newpagecolor}
% There have been requests (via e-mail and at \\
% \url{http://tex.stackexchange.com/q/25137}) to change the colour of just one
% (or two) page(s) only, similar to |\newgeometry| and |\restoregeometry| of the
% \xpackage{geometry} package (\url{http://ctan.org/pkg/geometry}). Therefore
% |\newpagecolor| and |\restorepagecolor| are introduced (as suggested by
% \textsc{haoyun\_tex}):
%
%    \begin{macrocode}
\newcommand{\newpagecolor}[1]{%
\xdef\pagecolourtmpc{\thepagecolornone}%
\pagecolor{#1}%
}

%    \end{macrocode}
%    \end{macro}
%
% |\newpagecolor{<|\textit{some colour}|>}| will execute |\pagecolor{somecolour}|
% and remember the page colour used before.\\
%
%    \begin{macro}{\restorepagecolor}
%    \begin{macrocode}
\newcommand{\restorepagecolor}{\pagecolor{\pagecolourtmpc}}

%    \end{macrocode}
%    \end{macro}
%
% |\restorecolor| (without argument) restores the page colour to the one used before
% use of the |\newpagecolor{...}| command.
%
%    \begin{macrocode}
\gdef\pagecolourtmpc{\thepagecolor}

%    \end{macrocode}
% is just a precaution for |\restorecolor| being used when no |\newpagecolor{...}|
% was used before it.\\
%
% When you want to change the colour for just one page and do not want to
% (or cannot) manually determine where the page ends,\\
% |\newpagecolor{<|\textit{some colour}|>}\afterpage{\restorepagecolor}|\\
% does the trick (and requires an additional |\usepackage{afterpage}| in the
% document's preamble).
%    \begin{macrocode}
%</package>
%    \end{macrocode}
%
% \newpage
%
% \section{Installation}
%
% \subsection{Downloads\label{ss:Downloads}}
%
% Everything is available on \CTAN{}, \url{http://www.ctan.org/tex-archive/},
% but may need additional packages themselves.\\
%
% \DescribeMacro{pagecolor.dtx}
% For unpacking the |pagecolor.dtx| file and constructing the documentation
% it is required:
% \begin{description}
% \item[-] \TeX{} Format \LaTeXe{}, \url{http://www.CTAN.org/}
%
% \item[-] document class \xpackage{ltxdoc}, 2007/11/11, v2.0u,
%           \url{http://ctan.org/pkg/ltxdoc}
%
% \item[-] package \xpackage{holtxdoc}, 2011/02/04, v0.21,
%           \url{http://ctan.org/pkg/holtxdoc}
% \end{description}
%
% \DescribeMacro{pagecolor.sty}
% The |pagecolor.sty| for \LaTeXe{} (i.\,e. each document using
% the \xpackage{pagecolor} package) requires:
% \begin{description}
% \item[-] \TeX{} Format \LaTeXe{}, \url{http://www.CTAN.org/}
%
% \item[-] package \xpackage{kvoptions}, 2010/12/23, v3.10,
%           \url{http://ctan.org/pkg/kvoptions}
% \end{description}
% and either
% \begin{description}
% \item[-] package \xpackage{color}, 2005/11/14, v1.0j,
%           \url{http://ctan.org/pkg/color} (from the\\
%           \xpackage{graphics} package bundle)
% \end{description}
% or
% \begin{description}
% \item[-] package \xpackage{xcolor}, 2007/01/21, v2.11,
%           \url{http://ctan.org/pkg/xcolor}
% \end{description}
%
% \DescribeMacro{pagecolor-example.tex}
% The |pagecolor-example.tex| requires the same files as all
% documents using the \xpackage{pagecolor} package and additionally:
% \begin{description}
% \item[-] class \xpackage{article}, 2007/10/19, v1.4h, from \xpackage{classes.dtx}:\\
%           \CTAN{macros/latex/base/classes.dtx}
%
% \item[-] package \xpackage{pagecolor}, 2012/02/23, v1.0e,
%           \url{http://ctan.org/pkg/pagecolor}\\
%           (Well, it is the example file for this package, and because you are
%            reading the documentation for the \xpackage{pagecolor} package,
%            it can be assumed that you already have some version of it --
%            is it the current one?)
%
% \item[-] package \xpackage{xcolor}, 2007/01/21, v2.11,
%           \url{http://ctan.org/pkg/xcolor}\\
%           This package would not be needed for the use of just base colours only,
%           the \xpackage{color} package would be sufficient for that.
% \end{description}
%
% \DescribeMacro{transparent}
% \DescribeMacro{hrefhide}
% As possible alternatives in section \ref{sec:Alternatives} there are listed
% \begin{description}
% \item[-] package \xpackage{transparent}, 2007/01/08, v1.0, \\
%           \url{http://ctan.org/pkg/transparent}
%
% \item[-] package \xpackage{hrefhide}, 2011/04/29, v1.0f, \\
%           \url{http://ctan.org/pkg/hrefhide}
% \end{description}
%
% \DescribeMacro{Oberdiek}
% \DescribeMacro{holtxdoc}
% \DescribeMacro{kvoptions}
% All packages of \textsc{Heiko Oberdiek}'s bundle `oberdiek'
% (especially \xpackage{holtxdoc} and \xpackage{kvoptions})
% are also available in a TDS compliant ZIP archive:\\
% \CTAN{install/macros/latex/contrib/oberdiek.tds.zip}.\\
% It is probably best to download and use this, because the packages in there
% are quite probably both recent and compatible among themselves.\\
%
% \DescribeMacro{hyperref}
% \noindent \xpackage{hyperref} is not included in that bundle and needs to be
% downloaded separately,\\
% \url{http://mirror.ctan.org/install/macros/latex/contrib/hyperref.tds.zip}.\\
%
% \DescribeMacro{M\"{u}nch}
% A hyperlinked list of my (other) packages can be found at
% \url{http://www.Uni-Bonn.de/~uzs5pv/LaTeX.html}.\\
%
% \subsection{Package, unpacking TDS}
% \paragraph{Package.} This package is available on \CTAN{}
% \begin{description}
% \item[\CTAN{macros/latex/contrib/pagecolor/pagecolor.dtx}]\hspace*{0.1cm} \\
%       The source file.
% \item[\CTAN{macros/latex/contrib/pagecolor/pagecolor.pdf}]\hspace*{0.1cm} \\
%       The documentation.
% \item[\CTAN{macros/latex/contrib/pagecolor/pagecolor-example.pdf}]\hspace*{0.1cm} \\
%       The compiled example file, as it should look like.
% \item[\CTAN{macros/latex/contrib/pagecolor/README}]\hspace*{0.1cm} \\
%       The README file.
% \end{description}
% There is also a pagecolor.tds.zip available:
% \begin{description}
% \item[\CTAN{install/macros/latex/contrib/pagecolor.tds.zip}]\hspace*{0.1cm} \\
%       Everything in \xfile{TDS} compliant, compiled format.
% \end{description}
% which additionally contains\\
% \begin{tabular}{ll}
% pagecolor.ins & The installation file.\\
% pagecolor.drv & The driver to generate the documentation.\\
% pagecolor.sty & The \xext{sty}le file.\\
% pagecolor-example.tex & The example file.
% \end{tabular}
%
% \bigskip
%
% \noindent For required other packages, please see the preceding subsection.
%
% \paragraph{Unpacking.} The \xfile{.dtx} file is a self-extracting
% \docstrip{} archive. The files are extracted by running the
% \xext{.dtx} through \plainTeX{}:
% \begin{quote}
%   \verb|tex pagecolor.dtx|
% \end{quote}
%
% About generating the documentation see paragraph~\ref{GenDoc} below.\\
%
% \paragraph{TDS.} Now the different files must be moved into
% the different directories in your installation TDS tree
% (also known as \xfile{texmf} tree):
% \begin{quote}
% \def\t{^^A
% \begin{tabular}{@{}>{\ttfamily}l@{ $\rightarrow$ }>{\ttfamily}l@{}}
%   pagecolor.sty & tex/latex/pagecolor/pagecolor.sty\\
%   pagecolor.pdf & doc/latex/pagecolor/pagecolor.pdf\\
%   pagecolor-example.tex & doc/latex/pagecolor/pagecolor-example.tex\\
%   pagecolor-example.pdf & doc/latex/pagecolor/pagecolor-example.pdf\\
%   pagecolor.dtx & source/latex/pagecolor/pagecolor.dtx\\
% \end{tabular}^^A
% }^^A
% \sbox0{\t}^^A
% \ifdim\wd0>\linewidth
%   \begingroup
%     \advance\linewidth by\leftmargin
%     \advance\linewidth by\rightmargin
%   \edef\x{\endgroup
%     \def\noexpand\lw{\the\linewidth}^^A
%   }\x
%   \def\lwbox{^^A
%     \leavevmode
%     \hbox to \linewidth{^^A
%       \kern-\leftmargin\relax
%       \hss
%       \usebox0
%       \hss
%       \kern-\rightmargin\relax
%     }^^A
%   }^^A
%   \ifdim\wd0>\lw
%     \sbox0{\small\t}^^A
%     \ifdim\wd0>\linewidth
%       \ifdim\wd0>\lw
%         \sbox0{\footnotesize\t}^^A
%         \ifdim\wd0>\linewidth
%           \ifdim\wd0>\lw
%             \sbox0{\scriptsize\t}^^A
%             \ifdim\wd0>\linewidth
%               \ifdim\wd0>\lw
%                 \sbox0{\tiny\t}^^A
%                 \ifdim\wd0>\linewidth
%                   \lwbox
%                 \else
%                   \usebox0
%                 \fi
%               \else
%                 \lwbox
%               \fi
%             \else
%               \usebox0
%             \fi
%           \else
%             \lwbox
%           \fi
%         \else
%           \usebox0
%         \fi
%       \else
%         \lwbox
%       \fi
%     \else
%       \usebox0
%     \fi
%   \else
%     \lwbox
%   \fi
% \else
%   \usebox0
% \fi
% \end{quote}
% If you have a \xfile{docstrip.cfg} that configures and enables \docstrip{}'s
% \xfile{TDS} installing feature, then some files can already be in the right
% place, see the documentation of \docstrip{}.
%
% \subsection{Refresh file name databases}
%
% If your \TeX{}~distribution (\teTeX{}, \mikTeX{},\dots{}) relies on file name
% databases, you must refresh these. For example, \teTeX{} users run
% \verb|texhash| or \verb|mktexlsr|.
%
% \subsection{Some details for the interested}
%
% \paragraph{Unpacking with \LaTeX{}.}
% The \xfile{.dtx} chooses its action depending on the format:
% \begin{description}
% \item[\plainTeX:] Run \docstrip{} and extract the files.
% \item[\LaTeX:] Generate the documentation.
% \end{description}
% If you insist on using \LaTeX{} for \docstrip{} (really,
% \docstrip{} does not need \LaTeX{}), then inform the autodetect routine
% about your intention:
% \begin{quote}
%   \verb|latex \let\install=y% \iffalse meta-comment
%
% File: pagecolor.dtx
% Version: 2012/02/23 v1.0e
%
% Copyright (C) 2011 - 2012 by
%    H.-Martin M"unch <Martin dot Muench at Uni-Bonn dot de>
%
% This work may be distributed and/or modified under the
% conditions of the LaTeX Project Public License, either
% version 1.3c of this license or (at your option) any later
% version. This version of this license is in
%    http://www.latex-project.org/lppl/lppl-1-3c.txt
% and the latest version of this license is in
%    http://www.latex-project.org/lppl.txt
% and version 1.3c or later is part of all distributions of
% LaTeX version 2005/12/01 or later.
%
% This work has the LPPL maintenance status "maintained".
%
% The Current Maintainer of this work is H.-Martin Muench.
%
% This work consists of the main source file pagecolor.dtx,
% the README, and the derived files
%    pagecolor.sty, pagecolor.pdf,
%    pagecolor.ins, pagecolor.drv,
%    pagecolor-example.tex, pagecolor-example.pdf.
%
% Distribution:
%    CTAN:macros/latex/contrib/pagecolor/pagecolor.dtx
%    CTAN:macros/latex/contrib/pagecolor/pagecolor.pdf
%    CTAN:install/macros/latex/contrib/pagecolor.tds.zip
%
% Cf. http://ctan.org/pkg/pagecolor and
% http://www.tex.ac.uk/tex-archive/help/Catalogue/entries/pagecolor.html
%
% Unpacking:
%    (a) If pagecolor.ins is present:
%           tex pagecolor.ins
%    (b) Without pagecolor.ins:
%           tex pagecolor.dtx
%    (c) If you insist on using LaTeX
%           latex \let\install=y\input{pagecolor.dtx}
%        (quote the arguments according to the demands of your shell)
%
% Documentation:
%    (a) If pagecolor.drv is present:
%           (pdf)latex pagecolor.drv
%           makeindex -s gind.ist pagecolor.idx
%           (pdf)latex pagecolor.drv
%           makeindex -s gind.ist pagecolor.idx
%           (pdf)latex pagecolor.drv
%    (b) Without pagecolor.drv:
%           (pdf)latex pagecolor.dtx
%           makeindex -s gind.ist pagecolor.idx
%           (pdf)latex pagecolor.dtx
%           makeindex -s gind.ist pagecolor.idx
%           (pdf)latex pagecolor.dtx
%
%    The class ltxdoc loads the configuration file ltxdoc.cfg
%    if available. Here you can specify further options, e.g.
%    use DIN A4 as paper format:
%       \PassOptionsToClass{a4paper}{article}
%
% Installation:
%    TDS:tex/latex/pagecolor/pagecolor.sty
%    TDS:doc/latex/pagecolor/pagecolor.pdf
%    TDS:doc/latex/pagecolor/pagecolor-example.tex
%    TDS:doc/latex/pagecolor/pagecolor-example.pdf
%    TDS:source/latex/pagecolor/pagecolor.dtx
%
%<*ignore>
\begingroup
  \catcode123=1 %
  \catcode125=2 %
  \def\x{LaTeX2e}%
\expandafter\endgroup
\ifcase 0\ifx\install y1\fi\expandafter
         \ifx\csname processbatchFile\endcsname\relax\else1\fi
         \ifx\fmtname\x\else 1\fi\relax
\else\csname fi\endcsname
%</ignore>
%<*install>
\input docstrip.tex
\Msg{*****************************************************************}
\Msg{* Installation                                                   }
\Msg{* Package: pagecolor 2012/02/23 v1.0e Provides thepagecolor (HMM)}
\Msg{*****************************************************************}

\keepsilent
\askforoverwritefalse

\let\MetaPrefix\relax
\preamble

This is a generated file.

Project: pagecolor
Version: 2012/02/23 v1.0e

Copyright (C) 2011 - 2012 by
    H.-Martin M"unch <Martin dot Muench at Uni-Bonn dot de>

The usual disclaimer applies:
If it doesn't work right that's your problem.
(Nevertheless, send an e-mail to the maintainer
 when you find an error in this package.)

This work may be distributed and/or modified under the
conditions of the LaTeX Project Public License, either
version 1.3c of this license or (at your option) any later
version. This version of this license is in
   http://www.latex-project.org/lppl/lppl-1-3c.txt
and the latest version of this license is in
   http://www.latex-project.org/lppl.txt
and version 1.3c or later is part of all distributions of
LaTeX version 2005/12/01 or later.

This work has the LPPL maintenance status "maintained".

The Current Maintainer of this work is H.-Martin Muench.

This work consists of the main source file pagecolor.dtx,
the README, and the derived files
   pagecolor.sty, pagecolor.pdf,
   pagecolor.ins, pagecolor.drv,
   pagecolor-example.tex, pagecolor-example.pdf.

\endpreamble
\let\MetaPrefix\DoubleperCent

\generate{%
  \file{pagecolor.ins}{\from{pagecolor.dtx}{install}}%
  \file{pagecolor.drv}{\from{pagecolor.dtx}{driver}}%
  \usedir{tex/latex/pagecolor}%
  \file{pagecolor.sty}{\from{pagecolor.dtx}{package}}%
  \usedir{doc/latex/pagecolor}%
  \file{pagecolor-example.tex}{\from{pagecolor.dtx}{example}}%
}

\catcode32=13\relax% active space
\let =\space%
\Msg{************************************************************************}
\Msg{*}
\Msg{* To finish the installation you have to move the following}
\Msg{* file into a directory searched by TeX:}
\Msg{*}
\Msg{*     pagecolor.sty}
\Msg{*}
\Msg{* To produce the documentation run the file `pagecolor.drv'}
\Msg{* through (pdf)LaTeX, e.g.}
\Msg{*  pdflatex pagecolor.drv}
\Msg{*  makeindex -s gind.ist pagecolor.idx}
\Msg{*  pdflatex pagecolor.drv}
\Msg{*  makeindex -s gind.ist pagecolor.idx}
\Msg{*  pdflatex pagecolor.drv}
\Msg{*}
\Msg{* At least three runs are necessary e.g. to get the}
\Msg{*  references right!}
\Msg{*}
\Msg{* Happy TeXing!}
\Msg{*}
\Msg{************************************************************************}

\endbatchfile
%</install>
%<*ignore>
\fi
%</ignore>
%
% \section{The documentation driver file}
%
% The next bit of code contains the documentation driver file for
% \TeX{}, i.\,e., the file that will produce the documentation you
% are currently reading. It will be extracted from this file by the
% \texttt{docstrip} programme. That is, run \LaTeX{} on \texttt{docstrip}
% and specify the \texttt{driver} option when \texttt{docstrip}
% asks for options.
%
%    \begin{macrocode}
%<*driver>
\NeedsTeXFormat{LaTeX2e}[2011/06/27]
\ProvidesFile{pagecolor.drv}%
  [2012/02/23 v1.0e Provides thepagecolor (HMM)]
\documentclass{ltxdoc}[2007/11/11]% v2.0u
\usepackage{holtxdoc}[2011/02/04]%  v0.21
%% pagecolor may work with earlier versions of LaTeX2e and those
%% class and package, but this was not tested.
%% Please consider updating your LaTeX, class, and package
%% to the most recent version (if they are not already the most
%% recent version).
\hypersetup{%
 pdfsubject={Providing the thepagecolor command (HMM)},%
 pdfkeywords={LaTeX, pagecolor, thepagecolor, H.-Martin Muench},%
 pdfencoding=auto,%
 pdflang={en},%
 breaklinks=true,%
 linktoc=all,%
 pdfstartview=FitH,%
 pdfpagelayout=OneColumn,%
 bookmarksnumbered=true,%
 bookmarksopen=true,%
 bookmarksopenlevel=3,%
 pdfmenubar=true,%
 pdftoolbar=true,%
 pdfwindowui=true,%
 pdfnewwindow=true%
}
\CodelineIndex
\hyphenation{printing docu-ment}
\gdef\unit#1{\mathord{\thinspace\mathrm{#1}}}%
\begin{document}
  \DocInput{pagecolor.dtx}%
\end{document}
%</driver>
%    \end{macrocode}
%
% \fi
%
% \CheckSum{137}
%
% \CharacterTable
%  {Upper-case    \A\B\C\D\E\F\G\H\I\J\K\L\M\N\O\P\Q\R\S\T\U\V\W\X\Y\Z
%   Lower-case    \a\b\c\d\e\f\g\h\i\j\k\l\m\n\o\p\q\r\s\t\u\v\w\x\y\z
%   Digits        \0\1\2\3\4\5\6\7\8\9
%   Exclamation   \!     Double quote  \"     Hash (number) \#
%   Dollar        \$     Percent       \%     Ampersand     \&
%   Acute accent  \'     Left paren    \(     Right paren   \)
%   Asterisk      \*     Plus          \+     Comma         \,
%   Minus         \-     Point         \.     Solidus       \/
%   Colon         \:     Semicolon     \;     Less than     \<
%   Equals        \=     Greater than  \>     Question mark \?
%   Commercial at \@     Left bracket  \[     Backslash     \\
%   Right bracket \]     Circumflex    \^     Underscore    \_
%   Grave accent  \`     Left brace    \{     Vertical bar  \|
%   Right brace   \}     Tilde         \~}
%
% \GetFileInfo{pagecolor.drv}
%
% \begingroup
%   \def\x{\#,\$,\^,\_,\~,\ ,\&,\{,\},\%}%
%   \makeatletter
%   \@onelevel@sanitize\x
% \expandafter\endgroup
% \expandafter\DoNotIndex\expandafter{\x}
% \expandafter\DoNotIndex\expandafter{\string\ }
% \begingroup
%   \makeatletter
%     \lccode`9=32\relax
%     \lowercase{%^^A
%       \edef\x{\noexpand\DoNotIndex{\@backslashchar9}}%^^A
%     }%^^A
%   \expandafter\endgroup\x
%
% \DoNotIndex{\,,\\}
% \DoNotIndex{\documentclass,\usepackage,\ProvidesPackage,\begin,\end}
% \DoNotIndex{\MessageBreak}
% \DoNotIndex{\NeedsTeXFormat,\DoNotIndex,\verb}
% \DoNotIndex{\def,\edef,\gdef,\global}
% \DoNotIndex{\ifx,\listfiles,\mathord,\mathrm}
% \DoNotIndex{\kvoptions,\SetupKeyvalOptions,\ProcessKeyvalOptions}
% \DoNotIndex{\bigskip,\space,\thinspace,\ldots}
% \DoNotIndex{\indent,\noindent,\newline,\linebreak,\pagebreak,\newpage}
% \DoNotIndex{\textbf,\textit,\textsf,\texttt,\textsc,\textquotedblleft,\textquotedblright}
% \DoNotIndex{\plainTeX,\TeX,\LaTeX,\pdfLaTeX}
% \DoNotIndex{\section,\color,\pagenumbering,\smallskip}
% \DoNotIndex{\@ifundefined,\@ifpackageloaded}
% \DoNotIndex{\DeclareStringOption}
%
% \title{The \xpackage{pagecolor} package}
% \date{2012/02/23 v1.0e}
% \author{H.-Martin M\"{u}nch\\\xemail{Martin.Muench at Uni-Bonn.de}}
%
% \maketitle
%
% \begin{abstract}
%  This \LaTeX{} package provides the command |\thepagecolor|,
% which gives the current page (background) colour,
% i.\,e.~the argument used with the most recent call of |\pagecolor{...}|.
% The command |\thepagecolornone| gives the same colour as |\thepagecolor|,
% except when the page background colour is
% \textquotedblleft|none|\textquotedblright. In that case |\thepagecolor| is
% |white| and |\thepagecolornone| is |none|.\\
% Similar to |\newgeometry| and |\restoregeometry| of the \xpackage{geometry}
% package\\
% |\newpagecolor{<|\textit{some colour} |>}| and |\restorepagecolor|
% are provided.
% \end{abstract}
%
% \bigskip
%
% \noindent Disclaimer for web links: The author is not responsible for any contents
% referred to in this work unless he has full knowledge of illegal contents.
% If any damage occurs by the use of information presented there, only the
% author of the respective pages might be liable, not the one who has referred
% to these pages.
%
% \bigskip
%
% \noindent {\color{green} Save per page about $200\unit{ml}$ water,
% $2\unit{g}$ CO$_{2}$ and $2\unit{g}$ wood:\\
% Therefore please print only if this is really necessary.}
%
% \pagebreak
%
% \tableofcontents
%
% \bigskip
%
% \section{Introduction\label{sec:Introduction}}
%
% \indent This \LaTeX{} package provides the command |\thepagecolor|,
% which gives the current page (background) colour,
% i.\,e.~the argument used with the most recent call of |\pagecolor{...}|.
% The package should be loaded before any package sets a page (background)
% colour, but after \xpackage{color} or \xpackage{xcolor} package.
% Its option |pagecolor={...}| is used to set the |\pagecolor{...}|.\\
% The command |\thepagecolornone| gives the same colour as |\thepagecolor|,
% except when the page background colour is
% \textquotedblleft|none|\textquotedblright (e.\,g. result of using the
% |\nopagecolor| command). In that case |\thepagecolor| is |white| and\linebreak
% |\thepagecolornone| is |none|.\\
% Similar to |\newgeometry| and |\restoregeometry| of the \xpackage{geometry}
% package\\
% |\newpagecolor{<|\textit{some colour}|>}| and |\restorepagecolor|
% are provided.
%
% \pagebreak
%
% \section{Usage}
%
% \indent Just load the package placing
% \begin{quote}
%   |\usepackage[<|\textit{option}|>]{pagecolor}|
% \end{quote}
% \noindent in the preamble of your \LaTeXe{} source file.
% This should be done before another package uses |\pagecolor|,
% but after |\nopagecolor| is defined (if that is defined in the document
% at all).\\
% Afterwards |\pagecolor{...}| can be used to change the page (background)
% colour as usual. Then |\thepagecolor| gives the current page (background)
% colour (in the same format as given with |\pagecolor{...}|).\\
% Similar to |\newgeometry| and |\restoregeometry| of the \xpackage{geometry}
% package\\
% |\newpagecolor{<|\textit{some colour} |>}| and |\restorepagecolor|
% are provided:\\
% |\newpagecolor{<|\textit{some colour} |>}| will execute
% |\pagecolor{<|\textit{some colour} |>}| and remember the page colour used before.
% |\restorecolor| (without argument) restores the page colour to the one used before
%  use of the |\newpagecolor{...}| command. When you want to change the colour for
%  just one page and do not want to (or cannot) manually determine where the page ends,\\
% |\newpagecolor{<|\textit{some colour}|>}\afterpage{\restorepagecolor}|\\
% does the trick (and requires a |\usepackage{afterpage}| in the document's
% preamble), or for short\\
% |\newcommand{\onepagecolor}[1]{%|\\
% |\newpagecolor{#1}\afterpage{\restorepagecolor}}|\\
% in the preamble and\\
% |\onepagecolor{<|\textit{some colour}|>}| in the document.
%
% \subsection{Option}
% \DescribeMacro{option}
% \indent The \xpackage{pagecolor} package takes the following option:
%
% \subsubsection{pagecolor\label{sss:pagecolor}}
% \DescribeMacro{pagecolor}
% \indent The option |pagecolor={...}| takes as value a colour. This
% could be as simple as |white| or |black|, but when e.\,g.~the
% \xpackage{xcolor} package is used (loaded before \xpackage{pagecolor}!),
% also colours like |red!50!green!20!blue| are possible.
% The default is |pagecolor={none}|.
%
% \section{Alternatives\label{sec:Alternatives}}
%
% As I neither know what exactly you want to accomplish
% when using this package (e.\,g.~hiding text), nor what resources you have
% (e.\,g.~\pdfTeX{} version), here is a list of some possible alternatives:\\
%
% \begin{description}
%  \item[-]
%  \DescribeMacro{transparent}
%  \xpackage{transparent} package: With it some object can be made
%  (fully or partially) transparent.
% \end{description}
%
% \begin{description}
%  \item[-]
%  \DescribeMacro{hrefhide}\xpackage{hrefhide} package: It allows to
%  \textquotedblleft hide\textquotedblright\ some (hyperlinked) text when
%  printing the document while keeping the layout.
% \end{description}
%
% \bigskip
%
% \noindent (You programmed or found another alternative,
%  which is available at \CTAN{}?\\
%  OK, send an e-mail to me with the name, location at \CTAN{},
%  and a short notice, and I will probably include it in
%  the list above.)\\
%
% \smallskip
%
% \noindent About how to get those packages, please see
% subsection~\ref{ss:Downloads}.
%
% \newpage
%
% \section{Example}
%
%    \begin{macrocode}
%<*example>
\documentclass[british]{article}[2007/10/19]% v1.4h
%%%%%%%%%%%%%%%%%%%%%%%%%%%%%%%%%%%%%%%%%%%%%%%%%%%%%%%%%%%%%%%%%%%%%
\usepackage[%
 extension=pdf,%
 plainpages=false,%
 pdfpagelabels=true,%
 hyperindex=false,%
 pdflang={en},%
 pdftitle={pagecolor package example},%
 pdfauthor={H.-Martin Muench},%
 pdfsubject={Example for the pagecolor package},%
 pdfkeywords={LaTeX, pagecolor, thepagecolor, page colour,%
  H.-Martin Muench},%
 pdfview=Fit,pdfstartview=Fit,%
 pdfpagelayout=SinglePage%
]{hyperref}[2012/02/06]% v6.82o
\usepackage[x11names]{xcolor}[2007/01/21]% v2.11
 % The xcolor package would not be needed for just using
 % the base colours. The color package would be sufficient for that.
\usepackage[pagecolor={LightGoldenrod1}]{pagecolor}[2012/02/23]% v1.0e

\usepackage{afterpage}[1995/10/27]% v1.08
 % The afterpage package is generally not needed,
 % but the |\newpagecolor{somecolour}\afterpage{\restorepagecolor}|
 % construct shall be demonstrated.

\usepackage{lipsum}[2011/04/14]% v1.2
 % The lipsum package is generally not needed,
 % but some blind text is needed for the example.

\gdef\unit#1{\mathord{\thinspace\mathrm{#1}}}%
\listfiles
\begin{document}
\pagenumbering{arabic}
\section*{Example for pagecolor}

This example demonstrates the use of package\newline
\textsf{pagecolor}, v1.0e as of 2012/02/23 (HMM).\newline
The used option was \verb|pagecolor={LightGoldenrod1}|.\newline
\verb|pagecolor={none}| would be the default.\newline

For more details please see the documentation!\newline

\noindent {\color{teal} Save per page about $200\unit{ml}$ water,
$2\unit{g}$ CO$_{2}$ and $2\unit{g}$ wood:\newline
Therefore please print only if this is really necessary.}\newline

The current page (background) colour is\newline
\verb|\thepagecolor|\ =\ \thepagecolor \newline
(and \verb|\thepagecolornone|\ =\ \thepagecolornone ,
which would only be different from \verb|\thepagecolor|,
when the page colour would be \verb|none|).

\pagebreak
\pagecolor{rgb:-green!40!yellow,3;green!40!yellow,2;red,1}

{\color{white} The current page (background) colour is\newline
\verb|\thepagecolor|\ =\ \thepagecolor . \newline}

{\color{\thepagecolor} And that makes this text practically invisible.
\newline}

{\color{white} Which made the preceding line of text practically
invisible.}

\pagebreak
\newpagecolor{red}

This page uses \verb|\newpagecolor{red}|.

\pagebreak
\restorepagecolor

{\color{white}And this page uses \verb|\restorepagecolor| to restore
the page colour to the value it had before the red page.}

\pagebreak
\pagecolor{none}

This page uses \verb|\pagecolor{none}|. If the \verb|\nopagecolor|
command is known (pdf\TeX and Lua\TeX; not yet for dvips, dvipdfm(x)
or Xe\TeX), the page colour is now \verb|none|, otherwise \verb|white|:
\verb|\thepagecolor|\ =\ \thepagecolor\ and
\verb|\thepagecolornone|\ =\ \thepagecolornone .

\pagebreak
\restorepagecolor

{\color{white}\verb|\restorepagecolor| restored the page colour again.}

\pagebreak
\pagecolor{green}

This page is green due to \verb|\pagecolor{green}|.

\pagebreak
\newpagecolor{blue}\afterpage{\restorepagecolor}

{\color{white}\verb|\newpagecolor{blue}\afterpage{\restorepagecolor}|%
\newline
was used here, i.\,e.~this page is blue, and the next one will
automatically have the same page colour before it was changed to blue
here (i.\,e. green).}

\smallskip
{\color{red}\textbf{\lipsum[1-11]}}
\bigskip

The page colour was changed back at the end of the page -
in mid-sentence!

\end{document}
%</example>
%    \end{macrocode}
%
% \StopEventually{}
%
% \newpage
%
% \section{The implementation}
%
% We start off by checking that we are loading into \LaTeXe{} and
% announcing the name and version of this package.
%
%    \begin{macrocode}
%<*package>
%    \end{macrocode}
%
%    \begin{macrocode}
\NeedsTeXFormat{LaTeX2e}[2011/06/27]
\ProvidesPackage{pagecolor}[2012/02/23 v1.0e
            Provides thepagecolor (HMM)]
%    \end{macrocode}
%
% A short description of the \xpackage{pagecolor} package:
%
%    \begin{macrocode}
%% Provides the \thepagecolor, \thepagecolornone, \newpagecolor{...},
%% and \restorepagecolor commands.
%    \end{macrocode}
%
% We need the \xpackage{kvoptions} package by \textsc{Heiko Oberdiek}:
%
%    \begin{macrocode}
\RequirePackage{kvoptions}[2010/12/23]% v3.10
%    \end{macrocode}
%
% and either the \xpackage{color} or the \xpackage{xcolor} package:
%
%    \begin{macrocode}
%% \RequirePackage{ either color or xcolor }:
\@ifpackageloaded{xcolor}{% xcolor loaded
   \@ifpackagelater{xcolor}{2007/01/21}{%
      % 2007/01/21, v2.11, or even more recent: OK
    }{% else: older package version
      \PackageWarning{pagecolor}{%
        It is requested version `2007/01/21' of package\MessageBreak%
        xcolor, but only an older version is available.\MessageBreak%
       }%
     }%
 }{% xcolor not loaded
   \@ifpackageloaded{color}{%
      \RequirePackage{color}[2005/11/14]% v1.0j
    }{% \else
      \PackageWarning{pagecolor}{%
        The pagecolor package must be loaded after either\MessageBreak%
        package color or after package xcolor (at your\MessageBreak%
        option). Neither package was loaded before package\MessageBreak%
        pagecolor. Loading of package xcolor will now be\MessageBreak%
        tried automatically.\MessageBreak%
        When the pagecolor package is used with option\MessageBreak%
        pagecolor using a colour requiring e. g. x11names\MessageBreak%
        option for xcolor package, this will not work!\MessageBreak%
       }
     }% \fi
   \RequirePackage{xcolor}[2007/01/21]% v2.11
  }% \fi
%    \end{macrocode}
%
% A last information for the user:
%
%    \begin{macrocode}
%% pagecolor may work with earlier versions of LaTeX and the
%% packages, but this was not tested. Please consider updating
%% your LaTeX and packages to the most recent version
%% (if they are not already the most recent version).

%    \end{macrocode}
%
% \indent See subsection~\ref{ss:Downloads} about how to get them.\\
%
% We process the options:
%
%    \begin{macrocode}
\SetupKeyvalOptions{family=pagecolor,prefix=pagecolor@}
\DeclareStringOption[none]{pagecolor}% \pagecolor@pagecolor
\ProcessKeyvalOptions*

%    \end{macrocode}
%
% We save the original |\pagecolor| command,
%
%    \begin{macrocode}
\let\origpagecolour\pagecolor

%    \end{macrocode}
%
%    \begin{macro}{\pagecolor}
% before we redefine it to include a definition of |\thepagecolor|
% and \linebreak
% |\thepagecolornone|:
%
%    \begin{macrocode}
\renewcommand{\pagecolor}[1]{\@bsphack%
  \edef\pagecolourtmpa{#1}%
  \def\pagecolourtmpb{none}%
  \ifx\pagecolourtmpa\pagecolourtmpb
    \@ifundefined{nopagecolor}{%
      \PackageError{pagecolor}{%
        pagecolor=none requested but \string\nopagecolor\space unknown%
       }{%
        \string\pagecolor{none} was used, but the command %
        \string\nopagecolor\space is undefined.\MessageBreak%
        Please use another colour. pagecolor=white will be used now.%
        \MessageBreak%
        }%
      \gdef\thepagecolor{white}%
      \gdef\thepagecolornone{white}% although it should be "none"
      \origpagecolour{white}%
     }{%
      \nopagecolor%
     }%
  \else%
    \xdef\thepagecolor{#1}%
    \xdef\thepagecolornone{#1}%
    \origpagecolour{\thepagecolornone}%
  \fi%
  \@esphack%
}

%    \end{macrocode}
%    \end{macro}
%
% |\nopagecolor| is only defined for \pdfTeX and \LuaTeX, but not yet for dvips,
% dvipdfm(x) or \XeTeX. Maybe
% \begin{verbatim}
% \@ifundefined{nopagecolor}{\newcommand{\nopagecolor}{\pagecolor{white}}}{}
% \end{verbatim}
% could be an alternative.\\
% When |\pagecolor{none}| is used and |\nopagecolor| is defined, then \\
% |\pagecolor{none}| is made into a synonym for |\nopagecolor|.\\
% If |\nopagecolor| has not been defined, nothing needs to be done.
% Otherwise we redefine |\nopagecolor|, thus that |\thepagecolor| is
% set to |white| and |\thepagecolornone| to |none| whenever
% |\nopagecolor| is used.
%
%    \begin{macrocode}
\@ifundefined{nopagecolor}{%
  \PackageWarning{pagecolor}{%
    \string\nopagecolor\space is undefined.\MessageBreak%
    (As long as it is neither used anywhere\MessageBreak%
    \space nor defined later, that will be no problem.)\MessageBreak%
   }
  \AtEndDocument{%
  \@ifundefined{nopagecolor}{% no problem
    }{%
    \PackageError{pagecolor}{%
      \string\nopagecolor\space defined after loading package%
      \MessageBreak%
      pagecolor%
      }{\string\nopagecolor\space was defined after loading the %
       pagecolor package.\MessageBreak%
       Either define \string\nopagecolor\space earlier or load the %
       pagecolor package later.\MessageBreak%
      }
    }%
  }%
  }{% \else
    \let\orignopagecolour\nopagecolor
    \renewcommand{\nopagecolor}{%
      \xdef\thepagecolor{white}
      \xdef\thepagecolornone{none}
      \orignopagecolour
      }
  }

%    \end{macrocode}
%
% If the page colour as given with option |pagecolor={...}| is |none|,
% but |\nopagecolor| is not known, |\pagecolor@pagecolor| is set to
% |white| and a warning is given.
%
%    \begin{macrocode}
\def\pagecolourtmpb{none}
\ifx\pagecolor@pagecolor\pagecolourtmpb
  \@ifundefined{nopagecolor}{%
    \PackageWarning{pagecolor}{%
      Option pagecolor=none (maybe by default) used,\MessageBreak%
      but \string\nopagecolor\space is unknown. Please use another%
      \MessageBreak%
      option value; white will be used now.\MessageBreak%
     }
    \setkeys{pagecolor}{pagecolor=white}%
  }{% ok
  }%
\fi

%    \end{macrocode}
%
% The (new) |\pagecolor| is now just carried out.
%
%    \begin{macrocode}
\pagecolor{\pagecolor@pagecolor}

%    \end{macrocode}
%
% Now the page (background) colour and |\thepagecolor| and \\
% |\thepagecolornone| are |\pagecolor@pagecolor| (or page (background) colour
% and |\thepagecolornone| are |none|=|\pagecolor@pagecolor| and
% |\thepagecolor| is |white|), and when the page (background) colour
% is changed, |\thepagecolor| and |\thepagecolornone| are changed
% accordingly.
%
% \newpage
%
%    \begin{macro}{\newpagecolor}
% There have been requests (via e-mail and at \\
% \url{http://tex.stackexchange.com/q/25137}) to change the colour of just one
% (or two) page(s) only, similar to |\newgeometry| and |\restoregeometry| of the
% \xpackage{geometry} package (\url{http://ctan.org/pkg/geometry}). Therefore
% |\newpagecolor| and |\restorepagecolor| are introduced (as suggested by
% \textsc{haoyun\_tex}):
%
%    \begin{macrocode}
\newcommand{\newpagecolor}[1]{%
\xdef\pagecolourtmpc{\thepagecolornone}%
\pagecolor{#1}%
}

%    \end{macrocode}
%    \end{macro}
%
% |\newpagecolor{<|\textit{some colour}|>}| will execute |\pagecolor{somecolour}|
% and remember the page colour used before.\\
%
%    \begin{macro}{\restorepagecolor}
%    \begin{macrocode}
\newcommand{\restorepagecolor}{\pagecolor{\pagecolourtmpc}}

%    \end{macrocode}
%    \end{macro}
%
% |\restorecolor| (without argument) restores the page colour to the one used before
% use of the |\newpagecolor{...}| command.
%
%    \begin{macrocode}
\gdef\pagecolourtmpc{\thepagecolor}

%    \end{macrocode}
% is just a precaution for |\restorecolor| being used when no |\newpagecolor{...}|
% was used before it.\\
%
% When you want to change the colour for just one page and do not want to
% (or cannot) manually determine where the page ends,\\
% |\newpagecolor{<|\textit{some colour}|>}\afterpage{\restorepagecolor}|\\
% does the trick (and requires an additional |\usepackage{afterpage}| in the
% document's preamble).
%    \begin{macrocode}
%</package>
%    \end{macrocode}
%
% \newpage
%
% \section{Installation}
%
% \subsection{Downloads\label{ss:Downloads}}
%
% Everything is available on \CTAN{}, \url{http://www.ctan.org/tex-archive/},
% but may need additional packages themselves.\\
%
% \DescribeMacro{pagecolor.dtx}
% For unpacking the |pagecolor.dtx| file and constructing the documentation
% it is required:
% \begin{description}
% \item[-] \TeX{} Format \LaTeXe{}, \url{http://www.CTAN.org/}
%
% \item[-] document class \xpackage{ltxdoc}, 2007/11/11, v2.0u,
%           \url{http://ctan.org/pkg/ltxdoc}
%
% \item[-] package \xpackage{holtxdoc}, 2011/02/04, v0.21,
%           \url{http://ctan.org/pkg/holtxdoc}
% \end{description}
%
% \DescribeMacro{pagecolor.sty}
% The |pagecolor.sty| for \LaTeXe{} (i.\,e. each document using
% the \xpackage{pagecolor} package) requires:
% \begin{description}
% \item[-] \TeX{} Format \LaTeXe{}, \url{http://www.CTAN.org/}
%
% \item[-] package \xpackage{kvoptions}, 2010/12/23, v3.10,
%           \url{http://ctan.org/pkg/kvoptions}
% \end{description}
% and either
% \begin{description}
% \item[-] package \xpackage{color}, 2005/11/14, v1.0j,
%           \url{http://ctan.org/pkg/color} (from the\\
%           \xpackage{graphics} package bundle)
% \end{description}
% or
% \begin{description}
% \item[-] package \xpackage{xcolor}, 2007/01/21, v2.11,
%           \url{http://ctan.org/pkg/xcolor}
% \end{description}
%
% \DescribeMacro{pagecolor-example.tex}
% The |pagecolor-example.tex| requires the same files as all
% documents using the \xpackage{pagecolor} package and additionally:
% \begin{description}
% \item[-] class \xpackage{article}, 2007/10/19, v1.4h, from \xpackage{classes.dtx}:\\
%           \CTAN{macros/latex/base/classes.dtx}
%
% \item[-] package \xpackage{pagecolor}, 2012/02/23, v1.0e,
%           \url{http://ctan.org/pkg/pagecolor}\\
%           (Well, it is the example file for this package, and because you are
%            reading the documentation for the \xpackage{pagecolor} package,
%            it can be assumed that you already have some version of it --
%            is it the current one?)
%
% \item[-] package \xpackage{xcolor}, 2007/01/21, v2.11,
%           \url{http://ctan.org/pkg/xcolor}\\
%           This package would not be needed for the use of just base colours only,
%           the \xpackage{color} package would be sufficient for that.
% \end{description}
%
% \DescribeMacro{transparent}
% \DescribeMacro{hrefhide}
% As possible alternatives in section \ref{sec:Alternatives} there are listed
% \begin{description}
% \item[-] package \xpackage{transparent}, 2007/01/08, v1.0, \\
%           \url{http://ctan.org/pkg/transparent}
%
% \item[-] package \xpackage{hrefhide}, 2011/04/29, v1.0f, \\
%           \url{http://ctan.org/pkg/hrefhide}
% \end{description}
%
% \DescribeMacro{Oberdiek}
% \DescribeMacro{holtxdoc}
% \DescribeMacro{kvoptions}
% All packages of \textsc{Heiko Oberdiek}'s bundle `oberdiek'
% (especially \xpackage{holtxdoc} and \xpackage{kvoptions})
% are also available in a TDS compliant ZIP archive:\\
% \CTAN{install/macros/latex/contrib/oberdiek.tds.zip}.\\
% It is probably best to download and use this, because the packages in there
% are quite probably both recent and compatible among themselves.\\
%
% \DescribeMacro{hyperref}
% \noindent \xpackage{hyperref} is not included in that bundle and needs to be
% downloaded separately,\\
% \url{http://mirror.ctan.org/install/macros/latex/contrib/hyperref.tds.zip}.\\
%
% \DescribeMacro{M\"{u}nch}
% A hyperlinked list of my (other) packages can be found at
% \url{http://www.Uni-Bonn.de/~uzs5pv/LaTeX.html}.\\
%
% \subsection{Package, unpacking TDS}
% \paragraph{Package.} This package is available on \CTAN{}
% \begin{description}
% \item[\CTAN{macros/latex/contrib/pagecolor/pagecolor.dtx}]\hspace*{0.1cm} \\
%       The source file.
% \item[\CTAN{macros/latex/contrib/pagecolor/pagecolor.pdf}]\hspace*{0.1cm} \\
%       The documentation.
% \item[\CTAN{macros/latex/contrib/pagecolor/pagecolor-example.pdf}]\hspace*{0.1cm} \\
%       The compiled example file, as it should look like.
% \item[\CTAN{macros/latex/contrib/pagecolor/README}]\hspace*{0.1cm} \\
%       The README file.
% \end{description}
% There is also a pagecolor.tds.zip available:
% \begin{description}
% \item[\CTAN{install/macros/latex/contrib/pagecolor.tds.zip}]\hspace*{0.1cm} \\
%       Everything in \xfile{TDS} compliant, compiled format.
% \end{description}
% which additionally contains\\
% \begin{tabular}{ll}
% pagecolor.ins & The installation file.\\
% pagecolor.drv & The driver to generate the documentation.\\
% pagecolor.sty & The \xext{sty}le file.\\
% pagecolor-example.tex & The example file.
% \end{tabular}
%
% \bigskip
%
% \noindent For required other packages, please see the preceding subsection.
%
% \paragraph{Unpacking.} The \xfile{.dtx} file is a self-extracting
% \docstrip{} archive. The files are extracted by running the
% \xext{.dtx} through \plainTeX{}:
% \begin{quote}
%   \verb|tex pagecolor.dtx|
% \end{quote}
%
% About generating the documentation see paragraph~\ref{GenDoc} below.\\
%
% \paragraph{TDS.} Now the different files must be moved into
% the different directories in your installation TDS tree
% (also known as \xfile{texmf} tree):
% \begin{quote}
% \def\t{^^A
% \begin{tabular}{@{}>{\ttfamily}l@{ $\rightarrow$ }>{\ttfamily}l@{}}
%   pagecolor.sty & tex/latex/pagecolor/pagecolor.sty\\
%   pagecolor.pdf & doc/latex/pagecolor/pagecolor.pdf\\
%   pagecolor-example.tex & doc/latex/pagecolor/pagecolor-example.tex\\
%   pagecolor-example.pdf & doc/latex/pagecolor/pagecolor-example.pdf\\
%   pagecolor.dtx & source/latex/pagecolor/pagecolor.dtx\\
% \end{tabular}^^A
% }^^A
% \sbox0{\t}^^A
% \ifdim\wd0>\linewidth
%   \begingroup
%     \advance\linewidth by\leftmargin
%     \advance\linewidth by\rightmargin
%   \edef\x{\endgroup
%     \def\noexpand\lw{\the\linewidth}^^A
%   }\x
%   \def\lwbox{^^A
%     \leavevmode
%     \hbox to \linewidth{^^A
%       \kern-\leftmargin\relax
%       \hss
%       \usebox0
%       \hss
%       \kern-\rightmargin\relax
%     }^^A
%   }^^A
%   \ifdim\wd0>\lw
%     \sbox0{\small\t}^^A
%     \ifdim\wd0>\linewidth
%       \ifdim\wd0>\lw
%         \sbox0{\footnotesize\t}^^A
%         \ifdim\wd0>\linewidth
%           \ifdim\wd0>\lw
%             \sbox0{\scriptsize\t}^^A
%             \ifdim\wd0>\linewidth
%               \ifdim\wd0>\lw
%                 \sbox0{\tiny\t}^^A
%                 \ifdim\wd0>\linewidth
%                   \lwbox
%                 \else
%                   \usebox0
%                 \fi
%               \else
%                 \lwbox
%               \fi
%             \else
%               \usebox0
%             \fi
%           \else
%             \lwbox
%           \fi
%         \else
%           \usebox0
%         \fi
%       \else
%         \lwbox
%       \fi
%     \else
%       \usebox0
%     \fi
%   \else
%     \lwbox
%   \fi
% \else
%   \usebox0
% \fi
% \end{quote}
% If you have a \xfile{docstrip.cfg} that configures and enables \docstrip{}'s
% \xfile{TDS} installing feature, then some files can already be in the right
% place, see the documentation of \docstrip{}.
%
% \subsection{Refresh file name databases}
%
% If your \TeX{}~distribution (\teTeX{}, \mikTeX{},\dots{}) relies on file name
% databases, you must refresh these. For example, \teTeX{} users run
% \verb|texhash| or \verb|mktexlsr|.
%
% \subsection{Some details for the interested}
%
% \paragraph{Unpacking with \LaTeX{}.}
% The \xfile{.dtx} chooses its action depending on the format:
% \begin{description}
% \item[\plainTeX:] Run \docstrip{} and extract the files.
% \item[\LaTeX:] Generate the documentation.
% \end{description}
% If you insist on using \LaTeX{} for \docstrip{} (really,
% \docstrip{} does not need \LaTeX{}), then inform the autodetect routine
% about your intention:
% \begin{quote}
%   \verb|latex \let\install=y\input{pagecolor.dtx}|
% \end{quote}
% Do not forget to quote the argument according to the demands
% of your shell.
%
% \paragraph{Generating the documentation.\label{GenDoc}}
% You can use both the \xfile{.dtx} or the \xfile{.drv} to generate
% the documentation. The process can be configured by a
% configuration file \xfile{ltxdoc.cfg}. For instance, put the following
% line into this file, if you want to have A4 as paper format:
% \begin{quote}
%   \verb|\PassOptionsToClass{a4paper}{article}|
% \end{quote}
%
% \noindent An example follows how to generate the
% documentation with \pdfLaTeX{}:
%
% \begin{quote}
%\begin{verbatim}
%pdflatex pagecolor.dtx
%makeindex -s gind.ist pagecolor.idx
%pdflatex pagecolor.dtx
%makeindex -s gind.ist pagecolor.idx
%pdflatex pagecolor.dtx
%\end{verbatim}
% \end{quote}
%
% \subsection{Compiling the example}
%
% The example file, \textsf{pagecolor-example.tex}, can be compiled via\\
% |(pdf)latex pagecolor-example.tex|.\\
%
% \section{Acknowledgements}
%
% I would like to thank \textsc{Heiko Oberdiek} for providing a~lot~(!) of
% useful packages (from which I also got everything I know about creating
% a file in \xext{dtx} format, ok, say it: copying), and the
% \Newsgroup{comp.text.tex} and \Newsgroup{de.comp.text.tex} newsgroups
% for their help in all things \TeX{}, especially all contributors to the
% discussion at
% \url{http://groups.google.com/group/comp.text.tex/browse_thread/thread/533576ebe44d60f6/e1654d70a19de63c?lnk=gst&q=Determination+of+the+current+page+colour#e1654d70a19de63c}
% \ (\textsc{H.~Oberdiek} \& \textsc{Gouailles}).\\
% I thank \textsc{haoyun\_tex} for suggesting the |\newpagecolor|/|\restorepagecolor|
% pair of commands and everyone at \url{http://tex.stackexchange.com/q/25137}
% for their contributions there.
%
% \bigskip
%
% \phantomsection
% \begin{History}\label{History}
%   \begin{Version}{2011/07/16 v1.0a}
%     \item First version discussed at \Newsgroup{comp.text.tex}.
%   \end{Version}
%   \begin{Version}{2011/08/06 v1.0b}
%     \item Changed version uploaded to \CTAN{}.
%   \end{Version}
%   \begin{Version}{2011/08/08 v1.0c}
%     \item Fixed a |\setkeys|.
%   \end{Version}
%   \begin{Version}{2012/02/01 v1.0d}
%     \item Bugfix: Obsolete installation path given in the documentation, updated.
%     \item New commands: |\newpagecolor{...}|, |\restorepagecolor|.
%     \item Update of documentation, README, and \xfile{dtx} internals.
%   \end{Version}
%   \begin{Version}{2012/02/23 v1.0e}
%     \item Fixed an error in the documentation.
%     \item Check for loading of \xpackage{color} or \xpackage{xcolor} package
%             and their versions has been changed, because \xpackage{xcolor} sets\\
%             |\@namedef{ver@color.sty}{1999/02/16}|\\
%             which gave a warning about old \xpackage{color} package even if a new
%             version was used.
%   \end{Version}
% \end{History}
%
% \bigskip
%
% When you find a mistake or have a suggestion for an improvement of this package,
% please send an e-mail to the maintainer, thanks! (Please see BUG REPORTS in the README.)
%
% \newpage
%
% \PrintIndex
%
% \Finale
\endinput|
% \end{quote}
% Do not forget to quote the argument according to the demands
% of your shell.
%
% \paragraph{Generating the documentation.\label{GenDoc}}
% You can use both the \xfile{.dtx} or the \xfile{.drv} to generate
% the documentation. The process can be configured by a
% configuration file \xfile{ltxdoc.cfg}. For instance, put the following
% line into this file, if you want to have A4 as paper format:
% \begin{quote}
%   \verb|\PassOptionsToClass{a4paper}{article}|
% \end{quote}
%
% \noindent An example follows how to generate the
% documentation with \pdfLaTeX{}:
%
% \begin{quote}
%\begin{verbatim}
%pdflatex pagecolor.dtx
%makeindex -s gind.ist pagecolor.idx
%pdflatex pagecolor.dtx
%makeindex -s gind.ist pagecolor.idx
%pdflatex pagecolor.dtx
%\end{verbatim}
% \end{quote}
%
% \subsection{Compiling the example}
%
% The example file, \textsf{pagecolor-example.tex}, can be compiled via\\
% |(pdf)latex pagecolor-example.tex|.\\
%
% \section{Acknowledgements}
%
% I would like to thank \textsc{Heiko Oberdiek} for providing a~lot~(!) of
% useful packages (from which I also got everything I know about creating
% a file in \xext{dtx} format, ok, say it: copying), and the
% \Newsgroup{comp.text.tex} and \Newsgroup{de.comp.text.tex} newsgroups
% for their help in all things \TeX{}, especially all contributors to the
% discussion at
% \url{http://groups.google.com/group/comp.text.tex/browse_thread/thread/533576ebe44d60f6/e1654d70a19de63c?lnk=gst&q=Determination+of+the+current+page+colour#e1654d70a19de63c}
% \ (\textsc{H.~Oberdiek} \& \textsc{Gouailles}).\\
% I thank \textsc{haoyun\_tex} for suggesting the |\newpagecolor|/|\restorepagecolor|
% pair of commands and everyone at \url{http://tex.stackexchange.com/q/25137}
% for their contributions there.
%
% \bigskip
%
% \phantomsection
% \begin{History}\label{History}
%   \begin{Version}{2011/07/16 v1.0a}
%     \item First version discussed at \Newsgroup{comp.text.tex}.
%   \end{Version}
%   \begin{Version}{2011/08/06 v1.0b}
%     \item Changed version uploaded to \CTAN{}.
%   \end{Version}
%   \begin{Version}{2011/08/08 v1.0c}
%     \item Fixed a |\setkeys|.
%   \end{Version}
%   \begin{Version}{2012/02/01 v1.0d}
%     \item Bugfix: Obsolete installation path given in the documentation, updated.
%     \item New commands: |\newpagecolor{...}|, |\restorepagecolor|.
%     \item Update of documentation, README, and \xfile{dtx} internals.
%   \end{Version}
%   \begin{Version}{2012/02/23 v1.0e}
%     \item Fixed an error in the documentation.
%     \item Check for loading of \xpackage{color} or \xpackage{xcolor} package
%             and their versions has been changed, because \xpackage{xcolor} sets\\
%             |\@namedef{ver@color.sty}{1999/02/16}|\\
%             which gave a warning about old \xpackage{color} package even if a new
%             version was used.
%   \end{Version}
% \end{History}
%
% \bigskip
%
% When you find a mistake or have a suggestion for an improvement of this package,
% please send an e-mail to the maintainer, thanks! (Please see BUG REPORTS in the README.)
%
% \newpage
%
% \PrintIndex
%
% \Finale
\endinput|
% \end{quote}
% Do not forget to quote the argument according to the demands
% of your shell.
%
% \paragraph{Generating the documentation.\label{GenDoc}}
% You can use both the \xfile{.dtx} or the \xfile{.drv} to generate
% the documentation. The process can be configured by a
% configuration file \xfile{ltxdoc.cfg}. For instance, put the following
% line into this file, if you want to have A4 as paper format:
% \begin{quote}
%   \verb|\PassOptionsToClass{a4paper}{article}|
% \end{quote}
%
% \noindent An example follows how to generate the
% documentation with \pdfLaTeX{}:
%
% \begin{quote}
%\begin{verbatim}
%pdflatex pagecolor.dtx
%makeindex -s gind.ist pagecolor.idx
%pdflatex pagecolor.dtx
%makeindex -s gind.ist pagecolor.idx
%pdflatex pagecolor.dtx
%\end{verbatim}
% \end{quote}
%
% \subsection{Compiling the example}
%
% The example file, \textsf{pagecolor-example.tex}, can be compiled via\\
% |(pdf)latex pagecolor-example.tex|.\\
%
% \section{Acknowledgements}
%
% I would like to thank \textsc{Heiko Oberdiek} for providing a~lot~(!) of
% useful packages (from which I also got everything I know about creating
% a file in \xext{dtx} format, ok, say it: copying), and the
% \Newsgroup{comp.text.tex} and \Newsgroup{de.comp.text.tex} newsgroups
% for their help in all things \TeX{}, especially all contributors to the
% discussion at
% \url{http://groups.google.com/group/comp.text.tex/browse_thread/thread/533576ebe44d60f6/e1654d70a19de63c?lnk=gst&q=Determination+of+the+current+page+colour#e1654d70a19de63c}
% \ (\textsc{H.~Oberdiek} \& \textsc{Gouailles}).\\
% I thank \textsc{haoyun\_tex} for suggesting the |\newpagecolor|/|\restorepagecolor|
% pair of commands and everyone at \url{http://tex.stackexchange.com/q/25137}
% for their contributions there.
%
% \bigskip
%
% \phantomsection
% \begin{History}\label{History}
%   \begin{Version}{2011/07/16 v1.0a}
%     \item First version discussed at \Newsgroup{comp.text.tex}.
%   \end{Version}
%   \begin{Version}{2011/08/06 v1.0b}
%     \item Changed version uploaded to \CTAN{}.
%   \end{Version}
%   \begin{Version}{2011/08/08 v1.0c}
%     \item Fixed a |\setkeys|.
%   \end{Version}
%   \begin{Version}{2012/02/01 v1.0d}
%     \item Bugfix: Obsolete installation path given in the documentation, updated.
%     \item New commands: |\newpagecolor{...}|, |\restorepagecolor|.
%     \item Update of documentation, README, and \xfile{dtx} internals.
%   \end{Version}
%   \begin{Version}{2012/02/23 v1.0e}
%     \item Fixed an error in the documentation.
%     \item Check for loading of \xpackage{color} or \xpackage{xcolor} package
%             and their versions has been changed, because \xpackage{xcolor} sets\\
%             |\@namedef{ver@color.sty}{1999/02/16}|\\
%             which gave a warning about old \xpackage{color} package even if a new
%             version was used.
%   \end{Version}
% \end{History}
%
% \bigskip
%
% When you find a mistake or have a suggestion for an improvement of this package,
% please send an e-mail to the maintainer, thanks! (Please see BUG REPORTS in the README.)
%
% \newpage
%
% \PrintIndex
%
% \Finale
\endinput
%        (quote the arguments according to the demands of your shell)
%
% Documentation:
%    (a) If pagecolor.drv is present:
%           (pdf)latex pagecolor.drv
%           makeindex -s gind.ist pagecolor.idx
%           (pdf)latex pagecolor.drv
%           makeindex -s gind.ist pagecolor.idx
%           (pdf)latex pagecolor.drv
%    (b) Without pagecolor.drv:
%           (pdf)latex pagecolor.dtx
%           makeindex -s gind.ist pagecolor.idx
%           (pdf)latex pagecolor.dtx
%           makeindex -s gind.ist pagecolor.idx
%           (pdf)latex pagecolor.dtx
%
%    The class ltxdoc loads the configuration file ltxdoc.cfg
%    if available. Here you can specify further options, e.g.
%    use DIN A4 as paper format:
%       \PassOptionsToClass{a4paper}{article}
%
% Installation:
%    TDS:tex/latex/pagecolor/pagecolor.sty
%    TDS:doc/latex/pagecolor/pagecolor.pdf
%    TDS:doc/latex/pagecolor/pagecolor-example.tex
%    TDS:doc/latex/pagecolor/pagecolor-example.pdf
%    TDS:source/latex/pagecolor/pagecolor.dtx
%
%<*ignore>
\begingroup
  \catcode123=1 %
  \catcode125=2 %
  \def\x{LaTeX2e}%
\expandafter\endgroup
\ifcase 0\ifx\install y1\fi\expandafter
         \ifx\csname processbatchFile\endcsname\relax\else1\fi
         \ifx\fmtname\x\else 1\fi\relax
\else\csname fi\endcsname
%</ignore>
%<*install>
\input docstrip.tex
\Msg{*****************************************************************}
\Msg{* Installation                                                   }
\Msg{* Package: pagecolor 2012/02/23 v1.0e Provides thepagecolor (HMM)}
\Msg{*****************************************************************}

\keepsilent
\askforoverwritefalse

\let\MetaPrefix\relax
\preamble

This is a generated file.

Project: pagecolor
Version: 2012/02/23 v1.0e

Copyright (C) 2011 - 2012 by
    H.-Martin M"unch <Martin dot Muench at Uni-Bonn dot de>

The usual disclaimer applies:
If it doesn't work right that's your problem.
(Nevertheless, send an e-mail to the maintainer
 when you find an error in this package.)

This work may be distributed and/or modified under the
conditions of the LaTeX Project Public License, either
version 1.3c of this license or (at your option) any later
version. This version of this license is in
   http://www.latex-project.org/lppl/lppl-1-3c.txt
and the latest version of this license is in
   http://www.latex-project.org/lppl.txt
and version 1.3c or later is part of all distributions of
LaTeX version 2005/12/01 or later.

This work has the LPPL maintenance status "maintained".

The Current Maintainer of this work is H.-Martin Muench.

This work consists of the main source file pagecolor.dtx,
the README, and the derived files
   pagecolor.sty, pagecolor.pdf,
   pagecolor.ins, pagecolor.drv,
   pagecolor-example.tex, pagecolor-example.pdf.

\endpreamble
\let\MetaPrefix\DoubleperCent

\generate{%
  \file{pagecolor.ins}{\from{pagecolor.dtx}{install}}%
  \file{pagecolor.drv}{\from{pagecolor.dtx}{driver}}%
  \usedir{tex/latex/pagecolor}%
  \file{pagecolor.sty}{\from{pagecolor.dtx}{package}}%
  \usedir{doc/latex/pagecolor}%
  \file{pagecolor-example.tex}{\from{pagecolor.dtx}{example}}%
}

\catcode32=13\relax% active space
\let =\space%
\Msg{************************************************************************}
\Msg{*}
\Msg{* To finish the installation you have to move the following}
\Msg{* file into a directory searched by TeX:}
\Msg{*}
\Msg{*     pagecolor.sty}
\Msg{*}
\Msg{* To produce the documentation run the file `pagecolor.drv'}
\Msg{* through (pdf)LaTeX, e.g.}
\Msg{*  pdflatex pagecolor.drv}
\Msg{*  makeindex -s gind.ist pagecolor.idx}
\Msg{*  pdflatex pagecolor.drv}
\Msg{*  makeindex -s gind.ist pagecolor.idx}
\Msg{*  pdflatex pagecolor.drv}
\Msg{*}
\Msg{* At least three runs are necessary e.g. to get the}
\Msg{*  references right!}
\Msg{*}
\Msg{* Happy TeXing!}
\Msg{*}
\Msg{************************************************************************}

\endbatchfile
%</install>
%<*ignore>
\fi
%</ignore>
%
% \section{The documentation driver file}
%
% The next bit of code contains the documentation driver file for
% \TeX{}, i.\,e., the file that will produce the documentation you
% are currently reading. It will be extracted from this file by the
% \texttt{docstrip} programme. That is, run \LaTeX{} on \texttt{docstrip}
% and specify the \texttt{driver} option when \texttt{docstrip}
% asks for options.
%
%    \begin{macrocode}
%<*driver>
\NeedsTeXFormat{LaTeX2e}[2011/06/27]
\ProvidesFile{pagecolor.drv}%
  [2012/02/23 v1.0e Provides thepagecolor (HMM)]
\documentclass{ltxdoc}[2007/11/11]% v2.0u
\usepackage{holtxdoc}[2011/02/04]%  v0.21
%% pagecolor may work with earlier versions of LaTeX2e and those
%% class and package, but this was not tested.
%% Please consider updating your LaTeX, class, and package
%% to the most recent version (if they are not already the most
%% recent version).
\hypersetup{%
 pdfsubject={Providing the thepagecolor command (HMM)},%
 pdfkeywords={LaTeX, pagecolor, thepagecolor, H.-Martin Muench},%
 pdfencoding=auto,%
 pdflang={en},%
 breaklinks=true,%
 linktoc=all,%
 pdfstartview=FitH,%
 pdfpagelayout=OneColumn,%
 bookmarksnumbered=true,%
 bookmarksopen=true,%
 bookmarksopenlevel=3,%
 pdfmenubar=true,%
 pdftoolbar=true,%
 pdfwindowui=true,%
 pdfnewwindow=true%
}
\CodelineIndex
\hyphenation{printing docu-ment}
\gdef\unit#1{\mathord{\thinspace\mathrm{#1}}}%
\begin{document}
  \DocInput{pagecolor.dtx}%
\end{document}
%</driver>
%    \end{macrocode}
%
% \fi
%
% \CheckSum{137}
%
% \CharacterTable
%  {Upper-case    \A\B\C\D\E\F\G\H\I\J\K\L\M\N\O\P\Q\R\S\T\U\V\W\X\Y\Z
%   Lower-case    \a\b\c\d\e\f\g\h\i\j\k\l\m\n\o\p\q\r\s\t\u\v\w\x\y\z
%   Digits        \0\1\2\3\4\5\6\7\8\9
%   Exclamation   \!     Double quote  \"     Hash (number) \#
%   Dollar        \$     Percent       \%     Ampersand     \&
%   Acute accent  \'     Left paren    \(     Right paren   \)
%   Asterisk      \*     Plus          \+     Comma         \,
%   Minus         \-     Point         \.     Solidus       \/
%   Colon         \:     Semicolon     \;     Less than     \<
%   Equals        \=     Greater than  \>     Question mark \?
%   Commercial at \@     Left bracket  \[     Backslash     \\
%   Right bracket \]     Circumflex    \^     Underscore    \_
%   Grave accent  \`     Left brace    \{     Vertical bar  \|
%   Right brace   \}     Tilde         \~}
%
% \GetFileInfo{pagecolor.drv}
%
% \begingroup
%   \def\x{\#,\$,\^,\_,\~,\ ,\&,\{,\},\%}%
%   \makeatletter
%   \@onelevel@sanitize\x
% \expandafter\endgroup
% \expandafter\DoNotIndex\expandafter{\x}
% \expandafter\DoNotIndex\expandafter{\string\ }
% \begingroup
%   \makeatletter
%     \lccode`9=32\relax
%     \lowercase{%^^A
%       \edef\x{\noexpand\DoNotIndex{\@backslashchar9}}%^^A
%     }%^^A
%   \expandafter\endgroup\x
%
% \DoNotIndex{\,,\\}
% \DoNotIndex{\documentclass,\usepackage,\ProvidesPackage,\begin,\end}
% \DoNotIndex{\MessageBreak}
% \DoNotIndex{\NeedsTeXFormat,\DoNotIndex,\verb}
% \DoNotIndex{\def,\edef,\gdef,\global}
% \DoNotIndex{\ifx,\listfiles,\mathord,\mathrm}
% \DoNotIndex{\kvoptions,\SetupKeyvalOptions,\ProcessKeyvalOptions}
% \DoNotIndex{\bigskip,\space,\thinspace,\ldots}
% \DoNotIndex{\indent,\noindent,\newline,\linebreak,\pagebreak,\newpage}
% \DoNotIndex{\textbf,\textit,\textsf,\texttt,\textsc,\textquotedblleft,\textquotedblright}
% \DoNotIndex{\plainTeX,\TeX,\LaTeX,\pdfLaTeX}
% \DoNotIndex{\section,\color,\pagenumbering,\smallskip}
% \DoNotIndex{\@ifundefined,\@ifpackageloaded}
% \DoNotIndex{\DeclareStringOption}
%
% \title{The \xpackage{pagecolor} package}
% \date{2012/02/23 v1.0e}
% \author{H.-Martin M\"{u}nch\\\xemail{Martin.Muench at Uni-Bonn.de}}
%
% \maketitle
%
% \begin{abstract}
%  This \LaTeX{} package provides the command |\thepagecolor|,
% which gives the current page (background) colour,
% i.\,e.~the argument used with the most recent call of |\pagecolor{...}|.
% The command |\thepagecolornone| gives the same colour as |\thepagecolor|,
% except when the page background colour is
% \textquotedblleft|none|\textquotedblright. In that case |\thepagecolor| is
% |white| and |\thepagecolornone| is |none|.\\
% Similar to |\newgeometry| and |\restoregeometry| of the \xpackage{geometry}
% package\\
% |\newpagecolor{<|\textit{some colour} |>}| and |\restorepagecolor|
% are provided.
% \end{abstract}
%
% \bigskip
%
% \noindent Disclaimer for web links: The author is not responsible for any contents
% referred to in this work unless he has full knowledge of illegal contents.
% If any damage occurs by the use of information presented there, only the
% author of the respective pages might be liable, not the one who has referred
% to these pages.
%
% \bigskip
%
% \noindent {\color{green} Save per page about $200\unit{ml}$ water,
% $2\unit{g}$ CO$_{2}$ and $2\unit{g}$ wood:\\
% Therefore please print only if this is really necessary.}
%
% \pagebreak
%
% \tableofcontents
%
% \bigskip
%
% \section{Introduction\label{sec:Introduction}}
%
% \indent This \LaTeX{} package provides the command |\thepagecolor|,
% which gives the current page (background) colour,
% i.\,e.~the argument used with the most recent call of |\pagecolor{...}|.
% The package should be loaded before any package sets a page (background)
% colour, but after \xpackage{color} or \xpackage{xcolor} package.
% Its option |pagecolor={...}| is used to set the |\pagecolor{...}|.\\
% The command |\thepagecolornone| gives the same colour as |\thepagecolor|,
% except when the page background colour is
% \textquotedblleft|none|\textquotedblright (e.\,g. result of using the
% |\nopagecolor| command). In that case |\thepagecolor| is |white| and\linebreak
% |\thepagecolornone| is |none|.\\
% Similar to |\newgeometry| and |\restoregeometry| of the \xpackage{geometry}
% package\\
% |\newpagecolor{<|\textit{some colour}|>}| and |\restorepagecolor|
% are provided.
%
% \pagebreak
%
% \section{Usage}
%
% \indent Just load the package placing
% \begin{quote}
%   |\usepackage[<|\textit{option}|>]{pagecolor}|
% \end{quote}
% \noindent in the preamble of your \LaTeXe{} source file.
% This should be done before another package uses |\pagecolor|,
% but after |\nopagecolor| is defined (if that is defined in the document
% at all).\\
% Afterwards |\pagecolor{...}| can be used to change the page (background)
% colour as usual. Then |\thepagecolor| gives the current page (background)
% colour (in the same format as given with |\pagecolor{...}|).\\
% Similar to |\newgeometry| and |\restoregeometry| of the \xpackage{geometry}
% package\\
% |\newpagecolor{<|\textit{some colour} |>}| and |\restorepagecolor|
% are provided:\\
% |\newpagecolor{<|\textit{some colour} |>}| will execute
% |\pagecolor{<|\textit{some colour} |>}| and remember the page colour used before.
% |\restorecolor| (without argument) restores the page colour to the one used before
%  use of the |\newpagecolor{...}| command. When you want to change the colour for
%  just one page and do not want to (or cannot) manually determine where the page ends,\\
% |\newpagecolor{<|\textit{some colour}|>}\afterpage{\restorepagecolor}|\\
% does the trick (and requires a |\usepackage{afterpage}| in the document's
% preamble), or for short\\
% |\newcommand{\onepagecolor}[1]{%|\\
% |\newpagecolor{#1}\afterpage{\restorepagecolor}}|\\
% in the preamble and\\
% |\onepagecolor{<|\textit{some colour}|>}| in the document.
%
% \subsection{Option}
% \DescribeMacro{option}
% \indent The \xpackage{pagecolor} package takes the following option:
%
% \subsubsection{pagecolor\label{sss:pagecolor}}
% \DescribeMacro{pagecolor}
% \indent The option |pagecolor={...}| takes as value a colour. This
% could be as simple as |white| or |black|, but when e.\,g.~the
% \xpackage{xcolor} package is used (loaded before \xpackage{pagecolor}!),
% also colours like |red!50!green!20!blue| are possible.
% The default is |pagecolor={none}|.
%
% \section{Alternatives\label{sec:Alternatives}}
%
% As I neither know what exactly you want to accomplish
% when using this package (e.\,g.~hiding text), nor what resources you have
% (e.\,g.~\pdfTeX{} version), here is a list of some possible alternatives:\\
%
% \begin{description}
%  \item[-]
%  \DescribeMacro{transparent}
%  \xpackage{transparent} package: With it some object can be made
%  (fully or partially) transparent.
% \end{description}
%
% \begin{description}
%  \item[-]
%  \DescribeMacro{hrefhide}\xpackage{hrefhide} package: It allows to
%  \textquotedblleft hide\textquotedblright\ some (hyperlinked) text when
%  printing the document while keeping the layout.
% \end{description}
%
% \bigskip
%
% \noindent (You programmed or found another alternative,
%  which is available at \CTAN{}?\\
%  OK, send an e-mail to me with the name, location at \CTAN{},
%  and a short notice, and I will probably include it in
%  the list above.)\\
%
% \smallskip
%
% \noindent About how to get those packages, please see
% subsection~\ref{ss:Downloads}.
%
% \newpage
%
% \section{Example}
%
%    \begin{macrocode}
%<*example>
\documentclass[british]{article}[2007/10/19]% v1.4h
%%%%%%%%%%%%%%%%%%%%%%%%%%%%%%%%%%%%%%%%%%%%%%%%%%%%%%%%%%%%%%%%%%%%%
\usepackage[%
 extension=pdf,%
 plainpages=false,%
 pdfpagelabels=true,%
 hyperindex=false,%
 pdflang={en},%
 pdftitle={pagecolor package example},%
 pdfauthor={H.-Martin Muench},%
 pdfsubject={Example for the pagecolor package},%
 pdfkeywords={LaTeX, pagecolor, thepagecolor, page colour,%
  H.-Martin Muench},%
 pdfview=Fit,pdfstartview=Fit,%
 pdfpagelayout=SinglePage%
]{hyperref}[2012/02/06]% v6.82o
\usepackage[x11names]{xcolor}[2007/01/21]% v2.11
 % The xcolor package would not be needed for just using
 % the base colours. The color package would be sufficient for that.
\usepackage[pagecolor={LightGoldenrod1}]{pagecolor}[2012/02/23]% v1.0e

\usepackage{afterpage}[1995/10/27]% v1.08
 % The afterpage package is generally not needed,
 % but the |\newpagecolor{somecolour}\afterpage{\restorepagecolor}|
 % construct shall be demonstrated.

\usepackage{lipsum}[2011/04/14]% v1.2
 % The lipsum package is generally not needed,
 % but some blind text is needed for the example.

\gdef\unit#1{\mathord{\thinspace\mathrm{#1}}}%
\listfiles
\begin{document}
\pagenumbering{arabic}
\section*{Example for pagecolor}

This example demonstrates the use of package\newline
\textsf{pagecolor}, v1.0e as of 2012/02/23 (HMM).\newline
The used option was \verb|pagecolor={LightGoldenrod1}|.\newline
\verb|pagecolor={none}| would be the default.\newline

For more details please see the documentation!\newline

\noindent {\color{teal} Save per page about $200\unit{ml}$ water,
$2\unit{g}$ CO$_{2}$ and $2\unit{g}$ wood:\newline
Therefore please print only if this is really necessary.}\newline

The current page (background) colour is\newline
\verb|\thepagecolor|\ =\ \thepagecolor \newline
(and \verb|\thepagecolornone|\ =\ \thepagecolornone ,
which would only be different from \verb|\thepagecolor|,
when the page colour would be \verb|none|).

\pagebreak
\pagecolor{rgb:-green!40!yellow,3;green!40!yellow,2;red,1}

{\color{white} The current page (background) colour is\newline
\verb|\thepagecolor|\ =\ \thepagecolor . \newline}

{\color{\thepagecolor} And that makes this text practically invisible.
\newline}

{\color{white} Which made the preceding line of text practically
invisible.}

\pagebreak
\newpagecolor{red}

This page uses \verb|\newpagecolor{red}|.

\pagebreak
\restorepagecolor

{\color{white}And this page uses \verb|\restorepagecolor| to restore
the page colour to the value it had before the red page.}

\pagebreak
\pagecolor{none}

This page uses \verb|\pagecolor{none}|. If the \verb|\nopagecolor|
command is known (pdf\TeX and Lua\TeX; not yet for dvips, dvipdfm(x)
or Xe\TeX), the page colour is now \verb|none|, otherwise \verb|white|:
\verb|\thepagecolor|\ =\ \thepagecolor\ and
\verb|\thepagecolornone|\ =\ \thepagecolornone .

\pagebreak
\restorepagecolor

{\color{white}\verb|\restorepagecolor| restored the page colour again.}

\pagebreak
\pagecolor{green}

This page is green due to \verb|\pagecolor{green}|.

\pagebreak
\newpagecolor{blue}\afterpage{\restorepagecolor}

{\color{white}\verb|\newpagecolor{blue}\afterpage{\restorepagecolor}|%
\newline
was used here, i.\,e.~this page is blue, and the next one will
automatically have the same page colour before it was changed to blue
here (i.\,e. green).}

\smallskip
{\color{red}\textbf{\lipsum[1-11]}}
\bigskip

The page colour was changed back at the end of the page -
in mid-sentence!

\end{document}
%</example>
%    \end{macrocode}
%
% \StopEventually{}
%
% \newpage
%
% \section{The implementation}
%
% We start off by checking that we are loading into \LaTeXe{} and
% announcing the name and version of this package.
%
%    \begin{macrocode}
%<*package>
%    \end{macrocode}
%
%    \begin{macrocode}
\NeedsTeXFormat{LaTeX2e}[2011/06/27]
\ProvidesPackage{pagecolor}[2012/02/23 v1.0e
            Provides thepagecolor (HMM)]
%    \end{macrocode}
%
% A short description of the \xpackage{pagecolor} package:
%
%    \begin{macrocode}
%% Provides the \thepagecolor, \thepagecolornone, \newpagecolor{...},
%% and \restorepagecolor commands.
%    \end{macrocode}
%
% We need the \xpackage{kvoptions} package by \textsc{Heiko Oberdiek}:
%
%    \begin{macrocode}
\RequirePackage{kvoptions}[2010/12/23]% v3.10
%    \end{macrocode}
%
% and either the \xpackage{color} or the \xpackage{xcolor} package:
%
%    \begin{macrocode}
%% \RequirePackage{ either color or xcolor }:
\@ifpackageloaded{xcolor}{% xcolor loaded
   \@ifpackagelater{xcolor}{2007/01/21}{%
      % 2007/01/21, v2.11, or even more recent: OK
    }{% else: older package version
      \PackageWarning{pagecolor}{%
        It is requested version `2007/01/21' of package\MessageBreak%
        xcolor, but only an older version is available.\MessageBreak%
       }%
     }%
 }{% xcolor not loaded
   \@ifpackageloaded{color}{%
      \RequirePackage{color}[2005/11/14]% v1.0j
    }{% \else
      \PackageWarning{pagecolor}{%
        The pagecolor package must be loaded after either\MessageBreak%
        package color or after package xcolor (at your\MessageBreak%
        option). Neither package was loaded before package\MessageBreak%
        pagecolor. Loading of package xcolor will now be\MessageBreak%
        tried automatically.\MessageBreak%
        When the pagecolor package is used with option\MessageBreak%
        pagecolor using a colour requiring e. g. x11names\MessageBreak%
        option for xcolor package, this will not work!\MessageBreak%
       }
     }% \fi
   \RequirePackage{xcolor}[2007/01/21]% v2.11
  }% \fi
%    \end{macrocode}
%
% A last information for the user:
%
%    \begin{macrocode}
%% pagecolor may work with earlier versions of LaTeX and the
%% packages, but this was not tested. Please consider updating
%% your LaTeX and packages to the most recent version
%% (if they are not already the most recent version).

%    \end{macrocode}
%
% \indent See subsection~\ref{ss:Downloads} about how to get them.\\
%
% We process the options:
%
%    \begin{macrocode}
\SetupKeyvalOptions{family=pagecolor,prefix=pagecolor@}
\DeclareStringOption[none]{pagecolor}% \pagecolor@pagecolor
\ProcessKeyvalOptions*

%    \end{macrocode}
%
% We save the original |\pagecolor| command,
%
%    \begin{macrocode}
\let\origpagecolour\pagecolor

%    \end{macrocode}
%
%    \begin{macro}{\pagecolor}
% before we redefine it to include a definition of |\thepagecolor|
% and \linebreak
% |\thepagecolornone|:
%
%    \begin{macrocode}
\renewcommand{\pagecolor}[1]{\@bsphack%
  \edef\pagecolourtmpa{#1}%
  \def\pagecolourtmpb{none}%
  \ifx\pagecolourtmpa\pagecolourtmpb
    \@ifundefined{nopagecolor}{%
      \PackageError{pagecolor}{%
        pagecolor=none requested but \string\nopagecolor\space unknown%
       }{%
        \string\pagecolor{none} was used, but the command %
        \string\nopagecolor\space is undefined.\MessageBreak%
        Please use another colour. pagecolor=white will be used now.%
        \MessageBreak%
        }%
      \gdef\thepagecolor{white}%
      \gdef\thepagecolornone{white}% although it should be "none"
      \origpagecolour{white}%
     }{%
      \nopagecolor%
     }%
  \else%
    \xdef\thepagecolor{#1}%
    \xdef\thepagecolornone{#1}%
    \origpagecolour{\thepagecolornone}%
  \fi%
  \@esphack%
}

%    \end{macrocode}
%    \end{macro}
%
% |\nopagecolor| is only defined for \pdfTeX and \LuaTeX, but not yet for dvips,
% dvipdfm(x) or \XeTeX. Maybe
% \begin{verbatim}
% \@ifundefined{nopagecolor}{\newcommand{\nopagecolor}{\pagecolor{white}}}{}
% \end{verbatim}
% could be an alternative.\\
% When |\pagecolor{none}| is used and |\nopagecolor| is defined, then \\
% |\pagecolor{none}| is made into a synonym for |\nopagecolor|.\\
% If |\nopagecolor| has not been defined, nothing needs to be done.
% Otherwise we redefine |\nopagecolor|, thus that |\thepagecolor| is
% set to |white| and |\thepagecolornone| to |none| whenever
% |\nopagecolor| is used.
%
%    \begin{macrocode}
\@ifundefined{nopagecolor}{%
  \PackageWarning{pagecolor}{%
    \string\nopagecolor\space is undefined.\MessageBreak%
    (As long as it is neither used anywhere\MessageBreak%
    \space nor defined later, that will be no problem.)\MessageBreak%
   }
  \AtEndDocument{%
  \@ifundefined{nopagecolor}{% no problem
    }{%
    \PackageError{pagecolor}{%
      \string\nopagecolor\space defined after loading package%
      \MessageBreak%
      pagecolor%
      }{\string\nopagecolor\space was defined after loading the %
       pagecolor package.\MessageBreak%
       Either define \string\nopagecolor\space earlier or load the %
       pagecolor package later.\MessageBreak%
      }
    }%
  }%
  }{% \else
    \let\orignopagecolour\nopagecolor
    \renewcommand{\nopagecolor}{%
      \xdef\thepagecolor{white}
      \xdef\thepagecolornone{none}
      \orignopagecolour
      }
  }

%    \end{macrocode}
%
% If the page colour as given with option |pagecolor={...}| is |none|,
% but |\nopagecolor| is not known, |\pagecolor@pagecolor| is set to
% |white| and a warning is given.
%
%    \begin{macrocode}
\def\pagecolourtmpb{none}
\ifx\pagecolor@pagecolor\pagecolourtmpb
  \@ifundefined{nopagecolor}{%
    \PackageWarning{pagecolor}{%
      Option pagecolor=none (maybe by default) used,\MessageBreak%
      but \string\nopagecolor\space is unknown. Please use another%
      \MessageBreak%
      option value; white will be used now.\MessageBreak%
     }
    \setkeys{pagecolor}{pagecolor=white}%
  }{% ok
  }%
\fi

%    \end{macrocode}
%
% The (new) |\pagecolor| is now just carried out.
%
%    \begin{macrocode}
\pagecolor{\pagecolor@pagecolor}

%    \end{macrocode}
%
% Now the page (background) colour and |\thepagecolor| and \\
% |\thepagecolornone| are |\pagecolor@pagecolor| (or page (background) colour
% and |\thepagecolornone| are |none|=|\pagecolor@pagecolor| and
% |\thepagecolor| is |white|), and when the page (background) colour
% is changed, |\thepagecolor| and |\thepagecolornone| are changed
% accordingly.
%
% \newpage
%
%    \begin{macro}{\newpagecolor}
% There have been requests (via e-mail and at \\
% \url{http://tex.stackexchange.com/q/25137}) to change the colour of just one
% (or two) page(s) only, similar to |\newgeometry| and |\restoregeometry| of the
% \xpackage{geometry} package (\url{http://ctan.org/pkg/geometry}). Therefore
% |\newpagecolor| and |\restorepagecolor| are introduced (as suggested by
% \textsc{haoyun\_tex}):
%
%    \begin{macrocode}
\newcommand{\newpagecolor}[1]{%
\xdef\pagecolourtmpc{\thepagecolornone}%
\pagecolor{#1}%
}

%    \end{macrocode}
%    \end{macro}
%
% |\newpagecolor{<|\textit{some colour}|>}| will execute |\pagecolor{somecolour}|
% and remember the page colour used before.\\
%
%    \begin{macro}{\restorepagecolor}
%    \begin{macrocode}
\newcommand{\restorepagecolor}{\pagecolor{\pagecolourtmpc}}

%    \end{macrocode}
%    \end{macro}
%
% |\restorecolor| (without argument) restores the page colour to the one used before
% use of the |\newpagecolor{...}| command.
%
%    \begin{macrocode}
\gdef\pagecolourtmpc{\thepagecolor}

%    \end{macrocode}
% is just a precaution for |\restorecolor| being used when no |\newpagecolor{...}|
% was used before it.\\
%
% When you want to change the colour for just one page and do not want to
% (or cannot) manually determine where the page ends,\\
% |\newpagecolor{<|\textit{some colour}|>}\afterpage{\restorepagecolor}|\\
% does the trick (and requires an additional |\usepackage{afterpage}| in the
% document's preamble).
%    \begin{macrocode}
%</package>
%    \end{macrocode}
%
% \newpage
%
% \section{Installation}
%
% \subsection{Downloads\label{ss:Downloads}}
%
% Everything is available on \CTAN{}, \url{http://www.ctan.org/tex-archive/},
% but may need additional packages themselves.\\
%
% \DescribeMacro{pagecolor.dtx}
% For unpacking the |pagecolor.dtx| file and constructing the documentation
% it is required:
% \begin{description}
% \item[-] \TeX{} Format \LaTeXe{}, \url{http://www.CTAN.org/}
%
% \item[-] document class \xpackage{ltxdoc}, 2007/11/11, v2.0u,
%           \url{http://ctan.org/pkg/ltxdoc}
%
% \item[-] package \xpackage{holtxdoc}, 2011/02/04, v0.21,
%           \url{http://ctan.org/pkg/holtxdoc}
% \end{description}
%
% \DescribeMacro{pagecolor.sty}
% The |pagecolor.sty| for \LaTeXe{} (i.\,e. each document using
% the \xpackage{pagecolor} package) requires:
% \begin{description}
% \item[-] \TeX{} Format \LaTeXe{}, \url{http://www.CTAN.org/}
%
% \item[-] package \xpackage{kvoptions}, 2010/12/23, v3.10,
%           \url{http://ctan.org/pkg/kvoptions}
% \end{description}
% and either
% \begin{description}
% \item[-] package \xpackage{color}, 2005/11/14, v1.0j,
%           \url{http://ctan.org/pkg/color} (from the\\
%           \xpackage{graphics} package bundle)
% \end{description}
% or
% \begin{description}
% \item[-] package \xpackage{xcolor}, 2007/01/21, v2.11,
%           \url{http://ctan.org/pkg/xcolor}
% \end{description}
%
% \DescribeMacro{pagecolor-example.tex}
% The |pagecolor-example.tex| requires the same files as all
% documents using the \xpackage{pagecolor} package and additionally:
% \begin{description}
% \item[-] class \xpackage{article}, 2007/10/19, v1.4h, from \xpackage{classes.dtx}:\\
%           \CTAN{macros/latex/base/classes.dtx}
%
% \item[-] package \xpackage{pagecolor}, 2012/02/23, v1.0e,
%           \url{http://ctan.org/pkg/pagecolor}\\
%           (Well, it is the example file for this package, and because you are
%            reading the documentation for the \xpackage{pagecolor} package,
%            it can be assumed that you already have some version of it --
%            is it the current one?)
%
% \item[-] package \xpackage{xcolor}, 2007/01/21, v2.11,
%           \url{http://ctan.org/pkg/xcolor}\\
%           This package would not be needed for the use of just base colours only,
%           the \xpackage{color} package would be sufficient for that.
% \end{description}
%
% \DescribeMacro{transparent}
% \DescribeMacro{hrefhide}
% As possible alternatives in section \ref{sec:Alternatives} there are listed
% \begin{description}
% \item[-] package \xpackage{transparent}, 2007/01/08, v1.0, \\
%           \url{http://ctan.org/pkg/transparent}
%
% \item[-] package \xpackage{hrefhide}, 2011/04/29, v1.0f, \\
%           \url{http://ctan.org/pkg/hrefhide}
% \end{description}
%
% \DescribeMacro{Oberdiek}
% \DescribeMacro{holtxdoc}
% \DescribeMacro{kvoptions}
% All packages of \textsc{Heiko Oberdiek}'s bundle `oberdiek'
% (especially \xpackage{holtxdoc} and \xpackage{kvoptions})
% are also available in a TDS compliant ZIP archive:\\
% \CTAN{install/macros/latex/contrib/oberdiek.tds.zip}.\\
% It is probably best to download and use this, because the packages in there
% are quite probably both recent and compatible among themselves.\\
%
% \DescribeMacro{hyperref}
% \noindent \xpackage{hyperref} is not included in that bundle and needs to be
% downloaded separately,\\
% \url{http://mirror.ctan.org/install/macros/latex/contrib/hyperref.tds.zip}.\\
%
% \DescribeMacro{M\"{u}nch}
% A hyperlinked list of my (other) packages can be found at
% \url{http://www.Uni-Bonn.de/~uzs5pv/LaTeX.html}.\\
%
% \subsection{Package, unpacking TDS}
% \paragraph{Package.} This package is available on \CTAN{}
% \begin{description}
% \item[\CTAN{macros/latex/contrib/pagecolor/pagecolor.dtx}]\hspace*{0.1cm} \\
%       The source file.
% \item[\CTAN{macros/latex/contrib/pagecolor/pagecolor.pdf}]\hspace*{0.1cm} \\
%       The documentation.
% \item[\CTAN{macros/latex/contrib/pagecolor/pagecolor-example.pdf}]\hspace*{0.1cm} \\
%       The compiled example file, as it should look like.
% \item[\CTAN{macros/latex/contrib/pagecolor/README}]\hspace*{0.1cm} \\
%       The README file.
% \end{description}
% There is also a pagecolor.tds.zip available:
% \begin{description}
% \item[\CTAN{install/macros/latex/contrib/pagecolor.tds.zip}]\hspace*{0.1cm} \\
%       Everything in \xfile{TDS} compliant, compiled format.
% \end{description}
% which additionally contains\\
% \begin{tabular}{ll}
% pagecolor.ins & The installation file.\\
% pagecolor.drv & The driver to generate the documentation.\\
% pagecolor.sty & The \xext{sty}le file.\\
% pagecolor-example.tex & The example file.
% \end{tabular}
%
% \bigskip
%
% \noindent For required other packages, please see the preceding subsection.
%
% \paragraph{Unpacking.} The \xfile{.dtx} file is a self-extracting
% \docstrip{} archive. The files are extracted by running the
% \xext{.dtx} through \plainTeX{}:
% \begin{quote}
%   \verb|tex pagecolor.dtx|
% \end{quote}
%
% About generating the documentation see paragraph~\ref{GenDoc} below.\\
%
% \paragraph{TDS.} Now the different files must be moved into
% the different directories in your installation TDS tree
% (also known as \xfile{texmf} tree):
% \begin{quote}
% \def\t{^^A
% \begin{tabular}{@{}>{\ttfamily}l@{ $\rightarrow$ }>{\ttfamily}l@{}}
%   pagecolor.sty & tex/latex/pagecolor/pagecolor.sty\\
%   pagecolor.pdf & doc/latex/pagecolor/pagecolor.pdf\\
%   pagecolor-example.tex & doc/latex/pagecolor/pagecolor-example.tex\\
%   pagecolor-example.pdf & doc/latex/pagecolor/pagecolor-example.pdf\\
%   pagecolor.dtx & source/latex/pagecolor/pagecolor.dtx\\
% \end{tabular}^^A
% }^^A
% \sbox0{\t}^^A
% \ifdim\wd0>\linewidth
%   \begingroup
%     \advance\linewidth by\leftmargin
%     \advance\linewidth by\rightmargin
%   \edef\x{\endgroup
%     \def\noexpand\lw{\the\linewidth}^^A
%   }\x
%   \def\lwbox{^^A
%     \leavevmode
%     \hbox to \linewidth{^^A
%       \kern-\leftmargin\relax
%       \hss
%       \usebox0
%       \hss
%       \kern-\rightmargin\relax
%     }^^A
%   }^^A
%   \ifdim\wd0>\lw
%     \sbox0{\small\t}^^A
%     \ifdim\wd0>\linewidth
%       \ifdim\wd0>\lw
%         \sbox0{\footnotesize\t}^^A
%         \ifdim\wd0>\linewidth
%           \ifdim\wd0>\lw
%             \sbox0{\scriptsize\t}^^A
%             \ifdim\wd0>\linewidth
%               \ifdim\wd0>\lw
%                 \sbox0{\tiny\t}^^A
%                 \ifdim\wd0>\linewidth
%                   \lwbox
%                 \else
%                   \usebox0
%                 \fi
%               \else
%                 \lwbox
%               \fi
%             \else
%               \usebox0
%             \fi
%           \else
%             \lwbox
%           \fi
%         \else
%           \usebox0
%         \fi
%       \else
%         \lwbox
%       \fi
%     \else
%       \usebox0
%     \fi
%   \else
%     \lwbox
%   \fi
% \else
%   \usebox0
% \fi
% \end{quote}
% If you have a \xfile{docstrip.cfg} that configures and enables \docstrip{}'s
% \xfile{TDS} installing feature, then some files can already be in the right
% place, see the documentation of \docstrip{}.
%
% \subsection{Refresh file name databases}
%
% If your \TeX{}~distribution (\teTeX{}, \mikTeX{},\dots{}) relies on file name
% databases, you must refresh these. For example, \teTeX{} users run
% \verb|texhash| or \verb|mktexlsr|.
%
% \subsection{Some details for the interested}
%
% \paragraph{Unpacking with \LaTeX{}.}
% The \xfile{.dtx} chooses its action depending on the format:
% \begin{description}
% \item[\plainTeX:] Run \docstrip{} and extract the files.
% \item[\LaTeX:] Generate the documentation.
% \end{description}
% If you insist on using \LaTeX{} for \docstrip{} (really,
% \docstrip{} does not need \LaTeX{}), then inform the autodetect routine
% about your intention:
% \begin{quote}
%   \verb|latex \let\install=y% \iffalse meta-comment
%
% File: pagecolor.dtx
% Version: 2012/02/23 v1.0e
%
% Copyright (C) 2011 - 2012 by
%    H.-Martin M"unch <Martin dot Muench at Uni-Bonn dot de>
%
% This work may be distributed and/or modified under the
% conditions of the LaTeX Project Public License, either
% version 1.3c of this license or (at your option) any later
% version. This version of this license is in
%    http://www.latex-project.org/lppl/lppl-1-3c.txt
% and the latest version of this license is in
%    http://www.latex-project.org/lppl.txt
% and version 1.3c or later is part of all distributions of
% LaTeX version 2005/12/01 or later.
%
% This work has the LPPL maintenance status "maintained".
%
% The Current Maintainer of this work is H.-Martin Muench.
%
% This work consists of the main source file pagecolor.dtx,
% the README, and the derived files
%    pagecolor.sty, pagecolor.pdf,
%    pagecolor.ins, pagecolor.drv,
%    pagecolor-example.tex, pagecolor-example.pdf.
%
% Distribution:
%    CTAN:macros/latex/contrib/pagecolor/pagecolor.dtx
%    CTAN:macros/latex/contrib/pagecolor/pagecolor.pdf
%    CTAN:install/macros/latex/contrib/pagecolor.tds.zip
%
% Cf. http://ctan.org/pkg/pagecolor and
% http://www.tex.ac.uk/tex-archive/help/Catalogue/entries/pagecolor.html
%
% Unpacking:
%    (a) If pagecolor.ins is present:
%           tex pagecolor.ins
%    (b) Without pagecolor.ins:
%           tex pagecolor.dtx
%    (c) If you insist on using LaTeX
%           latex \let\install=y% \iffalse meta-comment
%
% File: pagecolor.dtx
% Version: 2012/02/23 v1.0e
%
% Copyright (C) 2011 - 2012 by
%    H.-Martin M"unch <Martin dot Muench at Uni-Bonn dot de>
%
% This work may be distributed and/or modified under the
% conditions of the LaTeX Project Public License, either
% version 1.3c of this license or (at your option) any later
% version. This version of this license is in
%    http://www.latex-project.org/lppl/lppl-1-3c.txt
% and the latest version of this license is in
%    http://www.latex-project.org/lppl.txt
% and version 1.3c or later is part of all distributions of
% LaTeX version 2005/12/01 or later.
%
% This work has the LPPL maintenance status "maintained".
%
% The Current Maintainer of this work is H.-Martin Muench.
%
% This work consists of the main source file pagecolor.dtx,
% the README, and the derived files
%    pagecolor.sty, pagecolor.pdf,
%    pagecolor.ins, pagecolor.drv,
%    pagecolor-example.tex, pagecolor-example.pdf.
%
% Distribution:
%    CTAN:macros/latex/contrib/pagecolor/pagecolor.dtx
%    CTAN:macros/latex/contrib/pagecolor/pagecolor.pdf
%    CTAN:install/macros/latex/contrib/pagecolor.tds.zip
%
% Cf. http://ctan.org/pkg/pagecolor and
% http://www.tex.ac.uk/tex-archive/help/Catalogue/entries/pagecolor.html
%
% Unpacking:
%    (a) If pagecolor.ins is present:
%           tex pagecolor.ins
%    (b) Without pagecolor.ins:
%           tex pagecolor.dtx
%    (c) If you insist on using LaTeX
%           latex \let\install=y% \iffalse meta-comment
%
% File: pagecolor.dtx
% Version: 2012/02/23 v1.0e
%
% Copyright (C) 2011 - 2012 by
%    H.-Martin M"unch <Martin dot Muench at Uni-Bonn dot de>
%
% This work may be distributed and/or modified under the
% conditions of the LaTeX Project Public License, either
% version 1.3c of this license or (at your option) any later
% version. This version of this license is in
%    http://www.latex-project.org/lppl/lppl-1-3c.txt
% and the latest version of this license is in
%    http://www.latex-project.org/lppl.txt
% and version 1.3c or later is part of all distributions of
% LaTeX version 2005/12/01 or later.
%
% This work has the LPPL maintenance status "maintained".
%
% The Current Maintainer of this work is H.-Martin Muench.
%
% This work consists of the main source file pagecolor.dtx,
% the README, and the derived files
%    pagecolor.sty, pagecolor.pdf,
%    pagecolor.ins, pagecolor.drv,
%    pagecolor-example.tex, pagecolor-example.pdf.
%
% Distribution:
%    CTAN:macros/latex/contrib/pagecolor/pagecolor.dtx
%    CTAN:macros/latex/contrib/pagecolor/pagecolor.pdf
%    CTAN:install/macros/latex/contrib/pagecolor.tds.zip
%
% Cf. http://ctan.org/pkg/pagecolor and
% http://www.tex.ac.uk/tex-archive/help/Catalogue/entries/pagecolor.html
%
% Unpacking:
%    (a) If pagecolor.ins is present:
%           tex pagecolor.ins
%    (b) Without pagecolor.ins:
%           tex pagecolor.dtx
%    (c) If you insist on using LaTeX
%           latex \let\install=y\input{pagecolor.dtx}
%        (quote the arguments according to the demands of your shell)
%
% Documentation:
%    (a) If pagecolor.drv is present:
%           (pdf)latex pagecolor.drv
%           makeindex -s gind.ist pagecolor.idx
%           (pdf)latex pagecolor.drv
%           makeindex -s gind.ist pagecolor.idx
%           (pdf)latex pagecolor.drv
%    (b) Without pagecolor.drv:
%           (pdf)latex pagecolor.dtx
%           makeindex -s gind.ist pagecolor.idx
%           (pdf)latex pagecolor.dtx
%           makeindex -s gind.ist pagecolor.idx
%           (pdf)latex pagecolor.dtx
%
%    The class ltxdoc loads the configuration file ltxdoc.cfg
%    if available. Here you can specify further options, e.g.
%    use DIN A4 as paper format:
%       \PassOptionsToClass{a4paper}{article}
%
% Installation:
%    TDS:tex/latex/pagecolor/pagecolor.sty
%    TDS:doc/latex/pagecolor/pagecolor.pdf
%    TDS:doc/latex/pagecolor/pagecolor-example.tex
%    TDS:doc/latex/pagecolor/pagecolor-example.pdf
%    TDS:source/latex/pagecolor/pagecolor.dtx
%
%<*ignore>
\begingroup
  \catcode123=1 %
  \catcode125=2 %
  \def\x{LaTeX2e}%
\expandafter\endgroup
\ifcase 0\ifx\install y1\fi\expandafter
         \ifx\csname processbatchFile\endcsname\relax\else1\fi
         \ifx\fmtname\x\else 1\fi\relax
\else\csname fi\endcsname
%</ignore>
%<*install>
\input docstrip.tex
\Msg{*****************************************************************}
\Msg{* Installation                                                   }
\Msg{* Package: pagecolor 2012/02/23 v1.0e Provides thepagecolor (HMM)}
\Msg{*****************************************************************}

\keepsilent
\askforoverwritefalse

\let\MetaPrefix\relax
\preamble

This is a generated file.

Project: pagecolor
Version: 2012/02/23 v1.0e

Copyright (C) 2011 - 2012 by
    H.-Martin M"unch <Martin dot Muench at Uni-Bonn dot de>

The usual disclaimer applies:
If it doesn't work right that's your problem.
(Nevertheless, send an e-mail to the maintainer
 when you find an error in this package.)

This work may be distributed and/or modified under the
conditions of the LaTeX Project Public License, either
version 1.3c of this license or (at your option) any later
version. This version of this license is in
   http://www.latex-project.org/lppl/lppl-1-3c.txt
and the latest version of this license is in
   http://www.latex-project.org/lppl.txt
and version 1.3c or later is part of all distributions of
LaTeX version 2005/12/01 or later.

This work has the LPPL maintenance status "maintained".

The Current Maintainer of this work is H.-Martin Muench.

This work consists of the main source file pagecolor.dtx,
the README, and the derived files
   pagecolor.sty, pagecolor.pdf,
   pagecolor.ins, pagecolor.drv,
   pagecolor-example.tex, pagecolor-example.pdf.

\endpreamble
\let\MetaPrefix\DoubleperCent

\generate{%
  \file{pagecolor.ins}{\from{pagecolor.dtx}{install}}%
  \file{pagecolor.drv}{\from{pagecolor.dtx}{driver}}%
  \usedir{tex/latex/pagecolor}%
  \file{pagecolor.sty}{\from{pagecolor.dtx}{package}}%
  \usedir{doc/latex/pagecolor}%
  \file{pagecolor-example.tex}{\from{pagecolor.dtx}{example}}%
}

\catcode32=13\relax% active space
\let =\space%
\Msg{************************************************************************}
\Msg{*}
\Msg{* To finish the installation you have to move the following}
\Msg{* file into a directory searched by TeX:}
\Msg{*}
\Msg{*     pagecolor.sty}
\Msg{*}
\Msg{* To produce the documentation run the file `pagecolor.drv'}
\Msg{* through (pdf)LaTeX, e.g.}
\Msg{*  pdflatex pagecolor.drv}
\Msg{*  makeindex -s gind.ist pagecolor.idx}
\Msg{*  pdflatex pagecolor.drv}
\Msg{*  makeindex -s gind.ist pagecolor.idx}
\Msg{*  pdflatex pagecolor.drv}
\Msg{*}
\Msg{* At least three runs are necessary e.g. to get the}
\Msg{*  references right!}
\Msg{*}
\Msg{* Happy TeXing!}
\Msg{*}
\Msg{************************************************************************}

\endbatchfile
%</install>
%<*ignore>
\fi
%</ignore>
%
% \section{The documentation driver file}
%
% The next bit of code contains the documentation driver file for
% \TeX{}, i.\,e., the file that will produce the documentation you
% are currently reading. It will be extracted from this file by the
% \texttt{docstrip} programme. That is, run \LaTeX{} on \texttt{docstrip}
% and specify the \texttt{driver} option when \texttt{docstrip}
% asks for options.
%
%    \begin{macrocode}
%<*driver>
\NeedsTeXFormat{LaTeX2e}[2011/06/27]
\ProvidesFile{pagecolor.drv}%
  [2012/02/23 v1.0e Provides thepagecolor (HMM)]
\documentclass{ltxdoc}[2007/11/11]% v2.0u
\usepackage{holtxdoc}[2011/02/04]%  v0.21
%% pagecolor may work with earlier versions of LaTeX2e and those
%% class and package, but this was not tested.
%% Please consider updating your LaTeX, class, and package
%% to the most recent version (if they are not already the most
%% recent version).
\hypersetup{%
 pdfsubject={Providing the thepagecolor command (HMM)},%
 pdfkeywords={LaTeX, pagecolor, thepagecolor, H.-Martin Muench},%
 pdfencoding=auto,%
 pdflang={en},%
 breaklinks=true,%
 linktoc=all,%
 pdfstartview=FitH,%
 pdfpagelayout=OneColumn,%
 bookmarksnumbered=true,%
 bookmarksopen=true,%
 bookmarksopenlevel=3,%
 pdfmenubar=true,%
 pdftoolbar=true,%
 pdfwindowui=true,%
 pdfnewwindow=true%
}
\CodelineIndex
\hyphenation{printing docu-ment}
\gdef\unit#1{\mathord{\thinspace\mathrm{#1}}}%
\begin{document}
  \DocInput{pagecolor.dtx}%
\end{document}
%</driver>
%    \end{macrocode}
%
% \fi
%
% \CheckSum{137}
%
% \CharacterTable
%  {Upper-case    \A\B\C\D\E\F\G\H\I\J\K\L\M\N\O\P\Q\R\S\T\U\V\W\X\Y\Z
%   Lower-case    \a\b\c\d\e\f\g\h\i\j\k\l\m\n\o\p\q\r\s\t\u\v\w\x\y\z
%   Digits        \0\1\2\3\4\5\6\7\8\9
%   Exclamation   \!     Double quote  \"     Hash (number) \#
%   Dollar        \$     Percent       \%     Ampersand     \&
%   Acute accent  \'     Left paren    \(     Right paren   \)
%   Asterisk      \*     Plus          \+     Comma         \,
%   Minus         \-     Point         \.     Solidus       \/
%   Colon         \:     Semicolon     \;     Less than     \<
%   Equals        \=     Greater than  \>     Question mark \?
%   Commercial at \@     Left bracket  \[     Backslash     \\
%   Right bracket \]     Circumflex    \^     Underscore    \_
%   Grave accent  \`     Left brace    \{     Vertical bar  \|
%   Right brace   \}     Tilde         \~}
%
% \GetFileInfo{pagecolor.drv}
%
% \begingroup
%   \def\x{\#,\$,\^,\_,\~,\ ,\&,\{,\},\%}%
%   \makeatletter
%   \@onelevel@sanitize\x
% \expandafter\endgroup
% \expandafter\DoNotIndex\expandafter{\x}
% \expandafter\DoNotIndex\expandafter{\string\ }
% \begingroup
%   \makeatletter
%     \lccode`9=32\relax
%     \lowercase{%^^A
%       \edef\x{\noexpand\DoNotIndex{\@backslashchar9}}%^^A
%     }%^^A
%   \expandafter\endgroup\x
%
% \DoNotIndex{\,,\\}
% \DoNotIndex{\documentclass,\usepackage,\ProvidesPackage,\begin,\end}
% \DoNotIndex{\MessageBreak}
% \DoNotIndex{\NeedsTeXFormat,\DoNotIndex,\verb}
% \DoNotIndex{\def,\edef,\gdef,\global}
% \DoNotIndex{\ifx,\listfiles,\mathord,\mathrm}
% \DoNotIndex{\kvoptions,\SetupKeyvalOptions,\ProcessKeyvalOptions}
% \DoNotIndex{\bigskip,\space,\thinspace,\ldots}
% \DoNotIndex{\indent,\noindent,\newline,\linebreak,\pagebreak,\newpage}
% \DoNotIndex{\textbf,\textit,\textsf,\texttt,\textsc,\textquotedblleft,\textquotedblright}
% \DoNotIndex{\plainTeX,\TeX,\LaTeX,\pdfLaTeX}
% \DoNotIndex{\section,\color,\pagenumbering,\smallskip}
% \DoNotIndex{\@ifundefined,\@ifpackageloaded}
% \DoNotIndex{\DeclareStringOption}
%
% \title{The \xpackage{pagecolor} package}
% \date{2012/02/23 v1.0e}
% \author{H.-Martin M\"{u}nch\\\xemail{Martin.Muench at Uni-Bonn.de}}
%
% \maketitle
%
% \begin{abstract}
%  This \LaTeX{} package provides the command |\thepagecolor|,
% which gives the current page (background) colour,
% i.\,e.~the argument used with the most recent call of |\pagecolor{...}|.
% The command |\thepagecolornone| gives the same colour as |\thepagecolor|,
% except when the page background colour is
% \textquotedblleft|none|\textquotedblright. In that case |\thepagecolor| is
% |white| and |\thepagecolornone| is |none|.\\
% Similar to |\newgeometry| and |\restoregeometry| of the \xpackage{geometry}
% package\\
% |\newpagecolor{<|\textit{some colour} |>}| and |\restorepagecolor|
% are provided.
% \end{abstract}
%
% \bigskip
%
% \noindent Disclaimer for web links: The author is not responsible for any contents
% referred to in this work unless he has full knowledge of illegal contents.
% If any damage occurs by the use of information presented there, only the
% author of the respective pages might be liable, not the one who has referred
% to these pages.
%
% \bigskip
%
% \noindent {\color{green} Save per page about $200\unit{ml}$ water,
% $2\unit{g}$ CO$_{2}$ and $2\unit{g}$ wood:\\
% Therefore please print only if this is really necessary.}
%
% \pagebreak
%
% \tableofcontents
%
% \bigskip
%
% \section{Introduction\label{sec:Introduction}}
%
% \indent This \LaTeX{} package provides the command |\thepagecolor|,
% which gives the current page (background) colour,
% i.\,e.~the argument used with the most recent call of |\pagecolor{...}|.
% The package should be loaded before any package sets a page (background)
% colour, but after \xpackage{color} or \xpackage{xcolor} package.
% Its option |pagecolor={...}| is used to set the |\pagecolor{...}|.\\
% The command |\thepagecolornone| gives the same colour as |\thepagecolor|,
% except when the page background colour is
% \textquotedblleft|none|\textquotedblright (e.\,g. result of using the
% |\nopagecolor| command). In that case |\thepagecolor| is |white| and\linebreak
% |\thepagecolornone| is |none|.\\
% Similar to |\newgeometry| and |\restoregeometry| of the \xpackage{geometry}
% package\\
% |\newpagecolor{<|\textit{some colour}|>}| and |\restorepagecolor|
% are provided.
%
% \pagebreak
%
% \section{Usage}
%
% \indent Just load the package placing
% \begin{quote}
%   |\usepackage[<|\textit{option}|>]{pagecolor}|
% \end{quote}
% \noindent in the preamble of your \LaTeXe{} source file.
% This should be done before another package uses |\pagecolor|,
% but after |\nopagecolor| is defined (if that is defined in the document
% at all).\\
% Afterwards |\pagecolor{...}| can be used to change the page (background)
% colour as usual. Then |\thepagecolor| gives the current page (background)
% colour (in the same format as given with |\pagecolor{...}|).\\
% Similar to |\newgeometry| and |\restoregeometry| of the \xpackage{geometry}
% package\\
% |\newpagecolor{<|\textit{some colour} |>}| and |\restorepagecolor|
% are provided:\\
% |\newpagecolor{<|\textit{some colour} |>}| will execute
% |\pagecolor{<|\textit{some colour} |>}| and remember the page colour used before.
% |\restorecolor| (without argument) restores the page colour to the one used before
%  use of the |\newpagecolor{...}| command. When you want to change the colour for
%  just one page and do not want to (or cannot) manually determine where the page ends,\\
% |\newpagecolor{<|\textit{some colour}|>}\afterpage{\restorepagecolor}|\\
% does the trick (and requires a |\usepackage{afterpage}| in the document's
% preamble), or for short\\
% |\newcommand{\onepagecolor}[1]{%|\\
% |\newpagecolor{#1}\afterpage{\restorepagecolor}}|\\
% in the preamble and\\
% |\onepagecolor{<|\textit{some colour}|>}| in the document.
%
% \subsection{Option}
% \DescribeMacro{option}
% \indent The \xpackage{pagecolor} package takes the following option:
%
% \subsubsection{pagecolor\label{sss:pagecolor}}
% \DescribeMacro{pagecolor}
% \indent The option |pagecolor={...}| takes as value a colour. This
% could be as simple as |white| or |black|, but when e.\,g.~the
% \xpackage{xcolor} package is used (loaded before \xpackage{pagecolor}!),
% also colours like |red!50!green!20!blue| are possible.
% The default is |pagecolor={none}|.
%
% \section{Alternatives\label{sec:Alternatives}}
%
% As I neither know what exactly you want to accomplish
% when using this package (e.\,g.~hiding text), nor what resources you have
% (e.\,g.~\pdfTeX{} version), here is a list of some possible alternatives:\\
%
% \begin{description}
%  \item[-]
%  \DescribeMacro{transparent}
%  \xpackage{transparent} package: With it some object can be made
%  (fully or partially) transparent.
% \end{description}
%
% \begin{description}
%  \item[-]
%  \DescribeMacro{hrefhide}\xpackage{hrefhide} package: It allows to
%  \textquotedblleft hide\textquotedblright\ some (hyperlinked) text when
%  printing the document while keeping the layout.
% \end{description}
%
% \bigskip
%
% \noindent (You programmed or found another alternative,
%  which is available at \CTAN{}?\\
%  OK, send an e-mail to me with the name, location at \CTAN{},
%  and a short notice, and I will probably include it in
%  the list above.)\\
%
% \smallskip
%
% \noindent About how to get those packages, please see
% subsection~\ref{ss:Downloads}.
%
% \newpage
%
% \section{Example}
%
%    \begin{macrocode}
%<*example>
\documentclass[british]{article}[2007/10/19]% v1.4h
%%%%%%%%%%%%%%%%%%%%%%%%%%%%%%%%%%%%%%%%%%%%%%%%%%%%%%%%%%%%%%%%%%%%%
\usepackage[%
 extension=pdf,%
 plainpages=false,%
 pdfpagelabels=true,%
 hyperindex=false,%
 pdflang={en},%
 pdftitle={pagecolor package example},%
 pdfauthor={H.-Martin Muench},%
 pdfsubject={Example for the pagecolor package},%
 pdfkeywords={LaTeX, pagecolor, thepagecolor, page colour,%
  H.-Martin Muench},%
 pdfview=Fit,pdfstartview=Fit,%
 pdfpagelayout=SinglePage%
]{hyperref}[2012/02/06]% v6.82o
\usepackage[x11names]{xcolor}[2007/01/21]% v2.11
 % The xcolor package would not be needed for just using
 % the base colours. The color package would be sufficient for that.
\usepackage[pagecolor={LightGoldenrod1}]{pagecolor}[2012/02/23]% v1.0e

\usepackage{afterpage}[1995/10/27]% v1.08
 % The afterpage package is generally not needed,
 % but the |\newpagecolor{somecolour}\afterpage{\restorepagecolor}|
 % construct shall be demonstrated.

\usepackage{lipsum}[2011/04/14]% v1.2
 % The lipsum package is generally not needed,
 % but some blind text is needed for the example.

\gdef\unit#1{\mathord{\thinspace\mathrm{#1}}}%
\listfiles
\begin{document}
\pagenumbering{arabic}
\section*{Example for pagecolor}

This example demonstrates the use of package\newline
\textsf{pagecolor}, v1.0e as of 2012/02/23 (HMM).\newline
The used option was \verb|pagecolor={LightGoldenrod1}|.\newline
\verb|pagecolor={none}| would be the default.\newline

For more details please see the documentation!\newline

\noindent {\color{teal} Save per page about $200\unit{ml}$ water,
$2\unit{g}$ CO$_{2}$ and $2\unit{g}$ wood:\newline
Therefore please print only if this is really necessary.}\newline

The current page (background) colour is\newline
\verb|\thepagecolor|\ =\ \thepagecolor \newline
(and \verb|\thepagecolornone|\ =\ \thepagecolornone ,
which would only be different from \verb|\thepagecolor|,
when the page colour would be \verb|none|).

\pagebreak
\pagecolor{rgb:-green!40!yellow,3;green!40!yellow,2;red,1}

{\color{white} The current page (background) colour is\newline
\verb|\thepagecolor|\ =\ \thepagecolor . \newline}

{\color{\thepagecolor} And that makes this text practically invisible.
\newline}

{\color{white} Which made the preceding line of text practically
invisible.}

\pagebreak
\newpagecolor{red}

This page uses \verb|\newpagecolor{red}|.

\pagebreak
\restorepagecolor

{\color{white}And this page uses \verb|\restorepagecolor| to restore
the page colour to the value it had before the red page.}

\pagebreak
\pagecolor{none}

This page uses \verb|\pagecolor{none}|. If the \verb|\nopagecolor|
command is known (pdf\TeX and Lua\TeX; not yet for dvips, dvipdfm(x)
or Xe\TeX), the page colour is now \verb|none|, otherwise \verb|white|:
\verb|\thepagecolor|\ =\ \thepagecolor\ and
\verb|\thepagecolornone|\ =\ \thepagecolornone .

\pagebreak
\restorepagecolor

{\color{white}\verb|\restorepagecolor| restored the page colour again.}

\pagebreak
\pagecolor{green}

This page is green due to \verb|\pagecolor{green}|.

\pagebreak
\newpagecolor{blue}\afterpage{\restorepagecolor}

{\color{white}\verb|\newpagecolor{blue}\afterpage{\restorepagecolor}|%
\newline
was used here, i.\,e.~this page is blue, and the next one will
automatically have the same page colour before it was changed to blue
here (i.\,e. green).}

\smallskip
{\color{red}\textbf{\lipsum[1-11]}}
\bigskip

The page colour was changed back at the end of the page -
in mid-sentence!

\end{document}
%</example>
%    \end{macrocode}
%
% \StopEventually{}
%
% \newpage
%
% \section{The implementation}
%
% We start off by checking that we are loading into \LaTeXe{} and
% announcing the name and version of this package.
%
%    \begin{macrocode}
%<*package>
%    \end{macrocode}
%
%    \begin{macrocode}
\NeedsTeXFormat{LaTeX2e}[2011/06/27]
\ProvidesPackage{pagecolor}[2012/02/23 v1.0e
            Provides thepagecolor (HMM)]
%    \end{macrocode}
%
% A short description of the \xpackage{pagecolor} package:
%
%    \begin{macrocode}
%% Provides the \thepagecolor, \thepagecolornone, \newpagecolor{...},
%% and \restorepagecolor commands.
%    \end{macrocode}
%
% We need the \xpackage{kvoptions} package by \textsc{Heiko Oberdiek}:
%
%    \begin{macrocode}
\RequirePackage{kvoptions}[2010/12/23]% v3.10
%    \end{macrocode}
%
% and either the \xpackage{color} or the \xpackage{xcolor} package:
%
%    \begin{macrocode}
%% \RequirePackage{ either color or xcolor }:
\@ifpackageloaded{xcolor}{% xcolor loaded
   \@ifpackagelater{xcolor}{2007/01/21}{%
      % 2007/01/21, v2.11, or even more recent: OK
    }{% else: older package version
      \PackageWarning{pagecolor}{%
        It is requested version `2007/01/21' of package\MessageBreak%
        xcolor, but only an older version is available.\MessageBreak%
       }%
     }%
 }{% xcolor not loaded
   \@ifpackageloaded{color}{%
      \RequirePackage{color}[2005/11/14]% v1.0j
    }{% \else
      \PackageWarning{pagecolor}{%
        The pagecolor package must be loaded after either\MessageBreak%
        package color or after package xcolor (at your\MessageBreak%
        option). Neither package was loaded before package\MessageBreak%
        pagecolor. Loading of package xcolor will now be\MessageBreak%
        tried automatically.\MessageBreak%
        When the pagecolor package is used with option\MessageBreak%
        pagecolor using a colour requiring e. g. x11names\MessageBreak%
        option for xcolor package, this will not work!\MessageBreak%
       }
     }% \fi
   \RequirePackage{xcolor}[2007/01/21]% v2.11
  }% \fi
%    \end{macrocode}
%
% A last information for the user:
%
%    \begin{macrocode}
%% pagecolor may work with earlier versions of LaTeX and the
%% packages, but this was not tested. Please consider updating
%% your LaTeX and packages to the most recent version
%% (if they are not already the most recent version).

%    \end{macrocode}
%
% \indent See subsection~\ref{ss:Downloads} about how to get them.\\
%
% We process the options:
%
%    \begin{macrocode}
\SetupKeyvalOptions{family=pagecolor,prefix=pagecolor@}
\DeclareStringOption[none]{pagecolor}% \pagecolor@pagecolor
\ProcessKeyvalOptions*

%    \end{macrocode}
%
% We save the original |\pagecolor| command,
%
%    \begin{macrocode}
\let\origpagecolour\pagecolor

%    \end{macrocode}
%
%    \begin{macro}{\pagecolor}
% before we redefine it to include a definition of |\thepagecolor|
% and \linebreak
% |\thepagecolornone|:
%
%    \begin{macrocode}
\renewcommand{\pagecolor}[1]{\@bsphack%
  \edef\pagecolourtmpa{#1}%
  \def\pagecolourtmpb{none}%
  \ifx\pagecolourtmpa\pagecolourtmpb
    \@ifundefined{nopagecolor}{%
      \PackageError{pagecolor}{%
        pagecolor=none requested but \string\nopagecolor\space unknown%
       }{%
        \string\pagecolor{none} was used, but the command %
        \string\nopagecolor\space is undefined.\MessageBreak%
        Please use another colour. pagecolor=white will be used now.%
        \MessageBreak%
        }%
      \gdef\thepagecolor{white}%
      \gdef\thepagecolornone{white}% although it should be "none"
      \origpagecolour{white}%
     }{%
      \nopagecolor%
     }%
  \else%
    \xdef\thepagecolor{#1}%
    \xdef\thepagecolornone{#1}%
    \origpagecolour{\thepagecolornone}%
  \fi%
  \@esphack%
}

%    \end{macrocode}
%    \end{macro}
%
% |\nopagecolor| is only defined for \pdfTeX and \LuaTeX, but not yet for dvips,
% dvipdfm(x) or \XeTeX. Maybe
% \begin{verbatim}
% \@ifundefined{nopagecolor}{\newcommand{\nopagecolor}{\pagecolor{white}}}{}
% \end{verbatim}
% could be an alternative.\\
% When |\pagecolor{none}| is used and |\nopagecolor| is defined, then \\
% |\pagecolor{none}| is made into a synonym for |\nopagecolor|.\\
% If |\nopagecolor| has not been defined, nothing needs to be done.
% Otherwise we redefine |\nopagecolor|, thus that |\thepagecolor| is
% set to |white| and |\thepagecolornone| to |none| whenever
% |\nopagecolor| is used.
%
%    \begin{macrocode}
\@ifundefined{nopagecolor}{%
  \PackageWarning{pagecolor}{%
    \string\nopagecolor\space is undefined.\MessageBreak%
    (As long as it is neither used anywhere\MessageBreak%
    \space nor defined later, that will be no problem.)\MessageBreak%
   }
  \AtEndDocument{%
  \@ifundefined{nopagecolor}{% no problem
    }{%
    \PackageError{pagecolor}{%
      \string\nopagecolor\space defined after loading package%
      \MessageBreak%
      pagecolor%
      }{\string\nopagecolor\space was defined after loading the %
       pagecolor package.\MessageBreak%
       Either define \string\nopagecolor\space earlier or load the %
       pagecolor package later.\MessageBreak%
      }
    }%
  }%
  }{% \else
    \let\orignopagecolour\nopagecolor
    \renewcommand{\nopagecolor}{%
      \xdef\thepagecolor{white}
      \xdef\thepagecolornone{none}
      \orignopagecolour
      }
  }

%    \end{macrocode}
%
% If the page colour as given with option |pagecolor={...}| is |none|,
% but |\nopagecolor| is not known, |\pagecolor@pagecolor| is set to
% |white| and a warning is given.
%
%    \begin{macrocode}
\def\pagecolourtmpb{none}
\ifx\pagecolor@pagecolor\pagecolourtmpb
  \@ifundefined{nopagecolor}{%
    \PackageWarning{pagecolor}{%
      Option pagecolor=none (maybe by default) used,\MessageBreak%
      but \string\nopagecolor\space is unknown. Please use another%
      \MessageBreak%
      option value; white will be used now.\MessageBreak%
     }
    \setkeys{pagecolor}{pagecolor=white}%
  }{% ok
  }%
\fi

%    \end{macrocode}
%
% The (new) |\pagecolor| is now just carried out.
%
%    \begin{macrocode}
\pagecolor{\pagecolor@pagecolor}

%    \end{macrocode}
%
% Now the page (background) colour and |\thepagecolor| and \\
% |\thepagecolornone| are |\pagecolor@pagecolor| (or page (background) colour
% and |\thepagecolornone| are |none|=|\pagecolor@pagecolor| and
% |\thepagecolor| is |white|), and when the page (background) colour
% is changed, |\thepagecolor| and |\thepagecolornone| are changed
% accordingly.
%
% \newpage
%
%    \begin{macro}{\newpagecolor}
% There have been requests (via e-mail and at \\
% \url{http://tex.stackexchange.com/q/25137}) to change the colour of just one
% (or two) page(s) only, similar to |\newgeometry| and |\restoregeometry| of the
% \xpackage{geometry} package (\url{http://ctan.org/pkg/geometry}). Therefore
% |\newpagecolor| and |\restorepagecolor| are introduced (as suggested by
% \textsc{haoyun\_tex}):
%
%    \begin{macrocode}
\newcommand{\newpagecolor}[1]{%
\xdef\pagecolourtmpc{\thepagecolornone}%
\pagecolor{#1}%
}

%    \end{macrocode}
%    \end{macro}
%
% |\newpagecolor{<|\textit{some colour}|>}| will execute |\pagecolor{somecolour}|
% and remember the page colour used before.\\
%
%    \begin{macro}{\restorepagecolor}
%    \begin{macrocode}
\newcommand{\restorepagecolor}{\pagecolor{\pagecolourtmpc}}

%    \end{macrocode}
%    \end{macro}
%
% |\restorecolor| (without argument) restores the page colour to the one used before
% use of the |\newpagecolor{...}| command.
%
%    \begin{macrocode}
\gdef\pagecolourtmpc{\thepagecolor}

%    \end{macrocode}
% is just a precaution for |\restorecolor| being used when no |\newpagecolor{...}|
% was used before it.\\
%
% When you want to change the colour for just one page and do not want to
% (or cannot) manually determine where the page ends,\\
% |\newpagecolor{<|\textit{some colour}|>}\afterpage{\restorepagecolor}|\\
% does the trick (and requires an additional |\usepackage{afterpage}| in the
% document's preamble).
%    \begin{macrocode}
%</package>
%    \end{macrocode}
%
% \newpage
%
% \section{Installation}
%
% \subsection{Downloads\label{ss:Downloads}}
%
% Everything is available on \CTAN{}, \url{http://www.ctan.org/tex-archive/},
% but may need additional packages themselves.\\
%
% \DescribeMacro{pagecolor.dtx}
% For unpacking the |pagecolor.dtx| file and constructing the documentation
% it is required:
% \begin{description}
% \item[-] \TeX{} Format \LaTeXe{}, \url{http://www.CTAN.org/}
%
% \item[-] document class \xpackage{ltxdoc}, 2007/11/11, v2.0u,
%           \url{http://ctan.org/pkg/ltxdoc}
%
% \item[-] package \xpackage{holtxdoc}, 2011/02/04, v0.21,
%           \url{http://ctan.org/pkg/holtxdoc}
% \end{description}
%
% \DescribeMacro{pagecolor.sty}
% The |pagecolor.sty| for \LaTeXe{} (i.\,e. each document using
% the \xpackage{pagecolor} package) requires:
% \begin{description}
% \item[-] \TeX{} Format \LaTeXe{}, \url{http://www.CTAN.org/}
%
% \item[-] package \xpackage{kvoptions}, 2010/12/23, v3.10,
%           \url{http://ctan.org/pkg/kvoptions}
% \end{description}
% and either
% \begin{description}
% \item[-] package \xpackage{color}, 2005/11/14, v1.0j,
%           \url{http://ctan.org/pkg/color} (from the\\
%           \xpackage{graphics} package bundle)
% \end{description}
% or
% \begin{description}
% \item[-] package \xpackage{xcolor}, 2007/01/21, v2.11,
%           \url{http://ctan.org/pkg/xcolor}
% \end{description}
%
% \DescribeMacro{pagecolor-example.tex}
% The |pagecolor-example.tex| requires the same files as all
% documents using the \xpackage{pagecolor} package and additionally:
% \begin{description}
% \item[-] class \xpackage{article}, 2007/10/19, v1.4h, from \xpackage{classes.dtx}:\\
%           \CTAN{macros/latex/base/classes.dtx}
%
% \item[-] package \xpackage{pagecolor}, 2012/02/23, v1.0e,
%           \url{http://ctan.org/pkg/pagecolor}\\
%           (Well, it is the example file for this package, and because you are
%            reading the documentation for the \xpackage{pagecolor} package,
%            it can be assumed that you already have some version of it --
%            is it the current one?)
%
% \item[-] package \xpackage{xcolor}, 2007/01/21, v2.11,
%           \url{http://ctan.org/pkg/xcolor}\\
%           This package would not be needed for the use of just base colours only,
%           the \xpackage{color} package would be sufficient for that.
% \end{description}
%
% \DescribeMacro{transparent}
% \DescribeMacro{hrefhide}
% As possible alternatives in section \ref{sec:Alternatives} there are listed
% \begin{description}
% \item[-] package \xpackage{transparent}, 2007/01/08, v1.0, \\
%           \url{http://ctan.org/pkg/transparent}
%
% \item[-] package \xpackage{hrefhide}, 2011/04/29, v1.0f, \\
%           \url{http://ctan.org/pkg/hrefhide}
% \end{description}
%
% \DescribeMacro{Oberdiek}
% \DescribeMacro{holtxdoc}
% \DescribeMacro{kvoptions}
% All packages of \textsc{Heiko Oberdiek}'s bundle `oberdiek'
% (especially \xpackage{holtxdoc} and \xpackage{kvoptions})
% are also available in a TDS compliant ZIP archive:\\
% \CTAN{install/macros/latex/contrib/oberdiek.tds.zip}.\\
% It is probably best to download and use this, because the packages in there
% are quite probably both recent and compatible among themselves.\\
%
% \DescribeMacro{hyperref}
% \noindent \xpackage{hyperref} is not included in that bundle and needs to be
% downloaded separately,\\
% \url{http://mirror.ctan.org/install/macros/latex/contrib/hyperref.tds.zip}.\\
%
% \DescribeMacro{M\"{u}nch}
% A hyperlinked list of my (other) packages can be found at
% \url{http://www.Uni-Bonn.de/~uzs5pv/LaTeX.html}.\\
%
% \subsection{Package, unpacking TDS}
% \paragraph{Package.} This package is available on \CTAN{}
% \begin{description}
% \item[\CTAN{macros/latex/contrib/pagecolor/pagecolor.dtx}]\hspace*{0.1cm} \\
%       The source file.
% \item[\CTAN{macros/latex/contrib/pagecolor/pagecolor.pdf}]\hspace*{0.1cm} \\
%       The documentation.
% \item[\CTAN{macros/latex/contrib/pagecolor/pagecolor-example.pdf}]\hspace*{0.1cm} \\
%       The compiled example file, as it should look like.
% \item[\CTAN{macros/latex/contrib/pagecolor/README}]\hspace*{0.1cm} \\
%       The README file.
% \end{description}
% There is also a pagecolor.tds.zip available:
% \begin{description}
% \item[\CTAN{install/macros/latex/contrib/pagecolor.tds.zip}]\hspace*{0.1cm} \\
%       Everything in \xfile{TDS} compliant, compiled format.
% \end{description}
% which additionally contains\\
% \begin{tabular}{ll}
% pagecolor.ins & The installation file.\\
% pagecolor.drv & The driver to generate the documentation.\\
% pagecolor.sty & The \xext{sty}le file.\\
% pagecolor-example.tex & The example file.
% \end{tabular}
%
% \bigskip
%
% \noindent For required other packages, please see the preceding subsection.
%
% \paragraph{Unpacking.} The \xfile{.dtx} file is a self-extracting
% \docstrip{} archive. The files are extracted by running the
% \xext{.dtx} through \plainTeX{}:
% \begin{quote}
%   \verb|tex pagecolor.dtx|
% \end{quote}
%
% About generating the documentation see paragraph~\ref{GenDoc} below.\\
%
% \paragraph{TDS.} Now the different files must be moved into
% the different directories in your installation TDS tree
% (also known as \xfile{texmf} tree):
% \begin{quote}
% \def\t{^^A
% \begin{tabular}{@{}>{\ttfamily}l@{ $\rightarrow$ }>{\ttfamily}l@{}}
%   pagecolor.sty & tex/latex/pagecolor/pagecolor.sty\\
%   pagecolor.pdf & doc/latex/pagecolor/pagecolor.pdf\\
%   pagecolor-example.tex & doc/latex/pagecolor/pagecolor-example.tex\\
%   pagecolor-example.pdf & doc/latex/pagecolor/pagecolor-example.pdf\\
%   pagecolor.dtx & source/latex/pagecolor/pagecolor.dtx\\
% \end{tabular}^^A
% }^^A
% \sbox0{\t}^^A
% \ifdim\wd0>\linewidth
%   \begingroup
%     \advance\linewidth by\leftmargin
%     \advance\linewidth by\rightmargin
%   \edef\x{\endgroup
%     \def\noexpand\lw{\the\linewidth}^^A
%   }\x
%   \def\lwbox{^^A
%     \leavevmode
%     \hbox to \linewidth{^^A
%       \kern-\leftmargin\relax
%       \hss
%       \usebox0
%       \hss
%       \kern-\rightmargin\relax
%     }^^A
%   }^^A
%   \ifdim\wd0>\lw
%     \sbox0{\small\t}^^A
%     \ifdim\wd0>\linewidth
%       \ifdim\wd0>\lw
%         \sbox0{\footnotesize\t}^^A
%         \ifdim\wd0>\linewidth
%           \ifdim\wd0>\lw
%             \sbox0{\scriptsize\t}^^A
%             \ifdim\wd0>\linewidth
%               \ifdim\wd0>\lw
%                 \sbox0{\tiny\t}^^A
%                 \ifdim\wd0>\linewidth
%                   \lwbox
%                 \else
%                   \usebox0
%                 \fi
%               \else
%                 \lwbox
%               \fi
%             \else
%               \usebox0
%             \fi
%           \else
%             \lwbox
%           \fi
%         \else
%           \usebox0
%         \fi
%       \else
%         \lwbox
%       \fi
%     \else
%       \usebox0
%     \fi
%   \else
%     \lwbox
%   \fi
% \else
%   \usebox0
% \fi
% \end{quote}
% If you have a \xfile{docstrip.cfg} that configures and enables \docstrip{}'s
% \xfile{TDS} installing feature, then some files can already be in the right
% place, see the documentation of \docstrip{}.
%
% \subsection{Refresh file name databases}
%
% If your \TeX{}~distribution (\teTeX{}, \mikTeX{},\dots{}) relies on file name
% databases, you must refresh these. For example, \teTeX{} users run
% \verb|texhash| or \verb|mktexlsr|.
%
% \subsection{Some details for the interested}
%
% \paragraph{Unpacking with \LaTeX{}.}
% The \xfile{.dtx} chooses its action depending on the format:
% \begin{description}
% \item[\plainTeX:] Run \docstrip{} and extract the files.
% \item[\LaTeX:] Generate the documentation.
% \end{description}
% If you insist on using \LaTeX{} for \docstrip{} (really,
% \docstrip{} does not need \LaTeX{}), then inform the autodetect routine
% about your intention:
% \begin{quote}
%   \verb|latex \let\install=y\input{pagecolor.dtx}|
% \end{quote}
% Do not forget to quote the argument according to the demands
% of your shell.
%
% \paragraph{Generating the documentation.\label{GenDoc}}
% You can use both the \xfile{.dtx} or the \xfile{.drv} to generate
% the documentation. The process can be configured by a
% configuration file \xfile{ltxdoc.cfg}. For instance, put the following
% line into this file, if you want to have A4 as paper format:
% \begin{quote}
%   \verb|\PassOptionsToClass{a4paper}{article}|
% \end{quote}
%
% \noindent An example follows how to generate the
% documentation with \pdfLaTeX{}:
%
% \begin{quote}
%\begin{verbatim}
%pdflatex pagecolor.dtx
%makeindex -s gind.ist pagecolor.idx
%pdflatex pagecolor.dtx
%makeindex -s gind.ist pagecolor.idx
%pdflatex pagecolor.dtx
%\end{verbatim}
% \end{quote}
%
% \subsection{Compiling the example}
%
% The example file, \textsf{pagecolor-example.tex}, can be compiled via\\
% |(pdf)latex pagecolor-example.tex|.\\
%
% \section{Acknowledgements}
%
% I would like to thank \textsc{Heiko Oberdiek} for providing a~lot~(!) of
% useful packages (from which I also got everything I know about creating
% a file in \xext{dtx} format, ok, say it: copying), and the
% \Newsgroup{comp.text.tex} and \Newsgroup{de.comp.text.tex} newsgroups
% for their help in all things \TeX{}, especially all contributors to the
% discussion at
% \url{http://groups.google.com/group/comp.text.tex/browse_thread/thread/533576ebe44d60f6/e1654d70a19de63c?lnk=gst&q=Determination+of+the+current+page+colour#e1654d70a19de63c}
% \ (\textsc{H.~Oberdiek} \& \textsc{Gouailles}).\\
% I thank \textsc{haoyun\_tex} for suggesting the |\newpagecolor|/|\restorepagecolor|
% pair of commands and everyone at \url{http://tex.stackexchange.com/q/25137}
% for their contributions there.
%
% \bigskip
%
% \phantomsection
% \begin{History}\label{History}
%   \begin{Version}{2011/07/16 v1.0a}
%     \item First version discussed at \Newsgroup{comp.text.tex}.
%   \end{Version}
%   \begin{Version}{2011/08/06 v1.0b}
%     \item Changed version uploaded to \CTAN{}.
%   \end{Version}
%   \begin{Version}{2011/08/08 v1.0c}
%     \item Fixed a |\setkeys|.
%   \end{Version}
%   \begin{Version}{2012/02/01 v1.0d}
%     \item Bugfix: Obsolete installation path given in the documentation, updated.
%     \item New commands: |\newpagecolor{...}|, |\restorepagecolor|.
%     \item Update of documentation, README, and \xfile{dtx} internals.
%   \end{Version}
%   \begin{Version}{2012/02/23 v1.0e}
%     \item Fixed an error in the documentation.
%     \item Check for loading of \xpackage{color} or \xpackage{xcolor} package
%             and their versions has been changed, because \xpackage{xcolor} sets\\
%             |\@namedef{ver@color.sty}{1999/02/16}|\\
%             which gave a warning about old \xpackage{color} package even if a new
%             version was used.
%   \end{Version}
% \end{History}
%
% \bigskip
%
% When you find a mistake or have a suggestion for an improvement of this package,
% please send an e-mail to the maintainer, thanks! (Please see BUG REPORTS in the README.)
%
% \newpage
%
% \PrintIndex
%
% \Finale
\endinput
%        (quote the arguments according to the demands of your shell)
%
% Documentation:
%    (a) If pagecolor.drv is present:
%           (pdf)latex pagecolor.drv
%           makeindex -s gind.ist pagecolor.idx
%           (pdf)latex pagecolor.drv
%           makeindex -s gind.ist pagecolor.idx
%           (pdf)latex pagecolor.drv
%    (b) Without pagecolor.drv:
%           (pdf)latex pagecolor.dtx
%           makeindex -s gind.ist pagecolor.idx
%           (pdf)latex pagecolor.dtx
%           makeindex -s gind.ist pagecolor.idx
%           (pdf)latex pagecolor.dtx
%
%    The class ltxdoc loads the configuration file ltxdoc.cfg
%    if available. Here you can specify further options, e.g.
%    use DIN A4 as paper format:
%       \PassOptionsToClass{a4paper}{article}
%
% Installation:
%    TDS:tex/latex/pagecolor/pagecolor.sty
%    TDS:doc/latex/pagecolor/pagecolor.pdf
%    TDS:doc/latex/pagecolor/pagecolor-example.tex
%    TDS:doc/latex/pagecolor/pagecolor-example.pdf
%    TDS:source/latex/pagecolor/pagecolor.dtx
%
%<*ignore>
\begingroup
  \catcode123=1 %
  \catcode125=2 %
  \def\x{LaTeX2e}%
\expandafter\endgroup
\ifcase 0\ifx\install y1\fi\expandafter
         \ifx\csname processbatchFile\endcsname\relax\else1\fi
         \ifx\fmtname\x\else 1\fi\relax
\else\csname fi\endcsname
%</ignore>
%<*install>
\input docstrip.tex
\Msg{*****************************************************************}
\Msg{* Installation                                                   }
\Msg{* Package: pagecolor 2012/02/23 v1.0e Provides thepagecolor (HMM)}
\Msg{*****************************************************************}

\keepsilent
\askforoverwritefalse

\let\MetaPrefix\relax
\preamble

This is a generated file.

Project: pagecolor
Version: 2012/02/23 v1.0e

Copyright (C) 2011 - 2012 by
    H.-Martin M"unch <Martin dot Muench at Uni-Bonn dot de>

The usual disclaimer applies:
If it doesn't work right that's your problem.
(Nevertheless, send an e-mail to the maintainer
 when you find an error in this package.)

This work may be distributed and/or modified under the
conditions of the LaTeX Project Public License, either
version 1.3c of this license or (at your option) any later
version. This version of this license is in
   http://www.latex-project.org/lppl/lppl-1-3c.txt
and the latest version of this license is in
   http://www.latex-project.org/lppl.txt
and version 1.3c or later is part of all distributions of
LaTeX version 2005/12/01 or later.

This work has the LPPL maintenance status "maintained".

The Current Maintainer of this work is H.-Martin Muench.

This work consists of the main source file pagecolor.dtx,
the README, and the derived files
   pagecolor.sty, pagecolor.pdf,
   pagecolor.ins, pagecolor.drv,
   pagecolor-example.tex, pagecolor-example.pdf.

\endpreamble
\let\MetaPrefix\DoubleperCent

\generate{%
  \file{pagecolor.ins}{\from{pagecolor.dtx}{install}}%
  \file{pagecolor.drv}{\from{pagecolor.dtx}{driver}}%
  \usedir{tex/latex/pagecolor}%
  \file{pagecolor.sty}{\from{pagecolor.dtx}{package}}%
  \usedir{doc/latex/pagecolor}%
  \file{pagecolor-example.tex}{\from{pagecolor.dtx}{example}}%
}

\catcode32=13\relax% active space
\let =\space%
\Msg{************************************************************************}
\Msg{*}
\Msg{* To finish the installation you have to move the following}
\Msg{* file into a directory searched by TeX:}
\Msg{*}
\Msg{*     pagecolor.sty}
\Msg{*}
\Msg{* To produce the documentation run the file `pagecolor.drv'}
\Msg{* through (pdf)LaTeX, e.g.}
\Msg{*  pdflatex pagecolor.drv}
\Msg{*  makeindex -s gind.ist pagecolor.idx}
\Msg{*  pdflatex pagecolor.drv}
\Msg{*  makeindex -s gind.ist pagecolor.idx}
\Msg{*  pdflatex pagecolor.drv}
\Msg{*}
\Msg{* At least three runs are necessary e.g. to get the}
\Msg{*  references right!}
\Msg{*}
\Msg{* Happy TeXing!}
\Msg{*}
\Msg{************************************************************************}

\endbatchfile
%</install>
%<*ignore>
\fi
%</ignore>
%
% \section{The documentation driver file}
%
% The next bit of code contains the documentation driver file for
% \TeX{}, i.\,e., the file that will produce the documentation you
% are currently reading. It will be extracted from this file by the
% \texttt{docstrip} programme. That is, run \LaTeX{} on \texttt{docstrip}
% and specify the \texttt{driver} option when \texttt{docstrip}
% asks for options.
%
%    \begin{macrocode}
%<*driver>
\NeedsTeXFormat{LaTeX2e}[2011/06/27]
\ProvidesFile{pagecolor.drv}%
  [2012/02/23 v1.0e Provides thepagecolor (HMM)]
\documentclass{ltxdoc}[2007/11/11]% v2.0u
\usepackage{holtxdoc}[2011/02/04]%  v0.21
%% pagecolor may work with earlier versions of LaTeX2e and those
%% class and package, but this was not tested.
%% Please consider updating your LaTeX, class, and package
%% to the most recent version (if they are not already the most
%% recent version).
\hypersetup{%
 pdfsubject={Providing the thepagecolor command (HMM)},%
 pdfkeywords={LaTeX, pagecolor, thepagecolor, H.-Martin Muench},%
 pdfencoding=auto,%
 pdflang={en},%
 breaklinks=true,%
 linktoc=all,%
 pdfstartview=FitH,%
 pdfpagelayout=OneColumn,%
 bookmarksnumbered=true,%
 bookmarksopen=true,%
 bookmarksopenlevel=3,%
 pdfmenubar=true,%
 pdftoolbar=true,%
 pdfwindowui=true,%
 pdfnewwindow=true%
}
\CodelineIndex
\hyphenation{printing docu-ment}
\gdef\unit#1{\mathord{\thinspace\mathrm{#1}}}%
\begin{document}
  \DocInput{pagecolor.dtx}%
\end{document}
%</driver>
%    \end{macrocode}
%
% \fi
%
% \CheckSum{137}
%
% \CharacterTable
%  {Upper-case    \A\B\C\D\E\F\G\H\I\J\K\L\M\N\O\P\Q\R\S\T\U\V\W\X\Y\Z
%   Lower-case    \a\b\c\d\e\f\g\h\i\j\k\l\m\n\o\p\q\r\s\t\u\v\w\x\y\z
%   Digits        \0\1\2\3\4\5\6\7\8\9
%   Exclamation   \!     Double quote  \"     Hash (number) \#
%   Dollar        \$     Percent       \%     Ampersand     \&
%   Acute accent  \'     Left paren    \(     Right paren   \)
%   Asterisk      \*     Plus          \+     Comma         \,
%   Minus         \-     Point         \.     Solidus       \/
%   Colon         \:     Semicolon     \;     Less than     \<
%   Equals        \=     Greater than  \>     Question mark \?
%   Commercial at \@     Left bracket  \[     Backslash     \\
%   Right bracket \]     Circumflex    \^     Underscore    \_
%   Grave accent  \`     Left brace    \{     Vertical bar  \|
%   Right brace   \}     Tilde         \~}
%
% \GetFileInfo{pagecolor.drv}
%
% \begingroup
%   \def\x{\#,\$,\^,\_,\~,\ ,\&,\{,\},\%}%
%   \makeatletter
%   \@onelevel@sanitize\x
% \expandafter\endgroup
% \expandafter\DoNotIndex\expandafter{\x}
% \expandafter\DoNotIndex\expandafter{\string\ }
% \begingroup
%   \makeatletter
%     \lccode`9=32\relax
%     \lowercase{%^^A
%       \edef\x{\noexpand\DoNotIndex{\@backslashchar9}}%^^A
%     }%^^A
%   \expandafter\endgroup\x
%
% \DoNotIndex{\,,\\}
% \DoNotIndex{\documentclass,\usepackage,\ProvidesPackage,\begin,\end}
% \DoNotIndex{\MessageBreak}
% \DoNotIndex{\NeedsTeXFormat,\DoNotIndex,\verb}
% \DoNotIndex{\def,\edef,\gdef,\global}
% \DoNotIndex{\ifx,\listfiles,\mathord,\mathrm}
% \DoNotIndex{\kvoptions,\SetupKeyvalOptions,\ProcessKeyvalOptions}
% \DoNotIndex{\bigskip,\space,\thinspace,\ldots}
% \DoNotIndex{\indent,\noindent,\newline,\linebreak,\pagebreak,\newpage}
% \DoNotIndex{\textbf,\textit,\textsf,\texttt,\textsc,\textquotedblleft,\textquotedblright}
% \DoNotIndex{\plainTeX,\TeX,\LaTeX,\pdfLaTeX}
% \DoNotIndex{\section,\color,\pagenumbering,\smallskip}
% \DoNotIndex{\@ifundefined,\@ifpackageloaded}
% \DoNotIndex{\DeclareStringOption}
%
% \title{The \xpackage{pagecolor} package}
% \date{2012/02/23 v1.0e}
% \author{H.-Martin M\"{u}nch\\\xemail{Martin.Muench at Uni-Bonn.de}}
%
% \maketitle
%
% \begin{abstract}
%  This \LaTeX{} package provides the command |\thepagecolor|,
% which gives the current page (background) colour,
% i.\,e.~the argument used with the most recent call of |\pagecolor{...}|.
% The command |\thepagecolornone| gives the same colour as |\thepagecolor|,
% except when the page background colour is
% \textquotedblleft|none|\textquotedblright. In that case |\thepagecolor| is
% |white| and |\thepagecolornone| is |none|.\\
% Similar to |\newgeometry| and |\restoregeometry| of the \xpackage{geometry}
% package\\
% |\newpagecolor{<|\textit{some colour} |>}| and |\restorepagecolor|
% are provided.
% \end{abstract}
%
% \bigskip
%
% \noindent Disclaimer for web links: The author is not responsible for any contents
% referred to in this work unless he has full knowledge of illegal contents.
% If any damage occurs by the use of information presented there, only the
% author of the respective pages might be liable, not the one who has referred
% to these pages.
%
% \bigskip
%
% \noindent {\color{green} Save per page about $200\unit{ml}$ water,
% $2\unit{g}$ CO$_{2}$ and $2\unit{g}$ wood:\\
% Therefore please print only if this is really necessary.}
%
% \pagebreak
%
% \tableofcontents
%
% \bigskip
%
% \section{Introduction\label{sec:Introduction}}
%
% \indent This \LaTeX{} package provides the command |\thepagecolor|,
% which gives the current page (background) colour,
% i.\,e.~the argument used with the most recent call of |\pagecolor{...}|.
% The package should be loaded before any package sets a page (background)
% colour, but after \xpackage{color} or \xpackage{xcolor} package.
% Its option |pagecolor={...}| is used to set the |\pagecolor{...}|.\\
% The command |\thepagecolornone| gives the same colour as |\thepagecolor|,
% except when the page background colour is
% \textquotedblleft|none|\textquotedblright (e.\,g. result of using the
% |\nopagecolor| command). In that case |\thepagecolor| is |white| and\linebreak
% |\thepagecolornone| is |none|.\\
% Similar to |\newgeometry| and |\restoregeometry| of the \xpackage{geometry}
% package\\
% |\newpagecolor{<|\textit{some colour}|>}| and |\restorepagecolor|
% are provided.
%
% \pagebreak
%
% \section{Usage}
%
% \indent Just load the package placing
% \begin{quote}
%   |\usepackage[<|\textit{option}|>]{pagecolor}|
% \end{quote}
% \noindent in the preamble of your \LaTeXe{} source file.
% This should be done before another package uses |\pagecolor|,
% but after |\nopagecolor| is defined (if that is defined in the document
% at all).\\
% Afterwards |\pagecolor{...}| can be used to change the page (background)
% colour as usual. Then |\thepagecolor| gives the current page (background)
% colour (in the same format as given with |\pagecolor{...}|).\\
% Similar to |\newgeometry| and |\restoregeometry| of the \xpackage{geometry}
% package\\
% |\newpagecolor{<|\textit{some colour} |>}| and |\restorepagecolor|
% are provided:\\
% |\newpagecolor{<|\textit{some colour} |>}| will execute
% |\pagecolor{<|\textit{some colour} |>}| and remember the page colour used before.
% |\restorecolor| (without argument) restores the page colour to the one used before
%  use of the |\newpagecolor{...}| command. When you want to change the colour for
%  just one page and do not want to (or cannot) manually determine where the page ends,\\
% |\newpagecolor{<|\textit{some colour}|>}\afterpage{\restorepagecolor}|\\
% does the trick (and requires a |\usepackage{afterpage}| in the document's
% preamble), or for short\\
% |\newcommand{\onepagecolor}[1]{%|\\
% |\newpagecolor{#1}\afterpage{\restorepagecolor}}|\\
% in the preamble and\\
% |\onepagecolor{<|\textit{some colour}|>}| in the document.
%
% \subsection{Option}
% \DescribeMacro{option}
% \indent The \xpackage{pagecolor} package takes the following option:
%
% \subsubsection{pagecolor\label{sss:pagecolor}}
% \DescribeMacro{pagecolor}
% \indent The option |pagecolor={...}| takes as value a colour. This
% could be as simple as |white| or |black|, but when e.\,g.~the
% \xpackage{xcolor} package is used (loaded before \xpackage{pagecolor}!),
% also colours like |red!50!green!20!blue| are possible.
% The default is |pagecolor={none}|.
%
% \section{Alternatives\label{sec:Alternatives}}
%
% As I neither know what exactly you want to accomplish
% when using this package (e.\,g.~hiding text), nor what resources you have
% (e.\,g.~\pdfTeX{} version), here is a list of some possible alternatives:\\
%
% \begin{description}
%  \item[-]
%  \DescribeMacro{transparent}
%  \xpackage{transparent} package: With it some object can be made
%  (fully or partially) transparent.
% \end{description}
%
% \begin{description}
%  \item[-]
%  \DescribeMacro{hrefhide}\xpackage{hrefhide} package: It allows to
%  \textquotedblleft hide\textquotedblright\ some (hyperlinked) text when
%  printing the document while keeping the layout.
% \end{description}
%
% \bigskip
%
% \noindent (You programmed or found another alternative,
%  which is available at \CTAN{}?\\
%  OK, send an e-mail to me with the name, location at \CTAN{},
%  and a short notice, and I will probably include it in
%  the list above.)\\
%
% \smallskip
%
% \noindent About how to get those packages, please see
% subsection~\ref{ss:Downloads}.
%
% \newpage
%
% \section{Example}
%
%    \begin{macrocode}
%<*example>
\documentclass[british]{article}[2007/10/19]% v1.4h
%%%%%%%%%%%%%%%%%%%%%%%%%%%%%%%%%%%%%%%%%%%%%%%%%%%%%%%%%%%%%%%%%%%%%
\usepackage[%
 extension=pdf,%
 plainpages=false,%
 pdfpagelabels=true,%
 hyperindex=false,%
 pdflang={en},%
 pdftitle={pagecolor package example},%
 pdfauthor={H.-Martin Muench},%
 pdfsubject={Example for the pagecolor package},%
 pdfkeywords={LaTeX, pagecolor, thepagecolor, page colour,%
  H.-Martin Muench},%
 pdfview=Fit,pdfstartview=Fit,%
 pdfpagelayout=SinglePage%
]{hyperref}[2012/02/06]% v6.82o
\usepackage[x11names]{xcolor}[2007/01/21]% v2.11
 % The xcolor package would not be needed for just using
 % the base colours. The color package would be sufficient for that.
\usepackage[pagecolor={LightGoldenrod1}]{pagecolor}[2012/02/23]% v1.0e

\usepackage{afterpage}[1995/10/27]% v1.08
 % The afterpage package is generally not needed,
 % but the |\newpagecolor{somecolour}\afterpage{\restorepagecolor}|
 % construct shall be demonstrated.

\usepackage{lipsum}[2011/04/14]% v1.2
 % The lipsum package is generally not needed,
 % but some blind text is needed for the example.

\gdef\unit#1{\mathord{\thinspace\mathrm{#1}}}%
\listfiles
\begin{document}
\pagenumbering{arabic}
\section*{Example for pagecolor}

This example demonstrates the use of package\newline
\textsf{pagecolor}, v1.0e as of 2012/02/23 (HMM).\newline
The used option was \verb|pagecolor={LightGoldenrod1}|.\newline
\verb|pagecolor={none}| would be the default.\newline

For more details please see the documentation!\newline

\noindent {\color{teal} Save per page about $200\unit{ml}$ water,
$2\unit{g}$ CO$_{2}$ and $2\unit{g}$ wood:\newline
Therefore please print only if this is really necessary.}\newline

The current page (background) colour is\newline
\verb|\thepagecolor|\ =\ \thepagecolor \newline
(and \verb|\thepagecolornone|\ =\ \thepagecolornone ,
which would only be different from \verb|\thepagecolor|,
when the page colour would be \verb|none|).

\pagebreak
\pagecolor{rgb:-green!40!yellow,3;green!40!yellow,2;red,1}

{\color{white} The current page (background) colour is\newline
\verb|\thepagecolor|\ =\ \thepagecolor . \newline}

{\color{\thepagecolor} And that makes this text practically invisible.
\newline}

{\color{white} Which made the preceding line of text practically
invisible.}

\pagebreak
\newpagecolor{red}

This page uses \verb|\newpagecolor{red}|.

\pagebreak
\restorepagecolor

{\color{white}And this page uses \verb|\restorepagecolor| to restore
the page colour to the value it had before the red page.}

\pagebreak
\pagecolor{none}

This page uses \verb|\pagecolor{none}|. If the \verb|\nopagecolor|
command is known (pdf\TeX and Lua\TeX; not yet for dvips, dvipdfm(x)
or Xe\TeX), the page colour is now \verb|none|, otherwise \verb|white|:
\verb|\thepagecolor|\ =\ \thepagecolor\ and
\verb|\thepagecolornone|\ =\ \thepagecolornone .

\pagebreak
\restorepagecolor

{\color{white}\verb|\restorepagecolor| restored the page colour again.}

\pagebreak
\pagecolor{green}

This page is green due to \verb|\pagecolor{green}|.

\pagebreak
\newpagecolor{blue}\afterpage{\restorepagecolor}

{\color{white}\verb|\newpagecolor{blue}\afterpage{\restorepagecolor}|%
\newline
was used here, i.\,e.~this page is blue, and the next one will
automatically have the same page colour before it was changed to blue
here (i.\,e. green).}

\smallskip
{\color{red}\textbf{\lipsum[1-11]}}
\bigskip

The page colour was changed back at the end of the page -
in mid-sentence!

\end{document}
%</example>
%    \end{macrocode}
%
% \StopEventually{}
%
% \newpage
%
% \section{The implementation}
%
% We start off by checking that we are loading into \LaTeXe{} and
% announcing the name and version of this package.
%
%    \begin{macrocode}
%<*package>
%    \end{macrocode}
%
%    \begin{macrocode}
\NeedsTeXFormat{LaTeX2e}[2011/06/27]
\ProvidesPackage{pagecolor}[2012/02/23 v1.0e
            Provides thepagecolor (HMM)]
%    \end{macrocode}
%
% A short description of the \xpackage{pagecolor} package:
%
%    \begin{macrocode}
%% Provides the \thepagecolor, \thepagecolornone, \newpagecolor{...},
%% and \restorepagecolor commands.
%    \end{macrocode}
%
% We need the \xpackage{kvoptions} package by \textsc{Heiko Oberdiek}:
%
%    \begin{macrocode}
\RequirePackage{kvoptions}[2010/12/23]% v3.10
%    \end{macrocode}
%
% and either the \xpackage{color} or the \xpackage{xcolor} package:
%
%    \begin{macrocode}
%% \RequirePackage{ either color or xcolor }:
\@ifpackageloaded{xcolor}{% xcolor loaded
   \@ifpackagelater{xcolor}{2007/01/21}{%
      % 2007/01/21, v2.11, or even more recent: OK
    }{% else: older package version
      \PackageWarning{pagecolor}{%
        It is requested version `2007/01/21' of package\MessageBreak%
        xcolor, but only an older version is available.\MessageBreak%
       }%
     }%
 }{% xcolor not loaded
   \@ifpackageloaded{color}{%
      \RequirePackage{color}[2005/11/14]% v1.0j
    }{% \else
      \PackageWarning{pagecolor}{%
        The pagecolor package must be loaded after either\MessageBreak%
        package color or after package xcolor (at your\MessageBreak%
        option). Neither package was loaded before package\MessageBreak%
        pagecolor. Loading of package xcolor will now be\MessageBreak%
        tried automatically.\MessageBreak%
        When the pagecolor package is used with option\MessageBreak%
        pagecolor using a colour requiring e. g. x11names\MessageBreak%
        option for xcolor package, this will not work!\MessageBreak%
       }
     }% \fi
   \RequirePackage{xcolor}[2007/01/21]% v2.11
  }% \fi
%    \end{macrocode}
%
% A last information for the user:
%
%    \begin{macrocode}
%% pagecolor may work with earlier versions of LaTeX and the
%% packages, but this was not tested. Please consider updating
%% your LaTeX and packages to the most recent version
%% (if they are not already the most recent version).

%    \end{macrocode}
%
% \indent See subsection~\ref{ss:Downloads} about how to get them.\\
%
% We process the options:
%
%    \begin{macrocode}
\SetupKeyvalOptions{family=pagecolor,prefix=pagecolor@}
\DeclareStringOption[none]{pagecolor}% \pagecolor@pagecolor
\ProcessKeyvalOptions*

%    \end{macrocode}
%
% We save the original |\pagecolor| command,
%
%    \begin{macrocode}
\let\origpagecolour\pagecolor

%    \end{macrocode}
%
%    \begin{macro}{\pagecolor}
% before we redefine it to include a definition of |\thepagecolor|
% and \linebreak
% |\thepagecolornone|:
%
%    \begin{macrocode}
\renewcommand{\pagecolor}[1]{\@bsphack%
  \edef\pagecolourtmpa{#1}%
  \def\pagecolourtmpb{none}%
  \ifx\pagecolourtmpa\pagecolourtmpb
    \@ifundefined{nopagecolor}{%
      \PackageError{pagecolor}{%
        pagecolor=none requested but \string\nopagecolor\space unknown%
       }{%
        \string\pagecolor{none} was used, but the command %
        \string\nopagecolor\space is undefined.\MessageBreak%
        Please use another colour. pagecolor=white will be used now.%
        \MessageBreak%
        }%
      \gdef\thepagecolor{white}%
      \gdef\thepagecolornone{white}% although it should be "none"
      \origpagecolour{white}%
     }{%
      \nopagecolor%
     }%
  \else%
    \xdef\thepagecolor{#1}%
    \xdef\thepagecolornone{#1}%
    \origpagecolour{\thepagecolornone}%
  \fi%
  \@esphack%
}

%    \end{macrocode}
%    \end{macro}
%
% |\nopagecolor| is only defined for \pdfTeX and \LuaTeX, but not yet for dvips,
% dvipdfm(x) or \XeTeX. Maybe
% \begin{verbatim}
% \@ifundefined{nopagecolor}{\newcommand{\nopagecolor}{\pagecolor{white}}}{}
% \end{verbatim}
% could be an alternative.\\
% When |\pagecolor{none}| is used and |\nopagecolor| is defined, then \\
% |\pagecolor{none}| is made into a synonym for |\nopagecolor|.\\
% If |\nopagecolor| has not been defined, nothing needs to be done.
% Otherwise we redefine |\nopagecolor|, thus that |\thepagecolor| is
% set to |white| and |\thepagecolornone| to |none| whenever
% |\nopagecolor| is used.
%
%    \begin{macrocode}
\@ifundefined{nopagecolor}{%
  \PackageWarning{pagecolor}{%
    \string\nopagecolor\space is undefined.\MessageBreak%
    (As long as it is neither used anywhere\MessageBreak%
    \space nor defined later, that will be no problem.)\MessageBreak%
   }
  \AtEndDocument{%
  \@ifundefined{nopagecolor}{% no problem
    }{%
    \PackageError{pagecolor}{%
      \string\nopagecolor\space defined after loading package%
      \MessageBreak%
      pagecolor%
      }{\string\nopagecolor\space was defined after loading the %
       pagecolor package.\MessageBreak%
       Either define \string\nopagecolor\space earlier or load the %
       pagecolor package later.\MessageBreak%
      }
    }%
  }%
  }{% \else
    \let\orignopagecolour\nopagecolor
    \renewcommand{\nopagecolor}{%
      \xdef\thepagecolor{white}
      \xdef\thepagecolornone{none}
      \orignopagecolour
      }
  }

%    \end{macrocode}
%
% If the page colour as given with option |pagecolor={...}| is |none|,
% but |\nopagecolor| is not known, |\pagecolor@pagecolor| is set to
% |white| and a warning is given.
%
%    \begin{macrocode}
\def\pagecolourtmpb{none}
\ifx\pagecolor@pagecolor\pagecolourtmpb
  \@ifundefined{nopagecolor}{%
    \PackageWarning{pagecolor}{%
      Option pagecolor=none (maybe by default) used,\MessageBreak%
      but \string\nopagecolor\space is unknown. Please use another%
      \MessageBreak%
      option value; white will be used now.\MessageBreak%
     }
    \setkeys{pagecolor}{pagecolor=white}%
  }{% ok
  }%
\fi

%    \end{macrocode}
%
% The (new) |\pagecolor| is now just carried out.
%
%    \begin{macrocode}
\pagecolor{\pagecolor@pagecolor}

%    \end{macrocode}
%
% Now the page (background) colour and |\thepagecolor| and \\
% |\thepagecolornone| are |\pagecolor@pagecolor| (or page (background) colour
% and |\thepagecolornone| are |none|=|\pagecolor@pagecolor| and
% |\thepagecolor| is |white|), and when the page (background) colour
% is changed, |\thepagecolor| and |\thepagecolornone| are changed
% accordingly.
%
% \newpage
%
%    \begin{macro}{\newpagecolor}
% There have been requests (via e-mail and at \\
% \url{http://tex.stackexchange.com/q/25137}) to change the colour of just one
% (or two) page(s) only, similar to |\newgeometry| and |\restoregeometry| of the
% \xpackage{geometry} package (\url{http://ctan.org/pkg/geometry}). Therefore
% |\newpagecolor| and |\restorepagecolor| are introduced (as suggested by
% \textsc{haoyun\_tex}):
%
%    \begin{macrocode}
\newcommand{\newpagecolor}[1]{%
\xdef\pagecolourtmpc{\thepagecolornone}%
\pagecolor{#1}%
}

%    \end{macrocode}
%    \end{macro}
%
% |\newpagecolor{<|\textit{some colour}|>}| will execute |\pagecolor{somecolour}|
% and remember the page colour used before.\\
%
%    \begin{macro}{\restorepagecolor}
%    \begin{macrocode}
\newcommand{\restorepagecolor}{\pagecolor{\pagecolourtmpc}}

%    \end{macrocode}
%    \end{macro}
%
% |\restorecolor| (without argument) restores the page colour to the one used before
% use of the |\newpagecolor{...}| command.
%
%    \begin{macrocode}
\gdef\pagecolourtmpc{\thepagecolor}

%    \end{macrocode}
% is just a precaution for |\restorecolor| being used when no |\newpagecolor{...}|
% was used before it.\\
%
% When you want to change the colour for just one page and do not want to
% (or cannot) manually determine where the page ends,\\
% |\newpagecolor{<|\textit{some colour}|>}\afterpage{\restorepagecolor}|\\
% does the trick (and requires an additional |\usepackage{afterpage}| in the
% document's preamble).
%    \begin{macrocode}
%</package>
%    \end{macrocode}
%
% \newpage
%
% \section{Installation}
%
% \subsection{Downloads\label{ss:Downloads}}
%
% Everything is available on \CTAN{}, \url{http://www.ctan.org/tex-archive/},
% but may need additional packages themselves.\\
%
% \DescribeMacro{pagecolor.dtx}
% For unpacking the |pagecolor.dtx| file and constructing the documentation
% it is required:
% \begin{description}
% \item[-] \TeX{} Format \LaTeXe{}, \url{http://www.CTAN.org/}
%
% \item[-] document class \xpackage{ltxdoc}, 2007/11/11, v2.0u,
%           \url{http://ctan.org/pkg/ltxdoc}
%
% \item[-] package \xpackage{holtxdoc}, 2011/02/04, v0.21,
%           \url{http://ctan.org/pkg/holtxdoc}
% \end{description}
%
% \DescribeMacro{pagecolor.sty}
% The |pagecolor.sty| for \LaTeXe{} (i.\,e. each document using
% the \xpackage{pagecolor} package) requires:
% \begin{description}
% \item[-] \TeX{} Format \LaTeXe{}, \url{http://www.CTAN.org/}
%
% \item[-] package \xpackage{kvoptions}, 2010/12/23, v3.10,
%           \url{http://ctan.org/pkg/kvoptions}
% \end{description}
% and either
% \begin{description}
% \item[-] package \xpackage{color}, 2005/11/14, v1.0j,
%           \url{http://ctan.org/pkg/color} (from the\\
%           \xpackage{graphics} package bundle)
% \end{description}
% or
% \begin{description}
% \item[-] package \xpackage{xcolor}, 2007/01/21, v2.11,
%           \url{http://ctan.org/pkg/xcolor}
% \end{description}
%
% \DescribeMacro{pagecolor-example.tex}
% The |pagecolor-example.tex| requires the same files as all
% documents using the \xpackage{pagecolor} package and additionally:
% \begin{description}
% \item[-] class \xpackage{article}, 2007/10/19, v1.4h, from \xpackage{classes.dtx}:\\
%           \CTAN{macros/latex/base/classes.dtx}
%
% \item[-] package \xpackage{pagecolor}, 2012/02/23, v1.0e,
%           \url{http://ctan.org/pkg/pagecolor}\\
%           (Well, it is the example file for this package, and because you are
%            reading the documentation for the \xpackage{pagecolor} package,
%            it can be assumed that you already have some version of it --
%            is it the current one?)
%
% \item[-] package \xpackage{xcolor}, 2007/01/21, v2.11,
%           \url{http://ctan.org/pkg/xcolor}\\
%           This package would not be needed for the use of just base colours only,
%           the \xpackage{color} package would be sufficient for that.
% \end{description}
%
% \DescribeMacro{transparent}
% \DescribeMacro{hrefhide}
% As possible alternatives in section \ref{sec:Alternatives} there are listed
% \begin{description}
% \item[-] package \xpackage{transparent}, 2007/01/08, v1.0, \\
%           \url{http://ctan.org/pkg/transparent}
%
% \item[-] package \xpackage{hrefhide}, 2011/04/29, v1.0f, \\
%           \url{http://ctan.org/pkg/hrefhide}
% \end{description}
%
% \DescribeMacro{Oberdiek}
% \DescribeMacro{holtxdoc}
% \DescribeMacro{kvoptions}
% All packages of \textsc{Heiko Oberdiek}'s bundle `oberdiek'
% (especially \xpackage{holtxdoc} and \xpackage{kvoptions})
% are also available in a TDS compliant ZIP archive:\\
% \CTAN{install/macros/latex/contrib/oberdiek.tds.zip}.\\
% It is probably best to download and use this, because the packages in there
% are quite probably both recent and compatible among themselves.\\
%
% \DescribeMacro{hyperref}
% \noindent \xpackage{hyperref} is not included in that bundle and needs to be
% downloaded separately,\\
% \url{http://mirror.ctan.org/install/macros/latex/contrib/hyperref.tds.zip}.\\
%
% \DescribeMacro{M\"{u}nch}
% A hyperlinked list of my (other) packages can be found at
% \url{http://www.Uni-Bonn.de/~uzs5pv/LaTeX.html}.\\
%
% \subsection{Package, unpacking TDS}
% \paragraph{Package.} This package is available on \CTAN{}
% \begin{description}
% \item[\CTAN{macros/latex/contrib/pagecolor/pagecolor.dtx}]\hspace*{0.1cm} \\
%       The source file.
% \item[\CTAN{macros/latex/contrib/pagecolor/pagecolor.pdf}]\hspace*{0.1cm} \\
%       The documentation.
% \item[\CTAN{macros/latex/contrib/pagecolor/pagecolor-example.pdf}]\hspace*{0.1cm} \\
%       The compiled example file, as it should look like.
% \item[\CTAN{macros/latex/contrib/pagecolor/README}]\hspace*{0.1cm} \\
%       The README file.
% \end{description}
% There is also a pagecolor.tds.zip available:
% \begin{description}
% \item[\CTAN{install/macros/latex/contrib/pagecolor.tds.zip}]\hspace*{0.1cm} \\
%       Everything in \xfile{TDS} compliant, compiled format.
% \end{description}
% which additionally contains\\
% \begin{tabular}{ll}
% pagecolor.ins & The installation file.\\
% pagecolor.drv & The driver to generate the documentation.\\
% pagecolor.sty & The \xext{sty}le file.\\
% pagecolor-example.tex & The example file.
% \end{tabular}
%
% \bigskip
%
% \noindent For required other packages, please see the preceding subsection.
%
% \paragraph{Unpacking.} The \xfile{.dtx} file is a self-extracting
% \docstrip{} archive. The files are extracted by running the
% \xext{.dtx} through \plainTeX{}:
% \begin{quote}
%   \verb|tex pagecolor.dtx|
% \end{quote}
%
% About generating the documentation see paragraph~\ref{GenDoc} below.\\
%
% \paragraph{TDS.} Now the different files must be moved into
% the different directories in your installation TDS tree
% (also known as \xfile{texmf} tree):
% \begin{quote}
% \def\t{^^A
% \begin{tabular}{@{}>{\ttfamily}l@{ $\rightarrow$ }>{\ttfamily}l@{}}
%   pagecolor.sty & tex/latex/pagecolor/pagecolor.sty\\
%   pagecolor.pdf & doc/latex/pagecolor/pagecolor.pdf\\
%   pagecolor-example.tex & doc/latex/pagecolor/pagecolor-example.tex\\
%   pagecolor-example.pdf & doc/latex/pagecolor/pagecolor-example.pdf\\
%   pagecolor.dtx & source/latex/pagecolor/pagecolor.dtx\\
% \end{tabular}^^A
% }^^A
% \sbox0{\t}^^A
% \ifdim\wd0>\linewidth
%   \begingroup
%     \advance\linewidth by\leftmargin
%     \advance\linewidth by\rightmargin
%   \edef\x{\endgroup
%     \def\noexpand\lw{\the\linewidth}^^A
%   }\x
%   \def\lwbox{^^A
%     \leavevmode
%     \hbox to \linewidth{^^A
%       \kern-\leftmargin\relax
%       \hss
%       \usebox0
%       \hss
%       \kern-\rightmargin\relax
%     }^^A
%   }^^A
%   \ifdim\wd0>\lw
%     \sbox0{\small\t}^^A
%     \ifdim\wd0>\linewidth
%       \ifdim\wd0>\lw
%         \sbox0{\footnotesize\t}^^A
%         \ifdim\wd0>\linewidth
%           \ifdim\wd0>\lw
%             \sbox0{\scriptsize\t}^^A
%             \ifdim\wd0>\linewidth
%               \ifdim\wd0>\lw
%                 \sbox0{\tiny\t}^^A
%                 \ifdim\wd0>\linewidth
%                   \lwbox
%                 \else
%                   \usebox0
%                 \fi
%               \else
%                 \lwbox
%               \fi
%             \else
%               \usebox0
%             \fi
%           \else
%             \lwbox
%           \fi
%         \else
%           \usebox0
%         \fi
%       \else
%         \lwbox
%       \fi
%     \else
%       \usebox0
%     \fi
%   \else
%     \lwbox
%   \fi
% \else
%   \usebox0
% \fi
% \end{quote}
% If you have a \xfile{docstrip.cfg} that configures and enables \docstrip{}'s
% \xfile{TDS} installing feature, then some files can already be in the right
% place, see the documentation of \docstrip{}.
%
% \subsection{Refresh file name databases}
%
% If your \TeX{}~distribution (\teTeX{}, \mikTeX{},\dots{}) relies on file name
% databases, you must refresh these. For example, \teTeX{} users run
% \verb|texhash| or \verb|mktexlsr|.
%
% \subsection{Some details for the interested}
%
% \paragraph{Unpacking with \LaTeX{}.}
% The \xfile{.dtx} chooses its action depending on the format:
% \begin{description}
% \item[\plainTeX:] Run \docstrip{} and extract the files.
% \item[\LaTeX:] Generate the documentation.
% \end{description}
% If you insist on using \LaTeX{} for \docstrip{} (really,
% \docstrip{} does not need \LaTeX{}), then inform the autodetect routine
% about your intention:
% \begin{quote}
%   \verb|latex \let\install=y% \iffalse meta-comment
%
% File: pagecolor.dtx
% Version: 2012/02/23 v1.0e
%
% Copyright (C) 2011 - 2012 by
%    H.-Martin M"unch <Martin dot Muench at Uni-Bonn dot de>
%
% This work may be distributed and/or modified under the
% conditions of the LaTeX Project Public License, either
% version 1.3c of this license or (at your option) any later
% version. This version of this license is in
%    http://www.latex-project.org/lppl/lppl-1-3c.txt
% and the latest version of this license is in
%    http://www.latex-project.org/lppl.txt
% and version 1.3c or later is part of all distributions of
% LaTeX version 2005/12/01 or later.
%
% This work has the LPPL maintenance status "maintained".
%
% The Current Maintainer of this work is H.-Martin Muench.
%
% This work consists of the main source file pagecolor.dtx,
% the README, and the derived files
%    pagecolor.sty, pagecolor.pdf,
%    pagecolor.ins, pagecolor.drv,
%    pagecolor-example.tex, pagecolor-example.pdf.
%
% Distribution:
%    CTAN:macros/latex/contrib/pagecolor/pagecolor.dtx
%    CTAN:macros/latex/contrib/pagecolor/pagecolor.pdf
%    CTAN:install/macros/latex/contrib/pagecolor.tds.zip
%
% Cf. http://ctan.org/pkg/pagecolor and
% http://www.tex.ac.uk/tex-archive/help/Catalogue/entries/pagecolor.html
%
% Unpacking:
%    (a) If pagecolor.ins is present:
%           tex pagecolor.ins
%    (b) Without pagecolor.ins:
%           tex pagecolor.dtx
%    (c) If you insist on using LaTeX
%           latex \let\install=y\input{pagecolor.dtx}
%        (quote the arguments according to the demands of your shell)
%
% Documentation:
%    (a) If pagecolor.drv is present:
%           (pdf)latex pagecolor.drv
%           makeindex -s gind.ist pagecolor.idx
%           (pdf)latex pagecolor.drv
%           makeindex -s gind.ist pagecolor.idx
%           (pdf)latex pagecolor.drv
%    (b) Without pagecolor.drv:
%           (pdf)latex pagecolor.dtx
%           makeindex -s gind.ist pagecolor.idx
%           (pdf)latex pagecolor.dtx
%           makeindex -s gind.ist pagecolor.idx
%           (pdf)latex pagecolor.dtx
%
%    The class ltxdoc loads the configuration file ltxdoc.cfg
%    if available. Here you can specify further options, e.g.
%    use DIN A4 as paper format:
%       \PassOptionsToClass{a4paper}{article}
%
% Installation:
%    TDS:tex/latex/pagecolor/pagecolor.sty
%    TDS:doc/latex/pagecolor/pagecolor.pdf
%    TDS:doc/latex/pagecolor/pagecolor-example.tex
%    TDS:doc/latex/pagecolor/pagecolor-example.pdf
%    TDS:source/latex/pagecolor/pagecolor.dtx
%
%<*ignore>
\begingroup
  \catcode123=1 %
  \catcode125=2 %
  \def\x{LaTeX2e}%
\expandafter\endgroup
\ifcase 0\ifx\install y1\fi\expandafter
         \ifx\csname processbatchFile\endcsname\relax\else1\fi
         \ifx\fmtname\x\else 1\fi\relax
\else\csname fi\endcsname
%</ignore>
%<*install>
\input docstrip.tex
\Msg{*****************************************************************}
\Msg{* Installation                                                   }
\Msg{* Package: pagecolor 2012/02/23 v1.0e Provides thepagecolor (HMM)}
\Msg{*****************************************************************}

\keepsilent
\askforoverwritefalse

\let\MetaPrefix\relax
\preamble

This is a generated file.

Project: pagecolor
Version: 2012/02/23 v1.0e

Copyright (C) 2011 - 2012 by
    H.-Martin M"unch <Martin dot Muench at Uni-Bonn dot de>

The usual disclaimer applies:
If it doesn't work right that's your problem.
(Nevertheless, send an e-mail to the maintainer
 when you find an error in this package.)

This work may be distributed and/or modified under the
conditions of the LaTeX Project Public License, either
version 1.3c of this license or (at your option) any later
version. This version of this license is in
   http://www.latex-project.org/lppl/lppl-1-3c.txt
and the latest version of this license is in
   http://www.latex-project.org/lppl.txt
and version 1.3c or later is part of all distributions of
LaTeX version 2005/12/01 or later.

This work has the LPPL maintenance status "maintained".

The Current Maintainer of this work is H.-Martin Muench.

This work consists of the main source file pagecolor.dtx,
the README, and the derived files
   pagecolor.sty, pagecolor.pdf,
   pagecolor.ins, pagecolor.drv,
   pagecolor-example.tex, pagecolor-example.pdf.

\endpreamble
\let\MetaPrefix\DoubleperCent

\generate{%
  \file{pagecolor.ins}{\from{pagecolor.dtx}{install}}%
  \file{pagecolor.drv}{\from{pagecolor.dtx}{driver}}%
  \usedir{tex/latex/pagecolor}%
  \file{pagecolor.sty}{\from{pagecolor.dtx}{package}}%
  \usedir{doc/latex/pagecolor}%
  \file{pagecolor-example.tex}{\from{pagecolor.dtx}{example}}%
}

\catcode32=13\relax% active space
\let =\space%
\Msg{************************************************************************}
\Msg{*}
\Msg{* To finish the installation you have to move the following}
\Msg{* file into a directory searched by TeX:}
\Msg{*}
\Msg{*     pagecolor.sty}
\Msg{*}
\Msg{* To produce the documentation run the file `pagecolor.drv'}
\Msg{* through (pdf)LaTeX, e.g.}
\Msg{*  pdflatex pagecolor.drv}
\Msg{*  makeindex -s gind.ist pagecolor.idx}
\Msg{*  pdflatex pagecolor.drv}
\Msg{*  makeindex -s gind.ist pagecolor.idx}
\Msg{*  pdflatex pagecolor.drv}
\Msg{*}
\Msg{* At least three runs are necessary e.g. to get the}
\Msg{*  references right!}
\Msg{*}
\Msg{* Happy TeXing!}
\Msg{*}
\Msg{************************************************************************}

\endbatchfile
%</install>
%<*ignore>
\fi
%</ignore>
%
% \section{The documentation driver file}
%
% The next bit of code contains the documentation driver file for
% \TeX{}, i.\,e., the file that will produce the documentation you
% are currently reading. It will be extracted from this file by the
% \texttt{docstrip} programme. That is, run \LaTeX{} on \texttt{docstrip}
% and specify the \texttt{driver} option when \texttt{docstrip}
% asks for options.
%
%    \begin{macrocode}
%<*driver>
\NeedsTeXFormat{LaTeX2e}[2011/06/27]
\ProvidesFile{pagecolor.drv}%
  [2012/02/23 v1.0e Provides thepagecolor (HMM)]
\documentclass{ltxdoc}[2007/11/11]% v2.0u
\usepackage{holtxdoc}[2011/02/04]%  v0.21
%% pagecolor may work with earlier versions of LaTeX2e and those
%% class and package, but this was not tested.
%% Please consider updating your LaTeX, class, and package
%% to the most recent version (if they are not already the most
%% recent version).
\hypersetup{%
 pdfsubject={Providing the thepagecolor command (HMM)},%
 pdfkeywords={LaTeX, pagecolor, thepagecolor, H.-Martin Muench},%
 pdfencoding=auto,%
 pdflang={en},%
 breaklinks=true,%
 linktoc=all,%
 pdfstartview=FitH,%
 pdfpagelayout=OneColumn,%
 bookmarksnumbered=true,%
 bookmarksopen=true,%
 bookmarksopenlevel=3,%
 pdfmenubar=true,%
 pdftoolbar=true,%
 pdfwindowui=true,%
 pdfnewwindow=true%
}
\CodelineIndex
\hyphenation{printing docu-ment}
\gdef\unit#1{\mathord{\thinspace\mathrm{#1}}}%
\begin{document}
  \DocInput{pagecolor.dtx}%
\end{document}
%</driver>
%    \end{macrocode}
%
% \fi
%
% \CheckSum{137}
%
% \CharacterTable
%  {Upper-case    \A\B\C\D\E\F\G\H\I\J\K\L\M\N\O\P\Q\R\S\T\U\V\W\X\Y\Z
%   Lower-case    \a\b\c\d\e\f\g\h\i\j\k\l\m\n\o\p\q\r\s\t\u\v\w\x\y\z
%   Digits        \0\1\2\3\4\5\6\7\8\9
%   Exclamation   \!     Double quote  \"     Hash (number) \#
%   Dollar        \$     Percent       \%     Ampersand     \&
%   Acute accent  \'     Left paren    \(     Right paren   \)
%   Asterisk      \*     Plus          \+     Comma         \,
%   Minus         \-     Point         \.     Solidus       \/
%   Colon         \:     Semicolon     \;     Less than     \<
%   Equals        \=     Greater than  \>     Question mark \?
%   Commercial at \@     Left bracket  \[     Backslash     \\
%   Right bracket \]     Circumflex    \^     Underscore    \_
%   Grave accent  \`     Left brace    \{     Vertical bar  \|
%   Right brace   \}     Tilde         \~}
%
% \GetFileInfo{pagecolor.drv}
%
% \begingroup
%   \def\x{\#,\$,\^,\_,\~,\ ,\&,\{,\},\%}%
%   \makeatletter
%   \@onelevel@sanitize\x
% \expandafter\endgroup
% \expandafter\DoNotIndex\expandafter{\x}
% \expandafter\DoNotIndex\expandafter{\string\ }
% \begingroup
%   \makeatletter
%     \lccode`9=32\relax
%     \lowercase{%^^A
%       \edef\x{\noexpand\DoNotIndex{\@backslashchar9}}%^^A
%     }%^^A
%   \expandafter\endgroup\x
%
% \DoNotIndex{\,,\\}
% \DoNotIndex{\documentclass,\usepackage,\ProvidesPackage,\begin,\end}
% \DoNotIndex{\MessageBreak}
% \DoNotIndex{\NeedsTeXFormat,\DoNotIndex,\verb}
% \DoNotIndex{\def,\edef,\gdef,\global}
% \DoNotIndex{\ifx,\listfiles,\mathord,\mathrm}
% \DoNotIndex{\kvoptions,\SetupKeyvalOptions,\ProcessKeyvalOptions}
% \DoNotIndex{\bigskip,\space,\thinspace,\ldots}
% \DoNotIndex{\indent,\noindent,\newline,\linebreak,\pagebreak,\newpage}
% \DoNotIndex{\textbf,\textit,\textsf,\texttt,\textsc,\textquotedblleft,\textquotedblright}
% \DoNotIndex{\plainTeX,\TeX,\LaTeX,\pdfLaTeX}
% \DoNotIndex{\section,\color,\pagenumbering,\smallskip}
% \DoNotIndex{\@ifundefined,\@ifpackageloaded}
% \DoNotIndex{\DeclareStringOption}
%
% \title{The \xpackage{pagecolor} package}
% \date{2012/02/23 v1.0e}
% \author{H.-Martin M\"{u}nch\\\xemail{Martin.Muench at Uni-Bonn.de}}
%
% \maketitle
%
% \begin{abstract}
%  This \LaTeX{} package provides the command |\thepagecolor|,
% which gives the current page (background) colour,
% i.\,e.~the argument used with the most recent call of |\pagecolor{...}|.
% The command |\thepagecolornone| gives the same colour as |\thepagecolor|,
% except when the page background colour is
% \textquotedblleft|none|\textquotedblright. In that case |\thepagecolor| is
% |white| and |\thepagecolornone| is |none|.\\
% Similar to |\newgeometry| and |\restoregeometry| of the \xpackage{geometry}
% package\\
% |\newpagecolor{<|\textit{some colour} |>}| and |\restorepagecolor|
% are provided.
% \end{abstract}
%
% \bigskip
%
% \noindent Disclaimer for web links: The author is not responsible for any contents
% referred to in this work unless he has full knowledge of illegal contents.
% If any damage occurs by the use of information presented there, only the
% author of the respective pages might be liable, not the one who has referred
% to these pages.
%
% \bigskip
%
% \noindent {\color{green} Save per page about $200\unit{ml}$ water,
% $2\unit{g}$ CO$_{2}$ and $2\unit{g}$ wood:\\
% Therefore please print only if this is really necessary.}
%
% \pagebreak
%
% \tableofcontents
%
% \bigskip
%
% \section{Introduction\label{sec:Introduction}}
%
% \indent This \LaTeX{} package provides the command |\thepagecolor|,
% which gives the current page (background) colour,
% i.\,e.~the argument used with the most recent call of |\pagecolor{...}|.
% The package should be loaded before any package sets a page (background)
% colour, but after \xpackage{color} or \xpackage{xcolor} package.
% Its option |pagecolor={...}| is used to set the |\pagecolor{...}|.\\
% The command |\thepagecolornone| gives the same colour as |\thepagecolor|,
% except when the page background colour is
% \textquotedblleft|none|\textquotedblright (e.\,g. result of using the
% |\nopagecolor| command). In that case |\thepagecolor| is |white| and\linebreak
% |\thepagecolornone| is |none|.\\
% Similar to |\newgeometry| and |\restoregeometry| of the \xpackage{geometry}
% package\\
% |\newpagecolor{<|\textit{some colour}|>}| and |\restorepagecolor|
% are provided.
%
% \pagebreak
%
% \section{Usage}
%
% \indent Just load the package placing
% \begin{quote}
%   |\usepackage[<|\textit{option}|>]{pagecolor}|
% \end{quote}
% \noindent in the preamble of your \LaTeXe{} source file.
% This should be done before another package uses |\pagecolor|,
% but after |\nopagecolor| is defined (if that is defined in the document
% at all).\\
% Afterwards |\pagecolor{...}| can be used to change the page (background)
% colour as usual. Then |\thepagecolor| gives the current page (background)
% colour (in the same format as given with |\pagecolor{...}|).\\
% Similar to |\newgeometry| and |\restoregeometry| of the \xpackage{geometry}
% package\\
% |\newpagecolor{<|\textit{some colour} |>}| and |\restorepagecolor|
% are provided:\\
% |\newpagecolor{<|\textit{some colour} |>}| will execute
% |\pagecolor{<|\textit{some colour} |>}| and remember the page colour used before.
% |\restorecolor| (without argument) restores the page colour to the one used before
%  use of the |\newpagecolor{...}| command. When you want to change the colour for
%  just one page and do not want to (or cannot) manually determine where the page ends,\\
% |\newpagecolor{<|\textit{some colour}|>}\afterpage{\restorepagecolor}|\\
% does the trick (and requires a |\usepackage{afterpage}| in the document's
% preamble), or for short\\
% |\newcommand{\onepagecolor}[1]{%|\\
% |\newpagecolor{#1}\afterpage{\restorepagecolor}}|\\
% in the preamble and\\
% |\onepagecolor{<|\textit{some colour}|>}| in the document.
%
% \subsection{Option}
% \DescribeMacro{option}
% \indent The \xpackage{pagecolor} package takes the following option:
%
% \subsubsection{pagecolor\label{sss:pagecolor}}
% \DescribeMacro{pagecolor}
% \indent The option |pagecolor={...}| takes as value a colour. This
% could be as simple as |white| or |black|, but when e.\,g.~the
% \xpackage{xcolor} package is used (loaded before \xpackage{pagecolor}!),
% also colours like |red!50!green!20!blue| are possible.
% The default is |pagecolor={none}|.
%
% \section{Alternatives\label{sec:Alternatives}}
%
% As I neither know what exactly you want to accomplish
% when using this package (e.\,g.~hiding text), nor what resources you have
% (e.\,g.~\pdfTeX{} version), here is a list of some possible alternatives:\\
%
% \begin{description}
%  \item[-]
%  \DescribeMacro{transparent}
%  \xpackage{transparent} package: With it some object can be made
%  (fully or partially) transparent.
% \end{description}
%
% \begin{description}
%  \item[-]
%  \DescribeMacro{hrefhide}\xpackage{hrefhide} package: It allows to
%  \textquotedblleft hide\textquotedblright\ some (hyperlinked) text when
%  printing the document while keeping the layout.
% \end{description}
%
% \bigskip
%
% \noindent (You programmed or found another alternative,
%  which is available at \CTAN{}?\\
%  OK, send an e-mail to me with the name, location at \CTAN{},
%  and a short notice, and I will probably include it in
%  the list above.)\\
%
% \smallskip
%
% \noindent About how to get those packages, please see
% subsection~\ref{ss:Downloads}.
%
% \newpage
%
% \section{Example}
%
%    \begin{macrocode}
%<*example>
\documentclass[british]{article}[2007/10/19]% v1.4h
%%%%%%%%%%%%%%%%%%%%%%%%%%%%%%%%%%%%%%%%%%%%%%%%%%%%%%%%%%%%%%%%%%%%%
\usepackage[%
 extension=pdf,%
 plainpages=false,%
 pdfpagelabels=true,%
 hyperindex=false,%
 pdflang={en},%
 pdftitle={pagecolor package example},%
 pdfauthor={H.-Martin Muench},%
 pdfsubject={Example for the pagecolor package},%
 pdfkeywords={LaTeX, pagecolor, thepagecolor, page colour,%
  H.-Martin Muench},%
 pdfview=Fit,pdfstartview=Fit,%
 pdfpagelayout=SinglePage%
]{hyperref}[2012/02/06]% v6.82o
\usepackage[x11names]{xcolor}[2007/01/21]% v2.11
 % The xcolor package would not be needed for just using
 % the base colours. The color package would be sufficient for that.
\usepackage[pagecolor={LightGoldenrod1}]{pagecolor}[2012/02/23]% v1.0e

\usepackage{afterpage}[1995/10/27]% v1.08
 % The afterpage package is generally not needed,
 % but the |\newpagecolor{somecolour}\afterpage{\restorepagecolor}|
 % construct shall be demonstrated.

\usepackage{lipsum}[2011/04/14]% v1.2
 % The lipsum package is generally not needed,
 % but some blind text is needed for the example.

\gdef\unit#1{\mathord{\thinspace\mathrm{#1}}}%
\listfiles
\begin{document}
\pagenumbering{arabic}
\section*{Example for pagecolor}

This example demonstrates the use of package\newline
\textsf{pagecolor}, v1.0e as of 2012/02/23 (HMM).\newline
The used option was \verb|pagecolor={LightGoldenrod1}|.\newline
\verb|pagecolor={none}| would be the default.\newline

For more details please see the documentation!\newline

\noindent {\color{teal} Save per page about $200\unit{ml}$ water,
$2\unit{g}$ CO$_{2}$ and $2\unit{g}$ wood:\newline
Therefore please print only if this is really necessary.}\newline

The current page (background) colour is\newline
\verb|\thepagecolor|\ =\ \thepagecolor \newline
(and \verb|\thepagecolornone|\ =\ \thepagecolornone ,
which would only be different from \verb|\thepagecolor|,
when the page colour would be \verb|none|).

\pagebreak
\pagecolor{rgb:-green!40!yellow,3;green!40!yellow,2;red,1}

{\color{white} The current page (background) colour is\newline
\verb|\thepagecolor|\ =\ \thepagecolor . \newline}

{\color{\thepagecolor} And that makes this text practically invisible.
\newline}

{\color{white} Which made the preceding line of text practically
invisible.}

\pagebreak
\newpagecolor{red}

This page uses \verb|\newpagecolor{red}|.

\pagebreak
\restorepagecolor

{\color{white}And this page uses \verb|\restorepagecolor| to restore
the page colour to the value it had before the red page.}

\pagebreak
\pagecolor{none}

This page uses \verb|\pagecolor{none}|. If the \verb|\nopagecolor|
command is known (pdf\TeX and Lua\TeX; not yet for dvips, dvipdfm(x)
or Xe\TeX), the page colour is now \verb|none|, otherwise \verb|white|:
\verb|\thepagecolor|\ =\ \thepagecolor\ and
\verb|\thepagecolornone|\ =\ \thepagecolornone .

\pagebreak
\restorepagecolor

{\color{white}\verb|\restorepagecolor| restored the page colour again.}

\pagebreak
\pagecolor{green}

This page is green due to \verb|\pagecolor{green}|.

\pagebreak
\newpagecolor{blue}\afterpage{\restorepagecolor}

{\color{white}\verb|\newpagecolor{blue}\afterpage{\restorepagecolor}|%
\newline
was used here, i.\,e.~this page is blue, and the next one will
automatically have the same page colour before it was changed to blue
here (i.\,e. green).}

\smallskip
{\color{red}\textbf{\lipsum[1-11]}}
\bigskip

The page colour was changed back at the end of the page -
in mid-sentence!

\end{document}
%</example>
%    \end{macrocode}
%
% \StopEventually{}
%
% \newpage
%
% \section{The implementation}
%
% We start off by checking that we are loading into \LaTeXe{} and
% announcing the name and version of this package.
%
%    \begin{macrocode}
%<*package>
%    \end{macrocode}
%
%    \begin{macrocode}
\NeedsTeXFormat{LaTeX2e}[2011/06/27]
\ProvidesPackage{pagecolor}[2012/02/23 v1.0e
            Provides thepagecolor (HMM)]
%    \end{macrocode}
%
% A short description of the \xpackage{pagecolor} package:
%
%    \begin{macrocode}
%% Provides the \thepagecolor, \thepagecolornone, \newpagecolor{...},
%% and \restorepagecolor commands.
%    \end{macrocode}
%
% We need the \xpackage{kvoptions} package by \textsc{Heiko Oberdiek}:
%
%    \begin{macrocode}
\RequirePackage{kvoptions}[2010/12/23]% v3.10
%    \end{macrocode}
%
% and either the \xpackage{color} or the \xpackage{xcolor} package:
%
%    \begin{macrocode}
%% \RequirePackage{ either color or xcolor }:
\@ifpackageloaded{xcolor}{% xcolor loaded
   \@ifpackagelater{xcolor}{2007/01/21}{%
      % 2007/01/21, v2.11, or even more recent: OK
    }{% else: older package version
      \PackageWarning{pagecolor}{%
        It is requested version `2007/01/21' of package\MessageBreak%
        xcolor, but only an older version is available.\MessageBreak%
       }%
     }%
 }{% xcolor not loaded
   \@ifpackageloaded{color}{%
      \RequirePackage{color}[2005/11/14]% v1.0j
    }{% \else
      \PackageWarning{pagecolor}{%
        The pagecolor package must be loaded after either\MessageBreak%
        package color or after package xcolor (at your\MessageBreak%
        option). Neither package was loaded before package\MessageBreak%
        pagecolor. Loading of package xcolor will now be\MessageBreak%
        tried automatically.\MessageBreak%
        When the pagecolor package is used with option\MessageBreak%
        pagecolor using a colour requiring e. g. x11names\MessageBreak%
        option for xcolor package, this will not work!\MessageBreak%
       }
     }% \fi
   \RequirePackage{xcolor}[2007/01/21]% v2.11
  }% \fi
%    \end{macrocode}
%
% A last information for the user:
%
%    \begin{macrocode}
%% pagecolor may work with earlier versions of LaTeX and the
%% packages, but this was not tested. Please consider updating
%% your LaTeX and packages to the most recent version
%% (if they are not already the most recent version).

%    \end{macrocode}
%
% \indent See subsection~\ref{ss:Downloads} about how to get them.\\
%
% We process the options:
%
%    \begin{macrocode}
\SetupKeyvalOptions{family=pagecolor,prefix=pagecolor@}
\DeclareStringOption[none]{pagecolor}% \pagecolor@pagecolor
\ProcessKeyvalOptions*

%    \end{macrocode}
%
% We save the original |\pagecolor| command,
%
%    \begin{macrocode}
\let\origpagecolour\pagecolor

%    \end{macrocode}
%
%    \begin{macro}{\pagecolor}
% before we redefine it to include a definition of |\thepagecolor|
% and \linebreak
% |\thepagecolornone|:
%
%    \begin{macrocode}
\renewcommand{\pagecolor}[1]{\@bsphack%
  \edef\pagecolourtmpa{#1}%
  \def\pagecolourtmpb{none}%
  \ifx\pagecolourtmpa\pagecolourtmpb
    \@ifundefined{nopagecolor}{%
      \PackageError{pagecolor}{%
        pagecolor=none requested but \string\nopagecolor\space unknown%
       }{%
        \string\pagecolor{none} was used, but the command %
        \string\nopagecolor\space is undefined.\MessageBreak%
        Please use another colour. pagecolor=white will be used now.%
        \MessageBreak%
        }%
      \gdef\thepagecolor{white}%
      \gdef\thepagecolornone{white}% although it should be "none"
      \origpagecolour{white}%
     }{%
      \nopagecolor%
     }%
  \else%
    \xdef\thepagecolor{#1}%
    \xdef\thepagecolornone{#1}%
    \origpagecolour{\thepagecolornone}%
  \fi%
  \@esphack%
}

%    \end{macrocode}
%    \end{macro}
%
% |\nopagecolor| is only defined for \pdfTeX and \LuaTeX, but not yet for dvips,
% dvipdfm(x) or \XeTeX. Maybe
% \begin{verbatim}
% \@ifundefined{nopagecolor}{\newcommand{\nopagecolor}{\pagecolor{white}}}{}
% \end{verbatim}
% could be an alternative.\\
% When |\pagecolor{none}| is used and |\nopagecolor| is defined, then \\
% |\pagecolor{none}| is made into a synonym for |\nopagecolor|.\\
% If |\nopagecolor| has not been defined, nothing needs to be done.
% Otherwise we redefine |\nopagecolor|, thus that |\thepagecolor| is
% set to |white| and |\thepagecolornone| to |none| whenever
% |\nopagecolor| is used.
%
%    \begin{macrocode}
\@ifundefined{nopagecolor}{%
  \PackageWarning{pagecolor}{%
    \string\nopagecolor\space is undefined.\MessageBreak%
    (As long as it is neither used anywhere\MessageBreak%
    \space nor defined later, that will be no problem.)\MessageBreak%
   }
  \AtEndDocument{%
  \@ifundefined{nopagecolor}{% no problem
    }{%
    \PackageError{pagecolor}{%
      \string\nopagecolor\space defined after loading package%
      \MessageBreak%
      pagecolor%
      }{\string\nopagecolor\space was defined after loading the %
       pagecolor package.\MessageBreak%
       Either define \string\nopagecolor\space earlier or load the %
       pagecolor package later.\MessageBreak%
      }
    }%
  }%
  }{% \else
    \let\orignopagecolour\nopagecolor
    \renewcommand{\nopagecolor}{%
      \xdef\thepagecolor{white}
      \xdef\thepagecolornone{none}
      \orignopagecolour
      }
  }

%    \end{macrocode}
%
% If the page colour as given with option |pagecolor={...}| is |none|,
% but |\nopagecolor| is not known, |\pagecolor@pagecolor| is set to
% |white| and a warning is given.
%
%    \begin{macrocode}
\def\pagecolourtmpb{none}
\ifx\pagecolor@pagecolor\pagecolourtmpb
  \@ifundefined{nopagecolor}{%
    \PackageWarning{pagecolor}{%
      Option pagecolor=none (maybe by default) used,\MessageBreak%
      but \string\nopagecolor\space is unknown. Please use another%
      \MessageBreak%
      option value; white will be used now.\MessageBreak%
     }
    \setkeys{pagecolor}{pagecolor=white}%
  }{% ok
  }%
\fi

%    \end{macrocode}
%
% The (new) |\pagecolor| is now just carried out.
%
%    \begin{macrocode}
\pagecolor{\pagecolor@pagecolor}

%    \end{macrocode}
%
% Now the page (background) colour and |\thepagecolor| and \\
% |\thepagecolornone| are |\pagecolor@pagecolor| (or page (background) colour
% and |\thepagecolornone| are |none|=|\pagecolor@pagecolor| and
% |\thepagecolor| is |white|), and when the page (background) colour
% is changed, |\thepagecolor| and |\thepagecolornone| are changed
% accordingly.
%
% \newpage
%
%    \begin{macro}{\newpagecolor}
% There have been requests (via e-mail and at \\
% \url{http://tex.stackexchange.com/q/25137}) to change the colour of just one
% (or two) page(s) only, similar to |\newgeometry| and |\restoregeometry| of the
% \xpackage{geometry} package (\url{http://ctan.org/pkg/geometry}). Therefore
% |\newpagecolor| and |\restorepagecolor| are introduced (as suggested by
% \textsc{haoyun\_tex}):
%
%    \begin{macrocode}
\newcommand{\newpagecolor}[1]{%
\xdef\pagecolourtmpc{\thepagecolornone}%
\pagecolor{#1}%
}

%    \end{macrocode}
%    \end{macro}
%
% |\newpagecolor{<|\textit{some colour}|>}| will execute |\pagecolor{somecolour}|
% and remember the page colour used before.\\
%
%    \begin{macro}{\restorepagecolor}
%    \begin{macrocode}
\newcommand{\restorepagecolor}{\pagecolor{\pagecolourtmpc}}

%    \end{macrocode}
%    \end{macro}
%
% |\restorecolor| (without argument) restores the page colour to the one used before
% use of the |\newpagecolor{...}| command.
%
%    \begin{macrocode}
\gdef\pagecolourtmpc{\thepagecolor}

%    \end{macrocode}
% is just a precaution for |\restorecolor| being used when no |\newpagecolor{...}|
% was used before it.\\
%
% When you want to change the colour for just one page and do not want to
% (or cannot) manually determine where the page ends,\\
% |\newpagecolor{<|\textit{some colour}|>}\afterpage{\restorepagecolor}|\\
% does the trick (and requires an additional |\usepackage{afterpage}| in the
% document's preamble).
%    \begin{macrocode}
%</package>
%    \end{macrocode}
%
% \newpage
%
% \section{Installation}
%
% \subsection{Downloads\label{ss:Downloads}}
%
% Everything is available on \CTAN{}, \url{http://www.ctan.org/tex-archive/},
% but may need additional packages themselves.\\
%
% \DescribeMacro{pagecolor.dtx}
% For unpacking the |pagecolor.dtx| file and constructing the documentation
% it is required:
% \begin{description}
% \item[-] \TeX{} Format \LaTeXe{}, \url{http://www.CTAN.org/}
%
% \item[-] document class \xpackage{ltxdoc}, 2007/11/11, v2.0u,
%           \url{http://ctan.org/pkg/ltxdoc}
%
% \item[-] package \xpackage{holtxdoc}, 2011/02/04, v0.21,
%           \url{http://ctan.org/pkg/holtxdoc}
% \end{description}
%
% \DescribeMacro{pagecolor.sty}
% The |pagecolor.sty| for \LaTeXe{} (i.\,e. each document using
% the \xpackage{pagecolor} package) requires:
% \begin{description}
% \item[-] \TeX{} Format \LaTeXe{}, \url{http://www.CTAN.org/}
%
% \item[-] package \xpackage{kvoptions}, 2010/12/23, v3.10,
%           \url{http://ctan.org/pkg/kvoptions}
% \end{description}
% and either
% \begin{description}
% \item[-] package \xpackage{color}, 2005/11/14, v1.0j,
%           \url{http://ctan.org/pkg/color} (from the\\
%           \xpackage{graphics} package bundle)
% \end{description}
% or
% \begin{description}
% \item[-] package \xpackage{xcolor}, 2007/01/21, v2.11,
%           \url{http://ctan.org/pkg/xcolor}
% \end{description}
%
% \DescribeMacro{pagecolor-example.tex}
% The |pagecolor-example.tex| requires the same files as all
% documents using the \xpackage{pagecolor} package and additionally:
% \begin{description}
% \item[-] class \xpackage{article}, 2007/10/19, v1.4h, from \xpackage{classes.dtx}:\\
%           \CTAN{macros/latex/base/classes.dtx}
%
% \item[-] package \xpackage{pagecolor}, 2012/02/23, v1.0e,
%           \url{http://ctan.org/pkg/pagecolor}\\
%           (Well, it is the example file for this package, and because you are
%            reading the documentation for the \xpackage{pagecolor} package,
%            it can be assumed that you already have some version of it --
%            is it the current one?)
%
% \item[-] package \xpackage{xcolor}, 2007/01/21, v2.11,
%           \url{http://ctan.org/pkg/xcolor}\\
%           This package would not be needed for the use of just base colours only,
%           the \xpackage{color} package would be sufficient for that.
% \end{description}
%
% \DescribeMacro{transparent}
% \DescribeMacro{hrefhide}
% As possible alternatives in section \ref{sec:Alternatives} there are listed
% \begin{description}
% \item[-] package \xpackage{transparent}, 2007/01/08, v1.0, \\
%           \url{http://ctan.org/pkg/transparent}
%
% \item[-] package \xpackage{hrefhide}, 2011/04/29, v1.0f, \\
%           \url{http://ctan.org/pkg/hrefhide}
% \end{description}
%
% \DescribeMacro{Oberdiek}
% \DescribeMacro{holtxdoc}
% \DescribeMacro{kvoptions}
% All packages of \textsc{Heiko Oberdiek}'s bundle `oberdiek'
% (especially \xpackage{holtxdoc} and \xpackage{kvoptions})
% are also available in a TDS compliant ZIP archive:\\
% \CTAN{install/macros/latex/contrib/oberdiek.tds.zip}.\\
% It is probably best to download and use this, because the packages in there
% are quite probably both recent and compatible among themselves.\\
%
% \DescribeMacro{hyperref}
% \noindent \xpackage{hyperref} is not included in that bundle and needs to be
% downloaded separately,\\
% \url{http://mirror.ctan.org/install/macros/latex/contrib/hyperref.tds.zip}.\\
%
% \DescribeMacro{M\"{u}nch}
% A hyperlinked list of my (other) packages can be found at
% \url{http://www.Uni-Bonn.de/~uzs5pv/LaTeX.html}.\\
%
% \subsection{Package, unpacking TDS}
% \paragraph{Package.} This package is available on \CTAN{}
% \begin{description}
% \item[\CTAN{macros/latex/contrib/pagecolor/pagecolor.dtx}]\hspace*{0.1cm} \\
%       The source file.
% \item[\CTAN{macros/latex/contrib/pagecolor/pagecolor.pdf}]\hspace*{0.1cm} \\
%       The documentation.
% \item[\CTAN{macros/latex/contrib/pagecolor/pagecolor-example.pdf}]\hspace*{0.1cm} \\
%       The compiled example file, as it should look like.
% \item[\CTAN{macros/latex/contrib/pagecolor/README}]\hspace*{0.1cm} \\
%       The README file.
% \end{description}
% There is also a pagecolor.tds.zip available:
% \begin{description}
% \item[\CTAN{install/macros/latex/contrib/pagecolor.tds.zip}]\hspace*{0.1cm} \\
%       Everything in \xfile{TDS} compliant, compiled format.
% \end{description}
% which additionally contains\\
% \begin{tabular}{ll}
% pagecolor.ins & The installation file.\\
% pagecolor.drv & The driver to generate the documentation.\\
% pagecolor.sty & The \xext{sty}le file.\\
% pagecolor-example.tex & The example file.
% \end{tabular}
%
% \bigskip
%
% \noindent For required other packages, please see the preceding subsection.
%
% \paragraph{Unpacking.} The \xfile{.dtx} file is a self-extracting
% \docstrip{} archive. The files are extracted by running the
% \xext{.dtx} through \plainTeX{}:
% \begin{quote}
%   \verb|tex pagecolor.dtx|
% \end{quote}
%
% About generating the documentation see paragraph~\ref{GenDoc} below.\\
%
% \paragraph{TDS.} Now the different files must be moved into
% the different directories in your installation TDS tree
% (also known as \xfile{texmf} tree):
% \begin{quote}
% \def\t{^^A
% \begin{tabular}{@{}>{\ttfamily}l@{ $\rightarrow$ }>{\ttfamily}l@{}}
%   pagecolor.sty & tex/latex/pagecolor/pagecolor.sty\\
%   pagecolor.pdf & doc/latex/pagecolor/pagecolor.pdf\\
%   pagecolor-example.tex & doc/latex/pagecolor/pagecolor-example.tex\\
%   pagecolor-example.pdf & doc/latex/pagecolor/pagecolor-example.pdf\\
%   pagecolor.dtx & source/latex/pagecolor/pagecolor.dtx\\
% \end{tabular}^^A
% }^^A
% \sbox0{\t}^^A
% \ifdim\wd0>\linewidth
%   \begingroup
%     \advance\linewidth by\leftmargin
%     \advance\linewidth by\rightmargin
%   \edef\x{\endgroup
%     \def\noexpand\lw{\the\linewidth}^^A
%   }\x
%   \def\lwbox{^^A
%     \leavevmode
%     \hbox to \linewidth{^^A
%       \kern-\leftmargin\relax
%       \hss
%       \usebox0
%       \hss
%       \kern-\rightmargin\relax
%     }^^A
%   }^^A
%   \ifdim\wd0>\lw
%     \sbox0{\small\t}^^A
%     \ifdim\wd0>\linewidth
%       \ifdim\wd0>\lw
%         \sbox0{\footnotesize\t}^^A
%         \ifdim\wd0>\linewidth
%           \ifdim\wd0>\lw
%             \sbox0{\scriptsize\t}^^A
%             \ifdim\wd0>\linewidth
%               \ifdim\wd0>\lw
%                 \sbox0{\tiny\t}^^A
%                 \ifdim\wd0>\linewidth
%                   \lwbox
%                 \else
%                   \usebox0
%                 \fi
%               \else
%                 \lwbox
%               \fi
%             \else
%               \usebox0
%             \fi
%           \else
%             \lwbox
%           \fi
%         \else
%           \usebox0
%         \fi
%       \else
%         \lwbox
%       \fi
%     \else
%       \usebox0
%     \fi
%   \else
%     \lwbox
%   \fi
% \else
%   \usebox0
% \fi
% \end{quote}
% If you have a \xfile{docstrip.cfg} that configures and enables \docstrip{}'s
% \xfile{TDS} installing feature, then some files can already be in the right
% place, see the documentation of \docstrip{}.
%
% \subsection{Refresh file name databases}
%
% If your \TeX{}~distribution (\teTeX{}, \mikTeX{},\dots{}) relies on file name
% databases, you must refresh these. For example, \teTeX{} users run
% \verb|texhash| or \verb|mktexlsr|.
%
% \subsection{Some details for the interested}
%
% \paragraph{Unpacking with \LaTeX{}.}
% The \xfile{.dtx} chooses its action depending on the format:
% \begin{description}
% \item[\plainTeX:] Run \docstrip{} and extract the files.
% \item[\LaTeX:] Generate the documentation.
% \end{description}
% If you insist on using \LaTeX{} for \docstrip{} (really,
% \docstrip{} does not need \LaTeX{}), then inform the autodetect routine
% about your intention:
% \begin{quote}
%   \verb|latex \let\install=y\input{pagecolor.dtx}|
% \end{quote}
% Do not forget to quote the argument according to the demands
% of your shell.
%
% \paragraph{Generating the documentation.\label{GenDoc}}
% You can use both the \xfile{.dtx} or the \xfile{.drv} to generate
% the documentation. The process can be configured by a
% configuration file \xfile{ltxdoc.cfg}. For instance, put the following
% line into this file, if you want to have A4 as paper format:
% \begin{quote}
%   \verb|\PassOptionsToClass{a4paper}{article}|
% \end{quote}
%
% \noindent An example follows how to generate the
% documentation with \pdfLaTeX{}:
%
% \begin{quote}
%\begin{verbatim}
%pdflatex pagecolor.dtx
%makeindex -s gind.ist pagecolor.idx
%pdflatex pagecolor.dtx
%makeindex -s gind.ist pagecolor.idx
%pdflatex pagecolor.dtx
%\end{verbatim}
% \end{quote}
%
% \subsection{Compiling the example}
%
% The example file, \textsf{pagecolor-example.tex}, can be compiled via\\
% |(pdf)latex pagecolor-example.tex|.\\
%
% \section{Acknowledgements}
%
% I would like to thank \textsc{Heiko Oberdiek} for providing a~lot~(!) of
% useful packages (from which I also got everything I know about creating
% a file in \xext{dtx} format, ok, say it: copying), and the
% \Newsgroup{comp.text.tex} and \Newsgroup{de.comp.text.tex} newsgroups
% for their help in all things \TeX{}, especially all contributors to the
% discussion at
% \url{http://groups.google.com/group/comp.text.tex/browse_thread/thread/533576ebe44d60f6/e1654d70a19de63c?lnk=gst&q=Determination+of+the+current+page+colour#e1654d70a19de63c}
% \ (\textsc{H.~Oberdiek} \& \textsc{Gouailles}).\\
% I thank \textsc{haoyun\_tex} for suggesting the |\newpagecolor|/|\restorepagecolor|
% pair of commands and everyone at \url{http://tex.stackexchange.com/q/25137}
% for their contributions there.
%
% \bigskip
%
% \phantomsection
% \begin{History}\label{History}
%   \begin{Version}{2011/07/16 v1.0a}
%     \item First version discussed at \Newsgroup{comp.text.tex}.
%   \end{Version}
%   \begin{Version}{2011/08/06 v1.0b}
%     \item Changed version uploaded to \CTAN{}.
%   \end{Version}
%   \begin{Version}{2011/08/08 v1.0c}
%     \item Fixed a |\setkeys|.
%   \end{Version}
%   \begin{Version}{2012/02/01 v1.0d}
%     \item Bugfix: Obsolete installation path given in the documentation, updated.
%     \item New commands: |\newpagecolor{...}|, |\restorepagecolor|.
%     \item Update of documentation, README, and \xfile{dtx} internals.
%   \end{Version}
%   \begin{Version}{2012/02/23 v1.0e}
%     \item Fixed an error in the documentation.
%     \item Check for loading of \xpackage{color} or \xpackage{xcolor} package
%             and their versions has been changed, because \xpackage{xcolor} sets\\
%             |\@namedef{ver@color.sty}{1999/02/16}|\\
%             which gave a warning about old \xpackage{color} package even if a new
%             version was used.
%   \end{Version}
% \end{History}
%
% \bigskip
%
% When you find a mistake or have a suggestion for an improvement of this package,
% please send an e-mail to the maintainer, thanks! (Please see BUG REPORTS in the README.)
%
% \newpage
%
% \PrintIndex
%
% \Finale
\endinput|
% \end{quote}
% Do not forget to quote the argument according to the demands
% of your shell.
%
% \paragraph{Generating the documentation.\label{GenDoc}}
% You can use both the \xfile{.dtx} or the \xfile{.drv} to generate
% the documentation. The process can be configured by a
% configuration file \xfile{ltxdoc.cfg}. For instance, put the following
% line into this file, if you want to have A4 as paper format:
% \begin{quote}
%   \verb|\PassOptionsToClass{a4paper}{article}|
% \end{quote}
%
% \noindent An example follows how to generate the
% documentation with \pdfLaTeX{}:
%
% \begin{quote}
%\begin{verbatim}
%pdflatex pagecolor.dtx
%makeindex -s gind.ist pagecolor.idx
%pdflatex pagecolor.dtx
%makeindex -s gind.ist pagecolor.idx
%pdflatex pagecolor.dtx
%\end{verbatim}
% \end{quote}
%
% \subsection{Compiling the example}
%
% The example file, \textsf{pagecolor-example.tex}, can be compiled via\\
% |(pdf)latex pagecolor-example.tex|.\\
%
% \section{Acknowledgements}
%
% I would like to thank \textsc{Heiko Oberdiek} for providing a~lot~(!) of
% useful packages (from which I also got everything I know about creating
% a file in \xext{dtx} format, ok, say it: copying), and the
% \Newsgroup{comp.text.tex} and \Newsgroup{de.comp.text.tex} newsgroups
% for their help in all things \TeX{}, especially all contributors to the
% discussion at
% \url{http://groups.google.com/group/comp.text.tex/browse_thread/thread/533576ebe44d60f6/e1654d70a19de63c?lnk=gst&q=Determination+of+the+current+page+colour#e1654d70a19de63c}
% \ (\textsc{H.~Oberdiek} \& \textsc{Gouailles}).\\
% I thank \textsc{haoyun\_tex} for suggesting the |\newpagecolor|/|\restorepagecolor|
% pair of commands and everyone at \url{http://tex.stackexchange.com/q/25137}
% for their contributions there.
%
% \bigskip
%
% \phantomsection
% \begin{History}\label{History}
%   \begin{Version}{2011/07/16 v1.0a}
%     \item First version discussed at \Newsgroup{comp.text.tex}.
%   \end{Version}
%   \begin{Version}{2011/08/06 v1.0b}
%     \item Changed version uploaded to \CTAN{}.
%   \end{Version}
%   \begin{Version}{2011/08/08 v1.0c}
%     \item Fixed a |\setkeys|.
%   \end{Version}
%   \begin{Version}{2012/02/01 v1.0d}
%     \item Bugfix: Obsolete installation path given in the documentation, updated.
%     \item New commands: |\newpagecolor{...}|, |\restorepagecolor|.
%     \item Update of documentation, README, and \xfile{dtx} internals.
%   \end{Version}
%   \begin{Version}{2012/02/23 v1.0e}
%     \item Fixed an error in the documentation.
%     \item Check for loading of \xpackage{color} or \xpackage{xcolor} package
%             and their versions has been changed, because \xpackage{xcolor} sets\\
%             |\@namedef{ver@color.sty}{1999/02/16}|\\
%             which gave a warning about old \xpackage{color} package even if a new
%             version was used.
%   \end{Version}
% \end{History}
%
% \bigskip
%
% When you find a mistake or have a suggestion for an improvement of this package,
% please send an e-mail to the maintainer, thanks! (Please see BUG REPORTS in the README.)
%
% \newpage
%
% \PrintIndex
%
% \Finale
\endinput
%        (quote the arguments according to the demands of your shell)
%
% Documentation:
%    (a) If pagecolor.drv is present:
%           (pdf)latex pagecolor.drv
%           makeindex -s gind.ist pagecolor.idx
%           (pdf)latex pagecolor.drv
%           makeindex -s gind.ist pagecolor.idx
%           (pdf)latex pagecolor.drv
%    (b) Without pagecolor.drv:
%           (pdf)latex pagecolor.dtx
%           makeindex -s gind.ist pagecolor.idx
%           (pdf)latex pagecolor.dtx
%           makeindex -s gind.ist pagecolor.idx
%           (pdf)latex pagecolor.dtx
%
%    The class ltxdoc loads the configuration file ltxdoc.cfg
%    if available. Here you can specify further options, e.g.
%    use DIN A4 as paper format:
%       \PassOptionsToClass{a4paper}{article}
%
% Installation:
%    TDS:tex/latex/pagecolor/pagecolor.sty
%    TDS:doc/latex/pagecolor/pagecolor.pdf
%    TDS:doc/latex/pagecolor/pagecolor-example.tex
%    TDS:doc/latex/pagecolor/pagecolor-example.pdf
%    TDS:source/latex/pagecolor/pagecolor.dtx
%
%<*ignore>
\begingroup
  \catcode123=1 %
  \catcode125=2 %
  \def\x{LaTeX2e}%
\expandafter\endgroup
\ifcase 0\ifx\install y1\fi\expandafter
         \ifx\csname processbatchFile\endcsname\relax\else1\fi
         \ifx\fmtname\x\else 1\fi\relax
\else\csname fi\endcsname
%</ignore>
%<*install>
\input docstrip.tex
\Msg{*****************************************************************}
\Msg{* Installation                                                   }
\Msg{* Package: pagecolor 2012/02/23 v1.0e Provides thepagecolor (HMM)}
\Msg{*****************************************************************}

\keepsilent
\askforoverwritefalse

\let\MetaPrefix\relax
\preamble

This is a generated file.

Project: pagecolor
Version: 2012/02/23 v1.0e

Copyright (C) 2011 - 2012 by
    H.-Martin M"unch <Martin dot Muench at Uni-Bonn dot de>

The usual disclaimer applies:
If it doesn't work right that's your problem.
(Nevertheless, send an e-mail to the maintainer
 when you find an error in this package.)

This work may be distributed and/or modified under the
conditions of the LaTeX Project Public License, either
version 1.3c of this license or (at your option) any later
version. This version of this license is in
   http://www.latex-project.org/lppl/lppl-1-3c.txt
and the latest version of this license is in
   http://www.latex-project.org/lppl.txt
and version 1.3c or later is part of all distributions of
LaTeX version 2005/12/01 or later.

This work has the LPPL maintenance status "maintained".

The Current Maintainer of this work is H.-Martin Muench.

This work consists of the main source file pagecolor.dtx,
the README, and the derived files
   pagecolor.sty, pagecolor.pdf,
   pagecolor.ins, pagecolor.drv,
   pagecolor-example.tex, pagecolor-example.pdf.

\endpreamble
\let\MetaPrefix\DoubleperCent

\generate{%
  \file{pagecolor.ins}{\from{pagecolor.dtx}{install}}%
  \file{pagecolor.drv}{\from{pagecolor.dtx}{driver}}%
  \usedir{tex/latex/pagecolor}%
  \file{pagecolor.sty}{\from{pagecolor.dtx}{package}}%
  \usedir{doc/latex/pagecolor}%
  \file{pagecolor-example.tex}{\from{pagecolor.dtx}{example}}%
}

\catcode32=13\relax% active space
\let =\space%
\Msg{************************************************************************}
\Msg{*}
\Msg{* To finish the installation you have to move the following}
\Msg{* file into a directory searched by TeX:}
\Msg{*}
\Msg{*     pagecolor.sty}
\Msg{*}
\Msg{* To produce the documentation run the file `pagecolor.drv'}
\Msg{* through (pdf)LaTeX, e.g.}
\Msg{*  pdflatex pagecolor.drv}
\Msg{*  makeindex -s gind.ist pagecolor.idx}
\Msg{*  pdflatex pagecolor.drv}
\Msg{*  makeindex -s gind.ist pagecolor.idx}
\Msg{*  pdflatex pagecolor.drv}
\Msg{*}
\Msg{* At least three runs are necessary e.g. to get the}
\Msg{*  references right!}
\Msg{*}
\Msg{* Happy TeXing!}
\Msg{*}
\Msg{************************************************************************}

\endbatchfile
%</install>
%<*ignore>
\fi
%</ignore>
%
% \section{The documentation driver file}
%
% The next bit of code contains the documentation driver file for
% \TeX{}, i.\,e., the file that will produce the documentation you
% are currently reading. It will be extracted from this file by the
% \texttt{docstrip} programme. That is, run \LaTeX{} on \texttt{docstrip}
% and specify the \texttt{driver} option when \texttt{docstrip}
% asks for options.
%
%    \begin{macrocode}
%<*driver>
\NeedsTeXFormat{LaTeX2e}[2011/06/27]
\ProvidesFile{pagecolor.drv}%
  [2012/02/23 v1.0e Provides thepagecolor (HMM)]
\documentclass{ltxdoc}[2007/11/11]% v2.0u
\usepackage{holtxdoc}[2011/02/04]%  v0.21
%% pagecolor may work with earlier versions of LaTeX2e and those
%% class and package, but this was not tested.
%% Please consider updating your LaTeX, class, and package
%% to the most recent version (if they are not already the most
%% recent version).
\hypersetup{%
 pdfsubject={Providing the thepagecolor command (HMM)},%
 pdfkeywords={LaTeX, pagecolor, thepagecolor, H.-Martin Muench},%
 pdfencoding=auto,%
 pdflang={en},%
 breaklinks=true,%
 linktoc=all,%
 pdfstartview=FitH,%
 pdfpagelayout=OneColumn,%
 bookmarksnumbered=true,%
 bookmarksopen=true,%
 bookmarksopenlevel=3,%
 pdfmenubar=true,%
 pdftoolbar=true,%
 pdfwindowui=true,%
 pdfnewwindow=true%
}
\CodelineIndex
\hyphenation{printing docu-ment}
\gdef\unit#1{\mathord{\thinspace\mathrm{#1}}}%
\begin{document}
  \DocInput{pagecolor.dtx}%
\end{document}
%</driver>
%    \end{macrocode}
%
% \fi
%
% \CheckSum{137}
%
% \CharacterTable
%  {Upper-case    \A\B\C\D\E\F\G\H\I\J\K\L\M\N\O\P\Q\R\S\T\U\V\W\X\Y\Z
%   Lower-case    \a\b\c\d\e\f\g\h\i\j\k\l\m\n\o\p\q\r\s\t\u\v\w\x\y\z
%   Digits        \0\1\2\3\4\5\6\7\8\9
%   Exclamation   \!     Double quote  \"     Hash (number) \#
%   Dollar        \$     Percent       \%     Ampersand     \&
%   Acute accent  \'     Left paren    \(     Right paren   \)
%   Asterisk      \*     Plus          \+     Comma         \,
%   Minus         \-     Point         \.     Solidus       \/
%   Colon         \:     Semicolon     \;     Less than     \<
%   Equals        \=     Greater than  \>     Question mark \?
%   Commercial at \@     Left bracket  \[     Backslash     \\
%   Right bracket \]     Circumflex    \^     Underscore    \_
%   Grave accent  \`     Left brace    \{     Vertical bar  \|
%   Right brace   \}     Tilde         \~}
%
% \GetFileInfo{pagecolor.drv}
%
% \begingroup
%   \def\x{\#,\$,\^,\_,\~,\ ,\&,\{,\},\%}%
%   \makeatletter
%   \@onelevel@sanitize\x
% \expandafter\endgroup
% \expandafter\DoNotIndex\expandafter{\x}
% \expandafter\DoNotIndex\expandafter{\string\ }
% \begingroup
%   \makeatletter
%     \lccode`9=32\relax
%     \lowercase{%^^A
%       \edef\x{\noexpand\DoNotIndex{\@backslashchar9}}%^^A
%     }%^^A
%   \expandafter\endgroup\x
%
% \DoNotIndex{\,,\\}
% \DoNotIndex{\documentclass,\usepackage,\ProvidesPackage,\begin,\end}
% \DoNotIndex{\MessageBreak}
% \DoNotIndex{\NeedsTeXFormat,\DoNotIndex,\verb}
% \DoNotIndex{\def,\edef,\gdef,\global}
% \DoNotIndex{\ifx,\listfiles,\mathord,\mathrm}
% \DoNotIndex{\kvoptions,\SetupKeyvalOptions,\ProcessKeyvalOptions}
% \DoNotIndex{\bigskip,\space,\thinspace,\ldots}
% \DoNotIndex{\indent,\noindent,\newline,\linebreak,\pagebreak,\newpage}
% \DoNotIndex{\textbf,\textit,\textsf,\texttt,\textsc,\textquotedblleft,\textquotedblright}
% \DoNotIndex{\plainTeX,\TeX,\LaTeX,\pdfLaTeX}
% \DoNotIndex{\section,\color,\pagenumbering,\smallskip}
% \DoNotIndex{\@ifundefined,\@ifpackageloaded}
% \DoNotIndex{\DeclareStringOption}
%
% \title{The \xpackage{pagecolor} package}
% \date{2012/02/23 v1.0e}
% \author{H.-Martin M\"{u}nch\\\xemail{Martin.Muench at Uni-Bonn.de}}
%
% \maketitle
%
% \begin{abstract}
%  This \LaTeX{} package provides the command |\thepagecolor|,
% which gives the current page (background) colour,
% i.\,e.~the argument used with the most recent call of |\pagecolor{...}|.
% The command |\thepagecolornone| gives the same colour as |\thepagecolor|,
% except when the page background colour is
% \textquotedblleft|none|\textquotedblright. In that case |\thepagecolor| is
% |white| and |\thepagecolornone| is |none|.\\
% Similar to |\newgeometry| and |\restoregeometry| of the \xpackage{geometry}
% package\\
% |\newpagecolor{<|\textit{some colour} |>}| and |\restorepagecolor|
% are provided.
% \end{abstract}
%
% \bigskip
%
% \noindent Disclaimer for web links: The author is not responsible for any contents
% referred to in this work unless he has full knowledge of illegal contents.
% If any damage occurs by the use of information presented there, only the
% author of the respective pages might be liable, not the one who has referred
% to these pages.
%
% \bigskip
%
% \noindent {\color{green} Save per page about $200\unit{ml}$ water,
% $2\unit{g}$ CO$_{2}$ and $2\unit{g}$ wood:\\
% Therefore please print only if this is really necessary.}
%
% \pagebreak
%
% \tableofcontents
%
% \bigskip
%
% \section{Introduction\label{sec:Introduction}}
%
% \indent This \LaTeX{} package provides the command |\thepagecolor|,
% which gives the current page (background) colour,
% i.\,e.~the argument used with the most recent call of |\pagecolor{...}|.
% The package should be loaded before any package sets a page (background)
% colour, but after \xpackage{color} or \xpackage{xcolor} package.
% Its option |pagecolor={...}| is used to set the |\pagecolor{...}|.\\
% The command |\thepagecolornone| gives the same colour as |\thepagecolor|,
% except when the page background colour is
% \textquotedblleft|none|\textquotedblright (e.\,g. result of using the
% |\nopagecolor| command). In that case |\thepagecolor| is |white| and\linebreak
% |\thepagecolornone| is |none|.\\
% Similar to |\newgeometry| and |\restoregeometry| of the \xpackage{geometry}
% package\\
% |\newpagecolor{<|\textit{some colour}|>}| and |\restorepagecolor|
% are provided.
%
% \pagebreak
%
% \section{Usage}
%
% \indent Just load the package placing
% \begin{quote}
%   |\usepackage[<|\textit{option}|>]{pagecolor}|
% \end{quote}
% \noindent in the preamble of your \LaTeXe{} source file.
% This should be done before another package uses |\pagecolor|,
% but after |\nopagecolor| is defined (if that is defined in the document
% at all).\\
% Afterwards |\pagecolor{...}| can be used to change the page (background)
% colour as usual. Then |\thepagecolor| gives the current page (background)
% colour (in the same format as given with |\pagecolor{...}|).\\
% Similar to |\newgeometry| and |\restoregeometry| of the \xpackage{geometry}
% package\\
% |\newpagecolor{<|\textit{some colour} |>}| and |\restorepagecolor|
% are provided:\\
% |\newpagecolor{<|\textit{some colour} |>}| will execute
% |\pagecolor{<|\textit{some colour} |>}| and remember the page colour used before.
% |\restorecolor| (without argument) restores the page colour to the one used before
%  use of the |\newpagecolor{...}| command. When you want to change the colour for
%  just one page and do not want to (or cannot) manually determine where the page ends,\\
% |\newpagecolor{<|\textit{some colour}|>}\afterpage{\restorepagecolor}|\\
% does the trick (and requires a |\usepackage{afterpage}| in the document's
% preamble), or for short\\
% |\newcommand{\onepagecolor}[1]{%|\\
% |\newpagecolor{#1}\afterpage{\restorepagecolor}}|\\
% in the preamble and\\
% |\onepagecolor{<|\textit{some colour}|>}| in the document.
%
% \subsection{Option}
% \DescribeMacro{option}
% \indent The \xpackage{pagecolor} package takes the following option:
%
% \subsubsection{pagecolor\label{sss:pagecolor}}
% \DescribeMacro{pagecolor}
% \indent The option |pagecolor={...}| takes as value a colour. This
% could be as simple as |white| or |black|, but when e.\,g.~the
% \xpackage{xcolor} package is used (loaded before \xpackage{pagecolor}!),
% also colours like |red!50!green!20!blue| are possible.
% The default is |pagecolor={none}|.
%
% \section{Alternatives\label{sec:Alternatives}}
%
% As I neither know what exactly you want to accomplish
% when using this package (e.\,g.~hiding text), nor what resources you have
% (e.\,g.~\pdfTeX{} version), here is a list of some possible alternatives:\\
%
% \begin{description}
%  \item[-]
%  \DescribeMacro{transparent}
%  \xpackage{transparent} package: With it some object can be made
%  (fully or partially) transparent.
% \end{description}
%
% \begin{description}
%  \item[-]
%  \DescribeMacro{hrefhide}\xpackage{hrefhide} package: It allows to
%  \textquotedblleft hide\textquotedblright\ some (hyperlinked) text when
%  printing the document while keeping the layout.
% \end{description}
%
% \bigskip
%
% \noindent (You programmed or found another alternative,
%  which is available at \CTAN{}?\\
%  OK, send an e-mail to me with the name, location at \CTAN{},
%  and a short notice, and I will probably include it in
%  the list above.)\\
%
% \smallskip
%
% \noindent About how to get those packages, please see
% subsection~\ref{ss:Downloads}.
%
% \newpage
%
% \section{Example}
%
%    \begin{macrocode}
%<*example>
\documentclass[british]{article}[2007/10/19]% v1.4h
%%%%%%%%%%%%%%%%%%%%%%%%%%%%%%%%%%%%%%%%%%%%%%%%%%%%%%%%%%%%%%%%%%%%%
\usepackage[%
 extension=pdf,%
 plainpages=false,%
 pdfpagelabels=true,%
 hyperindex=false,%
 pdflang={en},%
 pdftitle={pagecolor package example},%
 pdfauthor={H.-Martin Muench},%
 pdfsubject={Example for the pagecolor package},%
 pdfkeywords={LaTeX, pagecolor, thepagecolor, page colour,%
  H.-Martin Muench},%
 pdfview=Fit,pdfstartview=Fit,%
 pdfpagelayout=SinglePage%
]{hyperref}[2012/02/06]% v6.82o
\usepackage[x11names]{xcolor}[2007/01/21]% v2.11
 % The xcolor package would not be needed for just using
 % the base colours. The color package would be sufficient for that.
\usepackage[pagecolor={LightGoldenrod1}]{pagecolor}[2012/02/23]% v1.0e

\usepackage{afterpage}[1995/10/27]% v1.08
 % The afterpage package is generally not needed,
 % but the |\newpagecolor{somecolour}\afterpage{\restorepagecolor}|
 % construct shall be demonstrated.

\usepackage{lipsum}[2011/04/14]% v1.2
 % The lipsum package is generally not needed,
 % but some blind text is needed for the example.

\gdef\unit#1{\mathord{\thinspace\mathrm{#1}}}%
\listfiles
\begin{document}
\pagenumbering{arabic}
\section*{Example for pagecolor}

This example demonstrates the use of package\newline
\textsf{pagecolor}, v1.0e as of 2012/02/23 (HMM).\newline
The used option was \verb|pagecolor={LightGoldenrod1}|.\newline
\verb|pagecolor={none}| would be the default.\newline

For more details please see the documentation!\newline

\noindent {\color{teal} Save per page about $200\unit{ml}$ water,
$2\unit{g}$ CO$_{2}$ and $2\unit{g}$ wood:\newline
Therefore please print only if this is really necessary.}\newline

The current page (background) colour is\newline
\verb|\thepagecolor|\ =\ \thepagecolor \newline
(and \verb|\thepagecolornone|\ =\ \thepagecolornone ,
which would only be different from \verb|\thepagecolor|,
when the page colour would be \verb|none|).

\pagebreak
\pagecolor{rgb:-green!40!yellow,3;green!40!yellow,2;red,1}

{\color{white} The current page (background) colour is\newline
\verb|\thepagecolor|\ =\ \thepagecolor . \newline}

{\color{\thepagecolor} And that makes this text practically invisible.
\newline}

{\color{white} Which made the preceding line of text practically
invisible.}

\pagebreak
\newpagecolor{red}

This page uses \verb|\newpagecolor{red}|.

\pagebreak
\restorepagecolor

{\color{white}And this page uses \verb|\restorepagecolor| to restore
the page colour to the value it had before the red page.}

\pagebreak
\pagecolor{none}

This page uses \verb|\pagecolor{none}|. If the \verb|\nopagecolor|
command is known (pdf\TeX and Lua\TeX; not yet for dvips, dvipdfm(x)
or Xe\TeX), the page colour is now \verb|none|, otherwise \verb|white|:
\verb|\thepagecolor|\ =\ \thepagecolor\ and
\verb|\thepagecolornone|\ =\ \thepagecolornone .

\pagebreak
\restorepagecolor

{\color{white}\verb|\restorepagecolor| restored the page colour again.}

\pagebreak
\pagecolor{green}

This page is green due to \verb|\pagecolor{green}|.

\pagebreak
\newpagecolor{blue}\afterpage{\restorepagecolor}

{\color{white}\verb|\newpagecolor{blue}\afterpage{\restorepagecolor}|%
\newline
was used here, i.\,e.~this page is blue, and the next one will
automatically have the same page colour before it was changed to blue
here (i.\,e. green).}

\smallskip
{\color{red}\textbf{\lipsum[1-11]}}
\bigskip

The page colour was changed back at the end of the page -
in mid-sentence!

\end{document}
%</example>
%    \end{macrocode}
%
% \StopEventually{}
%
% \newpage
%
% \section{The implementation}
%
% We start off by checking that we are loading into \LaTeXe{} and
% announcing the name and version of this package.
%
%    \begin{macrocode}
%<*package>
%    \end{macrocode}
%
%    \begin{macrocode}
\NeedsTeXFormat{LaTeX2e}[2011/06/27]
\ProvidesPackage{pagecolor}[2012/02/23 v1.0e
            Provides thepagecolor (HMM)]
%    \end{macrocode}
%
% A short description of the \xpackage{pagecolor} package:
%
%    \begin{macrocode}
%% Provides the \thepagecolor, \thepagecolornone, \newpagecolor{...},
%% and \restorepagecolor commands.
%    \end{macrocode}
%
% We need the \xpackage{kvoptions} package by \textsc{Heiko Oberdiek}:
%
%    \begin{macrocode}
\RequirePackage{kvoptions}[2010/12/23]% v3.10
%    \end{macrocode}
%
% and either the \xpackage{color} or the \xpackage{xcolor} package:
%
%    \begin{macrocode}
%% \RequirePackage{ either color or xcolor }:
\@ifpackageloaded{xcolor}{% xcolor loaded
   \@ifpackagelater{xcolor}{2007/01/21}{%
      % 2007/01/21, v2.11, or even more recent: OK
    }{% else: older package version
      \PackageWarning{pagecolor}{%
        It is requested version `2007/01/21' of package\MessageBreak%
        xcolor, but only an older version is available.\MessageBreak%
       }%
     }%
 }{% xcolor not loaded
   \@ifpackageloaded{color}{%
      \RequirePackage{color}[2005/11/14]% v1.0j
    }{% \else
      \PackageWarning{pagecolor}{%
        The pagecolor package must be loaded after either\MessageBreak%
        package color or after package xcolor (at your\MessageBreak%
        option). Neither package was loaded before package\MessageBreak%
        pagecolor. Loading of package xcolor will now be\MessageBreak%
        tried automatically.\MessageBreak%
        When the pagecolor package is used with option\MessageBreak%
        pagecolor using a colour requiring e. g. x11names\MessageBreak%
        option for xcolor package, this will not work!\MessageBreak%
       }
     }% \fi
   \RequirePackage{xcolor}[2007/01/21]% v2.11
  }% \fi
%    \end{macrocode}
%
% A last information for the user:
%
%    \begin{macrocode}
%% pagecolor may work with earlier versions of LaTeX and the
%% packages, but this was not tested. Please consider updating
%% your LaTeX and packages to the most recent version
%% (if they are not already the most recent version).

%    \end{macrocode}
%
% \indent See subsection~\ref{ss:Downloads} about how to get them.\\
%
% We process the options:
%
%    \begin{macrocode}
\SetupKeyvalOptions{family=pagecolor,prefix=pagecolor@}
\DeclareStringOption[none]{pagecolor}% \pagecolor@pagecolor
\ProcessKeyvalOptions*

%    \end{macrocode}
%
% We save the original |\pagecolor| command,
%
%    \begin{macrocode}
\let\origpagecolour\pagecolor

%    \end{macrocode}
%
%    \begin{macro}{\pagecolor}
% before we redefine it to include a definition of |\thepagecolor|
% and \linebreak
% |\thepagecolornone|:
%
%    \begin{macrocode}
\renewcommand{\pagecolor}[1]{\@bsphack%
  \edef\pagecolourtmpa{#1}%
  \def\pagecolourtmpb{none}%
  \ifx\pagecolourtmpa\pagecolourtmpb
    \@ifundefined{nopagecolor}{%
      \PackageError{pagecolor}{%
        pagecolor=none requested but \string\nopagecolor\space unknown%
       }{%
        \string\pagecolor{none} was used, but the command %
        \string\nopagecolor\space is undefined.\MessageBreak%
        Please use another colour. pagecolor=white will be used now.%
        \MessageBreak%
        }%
      \gdef\thepagecolor{white}%
      \gdef\thepagecolornone{white}% although it should be "none"
      \origpagecolour{white}%
     }{%
      \nopagecolor%
     }%
  \else%
    \xdef\thepagecolor{#1}%
    \xdef\thepagecolornone{#1}%
    \origpagecolour{\thepagecolornone}%
  \fi%
  \@esphack%
}

%    \end{macrocode}
%    \end{macro}
%
% |\nopagecolor| is only defined for \pdfTeX and \LuaTeX, but not yet for dvips,
% dvipdfm(x) or \XeTeX. Maybe
% \begin{verbatim}
% \@ifundefined{nopagecolor}{\newcommand{\nopagecolor}{\pagecolor{white}}}{}
% \end{verbatim}
% could be an alternative.\\
% When |\pagecolor{none}| is used and |\nopagecolor| is defined, then \\
% |\pagecolor{none}| is made into a synonym for |\nopagecolor|.\\
% If |\nopagecolor| has not been defined, nothing needs to be done.
% Otherwise we redefine |\nopagecolor|, thus that |\thepagecolor| is
% set to |white| and |\thepagecolornone| to |none| whenever
% |\nopagecolor| is used.
%
%    \begin{macrocode}
\@ifundefined{nopagecolor}{%
  \PackageWarning{pagecolor}{%
    \string\nopagecolor\space is undefined.\MessageBreak%
    (As long as it is neither used anywhere\MessageBreak%
    \space nor defined later, that will be no problem.)\MessageBreak%
   }
  \AtEndDocument{%
  \@ifundefined{nopagecolor}{% no problem
    }{%
    \PackageError{pagecolor}{%
      \string\nopagecolor\space defined after loading package%
      \MessageBreak%
      pagecolor%
      }{\string\nopagecolor\space was defined after loading the %
       pagecolor package.\MessageBreak%
       Either define \string\nopagecolor\space earlier or load the %
       pagecolor package later.\MessageBreak%
      }
    }%
  }%
  }{% \else
    \let\orignopagecolour\nopagecolor
    \renewcommand{\nopagecolor}{%
      \xdef\thepagecolor{white}
      \xdef\thepagecolornone{none}
      \orignopagecolour
      }
  }

%    \end{macrocode}
%
% If the page colour as given with option |pagecolor={...}| is |none|,
% but |\nopagecolor| is not known, |\pagecolor@pagecolor| is set to
% |white| and a warning is given.
%
%    \begin{macrocode}
\def\pagecolourtmpb{none}
\ifx\pagecolor@pagecolor\pagecolourtmpb
  \@ifundefined{nopagecolor}{%
    \PackageWarning{pagecolor}{%
      Option pagecolor=none (maybe by default) used,\MessageBreak%
      but \string\nopagecolor\space is unknown. Please use another%
      \MessageBreak%
      option value; white will be used now.\MessageBreak%
     }
    \setkeys{pagecolor}{pagecolor=white}%
  }{% ok
  }%
\fi

%    \end{macrocode}
%
% The (new) |\pagecolor| is now just carried out.
%
%    \begin{macrocode}
\pagecolor{\pagecolor@pagecolor}

%    \end{macrocode}
%
% Now the page (background) colour and |\thepagecolor| and \\
% |\thepagecolornone| are |\pagecolor@pagecolor| (or page (background) colour
% and |\thepagecolornone| are |none|=|\pagecolor@pagecolor| and
% |\thepagecolor| is |white|), and when the page (background) colour
% is changed, |\thepagecolor| and |\thepagecolornone| are changed
% accordingly.
%
% \newpage
%
%    \begin{macro}{\newpagecolor}
% There have been requests (via e-mail and at \\
% \url{http://tex.stackexchange.com/q/25137}) to change the colour of just one
% (or two) page(s) only, similar to |\newgeometry| and |\restoregeometry| of the
% \xpackage{geometry} package (\url{http://ctan.org/pkg/geometry}). Therefore
% |\newpagecolor| and |\restorepagecolor| are introduced (as suggested by
% \textsc{haoyun\_tex}):
%
%    \begin{macrocode}
\newcommand{\newpagecolor}[1]{%
\xdef\pagecolourtmpc{\thepagecolornone}%
\pagecolor{#1}%
}

%    \end{macrocode}
%    \end{macro}
%
% |\newpagecolor{<|\textit{some colour}|>}| will execute |\pagecolor{somecolour}|
% and remember the page colour used before.\\
%
%    \begin{macro}{\restorepagecolor}
%    \begin{macrocode}
\newcommand{\restorepagecolor}{\pagecolor{\pagecolourtmpc}}

%    \end{macrocode}
%    \end{macro}
%
% |\restorecolor| (without argument) restores the page colour to the one used before
% use of the |\newpagecolor{...}| command.
%
%    \begin{macrocode}
\gdef\pagecolourtmpc{\thepagecolor}

%    \end{macrocode}
% is just a precaution for |\restorecolor| being used when no |\newpagecolor{...}|
% was used before it.\\
%
% When you want to change the colour for just one page and do not want to
% (or cannot) manually determine where the page ends,\\
% |\newpagecolor{<|\textit{some colour}|>}\afterpage{\restorepagecolor}|\\
% does the trick (and requires an additional |\usepackage{afterpage}| in the
% document's preamble).
%    \begin{macrocode}
%</package>
%    \end{macrocode}
%
% \newpage
%
% \section{Installation}
%
% \subsection{Downloads\label{ss:Downloads}}
%
% Everything is available on \CTAN{}, \url{http://www.ctan.org/tex-archive/},
% but may need additional packages themselves.\\
%
% \DescribeMacro{pagecolor.dtx}
% For unpacking the |pagecolor.dtx| file and constructing the documentation
% it is required:
% \begin{description}
% \item[-] \TeX{} Format \LaTeXe{}, \url{http://www.CTAN.org/}
%
% \item[-] document class \xpackage{ltxdoc}, 2007/11/11, v2.0u,
%           \url{http://ctan.org/pkg/ltxdoc}
%
% \item[-] package \xpackage{holtxdoc}, 2011/02/04, v0.21,
%           \url{http://ctan.org/pkg/holtxdoc}
% \end{description}
%
% \DescribeMacro{pagecolor.sty}
% The |pagecolor.sty| for \LaTeXe{} (i.\,e. each document using
% the \xpackage{pagecolor} package) requires:
% \begin{description}
% \item[-] \TeX{} Format \LaTeXe{}, \url{http://www.CTAN.org/}
%
% \item[-] package \xpackage{kvoptions}, 2010/12/23, v3.10,
%           \url{http://ctan.org/pkg/kvoptions}
% \end{description}
% and either
% \begin{description}
% \item[-] package \xpackage{color}, 2005/11/14, v1.0j,
%           \url{http://ctan.org/pkg/color} (from the\\
%           \xpackage{graphics} package bundle)
% \end{description}
% or
% \begin{description}
% \item[-] package \xpackage{xcolor}, 2007/01/21, v2.11,
%           \url{http://ctan.org/pkg/xcolor}
% \end{description}
%
% \DescribeMacro{pagecolor-example.tex}
% The |pagecolor-example.tex| requires the same files as all
% documents using the \xpackage{pagecolor} package and additionally:
% \begin{description}
% \item[-] class \xpackage{article}, 2007/10/19, v1.4h, from \xpackage{classes.dtx}:\\
%           \CTAN{macros/latex/base/classes.dtx}
%
% \item[-] package \xpackage{pagecolor}, 2012/02/23, v1.0e,
%           \url{http://ctan.org/pkg/pagecolor}\\
%           (Well, it is the example file for this package, and because you are
%            reading the documentation for the \xpackage{pagecolor} package,
%            it can be assumed that you already have some version of it --
%            is it the current one?)
%
% \item[-] package \xpackage{xcolor}, 2007/01/21, v2.11,
%           \url{http://ctan.org/pkg/xcolor}\\
%           This package would not be needed for the use of just base colours only,
%           the \xpackage{color} package would be sufficient for that.
% \end{description}
%
% \DescribeMacro{transparent}
% \DescribeMacro{hrefhide}
% As possible alternatives in section \ref{sec:Alternatives} there are listed
% \begin{description}
% \item[-] package \xpackage{transparent}, 2007/01/08, v1.0, \\
%           \url{http://ctan.org/pkg/transparent}
%
% \item[-] package \xpackage{hrefhide}, 2011/04/29, v1.0f, \\
%           \url{http://ctan.org/pkg/hrefhide}
% \end{description}
%
% \DescribeMacro{Oberdiek}
% \DescribeMacro{holtxdoc}
% \DescribeMacro{kvoptions}
% All packages of \textsc{Heiko Oberdiek}'s bundle `oberdiek'
% (especially \xpackage{holtxdoc} and \xpackage{kvoptions})
% are also available in a TDS compliant ZIP archive:\\
% \CTAN{install/macros/latex/contrib/oberdiek.tds.zip}.\\
% It is probably best to download and use this, because the packages in there
% are quite probably both recent and compatible among themselves.\\
%
% \DescribeMacro{hyperref}
% \noindent \xpackage{hyperref} is not included in that bundle and needs to be
% downloaded separately,\\
% \url{http://mirror.ctan.org/install/macros/latex/contrib/hyperref.tds.zip}.\\
%
% \DescribeMacro{M\"{u}nch}
% A hyperlinked list of my (other) packages can be found at
% \url{http://www.Uni-Bonn.de/~uzs5pv/LaTeX.html}.\\
%
% \subsection{Package, unpacking TDS}
% \paragraph{Package.} This package is available on \CTAN{}
% \begin{description}
% \item[\CTAN{macros/latex/contrib/pagecolor/pagecolor.dtx}]\hspace*{0.1cm} \\
%       The source file.
% \item[\CTAN{macros/latex/contrib/pagecolor/pagecolor.pdf}]\hspace*{0.1cm} \\
%       The documentation.
% \item[\CTAN{macros/latex/contrib/pagecolor/pagecolor-example.pdf}]\hspace*{0.1cm} \\
%       The compiled example file, as it should look like.
% \item[\CTAN{macros/latex/contrib/pagecolor/README}]\hspace*{0.1cm} \\
%       The README file.
% \end{description}
% There is also a pagecolor.tds.zip available:
% \begin{description}
% \item[\CTAN{install/macros/latex/contrib/pagecolor.tds.zip}]\hspace*{0.1cm} \\
%       Everything in \xfile{TDS} compliant, compiled format.
% \end{description}
% which additionally contains\\
% \begin{tabular}{ll}
% pagecolor.ins & The installation file.\\
% pagecolor.drv & The driver to generate the documentation.\\
% pagecolor.sty & The \xext{sty}le file.\\
% pagecolor-example.tex & The example file.
% \end{tabular}
%
% \bigskip
%
% \noindent For required other packages, please see the preceding subsection.
%
% \paragraph{Unpacking.} The \xfile{.dtx} file is a self-extracting
% \docstrip{} archive. The files are extracted by running the
% \xext{.dtx} through \plainTeX{}:
% \begin{quote}
%   \verb|tex pagecolor.dtx|
% \end{quote}
%
% About generating the documentation see paragraph~\ref{GenDoc} below.\\
%
% \paragraph{TDS.} Now the different files must be moved into
% the different directories in your installation TDS tree
% (also known as \xfile{texmf} tree):
% \begin{quote}
% \def\t{^^A
% \begin{tabular}{@{}>{\ttfamily}l@{ $\rightarrow$ }>{\ttfamily}l@{}}
%   pagecolor.sty & tex/latex/pagecolor/pagecolor.sty\\
%   pagecolor.pdf & doc/latex/pagecolor/pagecolor.pdf\\
%   pagecolor-example.tex & doc/latex/pagecolor/pagecolor-example.tex\\
%   pagecolor-example.pdf & doc/latex/pagecolor/pagecolor-example.pdf\\
%   pagecolor.dtx & source/latex/pagecolor/pagecolor.dtx\\
% \end{tabular}^^A
% }^^A
% \sbox0{\t}^^A
% \ifdim\wd0>\linewidth
%   \begingroup
%     \advance\linewidth by\leftmargin
%     \advance\linewidth by\rightmargin
%   \edef\x{\endgroup
%     \def\noexpand\lw{\the\linewidth}^^A
%   }\x
%   \def\lwbox{^^A
%     \leavevmode
%     \hbox to \linewidth{^^A
%       \kern-\leftmargin\relax
%       \hss
%       \usebox0
%       \hss
%       \kern-\rightmargin\relax
%     }^^A
%   }^^A
%   \ifdim\wd0>\lw
%     \sbox0{\small\t}^^A
%     \ifdim\wd0>\linewidth
%       \ifdim\wd0>\lw
%         \sbox0{\footnotesize\t}^^A
%         \ifdim\wd0>\linewidth
%           \ifdim\wd0>\lw
%             \sbox0{\scriptsize\t}^^A
%             \ifdim\wd0>\linewidth
%               \ifdim\wd0>\lw
%                 \sbox0{\tiny\t}^^A
%                 \ifdim\wd0>\linewidth
%                   \lwbox
%                 \else
%                   \usebox0
%                 \fi
%               \else
%                 \lwbox
%               \fi
%             \else
%               \usebox0
%             \fi
%           \else
%             \lwbox
%           \fi
%         \else
%           \usebox0
%         \fi
%       \else
%         \lwbox
%       \fi
%     \else
%       \usebox0
%     \fi
%   \else
%     \lwbox
%   \fi
% \else
%   \usebox0
% \fi
% \end{quote}
% If you have a \xfile{docstrip.cfg} that configures and enables \docstrip{}'s
% \xfile{TDS} installing feature, then some files can already be in the right
% place, see the documentation of \docstrip{}.
%
% \subsection{Refresh file name databases}
%
% If your \TeX{}~distribution (\teTeX{}, \mikTeX{},\dots{}) relies on file name
% databases, you must refresh these. For example, \teTeX{} users run
% \verb|texhash| or \verb|mktexlsr|.
%
% \subsection{Some details for the interested}
%
% \paragraph{Unpacking with \LaTeX{}.}
% The \xfile{.dtx} chooses its action depending on the format:
% \begin{description}
% \item[\plainTeX:] Run \docstrip{} and extract the files.
% \item[\LaTeX:] Generate the documentation.
% \end{description}
% If you insist on using \LaTeX{} for \docstrip{} (really,
% \docstrip{} does not need \LaTeX{}), then inform the autodetect routine
% about your intention:
% \begin{quote}
%   \verb|latex \let\install=y% \iffalse meta-comment
%
% File: pagecolor.dtx
% Version: 2012/02/23 v1.0e
%
% Copyright (C) 2011 - 2012 by
%    H.-Martin M"unch <Martin dot Muench at Uni-Bonn dot de>
%
% This work may be distributed and/or modified under the
% conditions of the LaTeX Project Public License, either
% version 1.3c of this license or (at your option) any later
% version. This version of this license is in
%    http://www.latex-project.org/lppl/lppl-1-3c.txt
% and the latest version of this license is in
%    http://www.latex-project.org/lppl.txt
% and version 1.3c or later is part of all distributions of
% LaTeX version 2005/12/01 or later.
%
% This work has the LPPL maintenance status "maintained".
%
% The Current Maintainer of this work is H.-Martin Muench.
%
% This work consists of the main source file pagecolor.dtx,
% the README, and the derived files
%    pagecolor.sty, pagecolor.pdf,
%    pagecolor.ins, pagecolor.drv,
%    pagecolor-example.tex, pagecolor-example.pdf.
%
% Distribution:
%    CTAN:macros/latex/contrib/pagecolor/pagecolor.dtx
%    CTAN:macros/latex/contrib/pagecolor/pagecolor.pdf
%    CTAN:install/macros/latex/contrib/pagecolor.tds.zip
%
% Cf. http://ctan.org/pkg/pagecolor and
% http://www.tex.ac.uk/tex-archive/help/Catalogue/entries/pagecolor.html
%
% Unpacking:
%    (a) If pagecolor.ins is present:
%           tex pagecolor.ins
%    (b) Without pagecolor.ins:
%           tex pagecolor.dtx
%    (c) If you insist on using LaTeX
%           latex \let\install=y% \iffalse meta-comment
%
% File: pagecolor.dtx
% Version: 2012/02/23 v1.0e
%
% Copyright (C) 2011 - 2012 by
%    H.-Martin M"unch <Martin dot Muench at Uni-Bonn dot de>
%
% This work may be distributed and/or modified under the
% conditions of the LaTeX Project Public License, either
% version 1.3c of this license or (at your option) any later
% version. This version of this license is in
%    http://www.latex-project.org/lppl/lppl-1-3c.txt
% and the latest version of this license is in
%    http://www.latex-project.org/lppl.txt
% and version 1.3c or later is part of all distributions of
% LaTeX version 2005/12/01 or later.
%
% This work has the LPPL maintenance status "maintained".
%
% The Current Maintainer of this work is H.-Martin Muench.
%
% This work consists of the main source file pagecolor.dtx,
% the README, and the derived files
%    pagecolor.sty, pagecolor.pdf,
%    pagecolor.ins, pagecolor.drv,
%    pagecolor-example.tex, pagecolor-example.pdf.
%
% Distribution:
%    CTAN:macros/latex/contrib/pagecolor/pagecolor.dtx
%    CTAN:macros/latex/contrib/pagecolor/pagecolor.pdf
%    CTAN:install/macros/latex/contrib/pagecolor.tds.zip
%
% Cf. http://ctan.org/pkg/pagecolor and
% http://www.tex.ac.uk/tex-archive/help/Catalogue/entries/pagecolor.html
%
% Unpacking:
%    (a) If pagecolor.ins is present:
%           tex pagecolor.ins
%    (b) Without pagecolor.ins:
%           tex pagecolor.dtx
%    (c) If you insist on using LaTeX
%           latex \let\install=y\input{pagecolor.dtx}
%        (quote the arguments according to the demands of your shell)
%
% Documentation:
%    (a) If pagecolor.drv is present:
%           (pdf)latex pagecolor.drv
%           makeindex -s gind.ist pagecolor.idx
%           (pdf)latex pagecolor.drv
%           makeindex -s gind.ist pagecolor.idx
%           (pdf)latex pagecolor.drv
%    (b) Without pagecolor.drv:
%           (pdf)latex pagecolor.dtx
%           makeindex -s gind.ist pagecolor.idx
%           (pdf)latex pagecolor.dtx
%           makeindex -s gind.ist pagecolor.idx
%           (pdf)latex pagecolor.dtx
%
%    The class ltxdoc loads the configuration file ltxdoc.cfg
%    if available. Here you can specify further options, e.g.
%    use DIN A4 as paper format:
%       \PassOptionsToClass{a4paper}{article}
%
% Installation:
%    TDS:tex/latex/pagecolor/pagecolor.sty
%    TDS:doc/latex/pagecolor/pagecolor.pdf
%    TDS:doc/latex/pagecolor/pagecolor-example.tex
%    TDS:doc/latex/pagecolor/pagecolor-example.pdf
%    TDS:source/latex/pagecolor/pagecolor.dtx
%
%<*ignore>
\begingroup
  \catcode123=1 %
  \catcode125=2 %
  \def\x{LaTeX2e}%
\expandafter\endgroup
\ifcase 0\ifx\install y1\fi\expandafter
         \ifx\csname processbatchFile\endcsname\relax\else1\fi
         \ifx\fmtname\x\else 1\fi\relax
\else\csname fi\endcsname
%</ignore>
%<*install>
\input docstrip.tex
\Msg{*****************************************************************}
\Msg{* Installation                                                   }
\Msg{* Package: pagecolor 2012/02/23 v1.0e Provides thepagecolor (HMM)}
\Msg{*****************************************************************}

\keepsilent
\askforoverwritefalse

\let\MetaPrefix\relax
\preamble

This is a generated file.

Project: pagecolor
Version: 2012/02/23 v1.0e

Copyright (C) 2011 - 2012 by
    H.-Martin M"unch <Martin dot Muench at Uni-Bonn dot de>

The usual disclaimer applies:
If it doesn't work right that's your problem.
(Nevertheless, send an e-mail to the maintainer
 when you find an error in this package.)

This work may be distributed and/or modified under the
conditions of the LaTeX Project Public License, either
version 1.3c of this license or (at your option) any later
version. This version of this license is in
   http://www.latex-project.org/lppl/lppl-1-3c.txt
and the latest version of this license is in
   http://www.latex-project.org/lppl.txt
and version 1.3c or later is part of all distributions of
LaTeX version 2005/12/01 or later.

This work has the LPPL maintenance status "maintained".

The Current Maintainer of this work is H.-Martin Muench.

This work consists of the main source file pagecolor.dtx,
the README, and the derived files
   pagecolor.sty, pagecolor.pdf,
   pagecolor.ins, pagecolor.drv,
   pagecolor-example.tex, pagecolor-example.pdf.

\endpreamble
\let\MetaPrefix\DoubleperCent

\generate{%
  \file{pagecolor.ins}{\from{pagecolor.dtx}{install}}%
  \file{pagecolor.drv}{\from{pagecolor.dtx}{driver}}%
  \usedir{tex/latex/pagecolor}%
  \file{pagecolor.sty}{\from{pagecolor.dtx}{package}}%
  \usedir{doc/latex/pagecolor}%
  \file{pagecolor-example.tex}{\from{pagecolor.dtx}{example}}%
}

\catcode32=13\relax% active space
\let =\space%
\Msg{************************************************************************}
\Msg{*}
\Msg{* To finish the installation you have to move the following}
\Msg{* file into a directory searched by TeX:}
\Msg{*}
\Msg{*     pagecolor.sty}
\Msg{*}
\Msg{* To produce the documentation run the file `pagecolor.drv'}
\Msg{* through (pdf)LaTeX, e.g.}
\Msg{*  pdflatex pagecolor.drv}
\Msg{*  makeindex -s gind.ist pagecolor.idx}
\Msg{*  pdflatex pagecolor.drv}
\Msg{*  makeindex -s gind.ist pagecolor.idx}
\Msg{*  pdflatex pagecolor.drv}
\Msg{*}
\Msg{* At least three runs are necessary e.g. to get the}
\Msg{*  references right!}
\Msg{*}
\Msg{* Happy TeXing!}
\Msg{*}
\Msg{************************************************************************}

\endbatchfile
%</install>
%<*ignore>
\fi
%</ignore>
%
% \section{The documentation driver file}
%
% The next bit of code contains the documentation driver file for
% \TeX{}, i.\,e., the file that will produce the documentation you
% are currently reading. It will be extracted from this file by the
% \texttt{docstrip} programme. That is, run \LaTeX{} on \texttt{docstrip}
% and specify the \texttt{driver} option when \texttt{docstrip}
% asks for options.
%
%    \begin{macrocode}
%<*driver>
\NeedsTeXFormat{LaTeX2e}[2011/06/27]
\ProvidesFile{pagecolor.drv}%
  [2012/02/23 v1.0e Provides thepagecolor (HMM)]
\documentclass{ltxdoc}[2007/11/11]% v2.0u
\usepackage{holtxdoc}[2011/02/04]%  v0.21
%% pagecolor may work with earlier versions of LaTeX2e and those
%% class and package, but this was not tested.
%% Please consider updating your LaTeX, class, and package
%% to the most recent version (if they are not already the most
%% recent version).
\hypersetup{%
 pdfsubject={Providing the thepagecolor command (HMM)},%
 pdfkeywords={LaTeX, pagecolor, thepagecolor, H.-Martin Muench},%
 pdfencoding=auto,%
 pdflang={en},%
 breaklinks=true,%
 linktoc=all,%
 pdfstartview=FitH,%
 pdfpagelayout=OneColumn,%
 bookmarksnumbered=true,%
 bookmarksopen=true,%
 bookmarksopenlevel=3,%
 pdfmenubar=true,%
 pdftoolbar=true,%
 pdfwindowui=true,%
 pdfnewwindow=true%
}
\CodelineIndex
\hyphenation{printing docu-ment}
\gdef\unit#1{\mathord{\thinspace\mathrm{#1}}}%
\begin{document}
  \DocInput{pagecolor.dtx}%
\end{document}
%</driver>
%    \end{macrocode}
%
% \fi
%
% \CheckSum{137}
%
% \CharacterTable
%  {Upper-case    \A\B\C\D\E\F\G\H\I\J\K\L\M\N\O\P\Q\R\S\T\U\V\W\X\Y\Z
%   Lower-case    \a\b\c\d\e\f\g\h\i\j\k\l\m\n\o\p\q\r\s\t\u\v\w\x\y\z
%   Digits        \0\1\2\3\4\5\6\7\8\9
%   Exclamation   \!     Double quote  \"     Hash (number) \#
%   Dollar        \$     Percent       \%     Ampersand     \&
%   Acute accent  \'     Left paren    \(     Right paren   \)
%   Asterisk      \*     Plus          \+     Comma         \,
%   Minus         \-     Point         \.     Solidus       \/
%   Colon         \:     Semicolon     \;     Less than     \<
%   Equals        \=     Greater than  \>     Question mark \?
%   Commercial at \@     Left bracket  \[     Backslash     \\
%   Right bracket \]     Circumflex    \^     Underscore    \_
%   Grave accent  \`     Left brace    \{     Vertical bar  \|
%   Right brace   \}     Tilde         \~}
%
% \GetFileInfo{pagecolor.drv}
%
% \begingroup
%   \def\x{\#,\$,\^,\_,\~,\ ,\&,\{,\},\%}%
%   \makeatletter
%   \@onelevel@sanitize\x
% \expandafter\endgroup
% \expandafter\DoNotIndex\expandafter{\x}
% \expandafter\DoNotIndex\expandafter{\string\ }
% \begingroup
%   \makeatletter
%     \lccode`9=32\relax
%     \lowercase{%^^A
%       \edef\x{\noexpand\DoNotIndex{\@backslashchar9}}%^^A
%     }%^^A
%   \expandafter\endgroup\x
%
% \DoNotIndex{\,,\\}
% \DoNotIndex{\documentclass,\usepackage,\ProvidesPackage,\begin,\end}
% \DoNotIndex{\MessageBreak}
% \DoNotIndex{\NeedsTeXFormat,\DoNotIndex,\verb}
% \DoNotIndex{\def,\edef,\gdef,\global}
% \DoNotIndex{\ifx,\listfiles,\mathord,\mathrm}
% \DoNotIndex{\kvoptions,\SetupKeyvalOptions,\ProcessKeyvalOptions}
% \DoNotIndex{\bigskip,\space,\thinspace,\ldots}
% \DoNotIndex{\indent,\noindent,\newline,\linebreak,\pagebreak,\newpage}
% \DoNotIndex{\textbf,\textit,\textsf,\texttt,\textsc,\textquotedblleft,\textquotedblright}
% \DoNotIndex{\plainTeX,\TeX,\LaTeX,\pdfLaTeX}
% \DoNotIndex{\section,\color,\pagenumbering,\smallskip}
% \DoNotIndex{\@ifundefined,\@ifpackageloaded}
% \DoNotIndex{\DeclareStringOption}
%
% \title{The \xpackage{pagecolor} package}
% \date{2012/02/23 v1.0e}
% \author{H.-Martin M\"{u}nch\\\xemail{Martin.Muench at Uni-Bonn.de}}
%
% \maketitle
%
% \begin{abstract}
%  This \LaTeX{} package provides the command |\thepagecolor|,
% which gives the current page (background) colour,
% i.\,e.~the argument used with the most recent call of |\pagecolor{...}|.
% The command |\thepagecolornone| gives the same colour as |\thepagecolor|,
% except when the page background colour is
% \textquotedblleft|none|\textquotedblright. In that case |\thepagecolor| is
% |white| and |\thepagecolornone| is |none|.\\
% Similar to |\newgeometry| and |\restoregeometry| of the \xpackage{geometry}
% package\\
% |\newpagecolor{<|\textit{some colour} |>}| and |\restorepagecolor|
% are provided.
% \end{abstract}
%
% \bigskip
%
% \noindent Disclaimer for web links: The author is not responsible for any contents
% referred to in this work unless he has full knowledge of illegal contents.
% If any damage occurs by the use of information presented there, only the
% author of the respective pages might be liable, not the one who has referred
% to these pages.
%
% \bigskip
%
% \noindent {\color{green} Save per page about $200\unit{ml}$ water,
% $2\unit{g}$ CO$_{2}$ and $2\unit{g}$ wood:\\
% Therefore please print only if this is really necessary.}
%
% \pagebreak
%
% \tableofcontents
%
% \bigskip
%
% \section{Introduction\label{sec:Introduction}}
%
% \indent This \LaTeX{} package provides the command |\thepagecolor|,
% which gives the current page (background) colour,
% i.\,e.~the argument used with the most recent call of |\pagecolor{...}|.
% The package should be loaded before any package sets a page (background)
% colour, but after \xpackage{color} or \xpackage{xcolor} package.
% Its option |pagecolor={...}| is used to set the |\pagecolor{...}|.\\
% The command |\thepagecolornone| gives the same colour as |\thepagecolor|,
% except when the page background colour is
% \textquotedblleft|none|\textquotedblright (e.\,g. result of using the
% |\nopagecolor| command). In that case |\thepagecolor| is |white| and\linebreak
% |\thepagecolornone| is |none|.\\
% Similar to |\newgeometry| and |\restoregeometry| of the \xpackage{geometry}
% package\\
% |\newpagecolor{<|\textit{some colour}|>}| and |\restorepagecolor|
% are provided.
%
% \pagebreak
%
% \section{Usage}
%
% \indent Just load the package placing
% \begin{quote}
%   |\usepackage[<|\textit{option}|>]{pagecolor}|
% \end{quote}
% \noindent in the preamble of your \LaTeXe{} source file.
% This should be done before another package uses |\pagecolor|,
% but after |\nopagecolor| is defined (if that is defined in the document
% at all).\\
% Afterwards |\pagecolor{...}| can be used to change the page (background)
% colour as usual. Then |\thepagecolor| gives the current page (background)
% colour (in the same format as given with |\pagecolor{...}|).\\
% Similar to |\newgeometry| and |\restoregeometry| of the \xpackage{geometry}
% package\\
% |\newpagecolor{<|\textit{some colour} |>}| and |\restorepagecolor|
% are provided:\\
% |\newpagecolor{<|\textit{some colour} |>}| will execute
% |\pagecolor{<|\textit{some colour} |>}| and remember the page colour used before.
% |\restorecolor| (without argument) restores the page colour to the one used before
%  use of the |\newpagecolor{...}| command. When you want to change the colour for
%  just one page and do not want to (or cannot) manually determine where the page ends,\\
% |\newpagecolor{<|\textit{some colour}|>}\afterpage{\restorepagecolor}|\\
% does the trick (and requires a |\usepackage{afterpage}| in the document's
% preamble), or for short\\
% |\newcommand{\onepagecolor}[1]{%|\\
% |\newpagecolor{#1}\afterpage{\restorepagecolor}}|\\
% in the preamble and\\
% |\onepagecolor{<|\textit{some colour}|>}| in the document.
%
% \subsection{Option}
% \DescribeMacro{option}
% \indent The \xpackage{pagecolor} package takes the following option:
%
% \subsubsection{pagecolor\label{sss:pagecolor}}
% \DescribeMacro{pagecolor}
% \indent The option |pagecolor={...}| takes as value a colour. This
% could be as simple as |white| or |black|, but when e.\,g.~the
% \xpackage{xcolor} package is used (loaded before \xpackage{pagecolor}!),
% also colours like |red!50!green!20!blue| are possible.
% The default is |pagecolor={none}|.
%
% \section{Alternatives\label{sec:Alternatives}}
%
% As I neither know what exactly you want to accomplish
% when using this package (e.\,g.~hiding text), nor what resources you have
% (e.\,g.~\pdfTeX{} version), here is a list of some possible alternatives:\\
%
% \begin{description}
%  \item[-]
%  \DescribeMacro{transparent}
%  \xpackage{transparent} package: With it some object can be made
%  (fully or partially) transparent.
% \end{description}
%
% \begin{description}
%  \item[-]
%  \DescribeMacro{hrefhide}\xpackage{hrefhide} package: It allows to
%  \textquotedblleft hide\textquotedblright\ some (hyperlinked) text when
%  printing the document while keeping the layout.
% \end{description}
%
% \bigskip
%
% \noindent (You programmed or found another alternative,
%  which is available at \CTAN{}?\\
%  OK, send an e-mail to me with the name, location at \CTAN{},
%  and a short notice, and I will probably include it in
%  the list above.)\\
%
% \smallskip
%
% \noindent About how to get those packages, please see
% subsection~\ref{ss:Downloads}.
%
% \newpage
%
% \section{Example}
%
%    \begin{macrocode}
%<*example>
\documentclass[british]{article}[2007/10/19]% v1.4h
%%%%%%%%%%%%%%%%%%%%%%%%%%%%%%%%%%%%%%%%%%%%%%%%%%%%%%%%%%%%%%%%%%%%%
\usepackage[%
 extension=pdf,%
 plainpages=false,%
 pdfpagelabels=true,%
 hyperindex=false,%
 pdflang={en},%
 pdftitle={pagecolor package example},%
 pdfauthor={H.-Martin Muench},%
 pdfsubject={Example for the pagecolor package},%
 pdfkeywords={LaTeX, pagecolor, thepagecolor, page colour,%
  H.-Martin Muench},%
 pdfview=Fit,pdfstartview=Fit,%
 pdfpagelayout=SinglePage%
]{hyperref}[2012/02/06]% v6.82o
\usepackage[x11names]{xcolor}[2007/01/21]% v2.11
 % The xcolor package would not be needed for just using
 % the base colours. The color package would be sufficient for that.
\usepackage[pagecolor={LightGoldenrod1}]{pagecolor}[2012/02/23]% v1.0e

\usepackage{afterpage}[1995/10/27]% v1.08
 % The afterpage package is generally not needed,
 % but the |\newpagecolor{somecolour}\afterpage{\restorepagecolor}|
 % construct shall be demonstrated.

\usepackage{lipsum}[2011/04/14]% v1.2
 % The lipsum package is generally not needed,
 % but some blind text is needed for the example.

\gdef\unit#1{\mathord{\thinspace\mathrm{#1}}}%
\listfiles
\begin{document}
\pagenumbering{arabic}
\section*{Example for pagecolor}

This example demonstrates the use of package\newline
\textsf{pagecolor}, v1.0e as of 2012/02/23 (HMM).\newline
The used option was \verb|pagecolor={LightGoldenrod1}|.\newline
\verb|pagecolor={none}| would be the default.\newline

For more details please see the documentation!\newline

\noindent {\color{teal} Save per page about $200\unit{ml}$ water,
$2\unit{g}$ CO$_{2}$ and $2\unit{g}$ wood:\newline
Therefore please print only if this is really necessary.}\newline

The current page (background) colour is\newline
\verb|\thepagecolor|\ =\ \thepagecolor \newline
(and \verb|\thepagecolornone|\ =\ \thepagecolornone ,
which would only be different from \verb|\thepagecolor|,
when the page colour would be \verb|none|).

\pagebreak
\pagecolor{rgb:-green!40!yellow,3;green!40!yellow,2;red,1}

{\color{white} The current page (background) colour is\newline
\verb|\thepagecolor|\ =\ \thepagecolor . \newline}

{\color{\thepagecolor} And that makes this text practically invisible.
\newline}

{\color{white} Which made the preceding line of text practically
invisible.}

\pagebreak
\newpagecolor{red}

This page uses \verb|\newpagecolor{red}|.

\pagebreak
\restorepagecolor

{\color{white}And this page uses \verb|\restorepagecolor| to restore
the page colour to the value it had before the red page.}

\pagebreak
\pagecolor{none}

This page uses \verb|\pagecolor{none}|. If the \verb|\nopagecolor|
command is known (pdf\TeX and Lua\TeX; not yet for dvips, dvipdfm(x)
or Xe\TeX), the page colour is now \verb|none|, otherwise \verb|white|:
\verb|\thepagecolor|\ =\ \thepagecolor\ and
\verb|\thepagecolornone|\ =\ \thepagecolornone .

\pagebreak
\restorepagecolor

{\color{white}\verb|\restorepagecolor| restored the page colour again.}

\pagebreak
\pagecolor{green}

This page is green due to \verb|\pagecolor{green}|.

\pagebreak
\newpagecolor{blue}\afterpage{\restorepagecolor}

{\color{white}\verb|\newpagecolor{blue}\afterpage{\restorepagecolor}|%
\newline
was used here, i.\,e.~this page is blue, and the next one will
automatically have the same page colour before it was changed to blue
here (i.\,e. green).}

\smallskip
{\color{red}\textbf{\lipsum[1-11]}}
\bigskip

The page colour was changed back at the end of the page -
in mid-sentence!

\end{document}
%</example>
%    \end{macrocode}
%
% \StopEventually{}
%
% \newpage
%
% \section{The implementation}
%
% We start off by checking that we are loading into \LaTeXe{} and
% announcing the name and version of this package.
%
%    \begin{macrocode}
%<*package>
%    \end{macrocode}
%
%    \begin{macrocode}
\NeedsTeXFormat{LaTeX2e}[2011/06/27]
\ProvidesPackage{pagecolor}[2012/02/23 v1.0e
            Provides thepagecolor (HMM)]
%    \end{macrocode}
%
% A short description of the \xpackage{pagecolor} package:
%
%    \begin{macrocode}
%% Provides the \thepagecolor, \thepagecolornone, \newpagecolor{...},
%% and \restorepagecolor commands.
%    \end{macrocode}
%
% We need the \xpackage{kvoptions} package by \textsc{Heiko Oberdiek}:
%
%    \begin{macrocode}
\RequirePackage{kvoptions}[2010/12/23]% v3.10
%    \end{macrocode}
%
% and either the \xpackage{color} or the \xpackage{xcolor} package:
%
%    \begin{macrocode}
%% \RequirePackage{ either color or xcolor }:
\@ifpackageloaded{xcolor}{% xcolor loaded
   \@ifpackagelater{xcolor}{2007/01/21}{%
      % 2007/01/21, v2.11, or even more recent: OK
    }{% else: older package version
      \PackageWarning{pagecolor}{%
        It is requested version `2007/01/21' of package\MessageBreak%
        xcolor, but only an older version is available.\MessageBreak%
       }%
     }%
 }{% xcolor not loaded
   \@ifpackageloaded{color}{%
      \RequirePackage{color}[2005/11/14]% v1.0j
    }{% \else
      \PackageWarning{pagecolor}{%
        The pagecolor package must be loaded after either\MessageBreak%
        package color or after package xcolor (at your\MessageBreak%
        option). Neither package was loaded before package\MessageBreak%
        pagecolor. Loading of package xcolor will now be\MessageBreak%
        tried automatically.\MessageBreak%
        When the pagecolor package is used with option\MessageBreak%
        pagecolor using a colour requiring e. g. x11names\MessageBreak%
        option for xcolor package, this will not work!\MessageBreak%
       }
     }% \fi
   \RequirePackage{xcolor}[2007/01/21]% v2.11
  }% \fi
%    \end{macrocode}
%
% A last information for the user:
%
%    \begin{macrocode}
%% pagecolor may work with earlier versions of LaTeX and the
%% packages, but this was not tested. Please consider updating
%% your LaTeX and packages to the most recent version
%% (if they are not already the most recent version).

%    \end{macrocode}
%
% \indent See subsection~\ref{ss:Downloads} about how to get them.\\
%
% We process the options:
%
%    \begin{macrocode}
\SetupKeyvalOptions{family=pagecolor,prefix=pagecolor@}
\DeclareStringOption[none]{pagecolor}% \pagecolor@pagecolor
\ProcessKeyvalOptions*

%    \end{macrocode}
%
% We save the original |\pagecolor| command,
%
%    \begin{macrocode}
\let\origpagecolour\pagecolor

%    \end{macrocode}
%
%    \begin{macro}{\pagecolor}
% before we redefine it to include a definition of |\thepagecolor|
% and \linebreak
% |\thepagecolornone|:
%
%    \begin{macrocode}
\renewcommand{\pagecolor}[1]{\@bsphack%
  \edef\pagecolourtmpa{#1}%
  \def\pagecolourtmpb{none}%
  \ifx\pagecolourtmpa\pagecolourtmpb
    \@ifundefined{nopagecolor}{%
      \PackageError{pagecolor}{%
        pagecolor=none requested but \string\nopagecolor\space unknown%
       }{%
        \string\pagecolor{none} was used, but the command %
        \string\nopagecolor\space is undefined.\MessageBreak%
        Please use another colour. pagecolor=white will be used now.%
        \MessageBreak%
        }%
      \gdef\thepagecolor{white}%
      \gdef\thepagecolornone{white}% although it should be "none"
      \origpagecolour{white}%
     }{%
      \nopagecolor%
     }%
  \else%
    \xdef\thepagecolor{#1}%
    \xdef\thepagecolornone{#1}%
    \origpagecolour{\thepagecolornone}%
  \fi%
  \@esphack%
}

%    \end{macrocode}
%    \end{macro}
%
% |\nopagecolor| is only defined for \pdfTeX and \LuaTeX, but not yet for dvips,
% dvipdfm(x) or \XeTeX. Maybe
% \begin{verbatim}
% \@ifundefined{nopagecolor}{\newcommand{\nopagecolor}{\pagecolor{white}}}{}
% \end{verbatim}
% could be an alternative.\\
% When |\pagecolor{none}| is used and |\nopagecolor| is defined, then \\
% |\pagecolor{none}| is made into a synonym for |\nopagecolor|.\\
% If |\nopagecolor| has not been defined, nothing needs to be done.
% Otherwise we redefine |\nopagecolor|, thus that |\thepagecolor| is
% set to |white| and |\thepagecolornone| to |none| whenever
% |\nopagecolor| is used.
%
%    \begin{macrocode}
\@ifundefined{nopagecolor}{%
  \PackageWarning{pagecolor}{%
    \string\nopagecolor\space is undefined.\MessageBreak%
    (As long as it is neither used anywhere\MessageBreak%
    \space nor defined later, that will be no problem.)\MessageBreak%
   }
  \AtEndDocument{%
  \@ifundefined{nopagecolor}{% no problem
    }{%
    \PackageError{pagecolor}{%
      \string\nopagecolor\space defined after loading package%
      \MessageBreak%
      pagecolor%
      }{\string\nopagecolor\space was defined after loading the %
       pagecolor package.\MessageBreak%
       Either define \string\nopagecolor\space earlier or load the %
       pagecolor package later.\MessageBreak%
      }
    }%
  }%
  }{% \else
    \let\orignopagecolour\nopagecolor
    \renewcommand{\nopagecolor}{%
      \xdef\thepagecolor{white}
      \xdef\thepagecolornone{none}
      \orignopagecolour
      }
  }

%    \end{macrocode}
%
% If the page colour as given with option |pagecolor={...}| is |none|,
% but |\nopagecolor| is not known, |\pagecolor@pagecolor| is set to
% |white| and a warning is given.
%
%    \begin{macrocode}
\def\pagecolourtmpb{none}
\ifx\pagecolor@pagecolor\pagecolourtmpb
  \@ifundefined{nopagecolor}{%
    \PackageWarning{pagecolor}{%
      Option pagecolor=none (maybe by default) used,\MessageBreak%
      but \string\nopagecolor\space is unknown. Please use another%
      \MessageBreak%
      option value; white will be used now.\MessageBreak%
     }
    \setkeys{pagecolor}{pagecolor=white}%
  }{% ok
  }%
\fi

%    \end{macrocode}
%
% The (new) |\pagecolor| is now just carried out.
%
%    \begin{macrocode}
\pagecolor{\pagecolor@pagecolor}

%    \end{macrocode}
%
% Now the page (background) colour and |\thepagecolor| and \\
% |\thepagecolornone| are |\pagecolor@pagecolor| (or page (background) colour
% and |\thepagecolornone| are |none|=|\pagecolor@pagecolor| and
% |\thepagecolor| is |white|), and when the page (background) colour
% is changed, |\thepagecolor| and |\thepagecolornone| are changed
% accordingly.
%
% \newpage
%
%    \begin{macro}{\newpagecolor}
% There have been requests (via e-mail and at \\
% \url{http://tex.stackexchange.com/q/25137}) to change the colour of just one
% (or two) page(s) only, similar to |\newgeometry| and |\restoregeometry| of the
% \xpackage{geometry} package (\url{http://ctan.org/pkg/geometry}). Therefore
% |\newpagecolor| and |\restorepagecolor| are introduced (as suggested by
% \textsc{haoyun\_tex}):
%
%    \begin{macrocode}
\newcommand{\newpagecolor}[1]{%
\xdef\pagecolourtmpc{\thepagecolornone}%
\pagecolor{#1}%
}

%    \end{macrocode}
%    \end{macro}
%
% |\newpagecolor{<|\textit{some colour}|>}| will execute |\pagecolor{somecolour}|
% and remember the page colour used before.\\
%
%    \begin{macro}{\restorepagecolor}
%    \begin{macrocode}
\newcommand{\restorepagecolor}{\pagecolor{\pagecolourtmpc}}

%    \end{macrocode}
%    \end{macro}
%
% |\restorecolor| (without argument) restores the page colour to the one used before
% use of the |\newpagecolor{...}| command.
%
%    \begin{macrocode}
\gdef\pagecolourtmpc{\thepagecolor}

%    \end{macrocode}
% is just a precaution for |\restorecolor| being used when no |\newpagecolor{...}|
% was used before it.\\
%
% When you want to change the colour for just one page and do not want to
% (or cannot) manually determine where the page ends,\\
% |\newpagecolor{<|\textit{some colour}|>}\afterpage{\restorepagecolor}|\\
% does the trick (and requires an additional |\usepackage{afterpage}| in the
% document's preamble).
%    \begin{macrocode}
%</package>
%    \end{macrocode}
%
% \newpage
%
% \section{Installation}
%
% \subsection{Downloads\label{ss:Downloads}}
%
% Everything is available on \CTAN{}, \url{http://www.ctan.org/tex-archive/},
% but may need additional packages themselves.\\
%
% \DescribeMacro{pagecolor.dtx}
% For unpacking the |pagecolor.dtx| file and constructing the documentation
% it is required:
% \begin{description}
% \item[-] \TeX{} Format \LaTeXe{}, \url{http://www.CTAN.org/}
%
% \item[-] document class \xpackage{ltxdoc}, 2007/11/11, v2.0u,
%           \url{http://ctan.org/pkg/ltxdoc}
%
% \item[-] package \xpackage{holtxdoc}, 2011/02/04, v0.21,
%           \url{http://ctan.org/pkg/holtxdoc}
% \end{description}
%
% \DescribeMacro{pagecolor.sty}
% The |pagecolor.sty| for \LaTeXe{} (i.\,e. each document using
% the \xpackage{pagecolor} package) requires:
% \begin{description}
% \item[-] \TeX{} Format \LaTeXe{}, \url{http://www.CTAN.org/}
%
% \item[-] package \xpackage{kvoptions}, 2010/12/23, v3.10,
%           \url{http://ctan.org/pkg/kvoptions}
% \end{description}
% and either
% \begin{description}
% \item[-] package \xpackage{color}, 2005/11/14, v1.0j,
%           \url{http://ctan.org/pkg/color} (from the\\
%           \xpackage{graphics} package bundle)
% \end{description}
% or
% \begin{description}
% \item[-] package \xpackage{xcolor}, 2007/01/21, v2.11,
%           \url{http://ctan.org/pkg/xcolor}
% \end{description}
%
% \DescribeMacro{pagecolor-example.tex}
% The |pagecolor-example.tex| requires the same files as all
% documents using the \xpackage{pagecolor} package and additionally:
% \begin{description}
% \item[-] class \xpackage{article}, 2007/10/19, v1.4h, from \xpackage{classes.dtx}:\\
%           \CTAN{macros/latex/base/classes.dtx}
%
% \item[-] package \xpackage{pagecolor}, 2012/02/23, v1.0e,
%           \url{http://ctan.org/pkg/pagecolor}\\
%           (Well, it is the example file for this package, and because you are
%            reading the documentation for the \xpackage{pagecolor} package,
%            it can be assumed that you already have some version of it --
%            is it the current one?)
%
% \item[-] package \xpackage{xcolor}, 2007/01/21, v2.11,
%           \url{http://ctan.org/pkg/xcolor}\\
%           This package would not be needed for the use of just base colours only,
%           the \xpackage{color} package would be sufficient for that.
% \end{description}
%
% \DescribeMacro{transparent}
% \DescribeMacro{hrefhide}
% As possible alternatives in section \ref{sec:Alternatives} there are listed
% \begin{description}
% \item[-] package \xpackage{transparent}, 2007/01/08, v1.0, \\
%           \url{http://ctan.org/pkg/transparent}
%
% \item[-] package \xpackage{hrefhide}, 2011/04/29, v1.0f, \\
%           \url{http://ctan.org/pkg/hrefhide}
% \end{description}
%
% \DescribeMacro{Oberdiek}
% \DescribeMacro{holtxdoc}
% \DescribeMacro{kvoptions}
% All packages of \textsc{Heiko Oberdiek}'s bundle `oberdiek'
% (especially \xpackage{holtxdoc} and \xpackage{kvoptions})
% are also available in a TDS compliant ZIP archive:\\
% \CTAN{install/macros/latex/contrib/oberdiek.tds.zip}.\\
% It is probably best to download and use this, because the packages in there
% are quite probably both recent and compatible among themselves.\\
%
% \DescribeMacro{hyperref}
% \noindent \xpackage{hyperref} is not included in that bundle and needs to be
% downloaded separately,\\
% \url{http://mirror.ctan.org/install/macros/latex/contrib/hyperref.tds.zip}.\\
%
% \DescribeMacro{M\"{u}nch}
% A hyperlinked list of my (other) packages can be found at
% \url{http://www.Uni-Bonn.de/~uzs5pv/LaTeX.html}.\\
%
% \subsection{Package, unpacking TDS}
% \paragraph{Package.} This package is available on \CTAN{}
% \begin{description}
% \item[\CTAN{macros/latex/contrib/pagecolor/pagecolor.dtx}]\hspace*{0.1cm} \\
%       The source file.
% \item[\CTAN{macros/latex/contrib/pagecolor/pagecolor.pdf}]\hspace*{0.1cm} \\
%       The documentation.
% \item[\CTAN{macros/latex/contrib/pagecolor/pagecolor-example.pdf}]\hspace*{0.1cm} \\
%       The compiled example file, as it should look like.
% \item[\CTAN{macros/latex/contrib/pagecolor/README}]\hspace*{0.1cm} \\
%       The README file.
% \end{description}
% There is also a pagecolor.tds.zip available:
% \begin{description}
% \item[\CTAN{install/macros/latex/contrib/pagecolor.tds.zip}]\hspace*{0.1cm} \\
%       Everything in \xfile{TDS} compliant, compiled format.
% \end{description}
% which additionally contains\\
% \begin{tabular}{ll}
% pagecolor.ins & The installation file.\\
% pagecolor.drv & The driver to generate the documentation.\\
% pagecolor.sty & The \xext{sty}le file.\\
% pagecolor-example.tex & The example file.
% \end{tabular}
%
% \bigskip
%
% \noindent For required other packages, please see the preceding subsection.
%
% \paragraph{Unpacking.} The \xfile{.dtx} file is a self-extracting
% \docstrip{} archive. The files are extracted by running the
% \xext{.dtx} through \plainTeX{}:
% \begin{quote}
%   \verb|tex pagecolor.dtx|
% \end{quote}
%
% About generating the documentation see paragraph~\ref{GenDoc} below.\\
%
% \paragraph{TDS.} Now the different files must be moved into
% the different directories in your installation TDS tree
% (also known as \xfile{texmf} tree):
% \begin{quote}
% \def\t{^^A
% \begin{tabular}{@{}>{\ttfamily}l@{ $\rightarrow$ }>{\ttfamily}l@{}}
%   pagecolor.sty & tex/latex/pagecolor/pagecolor.sty\\
%   pagecolor.pdf & doc/latex/pagecolor/pagecolor.pdf\\
%   pagecolor-example.tex & doc/latex/pagecolor/pagecolor-example.tex\\
%   pagecolor-example.pdf & doc/latex/pagecolor/pagecolor-example.pdf\\
%   pagecolor.dtx & source/latex/pagecolor/pagecolor.dtx\\
% \end{tabular}^^A
% }^^A
% \sbox0{\t}^^A
% \ifdim\wd0>\linewidth
%   \begingroup
%     \advance\linewidth by\leftmargin
%     \advance\linewidth by\rightmargin
%   \edef\x{\endgroup
%     \def\noexpand\lw{\the\linewidth}^^A
%   }\x
%   \def\lwbox{^^A
%     \leavevmode
%     \hbox to \linewidth{^^A
%       \kern-\leftmargin\relax
%       \hss
%       \usebox0
%       \hss
%       \kern-\rightmargin\relax
%     }^^A
%   }^^A
%   \ifdim\wd0>\lw
%     \sbox0{\small\t}^^A
%     \ifdim\wd0>\linewidth
%       \ifdim\wd0>\lw
%         \sbox0{\footnotesize\t}^^A
%         \ifdim\wd0>\linewidth
%           \ifdim\wd0>\lw
%             \sbox0{\scriptsize\t}^^A
%             \ifdim\wd0>\linewidth
%               \ifdim\wd0>\lw
%                 \sbox0{\tiny\t}^^A
%                 \ifdim\wd0>\linewidth
%                   \lwbox
%                 \else
%                   \usebox0
%                 \fi
%               \else
%                 \lwbox
%               \fi
%             \else
%               \usebox0
%             \fi
%           \else
%             \lwbox
%           \fi
%         \else
%           \usebox0
%         \fi
%       \else
%         \lwbox
%       \fi
%     \else
%       \usebox0
%     \fi
%   \else
%     \lwbox
%   \fi
% \else
%   \usebox0
% \fi
% \end{quote}
% If you have a \xfile{docstrip.cfg} that configures and enables \docstrip{}'s
% \xfile{TDS} installing feature, then some files can already be in the right
% place, see the documentation of \docstrip{}.
%
% \subsection{Refresh file name databases}
%
% If your \TeX{}~distribution (\teTeX{}, \mikTeX{},\dots{}) relies on file name
% databases, you must refresh these. For example, \teTeX{} users run
% \verb|texhash| or \verb|mktexlsr|.
%
% \subsection{Some details for the interested}
%
% \paragraph{Unpacking with \LaTeX{}.}
% The \xfile{.dtx} chooses its action depending on the format:
% \begin{description}
% \item[\plainTeX:] Run \docstrip{} and extract the files.
% \item[\LaTeX:] Generate the documentation.
% \end{description}
% If you insist on using \LaTeX{} for \docstrip{} (really,
% \docstrip{} does not need \LaTeX{}), then inform the autodetect routine
% about your intention:
% \begin{quote}
%   \verb|latex \let\install=y\input{pagecolor.dtx}|
% \end{quote}
% Do not forget to quote the argument according to the demands
% of your shell.
%
% \paragraph{Generating the documentation.\label{GenDoc}}
% You can use both the \xfile{.dtx} or the \xfile{.drv} to generate
% the documentation. The process can be configured by a
% configuration file \xfile{ltxdoc.cfg}. For instance, put the following
% line into this file, if you want to have A4 as paper format:
% \begin{quote}
%   \verb|\PassOptionsToClass{a4paper}{article}|
% \end{quote}
%
% \noindent An example follows how to generate the
% documentation with \pdfLaTeX{}:
%
% \begin{quote}
%\begin{verbatim}
%pdflatex pagecolor.dtx
%makeindex -s gind.ist pagecolor.idx
%pdflatex pagecolor.dtx
%makeindex -s gind.ist pagecolor.idx
%pdflatex pagecolor.dtx
%\end{verbatim}
% \end{quote}
%
% \subsection{Compiling the example}
%
% The example file, \textsf{pagecolor-example.tex}, can be compiled via\\
% |(pdf)latex pagecolor-example.tex|.\\
%
% \section{Acknowledgements}
%
% I would like to thank \textsc{Heiko Oberdiek} for providing a~lot~(!) of
% useful packages (from which I also got everything I know about creating
% a file in \xext{dtx} format, ok, say it: copying), and the
% \Newsgroup{comp.text.tex} and \Newsgroup{de.comp.text.tex} newsgroups
% for their help in all things \TeX{}, especially all contributors to the
% discussion at
% \url{http://groups.google.com/group/comp.text.tex/browse_thread/thread/533576ebe44d60f6/e1654d70a19de63c?lnk=gst&q=Determination+of+the+current+page+colour#e1654d70a19de63c}
% \ (\textsc{H.~Oberdiek} \& \textsc{Gouailles}).\\
% I thank \textsc{haoyun\_tex} for suggesting the |\newpagecolor|/|\restorepagecolor|
% pair of commands and everyone at \url{http://tex.stackexchange.com/q/25137}
% for their contributions there.
%
% \bigskip
%
% \phantomsection
% \begin{History}\label{History}
%   \begin{Version}{2011/07/16 v1.0a}
%     \item First version discussed at \Newsgroup{comp.text.tex}.
%   \end{Version}
%   \begin{Version}{2011/08/06 v1.0b}
%     \item Changed version uploaded to \CTAN{}.
%   \end{Version}
%   \begin{Version}{2011/08/08 v1.0c}
%     \item Fixed a |\setkeys|.
%   \end{Version}
%   \begin{Version}{2012/02/01 v1.0d}
%     \item Bugfix: Obsolete installation path given in the documentation, updated.
%     \item New commands: |\newpagecolor{...}|, |\restorepagecolor|.
%     \item Update of documentation, README, and \xfile{dtx} internals.
%   \end{Version}
%   \begin{Version}{2012/02/23 v1.0e}
%     \item Fixed an error in the documentation.
%     \item Check for loading of \xpackage{color} or \xpackage{xcolor} package
%             and their versions has been changed, because \xpackage{xcolor} sets\\
%             |\@namedef{ver@color.sty}{1999/02/16}|\\
%             which gave a warning about old \xpackage{color} package even if a new
%             version was used.
%   \end{Version}
% \end{History}
%
% \bigskip
%
% When you find a mistake or have a suggestion for an improvement of this package,
% please send an e-mail to the maintainer, thanks! (Please see BUG REPORTS in the README.)
%
% \newpage
%
% \PrintIndex
%
% \Finale
\endinput
%        (quote the arguments according to the demands of your shell)
%
% Documentation:
%    (a) If pagecolor.drv is present:
%           (pdf)latex pagecolor.drv
%           makeindex -s gind.ist pagecolor.idx
%           (pdf)latex pagecolor.drv
%           makeindex -s gind.ist pagecolor.idx
%           (pdf)latex pagecolor.drv
%    (b) Without pagecolor.drv:
%           (pdf)latex pagecolor.dtx
%           makeindex -s gind.ist pagecolor.idx
%           (pdf)latex pagecolor.dtx
%           makeindex -s gind.ist pagecolor.idx
%           (pdf)latex pagecolor.dtx
%
%    The class ltxdoc loads the configuration file ltxdoc.cfg
%    if available. Here you can specify further options, e.g.
%    use DIN A4 as paper format:
%       \PassOptionsToClass{a4paper}{article}
%
% Installation:
%    TDS:tex/latex/pagecolor/pagecolor.sty
%    TDS:doc/latex/pagecolor/pagecolor.pdf
%    TDS:doc/latex/pagecolor/pagecolor-example.tex
%    TDS:doc/latex/pagecolor/pagecolor-example.pdf
%    TDS:source/latex/pagecolor/pagecolor.dtx
%
%<*ignore>
\begingroup
  \catcode123=1 %
  \catcode125=2 %
  \def\x{LaTeX2e}%
\expandafter\endgroup
\ifcase 0\ifx\install y1\fi\expandafter
         \ifx\csname processbatchFile\endcsname\relax\else1\fi
         \ifx\fmtname\x\else 1\fi\relax
\else\csname fi\endcsname
%</ignore>
%<*install>
\input docstrip.tex
\Msg{*****************************************************************}
\Msg{* Installation                                                   }
\Msg{* Package: pagecolor 2012/02/23 v1.0e Provides thepagecolor (HMM)}
\Msg{*****************************************************************}

\keepsilent
\askforoverwritefalse

\let\MetaPrefix\relax
\preamble

This is a generated file.

Project: pagecolor
Version: 2012/02/23 v1.0e

Copyright (C) 2011 - 2012 by
    H.-Martin M"unch <Martin dot Muench at Uni-Bonn dot de>

The usual disclaimer applies:
If it doesn't work right that's your problem.
(Nevertheless, send an e-mail to the maintainer
 when you find an error in this package.)

This work may be distributed and/or modified under the
conditions of the LaTeX Project Public License, either
version 1.3c of this license or (at your option) any later
version. This version of this license is in
   http://www.latex-project.org/lppl/lppl-1-3c.txt
and the latest version of this license is in
   http://www.latex-project.org/lppl.txt
and version 1.3c or later is part of all distributions of
LaTeX version 2005/12/01 or later.

This work has the LPPL maintenance status "maintained".

The Current Maintainer of this work is H.-Martin Muench.

This work consists of the main source file pagecolor.dtx,
the README, and the derived files
   pagecolor.sty, pagecolor.pdf,
   pagecolor.ins, pagecolor.drv,
   pagecolor-example.tex, pagecolor-example.pdf.

\endpreamble
\let\MetaPrefix\DoubleperCent

\generate{%
  \file{pagecolor.ins}{\from{pagecolor.dtx}{install}}%
  \file{pagecolor.drv}{\from{pagecolor.dtx}{driver}}%
  \usedir{tex/latex/pagecolor}%
  \file{pagecolor.sty}{\from{pagecolor.dtx}{package}}%
  \usedir{doc/latex/pagecolor}%
  \file{pagecolor-example.tex}{\from{pagecolor.dtx}{example}}%
}

\catcode32=13\relax% active space
\let =\space%
\Msg{************************************************************************}
\Msg{*}
\Msg{* To finish the installation you have to move the following}
\Msg{* file into a directory searched by TeX:}
\Msg{*}
\Msg{*     pagecolor.sty}
\Msg{*}
\Msg{* To produce the documentation run the file `pagecolor.drv'}
\Msg{* through (pdf)LaTeX, e.g.}
\Msg{*  pdflatex pagecolor.drv}
\Msg{*  makeindex -s gind.ist pagecolor.idx}
\Msg{*  pdflatex pagecolor.drv}
\Msg{*  makeindex -s gind.ist pagecolor.idx}
\Msg{*  pdflatex pagecolor.drv}
\Msg{*}
\Msg{* At least three runs are necessary e.g. to get the}
\Msg{*  references right!}
\Msg{*}
\Msg{* Happy TeXing!}
\Msg{*}
\Msg{************************************************************************}

\endbatchfile
%</install>
%<*ignore>
\fi
%</ignore>
%
% \section{The documentation driver file}
%
% The next bit of code contains the documentation driver file for
% \TeX{}, i.\,e., the file that will produce the documentation you
% are currently reading. It will be extracted from this file by the
% \texttt{docstrip} programme. That is, run \LaTeX{} on \texttt{docstrip}
% and specify the \texttt{driver} option when \texttt{docstrip}
% asks for options.
%
%    \begin{macrocode}
%<*driver>
\NeedsTeXFormat{LaTeX2e}[2011/06/27]
\ProvidesFile{pagecolor.drv}%
  [2012/02/23 v1.0e Provides thepagecolor (HMM)]
\documentclass{ltxdoc}[2007/11/11]% v2.0u
\usepackage{holtxdoc}[2011/02/04]%  v0.21
%% pagecolor may work with earlier versions of LaTeX2e and those
%% class and package, but this was not tested.
%% Please consider updating your LaTeX, class, and package
%% to the most recent version (if they are not already the most
%% recent version).
\hypersetup{%
 pdfsubject={Providing the thepagecolor command (HMM)},%
 pdfkeywords={LaTeX, pagecolor, thepagecolor, H.-Martin Muench},%
 pdfencoding=auto,%
 pdflang={en},%
 breaklinks=true,%
 linktoc=all,%
 pdfstartview=FitH,%
 pdfpagelayout=OneColumn,%
 bookmarksnumbered=true,%
 bookmarksopen=true,%
 bookmarksopenlevel=3,%
 pdfmenubar=true,%
 pdftoolbar=true,%
 pdfwindowui=true,%
 pdfnewwindow=true%
}
\CodelineIndex
\hyphenation{printing docu-ment}
\gdef\unit#1{\mathord{\thinspace\mathrm{#1}}}%
\begin{document}
  \DocInput{pagecolor.dtx}%
\end{document}
%</driver>
%    \end{macrocode}
%
% \fi
%
% \CheckSum{137}
%
% \CharacterTable
%  {Upper-case    \A\B\C\D\E\F\G\H\I\J\K\L\M\N\O\P\Q\R\S\T\U\V\W\X\Y\Z
%   Lower-case    \a\b\c\d\e\f\g\h\i\j\k\l\m\n\o\p\q\r\s\t\u\v\w\x\y\z
%   Digits        \0\1\2\3\4\5\6\7\8\9
%   Exclamation   \!     Double quote  \"     Hash (number) \#
%   Dollar        \$     Percent       \%     Ampersand     \&
%   Acute accent  \'     Left paren    \(     Right paren   \)
%   Asterisk      \*     Plus          \+     Comma         \,
%   Minus         \-     Point         \.     Solidus       \/
%   Colon         \:     Semicolon     \;     Less than     \<
%   Equals        \=     Greater than  \>     Question mark \?
%   Commercial at \@     Left bracket  \[     Backslash     \\
%   Right bracket \]     Circumflex    \^     Underscore    \_
%   Grave accent  \`     Left brace    \{     Vertical bar  \|
%   Right brace   \}     Tilde         \~}
%
% \GetFileInfo{pagecolor.drv}
%
% \begingroup
%   \def\x{\#,\$,\^,\_,\~,\ ,\&,\{,\},\%}%
%   \makeatletter
%   \@onelevel@sanitize\x
% \expandafter\endgroup
% \expandafter\DoNotIndex\expandafter{\x}
% \expandafter\DoNotIndex\expandafter{\string\ }
% \begingroup
%   \makeatletter
%     \lccode`9=32\relax
%     \lowercase{%^^A
%       \edef\x{\noexpand\DoNotIndex{\@backslashchar9}}%^^A
%     }%^^A
%   \expandafter\endgroup\x
%
% \DoNotIndex{\,,\\}
% \DoNotIndex{\documentclass,\usepackage,\ProvidesPackage,\begin,\end}
% \DoNotIndex{\MessageBreak}
% \DoNotIndex{\NeedsTeXFormat,\DoNotIndex,\verb}
% \DoNotIndex{\def,\edef,\gdef,\global}
% \DoNotIndex{\ifx,\listfiles,\mathord,\mathrm}
% \DoNotIndex{\kvoptions,\SetupKeyvalOptions,\ProcessKeyvalOptions}
% \DoNotIndex{\bigskip,\space,\thinspace,\ldots}
% \DoNotIndex{\indent,\noindent,\newline,\linebreak,\pagebreak,\newpage}
% \DoNotIndex{\textbf,\textit,\textsf,\texttt,\textsc,\textquotedblleft,\textquotedblright}
% \DoNotIndex{\plainTeX,\TeX,\LaTeX,\pdfLaTeX}
% \DoNotIndex{\section,\color,\pagenumbering,\smallskip}
% \DoNotIndex{\@ifundefined,\@ifpackageloaded}
% \DoNotIndex{\DeclareStringOption}
%
% \title{The \xpackage{pagecolor} package}
% \date{2012/02/23 v1.0e}
% \author{H.-Martin M\"{u}nch\\\xemail{Martin.Muench at Uni-Bonn.de}}
%
% \maketitle
%
% \begin{abstract}
%  This \LaTeX{} package provides the command |\thepagecolor|,
% which gives the current page (background) colour,
% i.\,e.~the argument used with the most recent call of |\pagecolor{...}|.
% The command |\thepagecolornone| gives the same colour as |\thepagecolor|,
% except when the page background colour is
% \textquotedblleft|none|\textquotedblright. In that case |\thepagecolor| is
% |white| and |\thepagecolornone| is |none|.\\
% Similar to |\newgeometry| and |\restoregeometry| of the \xpackage{geometry}
% package\\
% |\newpagecolor{<|\textit{some colour} |>}| and |\restorepagecolor|
% are provided.
% \end{abstract}
%
% \bigskip
%
% \noindent Disclaimer for web links: The author is not responsible for any contents
% referred to in this work unless he has full knowledge of illegal contents.
% If any damage occurs by the use of information presented there, only the
% author of the respective pages might be liable, not the one who has referred
% to these pages.
%
% \bigskip
%
% \noindent {\color{green} Save per page about $200\unit{ml}$ water,
% $2\unit{g}$ CO$_{2}$ and $2\unit{g}$ wood:\\
% Therefore please print only if this is really necessary.}
%
% \pagebreak
%
% \tableofcontents
%
% \bigskip
%
% \section{Introduction\label{sec:Introduction}}
%
% \indent This \LaTeX{} package provides the command |\thepagecolor|,
% which gives the current page (background) colour,
% i.\,e.~the argument used with the most recent call of |\pagecolor{...}|.
% The package should be loaded before any package sets a page (background)
% colour, but after \xpackage{color} or \xpackage{xcolor} package.
% Its option |pagecolor={...}| is used to set the |\pagecolor{...}|.\\
% The command |\thepagecolornone| gives the same colour as |\thepagecolor|,
% except when the page background colour is
% \textquotedblleft|none|\textquotedblright (e.\,g. result of using the
% |\nopagecolor| command). In that case |\thepagecolor| is |white| and\linebreak
% |\thepagecolornone| is |none|.\\
% Similar to |\newgeometry| and |\restoregeometry| of the \xpackage{geometry}
% package\\
% |\newpagecolor{<|\textit{some colour}|>}| and |\restorepagecolor|
% are provided.
%
% \pagebreak
%
% \section{Usage}
%
% \indent Just load the package placing
% \begin{quote}
%   |\usepackage[<|\textit{option}|>]{pagecolor}|
% \end{quote}
% \noindent in the preamble of your \LaTeXe{} source file.
% This should be done before another package uses |\pagecolor|,
% but after |\nopagecolor| is defined (if that is defined in the document
% at all).\\
% Afterwards |\pagecolor{...}| can be used to change the page (background)
% colour as usual. Then |\thepagecolor| gives the current page (background)
% colour (in the same format as given with |\pagecolor{...}|).\\
% Similar to |\newgeometry| and |\restoregeometry| of the \xpackage{geometry}
% package\\
% |\newpagecolor{<|\textit{some colour} |>}| and |\restorepagecolor|
% are provided:\\
% |\newpagecolor{<|\textit{some colour} |>}| will execute
% |\pagecolor{<|\textit{some colour} |>}| and remember the page colour used before.
% |\restorecolor| (without argument) restores the page colour to the one used before
%  use of the |\newpagecolor{...}| command. When you want to change the colour for
%  just one page and do not want to (or cannot) manually determine where the page ends,\\
% |\newpagecolor{<|\textit{some colour}|>}\afterpage{\restorepagecolor}|\\
% does the trick (and requires a |\usepackage{afterpage}| in the document's
% preamble), or for short\\
% |\newcommand{\onepagecolor}[1]{%|\\
% |\newpagecolor{#1}\afterpage{\restorepagecolor}}|\\
% in the preamble and\\
% |\onepagecolor{<|\textit{some colour}|>}| in the document.
%
% \subsection{Option}
% \DescribeMacro{option}
% \indent The \xpackage{pagecolor} package takes the following option:
%
% \subsubsection{pagecolor\label{sss:pagecolor}}
% \DescribeMacro{pagecolor}
% \indent The option |pagecolor={...}| takes as value a colour. This
% could be as simple as |white| or |black|, but when e.\,g.~the
% \xpackage{xcolor} package is used (loaded before \xpackage{pagecolor}!),
% also colours like |red!50!green!20!blue| are possible.
% The default is |pagecolor={none}|.
%
% \section{Alternatives\label{sec:Alternatives}}
%
% As I neither know what exactly you want to accomplish
% when using this package (e.\,g.~hiding text), nor what resources you have
% (e.\,g.~\pdfTeX{} version), here is a list of some possible alternatives:\\
%
% \begin{description}
%  \item[-]
%  \DescribeMacro{transparent}
%  \xpackage{transparent} package: With it some object can be made
%  (fully or partially) transparent.
% \end{description}
%
% \begin{description}
%  \item[-]
%  \DescribeMacro{hrefhide}\xpackage{hrefhide} package: It allows to
%  \textquotedblleft hide\textquotedblright\ some (hyperlinked) text when
%  printing the document while keeping the layout.
% \end{description}
%
% \bigskip
%
% \noindent (You programmed or found another alternative,
%  which is available at \CTAN{}?\\
%  OK, send an e-mail to me with the name, location at \CTAN{},
%  and a short notice, and I will probably include it in
%  the list above.)\\
%
% \smallskip
%
% \noindent About how to get those packages, please see
% subsection~\ref{ss:Downloads}.
%
% \newpage
%
% \section{Example}
%
%    \begin{macrocode}
%<*example>
\documentclass[british]{article}[2007/10/19]% v1.4h
%%%%%%%%%%%%%%%%%%%%%%%%%%%%%%%%%%%%%%%%%%%%%%%%%%%%%%%%%%%%%%%%%%%%%
\usepackage[%
 extension=pdf,%
 plainpages=false,%
 pdfpagelabels=true,%
 hyperindex=false,%
 pdflang={en},%
 pdftitle={pagecolor package example},%
 pdfauthor={H.-Martin Muench},%
 pdfsubject={Example for the pagecolor package},%
 pdfkeywords={LaTeX, pagecolor, thepagecolor, page colour,%
  H.-Martin Muench},%
 pdfview=Fit,pdfstartview=Fit,%
 pdfpagelayout=SinglePage%
]{hyperref}[2012/02/06]% v6.82o
\usepackage[x11names]{xcolor}[2007/01/21]% v2.11
 % The xcolor package would not be needed for just using
 % the base colours. The color package would be sufficient for that.
\usepackage[pagecolor={LightGoldenrod1}]{pagecolor}[2012/02/23]% v1.0e

\usepackage{afterpage}[1995/10/27]% v1.08
 % The afterpage package is generally not needed,
 % but the |\newpagecolor{somecolour}\afterpage{\restorepagecolor}|
 % construct shall be demonstrated.

\usepackage{lipsum}[2011/04/14]% v1.2
 % The lipsum package is generally not needed,
 % but some blind text is needed for the example.

\gdef\unit#1{\mathord{\thinspace\mathrm{#1}}}%
\listfiles
\begin{document}
\pagenumbering{arabic}
\section*{Example for pagecolor}

This example demonstrates the use of package\newline
\textsf{pagecolor}, v1.0e as of 2012/02/23 (HMM).\newline
The used option was \verb|pagecolor={LightGoldenrod1}|.\newline
\verb|pagecolor={none}| would be the default.\newline

For more details please see the documentation!\newline

\noindent {\color{teal} Save per page about $200\unit{ml}$ water,
$2\unit{g}$ CO$_{2}$ and $2\unit{g}$ wood:\newline
Therefore please print only if this is really necessary.}\newline

The current page (background) colour is\newline
\verb|\thepagecolor|\ =\ \thepagecolor \newline
(and \verb|\thepagecolornone|\ =\ \thepagecolornone ,
which would only be different from \verb|\thepagecolor|,
when the page colour would be \verb|none|).

\pagebreak
\pagecolor{rgb:-green!40!yellow,3;green!40!yellow,2;red,1}

{\color{white} The current page (background) colour is\newline
\verb|\thepagecolor|\ =\ \thepagecolor . \newline}

{\color{\thepagecolor} And that makes this text practically invisible.
\newline}

{\color{white} Which made the preceding line of text practically
invisible.}

\pagebreak
\newpagecolor{red}

This page uses \verb|\newpagecolor{red}|.

\pagebreak
\restorepagecolor

{\color{white}And this page uses \verb|\restorepagecolor| to restore
the page colour to the value it had before the red page.}

\pagebreak
\pagecolor{none}

This page uses \verb|\pagecolor{none}|. If the \verb|\nopagecolor|
command is known (pdf\TeX and Lua\TeX; not yet for dvips, dvipdfm(x)
or Xe\TeX), the page colour is now \verb|none|, otherwise \verb|white|:
\verb|\thepagecolor|\ =\ \thepagecolor\ and
\verb|\thepagecolornone|\ =\ \thepagecolornone .

\pagebreak
\restorepagecolor

{\color{white}\verb|\restorepagecolor| restored the page colour again.}

\pagebreak
\pagecolor{green}

This page is green due to \verb|\pagecolor{green}|.

\pagebreak
\newpagecolor{blue}\afterpage{\restorepagecolor}

{\color{white}\verb|\newpagecolor{blue}\afterpage{\restorepagecolor}|%
\newline
was used here, i.\,e.~this page is blue, and the next one will
automatically have the same page colour before it was changed to blue
here (i.\,e. green).}

\smallskip
{\color{red}\textbf{\lipsum[1-11]}}
\bigskip

The page colour was changed back at the end of the page -
in mid-sentence!

\end{document}
%</example>
%    \end{macrocode}
%
% \StopEventually{}
%
% \newpage
%
% \section{The implementation}
%
% We start off by checking that we are loading into \LaTeXe{} and
% announcing the name and version of this package.
%
%    \begin{macrocode}
%<*package>
%    \end{macrocode}
%
%    \begin{macrocode}
\NeedsTeXFormat{LaTeX2e}[2011/06/27]
\ProvidesPackage{pagecolor}[2012/02/23 v1.0e
            Provides thepagecolor (HMM)]
%    \end{macrocode}
%
% A short description of the \xpackage{pagecolor} package:
%
%    \begin{macrocode}
%% Provides the \thepagecolor, \thepagecolornone, \newpagecolor{...},
%% and \restorepagecolor commands.
%    \end{macrocode}
%
% We need the \xpackage{kvoptions} package by \textsc{Heiko Oberdiek}:
%
%    \begin{macrocode}
\RequirePackage{kvoptions}[2010/12/23]% v3.10
%    \end{macrocode}
%
% and either the \xpackage{color} or the \xpackage{xcolor} package:
%
%    \begin{macrocode}
%% \RequirePackage{ either color or xcolor }:
\@ifpackageloaded{xcolor}{% xcolor loaded
   \@ifpackagelater{xcolor}{2007/01/21}{%
      % 2007/01/21, v2.11, or even more recent: OK
    }{% else: older package version
      \PackageWarning{pagecolor}{%
        It is requested version `2007/01/21' of package\MessageBreak%
        xcolor, but only an older version is available.\MessageBreak%
       }%
     }%
 }{% xcolor not loaded
   \@ifpackageloaded{color}{%
      \RequirePackage{color}[2005/11/14]% v1.0j
    }{% \else
      \PackageWarning{pagecolor}{%
        The pagecolor package must be loaded after either\MessageBreak%
        package color or after package xcolor (at your\MessageBreak%
        option). Neither package was loaded before package\MessageBreak%
        pagecolor. Loading of package xcolor will now be\MessageBreak%
        tried automatically.\MessageBreak%
        When the pagecolor package is used with option\MessageBreak%
        pagecolor using a colour requiring e. g. x11names\MessageBreak%
        option for xcolor package, this will not work!\MessageBreak%
       }
     }% \fi
   \RequirePackage{xcolor}[2007/01/21]% v2.11
  }% \fi
%    \end{macrocode}
%
% A last information for the user:
%
%    \begin{macrocode}
%% pagecolor may work with earlier versions of LaTeX and the
%% packages, but this was not tested. Please consider updating
%% your LaTeX and packages to the most recent version
%% (if they are not already the most recent version).

%    \end{macrocode}
%
% \indent See subsection~\ref{ss:Downloads} about how to get them.\\
%
% We process the options:
%
%    \begin{macrocode}
\SetupKeyvalOptions{family=pagecolor,prefix=pagecolor@}
\DeclareStringOption[none]{pagecolor}% \pagecolor@pagecolor
\ProcessKeyvalOptions*

%    \end{macrocode}
%
% We save the original |\pagecolor| command,
%
%    \begin{macrocode}
\let\origpagecolour\pagecolor

%    \end{macrocode}
%
%    \begin{macro}{\pagecolor}
% before we redefine it to include a definition of |\thepagecolor|
% and \linebreak
% |\thepagecolornone|:
%
%    \begin{macrocode}
\renewcommand{\pagecolor}[1]{\@bsphack%
  \edef\pagecolourtmpa{#1}%
  \def\pagecolourtmpb{none}%
  \ifx\pagecolourtmpa\pagecolourtmpb
    \@ifundefined{nopagecolor}{%
      \PackageError{pagecolor}{%
        pagecolor=none requested but \string\nopagecolor\space unknown%
       }{%
        \string\pagecolor{none} was used, but the command %
        \string\nopagecolor\space is undefined.\MessageBreak%
        Please use another colour. pagecolor=white will be used now.%
        \MessageBreak%
        }%
      \gdef\thepagecolor{white}%
      \gdef\thepagecolornone{white}% although it should be "none"
      \origpagecolour{white}%
     }{%
      \nopagecolor%
     }%
  \else%
    \xdef\thepagecolor{#1}%
    \xdef\thepagecolornone{#1}%
    \origpagecolour{\thepagecolornone}%
  \fi%
  \@esphack%
}

%    \end{macrocode}
%    \end{macro}
%
% |\nopagecolor| is only defined for \pdfTeX and \LuaTeX, but not yet for dvips,
% dvipdfm(x) or \XeTeX. Maybe
% \begin{verbatim}
% \@ifundefined{nopagecolor}{\newcommand{\nopagecolor}{\pagecolor{white}}}{}
% \end{verbatim}
% could be an alternative.\\
% When |\pagecolor{none}| is used and |\nopagecolor| is defined, then \\
% |\pagecolor{none}| is made into a synonym for |\nopagecolor|.\\
% If |\nopagecolor| has not been defined, nothing needs to be done.
% Otherwise we redefine |\nopagecolor|, thus that |\thepagecolor| is
% set to |white| and |\thepagecolornone| to |none| whenever
% |\nopagecolor| is used.
%
%    \begin{macrocode}
\@ifundefined{nopagecolor}{%
  \PackageWarning{pagecolor}{%
    \string\nopagecolor\space is undefined.\MessageBreak%
    (As long as it is neither used anywhere\MessageBreak%
    \space nor defined later, that will be no problem.)\MessageBreak%
   }
  \AtEndDocument{%
  \@ifundefined{nopagecolor}{% no problem
    }{%
    \PackageError{pagecolor}{%
      \string\nopagecolor\space defined after loading package%
      \MessageBreak%
      pagecolor%
      }{\string\nopagecolor\space was defined after loading the %
       pagecolor package.\MessageBreak%
       Either define \string\nopagecolor\space earlier or load the %
       pagecolor package later.\MessageBreak%
      }
    }%
  }%
  }{% \else
    \let\orignopagecolour\nopagecolor
    \renewcommand{\nopagecolor}{%
      \xdef\thepagecolor{white}
      \xdef\thepagecolornone{none}
      \orignopagecolour
      }
  }

%    \end{macrocode}
%
% If the page colour as given with option |pagecolor={...}| is |none|,
% but |\nopagecolor| is not known, |\pagecolor@pagecolor| is set to
% |white| and a warning is given.
%
%    \begin{macrocode}
\def\pagecolourtmpb{none}
\ifx\pagecolor@pagecolor\pagecolourtmpb
  \@ifundefined{nopagecolor}{%
    \PackageWarning{pagecolor}{%
      Option pagecolor=none (maybe by default) used,\MessageBreak%
      but \string\nopagecolor\space is unknown. Please use another%
      \MessageBreak%
      option value; white will be used now.\MessageBreak%
     }
    \setkeys{pagecolor}{pagecolor=white}%
  }{% ok
  }%
\fi

%    \end{macrocode}
%
% The (new) |\pagecolor| is now just carried out.
%
%    \begin{macrocode}
\pagecolor{\pagecolor@pagecolor}

%    \end{macrocode}
%
% Now the page (background) colour and |\thepagecolor| and \\
% |\thepagecolornone| are |\pagecolor@pagecolor| (or page (background) colour
% and |\thepagecolornone| are |none|=|\pagecolor@pagecolor| and
% |\thepagecolor| is |white|), and when the page (background) colour
% is changed, |\thepagecolor| and |\thepagecolornone| are changed
% accordingly.
%
% \newpage
%
%    \begin{macro}{\newpagecolor}
% There have been requests (via e-mail and at \\
% \url{http://tex.stackexchange.com/q/25137}) to change the colour of just one
% (or two) page(s) only, similar to |\newgeometry| and |\restoregeometry| of the
% \xpackage{geometry} package (\url{http://ctan.org/pkg/geometry}). Therefore
% |\newpagecolor| and |\restorepagecolor| are introduced (as suggested by
% \textsc{haoyun\_tex}):
%
%    \begin{macrocode}
\newcommand{\newpagecolor}[1]{%
\xdef\pagecolourtmpc{\thepagecolornone}%
\pagecolor{#1}%
}

%    \end{macrocode}
%    \end{macro}
%
% |\newpagecolor{<|\textit{some colour}|>}| will execute |\pagecolor{somecolour}|
% and remember the page colour used before.\\
%
%    \begin{macro}{\restorepagecolor}
%    \begin{macrocode}
\newcommand{\restorepagecolor}{\pagecolor{\pagecolourtmpc}}

%    \end{macrocode}
%    \end{macro}
%
% |\restorecolor| (without argument) restores the page colour to the one used before
% use of the |\newpagecolor{...}| command.
%
%    \begin{macrocode}
\gdef\pagecolourtmpc{\thepagecolor}

%    \end{macrocode}
% is just a precaution for |\restorecolor| being used when no |\newpagecolor{...}|
% was used before it.\\
%
% When you want to change the colour for just one page and do not want to
% (or cannot) manually determine where the page ends,\\
% |\newpagecolor{<|\textit{some colour}|>}\afterpage{\restorepagecolor}|\\
% does the trick (and requires an additional |\usepackage{afterpage}| in the
% document's preamble).
%    \begin{macrocode}
%</package>
%    \end{macrocode}
%
% \newpage
%
% \section{Installation}
%
% \subsection{Downloads\label{ss:Downloads}}
%
% Everything is available on \CTAN{}, \url{http://www.ctan.org/tex-archive/},
% but may need additional packages themselves.\\
%
% \DescribeMacro{pagecolor.dtx}
% For unpacking the |pagecolor.dtx| file and constructing the documentation
% it is required:
% \begin{description}
% \item[-] \TeX{} Format \LaTeXe{}, \url{http://www.CTAN.org/}
%
% \item[-] document class \xpackage{ltxdoc}, 2007/11/11, v2.0u,
%           \url{http://ctan.org/pkg/ltxdoc}
%
% \item[-] package \xpackage{holtxdoc}, 2011/02/04, v0.21,
%           \url{http://ctan.org/pkg/holtxdoc}
% \end{description}
%
% \DescribeMacro{pagecolor.sty}
% The |pagecolor.sty| for \LaTeXe{} (i.\,e. each document using
% the \xpackage{pagecolor} package) requires:
% \begin{description}
% \item[-] \TeX{} Format \LaTeXe{}, \url{http://www.CTAN.org/}
%
% \item[-] package \xpackage{kvoptions}, 2010/12/23, v3.10,
%           \url{http://ctan.org/pkg/kvoptions}
% \end{description}
% and either
% \begin{description}
% \item[-] package \xpackage{color}, 2005/11/14, v1.0j,
%           \url{http://ctan.org/pkg/color} (from the\\
%           \xpackage{graphics} package bundle)
% \end{description}
% or
% \begin{description}
% \item[-] package \xpackage{xcolor}, 2007/01/21, v2.11,
%           \url{http://ctan.org/pkg/xcolor}
% \end{description}
%
% \DescribeMacro{pagecolor-example.tex}
% The |pagecolor-example.tex| requires the same files as all
% documents using the \xpackage{pagecolor} package and additionally:
% \begin{description}
% \item[-] class \xpackage{article}, 2007/10/19, v1.4h, from \xpackage{classes.dtx}:\\
%           \CTAN{macros/latex/base/classes.dtx}
%
% \item[-] package \xpackage{pagecolor}, 2012/02/23, v1.0e,
%           \url{http://ctan.org/pkg/pagecolor}\\
%           (Well, it is the example file for this package, and because you are
%            reading the documentation for the \xpackage{pagecolor} package,
%            it can be assumed that you already have some version of it --
%            is it the current one?)
%
% \item[-] package \xpackage{xcolor}, 2007/01/21, v2.11,
%           \url{http://ctan.org/pkg/xcolor}\\
%           This package would not be needed for the use of just base colours only,
%           the \xpackage{color} package would be sufficient for that.
% \end{description}
%
% \DescribeMacro{transparent}
% \DescribeMacro{hrefhide}
% As possible alternatives in section \ref{sec:Alternatives} there are listed
% \begin{description}
% \item[-] package \xpackage{transparent}, 2007/01/08, v1.0, \\
%           \url{http://ctan.org/pkg/transparent}
%
% \item[-] package \xpackage{hrefhide}, 2011/04/29, v1.0f, \\
%           \url{http://ctan.org/pkg/hrefhide}
% \end{description}
%
% \DescribeMacro{Oberdiek}
% \DescribeMacro{holtxdoc}
% \DescribeMacro{kvoptions}
% All packages of \textsc{Heiko Oberdiek}'s bundle `oberdiek'
% (especially \xpackage{holtxdoc} and \xpackage{kvoptions})
% are also available in a TDS compliant ZIP archive:\\
% \CTAN{install/macros/latex/contrib/oberdiek.tds.zip}.\\
% It is probably best to download and use this, because the packages in there
% are quite probably both recent and compatible among themselves.\\
%
% \DescribeMacro{hyperref}
% \noindent \xpackage{hyperref} is not included in that bundle and needs to be
% downloaded separately,\\
% \url{http://mirror.ctan.org/install/macros/latex/contrib/hyperref.tds.zip}.\\
%
% \DescribeMacro{M\"{u}nch}
% A hyperlinked list of my (other) packages can be found at
% \url{http://www.Uni-Bonn.de/~uzs5pv/LaTeX.html}.\\
%
% \subsection{Package, unpacking TDS}
% \paragraph{Package.} This package is available on \CTAN{}
% \begin{description}
% \item[\CTAN{macros/latex/contrib/pagecolor/pagecolor.dtx}]\hspace*{0.1cm} \\
%       The source file.
% \item[\CTAN{macros/latex/contrib/pagecolor/pagecolor.pdf}]\hspace*{0.1cm} \\
%       The documentation.
% \item[\CTAN{macros/latex/contrib/pagecolor/pagecolor-example.pdf}]\hspace*{0.1cm} \\
%       The compiled example file, as it should look like.
% \item[\CTAN{macros/latex/contrib/pagecolor/README}]\hspace*{0.1cm} \\
%       The README file.
% \end{description}
% There is also a pagecolor.tds.zip available:
% \begin{description}
% \item[\CTAN{install/macros/latex/contrib/pagecolor.tds.zip}]\hspace*{0.1cm} \\
%       Everything in \xfile{TDS} compliant, compiled format.
% \end{description}
% which additionally contains\\
% \begin{tabular}{ll}
% pagecolor.ins & The installation file.\\
% pagecolor.drv & The driver to generate the documentation.\\
% pagecolor.sty & The \xext{sty}le file.\\
% pagecolor-example.tex & The example file.
% \end{tabular}
%
% \bigskip
%
% \noindent For required other packages, please see the preceding subsection.
%
% \paragraph{Unpacking.} The \xfile{.dtx} file is a self-extracting
% \docstrip{} archive. The files are extracted by running the
% \xext{.dtx} through \plainTeX{}:
% \begin{quote}
%   \verb|tex pagecolor.dtx|
% \end{quote}
%
% About generating the documentation see paragraph~\ref{GenDoc} below.\\
%
% \paragraph{TDS.} Now the different files must be moved into
% the different directories in your installation TDS tree
% (also known as \xfile{texmf} tree):
% \begin{quote}
% \def\t{^^A
% \begin{tabular}{@{}>{\ttfamily}l@{ $\rightarrow$ }>{\ttfamily}l@{}}
%   pagecolor.sty & tex/latex/pagecolor/pagecolor.sty\\
%   pagecolor.pdf & doc/latex/pagecolor/pagecolor.pdf\\
%   pagecolor-example.tex & doc/latex/pagecolor/pagecolor-example.tex\\
%   pagecolor-example.pdf & doc/latex/pagecolor/pagecolor-example.pdf\\
%   pagecolor.dtx & source/latex/pagecolor/pagecolor.dtx\\
% \end{tabular}^^A
% }^^A
% \sbox0{\t}^^A
% \ifdim\wd0>\linewidth
%   \begingroup
%     \advance\linewidth by\leftmargin
%     \advance\linewidth by\rightmargin
%   \edef\x{\endgroup
%     \def\noexpand\lw{\the\linewidth}^^A
%   }\x
%   \def\lwbox{^^A
%     \leavevmode
%     \hbox to \linewidth{^^A
%       \kern-\leftmargin\relax
%       \hss
%       \usebox0
%       \hss
%       \kern-\rightmargin\relax
%     }^^A
%   }^^A
%   \ifdim\wd0>\lw
%     \sbox0{\small\t}^^A
%     \ifdim\wd0>\linewidth
%       \ifdim\wd0>\lw
%         \sbox0{\footnotesize\t}^^A
%         \ifdim\wd0>\linewidth
%           \ifdim\wd0>\lw
%             \sbox0{\scriptsize\t}^^A
%             \ifdim\wd0>\linewidth
%               \ifdim\wd0>\lw
%                 \sbox0{\tiny\t}^^A
%                 \ifdim\wd0>\linewidth
%                   \lwbox
%                 \else
%                   \usebox0
%                 \fi
%               \else
%                 \lwbox
%               \fi
%             \else
%               \usebox0
%             \fi
%           \else
%             \lwbox
%           \fi
%         \else
%           \usebox0
%         \fi
%       \else
%         \lwbox
%       \fi
%     \else
%       \usebox0
%     \fi
%   \else
%     \lwbox
%   \fi
% \else
%   \usebox0
% \fi
% \end{quote}
% If you have a \xfile{docstrip.cfg} that configures and enables \docstrip{}'s
% \xfile{TDS} installing feature, then some files can already be in the right
% place, see the documentation of \docstrip{}.
%
% \subsection{Refresh file name databases}
%
% If your \TeX{}~distribution (\teTeX{}, \mikTeX{},\dots{}) relies on file name
% databases, you must refresh these. For example, \teTeX{} users run
% \verb|texhash| or \verb|mktexlsr|.
%
% \subsection{Some details for the interested}
%
% \paragraph{Unpacking with \LaTeX{}.}
% The \xfile{.dtx} chooses its action depending on the format:
% \begin{description}
% \item[\plainTeX:] Run \docstrip{} and extract the files.
% \item[\LaTeX:] Generate the documentation.
% \end{description}
% If you insist on using \LaTeX{} for \docstrip{} (really,
% \docstrip{} does not need \LaTeX{}), then inform the autodetect routine
% about your intention:
% \begin{quote}
%   \verb|latex \let\install=y% \iffalse meta-comment
%
% File: pagecolor.dtx
% Version: 2012/02/23 v1.0e
%
% Copyright (C) 2011 - 2012 by
%    H.-Martin M"unch <Martin dot Muench at Uni-Bonn dot de>
%
% This work may be distributed and/or modified under the
% conditions of the LaTeX Project Public License, either
% version 1.3c of this license or (at your option) any later
% version. This version of this license is in
%    http://www.latex-project.org/lppl/lppl-1-3c.txt
% and the latest version of this license is in
%    http://www.latex-project.org/lppl.txt
% and version 1.3c or later is part of all distributions of
% LaTeX version 2005/12/01 or later.
%
% This work has the LPPL maintenance status "maintained".
%
% The Current Maintainer of this work is H.-Martin Muench.
%
% This work consists of the main source file pagecolor.dtx,
% the README, and the derived files
%    pagecolor.sty, pagecolor.pdf,
%    pagecolor.ins, pagecolor.drv,
%    pagecolor-example.tex, pagecolor-example.pdf.
%
% Distribution:
%    CTAN:macros/latex/contrib/pagecolor/pagecolor.dtx
%    CTAN:macros/latex/contrib/pagecolor/pagecolor.pdf
%    CTAN:install/macros/latex/contrib/pagecolor.tds.zip
%
% Cf. http://ctan.org/pkg/pagecolor and
% http://www.tex.ac.uk/tex-archive/help/Catalogue/entries/pagecolor.html
%
% Unpacking:
%    (a) If pagecolor.ins is present:
%           tex pagecolor.ins
%    (b) Without pagecolor.ins:
%           tex pagecolor.dtx
%    (c) If you insist on using LaTeX
%           latex \let\install=y\input{pagecolor.dtx}
%        (quote the arguments according to the demands of your shell)
%
% Documentation:
%    (a) If pagecolor.drv is present:
%           (pdf)latex pagecolor.drv
%           makeindex -s gind.ist pagecolor.idx
%           (pdf)latex pagecolor.drv
%           makeindex -s gind.ist pagecolor.idx
%           (pdf)latex pagecolor.drv
%    (b) Without pagecolor.drv:
%           (pdf)latex pagecolor.dtx
%           makeindex -s gind.ist pagecolor.idx
%           (pdf)latex pagecolor.dtx
%           makeindex -s gind.ist pagecolor.idx
%           (pdf)latex pagecolor.dtx
%
%    The class ltxdoc loads the configuration file ltxdoc.cfg
%    if available. Here you can specify further options, e.g.
%    use DIN A4 as paper format:
%       \PassOptionsToClass{a4paper}{article}
%
% Installation:
%    TDS:tex/latex/pagecolor/pagecolor.sty
%    TDS:doc/latex/pagecolor/pagecolor.pdf
%    TDS:doc/latex/pagecolor/pagecolor-example.tex
%    TDS:doc/latex/pagecolor/pagecolor-example.pdf
%    TDS:source/latex/pagecolor/pagecolor.dtx
%
%<*ignore>
\begingroup
  \catcode123=1 %
  \catcode125=2 %
  \def\x{LaTeX2e}%
\expandafter\endgroup
\ifcase 0\ifx\install y1\fi\expandafter
         \ifx\csname processbatchFile\endcsname\relax\else1\fi
         \ifx\fmtname\x\else 1\fi\relax
\else\csname fi\endcsname
%</ignore>
%<*install>
\input docstrip.tex
\Msg{*****************************************************************}
\Msg{* Installation                                                   }
\Msg{* Package: pagecolor 2012/02/23 v1.0e Provides thepagecolor (HMM)}
\Msg{*****************************************************************}

\keepsilent
\askforoverwritefalse

\let\MetaPrefix\relax
\preamble

This is a generated file.

Project: pagecolor
Version: 2012/02/23 v1.0e

Copyright (C) 2011 - 2012 by
    H.-Martin M"unch <Martin dot Muench at Uni-Bonn dot de>

The usual disclaimer applies:
If it doesn't work right that's your problem.
(Nevertheless, send an e-mail to the maintainer
 when you find an error in this package.)

This work may be distributed and/or modified under the
conditions of the LaTeX Project Public License, either
version 1.3c of this license or (at your option) any later
version. This version of this license is in
   http://www.latex-project.org/lppl/lppl-1-3c.txt
and the latest version of this license is in
   http://www.latex-project.org/lppl.txt
and version 1.3c or later is part of all distributions of
LaTeX version 2005/12/01 or later.

This work has the LPPL maintenance status "maintained".

The Current Maintainer of this work is H.-Martin Muench.

This work consists of the main source file pagecolor.dtx,
the README, and the derived files
   pagecolor.sty, pagecolor.pdf,
   pagecolor.ins, pagecolor.drv,
   pagecolor-example.tex, pagecolor-example.pdf.

\endpreamble
\let\MetaPrefix\DoubleperCent

\generate{%
  \file{pagecolor.ins}{\from{pagecolor.dtx}{install}}%
  \file{pagecolor.drv}{\from{pagecolor.dtx}{driver}}%
  \usedir{tex/latex/pagecolor}%
  \file{pagecolor.sty}{\from{pagecolor.dtx}{package}}%
  \usedir{doc/latex/pagecolor}%
  \file{pagecolor-example.tex}{\from{pagecolor.dtx}{example}}%
}

\catcode32=13\relax% active space
\let =\space%
\Msg{************************************************************************}
\Msg{*}
\Msg{* To finish the installation you have to move the following}
\Msg{* file into a directory searched by TeX:}
\Msg{*}
\Msg{*     pagecolor.sty}
\Msg{*}
\Msg{* To produce the documentation run the file `pagecolor.drv'}
\Msg{* through (pdf)LaTeX, e.g.}
\Msg{*  pdflatex pagecolor.drv}
\Msg{*  makeindex -s gind.ist pagecolor.idx}
\Msg{*  pdflatex pagecolor.drv}
\Msg{*  makeindex -s gind.ist pagecolor.idx}
\Msg{*  pdflatex pagecolor.drv}
\Msg{*}
\Msg{* At least three runs are necessary e.g. to get the}
\Msg{*  references right!}
\Msg{*}
\Msg{* Happy TeXing!}
\Msg{*}
\Msg{************************************************************************}

\endbatchfile
%</install>
%<*ignore>
\fi
%</ignore>
%
% \section{The documentation driver file}
%
% The next bit of code contains the documentation driver file for
% \TeX{}, i.\,e., the file that will produce the documentation you
% are currently reading. It will be extracted from this file by the
% \texttt{docstrip} programme. That is, run \LaTeX{} on \texttt{docstrip}
% and specify the \texttt{driver} option when \texttt{docstrip}
% asks for options.
%
%    \begin{macrocode}
%<*driver>
\NeedsTeXFormat{LaTeX2e}[2011/06/27]
\ProvidesFile{pagecolor.drv}%
  [2012/02/23 v1.0e Provides thepagecolor (HMM)]
\documentclass{ltxdoc}[2007/11/11]% v2.0u
\usepackage{holtxdoc}[2011/02/04]%  v0.21
%% pagecolor may work with earlier versions of LaTeX2e and those
%% class and package, but this was not tested.
%% Please consider updating your LaTeX, class, and package
%% to the most recent version (if they are not already the most
%% recent version).
\hypersetup{%
 pdfsubject={Providing the thepagecolor command (HMM)},%
 pdfkeywords={LaTeX, pagecolor, thepagecolor, H.-Martin Muench},%
 pdfencoding=auto,%
 pdflang={en},%
 breaklinks=true,%
 linktoc=all,%
 pdfstartview=FitH,%
 pdfpagelayout=OneColumn,%
 bookmarksnumbered=true,%
 bookmarksopen=true,%
 bookmarksopenlevel=3,%
 pdfmenubar=true,%
 pdftoolbar=true,%
 pdfwindowui=true,%
 pdfnewwindow=true%
}
\CodelineIndex
\hyphenation{printing docu-ment}
\gdef\unit#1{\mathord{\thinspace\mathrm{#1}}}%
\begin{document}
  \DocInput{pagecolor.dtx}%
\end{document}
%</driver>
%    \end{macrocode}
%
% \fi
%
% \CheckSum{137}
%
% \CharacterTable
%  {Upper-case    \A\B\C\D\E\F\G\H\I\J\K\L\M\N\O\P\Q\R\S\T\U\V\W\X\Y\Z
%   Lower-case    \a\b\c\d\e\f\g\h\i\j\k\l\m\n\o\p\q\r\s\t\u\v\w\x\y\z
%   Digits        \0\1\2\3\4\5\6\7\8\9
%   Exclamation   \!     Double quote  \"     Hash (number) \#
%   Dollar        \$     Percent       \%     Ampersand     \&
%   Acute accent  \'     Left paren    \(     Right paren   \)
%   Asterisk      \*     Plus          \+     Comma         \,
%   Minus         \-     Point         \.     Solidus       \/
%   Colon         \:     Semicolon     \;     Less than     \<
%   Equals        \=     Greater than  \>     Question mark \?
%   Commercial at \@     Left bracket  \[     Backslash     \\
%   Right bracket \]     Circumflex    \^     Underscore    \_
%   Grave accent  \`     Left brace    \{     Vertical bar  \|
%   Right brace   \}     Tilde         \~}
%
% \GetFileInfo{pagecolor.drv}
%
% \begingroup
%   \def\x{\#,\$,\^,\_,\~,\ ,\&,\{,\},\%}%
%   \makeatletter
%   \@onelevel@sanitize\x
% \expandafter\endgroup
% \expandafter\DoNotIndex\expandafter{\x}
% \expandafter\DoNotIndex\expandafter{\string\ }
% \begingroup
%   \makeatletter
%     \lccode`9=32\relax
%     \lowercase{%^^A
%       \edef\x{\noexpand\DoNotIndex{\@backslashchar9}}%^^A
%     }%^^A
%   \expandafter\endgroup\x
%
% \DoNotIndex{\,,\\}
% \DoNotIndex{\documentclass,\usepackage,\ProvidesPackage,\begin,\end}
% \DoNotIndex{\MessageBreak}
% \DoNotIndex{\NeedsTeXFormat,\DoNotIndex,\verb}
% \DoNotIndex{\def,\edef,\gdef,\global}
% \DoNotIndex{\ifx,\listfiles,\mathord,\mathrm}
% \DoNotIndex{\kvoptions,\SetupKeyvalOptions,\ProcessKeyvalOptions}
% \DoNotIndex{\bigskip,\space,\thinspace,\ldots}
% \DoNotIndex{\indent,\noindent,\newline,\linebreak,\pagebreak,\newpage}
% \DoNotIndex{\textbf,\textit,\textsf,\texttt,\textsc,\textquotedblleft,\textquotedblright}
% \DoNotIndex{\plainTeX,\TeX,\LaTeX,\pdfLaTeX}
% \DoNotIndex{\section,\color,\pagenumbering,\smallskip}
% \DoNotIndex{\@ifundefined,\@ifpackageloaded}
% \DoNotIndex{\DeclareStringOption}
%
% \title{The \xpackage{pagecolor} package}
% \date{2012/02/23 v1.0e}
% \author{H.-Martin M\"{u}nch\\\xemail{Martin.Muench at Uni-Bonn.de}}
%
% \maketitle
%
% \begin{abstract}
%  This \LaTeX{} package provides the command |\thepagecolor|,
% which gives the current page (background) colour,
% i.\,e.~the argument used with the most recent call of |\pagecolor{...}|.
% The command |\thepagecolornone| gives the same colour as |\thepagecolor|,
% except when the page background colour is
% \textquotedblleft|none|\textquotedblright. In that case |\thepagecolor| is
% |white| and |\thepagecolornone| is |none|.\\
% Similar to |\newgeometry| and |\restoregeometry| of the \xpackage{geometry}
% package\\
% |\newpagecolor{<|\textit{some colour} |>}| and |\restorepagecolor|
% are provided.
% \end{abstract}
%
% \bigskip
%
% \noindent Disclaimer for web links: The author is not responsible for any contents
% referred to in this work unless he has full knowledge of illegal contents.
% If any damage occurs by the use of information presented there, only the
% author of the respective pages might be liable, not the one who has referred
% to these pages.
%
% \bigskip
%
% \noindent {\color{green} Save per page about $200\unit{ml}$ water,
% $2\unit{g}$ CO$_{2}$ and $2\unit{g}$ wood:\\
% Therefore please print only if this is really necessary.}
%
% \pagebreak
%
% \tableofcontents
%
% \bigskip
%
% \section{Introduction\label{sec:Introduction}}
%
% \indent This \LaTeX{} package provides the command |\thepagecolor|,
% which gives the current page (background) colour,
% i.\,e.~the argument used with the most recent call of |\pagecolor{...}|.
% The package should be loaded before any package sets a page (background)
% colour, but after \xpackage{color} or \xpackage{xcolor} package.
% Its option |pagecolor={...}| is used to set the |\pagecolor{...}|.\\
% The command |\thepagecolornone| gives the same colour as |\thepagecolor|,
% except when the page background colour is
% \textquotedblleft|none|\textquotedblright (e.\,g. result of using the
% |\nopagecolor| command). In that case |\thepagecolor| is |white| and\linebreak
% |\thepagecolornone| is |none|.\\
% Similar to |\newgeometry| and |\restoregeometry| of the \xpackage{geometry}
% package\\
% |\newpagecolor{<|\textit{some colour}|>}| and |\restorepagecolor|
% are provided.
%
% \pagebreak
%
% \section{Usage}
%
% \indent Just load the package placing
% \begin{quote}
%   |\usepackage[<|\textit{option}|>]{pagecolor}|
% \end{quote}
% \noindent in the preamble of your \LaTeXe{} source file.
% This should be done before another package uses |\pagecolor|,
% but after |\nopagecolor| is defined (if that is defined in the document
% at all).\\
% Afterwards |\pagecolor{...}| can be used to change the page (background)
% colour as usual. Then |\thepagecolor| gives the current page (background)
% colour (in the same format as given with |\pagecolor{...}|).\\
% Similar to |\newgeometry| and |\restoregeometry| of the \xpackage{geometry}
% package\\
% |\newpagecolor{<|\textit{some colour} |>}| and |\restorepagecolor|
% are provided:\\
% |\newpagecolor{<|\textit{some colour} |>}| will execute
% |\pagecolor{<|\textit{some colour} |>}| and remember the page colour used before.
% |\restorecolor| (without argument) restores the page colour to the one used before
%  use of the |\newpagecolor{...}| command. When you want to change the colour for
%  just one page and do not want to (or cannot) manually determine where the page ends,\\
% |\newpagecolor{<|\textit{some colour}|>}\afterpage{\restorepagecolor}|\\
% does the trick (and requires a |\usepackage{afterpage}| in the document's
% preamble), or for short\\
% |\newcommand{\onepagecolor}[1]{%|\\
% |\newpagecolor{#1}\afterpage{\restorepagecolor}}|\\
% in the preamble and\\
% |\onepagecolor{<|\textit{some colour}|>}| in the document.
%
% \subsection{Option}
% \DescribeMacro{option}
% \indent The \xpackage{pagecolor} package takes the following option:
%
% \subsubsection{pagecolor\label{sss:pagecolor}}
% \DescribeMacro{pagecolor}
% \indent The option |pagecolor={...}| takes as value a colour. This
% could be as simple as |white| or |black|, but when e.\,g.~the
% \xpackage{xcolor} package is used (loaded before \xpackage{pagecolor}!),
% also colours like |red!50!green!20!blue| are possible.
% The default is |pagecolor={none}|.
%
% \section{Alternatives\label{sec:Alternatives}}
%
% As I neither know what exactly you want to accomplish
% when using this package (e.\,g.~hiding text), nor what resources you have
% (e.\,g.~\pdfTeX{} version), here is a list of some possible alternatives:\\
%
% \begin{description}
%  \item[-]
%  \DescribeMacro{transparent}
%  \xpackage{transparent} package: With it some object can be made
%  (fully or partially) transparent.
% \end{description}
%
% \begin{description}
%  \item[-]
%  \DescribeMacro{hrefhide}\xpackage{hrefhide} package: It allows to
%  \textquotedblleft hide\textquotedblright\ some (hyperlinked) text when
%  printing the document while keeping the layout.
% \end{description}
%
% \bigskip
%
% \noindent (You programmed or found another alternative,
%  which is available at \CTAN{}?\\
%  OK, send an e-mail to me with the name, location at \CTAN{},
%  and a short notice, and I will probably include it in
%  the list above.)\\
%
% \smallskip
%
% \noindent About how to get those packages, please see
% subsection~\ref{ss:Downloads}.
%
% \newpage
%
% \section{Example}
%
%    \begin{macrocode}
%<*example>
\documentclass[british]{article}[2007/10/19]% v1.4h
%%%%%%%%%%%%%%%%%%%%%%%%%%%%%%%%%%%%%%%%%%%%%%%%%%%%%%%%%%%%%%%%%%%%%
\usepackage[%
 extension=pdf,%
 plainpages=false,%
 pdfpagelabels=true,%
 hyperindex=false,%
 pdflang={en},%
 pdftitle={pagecolor package example},%
 pdfauthor={H.-Martin Muench},%
 pdfsubject={Example for the pagecolor package},%
 pdfkeywords={LaTeX, pagecolor, thepagecolor, page colour,%
  H.-Martin Muench},%
 pdfview=Fit,pdfstartview=Fit,%
 pdfpagelayout=SinglePage%
]{hyperref}[2012/02/06]% v6.82o
\usepackage[x11names]{xcolor}[2007/01/21]% v2.11
 % The xcolor package would not be needed for just using
 % the base colours. The color package would be sufficient for that.
\usepackage[pagecolor={LightGoldenrod1}]{pagecolor}[2012/02/23]% v1.0e

\usepackage{afterpage}[1995/10/27]% v1.08
 % The afterpage package is generally not needed,
 % but the |\newpagecolor{somecolour}\afterpage{\restorepagecolor}|
 % construct shall be demonstrated.

\usepackage{lipsum}[2011/04/14]% v1.2
 % The lipsum package is generally not needed,
 % but some blind text is needed for the example.

\gdef\unit#1{\mathord{\thinspace\mathrm{#1}}}%
\listfiles
\begin{document}
\pagenumbering{arabic}
\section*{Example for pagecolor}

This example demonstrates the use of package\newline
\textsf{pagecolor}, v1.0e as of 2012/02/23 (HMM).\newline
The used option was \verb|pagecolor={LightGoldenrod1}|.\newline
\verb|pagecolor={none}| would be the default.\newline

For more details please see the documentation!\newline

\noindent {\color{teal} Save per page about $200\unit{ml}$ water,
$2\unit{g}$ CO$_{2}$ and $2\unit{g}$ wood:\newline
Therefore please print only if this is really necessary.}\newline

The current page (background) colour is\newline
\verb|\thepagecolor|\ =\ \thepagecolor \newline
(and \verb|\thepagecolornone|\ =\ \thepagecolornone ,
which would only be different from \verb|\thepagecolor|,
when the page colour would be \verb|none|).

\pagebreak
\pagecolor{rgb:-green!40!yellow,3;green!40!yellow,2;red,1}

{\color{white} The current page (background) colour is\newline
\verb|\thepagecolor|\ =\ \thepagecolor . \newline}

{\color{\thepagecolor} And that makes this text practically invisible.
\newline}

{\color{white} Which made the preceding line of text practically
invisible.}

\pagebreak
\newpagecolor{red}

This page uses \verb|\newpagecolor{red}|.

\pagebreak
\restorepagecolor

{\color{white}And this page uses \verb|\restorepagecolor| to restore
the page colour to the value it had before the red page.}

\pagebreak
\pagecolor{none}

This page uses \verb|\pagecolor{none}|. If the \verb|\nopagecolor|
command is known (pdf\TeX and Lua\TeX; not yet for dvips, dvipdfm(x)
or Xe\TeX), the page colour is now \verb|none|, otherwise \verb|white|:
\verb|\thepagecolor|\ =\ \thepagecolor\ and
\verb|\thepagecolornone|\ =\ \thepagecolornone .

\pagebreak
\restorepagecolor

{\color{white}\verb|\restorepagecolor| restored the page colour again.}

\pagebreak
\pagecolor{green}

This page is green due to \verb|\pagecolor{green}|.

\pagebreak
\newpagecolor{blue}\afterpage{\restorepagecolor}

{\color{white}\verb|\newpagecolor{blue}\afterpage{\restorepagecolor}|%
\newline
was used here, i.\,e.~this page is blue, and the next one will
automatically have the same page colour before it was changed to blue
here (i.\,e. green).}

\smallskip
{\color{red}\textbf{\lipsum[1-11]}}
\bigskip

The page colour was changed back at the end of the page -
in mid-sentence!

\end{document}
%</example>
%    \end{macrocode}
%
% \StopEventually{}
%
% \newpage
%
% \section{The implementation}
%
% We start off by checking that we are loading into \LaTeXe{} and
% announcing the name and version of this package.
%
%    \begin{macrocode}
%<*package>
%    \end{macrocode}
%
%    \begin{macrocode}
\NeedsTeXFormat{LaTeX2e}[2011/06/27]
\ProvidesPackage{pagecolor}[2012/02/23 v1.0e
            Provides thepagecolor (HMM)]
%    \end{macrocode}
%
% A short description of the \xpackage{pagecolor} package:
%
%    \begin{macrocode}
%% Provides the \thepagecolor, \thepagecolornone, \newpagecolor{...},
%% and \restorepagecolor commands.
%    \end{macrocode}
%
% We need the \xpackage{kvoptions} package by \textsc{Heiko Oberdiek}:
%
%    \begin{macrocode}
\RequirePackage{kvoptions}[2010/12/23]% v3.10
%    \end{macrocode}
%
% and either the \xpackage{color} or the \xpackage{xcolor} package:
%
%    \begin{macrocode}
%% \RequirePackage{ either color or xcolor }:
\@ifpackageloaded{xcolor}{% xcolor loaded
   \@ifpackagelater{xcolor}{2007/01/21}{%
      % 2007/01/21, v2.11, or even more recent: OK
    }{% else: older package version
      \PackageWarning{pagecolor}{%
        It is requested version `2007/01/21' of package\MessageBreak%
        xcolor, but only an older version is available.\MessageBreak%
       }%
     }%
 }{% xcolor not loaded
   \@ifpackageloaded{color}{%
      \RequirePackage{color}[2005/11/14]% v1.0j
    }{% \else
      \PackageWarning{pagecolor}{%
        The pagecolor package must be loaded after either\MessageBreak%
        package color or after package xcolor (at your\MessageBreak%
        option). Neither package was loaded before package\MessageBreak%
        pagecolor. Loading of package xcolor will now be\MessageBreak%
        tried automatically.\MessageBreak%
        When the pagecolor package is used with option\MessageBreak%
        pagecolor using a colour requiring e. g. x11names\MessageBreak%
        option for xcolor package, this will not work!\MessageBreak%
       }
     }% \fi
   \RequirePackage{xcolor}[2007/01/21]% v2.11
  }% \fi
%    \end{macrocode}
%
% A last information for the user:
%
%    \begin{macrocode}
%% pagecolor may work with earlier versions of LaTeX and the
%% packages, but this was not tested. Please consider updating
%% your LaTeX and packages to the most recent version
%% (if they are not already the most recent version).

%    \end{macrocode}
%
% \indent See subsection~\ref{ss:Downloads} about how to get them.\\
%
% We process the options:
%
%    \begin{macrocode}
\SetupKeyvalOptions{family=pagecolor,prefix=pagecolor@}
\DeclareStringOption[none]{pagecolor}% \pagecolor@pagecolor
\ProcessKeyvalOptions*

%    \end{macrocode}
%
% We save the original |\pagecolor| command,
%
%    \begin{macrocode}
\let\origpagecolour\pagecolor

%    \end{macrocode}
%
%    \begin{macro}{\pagecolor}
% before we redefine it to include a definition of |\thepagecolor|
% and \linebreak
% |\thepagecolornone|:
%
%    \begin{macrocode}
\renewcommand{\pagecolor}[1]{\@bsphack%
  \edef\pagecolourtmpa{#1}%
  \def\pagecolourtmpb{none}%
  \ifx\pagecolourtmpa\pagecolourtmpb
    \@ifundefined{nopagecolor}{%
      \PackageError{pagecolor}{%
        pagecolor=none requested but \string\nopagecolor\space unknown%
       }{%
        \string\pagecolor{none} was used, but the command %
        \string\nopagecolor\space is undefined.\MessageBreak%
        Please use another colour. pagecolor=white will be used now.%
        \MessageBreak%
        }%
      \gdef\thepagecolor{white}%
      \gdef\thepagecolornone{white}% although it should be "none"
      \origpagecolour{white}%
     }{%
      \nopagecolor%
     }%
  \else%
    \xdef\thepagecolor{#1}%
    \xdef\thepagecolornone{#1}%
    \origpagecolour{\thepagecolornone}%
  \fi%
  \@esphack%
}

%    \end{macrocode}
%    \end{macro}
%
% |\nopagecolor| is only defined for \pdfTeX and \LuaTeX, but not yet for dvips,
% dvipdfm(x) or \XeTeX. Maybe
% \begin{verbatim}
% \@ifundefined{nopagecolor}{\newcommand{\nopagecolor}{\pagecolor{white}}}{}
% \end{verbatim}
% could be an alternative.\\
% When |\pagecolor{none}| is used and |\nopagecolor| is defined, then \\
% |\pagecolor{none}| is made into a synonym for |\nopagecolor|.\\
% If |\nopagecolor| has not been defined, nothing needs to be done.
% Otherwise we redefine |\nopagecolor|, thus that |\thepagecolor| is
% set to |white| and |\thepagecolornone| to |none| whenever
% |\nopagecolor| is used.
%
%    \begin{macrocode}
\@ifundefined{nopagecolor}{%
  \PackageWarning{pagecolor}{%
    \string\nopagecolor\space is undefined.\MessageBreak%
    (As long as it is neither used anywhere\MessageBreak%
    \space nor defined later, that will be no problem.)\MessageBreak%
   }
  \AtEndDocument{%
  \@ifundefined{nopagecolor}{% no problem
    }{%
    \PackageError{pagecolor}{%
      \string\nopagecolor\space defined after loading package%
      \MessageBreak%
      pagecolor%
      }{\string\nopagecolor\space was defined after loading the %
       pagecolor package.\MessageBreak%
       Either define \string\nopagecolor\space earlier or load the %
       pagecolor package later.\MessageBreak%
      }
    }%
  }%
  }{% \else
    \let\orignopagecolour\nopagecolor
    \renewcommand{\nopagecolor}{%
      \xdef\thepagecolor{white}
      \xdef\thepagecolornone{none}
      \orignopagecolour
      }
  }

%    \end{macrocode}
%
% If the page colour as given with option |pagecolor={...}| is |none|,
% but |\nopagecolor| is not known, |\pagecolor@pagecolor| is set to
% |white| and a warning is given.
%
%    \begin{macrocode}
\def\pagecolourtmpb{none}
\ifx\pagecolor@pagecolor\pagecolourtmpb
  \@ifundefined{nopagecolor}{%
    \PackageWarning{pagecolor}{%
      Option pagecolor=none (maybe by default) used,\MessageBreak%
      but \string\nopagecolor\space is unknown. Please use another%
      \MessageBreak%
      option value; white will be used now.\MessageBreak%
     }
    \setkeys{pagecolor}{pagecolor=white}%
  }{% ok
  }%
\fi

%    \end{macrocode}
%
% The (new) |\pagecolor| is now just carried out.
%
%    \begin{macrocode}
\pagecolor{\pagecolor@pagecolor}

%    \end{macrocode}
%
% Now the page (background) colour and |\thepagecolor| and \\
% |\thepagecolornone| are |\pagecolor@pagecolor| (or page (background) colour
% and |\thepagecolornone| are |none|=|\pagecolor@pagecolor| and
% |\thepagecolor| is |white|), and when the page (background) colour
% is changed, |\thepagecolor| and |\thepagecolornone| are changed
% accordingly.
%
% \newpage
%
%    \begin{macro}{\newpagecolor}
% There have been requests (via e-mail and at \\
% \url{http://tex.stackexchange.com/q/25137}) to change the colour of just one
% (or two) page(s) only, similar to |\newgeometry| and |\restoregeometry| of the
% \xpackage{geometry} package (\url{http://ctan.org/pkg/geometry}). Therefore
% |\newpagecolor| and |\restorepagecolor| are introduced (as suggested by
% \textsc{haoyun\_tex}):
%
%    \begin{macrocode}
\newcommand{\newpagecolor}[1]{%
\xdef\pagecolourtmpc{\thepagecolornone}%
\pagecolor{#1}%
}

%    \end{macrocode}
%    \end{macro}
%
% |\newpagecolor{<|\textit{some colour}|>}| will execute |\pagecolor{somecolour}|
% and remember the page colour used before.\\
%
%    \begin{macro}{\restorepagecolor}
%    \begin{macrocode}
\newcommand{\restorepagecolor}{\pagecolor{\pagecolourtmpc}}

%    \end{macrocode}
%    \end{macro}
%
% |\restorecolor| (without argument) restores the page colour to the one used before
% use of the |\newpagecolor{...}| command.
%
%    \begin{macrocode}
\gdef\pagecolourtmpc{\thepagecolor}

%    \end{macrocode}
% is just a precaution for |\restorecolor| being used when no |\newpagecolor{...}|
% was used before it.\\
%
% When you want to change the colour for just one page and do not want to
% (or cannot) manually determine where the page ends,\\
% |\newpagecolor{<|\textit{some colour}|>}\afterpage{\restorepagecolor}|\\
% does the trick (and requires an additional |\usepackage{afterpage}| in the
% document's preamble).
%    \begin{macrocode}
%</package>
%    \end{macrocode}
%
% \newpage
%
% \section{Installation}
%
% \subsection{Downloads\label{ss:Downloads}}
%
% Everything is available on \CTAN{}, \url{http://www.ctan.org/tex-archive/},
% but may need additional packages themselves.\\
%
% \DescribeMacro{pagecolor.dtx}
% For unpacking the |pagecolor.dtx| file and constructing the documentation
% it is required:
% \begin{description}
% \item[-] \TeX{} Format \LaTeXe{}, \url{http://www.CTAN.org/}
%
% \item[-] document class \xpackage{ltxdoc}, 2007/11/11, v2.0u,
%           \url{http://ctan.org/pkg/ltxdoc}
%
% \item[-] package \xpackage{holtxdoc}, 2011/02/04, v0.21,
%           \url{http://ctan.org/pkg/holtxdoc}
% \end{description}
%
% \DescribeMacro{pagecolor.sty}
% The |pagecolor.sty| for \LaTeXe{} (i.\,e. each document using
% the \xpackage{pagecolor} package) requires:
% \begin{description}
% \item[-] \TeX{} Format \LaTeXe{}, \url{http://www.CTAN.org/}
%
% \item[-] package \xpackage{kvoptions}, 2010/12/23, v3.10,
%           \url{http://ctan.org/pkg/kvoptions}
% \end{description}
% and either
% \begin{description}
% \item[-] package \xpackage{color}, 2005/11/14, v1.0j,
%           \url{http://ctan.org/pkg/color} (from the\\
%           \xpackage{graphics} package bundle)
% \end{description}
% or
% \begin{description}
% \item[-] package \xpackage{xcolor}, 2007/01/21, v2.11,
%           \url{http://ctan.org/pkg/xcolor}
% \end{description}
%
% \DescribeMacro{pagecolor-example.tex}
% The |pagecolor-example.tex| requires the same files as all
% documents using the \xpackage{pagecolor} package and additionally:
% \begin{description}
% \item[-] class \xpackage{article}, 2007/10/19, v1.4h, from \xpackage{classes.dtx}:\\
%           \CTAN{macros/latex/base/classes.dtx}
%
% \item[-] package \xpackage{pagecolor}, 2012/02/23, v1.0e,
%           \url{http://ctan.org/pkg/pagecolor}\\
%           (Well, it is the example file for this package, and because you are
%            reading the documentation for the \xpackage{pagecolor} package,
%            it can be assumed that you already have some version of it --
%            is it the current one?)
%
% \item[-] package \xpackage{xcolor}, 2007/01/21, v2.11,
%           \url{http://ctan.org/pkg/xcolor}\\
%           This package would not be needed for the use of just base colours only,
%           the \xpackage{color} package would be sufficient for that.
% \end{description}
%
% \DescribeMacro{transparent}
% \DescribeMacro{hrefhide}
% As possible alternatives in section \ref{sec:Alternatives} there are listed
% \begin{description}
% \item[-] package \xpackage{transparent}, 2007/01/08, v1.0, \\
%           \url{http://ctan.org/pkg/transparent}
%
% \item[-] package \xpackage{hrefhide}, 2011/04/29, v1.0f, \\
%           \url{http://ctan.org/pkg/hrefhide}
% \end{description}
%
% \DescribeMacro{Oberdiek}
% \DescribeMacro{holtxdoc}
% \DescribeMacro{kvoptions}
% All packages of \textsc{Heiko Oberdiek}'s bundle `oberdiek'
% (especially \xpackage{holtxdoc} and \xpackage{kvoptions})
% are also available in a TDS compliant ZIP archive:\\
% \CTAN{install/macros/latex/contrib/oberdiek.tds.zip}.\\
% It is probably best to download and use this, because the packages in there
% are quite probably both recent and compatible among themselves.\\
%
% \DescribeMacro{hyperref}
% \noindent \xpackage{hyperref} is not included in that bundle and needs to be
% downloaded separately,\\
% \url{http://mirror.ctan.org/install/macros/latex/contrib/hyperref.tds.zip}.\\
%
% \DescribeMacro{M\"{u}nch}
% A hyperlinked list of my (other) packages can be found at
% \url{http://www.Uni-Bonn.de/~uzs5pv/LaTeX.html}.\\
%
% \subsection{Package, unpacking TDS}
% \paragraph{Package.} This package is available on \CTAN{}
% \begin{description}
% \item[\CTAN{macros/latex/contrib/pagecolor/pagecolor.dtx}]\hspace*{0.1cm} \\
%       The source file.
% \item[\CTAN{macros/latex/contrib/pagecolor/pagecolor.pdf}]\hspace*{0.1cm} \\
%       The documentation.
% \item[\CTAN{macros/latex/contrib/pagecolor/pagecolor-example.pdf}]\hspace*{0.1cm} \\
%       The compiled example file, as it should look like.
% \item[\CTAN{macros/latex/contrib/pagecolor/README}]\hspace*{0.1cm} \\
%       The README file.
% \end{description}
% There is also a pagecolor.tds.zip available:
% \begin{description}
% \item[\CTAN{install/macros/latex/contrib/pagecolor.tds.zip}]\hspace*{0.1cm} \\
%       Everything in \xfile{TDS} compliant, compiled format.
% \end{description}
% which additionally contains\\
% \begin{tabular}{ll}
% pagecolor.ins & The installation file.\\
% pagecolor.drv & The driver to generate the documentation.\\
% pagecolor.sty & The \xext{sty}le file.\\
% pagecolor-example.tex & The example file.
% \end{tabular}
%
% \bigskip
%
% \noindent For required other packages, please see the preceding subsection.
%
% \paragraph{Unpacking.} The \xfile{.dtx} file is a self-extracting
% \docstrip{} archive. The files are extracted by running the
% \xext{.dtx} through \plainTeX{}:
% \begin{quote}
%   \verb|tex pagecolor.dtx|
% \end{quote}
%
% About generating the documentation see paragraph~\ref{GenDoc} below.\\
%
% \paragraph{TDS.} Now the different files must be moved into
% the different directories in your installation TDS tree
% (also known as \xfile{texmf} tree):
% \begin{quote}
% \def\t{^^A
% \begin{tabular}{@{}>{\ttfamily}l@{ $\rightarrow$ }>{\ttfamily}l@{}}
%   pagecolor.sty & tex/latex/pagecolor/pagecolor.sty\\
%   pagecolor.pdf & doc/latex/pagecolor/pagecolor.pdf\\
%   pagecolor-example.tex & doc/latex/pagecolor/pagecolor-example.tex\\
%   pagecolor-example.pdf & doc/latex/pagecolor/pagecolor-example.pdf\\
%   pagecolor.dtx & source/latex/pagecolor/pagecolor.dtx\\
% \end{tabular}^^A
% }^^A
% \sbox0{\t}^^A
% \ifdim\wd0>\linewidth
%   \begingroup
%     \advance\linewidth by\leftmargin
%     \advance\linewidth by\rightmargin
%   \edef\x{\endgroup
%     \def\noexpand\lw{\the\linewidth}^^A
%   }\x
%   \def\lwbox{^^A
%     \leavevmode
%     \hbox to \linewidth{^^A
%       \kern-\leftmargin\relax
%       \hss
%       \usebox0
%       \hss
%       \kern-\rightmargin\relax
%     }^^A
%   }^^A
%   \ifdim\wd0>\lw
%     \sbox0{\small\t}^^A
%     \ifdim\wd0>\linewidth
%       \ifdim\wd0>\lw
%         \sbox0{\footnotesize\t}^^A
%         \ifdim\wd0>\linewidth
%           \ifdim\wd0>\lw
%             \sbox0{\scriptsize\t}^^A
%             \ifdim\wd0>\linewidth
%               \ifdim\wd0>\lw
%                 \sbox0{\tiny\t}^^A
%                 \ifdim\wd0>\linewidth
%                   \lwbox
%                 \else
%                   \usebox0
%                 \fi
%               \else
%                 \lwbox
%               \fi
%             \else
%               \usebox0
%             \fi
%           \else
%             \lwbox
%           \fi
%         \else
%           \usebox0
%         \fi
%       \else
%         \lwbox
%       \fi
%     \else
%       \usebox0
%     \fi
%   \else
%     \lwbox
%   \fi
% \else
%   \usebox0
% \fi
% \end{quote}
% If you have a \xfile{docstrip.cfg} that configures and enables \docstrip{}'s
% \xfile{TDS} installing feature, then some files can already be in the right
% place, see the documentation of \docstrip{}.
%
% \subsection{Refresh file name databases}
%
% If your \TeX{}~distribution (\teTeX{}, \mikTeX{},\dots{}) relies on file name
% databases, you must refresh these. For example, \teTeX{} users run
% \verb|texhash| or \verb|mktexlsr|.
%
% \subsection{Some details for the interested}
%
% \paragraph{Unpacking with \LaTeX{}.}
% The \xfile{.dtx} chooses its action depending on the format:
% \begin{description}
% \item[\plainTeX:] Run \docstrip{} and extract the files.
% \item[\LaTeX:] Generate the documentation.
% \end{description}
% If you insist on using \LaTeX{} for \docstrip{} (really,
% \docstrip{} does not need \LaTeX{}), then inform the autodetect routine
% about your intention:
% \begin{quote}
%   \verb|latex \let\install=y\input{pagecolor.dtx}|
% \end{quote}
% Do not forget to quote the argument according to the demands
% of your shell.
%
% \paragraph{Generating the documentation.\label{GenDoc}}
% You can use both the \xfile{.dtx} or the \xfile{.drv} to generate
% the documentation. The process can be configured by a
% configuration file \xfile{ltxdoc.cfg}. For instance, put the following
% line into this file, if you want to have A4 as paper format:
% \begin{quote}
%   \verb|\PassOptionsToClass{a4paper}{article}|
% \end{quote}
%
% \noindent An example follows how to generate the
% documentation with \pdfLaTeX{}:
%
% \begin{quote}
%\begin{verbatim}
%pdflatex pagecolor.dtx
%makeindex -s gind.ist pagecolor.idx
%pdflatex pagecolor.dtx
%makeindex -s gind.ist pagecolor.idx
%pdflatex pagecolor.dtx
%\end{verbatim}
% \end{quote}
%
% \subsection{Compiling the example}
%
% The example file, \textsf{pagecolor-example.tex}, can be compiled via\\
% |(pdf)latex pagecolor-example.tex|.\\
%
% \section{Acknowledgements}
%
% I would like to thank \textsc{Heiko Oberdiek} for providing a~lot~(!) of
% useful packages (from which I also got everything I know about creating
% a file in \xext{dtx} format, ok, say it: copying), and the
% \Newsgroup{comp.text.tex} and \Newsgroup{de.comp.text.tex} newsgroups
% for their help in all things \TeX{}, especially all contributors to the
% discussion at
% \url{http://groups.google.com/group/comp.text.tex/browse_thread/thread/533576ebe44d60f6/e1654d70a19de63c?lnk=gst&q=Determination+of+the+current+page+colour#e1654d70a19de63c}
% \ (\textsc{H.~Oberdiek} \& \textsc{Gouailles}).\\
% I thank \textsc{haoyun\_tex} for suggesting the |\newpagecolor|/|\restorepagecolor|
% pair of commands and everyone at \url{http://tex.stackexchange.com/q/25137}
% for their contributions there.
%
% \bigskip
%
% \phantomsection
% \begin{History}\label{History}
%   \begin{Version}{2011/07/16 v1.0a}
%     \item First version discussed at \Newsgroup{comp.text.tex}.
%   \end{Version}
%   \begin{Version}{2011/08/06 v1.0b}
%     \item Changed version uploaded to \CTAN{}.
%   \end{Version}
%   \begin{Version}{2011/08/08 v1.0c}
%     \item Fixed a |\setkeys|.
%   \end{Version}
%   \begin{Version}{2012/02/01 v1.0d}
%     \item Bugfix: Obsolete installation path given in the documentation, updated.
%     \item New commands: |\newpagecolor{...}|, |\restorepagecolor|.
%     \item Update of documentation, README, and \xfile{dtx} internals.
%   \end{Version}
%   \begin{Version}{2012/02/23 v1.0e}
%     \item Fixed an error in the documentation.
%     \item Check for loading of \xpackage{color} or \xpackage{xcolor} package
%             and their versions has been changed, because \xpackage{xcolor} sets\\
%             |\@namedef{ver@color.sty}{1999/02/16}|\\
%             which gave a warning about old \xpackage{color} package even if a new
%             version was used.
%   \end{Version}
% \end{History}
%
% \bigskip
%
% When you find a mistake or have a suggestion for an improvement of this package,
% please send an e-mail to the maintainer, thanks! (Please see BUG REPORTS in the README.)
%
% \newpage
%
% \PrintIndex
%
% \Finale
\endinput|
% \end{quote}
% Do not forget to quote the argument according to the demands
% of your shell.
%
% \paragraph{Generating the documentation.\label{GenDoc}}
% You can use both the \xfile{.dtx} or the \xfile{.drv} to generate
% the documentation. The process can be configured by a
% configuration file \xfile{ltxdoc.cfg}. For instance, put the following
% line into this file, if you want to have A4 as paper format:
% \begin{quote}
%   \verb|\PassOptionsToClass{a4paper}{article}|
% \end{quote}
%
% \noindent An example follows how to generate the
% documentation with \pdfLaTeX{}:
%
% \begin{quote}
%\begin{verbatim}
%pdflatex pagecolor.dtx
%makeindex -s gind.ist pagecolor.idx
%pdflatex pagecolor.dtx
%makeindex -s gind.ist pagecolor.idx
%pdflatex pagecolor.dtx
%\end{verbatim}
% \end{quote}
%
% \subsection{Compiling the example}
%
% The example file, \textsf{pagecolor-example.tex}, can be compiled via\\
% |(pdf)latex pagecolor-example.tex|.\\
%
% \section{Acknowledgements}
%
% I would like to thank \textsc{Heiko Oberdiek} for providing a~lot~(!) of
% useful packages (from which I also got everything I know about creating
% a file in \xext{dtx} format, ok, say it: copying), and the
% \Newsgroup{comp.text.tex} and \Newsgroup{de.comp.text.tex} newsgroups
% for their help in all things \TeX{}, especially all contributors to the
% discussion at
% \url{http://groups.google.com/group/comp.text.tex/browse_thread/thread/533576ebe44d60f6/e1654d70a19de63c?lnk=gst&q=Determination+of+the+current+page+colour#e1654d70a19de63c}
% \ (\textsc{H.~Oberdiek} \& \textsc{Gouailles}).\\
% I thank \textsc{haoyun\_tex} for suggesting the |\newpagecolor|/|\restorepagecolor|
% pair of commands and everyone at \url{http://tex.stackexchange.com/q/25137}
% for their contributions there.
%
% \bigskip
%
% \phantomsection
% \begin{History}\label{History}
%   \begin{Version}{2011/07/16 v1.0a}
%     \item First version discussed at \Newsgroup{comp.text.tex}.
%   \end{Version}
%   \begin{Version}{2011/08/06 v1.0b}
%     \item Changed version uploaded to \CTAN{}.
%   \end{Version}
%   \begin{Version}{2011/08/08 v1.0c}
%     \item Fixed a |\setkeys|.
%   \end{Version}
%   \begin{Version}{2012/02/01 v1.0d}
%     \item Bugfix: Obsolete installation path given in the documentation, updated.
%     \item New commands: |\newpagecolor{...}|, |\restorepagecolor|.
%     \item Update of documentation, README, and \xfile{dtx} internals.
%   \end{Version}
%   \begin{Version}{2012/02/23 v1.0e}
%     \item Fixed an error in the documentation.
%     \item Check for loading of \xpackage{color} or \xpackage{xcolor} package
%             and their versions has been changed, because \xpackage{xcolor} sets\\
%             |\@namedef{ver@color.sty}{1999/02/16}|\\
%             which gave a warning about old \xpackage{color} package even if a new
%             version was used.
%   \end{Version}
% \end{History}
%
% \bigskip
%
% When you find a mistake or have a suggestion for an improvement of this package,
% please send an e-mail to the maintainer, thanks! (Please see BUG REPORTS in the README.)
%
% \newpage
%
% \PrintIndex
%
% \Finale
\endinput|
% \end{quote}
% Do not forget to quote the argument according to the demands
% of your shell.
%
% \paragraph{Generating the documentation.\label{GenDoc}}
% You can use both the \xfile{.dtx} or the \xfile{.drv} to generate
% the documentation. The process can be configured by a
% configuration file \xfile{ltxdoc.cfg}. For instance, put the following
% line into this file, if you want to have A4 as paper format:
% \begin{quote}
%   \verb|\PassOptionsToClass{a4paper}{article}|
% \end{quote}
%
% \noindent An example follows how to generate the
% documentation with \pdfLaTeX{}:
%
% \begin{quote}
%\begin{verbatim}
%pdflatex pagecolor.dtx
%makeindex -s gind.ist pagecolor.idx
%pdflatex pagecolor.dtx
%makeindex -s gind.ist pagecolor.idx
%pdflatex pagecolor.dtx
%\end{verbatim}
% \end{quote}
%
% \subsection{Compiling the example}
%
% The example file, \textsf{pagecolor-example.tex}, can be compiled via\\
% |(pdf)latex pagecolor-example.tex|.\\
%
% \section{Acknowledgements}
%
% I would like to thank \textsc{Heiko Oberdiek} for providing a~lot~(!) of
% useful packages (from which I also got everything I know about creating
% a file in \xext{dtx} format, ok, say it: copying), and the
% \Newsgroup{comp.text.tex} and \Newsgroup{de.comp.text.tex} newsgroups
% for their help in all things \TeX{}, especially all contributors to the
% discussion at
% \url{http://groups.google.com/group/comp.text.tex/browse_thread/thread/533576ebe44d60f6/e1654d70a19de63c?lnk=gst&q=Determination+of+the+current+page+colour#e1654d70a19de63c}
% \ (\textsc{H.~Oberdiek} \& \textsc{Gouailles}).\\
% I thank \textsc{haoyun\_tex} for suggesting the |\newpagecolor|/|\restorepagecolor|
% pair of commands and everyone at \url{http://tex.stackexchange.com/q/25137}
% for their contributions there.
%
% \bigskip
%
% \phantomsection
% \begin{History}\label{History}
%   \begin{Version}{2011/07/16 v1.0a}
%     \item First version discussed at \Newsgroup{comp.text.tex}.
%   \end{Version}
%   \begin{Version}{2011/08/06 v1.0b}
%     \item Changed version uploaded to \CTAN{}.
%   \end{Version}
%   \begin{Version}{2011/08/08 v1.0c}
%     \item Fixed a |\setkeys|.
%   \end{Version}
%   \begin{Version}{2012/02/01 v1.0d}
%     \item Bugfix: Obsolete installation path given in the documentation, updated.
%     \item New commands: |\newpagecolor{...}|, |\restorepagecolor|.
%     \item Update of documentation, README, and \xfile{dtx} internals.
%   \end{Version}
%   \begin{Version}{2012/02/23 v1.0e}
%     \item Fixed an error in the documentation.
%     \item Check for loading of \xpackage{color} or \xpackage{xcolor} package
%             and their versions has been changed, because \xpackage{xcolor} sets\\
%             |\@namedef{ver@color.sty}{1999/02/16}|\\
%             which gave a warning about old \xpackage{color} package even if a new
%             version was used.
%   \end{Version}
% \end{History}
%
% \bigskip
%
% When you find a mistake or have a suggestion for an improvement of this package,
% please send an e-mail to the maintainer, thanks! (Please see BUG REPORTS in the README.)
%
% \newpage
%
% \PrintIndex
%
% \Finale
\endinput|
% \end{quote}
% Do not forget to quote the argument according to the demands
% of your shell.
%
% \paragraph{Generating the documentation.\label{GenDoc}}
% You can use both the \xfile{.dtx} or the \xfile{.drv} to generate
% the documentation. The process can be configured by a
% configuration file \xfile{ltxdoc.cfg}. For instance, put the following
% line into this file, if you want to have A4 as paper format:
% \begin{quote}
%   \verb|\PassOptionsToClass{a4paper}{article}|
% \end{quote}
%
% \noindent An example follows how to generate the
% documentation with \pdfLaTeX{}:
%
% \begin{quote}
%\begin{verbatim}
%pdflatex pagecolor.dtx
%makeindex -s gind.ist pagecolor.idx
%pdflatex pagecolor.dtx
%makeindex -s gind.ist pagecolor.idx
%pdflatex pagecolor.dtx
%\end{verbatim}
% \end{quote}
%
% \subsection{Compiling the example}
%
% The example file, \textsf{pagecolor-example.tex}, can be compiled via\\
% |(pdf)latex pagecolor-example.tex|.\\
%
% \section{Acknowledgements}
%
% I would like to thank \textsc{Heiko Oberdiek} for providing a~lot~(!) of
% useful packages (from which I also got everything I know about creating
% a file in \xext{dtx} format, ok, say it: copying), and the
% \Newsgroup{comp.text.tex} and \Newsgroup{de.comp.text.tex} newsgroups
% for their help in all things \TeX{}, especially all contributors to the
% discussion at
% \url{http://groups.google.com/group/comp.text.tex/browse_thread/thread/533576ebe44d60f6/e1654d70a19de63c?lnk=gst&q=Determination+of+the+current+page+colour#e1654d70a19de63c}
% \ (\textsc{H.~Oberdiek} \& \textsc{Gouailles}).\\
% I thank \textsc{haoyun\_tex} for suggesting the |\newpagecolor|/|\restorepagecolor|
% pair of commands and everyone at \url{http://tex.stackexchange.com/q/25137}
% for their contributions there.
%
% \bigskip
%
% \phantomsection
% \begin{History}\label{History}
%   \begin{Version}{2011/07/16 v1.0a}
%     \item First version discussed at \Newsgroup{comp.text.tex}.
%   \end{Version}
%   \begin{Version}{2011/08/06 v1.0b}
%     \item Changed version uploaded to \CTAN{}.
%   \end{Version}
%   \begin{Version}{2011/08/08 v1.0c}
%     \item Fixed a |\setkeys|.
%   \end{Version}
%   \begin{Version}{2012/02/01 v1.0d}
%     \item Bugfix: Obsolete installation path given in the documentation, updated.
%     \item New commands: |\newpagecolor{...}|, |\restorepagecolor|.
%     \item Update of documentation, README, and \xfile{dtx} internals.
%   \end{Version}
%   \begin{Version}{2012/02/23 v1.0e}
%     \item Fixed an error in the documentation.
%     \item Check for loading of \xpackage{color} or \xpackage{xcolor} package
%             and their versions has been changed, because \xpackage{xcolor} sets\\
%             |\@namedef{ver@color.sty}{1999/02/16}|\\
%             which gave a warning about old \xpackage{color} package even if a new
%             version was used.
%   \end{Version}
% \end{History}
%
% \bigskip
%
% When you find a mistake or have a suggestion for an improvement of this package,
% please send an e-mail to the maintainer, thanks! (Please see BUG REPORTS in the README.)
%
% \newpage
%
% \PrintIndex
%
% \Finale
\endinput