% \iffalse meta-comment
%
% Copyright (C) 2003-2011 by Heiko Bauke <heiko.bauke@mpi-hd.mpg.de>
% -----------------------------------------------------------------------
% 
% This file may be distributed and/or modified under the
% conditions of the LaTeX Project Public License, either version 1.2
% of this license or (at your option) any later version.
% The latest version of this license is in:
%
%    http://www.latex-project.org/lppl.txt
%
% and version 1.2 or later is part of all distributions of LaTeX 
% version 1999/12/01 or later.
%
% \fi
%
% \iffalse
%<package>\NeedsTeXFormat{LaTeX2e}[1999/12/01]
%<package>\ProvidesPackage{isotope}
%<package>   [2011/08/26 v0.3 for typesetting isotopes] 
%
%<*driver>
\documentclass{ltxdoc}
\usepackage{isotope}
\usepackage{textcomp}
\EnableCrossrefs         
\CodelineIndex
\RecordChanges
\begin{document}
  \DocInput{isotope.dtx}
\end{document}
%</driver>
% \fi
%
%\CheckSum{47}
%
% \CharacterTable
%  {Upper-case    \A\B\C\D\E\F\G\H\I\J\K\L\M\N\O\P\Q\R\S\T\U\V\W\X\Y\Z
%   Lower-case    \a\b\c\d\e\f\g\h\i\j\k\l\m\n\o\p\q\r\s\t\u\v\w\x\y\z
%   Digits        \0\1\2\3\4\5\6\7\8\9
%   Exclamation   \!     Double quote  \"     Hash (number) \#
%   Dollar        \$     Percent       \%     Ampersand     \&
%   Acute accent  \'     Left paren    \(     Right paren   \)
%   Asterisk      \*     Plus          \+     Comma         \,
%   Minus         \-     Point         \.     Solidus       \/
%   Colon         \:     Semicolon     \;     Less than     \<
%   Equals        \=     Greater than  \>     Question mark \?
%   Commercial at \@     Left bracket  \[     Backslash     \\
%   Right bracket \]     Circumflex    \^     Underscore    \_
%   Grave accent  \`     Left brace    \{     Vertical bar  \|
%   Right brace   \}     Tilde         \~}
%
% \changes{v0.1}{2003/12/31}{Initial version.}
%
% \GetFileInfo{isotope.sty}
%
% \DoNotIndex{\newcommand,\newenvironment}
% 
%
% \title{The \textsf{isotope} package\thanks{This document
%   corresponds to \textsf{isotope}~\fileversion, dated \filedate.}}
% \author{Heiko Bauke\thanks{E-mail: \texttt{heiko.bauke@mpi-hd.mpg.de}}}
%
% \maketitle
%
% \section{Introduction}
%
% Despite its powerful typographic capabilities it is surprisingly
% difficult to typeset isotopes with \LaTeX. Ad hoc methods as for 
% example |$^{232}_{90}\mathrm{Th}$| give poor results as 
% $^{232}_{90}\mathrm{Th}$.
% This is not satisfactory because of the wrong alignment of atomic
% and nuclear numbers. The package \textsf{isotope} provides the |\isotope|
% macro for correct typesetting of isotopes.
%
% \section{Usage}
%
% The package \textsf{isotope} has to be included into the preamble of your 
% \LaTeX\ file by:
% \begin{quote}
% |\usepackage{isotope}|
% \end{quote}
% \DescribeMacro{\isotope}
% The usage of the |\isotope| macro is straight forward. Just provide the
% isotope's name and optionally its nucleon number and its atomic number.
% \begin{quote}
% |\isotope|\oarg{nucleon number}\oarg{atomic number}\marg{name}
% \end{quote}
% See \tablename~\ref{tab:example} for some examples. Note that the
% \verb#\alpha# has been enclosed by \verb#\mathnormal#.  Not doing so
% may give unexpected results for some math fonds.
%
% \begin{table}[t]
%   \centering
%   \caption{Examples for \texttt{\textbackslash isotope} macro usage.}
%   \label{tab:example}
%   \begin{tabular}{@{}ll@{}}
%   \hline
%   \hline
%   command & result \\
%   \hline
%   |\isotope{Ra}| & \isotope{Ra} \\
%   |\isotope[228]{Ra}| & \isotope[228]{Ra} \\
%   |\isotope[228][88]{Ra}| & \isotope[228][88]{Ra} \\
%   |$\isotope[A][Z]{X}\to\isotope[A-4][Z-2]{Y}+| & \\
%   |\isotope[4][2]{\mathnormal{\alpha}}$| & 
%   \raisebox{1.5ex}[-1.5ex]{$\isotope[A][Z]{X}\to\isotope[A-4][Z-2]{Y}+\isotope[4][2]{\mathnormal{\alpha}}$}\\
%   \hline
%   \hline
%   \end{tabular}
% \end{table}
% \DescribeMacro{\isotopestyle}
% The macro |\isotopestyle| determines the style which is used to typeset the 
% name of the iostope. It may be redefined. For example, the redefinition
% \begin{quote}
% |\renewcommand{\isotopestyle}{\mathsf}|\\
% |\isotope[228]{Ra}|
% \end{quote}
% gives {\renewcommand{\isotopestyle}{\mathsf}\isotope[228]{Ra}}.
%
% \StopEventually{\PrintChanges\PrintIndex}
%
% \section{Implementation}
%
% \begin{macro}{\isotopestyle}
% \changes{v0.2}{2004/01/04}{\texttt{\textbackslash isotopestyle} determines
%   also the style of nuclear and atomic number.}
% |\isotopestyle| determines the style which is used to typeset the 
% name of an iostope and its nucleon and atomic numbers. 
%    \begin{macrocode}
\newcommand{\isotopestyle}{\mathrm}
%    \end{macrocode}
% \end{macro}
% \begin{macro}{\isotope}
% \changes{v0.2}{2004/01/04}{Vertical spacing between nuclear and atomic 
%   number has been improved. The implementation is based no a suggestion
%   of Walter Schmidt in \texttt{de.comp.text.tex}.}
% Now it follows the implementation of the \verb#\isotope# macro.
%    \begin{macrocode}
\newcommand{\isotope@atomicnumber}{}
\newcommand{\isotope@nucleonnumber}{}
\newcommand{\isotope}[1][]{%
  \begingroup%
  \renewcommand{\isotope@nucleonnumber}{#1}%
  \isotope@two}%
  \newcommand{\isotope@two}[2][]{%
  \renewcommand{\isotope@atomicnumber}{#1}%
  {\m@th%
%    \end{macrocode}
%  Determine which has a larger width nucleon number or atomic number.
%    \begin{macrocode}
  \settowidth\@tempdimb{\ensuremath{%
    \scriptstyle\isotope@nucleonnumber}}%
  \settowidth\@tempdimc{\ensuremath{%
    \scriptstyle\isotope@atomicnumber}}%
  \ifdim\@tempdimb<\@tempdimc\@tempdimb=\@tempdimc\fi%
  \ensuremath{{}%
    ^{\makebox[\@tempdimb][r]{\ensuremath{%
       \scriptstyle\isotope@nucleonnumber}}}%
    _{\makebox[\@tempdimb][r]{\ensuremath{%
       \scriptstyle\isotope@atomicnumber}}}%
    \isotopestyle{#2}}%
  }%
  \endgroup%
}%	
%    \end{macrocode}
% \end{macro}
% \changes{v0.3}{2011/08/26}{Changed \texttt{\textbackslash
% isotopestyle} from \texttt{\textbackslash rm} to
% \texttt{\textbackslash mathrm} to improve typographic quality when
% sans serif math fonts are used.} 
% \PrintChanges
% \Finale
\endinput
