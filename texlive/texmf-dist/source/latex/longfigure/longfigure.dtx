% \iffalse meta-comment
%<*internal>
\iffalse
%</internal>
%<*readme>
----------------------------------------------------------------
longfigure --- A LaTeX package providing a table-like environment
               that can display a stream of subfigures as a single
               figure that can be broken by \TeX's
               standard page-breaking algorithm.

               Nearly identical to David Carlisle's |longtable| package,
               but with some obvious changes to use figures instead of
               tables. All credit goes to David Carlisle; this package
               represents small changes to his code and logic.
               I am responsible for the changes.

E-mail: tim.arnold@sas.com
Released under the LaTeX Project Public License v1.3c or later
See http://www.latex-project.org/lppl.txt
----------------------------------------------------------------

This work consists of the file longfigure.dtx
and the derived files:
   README (this file)
   longfigure.ins
   longfigure.pdf
   longfigure.sty
%</readme>
%<*internal>
\fi
\def\nameofplainTeX{plain}
\ifx\fmtname\nameofplainTeX\else
  \expandafter\begingroup
\fi
%</internal>
%<*install>
\input docstrip.tex
\keepsilent
\askforoverwritefalse
\preamble
----------------------------------------------------------------
longfigure --- A LaTeX package providing a table-like environment
               that can display a stream of subfigures as a single
               figure that can be broken by \TeX's
               standard page-breaking algorithm.

E-mail: tim.arnold@sas.com
----------------------------------------------------------------
\endpreamble
\postamble
Copyright (C) 2014 by SAS Institute Inc. <tim.arnold@sas.com>

This work may be distributed and/or modified under the
conditions of the LaTeX Project Public License (LPPL), either
version 1.3c of this license or any later version.
The latest version of this license is in the file:
http://www.latex-project.org/lppl.txt

This work is "maintained" (as per LPPL maintenance status) by
Tim Arnold.

This work consists of the file longfigure.dtx
and the derived files:
   README (this file)
   longfigure.ins
   longfigure.pdf
   longfigure.sty
\endpostamble
\generate{\file{longfigure.sty}{\from{longfigure.dtx}{longfigure}}}
%</install>
%<install>\endbatchfile
%<*internal>
\generate{\file{longfigure.ins}{\from{longfigure.dtx}{install}}}
\nopreamble\nopostamble
\generate{\file{README}{\from{longfigure.dtx}{readme}}}
\ifx\fmtname\nameofplainTeX
  \expandafter\endbatchfile
\else
  \expandafter\endgroup
\fi
%</internal>
%<*longfigure>
\NeedsTeXFormat{LaTeX2e}
\ProvidesPackage{longfigure}[2014/1/06 v1.0]
%</longfigure>
%<*driver>
\documentclass{ltxdoc}
\usepackage[T1]{fontenc}
\usepackage{lmodern}
\usepackage{\jobname}
\usepackage[numbered]{hypdoc}
\EnableCrossrefs
\CodelineIndex
\RecordChanges
\begin{document}
  \DocInput{\jobname.dtx}
\end{document}
%</driver>
% \fi
%\GetFileInfo{\jobname.sty}
% \def\fileversion{v1.0}
%
%\title{^^A
%  The \textsf{longfigure} Package\thanks{^^A
%    This file describes version \fileversion, last revised \filedate.^^A
%  }^^A
%}
%\author{^^A
%  Tim Arnold\thanks{E-mail: tim.arnold@sas.com}^^A
%}
%\date{Released \filedate}
%
%\maketitle
%
%\changes{v1.0}{2014/1/06}{First public release}
%    \section{Usage}
%    \label{longfigure}
%    \index{longfigure package|usage}
%
%    The |longfigure| package uses and relabels components of the
%    well-known |longtable| package, written by David Carlisle, to
%    provide a table-like environment that can display a stream
%    of subfigures as a single figure that can break across pages.
%
%    The |longtable| package defines a |longtable| environment,
%    which produces tables that can be broken by \TeX's standard page-breaking algorithm.
%    Similarly, the |longfigure| package defines a |longfigure| environment, which produces
%    figures that can be broken by \TeX's standard page-breaking algorithm.
%    The internal structure of a long figure is similar to a long table.
%    Rows might contain (for example) tables or graphics. Page breaks
%    can occur only between rows.
%
%     The |longfigure| package differs from the |longtable| package in the following ways:
%    \begin{itemize}
%      \item The |longfigure| package supports two additional key-value options:
%      \begin{itemize}
%        \item The |figname=| option specifies the counter for numbering |longfigure| environments.
%              You can specify any string; the default is |figure|.
%              When you specify a |figname=| value for which no counter already exists, the
%              |longfigure| package loads the |tocloft| package and creates the counter.
%              \index{figname option|usage}
%
%       \item The |resetby=| option specifies a counter (for example, |resetby=chapter|) such
%             that output numbering is reset each time the counter value changes.
%             If a counter is specified that does not exist, the |tocloft| package is
%             loaded to create the new counter.
%             \index{tocloft package|usage}%
%             For information about how the lists are typeset,
%             \index{list of outputs, creating|usage}%
%             see the |tocloft| package documentation.
%      \end{itemize}
%      \item The counters and macros that start with \cs{LT} in the |longtable| package
%            are renamed to start with \cs{LF} in the |longfigure| pacakge to avoid
%            namespace conflicts when the two packages are used together.
%            The generic macros that are defined in the |longtable| package
%            (\cs{endfirsthead}, \cs{endhead}, \cs{endfoot}, and \cs{endlastfoot}) are
%            also renamed with \cs{LF} as a prefix in the |longfigure| package.
%
%      \item The \cs{LF@name} macro is based on the \cs{fnum@table} macro from
%            the |longtable| package. The  \cs{LF@name} macro returns the capitalized
%            counter name and value. For example, if the counter is |figure| and the
%            macro is processing the second |longfigure|, the \cs{LF@name} macro
%            would contain the value ``Figure 2.''
%
%  \end{itemize}
%    You can use the |longfigure| package defaults to produce a \emph{List of Figures}
%    by inserting the following tag in your document at the point where you want
%    the list to appear:
% \iffalse
%<*example>
% \fi
\begin{verbatim}
   \listoffigures
\end{verbatim}
% \iffalse
%</example>
% \fi
%
%    The default counter used to display figures is the |figure| counter,
%    but you can specify a different counter. For example, if you want your
%    figures to be labeled as ``Display,'' specify |figname=display| when you
%    load the |longfigure| package; to display a \emph{List of Displays}, insert the following
%    command in your document at the point where you want the list to appear:
% \iffalse
%<*example>
% \fi
\begin{verbatim}
   \listofdisplay
\end{verbatim}
% \iffalse
%</example>
% \fi
%    \textbf{Note:}  If you specify a counter that does not exist,
%    an auxiliary file with extension |.lft| is created to contain the information
%    needed to create the list.
%
%    If you want to use more advanced features of the |tocloft| package,
%    load it before you load the |longfigure| package so that the |longfigure|
%    package sees that the counters specified by the |figname=| and |resetby=|
%    options are already defined and does not attempt to create them.
%
%
%\subsection{Example}
% The following lines produce a single figure that contains three images and
% one tabular environment. Each element is a row of the |longfigure| environment.
% Page breaks can occur between rows.
%
%\begin{verbatim}
%\documentclass{book}
%\usepackage{graphicx}
%\usepackage{longfigure}
%
%\begin{document}
%  \begin{longfigure}{c}
%    \caption{My Long Figure}\label{mlfig}\\
%    \includegraphics[width=3in]{myfig1}\\
%    \includegraphics[width=3in]{myfig2}\\
%    \includegraphics[width=3in]{myfig3}\\
%    \begin{tabular}{ll}
%      one & two \\
%      three & four\\
%    \end{tabular}
%  \end{longfigure}
%\end{document}
%\end{verbatim}
%
%  In this example, the |{c}| argument in the |\begin{longfigure}| command
%  specifies only a single centered column. You can also specify multiple columns and,
%  if needed, use the \cs{multicolumn} command for more flexibility.
%
% The following lines display another example that specifies a 'continued' heading
% when the figure breaks over a page. It also displays a double horizontal line at the
% end of the figure.
%\begin{verbatim}
%\documentclass{book}
%\usepackage{graphicx}
%\usepackage{longfigure}
%
%\begin{document}
%  \begin{longfigure}{c}
%    \caption{My Long Figure}\label{mlfig2}\\
%    \hline\endLFfirsthead
%    \caption{continued}\\
%    \hline\endLFhead
%    \hline\endLFfoot
%    \hline\hline\endLFlastfoot
%    \includegraphics[width=3in]{myfig1}\\
%    \includegraphics[width=3in]{myfig2}\\
%    \includegraphics[width=3in]{myfig3}\\
%    \begin{tabular}{ll}
%      one & two \\
%      three & four\\
%    \end{tabular}
%  \end{longfigure}
%\end{document}
%\end{verbatim}
%
%    \section{Implementation}
%
%    This section describes the implementation of the |longfigure| package.
%    The comments describe only the changes from the |longtable| package code.
%    For complete details about the logic and usage of the |longtable| environment,
%    see Carlisle (2004).
%\iffalse
%<*longfigure>
%\fi
%    \begin{macrocode}
\ProvidesPackage{longfigure}[2014/01/06 longfigure]
%    \end{macrocode}
%    The following statement loads the |xkeyval| package for declaring
%    and processing package options:
%    \begin{macrocode}
\RequirePackage{xkeyval}
%    \end{macrocode}
%    The following statement defines a new command, \cs{LFcounter}, to
%    contain the string |figure|.
%    Later code tests whether a counter with that name exists.
%    \begin{macrocode}
\newcommand*{\LFcounter}{figure}
%    \end{macrocode}
%
%    The following statement defines a new command, \cs{LFreset}, to contain
%    the name of the counter within which the |longfigure| number should reset.
%    If no value is specified, the long figures are numbered consecutively
%    through the document.
%    \begin{macrocode}
\newcommand*{\LFreset}{\@empty}
%    \end{macrocode}
% \subsection{Options}
%    The \cs{LFcounter} and \cs{LFreset} commands support the
%    package options |figname=| and |resetby=| as follows:
%    \begin{macrocode}
\DeclareOptionX{figname}[figure]{\renewcommand*{\LFcounter}{#1}}
\DeclareOptionX{resetby}{\renewcommand*{\LFreset}{#1}}
%    \end{macrocode}
%    The following statements further define the options
%    that the |longtable| package defines:
%    \begin{macrocode}
\DeclareOptionX{set}{}
\DeclareOptionX{final}{}
\DeclareOptionX{errorshow}{\def\LF@warn{\PackageInfo{longfigure}}}
\DeclareOptionX{pausing}{\def\LF@warn#1{\LF@err{#1}{This is not really an error}}}
\ProcessOptionsX
%    \end{macrocode}
%    The following statements process the options:
%    \begin{macrocode}
\def\LFProcessOptions#1{
  \@ifundefined{c@#1}{%
      \RequirePackage{tocloft}
      \def\LFuc##1##2{\MakeUppercase{##1}{##2}}
      \expandafter\def\csname list#1name\endcsname{List of \LFuc#1s}
      \ifx\@empty\LFreset%
        \newlistof{#1}{lft}{\csname list#1name\endcsname}
      \else
        \newlistof[\LFreset]{#1}{lft}{\csname list#1name\endcsname}
      \fi
  }{}%
}
\expandafter\LFProcessOptions\expandafter{\LFcounter}
%    \end{macrocode}
%    If a counter is specified that does not exist, its name (\cs{c@}\textit{countername})
%    is undefined and the |longfigure| package loads the |tocloft| package in order to
%    use its commands to create the new counters and list.
%
%    Thus, the |tocloft| package is required only when a new counter
%    is specified,
%    \index{tocloft package|usage}%
%    and this automatic loading takes place only if the counter that is  specified
%    in the package options is not defined.
%
%    You can load the |tocloft| package before loading the |longfigure| package
%    and retain all of the flexibility that the |tocloft| package offers.
%    However, you must define the new counters yourself by using the
%    \cs{newlistof} command in the |tocloft| package,
%    and you must define the new list to use an auxiliary |lft| file where
%    its auxiliary information is written.
%    \index{tocloft package|usage}%
%
% \subsection{Utilities}
% \DescribeMacro{\strcfstr}
%  The following macro, \cs{strcfstr}, checks whether two strings, which are
%  provided as arguments, are equal (Wilson, 2001).
%  A new boolean \cs{ifLF@same} contains the result of the test.
%    \begin{macrocode}
\newif\ifLF@same
\newcommand{\strcfstr}[2]{%
  \LF@samefalse
  \begingroup\def\2{#2}
    \ifx\2#1\endgroup\LF@sametrue
    \else\endgroup
    \fi
}
%    \end{macrocode}
%  \DescribeMacro{\LFupcase}
%    The following macro, \cs{LFupcase}, uppercases the first letter of
%    a string (Lazarides, 2010):
%    \begin{macrocode}
\def\LFupcase#1{%
  \def\x##1##2{%
    \MakeUppercase{##1}{##2}}\x#1%
}
%    \end{macrocode}
%    The following macro, \cs{LF@name}, creates a string to provide a label and number for
%    an output. Analogous to the \cs{fnum@table} macro in the |longtable| package,
%    it contains the capitalized version of the counter name and the counter
%    number (for example, |Figure~3|).
%    \begin{macrocode}
\def\LF@name{\expandafter\LFupcase%
             \expandafter{\LFcounter}~%
             \expandafter\csname the\LFcounter\endcsname}%
%    \end{macrocode}
%    The remainder of this package follows the |longtable| package
%    almost identically, except that macros, skips, counters, and so on
%    use an \cs{LF} prefix instead of the \cs{LT} prefix that the |longtable| package uses.
%
%    \begin{macrocode}
\def\LF@err{\PackageError{longfigure}}
\def\LF@warn{\PackageWarning{longfigure}}
\def\LF@final@warn{%
  \AtEndDocument{%
    \LF@warn{\LFcounter \@width s have changed. Rerun \LaTeX\.\@gobbletwo}}%
  \global\let\LF@final@warn\relax}
%
\newskip\LFleft       \LFleft=\fill
\newskip\LFright      \LFright=\fill
\newskip\LFpre        \LFpre=\bigskipamount
\newskip\LFpost       \LFpost=\bigskipamount
\newcount\LFchunksize \LFchunksize=20
\let\c@LFchunksize\LFchunksize
\newdimen\LFcapwidth  \LFcapwidth=4in
\newbox\LF@head
\newbox\LF@firsthead
\newbox\LF@foot
\newbox\LF@lastfoot
\newcount\LF@cols
\newcount\LF@rows
\newcounter{LF@tables}
\newcounter{LF@chunks}[LF@tables]
%
\newtoks\LF@p@ftn
\mathchardef\LF@end@pen=30000
\def\longfigure{%
  \par
  \ifx\multicols\@undefined
  \else
     \ifnum\col@number>\@ne
       \@twocolumntrue
     \fi
  \fi
  \if@twocolumn
    \LF@err{longfigure not in 1-column mode}\@ehc
  \fi
  \begingroup
  \@ifnextchar[\LF@array{\LF@array[x]}}
\def\LF@array[#1]#2{%
  \refstepcounter{\LFcounter}\stepcounter{LF@tables}%
  \if l#1%
    \LFleft\z@ \LFright\fill
  \else\if r#1%
    \LFleft\fill \LFright\z@
  \else\if c#1%
    \LFleft\fill \LFright\fill
  \fi\fi\fi
  \let\LF@mcol\multicolumn
  \let\LF@@tabarray\@tabarray
  \let\LF@@hl\hline
  \def\@tabarray{%
    \let\hline\LF@@hl
    \LF@@tabarray}%
  \let\\\LF@tabularcr\let\tabularnewline\\%
  \def\newpage{\noalign{\break}}%
  \def\pagebreak{\noalign{\ifnum`}=0\fi\@testopt{\LF@no@pgbk-}4}%
  \def\nopagebreak{\noalign{\ifnum`}=0\fi\@testopt\LF@no@pgbk4}%
  \let\hline\LF@hline \let\kill\LF@kill\let\caption\LF@caption
  \@tempdima\ht\strutbox
  \let\@endpbox\LF@endpbox
  \ifx\extrarowheight\@undefined
    \let\@acol\@tabacol
    \let\@classz\@tabclassz \let\@classiv\@tabclassiv
    \def\@startpbox{\vtop\LF@startpbox}%
    \let\@@startpbox\@startpbox
    \let\@@endpbox\@endpbox
    \let\LF@LL@FM@cr\@tabularcr
  \else
    \advance\@tempdima\extrarowheight
    \col@sep\tabcolsep
    \let\@startpbox\LF@startpbox\let\LF@LL@FM@cr\@arraycr
  \fi
  \setbox\@arstrutbox\hbox{\vrule
    \@height \arraystretch \@tempdima
    \@depth \arraystretch \dp \strutbox
    \@width \z@}%
  \let\@sharp##\let\protect\relax
   \begingroup
    \@mkpream{#2}%
    \xdef\LF@bchunk{%
       \global\advance\c@LF@chunks\@ne
       \global\LF@rows\z@\setbox\z@\vbox\bgroup
       \LF@setprevdepth
       \tabskip\LFleft \noexpand\halign to\hsize\bgroup
      \tabskip\z@ \@arstrut \@preamble \tabskip\LFright \cr}%
  \endgroup
  \expandafter\LF@nofcols\LF@bchunk&\LF@nofcols
  \LF@make@row
  \m@th\let\par\@empty
  \everycr{}\lineskip\z@\baselineskip\z@
  \LF@bchunk}
\def\LF@no@pgbk#1[#2]{\penalty #1\@getpen{#2}\ifnum`{=0\fi}}
\def\LF@start{%
  \let\LF@start\endgraf
  \endgraf\penalty\z@\vskip\LFpre
  \dimen@\pagetotal
  \advance\dimen@ \ht\ifvoid\LF@firsthead\LF@head\else\LF@firsthead\fi
  \advance\dimen@ \dp\ifvoid\LF@firsthead\LF@head\else\LF@firsthead\fi
  \advance\dimen@ \ht\LF@foot
  \dimen@ii\vfuzz
  \vfuzz\maxdimen
    \setbox\tw@\copy\z@
    \setbox\tw@\vsplit\tw@ to \ht\@arstrutbox
    \setbox\tw@\vbox{\unvbox\tw@}%
  \vfuzz\dimen@ii
  \advance\dimen@ \ht
        \ifdim\ht\@arstrutbox>\ht\tw@\@arstrutbox\else\tw@\fi
  \advance\dimen@\dp
        \ifdim\dp\@arstrutbox>\dp\tw@\@arstrutbox\else\tw@\fi
  \advance\dimen@ -\pagegoal
  \ifdim \dimen@>\z@\vfil\break\fi
      \global\@colroom\@colht
  \ifvoid\LF@foot\else
    \advance\vsize-\ht\LF@foot
    \global\advance\@colroom-\ht\LF@foot
    \dimen@\pagegoal\advance\dimen@-\ht\LF@foot\pagegoal\dimen@
    \maxdepth\z@
  \fi
  \ifvoid\LF@firsthead\copy\LF@head\else\box\LF@firsthead\fi\nobreak
  \output{\LF@output}}
\def\endlongfigure{%
  \crcr
  \noalign{%
    \let\LF@entry\LF@entry@chop
    \xdef\LF@save@row{\LF@save@row}}%
  \LF@echunk
  \LF@start
  \unvbox\z@
  \LF@get@widths
  \if@filesw
    {\let\LF@entry\LF@entry@write\immediate\write\@auxout{%
      \gdef\expandafter\noexpand
        \csname LF@\romannumeral\c@LF@tables\endcsname
          {\LF@save@row}}}%
  \fi
  \ifx\LF@save@row\LF@@save@row
  \else
    \LF@warn{Column \@width s have changed\MessageBreak
             in table \thetable}%
    \LF@final@warn
  \fi
  \endgraf\penalty -\LF@end@pen
  \endgroup
  \global\@mparbottom\z@
  \pagegoal\vsize
  \endgraf\penalty\z@\addvspace\LFpost
  \ifvoid\footins\else\insert\footins{}\fi}
\def\LF@nofcols#1&{%
  \futurelet\@let@token\LF@n@fcols}
\def\LF@n@fcols{%
  \advance\LF@cols\@ne
  \ifx\@let@token\LF@nofcols
    \expandafter\@gobble
  \else
    \expandafter\LF@nofcols
  \fi}
\def\LF@tabularcr{%
  \relax\iffalse{\fi\ifnum0=`}\fi
  \@ifstar
    {\def\crcr{\LF@crcr\noalign{\nobreak}}\let\cr\crcr
     \LF@t@bularcr}%
    {\LF@t@bularcr}}
\let\LF@crcr\crcr
\let\LF@setprevdepth\relax
\def\LF@t@bularcr{%
  \global\advance\LF@rows\@ne
  \ifnum\LF@rows=\LFchunksize
    \gdef\LF@setprevdepth{%
      \prevdepth\z@\global
      \global\let\LF@setprevdepth\relax}%
    \expandafter\LF@xtabularcr
  \else
    \ifnum0=`{}\fi
    \expandafter\LF@LL@FM@cr
  \fi}
\def\LF@xtabularcr{%
  \@ifnextchar[\LF@argtabularcr\LF@ntabularcr}
\def\LF@ntabularcr{%
  \ifnum0=`{}\fi
  \LF@echunk
  \LF@start
  \unvbox\z@
  \LF@get@widths
  \LF@bchunk}
\def\LF@argtabularcr[#1]{%
  \ifnum0=`{}\fi
  \ifdim #1>\z@
    \unskip\@xargarraycr{#1}%
  \else
    \@yargarraycr{#1}%
  \fi
  \LF@echunk
  \LF@start
  \unvbox\z@
  \LF@get@widths
  \LF@bchunk}
\def\LF@echunk{%
  \crcr\LF@save@row\cr\egroup
  \global\setbox\@ne\lastbox
    \unskip
  \egroup}
\def\LF@entry#1#2{%
  \ifhmode\@firstofone{&}\fi\omit
  \ifnum#1=\c@LF@chunks
  \else
    \kern#2\relax
  \fi}
\def\LF@entry@chop#1#2{%
  \noexpand\LF@entry
    {\ifnum#1>\c@LF@chunks
       1}{0pt%
     \else
       #1}{#2%
     \fi}}
\def\LF@entry@write{%
  \noexpand\LF@entry^^J%
  \@spaces}
\def\LF@kill{%
  \LF@echunk
  \LF@get@widths
  \expandafter\LF@rebox\LF@bchunk}
\def\LF@rebox#1\bgroup{%
  #1\bgroup
  \unvbox\z@
  \unskip
  \setbox\z@\lastbox}
\def\LF@blank@row{%
  \xdef\LF@save@row{\expandafter\LF@build@blank
    \romannumeral\number\LF@cols 001 }}
\def\LF@build@blank#1{%
  \if#1m%
    \noexpand\LF@entry{1}{0pt}%
    \expandafter\LF@build@blank
  \fi}
\def\LF@make@row{%
  \global\expandafter\let\expandafter\LF@save@row
    \csname LF@\romannumeral\c@LF@tables\endcsname
  \ifx\LF@save@row\relax
    \LF@blank@row
  \else
    {\let\LF@entry\or
     \if!%
         \ifcase\expandafter\expandafter\expandafter\LF@cols
         \expandafter\@gobble\LF@save@row
         \or
         \else
           \relax
         \fi
        !%
     \else
       \aftergroup\LF@blank@row
     \fi}%
  \fi}
\let\setlongfigures\relax
\def\LF@get@widths{%
  \setbox\tw@\hbox{%
    \unhbox\@ne
    \let\LF@old@row\LF@save@row
    \global\let\LF@save@row\@empty
    \count@\LF@cols
    \loop
      \unskip
      \setbox\tw@\lastbox
    \ifhbox\tw@
      \LF@def@row
      \advance\count@\m@ne
    \repeat}%
  \ifx\LF@@save@row\@undefined
    \let\LF@@save@row\LF@save@row
  \fi}
\def\LF@def@row{%
  \let\LF@entry\or
  \edef\@tempa{%
    \ifcase\expandafter\count@\LF@old@row
    \else
      {1}{0pt}%
    \fi}%
  \let\LF@entry\relax
  \xdef\LF@save@row{%
    \LF@entry
    \expandafter\LF@max@sel\@tempa
    \LF@save@row}}
\def\LF@max@sel#1#2{%
  {\ifdim#2=\wd\tw@
     #1%
   \else
     \number\c@LF@chunks
   \fi}%
  {\the\wd\tw@}}
\def\LF@hline{%
  \noalign{\ifnum0=`}\fi
    \penalty\@M
    \futurelet\@let@token\LF@@hline}
\def\LF@@hline{%
  \ifx\@let@token\hline
    \global\let\@gtempa\@gobble
    \gdef\LF@sep{\penalty-\@medpenalty\vskip\doublerulesep}%
  \else
    \global\let\@gtempa\@empty
    \gdef\LF@sep{\penalty-\@lowpenalty\vskip-\arrayrulewidth}%
  \fi
  \ifnum0=`{\fi}%
  \multispan\LF@cols
     \unskip\leaders\hrule\@height\arrayrulewidth\hfill\cr
  \noalign{\LF@sep}%
  \multispan\LF@cols
     \unskip\leaders\hrule\@height\arrayrulewidth\hfill\cr
  \noalign{\penalty\@M}%
  \@gtempa}
%    \end{macrocode}
% \subsection{Captioning}
%    You can easily change how a long figure is captioned by
%    redefining the \cs{LF@makecaption} macro after loading the |longfigure| package.
%    The following statements show the default definition of the
%    \cs{LF@makecaption}:
%
%    \begin{macrocode}
\def\LF@caption{%
  \noalign\bgroup
    \@ifnextchar[{\egroup\LF@c@ption\@firstofone}\LF@capti@n}
%    \end{macrocode}
%    The \cs{LF@caption} command begins the process.
%    If it includes an optional argument, it calls \cs{LF@c@ption}; otherwise
%    it calls \cs{LF@capti@n}, which then calls \cs{LF@c@ption}.
%    \begin{macrocode}
\def\LF@c@ption#1[#2]#3{%
  \LF@makecaption#1\LF@name{#3}%
  \def\@tempa{#2}%
  \ifx\@tempa\@empty\else
%    \end{macrocode}
%    If a list of long figures is requested, the following code uses the
%    previously defined \cs{strcfstr} macro and  \cs{ifLF@same} boolean
%    to determine the name of the counter and set the  output file to
%    contain the |longfigure| information.
%
%    The code writes to one of the following files.
%    \begin{itemize}
%     \item If the counter is |figure|, write to the |lof| file.
%     \item If the counter is |table|, write to the |lot| file.
%     \item Otherwise, write to |lft|, a file created here for this purpose.
%    \end{itemize}
%
%    \begin{macrocode}
     {\let\\\space
     \strcfstr{\LFcounter}{figure}
       \ifLF@same\def\LFoutfile{lof}\else
         \strcfstr{\LFcounter}{table}
           \ifLF@same\def\LFoutfile{lot}\else
             \def\LFoutfile{lft}\fi\fi
     \addcontentsline{\LFoutfile}{\LFcounter}
     {\expandafter\protect\expandafter\numberline\expandafter%
  {\expandafter\csname the\LFcounter\endcsname}{#2}}}%
  \fi
}
%    \end{macrocode}
%    The \cs{LF@c@ption} macro ends the process when it calls the
%    \cs{LF@makecaption} macro, which typesets the caption.
%    \begin{macrocode}
\def\LF@capti@n{%
  \@ifstar
    {\egroup\LF@c@ption\@gobble[]}%
    {\egroup\@xdblarg{\LF@c@ption\@firstofone}}}
%    \end{macrocode}
%
%    If you want to redefine how the |longfigure| is captioned,
%    you need to override the following macro.
%    The first argument is the name of the counter
%    (for example, |Figure|), the second argument is the number of the counter, and the
%    third argument is the caption itself.
%    \begin{macrocode}
\def\LF@makecaption#1#2#3{%
  \LF@mcol\LF@cols c{\hbox to\z@{\hss\parbox[t]\LFcapwidth{%
    \sbox\@tempboxa{#1{#2: }#3}%
    \ifdim\wd\@tempboxa>\hsize
      #1{#2: }#3%
    \else
      \hbox to\hsize{\hfil\box\@tempboxa\hfil}%
    \fi
    \endgraf\vskip\baselineskip}%
  \hss}}}
\def\LF@output{%
  \ifnum\outputpenalty <-\@Mi
    \ifnum\outputpenalty > -\LF@end@pen
      \LF@err{floats and marginpars not allowed in a longfigure}\@ehc
    \else
      \setbox\z@\vbox{\unvbox\@cclv}%
      \ifdim \ht\LF@lastfoot>\ht\LF@foot
        \dimen@\pagegoal
        \advance\dimen@-\ht\LF@lastfoot
        \ifdim\dimen@<\ht\z@
          \setbox\@cclv\vbox{\unvbox\z@\copy\LF@foot\vss}%
          \@makecol
          \@outputpage
          \setbox\z@\vbox{\box\LF@head}%
        \fi
      \fi
      \global\@colroom\@colht
      \global\vsize\@colht
      \vbox
        {\unvbox\z@\box\ifvoid\LF@lastfoot\LF@foot\else
         \LF@lastfoot\fi}%
    \fi
  \else
    \setbox\@cclv\vbox{\unvbox\@cclv\copy\LF@foot\vss}%
    \@makecol
    \@outputpage
      \global\vsize\@colroom
    \copy\LF@head\nobreak
  \fi}
\def\LF@end@hd@ft#1{%
  \LF@echunk
  \ifx\LF@start\endgraf
    \LF@err
     {Longfigure head or foot not at start of table}%
     {Increase LFchunksize}%
  \fi
  \setbox#1\box\z@
  \LF@get@widths
  \LF@bchunk}
%    \end{macrocode}
%    The following four macros do not have an \cs{LT} prefix in the |longtable| package,
%    but they must be redefined to have an \cs{LF} prefix in order to avoid a namespace clash;
%    \begin{macrocode}
\def\endLFfirsthead{\LF@end@hd@ft\LF@firsthead}
\def\endLFhead{\LF@end@hd@ft\LF@head}
\def\endLFfoot{\LF@end@hd@ft\LF@foot}
\def\endLFlastfoot{\LF@end@hd@ft\LF@lastfoot}
 %
\def\LF@startpbox#1{%
  \bgroup
    \let\@footnotetext\LF@p@ftntext
    \setlength\hsize{#1}%
    \@arrayparboxrestore
    \vrule \@height \ht\@arstrutbox \@width \z@}
\def\LF@endpbox{%
  \@finalstrut\@arstrutbox
  \egroup
  \the\LF@p@ftn
  \global\LF@p@ftn{}%
  \hfil}
\def\LF@p@ftntext#1{%
  \edef\@tempa{\the\LF@p@ftn\noexpand\footnotetext[\the\c@footnote]}%
  \global\LF@p@ftn\expandafter{\@tempa{#1}}}%
%    \end{macrocode}
%    \subsection{References}
%    \begin{description}
%    \item[\textmd{Carlisle, D. 2004.}]
%    \emph{The longtable Package}.
%    Included in the ``Comprehensive \TeX\ Archive Network.'' \url{http://ctan.org}.
%
%    \item[\textmd{Lazarides, Y. 2010.}]
%    \TeX stackexchange, online forum. \url{http://tex.stackexchange.com/questions/7992}.
%
%    \item[\textmd{Sch\"opf, R., B. Raichle, and C. Rowley. 2001.}]
%    \emph{A New Implementation of \LaTeX's \texttt{verbatim} and \texttt{verbatim*} Environments}.
%    Originally appeared in TUGboat 1990, 11(2), 284--296.
%
%    \item[\textmd{Thanh, H., S. Rahtz, H. Hagen, and H. Henkel. 2009.}]
%    ``The pdf\TeX\ User's Manual'' Revision 655, corresponding to pdf\TeX\ 1.40.11.
%    \url{www.tug.org/applications/pdftex}
%
%    \item[\textmd{Wilson, Peter. 2001.}]
%    \emph{Glisterings} in TUGboat 22(4), 339--340.
%    \end{description}
%
%\StopEventually{^^A
%  \PrintChanges
%  \PrintIndex
%}
%
%    \begin{macrocode}
%</longfigure>
%    \end{macrocode}
%\Finale
