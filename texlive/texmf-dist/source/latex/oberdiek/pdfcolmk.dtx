% \iffalse meta-comment
%
% File: pdfcolmk.dtx
% Version: 2008/08/11 v1.2
% Info: Color support for pdfTeX via marks
%
% Copyright (C) 2000, 2005-2008 by
%    Heiko Oberdiek <heiko.oberdiek at googlemail.com>
%
% This work may be distributed and/or modified under the
% conditions of the LaTeX Project Public License, either
% version 1.3c of this license or (at your option) any later
% version. This version of this license is in
%    http://www.latex-project.org/lppl/lppl-1-3c.txt
% and the latest version of this license is in
%    http://www.latex-project.org/lppl.txt
% and version 1.3 or later is part of all distributions of
% LaTeX version 2005/12/01 or later.
%
% This work has the LPPL maintenance status "maintained".
%
% This Current Maintainer of this work is Heiko Oberdiek.
%
% This work consists of the main source file pdfcolmk.dtx
% and the derived files
%    pdfcolmk.sty, pdfcolmk.pdf, pdfcolmk.ins, pdfcolmk.drv.
%
% Distribution:
%    CTAN:macros/latex/contrib/oberdiek/pdfcolmk.dtx
%    CTAN:macros/latex/contrib/oberdiek/pdfcolmk.pdf
%
% Unpacking:
%    (a) If pdfcolmk.ins is present:
%           tex pdfcolmk.ins
%    (b) Without pdfcolmk.ins:
%           tex pdfcolmk.dtx
%    (c) If you insist on using LaTeX
%           latex \let\install=y% \iffalse meta-comment
%
% File: pdfcolmk.dtx
% Version: 2008/08/11 v1.2
% Info: Color support for pdfTeX via marks
%
% Copyright (C) 2000, 2005-2008 by
%    Heiko Oberdiek <heiko.oberdiek at googlemail.com>
%
% This work may be distributed and/or modified under the
% conditions of the LaTeX Project Public License, either
% version 1.3c of this license or (at your option) any later
% version. This version of this license is in
%    http://www.latex-project.org/lppl/lppl-1-3c.txt
% and the latest version of this license is in
%    http://www.latex-project.org/lppl.txt
% and version 1.3 or later is part of all distributions of
% LaTeX version 2005/12/01 or later.
%
% This work has the LPPL maintenance status "maintained".
%
% This Current Maintainer of this work is Heiko Oberdiek.
%
% This work consists of the main source file pdfcolmk.dtx
% and the derived files
%    pdfcolmk.sty, pdfcolmk.pdf, pdfcolmk.ins, pdfcolmk.drv.
%
% Distribution:
%    CTAN:macros/latex/contrib/oberdiek/pdfcolmk.dtx
%    CTAN:macros/latex/contrib/oberdiek/pdfcolmk.pdf
%
% Unpacking:
%    (a) If pdfcolmk.ins is present:
%           tex pdfcolmk.ins
%    (b) Without pdfcolmk.ins:
%           tex pdfcolmk.dtx
%    (c) If you insist on using LaTeX
%           latex \let\install=y% \iffalse meta-comment
%
% File: pdfcolmk.dtx
% Version: 2008/08/11 v1.2
% Info: Color support for pdfTeX via marks
%
% Copyright (C) 2000, 2005-2008 by
%    Heiko Oberdiek <heiko.oberdiek at googlemail.com>
%
% This work may be distributed and/or modified under the
% conditions of the LaTeX Project Public License, either
% version 1.3c of this license or (at your option) any later
% version. This version of this license is in
%    http://www.latex-project.org/lppl/lppl-1-3c.txt
% and the latest version of this license is in
%    http://www.latex-project.org/lppl.txt
% and version 1.3 or later is part of all distributions of
% LaTeX version 2005/12/01 or later.
%
% This work has the LPPL maintenance status "maintained".
%
% This Current Maintainer of this work is Heiko Oberdiek.
%
% This work consists of the main source file pdfcolmk.dtx
% and the derived files
%    pdfcolmk.sty, pdfcolmk.pdf, pdfcolmk.ins, pdfcolmk.drv.
%
% Distribution:
%    CTAN:macros/latex/contrib/oberdiek/pdfcolmk.dtx
%    CTAN:macros/latex/contrib/oberdiek/pdfcolmk.pdf
%
% Unpacking:
%    (a) If pdfcolmk.ins is present:
%           tex pdfcolmk.ins
%    (b) Without pdfcolmk.ins:
%           tex pdfcolmk.dtx
%    (c) If you insist on using LaTeX
%           latex \let\install=y% \iffalse meta-comment
%
% File: pdfcolmk.dtx
% Version: 2008/08/11 v1.2
% Info: Color support for pdfTeX via marks
%
% Copyright (C) 2000, 2005-2008 by
%    Heiko Oberdiek <heiko.oberdiek at googlemail.com>
%
% This work may be distributed and/or modified under the
% conditions of the LaTeX Project Public License, either
% version 1.3c of this license or (at your option) any later
% version. This version of this license is in
%    http://www.latex-project.org/lppl/lppl-1-3c.txt
% and the latest version of this license is in
%    http://www.latex-project.org/lppl.txt
% and version 1.3 or later is part of all distributions of
% LaTeX version 2005/12/01 or later.
%
% This work has the LPPL maintenance status "maintained".
%
% This Current Maintainer of this work is Heiko Oberdiek.
%
% This work consists of the main source file pdfcolmk.dtx
% and the derived files
%    pdfcolmk.sty, pdfcolmk.pdf, pdfcolmk.ins, pdfcolmk.drv.
%
% Distribution:
%    CTAN:macros/latex/contrib/oberdiek/pdfcolmk.dtx
%    CTAN:macros/latex/contrib/oberdiek/pdfcolmk.pdf
%
% Unpacking:
%    (a) If pdfcolmk.ins is present:
%           tex pdfcolmk.ins
%    (b) Without pdfcolmk.ins:
%           tex pdfcolmk.dtx
%    (c) If you insist on using LaTeX
%           latex \let\install=y\input{pdfcolmk.dtx}
%        (quote the arguments according to the demands of your shell)
%
% Documentation:
%    (a) If pdfcolmk.drv is present:
%           latex pdfcolmk.drv
%    (b) Without pdfcolmk.drv:
%           latex pdfcolmk.dtx; ...
%    The class ltxdoc loads the configuration file ltxdoc.cfg
%    if available. Here you can specify further options, e.g.
%    use A4 as paper format:
%       \PassOptionsToClass{a4paper}{article}
%
%    Programm calls to get the documentation (example):
%       pdflatex pdfcolmk.dtx
%       makeindex -s gind.ist pdfcolmk.idx
%       pdflatex pdfcolmk.dtx
%       makeindex -s gind.ist pdfcolmk.idx
%       pdflatex pdfcolmk.dtx
%
% Installation:
%    TDS:tex/latex/oberdiek/pdfcolmk.sty
%    TDS:doc/latex/oberdiek/pdfcolmk.pdf
%    TDS:source/latex/oberdiek/pdfcolmk.dtx
%
%<*ignore>
\begingroup
  \catcode123=1 %
  \catcode125=2 %
  \def\x{LaTeX2e}%
\expandafter\endgroup
\ifcase 0\ifx\install y1\fi\expandafter
         \ifx\csname processbatchFile\endcsname\relax\else1\fi
         \ifx\fmtname\x\else 1\fi\relax
\else\csname fi\endcsname
%</ignore>
%<*install>
\input docstrip.tex
\Msg{************************************************************************}
\Msg{* Installation}
\Msg{* Package: pdfcolmk 2008/08/11 v1.2 Color support for pdfTeX via marks (HO)}
\Msg{************************************************************************}

\keepsilent
\askforoverwritefalse

\let\MetaPrefix\relax
\preamble

This is a generated file.

Project: pdfcolmk
Version: 2008/08/11 v1.2

Copyright (C) 2000, 2005-2008 by
   Heiko Oberdiek <heiko.oberdiek at googlemail.com>

This work may be distributed and/or modified under the
conditions of the LaTeX Project Public License, either
version 1.3c of this license or (at your option) any later
version. This version of this license is in
   http://www.latex-project.org/lppl/lppl-1-3c.txt
and the latest version of this license is in
   http://www.latex-project.org/lppl.txt
and version 1.3 or later is part of all distributions of
LaTeX version 2005/12/01 or later.

This work has the LPPL maintenance status "maintained".

This Current Maintainer of this work is Heiko Oberdiek.

This work consists of the main source file pdfcolmk.dtx
and the derived files
   pdfcolmk.sty, pdfcolmk.pdf, pdfcolmk.ins, pdfcolmk.drv.

\endpreamble
\let\MetaPrefix\DoubleperCent

\generate{%
  \file{pdfcolmk.ins}{\from{pdfcolmk.dtx}{install}}%
  \file{pdfcolmk.drv}{\from{pdfcolmk.dtx}{driver}}%
  \usedir{tex/latex/oberdiek}%
  \file{pdfcolmk.sty}{\from{pdfcolmk.dtx}{package}}%
  \nopreamble
  \nopostamble
  \usedir{source/latex/oberdiek/catalogue}%
  \file{pdfcolmk.xml}{\from{pdfcolmk.dtx}{catalogue}}%
}

\catcode32=13\relax% active space
\let =\space%
\Msg{************************************************************************}
\Msg{*}
\Msg{* To finish the installation you have to move the following}
\Msg{* file into a directory searched by TeX:}
\Msg{*}
\Msg{*     pdfcolmk.sty}
\Msg{*}
\Msg{* To produce the documentation run the file `pdfcolmk.drv'}
\Msg{* through LaTeX.}
\Msg{*}
\Msg{* Happy TeXing!}
\Msg{*}
\Msg{************************************************************************}

\endbatchfile
%</install>
%<*ignore>
\fi
%</ignore>
%<*driver>
\NeedsTeXFormat{LaTeX2e}
\ProvidesFile{pdfcolmk.drv}%
  [2008/08/11 v1.2 Color support for pdfTeX via marks (HO)]%
\documentclass{ltxdoc}
\usepackage{holtxdoc}[2011/11/22]
\begin{document}
  \DocInput{pdfcolmk.dtx}%
\end{document}
%</driver>
% \fi
%
% \CheckSum{843}
%
% \CharacterTable
%  {Upper-case    \A\B\C\D\E\F\G\H\I\J\K\L\M\N\O\P\Q\R\S\T\U\V\W\X\Y\Z
%   Lower-case    \a\b\c\d\e\f\g\h\i\j\k\l\m\n\o\p\q\r\s\t\u\v\w\x\y\z
%   Digits        \0\1\2\3\4\5\6\7\8\9
%   Exclamation   \!     Double quote  \"     Hash (number) \#
%   Dollar        \$     Percent       \%     Ampersand     \&
%   Acute accent  \'     Left paren    \(     Right paren   \)
%   Asterisk      \*     Plus          \+     Comma         \,
%   Minus         \-     Point         \.     Solidus       \/
%   Colon         \:     Semicolon     \;     Less than     \<
%   Equals        \=     Greater than  \>     Question mark \?
%   Commercial at \@     Left bracket  \[     Backslash     \\
%   Right bracket \]     Circumflex    \^     Underscore    \_
%   Grave accent  \`     Left brace    \{     Vertical bar  \|
%   Right brace   \}     Tilde         \~}
%
% \GetFileInfo{pdfcolmk.drv}
%
% \title{The \xpackage{pdfcolmk} package}
% \date{2008/08/11 v1.2}
% \author{Heiko Oberdiek\\\xemail{heiko.oberdiek at googlemail.com}}
%
% \maketitle
%
% \begin{abstract}
% This package tries a solution for the missing color
% stack of \pdfTeX.
% \end{abstract}
%
% \tableofcontents
%
% \section{Documentation}
%
% \subsection{Introduction}
%
% This package uses a mark register in order to solve the
% problem of a missing color stack of \pdfTeX\ prior 1.40.0.
% Since this version of \pdfTeX\ a color stack is available
% and supported by \xfile{pdftex.def} 2007/01/01 v0.04a.
% In this case this package is obsolete and the package
% stops its loading.
%
% \subsection{Background}
%
% After the Dante meeting (Clausthal 2000) I have started
% to experiment with the eTeX method of a \emph{colour} mark.
% One of the major problems is the understanding of the
% output routine and the need to rewrite it because of
% missing hooks. Currently I have made some tests in
% in onecolumn and twocolumn mode, but the state is
% experimental.
%
% \subsection{Limitations}
%
% \begin{itemize}
% \item Mark limitations: page breaks in math.
% \item \LaTeX's output routine is redefinded.
%   \begin{itemize}
%   \item Changes in the output routine of newer versions
%         of LaTeX are not detected.
%   \item Packages that change the output routine are not
%         supported.
%   \end{itemize}
% \item It does not support several independent text
%       streams like footnotes.
% \item Limitations in float and marginpar support.
% \end{itemize}
%
% \subsection{Recommendation}
%
% \eTeX\ (for additional mark register)
% Without \eTeX\ \LaTeX's mark commands are redefined
% to store an additional color value.
%
% \subsection{Usage}
%
% Load after package color:
% \begin{quote}
%   |\usepackage[pdftex]{color}|\\
%   |\usepackage{pdfcolmk}|
% \end{quote}
%
% \subsection{Compatibility}
%
% \begin{itemize}
% \item Load the following packages after \xpackage{pdfcolmk}:
%   \begin{quote}
%       \xpackage{mparhack.sty}
%   \end{quote}
% \item Load the following packages before \xpackage{pdfcolmk}:
%   \begin{quote}
%       \xpackage{marn.sty}\\
%       \xpackage{newmarn.sty}
%   \end{quote}
% \item Supported \cs{@addmarginpar} patch:
%   \begin{quote}
%       \xpackage{latex/base/latex.ltx}\\
%       \xpackage{memoir.cls}\\
%       \xpackage{poemscol/marn.sty}, \xpackage{poemscol/newmarn.sty}\\
%       \xpackage{mparhack.sty}
%   \end{quote}
% \item Unsupported \cs{@addmarginpar} patch:
%   \begin{quote}
%       \xpackage{lineno.sty}\\
%       \xpackage{sttools/marginal.sty}\\
%       \xpackage{revtex4.cls}
%   \end{quote}
% \end{itemize}
%
% \StopEventually{
% }
%
% \section{Implementation}
%
%    \begin{macrocode}
%<*package>
%    \end{macrocode}
%    Package identification.
%    \begin{macrocode}
\NeedsTeXFormat{LaTeX2e}
\ProvidesPackage{pdfcolmk}%
  [2008/08/11 v1.2 Color support for pdfTeX via marks (HO)]
%    \end{macrocode}
%
%    \begin{macrocode}
\@ifundefined{ver@pdftex.def}{%
  \PackageWarningNoLine{pdfcolmk}{%
    Nothing to fix, because \string`pdftex.def\string' not loaded%
  }%
  \endinput
}{}
\@ifpackageloaded{color}{}{%
  \PackageWarningNoLine{pdfcolmk}{%
    Nothing to fix, because \string`color.sty\string' not loaded%
  }%
  \endinput
}
\begingroup\expandafter\expandafter\expandafter\endgroup
\expandafter\ifx\csname main@pdfcolorstack\endcsname\relax
\else
  % pdftex.def >= 2007/01/01 0.04a and pdfTeX >= 1.40.0
  \begingroup
    \let\on@line\@empty
    \PackageInfo{pdfcolmk}{%
      The color stack of pdfTeX \string>\string= 1.40 is used. %
      Therefore\MessageBreak
      this package is not necessary and not loaded%
    }%
  \endgroup
  \expandafter\endinput
\fi

\PackageInfo{pdfcolmk}{%
  This package tries to simulate dvips's color stack\MessageBreak
  for pdfTeX based on a mark register of e-TeX.\MessageBreak
  It redefines LaTeX's output routine. Therefore\MessageBreak
  use with care, no warranties%
}

\ifx\marks\@undefined

  \let\pec@mark\mark
  \let\pec@value\empty
  \long\def\mark#1{%
    \protected@xdef\pec@value{#1}%
    \pec@setmark
  }%
  \def\pec@setmark{%
    \begingroup
      \@temptokena\expandafter{\pec@value}%
      \pec@mark{{\current@color}\the\@temptokena}%
    \endgroup
  }%
  \def\pec@getmark{%
    \xdef\pec@botcolor{%
      \expandafter\@firstofthree\botmark\@empty\@empty\@empty
    }%
  }%
  \long\def\@firstofthree#1#2#3{#1}%
  \CheckCommand{\@leftmark}[2]{#1}%
  \CheckCommand{\@rightmark}[2]{#2}%
  \CheckCommand*{\leftmark}{%
    \expandafter\@leftmark\botmark\@empty\@empty
  }%
  \CheckCommand*{\rightmark}{%
    \expandafter\@rightmark\firstmark\@empty\@empty
  }%
  \long\def\@leftmark#1#2#3{#2}%
  \long\def\@rightmark#1#2#3{#3}%
  \g@addto@macro\leftmark\@empty
  \g@addto@macro\rightmark\@empty

\else

  \RequirePackage{etex}[1998/03/26]%
  \newmarks\pec@marks
  \def\pec@setmark{\marks\pec@marks{\current@color}}%
  \def\pec@getmark{\xdef\pec@botcolor{\botmarks\pec@marks}}%

\fi
%    \end{macrocode}
%
% \subsection{\cs{marginpar} fix}
%
%    \begin{macrocode}
\chardef\pec@result\z@
\def\pec@temp#1{%
  \chardef\pec@result\@ne
  \begingroup
    \let\on@line\@empty
    \PackageInfo{pdfcolmk}{%
      Patch for \string\@addmarginpar\space applied (#1)%
    }%
  \endgroup
}
%    \end{macrocode}
%
% \subsubsection{latex/base/latex.ltx}
%
%    \begin{macrocode}
\def\pec@addmarginpar{%
  \@next\@marbox\@currlist{%
    \@cons\@freelist\@marbox
    \@cons\@freelist\@currbox
  }\@latexbug
  \@tempcnta\@ne
  \if@twocolumn
    \if@firstcolumn
      \@tempcnta\m@ne
    \fi
  \else
    \if@mparswitch
      \ifodd\c@page
      \else
        \@tempcnta\m@ne
      \fi
    \fi
    \if@reversemargin \@tempcnta -\@tempcnta \fi
  \fi
  \ifnum\@tempcnta <\z@  \global\setbox\@marbox\box\@currbox \fi
  \@tempdima\@mparbottom
  \advance\@tempdima -\@pageht
  \advance\@tempdima\ht\@marbox
  \ifdim\@tempdima >\z@
    \@latex@warning@no@line{Marginpar on page \thepage\space moved}%
  \else
    \@tempdima\z@
  \fi
  \global\@mparbottom\@pageht
  \global\advance\@mparbottom\@tempdima
  \global\advance\@mparbottom\dp\@marbox
  \global\advance\@mparbottom\marginparpush
  \advance\@tempdima -\ht\@marbox
  \global\setbox\@marbox\vbox{%
    \vskip \@tempdima
    \box \@marbox
  }%
  \global \ht\@marbox \z@
  \global \dp\@marbox \z@
  \kern -\@pagedp
  \nointerlineskip
  \hb@xt@\columnwidth{%
    \ifnum \@tempcnta >\z@
      \hskip\columnwidth
      \hskip\marginparsep
    \else
      \hskip -\marginparsep
      \hskip -\marginparwidth
    \fi
    \box\@marbox \hss
  }%
  \nointerlineskip
  \hbox{\vrule \@height\z@ \@width\z@ \@depth\@pagedp}%
}
\ifx\pec@addmarginpar\@addmarginpar
  \pec@temp{latex/base}%
\fi
%    \end{macrocode}
%
% \subsubsection{memoir.cls}
%
%    \begin{macrocode}
\def\pec@addmarginpar{%
  \checkoddpage
  \@next\@marbox\@currlist{%
    \@cons\@freelist\@marbox
    \@cons\@freelist\@currbox
  }\@latexbug
  \@tempcnta\@ne
  \if@twocolumn
    \if@firstcolumn
      \@tempcnta\m@ne
    \fi
  \else
    \if@mparswitch
      \ifoddpage
      \else
        \@tempcnta\m@ne
      \fi
    \fi
    \if@reversemargin
      \@tempcnta -\@tempcnta
    \fi
  \fi
  \ifnum\@tempcnta <\z@
    \global\setbox\@marbox\box\@currbox
  \fi
  \@tempdima\@mparbottom
  \advance\@tempdima -\@pageht
  \advance\@tempdima\ht\@marbox
  \ifdim\@tempdima >\z@
    \@latex@warning@no@line{%
      Marginpar on page \thepage\space moved by \the\@tempdima
    }%
  \else
    \@tempdima\z@
  \fi
  \global\@mparbottom\@pageht
  \global\advance\@mparbottom\@tempdima
  \global\advance\@mparbottom\dp\@marbox
  \global\advance\@mparbottom\marginparpush
  \advance\@tempdima -\ht\@marbox
  \global\setbox\@marbox\vbox{%
    \vskip \@tempdima
    \box \@marbox
  }%
  \global \ht\@marbox \z@
  \global \dp\@marbox \z@
  \kern -\@pagedp
  \nointerlineskip
  \hb@xt@\columnwidth{%
    \ifnum \@tempcnta >\z@
      \hskip\columnwidth
      \hskip\marginparsep
    \else
      \hskip -\marginparsep
      \hskip -\marginparwidth
    \fi
    \box\@marbox
    \hss
  }%
  \nointerlineskip
  \hbox{\vrule \@height\z@ \@width\z@ \@depth\@pagedp}%
}%
\ifx\pec@addmarginpar\@addmarginpar
  \pec@temp{memoir.cls}%
\fi
%    \end{macrocode}
%
% \subsubsection{poemscol/marn.sty, poemscol/newmarn.sty}
%
%    \begin{macrocode}
\def\pec@addmarginpar{%
  \@next \@marbox\@currlist{%
    \@cons\@freelist\@marbox
    \@cons\@freelist\@currbox
  }\@latexbug
  \global\advance\@mpar@count\m@ne
  \@ifundefined{@marn@\the\@mpar@count @}{% was location logged last time?
    \@tempcnta\@ne % NO: use original LaTeX logic
    \if@twocolumn
      \if@firstcolumn
        \@tempcnta\m@ne
      \fi
    \else
      \if@mparswitch
        \ifodd\c@page
        \else
          \@tempcnta\m@ne
        \fi
      \fi
      \if@reversemargin
        \@tempcnta -\@tempcnta
      \fi
    \fi
  }{%
    \@tempcnta %    YES: use record from last time to decide side.
    \@nameuse{@marn@\the\@mpar@count @}%
    \if@reversemargin -\fi \@ne
  }%
  \ifnum\@tempcnta <\z@
    \global\setbox\@marbox\box\@currbox
    \global\let\@marnbottom\@mparbottoml
  \else
    \global\let\@marnbottom\@mparbottom
  \fi
  \@tempdima\@marnbottom \advance\@tempdima -\@pageht
  \advance\@tempdima\ht\@marbox
  \ifdim\@tempdima >\z@
    \@@warning{Marginpar on page \thepage\space moved}%
  \else
    \@tempdima\z@
  \fi
  \global\@marnbottom\@pageht
  \global\advance\@marnbottom\@tempdima
  \global\advance\@marnbottom\dp\@marbox
  \global\advance\@marnbottom\marginparpush
  \advance\@tempdima -\ht\@marbox
  \global\ht\@marbox\z@
  \global\dp\@marbox\z@
  \vskip -\@pagedp
  \vskip\@tempdima\nointerlineskip
  \hbox to\columnwidth{%
    \ifnum \@tempcnta >\z@
      \hskip\columnwidth
      \hskip\marginparsep
    \else
      \hskip -\marginparsep
      \hskip -\marginparwidth
    \fi
    \if@filesw % record where this is for use next time:
       \@marn@log\@mpar@count
    \fi
    \box\@marbox
    \hss
  }%
  \nobreak   %% RmS 91/06/21 \nobreak added
  \vskip -\@tempdima
  \nointerlineskip
  \hbox{\vrule \@height\z@ \@width\z@ \@depth\@pagedp}%
}
\ifx\pec@addmarginpar\@addmarginpar
  \pec@temp{poemscol/(new)marn.sty}%
\fi
%    \end{macrocode}
%
% \subsubsection{refman/refart.cls, refnam/refrep.cls}
%
%    \begin{macrocode}
\def\pec@addmarginpar{%
  \@next\@marbox\@currlist{%
    \@cons\@freelist\@marbox
    \@cons\@freelist\@currbox
  }\@latexbug
  \@tempcnta\@ne
  \if@twocolumn
    \if@firstcolumn
      \@tempcnta\m@ne
    \fi
  \else
    \@tempcnta\m@ne
  \fi
  \ifnum\@tempcnta <\z@
    \global\setbox\@marbox\box\@currbox
  \fi
  \@tempdima\@mparbottom
  \advance\@tempdima -\@pageht
  \advance\@tempdima\ht\@marbox
  \ifdim\@tempdima >\z@
     \@@warning{Marginpar on page \thepage\space moved}%
  \else
     \@tempdima\z@
  \fi
  \global\@mparbottom\@pageht
  \global\advance\@mparbottom\@tempdima
  \global\advance\@mparbottom\dp\@marbox
  \global\advance\@mparbottom\marginparpush
  \advance\@tempdima -\ht\@marbox
  \global\setbox\@marbox\vbox{%
    \vskip \@tempdima \box \@marbox
  }%
  \global \ht\@marbox \z@
  \global \dp\@marbox \z@
  \kern -\@pagedp
  \nointerlineskip
  \hb@xt@\columnwidth{%
    \ifnum \@tempcnta >\z@
      \hskip\columnwidth
      \hskip\marginparsep
    \else
      \hskip -\marginparsep
      \hskip -\marginparwidth
    \fi
    \box\@marbox
    \hss
  }%
  \nointerlineskip
  \hbox{\vrule \@height\z@ \@width\z@ \@depth\@pagedp}%
}
\ifx\pec@addmarginpar\@addmarginpar
  \pec@temp{ref(art|rep).cls}%
\fi

\ifcase\pec@result
  \PackageInfo{pdfcolmk}{%
    Fix for \string\@addmarginpar\space is omitted, %
    because this variant\MessageBreak
    of \string\@addmarginpar\space
      is not recognized%
  }%
\else
  % apply patch for \@addmarginpar
  \def\pec@PatchAddMarginpar#1\columnwidth#2#3\@nil{%
    \pec@PatchAddMarginparI#2\@nil{#1}{#3}%
  }%
  \def\pec@PatchAddMarginparI#1\box\@marbox\hss#2\@nil#3#4{%
    \def\@addmarginpar{%
      #3%
      \columnwidth{%
        #1%
        \pdfliteral{q}%
        \rlap{%
          \box\@marbox
        }%
        \pdfliteral{Q}%
        \hss
        #2%
      }%
      #4%
    }%
  }%
  \expandafter\pec@PatchAddMarginpar\@addmarginpar\@nil
\fi
%    \end{macrocode}
%
% \subsection{Color fix}
%
%    \begin{macrocode}
\def\set@color{%
  \pdfliteral{\current@color}%
  \ifinner
  \else
    \pec@setmark
  \fi
  \aftergroup\reset@color
}
\def\reset@color{%
  \pdfliteral{\current@color}%
  \ifinner
  \else
    \pec@setmark
  \fi
}

\let\pec@botcolor\current@color

\def\pec@PatchVBoxCCLV{%
  \ifx\pec@botcolor\@empty
  \else
    \setbox\@cclv\vbox{%
      \pdfliteral{\pec@botcolor}%
      \unvbox\@cclv
    }%
  \fi
  \pec@getmark
}

\def\pec@PatchAlreadyInBox{%
  \ifx\pec@botcolor\@empty
  \else
    \pdfliteral{\pec@botcolor}%
  \fi
  \pec@getmark
}

\@ifclassloaded{memoir}{%
  \expandafter\def\expandafter\mem@makecol\expandafter{%
    \expandafter\pec@PatchVBoxCCLV
    \mem@makecol
  }%
  \endinput
}{}

\@ifclassloaded{seminar}{%
  \newcommand\pec@org@makeslide{}%
  \let\pec@org@makeslide\@makeslide
  \def\@makeslide{%
    \pec@PatchVBoxCCLV
    \pec@org@makeslide
  }%
  \endinput
}{}

\long\def\pec@output#1\@specialoutput\else#2\pec@end{%
  \begingroup
    \def\x{#2}%
  \expandafter\endgroup
  \ifx\x\@empty
    \PackageWarningNoLine{pdfcolmk}{%
      Unexpected \string\output\space routine detected,%
      \MessageBreak
      loading of package stopped%
    }%
    \expandafter\endinput
  \fi
}
\expandafter\expandafter\expandafter\pec@output
\expandafter\@firstofone\the\output\@specialoutput\else\pec@end

\long\def\pec@output#1\@specialoutput\else#2\pec@end{%
  \output{%
    #1\@specialoutput\else
    \pec@PatchVBoxCCLV
    #2%
  }%
}
\expandafter\expandafter\expandafter\pec@output
\expandafter\@firstofone\the\output\pec@end
%    \end{macrocode}
%
%    \begin{macrocode}
%</package>
%    \end{macrocode}
%
% \section{Installation}
%
% \subsection{Download}
%
% \paragraph{Package.} This package is available on
% CTAN\footnote{\url{ftp://ftp.ctan.org/tex-archive/}}:
% \begin{description}
% \item[\CTAN{macros/latex/contrib/oberdiek/pdfcolmk.dtx}] The source file.
% \item[\CTAN{macros/latex/contrib/oberdiek/pdfcolmk.pdf}] Documentation.
% \end{description}
%
%
% \paragraph{Bundle.} All the packages of the bundle `oberdiek'
% are also available in a TDS compliant ZIP archive. There
% the packages are already unpacked and the documentation files
% are generated. The files and directories obey the TDS standard.
% \begin{description}
% \item[\CTAN{install/macros/latex/contrib/oberdiek.tds.zip}]
% \end{description}
% \emph{TDS} refers to the standard ``A Directory Structure
% for \TeX\ Files'' (\CTAN{tds/tds.pdf}). Directories
% with \xfile{texmf} in their name are usually organized this way.
%
% \subsection{Bundle installation}
%
% \paragraph{Unpacking.} Unpack the \xfile{oberdiek.tds.zip} in the
% TDS tree (also known as \xfile{texmf} tree) of your choice.
% Example (linux):
% \begin{quote}
%   |unzip oberdiek.tds.zip -d ~/texmf|
% \end{quote}
%
% \paragraph{Script installation.}
% Check the directory \xfile{TDS:scripts/oberdiek/} for
% scripts that need further installation steps.
% Package \xpackage{attachfile2} comes with the Perl script
% \xfile{pdfatfi.pl} that should be installed in such a way
% that it can be called as \texttt{pdfatfi}.
% Example (linux):
% \begin{quote}
%   |chmod +x scripts/oberdiek/pdfatfi.pl|\\
%   |cp scripts/oberdiek/pdfatfi.pl /usr/local/bin/|
% \end{quote}
%
% \subsection{Package installation}
%
% \paragraph{Unpacking.} The \xfile{.dtx} file is a self-extracting
% \docstrip\ archive. The files are extracted by running the
% \xfile{.dtx} through \plainTeX:
% \begin{quote}
%   \verb|tex pdfcolmk.dtx|
% \end{quote}
%
% \paragraph{TDS.} Now the different files must be moved into
% the different directories in your installation TDS tree
% (also known as \xfile{texmf} tree):
% \begin{quote}
% \def\t{^^A
% \begin{tabular}{@{}>{\ttfamily}l@{ $\rightarrow$ }>{\ttfamily}l@{}}
%   pdfcolmk.sty & tex/latex/oberdiek/pdfcolmk.sty\\
%   pdfcolmk.pdf & doc/latex/oberdiek/pdfcolmk.pdf\\
%   pdfcolmk.dtx & source/latex/oberdiek/pdfcolmk.dtx\\
% \end{tabular}^^A
% }^^A
% \sbox0{\t}^^A
% \ifdim\wd0>\linewidth
%   \begingroup
%     \advance\linewidth by\leftmargin
%     \advance\linewidth by\rightmargin
%   \edef\x{\endgroup
%     \def\noexpand\lw{\the\linewidth}^^A
%   }\x
%   \def\lwbox{^^A
%     \leavevmode
%     \hbox to \linewidth{^^A
%       \kern-\leftmargin\relax
%       \hss
%       \usebox0
%       \hss
%       \kern-\rightmargin\relax
%     }^^A
%   }^^A
%   \ifdim\wd0>\lw
%     \sbox0{\small\t}^^A
%     \ifdim\wd0>\linewidth
%       \ifdim\wd0>\lw
%         \sbox0{\footnotesize\t}^^A
%         \ifdim\wd0>\linewidth
%           \ifdim\wd0>\lw
%             \sbox0{\scriptsize\t}^^A
%             \ifdim\wd0>\linewidth
%               \ifdim\wd0>\lw
%                 \sbox0{\tiny\t}^^A
%                 \ifdim\wd0>\linewidth
%                   \lwbox
%                 \else
%                   \usebox0
%                 \fi
%               \else
%                 \lwbox
%               \fi
%             \else
%               \usebox0
%             \fi
%           \else
%             \lwbox
%           \fi
%         \else
%           \usebox0
%         \fi
%       \else
%         \lwbox
%       \fi
%     \else
%       \usebox0
%     \fi
%   \else
%     \lwbox
%   \fi
% \else
%   \usebox0
% \fi
% \end{quote}
% If you have a \xfile{docstrip.cfg} that configures and enables \docstrip's
% TDS installing feature, then some files can already be in the right
% place, see the documentation of \docstrip.
%
% \subsection{Refresh file name databases}
%
% If your \TeX~distribution
% (\teTeX, \mikTeX, \dots) relies on file name databases, you must refresh
% these. For example, \teTeX\ users run \verb|texhash| or
% \verb|mktexlsr|.
%
% \subsection{Some details for the interested}
%
% \paragraph{Attached source.}
%
% The PDF documentation on CTAN also includes the
% \xfile{.dtx} source file. It can be extracted by
% AcrobatReader 6 or higher. Another option is \textsf{pdftk},
% e.g. unpack the file into the current directory:
% \begin{quote}
%   \verb|pdftk pdfcolmk.pdf unpack_files output .|
% \end{quote}
%
% \paragraph{Unpacking with \LaTeX.}
% The \xfile{.dtx} chooses its action depending on the format:
% \begin{description}
% \item[\plainTeX:] Run \docstrip\ and extract the files.
% \item[\LaTeX:] Generate the documentation.
% \end{description}
% If you insist on using \LaTeX\ for \docstrip\ (really,
% \docstrip\ does not need \LaTeX), then inform the autodetect routine
% about your intention:
% \begin{quote}
%   \verb|latex \let\install=y\input{pdfcolmk.dtx}|
% \end{quote}
% Do not forget to quote the argument according to the demands
% of your shell.
%
% \paragraph{Generating the documentation.}
% You can use both the \xfile{.dtx} or the \xfile{.drv} to generate
% the documentation. The process can be configured by the
% configuration file \xfile{ltxdoc.cfg}. For instance, put this
% line into this file, if you want to have A4 as paper format:
% \begin{quote}
%   \verb|\PassOptionsToClass{a4paper}{article}|
% \end{quote}
% An example follows how to generate the
% documentation with pdf\LaTeX:
% \begin{quote}
%\begin{verbatim}
%pdflatex pdfcolmk.dtx
%makeindex -s gind.ist pdfcolmk.idx
%pdflatex pdfcolmk.dtx
%makeindex -s gind.ist pdfcolmk.idx
%pdflatex pdfcolmk.dtx
%\end{verbatim}
% \end{quote}
%
% \section{Catalogue}
%
% The following XML file can be used as source for the
% \href{http://mirror.ctan.org/help/Catalogue/catalogue.html}{\TeX\ Catalogue}.
% The elements \texttt{caption} and \texttt{description} are imported
% from the original XML file from the Catalogue.
% The name of the XML file in the Catalogue is \xfile{pdfcolmk.xml}.
%    \begin{macrocode}
%<*catalogue>
<?xml version='1.0' encoding='us-ascii'?>
<!DOCTYPE entry SYSTEM 'catalogue.dtd'>
<entry datestamp='$Date$' modifier='$Author$' id='pdfcolmk'>
  <name>pdfcolmk</name>
  <caption>Improving colour support under pdftex.</caption>
  <authorref id='auth:oberdiek'/>
  <copyright owner='Heiko Oberdiek' year='2000,2005-2008'/>
  <license type='lppl1.3'/>
  <version number='1.2'/>
  <description>
    The package provides macros that emulate the &#x2018;colour stack&#x2019;
    functionality of dvips.  The colour stack deals with colour
    manipulations when asynchronous events (like page-breaking) occur;
    pdftex does not (yet) have such a stack, but dvips does, and the
    <xref refid='color'>color</xref> package makes extensive use of
    it.
    <p/>
    This package is an experimental solution to the problem, and works
    best with pdf-e-tex.
    <p/>
    The package is part of the <xref refid='oberdiek'>oberdiek</xref> bundle.
  </description>
  <documentation details='Package documentation'
      href='ctan:/macros/latex/contrib/oberdiek/pdfcolmk.pdf'/>
  <ctan file='true' path='/macros/latex/contrib/oberdiek/pdfcolmk.dtx'/>
  <miktex location='oberdiek'/>
  <texlive location='oberdiek'/>
  <install path='/macros/latex/contrib/oberdiek/oberdiek.tds.zip'/>
</entry>
%</catalogue>
%    \end{macrocode}
%
% \begin{History}
%   \begin{Version}{2000/08/27 v0.1}
%   \item
%     First published version in newsgroup \xnewsgroup{comp.text.tex}:\\
%     \URL{``\link{pdftex: bug with colors?}''}^^A
%     {http://groups.google.com/group/comp.text.tex/msg/6f088e69e4085d2c}
%   \end{Version}
%   \begin{Version}{2000/09/02 v0.2}
%   \item
%     Next try.
%   \end{Version}
%   \begin{Version}{2000/09/02 v0.3}
%   \item
%     Solution without \eTeX\ added.
%   \end{Version}
%   \begin{Version}{2000/09/06 v0.4}
%   \item
%     Patch commands added.
%   \item
%     Patch for seminar.cls added.
%   \end{Version}
%   \begin{Version}{2000/09/06 v0.5}
%   \item
%     Bug fix: initialization of \cs{pec@value} added.
%   \end{Version}
%   \begin{Version}{2005/06/15 v0.6}
%   \item
%     Support for \cs{marginpar} added.
%     See thread in \xnewsgroup{comp.text.tex}:\\
%     \URL{``\link{Using \cs{textcolor} and \cs{marginpar} together}''}^^A
%     {http://groups.google.com/group/comp.text.tex/msg/38ed58f8845a2a4f}
%   \end{Version}
%   \begin{Version}{2005/07/09 v0.7}
%   \item
%     Output support added for \xpackage{memoir},
%     provided by Lars Madsen.
%   \end{Version}
%   \begin{Version}{2006/02/20 v0.8}
%   \item
%     Code is not changed.
%   \item
%     DTX framework.
%   \end{Version}
%   \begin{Version}{2007/01/01 v1.0}
%   \item
%     If \xfile{pdftex.def} \textgreater= 2007/01/01 v0.04a is used with
%     \pdfTeX\ \textgreater= 1.40.0, then package \xpackage{pdfcolmk} is obsolete.
%   \end{Version}
%   \begin{Version}{2007/04/11 v1.1}
%   \item
%     Line ends sanitized.
%   \end{Version}
%   \begin{Version}{2008/08/11 v1.2}
%   \item
%     Code is not changed.
%   \item
%     URLs updated.
%   \end{Version}
% \end{History}
%
% \PrintIndex
%
% \Finale
\endinput

%        (quote the arguments according to the demands of your shell)
%
% Documentation:
%    (a) If pdfcolmk.drv is present:
%           latex pdfcolmk.drv
%    (b) Without pdfcolmk.drv:
%           latex pdfcolmk.dtx; ...
%    The class ltxdoc loads the configuration file ltxdoc.cfg
%    if available. Here you can specify further options, e.g.
%    use A4 as paper format:
%       \PassOptionsToClass{a4paper}{article}
%
%    Programm calls to get the documentation (example):
%       pdflatex pdfcolmk.dtx
%       makeindex -s gind.ist pdfcolmk.idx
%       pdflatex pdfcolmk.dtx
%       makeindex -s gind.ist pdfcolmk.idx
%       pdflatex pdfcolmk.dtx
%
% Installation:
%    TDS:tex/latex/oberdiek/pdfcolmk.sty
%    TDS:doc/latex/oberdiek/pdfcolmk.pdf
%    TDS:source/latex/oberdiek/pdfcolmk.dtx
%
%<*ignore>
\begingroup
  \catcode123=1 %
  \catcode125=2 %
  \def\x{LaTeX2e}%
\expandafter\endgroup
\ifcase 0\ifx\install y1\fi\expandafter
         \ifx\csname processbatchFile\endcsname\relax\else1\fi
         \ifx\fmtname\x\else 1\fi\relax
\else\csname fi\endcsname
%</ignore>
%<*install>
\input docstrip.tex
\Msg{************************************************************************}
\Msg{* Installation}
\Msg{* Package: pdfcolmk 2008/08/11 v1.2 Color support for pdfTeX via marks (HO)}
\Msg{************************************************************************}

\keepsilent
\askforoverwritefalse

\let\MetaPrefix\relax
\preamble

This is a generated file.

Project: pdfcolmk
Version: 2008/08/11 v1.2

Copyright (C) 2000, 2005-2008 by
   Heiko Oberdiek <heiko.oberdiek at googlemail.com>

This work may be distributed and/or modified under the
conditions of the LaTeX Project Public License, either
version 1.3c of this license or (at your option) any later
version. This version of this license is in
   http://www.latex-project.org/lppl/lppl-1-3c.txt
and the latest version of this license is in
   http://www.latex-project.org/lppl.txt
and version 1.3 or later is part of all distributions of
LaTeX version 2005/12/01 or later.

This work has the LPPL maintenance status "maintained".

This Current Maintainer of this work is Heiko Oberdiek.

This work consists of the main source file pdfcolmk.dtx
and the derived files
   pdfcolmk.sty, pdfcolmk.pdf, pdfcolmk.ins, pdfcolmk.drv.

\endpreamble
\let\MetaPrefix\DoubleperCent

\generate{%
  \file{pdfcolmk.ins}{\from{pdfcolmk.dtx}{install}}%
  \file{pdfcolmk.drv}{\from{pdfcolmk.dtx}{driver}}%
  \usedir{tex/latex/oberdiek}%
  \file{pdfcolmk.sty}{\from{pdfcolmk.dtx}{package}}%
  \nopreamble
  \nopostamble
  \usedir{source/latex/oberdiek/catalogue}%
  \file{pdfcolmk.xml}{\from{pdfcolmk.dtx}{catalogue}}%
}

\catcode32=13\relax% active space
\let =\space%
\Msg{************************************************************************}
\Msg{*}
\Msg{* To finish the installation you have to move the following}
\Msg{* file into a directory searched by TeX:}
\Msg{*}
\Msg{*     pdfcolmk.sty}
\Msg{*}
\Msg{* To produce the documentation run the file `pdfcolmk.drv'}
\Msg{* through LaTeX.}
\Msg{*}
\Msg{* Happy TeXing!}
\Msg{*}
\Msg{************************************************************************}

\endbatchfile
%</install>
%<*ignore>
\fi
%</ignore>
%<*driver>
\NeedsTeXFormat{LaTeX2e}
\ProvidesFile{pdfcolmk.drv}%
  [2008/08/11 v1.2 Color support for pdfTeX via marks (HO)]%
\documentclass{ltxdoc}
\usepackage{holtxdoc}[2011/11/22]
\begin{document}
  \DocInput{pdfcolmk.dtx}%
\end{document}
%</driver>
% \fi
%
% \CheckSum{843}
%
% \CharacterTable
%  {Upper-case    \A\B\C\D\E\F\G\H\I\J\K\L\M\N\O\P\Q\R\S\T\U\V\W\X\Y\Z
%   Lower-case    \a\b\c\d\e\f\g\h\i\j\k\l\m\n\o\p\q\r\s\t\u\v\w\x\y\z
%   Digits        \0\1\2\3\4\5\6\7\8\9
%   Exclamation   \!     Double quote  \"     Hash (number) \#
%   Dollar        \$     Percent       \%     Ampersand     \&
%   Acute accent  \'     Left paren    \(     Right paren   \)
%   Asterisk      \*     Plus          \+     Comma         \,
%   Minus         \-     Point         \.     Solidus       \/
%   Colon         \:     Semicolon     \;     Less than     \<
%   Equals        \=     Greater than  \>     Question mark \?
%   Commercial at \@     Left bracket  \[     Backslash     \\
%   Right bracket \]     Circumflex    \^     Underscore    \_
%   Grave accent  \`     Left brace    \{     Vertical bar  \|
%   Right brace   \}     Tilde         \~}
%
% \GetFileInfo{pdfcolmk.drv}
%
% \title{The \xpackage{pdfcolmk} package}
% \date{2008/08/11 v1.2}
% \author{Heiko Oberdiek\\\xemail{heiko.oberdiek at googlemail.com}}
%
% \maketitle
%
% \begin{abstract}
% This package tries a solution for the missing color
% stack of \pdfTeX.
% \end{abstract}
%
% \tableofcontents
%
% \section{Documentation}
%
% \subsection{Introduction}
%
% This package uses a mark register in order to solve the
% problem of a missing color stack of \pdfTeX\ prior 1.40.0.
% Since this version of \pdfTeX\ a color stack is available
% and supported by \xfile{pdftex.def} 2007/01/01 v0.04a.
% In this case this package is obsolete and the package
% stops its loading.
%
% \subsection{Background}
%
% After the Dante meeting (Clausthal 2000) I have started
% to experiment with the eTeX method of a \emph{colour} mark.
% One of the major problems is the understanding of the
% output routine and the need to rewrite it because of
% missing hooks. Currently I have made some tests in
% in onecolumn and twocolumn mode, but the state is
% experimental.
%
% \subsection{Limitations}
%
% \begin{itemize}
% \item Mark limitations: page breaks in math.
% \item \LaTeX's output routine is redefinded.
%   \begin{itemize}
%   \item Changes in the output routine of newer versions
%         of LaTeX are not detected.
%   \item Packages that change the output routine are not
%         supported.
%   \end{itemize}
% \item It does not support several independent text
%       streams like footnotes.
% \item Limitations in float and marginpar support.
% \end{itemize}
%
% \subsection{Recommendation}
%
% \eTeX\ (for additional mark register)
% Without \eTeX\ \LaTeX's mark commands are redefined
% to store an additional color value.
%
% \subsection{Usage}
%
% Load after package color:
% \begin{quote}
%   |\usepackage[pdftex]{color}|\\
%   |\usepackage{pdfcolmk}|
% \end{quote}
%
% \subsection{Compatibility}
%
% \begin{itemize}
% \item Load the following packages after \xpackage{pdfcolmk}:
%   \begin{quote}
%       \xpackage{mparhack.sty}
%   \end{quote}
% \item Load the following packages before \xpackage{pdfcolmk}:
%   \begin{quote}
%       \xpackage{marn.sty}\\
%       \xpackage{newmarn.sty}
%   \end{quote}
% \item Supported \cs{@addmarginpar} patch:
%   \begin{quote}
%       \xpackage{latex/base/latex.ltx}\\
%       \xpackage{memoir.cls}\\
%       \xpackage{poemscol/marn.sty}, \xpackage{poemscol/newmarn.sty}\\
%       \xpackage{mparhack.sty}
%   \end{quote}
% \item Unsupported \cs{@addmarginpar} patch:
%   \begin{quote}
%       \xpackage{lineno.sty}\\
%       \xpackage{sttools/marginal.sty}\\
%       \xpackage{revtex4.cls}
%   \end{quote}
% \end{itemize}
%
% \StopEventually{
% }
%
% \section{Implementation}
%
%    \begin{macrocode}
%<*package>
%    \end{macrocode}
%    Package identification.
%    \begin{macrocode}
\NeedsTeXFormat{LaTeX2e}
\ProvidesPackage{pdfcolmk}%
  [2008/08/11 v1.2 Color support for pdfTeX via marks (HO)]
%    \end{macrocode}
%
%    \begin{macrocode}
\@ifundefined{ver@pdftex.def}{%
  \PackageWarningNoLine{pdfcolmk}{%
    Nothing to fix, because \string`pdftex.def\string' not loaded%
  }%
  \endinput
}{}
\@ifpackageloaded{color}{}{%
  \PackageWarningNoLine{pdfcolmk}{%
    Nothing to fix, because \string`color.sty\string' not loaded%
  }%
  \endinput
}
\begingroup\expandafter\expandafter\expandafter\endgroup
\expandafter\ifx\csname main@pdfcolorstack\endcsname\relax
\else
  % pdftex.def >= 2007/01/01 0.04a and pdfTeX >= 1.40.0
  \begingroup
    \let\on@line\@empty
    \PackageInfo{pdfcolmk}{%
      The color stack of pdfTeX \string>\string= 1.40 is used. %
      Therefore\MessageBreak
      this package is not necessary and not loaded%
    }%
  \endgroup
  \expandafter\endinput
\fi

\PackageInfo{pdfcolmk}{%
  This package tries to simulate dvips's color stack\MessageBreak
  for pdfTeX based on a mark register of e-TeX.\MessageBreak
  It redefines LaTeX's output routine. Therefore\MessageBreak
  use with care, no warranties%
}

\ifx\marks\@undefined

  \let\pec@mark\mark
  \let\pec@value\empty
  \long\def\mark#1{%
    \protected@xdef\pec@value{#1}%
    \pec@setmark
  }%
  \def\pec@setmark{%
    \begingroup
      \@temptokena\expandafter{\pec@value}%
      \pec@mark{{\current@color}\the\@temptokena}%
    \endgroup
  }%
  \def\pec@getmark{%
    \xdef\pec@botcolor{%
      \expandafter\@firstofthree\botmark\@empty\@empty\@empty
    }%
  }%
  \long\def\@firstofthree#1#2#3{#1}%
  \CheckCommand{\@leftmark}[2]{#1}%
  \CheckCommand{\@rightmark}[2]{#2}%
  \CheckCommand*{\leftmark}{%
    \expandafter\@leftmark\botmark\@empty\@empty
  }%
  \CheckCommand*{\rightmark}{%
    \expandafter\@rightmark\firstmark\@empty\@empty
  }%
  \long\def\@leftmark#1#2#3{#2}%
  \long\def\@rightmark#1#2#3{#3}%
  \g@addto@macro\leftmark\@empty
  \g@addto@macro\rightmark\@empty

\else

  \RequirePackage{etex}[1998/03/26]%
  \newmarks\pec@marks
  \def\pec@setmark{\marks\pec@marks{\current@color}}%
  \def\pec@getmark{\xdef\pec@botcolor{\botmarks\pec@marks}}%

\fi
%    \end{macrocode}
%
% \subsection{\cs{marginpar} fix}
%
%    \begin{macrocode}
\chardef\pec@result\z@
\def\pec@temp#1{%
  \chardef\pec@result\@ne
  \begingroup
    \let\on@line\@empty
    \PackageInfo{pdfcolmk}{%
      Patch for \string\@addmarginpar\space applied (#1)%
    }%
  \endgroup
}
%    \end{macrocode}
%
% \subsubsection{latex/base/latex.ltx}
%
%    \begin{macrocode}
\def\pec@addmarginpar{%
  \@next\@marbox\@currlist{%
    \@cons\@freelist\@marbox
    \@cons\@freelist\@currbox
  }\@latexbug
  \@tempcnta\@ne
  \if@twocolumn
    \if@firstcolumn
      \@tempcnta\m@ne
    \fi
  \else
    \if@mparswitch
      \ifodd\c@page
      \else
        \@tempcnta\m@ne
      \fi
    \fi
    \if@reversemargin \@tempcnta -\@tempcnta \fi
  \fi
  \ifnum\@tempcnta <\z@  \global\setbox\@marbox\box\@currbox \fi
  \@tempdima\@mparbottom
  \advance\@tempdima -\@pageht
  \advance\@tempdima\ht\@marbox
  \ifdim\@tempdima >\z@
    \@latex@warning@no@line{Marginpar on page \thepage\space moved}%
  \else
    \@tempdima\z@
  \fi
  \global\@mparbottom\@pageht
  \global\advance\@mparbottom\@tempdima
  \global\advance\@mparbottom\dp\@marbox
  \global\advance\@mparbottom\marginparpush
  \advance\@tempdima -\ht\@marbox
  \global\setbox\@marbox\vbox{%
    \vskip \@tempdima
    \box \@marbox
  }%
  \global \ht\@marbox \z@
  \global \dp\@marbox \z@
  \kern -\@pagedp
  \nointerlineskip
  \hb@xt@\columnwidth{%
    \ifnum \@tempcnta >\z@
      \hskip\columnwidth
      \hskip\marginparsep
    \else
      \hskip -\marginparsep
      \hskip -\marginparwidth
    \fi
    \box\@marbox \hss
  }%
  \nointerlineskip
  \hbox{\vrule \@height\z@ \@width\z@ \@depth\@pagedp}%
}
\ifx\pec@addmarginpar\@addmarginpar
  \pec@temp{latex/base}%
\fi
%    \end{macrocode}
%
% \subsubsection{memoir.cls}
%
%    \begin{macrocode}
\def\pec@addmarginpar{%
  \checkoddpage
  \@next\@marbox\@currlist{%
    \@cons\@freelist\@marbox
    \@cons\@freelist\@currbox
  }\@latexbug
  \@tempcnta\@ne
  \if@twocolumn
    \if@firstcolumn
      \@tempcnta\m@ne
    \fi
  \else
    \if@mparswitch
      \ifoddpage
      \else
        \@tempcnta\m@ne
      \fi
    \fi
    \if@reversemargin
      \@tempcnta -\@tempcnta
    \fi
  \fi
  \ifnum\@tempcnta <\z@
    \global\setbox\@marbox\box\@currbox
  \fi
  \@tempdima\@mparbottom
  \advance\@tempdima -\@pageht
  \advance\@tempdima\ht\@marbox
  \ifdim\@tempdima >\z@
    \@latex@warning@no@line{%
      Marginpar on page \thepage\space moved by \the\@tempdima
    }%
  \else
    \@tempdima\z@
  \fi
  \global\@mparbottom\@pageht
  \global\advance\@mparbottom\@tempdima
  \global\advance\@mparbottom\dp\@marbox
  \global\advance\@mparbottom\marginparpush
  \advance\@tempdima -\ht\@marbox
  \global\setbox\@marbox\vbox{%
    \vskip \@tempdima
    \box \@marbox
  }%
  \global \ht\@marbox \z@
  \global \dp\@marbox \z@
  \kern -\@pagedp
  \nointerlineskip
  \hb@xt@\columnwidth{%
    \ifnum \@tempcnta >\z@
      \hskip\columnwidth
      \hskip\marginparsep
    \else
      \hskip -\marginparsep
      \hskip -\marginparwidth
    \fi
    \box\@marbox
    \hss
  }%
  \nointerlineskip
  \hbox{\vrule \@height\z@ \@width\z@ \@depth\@pagedp}%
}%
\ifx\pec@addmarginpar\@addmarginpar
  \pec@temp{memoir.cls}%
\fi
%    \end{macrocode}
%
% \subsubsection{poemscol/marn.sty, poemscol/newmarn.sty}
%
%    \begin{macrocode}
\def\pec@addmarginpar{%
  \@next \@marbox\@currlist{%
    \@cons\@freelist\@marbox
    \@cons\@freelist\@currbox
  }\@latexbug
  \global\advance\@mpar@count\m@ne
  \@ifundefined{@marn@\the\@mpar@count @}{% was location logged last time?
    \@tempcnta\@ne % NO: use original LaTeX logic
    \if@twocolumn
      \if@firstcolumn
        \@tempcnta\m@ne
      \fi
    \else
      \if@mparswitch
        \ifodd\c@page
        \else
          \@tempcnta\m@ne
        \fi
      \fi
      \if@reversemargin
        \@tempcnta -\@tempcnta
      \fi
    \fi
  }{%
    \@tempcnta %    YES: use record from last time to decide side.
    \@nameuse{@marn@\the\@mpar@count @}%
    \if@reversemargin -\fi \@ne
  }%
  \ifnum\@tempcnta <\z@
    \global\setbox\@marbox\box\@currbox
    \global\let\@marnbottom\@mparbottoml
  \else
    \global\let\@marnbottom\@mparbottom
  \fi
  \@tempdima\@marnbottom \advance\@tempdima -\@pageht
  \advance\@tempdima\ht\@marbox
  \ifdim\@tempdima >\z@
    \@@warning{Marginpar on page \thepage\space moved}%
  \else
    \@tempdima\z@
  \fi
  \global\@marnbottom\@pageht
  \global\advance\@marnbottom\@tempdima
  \global\advance\@marnbottom\dp\@marbox
  \global\advance\@marnbottom\marginparpush
  \advance\@tempdima -\ht\@marbox
  \global\ht\@marbox\z@
  \global\dp\@marbox\z@
  \vskip -\@pagedp
  \vskip\@tempdima\nointerlineskip
  \hbox to\columnwidth{%
    \ifnum \@tempcnta >\z@
      \hskip\columnwidth
      \hskip\marginparsep
    \else
      \hskip -\marginparsep
      \hskip -\marginparwidth
    \fi
    \if@filesw % record where this is for use next time:
       \@marn@log\@mpar@count
    \fi
    \box\@marbox
    \hss
  }%
  \nobreak   %% RmS 91/06/21 \nobreak added
  \vskip -\@tempdima
  \nointerlineskip
  \hbox{\vrule \@height\z@ \@width\z@ \@depth\@pagedp}%
}
\ifx\pec@addmarginpar\@addmarginpar
  \pec@temp{poemscol/(new)marn.sty}%
\fi
%    \end{macrocode}
%
% \subsubsection{refman/refart.cls, refnam/refrep.cls}
%
%    \begin{macrocode}
\def\pec@addmarginpar{%
  \@next\@marbox\@currlist{%
    \@cons\@freelist\@marbox
    \@cons\@freelist\@currbox
  }\@latexbug
  \@tempcnta\@ne
  \if@twocolumn
    \if@firstcolumn
      \@tempcnta\m@ne
    \fi
  \else
    \@tempcnta\m@ne
  \fi
  \ifnum\@tempcnta <\z@
    \global\setbox\@marbox\box\@currbox
  \fi
  \@tempdima\@mparbottom
  \advance\@tempdima -\@pageht
  \advance\@tempdima\ht\@marbox
  \ifdim\@tempdima >\z@
     \@@warning{Marginpar on page \thepage\space moved}%
  \else
     \@tempdima\z@
  \fi
  \global\@mparbottom\@pageht
  \global\advance\@mparbottom\@tempdima
  \global\advance\@mparbottom\dp\@marbox
  \global\advance\@mparbottom\marginparpush
  \advance\@tempdima -\ht\@marbox
  \global\setbox\@marbox\vbox{%
    \vskip \@tempdima \box \@marbox
  }%
  \global \ht\@marbox \z@
  \global \dp\@marbox \z@
  \kern -\@pagedp
  \nointerlineskip
  \hb@xt@\columnwidth{%
    \ifnum \@tempcnta >\z@
      \hskip\columnwidth
      \hskip\marginparsep
    \else
      \hskip -\marginparsep
      \hskip -\marginparwidth
    \fi
    \box\@marbox
    \hss
  }%
  \nointerlineskip
  \hbox{\vrule \@height\z@ \@width\z@ \@depth\@pagedp}%
}
\ifx\pec@addmarginpar\@addmarginpar
  \pec@temp{ref(art|rep).cls}%
\fi

\ifcase\pec@result
  \PackageInfo{pdfcolmk}{%
    Fix for \string\@addmarginpar\space is omitted, %
    because this variant\MessageBreak
    of \string\@addmarginpar\space
      is not recognized%
  }%
\else
  % apply patch for \@addmarginpar
  \def\pec@PatchAddMarginpar#1\columnwidth#2#3\@nil{%
    \pec@PatchAddMarginparI#2\@nil{#1}{#3}%
  }%
  \def\pec@PatchAddMarginparI#1\box\@marbox\hss#2\@nil#3#4{%
    \def\@addmarginpar{%
      #3%
      \columnwidth{%
        #1%
        \pdfliteral{q}%
        \rlap{%
          \box\@marbox
        }%
        \pdfliteral{Q}%
        \hss
        #2%
      }%
      #4%
    }%
  }%
  \expandafter\pec@PatchAddMarginpar\@addmarginpar\@nil
\fi
%    \end{macrocode}
%
% \subsection{Color fix}
%
%    \begin{macrocode}
\def\set@color{%
  \pdfliteral{\current@color}%
  \ifinner
  \else
    \pec@setmark
  \fi
  \aftergroup\reset@color
}
\def\reset@color{%
  \pdfliteral{\current@color}%
  \ifinner
  \else
    \pec@setmark
  \fi
}

\let\pec@botcolor\current@color

\def\pec@PatchVBoxCCLV{%
  \ifx\pec@botcolor\@empty
  \else
    \setbox\@cclv\vbox{%
      \pdfliteral{\pec@botcolor}%
      \unvbox\@cclv
    }%
  \fi
  \pec@getmark
}

\def\pec@PatchAlreadyInBox{%
  \ifx\pec@botcolor\@empty
  \else
    \pdfliteral{\pec@botcolor}%
  \fi
  \pec@getmark
}

\@ifclassloaded{memoir}{%
  \expandafter\def\expandafter\mem@makecol\expandafter{%
    \expandafter\pec@PatchVBoxCCLV
    \mem@makecol
  }%
  \endinput
}{}

\@ifclassloaded{seminar}{%
  \newcommand\pec@org@makeslide{}%
  \let\pec@org@makeslide\@makeslide
  \def\@makeslide{%
    \pec@PatchVBoxCCLV
    \pec@org@makeslide
  }%
  \endinput
}{}

\long\def\pec@output#1\@specialoutput\else#2\pec@end{%
  \begingroup
    \def\x{#2}%
  \expandafter\endgroup
  \ifx\x\@empty
    \PackageWarningNoLine{pdfcolmk}{%
      Unexpected \string\output\space routine detected,%
      \MessageBreak
      loading of package stopped%
    }%
    \expandafter\endinput
  \fi
}
\expandafter\expandafter\expandafter\pec@output
\expandafter\@firstofone\the\output\@specialoutput\else\pec@end

\long\def\pec@output#1\@specialoutput\else#2\pec@end{%
  \output{%
    #1\@specialoutput\else
    \pec@PatchVBoxCCLV
    #2%
  }%
}
\expandafter\expandafter\expandafter\pec@output
\expandafter\@firstofone\the\output\pec@end
%    \end{macrocode}
%
%    \begin{macrocode}
%</package>
%    \end{macrocode}
%
% \section{Installation}
%
% \subsection{Download}
%
% \paragraph{Package.} This package is available on
% CTAN\footnote{\url{ftp://ftp.ctan.org/tex-archive/}}:
% \begin{description}
% \item[\CTAN{macros/latex/contrib/oberdiek/pdfcolmk.dtx}] The source file.
% \item[\CTAN{macros/latex/contrib/oberdiek/pdfcolmk.pdf}] Documentation.
% \end{description}
%
%
% \paragraph{Bundle.} All the packages of the bundle `oberdiek'
% are also available in a TDS compliant ZIP archive. There
% the packages are already unpacked and the documentation files
% are generated. The files and directories obey the TDS standard.
% \begin{description}
% \item[\CTAN{install/macros/latex/contrib/oberdiek.tds.zip}]
% \end{description}
% \emph{TDS} refers to the standard ``A Directory Structure
% for \TeX\ Files'' (\CTAN{tds/tds.pdf}). Directories
% with \xfile{texmf} in their name are usually organized this way.
%
% \subsection{Bundle installation}
%
% \paragraph{Unpacking.} Unpack the \xfile{oberdiek.tds.zip} in the
% TDS tree (also known as \xfile{texmf} tree) of your choice.
% Example (linux):
% \begin{quote}
%   |unzip oberdiek.tds.zip -d ~/texmf|
% \end{quote}
%
% \paragraph{Script installation.}
% Check the directory \xfile{TDS:scripts/oberdiek/} for
% scripts that need further installation steps.
% Package \xpackage{attachfile2} comes with the Perl script
% \xfile{pdfatfi.pl} that should be installed in such a way
% that it can be called as \texttt{pdfatfi}.
% Example (linux):
% \begin{quote}
%   |chmod +x scripts/oberdiek/pdfatfi.pl|\\
%   |cp scripts/oberdiek/pdfatfi.pl /usr/local/bin/|
% \end{quote}
%
% \subsection{Package installation}
%
% \paragraph{Unpacking.} The \xfile{.dtx} file is a self-extracting
% \docstrip\ archive. The files are extracted by running the
% \xfile{.dtx} through \plainTeX:
% \begin{quote}
%   \verb|tex pdfcolmk.dtx|
% \end{quote}
%
% \paragraph{TDS.} Now the different files must be moved into
% the different directories in your installation TDS tree
% (also known as \xfile{texmf} tree):
% \begin{quote}
% \def\t{^^A
% \begin{tabular}{@{}>{\ttfamily}l@{ $\rightarrow$ }>{\ttfamily}l@{}}
%   pdfcolmk.sty & tex/latex/oberdiek/pdfcolmk.sty\\
%   pdfcolmk.pdf & doc/latex/oberdiek/pdfcolmk.pdf\\
%   pdfcolmk.dtx & source/latex/oberdiek/pdfcolmk.dtx\\
% \end{tabular}^^A
% }^^A
% \sbox0{\t}^^A
% \ifdim\wd0>\linewidth
%   \begingroup
%     \advance\linewidth by\leftmargin
%     \advance\linewidth by\rightmargin
%   \edef\x{\endgroup
%     \def\noexpand\lw{\the\linewidth}^^A
%   }\x
%   \def\lwbox{^^A
%     \leavevmode
%     \hbox to \linewidth{^^A
%       \kern-\leftmargin\relax
%       \hss
%       \usebox0
%       \hss
%       \kern-\rightmargin\relax
%     }^^A
%   }^^A
%   \ifdim\wd0>\lw
%     \sbox0{\small\t}^^A
%     \ifdim\wd0>\linewidth
%       \ifdim\wd0>\lw
%         \sbox0{\footnotesize\t}^^A
%         \ifdim\wd0>\linewidth
%           \ifdim\wd0>\lw
%             \sbox0{\scriptsize\t}^^A
%             \ifdim\wd0>\linewidth
%               \ifdim\wd0>\lw
%                 \sbox0{\tiny\t}^^A
%                 \ifdim\wd0>\linewidth
%                   \lwbox
%                 \else
%                   \usebox0
%                 \fi
%               \else
%                 \lwbox
%               \fi
%             \else
%               \usebox0
%             \fi
%           \else
%             \lwbox
%           \fi
%         \else
%           \usebox0
%         \fi
%       \else
%         \lwbox
%       \fi
%     \else
%       \usebox0
%     \fi
%   \else
%     \lwbox
%   \fi
% \else
%   \usebox0
% \fi
% \end{quote}
% If you have a \xfile{docstrip.cfg} that configures and enables \docstrip's
% TDS installing feature, then some files can already be in the right
% place, see the documentation of \docstrip.
%
% \subsection{Refresh file name databases}
%
% If your \TeX~distribution
% (\teTeX, \mikTeX, \dots) relies on file name databases, you must refresh
% these. For example, \teTeX\ users run \verb|texhash| or
% \verb|mktexlsr|.
%
% \subsection{Some details for the interested}
%
% \paragraph{Attached source.}
%
% The PDF documentation on CTAN also includes the
% \xfile{.dtx} source file. It can be extracted by
% AcrobatReader 6 or higher. Another option is \textsf{pdftk},
% e.g. unpack the file into the current directory:
% \begin{quote}
%   \verb|pdftk pdfcolmk.pdf unpack_files output .|
% \end{quote}
%
% \paragraph{Unpacking with \LaTeX.}
% The \xfile{.dtx} chooses its action depending on the format:
% \begin{description}
% \item[\plainTeX:] Run \docstrip\ and extract the files.
% \item[\LaTeX:] Generate the documentation.
% \end{description}
% If you insist on using \LaTeX\ for \docstrip\ (really,
% \docstrip\ does not need \LaTeX), then inform the autodetect routine
% about your intention:
% \begin{quote}
%   \verb|latex \let\install=y% \iffalse meta-comment
%
% File: pdfcolmk.dtx
% Version: 2008/08/11 v1.2
% Info: Color support for pdfTeX via marks
%
% Copyright (C) 2000, 2005-2008 by
%    Heiko Oberdiek <heiko.oberdiek at googlemail.com>
%
% This work may be distributed and/or modified under the
% conditions of the LaTeX Project Public License, either
% version 1.3c of this license or (at your option) any later
% version. This version of this license is in
%    http://www.latex-project.org/lppl/lppl-1-3c.txt
% and the latest version of this license is in
%    http://www.latex-project.org/lppl.txt
% and version 1.3 or later is part of all distributions of
% LaTeX version 2005/12/01 or later.
%
% This work has the LPPL maintenance status "maintained".
%
% This Current Maintainer of this work is Heiko Oberdiek.
%
% This work consists of the main source file pdfcolmk.dtx
% and the derived files
%    pdfcolmk.sty, pdfcolmk.pdf, pdfcolmk.ins, pdfcolmk.drv.
%
% Distribution:
%    CTAN:macros/latex/contrib/oberdiek/pdfcolmk.dtx
%    CTAN:macros/latex/contrib/oberdiek/pdfcolmk.pdf
%
% Unpacking:
%    (a) If pdfcolmk.ins is present:
%           tex pdfcolmk.ins
%    (b) Without pdfcolmk.ins:
%           tex pdfcolmk.dtx
%    (c) If you insist on using LaTeX
%           latex \let\install=y\input{pdfcolmk.dtx}
%        (quote the arguments according to the demands of your shell)
%
% Documentation:
%    (a) If pdfcolmk.drv is present:
%           latex pdfcolmk.drv
%    (b) Without pdfcolmk.drv:
%           latex pdfcolmk.dtx; ...
%    The class ltxdoc loads the configuration file ltxdoc.cfg
%    if available. Here you can specify further options, e.g.
%    use A4 as paper format:
%       \PassOptionsToClass{a4paper}{article}
%
%    Programm calls to get the documentation (example):
%       pdflatex pdfcolmk.dtx
%       makeindex -s gind.ist pdfcolmk.idx
%       pdflatex pdfcolmk.dtx
%       makeindex -s gind.ist pdfcolmk.idx
%       pdflatex pdfcolmk.dtx
%
% Installation:
%    TDS:tex/latex/oberdiek/pdfcolmk.sty
%    TDS:doc/latex/oberdiek/pdfcolmk.pdf
%    TDS:source/latex/oberdiek/pdfcolmk.dtx
%
%<*ignore>
\begingroup
  \catcode123=1 %
  \catcode125=2 %
  \def\x{LaTeX2e}%
\expandafter\endgroup
\ifcase 0\ifx\install y1\fi\expandafter
         \ifx\csname processbatchFile\endcsname\relax\else1\fi
         \ifx\fmtname\x\else 1\fi\relax
\else\csname fi\endcsname
%</ignore>
%<*install>
\input docstrip.tex
\Msg{************************************************************************}
\Msg{* Installation}
\Msg{* Package: pdfcolmk 2008/08/11 v1.2 Color support for pdfTeX via marks (HO)}
\Msg{************************************************************************}

\keepsilent
\askforoverwritefalse

\let\MetaPrefix\relax
\preamble

This is a generated file.

Project: pdfcolmk
Version: 2008/08/11 v1.2

Copyright (C) 2000, 2005-2008 by
   Heiko Oberdiek <heiko.oberdiek at googlemail.com>

This work may be distributed and/or modified under the
conditions of the LaTeX Project Public License, either
version 1.3c of this license or (at your option) any later
version. This version of this license is in
   http://www.latex-project.org/lppl/lppl-1-3c.txt
and the latest version of this license is in
   http://www.latex-project.org/lppl.txt
and version 1.3 or later is part of all distributions of
LaTeX version 2005/12/01 or later.

This work has the LPPL maintenance status "maintained".

This Current Maintainer of this work is Heiko Oberdiek.

This work consists of the main source file pdfcolmk.dtx
and the derived files
   pdfcolmk.sty, pdfcolmk.pdf, pdfcolmk.ins, pdfcolmk.drv.

\endpreamble
\let\MetaPrefix\DoubleperCent

\generate{%
  \file{pdfcolmk.ins}{\from{pdfcolmk.dtx}{install}}%
  \file{pdfcolmk.drv}{\from{pdfcolmk.dtx}{driver}}%
  \usedir{tex/latex/oberdiek}%
  \file{pdfcolmk.sty}{\from{pdfcolmk.dtx}{package}}%
  \nopreamble
  \nopostamble
  \usedir{source/latex/oberdiek/catalogue}%
  \file{pdfcolmk.xml}{\from{pdfcolmk.dtx}{catalogue}}%
}

\catcode32=13\relax% active space
\let =\space%
\Msg{************************************************************************}
\Msg{*}
\Msg{* To finish the installation you have to move the following}
\Msg{* file into a directory searched by TeX:}
\Msg{*}
\Msg{*     pdfcolmk.sty}
\Msg{*}
\Msg{* To produce the documentation run the file `pdfcolmk.drv'}
\Msg{* through LaTeX.}
\Msg{*}
\Msg{* Happy TeXing!}
\Msg{*}
\Msg{************************************************************************}

\endbatchfile
%</install>
%<*ignore>
\fi
%</ignore>
%<*driver>
\NeedsTeXFormat{LaTeX2e}
\ProvidesFile{pdfcolmk.drv}%
  [2008/08/11 v1.2 Color support for pdfTeX via marks (HO)]%
\documentclass{ltxdoc}
\usepackage{holtxdoc}[2011/11/22]
\begin{document}
  \DocInput{pdfcolmk.dtx}%
\end{document}
%</driver>
% \fi
%
% \CheckSum{843}
%
% \CharacterTable
%  {Upper-case    \A\B\C\D\E\F\G\H\I\J\K\L\M\N\O\P\Q\R\S\T\U\V\W\X\Y\Z
%   Lower-case    \a\b\c\d\e\f\g\h\i\j\k\l\m\n\o\p\q\r\s\t\u\v\w\x\y\z
%   Digits        \0\1\2\3\4\5\6\7\8\9
%   Exclamation   \!     Double quote  \"     Hash (number) \#
%   Dollar        \$     Percent       \%     Ampersand     \&
%   Acute accent  \'     Left paren    \(     Right paren   \)
%   Asterisk      \*     Plus          \+     Comma         \,
%   Minus         \-     Point         \.     Solidus       \/
%   Colon         \:     Semicolon     \;     Less than     \<
%   Equals        \=     Greater than  \>     Question mark \?
%   Commercial at \@     Left bracket  \[     Backslash     \\
%   Right bracket \]     Circumflex    \^     Underscore    \_
%   Grave accent  \`     Left brace    \{     Vertical bar  \|
%   Right brace   \}     Tilde         \~}
%
% \GetFileInfo{pdfcolmk.drv}
%
% \title{The \xpackage{pdfcolmk} package}
% \date{2008/08/11 v1.2}
% \author{Heiko Oberdiek\\\xemail{heiko.oberdiek at googlemail.com}}
%
% \maketitle
%
% \begin{abstract}
% This package tries a solution for the missing color
% stack of \pdfTeX.
% \end{abstract}
%
% \tableofcontents
%
% \section{Documentation}
%
% \subsection{Introduction}
%
% This package uses a mark register in order to solve the
% problem of a missing color stack of \pdfTeX\ prior 1.40.0.
% Since this version of \pdfTeX\ a color stack is available
% and supported by \xfile{pdftex.def} 2007/01/01 v0.04a.
% In this case this package is obsolete and the package
% stops its loading.
%
% \subsection{Background}
%
% After the Dante meeting (Clausthal 2000) I have started
% to experiment with the eTeX method of a \emph{colour} mark.
% One of the major problems is the understanding of the
% output routine and the need to rewrite it because of
% missing hooks. Currently I have made some tests in
% in onecolumn and twocolumn mode, but the state is
% experimental.
%
% \subsection{Limitations}
%
% \begin{itemize}
% \item Mark limitations: page breaks in math.
% \item \LaTeX's output routine is redefinded.
%   \begin{itemize}
%   \item Changes in the output routine of newer versions
%         of LaTeX are not detected.
%   \item Packages that change the output routine are not
%         supported.
%   \end{itemize}
% \item It does not support several independent text
%       streams like footnotes.
% \item Limitations in float and marginpar support.
% \end{itemize}
%
% \subsection{Recommendation}
%
% \eTeX\ (for additional mark register)
% Without \eTeX\ \LaTeX's mark commands are redefined
% to store an additional color value.
%
% \subsection{Usage}
%
% Load after package color:
% \begin{quote}
%   |\usepackage[pdftex]{color}|\\
%   |\usepackage{pdfcolmk}|
% \end{quote}
%
% \subsection{Compatibility}
%
% \begin{itemize}
% \item Load the following packages after \xpackage{pdfcolmk}:
%   \begin{quote}
%       \xpackage{mparhack.sty}
%   \end{quote}
% \item Load the following packages before \xpackage{pdfcolmk}:
%   \begin{quote}
%       \xpackage{marn.sty}\\
%       \xpackage{newmarn.sty}
%   \end{quote}
% \item Supported \cs{@addmarginpar} patch:
%   \begin{quote}
%       \xpackage{latex/base/latex.ltx}\\
%       \xpackage{memoir.cls}\\
%       \xpackage{poemscol/marn.sty}, \xpackage{poemscol/newmarn.sty}\\
%       \xpackage{mparhack.sty}
%   \end{quote}
% \item Unsupported \cs{@addmarginpar} patch:
%   \begin{quote}
%       \xpackage{lineno.sty}\\
%       \xpackage{sttools/marginal.sty}\\
%       \xpackage{revtex4.cls}
%   \end{quote}
% \end{itemize}
%
% \StopEventually{
% }
%
% \section{Implementation}
%
%    \begin{macrocode}
%<*package>
%    \end{macrocode}
%    Package identification.
%    \begin{macrocode}
\NeedsTeXFormat{LaTeX2e}
\ProvidesPackage{pdfcolmk}%
  [2008/08/11 v1.2 Color support for pdfTeX via marks (HO)]
%    \end{macrocode}
%
%    \begin{macrocode}
\@ifundefined{ver@pdftex.def}{%
  \PackageWarningNoLine{pdfcolmk}{%
    Nothing to fix, because \string`pdftex.def\string' not loaded%
  }%
  \endinput
}{}
\@ifpackageloaded{color}{}{%
  \PackageWarningNoLine{pdfcolmk}{%
    Nothing to fix, because \string`color.sty\string' not loaded%
  }%
  \endinput
}
\begingroup\expandafter\expandafter\expandafter\endgroup
\expandafter\ifx\csname main@pdfcolorstack\endcsname\relax
\else
  % pdftex.def >= 2007/01/01 0.04a and pdfTeX >= 1.40.0
  \begingroup
    \let\on@line\@empty
    \PackageInfo{pdfcolmk}{%
      The color stack of pdfTeX \string>\string= 1.40 is used. %
      Therefore\MessageBreak
      this package is not necessary and not loaded%
    }%
  \endgroup
  \expandafter\endinput
\fi

\PackageInfo{pdfcolmk}{%
  This package tries to simulate dvips's color stack\MessageBreak
  for pdfTeX based on a mark register of e-TeX.\MessageBreak
  It redefines LaTeX's output routine. Therefore\MessageBreak
  use with care, no warranties%
}

\ifx\marks\@undefined

  \let\pec@mark\mark
  \let\pec@value\empty
  \long\def\mark#1{%
    \protected@xdef\pec@value{#1}%
    \pec@setmark
  }%
  \def\pec@setmark{%
    \begingroup
      \@temptokena\expandafter{\pec@value}%
      \pec@mark{{\current@color}\the\@temptokena}%
    \endgroup
  }%
  \def\pec@getmark{%
    \xdef\pec@botcolor{%
      \expandafter\@firstofthree\botmark\@empty\@empty\@empty
    }%
  }%
  \long\def\@firstofthree#1#2#3{#1}%
  \CheckCommand{\@leftmark}[2]{#1}%
  \CheckCommand{\@rightmark}[2]{#2}%
  \CheckCommand*{\leftmark}{%
    \expandafter\@leftmark\botmark\@empty\@empty
  }%
  \CheckCommand*{\rightmark}{%
    \expandafter\@rightmark\firstmark\@empty\@empty
  }%
  \long\def\@leftmark#1#2#3{#2}%
  \long\def\@rightmark#1#2#3{#3}%
  \g@addto@macro\leftmark\@empty
  \g@addto@macro\rightmark\@empty

\else

  \RequirePackage{etex}[1998/03/26]%
  \newmarks\pec@marks
  \def\pec@setmark{\marks\pec@marks{\current@color}}%
  \def\pec@getmark{\xdef\pec@botcolor{\botmarks\pec@marks}}%

\fi
%    \end{macrocode}
%
% \subsection{\cs{marginpar} fix}
%
%    \begin{macrocode}
\chardef\pec@result\z@
\def\pec@temp#1{%
  \chardef\pec@result\@ne
  \begingroup
    \let\on@line\@empty
    \PackageInfo{pdfcolmk}{%
      Patch for \string\@addmarginpar\space applied (#1)%
    }%
  \endgroup
}
%    \end{macrocode}
%
% \subsubsection{latex/base/latex.ltx}
%
%    \begin{macrocode}
\def\pec@addmarginpar{%
  \@next\@marbox\@currlist{%
    \@cons\@freelist\@marbox
    \@cons\@freelist\@currbox
  }\@latexbug
  \@tempcnta\@ne
  \if@twocolumn
    \if@firstcolumn
      \@tempcnta\m@ne
    \fi
  \else
    \if@mparswitch
      \ifodd\c@page
      \else
        \@tempcnta\m@ne
      \fi
    \fi
    \if@reversemargin \@tempcnta -\@tempcnta \fi
  \fi
  \ifnum\@tempcnta <\z@  \global\setbox\@marbox\box\@currbox \fi
  \@tempdima\@mparbottom
  \advance\@tempdima -\@pageht
  \advance\@tempdima\ht\@marbox
  \ifdim\@tempdima >\z@
    \@latex@warning@no@line{Marginpar on page \thepage\space moved}%
  \else
    \@tempdima\z@
  \fi
  \global\@mparbottom\@pageht
  \global\advance\@mparbottom\@tempdima
  \global\advance\@mparbottom\dp\@marbox
  \global\advance\@mparbottom\marginparpush
  \advance\@tempdima -\ht\@marbox
  \global\setbox\@marbox\vbox{%
    \vskip \@tempdima
    \box \@marbox
  }%
  \global \ht\@marbox \z@
  \global \dp\@marbox \z@
  \kern -\@pagedp
  \nointerlineskip
  \hb@xt@\columnwidth{%
    \ifnum \@tempcnta >\z@
      \hskip\columnwidth
      \hskip\marginparsep
    \else
      \hskip -\marginparsep
      \hskip -\marginparwidth
    \fi
    \box\@marbox \hss
  }%
  \nointerlineskip
  \hbox{\vrule \@height\z@ \@width\z@ \@depth\@pagedp}%
}
\ifx\pec@addmarginpar\@addmarginpar
  \pec@temp{latex/base}%
\fi
%    \end{macrocode}
%
% \subsubsection{memoir.cls}
%
%    \begin{macrocode}
\def\pec@addmarginpar{%
  \checkoddpage
  \@next\@marbox\@currlist{%
    \@cons\@freelist\@marbox
    \@cons\@freelist\@currbox
  }\@latexbug
  \@tempcnta\@ne
  \if@twocolumn
    \if@firstcolumn
      \@tempcnta\m@ne
    \fi
  \else
    \if@mparswitch
      \ifoddpage
      \else
        \@tempcnta\m@ne
      \fi
    \fi
    \if@reversemargin
      \@tempcnta -\@tempcnta
    \fi
  \fi
  \ifnum\@tempcnta <\z@
    \global\setbox\@marbox\box\@currbox
  \fi
  \@tempdima\@mparbottom
  \advance\@tempdima -\@pageht
  \advance\@tempdima\ht\@marbox
  \ifdim\@tempdima >\z@
    \@latex@warning@no@line{%
      Marginpar on page \thepage\space moved by \the\@tempdima
    }%
  \else
    \@tempdima\z@
  \fi
  \global\@mparbottom\@pageht
  \global\advance\@mparbottom\@tempdima
  \global\advance\@mparbottom\dp\@marbox
  \global\advance\@mparbottom\marginparpush
  \advance\@tempdima -\ht\@marbox
  \global\setbox\@marbox\vbox{%
    \vskip \@tempdima
    \box \@marbox
  }%
  \global \ht\@marbox \z@
  \global \dp\@marbox \z@
  \kern -\@pagedp
  \nointerlineskip
  \hb@xt@\columnwidth{%
    \ifnum \@tempcnta >\z@
      \hskip\columnwidth
      \hskip\marginparsep
    \else
      \hskip -\marginparsep
      \hskip -\marginparwidth
    \fi
    \box\@marbox
    \hss
  }%
  \nointerlineskip
  \hbox{\vrule \@height\z@ \@width\z@ \@depth\@pagedp}%
}%
\ifx\pec@addmarginpar\@addmarginpar
  \pec@temp{memoir.cls}%
\fi
%    \end{macrocode}
%
% \subsubsection{poemscol/marn.sty, poemscol/newmarn.sty}
%
%    \begin{macrocode}
\def\pec@addmarginpar{%
  \@next \@marbox\@currlist{%
    \@cons\@freelist\@marbox
    \@cons\@freelist\@currbox
  }\@latexbug
  \global\advance\@mpar@count\m@ne
  \@ifundefined{@marn@\the\@mpar@count @}{% was location logged last time?
    \@tempcnta\@ne % NO: use original LaTeX logic
    \if@twocolumn
      \if@firstcolumn
        \@tempcnta\m@ne
      \fi
    \else
      \if@mparswitch
        \ifodd\c@page
        \else
          \@tempcnta\m@ne
        \fi
      \fi
      \if@reversemargin
        \@tempcnta -\@tempcnta
      \fi
    \fi
  }{%
    \@tempcnta %    YES: use record from last time to decide side.
    \@nameuse{@marn@\the\@mpar@count @}%
    \if@reversemargin -\fi \@ne
  }%
  \ifnum\@tempcnta <\z@
    \global\setbox\@marbox\box\@currbox
    \global\let\@marnbottom\@mparbottoml
  \else
    \global\let\@marnbottom\@mparbottom
  \fi
  \@tempdima\@marnbottom \advance\@tempdima -\@pageht
  \advance\@tempdima\ht\@marbox
  \ifdim\@tempdima >\z@
    \@@warning{Marginpar on page \thepage\space moved}%
  \else
    \@tempdima\z@
  \fi
  \global\@marnbottom\@pageht
  \global\advance\@marnbottom\@tempdima
  \global\advance\@marnbottom\dp\@marbox
  \global\advance\@marnbottom\marginparpush
  \advance\@tempdima -\ht\@marbox
  \global\ht\@marbox\z@
  \global\dp\@marbox\z@
  \vskip -\@pagedp
  \vskip\@tempdima\nointerlineskip
  \hbox to\columnwidth{%
    \ifnum \@tempcnta >\z@
      \hskip\columnwidth
      \hskip\marginparsep
    \else
      \hskip -\marginparsep
      \hskip -\marginparwidth
    \fi
    \if@filesw % record where this is for use next time:
       \@marn@log\@mpar@count
    \fi
    \box\@marbox
    \hss
  }%
  \nobreak   %% RmS 91/06/21 \nobreak added
  \vskip -\@tempdima
  \nointerlineskip
  \hbox{\vrule \@height\z@ \@width\z@ \@depth\@pagedp}%
}
\ifx\pec@addmarginpar\@addmarginpar
  \pec@temp{poemscol/(new)marn.sty}%
\fi
%    \end{macrocode}
%
% \subsubsection{refman/refart.cls, refnam/refrep.cls}
%
%    \begin{macrocode}
\def\pec@addmarginpar{%
  \@next\@marbox\@currlist{%
    \@cons\@freelist\@marbox
    \@cons\@freelist\@currbox
  }\@latexbug
  \@tempcnta\@ne
  \if@twocolumn
    \if@firstcolumn
      \@tempcnta\m@ne
    \fi
  \else
    \@tempcnta\m@ne
  \fi
  \ifnum\@tempcnta <\z@
    \global\setbox\@marbox\box\@currbox
  \fi
  \@tempdima\@mparbottom
  \advance\@tempdima -\@pageht
  \advance\@tempdima\ht\@marbox
  \ifdim\@tempdima >\z@
     \@@warning{Marginpar on page \thepage\space moved}%
  \else
     \@tempdima\z@
  \fi
  \global\@mparbottom\@pageht
  \global\advance\@mparbottom\@tempdima
  \global\advance\@mparbottom\dp\@marbox
  \global\advance\@mparbottom\marginparpush
  \advance\@tempdima -\ht\@marbox
  \global\setbox\@marbox\vbox{%
    \vskip \@tempdima \box \@marbox
  }%
  \global \ht\@marbox \z@
  \global \dp\@marbox \z@
  \kern -\@pagedp
  \nointerlineskip
  \hb@xt@\columnwidth{%
    \ifnum \@tempcnta >\z@
      \hskip\columnwidth
      \hskip\marginparsep
    \else
      \hskip -\marginparsep
      \hskip -\marginparwidth
    \fi
    \box\@marbox
    \hss
  }%
  \nointerlineskip
  \hbox{\vrule \@height\z@ \@width\z@ \@depth\@pagedp}%
}
\ifx\pec@addmarginpar\@addmarginpar
  \pec@temp{ref(art|rep).cls}%
\fi

\ifcase\pec@result
  \PackageInfo{pdfcolmk}{%
    Fix for \string\@addmarginpar\space is omitted, %
    because this variant\MessageBreak
    of \string\@addmarginpar\space
      is not recognized%
  }%
\else
  % apply patch for \@addmarginpar
  \def\pec@PatchAddMarginpar#1\columnwidth#2#3\@nil{%
    \pec@PatchAddMarginparI#2\@nil{#1}{#3}%
  }%
  \def\pec@PatchAddMarginparI#1\box\@marbox\hss#2\@nil#3#4{%
    \def\@addmarginpar{%
      #3%
      \columnwidth{%
        #1%
        \pdfliteral{q}%
        \rlap{%
          \box\@marbox
        }%
        \pdfliteral{Q}%
        \hss
        #2%
      }%
      #4%
    }%
  }%
  \expandafter\pec@PatchAddMarginpar\@addmarginpar\@nil
\fi
%    \end{macrocode}
%
% \subsection{Color fix}
%
%    \begin{macrocode}
\def\set@color{%
  \pdfliteral{\current@color}%
  \ifinner
  \else
    \pec@setmark
  \fi
  \aftergroup\reset@color
}
\def\reset@color{%
  \pdfliteral{\current@color}%
  \ifinner
  \else
    \pec@setmark
  \fi
}

\let\pec@botcolor\current@color

\def\pec@PatchVBoxCCLV{%
  \ifx\pec@botcolor\@empty
  \else
    \setbox\@cclv\vbox{%
      \pdfliteral{\pec@botcolor}%
      \unvbox\@cclv
    }%
  \fi
  \pec@getmark
}

\def\pec@PatchAlreadyInBox{%
  \ifx\pec@botcolor\@empty
  \else
    \pdfliteral{\pec@botcolor}%
  \fi
  \pec@getmark
}

\@ifclassloaded{memoir}{%
  \expandafter\def\expandafter\mem@makecol\expandafter{%
    \expandafter\pec@PatchVBoxCCLV
    \mem@makecol
  }%
  \endinput
}{}

\@ifclassloaded{seminar}{%
  \newcommand\pec@org@makeslide{}%
  \let\pec@org@makeslide\@makeslide
  \def\@makeslide{%
    \pec@PatchVBoxCCLV
    \pec@org@makeslide
  }%
  \endinput
}{}

\long\def\pec@output#1\@specialoutput\else#2\pec@end{%
  \begingroup
    \def\x{#2}%
  \expandafter\endgroup
  \ifx\x\@empty
    \PackageWarningNoLine{pdfcolmk}{%
      Unexpected \string\output\space routine detected,%
      \MessageBreak
      loading of package stopped%
    }%
    \expandafter\endinput
  \fi
}
\expandafter\expandafter\expandafter\pec@output
\expandafter\@firstofone\the\output\@specialoutput\else\pec@end

\long\def\pec@output#1\@specialoutput\else#2\pec@end{%
  \output{%
    #1\@specialoutput\else
    \pec@PatchVBoxCCLV
    #2%
  }%
}
\expandafter\expandafter\expandafter\pec@output
\expandafter\@firstofone\the\output\pec@end
%    \end{macrocode}
%
%    \begin{macrocode}
%</package>
%    \end{macrocode}
%
% \section{Installation}
%
% \subsection{Download}
%
% \paragraph{Package.} This package is available on
% CTAN\footnote{\url{ftp://ftp.ctan.org/tex-archive/}}:
% \begin{description}
% \item[\CTAN{macros/latex/contrib/oberdiek/pdfcolmk.dtx}] The source file.
% \item[\CTAN{macros/latex/contrib/oberdiek/pdfcolmk.pdf}] Documentation.
% \end{description}
%
%
% \paragraph{Bundle.} All the packages of the bundle `oberdiek'
% are also available in a TDS compliant ZIP archive. There
% the packages are already unpacked and the documentation files
% are generated. The files and directories obey the TDS standard.
% \begin{description}
% \item[\CTAN{install/macros/latex/contrib/oberdiek.tds.zip}]
% \end{description}
% \emph{TDS} refers to the standard ``A Directory Structure
% for \TeX\ Files'' (\CTAN{tds/tds.pdf}). Directories
% with \xfile{texmf} in their name are usually organized this way.
%
% \subsection{Bundle installation}
%
% \paragraph{Unpacking.} Unpack the \xfile{oberdiek.tds.zip} in the
% TDS tree (also known as \xfile{texmf} tree) of your choice.
% Example (linux):
% \begin{quote}
%   |unzip oberdiek.tds.zip -d ~/texmf|
% \end{quote}
%
% \paragraph{Script installation.}
% Check the directory \xfile{TDS:scripts/oberdiek/} for
% scripts that need further installation steps.
% Package \xpackage{attachfile2} comes with the Perl script
% \xfile{pdfatfi.pl} that should be installed in such a way
% that it can be called as \texttt{pdfatfi}.
% Example (linux):
% \begin{quote}
%   |chmod +x scripts/oberdiek/pdfatfi.pl|\\
%   |cp scripts/oberdiek/pdfatfi.pl /usr/local/bin/|
% \end{quote}
%
% \subsection{Package installation}
%
% \paragraph{Unpacking.} The \xfile{.dtx} file is a self-extracting
% \docstrip\ archive. The files are extracted by running the
% \xfile{.dtx} through \plainTeX:
% \begin{quote}
%   \verb|tex pdfcolmk.dtx|
% \end{quote}
%
% \paragraph{TDS.} Now the different files must be moved into
% the different directories in your installation TDS tree
% (also known as \xfile{texmf} tree):
% \begin{quote}
% \def\t{^^A
% \begin{tabular}{@{}>{\ttfamily}l@{ $\rightarrow$ }>{\ttfamily}l@{}}
%   pdfcolmk.sty & tex/latex/oberdiek/pdfcolmk.sty\\
%   pdfcolmk.pdf & doc/latex/oberdiek/pdfcolmk.pdf\\
%   pdfcolmk.dtx & source/latex/oberdiek/pdfcolmk.dtx\\
% \end{tabular}^^A
% }^^A
% \sbox0{\t}^^A
% \ifdim\wd0>\linewidth
%   \begingroup
%     \advance\linewidth by\leftmargin
%     \advance\linewidth by\rightmargin
%   \edef\x{\endgroup
%     \def\noexpand\lw{\the\linewidth}^^A
%   }\x
%   \def\lwbox{^^A
%     \leavevmode
%     \hbox to \linewidth{^^A
%       \kern-\leftmargin\relax
%       \hss
%       \usebox0
%       \hss
%       \kern-\rightmargin\relax
%     }^^A
%   }^^A
%   \ifdim\wd0>\lw
%     \sbox0{\small\t}^^A
%     \ifdim\wd0>\linewidth
%       \ifdim\wd0>\lw
%         \sbox0{\footnotesize\t}^^A
%         \ifdim\wd0>\linewidth
%           \ifdim\wd0>\lw
%             \sbox0{\scriptsize\t}^^A
%             \ifdim\wd0>\linewidth
%               \ifdim\wd0>\lw
%                 \sbox0{\tiny\t}^^A
%                 \ifdim\wd0>\linewidth
%                   \lwbox
%                 \else
%                   \usebox0
%                 \fi
%               \else
%                 \lwbox
%               \fi
%             \else
%               \usebox0
%             \fi
%           \else
%             \lwbox
%           \fi
%         \else
%           \usebox0
%         \fi
%       \else
%         \lwbox
%       \fi
%     \else
%       \usebox0
%     \fi
%   \else
%     \lwbox
%   \fi
% \else
%   \usebox0
% \fi
% \end{quote}
% If you have a \xfile{docstrip.cfg} that configures and enables \docstrip's
% TDS installing feature, then some files can already be in the right
% place, see the documentation of \docstrip.
%
% \subsection{Refresh file name databases}
%
% If your \TeX~distribution
% (\teTeX, \mikTeX, \dots) relies on file name databases, you must refresh
% these. For example, \teTeX\ users run \verb|texhash| or
% \verb|mktexlsr|.
%
% \subsection{Some details for the interested}
%
% \paragraph{Attached source.}
%
% The PDF documentation on CTAN also includes the
% \xfile{.dtx} source file. It can be extracted by
% AcrobatReader 6 or higher. Another option is \textsf{pdftk},
% e.g. unpack the file into the current directory:
% \begin{quote}
%   \verb|pdftk pdfcolmk.pdf unpack_files output .|
% \end{quote}
%
% \paragraph{Unpacking with \LaTeX.}
% The \xfile{.dtx} chooses its action depending on the format:
% \begin{description}
% \item[\plainTeX:] Run \docstrip\ and extract the files.
% \item[\LaTeX:] Generate the documentation.
% \end{description}
% If you insist on using \LaTeX\ for \docstrip\ (really,
% \docstrip\ does not need \LaTeX), then inform the autodetect routine
% about your intention:
% \begin{quote}
%   \verb|latex \let\install=y\input{pdfcolmk.dtx}|
% \end{quote}
% Do not forget to quote the argument according to the demands
% of your shell.
%
% \paragraph{Generating the documentation.}
% You can use both the \xfile{.dtx} or the \xfile{.drv} to generate
% the documentation. The process can be configured by the
% configuration file \xfile{ltxdoc.cfg}. For instance, put this
% line into this file, if you want to have A4 as paper format:
% \begin{quote}
%   \verb|\PassOptionsToClass{a4paper}{article}|
% \end{quote}
% An example follows how to generate the
% documentation with pdf\LaTeX:
% \begin{quote}
%\begin{verbatim}
%pdflatex pdfcolmk.dtx
%makeindex -s gind.ist pdfcolmk.idx
%pdflatex pdfcolmk.dtx
%makeindex -s gind.ist pdfcolmk.idx
%pdflatex pdfcolmk.dtx
%\end{verbatim}
% \end{quote}
%
% \section{Catalogue}
%
% The following XML file can be used as source for the
% \href{http://mirror.ctan.org/help/Catalogue/catalogue.html}{\TeX\ Catalogue}.
% The elements \texttt{caption} and \texttt{description} are imported
% from the original XML file from the Catalogue.
% The name of the XML file in the Catalogue is \xfile{pdfcolmk.xml}.
%    \begin{macrocode}
%<*catalogue>
<?xml version='1.0' encoding='us-ascii'?>
<!DOCTYPE entry SYSTEM 'catalogue.dtd'>
<entry datestamp='$Date$' modifier='$Author$' id='pdfcolmk'>
  <name>pdfcolmk</name>
  <caption>Improving colour support under pdftex.</caption>
  <authorref id='auth:oberdiek'/>
  <copyright owner='Heiko Oberdiek' year='2000,2005-2008'/>
  <license type='lppl1.3'/>
  <version number='1.2'/>
  <description>
    The package provides macros that emulate the &#x2018;colour stack&#x2019;
    functionality of dvips.  The colour stack deals with colour
    manipulations when asynchronous events (like page-breaking) occur;
    pdftex does not (yet) have such a stack, but dvips does, and the
    <xref refid='color'>color</xref> package makes extensive use of
    it.
    <p/>
    This package is an experimental solution to the problem, and works
    best with pdf-e-tex.
    <p/>
    The package is part of the <xref refid='oberdiek'>oberdiek</xref> bundle.
  </description>
  <documentation details='Package documentation'
      href='ctan:/macros/latex/contrib/oberdiek/pdfcolmk.pdf'/>
  <ctan file='true' path='/macros/latex/contrib/oberdiek/pdfcolmk.dtx'/>
  <miktex location='oberdiek'/>
  <texlive location='oberdiek'/>
  <install path='/macros/latex/contrib/oberdiek/oberdiek.tds.zip'/>
</entry>
%</catalogue>
%    \end{macrocode}
%
% \begin{History}
%   \begin{Version}{2000/08/27 v0.1}
%   \item
%     First published version in newsgroup \xnewsgroup{comp.text.tex}:\\
%     \URL{``\link{pdftex: bug with colors?}''}^^A
%     {http://groups.google.com/group/comp.text.tex/msg/6f088e69e4085d2c}
%   \end{Version}
%   \begin{Version}{2000/09/02 v0.2}
%   \item
%     Next try.
%   \end{Version}
%   \begin{Version}{2000/09/02 v0.3}
%   \item
%     Solution without \eTeX\ added.
%   \end{Version}
%   \begin{Version}{2000/09/06 v0.4}
%   \item
%     Patch commands added.
%   \item
%     Patch for seminar.cls added.
%   \end{Version}
%   \begin{Version}{2000/09/06 v0.5}
%   \item
%     Bug fix: initialization of \cs{pec@value} added.
%   \end{Version}
%   \begin{Version}{2005/06/15 v0.6}
%   \item
%     Support for \cs{marginpar} added.
%     See thread in \xnewsgroup{comp.text.tex}:\\
%     \URL{``\link{Using \cs{textcolor} and \cs{marginpar} together}''}^^A
%     {http://groups.google.com/group/comp.text.tex/msg/38ed58f8845a2a4f}
%   \end{Version}
%   \begin{Version}{2005/07/09 v0.7}
%   \item
%     Output support added for \xpackage{memoir},
%     provided by Lars Madsen.
%   \end{Version}
%   \begin{Version}{2006/02/20 v0.8}
%   \item
%     Code is not changed.
%   \item
%     DTX framework.
%   \end{Version}
%   \begin{Version}{2007/01/01 v1.0}
%   \item
%     If \xfile{pdftex.def} \textgreater= 2007/01/01 v0.04a is used with
%     \pdfTeX\ \textgreater= 1.40.0, then package \xpackage{pdfcolmk} is obsolete.
%   \end{Version}
%   \begin{Version}{2007/04/11 v1.1}
%   \item
%     Line ends sanitized.
%   \end{Version}
%   \begin{Version}{2008/08/11 v1.2}
%   \item
%     Code is not changed.
%   \item
%     URLs updated.
%   \end{Version}
% \end{History}
%
% \PrintIndex
%
% \Finale
\endinput
|
% \end{quote}
% Do not forget to quote the argument according to the demands
% of your shell.
%
% \paragraph{Generating the documentation.}
% You can use both the \xfile{.dtx} or the \xfile{.drv} to generate
% the documentation. The process can be configured by the
% configuration file \xfile{ltxdoc.cfg}. For instance, put this
% line into this file, if you want to have A4 as paper format:
% \begin{quote}
%   \verb|\PassOptionsToClass{a4paper}{article}|
% \end{quote}
% An example follows how to generate the
% documentation with pdf\LaTeX:
% \begin{quote}
%\begin{verbatim}
%pdflatex pdfcolmk.dtx
%makeindex -s gind.ist pdfcolmk.idx
%pdflatex pdfcolmk.dtx
%makeindex -s gind.ist pdfcolmk.idx
%pdflatex pdfcolmk.dtx
%\end{verbatim}
% \end{quote}
%
% \section{Catalogue}
%
% The following XML file can be used as source for the
% \href{http://mirror.ctan.org/help/Catalogue/catalogue.html}{\TeX\ Catalogue}.
% The elements \texttt{caption} and \texttt{description} are imported
% from the original XML file from the Catalogue.
% The name of the XML file in the Catalogue is \xfile{pdfcolmk.xml}.
%    \begin{macrocode}
%<*catalogue>
<?xml version='1.0' encoding='us-ascii'?>
<!DOCTYPE entry SYSTEM 'catalogue.dtd'>
<entry datestamp='$Date$' modifier='$Author$' id='pdfcolmk'>
  <name>pdfcolmk</name>
  <caption>Improving colour support under pdftex.</caption>
  <authorref id='auth:oberdiek'/>
  <copyright owner='Heiko Oberdiek' year='2000,2005-2008'/>
  <license type='lppl1.3'/>
  <version number='1.2'/>
  <description>
    The package provides macros that emulate the &#x2018;colour stack&#x2019;
    functionality of dvips.  The colour stack deals with colour
    manipulations when asynchronous events (like page-breaking) occur;
    pdftex does not (yet) have such a stack, but dvips does, and the
    <xref refid='color'>color</xref> package makes extensive use of
    it.
    <p/>
    This package is an experimental solution to the problem, and works
    best with pdf-e-tex.
    <p/>
    The package is part of the <xref refid='oberdiek'>oberdiek</xref> bundle.
  </description>
  <documentation details='Package documentation'
      href='ctan:/macros/latex/contrib/oberdiek/pdfcolmk.pdf'/>
  <ctan file='true' path='/macros/latex/contrib/oberdiek/pdfcolmk.dtx'/>
  <miktex location='oberdiek'/>
  <texlive location='oberdiek'/>
  <install path='/macros/latex/contrib/oberdiek/oberdiek.tds.zip'/>
</entry>
%</catalogue>
%    \end{macrocode}
%
% \begin{History}
%   \begin{Version}{2000/08/27 v0.1}
%   \item
%     First published version in newsgroup \xnewsgroup{comp.text.tex}:\\
%     \URL{``\link{pdftex: bug with colors?}''}^^A
%     {http://groups.google.com/group/comp.text.tex/msg/6f088e69e4085d2c}
%   \end{Version}
%   \begin{Version}{2000/09/02 v0.2}
%   \item
%     Next try.
%   \end{Version}
%   \begin{Version}{2000/09/02 v0.3}
%   \item
%     Solution without \eTeX\ added.
%   \end{Version}
%   \begin{Version}{2000/09/06 v0.4}
%   \item
%     Patch commands added.
%   \item
%     Patch for seminar.cls added.
%   \end{Version}
%   \begin{Version}{2000/09/06 v0.5}
%   \item
%     Bug fix: initialization of \cs{pec@value} added.
%   \end{Version}
%   \begin{Version}{2005/06/15 v0.6}
%   \item
%     Support for \cs{marginpar} added.
%     See thread in \xnewsgroup{comp.text.tex}:\\
%     \URL{``\link{Using \cs{textcolor} and \cs{marginpar} together}''}^^A
%     {http://groups.google.com/group/comp.text.tex/msg/38ed58f8845a2a4f}
%   \end{Version}
%   \begin{Version}{2005/07/09 v0.7}
%   \item
%     Output support added for \xpackage{memoir},
%     provided by Lars Madsen.
%   \end{Version}
%   \begin{Version}{2006/02/20 v0.8}
%   \item
%     Code is not changed.
%   \item
%     DTX framework.
%   \end{Version}
%   \begin{Version}{2007/01/01 v1.0}
%   \item
%     If \xfile{pdftex.def} \textgreater= 2007/01/01 v0.04a is used with
%     \pdfTeX\ \textgreater= 1.40.0, then package \xpackage{pdfcolmk} is obsolete.
%   \end{Version}
%   \begin{Version}{2007/04/11 v1.1}
%   \item
%     Line ends sanitized.
%   \end{Version}
%   \begin{Version}{2008/08/11 v1.2}
%   \item
%     Code is not changed.
%   \item
%     URLs updated.
%   \end{Version}
% \end{History}
%
% \PrintIndex
%
% \Finale
\endinput

%        (quote the arguments according to the demands of your shell)
%
% Documentation:
%    (a) If pdfcolmk.drv is present:
%           latex pdfcolmk.drv
%    (b) Without pdfcolmk.drv:
%           latex pdfcolmk.dtx; ...
%    The class ltxdoc loads the configuration file ltxdoc.cfg
%    if available. Here you can specify further options, e.g.
%    use A4 as paper format:
%       \PassOptionsToClass{a4paper}{article}
%
%    Programm calls to get the documentation (example):
%       pdflatex pdfcolmk.dtx
%       makeindex -s gind.ist pdfcolmk.idx
%       pdflatex pdfcolmk.dtx
%       makeindex -s gind.ist pdfcolmk.idx
%       pdflatex pdfcolmk.dtx
%
% Installation:
%    TDS:tex/latex/oberdiek/pdfcolmk.sty
%    TDS:doc/latex/oberdiek/pdfcolmk.pdf
%    TDS:source/latex/oberdiek/pdfcolmk.dtx
%
%<*ignore>
\begingroup
  \catcode123=1 %
  \catcode125=2 %
  \def\x{LaTeX2e}%
\expandafter\endgroup
\ifcase 0\ifx\install y1\fi\expandafter
         \ifx\csname processbatchFile\endcsname\relax\else1\fi
         \ifx\fmtname\x\else 1\fi\relax
\else\csname fi\endcsname
%</ignore>
%<*install>
\input docstrip.tex
\Msg{************************************************************************}
\Msg{* Installation}
\Msg{* Package: pdfcolmk 2008/08/11 v1.2 Color support for pdfTeX via marks (HO)}
\Msg{************************************************************************}

\keepsilent
\askforoverwritefalse

\let\MetaPrefix\relax
\preamble

This is a generated file.

Project: pdfcolmk
Version: 2008/08/11 v1.2

Copyright (C) 2000, 2005-2008 by
   Heiko Oberdiek <heiko.oberdiek at googlemail.com>

This work may be distributed and/or modified under the
conditions of the LaTeX Project Public License, either
version 1.3c of this license or (at your option) any later
version. This version of this license is in
   http://www.latex-project.org/lppl/lppl-1-3c.txt
and the latest version of this license is in
   http://www.latex-project.org/lppl.txt
and version 1.3 or later is part of all distributions of
LaTeX version 2005/12/01 or later.

This work has the LPPL maintenance status "maintained".

This Current Maintainer of this work is Heiko Oberdiek.

This work consists of the main source file pdfcolmk.dtx
and the derived files
   pdfcolmk.sty, pdfcolmk.pdf, pdfcolmk.ins, pdfcolmk.drv.

\endpreamble
\let\MetaPrefix\DoubleperCent

\generate{%
  \file{pdfcolmk.ins}{\from{pdfcolmk.dtx}{install}}%
  \file{pdfcolmk.drv}{\from{pdfcolmk.dtx}{driver}}%
  \usedir{tex/latex/oberdiek}%
  \file{pdfcolmk.sty}{\from{pdfcolmk.dtx}{package}}%
  \nopreamble
  \nopostamble
  \usedir{source/latex/oberdiek/catalogue}%
  \file{pdfcolmk.xml}{\from{pdfcolmk.dtx}{catalogue}}%
}

\catcode32=13\relax% active space
\let =\space%
\Msg{************************************************************************}
\Msg{*}
\Msg{* To finish the installation you have to move the following}
\Msg{* file into a directory searched by TeX:}
\Msg{*}
\Msg{*     pdfcolmk.sty}
\Msg{*}
\Msg{* To produce the documentation run the file `pdfcolmk.drv'}
\Msg{* through LaTeX.}
\Msg{*}
\Msg{* Happy TeXing!}
\Msg{*}
\Msg{************************************************************************}

\endbatchfile
%</install>
%<*ignore>
\fi
%</ignore>
%<*driver>
\NeedsTeXFormat{LaTeX2e}
\ProvidesFile{pdfcolmk.drv}%
  [2008/08/11 v1.2 Color support for pdfTeX via marks (HO)]%
\documentclass{ltxdoc}
\usepackage{holtxdoc}[2011/11/22]
\begin{document}
  \DocInput{pdfcolmk.dtx}%
\end{document}
%</driver>
% \fi
%
% \CheckSum{843}
%
% \CharacterTable
%  {Upper-case    \A\B\C\D\E\F\G\H\I\J\K\L\M\N\O\P\Q\R\S\T\U\V\W\X\Y\Z
%   Lower-case    \a\b\c\d\e\f\g\h\i\j\k\l\m\n\o\p\q\r\s\t\u\v\w\x\y\z
%   Digits        \0\1\2\3\4\5\6\7\8\9
%   Exclamation   \!     Double quote  \"     Hash (number) \#
%   Dollar        \$     Percent       \%     Ampersand     \&
%   Acute accent  \'     Left paren    \(     Right paren   \)
%   Asterisk      \*     Plus          \+     Comma         \,
%   Minus         \-     Point         \.     Solidus       \/
%   Colon         \:     Semicolon     \;     Less than     \<
%   Equals        \=     Greater than  \>     Question mark \?
%   Commercial at \@     Left bracket  \[     Backslash     \\
%   Right bracket \]     Circumflex    \^     Underscore    \_
%   Grave accent  \`     Left brace    \{     Vertical bar  \|
%   Right brace   \}     Tilde         \~}
%
% \GetFileInfo{pdfcolmk.drv}
%
% \title{The \xpackage{pdfcolmk} package}
% \date{2008/08/11 v1.2}
% \author{Heiko Oberdiek\\\xemail{heiko.oberdiek at googlemail.com}}
%
% \maketitle
%
% \begin{abstract}
% This package tries a solution for the missing color
% stack of \pdfTeX.
% \end{abstract}
%
% \tableofcontents
%
% \section{Documentation}
%
% \subsection{Introduction}
%
% This package uses a mark register in order to solve the
% problem of a missing color stack of \pdfTeX\ prior 1.40.0.
% Since this version of \pdfTeX\ a color stack is available
% and supported by \xfile{pdftex.def} 2007/01/01 v0.04a.
% In this case this package is obsolete and the package
% stops its loading.
%
% \subsection{Background}
%
% After the Dante meeting (Clausthal 2000) I have started
% to experiment with the eTeX method of a \emph{colour} mark.
% One of the major problems is the understanding of the
% output routine and the need to rewrite it because of
% missing hooks. Currently I have made some tests in
% in onecolumn and twocolumn mode, but the state is
% experimental.
%
% \subsection{Limitations}
%
% \begin{itemize}
% \item Mark limitations: page breaks in math.
% \item \LaTeX's output routine is redefinded.
%   \begin{itemize}
%   \item Changes in the output routine of newer versions
%         of LaTeX are not detected.
%   \item Packages that change the output routine are not
%         supported.
%   \end{itemize}
% \item It does not support several independent text
%       streams like footnotes.
% \item Limitations in float and marginpar support.
% \end{itemize}
%
% \subsection{Recommendation}
%
% \eTeX\ (for additional mark register)
% Without \eTeX\ \LaTeX's mark commands are redefined
% to store an additional color value.
%
% \subsection{Usage}
%
% Load after package color:
% \begin{quote}
%   |\usepackage[pdftex]{color}|\\
%   |\usepackage{pdfcolmk}|
% \end{quote}
%
% \subsection{Compatibility}
%
% \begin{itemize}
% \item Load the following packages after \xpackage{pdfcolmk}:
%   \begin{quote}
%       \xpackage{mparhack.sty}
%   \end{quote}
% \item Load the following packages before \xpackage{pdfcolmk}:
%   \begin{quote}
%       \xpackage{marn.sty}\\
%       \xpackage{newmarn.sty}
%   \end{quote}
% \item Supported \cs{@addmarginpar} patch:
%   \begin{quote}
%       \xpackage{latex/base/latex.ltx}\\
%       \xpackage{memoir.cls}\\
%       \xpackage{poemscol/marn.sty}, \xpackage{poemscol/newmarn.sty}\\
%       \xpackage{mparhack.sty}
%   \end{quote}
% \item Unsupported \cs{@addmarginpar} patch:
%   \begin{quote}
%       \xpackage{lineno.sty}\\
%       \xpackage{sttools/marginal.sty}\\
%       \xpackage{revtex4.cls}
%   \end{quote}
% \end{itemize}
%
% \StopEventually{
% }
%
% \section{Implementation}
%
%    \begin{macrocode}
%<*package>
%    \end{macrocode}
%    Package identification.
%    \begin{macrocode}
\NeedsTeXFormat{LaTeX2e}
\ProvidesPackage{pdfcolmk}%
  [2008/08/11 v1.2 Color support for pdfTeX via marks (HO)]
%    \end{macrocode}
%
%    \begin{macrocode}
\@ifundefined{ver@pdftex.def}{%
  \PackageWarningNoLine{pdfcolmk}{%
    Nothing to fix, because \string`pdftex.def\string' not loaded%
  }%
  \endinput
}{}
\@ifpackageloaded{color}{}{%
  \PackageWarningNoLine{pdfcolmk}{%
    Nothing to fix, because \string`color.sty\string' not loaded%
  }%
  \endinput
}
\begingroup\expandafter\expandafter\expandafter\endgroup
\expandafter\ifx\csname main@pdfcolorstack\endcsname\relax
\else
  % pdftex.def >= 2007/01/01 0.04a and pdfTeX >= 1.40.0
  \begingroup
    \let\on@line\@empty
    \PackageInfo{pdfcolmk}{%
      The color stack of pdfTeX \string>\string= 1.40 is used. %
      Therefore\MessageBreak
      this package is not necessary and not loaded%
    }%
  \endgroup
  \expandafter\endinput
\fi

\PackageInfo{pdfcolmk}{%
  This package tries to simulate dvips's color stack\MessageBreak
  for pdfTeX based on a mark register of e-TeX.\MessageBreak
  It redefines LaTeX's output routine. Therefore\MessageBreak
  use with care, no warranties%
}

\ifx\marks\@undefined

  \let\pec@mark\mark
  \let\pec@value\empty
  \long\def\mark#1{%
    \protected@xdef\pec@value{#1}%
    \pec@setmark
  }%
  \def\pec@setmark{%
    \begingroup
      \@temptokena\expandafter{\pec@value}%
      \pec@mark{{\current@color}\the\@temptokena}%
    \endgroup
  }%
  \def\pec@getmark{%
    \xdef\pec@botcolor{%
      \expandafter\@firstofthree\botmark\@empty\@empty\@empty
    }%
  }%
  \long\def\@firstofthree#1#2#3{#1}%
  \CheckCommand{\@leftmark}[2]{#1}%
  \CheckCommand{\@rightmark}[2]{#2}%
  \CheckCommand*{\leftmark}{%
    \expandafter\@leftmark\botmark\@empty\@empty
  }%
  \CheckCommand*{\rightmark}{%
    \expandafter\@rightmark\firstmark\@empty\@empty
  }%
  \long\def\@leftmark#1#2#3{#2}%
  \long\def\@rightmark#1#2#3{#3}%
  \g@addto@macro\leftmark\@empty
  \g@addto@macro\rightmark\@empty

\else

  \RequirePackage{etex}[1998/03/26]%
  \newmarks\pec@marks
  \def\pec@setmark{\marks\pec@marks{\current@color}}%
  \def\pec@getmark{\xdef\pec@botcolor{\botmarks\pec@marks}}%

\fi
%    \end{macrocode}
%
% \subsection{\cs{marginpar} fix}
%
%    \begin{macrocode}
\chardef\pec@result\z@
\def\pec@temp#1{%
  \chardef\pec@result\@ne
  \begingroup
    \let\on@line\@empty
    \PackageInfo{pdfcolmk}{%
      Patch for \string\@addmarginpar\space applied (#1)%
    }%
  \endgroup
}
%    \end{macrocode}
%
% \subsubsection{latex/base/latex.ltx}
%
%    \begin{macrocode}
\def\pec@addmarginpar{%
  \@next\@marbox\@currlist{%
    \@cons\@freelist\@marbox
    \@cons\@freelist\@currbox
  }\@latexbug
  \@tempcnta\@ne
  \if@twocolumn
    \if@firstcolumn
      \@tempcnta\m@ne
    \fi
  \else
    \if@mparswitch
      \ifodd\c@page
      \else
        \@tempcnta\m@ne
      \fi
    \fi
    \if@reversemargin \@tempcnta -\@tempcnta \fi
  \fi
  \ifnum\@tempcnta <\z@  \global\setbox\@marbox\box\@currbox \fi
  \@tempdima\@mparbottom
  \advance\@tempdima -\@pageht
  \advance\@tempdima\ht\@marbox
  \ifdim\@tempdima >\z@
    \@latex@warning@no@line{Marginpar on page \thepage\space moved}%
  \else
    \@tempdima\z@
  \fi
  \global\@mparbottom\@pageht
  \global\advance\@mparbottom\@tempdima
  \global\advance\@mparbottom\dp\@marbox
  \global\advance\@mparbottom\marginparpush
  \advance\@tempdima -\ht\@marbox
  \global\setbox\@marbox\vbox{%
    \vskip \@tempdima
    \box \@marbox
  }%
  \global \ht\@marbox \z@
  \global \dp\@marbox \z@
  \kern -\@pagedp
  \nointerlineskip
  \hb@xt@\columnwidth{%
    \ifnum \@tempcnta >\z@
      \hskip\columnwidth
      \hskip\marginparsep
    \else
      \hskip -\marginparsep
      \hskip -\marginparwidth
    \fi
    \box\@marbox \hss
  }%
  \nointerlineskip
  \hbox{\vrule \@height\z@ \@width\z@ \@depth\@pagedp}%
}
\ifx\pec@addmarginpar\@addmarginpar
  \pec@temp{latex/base}%
\fi
%    \end{macrocode}
%
% \subsubsection{memoir.cls}
%
%    \begin{macrocode}
\def\pec@addmarginpar{%
  \checkoddpage
  \@next\@marbox\@currlist{%
    \@cons\@freelist\@marbox
    \@cons\@freelist\@currbox
  }\@latexbug
  \@tempcnta\@ne
  \if@twocolumn
    \if@firstcolumn
      \@tempcnta\m@ne
    \fi
  \else
    \if@mparswitch
      \ifoddpage
      \else
        \@tempcnta\m@ne
      \fi
    \fi
    \if@reversemargin
      \@tempcnta -\@tempcnta
    \fi
  \fi
  \ifnum\@tempcnta <\z@
    \global\setbox\@marbox\box\@currbox
  \fi
  \@tempdima\@mparbottom
  \advance\@tempdima -\@pageht
  \advance\@tempdima\ht\@marbox
  \ifdim\@tempdima >\z@
    \@latex@warning@no@line{%
      Marginpar on page \thepage\space moved by \the\@tempdima
    }%
  \else
    \@tempdima\z@
  \fi
  \global\@mparbottom\@pageht
  \global\advance\@mparbottom\@tempdima
  \global\advance\@mparbottom\dp\@marbox
  \global\advance\@mparbottom\marginparpush
  \advance\@tempdima -\ht\@marbox
  \global\setbox\@marbox\vbox{%
    \vskip \@tempdima
    \box \@marbox
  }%
  \global \ht\@marbox \z@
  \global \dp\@marbox \z@
  \kern -\@pagedp
  \nointerlineskip
  \hb@xt@\columnwidth{%
    \ifnum \@tempcnta >\z@
      \hskip\columnwidth
      \hskip\marginparsep
    \else
      \hskip -\marginparsep
      \hskip -\marginparwidth
    \fi
    \box\@marbox
    \hss
  }%
  \nointerlineskip
  \hbox{\vrule \@height\z@ \@width\z@ \@depth\@pagedp}%
}%
\ifx\pec@addmarginpar\@addmarginpar
  \pec@temp{memoir.cls}%
\fi
%    \end{macrocode}
%
% \subsubsection{poemscol/marn.sty, poemscol/newmarn.sty}
%
%    \begin{macrocode}
\def\pec@addmarginpar{%
  \@next \@marbox\@currlist{%
    \@cons\@freelist\@marbox
    \@cons\@freelist\@currbox
  }\@latexbug
  \global\advance\@mpar@count\m@ne
  \@ifundefined{@marn@\the\@mpar@count @}{% was location logged last time?
    \@tempcnta\@ne % NO: use original LaTeX logic
    \if@twocolumn
      \if@firstcolumn
        \@tempcnta\m@ne
      \fi
    \else
      \if@mparswitch
        \ifodd\c@page
        \else
          \@tempcnta\m@ne
        \fi
      \fi
      \if@reversemargin
        \@tempcnta -\@tempcnta
      \fi
    \fi
  }{%
    \@tempcnta %    YES: use record from last time to decide side.
    \@nameuse{@marn@\the\@mpar@count @}%
    \if@reversemargin -\fi \@ne
  }%
  \ifnum\@tempcnta <\z@
    \global\setbox\@marbox\box\@currbox
    \global\let\@marnbottom\@mparbottoml
  \else
    \global\let\@marnbottom\@mparbottom
  \fi
  \@tempdima\@marnbottom \advance\@tempdima -\@pageht
  \advance\@tempdima\ht\@marbox
  \ifdim\@tempdima >\z@
    \@@warning{Marginpar on page \thepage\space moved}%
  \else
    \@tempdima\z@
  \fi
  \global\@marnbottom\@pageht
  \global\advance\@marnbottom\@tempdima
  \global\advance\@marnbottom\dp\@marbox
  \global\advance\@marnbottom\marginparpush
  \advance\@tempdima -\ht\@marbox
  \global\ht\@marbox\z@
  \global\dp\@marbox\z@
  \vskip -\@pagedp
  \vskip\@tempdima\nointerlineskip
  \hbox to\columnwidth{%
    \ifnum \@tempcnta >\z@
      \hskip\columnwidth
      \hskip\marginparsep
    \else
      \hskip -\marginparsep
      \hskip -\marginparwidth
    \fi
    \if@filesw % record where this is for use next time:
       \@marn@log\@mpar@count
    \fi
    \box\@marbox
    \hss
  }%
  \nobreak   %% RmS 91/06/21 \nobreak added
  \vskip -\@tempdima
  \nointerlineskip
  \hbox{\vrule \@height\z@ \@width\z@ \@depth\@pagedp}%
}
\ifx\pec@addmarginpar\@addmarginpar
  \pec@temp{poemscol/(new)marn.sty}%
\fi
%    \end{macrocode}
%
% \subsubsection{refman/refart.cls, refnam/refrep.cls}
%
%    \begin{macrocode}
\def\pec@addmarginpar{%
  \@next\@marbox\@currlist{%
    \@cons\@freelist\@marbox
    \@cons\@freelist\@currbox
  }\@latexbug
  \@tempcnta\@ne
  \if@twocolumn
    \if@firstcolumn
      \@tempcnta\m@ne
    \fi
  \else
    \@tempcnta\m@ne
  \fi
  \ifnum\@tempcnta <\z@
    \global\setbox\@marbox\box\@currbox
  \fi
  \@tempdima\@mparbottom
  \advance\@tempdima -\@pageht
  \advance\@tempdima\ht\@marbox
  \ifdim\@tempdima >\z@
     \@@warning{Marginpar on page \thepage\space moved}%
  \else
     \@tempdima\z@
  \fi
  \global\@mparbottom\@pageht
  \global\advance\@mparbottom\@tempdima
  \global\advance\@mparbottom\dp\@marbox
  \global\advance\@mparbottom\marginparpush
  \advance\@tempdima -\ht\@marbox
  \global\setbox\@marbox\vbox{%
    \vskip \@tempdima \box \@marbox
  }%
  \global \ht\@marbox \z@
  \global \dp\@marbox \z@
  \kern -\@pagedp
  \nointerlineskip
  \hb@xt@\columnwidth{%
    \ifnum \@tempcnta >\z@
      \hskip\columnwidth
      \hskip\marginparsep
    \else
      \hskip -\marginparsep
      \hskip -\marginparwidth
    \fi
    \box\@marbox
    \hss
  }%
  \nointerlineskip
  \hbox{\vrule \@height\z@ \@width\z@ \@depth\@pagedp}%
}
\ifx\pec@addmarginpar\@addmarginpar
  \pec@temp{ref(art|rep).cls}%
\fi

\ifcase\pec@result
  \PackageInfo{pdfcolmk}{%
    Fix for \string\@addmarginpar\space is omitted, %
    because this variant\MessageBreak
    of \string\@addmarginpar\space
      is not recognized%
  }%
\else
  % apply patch for \@addmarginpar
  \def\pec@PatchAddMarginpar#1\columnwidth#2#3\@nil{%
    \pec@PatchAddMarginparI#2\@nil{#1}{#3}%
  }%
  \def\pec@PatchAddMarginparI#1\box\@marbox\hss#2\@nil#3#4{%
    \def\@addmarginpar{%
      #3%
      \columnwidth{%
        #1%
        \pdfliteral{q}%
        \rlap{%
          \box\@marbox
        }%
        \pdfliteral{Q}%
        \hss
        #2%
      }%
      #4%
    }%
  }%
  \expandafter\pec@PatchAddMarginpar\@addmarginpar\@nil
\fi
%    \end{macrocode}
%
% \subsection{Color fix}
%
%    \begin{macrocode}
\def\set@color{%
  \pdfliteral{\current@color}%
  \ifinner
  \else
    \pec@setmark
  \fi
  \aftergroup\reset@color
}
\def\reset@color{%
  \pdfliteral{\current@color}%
  \ifinner
  \else
    \pec@setmark
  \fi
}

\let\pec@botcolor\current@color

\def\pec@PatchVBoxCCLV{%
  \ifx\pec@botcolor\@empty
  \else
    \setbox\@cclv\vbox{%
      \pdfliteral{\pec@botcolor}%
      \unvbox\@cclv
    }%
  \fi
  \pec@getmark
}

\def\pec@PatchAlreadyInBox{%
  \ifx\pec@botcolor\@empty
  \else
    \pdfliteral{\pec@botcolor}%
  \fi
  \pec@getmark
}

\@ifclassloaded{memoir}{%
  \expandafter\def\expandafter\mem@makecol\expandafter{%
    \expandafter\pec@PatchVBoxCCLV
    \mem@makecol
  }%
  \endinput
}{}

\@ifclassloaded{seminar}{%
  \newcommand\pec@org@makeslide{}%
  \let\pec@org@makeslide\@makeslide
  \def\@makeslide{%
    \pec@PatchVBoxCCLV
    \pec@org@makeslide
  }%
  \endinput
}{}

\long\def\pec@output#1\@specialoutput\else#2\pec@end{%
  \begingroup
    \def\x{#2}%
  \expandafter\endgroup
  \ifx\x\@empty
    \PackageWarningNoLine{pdfcolmk}{%
      Unexpected \string\output\space routine detected,%
      \MessageBreak
      loading of package stopped%
    }%
    \expandafter\endinput
  \fi
}
\expandafter\expandafter\expandafter\pec@output
\expandafter\@firstofone\the\output\@specialoutput\else\pec@end

\long\def\pec@output#1\@specialoutput\else#2\pec@end{%
  \output{%
    #1\@specialoutput\else
    \pec@PatchVBoxCCLV
    #2%
  }%
}
\expandafter\expandafter\expandafter\pec@output
\expandafter\@firstofone\the\output\pec@end
%    \end{macrocode}
%
%    \begin{macrocode}
%</package>
%    \end{macrocode}
%
% \section{Installation}
%
% \subsection{Download}
%
% \paragraph{Package.} This package is available on
% CTAN\footnote{\url{ftp://ftp.ctan.org/tex-archive/}}:
% \begin{description}
% \item[\CTAN{macros/latex/contrib/oberdiek/pdfcolmk.dtx}] The source file.
% \item[\CTAN{macros/latex/contrib/oberdiek/pdfcolmk.pdf}] Documentation.
% \end{description}
%
%
% \paragraph{Bundle.} All the packages of the bundle `oberdiek'
% are also available in a TDS compliant ZIP archive. There
% the packages are already unpacked and the documentation files
% are generated. The files and directories obey the TDS standard.
% \begin{description}
% \item[\CTAN{install/macros/latex/contrib/oberdiek.tds.zip}]
% \end{description}
% \emph{TDS} refers to the standard ``A Directory Structure
% for \TeX\ Files'' (\CTAN{tds/tds.pdf}). Directories
% with \xfile{texmf} in their name are usually organized this way.
%
% \subsection{Bundle installation}
%
% \paragraph{Unpacking.} Unpack the \xfile{oberdiek.tds.zip} in the
% TDS tree (also known as \xfile{texmf} tree) of your choice.
% Example (linux):
% \begin{quote}
%   |unzip oberdiek.tds.zip -d ~/texmf|
% \end{quote}
%
% \paragraph{Script installation.}
% Check the directory \xfile{TDS:scripts/oberdiek/} for
% scripts that need further installation steps.
% Package \xpackage{attachfile2} comes with the Perl script
% \xfile{pdfatfi.pl} that should be installed in such a way
% that it can be called as \texttt{pdfatfi}.
% Example (linux):
% \begin{quote}
%   |chmod +x scripts/oberdiek/pdfatfi.pl|\\
%   |cp scripts/oberdiek/pdfatfi.pl /usr/local/bin/|
% \end{quote}
%
% \subsection{Package installation}
%
% \paragraph{Unpacking.} The \xfile{.dtx} file is a self-extracting
% \docstrip\ archive. The files are extracted by running the
% \xfile{.dtx} through \plainTeX:
% \begin{quote}
%   \verb|tex pdfcolmk.dtx|
% \end{quote}
%
% \paragraph{TDS.} Now the different files must be moved into
% the different directories in your installation TDS tree
% (also known as \xfile{texmf} tree):
% \begin{quote}
% \def\t{^^A
% \begin{tabular}{@{}>{\ttfamily}l@{ $\rightarrow$ }>{\ttfamily}l@{}}
%   pdfcolmk.sty & tex/latex/oberdiek/pdfcolmk.sty\\
%   pdfcolmk.pdf & doc/latex/oberdiek/pdfcolmk.pdf\\
%   pdfcolmk.dtx & source/latex/oberdiek/pdfcolmk.dtx\\
% \end{tabular}^^A
% }^^A
% \sbox0{\t}^^A
% \ifdim\wd0>\linewidth
%   \begingroup
%     \advance\linewidth by\leftmargin
%     \advance\linewidth by\rightmargin
%   \edef\x{\endgroup
%     \def\noexpand\lw{\the\linewidth}^^A
%   }\x
%   \def\lwbox{^^A
%     \leavevmode
%     \hbox to \linewidth{^^A
%       \kern-\leftmargin\relax
%       \hss
%       \usebox0
%       \hss
%       \kern-\rightmargin\relax
%     }^^A
%   }^^A
%   \ifdim\wd0>\lw
%     \sbox0{\small\t}^^A
%     \ifdim\wd0>\linewidth
%       \ifdim\wd0>\lw
%         \sbox0{\footnotesize\t}^^A
%         \ifdim\wd0>\linewidth
%           \ifdim\wd0>\lw
%             \sbox0{\scriptsize\t}^^A
%             \ifdim\wd0>\linewidth
%               \ifdim\wd0>\lw
%                 \sbox0{\tiny\t}^^A
%                 \ifdim\wd0>\linewidth
%                   \lwbox
%                 \else
%                   \usebox0
%                 \fi
%               \else
%                 \lwbox
%               \fi
%             \else
%               \usebox0
%             \fi
%           \else
%             \lwbox
%           \fi
%         \else
%           \usebox0
%         \fi
%       \else
%         \lwbox
%       \fi
%     \else
%       \usebox0
%     \fi
%   \else
%     \lwbox
%   \fi
% \else
%   \usebox0
% \fi
% \end{quote}
% If you have a \xfile{docstrip.cfg} that configures and enables \docstrip's
% TDS installing feature, then some files can already be in the right
% place, see the documentation of \docstrip.
%
% \subsection{Refresh file name databases}
%
% If your \TeX~distribution
% (\teTeX, \mikTeX, \dots) relies on file name databases, you must refresh
% these. For example, \teTeX\ users run \verb|texhash| or
% \verb|mktexlsr|.
%
% \subsection{Some details for the interested}
%
% \paragraph{Attached source.}
%
% The PDF documentation on CTAN also includes the
% \xfile{.dtx} source file. It can be extracted by
% AcrobatReader 6 or higher. Another option is \textsf{pdftk},
% e.g. unpack the file into the current directory:
% \begin{quote}
%   \verb|pdftk pdfcolmk.pdf unpack_files output .|
% \end{quote}
%
% \paragraph{Unpacking with \LaTeX.}
% The \xfile{.dtx} chooses its action depending on the format:
% \begin{description}
% \item[\plainTeX:] Run \docstrip\ and extract the files.
% \item[\LaTeX:] Generate the documentation.
% \end{description}
% If you insist on using \LaTeX\ for \docstrip\ (really,
% \docstrip\ does not need \LaTeX), then inform the autodetect routine
% about your intention:
% \begin{quote}
%   \verb|latex \let\install=y% \iffalse meta-comment
%
% File: pdfcolmk.dtx
% Version: 2008/08/11 v1.2
% Info: Color support for pdfTeX via marks
%
% Copyright (C) 2000, 2005-2008 by
%    Heiko Oberdiek <heiko.oberdiek at googlemail.com>
%
% This work may be distributed and/or modified under the
% conditions of the LaTeX Project Public License, either
% version 1.3c of this license or (at your option) any later
% version. This version of this license is in
%    http://www.latex-project.org/lppl/lppl-1-3c.txt
% and the latest version of this license is in
%    http://www.latex-project.org/lppl.txt
% and version 1.3 or later is part of all distributions of
% LaTeX version 2005/12/01 or later.
%
% This work has the LPPL maintenance status "maintained".
%
% This Current Maintainer of this work is Heiko Oberdiek.
%
% This work consists of the main source file pdfcolmk.dtx
% and the derived files
%    pdfcolmk.sty, pdfcolmk.pdf, pdfcolmk.ins, pdfcolmk.drv.
%
% Distribution:
%    CTAN:macros/latex/contrib/oberdiek/pdfcolmk.dtx
%    CTAN:macros/latex/contrib/oberdiek/pdfcolmk.pdf
%
% Unpacking:
%    (a) If pdfcolmk.ins is present:
%           tex pdfcolmk.ins
%    (b) Without pdfcolmk.ins:
%           tex pdfcolmk.dtx
%    (c) If you insist on using LaTeX
%           latex \let\install=y% \iffalse meta-comment
%
% File: pdfcolmk.dtx
% Version: 2008/08/11 v1.2
% Info: Color support for pdfTeX via marks
%
% Copyright (C) 2000, 2005-2008 by
%    Heiko Oberdiek <heiko.oberdiek at googlemail.com>
%
% This work may be distributed and/or modified under the
% conditions of the LaTeX Project Public License, either
% version 1.3c of this license or (at your option) any later
% version. This version of this license is in
%    http://www.latex-project.org/lppl/lppl-1-3c.txt
% and the latest version of this license is in
%    http://www.latex-project.org/lppl.txt
% and version 1.3 or later is part of all distributions of
% LaTeX version 2005/12/01 or later.
%
% This work has the LPPL maintenance status "maintained".
%
% This Current Maintainer of this work is Heiko Oberdiek.
%
% This work consists of the main source file pdfcolmk.dtx
% and the derived files
%    pdfcolmk.sty, pdfcolmk.pdf, pdfcolmk.ins, pdfcolmk.drv.
%
% Distribution:
%    CTAN:macros/latex/contrib/oberdiek/pdfcolmk.dtx
%    CTAN:macros/latex/contrib/oberdiek/pdfcolmk.pdf
%
% Unpacking:
%    (a) If pdfcolmk.ins is present:
%           tex pdfcolmk.ins
%    (b) Without pdfcolmk.ins:
%           tex pdfcolmk.dtx
%    (c) If you insist on using LaTeX
%           latex \let\install=y\input{pdfcolmk.dtx}
%        (quote the arguments according to the demands of your shell)
%
% Documentation:
%    (a) If pdfcolmk.drv is present:
%           latex pdfcolmk.drv
%    (b) Without pdfcolmk.drv:
%           latex pdfcolmk.dtx; ...
%    The class ltxdoc loads the configuration file ltxdoc.cfg
%    if available. Here you can specify further options, e.g.
%    use A4 as paper format:
%       \PassOptionsToClass{a4paper}{article}
%
%    Programm calls to get the documentation (example):
%       pdflatex pdfcolmk.dtx
%       makeindex -s gind.ist pdfcolmk.idx
%       pdflatex pdfcolmk.dtx
%       makeindex -s gind.ist pdfcolmk.idx
%       pdflatex pdfcolmk.dtx
%
% Installation:
%    TDS:tex/latex/oberdiek/pdfcolmk.sty
%    TDS:doc/latex/oberdiek/pdfcolmk.pdf
%    TDS:source/latex/oberdiek/pdfcolmk.dtx
%
%<*ignore>
\begingroup
  \catcode123=1 %
  \catcode125=2 %
  \def\x{LaTeX2e}%
\expandafter\endgroup
\ifcase 0\ifx\install y1\fi\expandafter
         \ifx\csname processbatchFile\endcsname\relax\else1\fi
         \ifx\fmtname\x\else 1\fi\relax
\else\csname fi\endcsname
%</ignore>
%<*install>
\input docstrip.tex
\Msg{************************************************************************}
\Msg{* Installation}
\Msg{* Package: pdfcolmk 2008/08/11 v1.2 Color support for pdfTeX via marks (HO)}
\Msg{************************************************************************}

\keepsilent
\askforoverwritefalse

\let\MetaPrefix\relax
\preamble

This is a generated file.

Project: pdfcolmk
Version: 2008/08/11 v1.2

Copyright (C) 2000, 2005-2008 by
   Heiko Oberdiek <heiko.oberdiek at googlemail.com>

This work may be distributed and/or modified under the
conditions of the LaTeX Project Public License, either
version 1.3c of this license or (at your option) any later
version. This version of this license is in
   http://www.latex-project.org/lppl/lppl-1-3c.txt
and the latest version of this license is in
   http://www.latex-project.org/lppl.txt
and version 1.3 or later is part of all distributions of
LaTeX version 2005/12/01 or later.

This work has the LPPL maintenance status "maintained".

This Current Maintainer of this work is Heiko Oberdiek.

This work consists of the main source file pdfcolmk.dtx
and the derived files
   pdfcolmk.sty, pdfcolmk.pdf, pdfcolmk.ins, pdfcolmk.drv.

\endpreamble
\let\MetaPrefix\DoubleperCent

\generate{%
  \file{pdfcolmk.ins}{\from{pdfcolmk.dtx}{install}}%
  \file{pdfcolmk.drv}{\from{pdfcolmk.dtx}{driver}}%
  \usedir{tex/latex/oberdiek}%
  \file{pdfcolmk.sty}{\from{pdfcolmk.dtx}{package}}%
  \nopreamble
  \nopostamble
  \usedir{source/latex/oberdiek/catalogue}%
  \file{pdfcolmk.xml}{\from{pdfcolmk.dtx}{catalogue}}%
}

\catcode32=13\relax% active space
\let =\space%
\Msg{************************************************************************}
\Msg{*}
\Msg{* To finish the installation you have to move the following}
\Msg{* file into a directory searched by TeX:}
\Msg{*}
\Msg{*     pdfcolmk.sty}
\Msg{*}
\Msg{* To produce the documentation run the file `pdfcolmk.drv'}
\Msg{* through LaTeX.}
\Msg{*}
\Msg{* Happy TeXing!}
\Msg{*}
\Msg{************************************************************************}

\endbatchfile
%</install>
%<*ignore>
\fi
%</ignore>
%<*driver>
\NeedsTeXFormat{LaTeX2e}
\ProvidesFile{pdfcolmk.drv}%
  [2008/08/11 v1.2 Color support for pdfTeX via marks (HO)]%
\documentclass{ltxdoc}
\usepackage{holtxdoc}[2011/11/22]
\begin{document}
  \DocInput{pdfcolmk.dtx}%
\end{document}
%</driver>
% \fi
%
% \CheckSum{843}
%
% \CharacterTable
%  {Upper-case    \A\B\C\D\E\F\G\H\I\J\K\L\M\N\O\P\Q\R\S\T\U\V\W\X\Y\Z
%   Lower-case    \a\b\c\d\e\f\g\h\i\j\k\l\m\n\o\p\q\r\s\t\u\v\w\x\y\z
%   Digits        \0\1\2\3\4\5\6\7\8\9
%   Exclamation   \!     Double quote  \"     Hash (number) \#
%   Dollar        \$     Percent       \%     Ampersand     \&
%   Acute accent  \'     Left paren    \(     Right paren   \)
%   Asterisk      \*     Plus          \+     Comma         \,
%   Minus         \-     Point         \.     Solidus       \/
%   Colon         \:     Semicolon     \;     Less than     \<
%   Equals        \=     Greater than  \>     Question mark \?
%   Commercial at \@     Left bracket  \[     Backslash     \\
%   Right bracket \]     Circumflex    \^     Underscore    \_
%   Grave accent  \`     Left brace    \{     Vertical bar  \|
%   Right brace   \}     Tilde         \~}
%
% \GetFileInfo{pdfcolmk.drv}
%
% \title{The \xpackage{pdfcolmk} package}
% \date{2008/08/11 v1.2}
% \author{Heiko Oberdiek\\\xemail{heiko.oberdiek at googlemail.com}}
%
% \maketitle
%
% \begin{abstract}
% This package tries a solution for the missing color
% stack of \pdfTeX.
% \end{abstract}
%
% \tableofcontents
%
% \section{Documentation}
%
% \subsection{Introduction}
%
% This package uses a mark register in order to solve the
% problem of a missing color stack of \pdfTeX\ prior 1.40.0.
% Since this version of \pdfTeX\ a color stack is available
% and supported by \xfile{pdftex.def} 2007/01/01 v0.04a.
% In this case this package is obsolete and the package
% stops its loading.
%
% \subsection{Background}
%
% After the Dante meeting (Clausthal 2000) I have started
% to experiment with the eTeX method of a \emph{colour} mark.
% One of the major problems is the understanding of the
% output routine and the need to rewrite it because of
% missing hooks. Currently I have made some tests in
% in onecolumn and twocolumn mode, but the state is
% experimental.
%
% \subsection{Limitations}
%
% \begin{itemize}
% \item Mark limitations: page breaks in math.
% \item \LaTeX's output routine is redefinded.
%   \begin{itemize}
%   \item Changes in the output routine of newer versions
%         of LaTeX are not detected.
%   \item Packages that change the output routine are not
%         supported.
%   \end{itemize}
% \item It does not support several independent text
%       streams like footnotes.
% \item Limitations in float and marginpar support.
% \end{itemize}
%
% \subsection{Recommendation}
%
% \eTeX\ (for additional mark register)
% Without \eTeX\ \LaTeX's mark commands are redefined
% to store an additional color value.
%
% \subsection{Usage}
%
% Load after package color:
% \begin{quote}
%   |\usepackage[pdftex]{color}|\\
%   |\usepackage{pdfcolmk}|
% \end{quote}
%
% \subsection{Compatibility}
%
% \begin{itemize}
% \item Load the following packages after \xpackage{pdfcolmk}:
%   \begin{quote}
%       \xpackage{mparhack.sty}
%   \end{quote}
% \item Load the following packages before \xpackage{pdfcolmk}:
%   \begin{quote}
%       \xpackage{marn.sty}\\
%       \xpackage{newmarn.sty}
%   \end{quote}
% \item Supported \cs{@addmarginpar} patch:
%   \begin{quote}
%       \xpackage{latex/base/latex.ltx}\\
%       \xpackage{memoir.cls}\\
%       \xpackage{poemscol/marn.sty}, \xpackage{poemscol/newmarn.sty}\\
%       \xpackage{mparhack.sty}
%   \end{quote}
% \item Unsupported \cs{@addmarginpar} patch:
%   \begin{quote}
%       \xpackage{lineno.sty}\\
%       \xpackage{sttools/marginal.sty}\\
%       \xpackage{revtex4.cls}
%   \end{quote}
% \end{itemize}
%
% \StopEventually{
% }
%
% \section{Implementation}
%
%    \begin{macrocode}
%<*package>
%    \end{macrocode}
%    Package identification.
%    \begin{macrocode}
\NeedsTeXFormat{LaTeX2e}
\ProvidesPackage{pdfcolmk}%
  [2008/08/11 v1.2 Color support for pdfTeX via marks (HO)]
%    \end{macrocode}
%
%    \begin{macrocode}
\@ifundefined{ver@pdftex.def}{%
  \PackageWarningNoLine{pdfcolmk}{%
    Nothing to fix, because \string`pdftex.def\string' not loaded%
  }%
  \endinput
}{}
\@ifpackageloaded{color}{}{%
  \PackageWarningNoLine{pdfcolmk}{%
    Nothing to fix, because \string`color.sty\string' not loaded%
  }%
  \endinput
}
\begingroup\expandafter\expandafter\expandafter\endgroup
\expandafter\ifx\csname main@pdfcolorstack\endcsname\relax
\else
  % pdftex.def >= 2007/01/01 0.04a and pdfTeX >= 1.40.0
  \begingroup
    \let\on@line\@empty
    \PackageInfo{pdfcolmk}{%
      The color stack of pdfTeX \string>\string= 1.40 is used. %
      Therefore\MessageBreak
      this package is not necessary and not loaded%
    }%
  \endgroup
  \expandafter\endinput
\fi

\PackageInfo{pdfcolmk}{%
  This package tries to simulate dvips's color stack\MessageBreak
  for pdfTeX based on a mark register of e-TeX.\MessageBreak
  It redefines LaTeX's output routine. Therefore\MessageBreak
  use with care, no warranties%
}

\ifx\marks\@undefined

  \let\pec@mark\mark
  \let\pec@value\empty
  \long\def\mark#1{%
    \protected@xdef\pec@value{#1}%
    \pec@setmark
  }%
  \def\pec@setmark{%
    \begingroup
      \@temptokena\expandafter{\pec@value}%
      \pec@mark{{\current@color}\the\@temptokena}%
    \endgroup
  }%
  \def\pec@getmark{%
    \xdef\pec@botcolor{%
      \expandafter\@firstofthree\botmark\@empty\@empty\@empty
    }%
  }%
  \long\def\@firstofthree#1#2#3{#1}%
  \CheckCommand{\@leftmark}[2]{#1}%
  \CheckCommand{\@rightmark}[2]{#2}%
  \CheckCommand*{\leftmark}{%
    \expandafter\@leftmark\botmark\@empty\@empty
  }%
  \CheckCommand*{\rightmark}{%
    \expandafter\@rightmark\firstmark\@empty\@empty
  }%
  \long\def\@leftmark#1#2#3{#2}%
  \long\def\@rightmark#1#2#3{#3}%
  \g@addto@macro\leftmark\@empty
  \g@addto@macro\rightmark\@empty

\else

  \RequirePackage{etex}[1998/03/26]%
  \newmarks\pec@marks
  \def\pec@setmark{\marks\pec@marks{\current@color}}%
  \def\pec@getmark{\xdef\pec@botcolor{\botmarks\pec@marks}}%

\fi
%    \end{macrocode}
%
% \subsection{\cs{marginpar} fix}
%
%    \begin{macrocode}
\chardef\pec@result\z@
\def\pec@temp#1{%
  \chardef\pec@result\@ne
  \begingroup
    \let\on@line\@empty
    \PackageInfo{pdfcolmk}{%
      Patch for \string\@addmarginpar\space applied (#1)%
    }%
  \endgroup
}
%    \end{macrocode}
%
% \subsubsection{latex/base/latex.ltx}
%
%    \begin{macrocode}
\def\pec@addmarginpar{%
  \@next\@marbox\@currlist{%
    \@cons\@freelist\@marbox
    \@cons\@freelist\@currbox
  }\@latexbug
  \@tempcnta\@ne
  \if@twocolumn
    \if@firstcolumn
      \@tempcnta\m@ne
    \fi
  \else
    \if@mparswitch
      \ifodd\c@page
      \else
        \@tempcnta\m@ne
      \fi
    \fi
    \if@reversemargin \@tempcnta -\@tempcnta \fi
  \fi
  \ifnum\@tempcnta <\z@  \global\setbox\@marbox\box\@currbox \fi
  \@tempdima\@mparbottom
  \advance\@tempdima -\@pageht
  \advance\@tempdima\ht\@marbox
  \ifdim\@tempdima >\z@
    \@latex@warning@no@line{Marginpar on page \thepage\space moved}%
  \else
    \@tempdima\z@
  \fi
  \global\@mparbottom\@pageht
  \global\advance\@mparbottom\@tempdima
  \global\advance\@mparbottom\dp\@marbox
  \global\advance\@mparbottom\marginparpush
  \advance\@tempdima -\ht\@marbox
  \global\setbox\@marbox\vbox{%
    \vskip \@tempdima
    \box \@marbox
  }%
  \global \ht\@marbox \z@
  \global \dp\@marbox \z@
  \kern -\@pagedp
  \nointerlineskip
  \hb@xt@\columnwidth{%
    \ifnum \@tempcnta >\z@
      \hskip\columnwidth
      \hskip\marginparsep
    \else
      \hskip -\marginparsep
      \hskip -\marginparwidth
    \fi
    \box\@marbox \hss
  }%
  \nointerlineskip
  \hbox{\vrule \@height\z@ \@width\z@ \@depth\@pagedp}%
}
\ifx\pec@addmarginpar\@addmarginpar
  \pec@temp{latex/base}%
\fi
%    \end{macrocode}
%
% \subsubsection{memoir.cls}
%
%    \begin{macrocode}
\def\pec@addmarginpar{%
  \checkoddpage
  \@next\@marbox\@currlist{%
    \@cons\@freelist\@marbox
    \@cons\@freelist\@currbox
  }\@latexbug
  \@tempcnta\@ne
  \if@twocolumn
    \if@firstcolumn
      \@tempcnta\m@ne
    \fi
  \else
    \if@mparswitch
      \ifoddpage
      \else
        \@tempcnta\m@ne
      \fi
    \fi
    \if@reversemargin
      \@tempcnta -\@tempcnta
    \fi
  \fi
  \ifnum\@tempcnta <\z@
    \global\setbox\@marbox\box\@currbox
  \fi
  \@tempdima\@mparbottom
  \advance\@tempdima -\@pageht
  \advance\@tempdima\ht\@marbox
  \ifdim\@tempdima >\z@
    \@latex@warning@no@line{%
      Marginpar on page \thepage\space moved by \the\@tempdima
    }%
  \else
    \@tempdima\z@
  \fi
  \global\@mparbottom\@pageht
  \global\advance\@mparbottom\@tempdima
  \global\advance\@mparbottom\dp\@marbox
  \global\advance\@mparbottom\marginparpush
  \advance\@tempdima -\ht\@marbox
  \global\setbox\@marbox\vbox{%
    \vskip \@tempdima
    \box \@marbox
  }%
  \global \ht\@marbox \z@
  \global \dp\@marbox \z@
  \kern -\@pagedp
  \nointerlineskip
  \hb@xt@\columnwidth{%
    \ifnum \@tempcnta >\z@
      \hskip\columnwidth
      \hskip\marginparsep
    \else
      \hskip -\marginparsep
      \hskip -\marginparwidth
    \fi
    \box\@marbox
    \hss
  }%
  \nointerlineskip
  \hbox{\vrule \@height\z@ \@width\z@ \@depth\@pagedp}%
}%
\ifx\pec@addmarginpar\@addmarginpar
  \pec@temp{memoir.cls}%
\fi
%    \end{macrocode}
%
% \subsubsection{poemscol/marn.sty, poemscol/newmarn.sty}
%
%    \begin{macrocode}
\def\pec@addmarginpar{%
  \@next \@marbox\@currlist{%
    \@cons\@freelist\@marbox
    \@cons\@freelist\@currbox
  }\@latexbug
  \global\advance\@mpar@count\m@ne
  \@ifundefined{@marn@\the\@mpar@count @}{% was location logged last time?
    \@tempcnta\@ne % NO: use original LaTeX logic
    \if@twocolumn
      \if@firstcolumn
        \@tempcnta\m@ne
      \fi
    \else
      \if@mparswitch
        \ifodd\c@page
        \else
          \@tempcnta\m@ne
        \fi
      \fi
      \if@reversemargin
        \@tempcnta -\@tempcnta
      \fi
    \fi
  }{%
    \@tempcnta %    YES: use record from last time to decide side.
    \@nameuse{@marn@\the\@mpar@count @}%
    \if@reversemargin -\fi \@ne
  }%
  \ifnum\@tempcnta <\z@
    \global\setbox\@marbox\box\@currbox
    \global\let\@marnbottom\@mparbottoml
  \else
    \global\let\@marnbottom\@mparbottom
  \fi
  \@tempdima\@marnbottom \advance\@tempdima -\@pageht
  \advance\@tempdima\ht\@marbox
  \ifdim\@tempdima >\z@
    \@@warning{Marginpar on page \thepage\space moved}%
  \else
    \@tempdima\z@
  \fi
  \global\@marnbottom\@pageht
  \global\advance\@marnbottom\@tempdima
  \global\advance\@marnbottom\dp\@marbox
  \global\advance\@marnbottom\marginparpush
  \advance\@tempdima -\ht\@marbox
  \global\ht\@marbox\z@
  \global\dp\@marbox\z@
  \vskip -\@pagedp
  \vskip\@tempdima\nointerlineskip
  \hbox to\columnwidth{%
    \ifnum \@tempcnta >\z@
      \hskip\columnwidth
      \hskip\marginparsep
    \else
      \hskip -\marginparsep
      \hskip -\marginparwidth
    \fi
    \if@filesw % record where this is for use next time:
       \@marn@log\@mpar@count
    \fi
    \box\@marbox
    \hss
  }%
  \nobreak   %% RmS 91/06/21 \nobreak added
  \vskip -\@tempdima
  \nointerlineskip
  \hbox{\vrule \@height\z@ \@width\z@ \@depth\@pagedp}%
}
\ifx\pec@addmarginpar\@addmarginpar
  \pec@temp{poemscol/(new)marn.sty}%
\fi
%    \end{macrocode}
%
% \subsubsection{refman/refart.cls, refnam/refrep.cls}
%
%    \begin{macrocode}
\def\pec@addmarginpar{%
  \@next\@marbox\@currlist{%
    \@cons\@freelist\@marbox
    \@cons\@freelist\@currbox
  }\@latexbug
  \@tempcnta\@ne
  \if@twocolumn
    \if@firstcolumn
      \@tempcnta\m@ne
    \fi
  \else
    \@tempcnta\m@ne
  \fi
  \ifnum\@tempcnta <\z@
    \global\setbox\@marbox\box\@currbox
  \fi
  \@tempdima\@mparbottom
  \advance\@tempdima -\@pageht
  \advance\@tempdima\ht\@marbox
  \ifdim\@tempdima >\z@
     \@@warning{Marginpar on page \thepage\space moved}%
  \else
     \@tempdima\z@
  \fi
  \global\@mparbottom\@pageht
  \global\advance\@mparbottom\@tempdima
  \global\advance\@mparbottom\dp\@marbox
  \global\advance\@mparbottom\marginparpush
  \advance\@tempdima -\ht\@marbox
  \global\setbox\@marbox\vbox{%
    \vskip \@tempdima \box \@marbox
  }%
  \global \ht\@marbox \z@
  \global \dp\@marbox \z@
  \kern -\@pagedp
  \nointerlineskip
  \hb@xt@\columnwidth{%
    \ifnum \@tempcnta >\z@
      \hskip\columnwidth
      \hskip\marginparsep
    \else
      \hskip -\marginparsep
      \hskip -\marginparwidth
    \fi
    \box\@marbox
    \hss
  }%
  \nointerlineskip
  \hbox{\vrule \@height\z@ \@width\z@ \@depth\@pagedp}%
}
\ifx\pec@addmarginpar\@addmarginpar
  \pec@temp{ref(art|rep).cls}%
\fi

\ifcase\pec@result
  \PackageInfo{pdfcolmk}{%
    Fix for \string\@addmarginpar\space is omitted, %
    because this variant\MessageBreak
    of \string\@addmarginpar\space
      is not recognized%
  }%
\else
  % apply patch for \@addmarginpar
  \def\pec@PatchAddMarginpar#1\columnwidth#2#3\@nil{%
    \pec@PatchAddMarginparI#2\@nil{#1}{#3}%
  }%
  \def\pec@PatchAddMarginparI#1\box\@marbox\hss#2\@nil#3#4{%
    \def\@addmarginpar{%
      #3%
      \columnwidth{%
        #1%
        \pdfliteral{q}%
        \rlap{%
          \box\@marbox
        }%
        \pdfliteral{Q}%
        \hss
        #2%
      }%
      #4%
    }%
  }%
  \expandafter\pec@PatchAddMarginpar\@addmarginpar\@nil
\fi
%    \end{macrocode}
%
% \subsection{Color fix}
%
%    \begin{macrocode}
\def\set@color{%
  \pdfliteral{\current@color}%
  \ifinner
  \else
    \pec@setmark
  \fi
  \aftergroup\reset@color
}
\def\reset@color{%
  \pdfliteral{\current@color}%
  \ifinner
  \else
    \pec@setmark
  \fi
}

\let\pec@botcolor\current@color

\def\pec@PatchVBoxCCLV{%
  \ifx\pec@botcolor\@empty
  \else
    \setbox\@cclv\vbox{%
      \pdfliteral{\pec@botcolor}%
      \unvbox\@cclv
    }%
  \fi
  \pec@getmark
}

\def\pec@PatchAlreadyInBox{%
  \ifx\pec@botcolor\@empty
  \else
    \pdfliteral{\pec@botcolor}%
  \fi
  \pec@getmark
}

\@ifclassloaded{memoir}{%
  \expandafter\def\expandafter\mem@makecol\expandafter{%
    \expandafter\pec@PatchVBoxCCLV
    \mem@makecol
  }%
  \endinput
}{}

\@ifclassloaded{seminar}{%
  \newcommand\pec@org@makeslide{}%
  \let\pec@org@makeslide\@makeslide
  \def\@makeslide{%
    \pec@PatchVBoxCCLV
    \pec@org@makeslide
  }%
  \endinput
}{}

\long\def\pec@output#1\@specialoutput\else#2\pec@end{%
  \begingroup
    \def\x{#2}%
  \expandafter\endgroup
  \ifx\x\@empty
    \PackageWarningNoLine{pdfcolmk}{%
      Unexpected \string\output\space routine detected,%
      \MessageBreak
      loading of package stopped%
    }%
    \expandafter\endinput
  \fi
}
\expandafter\expandafter\expandafter\pec@output
\expandafter\@firstofone\the\output\@specialoutput\else\pec@end

\long\def\pec@output#1\@specialoutput\else#2\pec@end{%
  \output{%
    #1\@specialoutput\else
    \pec@PatchVBoxCCLV
    #2%
  }%
}
\expandafter\expandafter\expandafter\pec@output
\expandafter\@firstofone\the\output\pec@end
%    \end{macrocode}
%
%    \begin{macrocode}
%</package>
%    \end{macrocode}
%
% \section{Installation}
%
% \subsection{Download}
%
% \paragraph{Package.} This package is available on
% CTAN\footnote{\url{ftp://ftp.ctan.org/tex-archive/}}:
% \begin{description}
% \item[\CTAN{macros/latex/contrib/oberdiek/pdfcolmk.dtx}] The source file.
% \item[\CTAN{macros/latex/contrib/oberdiek/pdfcolmk.pdf}] Documentation.
% \end{description}
%
%
% \paragraph{Bundle.} All the packages of the bundle `oberdiek'
% are also available in a TDS compliant ZIP archive. There
% the packages are already unpacked and the documentation files
% are generated. The files and directories obey the TDS standard.
% \begin{description}
% \item[\CTAN{install/macros/latex/contrib/oberdiek.tds.zip}]
% \end{description}
% \emph{TDS} refers to the standard ``A Directory Structure
% for \TeX\ Files'' (\CTAN{tds/tds.pdf}). Directories
% with \xfile{texmf} in their name are usually organized this way.
%
% \subsection{Bundle installation}
%
% \paragraph{Unpacking.} Unpack the \xfile{oberdiek.tds.zip} in the
% TDS tree (also known as \xfile{texmf} tree) of your choice.
% Example (linux):
% \begin{quote}
%   |unzip oberdiek.tds.zip -d ~/texmf|
% \end{quote}
%
% \paragraph{Script installation.}
% Check the directory \xfile{TDS:scripts/oberdiek/} for
% scripts that need further installation steps.
% Package \xpackage{attachfile2} comes with the Perl script
% \xfile{pdfatfi.pl} that should be installed in such a way
% that it can be called as \texttt{pdfatfi}.
% Example (linux):
% \begin{quote}
%   |chmod +x scripts/oberdiek/pdfatfi.pl|\\
%   |cp scripts/oberdiek/pdfatfi.pl /usr/local/bin/|
% \end{quote}
%
% \subsection{Package installation}
%
% \paragraph{Unpacking.} The \xfile{.dtx} file is a self-extracting
% \docstrip\ archive. The files are extracted by running the
% \xfile{.dtx} through \plainTeX:
% \begin{quote}
%   \verb|tex pdfcolmk.dtx|
% \end{quote}
%
% \paragraph{TDS.} Now the different files must be moved into
% the different directories in your installation TDS tree
% (also known as \xfile{texmf} tree):
% \begin{quote}
% \def\t{^^A
% \begin{tabular}{@{}>{\ttfamily}l@{ $\rightarrow$ }>{\ttfamily}l@{}}
%   pdfcolmk.sty & tex/latex/oberdiek/pdfcolmk.sty\\
%   pdfcolmk.pdf & doc/latex/oberdiek/pdfcolmk.pdf\\
%   pdfcolmk.dtx & source/latex/oberdiek/pdfcolmk.dtx\\
% \end{tabular}^^A
% }^^A
% \sbox0{\t}^^A
% \ifdim\wd0>\linewidth
%   \begingroup
%     \advance\linewidth by\leftmargin
%     \advance\linewidth by\rightmargin
%   \edef\x{\endgroup
%     \def\noexpand\lw{\the\linewidth}^^A
%   }\x
%   \def\lwbox{^^A
%     \leavevmode
%     \hbox to \linewidth{^^A
%       \kern-\leftmargin\relax
%       \hss
%       \usebox0
%       \hss
%       \kern-\rightmargin\relax
%     }^^A
%   }^^A
%   \ifdim\wd0>\lw
%     \sbox0{\small\t}^^A
%     \ifdim\wd0>\linewidth
%       \ifdim\wd0>\lw
%         \sbox0{\footnotesize\t}^^A
%         \ifdim\wd0>\linewidth
%           \ifdim\wd0>\lw
%             \sbox0{\scriptsize\t}^^A
%             \ifdim\wd0>\linewidth
%               \ifdim\wd0>\lw
%                 \sbox0{\tiny\t}^^A
%                 \ifdim\wd0>\linewidth
%                   \lwbox
%                 \else
%                   \usebox0
%                 \fi
%               \else
%                 \lwbox
%               \fi
%             \else
%               \usebox0
%             \fi
%           \else
%             \lwbox
%           \fi
%         \else
%           \usebox0
%         \fi
%       \else
%         \lwbox
%       \fi
%     \else
%       \usebox0
%     \fi
%   \else
%     \lwbox
%   \fi
% \else
%   \usebox0
% \fi
% \end{quote}
% If you have a \xfile{docstrip.cfg} that configures and enables \docstrip's
% TDS installing feature, then some files can already be in the right
% place, see the documentation of \docstrip.
%
% \subsection{Refresh file name databases}
%
% If your \TeX~distribution
% (\teTeX, \mikTeX, \dots) relies on file name databases, you must refresh
% these. For example, \teTeX\ users run \verb|texhash| or
% \verb|mktexlsr|.
%
% \subsection{Some details for the interested}
%
% \paragraph{Attached source.}
%
% The PDF documentation on CTAN also includes the
% \xfile{.dtx} source file. It can be extracted by
% AcrobatReader 6 or higher. Another option is \textsf{pdftk},
% e.g. unpack the file into the current directory:
% \begin{quote}
%   \verb|pdftk pdfcolmk.pdf unpack_files output .|
% \end{quote}
%
% \paragraph{Unpacking with \LaTeX.}
% The \xfile{.dtx} chooses its action depending on the format:
% \begin{description}
% \item[\plainTeX:] Run \docstrip\ and extract the files.
% \item[\LaTeX:] Generate the documentation.
% \end{description}
% If you insist on using \LaTeX\ for \docstrip\ (really,
% \docstrip\ does not need \LaTeX), then inform the autodetect routine
% about your intention:
% \begin{quote}
%   \verb|latex \let\install=y\input{pdfcolmk.dtx}|
% \end{quote}
% Do not forget to quote the argument according to the demands
% of your shell.
%
% \paragraph{Generating the documentation.}
% You can use both the \xfile{.dtx} or the \xfile{.drv} to generate
% the documentation. The process can be configured by the
% configuration file \xfile{ltxdoc.cfg}. For instance, put this
% line into this file, if you want to have A4 as paper format:
% \begin{quote}
%   \verb|\PassOptionsToClass{a4paper}{article}|
% \end{quote}
% An example follows how to generate the
% documentation with pdf\LaTeX:
% \begin{quote}
%\begin{verbatim}
%pdflatex pdfcolmk.dtx
%makeindex -s gind.ist pdfcolmk.idx
%pdflatex pdfcolmk.dtx
%makeindex -s gind.ist pdfcolmk.idx
%pdflatex pdfcolmk.dtx
%\end{verbatim}
% \end{quote}
%
% \section{Catalogue}
%
% The following XML file can be used as source for the
% \href{http://mirror.ctan.org/help/Catalogue/catalogue.html}{\TeX\ Catalogue}.
% The elements \texttt{caption} and \texttt{description} are imported
% from the original XML file from the Catalogue.
% The name of the XML file in the Catalogue is \xfile{pdfcolmk.xml}.
%    \begin{macrocode}
%<*catalogue>
<?xml version='1.0' encoding='us-ascii'?>
<!DOCTYPE entry SYSTEM 'catalogue.dtd'>
<entry datestamp='$Date$' modifier='$Author$' id='pdfcolmk'>
  <name>pdfcolmk</name>
  <caption>Improving colour support under pdftex.</caption>
  <authorref id='auth:oberdiek'/>
  <copyright owner='Heiko Oberdiek' year='2000,2005-2008'/>
  <license type='lppl1.3'/>
  <version number='1.2'/>
  <description>
    The package provides macros that emulate the &#x2018;colour stack&#x2019;
    functionality of dvips.  The colour stack deals with colour
    manipulations when asynchronous events (like page-breaking) occur;
    pdftex does not (yet) have such a stack, but dvips does, and the
    <xref refid='color'>color</xref> package makes extensive use of
    it.
    <p/>
    This package is an experimental solution to the problem, and works
    best with pdf-e-tex.
    <p/>
    The package is part of the <xref refid='oberdiek'>oberdiek</xref> bundle.
  </description>
  <documentation details='Package documentation'
      href='ctan:/macros/latex/contrib/oberdiek/pdfcolmk.pdf'/>
  <ctan file='true' path='/macros/latex/contrib/oberdiek/pdfcolmk.dtx'/>
  <miktex location='oberdiek'/>
  <texlive location='oberdiek'/>
  <install path='/macros/latex/contrib/oberdiek/oberdiek.tds.zip'/>
</entry>
%</catalogue>
%    \end{macrocode}
%
% \begin{History}
%   \begin{Version}{2000/08/27 v0.1}
%   \item
%     First published version in newsgroup \xnewsgroup{comp.text.tex}:\\
%     \URL{``\link{pdftex: bug with colors?}''}^^A
%     {http://groups.google.com/group/comp.text.tex/msg/6f088e69e4085d2c}
%   \end{Version}
%   \begin{Version}{2000/09/02 v0.2}
%   \item
%     Next try.
%   \end{Version}
%   \begin{Version}{2000/09/02 v0.3}
%   \item
%     Solution without \eTeX\ added.
%   \end{Version}
%   \begin{Version}{2000/09/06 v0.4}
%   \item
%     Patch commands added.
%   \item
%     Patch for seminar.cls added.
%   \end{Version}
%   \begin{Version}{2000/09/06 v0.5}
%   \item
%     Bug fix: initialization of \cs{pec@value} added.
%   \end{Version}
%   \begin{Version}{2005/06/15 v0.6}
%   \item
%     Support for \cs{marginpar} added.
%     See thread in \xnewsgroup{comp.text.tex}:\\
%     \URL{``\link{Using \cs{textcolor} and \cs{marginpar} together}''}^^A
%     {http://groups.google.com/group/comp.text.tex/msg/38ed58f8845a2a4f}
%   \end{Version}
%   \begin{Version}{2005/07/09 v0.7}
%   \item
%     Output support added for \xpackage{memoir},
%     provided by Lars Madsen.
%   \end{Version}
%   \begin{Version}{2006/02/20 v0.8}
%   \item
%     Code is not changed.
%   \item
%     DTX framework.
%   \end{Version}
%   \begin{Version}{2007/01/01 v1.0}
%   \item
%     If \xfile{pdftex.def} \textgreater= 2007/01/01 v0.04a is used with
%     \pdfTeX\ \textgreater= 1.40.0, then package \xpackage{pdfcolmk} is obsolete.
%   \end{Version}
%   \begin{Version}{2007/04/11 v1.1}
%   \item
%     Line ends sanitized.
%   \end{Version}
%   \begin{Version}{2008/08/11 v1.2}
%   \item
%     Code is not changed.
%   \item
%     URLs updated.
%   \end{Version}
% \end{History}
%
% \PrintIndex
%
% \Finale
\endinput

%        (quote the arguments according to the demands of your shell)
%
% Documentation:
%    (a) If pdfcolmk.drv is present:
%           latex pdfcolmk.drv
%    (b) Without pdfcolmk.drv:
%           latex pdfcolmk.dtx; ...
%    The class ltxdoc loads the configuration file ltxdoc.cfg
%    if available. Here you can specify further options, e.g.
%    use A4 as paper format:
%       \PassOptionsToClass{a4paper}{article}
%
%    Programm calls to get the documentation (example):
%       pdflatex pdfcolmk.dtx
%       makeindex -s gind.ist pdfcolmk.idx
%       pdflatex pdfcolmk.dtx
%       makeindex -s gind.ist pdfcolmk.idx
%       pdflatex pdfcolmk.dtx
%
% Installation:
%    TDS:tex/latex/oberdiek/pdfcolmk.sty
%    TDS:doc/latex/oberdiek/pdfcolmk.pdf
%    TDS:source/latex/oberdiek/pdfcolmk.dtx
%
%<*ignore>
\begingroup
  \catcode123=1 %
  \catcode125=2 %
  \def\x{LaTeX2e}%
\expandafter\endgroup
\ifcase 0\ifx\install y1\fi\expandafter
         \ifx\csname processbatchFile\endcsname\relax\else1\fi
         \ifx\fmtname\x\else 1\fi\relax
\else\csname fi\endcsname
%</ignore>
%<*install>
\input docstrip.tex
\Msg{************************************************************************}
\Msg{* Installation}
\Msg{* Package: pdfcolmk 2008/08/11 v1.2 Color support for pdfTeX via marks (HO)}
\Msg{************************************************************************}

\keepsilent
\askforoverwritefalse

\let\MetaPrefix\relax
\preamble

This is a generated file.

Project: pdfcolmk
Version: 2008/08/11 v1.2

Copyright (C) 2000, 2005-2008 by
   Heiko Oberdiek <heiko.oberdiek at googlemail.com>

This work may be distributed and/or modified under the
conditions of the LaTeX Project Public License, either
version 1.3c of this license or (at your option) any later
version. This version of this license is in
   http://www.latex-project.org/lppl/lppl-1-3c.txt
and the latest version of this license is in
   http://www.latex-project.org/lppl.txt
and version 1.3 or later is part of all distributions of
LaTeX version 2005/12/01 or later.

This work has the LPPL maintenance status "maintained".

This Current Maintainer of this work is Heiko Oberdiek.

This work consists of the main source file pdfcolmk.dtx
and the derived files
   pdfcolmk.sty, pdfcolmk.pdf, pdfcolmk.ins, pdfcolmk.drv.

\endpreamble
\let\MetaPrefix\DoubleperCent

\generate{%
  \file{pdfcolmk.ins}{\from{pdfcolmk.dtx}{install}}%
  \file{pdfcolmk.drv}{\from{pdfcolmk.dtx}{driver}}%
  \usedir{tex/latex/oberdiek}%
  \file{pdfcolmk.sty}{\from{pdfcolmk.dtx}{package}}%
  \nopreamble
  \nopostamble
  \usedir{source/latex/oberdiek/catalogue}%
  \file{pdfcolmk.xml}{\from{pdfcolmk.dtx}{catalogue}}%
}

\catcode32=13\relax% active space
\let =\space%
\Msg{************************************************************************}
\Msg{*}
\Msg{* To finish the installation you have to move the following}
\Msg{* file into a directory searched by TeX:}
\Msg{*}
\Msg{*     pdfcolmk.sty}
\Msg{*}
\Msg{* To produce the documentation run the file `pdfcolmk.drv'}
\Msg{* through LaTeX.}
\Msg{*}
\Msg{* Happy TeXing!}
\Msg{*}
\Msg{************************************************************************}

\endbatchfile
%</install>
%<*ignore>
\fi
%</ignore>
%<*driver>
\NeedsTeXFormat{LaTeX2e}
\ProvidesFile{pdfcolmk.drv}%
  [2008/08/11 v1.2 Color support for pdfTeX via marks (HO)]%
\documentclass{ltxdoc}
\usepackage{holtxdoc}[2011/11/22]
\begin{document}
  \DocInput{pdfcolmk.dtx}%
\end{document}
%</driver>
% \fi
%
% \CheckSum{843}
%
% \CharacterTable
%  {Upper-case    \A\B\C\D\E\F\G\H\I\J\K\L\M\N\O\P\Q\R\S\T\U\V\W\X\Y\Z
%   Lower-case    \a\b\c\d\e\f\g\h\i\j\k\l\m\n\o\p\q\r\s\t\u\v\w\x\y\z
%   Digits        \0\1\2\3\4\5\6\7\8\9
%   Exclamation   \!     Double quote  \"     Hash (number) \#
%   Dollar        \$     Percent       \%     Ampersand     \&
%   Acute accent  \'     Left paren    \(     Right paren   \)
%   Asterisk      \*     Plus          \+     Comma         \,
%   Minus         \-     Point         \.     Solidus       \/
%   Colon         \:     Semicolon     \;     Less than     \<
%   Equals        \=     Greater than  \>     Question mark \?
%   Commercial at \@     Left bracket  \[     Backslash     \\
%   Right bracket \]     Circumflex    \^     Underscore    \_
%   Grave accent  \`     Left brace    \{     Vertical bar  \|
%   Right brace   \}     Tilde         \~}
%
% \GetFileInfo{pdfcolmk.drv}
%
% \title{The \xpackage{pdfcolmk} package}
% \date{2008/08/11 v1.2}
% \author{Heiko Oberdiek\\\xemail{heiko.oberdiek at googlemail.com}}
%
% \maketitle
%
% \begin{abstract}
% This package tries a solution for the missing color
% stack of \pdfTeX.
% \end{abstract}
%
% \tableofcontents
%
% \section{Documentation}
%
% \subsection{Introduction}
%
% This package uses a mark register in order to solve the
% problem of a missing color stack of \pdfTeX\ prior 1.40.0.
% Since this version of \pdfTeX\ a color stack is available
% and supported by \xfile{pdftex.def} 2007/01/01 v0.04a.
% In this case this package is obsolete and the package
% stops its loading.
%
% \subsection{Background}
%
% After the Dante meeting (Clausthal 2000) I have started
% to experiment with the eTeX method of a \emph{colour} mark.
% One of the major problems is the understanding of the
% output routine and the need to rewrite it because of
% missing hooks. Currently I have made some tests in
% in onecolumn and twocolumn mode, but the state is
% experimental.
%
% \subsection{Limitations}
%
% \begin{itemize}
% \item Mark limitations: page breaks in math.
% \item \LaTeX's output routine is redefinded.
%   \begin{itemize}
%   \item Changes in the output routine of newer versions
%         of LaTeX are not detected.
%   \item Packages that change the output routine are not
%         supported.
%   \end{itemize}
% \item It does not support several independent text
%       streams like footnotes.
% \item Limitations in float and marginpar support.
% \end{itemize}
%
% \subsection{Recommendation}
%
% \eTeX\ (for additional mark register)
% Without \eTeX\ \LaTeX's mark commands are redefined
% to store an additional color value.
%
% \subsection{Usage}
%
% Load after package color:
% \begin{quote}
%   |\usepackage[pdftex]{color}|\\
%   |\usepackage{pdfcolmk}|
% \end{quote}
%
% \subsection{Compatibility}
%
% \begin{itemize}
% \item Load the following packages after \xpackage{pdfcolmk}:
%   \begin{quote}
%       \xpackage{mparhack.sty}
%   \end{quote}
% \item Load the following packages before \xpackage{pdfcolmk}:
%   \begin{quote}
%       \xpackage{marn.sty}\\
%       \xpackage{newmarn.sty}
%   \end{quote}
% \item Supported \cs{@addmarginpar} patch:
%   \begin{quote}
%       \xpackage{latex/base/latex.ltx}\\
%       \xpackage{memoir.cls}\\
%       \xpackage{poemscol/marn.sty}, \xpackage{poemscol/newmarn.sty}\\
%       \xpackage{mparhack.sty}
%   \end{quote}
% \item Unsupported \cs{@addmarginpar} patch:
%   \begin{quote}
%       \xpackage{lineno.sty}\\
%       \xpackage{sttools/marginal.sty}\\
%       \xpackage{revtex4.cls}
%   \end{quote}
% \end{itemize}
%
% \StopEventually{
% }
%
% \section{Implementation}
%
%    \begin{macrocode}
%<*package>
%    \end{macrocode}
%    Package identification.
%    \begin{macrocode}
\NeedsTeXFormat{LaTeX2e}
\ProvidesPackage{pdfcolmk}%
  [2008/08/11 v1.2 Color support for pdfTeX via marks (HO)]
%    \end{macrocode}
%
%    \begin{macrocode}
\@ifundefined{ver@pdftex.def}{%
  \PackageWarningNoLine{pdfcolmk}{%
    Nothing to fix, because \string`pdftex.def\string' not loaded%
  }%
  \endinput
}{}
\@ifpackageloaded{color}{}{%
  \PackageWarningNoLine{pdfcolmk}{%
    Nothing to fix, because \string`color.sty\string' not loaded%
  }%
  \endinput
}
\begingroup\expandafter\expandafter\expandafter\endgroup
\expandafter\ifx\csname main@pdfcolorstack\endcsname\relax
\else
  % pdftex.def >= 2007/01/01 0.04a and pdfTeX >= 1.40.0
  \begingroup
    \let\on@line\@empty
    \PackageInfo{pdfcolmk}{%
      The color stack of pdfTeX \string>\string= 1.40 is used. %
      Therefore\MessageBreak
      this package is not necessary and not loaded%
    }%
  \endgroup
  \expandafter\endinput
\fi

\PackageInfo{pdfcolmk}{%
  This package tries to simulate dvips's color stack\MessageBreak
  for pdfTeX based on a mark register of e-TeX.\MessageBreak
  It redefines LaTeX's output routine. Therefore\MessageBreak
  use with care, no warranties%
}

\ifx\marks\@undefined

  \let\pec@mark\mark
  \let\pec@value\empty
  \long\def\mark#1{%
    \protected@xdef\pec@value{#1}%
    \pec@setmark
  }%
  \def\pec@setmark{%
    \begingroup
      \@temptokena\expandafter{\pec@value}%
      \pec@mark{{\current@color}\the\@temptokena}%
    \endgroup
  }%
  \def\pec@getmark{%
    \xdef\pec@botcolor{%
      \expandafter\@firstofthree\botmark\@empty\@empty\@empty
    }%
  }%
  \long\def\@firstofthree#1#2#3{#1}%
  \CheckCommand{\@leftmark}[2]{#1}%
  \CheckCommand{\@rightmark}[2]{#2}%
  \CheckCommand*{\leftmark}{%
    \expandafter\@leftmark\botmark\@empty\@empty
  }%
  \CheckCommand*{\rightmark}{%
    \expandafter\@rightmark\firstmark\@empty\@empty
  }%
  \long\def\@leftmark#1#2#3{#2}%
  \long\def\@rightmark#1#2#3{#3}%
  \g@addto@macro\leftmark\@empty
  \g@addto@macro\rightmark\@empty

\else

  \RequirePackage{etex}[1998/03/26]%
  \newmarks\pec@marks
  \def\pec@setmark{\marks\pec@marks{\current@color}}%
  \def\pec@getmark{\xdef\pec@botcolor{\botmarks\pec@marks}}%

\fi
%    \end{macrocode}
%
% \subsection{\cs{marginpar} fix}
%
%    \begin{macrocode}
\chardef\pec@result\z@
\def\pec@temp#1{%
  \chardef\pec@result\@ne
  \begingroup
    \let\on@line\@empty
    \PackageInfo{pdfcolmk}{%
      Patch for \string\@addmarginpar\space applied (#1)%
    }%
  \endgroup
}
%    \end{macrocode}
%
% \subsubsection{latex/base/latex.ltx}
%
%    \begin{macrocode}
\def\pec@addmarginpar{%
  \@next\@marbox\@currlist{%
    \@cons\@freelist\@marbox
    \@cons\@freelist\@currbox
  }\@latexbug
  \@tempcnta\@ne
  \if@twocolumn
    \if@firstcolumn
      \@tempcnta\m@ne
    \fi
  \else
    \if@mparswitch
      \ifodd\c@page
      \else
        \@tempcnta\m@ne
      \fi
    \fi
    \if@reversemargin \@tempcnta -\@tempcnta \fi
  \fi
  \ifnum\@tempcnta <\z@  \global\setbox\@marbox\box\@currbox \fi
  \@tempdima\@mparbottom
  \advance\@tempdima -\@pageht
  \advance\@tempdima\ht\@marbox
  \ifdim\@tempdima >\z@
    \@latex@warning@no@line{Marginpar on page \thepage\space moved}%
  \else
    \@tempdima\z@
  \fi
  \global\@mparbottom\@pageht
  \global\advance\@mparbottom\@tempdima
  \global\advance\@mparbottom\dp\@marbox
  \global\advance\@mparbottom\marginparpush
  \advance\@tempdima -\ht\@marbox
  \global\setbox\@marbox\vbox{%
    \vskip \@tempdima
    \box \@marbox
  }%
  \global \ht\@marbox \z@
  \global \dp\@marbox \z@
  \kern -\@pagedp
  \nointerlineskip
  \hb@xt@\columnwidth{%
    \ifnum \@tempcnta >\z@
      \hskip\columnwidth
      \hskip\marginparsep
    \else
      \hskip -\marginparsep
      \hskip -\marginparwidth
    \fi
    \box\@marbox \hss
  }%
  \nointerlineskip
  \hbox{\vrule \@height\z@ \@width\z@ \@depth\@pagedp}%
}
\ifx\pec@addmarginpar\@addmarginpar
  \pec@temp{latex/base}%
\fi
%    \end{macrocode}
%
% \subsubsection{memoir.cls}
%
%    \begin{macrocode}
\def\pec@addmarginpar{%
  \checkoddpage
  \@next\@marbox\@currlist{%
    \@cons\@freelist\@marbox
    \@cons\@freelist\@currbox
  }\@latexbug
  \@tempcnta\@ne
  \if@twocolumn
    \if@firstcolumn
      \@tempcnta\m@ne
    \fi
  \else
    \if@mparswitch
      \ifoddpage
      \else
        \@tempcnta\m@ne
      \fi
    \fi
    \if@reversemargin
      \@tempcnta -\@tempcnta
    \fi
  \fi
  \ifnum\@tempcnta <\z@
    \global\setbox\@marbox\box\@currbox
  \fi
  \@tempdima\@mparbottom
  \advance\@tempdima -\@pageht
  \advance\@tempdima\ht\@marbox
  \ifdim\@tempdima >\z@
    \@latex@warning@no@line{%
      Marginpar on page \thepage\space moved by \the\@tempdima
    }%
  \else
    \@tempdima\z@
  \fi
  \global\@mparbottom\@pageht
  \global\advance\@mparbottom\@tempdima
  \global\advance\@mparbottom\dp\@marbox
  \global\advance\@mparbottom\marginparpush
  \advance\@tempdima -\ht\@marbox
  \global\setbox\@marbox\vbox{%
    \vskip \@tempdima
    \box \@marbox
  }%
  \global \ht\@marbox \z@
  \global \dp\@marbox \z@
  \kern -\@pagedp
  \nointerlineskip
  \hb@xt@\columnwidth{%
    \ifnum \@tempcnta >\z@
      \hskip\columnwidth
      \hskip\marginparsep
    \else
      \hskip -\marginparsep
      \hskip -\marginparwidth
    \fi
    \box\@marbox
    \hss
  }%
  \nointerlineskip
  \hbox{\vrule \@height\z@ \@width\z@ \@depth\@pagedp}%
}%
\ifx\pec@addmarginpar\@addmarginpar
  \pec@temp{memoir.cls}%
\fi
%    \end{macrocode}
%
% \subsubsection{poemscol/marn.sty, poemscol/newmarn.sty}
%
%    \begin{macrocode}
\def\pec@addmarginpar{%
  \@next \@marbox\@currlist{%
    \@cons\@freelist\@marbox
    \@cons\@freelist\@currbox
  }\@latexbug
  \global\advance\@mpar@count\m@ne
  \@ifundefined{@marn@\the\@mpar@count @}{% was location logged last time?
    \@tempcnta\@ne % NO: use original LaTeX logic
    \if@twocolumn
      \if@firstcolumn
        \@tempcnta\m@ne
      \fi
    \else
      \if@mparswitch
        \ifodd\c@page
        \else
          \@tempcnta\m@ne
        \fi
      \fi
      \if@reversemargin
        \@tempcnta -\@tempcnta
      \fi
    \fi
  }{%
    \@tempcnta %    YES: use record from last time to decide side.
    \@nameuse{@marn@\the\@mpar@count @}%
    \if@reversemargin -\fi \@ne
  }%
  \ifnum\@tempcnta <\z@
    \global\setbox\@marbox\box\@currbox
    \global\let\@marnbottom\@mparbottoml
  \else
    \global\let\@marnbottom\@mparbottom
  \fi
  \@tempdima\@marnbottom \advance\@tempdima -\@pageht
  \advance\@tempdima\ht\@marbox
  \ifdim\@tempdima >\z@
    \@@warning{Marginpar on page \thepage\space moved}%
  \else
    \@tempdima\z@
  \fi
  \global\@marnbottom\@pageht
  \global\advance\@marnbottom\@tempdima
  \global\advance\@marnbottom\dp\@marbox
  \global\advance\@marnbottom\marginparpush
  \advance\@tempdima -\ht\@marbox
  \global\ht\@marbox\z@
  \global\dp\@marbox\z@
  \vskip -\@pagedp
  \vskip\@tempdima\nointerlineskip
  \hbox to\columnwidth{%
    \ifnum \@tempcnta >\z@
      \hskip\columnwidth
      \hskip\marginparsep
    \else
      \hskip -\marginparsep
      \hskip -\marginparwidth
    \fi
    \if@filesw % record where this is for use next time:
       \@marn@log\@mpar@count
    \fi
    \box\@marbox
    \hss
  }%
  \nobreak   %% RmS 91/06/21 \nobreak added
  \vskip -\@tempdima
  \nointerlineskip
  \hbox{\vrule \@height\z@ \@width\z@ \@depth\@pagedp}%
}
\ifx\pec@addmarginpar\@addmarginpar
  \pec@temp{poemscol/(new)marn.sty}%
\fi
%    \end{macrocode}
%
% \subsubsection{refman/refart.cls, refnam/refrep.cls}
%
%    \begin{macrocode}
\def\pec@addmarginpar{%
  \@next\@marbox\@currlist{%
    \@cons\@freelist\@marbox
    \@cons\@freelist\@currbox
  }\@latexbug
  \@tempcnta\@ne
  \if@twocolumn
    \if@firstcolumn
      \@tempcnta\m@ne
    \fi
  \else
    \@tempcnta\m@ne
  \fi
  \ifnum\@tempcnta <\z@
    \global\setbox\@marbox\box\@currbox
  \fi
  \@tempdima\@mparbottom
  \advance\@tempdima -\@pageht
  \advance\@tempdima\ht\@marbox
  \ifdim\@tempdima >\z@
     \@@warning{Marginpar on page \thepage\space moved}%
  \else
     \@tempdima\z@
  \fi
  \global\@mparbottom\@pageht
  \global\advance\@mparbottom\@tempdima
  \global\advance\@mparbottom\dp\@marbox
  \global\advance\@mparbottom\marginparpush
  \advance\@tempdima -\ht\@marbox
  \global\setbox\@marbox\vbox{%
    \vskip \@tempdima \box \@marbox
  }%
  \global \ht\@marbox \z@
  \global \dp\@marbox \z@
  \kern -\@pagedp
  \nointerlineskip
  \hb@xt@\columnwidth{%
    \ifnum \@tempcnta >\z@
      \hskip\columnwidth
      \hskip\marginparsep
    \else
      \hskip -\marginparsep
      \hskip -\marginparwidth
    \fi
    \box\@marbox
    \hss
  }%
  \nointerlineskip
  \hbox{\vrule \@height\z@ \@width\z@ \@depth\@pagedp}%
}
\ifx\pec@addmarginpar\@addmarginpar
  \pec@temp{ref(art|rep).cls}%
\fi

\ifcase\pec@result
  \PackageInfo{pdfcolmk}{%
    Fix for \string\@addmarginpar\space is omitted, %
    because this variant\MessageBreak
    of \string\@addmarginpar\space
      is not recognized%
  }%
\else
  % apply patch for \@addmarginpar
  \def\pec@PatchAddMarginpar#1\columnwidth#2#3\@nil{%
    \pec@PatchAddMarginparI#2\@nil{#1}{#3}%
  }%
  \def\pec@PatchAddMarginparI#1\box\@marbox\hss#2\@nil#3#4{%
    \def\@addmarginpar{%
      #3%
      \columnwidth{%
        #1%
        \pdfliteral{q}%
        \rlap{%
          \box\@marbox
        }%
        \pdfliteral{Q}%
        \hss
        #2%
      }%
      #4%
    }%
  }%
  \expandafter\pec@PatchAddMarginpar\@addmarginpar\@nil
\fi
%    \end{macrocode}
%
% \subsection{Color fix}
%
%    \begin{macrocode}
\def\set@color{%
  \pdfliteral{\current@color}%
  \ifinner
  \else
    \pec@setmark
  \fi
  \aftergroup\reset@color
}
\def\reset@color{%
  \pdfliteral{\current@color}%
  \ifinner
  \else
    \pec@setmark
  \fi
}

\let\pec@botcolor\current@color

\def\pec@PatchVBoxCCLV{%
  \ifx\pec@botcolor\@empty
  \else
    \setbox\@cclv\vbox{%
      \pdfliteral{\pec@botcolor}%
      \unvbox\@cclv
    }%
  \fi
  \pec@getmark
}

\def\pec@PatchAlreadyInBox{%
  \ifx\pec@botcolor\@empty
  \else
    \pdfliteral{\pec@botcolor}%
  \fi
  \pec@getmark
}

\@ifclassloaded{memoir}{%
  \expandafter\def\expandafter\mem@makecol\expandafter{%
    \expandafter\pec@PatchVBoxCCLV
    \mem@makecol
  }%
  \endinput
}{}

\@ifclassloaded{seminar}{%
  \newcommand\pec@org@makeslide{}%
  \let\pec@org@makeslide\@makeslide
  \def\@makeslide{%
    \pec@PatchVBoxCCLV
    \pec@org@makeslide
  }%
  \endinput
}{}

\long\def\pec@output#1\@specialoutput\else#2\pec@end{%
  \begingroup
    \def\x{#2}%
  \expandafter\endgroup
  \ifx\x\@empty
    \PackageWarningNoLine{pdfcolmk}{%
      Unexpected \string\output\space routine detected,%
      \MessageBreak
      loading of package stopped%
    }%
    \expandafter\endinput
  \fi
}
\expandafter\expandafter\expandafter\pec@output
\expandafter\@firstofone\the\output\@specialoutput\else\pec@end

\long\def\pec@output#1\@specialoutput\else#2\pec@end{%
  \output{%
    #1\@specialoutput\else
    \pec@PatchVBoxCCLV
    #2%
  }%
}
\expandafter\expandafter\expandafter\pec@output
\expandafter\@firstofone\the\output\pec@end
%    \end{macrocode}
%
%    \begin{macrocode}
%</package>
%    \end{macrocode}
%
% \section{Installation}
%
% \subsection{Download}
%
% \paragraph{Package.} This package is available on
% CTAN\footnote{\url{ftp://ftp.ctan.org/tex-archive/}}:
% \begin{description}
% \item[\CTAN{macros/latex/contrib/oberdiek/pdfcolmk.dtx}] The source file.
% \item[\CTAN{macros/latex/contrib/oberdiek/pdfcolmk.pdf}] Documentation.
% \end{description}
%
%
% \paragraph{Bundle.} All the packages of the bundle `oberdiek'
% are also available in a TDS compliant ZIP archive. There
% the packages are already unpacked and the documentation files
% are generated. The files and directories obey the TDS standard.
% \begin{description}
% \item[\CTAN{install/macros/latex/contrib/oberdiek.tds.zip}]
% \end{description}
% \emph{TDS} refers to the standard ``A Directory Structure
% for \TeX\ Files'' (\CTAN{tds/tds.pdf}). Directories
% with \xfile{texmf} in their name are usually organized this way.
%
% \subsection{Bundle installation}
%
% \paragraph{Unpacking.} Unpack the \xfile{oberdiek.tds.zip} in the
% TDS tree (also known as \xfile{texmf} tree) of your choice.
% Example (linux):
% \begin{quote}
%   |unzip oberdiek.tds.zip -d ~/texmf|
% \end{quote}
%
% \paragraph{Script installation.}
% Check the directory \xfile{TDS:scripts/oberdiek/} for
% scripts that need further installation steps.
% Package \xpackage{attachfile2} comes with the Perl script
% \xfile{pdfatfi.pl} that should be installed in such a way
% that it can be called as \texttt{pdfatfi}.
% Example (linux):
% \begin{quote}
%   |chmod +x scripts/oberdiek/pdfatfi.pl|\\
%   |cp scripts/oberdiek/pdfatfi.pl /usr/local/bin/|
% \end{quote}
%
% \subsection{Package installation}
%
% \paragraph{Unpacking.} The \xfile{.dtx} file is a self-extracting
% \docstrip\ archive. The files are extracted by running the
% \xfile{.dtx} through \plainTeX:
% \begin{quote}
%   \verb|tex pdfcolmk.dtx|
% \end{quote}
%
% \paragraph{TDS.} Now the different files must be moved into
% the different directories in your installation TDS tree
% (also known as \xfile{texmf} tree):
% \begin{quote}
% \def\t{^^A
% \begin{tabular}{@{}>{\ttfamily}l@{ $\rightarrow$ }>{\ttfamily}l@{}}
%   pdfcolmk.sty & tex/latex/oberdiek/pdfcolmk.sty\\
%   pdfcolmk.pdf & doc/latex/oberdiek/pdfcolmk.pdf\\
%   pdfcolmk.dtx & source/latex/oberdiek/pdfcolmk.dtx\\
% \end{tabular}^^A
% }^^A
% \sbox0{\t}^^A
% \ifdim\wd0>\linewidth
%   \begingroup
%     \advance\linewidth by\leftmargin
%     \advance\linewidth by\rightmargin
%   \edef\x{\endgroup
%     \def\noexpand\lw{\the\linewidth}^^A
%   }\x
%   \def\lwbox{^^A
%     \leavevmode
%     \hbox to \linewidth{^^A
%       \kern-\leftmargin\relax
%       \hss
%       \usebox0
%       \hss
%       \kern-\rightmargin\relax
%     }^^A
%   }^^A
%   \ifdim\wd0>\lw
%     \sbox0{\small\t}^^A
%     \ifdim\wd0>\linewidth
%       \ifdim\wd0>\lw
%         \sbox0{\footnotesize\t}^^A
%         \ifdim\wd0>\linewidth
%           \ifdim\wd0>\lw
%             \sbox0{\scriptsize\t}^^A
%             \ifdim\wd0>\linewidth
%               \ifdim\wd0>\lw
%                 \sbox0{\tiny\t}^^A
%                 \ifdim\wd0>\linewidth
%                   \lwbox
%                 \else
%                   \usebox0
%                 \fi
%               \else
%                 \lwbox
%               \fi
%             \else
%               \usebox0
%             \fi
%           \else
%             \lwbox
%           \fi
%         \else
%           \usebox0
%         \fi
%       \else
%         \lwbox
%       \fi
%     \else
%       \usebox0
%     \fi
%   \else
%     \lwbox
%   \fi
% \else
%   \usebox0
% \fi
% \end{quote}
% If you have a \xfile{docstrip.cfg} that configures and enables \docstrip's
% TDS installing feature, then some files can already be in the right
% place, see the documentation of \docstrip.
%
% \subsection{Refresh file name databases}
%
% If your \TeX~distribution
% (\teTeX, \mikTeX, \dots) relies on file name databases, you must refresh
% these. For example, \teTeX\ users run \verb|texhash| or
% \verb|mktexlsr|.
%
% \subsection{Some details for the interested}
%
% \paragraph{Attached source.}
%
% The PDF documentation on CTAN also includes the
% \xfile{.dtx} source file. It can be extracted by
% AcrobatReader 6 or higher. Another option is \textsf{pdftk},
% e.g. unpack the file into the current directory:
% \begin{quote}
%   \verb|pdftk pdfcolmk.pdf unpack_files output .|
% \end{quote}
%
% \paragraph{Unpacking with \LaTeX.}
% The \xfile{.dtx} chooses its action depending on the format:
% \begin{description}
% \item[\plainTeX:] Run \docstrip\ and extract the files.
% \item[\LaTeX:] Generate the documentation.
% \end{description}
% If you insist on using \LaTeX\ for \docstrip\ (really,
% \docstrip\ does not need \LaTeX), then inform the autodetect routine
% about your intention:
% \begin{quote}
%   \verb|latex \let\install=y% \iffalse meta-comment
%
% File: pdfcolmk.dtx
% Version: 2008/08/11 v1.2
% Info: Color support for pdfTeX via marks
%
% Copyright (C) 2000, 2005-2008 by
%    Heiko Oberdiek <heiko.oberdiek at googlemail.com>
%
% This work may be distributed and/or modified under the
% conditions of the LaTeX Project Public License, either
% version 1.3c of this license or (at your option) any later
% version. This version of this license is in
%    http://www.latex-project.org/lppl/lppl-1-3c.txt
% and the latest version of this license is in
%    http://www.latex-project.org/lppl.txt
% and version 1.3 or later is part of all distributions of
% LaTeX version 2005/12/01 or later.
%
% This work has the LPPL maintenance status "maintained".
%
% This Current Maintainer of this work is Heiko Oberdiek.
%
% This work consists of the main source file pdfcolmk.dtx
% and the derived files
%    pdfcolmk.sty, pdfcolmk.pdf, pdfcolmk.ins, pdfcolmk.drv.
%
% Distribution:
%    CTAN:macros/latex/contrib/oberdiek/pdfcolmk.dtx
%    CTAN:macros/latex/contrib/oberdiek/pdfcolmk.pdf
%
% Unpacking:
%    (a) If pdfcolmk.ins is present:
%           tex pdfcolmk.ins
%    (b) Without pdfcolmk.ins:
%           tex pdfcolmk.dtx
%    (c) If you insist on using LaTeX
%           latex \let\install=y\input{pdfcolmk.dtx}
%        (quote the arguments according to the demands of your shell)
%
% Documentation:
%    (a) If pdfcolmk.drv is present:
%           latex pdfcolmk.drv
%    (b) Without pdfcolmk.drv:
%           latex pdfcolmk.dtx; ...
%    The class ltxdoc loads the configuration file ltxdoc.cfg
%    if available. Here you can specify further options, e.g.
%    use A4 as paper format:
%       \PassOptionsToClass{a4paper}{article}
%
%    Programm calls to get the documentation (example):
%       pdflatex pdfcolmk.dtx
%       makeindex -s gind.ist pdfcolmk.idx
%       pdflatex pdfcolmk.dtx
%       makeindex -s gind.ist pdfcolmk.idx
%       pdflatex pdfcolmk.dtx
%
% Installation:
%    TDS:tex/latex/oberdiek/pdfcolmk.sty
%    TDS:doc/latex/oberdiek/pdfcolmk.pdf
%    TDS:source/latex/oberdiek/pdfcolmk.dtx
%
%<*ignore>
\begingroup
  \catcode123=1 %
  \catcode125=2 %
  \def\x{LaTeX2e}%
\expandafter\endgroup
\ifcase 0\ifx\install y1\fi\expandafter
         \ifx\csname processbatchFile\endcsname\relax\else1\fi
         \ifx\fmtname\x\else 1\fi\relax
\else\csname fi\endcsname
%</ignore>
%<*install>
\input docstrip.tex
\Msg{************************************************************************}
\Msg{* Installation}
\Msg{* Package: pdfcolmk 2008/08/11 v1.2 Color support for pdfTeX via marks (HO)}
\Msg{************************************************************************}

\keepsilent
\askforoverwritefalse

\let\MetaPrefix\relax
\preamble

This is a generated file.

Project: pdfcolmk
Version: 2008/08/11 v1.2

Copyright (C) 2000, 2005-2008 by
   Heiko Oberdiek <heiko.oberdiek at googlemail.com>

This work may be distributed and/or modified under the
conditions of the LaTeX Project Public License, either
version 1.3c of this license or (at your option) any later
version. This version of this license is in
   http://www.latex-project.org/lppl/lppl-1-3c.txt
and the latest version of this license is in
   http://www.latex-project.org/lppl.txt
and version 1.3 or later is part of all distributions of
LaTeX version 2005/12/01 or later.

This work has the LPPL maintenance status "maintained".

This Current Maintainer of this work is Heiko Oberdiek.

This work consists of the main source file pdfcolmk.dtx
and the derived files
   pdfcolmk.sty, pdfcolmk.pdf, pdfcolmk.ins, pdfcolmk.drv.

\endpreamble
\let\MetaPrefix\DoubleperCent

\generate{%
  \file{pdfcolmk.ins}{\from{pdfcolmk.dtx}{install}}%
  \file{pdfcolmk.drv}{\from{pdfcolmk.dtx}{driver}}%
  \usedir{tex/latex/oberdiek}%
  \file{pdfcolmk.sty}{\from{pdfcolmk.dtx}{package}}%
  \nopreamble
  \nopostamble
  \usedir{source/latex/oberdiek/catalogue}%
  \file{pdfcolmk.xml}{\from{pdfcolmk.dtx}{catalogue}}%
}

\catcode32=13\relax% active space
\let =\space%
\Msg{************************************************************************}
\Msg{*}
\Msg{* To finish the installation you have to move the following}
\Msg{* file into a directory searched by TeX:}
\Msg{*}
\Msg{*     pdfcolmk.sty}
\Msg{*}
\Msg{* To produce the documentation run the file `pdfcolmk.drv'}
\Msg{* through LaTeX.}
\Msg{*}
\Msg{* Happy TeXing!}
\Msg{*}
\Msg{************************************************************************}

\endbatchfile
%</install>
%<*ignore>
\fi
%</ignore>
%<*driver>
\NeedsTeXFormat{LaTeX2e}
\ProvidesFile{pdfcolmk.drv}%
  [2008/08/11 v1.2 Color support for pdfTeX via marks (HO)]%
\documentclass{ltxdoc}
\usepackage{holtxdoc}[2011/11/22]
\begin{document}
  \DocInput{pdfcolmk.dtx}%
\end{document}
%</driver>
% \fi
%
% \CheckSum{843}
%
% \CharacterTable
%  {Upper-case    \A\B\C\D\E\F\G\H\I\J\K\L\M\N\O\P\Q\R\S\T\U\V\W\X\Y\Z
%   Lower-case    \a\b\c\d\e\f\g\h\i\j\k\l\m\n\o\p\q\r\s\t\u\v\w\x\y\z
%   Digits        \0\1\2\3\4\5\6\7\8\9
%   Exclamation   \!     Double quote  \"     Hash (number) \#
%   Dollar        \$     Percent       \%     Ampersand     \&
%   Acute accent  \'     Left paren    \(     Right paren   \)
%   Asterisk      \*     Plus          \+     Comma         \,
%   Minus         \-     Point         \.     Solidus       \/
%   Colon         \:     Semicolon     \;     Less than     \<
%   Equals        \=     Greater than  \>     Question mark \?
%   Commercial at \@     Left bracket  \[     Backslash     \\
%   Right bracket \]     Circumflex    \^     Underscore    \_
%   Grave accent  \`     Left brace    \{     Vertical bar  \|
%   Right brace   \}     Tilde         \~}
%
% \GetFileInfo{pdfcolmk.drv}
%
% \title{The \xpackage{pdfcolmk} package}
% \date{2008/08/11 v1.2}
% \author{Heiko Oberdiek\\\xemail{heiko.oberdiek at googlemail.com}}
%
% \maketitle
%
% \begin{abstract}
% This package tries a solution for the missing color
% stack of \pdfTeX.
% \end{abstract}
%
% \tableofcontents
%
% \section{Documentation}
%
% \subsection{Introduction}
%
% This package uses a mark register in order to solve the
% problem of a missing color stack of \pdfTeX\ prior 1.40.0.
% Since this version of \pdfTeX\ a color stack is available
% and supported by \xfile{pdftex.def} 2007/01/01 v0.04a.
% In this case this package is obsolete and the package
% stops its loading.
%
% \subsection{Background}
%
% After the Dante meeting (Clausthal 2000) I have started
% to experiment with the eTeX method of a \emph{colour} mark.
% One of the major problems is the understanding of the
% output routine and the need to rewrite it because of
% missing hooks. Currently I have made some tests in
% in onecolumn and twocolumn mode, but the state is
% experimental.
%
% \subsection{Limitations}
%
% \begin{itemize}
% \item Mark limitations: page breaks in math.
% \item \LaTeX's output routine is redefinded.
%   \begin{itemize}
%   \item Changes in the output routine of newer versions
%         of LaTeX are not detected.
%   \item Packages that change the output routine are not
%         supported.
%   \end{itemize}
% \item It does not support several independent text
%       streams like footnotes.
% \item Limitations in float and marginpar support.
% \end{itemize}
%
% \subsection{Recommendation}
%
% \eTeX\ (for additional mark register)
% Without \eTeX\ \LaTeX's mark commands are redefined
% to store an additional color value.
%
% \subsection{Usage}
%
% Load after package color:
% \begin{quote}
%   |\usepackage[pdftex]{color}|\\
%   |\usepackage{pdfcolmk}|
% \end{quote}
%
% \subsection{Compatibility}
%
% \begin{itemize}
% \item Load the following packages after \xpackage{pdfcolmk}:
%   \begin{quote}
%       \xpackage{mparhack.sty}
%   \end{quote}
% \item Load the following packages before \xpackage{pdfcolmk}:
%   \begin{quote}
%       \xpackage{marn.sty}\\
%       \xpackage{newmarn.sty}
%   \end{quote}
% \item Supported \cs{@addmarginpar} patch:
%   \begin{quote}
%       \xpackage{latex/base/latex.ltx}\\
%       \xpackage{memoir.cls}\\
%       \xpackage{poemscol/marn.sty}, \xpackage{poemscol/newmarn.sty}\\
%       \xpackage{mparhack.sty}
%   \end{quote}
% \item Unsupported \cs{@addmarginpar} patch:
%   \begin{quote}
%       \xpackage{lineno.sty}\\
%       \xpackage{sttools/marginal.sty}\\
%       \xpackage{revtex4.cls}
%   \end{quote}
% \end{itemize}
%
% \StopEventually{
% }
%
% \section{Implementation}
%
%    \begin{macrocode}
%<*package>
%    \end{macrocode}
%    Package identification.
%    \begin{macrocode}
\NeedsTeXFormat{LaTeX2e}
\ProvidesPackage{pdfcolmk}%
  [2008/08/11 v1.2 Color support for pdfTeX via marks (HO)]
%    \end{macrocode}
%
%    \begin{macrocode}
\@ifundefined{ver@pdftex.def}{%
  \PackageWarningNoLine{pdfcolmk}{%
    Nothing to fix, because \string`pdftex.def\string' not loaded%
  }%
  \endinput
}{}
\@ifpackageloaded{color}{}{%
  \PackageWarningNoLine{pdfcolmk}{%
    Nothing to fix, because \string`color.sty\string' not loaded%
  }%
  \endinput
}
\begingroup\expandafter\expandafter\expandafter\endgroup
\expandafter\ifx\csname main@pdfcolorstack\endcsname\relax
\else
  % pdftex.def >= 2007/01/01 0.04a and pdfTeX >= 1.40.0
  \begingroup
    \let\on@line\@empty
    \PackageInfo{pdfcolmk}{%
      The color stack of pdfTeX \string>\string= 1.40 is used. %
      Therefore\MessageBreak
      this package is not necessary and not loaded%
    }%
  \endgroup
  \expandafter\endinput
\fi

\PackageInfo{pdfcolmk}{%
  This package tries to simulate dvips's color stack\MessageBreak
  for pdfTeX based on a mark register of e-TeX.\MessageBreak
  It redefines LaTeX's output routine. Therefore\MessageBreak
  use with care, no warranties%
}

\ifx\marks\@undefined

  \let\pec@mark\mark
  \let\pec@value\empty
  \long\def\mark#1{%
    \protected@xdef\pec@value{#1}%
    \pec@setmark
  }%
  \def\pec@setmark{%
    \begingroup
      \@temptokena\expandafter{\pec@value}%
      \pec@mark{{\current@color}\the\@temptokena}%
    \endgroup
  }%
  \def\pec@getmark{%
    \xdef\pec@botcolor{%
      \expandafter\@firstofthree\botmark\@empty\@empty\@empty
    }%
  }%
  \long\def\@firstofthree#1#2#3{#1}%
  \CheckCommand{\@leftmark}[2]{#1}%
  \CheckCommand{\@rightmark}[2]{#2}%
  \CheckCommand*{\leftmark}{%
    \expandafter\@leftmark\botmark\@empty\@empty
  }%
  \CheckCommand*{\rightmark}{%
    \expandafter\@rightmark\firstmark\@empty\@empty
  }%
  \long\def\@leftmark#1#2#3{#2}%
  \long\def\@rightmark#1#2#3{#3}%
  \g@addto@macro\leftmark\@empty
  \g@addto@macro\rightmark\@empty

\else

  \RequirePackage{etex}[1998/03/26]%
  \newmarks\pec@marks
  \def\pec@setmark{\marks\pec@marks{\current@color}}%
  \def\pec@getmark{\xdef\pec@botcolor{\botmarks\pec@marks}}%

\fi
%    \end{macrocode}
%
% \subsection{\cs{marginpar} fix}
%
%    \begin{macrocode}
\chardef\pec@result\z@
\def\pec@temp#1{%
  \chardef\pec@result\@ne
  \begingroup
    \let\on@line\@empty
    \PackageInfo{pdfcolmk}{%
      Patch for \string\@addmarginpar\space applied (#1)%
    }%
  \endgroup
}
%    \end{macrocode}
%
% \subsubsection{latex/base/latex.ltx}
%
%    \begin{macrocode}
\def\pec@addmarginpar{%
  \@next\@marbox\@currlist{%
    \@cons\@freelist\@marbox
    \@cons\@freelist\@currbox
  }\@latexbug
  \@tempcnta\@ne
  \if@twocolumn
    \if@firstcolumn
      \@tempcnta\m@ne
    \fi
  \else
    \if@mparswitch
      \ifodd\c@page
      \else
        \@tempcnta\m@ne
      \fi
    \fi
    \if@reversemargin \@tempcnta -\@tempcnta \fi
  \fi
  \ifnum\@tempcnta <\z@  \global\setbox\@marbox\box\@currbox \fi
  \@tempdima\@mparbottom
  \advance\@tempdima -\@pageht
  \advance\@tempdima\ht\@marbox
  \ifdim\@tempdima >\z@
    \@latex@warning@no@line{Marginpar on page \thepage\space moved}%
  \else
    \@tempdima\z@
  \fi
  \global\@mparbottom\@pageht
  \global\advance\@mparbottom\@tempdima
  \global\advance\@mparbottom\dp\@marbox
  \global\advance\@mparbottom\marginparpush
  \advance\@tempdima -\ht\@marbox
  \global\setbox\@marbox\vbox{%
    \vskip \@tempdima
    \box \@marbox
  }%
  \global \ht\@marbox \z@
  \global \dp\@marbox \z@
  \kern -\@pagedp
  \nointerlineskip
  \hb@xt@\columnwidth{%
    \ifnum \@tempcnta >\z@
      \hskip\columnwidth
      \hskip\marginparsep
    \else
      \hskip -\marginparsep
      \hskip -\marginparwidth
    \fi
    \box\@marbox \hss
  }%
  \nointerlineskip
  \hbox{\vrule \@height\z@ \@width\z@ \@depth\@pagedp}%
}
\ifx\pec@addmarginpar\@addmarginpar
  \pec@temp{latex/base}%
\fi
%    \end{macrocode}
%
% \subsubsection{memoir.cls}
%
%    \begin{macrocode}
\def\pec@addmarginpar{%
  \checkoddpage
  \@next\@marbox\@currlist{%
    \@cons\@freelist\@marbox
    \@cons\@freelist\@currbox
  }\@latexbug
  \@tempcnta\@ne
  \if@twocolumn
    \if@firstcolumn
      \@tempcnta\m@ne
    \fi
  \else
    \if@mparswitch
      \ifoddpage
      \else
        \@tempcnta\m@ne
      \fi
    \fi
    \if@reversemargin
      \@tempcnta -\@tempcnta
    \fi
  \fi
  \ifnum\@tempcnta <\z@
    \global\setbox\@marbox\box\@currbox
  \fi
  \@tempdima\@mparbottom
  \advance\@tempdima -\@pageht
  \advance\@tempdima\ht\@marbox
  \ifdim\@tempdima >\z@
    \@latex@warning@no@line{%
      Marginpar on page \thepage\space moved by \the\@tempdima
    }%
  \else
    \@tempdima\z@
  \fi
  \global\@mparbottom\@pageht
  \global\advance\@mparbottom\@tempdima
  \global\advance\@mparbottom\dp\@marbox
  \global\advance\@mparbottom\marginparpush
  \advance\@tempdima -\ht\@marbox
  \global\setbox\@marbox\vbox{%
    \vskip \@tempdima
    \box \@marbox
  }%
  \global \ht\@marbox \z@
  \global \dp\@marbox \z@
  \kern -\@pagedp
  \nointerlineskip
  \hb@xt@\columnwidth{%
    \ifnum \@tempcnta >\z@
      \hskip\columnwidth
      \hskip\marginparsep
    \else
      \hskip -\marginparsep
      \hskip -\marginparwidth
    \fi
    \box\@marbox
    \hss
  }%
  \nointerlineskip
  \hbox{\vrule \@height\z@ \@width\z@ \@depth\@pagedp}%
}%
\ifx\pec@addmarginpar\@addmarginpar
  \pec@temp{memoir.cls}%
\fi
%    \end{macrocode}
%
% \subsubsection{poemscol/marn.sty, poemscol/newmarn.sty}
%
%    \begin{macrocode}
\def\pec@addmarginpar{%
  \@next \@marbox\@currlist{%
    \@cons\@freelist\@marbox
    \@cons\@freelist\@currbox
  }\@latexbug
  \global\advance\@mpar@count\m@ne
  \@ifundefined{@marn@\the\@mpar@count @}{% was location logged last time?
    \@tempcnta\@ne % NO: use original LaTeX logic
    \if@twocolumn
      \if@firstcolumn
        \@tempcnta\m@ne
      \fi
    \else
      \if@mparswitch
        \ifodd\c@page
        \else
          \@tempcnta\m@ne
        \fi
      \fi
      \if@reversemargin
        \@tempcnta -\@tempcnta
      \fi
    \fi
  }{%
    \@tempcnta %    YES: use record from last time to decide side.
    \@nameuse{@marn@\the\@mpar@count @}%
    \if@reversemargin -\fi \@ne
  }%
  \ifnum\@tempcnta <\z@
    \global\setbox\@marbox\box\@currbox
    \global\let\@marnbottom\@mparbottoml
  \else
    \global\let\@marnbottom\@mparbottom
  \fi
  \@tempdima\@marnbottom \advance\@tempdima -\@pageht
  \advance\@tempdima\ht\@marbox
  \ifdim\@tempdima >\z@
    \@@warning{Marginpar on page \thepage\space moved}%
  \else
    \@tempdima\z@
  \fi
  \global\@marnbottom\@pageht
  \global\advance\@marnbottom\@tempdima
  \global\advance\@marnbottom\dp\@marbox
  \global\advance\@marnbottom\marginparpush
  \advance\@tempdima -\ht\@marbox
  \global\ht\@marbox\z@
  \global\dp\@marbox\z@
  \vskip -\@pagedp
  \vskip\@tempdima\nointerlineskip
  \hbox to\columnwidth{%
    \ifnum \@tempcnta >\z@
      \hskip\columnwidth
      \hskip\marginparsep
    \else
      \hskip -\marginparsep
      \hskip -\marginparwidth
    \fi
    \if@filesw % record where this is for use next time:
       \@marn@log\@mpar@count
    \fi
    \box\@marbox
    \hss
  }%
  \nobreak   %% RmS 91/06/21 \nobreak added
  \vskip -\@tempdima
  \nointerlineskip
  \hbox{\vrule \@height\z@ \@width\z@ \@depth\@pagedp}%
}
\ifx\pec@addmarginpar\@addmarginpar
  \pec@temp{poemscol/(new)marn.sty}%
\fi
%    \end{macrocode}
%
% \subsubsection{refman/refart.cls, refnam/refrep.cls}
%
%    \begin{macrocode}
\def\pec@addmarginpar{%
  \@next\@marbox\@currlist{%
    \@cons\@freelist\@marbox
    \@cons\@freelist\@currbox
  }\@latexbug
  \@tempcnta\@ne
  \if@twocolumn
    \if@firstcolumn
      \@tempcnta\m@ne
    \fi
  \else
    \@tempcnta\m@ne
  \fi
  \ifnum\@tempcnta <\z@
    \global\setbox\@marbox\box\@currbox
  \fi
  \@tempdima\@mparbottom
  \advance\@tempdima -\@pageht
  \advance\@tempdima\ht\@marbox
  \ifdim\@tempdima >\z@
     \@@warning{Marginpar on page \thepage\space moved}%
  \else
     \@tempdima\z@
  \fi
  \global\@mparbottom\@pageht
  \global\advance\@mparbottom\@tempdima
  \global\advance\@mparbottom\dp\@marbox
  \global\advance\@mparbottom\marginparpush
  \advance\@tempdima -\ht\@marbox
  \global\setbox\@marbox\vbox{%
    \vskip \@tempdima \box \@marbox
  }%
  \global \ht\@marbox \z@
  \global \dp\@marbox \z@
  \kern -\@pagedp
  \nointerlineskip
  \hb@xt@\columnwidth{%
    \ifnum \@tempcnta >\z@
      \hskip\columnwidth
      \hskip\marginparsep
    \else
      \hskip -\marginparsep
      \hskip -\marginparwidth
    \fi
    \box\@marbox
    \hss
  }%
  \nointerlineskip
  \hbox{\vrule \@height\z@ \@width\z@ \@depth\@pagedp}%
}
\ifx\pec@addmarginpar\@addmarginpar
  \pec@temp{ref(art|rep).cls}%
\fi

\ifcase\pec@result
  \PackageInfo{pdfcolmk}{%
    Fix for \string\@addmarginpar\space is omitted, %
    because this variant\MessageBreak
    of \string\@addmarginpar\space
      is not recognized%
  }%
\else
  % apply patch for \@addmarginpar
  \def\pec@PatchAddMarginpar#1\columnwidth#2#3\@nil{%
    \pec@PatchAddMarginparI#2\@nil{#1}{#3}%
  }%
  \def\pec@PatchAddMarginparI#1\box\@marbox\hss#2\@nil#3#4{%
    \def\@addmarginpar{%
      #3%
      \columnwidth{%
        #1%
        \pdfliteral{q}%
        \rlap{%
          \box\@marbox
        }%
        \pdfliteral{Q}%
        \hss
        #2%
      }%
      #4%
    }%
  }%
  \expandafter\pec@PatchAddMarginpar\@addmarginpar\@nil
\fi
%    \end{macrocode}
%
% \subsection{Color fix}
%
%    \begin{macrocode}
\def\set@color{%
  \pdfliteral{\current@color}%
  \ifinner
  \else
    \pec@setmark
  \fi
  \aftergroup\reset@color
}
\def\reset@color{%
  \pdfliteral{\current@color}%
  \ifinner
  \else
    \pec@setmark
  \fi
}

\let\pec@botcolor\current@color

\def\pec@PatchVBoxCCLV{%
  \ifx\pec@botcolor\@empty
  \else
    \setbox\@cclv\vbox{%
      \pdfliteral{\pec@botcolor}%
      \unvbox\@cclv
    }%
  \fi
  \pec@getmark
}

\def\pec@PatchAlreadyInBox{%
  \ifx\pec@botcolor\@empty
  \else
    \pdfliteral{\pec@botcolor}%
  \fi
  \pec@getmark
}

\@ifclassloaded{memoir}{%
  \expandafter\def\expandafter\mem@makecol\expandafter{%
    \expandafter\pec@PatchVBoxCCLV
    \mem@makecol
  }%
  \endinput
}{}

\@ifclassloaded{seminar}{%
  \newcommand\pec@org@makeslide{}%
  \let\pec@org@makeslide\@makeslide
  \def\@makeslide{%
    \pec@PatchVBoxCCLV
    \pec@org@makeslide
  }%
  \endinput
}{}

\long\def\pec@output#1\@specialoutput\else#2\pec@end{%
  \begingroup
    \def\x{#2}%
  \expandafter\endgroup
  \ifx\x\@empty
    \PackageWarningNoLine{pdfcolmk}{%
      Unexpected \string\output\space routine detected,%
      \MessageBreak
      loading of package stopped%
    }%
    \expandafter\endinput
  \fi
}
\expandafter\expandafter\expandafter\pec@output
\expandafter\@firstofone\the\output\@specialoutput\else\pec@end

\long\def\pec@output#1\@specialoutput\else#2\pec@end{%
  \output{%
    #1\@specialoutput\else
    \pec@PatchVBoxCCLV
    #2%
  }%
}
\expandafter\expandafter\expandafter\pec@output
\expandafter\@firstofone\the\output\pec@end
%    \end{macrocode}
%
%    \begin{macrocode}
%</package>
%    \end{macrocode}
%
% \section{Installation}
%
% \subsection{Download}
%
% \paragraph{Package.} This package is available on
% CTAN\footnote{\url{ftp://ftp.ctan.org/tex-archive/}}:
% \begin{description}
% \item[\CTAN{macros/latex/contrib/oberdiek/pdfcolmk.dtx}] The source file.
% \item[\CTAN{macros/latex/contrib/oberdiek/pdfcolmk.pdf}] Documentation.
% \end{description}
%
%
% \paragraph{Bundle.} All the packages of the bundle `oberdiek'
% are also available in a TDS compliant ZIP archive. There
% the packages are already unpacked and the documentation files
% are generated. The files and directories obey the TDS standard.
% \begin{description}
% \item[\CTAN{install/macros/latex/contrib/oberdiek.tds.zip}]
% \end{description}
% \emph{TDS} refers to the standard ``A Directory Structure
% for \TeX\ Files'' (\CTAN{tds/tds.pdf}). Directories
% with \xfile{texmf} in their name are usually organized this way.
%
% \subsection{Bundle installation}
%
% \paragraph{Unpacking.} Unpack the \xfile{oberdiek.tds.zip} in the
% TDS tree (also known as \xfile{texmf} tree) of your choice.
% Example (linux):
% \begin{quote}
%   |unzip oberdiek.tds.zip -d ~/texmf|
% \end{quote}
%
% \paragraph{Script installation.}
% Check the directory \xfile{TDS:scripts/oberdiek/} for
% scripts that need further installation steps.
% Package \xpackage{attachfile2} comes with the Perl script
% \xfile{pdfatfi.pl} that should be installed in such a way
% that it can be called as \texttt{pdfatfi}.
% Example (linux):
% \begin{quote}
%   |chmod +x scripts/oberdiek/pdfatfi.pl|\\
%   |cp scripts/oberdiek/pdfatfi.pl /usr/local/bin/|
% \end{quote}
%
% \subsection{Package installation}
%
% \paragraph{Unpacking.} The \xfile{.dtx} file is a self-extracting
% \docstrip\ archive. The files are extracted by running the
% \xfile{.dtx} through \plainTeX:
% \begin{quote}
%   \verb|tex pdfcolmk.dtx|
% \end{quote}
%
% \paragraph{TDS.} Now the different files must be moved into
% the different directories in your installation TDS tree
% (also known as \xfile{texmf} tree):
% \begin{quote}
% \def\t{^^A
% \begin{tabular}{@{}>{\ttfamily}l@{ $\rightarrow$ }>{\ttfamily}l@{}}
%   pdfcolmk.sty & tex/latex/oberdiek/pdfcolmk.sty\\
%   pdfcolmk.pdf & doc/latex/oberdiek/pdfcolmk.pdf\\
%   pdfcolmk.dtx & source/latex/oberdiek/pdfcolmk.dtx\\
% \end{tabular}^^A
% }^^A
% \sbox0{\t}^^A
% \ifdim\wd0>\linewidth
%   \begingroup
%     \advance\linewidth by\leftmargin
%     \advance\linewidth by\rightmargin
%   \edef\x{\endgroup
%     \def\noexpand\lw{\the\linewidth}^^A
%   }\x
%   \def\lwbox{^^A
%     \leavevmode
%     \hbox to \linewidth{^^A
%       \kern-\leftmargin\relax
%       \hss
%       \usebox0
%       \hss
%       \kern-\rightmargin\relax
%     }^^A
%   }^^A
%   \ifdim\wd0>\lw
%     \sbox0{\small\t}^^A
%     \ifdim\wd0>\linewidth
%       \ifdim\wd0>\lw
%         \sbox0{\footnotesize\t}^^A
%         \ifdim\wd0>\linewidth
%           \ifdim\wd0>\lw
%             \sbox0{\scriptsize\t}^^A
%             \ifdim\wd0>\linewidth
%               \ifdim\wd0>\lw
%                 \sbox0{\tiny\t}^^A
%                 \ifdim\wd0>\linewidth
%                   \lwbox
%                 \else
%                   \usebox0
%                 \fi
%               \else
%                 \lwbox
%               \fi
%             \else
%               \usebox0
%             \fi
%           \else
%             \lwbox
%           \fi
%         \else
%           \usebox0
%         \fi
%       \else
%         \lwbox
%       \fi
%     \else
%       \usebox0
%     \fi
%   \else
%     \lwbox
%   \fi
% \else
%   \usebox0
% \fi
% \end{quote}
% If you have a \xfile{docstrip.cfg} that configures and enables \docstrip's
% TDS installing feature, then some files can already be in the right
% place, see the documentation of \docstrip.
%
% \subsection{Refresh file name databases}
%
% If your \TeX~distribution
% (\teTeX, \mikTeX, \dots) relies on file name databases, you must refresh
% these. For example, \teTeX\ users run \verb|texhash| or
% \verb|mktexlsr|.
%
% \subsection{Some details for the interested}
%
% \paragraph{Attached source.}
%
% The PDF documentation on CTAN also includes the
% \xfile{.dtx} source file. It can be extracted by
% AcrobatReader 6 or higher. Another option is \textsf{pdftk},
% e.g. unpack the file into the current directory:
% \begin{quote}
%   \verb|pdftk pdfcolmk.pdf unpack_files output .|
% \end{quote}
%
% \paragraph{Unpacking with \LaTeX.}
% The \xfile{.dtx} chooses its action depending on the format:
% \begin{description}
% \item[\plainTeX:] Run \docstrip\ and extract the files.
% \item[\LaTeX:] Generate the documentation.
% \end{description}
% If you insist on using \LaTeX\ for \docstrip\ (really,
% \docstrip\ does not need \LaTeX), then inform the autodetect routine
% about your intention:
% \begin{quote}
%   \verb|latex \let\install=y\input{pdfcolmk.dtx}|
% \end{quote}
% Do not forget to quote the argument according to the demands
% of your shell.
%
% \paragraph{Generating the documentation.}
% You can use both the \xfile{.dtx} or the \xfile{.drv} to generate
% the documentation. The process can be configured by the
% configuration file \xfile{ltxdoc.cfg}. For instance, put this
% line into this file, if you want to have A4 as paper format:
% \begin{quote}
%   \verb|\PassOptionsToClass{a4paper}{article}|
% \end{quote}
% An example follows how to generate the
% documentation with pdf\LaTeX:
% \begin{quote}
%\begin{verbatim}
%pdflatex pdfcolmk.dtx
%makeindex -s gind.ist pdfcolmk.idx
%pdflatex pdfcolmk.dtx
%makeindex -s gind.ist pdfcolmk.idx
%pdflatex pdfcolmk.dtx
%\end{verbatim}
% \end{quote}
%
% \section{Catalogue}
%
% The following XML file can be used as source for the
% \href{http://mirror.ctan.org/help/Catalogue/catalogue.html}{\TeX\ Catalogue}.
% The elements \texttt{caption} and \texttt{description} are imported
% from the original XML file from the Catalogue.
% The name of the XML file in the Catalogue is \xfile{pdfcolmk.xml}.
%    \begin{macrocode}
%<*catalogue>
<?xml version='1.0' encoding='us-ascii'?>
<!DOCTYPE entry SYSTEM 'catalogue.dtd'>
<entry datestamp='$Date$' modifier='$Author$' id='pdfcolmk'>
  <name>pdfcolmk</name>
  <caption>Improving colour support under pdftex.</caption>
  <authorref id='auth:oberdiek'/>
  <copyright owner='Heiko Oberdiek' year='2000,2005-2008'/>
  <license type='lppl1.3'/>
  <version number='1.2'/>
  <description>
    The package provides macros that emulate the &#x2018;colour stack&#x2019;
    functionality of dvips.  The colour stack deals with colour
    manipulations when asynchronous events (like page-breaking) occur;
    pdftex does not (yet) have such a stack, but dvips does, and the
    <xref refid='color'>color</xref> package makes extensive use of
    it.
    <p/>
    This package is an experimental solution to the problem, and works
    best with pdf-e-tex.
    <p/>
    The package is part of the <xref refid='oberdiek'>oberdiek</xref> bundle.
  </description>
  <documentation details='Package documentation'
      href='ctan:/macros/latex/contrib/oberdiek/pdfcolmk.pdf'/>
  <ctan file='true' path='/macros/latex/contrib/oberdiek/pdfcolmk.dtx'/>
  <miktex location='oberdiek'/>
  <texlive location='oberdiek'/>
  <install path='/macros/latex/contrib/oberdiek/oberdiek.tds.zip'/>
</entry>
%</catalogue>
%    \end{macrocode}
%
% \begin{History}
%   \begin{Version}{2000/08/27 v0.1}
%   \item
%     First published version in newsgroup \xnewsgroup{comp.text.tex}:\\
%     \URL{``\link{pdftex: bug with colors?}''}^^A
%     {http://groups.google.com/group/comp.text.tex/msg/6f088e69e4085d2c}
%   \end{Version}
%   \begin{Version}{2000/09/02 v0.2}
%   \item
%     Next try.
%   \end{Version}
%   \begin{Version}{2000/09/02 v0.3}
%   \item
%     Solution without \eTeX\ added.
%   \end{Version}
%   \begin{Version}{2000/09/06 v0.4}
%   \item
%     Patch commands added.
%   \item
%     Patch for seminar.cls added.
%   \end{Version}
%   \begin{Version}{2000/09/06 v0.5}
%   \item
%     Bug fix: initialization of \cs{pec@value} added.
%   \end{Version}
%   \begin{Version}{2005/06/15 v0.6}
%   \item
%     Support for \cs{marginpar} added.
%     See thread in \xnewsgroup{comp.text.tex}:\\
%     \URL{``\link{Using \cs{textcolor} and \cs{marginpar} together}''}^^A
%     {http://groups.google.com/group/comp.text.tex/msg/38ed58f8845a2a4f}
%   \end{Version}
%   \begin{Version}{2005/07/09 v0.7}
%   \item
%     Output support added for \xpackage{memoir},
%     provided by Lars Madsen.
%   \end{Version}
%   \begin{Version}{2006/02/20 v0.8}
%   \item
%     Code is not changed.
%   \item
%     DTX framework.
%   \end{Version}
%   \begin{Version}{2007/01/01 v1.0}
%   \item
%     If \xfile{pdftex.def} \textgreater= 2007/01/01 v0.04a is used with
%     \pdfTeX\ \textgreater= 1.40.0, then package \xpackage{pdfcolmk} is obsolete.
%   \end{Version}
%   \begin{Version}{2007/04/11 v1.1}
%   \item
%     Line ends sanitized.
%   \end{Version}
%   \begin{Version}{2008/08/11 v1.2}
%   \item
%     Code is not changed.
%   \item
%     URLs updated.
%   \end{Version}
% \end{History}
%
% \PrintIndex
%
% \Finale
\endinput
|
% \end{quote}
% Do not forget to quote the argument according to the demands
% of your shell.
%
% \paragraph{Generating the documentation.}
% You can use both the \xfile{.dtx} or the \xfile{.drv} to generate
% the documentation. The process can be configured by the
% configuration file \xfile{ltxdoc.cfg}. For instance, put this
% line into this file, if you want to have A4 as paper format:
% \begin{quote}
%   \verb|\PassOptionsToClass{a4paper}{article}|
% \end{quote}
% An example follows how to generate the
% documentation with pdf\LaTeX:
% \begin{quote}
%\begin{verbatim}
%pdflatex pdfcolmk.dtx
%makeindex -s gind.ist pdfcolmk.idx
%pdflatex pdfcolmk.dtx
%makeindex -s gind.ist pdfcolmk.idx
%pdflatex pdfcolmk.dtx
%\end{verbatim}
% \end{quote}
%
% \section{Catalogue}
%
% The following XML file can be used as source for the
% \href{http://mirror.ctan.org/help/Catalogue/catalogue.html}{\TeX\ Catalogue}.
% The elements \texttt{caption} and \texttt{description} are imported
% from the original XML file from the Catalogue.
% The name of the XML file in the Catalogue is \xfile{pdfcolmk.xml}.
%    \begin{macrocode}
%<*catalogue>
<?xml version='1.0' encoding='us-ascii'?>
<!DOCTYPE entry SYSTEM 'catalogue.dtd'>
<entry datestamp='$Date$' modifier='$Author$' id='pdfcolmk'>
  <name>pdfcolmk</name>
  <caption>Improving colour support under pdftex.</caption>
  <authorref id='auth:oberdiek'/>
  <copyright owner='Heiko Oberdiek' year='2000,2005-2008'/>
  <license type='lppl1.3'/>
  <version number='1.2'/>
  <description>
    The package provides macros that emulate the &#x2018;colour stack&#x2019;
    functionality of dvips.  The colour stack deals with colour
    manipulations when asynchronous events (like page-breaking) occur;
    pdftex does not (yet) have such a stack, but dvips does, and the
    <xref refid='color'>color</xref> package makes extensive use of
    it.
    <p/>
    This package is an experimental solution to the problem, and works
    best with pdf-e-tex.
    <p/>
    The package is part of the <xref refid='oberdiek'>oberdiek</xref> bundle.
  </description>
  <documentation details='Package documentation'
      href='ctan:/macros/latex/contrib/oberdiek/pdfcolmk.pdf'/>
  <ctan file='true' path='/macros/latex/contrib/oberdiek/pdfcolmk.dtx'/>
  <miktex location='oberdiek'/>
  <texlive location='oberdiek'/>
  <install path='/macros/latex/contrib/oberdiek/oberdiek.tds.zip'/>
</entry>
%</catalogue>
%    \end{macrocode}
%
% \begin{History}
%   \begin{Version}{2000/08/27 v0.1}
%   \item
%     First published version in newsgroup \xnewsgroup{comp.text.tex}:\\
%     \URL{``\link{pdftex: bug with colors?}''}^^A
%     {http://groups.google.com/group/comp.text.tex/msg/6f088e69e4085d2c}
%   \end{Version}
%   \begin{Version}{2000/09/02 v0.2}
%   \item
%     Next try.
%   \end{Version}
%   \begin{Version}{2000/09/02 v0.3}
%   \item
%     Solution without \eTeX\ added.
%   \end{Version}
%   \begin{Version}{2000/09/06 v0.4}
%   \item
%     Patch commands added.
%   \item
%     Patch for seminar.cls added.
%   \end{Version}
%   \begin{Version}{2000/09/06 v0.5}
%   \item
%     Bug fix: initialization of \cs{pec@value} added.
%   \end{Version}
%   \begin{Version}{2005/06/15 v0.6}
%   \item
%     Support for \cs{marginpar} added.
%     See thread in \xnewsgroup{comp.text.tex}:\\
%     \URL{``\link{Using \cs{textcolor} and \cs{marginpar} together}''}^^A
%     {http://groups.google.com/group/comp.text.tex/msg/38ed58f8845a2a4f}
%   \end{Version}
%   \begin{Version}{2005/07/09 v0.7}
%   \item
%     Output support added for \xpackage{memoir},
%     provided by Lars Madsen.
%   \end{Version}
%   \begin{Version}{2006/02/20 v0.8}
%   \item
%     Code is not changed.
%   \item
%     DTX framework.
%   \end{Version}
%   \begin{Version}{2007/01/01 v1.0}
%   \item
%     If \xfile{pdftex.def} \textgreater= 2007/01/01 v0.04a is used with
%     \pdfTeX\ \textgreater= 1.40.0, then package \xpackage{pdfcolmk} is obsolete.
%   \end{Version}
%   \begin{Version}{2007/04/11 v1.1}
%   \item
%     Line ends sanitized.
%   \end{Version}
%   \begin{Version}{2008/08/11 v1.2}
%   \item
%     Code is not changed.
%   \item
%     URLs updated.
%   \end{Version}
% \end{History}
%
% \PrintIndex
%
% \Finale
\endinput
|
% \end{quote}
% Do not forget to quote the argument according to the demands
% of your shell.
%
% \paragraph{Generating the documentation.}
% You can use both the \xfile{.dtx} or the \xfile{.drv} to generate
% the documentation. The process can be configured by the
% configuration file \xfile{ltxdoc.cfg}. For instance, put this
% line into this file, if you want to have A4 as paper format:
% \begin{quote}
%   \verb|\PassOptionsToClass{a4paper}{article}|
% \end{quote}
% An example follows how to generate the
% documentation with pdf\LaTeX:
% \begin{quote}
%\begin{verbatim}
%pdflatex pdfcolmk.dtx
%makeindex -s gind.ist pdfcolmk.idx
%pdflatex pdfcolmk.dtx
%makeindex -s gind.ist pdfcolmk.idx
%pdflatex pdfcolmk.dtx
%\end{verbatim}
% \end{quote}
%
% \section{Catalogue}
%
% The following XML file can be used as source for the
% \href{http://mirror.ctan.org/help/Catalogue/catalogue.html}{\TeX\ Catalogue}.
% The elements \texttt{caption} and \texttt{description} are imported
% from the original XML file from the Catalogue.
% The name of the XML file in the Catalogue is \xfile{pdfcolmk.xml}.
%    \begin{macrocode}
%<*catalogue>
<?xml version='1.0' encoding='us-ascii'?>
<!DOCTYPE entry SYSTEM 'catalogue.dtd'>
<entry datestamp='$Date$' modifier='$Author$' id='pdfcolmk'>
  <name>pdfcolmk</name>
  <caption>Improving colour support under pdftex.</caption>
  <authorref id='auth:oberdiek'/>
  <copyright owner='Heiko Oberdiek' year='2000,2005-2008'/>
  <license type='lppl1.3'/>
  <version number='1.2'/>
  <description>
    The package provides macros that emulate the &#x2018;colour stack&#x2019;
    functionality of dvips.  The colour stack deals with colour
    manipulations when asynchronous events (like page-breaking) occur;
    pdftex does not (yet) have such a stack, but dvips does, and the
    <xref refid='color'>color</xref> package makes extensive use of
    it.
    <p/>
    This package is an experimental solution to the problem, and works
    best with pdf-e-tex.
    <p/>
    The package is part of the <xref refid='oberdiek'>oberdiek</xref> bundle.
  </description>
  <documentation details='Package documentation'
      href='ctan:/macros/latex/contrib/oberdiek/pdfcolmk.pdf'/>
  <ctan file='true' path='/macros/latex/contrib/oberdiek/pdfcolmk.dtx'/>
  <miktex location='oberdiek'/>
  <texlive location='oberdiek'/>
  <install path='/macros/latex/contrib/oberdiek/oberdiek.tds.zip'/>
</entry>
%</catalogue>
%    \end{macrocode}
%
% \begin{History}
%   \begin{Version}{2000/08/27 v0.1}
%   \item
%     First published version in newsgroup \xnewsgroup{comp.text.tex}:\\
%     \URL{``\link{pdftex: bug with colors?}''}^^A
%     {http://groups.google.com/group/comp.text.tex/msg/6f088e69e4085d2c}
%   \end{Version}
%   \begin{Version}{2000/09/02 v0.2}
%   \item
%     Next try.
%   \end{Version}
%   \begin{Version}{2000/09/02 v0.3}
%   \item
%     Solution without \eTeX\ added.
%   \end{Version}
%   \begin{Version}{2000/09/06 v0.4}
%   \item
%     Patch commands added.
%   \item
%     Patch for seminar.cls added.
%   \end{Version}
%   \begin{Version}{2000/09/06 v0.5}
%   \item
%     Bug fix: initialization of \cs{pec@value} added.
%   \end{Version}
%   \begin{Version}{2005/06/15 v0.6}
%   \item
%     Support for \cs{marginpar} added.
%     See thread in \xnewsgroup{comp.text.tex}:\\
%     \URL{``\link{Using \cs{textcolor} and \cs{marginpar} together}''}^^A
%     {http://groups.google.com/group/comp.text.tex/msg/38ed58f8845a2a4f}
%   \end{Version}
%   \begin{Version}{2005/07/09 v0.7}
%   \item
%     Output support added for \xpackage{memoir},
%     provided by Lars Madsen.
%   \end{Version}
%   \begin{Version}{2006/02/20 v0.8}
%   \item
%     Code is not changed.
%   \item
%     DTX framework.
%   \end{Version}
%   \begin{Version}{2007/01/01 v1.0}
%   \item
%     If \xfile{pdftex.def} \textgreater= 2007/01/01 v0.04a is used with
%     \pdfTeX\ \textgreater= 1.40.0, then package \xpackage{pdfcolmk} is obsolete.
%   \end{Version}
%   \begin{Version}{2007/04/11 v1.1}
%   \item
%     Line ends sanitized.
%   \end{Version}
%   \begin{Version}{2008/08/11 v1.2}
%   \item
%     Code is not changed.
%   \item
%     URLs updated.
%   \end{Version}
% \end{History}
%
% \PrintIndex
%
% \Finale
\endinput

%        (quote the arguments according to the demands of your shell)
%
% Documentation:
%    (a) If pdfcolmk.drv is present:
%           latex pdfcolmk.drv
%    (b) Without pdfcolmk.drv:
%           latex pdfcolmk.dtx; ...
%    The class ltxdoc loads the configuration file ltxdoc.cfg
%    if available. Here you can specify further options, e.g.
%    use A4 as paper format:
%       \PassOptionsToClass{a4paper}{article}
%
%    Programm calls to get the documentation (example):
%       pdflatex pdfcolmk.dtx
%       makeindex -s gind.ist pdfcolmk.idx
%       pdflatex pdfcolmk.dtx
%       makeindex -s gind.ist pdfcolmk.idx
%       pdflatex pdfcolmk.dtx
%
% Installation:
%    TDS:tex/latex/oberdiek/pdfcolmk.sty
%    TDS:doc/latex/oberdiek/pdfcolmk.pdf
%    TDS:source/latex/oberdiek/pdfcolmk.dtx
%
%<*ignore>
\begingroup
  \catcode123=1 %
  \catcode125=2 %
  \def\x{LaTeX2e}%
\expandafter\endgroup
\ifcase 0\ifx\install y1\fi\expandafter
         \ifx\csname processbatchFile\endcsname\relax\else1\fi
         \ifx\fmtname\x\else 1\fi\relax
\else\csname fi\endcsname
%</ignore>
%<*install>
\input docstrip.tex
\Msg{************************************************************************}
\Msg{* Installation}
\Msg{* Package: pdfcolmk 2008/08/11 v1.2 Color support for pdfTeX via marks (HO)}
\Msg{************************************************************************}

\keepsilent
\askforoverwritefalse

\let\MetaPrefix\relax
\preamble

This is a generated file.

Project: pdfcolmk
Version: 2008/08/11 v1.2

Copyright (C) 2000, 2005-2008 by
   Heiko Oberdiek <heiko.oberdiek at googlemail.com>

This work may be distributed and/or modified under the
conditions of the LaTeX Project Public License, either
version 1.3c of this license or (at your option) any later
version. This version of this license is in
   http://www.latex-project.org/lppl/lppl-1-3c.txt
and the latest version of this license is in
   http://www.latex-project.org/lppl.txt
and version 1.3 or later is part of all distributions of
LaTeX version 2005/12/01 or later.

This work has the LPPL maintenance status "maintained".

This Current Maintainer of this work is Heiko Oberdiek.

This work consists of the main source file pdfcolmk.dtx
and the derived files
   pdfcolmk.sty, pdfcolmk.pdf, pdfcolmk.ins, pdfcolmk.drv.

\endpreamble
\let\MetaPrefix\DoubleperCent

\generate{%
  \file{pdfcolmk.ins}{\from{pdfcolmk.dtx}{install}}%
  \file{pdfcolmk.drv}{\from{pdfcolmk.dtx}{driver}}%
  \usedir{tex/latex/oberdiek}%
  \file{pdfcolmk.sty}{\from{pdfcolmk.dtx}{package}}%
  \nopreamble
  \nopostamble
  \usedir{source/latex/oberdiek/catalogue}%
  \file{pdfcolmk.xml}{\from{pdfcolmk.dtx}{catalogue}}%
}

\catcode32=13\relax% active space
\let =\space%
\Msg{************************************************************************}
\Msg{*}
\Msg{* To finish the installation you have to move the following}
\Msg{* file into a directory searched by TeX:}
\Msg{*}
\Msg{*     pdfcolmk.sty}
\Msg{*}
\Msg{* To produce the documentation run the file `pdfcolmk.drv'}
\Msg{* through LaTeX.}
\Msg{*}
\Msg{* Happy TeXing!}
\Msg{*}
\Msg{************************************************************************}

\endbatchfile
%</install>
%<*ignore>
\fi
%</ignore>
%<*driver>
\NeedsTeXFormat{LaTeX2e}
\ProvidesFile{pdfcolmk.drv}%
  [2008/08/11 v1.2 Color support for pdfTeX via marks (HO)]%
\documentclass{ltxdoc}
\usepackage{holtxdoc}[2011/11/22]
\begin{document}
  \DocInput{pdfcolmk.dtx}%
\end{document}
%</driver>
% \fi
%
% \CheckSum{843}
%
% \CharacterTable
%  {Upper-case    \A\B\C\D\E\F\G\H\I\J\K\L\M\N\O\P\Q\R\S\T\U\V\W\X\Y\Z
%   Lower-case    \a\b\c\d\e\f\g\h\i\j\k\l\m\n\o\p\q\r\s\t\u\v\w\x\y\z
%   Digits        \0\1\2\3\4\5\6\7\8\9
%   Exclamation   \!     Double quote  \"     Hash (number) \#
%   Dollar        \$     Percent       \%     Ampersand     \&
%   Acute accent  \'     Left paren    \(     Right paren   \)
%   Asterisk      \*     Plus          \+     Comma         \,
%   Minus         \-     Point         \.     Solidus       \/
%   Colon         \:     Semicolon     \;     Less than     \<
%   Equals        \=     Greater than  \>     Question mark \?
%   Commercial at \@     Left bracket  \[     Backslash     \\
%   Right bracket \]     Circumflex    \^     Underscore    \_
%   Grave accent  \`     Left brace    \{     Vertical bar  \|
%   Right brace   \}     Tilde         \~}
%
% \GetFileInfo{pdfcolmk.drv}
%
% \title{The \xpackage{pdfcolmk} package}
% \date{2008/08/11 v1.2}
% \author{Heiko Oberdiek\\\xemail{heiko.oberdiek at googlemail.com}}
%
% \maketitle
%
% \begin{abstract}
% This package tries a solution for the missing color
% stack of \pdfTeX.
% \end{abstract}
%
% \tableofcontents
%
% \section{Documentation}
%
% \subsection{Introduction}
%
% This package uses a mark register in order to solve the
% problem of a missing color stack of \pdfTeX\ prior 1.40.0.
% Since this version of \pdfTeX\ a color stack is available
% and supported by \xfile{pdftex.def} 2007/01/01 v0.04a.
% In this case this package is obsolete and the package
% stops its loading.
%
% \subsection{Background}
%
% After the Dante meeting (Clausthal 2000) I have started
% to experiment with the eTeX method of a \emph{colour} mark.
% One of the major problems is the understanding of the
% output routine and the need to rewrite it because of
% missing hooks. Currently I have made some tests in
% in onecolumn and twocolumn mode, but the state is
% experimental.
%
% \subsection{Limitations}
%
% \begin{itemize}
% \item Mark limitations: page breaks in math.
% \item \LaTeX's output routine is redefinded.
%   \begin{itemize}
%   \item Changes in the output routine of newer versions
%         of LaTeX are not detected.
%   \item Packages that change the output routine are not
%         supported.
%   \end{itemize}
% \item It does not support several independent text
%       streams like footnotes.
% \item Limitations in float and marginpar support.
% \end{itemize}
%
% \subsection{Recommendation}
%
% \eTeX\ (for additional mark register)
% Without \eTeX\ \LaTeX's mark commands are redefined
% to store an additional color value.
%
% \subsection{Usage}
%
% Load after package color:
% \begin{quote}
%   |\usepackage[pdftex]{color}|\\
%   |\usepackage{pdfcolmk}|
% \end{quote}
%
% \subsection{Compatibility}
%
% \begin{itemize}
% \item Load the following packages after \xpackage{pdfcolmk}:
%   \begin{quote}
%       \xpackage{mparhack.sty}
%   \end{quote}
% \item Load the following packages before \xpackage{pdfcolmk}:
%   \begin{quote}
%       \xpackage{marn.sty}\\
%       \xpackage{newmarn.sty}
%   \end{quote}
% \item Supported \cs{@addmarginpar} patch:
%   \begin{quote}
%       \xpackage{latex/base/latex.ltx}\\
%       \xpackage{memoir.cls}\\
%       \xpackage{poemscol/marn.sty}, \xpackage{poemscol/newmarn.sty}\\
%       \xpackage{mparhack.sty}
%   \end{quote}
% \item Unsupported \cs{@addmarginpar} patch:
%   \begin{quote}
%       \xpackage{lineno.sty}\\
%       \xpackage{sttools/marginal.sty}\\
%       \xpackage{revtex4.cls}
%   \end{quote}
% \end{itemize}
%
% \StopEventually{
% }
%
% \section{Implementation}
%
%    \begin{macrocode}
%<*package>
%    \end{macrocode}
%    Package identification.
%    \begin{macrocode}
\NeedsTeXFormat{LaTeX2e}
\ProvidesPackage{pdfcolmk}%
  [2008/08/11 v1.2 Color support for pdfTeX via marks (HO)]
%    \end{macrocode}
%
%    \begin{macrocode}
\@ifundefined{ver@pdftex.def}{%
  \PackageWarningNoLine{pdfcolmk}{%
    Nothing to fix, because \string`pdftex.def\string' not loaded%
  }%
  \endinput
}{}
\@ifpackageloaded{color}{}{%
  \PackageWarningNoLine{pdfcolmk}{%
    Nothing to fix, because \string`color.sty\string' not loaded%
  }%
  \endinput
}
\begingroup\expandafter\expandafter\expandafter\endgroup
\expandafter\ifx\csname main@pdfcolorstack\endcsname\relax
\else
  % pdftex.def >= 2007/01/01 0.04a and pdfTeX >= 1.40.0
  \begingroup
    \let\on@line\@empty
    \PackageInfo{pdfcolmk}{%
      The color stack of pdfTeX \string>\string= 1.40 is used. %
      Therefore\MessageBreak
      this package is not necessary and not loaded%
    }%
  \endgroup
  \expandafter\endinput
\fi

\PackageInfo{pdfcolmk}{%
  This package tries to simulate dvips's color stack\MessageBreak
  for pdfTeX based on a mark register of e-TeX.\MessageBreak
  It redefines LaTeX's output routine. Therefore\MessageBreak
  use with care, no warranties%
}

\ifx\marks\@undefined

  \let\pec@mark\mark
  \let\pec@value\empty
  \long\def\mark#1{%
    \protected@xdef\pec@value{#1}%
    \pec@setmark
  }%
  \def\pec@setmark{%
    \begingroup
      \@temptokena\expandafter{\pec@value}%
      \pec@mark{{\current@color}\the\@temptokena}%
    \endgroup
  }%
  \def\pec@getmark{%
    \xdef\pec@botcolor{%
      \expandafter\@firstofthree\botmark\@empty\@empty\@empty
    }%
  }%
  \long\def\@firstofthree#1#2#3{#1}%
  \CheckCommand{\@leftmark}[2]{#1}%
  \CheckCommand{\@rightmark}[2]{#2}%
  \CheckCommand*{\leftmark}{%
    \expandafter\@leftmark\botmark\@empty\@empty
  }%
  \CheckCommand*{\rightmark}{%
    \expandafter\@rightmark\firstmark\@empty\@empty
  }%
  \long\def\@leftmark#1#2#3{#2}%
  \long\def\@rightmark#1#2#3{#3}%
  \g@addto@macro\leftmark\@empty
  \g@addto@macro\rightmark\@empty

\else

  \RequirePackage{etex}[1998/03/26]%
  \newmarks\pec@marks
  \def\pec@setmark{\marks\pec@marks{\current@color}}%
  \def\pec@getmark{\xdef\pec@botcolor{\botmarks\pec@marks}}%

\fi
%    \end{macrocode}
%
% \subsection{\cs{marginpar} fix}
%
%    \begin{macrocode}
\chardef\pec@result\z@
\def\pec@temp#1{%
  \chardef\pec@result\@ne
  \begingroup
    \let\on@line\@empty
    \PackageInfo{pdfcolmk}{%
      Patch for \string\@addmarginpar\space applied (#1)%
    }%
  \endgroup
}
%    \end{macrocode}
%
% \subsubsection{latex/base/latex.ltx}
%
%    \begin{macrocode}
\def\pec@addmarginpar{%
  \@next\@marbox\@currlist{%
    \@cons\@freelist\@marbox
    \@cons\@freelist\@currbox
  }\@latexbug
  \@tempcnta\@ne
  \if@twocolumn
    \if@firstcolumn
      \@tempcnta\m@ne
    \fi
  \else
    \if@mparswitch
      \ifodd\c@page
      \else
        \@tempcnta\m@ne
      \fi
    \fi
    \if@reversemargin \@tempcnta -\@tempcnta \fi
  \fi
  \ifnum\@tempcnta <\z@  \global\setbox\@marbox\box\@currbox \fi
  \@tempdima\@mparbottom
  \advance\@tempdima -\@pageht
  \advance\@tempdima\ht\@marbox
  \ifdim\@tempdima >\z@
    \@latex@warning@no@line{Marginpar on page \thepage\space moved}%
  \else
    \@tempdima\z@
  \fi
  \global\@mparbottom\@pageht
  \global\advance\@mparbottom\@tempdima
  \global\advance\@mparbottom\dp\@marbox
  \global\advance\@mparbottom\marginparpush
  \advance\@tempdima -\ht\@marbox
  \global\setbox\@marbox\vbox{%
    \vskip \@tempdima
    \box \@marbox
  }%
  \global \ht\@marbox \z@
  \global \dp\@marbox \z@
  \kern -\@pagedp
  \nointerlineskip
  \hb@xt@\columnwidth{%
    \ifnum \@tempcnta >\z@
      \hskip\columnwidth
      \hskip\marginparsep
    \else
      \hskip -\marginparsep
      \hskip -\marginparwidth
    \fi
    \box\@marbox \hss
  }%
  \nointerlineskip
  \hbox{\vrule \@height\z@ \@width\z@ \@depth\@pagedp}%
}
\ifx\pec@addmarginpar\@addmarginpar
  \pec@temp{latex/base}%
\fi
%    \end{macrocode}
%
% \subsubsection{memoir.cls}
%
%    \begin{macrocode}
\def\pec@addmarginpar{%
  \checkoddpage
  \@next\@marbox\@currlist{%
    \@cons\@freelist\@marbox
    \@cons\@freelist\@currbox
  }\@latexbug
  \@tempcnta\@ne
  \if@twocolumn
    \if@firstcolumn
      \@tempcnta\m@ne
    \fi
  \else
    \if@mparswitch
      \ifoddpage
      \else
        \@tempcnta\m@ne
      \fi
    \fi
    \if@reversemargin
      \@tempcnta -\@tempcnta
    \fi
  \fi
  \ifnum\@tempcnta <\z@
    \global\setbox\@marbox\box\@currbox
  \fi
  \@tempdima\@mparbottom
  \advance\@tempdima -\@pageht
  \advance\@tempdima\ht\@marbox
  \ifdim\@tempdima >\z@
    \@latex@warning@no@line{%
      Marginpar on page \thepage\space moved by \the\@tempdima
    }%
  \else
    \@tempdima\z@
  \fi
  \global\@mparbottom\@pageht
  \global\advance\@mparbottom\@tempdima
  \global\advance\@mparbottom\dp\@marbox
  \global\advance\@mparbottom\marginparpush
  \advance\@tempdima -\ht\@marbox
  \global\setbox\@marbox\vbox{%
    \vskip \@tempdima
    \box \@marbox
  }%
  \global \ht\@marbox \z@
  \global \dp\@marbox \z@
  \kern -\@pagedp
  \nointerlineskip
  \hb@xt@\columnwidth{%
    \ifnum \@tempcnta >\z@
      \hskip\columnwidth
      \hskip\marginparsep
    \else
      \hskip -\marginparsep
      \hskip -\marginparwidth
    \fi
    \box\@marbox
    \hss
  }%
  \nointerlineskip
  \hbox{\vrule \@height\z@ \@width\z@ \@depth\@pagedp}%
}%
\ifx\pec@addmarginpar\@addmarginpar
  \pec@temp{memoir.cls}%
\fi
%    \end{macrocode}
%
% \subsubsection{poemscol/marn.sty, poemscol/newmarn.sty}
%
%    \begin{macrocode}
\def\pec@addmarginpar{%
  \@next \@marbox\@currlist{%
    \@cons\@freelist\@marbox
    \@cons\@freelist\@currbox
  }\@latexbug
  \global\advance\@mpar@count\m@ne
  \@ifundefined{@marn@\the\@mpar@count @}{% was location logged last time?
    \@tempcnta\@ne % NO: use original LaTeX logic
    \if@twocolumn
      \if@firstcolumn
        \@tempcnta\m@ne
      \fi
    \else
      \if@mparswitch
        \ifodd\c@page
        \else
          \@tempcnta\m@ne
        \fi
      \fi
      \if@reversemargin
        \@tempcnta -\@tempcnta
      \fi
    \fi
  }{%
    \@tempcnta %    YES: use record from last time to decide side.
    \@nameuse{@marn@\the\@mpar@count @}%
    \if@reversemargin -\fi \@ne
  }%
  \ifnum\@tempcnta <\z@
    \global\setbox\@marbox\box\@currbox
    \global\let\@marnbottom\@mparbottoml
  \else
    \global\let\@marnbottom\@mparbottom
  \fi
  \@tempdima\@marnbottom \advance\@tempdima -\@pageht
  \advance\@tempdima\ht\@marbox
  \ifdim\@tempdima >\z@
    \@@warning{Marginpar on page \thepage\space moved}%
  \else
    \@tempdima\z@
  \fi
  \global\@marnbottom\@pageht
  \global\advance\@marnbottom\@tempdima
  \global\advance\@marnbottom\dp\@marbox
  \global\advance\@marnbottom\marginparpush
  \advance\@tempdima -\ht\@marbox
  \global\ht\@marbox\z@
  \global\dp\@marbox\z@
  \vskip -\@pagedp
  \vskip\@tempdima\nointerlineskip
  \hbox to\columnwidth{%
    \ifnum \@tempcnta >\z@
      \hskip\columnwidth
      \hskip\marginparsep
    \else
      \hskip -\marginparsep
      \hskip -\marginparwidth
    \fi
    \if@filesw % record where this is for use next time:
       \@marn@log\@mpar@count
    \fi
    \box\@marbox
    \hss
  }%
  \nobreak   %% RmS 91/06/21 \nobreak added
  \vskip -\@tempdima
  \nointerlineskip
  \hbox{\vrule \@height\z@ \@width\z@ \@depth\@pagedp}%
}
\ifx\pec@addmarginpar\@addmarginpar
  \pec@temp{poemscol/(new)marn.sty}%
\fi
%    \end{macrocode}
%
% \subsubsection{refman/refart.cls, refnam/refrep.cls}
%
%    \begin{macrocode}
\def\pec@addmarginpar{%
  \@next\@marbox\@currlist{%
    \@cons\@freelist\@marbox
    \@cons\@freelist\@currbox
  }\@latexbug
  \@tempcnta\@ne
  \if@twocolumn
    \if@firstcolumn
      \@tempcnta\m@ne
    \fi
  \else
    \@tempcnta\m@ne
  \fi
  \ifnum\@tempcnta <\z@
    \global\setbox\@marbox\box\@currbox
  \fi
  \@tempdima\@mparbottom
  \advance\@tempdima -\@pageht
  \advance\@tempdima\ht\@marbox
  \ifdim\@tempdima >\z@
     \@@warning{Marginpar on page \thepage\space moved}%
  \else
     \@tempdima\z@
  \fi
  \global\@mparbottom\@pageht
  \global\advance\@mparbottom\@tempdima
  \global\advance\@mparbottom\dp\@marbox
  \global\advance\@mparbottom\marginparpush
  \advance\@tempdima -\ht\@marbox
  \global\setbox\@marbox\vbox{%
    \vskip \@tempdima \box \@marbox
  }%
  \global \ht\@marbox \z@
  \global \dp\@marbox \z@
  \kern -\@pagedp
  \nointerlineskip
  \hb@xt@\columnwidth{%
    \ifnum \@tempcnta >\z@
      \hskip\columnwidth
      \hskip\marginparsep
    \else
      \hskip -\marginparsep
      \hskip -\marginparwidth
    \fi
    \box\@marbox
    \hss
  }%
  \nointerlineskip
  \hbox{\vrule \@height\z@ \@width\z@ \@depth\@pagedp}%
}
\ifx\pec@addmarginpar\@addmarginpar
  \pec@temp{ref(art|rep).cls}%
\fi

\ifcase\pec@result
  \PackageInfo{pdfcolmk}{%
    Fix for \string\@addmarginpar\space is omitted, %
    because this variant\MessageBreak
    of \string\@addmarginpar\space
      is not recognized%
  }%
\else
  % apply patch for \@addmarginpar
  \def\pec@PatchAddMarginpar#1\columnwidth#2#3\@nil{%
    \pec@PatchAddMarginparI#2\@nil{#1}{#3}%
  }%
  \def\pec@PatchAddMarginparI#1\box\@marbox\hss#2\@nil#3#4{%
    \def\@addmarginpar{%
      #3%
      \columnwidth{%
        #1%
        \pdfliteral{q}%
        \rlap{%
          \box\@marbox
        }%
        \pdfliteral{Q}%
        \hss
        #2%
      }%
      #4%
    }%
  }%
  \expandafter\pec@PatchAddMarginpar\@addmarginpar\@nil
\fi
%    \end{macrocode}
%
% \subsection{Color fix}
%
%    \begin{macrocode}
\def\set@color{%
  \pdfliteral{\current@color}%
  \ifinner
  \else
    \pec@setmark
  \fi
  \aftergroup\reset@color
}
\def\reset@color{%
  \pdfliteral{\current@color}%
  \ifinner
  \else
    \pec@setmark
  \fi
}

\let\pec@botcolor\current@color

\def\pec@PatchVBoxCCLV{%
  \ifx\pec@botcolor\@empty
  \else
    \setbox\@cclv\vbox{%
      \pdfliteral{\pec@botcolor}%
      \unvbox\@cclv
    }%
  \fi
  \pec@getmark
}

\def\pec@PatchAlreadyInBox{%
  \ifx\pec@botcolor\@empty
  \else
    \pdfliteral{\pec@botcolor}%
  \fi
  \pec@getmark
}

\@ifclassloaded{memoir}{%
  \expandafter\def\expandafter\mem@makecol\expandafter{%
    \expandafter\pec@PatchVBoxCCLV
    \mem@makecol
  }%
  \endinput
}{}

\@ifclassloaded{seminar}{%
  \newcommand\pec@org@makeslide{}%
  \let\pec@org@makeslide\@makeslide
  \def\@makeslide{%
    \pec@PatchVBoxCCLV
    \pec@org@makeslide
  }%
  \endinput
}{}

\long\def\pec@output#1\@specialoutput\else#2\pec@end{%
  \begingroup
    \def\x{#2}%
  \expandafter\endgroup
  \ifx\x\@empty
    \PackageWarningNoLine{pdfcolmk}{%
      Unexpected \string\output\space routine detected,%
      \MessageBreak
      loading of package stopped%
    }%
    \expandafter\endinput
  \fi
}
\expandafter\expandafter\expandafter\pec@output
\expandafter\@firstofone\the\output\@specialoutput\else\pec@end

\long\def\pec@output#1\@specialoutput\else#2\pec@end{%
  \output{%
    #1\@specialoutput\else
    \pec@PatchVBoxCCLV
    #2%
  }%
}
\expandafter\expandafter\expandafter\pec@output
\expandafter\@firstofone\the\output\pec@end
%    \end{macrocode}
%
%    \begin{macrocode}
%</package>
%    \end{macrocode}
%
% \section{Installation}
%
% \subsection{Download}
%
% \paragraph{Package.} This package is available on
% CTAN\footnote{\url{ftp://ftp.ctan.org/tex-archive/}}:
% \begin{description}
% \item[\CTAN{macros/latex/contrib/oberdiek/pdfcolmk.dtx}] The source file.
% \item[\CTAN{macros/latex/contrib/oberdiek/pdfcolmk.pdf}] Documentation.
% \end{description}
%
%
% \paragraph{Bundle.} All the packages of the bundle `oberdiek'
% are also available in a TDS compliant ZIP archive. There
% the packages are already unpacked and the documentation files
% are generated. The files and directories obey the TDS standard.
% \begin{description}
% \item[\CTAN{install/macros/latex/contrib/oberdiek.tds.zip}]
% \end{description}
% \emph{TDS} refers to the standard ``A Directory Structure
% for \TeX\ Files'' (\CTAN{tds/tds.pdf}). Directories
% with \xfile{texmf} in their name are usually organized this way.
%
% \subsection{Bundle installation}
%
% \paragraph{Unpacking.} Unpack the \xfile{oberdiek.tds.zip} in the
% TDS tree (also known as \xfile{texmf} tree) of your choice.
% Example (linux):
% \begin{quote}
%   |unzip oberdiek.tds.zip -d ~/texmf|
% \end{quote}
%
% \paragraph{Script installation.}
% Check the directory \xfile{TDS:scripts/oberdiek/} for
% scripts that need further installation steps.
% Package \xpackage{attachfile2} comes with the Perl script
% \xfile{pdfatfi.pl} that should be installed in such a way
% that it can be called as \texttt{pdfatfi}.
% Example (linux):
% \begin{quote}
%   |chmod +x scripts/oberdiek/pdfatfi.pl|\\
%   |cp scripts/oberdiek/pdfatfi.pl /usr/local/bin/|
% \end{quote}
%
% \subsection{Package installation}
%
% \paragraph{Unpacking.} The \xfile{.dtx} file is a self-extracting
% \docstrip\ archive. The files are extracted by running the
% \xfile{.dtx} through \plainTeX:
% \begin{quote}
%   \verb|tex pdfcolmk.dtx|
% \end{quote}
%
% \paragraph{TDS.} Now the different files must be moved into
% the different directories in your installation TDS tree
% (also known as \xfile{texmf} tree):
% \begin{quote}
% \def\t{^^A
% \begin{tabular}{@{}>{\ttfamily}l@{ $\rightarrow$ }>{\ttfamily}l@{}}
%   pdfcolmk.sty & tex/latex/oberdiek/pdfcolmk.sty\\
%   pdfcolmk.pdf & doc/latex/oberdiek/pdfcolmk.pdf\\
%   pdfcolmk.dtx & source/latex/oberdiek/pdfcolmk.dtx\\
% \end{tabular}^^A
% }^^A
% \sbox0{\t}^^A
% \ifdim\wd0>\linewidth
%   \begingroup
%     \advance\linewidth by\leftmargin
%     \advance\linewidth by\rightmargin
%   \edef\x{\endgroup
%     \def\noexpand\lw{\the\linewidth}^^A
%   }\x
%   \def\lwbox{^^A
%     \leavevmode
%     \hbox to \linewidth{^^A
%       \kern-\leftmargin\relax
%       \hss
%       \usebox0
%       \hss
%       \kern-\rightmargin\relax
%     }^^A
%   }^^A
%   \ifdim\wd0>\lw
%     \sbox0{\small\t}^^A
%     \ifdim\wd0>\linewidth
%       \ifdim\wd0>\lw
%         \sbox0{\footnotesize\t}^^A
%         \ifdim\wd0>\linewidth
%           \ifdim\wd0>\lw
%             \sbox0{\scriptsize\t}^^A
%             \ifdim\wd0>\linewidth
%               \ifdim\wd0>\lw
%                 \sbox0{\tiny\t}^^A
%                 \ifdim\wd0>\linewidth
%                   \lwbox
%                 \else
%                   \usebox0
%                 \fi
%               \else
%                 \lwbox
%               \fi
%             \else
%               \usebox0
%             \fi
%           \else
%             \lwbox
%           \fi
%         \else
%           \usebox0
%         \fi
%       \else
%         \lwbox
%       \fi
%     \else
%       \usebox0
%     \fi
%   \else
%     \lwbox
%   \fi
% \else
%   \usebox0
% \fi
% \end{quote}
% If you have a \xfile{docstrip.cfg} that configures and enables \docstrip's
% TDS installing feature, then some files can already be in the right
% place, see the documentation of \docstrip.
%
% \subsection{Refresh file name databases}
%
% If your \TeX~distribution
% (\teTeX, \mikTeX, \dots) relies on file name databases, you must refresh
% these. For example, \teTeX\ users run \verb|texhash| or
% \verb|mktexlsr|.
%
% \subsection{Some details for the interested}
%
% \paragraph{Attached source.}
%
% The PDF documentation on CTAN also includes the
% \xfile{.dtx} source file. It can be extracted by
% AcrobatReader 6 or higher. Another option is \textsf{pdftk},
% e.g. unpack the file into the current directory:
% \begin{quote}
%   \verb|pdftk pdfcolmk.pdf unpack_files output .|
% \end{quote}
%
% \paragraph{Unpacking with \LaTeX.}
% The \xfile{.dtx} chooses its action depending on the format:
% \begin{description}
% \item[\plainTeX:] Run \docstrip\ and extract the files.
% \item[\LaTeX:] Generate the documentation.
% \end{description}
% If you insist on using \LaTeX\ for \docstrip\ (really,
% \docstrip\ does not need \LaTeX), then inform the autodetect routine
% about your intention:
% \begin{quote}
%   \verb|latex \let\install=y% \iffalse meta-comment
%
% File: pdfcolmk.dtx
% Version: 2008/08/11 v1.2
% Info: Color support for pdfTeX via marks
%
% Copyright (C) 2000, 2005-2008 by
%    Heiko Oberdiek <heiko.oberdiek at googlemail.com>
%
% This work may be distributed and/or modified under the
% conditions of the LaTeX Project Public License, either
% version 1.3c of this license or (at your option) any later
% version. This version of this license is in
%    http://www.latex-project.org/lppl/lppl-1-3c.txt
% and the latest version of this license is in
%    http://www.latex-project.org/lppl.txt
% and version 1.3 or later is part of all distributions of
% LaTeX version 2005/12/01 or later.
%
% This work has the LPPL maintenance status "maintained".
%
% This Current Maintainer of this work is Heiko Oberdiek.
%
% This work consists of the main source file pdfcolmk.dtx
% and the derived files
%    pdfcolmk.sty, pdfcolmk.pdf, pdfcolmk.ins, pdfcolmk.drv.
%
% Distribution:
%    CTAN:macros/latex/contrib/oberdiek/pdfcolmk.dtx
%    CTAN:macros/latex/contrib/oberdiek/pdfcolmk.pdf
%
% Unpacking:
%    (a) If pdfcolmk.ins is present:
%           tex pdfcolmk.ins
%    (b) Without pdfcolmk.ins:
%           tex pdfcolmk.dtx
%    (c) If you insist on using LaTeX
%           latex \let\install=y% \iffalse meta-comment
%
% File: pdfcolmk.dtx
% Version: 2008/08/11 v1.2
% Info: Color support for pdfTeX via marks
%
% Copyright (C) 2000, 2005-2008 by
%    Heiko Oberdiek <heiko.oberdiek at googlemail.com>
%
% This work may be distributed and/or modified under the
% conditions of the LaTeX Project Public License, either
% version 1.3c of this license or (at your option) any later
% version. This version of this license is in
%    http://www.latex-project.org/lppl/lppl-1-3c.txt
% and the latest version of this license is in
%    http://www.latex-project.org/lppl.txt
% and version 1.3 or later is part of all distributions of
% LaTeX version 2005/12/01 or later.
%
% This work has the LPPL maintenance status "maintained".
%
% This Current Maintainer of this work is Heiko Oberdiek.
%
% This work consists of the main source file pdfcolmk.dtx
% and the derived files
%    pdfcolmk.sty, pdfcolmk.pdf, pdfcolmk.ins, pdfcolmk.drv.
%
% Distribution:
%    CTAN:macros/latex/contrib/oberdiek/pdfcolmk.dtx
%    CTAN:macros/latex/contrib/oberdiek/pdfcolmk.pdf
%
% Unpacking:
%    (a) If pdfcolmk.ins is present:
%           tex pdfcolmk.ins
%    (b) Without pdfcolmk.ins:
%           tex pdfcolmk.dtx
%    (c) If you insist on using LaTeX
%           latex \let\install=y% \iffalse meta-comment
%
% File: pdfcolmk.dtx
% Version: 2008/08/11 v1.2
% Info: Color support for pdfTeX via marks
%
% Copyright (C) 2000, 2005-2008 by
%    Heiko Oberdiek <heiko.oberdiek at googlemail.com>
%
% This work may be distributed and/or modified under the
% conditions of the LaTeX Project Public License, either
% version 1.3c of this license or (at your option) any later
% version. This version of this license is in
%    http://www.latex-project.org/lppl/lppl-1-3c.txt
% and the latest version of this license is in
%    http://www.latex-project.org/lppl.txt
% and version 1.3 or later is part of all distributions of
% LaTeX version 2005/12/01 or later.
%
% This work has the LPPL maintenance status "maintained".
%
% This Current Maintainer of this work is Heiko Oberdiek.
%
% This work consists of the main source file pdfcolmk.dtx
% and the derived files
%    pdfcolmk.sty, pdfcolmk.pdf, pdfcolmk.ins, pdfcolmk.drv.
%
% Distribution:
%    CTAN:macros/latex/contrib/oberdiek/pdfcolmk.dtx
%    CTAN:macros/latex/contrib/oberdiek/pdfcolmk.pdf
%
% Unpacking:
%    (a) If pdfcolmk.ins is present:
%           tex pdfcolmk.ins
%    (b) Without pdfcolmk.ins:
%           tex pdfcolmk.dtx
%    (c) If you insist on using LaTeX
%           latex \let\install=y\input{pdfcolmk.dtx}
%        (quote the arguments according to the demands of your shell)
%
% Documentation:
%    (a) If pdfcolmk.drv is present:
%           latex pdfcolmk.drv
%    (b) Without pdfcolmk.drv:
%           latex pdfcolmk.dtx; ...
%    The class ltxdoc loads the configuration file ltxdoc.cfg
%    if available. Here you can specify further options, e.g.
%    use A4 as paper format:
%       \PassOptionsToClass{a4paper}{article}
%
%    Programm calls to get the documentation (example):
%       pdflatex pdfcolmk.dtx
%       makeindex -s gind.ist pdfcolmk.idx
%       pdflatex pdfcolmk.dtx
%       makeindex -s gind.ist pdfcolmk.idx
%       pdflatex pdfcolmk.dtx
%
% Installation:
%    TDS:tex/latex/oberdiek/pdfcolmk.sty
%    TDS:doc/latex/oberdiek/pdfcolmk.pdf
%    TDS:source/latex/oberdiek/pdfcolmk.dtx
%
%<*ignore>
\begingroup
  \catcode123=1 %
  \catcode125=2 %
  \def\x{LaTeX2e}%
\expandafter\endgroup
\ifcase 0\ifx\install y1\fi\expandafter
         \ifx\csname processbatchFile\endcsname\relax\else1\fi
         \ifx\fmtname\x\else 1\fi\relax
\else\csname fi\endcsname
%</ignore>
%<*install>
\input docstrip.tex
\Msg{************************************************************************}
\Msg{* Installation}
\Msg{* Package: pdfcolmk 2008/08/11 v1.2 Color support for pdfTeX via marks (HO)}
\Msg{************************************************************************}

\keepsilent
\askforoverwritefalse

\let\MetaPrefix\relax
\preamble

This is a generated file.

Project: pdfcolmk
Version: 2008/08/11 v1.2

Copyright (C) 2000, 2005-2008 by
   Heiko Oberdiek <heiko.oberdiek at googlemail.com>

This work may be distributed and/or modified under the
conditions of the LaTeX Project Public License, either
version 1.3c of this license or (at your option) any later
version. This version of this license is in
   http://www.latex-project.org/lppl/lppl-1-3c.txt
and the latest version of this license is in
   http://www.latex-project.org/lppl.txt
and version 1.3 or later is part of all distributions of
LaTeX version 2005/12/01 or later.

This work has the LPPL maintenance status "maintained".

This Current Maintainer of this work is Heiko Oberdiek.

This work consists of the main source file pdfcolmk.dtx
and the derived files
   pdfcolmk.sty, pdfcolmk.pdf, pdfcolmk.ins, pdfcolmk.drv.

\endpreamble
\let\MetaPrefix\DoubleperCent

\generate{%
  \file{pdfcolmk.ins}{\from{pdfcolmk.dtx}{install}}%
  \file{pdfcolmk.drv}{\from{pdfcolmk.dtx}{driver}}%
  \usedir{tex/latex/oberdiek}%
  \file{pdfcolmk.sty}{\from{pdfcolmk.dtx}{package}}%
  \nopreamble
  \nopostamble
  \usedir{source/latex/oberdiek/catalogue}%
  \file{pdfcolmk.xml}{\from{pdfcolmk.dtx}{catalogue}}%
}

\catcode32=13\relax% active space
\let =\space%
\Msg{************************************************************************}
\Msg{*}
\Msg{* To finish the installation you have to move the following}
\Msg{* file into a directory searched by TeX:}
\Msg{*}
\Msg{*     pdfcolmk.sty}
\Msg{*}
\Msg{* To produce the documentation run the file `pdfcolmk.drv'}
\Msg{* through LaTeX.}
\Msg{*}
\Msg{* Happy TeXing!}
\Msg{*}
\Msg{************************************************************************}

\endbatchfile
%</install>
%<*ignore>
\fi
%</ignore>
%<*driver>
\NeedsTeXFormat{LaTeX2e}
\ProvidesFile{pdfcolmk.drv}%
  [2008/08/11 v1.2 Color support for pdfTeX via marks (HO)]%
\documentclass{ltxdoc}
\usepackage{holtxdoc}[2011/11/22]
\begin{document}
  \DocInput{pdfcolmk.dtx}%
\end{document}
%</driver>
% \fi
%
% \CheckSum{843}
%
% \CharacterTable
%  {Upper-case    \A\B\C\D\E\F\G\H\I\J\K\L\M\N\O\P\Q\R\S\T\U\V\W\X\Y\Z
%   Lower-case    \a\b\c\d\e\f\g\h\i\j\k\l\m\n\o\p\q\r\s\t\u\v\w\x\y\z
%   Digits        \0\1\2\3\4\5\6\7\8\9
%   Exclamation   \!     Double quote  \"     Hash (number) \#
%   Dollar        \$     Percent       \%     Ampersand     \&
%   Acute accent  \'     Left paren    \(     Right paren   \)
%   Asterisk      \*     Plus          \+     Comma         \,
%   Minus         \-     Point         \.     Solidus       \/
%   Colon         \:     Semicolon     \;     Less than     \<
%   Equals        \=     Greater than  \>     Question mark \?
%   Commercial at \@     Left bracket  \[     Backslash     \\
%   Right bracket \]     Circumflex    \^     Underscore    \_
%   Grave accent  \`     Left brace    \{     Vertical bar  \|
%   Right brace   \}     Tilde         \~}
%
% \GetFileInfo{pdfcolmk.drv}
%
% \title{The \xpackage{pdfcolmk} package}
% \date{2008/08/11 v1.2}
% \author{Heiko Oberdiek\\\xemail{heiko.oberdiek at googlemail.com}}
%
% \maketitle
%
% \begin{abstract}
% This package tries a solution for the missing color
% stack of \pdfTeX.
% \end{abstract}
%
% \tableofcontents
%
% \section{Documentation}
%
% \subsection{Introduction}
%
% This package uses a mark register in order to solve the
% problem of a missing color stack of \pdfTeX\ prior 1.40.0.
% Since this version of \pdfTeX\ a color stack is available
% and supported by \xfile{pdftex.def} 2007/01/01 v0.04a.
% In this case this package is obsolete and the package
% stops its loading.
%
% \subsection{Background}
%
% After the Dante meeting (Clausthal 2000) I have started
% to experiment with the eTeX method of a \emph{colour} mark.
% One of the major problems is the understanding of the
% output routine and the need to rewrite it because of
% missing hooks. Currently I have made some tests in
% in onecolumn and twocolumn mode, but the state is
% experimental.
%
% \subsection{Limitations}
%
% \begin{itemize}
% \item Mark limitations: page breaks in math.
% \item \LaTeX's output routine is redefinded.
%   \begin{itemize}
%   \item Changes in the output routine of newer versions
%         of LaTeX are not detected.
%   \item Packages that change the output routine are not
%         supported.
%   \end{itemize}
% \item It does not support several independent text
%       streams like footnotes.
% \item Limitations in float and marginpar support.
% \end{itemize}
%
% \subsection{Recommendation}
%
% \eTeX\ (for additional mark register)
% Without \eTeX\ \LaTeX's mark commands are redefined
% to store an additional color value.
%
% \subsection{Usage}
%
% Load after package color:
% \begin{quote}
%   |\usepackage[pdftex]{color}|\\
%   |\usepackage{pdfcolmk}|
% \end{quote}
%
% \subsection{Compatibility}
%
% \begin{itemize}
% \item Load the following packages after \xpackage{pdfcolmk}:
%   \begin{quote}
%       \xpackage{mparhack.sty}
%   \end{quote}
% \item Load the following packages before \xpackage{pdfcolmk}:
%   \begin{quote}
%       \xpackage{marn.sty}\\
%       \xpackage{newmarn.sty}
%   \end{quote}
% \item Supported \cs{@addmarginpar} patch:
%   \begin{quote}
%       \xpackage{latex/base/latex.ltx}\\
%       \xpackage{memoir.cls}\\
%       \xpackage{poemscol/marn.sty}, \xpackage{poemscol/newmarn.sty}\\
%       \xpackage{mparhack.sty}
%   \end{quote}
% \item Unsupported \cs{@addmarginpar} patch:
%   \begin{quote}
%       \xpackage{lineno.sty}\\
%       \xpackage{sttools/marginal.sty}\\
%       \xpackage{revtex4.cls}
%   \end{quote}
% \end{itemize}
%
% \StopEventually{
% }
%
% \section{Implementation}
%
%    \begin{macrocode}
%<*package>
%    \end{macrocode}
%    Package identification.
%    \begin{macrocode}
\NeedsTeXFormat{LaTeX2e}
\ProvidesPackage{pdfcolmk}%
  [2008/08/11 v1.2 Color support for pdfTeX via marks (HO)]
%    \end{macrocode}
%
%    \begin{macrocode}
\@ifundefined{ver@pdftex.def}{%
  \PackageWarningNoLine{pdfcolmk}{%
    Nothing to fix, because \string`pdftex.def\string' not loaded%
  }%
  \endinput
}{}
\@ifpackageloaded{color}{}{%
  \PackageWarningNoLine{pdfcolmk}{%
    Nothing to fix, because \string`color.sty\string' not loaded%
  }%
  \endinput
}
\begingroup\expandafter\expandafter\expandafter\endgroup
\expandafter\ifx\csname main@pdfcolorstack\endcsname\relax
\else
  % pdftex.def >= 2007/01/01 0.04a and pdfTeX >= 1.40.0
  \begingroup
    \let\on@line\@empty
    \PackageInfo{pdfcolmk}{%
      The color stack of pdfTeX \string>\string= 1.40 is used. %
      Therefore\MessageBreak
      this package is not necessary and not loaded%
    }%
  \endgroup
  \expandafter\endinput
\fi

\PackageInfo{pdfcolmk}{%
  This package tries to simulate dvips's color stack\MessageBreak
  for pdfTeX based on a mark register of e-TeX.\MessageBreak
  It redefines LaTeX's output routine. Therefore\MessageBreak
  use with care, no warranties%
}

\ifx\marks\@undefined

  \let\pec@mark\mark
  \let\pec@value\empty
  \long\def\mark#1{%
    \protected@xdef\pec@value{#1}%
    \pec@setmark
  }%
  \def\pec@setmark{%
    \begingroup
      \@temptokena\expandafter{\pec@value}%
      \pec@mark{{\current@color}\the\@temptokena}%
    \endgroup
  }%
  \def\pec@getmark{%
    \xdef\pec@botcolor{%
      \expandafter\@firstofthree\botmark\@empty\@empty\@empty
    }%
  }%
  \long\def\@firstofthree#1#2#3{#1}%
  \CheckCommand{\@leftmark}[2]{#1}%
  \CheckCommand{\@rightmark}[2]{#2}%
  \CheckCommand*{\leftmark}{%
    \expandafter\@leftmark\botmark\@empty\@empty
  }%
  \CheckCommand*{\rightmark}{%
    \expandafter\@rightmark\firstmark\@empty\@empty
  }%
  \long\def\@leftmark#1#2#3{#2}%
  \long\def\@rightmark#1#2#3{#3}%
  \g@addto@macro\leftmark\@empty
  \g@addto@macro\rightmark\@empty

\else

  \RequirePackage{etex}[1998/03/26]%
  \newmarks\pec@marks
  \def\pec@setmark{\marks\pec@marks{\current@color}}%
  \def\pec@getmark{\xdef\pec@botcolor{\botmarks\pec@marks}}%

\fi
%    \end{macrocode}
%
% \subsection{\cs{marginpar} fix}
%
%    \begin{macrocode}
\chardef\pec@result\z@
\def\pec@temp#1{%
  \chardef\pec@result\@ne
  \begingroup
    \let\on@line\@empty
    \PackageInfo{pdfcolmk}{%
      Patch for \string\@addmarginpar\space applied (#1)%
    }%
  \endgroup
}
%    \end{macrocode}
%
% \subsubsection{latex/base/latex.ltx}
%
%    \begin{macrocode}
\def\pec@addmarginpar{%
  \@next\@marbox\@currlist{%
    \@cons\@freelist\@marbox
    \@cons\@freelist\@currbox
  }\@latexbug
  \@tempcnta\@ne
  \if@twocolumn
    \if@firstcolumn
      \@tempcnta\m@ne
    \fi
  \else
    \if@mparswitch
      \ifodd\c@page
      \else
        \@tempcnta\m@ne
      \fi
    \fi
    \if@reversemargin \@tempcnta -\@tempcnta \fi
  \fi
  \ifnum\@tempcnta <\z@  \global\setbox\@marbox\box\@currbox \fi
  \@tempdima\@mparbottom
  \advance\@tempdima -\@pageht
  \advance\@tempdima\ht\@marbox
  \ifdim\@tempdima >\z@
    \@latex@warning@no@line{Marginpar on page \thepage\space moved}%
  \else
    \@tempdima\z@
  \fi
  \global\@mparbottom\@pageht
  \global\advance\@mparbottom\@tempdima
  \global\advance\@mparbottom\dp\@marbox
  \global\advance\@mparbottom\marginparpush
  \advance\@tempdima -\ht\@marbox
  \global\setbox\@marbox\vbox{%
    \vskip \@tempdima
    \box \@marbox
  }%
  \global \ht\@marbox \z@
  \global \dp\@marbox \z@
  \kern -\@pagedp
  \nointerlineskip
  \hb@xt@\columnwidth{%
    \ifnum \@tempcnta >\z@
      \hskip\columnwidth
      \hskip\marginparsep
    \else
      \hskip -\marginparsep
      \hskip -\marginparwidth
    \fi
    \box\@marbox \hss
  }%
  \nointerlineskip
  \hbox{\vrule \@height\z@ \@width\z@ \@depth\@pagedp}%
}
\ifx\pec@addmarginpar\@addmarginpar
  \pec@temp{latex/base}%
\fi
%    \end{macrocode}
%
% \subsubsection{memoir.cls}
%
%    \begin{macrocode}
\def\pec@addmarginpar{%
  \checkoddpage
  \@next\@marbox\@currlist{%
    \@cons\@freelist\@marbox
    \@cons\@freelist\@currbox
  }\@latexbug
  \@tempcnta\@ne
  \if@twocolumn
    \if@firstcolumn
      \@tempcnta\m@ne
    \fi
  \else
    \if@mparswitch
      \ifoddpage
      \else
        \@tempcnta\m@ne
      \fi
    \fi
    \if@reversemargin
      \@tempcnta -\@tempcnta
    \fi
  \fi
  \ifnum\@tempcnta <\z@
    \global\setbox\@marbox\box\@currbox
  \fi
  \@tempdima\@mparbottom
  \advance\@tempdima -\@pageht
  \advance\@tempdima\ht\@marbox
  \ifdim\@tempdima >\z@
    \@latex@warning@no@line{%
      Marginpar on page \thepage\space moved by \the\@tempdima
    }%
  \else
    \@tempdima\z@
  \fi
  \global\@mparbottom\@pageht
  \global\advance\@mparbottom\@tempdima
  \global\advance\@mparbottom\dp\@marbox
  \global\advance\@mparbottom\marginparpush
  \advance\@tempdima -\ht\@marbox
  \global\setbox\@marbox\vbox{%
    \vskip \@tempdima
    \box \@marbox
  }%
  \global \ht\@marbox \z@
  \global \dp\@marbox \z@
  \kern -\@pagedp
  \nointerlineskip
  \hb@xt@\columnwidth{%
    \ifnum \@tempcnta >\z@
      \hskip\columnwidth
      \hskip\marginparsep
    \else
      \hskip -\marginparsep
      \hskip -\marginparwidth
    \fi
    \box\@marbox
    \hss
  }%
  \nointerlineskip
  \hbox{\vrule \@height\z@ \@width\z@ \@depth\@pagedp}%
}%
\ifx\pec@addmarginpar\@addmarginpar
  \pec@temp{memoir.cls}%
\fi
%    \end{macrocode}
%
% \subsubsection{poemscol/marn.sty, poemscol/newmarn.sty}
%
%    \begin{macrocode}
\def\pec@addmarginpar{%
  \@next \@marbox\@currlist{%
    \@cons\@freelist\@marbox
    \@cons\@freelist\@currbox
  }\@latexbug
  \global\advance\@mpar@count\m@ne
  \@ifundefined{@marn@\the\@mpar@count @}{% was location logged last time?
    \@tempcnta\@ne % NO: use original LaTeX logic
    \if@twocolumn
      \if@firstcolumn
        \@tempcnta\m@ne
      \fi
    \else
      \if@mparswitch
        \ifodd\c@page
        \else
          \@tempcnta\m@ne
        \fi
      \fi
      \if@reversemargin
        \@tempcnta -\@tempcnta
      \fi
    \fi
  }{%
    \@tempcnta %    YES: use record from last time to decide side.
    \@nameuse{@marn@\the\@mpar@count @}%
    \if@reversemargin -\fi \@ne
  }%
  \ifnum\@tempcnta <\z@
    \global\setbox\@marbox\box\@currbox
    \global\let\@marnbottom\@mparbottoml
  \else
    \global\let\@marnbottom\@mparbottom
  \fi
  \@tempdima\@marnbottom \advance\@tempdima -\@pageht
  \advance\@tempdima\ht\@marbox
  \ifdim\@tempdima >\z@
    \@@warning{Marginpar on page \thepage\space moved}%
  \else
    \@tempdima\z@
  \fi
  \global\@marnbottom\@pageht
  \global\advance\@marnbottom\@tempdima
  \global\advance\@marnbottom\dp\@marbox
  \global\advance\@marnbottom\marginparpush
  \advance\@tempdima -\ht\@marbox
  \global\ht\@marbox\z@
  \global\dp\@marbox\z@
  \vskip -\@pagedp
  \vskip\@tempdima\nointerlineskip
  \hbox to\columnwidth{%
    \ifnum \@tempcnta >\z@
      \hskip\columnwidth
      \hskip\marginparsep
    \else
      \hskip -\marginparsep
      \hskip -\marginparwidth
    \fi
    \if@filesw % record where this is for use next time:
       \@marn@log\@mpar@count
    \fi
    \box\@marbox
    \hss
  }%
  \nobreak   %% RmS 91/06/21 \nobreak added
  \vskip -\@tempdima
  \nointerlineskip
  \hbox{\vrule \@height\z@ \@width\z@ \@depth\@pagedp}%
}
\ifx\pec@addmarginpar\@addmarginpar
  \pec@temp{poemscol/(new)marn.sty}%
\fi
%    \end{macrocode}
%
% \subsubsection{refman/refart.cls, refnam/refrep.cls}
%
%    \begin{macrocode}
\def\pec@addmarginpar{%
  \@next\@marbox\@currlist{%
    \@cons\@freelist\@marbox
    \@cons\@freelist\@currbox
  }\@latexbug
  \@tempcnta\@ne
  \if@twocolumn
    \if@firstcolumn
      \@tempcnta\m@ne
    \fi
  \else
    \@tempcnta\m@ne
  \fi
  \ifnum\@tempcnta <\z@
    \global\setbox\@marbox\box\@currbox
  \fi
  \@tempdima\@mparbottom
  \advance\@tempdima -\@pageht
  \advance\@tempdima\ht\@marbox
  \ifdim\@tempdima >\z@
     \@@warning{Marginpar on page \thepage\space moved}%
  \else
     \@tempdima\z@
  \fi
  \global\@mparbottom\@pageht
  \global\advance\@mparbottom\@tempdima
  \global\advance\@mparbottom\dp\@marbox
  \global\advance\@mparbottom\marginparpush
  \advance\@tempdima -\ht\@marbox
  \global\setbox\@marbox\vbox{%
    \vskip \@tempdima \box \@marbox
  }%
  \global \ht\@marbox \z@
  \global \dp\@marbox \z@
  \kern -\@pagedp
  \nointerlineskip
  \hb@xt@\columnwidth{%
    \ifnum \@tempcnta >\z@
      \hskip\columnwidth
      \hskip\marginparsep
    \else
      \hskip -\marginparsep
      \hskip -\marginparwidth
    \fi
    \box\@marbox
    \hss
  }%
  \nointerlineskip
  \hbox{\vrule \@height\z@ \@width\z@ \@depth\@pagedp}%
}
\ifx\pec@addmarginpar\@addmarginpar
  \pec@temp{ref(art|rep).cls}%
\fi

\ifcase\pec@result
  \PackageInfo{pdfcolmk}{%
    Fix for \string\@addmarginpar\space is omitted, %
    because this variant\MessageBreak
    of \string\@addmarginpar\space
      is not recognized%
  }%
\else
  % apply patch for \@addmarginpar
  \def\pec@PatchAddMarginpar#1\columnwidth#2#3\@nil{%
    \pec@PatchAddMarginparI#2\@nil{#1}{#3}%
  }%
  \def\pec@PatchAddMarginparI#1\box\@marbox\hss#2\@nil#3#4{%
    \def\@addmarginpar{%
      #3%
      \columnwidth{%
        #1%
        \pdfliteral{q}%
        \rlap{%
          \box\@marbox
        }%
        \pdfliteral{Q}%
        \hss
        #2%
      }%
      #4%
    }%
  }%
  \expandafter\pec@PatchAddMarginpar\@addmarginpar\@nil
\fi
%    \end{macrocode}
%
% \subsection{Color fix}
%
%    \begin{macrocode}
\def\set@color{%
  \pdfliteral{\current@color}%
  \ifinner
  \else
    \pec@setmark
  \fi
  \aftergroup\reset@color
}
\def\reset@color{%
  \pdfliteral{\current@color}%
  \ifinner
  \else
    \pec@setmark
  \fi
}

\let\pec@botcolor\current@color

\def\pec@PatchVBoxCCLV{%
  \ifx\pec@botcolor\@empty
  \else
    \setbox\@cclv\vbox{%
      \pdfliteral{\pec@botcolor}%
      \unvbox\@cclv
    }%
  \fi
  \pec@getmark
}

\def\pec@PatchAlreadyInBox{%
  \ifx\pec@botcolor\@empty
  \else
    \pdfliteral{\pec@botcolor}%
  \fi
  \pec@getmark
}

\@ifclassloaded{memoir}{%
  \expandafter\def\expandafter\mem@makecol\expandafter{%
    \expandafter\pec@PatchVBoxCCLV
    \mem@makecol
  }%
  \endinput
}{}

\@ifclassloaded{seminar}{%
  \newcommand\pec@org@makeslide{}%
  \let\pec@org@makeslide\@makeslide
  \def\@makeslide{%
    \pec@PatchVBoxCCLV
    \pec@org@makeslide
  }%
  \endinput
}{}

\long\def\pec@output#1\@specialoutput\else#2\pec@end{%
  \begingroup
    \def\x{#2}%
  \expandafter\endgroup
  \ifx\x\@empty
    \PackageWarningNoLine{pdfcolmk}{%
      Unexpected \string\output\space routine detected,%
      \MessageBreak
      loading of package stopped%
    }%
    \expandafter\endinput
  \fi
}
\expandafter\expandafter\expandafter\pec@output
\expandafter\@firstofone\the\output\@specialoutput\else\pec@end

\long\def\pec@output#1\@specialoutput\else#2\pec@end{%
  \output{%
    #1\@specialoutput\else
    \pec@PatchVBoxCCLV
    #2%
  }%
}
\expandafter\expandafter\expandafter\pec@output
\expandafter\@firstofone\the\output\pec@end
%    \end{macrocode}
%
%    \begin{macrocode}
%</package>
%    \end{macrocode}
%
% \section{Installation}
%
% \subsection{Download}
%
% \paragraph{Package.} This package is available on
% CTAN\footnote{\url{ftp://ftp.ctan.org/tex-archive/}}:
% \begin{description}
% \item[\CTAN{macros/latex/contrib/oberdiek/pdfcolmk.dtx}] The source file.
% \item[\CTAN{macros/latex/contrib/oberdiek/pdfcolmk.pdf}] Documentation.
% \end{description}
%
%
% \paragraph{Bundle.} All the packages of the bundle `oberdiek'
% are also available in a TDS compliant ZIP archive. There
% the packages are already unpacked and the documentation files
% are generated. The files and directories obey the TDS standard.
% \begin{description}
% \item[\CTAN{install/macros/latex/contrib/oberdiek.tds.zip}]
% \end{description}
% \emph{TDS} refers to the standard ``A Directory Structure
% for \TeX\ Files'' (\CTAN{tds/tds.pdf}). Directories
% with \xfile{texmf} in their name are usually organized this way.
%
% \subsection{Bundle installation}
%
% \paragraph{Unpacking.} Unpack the \xfile{oberdiek.tds.zip} in the
% TDS tree (also known as \xfile{texmf} tree) of your choice.
% Example (linux):
% \begin{quote}
%   |unzip oberdiek.tds.zip -d ~/texmf|
% \end{quote}
%
% \paragraph{Script installation.}
% Check the directory \xfile{TDS:scripts/oberdiek/} for
% scripts that need further installation steps.
% Package \xpackage{attachfile2} comes with the Perl script
% \xfile{pdfatfi.pl} that should be installed in such a way
% that it can be called as \texttt{pdfatfi}.
% Example (linux):
% \begin{quote}
%   |chmod +x scripts/oberdiek/pdfatfi.pl|\\
%   |cp scripts/oberdiek/pdfatfi.pl /usr/local/bin/|
% \end{quote}
%
% \subsection{Package installation}
%
% \paragraph{Unpacking.} The \xfile{.dtx} file is a self-extracting
% \docstrip\ archive. The files are extracted by running the
% \xfile{.dtx} through \plainTeX:
% \begin{quote}
%   \verb|tex pdfcolmk.dtx|
% \end{quote}
%
% \paragraph{TDS.} Now the different files must be moved into
% the different directories in your installation TDS tree
% (also known as \xfile{texmf} tree):
% \begin{quote}
% \def\t{^^A
% \begin{tabular}{@{}>{\ttfamily}l@{ $\rightarrow$ }>{\ttfamily}l@{}}
%   pdfcolmk.sty & tex/latex/oberdiek/pdfcolmk.sty\\
%   pdfcolmk.pdf & doc/latex/oberdiek/pdfcolmk.pdf\\
%   pdfcolmk.dtx & source/latex/oberdiek/pdfcolmk.dtx\\
% \end{tabular}^^A
% }^^A
% \sbox0{\t}^^A
% \ifdim\wd0>\linewidth
%   \begingroup
%     \advance\linewidth by\leftmargin
%     \advance\linewidth by\rightmargin
%   \edef\x{\endgroup
%     \def\noexpand\lw{\the\linewidth}^^A
%   }\x
%   \def\lwbox{^^A
%     \leavevmode
%     \hbox to \linewidth{^^A
%       \kern-\leftmargin\relax
%       \hss
%       \usebox0
%       \hss
%       \kern-\rightmargin\relax
%     }^^A
%   }^^A
%   \ifdim\wd0>\lw
%     \sbox0{\small\t}^^A
%     \ifdim\wd0>\linewidth
%       \ifdim\wd0>\lw
%         \sbox0{\footnotesize\t}^^A
%         \ifdim\wd0>\linewidth
%           \ifdim\wd0>\lw
%             \sbox0{\scriptsize\t}^^A
%             \ifdim\wd0>\linewidth
%               \ifdim\wd0>\lw
%                 \sbox0{\tiny\t}^^A
%                 \ifdim\wd0>\linewidth
%                   \lwbox
%                 \else
%                   \usebox0
%                 \fi
%               \else
%                 \lwbox
%               \fi
%             \else
%               \usebox0
%             \fi
%           \else
%             \lwbox
%           \fi
%         \else
%           \usebox0
%         \fi
%       \else
%         \lwbox
%       \fi
%     \else
%       \usebox0
%     \fi
%   \else
%     \lwbox
%   \fi
% \else
%   \usebox0
% \fi
% \end{quote}
% If you have a \xfile{docstrip.cfg} that configures and enables \docstrip's
% TDS installing feature, then some files can already be in the right
% place, see the documentation of \docstrip.
%
% \subsection{Refresh file name databases}
%
% If your \TeX~distribution
% (\teTeX, \mikTeX, \dots) relies on file name databases, you must refresh
% these. For example, \teTeX\ users run \verb|texhash| or
% \verb|mktexlsr|.
%
% \subsection{Some details for the interested}
%
% \paragraph{Attached source.}
%
% The PDF documentation on CTAN also includes the
% \xfile{.dtx} source file. It can be extracted by
% AcrobatReader 6 or higher. Another option is \textsf{pdftk},
% e.g. unpack the file into the current directory:
% \begin{quote}
%   \verb|pdftk pdfcolmk.pdf unpack_files output .|
% \end{quote}
%
% \paragraph{Unpacking with \LaTeX.}
% The \xfile{.dtx} chooses its action depending on the format:
% \begin{description}
% \item[\plainTeX:] Run \docstrip\ and extract the files.
% \item[\LaTeX:] Generate the documentation.
% \end{description}
% If you insist on using \LaTeX\ for \docstrip\ (really,
% \docstrip\ does not need \LaTeX), then inform the autodetect routine
% about your intention:
% \begin{quote}
%   \verb|latex \let\install=y\input{pdfcolmk.dtx}|
% \end{quote}
% Do not forget to quote the argument according to the demands
% of your shell.
%
% \paragraph{Generating the documentation.}
% You can use both the \xfile{.dtx} or the \xfile{.drv} to generate
% the documentation. The process can be configured by the
% configuration file \xfile{ltxdoc.cfg}. For instance, put this
% line into this file, if you want to have A4 as paper format:
% \begin{quote}
%   \verb|\PassOptionsToClass{a4paper}{article}|
% \end{quote}
% An example follows how to generate the
% documentation with pdf\LaTeX:
% \begin{quote}
%\begin{verbatim}
%pdflatex pdfcolmk.dtx
%makeindex -s gind.ist pdfcolmk.idx
%pdflatex pdfcolmk.dtx
%makeindex -s gind.ist pdfcolmk.idx
%pdflatex pdfcolmk.dtx
%\end{verbatim}
% \end{quote}
%
% \section{Catalogue}
%
% The following XML file can be used as source for the
% \href{http://mirror.ctan.org/help/Catalogue/catalogue.html}{\TeX\ Catalogue}.
% The elements \texttt{caption} and \texttt{description} are imported
% from the original XML file from the Catalogue.
% The name of the XML file in the Catalogue is \xfile{pdfcolmk.xml}.
%    \begin{macrocode}
%<*catalogue>
<?xml version='1.0' encoding='us-ascii'?>
<!DOCTYPE entry SYSTEM 'catalogue.dtd'>
<entry datestamp='$Date$' modifier='$Author$' id='pdfcolmk'>
  <name>pdfcolmk</name>
  <caption>Improving colour support under pdftex.</caption>
  <authorref id='auth:oberdiek'/>
  <copyright owner='Heiko Oberdiek' year='2000,2005-2008'/>
  <license type='lppl1.3'/>
  <version number='1.2'/>
  <description>
    The package provides macros that emulate the &#x2018;colour stack&#x2019;
    functionality of dvips.  The colour stack deals with colour
    manipulations when asynchronous events (like page-breaking) occur;
    pdftex does not (yet) have such a stack, but dvips does, and the
    <xref refid='color'>color</xref> package makes extensive use of
    it.
    <p/>
    This package is an experimental solution to the problem, and works
    best with pdf-e-tex.
    <p/>
    The package is part of the <xref refid='oberdiek'>oberdiek</xref> bundle.
  </description>
  <documentation details='Package documentation'
      href='ctan:/macros/latex/contrib/oberdiek/pdfcolmk.pdf'/>
  <ctan file='true' path='/macros/latex/contrib/oberdiek/pdfcolmk.dtx'/>
  <miktex location='oberdiek'/>
  <texlive location='oberdiek'/>
  <install path='/macros/latex/contrib/oberdiek/oberdiek.tds.zip'/>
</entry>
%</catalogue>
%    \end{macrocode}
%
% \begin{History}
%   \begin{Version}{2000/08/27 v0.1}
%   \item
%     First published version in newsgroup \xnewsgroup{comp.text.tex}:\\
%     \URL{``\link{pdftex: bug with colors?}''}^^A
%     {http://groups.google.com/group/comp.text.tex/msg/6f088e69e4085d2c}
%   \end{Version}
%   \begin{Version}{2000/09/02 v0.2}
%   \item
%     Next try.
%   \end{Version}
%   \begin{Version}{2000/09/02 v0.3}
%   \item
%     Solution without \eTeX\ added.
%   \end{Version}
%   \begin{Version}{2000/09/06 v0.4}
%   \item
%     Patch commands added.
%   \item
%     Patch for seminar.cls added.
%   \end{Version}
%   \begin{Version}{2000/09/06 v0.5}
%   \item
%     Bug fix: initialization of \cs{pec@value} added.
%   \end{Version}
%   \begin{Version}{2005/06/15 v0.6}
%   \item
%     Support for \cs{marginpar} added.
%     See thread in \xnewsgroup{comp.text.tex}:\\
%     \URL{``\link{Using \cs{textcolor} and \cs{marginpar} together}''}^^A
%     {http://groups.google.com/group/comp.text.tex/msg/38ed58f8845a2a4f}
%   \end{Version}
%   \begin{Version}{2005/07/09 v0.7}
%   \item
%     Output support added for \xpackage{memoir},
%     provided by Lars Madsen.
%   \end{Version}
%   \begin{Version}{2006/02/20 v0.8}
%   \item
%     Code is not changed.
%   \item
%     DTX framework.
%   \end{Version}
%   \begin{Version}{2007/01/01 v1.0}
%   \item
%     If \xfile{pdftex.def} \textgreater= 2007/01/01 v0.04a is used with
%     \pdfTeX\ \textgreater= 1.40.0, then package \xpackage{pdfcolmk} is obsolete.
%   \end{Version}
%   \begin{Version}{2007/04/11 v1.1}
%   \item
%     Line ends sanitized.
%   \end{Version}
%   \begin{Version}{2008/08/11 v1.2}
%   \item
%     Code is not changed.
%   \item
%     URLs updated.
%   \end{Version}
% \end{History}
%
% \PrintIndex
%
% \Finale
\endinput

%        (quote the arguments according to the demands of your shell)
%
% Documentation:
%    (a) If pdfcolmk.drv is present:
%           latex pdfcolmk.drv
%    (b) Without pdfcolmk.drv:
%           latex pdfcolmk.dtx; ...
%    The class ltxdoc loads the configuration file ltxdoc.cfg
%    if available. Here you can specify further options, e.g.
%    use A4 as paper format:
%       \PassOptionsToClass{a4paper}{article}
%
%    Programm calls to get the documentation (example):
%       pdflatex pdfcolmk.dtx
%       makeindex -s gind.ist pdfcolmk.idx
%       pdflatex pdfcolmk.dtx
%       makeindex -s gind.ist pdfcolmk.idx
%       pdflatex pdfcolmk.dtx
%
% Installation:
%    TDS:tex/latex/oberdiek/pdfcolmk.sty
%    TDS:doc/latex/oberdiek/pdfcolmk.pdf
%    TDS:source/latex/oberdiek/pdfcolmk.dtx
%
%<*ignore>
\begingroup
  \catcode123=1 %
  \catcode125=2 %
  \def\x{LaTeX2e}%
\expandafter\endgroup
\ifcase 0\ifx\install y1\fi\expandafter
         \ifx\csname processbatchFile\endcsname\relax\else1\fi
         \ifx\fmtname\x\else 1\fi\relax
\else\csname fi\endcsname
%</ignore>
%<*install>
\input docstrip.tex
\Msg{************************************************************************}
\Msg{* Installation}
\Msg{* Package: pdfcolmk 2008/08/11 v1.2 Color support for pdfTeX via marks (HO)}
\Msg{************************************************************************}

\keepsilent
\askforoverwritefalse

\let\MetaPrefix\relax
\preamble

This is a generated file.

Project: pdfcolmk
Version: 2008/08/11 v1.2

Copyright (C) 2000, 2005-2008 by
   Heiko Oberdiek <heiko.oberdiek at googlemail.com>

This work may be distributed and/or modified under the
conditions of the LaTeX Project Public License, either
version 1.3c of this license or (at your option) any later
version. This version of this license is in
   http://www.latex-project.org/lppl/lppl-1-3c.txt
and the latest version of this license is in
   http://www.latex-project.org/lppl.txt
and version 1.3 or later is part of all distributions of
LaTeX version 2005/12/01 or later.

This work has the LPPL maintenance status "maintained".

This Current Maintainer of this work is Heiko Oberdiek.

This work consists of the main source file pdfcolmk.dtx
and the derived files
   pdfcolmk.sty, pdfcolmk.pdf, pdfcolmk.ins, pdfcolmk.drv.

\endpreamble
\let\MetaPrefix\DoubleperCent

\generate{%
  \file{pdfcolmk.ins}{\from{pdfcolmk.dtx}{install}}%
  \file{pdfcolmk.drv}{\from{pdfcolmk.dtx}{driver}}%
  \usedir{tex/latex/oberdiek}%
  \file{pdfcolmk.sty}{\from{pdfcolmk.dtx}{package}}%
  \nopreamble
  \nopostamble
  \usedir{source/latex/oberdiek/catalogue}%
  \file{pdfcolmk.xml}{\from{pdfcolmk.dtx}{catalogue}}%
}

\catcode32=13\relax% active space
\let =\space%
\Msg{************************************************************************}
\Msg{*}
\Msg{* To finish the installation you have to move the following}
\Msg{* file into a directory searched by TeX:}
\Msg{*}
\Msg{*     pdfcolmk.sty}
\Msg{*}
\Msg{* To produce the documentation run the file `pdfcolmk.drv'}
\Msg{* through LaTeX.}
\Msg{*}
\Msg{* Happy TeXing!}
\Msg{*}
\Msg{************************************************************************}

\endbatchfile
%</install>
%<*ignore>
\fi
%</ignore>
%<*driver>
\NeedsTeXFormat{LaTeX2e}
\ProvidesFile{pdfcolmk.drv}%
  [2008/08/11 v1.2 Color support for pdfTeX via marks (HO)]%
\documentclass{ltxdoc}
\usepackage{holtxdoc}[2011/11/22]
\begin{document}
  \DocInput{pdfcolmk.dtx}%
\end{document}
%</driver>
% \fi
%
% \CheckSum{843}
%
% \CharacterTable
%  {Upper-case    \A\B\C\D\E\F\G\H\I\J\K\L\M\N\O\P\Q\R\S\T\U\V\W\X\Y\Z
%   Lower-case    \a\b\c\d\e\f\g\h\i\j\k\l\m\n\o\p\q\r\s\t\u\v\w\x\y\z
%   Digits        \0\1\2\3\4\5\6\7\8\9
%   Exclamation   \!     Double quote  \"     Hash (number) \#
%   Dollar        \$     Percent       \%     Ampersand     \&
%   Acute accent  \'     Left paren    \(     Right paren   \)
%   Asterisk      \*     Plus          \+     Comma         \,
%   Minus         \-     Point         \.     Solidus       \/
%   Colon         \:     Semicolon     \;     Less than     \<
%   Equals        \=     Greater than  \>     Question mark \?
%   Commercial at \@     Left bracket  \[     Backslash     \\
%   Right bracket \]     Circumflex    \^     Underscore    \_
%   Grave accent  \`     Left brace    \{     Vertical bar  \|
%   Right brace   \}     Tilde         \~}
%
% \GetFileInfo{pdfcolmk.drv}
%
% \title{The \xpackage{pdfcolmk} package}
% \date{2008/08/11 v1.2}
% \author{Heiko Oberdiek\\\xemail{heiko.oberdiek at googlemail.com}}
%
% \maketitle
%
% \begin{abstract}
% This package tries a solution for the missing color
% stack of \pdfTeX.
% \end{abstract}
%
% \tableofcontents
%
% \section{Documentation}
%
% \subsection{Introduction}
%
% This package uses a mark register in order to solve the
% problem of a missing color stack of \pdfTeX\ prior 1.40.0.
% Since this version of \pdfTeX\ a color stack is available
% and supported by \xfile{pdftex.def} 2007/01/01 v0.04a.
% In this case this package is obsolete and the package
% stops its loading.
%
% \subsection{Background}
%
% After the Dante meeting (Clausthal 2000) I have started
% to experiment with the eTeX method of a \emph{colour} mark.
% One of the major problems is the understanding of the
% output routine and the need to rewrite it because of
% missing hooks. Currently I have made some tests in
% in onecolumn and twocolumn mode, but the state is
% experimental.
%
% \subsection{Limitations}
%
% \begin{itemize}
% \item Mark limitations: page breaks in math.
% \item \LaTeX's output routine is redefinded.
%   \begin{itemize}
%   \item Changes in the output routine of newer versions
%         of LaTeX are not detected.
%   \item Packages that change the output routine are not
%         supported.
%   \end{itemize}
% \item It does not support several independent text
%       streams like footnotes.
% \item Limitations in float and marginpar support.
% \end{itemize}
%
% \subsection{Recommendation}
%
% \eTeX\ (for additional mark register)
% Without \eTeX\ \LaTeX's mark commands are redefined
% to store an additional color value.
%
% \subsection{Usage}
%
% Load after package color:
% \begin{quote}
%   |\usepackage[pdftex]{color}|\\
%   |\usepackage{pdfcolmk}|
% \end{quote}
%
% \subsection{Compatibility}
%
% \begin{itemize}
% \item Load the following packages after \xpackage{pdfcolmk}:
%   \begin{quote}
%       \xpackage{mparhack.sty}
%   \end{quote}
% \item Load the following packages before \xpackage{pdfcolmk}:
%   \begin{quote}
%       \xpackage{marn.sty}\\
%       \xpackage{newmarn.sty}
%   \end{quote}
% \item Supported \cs{@addmarginpar} patch:
%   \begin{quote}
%       \xpackage{latex/base/latex.ltx}\\
%       \xpackage{memoir.cls}\\
%       \xpackage{poemscol/marn.sty}, \xpackage{poemscol/newmarn.sty}\\
%       \xpackage{mparhack.sty}
%   \end{quote}
% \item Unsupported \cs{@addmarginpar} patch:
%   \begin{quote}
%       \xpackage{lineno.sty}\\
%       \xpackage{sttools/marginal.sty}\\
%       \xpackage{revtex4.cls}
%   \end{quote}
% \end{itemize}
%
% \StopEventually{
% }
%
% \section{Implementation}
%
%    \begin{macrocode}
%<*package>
%    \end{macrocode}
%    Package identification.
%    \begin{macrocode}
\NeedsTeXFormat{LaTeX2e}
\ProvidesPackage{pdfcolmk}%
  [2008/08/11 v1.2 Color support for pdfTeX via marks (HO)]
%    \end{macrocode}
%
%    \begin{macrocode}
\@ifundefined{ver@pdftex.def}{%
  \PackageWarningNoLine{pdfcolmk}{%
    Nothing to fix, because \string`pdftex.def\string' not loaded%
  }%
  \endinput
}{}
\@ifpackageloaded{color}{}{%
  \PackageWarningNoLine{pdfcolmk}{%
    Nothing to fix, because \string`color.sty\string' not loaded%
  }%
  \endinput
}
\begingroup\expandafter\expandafter\expandafter\endgroup
\expandafter\ifx\csname main@pdfcolorstack\endcsname\relax
\else
  % pdftex.def >= 2007/01/01 0.04a and pdfTeX >= 1.40.0
  \begingroup
    \let\on@line\@empty
    \PackageInfo{pdfcolmk}{%
      The color stack of pdfTeX \string>\string= 1.40 is used. %
      Therefore\MessageBreak
      this package is not necessary and not loaded%
    }%
  \endgroup
  \expandafter\endinput
\fi

\PackageInfo{pdfcolmk}{%
  This package tries to simulate dvips's color stack\MessageBreak
  for pdfTeX based on a mark register of e-TeX.\MessageBreak
  It redefines LaTeX's output routine. Therefore\MessageBreak
  use with care, no warranties%
}

\ifx\marks\@undefined

  \let\pec@mark\mark
  \let\pec@value\empty
  \long\def\mark#1{%
    \protected@xdef\pec@value{#1}%
    \pec@setmark
  }%
  \def\pec@setmark{%
    \begingroup
      \@temptokena\expandafter{\pec@value}%
      \pec@mark{{\current@color}\the\@temptokena}%
    \endgroup
  }%
  \def\pec@getmark{%
    \xdef\pec@botcolor{%
      \expandafter\@firstofthree\botmark\@empty\@empty\@empty
    }%
  }%
  \long\def\@firstofthree#1#2#3{#1}%
  \CheckCommand{\@leftmark}[2]{#1}%
  \CheckCommand{\@rightmark}[2]{#2}%
  \CheckCommand*{\leftmark}{%
    \expandafter\@leftmark\botmark\@empty\@empty
  }%
  \CheckCommand*{\rightmark}{%
    \expandafter\@rightmark\firstmark\@empty\@empty
  }%
  \long\def\@leftmark#1#2#3{#2}%
  \long\def\@rightmark#1#2#3{#3}%
  \g@addto@macro\leftmark\@empty
  \g@addto@macro\rightmark\@empty

\else

  \RequirePackage{etex}[1998/03/26]%
  \newmarks\pec@marks
  \def\pec@setmark{\marks\pec@marks{\current@color}}%
  \def\pec@getmark{\xdef\pec@botcolor{\botmarks\pec@marks}}%

\fi
%    \end{macrocode}
%
% \subsection{\cs{marginpar} fix}
%
%    \begin{macrocode}
\chardef\pec@result\z@
\def\pec@temp#1{%
  \chardef\pec@result\@ne
  \begingroup
    \let\on@line\@empty
    \PackageInfo{pdfcolmk}{%
      Patch for \string\@addmarginpar\space applied (#1)%
    }%
  \endgroup
}
%    \end{macrocode}
%
% \subsubsection{latex/base/latex.ltx}
%
%    \begin{macrocode}
\def\pec@addmarginpar{%
  \@next\@marbox\@currlist{%
    \@cons\@freelist\@marbox
    \@cons\@freelist\@currbox
  }\@latexbug
  \@tempcnta\@ne
  \if@twocolumn
    \if@firstcolumn
      \@tempcnta\m@ne
    \fi
  \else
    \if@mparswitch
      \ifodd\c@page
      \else
        \@tempcnta\m@ne
      \fi
    \fi
    \if@reversemargin \@tempcnta -\@tempcnta \fi
  \fi
  \ifnum\@tempcnta <\z@  \global\setbox\@marbox\box\@currbox \fi
  \@tempdima\@mparbottom
  \advance\@tempdima -\@pageht
  \advance\@tempdima\ht\@marbox
  \ifdim\@tempdima >\z@
    \@latex@warning@no@line{Marginpar on page \thepage\space moved}%
  \else
    \@tempdima\z@
  \fi
  \global\@mparbottom\@pageht
  \global\advance\@mparbottom\@tempdima
  \global\advance\@mparbottom\dp\@marbox
  \global\advance\@mparbottom\marginparpush
  \advance\@tempdima -\ht\@marbox
  \global\setbox\@marbox\vbox{%
    \vskip \@tempdima
    \box \@marbox
  }%
  \global \ht\@marbox \z@
  \global \dp\@marbox \z@
  \kern -\@pagedp
  \nointerlineskip
  \hb@xt@\columnwidth{%
    \ifnum \@tempcnta >\z@
      \hskip\columnwidth
      \hskip\marginparsep
    \else
      \hskip -\marginparsep
      \hskip -\marginparwidth
    \fi
    \box\@marbox \hss
  }%
  \nointerlineskip
  \hbox{\vrule \@height\z@ \@width\z@ \@depth\@pagedp}%
}
\ifx\pec@addmarginpar\@addmarginpar
  \pec@temp{latex/base}%
\fi
%    \end{macrocode}
%
% \subsubsection{memoir.cls}
%
%    \begin{macrocode}
\def\pec@addmarginpar{%
  \checkoddpage
  \@next\@marbox\@currlist{%
    \@cons\@freelist\@marbox
    \@cons\@freelist\@currbox
  }\@latexbug
  \@tempcnta\@ne
  \if@twocolumn
    \if@firstcolumn
      \@tempcnta\m@ne
    \fi
  \else
    \if@mparswitch
      \ifoddpage
      \else
        \@tempcnta\m@ne
      \fi
    \fi
    \if@reversemargin
      \@tempcnta -\@tempcnta
    \fi
  \fi
  \ifnum\@tempcnta <\z@
    \global\setbox\@marbox\box\@currbox
  \fi
  \@tempdima\@mparbottom
  \advance\@tempdima -\@pageht
  \advance\@tempdima\ht\@marbox
  \ifdim\@tempdima >\z@
    \@latex@warning@no@line{%
      Marginpar on page \thepage\space moved by \the\@tempdima
    }%
  \else
    \@tempdima\z@
  \fi
  \global\@mparbottom\@pageht
  \global\advance\@mparbottom\@tempdima
  \global\advance\@mparbottom\dp\@marbox
  \global\advance\@mparbottom\marginparpush
  \advance\@tempdima -\ht\@marbox
  \global\setbox\@marbox\vbox{%
    \vskip \@tempdima
    \box \@marbox
  }%
  \global \ht\@marbox \z@
  \global \dp\@marbox \z@
  \kern -\@pagedp
  \nointerlineskip
  \hb@xt@\columnwidth{%
    \ifnum \@tempcnta >\z@
      \hskip\columnwidth
      \hskip\marginparsep
    \else
      \hskip -\marginparsep
      \hskip -\marginparwidth
    \fi
    \box\@marbox
    \hss
  }%
  \nointerlineskip
  \hbox{\vrule \@height\z@ \@width\z@ \@depth\@pagedp}%
}%
\ifx\pec@addmarginpar\@addmarginpar
  \pec@temp{memoir.cls}%
\fi
%    \end{macrocode}
%
% \subsubsection{poemscol/marn.sty, poemscol/newmarn.sty}
%
%    \begin{macrocode}
\def\pec@addmarginpar{%
  \@next \@marbox\@currlist{%
    \@cons\@freelist\@marbox
    \@cons\@freelist\@currbox
  }\@latexbug
  \global\advance\@mpar@count\m@ne
  \@ifundefined{@marn@\the\@mpar@count @}{% was location logged last time?
    \@tempcnta\@ne % NO: use original LaTeX logic
    \if@twocolumn
      \if@firstcolumn
        \@tempcnta\m@ne
      \fi
    \else
      \if@mparswitch
        \ifodd\c@page
        \else
          \@tempcnta\m@ne
        \fi
      \fi
      \if@reversemargin
        \@tempcnta -\@tempcnta
      \fi
    \fi
  }{%
    \@tempcnta %    YES: use record from last time to decide side.
    \@nameuse{@marn@\the\@mpar@count @}%
    \if@reversemargin -\fi \@ne
  }%
  \ifnum\@tempcnta <\z@
    \global\setbox\@marbox\box\@currbox
    \global\let\@marnbottom\@mparbottoml
  \else
    \global\let\@marnbottom\@mparbottom
  \fi
  \@tempdima\@marnbottom \advance\@tempdima -\@pageht
  \advance\@tempdima\ht\@marbox
  \ifdim\@tempdima >\z@
    \@@warning{Marginpar on page \thepage\space moved}%
  \else
    \@tempdima\z@
  \fi
  \global\@marnbottom\@pageht
  \global\advance\@marnbottom\@tempdima
  \global\advance\@marnbottom\dp\@marbox
  \global\advance\@marnbottom\marginparpush
  \advance\@tempdima -\ht\@marbox
  \global\ht\@marbox\z@
  \global\dp\@marbox\z@
  \vskip -\@pagedp
  \vskip\@tempdima\nointerlineskip
  \hbox to\columnwidth{%
    \ifnum \@tempcnta >\z@
      \hskip\columnwidth
      \hskip\marginparsep
    \else
      \hskip -\marginparsep
      \hskip -\marginparwidth
    \fi
    \if@filesw % record where this is for use next time:
       \@marn@log\@mpar@count
    \fi
    \box\@marbox
    \hss
  }%
  \nobreak   %% RmS 91/06/21 \nobreak added
  \vskip -\@tempdima
  \nointerlineskip
  \hbox{\vrule \@height\z@ \@width\z@ \@depth\@pagedp}%
}
\ifx\pec@addmarginpar\@addmarginpar
  \pec@temp{poemscol/(new)marn.sty}%
\fi
%    \end{macrocode}
%
% \subsubsection{refman/refart.cls, refnam/refrep.cls}
%
%    \begin{macrocode}
\def\pec@addmarginpar{%
  \@next\@marbox\@currlist{%
    \@cons\@freelist\@marbox
    \@cons\@freelist\@currbox
  }\@latexbug
  \@tempcnta\@ne
  \if@twocolumn
    \if@firstcolumn
      \@tempcnta\m@ne
    \fi
  \else
    \@tempcnta\m@ne
  \fi
  \ifnum\@tempcnta <\z@
    \global\setbox\@marbox\box\@currbox
  \fi
  \@tempdima\@mparbottom
  \advance\@tempdima -\@pageht
  \advance\@tempdima\ht\@marbox
  \ifdim\@tempdima >\z@
     \@@warning{Marginpar on page \thepage\space moved}%
  \else
     \@tempdima\z@
  \fi
  \global\@mparbottom\@pageht
  \global\advance\@mparbottom\@tempdima
  \global\advance\@mparbottom\dp\@marbox
  \global\advance\@mparbottom\marginparpush
  \advance\@tempdima -\ht\@marbox
  \global\setbox\@marbox\vbox{%
    \vskip \@tempdima \box \@marbox
  }%
  \global \ht\@marbox \z@
  \global \dp\@marbox \z@
  \kern -\@pagedp
  \nointerlineskip
  \hb@xt@\columnwidth{%
    \ifnum \@tempcnta >\z@
      \hskip\columnwidth
      \hskip\marginparsep
    \else
      \hskip -\marginparsep
      \hskip -\marginparwidth
    \fi
    \box\@marbox
    \hss
  }%
  \nointerlineskip
  \hbox{\vrule \@height\z@ \@width\z@ \@depth\@pagedp}%
}
\ifx\pec@addmarginpar\@addmarginpar
  \pec@temp{ref(art|rep).cls}%
\fi

\ifcase\pec@result
  \PackageInfo{pdfcolmk}{%
    Fix for \string\@addmarginpar\space is omitted, %
    because this variant\MessageBreak
    of \string\@addmarginpar\space
      is not recognized%
  }%
\else
  % apply patch for \@addmarginpar
  \def\pec@PatchAddMarginpar#1\columnwidth#2#3\@nil{%
    \pec@PatchAddMarginparI#2\@nil{#1}{#3}%
  }%
  \def\pec@PatchAddMarginparI#1\box\@marbox\hss#2\@nil#3#4{%
    \def\@addmarginpar{%
      #3%
      \columnwidth{%
        #1%
        \pdfliteral{q}%
        \rlap{%
          \box\@marbox
        }%
        \pdfliteral{Q}%
        \hss
        #2%
      }%
      #4%
    }%
  }%
  \expandafter\pec@PatchAddMarginpar\@addmarginpar\@nil
\fi
%    \end{macrocode}
%
% \subsection{Color fix}
%
%    \begin{macrocode}
\def\set@color{%
  \pdfliteral{\current@color}%
  \ifinner
  \else
    \pec@setmark
  \fi
  \aftergroup\reset@color
}
\def\reset@color{%
  \pdfliteral{\current@color}%
  \ifinner
  \else
    \pec@setmark
  \fi
}

\let\pec@botcolor\current@color

\def\pec@PatchVBoxCCLV{%
  \ifx\pec@botcolor\@empty
  \else
    \setbox\@cclv\vbox{%
      \pdfliteral{\pec@botcolor}%
      \unvbox\@cclv
    }%
  \fi
  \pec@getmark
}

\def\pec@PatchAlreadyInBox{%
  \ifx\pec@botcolor\@empty
  \else
    \pdfliteral{\pec@botcolor}%
  \fi
  \pec@getmark
}

\@ifclassloaded{memoir}{%
  \expandafter\def\expandafter\mem@makecol\expandafter{%
    \expandafter\pec@PatchVBoxCCLV
    \mem@makecol
  }%
  \endinput
}{}

\@ifclassloaded{seminar}{%
  \newcommand\pec@org@makeslide{}%
  \let\pec@org@makeslide\@makeslide
  \def\@makeslide{%
    \pec@PatchVBoxCCLV
    \pec@org@makeslide
  }%
  \endinput
}{}

\long\def\pec@output#1\@specialoutput\else#2\pec@end{%
  \begingroup
    \def\x{#2}%
  \expandafter\endgroup
  \ifx\x\@empty
    \PackageWarningNoLine{pdfcolmk}{%
      Unexpected \string\output\space routine detected,%
      \MessageBreak
      loading of package stopped%
    }%
    \expandafter\endinput
  \fi
}
\expandafter\expandafter\expandafter\pec@output
\expandafter\@firstofone\the\output\@specialoutput\else\pec@end

\long\def\pec@output#1\@specialoutput\else#2\pec@end{%
  \output{%
    #1\@specialoutput\else
    \pec@PatchVBoxCCLV
    #2%
  }%
}
\expandafter\expandafter\expandafter\pec@output
\expandafter\@firstofone\the\output\pec@end
%    \end{macrocode}
%
%    \begin{macrocode}
%</package>
%    \end{macrocode}
%
% \section{Installation}
%
% \subsection{Download}
%
% \paragraph{Package.} This package is available on
% CTAN\footnote{\url{ftp://ftp.ctan.org/tex-archive/}}:
% \begin{description}
% \item[\CTAN{macros/latex/contrib/oberdiek/pdfcolmk.dtx}] The source file.
% \item[\CTAN{macros/latex/contrib/oberdiek/pdfcolmk.pdf}] Documentation.
% \end{description}
%
%
% \paragraph{Bundle.} All the packages of the bundle `oberdiek'
% are also available in a TDS compliant ZIP archive. There
% the packages are already unpacked and the documentation files
% are generated. The files and directories obey the TDS standard.
% \begin{description}
% \item[\CTAN{install/macros/latex/contrib/oberdiek.tds.zip}]
% \end{description}
% \emph{TDS} refers to the standard ``A Directory Structure
% for \TeX\ Files'' (\CTAN{tds/tds.pdf}). Directories
% with \xfile{texmf} in their name are usually organized this way.
%
% \subsection{Bundle installation}
%
% \paragraph{Unpacking.} Unpack the \xfile{oberdiek.tds.zip} in the
% TDS tree (also known as \xfile{texmf} tree) of your choice.
% Example (linux):
% \begin{quote}
%   |unzip oberdiek.tds.zip -d ~/texmf|
% \end{quote}
%
% \paragraph{Script installation.}
% Check the directory \xfile{TDS:scripts/oberdiek/} for
% scripts that need further installation steps.
% Package \xpackage{attachfile2} comes with the Perl script
% \xfile{pdfatfi.pl} that should be installed in such a way
% that it can be called as \texttt{pdfatfi}.
% Example (linux):
% \begin{quote}
%   |chmod +x scripts/oberdiek/pdfatfi.pl|\\
%   |cp scripts/oberdiek/pdfatfi.pl /usr/local/bin/|
% \end{quote}
%
% \subsection{Package installation}
%
% \paragraph{Unpacking.} The \xfile{.dtx} file is a self-extracting
% \docstrip\ archive. The files are extracted by running the
% \xfile{.dtx} through \plainTeX:
% \begin{quote}
%   \verb|tex pdfcolmk.dtx|
% \end{quote}
%
% \paragraph{TDS.} Now the different files must be moved into
% the different directories in your installation TDS tree
% (also known as \xfile{texmf} tree):
% \begin{quote}
% \def\t{^^A
% \begin{tabular}{@{}>{\ttfamily}l@{ $\rightarrow$ }>{\ttfamily}l@{}}
%   pdfcolmk.sty & tex/latex/oberdiek/pdfcolmk.sty\\
%   pdfcolmk.pdf & doc/latex/oberdiek/pdfcolmk.pdf\\
%   pdfcolmk.dtx & source/latex/oberdiek/pdfcolmk.dtx\\
% \end{tabular}^^A
% }^^A
% \sbox0{\t}^^A
% \ifdim\wd0>\linewidth
%   \begingroup
%     \advance\linewidth by\leftmargin
%     \advance\linewidth by\rightmargin
%   \edef\x{\endgroup
%     \def\noexpand\lw{\the\linewidth}^^A
%   }\x
%   \def\lwbox{^^A
%     \leavevmode
%     \hbox to \linewidth{^^A
%       \kern-\leftmargin\relax
%       \hss
%       \usebox0
%       \hss
%       \kern-\rightmargin\relax
%     }^^A
%   }^^A
%   \ifdim\wd0>\lw
%     \sbox0{\small\t}^^A
%     \ifdim\wd0>\linewidth
%       \ifdim\wd0>\lw
%         \sbox0{\footnotesize\t}^^A
%         \ifdim\wd0>\linewidth
%           \ifdim\wd0>\lw
%             \sbox0{\scriptsize\t}^^A
%             \ifdim\wd0>\linewidth
%               \ifdim\wd0>\lw
%                 \sbox0{\tiny\t}^^A
%                 \ifdim\wd0>\linewidth
%                   \lwbox
%                 \else
%                   \usebox0
%                 \fi
%               \else
%                 \lwbox
%               \fi
%             \else
%               \usebox0
%             \fi
%           \else
%             \lwbox
%           \fi
%         \else
%           \usebox0
%         \fi
%       \else
%         \lwbox
%       \fi
%     \else
%       \usebox0
%     \fi
%   \else
%     \lwbox
%   \fi
% \else
%   \usebox0
% \fi
% \end{quote}
% If you have a \xfile{docstrip.cfg} that configures and enables \docstrip's
% TDS installing feature, then some files can already be in the right
% place, see the documentation of \docstrip.
%
% \subsection{Refresh file name databases}
%
% If your \TeX~distribution
% (\teTeX, \mikTeX, \dots) relies on file name databases, you must refresh
% these. For example, \teTeX\ users run \verb|texhash| or
% \verb|mktexlsr|.
%
% \subsection{Some details for the interested}
%
% \paragraph{Attached source.}
%
% The PDF documentation on CTAN also includes the
% \xfile{.dtx} source file. It can be extracted by
% AcrobatReader 6 or higher. Another option is \textsf{pdftk},
% e.g. unpack the file into the current directory:
% \begin{quote}
%   \verb|pdftk pdfcolmk.pdf unpack_files output .|
% \end{quote}
%
% \paragraph{Unpacking with \LaTeX.}
% The \xfile{.dtx} chooses its action depending on the format:
% \begin{description}
% \item[\plainTeX:] Run \docstrip\ and extract the files.
% \item[\LaTeX:] Generate the documentation.
% \end{description}
% If you insist on using \LaTeX\ for \docstrip\ (really,
% \docstrip\ does not need \LaTeX), then inform the autodetect routine
% about your intention:
% \begin{quote}
%   \verb|latex \let\install=y% \iffalse meta-comment
%
% File: pdfcolmk.dtx
% Version: 2008/08/11 v1.2
% Info: Color support for pdfTeX via marks
%
% Copyright (C) 2000, 2005-2008 by
%    Heiko Oberdiek <heiko.oberdiek at googlemail.com>
%
% This work may be distributed and/or modified under the
% conditions of the LaTeX Project Public License, either
% version 1.3c of this license or (at your option) any later
% version. This version of this license is in
%    http://www.latex-project.org/lppl/lppl-1-3c.txt
% and the latest version of this license is in
%    http://www.latex-project.org/lppl.txt
% and version 1.3 or later is part of all distributions of
% LaTeX version 2005/12/01 or later.
%
% This work has the LPPL maintenance status "maintained".
%
% This Current Maintainer of this work is Heiko Oberdiek.
%
% This work consists of the main source file pdfcolmk.dtx
% and the derived files
%    pdfcolmk.sty, pdfcolmk.pdf, pdfcolmk.ins, pdfcolmk.drv.
%
% Distribution:
%    CTAN:macros/latex/contrib/oberdiek/pdfcolmk.dtx
%    CTAN:macros/latex/contrib/oberdiek/pdfcolmk.pdf
%
% Unpacking:
%    (a) If pdfcolmk.ins is present:
%           tex pdfcolmk.ins
%    (b) Without pdfcolmk.ins:
%           tex pdfcolmk.dtx
%    (c) If you insist on using LaTeX
%           latex \let\install=y\input{pdfcolmk.dtx}
%        (quote the arguments according to the demands of your shell)
%
% Documentation:
%    (a) If pdfcolmk.drv is present:
%           latex pdfcolmk.drv
%    (b) Without pdfcolmk.drv:
%           latex pdfcolmk.dtx; ...
%    The class ltxdoc loads the configuration file ltxdoc.cfg
%    if available. Here you can specify further options, e.g.
%    use A4 as paper format:
%       \PassOptionsToClass{a4paper}{article}
%
%    Programm calls to get the documentation (example):
%       pdflatex pdfcolmk.dtx
%       makeindex -s gind.ist pdfcolmk.idx
%       pdflatex pdfcolmk.dtx
%       makeindex -s gind.ist pdfcolmk.idx
%       pdflatex pdfcolmk.dtx
%
% Installation:
%    TDS:tex/latex/oberdiek/pdfcolmk.sty
%    TDS:doc/latex/oberdiek/pdfcolmk.pdf
%    TDS:source/latex/oberdiek/pdfcolmk.dtx
%
%<*ignore>
\begingroup
  \catcode123=1 %
  \catcode125=2 %
  \def\x{LaTeX2e}%
\expandafter\endgroup
\ifcase 0\ifx\install y1\fi\expandafter
         \ifx\csname processbatchFile\endcsname\relax\else1\fi
         \ifx\fmtname\x\else 1\fi\relax
\else\csname fi\endcsname
%</ignore>
%<*install>
\input docstrip.tex
\Msg{************************************************************************}
\Msg{* Installation}
\Msg{* Package: pdfcolmk 2008/08/11 v1.2 Color support for pdfTeX via marks (HO)}
\Msg{************************************************************************}

\keepsilent
\askforoverwritefalse

\let\MetaPrefix\relax
\preamble

This is a generated file.

Project: pdfcolmk
Version: 2008/08/11 v1.2

Copyright (C) 2000, 2005-2008 by
   Heiko Oberdiek <heiko.oberdiek at googlemail.com>

This work may be distributed and/or modified under the
conditions of the LaTeX Project Public License, either
version 1.3c of this license or (at your option) any later
version. This version of this license is in
   http://www.latex-project.org/lppl/lppl-1-3c.txt
and the latest version of this license is in
   http://www.latex-project.org/lppl.txt
and version 1.3 or later is part of all distributions of
LaTeX version 2005/12/01 or later.

This work has the LPPL maintenance status "maintained".

This Current Maintainer of this work is Heiko Oberdiek.

This work consists of the main source file pdfcolmk.dtx
and the derived files
   pdfcolmk.sty, pdfcolmk.pdf, pdfcolmk.ins, pdfcolmk.drv.

\endpreamble
\let\MetaPrefix\DoubleperCent

\generate{%
  \file{pdfcolmk.ins}{\from{pdfcolmk.dtx}{install}}%
  \file{pdfcolmk.drv}{\from{pdfcolmk.dtx}{driver}}%
  \usedir{tex/latex/oberdiek}%
  \file{pdfcolmk.sty}{\from{pdfcolmk.dtx}{package}}%
  \nopreamble
  \nopostamble
  \usedir{source/latex/oberdiek/catalogue}%
  \file{pdfcolmk.xml}{\from{pdfcolmk.dtx}{catalogue}}%
}

\catcode32=13\relax% active space
\let =\space%
\Msg{************************************************************************}
\Msg{*}
\Msg{* To finish the installation you have to move the following}
\Msg{* file into a directory searched by TeX:}
\Msg{*}
\Msg{*     pdfcolmk.sty}
\Msg{*}
\Msg{* To produce the documentation run the file `pdfcolmk.drv'}
\Msg{* through LaTeX.}
\Msg{*}
\Msg{* Happy TeXing!}
\Msg{*}
\Msg{************************************************************************}

\endbatchfile
%</install>
%<*ignore>
\fi
%</ignore>
%<*driver>
\NeedsTeXFormat{LaTeX2e}
\ProvidesFile{pdfcolmk.drv}%
  [2008/08/11 v1.2 Color support for pdfTeX via marks (HO)]%
\documentclass{ltxdoc}
\usepackage{holtxdoc}[2011/11/22]
\begin{document}
  \DocInput{pdfcolmk.dtx}%
\end{document}
%</driver>
% \fi
%
% \CheckSum{843}
%
% \CharacterTable
%  {Upper-case    \A\B\C\D\E\F\G\H\I\J\K\L\M\N\O\P\Q\R\S\T\U\V\W\X\Y\Z
%   Lower-case    \a\b\c\d\e\f\g\h\i\j\k\l\m\n\o\p\q\r\s\t\u\v\w\x\y\z
%   Digits        \0\1\2\3\4\5\6\7\8\9
%   Exclamation   \!     Double quote  \"     Hash (number) \#
%   Dollar        \$     Percent       \%     Ampersand     \&
%   Acute accent  \'     Left paren    \(     Right paren   \)
%   Asterisk      \*     Plus          \+     Comma         \,
%   Minus         \-     Point         \.     Solidus       \/
%   Colon         \:     Semicolon     \;     Less than     \<
%   Equals        \=     Greater than  \>     Question mark \?
%   Commercial at \@     Left bracket  \[     Backslash     \\
%   Right bracket \]     Circumflex    \^     Underscore    \_
%   Grave accent  \`     Left brace    \{     Vertical bar  \|
%   Right brace   \}     Tilde         \~}
%
% \GetFileInfo{pdfcolmk.drv}
%
% \title{The \xpackage{pdfcolmk} package}
% \date{2008/08/11 v1.2}
% \author{Heiko Oberdiek\\\xemail{heiko.oberdiek at googlemail.com}}
%
% \maketitle
%
% \begin{abstract}
% This package tries a solution for the missing color
% stack of \pdfTeX.
% \end{abstract}
%
% \tableofcontents
%
% \section{Documentation}
%
% \subsection{Introduction}
%
% This package uses a mark register in order to solve the
% problem of a missing color stack of \pdfTeX\ prior 1.40.0.
% Since this version of \pdfTeX\ a color stack is available
% and supported by \xfile{pdftex.def} 2007/01/01 v0.04a.
% In this case this package is obsolete and the package
% stops its loading.
%
% \subsection{Background}
%
% After the Dante meeting (Clausthal 2000) I have started
% to experiment with the eTeX method of a \emph{colour} mark.
% One of the major problems is the understanding of the
% output routine and the need to rewrite it because of
% missing hooks. Currently I have made some tests in
% in onecolumn and twocolumn mode, but the state is
% experimental.
%
% \subsection{Limitations}
%
% \begin{itemize}
% \item Mark limitations: page breaks in math.
% \item \LaTeX's output routine is redefinded.
%   \begin{itemize}
%   \item Changes in the output routine of newer versions
%         of LaTeX are not detected.
%   \item Packages that change the output routine are not
%         supported.
%   \end{itemize}
% \item It does not support several independent text
%       streams like footnotes.
% \item Limitations in float and marginpar support.
% \end{itemize}
%
% \subsection{Recommendation}
%
% \eTeX\ (for additional mark register)
% Without \eTeX\ \LaTeX's mark commands are redefined
% to store an additional color value.
%
% \subsection{Usage}
%
% Load after package color:
% \begin{quote}
%   |\usepackage[pdftex]{color}|\\
%   |\usepackage{pdfcolmk}|
% \end{quote}
%
% \subsection{Compatibility}
%
% \begin{itemize}
% \item Load the following packages after \xpackage{pdfcolmk}:
%   \begin{quote}
%       \xpackage{mparhack.sty}
%   \end{quote}
% \item Load the following packages before \xpackage{pdfcolmk}:
%   \begin{quote}
%       \xpackage{marn.sty}\\
%       \xpackage{newmarn.sty}
%   \end{quote}
% \item Supported \cs{@addmarginpar} patch:
%   \begin{quote}
%       \xpackage{latex/base/latex.ltx}\\
%       \xpackage{memoir.cls}\\
%       \xpackage{poemscol/marn.sty}, \xpackage{poemscol/newmarn.sty}\\
%       \xpackage{mparhack.sty}
%   \end{quote}
% \item Unsupported \cs{@addmarginpar} patch:
%   \begin{quote}
%       \xpackage{lineno.sty}\\
%       \xpackage{sttools/marginal.sty}\\
%       \xpackage{revtex4.cls}
%   \end{quote}
% \end{itemize}
%
% \StopEventually{
% }
%
% \section{Implementation}
%
%    \begin{macrocode}
%<*package>
%    \end{macrocode}
%    Package identification.
%    \begin{macrocode}
\NeedsTeXFormat{LaTeX2e}
\ProvidesPackage{pdfcolmk}%
  [2008/08/11 v1.2 Color support for pdfTeX via marks (HO)]
%    \end{macrocode}
%
%    \begin{macrocode}
\@ifundefined{ver@pdftex.def}{%
  \PackageWarningNoLine{pdfcolmk}{%
    Nothing to fix, because \string`pdftex.def\string' not loaded%
  }%
  \endinput
}{}
\@ifpackageloaded{color}{}{%
  \PackageWarningNoLine{pdfcolmk}{%
    Nothing to fix, because \string`color.sty\string' not loaded%
  }%
  \endinput
}
\begingroup\expandafter\expandafter\expandafter\endgroup
\expandafter\ifx\csname main@pdfcolorstack\endcsname\relax
\else
  % pdftex.def >= 2007/01/01 0.04a and pdfTeX >= 1.40.0
  \begingroup
    \let\on@line\@empty
    \PackageInfo{pdfcolmk}{%
      The color stack of pdfTeX \string>\string= 1.40 is used. %
      Therefore\MessageBreak
      this package is not necessary and not loaded%
    }%
  \endgroup
  \expandafter\endinput
\fi

\PackageInfo{pdfcolmk}{%
  This package tries to simulate dvips's color stack\MessageBreak
  for pdfTeX based on a mark register of e-TeX.\MessageBreak
  It redefines LaTeX's output routine. Therefore\MessageBreak
  use with care, no warranties%
}

\ifx\marks\@undefined

  \let\pec@mark\mark
  \let\pec@value\empty
  \long\def\mark#1{%
    \protected@xdef\pec@value{#1}%
    \pec@setmark
  }%
  \def\pec@setmark{%
    \begingroup
      \@temptokena\expandafter{\pec@value}%
      \pec@mark{{\current@color}\the\@temptokena}%
    \endgroup
  }%
  \def\pec@getmark{%
    \xdef\pec@botcolor{%
      \expandafter\@firstofthree\botmark\@empty\@empty\@empty
    }%
  }%
  \long\def\@firstofthree#1#2#3{#1}%
  \CheckCommand{\@leftmark}[2]{#1}%
  \CheckCommand{\@rightmark}[2]{#2}%
  \CheckCommand*{\leftmark}{%
    \expandafter\@leftmark\botmark\@empty\@empty
  }%
  \CheckCommand*{\rightmark}{%
    \expandafter\@rightmark\firstmark\@empty\@empty
  }%
  \long\def\@leftmark#1#2#3{#2}%
  \long\def\@rightmark#1#2#3{#3}%
  \g@addto@macro\leftmark\@empty
  \g@addto@macro\rightmark\@empty

\else

  \RequirePackage{etex}[1998/03/26]%
  \newmarks\pec@marks
  \def\pec@setmark{\marks\pec@marks{\current@color}}%
  \def\pec@getmark{\xdef\pec@botcolor{\botmarks\pec@marks}}%

\fi
%    \end{macrocode}
%
% \subsection{\cs{marginpar} fix}
%
%    \begin{macrocode}
\chardef\pec@result\z@
\def\pec@temp#1{%
  \chardef\pec@result\@ne
  \begingroup
    \let\on@line\@empty
    \PackageInfo{pdfcolmk}{%
      Patch for \string\@addmarginpar\space applied (#1)%
    }%
  \endgroup
}
%    \end{macrocode}
%
% \subsubsection{latex/base/latex.ltx}
%
%    \begin{macrocode}
\def\pec@addmarginpar{%
  \@next\@marbox\@currlist{%
    \@cons\@freelist\@marbox
    \@cons\@freelist\@currbox
  }\@latexbug
  \@tempcnta\@ne
  \if@twocolumn
    \if@firstcolumn
      \@tempcnta\m@ne
    \fi
  \else
    \if@mparswitch
      \ifodd\c@page
      \else
        \@tempcnta\m@ne
      \fi
    \fi
    \if@reversemargin \@tempcnta -\@tempcnta \fi
  \fi
  \ifnum\@tempcnta <\z@  \global\setbox\@marbox\box\@currbox \fi
  \@tempdima\@mparbottom
  \advance\@tempdima -\@pageht
  \advance\@tempdima\ht\@marbox
  \ifdim\@tempdima >\z@
    \@latex@warning@no@line{Marginpar on page \thepage\space moved}%
  \else
    \@tempdima\z@
  \fi
  \global\@mparbottom\@pageht
  \global\advance\@mparbottom\@tempdima
  \global\advance\@mparbottom\dp\@marbox
  \global\advance\@mparbottom\marginparpush
  \advance\@tempdima -\ht\@marbox
  \global\setbox\@marbox\vbox{%
    \vskip \@tempdima
    \box \@marbox
  }%
  \global \ht\@marbox \z@
  \global \dp\@marbox \z@
  \kern -\@pagedp
  \nointerlineskip
  \hb@xt@\columnwidth{%
    \ifnum \@tempcnta >\z@
      \hskip\columnwidth
      \hskip\marginparsep
    \else
      \hskip -\marginparsep
      \hskip -\marginparwidth
    \fi
    \box\@marbox \hss
  }%
  \nointerlineskip
  \hbox{\vrule \@height\z@ \@width\z@ \@depth\@pagedp}%
}
\ifx\pec@addmarginpar\@addmarginpar
  \pec@temp{latex/base}%
\fi
%    \end{macrocode}
%
% \subsubsection{memoir.cls}
%
%    \begin{macrocode}
\def\pec@addmarginpar{%
  \checkoddpage
  \@next\@marbox\@currlist{%
    \@cons\@freelist\@marbox
    \@cons\@freelist\@currbox
  }\@latexbug
  \@tempcnta\@ne
  \if@twocolumn
    \if@firstcolumn
      \@tempcnta\m@ne
    \fi
  \else
    \if@mparswitch
      \ifoddpage
      \else
        \@tempcnta\m@ne
      \fi
    \fi
    \if@reversemargin
      \@tempcnta -\@tempcnta
    \fi
  \fi
  \ifnum\@tempcnta <\z@
    \global\setbox\@marbox\box\@currbox
  \fi
  \@tempdima\@mparbottom
  \advance\@tempdima -\@pageht
  \advance\@tempdima\ht\@marbox
  \ifdim\@tempdima >\z@
    \@latex@warning@no@line{%
      Marginpar on page \thepage\space moved by \the\@tempdima
    }%
  \else
    \@tempdima\z@
  \fi
  \global\@mparbottom\@pageht
  \global\advance\@mparbottom\@tempdima
  \global\advance\@mparbottom\dp\@marbox
  \global\advance\@mparbottom\marginparpush
  \advance\@tempdima -\ht\@marbox
  \global\setbox\@marbox\vbox{%
    \vskip \@tempdima
    \box \@marbox
  }%
  \global \ht\@marbox \z@
  \global \dp\@marbox \z@
  \kern -\@pagedp
  \nointerlineskip
  \hb@xt@\columnwidth{%
    \ifnum \@tempcnta >\z@
      \hskip\columnwidth
      \hskip\marginparsep
    \else
      \hskip -\marginparsep
      \hskip -\marginparwidth
    \fi
    \box\@marbox
    \hss
  }%
  \nointerlineskip
  \hbox{\vrule \@height\z@ \@width\z@ \@depth\@pagedp}%
}%
\ifx\pec@addmarginpar\@addmarginpar
  \pec@temp{memoir.cls}%
\fi
%    \end{macrocode}
%
% \subsubsection{poemscol/marn.sty, poemscol/newmarn.sty}
%
%    \begin{macrocode}
\def\pec@addmarginpar{%
  \@next \@marbox\@currlist{%
    \@cons\@freelist\@marbox
    \@cons\@freelist\@currbox
  }\@latexbug
  \global\advance\@mpar@count\m@ne
  \@ifundefined{@marn@\the\@mpar@count @}{% was location logged last time?
    \@tempcnta\@ne % NO: use original LaTeX logic
    \if@twocolumn
      \if@firstcolumn
        \@tempcnta\m@ne
      \fi
    \else
      \if@mparswitch
        \ifodd\c@page
        \else
          \@tempcnta\m@ne
        \fi
      \fi
      \if@reversemargin
        \@tempcnta -\@tempcnta
      \fi
    \fi
  }{%
    \@tempcnta %    YES: use record from last time to decide side.
    \@nameuse{@marn@\the\@mpar@count @}%
    \if@reversemargin -\fi \@ne
  }%
  \ifnum\@tempcnta <\z@
    \global\setbox\@marbox\box\@currbox
    \global\let\@marnbottom\@mparbottoml
  \else
    \global\let\@marnbottom\@mparbottom
  \fi
  \@tempdima\@marnbottom \advance\@tempdima -\@pageht
  \advance\@tempdima\ht\@marbox
  \ifdim\@tempdima >\z@
    \@@warning{Marginpar on page \thepage\space moved}%
  \else
    \@tempdima\z@
  \fi
  \global\@marnbottom\@pageht
  \global\advance\@marnbottom\@tempdima
  \global\advance\@marnbottom\dp\@marbox
  \global\advance\@marnbottom\marginparpush
  \advance\@tempdima -\ht\@marbox
  \global\ht\@marbox\z@
  \global\dp\@marbox\z@
  \vskip -\@pagedp
  \vskip\@tempdima\nointerlineskip
  \hbox to\columnwidth{%
    \ifnum \@tempcnta >\z@
      \hskip\columnwidth
      \hskip\marginparsep
    \else
      \hskip -\marginparsep
      \hskip -\marginparwidth
    \fi
    \if@filesw % record where this is for use next time:
       \@marn@log\@mpar@count
    \fi
    \box\@marbox
    \hss
  }%
  \nobreak   %% RmS 91/06/21 \nobreak added
  \vskip -\@tempdima
  \nointerlineskip
  \hbox{\vrule \@height\z@ \@width\z@ \@depth\@pagedp}%
}
\ifx\pec@addmarginpar\@addmarginpar
  \pec@temp{poemscol/(new)marn.sty}%
\fi
%    \end{macrocode}
%
% \subsubsection{refman/refart.cls, refnam/refrep.cls}
%
%    \begin{macrocode}
\def\pec@addmarginpar{%
  \@next\@marbox\@currlist{%
    \@cons\@freelist\@marbox
    \@cons\@freelist\@currbox
  }\@latexbug
  \@tempcnta\@ne
  \if@twocolumn
    \if@firstcolumn
      \@tempcnta\m@ne
    \fi
  \else
    \@tempcnta\m@ne
  \fi
  \ifnum\@tempcnta <\z@
    \global\setbox\@marbox\box\@currbox
  \fi
  \@tempdima\@mparbottom
  \advance\@tempdima -\@pageht
  \advance\@tempdima\ht\@marbox
  \ifdim\@tempdima >\z@
     \@@warning{Marginpar on page \thepage\space moved}%
  \else
     \@tempdima\z@
  \fi
  \global\@mparbottom\@pageht
  \global\advance\@mparbottom\@tempdima
  \global\advance\@mparbottom\dp\@marbox
  \global\advance\@mparbottom\marginparpush
  \advance\@tempdima -\ht\@marbox
  \global\setbox\@marbox\vbox{%
    \vskip \@tempdima \box \@marbox
  }%
  \global \ht\@marbox \z@
  \global \dp\@marbox \z@
  \kern -\@pagedp
  \nointerlineskip
  \hb@xt@\columnwidth{%
    \ifnum \@tempcnta >\z@
      \hskip\columnwidth
      \hskip\marginparsep
    \else
      \hskip -\marginparsep
      \hskip -\marginparwidth
    \fi
    \box\@marbox
    \hss
  }%
  \nointerlineskip
  \hbox{\vrule \@height\z@ \@width\z@ \@depth\@pagedp}%
}
\ifx\pec@addmarginpar\@addmarginpar
  \pec@temp{ref(art|rep).cls}%
\fi

\ifcase\pec@result
  \PackageInfo{pdfcolmk}{%
    Fix for \string\@addmarginpar\space is omitted, %
    because this variant\MessageBreak
    of \string\@addmarginpar\space
      is not recognized%
  }%
\else
  % apply patch for \@addmarginpar
  \def\pec@PatchAddMarginpar#1\columnwidth#2#3\@nil{%
    \pec@PatchAddMarginparI#2\@nil{#1}{#3}%
  }%
  \def\pec@PatchAddMarginparI#1\box\@marbox\hss#2\@nil#3#4{%
    \def\@addmarginpar{%
      #3%
      \columnwidth{%
        #1%
        \pdfliteral{q}%
        \rlap{%
          \box\@marbox
        }%
        \pdfliteral{Q}%
        \hss
        #2%
      }%
      #4%
    }%
  }%
  \expandafter\pec@PatchAddMarginpar\@addmarginpar\@nil
\fi
%    \end{macrocode}
%
% \subsection{Color fix}
%
%    \begin{macrocode}
\def\set@color{%
  \pdfliteral{\current@color}%
  \ifinner
  \else
    \pec@setmark
  \fi
  \aftergroup\reset@color
}
\def\reset@color{%
  \pdfliteral{\current@color}%
  \ifinner
  \else
    \pec@setmark
  \fi
}

\let\pec@botcolor\current@color

\def\pec@PatchVBoxCCLV{%
  \ifx\pec@botcolor\@empty
  \else
    \setbox\@cclv\vbox{%
      \pdfliteral{\pec@botcolor}%
      \unvbox\@cclv
    }%
  \fi
  \pec@getmark
}

\def\pec@PatchAlreadyInBox{%
  \ifx\pec@botcolor\@empty
  \else
    \pdfliteral{\pec@botcolor}%
  \fi
  \pec@getmark
}

\@ifclassloaded{memoir}{%
  \expandafter\def\expandafter\mem@makecol\expandafter{%
    \expandafter\pec@PatchVBoxCCLV
    \mem@makecol
  }%
  \endinput
}{}

\@ifclassloaded{seminar}{%
  \newcommand\pec@org@makeslide{}%
  \let\pec@org@makeslide\@makeslide
  \def\@makeslide{%
    \pec@PatchVBoxCCLV
    \pec@org@makeslide
  }%
  \endinput
}{}

\long\def\pec@output#1\@specialoutput\else#2\pec@end{%
  \begingroup
    \def\x{#2}%
  \expandafter\endgroup
  \ifx\x\@empty
    \PackageWarningNoLine{pdfcolmk}{%
      Unexpected \string\output\space routine detected,%
      \MessageBreak
      loading of package stopped%
    }%
    \expandafter\endinput
  \fi
}
\expandafter\expandafter\expandafter\pec@output
\expandafter\@firstofone\the\output\@specialoutput\else\pec@end

\long\def\pec@output#1\@specialoutput\else#2\pec@end{%
  \output{%
    #1\@specialoutput\else
    \pec@PatchVBoxCCLV
    #2%
  }%
}
\expandafter\expandafter\expandafter\pec@output
\expandafter\@firstofone\the\output\pec@end
%    \end{macrocode}
%
%    \begin{macrocode}
%</package>
%    \end{macrocode}
%
% \section{Installation}
%
% \subsection{Download}
%
% \paragraph{Package.} This package is available on
% CTAN\footnote{\url{ftp://ftp.ctan.org/tex-archive/}}:
% \begin{description}
% \item[\CTAN{macros/latex/contrib/oberdiek/pdfcolmk.dtx}] The source file.
% \item[\CTAN{macros/latex/contrib/oberdiek/pdfcolmk.pdf}] Documentation.
% \end{description}
%
%
% \paragraph{Bundle.} All the packages of the bundle `oberdiek'
% are also available in a TDS compliant ZIP archive. There
% the packages are already unpacked and the documentation files
% are generated. The files and directories obey the TDS standard.
% \begin{description}
% \item[\CTAN{install/macros/latex/contrib/oberdiek.tds.zip}]
% \end{description}
% \emph{TDS} refers to the standard ``A Directory Structure
% for \TeX\ Files'' (\CTAN{tds/tds.pdf}). Directories
% with \xfile{texmf} in their name are usually organized this way.
%
% \subsection{Bundle installation}
%
% \paragraph{Unpacking.} Unpack the \xfile{oberdiek.tds.zip} in the
% TDS tree (also known as \xfile{texmf} tree) of your choice.
% Example (linux):
% \begin{quote}
%   |unzip oberdiek.tds.zip -d ~/texmf|
% \end{quote}
%
% \paragraph{Script installation.}
% Check the directory \xfile{TDS:scripts/oberdiek/} for
% scripts that need further installation steps.
% Package \xpackage{attachfile2} comes with the Perl script
% \xfile{pdfatfi.pl} that should be installed in such a way
% that it can be called as \texttt{pdfatfi}.
% Example (linux):
% \begin{quote}
%   |chmod +x scripts/oberdiek/pdfatfi.pl|\\
%   |cp scripts/oberdiek/pdfatfi.pl /usr/local/bin/|
% \end{quote}
%
% \subsection{Package installation}
%
% \paragraph{Unpacking.} The \xfile{.dtx} file is a self-extracting
% \docstrip\ archive. The files are extracted by running the
% \xfile{.dtx} through \plainTeX:
% \begin{quote}
%   \verb|tex pdfcolmk.dtx|
% \end{quote}
%
% \paragraph{TDS.} Now the different files must be moved into
% the different directories in your installation TDS tree
% (also known as \xfile{texmf} tree):
% \begin{quote}
% \def\t{^^A
% \begin{tabular}{@{}>{\ttfamily}l@{ $\rightarrow$ }>{\ttfamily}l@{}}
%   pdfcolmk.sty & tex/latex/oberdiek/pdfcolmk.sty\\
%   pdfcolmk.pdf & doc/latex/oberdiek/pdfcolmk.pdf\\
%   pdfcolmk.dtx & source/latex/oberdiek/pdfcolmk.dtx\\
% \end{tabular}^^A
% }^^A
% \sbox0{\t}^^A
% \ifdim\wd0>\linewidth
%   \begingroup
%     \advance\linewidth by\leftmargin
%     \advance\linewidth by\rightmargin
%   \edef\x{\endgroup
%     \def\noexpand\lw{\the\linewidth}^^A
%   }\x
%   \def\lwbox{^^A
%     \leavevmode
%     \hbox to \linewidth{^^A
%       \kern-\leftmargin\relax
%       \hss
%       \usebox0
%       \hss
%       \kern-\rightmargin\relax
%     }^^A
%   }^^A
%   \ifdim\wd0>\lw
%     \sbox0{\small\t}^^A
%     \ifdim\wd0>\linewidth
%       \ifdim\wd0>\lw
%         \sbox0{\footnotesize\t}^^A
%         \ifdim\wd0>\linewidth
%           \ifdim\wd0>\lw
%             \sbox0{\scriptsize\t}^^A
%             \ifdim\wd0>\linewidth
%               \ifdim\wd0>\lw
%                 \sbox0{\tiny\t}^^A
%                 \ifdim\wd0>\linewidth
%                   \lwbox
%                 \else
%                   \usebox0
%                 \fi
%               \else
%                 \lwbox
%               \fi
%             \else
%               \usebox0
%             \fi
%           \else
%             \lwbox
%           \fi
%         \else
%           \usebox0
%         \fi
%       \else
%         \lwbox
%       \fi
%     \else
%       \usebox0
%     \fi
%   \else
%     \lwbox
%   \fi
% \else
%   \usebox0
% \fi
% \end{quote}
% If you have a \xfile{docstrip.cfg} that configures and enables \docstrip's
% TDS installing feature, then some files can already be in the right
% place, see the documentation of \docstrip.
%
% \subsection{Refresh file name databases}
%
% If your \TeX~distribution
% (\teTeX, \mikTeX, \dots) relies on file name databases, you must refresh
% these. For example, \teTeX\ users run \verb|texhash| or
% \verb|mktexlsr|.
%
% \subsection{Some details for the interested}
%
% \paragraph{Attached source.}
%
% The PDF documentation on CTAN also includes the
% \xfile{.dtx} source file. It can be extracted by
% AcrobatReader 6 or higher. Another option is \textsf{pdftk},
% e.g. unpack the file into the current directory:
% \begin{quote}
%   \verb|pdftk pdfcolmk.pdf unpack_files output .|
% \end{quote}
%
% \paragraph{Unpacking with \LaTeX.}
% The \xfile{.dtx} chooses its action depending on the format:
% \begin{description}
% \item[\plainTeX:] Run \docstrip\ and extract the files.
% \item[\LaTeX:] Generate the documentation.
% \end{description}
% If you insist on using \LaTeX\ for \docstrip\ (really,
% \docstrip\ does not need \LaTeX), then inform the autodetect routine
% about your intention:
% \begin{quote}
%   \verb|latex \let\install=y\input{pdfcolmk.dtx}|
% \end{quote}
% Do not forget to quote the argument according to the demands
% of your shell.
%
% \paragraph{Generating the documentation.}
% You can use both the \xfile{.dtx} or the \xfile{.drv} to generate
% the documentation. The process can be configured by the
% configuration file \xfile{ltxdoc.cfg}. For instance, put this
% line into this file, if you want to have A4 as paper format:
% \begin{quote}
%   \verb|\PassOptionsToClass{a4paper}{article}|
% \end{quote}
% An example follows how to generate the
% documentation with pdf\LaTeX:
% \begin{quote}
%\begin{verbatim}
%pdflatex pdfcolmk.dtx
%makeindex -s gind.ist pdfcolmk.idx
%pdflatex pdfcolmk.dtx
%makeindex -s gind.ist pdfcolmk.idx
%pdflatex pdfcolmk.dtx
%\end{verbatim}
% \end{quote}
%
% \section{Catalogue}
%
% The following XML file can be used as source for the
% \href{http://mirror.ctan.org/help/Catalogue/catalogue.html}{\TeX\ Catalogue}.
% The elements \texttt{caption} and \texttt{description} are imported
% from the original XML file from the Catalogue.
% The name of the XML file in the Catalogue is \xfile{pdfcolmk.xml}.
%    \begin{macrocode}
%<*catalogue>
<?xml version='1.0' encoding='us-ascii'?>
<!DOCTYPE entry SYSTEM 'catalogue.dtd'>
<entry datestamp='$Date$' modifier='$Author$' id='pdfcolmk'>
  <name>pdfcolmk</name>
  <caption>Improving colour support under pdftex.</caption>
  <authorref id='auth:oberdiek'/>
  <copyright owner='Heiko Oberdiek' year='2000,2005-2008'/>
  <license type='lppl1.3'/>
  <version number='1.2'/>
  <description>
    The package provides macros that emulate the &#x2018;colour stack&#x2019;
    functionality of dvips.  The colour stack deals with colour
    manipulations when asynchronous events (like page-breaking) occur;
    pdftex does not (yet) have such a stack, but dvips does, and the
    <xref refid='color'>color</xref> package makes extensive use of
    it.
    <p/>
    This package is an experimental solution to the problem, and works
    best with pdf-e-tex.
    <p/>
    The package is part of the <xref refid='oberdiek'>oberdiek</xref> bundle.
  </description>
  <documentation details='Package documentation'
      href='ctan:/macros/latex/contrib/oberdiek/pdfcolmk.pdf'/>
  <ctan file='true' path='/macros/latex/contrib/oberdiek/pdfcolmk.dtx'/>
  <miktex location='oberdiek'/>
  <texlive location='oberdiek'/>
  <install path='/macros/latex/contrib/oberdiek/oberdiek.tds.zip'/>
</entry>
%</catalogue>
%    \end{macrocode}
%
% \begin{History}
%   \begin{Version}{2000/08/27 v0.1}
%   \item
%     First published version in newsgroup \xnewsgroup{comp.text.tex}:\\
%     \URL{``\link{pdftex: bug with colors?}''}^^A
%     {http://groups.google.com/group/comp.text.tex/msg/6f088e69e4085d2c}
%   \end{Version}
%   \begin{Version}{2000/09/02 v0.2}
%   \item
%     Next try.
%   \end{Version}
%   \begin{Version}{2000/09/02 v0.3}
%   \item
%     Solution without \eTeX\ added.
%   \end{Version}
%   \begin{Version}{2000/09/06 v0.4}
%   \item
%     Patch commands added.
%   \item
%     Patch for seminar.cls added.
%   \end{Version}
%   \begin{Version}{2000/09/06 v0.5}
%   \item
%     Bug fix: initialization of \cs{pec@value} added.
%   \end{Version}
%   \begin{Version}{2005/06/15 v0.6}
%   \item
%     Support for \cs{marginpar} added.
%     See thread in \xnewsgroup{comp.text.tex}:\\
%     \URL{``\link{Using \cs{textcolor} and \cs{marginpar} together}''}^^A
%     {http://groups.google.com/group/comp.text.tex/msg/38ed58f8845a2a4f}
%   \end{Version}
%   \begin{Version}{2005/07/09 v0.7}
%   \item
%     Output support added for \xpackage{memoir},
%     provided by Lars Madsen.
%   \end{Version}
%   \begin{Version}{2006/02/20 v0.8}
%   \item
%     Code is not changed.
%   \item
%     DTX framework.
%   \end{Version}
%   \begin{Version}{2007/01/01 v1.0}
%   \item
%     If \xfile{pdftex.def} \textgreater= 2007/01/01 v0.04a is used with
%     \pdfTeX\ \textgreater= 1.40.0, then package \xpackage{pdfcolmk} is obsolete.
%   \end{Version}
%   \begin{Version}{2007/04/11 v1.1}
%   \item
%     Line ends sanitized.
%   \end{Version}
%   \begin{Version}{2008/08/11 v1.2}
%   \item
%     Code is not changed.
%   \item
%     URLs updated.
%   \end{Version}
% \end{History}
%
% \PrintIndex
%
% \Finale
\endinput
|
% \end{quote}
% Do not forget to quote the argument according to the demands
% of your shell.
%
% \paragraph{Generating the documentation.}
% You can use both the \xfile{.dtx} or the \xfile{.drv} to generate
% the documentation. The process can be configured by the
% configuration file \xfile{ltxdoc.cfg}. For instance, put this
% line into this file, if you want to have A4 as paper format:
% \begin{quote}
%   \verb|\PassOptionsToClass{a4paper}{article}|
% \end{quote}
% An example follows how to generate the
% documentation with pdf\LaTeX:
% \begin{quote}
%\begin{verbatim}
%pdflatex pdfcolmk.dtx
%makeindex -s gind.ist pdfcolmk.idx
%pdflatex pdfcolmk.dtx
%makeindex -s gind.ist pdfcolmk.idx
%pdflatex pdfcolmk.dtx
%\end{verbatim}
% \end{quote}
%
% \section{Catalogue}
%
% The following XML file can be used as source for the
% \href{http://mirror.ctan.org/help/Catalogue/catalogue.html}{\TeX\ Catalogue}.
% The elements \texttt{caption} and \texttt{description} are imported
% from the original XML file from the Catalogue.
% The name of the XML file in the Catalogue is \xfile{pdfcolmk.xml}.
%    \begin{macrocode}
%<*catalogue>
<?xml version='1.0' encoding='us-ascii'?>
<!DOCTYPE entry SYSTEM 'catalogue.dtd'>
<entry datestamp='$Date$' modifier='$Author$' id='pdfcolmk'>
  <name>pdfcolmk</name>
  <caption>Improving colour support under pdftex.</caption>
  <authorref id='auth:oberdiek'/>
  <copyright owner='Heiko Oberdiek' year='2000,2005-2008'/>
  <license type='lppl1.3'/>
  <version number='1.2'/>
  <description>
    The package provides macros that emulate the &#x2018;colour stack&#x2019;
    functionality of dvips.  The colour stack deals with colour
    manipulations when asynchronous events (like page-breaking) occur;
    pdftex does not (yet) have such a stack, but dvips does, and the
    <xref refid='color'>color</xref> package makes extensive use of
    it.
    <p/>
    This package is an experimental solution to the problem, and works
    best with pdf-e-tex.
    <p/>
    The package is part of the <xref refid='oberdiek'>oberdiek</xref> bundle.
  </description>
  <documentation details='Package documentation'
      href='ctan:/macros/latex/contrib/oberdiek/pdfcolmk.pdf'/>
  <ctan file='true' path='/macros/latex/contrib/oberdiek/pdfcolmk.dtx'/>
  <miktex location='oberdiek'/>
  <texlive location='oberdiek'/>
  <install path='/macros/latex/contrib/oberdiek/oberdiek.tds.zip'/>
</entry>
%</catalogue>
%    \end{macrocode}
%
% \begin{History}
%   \begin{Version}{2000/08/27 v0.1}
%   \item
%     First published version in newsgroup \xnewsgroup{comp.text.tex}:\\
%     \URL{``\link{pdftex: bug with colors?}''}^^A
%     {http://groups.google.com/group/comp.text.tex/msg/6f088e69e4085d2c}
%   \end{Version}
%   \begin{Version}{2000/09/02 v0.2}
%   \item
%     Next try.
%   \end{Version}
%   \begin{Version}{2000/09/02 v0.3}
%   \item
%     Solution without \eTeX\ added.
%   \end{Version}
%   \begin{Version}{2000/09/06 v0.4}
%   \item
%     Patch commands added.
%   \item
%     Patch for seminar.cls added.
%   \end{Version}
%   \begin{Version}{2000/09/06 v0.5}
%   \item
%     Bug fix: initialization of \cs{pec@value} added.
%   \end{Version}
%   \begin{Version}{2005/06/15 v0.6}
%   \item
%     Support for \cs{marginpar} added.
%     See thread in \xnewsgroup{comp.text.tex}:\\
%     \URL{``\link{Using \cs{textcolor} and \cs{marginpar} together}''}^^A
%     {http://groups.google.com/group/comp.text.tex/msg/38ed58f8845a2a4f}
%   \end{Version}
%   \begin{Version}{2005/07/09 v0.7}
%   \item
%     Output support added for \xpackage{memoir},
%     provided by Lars Madsen.
%   \end{Version}
%   \begin{Version}{2006/02/20 v0.8}
%   \item
%     Code is not changed.
%   \item
%     DTX framework.
%   \end{Version}
%   \begin{Version}{2007/01/01 v1.0}
%   \item
%     If \xfile{pdftex.def} \textgreater= 2007/01/01 v0.04a is used with
%     \pdfTeX\ \textgreater= 1.40.0, then package \xpackage{pdfcolmk} is obsolete.
%   \end{Version}
%   \begin{Version}{2007/04/11 v1.1}
%   \item
%     Line ends sanitized.
%   \end{Version}
%   \begin{Version}{2008/08/11 v1.2}
%   \item
%     Code is not changed.
%   \item
%     URLs updated.
%   \end{Version}
% \end{History}
%
% \PrintIndex
%
% \Finale
\endinput

%        (quote the arguments according to the demands of your shell)
%
% Documentation:
%    (a) If pdfcolmk.drv is present:
%           latex pdfcolmk.drv
%    (b) Without pdfcolmk.drv:
%           latex pdfcolmk.dtx; ...
%    The class ltxdoc loads the configuration file ltxdoc.cfg
%    if available. Here you can specify further options, e.g.
%    use A4 as paper format:
%       \PassOptionsToClass{a4paper}{article}
%
%    Programm calls to get the documentation (example):
%       pdflatex pdfcolmk.dtx
%       makeindex -s gind.ist pdfcolmk.idx
%       pdflatex pdfcolmk.dtx
%       makeindex -s gind.ist pdfcolmk.idx
%       pdflatex pdfcolmk.dtx
%
% Installation:
%    TDS:tex/latex/oberdiek/pdfcolmk.sty
%    TDS:doc/latex/oberdiek/pdfcolmk.pdf
%    TDS:source/latex/oberdiek/pdfcolmk.dtx
%
%<*ignore>
\begingroup
  \catcode123=1 %
  \catcode125=2 %
  \def\x{LaTeX2e}%
\expandafter\endgroup
\ifcase 0\ifx\install y1\fi\expandafter
         \ifx\csname processbatchFile\endcsname\relax\else1\fi
         \ifx\fmtname\x\else 1\fi\relax
\else\csname fi\endcsname
%</ignore>
%<*install>
\input docstrip.tex
\Msg{************************************************************************}
\Msg{* Installation}
\Msg{* Package: pdfcolmk 2008/08/11 v1.2 Color support for pdfTeX via marks (HO)}
\Msg{************************************************************************}

\keepsilent
\askforoverwritefalse

\let\MetaPrefix\relax
\preamble

This is a generated file.

Project: pdfcolmk
Version: 2008/08/11 v1.2

Copyright (C) 2000, 2005-2008 by
   Heiko Oberdiek <heiko.oberdiek at googlemail.com>

This work may be distributed and/or modified under the
conditions of the LaTeX Project Public License, either
version 1.3c of this license or (at your option) any later
version. This version of this license is in
   http://www.latex-project.org/lppl/lppl-1-3c.txt
and the latest version of this license is in
   http://www.latex-project.org/lppl.txt
and version 1.3 or later is part of all distributions of
LaTeX version 2005/12/01 or later.

This work has the LPPL maintenance status "maintained".

This Current Maintainer of this work is Heiko Oberdiek.

This work consists of the main source file pdfcolmk.dtx
and the derived files
   pdfcolmk.sty, pdfcolmk.pdf, pdfcolmk.ins, pdfcolmk.drv.

\endpreamble
\let\MetaPrefix\DoubleperCent

\generate{%
  \file{pdfcolmk.ins}{\from{pdfcolmk.dtx}{install}}%
  \file{pdfcolmk.drv}{\from{pdfcolmk.dtx}{driver}}%
  \usedir{tex/latex/oberdiek}%
  \file{pdfcolmk.sty}{\from{pdfcolmk.dtx}{package}}%
  \nopreamble
  \nopostamble
  \usedir{source/latex/oberdiek/catalogue}%
  \file{pdfcolmk.xml}{\from{pdfcolmk.dtx}{catalogue}}%
}

\catcode32=13\relax% active space
\let =\space%
\Msg{************************************************************************}
\Msg{*}
\Msg{* To finish the installation you have to move the following}
\Msg{* file into a directory searched by TeX:}
\Msg{*}
\Msg{*     pdfcolmk.sty}
\Msg{*}
\Msg{* To produce the documentation run the file `pdfcolmk.drv'}
\Msg{* through LaTeX.}
\Msg{*}
\Msg{* Happy TeXing!}
\Msg{*}
\Msg{************************************************************************}

\endbatchfile
%</install>
%<*ignore>
\fi
%</ignore>
%<*driver>
\NeedsTeXFormat{LaTeX2e}
\ProvidesFile{pdfcolmk.drv}%
  [2008/08/11 v1.2 Color support for pdfTeX via marks (HO)]%
\documentclass{ltxdoc}
\usepackage{holtxdoc}[2011/11/22]
\begin{document}
  \DocInput{pdfcolmk.dtx}%
\end{document}
%</driver>
% \fi
%
% \CheckSum{843}
%
% \CharacterTable
%  {Upper-case    \A\B\C\D\E\F\G\H\I\J\K\L\M\N\O\P\Q\R\S\T\U\V\W\X\Y\Z
%   Lower-case    \a\b\c\d\e\f\g\h\i\j\k\l\m\n\o\p\q\r\s\t\u\v\w\x\y\z
%   Digits        \0\1\2\3\4\5\6\7\8\9
%   Exclamation   \!     Double quote  \"     Hash (number) \#
%   Dollar        \$     Percent       \%     Ampersand     \&
%   Acute accent  \'     Left paren    \(     Right paren   \)
%   Asterisk      \*     Plus          \+     Comma         \,
%   Minus         \-     Point         \.     Solidus       \/
%   Colon         \:     Semicolon     \;     Less than     \<
%   Equals        \=     Greater than  \>     Question mark \?
%   Commercial at \@     Left bracket  \[     Backslash     \\
%   Right bracket \]     Circumflex    \^     Underscore    \_
%   Grave accent  \`     Left brace    \{     Vertical bar  \|
%   Right brace   \}     Tilde         \~}
%
% \GetFileInfo{pdfcolmk.drv}
%
% \title{The \xpackage{pdfcolmk} package}
% \date{2008/08/11 v1.2}
% \author{Heiko Oberdiek\\\xemail{heiko.oberdiek at googlemail.com}}
%
% \maketitle
%
% \begin{abstract}
% This package tries a solution for the missing color
% stack of \pdfTeX.
% \end{abstract}
%
% \tableofcontents
%
% \section{Documentation}
%
% \subsection{Introduction}
%
% This package uses a mark register in order to solve the
% problem of a missing color stack of \pdfTeX\ prior 1.40.0.
% Since this version of \pdfTeX\ a color stack is available
% and supported by \xfile{pdftex.def} 2007/01/01 v0.04a.
% In this case this package is obsolete and the package
% stops its loading.
%
% \subsection{Background}
%
% After the Dante meeting (Clausthal 2000) I have started
% to experiment with the eTeX method of a \emph{colour} mark.
% One of the major problems is the understanding of the
% output routine and the need to rewrite it because of
% missing hooks. Currently I have made some tests in
% in onecolumn and twocolumn mode, but the state is
% experimental.
%
% \subsection{Limitations}
%
% \begin{itemize}
% \item Mark limitations: page breaks in math.
% \item \LaTeX's output routine is redefinded.
%   \begin{itemize}
%   \item Changes in the output routine of newer versions
%         of LaTeX are not detected.
%   \item Packages that change the output routine are not
%         supported.
%   \end{itemize}
% \item It does not support several independent text
%       streams like footnotes.
% \item Limitations in float and marginpar support.
% \end{itemize}
%
% \subsection{Recommendation}
%
% \eTeX\ (for additional mark register)
% Without \eTeX\ \LaTeX's mark commands are redefined
% to store an additional color value.
%
% \subsection{Usage}
%
% Load after package color:
% \begin{quote}
%   |\usepackage[pdftex]{color}|\\
%   |\usepackage{pdfcolmk}|
% \end{quote}
%
% \subsection{Compatibility}
%
% \begin{itemize}
% \item Load the following packages after \xpackage{pdfcolmk}:
%   \begin{quote}
%       \xpackage{mparhack.sty}
%   \end{quote}
% \item Load the following packages before \xpackage{pdfcolmk}:
%   \begin{quote}
%       \xpackage{marn.sty}\\
%       \xpackage{newmarn.sty}
%   \end{quote}
% \item Supported \cs{@addmarginpar} patch:
%   \begin{quote}
%       \xpackage{latex/base/latex.ltx}\\
%       \xpackage{memoir.cls}\\
%       \xpackage{poemscol/marn.sty}, \xpackage{poemscol/newmarn.sty}\\
%       \xpackage{mparhack.sty}
%   \end{quote}
% \item Unsupported \cs{@addmarginpar} patch:
%   \begin{quote}
%       \xpackage{lineno.sty}\\
%       \xpackage{sttools/marginal.sty}\\
%       \xpackage{revtex4.cls}
%   \end{quote}
% \end{itemize}
%
% \StopEventually{
% }
%
% \section{Implementation}
%
%    \begin{macrocode}
%<*package>
%    \end{macrocode}
%    Package identification.
%    \begin{macrocode}
\NeedsTeXFormat{LaTeX2e}
\ProvidesPackage{pdfcolmk}%
  [2008/08/11 v1.2 Color support for pdfTeX via marks (HO)]
%    \end{macrocode}
%
%    \begin{macrocode}
\@ifundefined{ver@pdftex.def}{%
  \PackageWarningNoLine{pdfcolmk}{%
    Nothing to fix, because \string`pdftex.def\string' not loaded%
  }%
  \endinput
}{}
\@ifpackageloaded{color}{}{%
  \PackageWarningNoLine{pdfcolmk}{%
    Nothing to fix, because \string`color.sty\string' not loaded%
  }%
  \endinput
}
\begingroup\expandafter\expandafter\expandafter\endgroup
\expandafter\ifx\csname main@pdfcolorstack\endcsname\relax
\else
  % pdftex.def >= 2007/01/01 0.04a and pdfTeX >= 1.40.0
  \begingroup
    \let\on@line\@empty
    \PackageInfo{pdfcolmk}{%
      The color stack of pdfTeX \string>\string= 1.40 is used. %
      Therefore\MessageBreak
      this package is not necessary and not loaded%
    }%
  \endgroup
  \expandafter\endinput
\fi

\PackageInfo{pdfcolmk}{%
  This package tries to simulate dvips's color stack\MessageBreak
  for pdfTeX based on a mark register of e-TeX.\MessageBreak
  It redefines LaTeX's output routine. Therefore\MessageBreak
  use with care, no warranties%
}

\ifx\marks\@undefined

  \let\pec@mark\mark
  \let\pec@value\empty
  \long\def\mark#1{%
    \protected@xdef\pec@value{#1}%
    \pec@setmark
  }%
  \def\pec@setmark{%
    \begingroup
      \@temptokena\expandafter{\pec@value}%
      \pec@mark{{\current@color}\the\@temptokena}%
    \endgroup
  }%
  \def\pec@getmark{%
    \xdef\pec@botcolor{%
      \expandafter\@firstofthree\botmark\@empty\@empty\@empty
    }%
  }%
  \long\def\@firstofthree#1#2#3{#1}%
  \CheckCommand{\@leftmark}[2]{#1}%
  \CheckCommand{\@rightmark}[2]{#2}%
  \CheckCommand*{\leftmark}{%
    \expandafter\@leftmark\botmark\@empty\@empty
  }%
  \CheckCommand*{\rightmark}{%
    \expandafter\@rightmark\firstmark\@empty\@empty
  }%
  \long\def\@leftmark#1#2#3{#2}%
  \long\def\@rightmark#1#2#3{#3}%
  \g@addto@macro\leftmark\@empty
  \g@addto@macro\rightmark\@empty

\else

  \RequirePackage{etex}[1998/03/26]%
  \newmarks\pec@marks
  \def\pec@setmark{\marks\pec@marks{\current@color}}%
  \def\pec@getmark{\xdef\pec@botcolor{\botmarks\pec@marks}}%

\fi
%    \end{macrocode}
%
% \subsection{\cs{marginpar} fix}
%
%    \begin{macrocode}
\chardef\pec@result\z@
\def\pec@temp#1{%
  \chardef\pec@result\@ne
  \begingroup
    \let\on@line\@empty
    \PackageInfo{pdfcolmk}{%
      Patch for \string\@addmarginpar\space applied (#1)%
    }%
  \endgroup
}
%    \end{macrocode}
%
% \subsubsection{latex/base/latex.ltx}
%
%    \begin{macrocode}
\def\pec@addmarginpar{%
  \@next\@marbox\@currlist{%
    \@cons\@freelist\@marbox
    \@cons\@freelist\@currbox
  }\@latexbug
  \@tempcnta\@ne
  \if@twocolumn
    \if@firstcolumn
      \@tempcnta\m@ne
    \fi
  \else
    \if@mparswitch
      \ifodd\c@page
      \else
        \@tempcnta\m@ne
      \fi
    \fi
    \if@reversemargin \@tempcnta -\@tempcnta \fi
  \fi
  \ifnum\@tempcnta <\z@  \global\setbox\@marbox\box\@currbox \fi
  \@tempdima\@mparbottom
  \advance\@tempdima -\@pageht
  \advance\@tempdima\ht\@marbox
  \ifdim\@tempdima >\z@
    \@latex@warning@no@line{Marginpar on page \thepage\space moved}%
  \else
    \@tempdima\z@
  \fi
  \global\@mparbottom\@pageht
  \global\advance\@mparbottom\@tempdima
  \global\advance\@mparbottom\dp\@marbox
  \global\advance\@mparbottom\marginparpush
  \advance\@tempdima -\ht\@marbox
  \global\setbox\@marbox\vbox{%
    \vskip \@tempdima
    \box \@marbox
  }%
  \global \ht\@marbox \z@
  \global \dp\@marbox \z@
  \kern -\@pagedp
  \nointerlineskip
  \hb@xt@\columnwidth{%
    \ifnum \@tempcnta >\z@
      \hskip\columnwidth
      \hskip\marginparsep
    \else
      \hskip -\marginparsep
      \hskip -\marginparwidth
    \fi
    \box\@marbox \hss
  }%
  \nointerlineskip
  \hbox{\vrule \@height\z@ \@width\z@ \@depth\@pagedp}%
}
\ifx\pec@addmarginpar\@addmarginpar
  \pec@temp{latex/base}%
\fi
%    \end{macrocode}
%
% \subsubsection{memoir.cls}
%
%    \begin{macrocode}
\def\pec@addmarginpar{%
  \checkoddpage
  \@next\@marbox\@currlist{%
    \@cons\@freelist\@marbox
    \@cons\@freelist\@currbox
  }\@latexbug
  \@tempcnta\@ne
  \if@twocolumn
    \if@firstcolumn
      \@tempcnta\m@ne
    \fi
  \else
    \if@mparswitch
      \ifoddpage
      \else
        \@tempcnta\m@ne
      \fi
    \fi
    \if@reversemargin
      \@tempcnta -\@tempcnta
    \fi
  \fi
  \ifnum\@tempcnta <\z@
    \global\setbox\@marbox\box\@currbox
  \fi
  \@tempdima\@mparbottom
  \advance\@tempdima -\@pageht
  \advance\@tempdima\ht\@marbox
  \ifdim\@tempdima >\z@
    \@latex@warning@no@line{%
      Marginpar on page \thepage\space moved by \the\@tempdima
    }%
  \else
    \@tempdima\z@
  \fi
  \global\@mparbottom\@pageht
  \global\advance\@mparbottom\@tempdima
  \global\advance\@mparbottom\dp\@marbox
  \global\advance\@mparbottom\marginparpush
  \advance\@tempdima -\ht\@marbox
  \global\setbox\@marbox\vbox{%
    \vskip \@tempdima
    \box \@marbox
  }%
  \global \ht\@marbox \z@
  \global \dp\@marbox \z@
  \kern -\@pagedp
  \nointerlineskip
  \hb@xt@\columnwidth{%
    \ifnum \@tempcnta >\z@
      \hskip\columnwidth
      \hskip\marginparsep
    \else
      \hskip -\marginparsep
      \hskip -\marginparwidth
    \fi
    \box\@marbox
    \hss
  }%
  \nointerlineskip
  \hbox{\vrule \@height\z@ \@width\z@ \@depth\@pagedp}%
}%
\ifx\pec@addmarginpar\@addmarginpar
  \pec@temp{memoir.cls}%
\fi
%    \end{macrocode}
%
% \subsubsection{poemscol/marn.sty, poemscol/newmarn.sty}
%
%    \begin{macrocode}
\def\pec@addmarginpar{%
  \@next \@marbox\@currlist{%
    \@cons\@freelist\@marbox
    \@cons\@freelist\@currbox
  }\@latexbug
  \global\advance\@mpar@count\m@ne
  \@ifundefined{@marn@\the\@mpar@count @}{% was location logged last time?
    \@tempcnta\@ne % NO: use original LaTeX logic
    \if@twocolumn
      \if@firstcolumn
        \@tempcnta\m@ne
      \fi
    \else
      \if@mparswitch
        \ifodd\c@page
        \else
          \@tempcnta\m@ne
        \fi
      \fi
      \if@reversemargin
        \@tempcnta -\@tempcnta
      \fi
    \fi
  }{%
    \@tempcnta %    YES: use record from last time to decide side.
    \@nameuse{@marn@\the\@mpar@count @}%
    \if@reversemargin -\fi \@ne
  }%
  \ifnum\@tempcnta <\z@
    \global\setbox\@marbox\box\@currbox
    \global\let\@marnbottom\@mparbottoml
  \else
    \global\let\@marnbottom\@mparbottom
  \fi
  \@tempdima\@marnbottom \advance\@tempdima -\@pageht
  \advance\@tempdima\ht\@marbox
  \ifdim\@tempdima >\z@
    \@@warning{Marginpar on page \thepage\space moved}%
  \else
    \@tempdima\z@
  \fi
  \global\@marnbottom\@pageht
  \global\advance\@marnbottom\@tempdima
  \global\advance\@marnbottom\dp\@marbox
  \global\advance\@marnbottom\marginparpush
  \advance\@tempdima -\ht\@marbox
  \global\ht\@marbox\z@
  \global\dp\@marbox\z@
  \vskip -\@pagedp
  \vskip\@tempdima\nointerlineskip
  \hbox to\columnwidth{%
    \ifnum \@tempcnta >\z@
      \hskip\columnwidth
      \hskip\marginparsep
    \else
      \hskip -\marginparsep
      \hskip -\marginparwidth
    \fi
    \if@filesw % record where this is for use next time:
       \@marn@log\@mpar@count
    \fi
    \box\@marbox
    \hss
  }%
  \nobreak   %% RmS 91/06/21 \nobreak added
  \vskip -\@tempdima
  \nointerlineskip
  \hbox{\vrule \@height\z@ \@width\z@ \@depth\@pagedp}%
}
\ifx\pec@addmarginpar\@addmarginpar
  \pec@temp{poemscol/(new)marn.sty}%
\fi
%    \end{macrocode}
%
% \subsubsection{refman/refart.cls, refnam/refrep.cls}
%
%    \begin{macrocode}
\def\pec@addmarginpar{%
  \@next\@marbox\@currlist{%
    \@cons\@freelist\@marbox
    \@cons\@freelist\@currbox
  }\@latexbug
  \@tempcnta\@ne
  \if@twocolumn
    \if@firstcolumn
      \@tempcnta\m@ne
    \fi
  \else
    \@tempcnta\m@ne
  \fi
  \ifnum\@tempcnta <\z@
    \global\setbox\@marbox\box\@currbox
  \fi
  \@tempdima\@mparbottom
  \advance\@tempdima -\@pageht
  \advance\@tempdima\ht\@marbox
  \ifdim\@tempdima >\z@
     \@@warning{Marginpar on page \thepage\space moved}%
  \else
     \@tempdima\z@
  \fi
  \global\@mparbottom\@pageht
  \global\advance\@mparbottom\@tempdima
  \global\advance\@mparbottom\dp\@marbox
  \global\advance\@mparbottom\marginparpush
  \advance\@tempdima -\ht\@marbox
  \global\setbox\@marbox\vbox{%
    \vskip \@tempdima \box \@marbox
  }%
  \global \ht\@marbox \z@
  \global \dp\@marbox \z@
  \kern -\@pagedp
  \nointerlineskip
  \hb@xt@\columnwidth{%
    \ifnum \@tempcnta >\z@
      \hskip\columnwidth
      \hskip\marginparsep
    \else
      \hskip -\marginparsep
      \hskip -\marginparwidth
    \fi
    \box\@marbox
    \hss
  }%
  \nointerlineskip
  \hbox{\vrule \@height\z@ \@width\z@ \@depth\@pagedp}%
}
\ifx\pec@addmarginpar\@addmarginpar
  \pec@temp{ref(art|rep).cls}%
\fi

\ifcase\pec@result
  \PackageInfo{pdfcolmk}{%
    Fix for \string\@addmarginpar\space is omitted, %
    because this variant\MessageBreak
    of \string\@addmarginpar\space
      is not recognized%
  }%
\else
  % apply patch for \@addmarginpar
  \def\pec@PatchAddMarginpar#1\columnwidth#2#3\@nil{%
    \pec@PatchAddMarginparI#2\@nil{#1}{#3}%
  }%
  \def\pec@PatchAddMarginparI#1\box\@marbox\hss#2\@nil#3#4{%
    \def\@addmarginpar{%
      #3%
      \columnwidth{%
        #1%
        \pdfliteral{q}%
        \rlap{%
          \box\@marbox
        }%
        \pdfliteral{Q}%
        \hss
        #2%
      }%
      #4%
    }%
  }%
  \expandafter\pec@PatchAddMarginpar\@addmarginpar\@nil
\fi
%    \end{macrocode}
%
% \subsection{Color fix}
%
%    \begin{macrocode}
\def\set@color{%
  \pdfliteral{\current@color}%
  \ifinner
  \else
    \pec@setmark
  \fi
  \aftergroup\reset@color
}
\def\reset@color{%
  \pdfliteral{\current@color}%
  \ifinner
  \else
    \pec@setmark
  \fi
}

\let\pec@botcolor\current@color

\def\pec@PatchVBoxCCLV{%
  \ifx\pec@botcolor\@empty
  \else
    \setbox\@cclv\vbox{%
      \pdfliteral{\pec@botcolor}%
      \unvbox\@cclv
    }%
  \fi
  \pec@getmark
}

\def\pec@PatchAlreadyInBox{%
  \ifx\pec@botcolor\@empty
  \else
    \pdfliteral{\pec@botcolor}%
  \fi
  \pec@getmark
}

\@ifclassloaded{memoir}{%
  \expandafter\def\expandafter\mem@makecol\expandafter{%
    \expandafter\pec@PatchVBoxCCLV
    \mem@makecol
  }%
  \endinput
}{}

\@ifclassloaded{seminar}{%
  \newcommand\pec@org@makeslide{}%
  \let\pec@org@makeslide\@makeslide
  \def\@makeslide{%
    \pec@PatchVBoxCCLV
    \pec@org@makeslide
  }%
  \endinput
}{}

\long\def\pec@output#1\@specialoutput\else#2\pec@end{%
  \begingroup
    \def\x{#2}%
  \expandafter\endgroup
  \ifx\x\@empty
    \PackageWarningNoLine{pdfcolmk}{%
      Unexpected \string\output\space routine detected,%
      \MessageBreak
      loading of package stopped%
    }%
    \expandafter\endinput
  \fi
}
\expandafter\expandafter\expandafter\pec@output
\expandafter\@firstofone\the\output\@specialoutput\else\pec@end

\long\def\pec@output#1\@specialoutput\else#2\pec@end{%
  \output{%
    #1\@specialoutput\else
    \pec@PatchVBoxCCLV
    #2%
  }%
}
\expandafter\expandafter\expandafter\pec@output
\expandafter\@firstofone\the\output\pec@end
%    \end{macrocode}
%
%    \begin{macrocode}
%</package>
%    \end{macrocode}
%
% \section{Installation}
%
% \subsection{Download}
%
% \paragraph{Package.} This package is available on
% CTAN\footnote{\url{ftp://ftp.ctan.org/tex-archive/}}:
% \begin{description}
% \item[\CTAN{macros/latex/contrib/oberdiek/pdfcolmk.dtx}] The source file.
% \item[\CTAN{macros/latex/contrib/oberdiek/pdfcolmk.pdf}] Documentation.
% \end{description}
%
%
% \paragraph{Bundle.} All the packages of the bundle `oberdiek'
% are also available in a TDS compliant ZIP archive. There
% the packages are already unpacked and the documentation files
% are generated. The files and directories obey the TDS standard.
% \begin{description}
% \item[\CTAN{install/macros/latex/contrib/oberdiek.tds.zip}]
% \end{description}
% \emph{TDS} refers to the standard ``A Directory Structure
% for \TeX\ Files'' (\CTAN{tds/tds.pdf}). Directories
% with \xfile{texmf} in their name are usually organized this way.
%
% \subsection{Bundle installation}
%
% \paragraph{Unpacking.} Unpack the \xfile{oberdiek.tds.zip} in the
% TDS tree (also known as \xfile{texmf} tree) of your choice.
% Example (linux):
% \begin{quote}
%   |unzip oberdiek.tds.zip -d ~/texmf|
% \end{quote}
%
% \paragraph{Script installation.}
% Check the directory \xfile{TDS:scripts/oberdiek/} for
% scripts that need further installation steps.
% Package \xpackage{attachfile2} comes with the Perl script
% \xfile{pdfatfi.pl} that should be installed in such a way
% that it can be called as \texttt{pdfatfi}.
% Example (linux):
% \begin{quote}
%   |chmod +x scripts/oberdiek/pdfatfi.pl|\\
%   |cp scripts/oberdiek/pdfatfi.pl /usr/local/bin/|
% \end{quote}
%
% \subsection{Package installation}
%
% \paragraph{Unpacking.} The \xfile{.dtx} file is a self-extracting
% \docstrip\ archive. The files are extracted by running the
% \xfile{.dtx} through \plainTeX:
% \begin{quote}
%   \verb|tex pdfcolmk.dtx|
% \end{quote}
%
% \paragraph{TDS.} Now the different files must be moved into
% the different directories in your installation TDS tree
% (also known as \xfile{texmf} tree):
% \begin{quote}
% \def\t{^^A
% \begin{tabular}{@{}>{\ttfamily}l@{ $\rightarrow$ }>{\ttfamily}l@{}}
%   pdfcolmk.sty & tex/latex/oberdiek/pdfcolmk.sty\\
%   pdfcolmk.pdf & doc/latex/oberdiek/pdfcolmk.pdf\\
%   pdfcolmk.dtx & source/latex/oberdiek/pdfcolmk.dtx\\
% \end{tabular}^^A
% }^^A
% \sbox0{\t}^^A
% \ifdim\wd0>\linewidth
%   \begingroup
%     \advance\linewidth by\leftmargin
%     \advance\linewidth by\rightmargin
%   \edef\x{\endgroup
%     \def\noexpand\lw{\the\linewidth}^^A
%   }\x
%   \def\lwbox{^^A
%     \leavevmode
%     \hbox to \linewidth{^^A
%       \kern-\leftmargin\relax
%       \hss
%       \usebox0
%       \hss
%       \kern-\rightmargin\relax
%     }^^A
%   }^^A
%   \ifdim\wd0>\lw
%     \sbox0{\small\t}^^A
%     \ifdim\wd0>\linewidth
%       \ifdim\wd0>\lw
%         \sbox0{\footnotesize\t}^^A
%         \ifdim\wd0>\linewidth
%           \ifdim\wd0>\lw
%             \sbox0{\scriptsize\t}^^A
%             \ifdim\wd0>\linewidth
%               \ifdim\wd0>\lw
%                 \sbox0{\tiny\t}^^A
%                 \ifdim\wd0>\linewidth
%                   \lwbox
%                 \else
%                   \usebox0
%                 \fi
%               \else
%                 \lwbox
%               \fi
%             \else
%               \usebox0
%             \fi
%           \else
%             \lwbox
%           \fi
%         \else
%           \usebox0
%         \fi
%       \else
%         \lwbox
%       \fi
%     \else
%       \usebox0
%     \fi
%   \else
%     \lwbox
%   \fi
% \else
%   \usebox0
% \fi
% \end{quote}
% If you have a \xfile{docstrip.cfg} that configures and enables \docstrip's
% TDS installing feature, then some files can already be in the right
% place, see the documentation of \docstrip.
%
% \subsection{Refresh file name databases}
%
% If your \TeX~distribution
% (\teTeX, \mikTeX, \dots) relies on file name databases, you must refresh
% these. For example, \teTeX\ users run \verb|texhash| or
% \verb|mktexlsr|.
%
% \subsection{Some details for the interested}
%
% \paragraph{Attached source.}
%
% The PDF documentation on CTAN also includes the
% \xfile{.dtx} source file. It can be extracted by
% AcrobatReader 6 or higher. Another option is \textsf{pdftk},
% e.g. unpack the file into the current directory:
% \begin{quote}
%   \verb|pdftk pdfcolmk.pdf unpack_files output .|
% \end{quote}
%
% \paragraph{Unpacking with \LaTeX.}
% The \xfile{.dtx} chooses its action depending on the format:
% \begin{description}
% \item[\plainTeX:] Run \docstrip\ and extract the files.
% \item[\LaTeX:] Generate the documentation.
% \end{description}
% If you insist on using \LaTeX\ for \docstrip\ (really,
% \docstrip\ does not need \LaTeX), then inform the autodetect routine
% about your intention:
% \begin{quote}
%   \verb|latex \let\install=y% \iffalse meta-comment
%
% File: pdfcolmk.dtx
% Version: 2008/08/11 v1.2
% Info: Color support for pdfTeX via marks
%
% Copyright (C) 2000, 2005-2008 by
%    Heiko Oberdiek <heiko.oberdiek at googlemail.com>
%
% This work may be distributed and/or modified under the
% conditions of the LaTeX Project Public License, either
% version 1.3c of this license or (at your option) any later
% version. This version of this license is in
%    http://www.latex-project.org/lppl/lppl-1-3c.txt
% and the latest version of this license is in
%    http://www.latex-project.org/lppl.txt
% and version 1.3 or later is part of all distributions of
% LaTeX version 2005/12/01 or later.
%
% This work has the LPPL maintenance status "maintained".
%
% This Current Maintainer of this work is Heiko Oberdiek.
%
% This work consists of the main source file pdfcolmk.dtx
% and the derived files
%    pdfcolmk.sty, pdfcolmk.pdf, pdfcolmk.ins, pdfcolmk.drv.
%
% Distribution:
%    CTAN:macros/latex/contrib/oberdiek/pdfcolmk.dtx
%    CTAN:macros/latex/contrib/oberdiek/pdfcolmk.pdf
%
% Unpacking:
%    (a) If pdfcolmk.ins is present:
%           tex pdfcolmk.ins
%    (b) Without pdfcolmk.ins:
%           tex pdfcolmk.dtx
%    (c) If you insist on using LaTeX
%           latex \let\install=y% \iffalse meta-comment
%
% File: pdfcolmk.dtx
% Version: 2008/08/11 v1.2
% Info: Color support for pdfTeX via marks
%
% Copyright (C) 2000, 2005-2008 by
%    Heiko Oberdiek <heiko.oberdiek at googlemail.com>
%
% This work may be distributed and/or modified under the
% conditions of the LaTeX Project Public License, either
% version 1.3c of this license or (at your option) any later
% version. This version of this license is in
%    http://www.latex-project.org/lppl/lppl-1-3c.txt
% and the latest version of this license is in
%    http://www.latex-project.org/lppl.txt
% and version 1.3 or later is part of all distributions of
% LaTeX version 2005/12/01 or later.
%
% This work has the LPPL maintenance status "maintained".
%
% This Current Maintainer of this work is Heiko Oberdiek.
%
% This work consists of the main source file pdfcolmk.dtx
% and the derived files
%    pdfcolmk.sty, pdfcolmk.pdf, pdfcolmk.ins, pdfcolmk.drv.
%
% Distribution:
%    CTAN:macros/latex/contrib/oberdiek/pdfcolmk.dtx
%    CTAN:macros/latex/contrib/oberdiek/pdfcolmk.pdf
%
% Unpacking:
%    (a) If pdfcolmk.ins is present:
%           tex pdfcolmk.ins
%    (b) Without pdfcolmk.ins:
%           tex pdfcolmk.dtx
%    (c) If you insist on using LaTeX
%           latex \let\install=y\input{pdfcolmk.dtx}
%        (quote the arguments according to the demands of your shell)
%
% Documentation:
%    (a) If pdfcolmk.drv is present:
%           latex pdfcolmk.drv
%    (b) Without pdfcolmk.drv:
%           latex pdfcolmk.dtx; ...
%    The class ltxdoc loads the configuration file ltxdoc.cfg
%    if available. Here you can specify further options, e.g.
%    use A4 as paper format:
%       \PassOptionsToClass{a4paper}{article}
%
%    Programm calls to get the documentation (example):
%       pdflatex pdfcolmk.dtx
%       makeindex -s gind.ist pdfcolmk.idx
%       pdflatex pdfcolmk.dtx
%       makeindex -s gind.ist pdfcolmk.idx
%       pdflatex pdfcolmk.dtx
%
% Installation:
%    TDS:tex/latex/oberdiek/pdfcolmk.sty
%    TDS:doc/latex/oberdiek/pdfcolmk.pdf
%    TDS:source/latex/oberdiek/pdfcolmk.dtx
%
%<*ignore>
\begingroup
  \catcode123=1 %
  \catcode125=2 %
  \def\x{LaTeX2e}%
\expandafter\endgroup
\ifcase 0\ifx\install y1\fi\expandafter
         \ifx\csname processbatchFile\endcsname\relax\else1\fi
         \ifx\fmtname\x\else 1\fi\relax
\else\csname fi\endcsname
%</ignore>
%<*install>
\input docstrip.tex
\Msg{************************************************************************}
\Msg{* Installation}
\Msg{* Package: pdfcolmk 2008/08/11 v1.2 Color support for pdfTeX via marks (HO)}
\Msg{************************************************************************}

\keepsilent
\askforoverwritefalse

\let\MetaPrefix\relax
\preamble

This is a generated file.

Project: pdfcolmk
Version: 2008/08/11 v1.2

Copyright (C) 2000, 2005-2008 by
   Heiko Oberdiek <heiko.oberdiek at googlemail.com>

This work may be distributed and/or modified under the
conditions of the LaTeX Project Public License, either
version 1.3c of this license or (at your option) any later
version. This version of this license is in
   http://www.latex-project.org/lppl/lppl-1-3c.txt
and the latest version of this license is in
   http://www.latex-project.org/lppl.txt
and version 1.3 or later is part of all distributions of
LaTeX version 2005/12/01 or later.

This work has the LPPL maintenance status "maintained".

This Current Maintainer of this work is Heiko Oberdiek.

This work consists of the main source file pdfcolmk.dtx
and the derived files
   pdfcolmk.sty, pdfcolmk.pdf, pdfcolmk.ins, pdfcolmk.drv.

\endpreamble
\let\MetaPrefix\DoubleperCent

\generate{%
  \file{pdfcolmk.ins}{\from{pdfcolmk.dtx}{install}}%
  \file{pdfcolmk.drv}{\from{pdfcolmk.dtx}{driver}}%
  \usedir{tex/latex/oberdiek}%
  \file{pdfcolmk.sty}{\from{pdfcolmk.dtx}{package}}%
  \nopreamble
  \nopostamble
  \usedir{source/latex/oberdiek/catalogue}%
  \file{pdfcolmk.xml}{\from{pdfcolmk.dtx}{catalogue}}%
}

\catcode32=13\relax% active space
\let =\space%
\Msg{************************************************************************}
\Msg{*}
\Msg{* To finish the installation you have to move the following}
\Msg{* file into a directory searched by TeX:}
\Msg{*}
\Msg{*     pdfcolmk.sty}
\Msg{*}
\Msg{* To produce the documentation run the file `pdfcolmk.drv'}
\Msg{* through LaTeX.}
\Msg{*}
\Msg{* Happy TeXing!}
\Msg{*}
\Msg{************************************************************************}

\endbatchfile
%</install>
%<*ignore>
\fi
%</ignore>
%<*driver>
\NeedsTeXFormat{LaTeX2e}
\ProvidesFile{pdfcolmk.drv}%
  [2008/08/11 v1.2 Color support for pdfTeX via marks (HO)]%
\documentclass{ltxdoc}
\usepackage{holtxdoc}[2011/11/22]
\begin{document}
  \DocInput{pdfcolmk.dtx}%
\end{document}
%</driver>
% \fi
%
% \CheckSum{843}
%
% \CharacterTable
%  {Upper-case    \A\B\C\D\E\F\G\H\I\J\K\L\M\N\O\P\Q\R\S\T\U\V\W\X\Y\Z
%   Lower-case    \a\b\c\d\e\f\g\h\i\j\k\l\m\n\o\p\q\r\s\t\u\v\w\x\y\z
%   Digits        \0\1\2\3\4\5\6\7\8\9
%   Exclamation   \!     Double quote  \"     Hash (number) \#
%   Dollar        \$     Percent       \%     Ampersand     \&
%   Acute accent  \'     Left paren    \(     Right paren   \)
%   Asterisk      \*     Plus          \+     Comma         \,
%   Minus         \-     Point         \.     Solidus       \/
%   Colon         \:     Semicolon     \;     Less than     \<
%   Equals        \=     Greater than  \>     Question mark \?
%   Commercial at \@     Left bracket  \[     Backslash     \\
%   Right bracket \]     Circumflex    \^     Underscore    \_
%   Grave accent  \`     Left brace    \{     Vertical bar  \|
%   Right brace   \}     Tilde         \~}
%
% \GetFileInfo{pdfcolmk.drv}
%
% \title{The \xpackage{pdfcolmk} package}
% \date{2008/08/11 v1.2}
% \author{Heiko Oberdiek\\\xemail{heiko.oberdiek at googlemail.com}}
%
% \maketitle
%
% \begin{abstract}
% This package tries a solution for the missing color
% stack of \pdfTeX.
% \end{abstract}
%
% \tableofcontents
%
% \section{Documentation}
%
% \subsection{Introduction}
%
% This package uses a mark register in order to solve the
% problem of a missing color stack of \pdfTeX\ prior 1.40.0.
% Since this version of \pdfTeX\ a color stack is available
% and supported by \xfile{pdftex.def} 2007/01/01 v0.04a.
% In this case this package is obsolete and the package
% stops its loading.
%
% \subsection{Background}
%
% After the Dante meeting (Clausthal 2000) I have started
% to experiment with the eTeX method of a \emph{colour} mark.
% One of the major problems is the understanding of the
% output routine and the need to rewrite it because of
% missing hooks. Currently I have made some tests in
% in onecolumn and twocolumn mode, but the state is
% experimental.
%
% \subsection{Limitations}
%
% \begin{itemize}
% \item Mark limitations: page breaks in math.
% \item \LaTeX's output routine is redefinded.
%   \begin{itemize}
%   \item Changes in the output routine of newer versions
%         of LaTeX are not detected.
%   \item Packages that change the output routine are not
%         supported.
%   \end{itemize}
% \item It does not support several independent text
%       streams like footnotes.
% \item Limitations in float and marginpar support.
% \end{itemize}
%
% \subsection{Recommendation}
%
% \eTeX\ (for additional mark register)
% Without \eTeX\ \LaTeX's mark commands are redefined
% to store an additional color value.
%
% \subsection{Usage}
%
% Load after package color:
% \begin{quote}
%   |\usepackage[pdftex]{color}|\\
%   |\usepackage{pdfcolmk}|
% \end{quote}
%
% \subsection{Compatibility}
%
% \begin{itemize}
% \item Load the following packages after \xpackage{pdfcolmk}:
%   \begin{quote}
%       \xpackage{mparhack.sty}
%   \end{quote}
% \item Load the following packages before \xpackage{pdfcolmk}:
%   \begin{quote}
%       \xpackage{marn.sty}\\
%       \xpackage{newmarn.sty}
%   \end{quote}
% \item Supported \cs{@addmarginpar} patch:
%   \begin{quote}
%       \xpackage{latex/base/latex.ltx}\\
%       \xpackage{memoir.cls}\\
%       \xpackage{poemscol/marn.sty}, \xpackage{poemscol/newmarn.sty}\\
%       \xpackage{mparhack.sty}
%   \end{quote}
% \item Unsupported \cs{@addmarginpar} patch:
%   \begin{quote}
%       \xpackage{lineno.sty}\\
%       \xpackage{sttools/marginal.sty}\\
%       \xpackage{revtex4.cls}
%   \end{quote}
% \end{itemize}
%
% \StopEventually{
% }
%
% \section{Implementation}
%
%    \begin{macrocode}
%<*package>
%    \end{macrocode}
%    Package identification.
%    \begin{macrocode}
\NeedsTeXFormat{LaTeX2e}
\ProvidesPackage{pdfcolmk}%
  [2008/08/11 v1.2 Color support for pdfTeX via marks (HO)]
%    \end{macrocode}
%
%    \begin{macrocode}
\@ifundefined{ver@pdftex.def}{%
  \PackageWarningNoLine{pdfcolmk}{%
    Nothing to fix, because \string`pdftex.def\string' not loaded%
  }%
  \endinput
}{}
\@ifpackageloaded{color}{}{%
  \PackageWarningNoLine{pdfcolmk}{%
    Nothing to fix, because \string`color.sty\string' not loaded%
  }%
  \endinput
}
\begingroup\expandafter\expandafter\expandafter\endgroup
\expandafter\ifx\csname main@pdfcolorstack\endcsname\relax
\else
  % pdftex.def >= 2007/01/01 0.04a and pdfTeX >= 1.40.0
  \begingroup
    \let\on@line\@empty
    \PackageInfo{pdfcolmk}{%
      The color stack of pdfTeX \string>\string= 1.40 is used. %
      Therefore\MessageBreak
      this package is not necessary and not loaded%
    }%
  \endgroup
  \expandafter\endinput
\fi

\PackageInfo{pdfcolmk}{%
  This package tries to simulate dvips's color stack\MessageBreak
  for pdfTeX based on a mark register of e-TeX.\MessageBreak
  It redefines LaTeX's output routine. Therefore\MessageBreak
  use with care, no warranties%
}

\ifx\marks\@undefined

  \let\pec@mark\mark
  \let\pec@value\empty
  \long\def\mark#1{%
    \protected@xdef\pec@value{#1}%
    \pec@setmark
  }%
  \def\pec@setmark{%
    \begingroup
      \@temptokena\expandafter{\pec@value}%
      \pec@mark{{\current@color}\the\@temptokena}%
    \endgroup
  }%
  \def\pec@getmark{%
    \xdef\pec@botcolor{%
      \expandafter\@firstofthree\botmark\@empty\@empty\@empty
    }%
  }%
  \long\def\@firstofthree#1#2#3{#1}%
  \CheckCommand{\@leftmark}[2]{#1}%
  \CheckCommand{\@rightmark}[2]{#2}%
  \CheckCommand*{\leftmark}{%
    \expandafter\@leftmark\botmark\@empty\@empty
  }%
  \CheckCommand*{\rightmark}{%
    \expandafter\@rightmark\firstmark\@empty\@empty
  }%
  \long\def\@leftmark#1#2#3{#2}%
  \long\def\@rightmark#1#2#3{#3}%
  \g@addto@macro\leftmark\@empty
  \g@addto@macro\rightmark\@empty

\else

  \RequirePackage{etex}[1998/03/26]%
  \newmarks\pec@marks
  \def\pec@setmark{\marks\pec@marks{\current@color}}%
  \def\pec@getmark{\xdef\pec@botcolor{\botmarks\pec@marks}}%

\fi
%    \end{macrocode}
%
% \subsection{\cs{marginpar} fix}
%
%    \begin{macrocode}
\chardef\pec@result\z@
\def\pec@temp#1{%
  \chardef\pec@result\@ne
  \begingroup
    \let\on@line\@empty
    \PackageInfo{pdfcolmk}{%
      Patch for \string\@addmarginpar\space applied (#1)%
    }%
  \endgroup
}
%    \end{macrocode}
%
% \subsubsection{latex/base/latex.ltx}
%
%    \begin{macrocode}
\def\pec@addmarginpar{%
  \@next\@marbox\@currlist{%
    \@cons\@freelist\@marbox
    \@cons\@freelist\@currbox
  }\@latexbug
  \@tempcnta\@ne
  \if@twocolumn
    \if@firstcolumn
      \@tempcnta\m@ne
    \fi
  \else
    \if@mparswitch
      \ifodd\c@page
      \else
        \@tempcnta\m@ne
      \fi
    \fi
    \if@reversemargin \@tempcnta -\@tempcnta \fi
  \fi
  \ifnum\@tempcnta <\z@  \global\setbox\@marbox\box\@currbox \fi
  \@tempdima\@mparbottom
  \advance\@tempdima -\@pageht
  \advance\@tempdima\ht\@marbox
  \ifdim\@tempdima >\z@
    \@latex@warning@no@line{Marginpar on page \thepage\space moved}%
  \else
    \@tempdima\z@
  \fi
  \global\@mparbottom\@pageht
  \global\advance\@mparbottom\@tempdima
  \global\advance\@mparbottom\dp\@marbox
  \global\advance\@mparbottom\marginparpush
  \advance\@tempdima -\ht\@marbox
  \global\setbox\@marbox\vbox{%
    \vskip \@tempdima
    \box \@marbox
  }%
  \global \ht\@marbox \z@
  \global \dp\@marbox \z@
  \kern -\@pagedp
  \nointerlineskip
  \hb@xt@\columnwidth{%
    \ifnum \@tempcnta >\z@
      \hskip\columnwidth
      \hskip\marginparsep
    \else
      \hskip -\marginparsep
      \hskip -\marginparwidth
    \fi
    \box\@marbox \hss
  }%
  \nointerlineskip
  \hbox{\vrule \@height\z@ \@width\z@ \@depth\@pagedp}%
}
\ifx\pec@addmarginpar\@addmarginpar
  \pec@temp{latex/base}%
\fi
%    \end{macrocode}
%
% \subsubsection{memoir.cls}
%
%    \begin{macrocode}
\def\pec@addmarginpar{%
  \checkoddpage
  \@next\@marbox\@currlist{%
    \@cons\@freelist\@marbox
    \@cons\@freelist\@currbox
  }\@latexbug
  \@tempcnta\@ne
  \if@twocolumn
    \if@firstcolumn
      \@tempcnta\m@ne
    \fi
  \else
    \if@mparswitch
      \ifoddpage
      \else
        \@tempcnta\m@ne
      \fi
    \fi
    \if@reversemargin
      \@tempcnta -\@tempcnta
    \fi
  \fi
  \ifnum\@tempcnta <\z@
    \global\setbox\@marbox\box\@currbox
  \fi
  \@tempdima\@mparbottom
  \advance\@tempdima -\@pageht
  \advance\@tempdima\ht\@marbox
  \ifdim\@tempdima >\z@
    \@latex@warning@no@line{%
      Marginpar on page \thepage\space moved by \the\@tempdima
    }%
  \else
    \@tempdima\z@
  \fi
  \global\@mparbottom\@pageht
  \global\advance\@mparbottom\@tempdima
  \global\advance\@mparbottom\dp\@marbox
  \global\advance\@mparbottom\marginparpush
  \advance\@tempdima -\ht\@marbox
  \global\setbox\@marbox\vbox{%
    \vskip \@tempdima
    \box \@marbox
  }%
  \global \ht\@marbox \z@
  \global \dp\@marbox \z@
  \kern -\@pagedp
  \nointerlineskip
  \hb@xt@\columnwidth{%
    \ifnum \@tempcnta >\z@
      \hskip\columnwidth
      \hskip\marginparsep
    \else
      \hskip -\marginparsep
      \hskip -\marginparwidth
    \fi
    \box\@marbox
    \hss
  }%
  \nointerlineskip
  \hbox{\vrule \@height\z@ \@width\z@ \@depth\@pagedp}%
}%
\ifx\pec@addmarginpar\@addmarginpar
  \pec@temp{memoir.cls}%
\fi
%    \end{macrocode}
%
% \subsubsection{poemscol/marn.sty, poemscol/newmarn.sty}
%
%    \begin{macrocode}
\def\pec@addmarginpar{%
  \@next \@marbox\@currlist{%
    \@cons\@freelist\@marbox
    \@cons\@freelist\@currbox
  }\@latexbug
  \global\advance\@mpar@count\m@ne
  \@ifundefined{@marn@\the\@mpar@count @}{% was location logged last time?
    \@tempcnta\@ne % NO: use original LaTeX logic
    \if@twocolumn
      \if@firstcolumn
        \@tempcnta\m@ne
      \fi
    \else
      \if@mparswitch
        \ifodd\c@page
        \else
          \@tempcnta\m@ne
        \fi
      \fi
      \if@reversemargin
        \@tempcnta -\@tempcnta
      \fi
    \fi
  }{%
    \@tempcnta %    YES: use record from last time to decide side.
    \@nameuse{@marn@\the\@mpar@count @}%
    \if@reversemargin -\fi \@ne
  }%
  \ifnum\@tempcnta <\z@
    \global\setbox\@marbox\box\@currbox
    \global\let\@marnbottom\@mparbottoml
  \else
    \global\let\@marnbottom\@mparbottom
  \fi
  \@tempdima\@marnbottom \advance\@tempdima -\@pageht
  \advance\@tempdima\ht\@marbox
  \ifdim\@tempdima >\z@
    \@@warning{Marginpar on page \thepage\space moved}%
  \else
    \@tempdima\z@
  \fi
  \global\@marnbottom\@pageht
  \global\advance\@marnbottom\@tempdima
  \global\advance\@marnbottom\dp\@marbox
  \global\advance\@marnbottom\marginparpush
  \advance\@tempdima -\ht\@marbox
  \global\ht\@marbox\z@
  \global\dp\@marbox\z@
  \vskip -\@pagedp
  \vskip\@tempdima\nointerlineskip
  \hbox to\columnwidth{%
    \ifnum \@tempcnta >\z@
      \hskip\columnwidth
      \hskip\marginparsep
    \else
      \hskip -\marginparsep
      \hskip -\marginparwidth
    \fi
    \if@filesw % record where this is for use next time:
       \@marn@log\@mpar@count
    \fi
    \box\@marbox
    \hss
  }%
  \nobreak   %% RmS 91/06/21 \nobreak added
  \vskip -\@tempdima
  \nointerlineskip
  \hbox{\vrule \@height\z@ \@width\z@ \@depth\@pagedp}%
}
\ifx\pec@addmarginpar\@addmarginpar
  \pec@temp{poemscol/(new)marn.sty}%
\fi
%    \end{macrocode}
%
% \subsubsection{refman/refart.cls, refnam/refrep.cls}
%
%    \begin{macrocode}
\def\pec@addmarginpar{%
  \@next\@marbox\@currlist{%
    \@cons\@freelist\@marbox
    \@cons\@freelist\@currbox
  }\@latexbug
  \@tempcnta\@ne
  \if@twocolumn
    \if@firstcolumn
      \@tempcnta\m@ne
    \fi
  \else
    \@tempcnta\m@ne
  \fi
  \ifnum\@tempcnta <\z@
    \global\setbox\@marbox\box\@currbox
  \fi
  \@tempdima\@mparbottom
  \advance\@tempdima -\@pageht
  \advance\@tempdima\ht\@marbox
  \ifdim\@tempdima >\z@
     \@@warning{Marginpar on page \thepage\space moved}%
  \else
     \@tempdima\z@
  \fi
  \global\@mparbottom\@pageht
  \global\advance\@mparbottom\@tempdima
  \global\advance\@mparbottom\dp\@marbox
  \global\advance\@mparbottom\marginparpush
  \advance\@tempdima -\ht\@marbox
  \global\setbox\@marbox\vbox{%
    \vskip \@tempdima \box \@marbox
  }%
  \global \ht\@marbox \z@
  \global \dp\@marbox \z@
  \kern -\@pagedp
  \nointerlineskip
  \hb@xt@\columnwidth{%
    \ifnum \@tempcnta >\z@
      \hskip\columnwidth
      \hskip\marginparsep
    \else
      \hskip -\marginparsep
      \hskip -\marginparwidth
    \fi
    \box\@marbox
    \hss
  }%
  \nointerlineskip
  \hbox{\vrule \@height\z@ \@width\z@ \@depth\@pagedp}%
}
\ifx\pec@addmarginpar\@addmarginpar
  \pec@temp{ref(art|rep).cls}%
\fi

\ifcase\pec@result
  \PackageInfo{pdfcolmk}{%
    Fix for \string\@addmarginpar\space is omitted, %
    because this variant\MessageBreak
    of \string\@addmarginpar\space
      is not recognized%
  }%
\else
  % apply patch for \@addmarginpar
  \def\pec@PatchAddMarginpar#1\columnwidth#2#3\@nil{%
    \pec@PatchAddMarginparI#2\@nil{#1}{#3}%
  }%
  \def\pec@PatchAddMarginparI#1\box\@marbox\hss#2\@nil#3#4{%
    \def\@addmarginpar{%
      #3%
      \columnwidth{%
        #1%
        \pdfliteral{q}%
        \rlap{%
          \box\@marbox
        }%
        \pdfliteral{Q}%
        \hss
        #2%
      }%
      #4%
    }%
  }%
  \expandafter\pec@PatchAddMarginpar\@addmarginpar\@nil
\fi
%    \end{macrocode}
%
% \subsection{Color fix}
%
%    \begin{macrocode}
\def\set@color{%
  \pdfliteral{\current@color}%
  \ifinner
  \else
    \pec@setmark
  \fi
  \aftergroup\reset@color
}
\def\reset@color{%
  \pdfliteral{\current@color}%
  \ifinner
  \else
    \pec@setmark
  \fi
}

\let\pec@botcolor\current@color

\def\pec@PatchVBoxCCLV{%
  \ifx\pec@botcolor\@empty
  \else
    \setbox\@cclv\vbox{%
      \pdfliteral{\pec@botcolor}%
      \unvbox\@cclv
    }%
  \fi
  \pec@getmark
}

\def\pec@PatchAlreadyInBox{%
  \ifx\pec@botcolor\@empty
  \else
    \pdfliteral{\pec@botcolor}%
  \fi
  \pec@getmark
}

\@ifclassloaded{memoir}{%
  \expandafter\def\expandafter\mem@makecol\expandafter{%
    \expandafter\pec@PatchVBoxCCLV
    \mem@makecol
  }%
  \endinput
}{}

\@ifclassloaded{seminar}{%
  \newcommand\pec@org@makeslide{}%
  \let\pec@org@makeslide\@makeslide
  \def\@makeslide{%
    \pec@PatchVBoxCCLV
    \pec@org@makeslide
  }%
  \endinput
}{}

\long\def\pec@output#1\@specialoutput\else#2\pec@end{%
  \begingroup
    \def\x{#2}%
  \expandafter\endgroup
  \ifx\x\@empty
    \PackageWarningNoLine{pdfcolmk}{%
      Unexpected \string\output\space routine detected,%
      \MessageBreak
      loading of package stopped%
    }%
    \expandafter\endinput
  \fi
}
\expandafter\expandafter\expandafter\pec@output
\expandafter\@firstofone\the\output\@specialoutput\else\pec@end

\long\def\pec@output#1\@specialoutput\else#2\pec@end{%
  \output{%
    #1\@specialoutput\else
    \pec@PatchVBoxCCLV
    #2%
  }%
}
\expandafter\expandafter\expandafter\pec@output
\expandafter\@firstofone\the\output\pec@end
%    \end{macrocode}
%
%    \begin{macrocode}
%</package>
%    \end{macrocode}
%
% \section{Installation}
%
% \subsection{Download}
%
% \paragraph{Package.} This package is available on
% CTAN\footnote{\url{ftp://ftp.ctan.org/tex-archive/}}:
% \begin{description}
% \item[\CTAN{macros/latex/contrib/oberdiek/pdfcolmk.dtx}] The source file.
% \item[\CTAN{macros/latex/contrib/oberdiek/pdfcolmk.pdf}] Documentation.
% \end{description}
%
%
% \paragraph{Bundle.} All the packages of the bundle `oberdiek'
% are also available in a TDS compliant ZIP archive. There
% the packages are already unpacked and the documentation files
% are generated. The files and directories obey the TDS standard.
% \begin{description}
% \item[\CTAN{install/macros/latex/contrib/oberdiek.tds.zip}]
% \end{description}
% \emph{TDS} refers to the standard ``A Directory Structure
% for \TeX\ Files'' (\CTAN{tds/tds.pdf}). Directories
% with \xfile{texmf} in their name are usually organized this way.
%
% \subsection{Bundle installation}
%
% \paragraph{Unpacking.} Unpack the \xfile{oberdiek.tds.zip} in the
% TDS tree (also known as \xfile{texmf} tree) of your choice.
% Example (linux):
% \begin{quote}
%   |unzip oberdiek.tds.zip -d ~/texmf|
% \end{quote}
%
% \paragraph{Script installation.}
% Check the directory \xfile{TDS:scripts/oberdiek/} for
% scripts that need further installation steps.
% Package \xpackage{attachfile2} comes with the Perl script
% \xfile{pdfatfi.pl} that should be installed in such a way
% that it can be called as \texttt{pdfatfi}.
% Example (linux):
% \begin{quote}
%   |chmod +x scripts/oberdiek/pdfatfi.pl|\\
%   |cp scripts/oberdiek/pdfatfi.pl /usr/local/bin/|
% \end{quote}
%
% \subsection{Package installation}
%
% \paragraph{Unpacking.} The \xfile{.dtx} file is a self-extracting
% \docstrip\ archive. The files are extracted by running the
% \xfile{.dtx} through \plainTeX:
% \begin{quote}
%   \verb|tex pdfcolmk.dtx|
% \end{quote}
%
% \paragraph{TDS.} Now the different files must be moved into
% the different directories in your installation TDS tree
% (also known as \xfile{texmf} tree):
% \begin{quote}
% \def\t{^^A
% \begin{tabular}{@{}>{\ttfamily}l@{ $\rightarrow$ }>{\ttfamily}l@{}}
%   pdfcolmk.sty & tex/latex/oberdiek/pdfcolmk.sty\\
%   pdfcolmk.pdf & doc/latex/oberdiek/pdfcolmk.pdf\\
%   pdfcolmk.dtx & source/latex/oberdiek/pdfcolmk.dtx\\
% \end{tabular}^^A
% }^^A
% \sbox0{\t}^^A
% \ifdim\wd0>\linewidth
%   \begingroup
%     \advance\linewidth by\leftmargin
%     \advance\linewidth by\rightmargin
%   \edef\x{\endgroup
%     \def\noexpand\lw{\the\linewidth}^^A
%   }\x
%   \def\lwbox{^^A
%     \leavevmode
%     \hbox to \linewidth{^^A
%       \kern-\leftmargin\relax
%       \hss
%       \usebox0
%       \hss
%       \kern-\rightmargin\relax
%     }^^A
%   }^^A
%   \ifdim\wd0>\lw
%     \sbox0{\small\t}^^A
%     \ifdim\wd0>\linewidth
%       \ifdim\wd0>\lw
%         \sbox0{\footnotesize\t}^^A
%         \ifdim\wd0>\linewidth
%           \ifdim\wd0>\lw
%             \sbox0{\scriptsize\t}^^A
%             \ifdim\wd0>\linewidth
%               \ifdim\wd0>\lw
%                 \sbox0{\tiny\t}^^A
%                 \ifdim\wd0>\linewidth
%                   \lwbox
%                 \else
%                   \usebox0
%                 \fi
%               \else
%                 \lwbox
%               \fi
%             \else
%               \usebox0
%             \fi
%           \else
%             \lwbox
%           \fi
%         \else
%           \usebox0
%         \fi
%       \else
%         \lwbox
%       \fi
%     \else
%       \usebox0
%     \fi
%   \else
%     \lwbox
%   \fi
% \else
%   \usebox0
% \fi
% \end{quote}
% If you have a \xfile{docstrip.cfg} that configures and enables \docstrip's
% TDS installing feature, then some files can already be in the right
% place, see the documentation of \docstrip.
%
% \subsection{Refresh file name databases}
%
% If your \TeX~distribution
% (\teTeX, \mikTeX, \dots) relies on file name databases, you must refresh
% these. For example, \teTeX\ users run \verb|texhash| or
% \verb|mktexlsr|.
%
% \subsection{Some details for the interested}
%
% \paragraph{Attached source.}
%
% The PDF documentation on CTAN also includes the
% \xfile{.dtx} source file. It can be extracted by
% AcrobatReader 6 or higher. Another option is \textsf{pdftk},
% e.g. unpack the file into the current directory:
% \begin{quote}
%   \verb|pdftk pdfcolmk.pdf unpack_files output .|
% \end{quote}
%
% \paragraph{Unpacking with \LaTeX.}
% The \xfile{.dtx} chooses its action depending on the format:
% \begin{description}
% \item[\plainTeX:] Run \docstrip\ and extract the files.
% \item[\LaTeX:] Generate the documentation.
% \end{description}
% If you insist on using \LaTeX\ for \docstrip\ (really,
% \docstrip\ does not need \LaTeX), then inform the autodetect routine
% about your intention:
% \begin{quote}
%   \verb|latex \let\install=y\input{pdfcolmk.dtx}|
% \end{quote}
% Do not forget to quote the argument according to the demands
% of your shell.
%
% \paragraph{Generating the documentation.}
% You can use both the \xfile{.dtx} or the \xfile{.drv} to generate
% the documentation. The process can be configured by the
% configuration file \xfile{ltxdoc.cfg}. For instance, put this
% line into this file, if you want to have A4 as paper format:
% \begin{quote}
%   \verb|\PassOptionsToClass{a4paper}{article}|
% \end{quote}
% An example follows how to generate the
% documentation with pdf\LaTeX:
% \begin{quote}
%\begin{verbatim}
%pdflatex pdfcolmk.dtx
%makeindex -s gind.ist pdfcolmk.idx
%pdflatex pdfcolmk.dtx
%makeindex -s gind.ist pdfcolmk.idx
%pdflatex pdfcolmk.dtx
%\end{verbatim}
% \end{quote}
%
% \section{Catalogue}
%
% The following XML file can be used as source for the
% \href{http://mirror.ctan.org/help/Catalogue/catalogue.html}{\TeX\ Catalogue}.
% The elements \texttt{caption} and \texttt{description} are imported
% from the original XML file from the Catalogue.
% The name of the XML file in the Catalogue is \xfile{pdfcolmk.xml}.
%    \begin{macrocode}
%<*catalogue>
<?xml version='1.0' encoding='us-ascii'?>
<!DOCTYPE entry SYSTEM 'catalogue.dtd'>
<entry datestamp='$Date$' modifier='$Author$' id='pdfcolmk'>
  <name>pdfcolmk</name>
  <caption>Improving colour support under pdftex.</caption>
  <authorref id='auth:oberdiek'/>
  <copyright owner='Heiko Oberdiek' year='2000,2005-2008'/>
  <license type='lppl1.3'/>
  <version number='1.2'/>
  <description>
    The package provides macros that emulate the &#x2018;colour stack&#x2019;
    functionality of dvips.  The colour stack deals with colour
    manipulations when asynchronous events (like page-breaking) occur;
    pdftex does not (yet) have such a stack, but dvips does, and the
    <xref refid='color'>color</xref> package makes extensive use of
    it.
    <p/>
    This package is an experimental solution to the problem, and works
    best with pdf-e-tex.
    <p/>
    The package is part of the <xref refid='oberdiek'>oberdiek</xref> bundle.
  </description>
  <documentation details='Package documentation'
      href='ctan:/macros/latex/contrib/oberdiek/pdfcolmk.pdf'/>
  <ctan file='true' path='/macros/latex/contrib/oberdiek/pdfcolmk.dtx'/>
  <miktex location='oberdiek'/>
  <texlive location='oberdiek'/>
  <install path='/macros/latex/contrib/oberdiek/oberdiek.tds.zip'/>
</entry>
%</catalogue>
%    \end{macrocode}
%
% \begin{History}
%   \begin{Version}{2000/08/27 v0.1}
%   \item
%     First published version in newsgroup \xnewsgroup{comp.text.tex}:\\
%     \URL{``\link{pdftex: bug with colors?}''}^^A
%     {http://groups.google.com/group/comp.text.tex/msg/6f088e69e4085d2c}
%   \end{Version}
%   \begin{Version}{2000/09/02 v0.2}
%   \item
%     Next try.
%   \end{Version}
%   \begin{Version}{2000/09/02 v0.3}
%   \item
%     Solution without \eTeX\ added.
%   \end{Version}
%   \begin{Version}{2000/09/06 v0.4}
%   \item
%     Patch commands added.
%   \item
%     Patch for seminar.cls added.
%   \end{Version}
%   \begin{Version}{2000/09/06 v0.5}
%   \item
%     Bug fix: initialization of \cs{pec@value} added.
%   \end{Version}
%   \begin{Version}{2005/06/15 v0.6}
%   \item
%     Support for \cs{marginpar} added.
%     See thread in \xnewsgroup{comp.text.tex}:\\
%     \URL{``\link{Using \cs{textcolor} and \cs{marginpar} together}''}^^A
%     {http://groups.google.com/group/comp.text.tex/msg/38ed58f8845a2a4f}
%   \end{Version}
%   \begin{Version}{2005/07/09 v0.7}
%   \item
%     Output support added for \xpackage{memoir},
%     provided by Lars Madsen.
%   \end{Version}
%   \begin{Version}{2006/02/20 v0.8}
%   \item
%     Code is not changed.
%   \item
%     DTX framework.
%   \end{Version}
%   \begin{Version}{2007/01/01 v1.0}
%   \item
%     If \xfile{pdftex.def} \textgreater= 2007/01/01 v0.04a is used with
%     \pdfTeX\ \textgreater= 1.40.0, then package \xpackage{pdfcolmk} is obsolete.
%   \end{Version}
%   \begin{Version}{2007/04/11 v1.1}
%   \item
%     Line ends sanitized.
%   \end{Version}
%   \begin{Version}{2008/08/11 v1.2}
%   \item
%     Code is not changed.
%   \item
%     URLs updated.
%   \end{Version}
% \end{History}
%
% \PrintIndex
%
% \Finale
\endinput

%        (quote the arguments according to the demands of your shell)
%
% Documentation:
%    (a) If pdfcolmk.drv is present:
%           latex pdfcolmk.drv
%    (b) Without pdfcolmk.drv:
%           latex pdfcolmk.dtx; ...
%    The class ltxdoc loads the configuration file ltxdoc.cfg
%    if available. Here you can specify further options, e.g.
%    use A4 as paper format:
%       \PassOptionsToClass{a4paper}{article}
%
%    Programm calls to get the documentation (example):
%       pdflatex pdfcolmk.dtx
%       makeindex -s gind.ist pdfcolmk.idx
%       pdflatex pdfcolmk.dtx
%       makeindex -s gind.ist pdfcolmk.idx
%       pdflatex pdfcolmk.dtx
%
% Installation:
%    TDS:tex/latex/oberdiek/pdfcolmk.sty
%    TDS:doc/latex/oberdiek/pdfcolmk.pdf
%    TDS:source/latex/oberdiek/pdfcolmk.dtx
%
%<*ignore>
\begingroup
  \catcode123=1 %
  \catcode125=2 %
  \def\x{LaTeX2e}%
\expandafter\endgroup
\ifcase 0\ifx\install y1\fi\expandafter
         \ifx\csname processbatchFile\endcsname\relax\else1\fi
         \ifx\fmtname\x\else 1\fi\relax
\else\csname fi\endcsname
%</ignore>
%<*install>
\input docstrip.tex
\Msg{************************************************************************}
\Msg{* Installation}
\Msg{* Package: pdfcolmk 2008/08/11 v1.2 Color support for pdfTeX via marks (HO)}
\Msg{************************************************************************}

\keepsilent
\askforoverwritefalse

\let\MetaPrefix\relax
\preamble

This is a generated file.

Project: pdfcolmk
Version: 2008/08/11 v1.2

Copyright (C) 2000, 2005-2008 by
   Heiko Oberdiek <heiko.oberdiek at googlemail.com>

This work may be distributed and/or modified under the
conditions of the LaTeX Project Public License, either
version 1.3c of this license or (at your option) any later
version. This version of this license is in
   http://www.latex-project.org/lppl/lppl-1-3c.txt
and the latest version of this license is in
   http://www.latex-project.org/lppl.txt
and version 1.3 or later is part of all distributions of
LaTeX version 2005/12/01 or later.

This work has the LPPL maintenance status "maintained".

This Current Maintainer of this work is Heiko Oberdiek.

This work consists of the main source file pdfcolmk.dtx
and the derived files
   pdfcolmk.sty, pdfcolmk.pdf, pdfcolmk.ins, pdfcolmk.drv.

\endpreamble
\let\MetaPrefix\DoubleperCent

\generate{%
  \file{pdfcolmk.ins}{\from{pdfcolmk.dtx}{install}}%
  \file{pdfcolmk.drv}{\from{pdfcolmk.dtx}{driver}}%
  \usedir{tex/latex/oberdiek}%
  \file{pdfcolmk.sty}{\from{pdfcolmk.dtx}{package}}%
  \nopreamble
  \nopostamble
  \usedir{source/latex/oberdiek/catalogue}%
  \file{pdfcolmk.xml}{\from{pdfcolmk.dtx}{catalogue}}%
}

\catcode32=13\relax% active space
\let =\space%
\Msg{************************************************************************}
\Msg{*}
\Msg{* To finish the installation you have to move the following}
\Msg{* file into a directory searched by TeX:}
\Msg{*}
\Msg{*     pdfcolmk.sty}
\Msg{*}
\Msg{* To produce the documentation run the file `pdfcolmk.drv'}
\Msg{* through LaTeX.}
\Msg{*}
\Msg{* Happy TeXing!}
\Msg{*}
\Msg{************************************************************************}

\endbatchfile
%</install>
%<*ignore>
\fi
%</ignore>
%<*driver>
\NeedsTeXFormat{LaTeX2e}
\ProvidesFile{pdfcolmk.drv}%
  [2008/08/11 v1.2 Color support for pdfTeX via marks (HO)]%
\documentclass{ltxdoc}
\usepackage{holtxdoc}[2011/11/22]
\begin{document}
  \DocInput{pdfcolmk.dtx}%
\end{document}
%</driver>
% \fi
%
% \CheckSum{843}
%
% \CharacterTable
%  {Upper-case    \A\B\C\D\E\F\G\H\I\J\K\L\M\N\O\P\Q\R\S\T\U\V\W\X\Y\Z
%   Lower-case    \a\b\c\d\e\f\g\h\i\j\k\l\m\n\o\p\q\r\s\t\u\v\w\x\y\z
%   Digits        \0\1\2\3\4\5\6\7\8\9
%   Exclamation   \!     Double quote  \"     Hash (number) \#
%   Dollar        \$     Percent       \%     Ampersand     \&
%   Acute accent  \'     Left paren    \(     Right paren   \)
%   Asterisk      \*     Plus          \+     Comma         \,
%   Minus         \-     Point         \.     Solidus       \/
%   Colon         \:     Semicolon     \;     Less than     \<
%   Equals        \=     Greater than  \>     Question mark \?
%   Commercial at \@     Left bracket  \[     Backslash     \\
%   Right bracket \]     Circumflex    \^     Underscore    \_
%   Grave accent  \`     Left brace    \{     Vertical bar  \|
%   Right brace   \}     Tilde         \~}
%
% \GetFileInfo{pdfcolmk.drv}
%
% \title{The \xpackage{pdfcolmk} package}
% \date{2008/08/11 v1.2}
% \author{Heiko Oberdiek\\\xemail{heiko.oberdiek at googlemail.com}}
%
% \maketitle
%
% \begin{abstract}
% This package tries a solution for the missing color
% stack of \pdfTeX.
% \end{abstract}
%
% \tableofcontents
%
% \section{Documentation}
%
% \subsection{Introduction}
%
% This package uses a mark register in order to solve the
% problem of a missing color stack of \pdfTeX\ prior 1.40.0.
% Since this version of \pdfTeX\ a color stack is available
% and supported by \xfile{pdftex.def} 2007/01/01 v0.04a.
% In this case this package is obsolete and the package
% stops its loading.
%
% \subsection{Background}
%
% After the Dante meeting (Clausthal 2000) I have started
% to experiment with the eTeX method of a \emph{colour} mark.
% One of the major problems is the understanding of the
% output routine and the need to rewrite it because of
% missing hooks. Currently I have made some tests in
% in onecolumn and twocolumn mode, but the state is
% experimental.
%
% \subsection{Limitations}
%
% \begin{itemize}
% \item Mark limitations: page breaks in math.
% \item \LaTeX's output routine is redefinded.
%   \begin{itemize}
%   \item Changes in the output routine of newer versions
%         of LaTeX are not detected.
%   \item Packages that change the output routine are not
%         supported.
%   \end{itemize}
% \item It does not support several independent text
%       streams like footnotes.
% \item Limitations in float and marginpar support.
% \end{itemize}
%
% \subsection{Recommendation}
%
% \eTeX\ (for additional mark register)
% Without \eTeX\ \LaTeX's mark commands are redefined
% to store an additional color value.
%
% \subsection{Usage}
%
% Load after package color:
% \begin{quote}
%   |\usepackage[pdftex]{color}|\\
%   |\usepackage{pdfcolmk}|
% \end{quote}
%
% \subsection{Compatibility}
%
% \begin{itemize}
% \item Load the following packages after \xpackage{pdfcolmk}:
%   \begin{quote}
%       \xpackage{mparhack.sty}
%   \end{quote}
% \item Load the following packages before \xpackage{pdfcolmk}:
%   \begin{quote}
%       \xpackage{marn.sty}\\
%       \xpackage{newmarn.sty}
%   \end{quote}
% \item Supported \cs{@addmarginpar} patch:
%   \begin{quote}
%       \xpackage{latex/base/latex.ltx}\\
%       \xpackage{memoir.cls}\\
%       \xpackage{poemscol/marn.sty}, \xpackage{poemscol/newmarn.sty}\\
%       \xpackage{mparhack.sty}
%   \end{quote}
% \item Unsupported \cs{@addmarginpar} patch:
%   \begin{quote}
%       \xpackage{lineno.sty}\\
%       \xpackage{sttools/marginal.sty}\\
%       \xpackage{revtex4.cls}
%   \end{quote}
% \end{itemize}
%
% \StopEventually{
% }
%
% \section{Implementation}
%
%    \begin{macrocode}
%<*package>
%    \end{macrocode}
%    Package identification.
%    \begin{macrocode}
\NeedsTeXFormat{LaTeX2e}
\ProvidesPackage{pdfcolmk}%
  [2008/08/11 v1.2 Color support for pdfTeX via marks (HO)]
%    \end{macrocode}
%
%    \begin{macrocode}
\@ifundefined{ver@pdftex.def}{%
  \PackageWarningNoLine{pdfcolmk}{%
    Nothing to fix, because \string`pdftex.def\string' not loaded%
  }%
  \endinput
}{}
\@ifpackageloaded{color}{}{%
  \PackageWarningNoLine{pdfcolmk}{%
    Nothing to fix, because \string`color.sty\string' not loaded%
  }%
  \endinput
}
\begingroup\expandafter\expandafter\expandafter\endgroup
\expandafter\ifx\csname main@pdfcolorstack\endcsname\relax
\else
  % pdftex.def >= 2007/01/01 0.04a and pdfTeX >= 1.40.0
  \begingroup
    \let\on@line\@empty
    \PackageInfo{pdfcolmk}{%
      The color stack of pdfTeX \string>\string= 1.40 is used. %
      Therefore\MessageBreak
      this package is not necessary and not loaded%
    }%
  \endgroup
  \expandafter\endinput
\fi

\PackageInfo{pdfcolmk}{%
  This package tries to simulate dvips's color stack\MessageBreak
  for pdfTeX based on a mark register of e-TeX.\MessageBreak
  It redefines LaTeX's output routine. Therefore\MessageBreak
  use with care, no warranties%
}

\ifx\marks\@undefined

  \let\pec@mark\mark
  \let\pec@value\empty
  \long\def\mark#1{%
    \protected@xdef\pec@value{#1}%
    \pec@setmark
  }%
  \def\pec@setmark{%
    \begingroup
      \@temptokena\expandafter{\pec@value}%
      \pec@mark{{\current@color}\the\@temptokena}%
    \endgroup
  }%
  \def\pec@getmark{%
    \xdef\pec@botcolor{%
      \expandafter\@firstofthree\botmark\@empty\@empty\@empty
    }%
  }%
  \long\def\@firstofthree#1#2#3{#1}%
  \CheckCommand{\@leftmark}[2]{#1}%
  \CheckCommand{\@rightmark}[2]{#2}%
  \CheckCommand*{\leftmark}{%
    \expandafter\@leftmark\botmark\@empty\@empty
  }%
  \CheckCommand*{\rightmark}{%
    \expandafter\@rightmark\firstmark\@empty\@empty
  }%
  \long\def\@leftmark#1#2#3{#2}%
  \long\def\@rightmark#1#2#3{#3}%
  \g@addto@macro\leftmark\@empty
  \g@addto@macro\rightmark\@empty

\else

  \RequirePackage{etex}[1998/03/26]%
  \newmarks\pec@marks
  \def\pec@setmark{\marks\pec@marks{\current@color}}%
  \def\pec@getmark{\xdef\pec@botcolor{\botmarks\pec@marks}}%

\fi
%    \end{macrocode}
%
% \subsection{\cs{marginpar} fix}
%
%    \begin{macrocode}
\chardef\pec@result\z@
\def\pec@temp#1{%
  \chardef\pec@result\@ne
  \begingroup
    \let\on@line\@empty
    \PackageInfo{pdfcolmk}{%
      Patch for \string\@addmarginpar\space applied (#1)%
    }%
  \endgroup
}
%    \end{macrocode}
%
% \subsubsection{latex/base/latex.ltx}
%
%    \begin{macrocode}
\def\pec@addmarginpar{%
  \@next\@marbox\@currlist{%
    \@cons\@freelist\@marbox
    \@cons\@freelist\@currbox
  }\@latexbug
  \@tempcnta\@ne
  \if@twocolumn
    \if@firstcolumn
      \@tempcnta\m@ne
    \fi
  \else
    \if@mparswitch
      \ifodd\c@page
      \else
        \@tempcnta\m@ne
      \fi
    \fi
    \if@reversemargin \@tempcnta -\@tempcnta \fi
  \fi
  \ifnum\@tempcnta <\z@  \global\setbox\@marbox\box\@currbox \fi
  \@tempdima\@mparbottom
  \advance\@tempdima -\@pageht
  \advance\@tempdima\ht\@marbox
  \ifdim\@tempdima >\z@
    \@latex@warning@no@line{Marginpar on page \thepage\space moved}%
  \else
    \@tempdima\z@
  \fi
  \global\@mparbottom\@pageht
  \global\advance\@mparbottom\@tempdima
  \global\advance\@mparbottom\dp\@marbox
  \global\advance\@mparbottom\marginparpush
  \advance\@tempdima -\ht\@marbox
  \global\setbox\@marbox\vbox{%
    \vskip \@tempdima
    \box \@marbox
  }%
  \global \ht\@marbox \z@
  \global \dp\@marbox \z@
  \kern -\@pagedp
  \nointerlineskip
  \hb@xt@\columnwidth{%
    \ifnum \@tempcnta >\z@
      \hskip\columnwidth
      \hskip\marginparsep
    \else
      \hskip -\marginparsep
      \hskip -\marginparwidth
    \fi
    \box\@marbox \hss
  }%
  \nointerlineskip
  \hbox{\vrule \@height\z@ \@width\z@ \@depth\@pagedp}%
}
\ifx\pec@addmarginpar\@addmarginpar
  \pec@temp{latex/base}%
\fi
%    \end{macrocode}
%
% \subsubsection{memoir.cls}
%
%    \begin{macrocode}
\def\pec@addmarginpar{%
  \checkoddpage
  \@next\@marbox\@currlist{%
    \@cons\@freelist\@marbox
    \@cons\@freelist\@currbox
  }\@latexbug
  \@tempcnta\@ne
  \if@twocolumn
    \if@firstcolumn
      \@tempcnta\m@ne
    \fi
  \else
    \if@mparswitch
      \ifoddpage
      \else
        \@tempcnta\m@ne
      \fi
    \fi
    \if@reversemargin
      \@tempcnta -\@tempcnta
    \fi
  \fi
  \ifnum\@tempcnta <\z@
    \global\setbox\@marbox\box\@currbox
  \fi
  \@tempdima\@mparbottom
  \advance\@tempdima -\@pageht
  \advance\@tempdima\ht\@marbox
  \ifdim\@tempdima >\z@
    \@latex@warning@no@line{%
      Marginpar on page \thepage\space moved by \the\@tempdima
    }%
  \else
    \@tempdima\z@
  \fi
  \global\@mparbottom\@pageht
  \global\advance\@mparbottom\@tempdima
  \global\advance\@mparbottom\dp\@marbox
  \global\advance\@mparbottom\marginparpush
  \advance\@tempdima -\ht\@marbox
  \global\setbox\@marbox\vbox{%
    \vskip \@tempdima
    \box \@marbox
  }%
  \global \ht\@marbox \z@
  \global \dp\@marbox \z@
  \kern -\@pagedp
  \nointerlineskip
  \hb@xt@\columnwidth{%
    \ifnum \@tempcnta >\z@
      \hskip\columnwidth
      \hskip\marginparsep
    \else
      \hskip -\marginparsep
      \hskip -\marginparwidth
    \fi
    \box\@marbox
    \hss
  }%
  \nointerlineskip
  \hbox{\vrule \@height\z@ \@width\z@ \@depth\@pagedp}%
}%
\ifx\pec@addmarginpar\@addmarginpar
  \pec@temp{memoir.cls}%
\fi
%    \end{macrocode}
%
% \subsubsection{poemscol/marn.sty, poemscol/newmarn.sty}
%
%    \begin{macrocode}
\def\pec@addmarginpar{%
  \@next \@marbox\@currlist{%
    \@cons\@freelist\@marbox
    \@cons\@freelist\@currbox
  }\@latexbug
  \global\advance\@mpar@count\m@ne
  \@ifundefined{@marn@\the\@mpar@count @}{% was location logged last time?
    \@tempcnta\@ne % NO: use original LaTeX logic
    \if@twocolumn
      \if@firstcolumn
        \@tempcnta\m@ne
      \fi
    \else
      \if@mparswitch
        \ifodd\c@page
        \else
          \@tempcnta\m@ne
        \fi
      \fi
      \if@reversemargin
        \@tempcnta -\@tempcnta
      \fi
    \fi
  }{%
    \@tempcnta %    YES: use record from last time to decide side.
    \@nameuse{@marn@\the\@mpar@count @}%
    \if@reversemargin -\fi \@ne
  }%
  \ifnum\@tempcnta <\z@
    \global\setbox\@marbox\box\@currbox
    \global\let\@marnbottom\@mparbottoml
  \else
    \global\let\@marnbottom\@mparbottom
  \fi
  \@tempdima\@marnbottom \advance\@tempdima -\@pageht
  \advance\@tempdima\ht\@marbox
  \ifdim\@tempdima >\z@
    \@@warning{Marginpar on page \thepage\space moved}%
  \else
    \@tempdima\z@
  \fi
  \global\@marnbottom\@pageht
  \global\advance\@marnbottom\@tempdima
  \global\advance\@marnbottom\dp\@marbox
  \global\advance\@marnbottom\marginparpush
  \advance\@tempdima -\ht\@marbox
  \global\ht\@marbox\z@
  \global\dp\@marbox\z@
  \vskip -\@pagedp
  \vskip\@tempdima\nointerlineskip
  \hbox to\columnwidth{%
    \ifnum \@tempcnta >\z@
      \hskip\columnwidth
      \hskip\marginparsep
    \else
      \hskip -\marginparsep
      \hskip -\marginparwidth
    \fi
    \if@filesw % record where this is for use next time:
       \@marn@log\@mpar@count
    \fi
    \box\@marbox
    \hss
  }%
  \nobreak   %% RmS 91/06/21 \nobreak added
  \vskip -\@tempdima
  \nointerlineskip
  \hbox{\vrule \@height\z@ \@width\z@ \@depth\@pagedp}%
}
\ifx\pec@addmarginpar\@addmarginpar
  \pec@temp{poemscol/(new)marn.sty}%
\fi
%    \end{macrocode}
%
% \subsubsection{refman/refart.cls, refnam/refrep.cls}
%
%    \begin{macrocode}
\def\pec@addmarginpar{%
  \@next\@marbox\@currlist{%
    \@cons\@freelist\@marbox
    \@cons\@freelist\@currbox
  }\@latexbug
  \@tempcnta\@ne
  \if@twocolumn
    \if@firstcolumn
      \@tempcnta\m@ne
    \fi
  \else
    \@tempcnta\m@ne
  \fi
  \ifnum\@tempcnta <\z@
    \global\setbox\@marbox\box\@currbox
  \fi
  \@tempdima\@mparbottom
  \advance\@tempdima -\@pageht
  \advance\@tempdima\ht\@marbox
  \ifdim\@tempdima >\z@
     \@@warning{Marginpar on page \thepage\space moved}%
  \else
     \@tempdima\z@
  \fi
  \global\@mparbottom\@pageht
  \global\advance\@mparbottom\@tempdima
  \global\advance\@mparbottom\dp\@marbox
  \global\advance\@mparbottom\marginparpush
  \advance\@tempdima -\ht\@marbox
  \global\setbox\@marbox\vbox{%
    \vskip \@tempdima \box \@marbox
  }%
  \global \ht\@marbox \z@
  \global \dp\@marbox \z@
  \kern -\@pagedp
  \nointerlineskip
  \hb@xt@\columnwidth{%
    \ifnum \@tempcnta >\z@
      \hskip\columnwidth
      \hskip\marginparsep
    \else
      \hskip -\marginparsep
      \hskip -\marginparwidth
    \fi
    \box\@marbox
    \hss
  }%
  \nointerlineskip
  \hbox{\vrule \@height\z@ \@width\z@ \@depth\@pagedp}%
}
\ifx\pec@addmarginpar\@addmarginpar
  \pec@temp{ref(art|rep).cls}%
\fi

\ifcase\pec@result
  \PackageInfo{pdfcolmk}{%
    Fix for \string\@addmarginpar\space is omitted, %
    because this variant\MessageBreak
    of \string\@addmarginpar\space
      is not recognized%
  }%
\else
  % apply patch for \@addmarginpar
  \def\pec@PatchAddMarginpar#1\columnwidth#2#3\@nil{%
    \pec@PatchAddMarginparI#2\@nil{#1}{#3}%
  }%
  \def\pec@PatchAddMarginparI#1\box\@marbox\hss#2\@nil#3#4{%
    \def\@addmarginpar{%
      #3%
      \columnwidth{%
        #1%
        \pdfliteral{q}%
        \rlap{%
          \box\@marbox
        }%
        \pdfliteral{Q}%
        \hss
        #2%
      }%
      #4%
    }%
  }%
  \expandafter\pec@PatchAddMarginpar\@addmarginpar\@nil
\fi
%    \end{macrocode}
%
% \subsection{Color fix}
%
%    \begin{macrocode}
\def\set@color{%
  \pdfliteral{\current@color}%
  \ifinner
  \else
    \pec@setmark
  \fi
  \aftergroup\reset@color
}
\def\reset@color{%
  \pdfliteral{\current@color}%
  \ifinner
  \else
    \pec@setmark
  \fi
}

\let\pec@botcolor\current@color

\def\pec@PatchVBoxCCLV{%
  \ifx\pec@botcolor\@empty
  \else
    \setbox\@cclv\vbox{%
      \pdfliteral{\pec@botcolor}%
      \unvbox\@cclv
    }%
  \fi
  \pec@getmark
}

\def\pec@PatchAlreadyInBox{%
  \ifx\pec@botcolor\@empty
  \else
    \pdfliteral{\pec@botcolor}%
  \fi
  \pec@getmark
}

\@ifclassloaded{memoir}{%
  \expandafter\def\expandafter\mem@makecol\expandafter{%
    \expandafter\pec@PatchVBoxCCLV
    \mem@makecol
  }%
  \endinput
}{}

\@ifclassloaded{seminar}{%
  \newcommand\pec@org@makeslide{}%
  \let\pec@org@makeslide\@makeslide
  \def\@makeslide{%
    \pec@PatchVBoxCCLV
    \pec@org@makeslide
  }%
  \endinput
}{}

\long\def\pec@output#1\@specialoutput\else#2\pec@end{%
  \begingroup
    \def\x{#2}%
  \expandafter\endgroup
  \ifx\x\@empty
    \PackageWarningNoLine{pdfcolmk}{%
      Unexpected \string\output\space routine detected,%
      \MessageBreak
      loading of package stopped%
    }%
    \expandafter\endinput
  \fi
}
\expandafter\expandafter\expandafter\pec@output
\expandafter\@firstofone\the\output\@specialoutput\else\pec@end

\long\def\pec@output#1\@specialoutput\else#2\pec@end{%
  \output{%
    #1\@specialoutput\else
    \pec@PatchVBoxCCLV
    #2%
  }%
}
\expandafter\expandafter\expandafter\pec@output
\expandafter\@firstofone\the\output\pec@end
%    \end{macrocode}
%
%    \begin{macrocode}
%</package>
%    \end{macrocode}
%
% \section{Installation}
%
% \subsection{Download}
%
% \paragraph{Package.} This package is available on
% CTAN\footnote{\url{ftp://ftp.ctan.org/tex-archive/}}:
% \begin{description}
% \item[\CTAN{macros/latex/contrib/oberdiek/pdfcolmk.dtx}] The source file.
% \item[\CTAN{macros/latex/contrib/oberdiek/pdfcolmk.pdf}] Documentation.
% \end{description}
%
%
% \paragraph{Bundle.} All the packages of the bundle `oberdiek'
% are also available in a TDS compliant ZIP archive. There
% the packages are already unpacked and the documentation files
% are generated. The files and directories obey the TDS standard.
% \begin{description}
% \item[\CTAN{install/macros/latex/contrib/oberdiek.tds.zip}]
% \end{description}
% \emph{TDS} refers to the standard ``A Directory Structure
% for \TeX\ Files'' (\CTAN{tds/tds.pdf}). Directories
% with \xfile{texmf} in their name are usually organized this way.
%
% \subsection{Bundle installation}
%
% \paragraph{Unpacking.} Unpack the \xfile{oberdiek.tds.zip} in the
% TDS tree (also known as \xfile{texmf} tree) of your choice.
% Example (linux):
% \begin{quote}
%   |unzip oberdiek.tds.zip -d ~/texmf|
% \end{quote}
%
% \paragraph{Script installation.}
% Check the directory \xfile{TDS:scripts/oberdiek/} for
% scripts that need further installation steps.
% Package \xpackage{attachfile2} comes with the Perl script
% \xfile{pdfatfi.pl} that should be installed in such a way
% that it can be called as \texttt{pdfatfi}.
% Example (linux):
% \begin{quote}
%   |chmod +x scripts/oberdiek/pdfatfi.pl|\\
%   |cp scripts/oberdiek/pdfatfi.pl /usr/local/bin/|
% \end{quote}
%
% \subsection{Package installation}
%
% \paragraph{Unpacking.} The \xfile{.dtx} file is a self-extracting
% \docstrip\ archive. The files are extracted by running the
% \xfile{.dtx} through \plainTeX:
% \begin{quote}
%   \verb|tex pdfcolmk.dtx|
% \end{quote}
%
% \paragraph{TDS.} Now the different files must be moved into
% the different directories in your installation TDS tree
% (also known as \xfile{texmf} tree):
% \begin{quote}
% \def\t{^^A
% \begin{tabular}{@{}>{\ttfamily}l@{ $\rightarrow$ }>{\ttfamily}l@{}}
%   pdfcolmk.sty & tex/latex/oberdiek/pdfcolmk.sty\\
%   pdfcolmk.pdf & doc/latex/oberdiek/pdfcolmk.pdf\\
%   pdfcolmk.dtx & source/latex/oberdiek/pdfcolmk.dtx\\
% \end{tabular}^^A
% }^^A
% \sbox0{\t}^^A
% \ifdim\wd0>\linewidth
%   \begingroup
%     \advance\linewidth by\leftmargin
%     \advance\linewidth by\rightmargin
%   \edef\x{\endgroup
%     \def\noexpand\lw{\the\linewidth}^^A
%   }\x
%   \def\lwbox{^^A
%     \leavevmode
%     \hbox to \linewidth{^^A
%       \kern-\leftmargin\relax
%       \hss
%       \usebox0
%       \hss
%       \kern-\rightmargin\relax
%     }^^A
%   }^^A
%   \ifdim\wd0>\lw
%     \sbox0{\small\t}^^A
%     \ifdim\wd0>\linewidth
%       \ifdim\wd0>\lw
%         \sbox0{\footnotesize\t}^^A
%         \ifdim\wd0>\linewidth
%           \ifdim\wd0>\lw
%             \sbox0{\scriptsize\t}^^A
%             \ifdim\wd0>\linewidth
%               \ifdim\wd0>\lw
%                 \sbox0{\tiny\t}^^A
%                 \ifdim\wd0>\linewidth
%                   \lwbox
%                 \else
%                   \usebox0
%                 \fi
%               \else
%                 \lwbox
%               \fi
%             \else
%               \usebox0
%             \fi
%           \else
%             \lwbox
%           \fi
%         \else
%           \usebox0
%         \fi
%       \else
%         \lwbox
%       \fi
%     \else
%       \usebox0
%     \fi
%   \else
%     \lwbox
%   \fi
% \else
%   \usebox0
% \fi
% \end{quote}
% If you have a \xfile{docstrip.cfg} that configures and enables \docstrip's
% TDS installing feature, then some files can already be in the right
% place, see the documentation of \docstrip.
%
% \subsection{Refresh file name databases}
%
% If your \TeX~distribution
% (\teTeX, \mikTeX, \dots) relies on file name databases, you must refresh
% these. For example, \teTeX\ users run \verb|texhash| or
% \verb|mktexlsr|.
%
% \subsection{Some details for the interested}
%
% \paragraph{Attached source.}
%
% The PDF documentation on CTAN also includes the
% \xfile{.dtx} source file. It can be extracted by
% AcrobatReader 6 or higher. Another option is \textsf{pdftk},
% e.g. unpack the file into the current directory:
% \begin{quote}
%   \verb|pdftk pdfcolmk.pdf unpack_files output .|
% \end{quote}
%
% \paragraph{Unpacking with \LaTeX.}
% The \xfile{.dtx} chooses its action depending on the format:
% \begin{description}
% \item[\plainTeX:] Run \docstrip\ and extract the files.
% \item[\LaTeX:] Generate the documentation.
% \end{description}
% If you insist on using \LaTeX\ for \docstrip\ (really,
% \docstrip\ does not need \LaTeX), then inform the autodetect routine
% about your intention:
% \begin{quote}
%   \verb|latex \let\install=y% \iffalse meta-comment
%
% File: pdfcolmk.dtx
% Version: 2008/08/11 v1.2
% Info: Color support for pdfTeX via marks
%
% Copyright (C) 2000, 2005-2008 by
%    Heiko Oberdiek <heiko.oberdiek at googlemail.com>
%
% This work may be distributed and/or modified under the
% conditions of the LaTeX Project Public License, either
% version 1.3c of this license or (at your option) any later
% version. This version of this license is in
%    http://www.latex-project.org/lppl/lppl-1-3c.txt
% and the latest version of this license is in
%    http://www.latex-project.org/lppl.txt
% and version 1.3 or later is part of all distributions of
% LaTeX version 2005/12/01 or later.
%
% This work has the LPPL maintenance status "maintained".
%
% This Current Maintainer of this work is Heiko Oberdiek.
%
% This work consists of the main source file pdfcolmk.dtx
% and the derived files
%    pdfcolmk.sty, pdfcolmk.pdf, pdfcolmk.ins, pdfcolmk.drv.
%
% Distribution:
%    CTAN:macros/latex/contrib/oberdiek/pdfcolmk.dtx
%    CTAN:macros/latex/contrib/oberdiek/pdfcolmk.pdf
%
% Unpacking:
%    (a) If pdfcolmk.ins is present:
%           tex pdfcolmk.ins
%    (b) Without pdfcolmk.ins:
%           tex pdfcolmk.dtx
%    (c) If you insist on using LaTeX
%           latex \let\install=y\input{pdfcolmk.dtx}
%        (quote the arguments according to the demands of your shell)
%
% Documentation:
%    (a) If pdfcolmk.drv is present:
%           latex pdfcolmk.drv
%    (b) Without pdfcolmk.drv:
%           latex pdfcolmk.dtx; ...
%    The class ltxdoc loads the configuration file ltxdoc.cfg
%    if available. Here you can specify further options, e.g.
%    use A4 as paper format:
%       \PassOptionsToClass{a4paper}{article}
%
%    Programm calls to get the documentation (example):
%       pdflatex pdfcolmk.dtx
%       makeindex -s gind.ist pdfcolmk.idx
%       pdflatex pdfcolmk.dtx
%       makeindex -s gind.ist pdfcolmk.idx
%       pdflatex pdfcolmk.dtx
%
% Installation:
%    TDS:tex/latex/oberdiek/pdfcolmk.sty
%    TDS:doc/latex/oberdiek/pdfcolmk.pdf
%    TDS:source/latex/oberdiek/pdfcolmk.dtx
%
%<*ignore>
\begingroup
  \catcode123=1 %
  \catcode125=2 %
  \def\x{LaTeX2e}%
\expandafter\endgroup
\ifcase 0\ifx\install y1\fi\expandafter
         \ifx\csname processbatchFile\endcsname\relax\else1\fi
         \ifx\fmtname\x\else 1\fi\relax
\else\csname fi\endcsname
%</ignore>
%<*install>
\input docstrip.tex
\Msg{************************************************************************}
\Msg{* Installation}
\Msg{* Package: pdfcolmk 2008/08/11 v1.2 Color support for pdfTeX via marks (HO)}
\Msg{************************************************************************}

\keepsilent
\askforoverwritefalse

\let\MetaPrefix\relax
\preamble

This is a generated file.

Project: pdfcolmk
Version: 2008/08/11 v1.2

Copyright (C) 2000, 2005-2008 by
   Heiko Oberdiek <heiko.oberdiek at googlemail.com>

This work may be distributed and/or modified under the
conditions of the LaTeX Project Public License, either
version 1.3c of this license or (at your option) any later
version. This version of this license is in
   http://www.latex-project.org/lppl/lppl-1-3c.txt
and the latest version of this license is in
   http://www.latex-project.org/lppl.txt
and version 1.3 or later is part of all distributions of
LaTeX version 2005/12/01 or later.

This work has the LPPL maintenance status "maintained".

This Current Maintainer of this work is Heiko Oberdiek.

This work consists of the main source file pdfcolmk.dtx
and the derived files
   pdfcolmk.sty, pdfcolmk.pdf, pdfcolmk.ins, pdfcolmk.drv.

\endpreamble
\let\MetaPrefix\DoubleperCent

\generate{%
  \file{pdfcolmk.ins}{\from{pdfcolmk.dtx}{install}}%
  \file{pdfcolmk.drv}{\from{pdfcolmk.dtx}{driver}}%
  \usedir{tex/latex/oberdiek}%
  \file{pdfcolmk.sty}{\from{pdfcolmk.dtx}{package}}%
  \nopreamble
  \nopostamble
  \usedir{source/latex/oberdiek/catalogue}%
  \file{pdfcolmk.xml}{\from{pdfcolmk.dtx}{catalogue}}%
}

\catcode32=13\relax% active space
\let =\space%
\Msg{************************************************************************}
\Msg{*}
\Msg{* To finish the installation you have to move the following}
\Msg{* file into a directory searched by TeX:}
\Msg{*}
\Msg{*     pdfcolmk.sty}
\Msg{*}
\Msg{* To produce the documentation run the file `pdfcolmk.drv'}
\Msg{* through LaTeX.}
\Msg{*}
\Msg{* Happy TeXing!}
\Msg{*}
\Msg{************************************************************************}

\endbatchfile
%</install>
%<*ignore>
\fi
%</ignore>
%<*driver>
\NeedsTeXFormat{LaTeX2e}
\ProvidesFile{pdfcolmk.drv}%
  [2008/08/11 v1.2 Color support for pdfTeX via marks (HO)]%
\documentclass{ltxdoc}
\usepackage{holtxdoc}[2011/11/22]
\begin{document}
  \DocInput{pdfcolmk.dtx}%
\end{document}
%</driver>
% \fi
%
% \CheckSum{843}
%
% \CharacterTable
%  {Upper-case    \A\B\C\D\E\F\G\H\I\J\K\L\M\N\O\P\Q\R\S\T\U\V\W\X\Y\Z
%   Lower-case    \a\b\c\d\e\f\g\h\i\j\k\l\m\n\o\p\q\r\s\t\u\v\w\x\y\z
%   Digits        \0\1\2\3\4\5\6\7\8\9
%   Exclamation   \!     Double quote  \"     Hash (number) \#
%   Dollar        \$     Percent       \%     Ampersand     \&
%   Acute accent  \'     Left paren    \(     Right paren   \)
%   Asterisk      \*     Plus          \+     Comma         \,
%   Minus         \-     Point         \.     Solidus       \/
%   Colon         \:     Semicolon     \;     Less than     \<
%   Equals        \=     Greater than  \>     Question mark \?
%   Commercial at \@     Left bracket  \[     Backslash     \\
%   Right bracket \]     Circumflex    \^     Underscore    \_
%   Grave accent  \`     Left brace    \{     Vertical bar  \|
%   Right brace   \}     Tilde         \~}
%
% \GetFileInfo{pdfcolmk.drv}
%
% \title{The \xpackage{pdfcolmk} package}
% \date{2008/08/11 v1.2}
% \author{Heiko Oberdiek\\\xemail{heiko.oberdiek at googlemail.com}}
%
% \maketitle
%
% \begin{abstract}
% This package tries a solution for the missing color
% stack of \pdfTeX.
% \end{abstract}
%
% \tableofcontents
%
% \section{Documentation}
%
% \subsection{Introduction}
%
% This package uses a mark register in order to solve the
% problem of a missing color stack of \pdfTeX\ prior 1.40.0.
% Since this version of \pdfTeX\ a color stack is available
% and supported by \xfile{pdftex.def} 2007/01/01 v0.04a.
% In this case this package is obsolete and the package
% stops its loading.
%
% \subsection{Background}
%
% After the Dante meeting (Clausthal 2000) I have started
% to experiment with the eTeX method of a \emph{colour} mark.
% One of the major problems is the understanding of the
% output routine and the need to rewrite it because of
% missing hooks. Currently I have made some tests in
% in onecolumn and twocolumn mode, but the state is
% experimental.
%
% \subsection{Limitations}
%
% \begin{itemize}
% \item Mark limitations: page breaks in math.
% \item \LaTeX's output routine is redefinded.
%   \begin{itemize}
%   \item Changes in the output routine of newer versions
%         of LaTeX are not detected.
%   \item Packages that change the output routine are not
%         supported.
%   \end{itemize}
% \item It does not support several independent text
%       streams like footnotes.
% \item Limitations in float and marginpar support.
% \end{itemize}
%
% \subsection{Recommendation}
%
% \eTeX\ (for additional mark register)
% Without \eTeX\ \LaTeX's mark commands are redefined
% to store an additional color value.
%
% \subsection{Usage}
%
% Load after package color:
% \begin{quote}
%   |\usepackage[pdftex]{color}|\\
%   |\usepackage{pdfcolmk}|
% \end{quote}
%
% \subsection{Compatibility}
%
% \begin{itemize}
% \item Load the following packages after \xpackage{pdfcolmk}:
%   \begin{quote}
%       \xpackage{mparhack.sty}
%   \end{quote}
% \item Load the following packages before \xpackage{pdfcolmk}:
%   \begin{quote}
%       \xpackage{marn.sty}\\
%       \xpackage{newmarn.sty}
%   \end{quote}
% \item Supported \cs{@addmarginpar} patch:
%   \begin{quote}
%       \xpackage{latex/base/latex.ltx}\\
%       \xpackage{memoir.cls}\\
%       \xpackage{poemscol/marn.sty}, \xpackage{poemscol/newmarn.sty}\\
%       \xpackage{mparhack.sty}
%   \end{quote}
% \item Unsupported \cs{@addmarginpar} patch:
%   \begin{quote}
%       \xpackage{lineno.sty}\\
%       \xpackage{sttools/marginal.sty}\\
%       \xpackage{revtex4.cls}
%   \end{quote}
% \end{itemize}
%
% \StopEventually{
% }
%
% \section{Implementation}
%
%    \begin{macrocode}
%<*package>
%    \end{macrocode}
%    Package identification.
%    \begin{macrocode}
\NeedsTeXFormat{LaTeX2e}
\ProvidesPackage{pdfcolmk}%
  [2008/08/11 v1.2 Color support for pdfTeX via marks (HO)]
%    \end{macrocode}
%
%    \begin{macrocode}
\@ifundefined{ver@pdftex.def}{%
  \PackageWarningNoLine{pdfcolmk}{%
    Nothing to fix, because \string`pdftex.def\string' not loaded%
  }%
  \endinput
}{}
\@ifpackageloaded{color}{}{%
  \PackageWarningNoLine{pdfcolmk}{%
    Nothing to fix, because \string`color.sty\string' not loaded%
  }%
  \endinput
}
\begingroup\expandafter\expandafter\expandafter\endgroup
\expandafter\ifx\csname main@pdfcolorstack\endcsname\relax
\else
  % pdftex.def >= 2007/01/01 0.04a and pdfTeX >= 1.40.0
  \begingroup
    \let\on@line\@empty
    \PackageInfo{pdfcolmk}{%
      The color stack of pdfTeX \string>\string= 1.40 is used. %
      Therefore\MessageBreak
      this package is not necessary and not loaded%
    }%
  \endgroup
  \expandafter\endinput
\fi

\PackageInfo{pdfcolmk}{%
  This package tries to simulate dvips's color stack\MessageBreak
  for pdfTeX based on a mark register of e-TeX.\MessageBreak
  It redefines LaTeX's output routine. Therefore\MessageBreak
  use with care, no warranties%
}

\ifx\marks\@undefined

  \let\pec@mark\mark
  \let\pec@value\empty
  \long\def\mark#1{%
    \protected@xdef\pec@value{#1}%
    \pec@setmark
  }%
  \def\pec@setmark{%
    \begingroup
      \@temptokena\expandafter{\pec@value}%
      \pec@mark{{\current@color}\the\@temptokena}%
    \endgroup
  }%
  \def\pec@getmark{%
    \xdef\pec@botcolor{%
      \expandafter\@firstofthree\botmark\@empty\@empty\@empty
    }%
  }%
  \long\def\@firstofthree#1#2#3{#1}%
  \CheckCommand{\@leftmark}[2]{#1}%
  \CheckCommand{\@rightmark}[2]{#2}%
  \CheckCommand*{\leftmark}{%
    \expandafter\@leftmark\botmark\@empty\@empty
  }%
  \CheckCommand*{\rightmark}{%
    \expandafter\@rightmark\firstmark\@empty\@empty
  }%
  \long\def\@leftmark#1#2#3{#2}%
  \long\def\@rightmark#1#2#3{#3}%
  \g@addto@macro\leftmark\@empty
  \g@addto@macro\rightmark\@empty

\else

  \RequirePackage{etex}[1998/03/26]%
  \newmarks\pec@marks
  \def\pec@setmark{\marks\pec@marks{\current@color}}%
  \def\pec@getmark{\xdef\pec@botcolor{\botmarks\pec@marks}}%

\fi
%    \end{macrocode}
%
% \subsection{\cs{marginpar} fix}
%
%    \begin{macrocode}
\chardef\pec@result\z@
\def\pec@temp#1{%
  \chardef\pec@result\@ne
  \begingroup
    \let\on@line\@empty
    \PackageInfo{pdfcolmk}{%
      Patch for \string\@addmarginpar\space applied (#1)%
    }%
  \endgroup
}
%    \end{macrocode}
%
% \subsubsection{latex/base/latex.ltx}
%
%    \begin{macrocode}
\def\pec@addmarginpar{%
  \@next\@marbox\@currlist{%
    \@cons\@freelist\@marbox
    \@cons\@freelist\@currbox
  }\@latexbug
  \@tempcnta\@ne
  \if@twocolumn
    \if@firstcolumn
      \@tempcnta\m@ne
    \fi
  \else
    \if@mparswitch
      \ifodd\c@page
      \else
        \@tempcnta\m@ne
      \fi
    \fi
    \if@reversemargin \@tempcnta -\@tempcnta \fi
  \fi
  \ifnum\@tempcnta <\z@  \global\setbox\@marbox\box\@currbox \fi
  \@tempdima\@mparbottom
  \advance\@tempdima -\@pageht
  \advance\@tempdima\ht\@marbox
  \ifdim\@tempdima >\z@
    \@latex@warning@no@line{Marginpar on page \thepage\space moved}%
  \else
    \@tempdima\z@
  \fi
  \global\@mparbottom\@pageht
  \global\advance\@mparbottom\@tempdima
  \global\advance\@mparbottom\dp\@marbox
  \global\advance\@mparbottom\marginparpush
  \advance\@tempdima -\ht\@marbox
  \global\setbox\@marbox\vbox{%
    \vskip \@tempdima
    \box \@marbox
  }%
  \global \ht\@marbox \z@
  \global \dp\@marbox \z@
  \kern -\@pagedp
  \nointerlineskip
  \hb@xt@\columnwidth{%
    \ifnum \@tempcnta >\z@
      \hskip\columnwidth
      \hskip\marginparsep
    \else
      \hskip -\marginparsep
      \hskip -\marginparwidth
    \fi
    \box\@marbox \hss
  }%
  \nointerlineskip
  \hbox{\vrule \@height\z@ \@width\z@ \@depth\@pagedp}%
}
\ifx\pec@addmarginpar\@addmarginpar
  \pec@temp{latex/base}%
\fi
%    \end{macrocode}
%
% \subsubsection{memoir.cls}
%
%    \begin{macrocode}
\def\pec@addmarginpar{%
  \checkoddpage
  \@next\@marbox\@currlist{%
    \@cons\@freelist\@marbox
    \@cons\@freelist\@currbox
  }\@latexbug
  \@tempcnta\@ne
  \if@twocolumn
    \if@firstcolumn
      \@tempcnta\m@ne
    \fi
  \else
    \if@mparswitch
      \ifoddpage
      \else
        \@tempcnta\m@ne
      \fi
    \fi
    \if@reversemargin
      \@tempcnta -\@tempcnta
    \fi
  \fi
  \ifnum\@tempcnta <\z@
    \global\setbox\@marbox\box\@currbox
  \fi
  \@tempdima\@mparbottom
  \advance\@tempdima -\@pageht
  \advance\@tempdima\ht\@marbox
  \ifdim\@tempdima >\z@
    \@latex@warning@no@line{%
      Marginpar on page \thepage\space moved by \the\@tempdima
    }%
  \else
    \@tempdima\z@
  \fi
  \global\@mparbottom\@pageht
  \global\advance\@mparbottom\@tempdima
  \global\advance\@mparbottom\dp\@marbox
  \global\advance\@mparbottom\marginparpush
  \advance\@tempdima -\ht\@marbox
  \global\setbox\@marbox\vbox{%
    \vskip \@tempdima
    \box \@marbox
  }%
  \global \ht\@marbox \z@
  \global \dp\@marbox \z@
  \kern -\@pagedp
  \nointerlineskip
  \hb@xt@\columnwidth{%
    \ifnum \@tempcnta >\z@
      \hskip\columnwidth
      \hskip\marginparsep
    \else
      \hskip -\marginparsep
      \hskip -\marginparwidth
    \fi
    \box\@marbox
    \hss
  }%
  \nointerlineskip
  \hbox{\vrule \@height\z@ \@width\z@ \@depth\@pagedp}%
}%
\ifx\pec@addmarginpar\@addmarginpar
  \pec@temp{memoir.cls}%
\fi
%    \end{macrocode}
%
% \subsubsection{poemscol/marn.sty, poemscol/newmarn.sty}
%
%    \begin{macrocode}
\def\pec@addmarginpar{%
  \@next \@marbox\@currlist{%
    \@cons\@freelist\@marbox
    \@cons\@freelist\@currbox
  }\@latexbug
  \global\advance\@mpar@count\m@ne
  \@ifundefined{@marn@\the\@mpar@count @}{% was location logged last time?
    \@tempcnta\@ne % NO: use original LaTeX logic
    \if@twocolumn
      \if@firstcolumn
        \@tempcnta\m@ne
      \fi
    \else
      \if@mparswitch
        \ifodd\c@page
        \else
          \@tempcnta\m@ne
        \fi
      \fi
      \if@reversemargin
        \@tempcnta -\@tempcnta
      \fi
    \fi
  }{%
    \@tempcnta %    YES: use record from last time to decide side.
    \@nameuse{@marn@\the\@mpar@count @}%
    \if@reversemargin -\fi \@ne
  }%
  \ifnum\@tempcnta <\z@
    \global\setbox\@marbox\box\@currbox
    \global\let\@marnbottom\@mparbottoml
  \else
    \global\let\@marnbottom\@mparbottom
  \fi
  \@tempdima\@marnbottom \advance\@tempdima -\@pageht
  \advance\@tempdima\ht\@marbox
  \ifdim\@tempdima >\z@
    \@@warning{Marginpar on page \thepage\space moved}%
  \else
    \@tempdima\z@
  \fi
  \global\@marnbottom\@pageht
  \global\advance\@marnbottom\@tempdima
  \global\advance\@marnbottom\dp\@marbox
  \global\advance\@marnbottom\marginparpush
  \advance\@tempdima -\ht\@marbox
  \global\ht\@marbox\z@
  \global\dp\@marbox\z@
  \vskip -\@pagedp
  \vskip\@tempdima\nointerlineskip
  \hbox to\columnwidth{%
    \ifnum \@tempcnta >\z@
      \hskip\columnwidth
      \hskip\marginparsep
    \else
      \hskip -\marginparsep
      \hskip -\marginparwidth
    \fi
    \if@filesw % record where this is for use next time:
       \@marn@log\@mpar@count
    \fi
    \box\@marbox
    \hss
  }%
  \nobreak   %% RmS 91/06/21 \nobreak added
  \vskip -\@tempdima
  \nointerlineskip
  \hbox{\vrule \@height\z@ \@width\z@ \@depth\@pagedp}%
}
\ifx\pec@addmarginpar\@addmarginpar
  \pec@temp{poemscol/(new)marn.sty}%
\fi
%    \end{macrocode}
%
% \subsubsection{refman/refart.cls, refnam/refrep.cls}
%
%    \begin{macrocode}
\def\pec@addmarginpar{%
  \@next\@marbox\@currlist{%
    \@cons\@freelist\@marbox
    \@cons\@freelist\@currbox
  }\@latexbug
  \@tempcnta\@ne
  \if@twocolumn
    \if@firstcolumn
      \@tempcnta\m@ne
    \fi
  \else
    \@tempcnta\m@ne
  \fi
  \ifnum\@tempcnta <\z@
    \global\setbox\@marbox\box\@currbox
  \fi
  \@tempdima\@mparbottom
  \advance\@tempdima -\@pageht
  \advance\@tempdima\ht\@marbox
  \ifdim\@tempdima >\z@
     \@@warning{Marginpar on page \thepage\space moved}%
  \else
     \@tempdima\z@
  \fi
  \global\@mparbottom\@pageht
  \global\advance\@mparbottom\@tempdima
  \global\advance\@mparbottom\dp\@marbox
  \global\advance\@mparbottom\marginparpush
  \advance\@tempdima -\ht\@marbox
  \global\setbox\@marbox\vbox{%
    \vskip \@tempdima \box \@marbox
  }%
  \global \ht\@marbox \z@
  \global \dp\@marbox \z@
  \kern -\@pagedp
  \nointerlineskip
  \hb@xt@\columnwidth{%
    \ifnum \@tempcnta >\z@
      \hskip\columnwidth
      \hskip\marginparsep
    \else
      \hskip -\marginparsep
      \hskip -\marginparwidth
    \fi
    \box\@marbox
    \hss
  }%
  \nointerlineskip
  \hbox{\vrule \@height\z@ \@width\z@ \@depth\@pagedp}%
}
\ifx\pec@addmarginpar\@addmarginpar
  \pec@temp{ref(art|rep).cls}%
\fi

\ifcase\pec@result
  \PackageInfo{pdfcolmk}{%
    Fix for \string\@addmarginpar\space is omitted, %
    because this variant\MessageBreak
    of \string\@addmarginpar\space
      is not recognized%
  }%
\else
  % apply patch for \@addmarginpar
  \def\pec@PatchAddMarginpar#1\columnwidth#2#3\@nil{%
    \pec@PatchAddMarginparI#2\@nil{#1}{#3}%
  }%
  \def\pec@PatchAddMarginparI#1\box\@marbox\hss#2\@nil#3#4{%
    \def\@addmarginpar{%
      #3%
      \columnwidth{%
        #1%
        \pdfliteral{q}%
        \rlap{%
          \box\@marbox
        }%
        \pdfliteral{Q}%
        \hss
        #2%
      }%
      #4%
    }%
  }%
  \expandafter\pec@PatchAddMarginpar\@addmarginpar\@nil
\fi
%    \end{macrocode}
%
% \subsection{Color fix}
%
%    \begin{macrocode}
\def\set@color{%
  \pdfliteral{\current@color}%
  \ifinner
  \else
    \pec@setmark
  \fi
  \aftergroup\reset@color
}
\def\reset@color{%
  \pdfliteral{\current@color}%
  \ifinner
  \else
    \pec@setmark
  \fi
}

\let\pec@botcolor\current@color

\def\pec@PatchVBoxCCLV{%
  \ifx\pec@botcolor\@empty
  \else
    \setbox\@cclv\vbox{%
      \pdfliteral{\pec@botcolor}%
      \unvbox\@cclv
    }%
  \fi
  \pec@getmark
}

\def\pec@PatchAlreadyInBox{%
  \ifx\pec@botcolor\@empty
  \else
    \pdfliteral{\pec@botcolor}%
  \fi
  \pec@getmark
}

\@ifclassloaded{memoir}{%
  \expandafter\def\expandafter\mem@makecol\expandafter{%
    \expandafter\pec@PatchVBoxCCLV
    \mem@makecol
  }%
  \endinput
}{}

\@ifclassloaded{seminar}{%
  \newcommand\pec@org@makeslide{}%
  \let\pec@org@makeslide\@makeslide
  \def\@makeslide{%
    \pec@PatchVBoxCCLV
    \pec@org@makeslide
  }%
  \endinput
}{}

\long\def\pec@output#1\@specialoutput\else#2\pec@end{%
  \begingroup
    \def\x{#2}%
  \expandafter\endgroup
  \ifx\x\@empty
    \PackageWarningNoLine{pdfcolmk}{%
      Unexpected \string\output\space routine detected,%
      \MessageBreak
      loading of package stopped%
    }%
    \expandafter\endinput
  \fi
}
\expandafter\expandafter\expandafter\pec@output
\expandafter\@firstofone\the\output\@specialoutput\else\pec@end

\long\def\pec@output#1\@specialoutput\else#2\pec@end{%
  \output{%
    #1\@specialoutput\else
    \pec@PatchVBoxCCLV
    #2%
  }%
}
\expandafter\expandafter\expandafter\pec@output
\expandafter\@firstofone\the\output\pec@end
%    \end{macrocode}
%
%    \begin{macrocode}
%</package>
%    \end{macrocode}
%
% \section{Installation}
%
% \subsection{Download}
%
% \paragraph{Package.} This package is available on
% CTAN\footnote{\url{ftp://ftp.ctan.org/tex-archive/}}:
% \begin{description}
% \item[\CTAN{macros/latex/contrib/oberdiek/pdfcolmk.dtx}] The source file.
% \item[\CTAN{macros/latex/contrib/oberdiek/pdfcolmk.pdf}] Documentation.
% \end{description}
%
%
% \paragraph{Bundle.} All the packages of the bundle `oberdiek'
% are also available in a TDS compliant ZIP archive. There
% the packages are already unpacked and the documentation files
% are generated. The files and directories obey the TDS standard.
% \begin{description}
% \item[\CTAN{install/macros/latex/contrib/oberdiek.tds.zip}]
% \end{description}
% \emph{TDS} refers to the standard ``A Directory Structure
% for \TeX\ Files'' (\CTAN{tds/tds.pdf}). Directories
% with \xfile{texmf} in their name are usually organized this way.
%
% \subsection{Bundle installation}
%
% \paragraph{Unpacking.} Unpack the \xfile{oberdiek.tds.zip} in the
% TDS tree (also known as \xfile{texmf} tree) of your choice.
% Example (linux):
% \begin{quote}
%   |unzip oberdiek.tds.zip -d ~/texmf|
% \end{quote}
%
% \paragraph{Script installation.}
% Check the directory \xfile{TDS:scripts/oberdiek/} for
% scripts that need further installation steps.
% Package \xpackage{attachfile2} comes with the Perl script
% \xfile{pdfatfi.pl} that should be installed in such a way
% that it can be called as \texttt{pdfatfi}.
% Example (linux):
% \begin{quote}
%   |chmod +x scripts/oberdiek/pdfatfi.pl|\\
%   |cp scripts/oberdiek/pdfatfi.pl /usr/local/bin/|
% \end{quote}
%
% \subsection{Package installation}
%
% \paragraph{Unpacking.} The \xfile{.dtx} file is a self-extracting
% \docstrip\ archive. The files are extracted by running the
% \xfile{.dtx} through \plainTeX:
% \begin{quote}
%   \verb|tex pdfcolmk.dtx|
% \end{quote}
%
% \paragraph{TDS.} Now the different files must be moved into
% the different directories in your installation TDS tree
% (also known as \xfile{texmf} tree):
% \begin{quote}
% \def\t{^^A
% \begin{tabular}{@{}>{\ttfamily}l@{ $\rightarrow$ }>{\ttfamily}l@{}}
%   pdfcolmk.sty & tex/latex/oberdiek/pdfcolmk.sty\\
%   pdfcolmk.pdf & doc/latex/oberdiek/pdfcolmk.pdf\\
%   pdfcolmk.dtx & source/latex/oberdiek/pdfcolmk.dtx\\
% \end{tabular}^^A
% }^^A
% \sbox0{\t}^^A
% \ifdim\wd0>\linewidth
%   \begingroup
%     \advance\linewidth by\leftmargin
%     \advance\linewidth by\rightmargin
%   \edef\x{\endgroup
%     \def\noexpand\lw{\the\linewidth}^^A
%   }\x
%   \def\lwbox{^^A
%     \leavevmode
%     \hbox to \linewidth{^^A
%       \kern-\leftmargin\relax
%       \hss
%       \usebox0
%       \hss
%       \kern-\rightmargin\relax
%     }^^A
%   }^^A
%   \ifdim\wd0>\lw
%     \sbox0{\small\t}^^A
%     \ifdim\wd0>\linewidth
%       \ifdim\wd0>\lw
%         \sbox0{\footnotesize\t}^^A
%         \ifdim\wd0>\linewidth
%           \ifdim\wd0>\lw
%             \sbox0{\scriptsize\t}^^A
%             \ifdim\wd0>\linewidth
%               \ifdim\wd0>\lw
%                 \sbox0{\tiny\t}^^A
%                 \ifdim\wd0>\linewidth
%                   \lwbox
%                 \else
%                   \usebox0
%                 \fi
%               \else
%                 \lwbox
%               \fi
%             \else
%               \usebox0
%             \fi
%           \else
%             \lwbox
%           \fi
%         \else
%           \usebox0
%         \fi
%       \else
%         \lwbox
%       \fi
%     \else
%       \usebox0
%     \fi
%   \else
%     \lwbox
%   \fi
% \else
%   \usebox0
% \fi
% \end{quote}
% If you have a \xfile{docstrip.cfg} that configures and enables \docstrip's
% TDS installing feature, then some files can already be in the right
% place, see the documentation of \docstrip.
%
% \subsection{Refresh file name databases}
%
% If your \TeX~distribution
% (\teTeX, \mikTeX, \dots) relies on file name databases, you must refresh
% these. For example, \teTeX\ users run \verb|texhash| or
% \verb|mktexlsr|.
%
% \subsection{Some details for the interested}
%
% \paragraph{Attached source.}
%
% The PDF documentation on CTAN also includes the
% \xfile{.dtx} source file. It can be extracted by
% AcrobatReader 6 or higher. Another option is \textsf{pdftk},
% e.g. unpack the file into the current directory:
% \begin{quote}
%   \verb|pdftk pdfcolmk.pdf unpack_files output .|
% \end{quote}
%
% \paragraph{Unpacking with \LaTeX.}
% The \xfile{.dtx} chooses its action depending on the format:
% \begin{description}
% \item[\plainTeX:] Run \docstrip\ and extract the files.
% \item[\LaTeX:] Generate the documentation.
% \end{description}
% If you insist on using \LaTeX\ for \docstrip\ (really,
% \docstrip\ does not need \LaTeX), then inform the autodetect routine
% about your intention:
% \begin{quote}
%   \verb|latex \let\install=y\input{pdfcolmk.dtx}|
% \end{quote}
% Do not forget to quote the argument according to the demands
% of your shell.
%
% \paragraph{Generating the documentation.}
% You can use both the \xfile{.dtx} or the \xfile{.drv} to generate
% the documentation. The process can be configured by the
% configuration file \xfile{ltxdoc.cfg}. For instance, put this
% line into this file, if you want to have A4 as paper format:
% \begin{quote}
%   \verb|\PassOptionsToClass{a4paper}{article}|
% \end{quote}
% An example follows how to generate the
% documentation with pdf\LaTeX:
% \begin{quote}
%\begin{verbatim}
%pdflatex pdfcolmk.dtx
%makeindex -s gind.ist pdfcolmk.idx
%pdflatex pdfcolmk.dtx
%makeindex -s gind.ist pdfcolmk.idx
%pdflatex pdfcolmk.dtx
%\end{verbatim}
% \end{quote}
%
% \section{Catalogue}
%
% The following XML file can be used as source for the
% \href{http://mirror.ctan.org/help/Catalogue/catalogue.html}{\TeX\ Catalogue}.
% The elements \texttt{caption} and \texttt{description} are imported
% from the original XML file from the Catalogue.
% The name of the XML file in the Catalogue is \xfile{pdfcolmk.xml}.
%    \begin{macrocode}
%<*catalogue>
<?xml version='1.0' encoding='us-ascii'?>
<!DOCTYPE entry SYSTEM 'catalogue.dtd'>
<entry datestamp='$Date$' modifier='$Author$' id='pdfcolmk'>
  <name>pdfcolmk</name>
  <caption>Improving colour support under pdftex.</caption>
  <authorref id='auth:oberdiek'/>
  <copyright owner='Heiko Oberdiek' year='2000,2005-2008'/>
  <license type='lppl1.3'/>
  <version number='1.2'/>
  <description>
    The package provides macros that emulate the &#x2018;colour stack&#x2019;
    functionality of dvips.  The colour stack deals with colour
    manipulations when asynchronous events (like page-breaking) occur;
    pdftex does not (yet) have such a stack, but dvips does, and the
    <xref refid='color'>color</xref> package makes extensive use of
    it.
    <p/>
    This package is an experimental solution to the problem, and works
    best with pdf-e-tex.
    <p/>
    The package is part of the <xref refid='oberdiek'>oberdiek</xref> bundle.
  </description>
  <documentation details='Package documentation'
      href='ctan:/macros/latex/contrib/oberdiek/pdfcolmk.pdf'/>
  <ctan file='true' path='/macros/latex/contrib/oberdiek/pdfcolmk.dtx'/>
  <miktex location='oberdiek'/>
  <texlive location='oberdiek'/>
  <install path='/macros/latex/contrib/oberdiek/oberdiek.tds.zip'/>
</entry>
%</catalogue>
%    \end{macrocode}
%
% \begin{History}
%   \begin{Version}{2000/08/27 v0.1}
%   \item
%     First published version in newsgroup \xnewsgroup{comp.text.tex}:\\
%     \URL{``\link{pdftex: bug with colors?}''}^^A
%     {http://groups.google.com/group/comp.text.tex/msg/6f088e69e4085d2c}
%   \end{Version}
%   \begin{Version}{2000/09/02 v0.2}
%   \item
%     Next try.
%   \end{Version}
%   \begin{Version}{2000/09/02 v0.3}
%   \item
%     Solution without \eTeX\ added.
%   \end{Version}
%   \begin{Version}{2000/09/06 v0.4}
%   \item
%     Patch commands added.
%   \item
%     Patch for seminar.cls added.
%   \end{Version}
%   \begin{Version}{2000/09/06 v0.5}
%   \item
%     Bug fix: initialization of \cs{pec@value} added.
%   \end{Version}
%   \begin{Version}{2005/06/15 v0.6}
%   \item
%     Support for \cs{marginpar} added.
%     See thread in \xnewsgroup{comp.text.tex}:\\
%     \URL{``\link{Using \cs{textcolor} and \cs{marginpar} together}''}^^A
%     {http://groups.google.com/group/comp.text.tex/msg/38ed58f8845a2a4f}
%   \end{Version}
%   \begin{Version}{2005/07/09 v0.7}
%   \item
%     Output support added for \xpackage{memoir},
%     provided by Lars Madsen.
%   \end{Version}
%   \begin{Version}{2006/02/20 v0.8}
%   \item
%     Code is not changed.
%   \item
%     DTX framework.
%   \end{Version}
%   \begin{Version}{2007/01/01 v1.0}
%   \item
%     If \xfile{pdftex.def} \textgreater= 2007/01/01 v0.04a is used with
%     \pdfTeX\ \textgreater= 1.40.0, then package \xpackage{pdfcolmk} is obsolete.
%   \end{Version}
%   \begin{Version}{2007/04/11 v1.1}
%   \item
%     Line ends sanitized.
%   \end{Version}
%   \begin{Version}{2008/08/11 v1.2}
%   \item
%     Code is not changed.
%   \item
%     URLs updated.
%   \end{Version}
% \end{History}
%
% \PrintIndex
%
% \Finale
\endinput
|
% \end{quote}
% Do not forget to quote the argument according to the demands
% of your shell.
%
% \paragraph{Generating the documentation.}
% You can use both the \xfile{.dtx} or the \xfile{.drv} to generate
% the documentation. The process can be configured by the
% configuration file \xfile{ltxdoc.cfg}. For instance, put this
% line into this file, if you want to have A4 as paper format:
% \begin{quote}
%   \verb|\PassOptionsToClass{a4paper}{article}|
% \end{quote}
% An example follows how to generate the
% documentation with pdf\LaTeX:
% \begin{quote}
%\begin{verbatim}
%pdflatex pdfcolmk.dtx
%makeindex -s gind.ist pdfcolmk.idx
%pdflatex pdfcolmk.dtx
%makeindex -s gind.ist pdfcolmk.idx
%pdflatex pdfcolmk.dtx
%\end{verbatim}
% \end{quote}
%
% \section{Catalogue}
%
% The following XML file can be used as source for the
% \href{http://mirror.ctan.org/help/Catalogue/catalogue.html}{\TeX\ Catalogue}.
% The elements \texttt{caption} and \texttt{description} are imported
% from the original XML file from the Catalogue.
% The name of the XML file in the Catalogue is \xfile{pdfcolmk.xml}.
%    \begin{macrocode}
%<*catalogue>
<?xml version='1.0' encoding='us-ascii'?>
<!DOCTYPE entry SYSTEM 'catalogue.dtd'>
<entry datestamp='$Date$' modifier='$Author$' id='pdfcolmk'>
  <name>pdfcolmk</name>
  <caption>Improving colour support under pdftex.</caption>
  <authorref id='auth:oberdiek'/>
  <copyright owner='Heiko Oberdiek' year='2000,2005-2008'/>
  <license type='lppl1.3'/>
  <version number='1.2'/>
  <description>
    The package provides macros that emulate the &#x2018;colour stack&#x2019;
    functionality of dvips.  The colour stack deals with colour
    manipulations when asynchronous events (like page-breaking) occur;
    pdftex does not (yet) have such a stack, but dvips does, and the
    <xref refid='color'>color</xref> package makes extensive use of
    it.
    <p/>
    This package is an experimental solution to the problem, and works
    best with pdf-e-tex.
    <p/>
    The package is part of the <xref refid='oberdiek'>oberdiek</xref> bundle.
  </description>
  <documentation details='Package documentation'
      href='ctan:/macros/latex/contrib/oberdiek/pdfcolmk.pdf'/>
  <ctan file='true' path='/macros/latex/contrib/oberdiek/pdfcolmk.dtx'/>
  <miktex location='oberdiek'/>
  <texlive location='oberdiek'/>
  <install path='/macros/latex/contrib/oberdiek/oberdiek.tds.zip'/>
</entry>
%</catalogue>
%    \end{macrocode}
%
% \begin{History}
%   \begin{Version}{2000/08/27 v0.1}
%   \item
%     First published version in newsgroup \xnewsgroup{comp.text.tex}:\\
%     \URL{``\link{pdftex: bug with colors?}''}^^A
%     {http://groups.google.com/group/comp.text.tex/msg/6f088e69e4085d2c}
%   \end{Version}
%   \begin{Version}{2000/09/02 v0.2}
%   \item
%     Next try.
%   \end{Version}
%   \begin{Version}{2000/09/02 v0.3}
%   \item
%     Solution without \eTeX\ added.
%   \end{Version}
%   \begin{Version}{2000/09/06 v0.4}
%   \item
%     Patch commands added.
%   \item
%     Patch for seminar.cls added.
%   \end{Version}
%   \begin{Version}{2000/09/06 v0.5}
%   \item
%     Bug fix: initialization of \cs{pec@value} added.
%   \end{Version}
%   \begin{Version}{2005/06/15 v0.6}
%   \item
%     Support for \cs{marginpar} added.
%     See thread in \xnewsgroup{comp.text.tex}:\\
%     \URL{``\link{Using \cs{textcolor} and \cs{marginpar} together}''}^^A
%     {http://groups.google.com/group/comp.text.tex/msg/38ed58f8845a2a4f}
%   \end{Version}
%   \begin{Version}{2005/07/09 v0.7}
%   \item
%     Output support added for \xpackage{memoir},
%     provided by Lars Madsen.
%   \end{Version}
%   \begin{Version}{2006/02/20 v0.8}
%   \item
%     Code is not changed.
%   \item
%     DTX framework.
%   \end{Version}
%   \begin{Version}{2007/01/01 v1.0}
%   \item
%     If \xfile{pdftex.def} \textgreater= 2007/01/01 v0.04a is used with
%     \pdfTeX\ \textgreater= 1.40.0, then package \xpackage{pdfcolmk} is obsolete.
%   \end{Version}
%   \begin{Version}{2007/04/11 v1.1}
%   \item
%     Line ends sanitized.
%   \end{Version}
%   \begin{Version}{2008/08/11 v1.2}
%   \item
%     Code is not changed.
%   \item
%     URLs updated.
%   \end{Version}
% \end{History}
%
% \PrintIndex
%
% \Finale
\endinput
|
% \end{quote}
% Do not forget to quote the argument according to the demands
% of your shell.
%
% \paragraph{Generating the documentation.}
% You can use both the \xfile{.dtx} or the \xfile{.drv} to generate
% the documentation. The process can be configured by the
% configuration file \xfile{ltxdoc.cfg}. For instance, put this
% line into this file, if you want to have A4 as paper format:
% \begin{quote}
%   \verb|\PassOptionsToClass{a4paper}{article}|
% \end{quote}
% An example follows how to generate the
% documentation with pdf\LaTeX:
% \begin{quote}
%\begin{verbatim}
%pdflatex pdfcolmk.dtx
%makeindex -s gind.ist pdfcolmk.idx
%pdflatex pdfcolmk.dtx
%makeindex -s gind.ist pdfcolmk.idx
%pdflatex pdfcolmk.dtx
%\end{verbatim}
% \end{quote}
%
% \section{Catalogue}
%
% The following XML file can be used as source for the
% \href{http://mirror.ctan.org/help/Catalogue/catalogue.html}{\TeX\ Catalogue}.
% The elements \texttt{caption} and \texttt{description} are imported
% from the original XML file from the Catalogue.
% The name of the XML file in the Catalogue is \xfile{pdfcolmk.xml}.
%    \begin{macrocode}
%<*catalogue>
<?xml version='1.0' encoding='us-ascii'?>
<!DOCTYPE entry SYSTEM 'catalogue.dtd'>
<entry datestamp='$Date$' modifier='$Author$' id='pdfcolmk'>
  <name>pdfcolmk</name>
  <caption>Improving colour support under pdftex.</caption>
  <authorref id='auth:oberdiek'/>
  <copyright owner='Heiko Oberdiek' year='2000,2005-2008'/>
  <license type='lppl1.3'/>
  <version number='1.2'/>
  <description>
    The package provides macros that emulate the &#x2018;colour stack&#x2019;
    functionality of dvips.  The colour stack deals with colour
    manipulations when asynchronous events (like page-breaking) occur;
    pdftex does not (yet) have such a stack, but dvips does, and the
    <xref refid='color'>color</xref> package makes extensive use of
    it.
    <p/>
    This package is an experimental solution to the problem, and works
    best with pdf-e-tex.
    <p/>
    The package is part of the <xref refid='oberdiek'>oberdiek</xref> bundle.
  </description>
  <documentation details='Package documentation'
      href='ctan:/macros/latex/contrib/oberdiek/pdfcolmk.pdf'/>
  <ctan file='true' path='/macros/latex/contrib/oberdiek/pdfcolmk.dtx'/>
  <miktex location='oberdiek'/>
  <texlive location='oberdiek'/>
  <install path='/macros/latex/contrib/oberdiek/oberdiek.tds.zip'/>
</entry>
%</catalogue>
%    \end{macrocode}
%
% \begin{History}
%   \begin{Version}{2000/08/27 v0.1}
%   \item
%     First published version in newsgroup \xnewsgroup{comp.text.tex}:\\
%     \URL{``\link{pdftex: bug with colors?}''}^^A
%     {http://groups.google.com/group/comp.text.tex/msg/6f088e69e4085d2c}
%   \end{Version}
%   \begin{Version}{2000/09/02 v0.2}
%   \item
%     Next try.
%   \end{Version}
%   \begin{Version}{2000/09/02 v0.3}
%   \item
%     Solution without \eTeX\ added.
%   \end{Version}
%   \begin{Version}{2000/09/06 v0.4}
%   \item
%     Patch commands added.
%   \item
%     Patch for seminar.cls added.
%   \end{Version}
%   \begin{Version}{2000/09/06 v0.5}
%   \item
%     Bug fix: initialization of \cs{pec@value} added.
%   \end{Version}
%   \begin{Version}{2005/06/15 v0.6}
%   \item
%     Support for \cs{marginpar} added.
%     See thread in \xnewsgroup{comp.text.tex}:\\
%     \URL{``\link{Using \cs{textcolor} and \cs{marginpar} together}''}^^A
%     {http://groups.google.com/group/comp.text.tex/msg/38ed58f8845a2a4f}
%   \end{Version}
%   \begin{Version}{2005/07/09 v0.7}
%   \item
%     Output support added for \xpackage{memoir},
%     provided by Lars Madsen.
%   \end{Version}
%   \begin{Version}{2006/02/20 v0.8}
%   \item
%     Code is not changed.
%   \item
%     DTX framework.
%   \end{Version}
%   \begin{Version}{2007/01/01 v1.0}
%   \item
%     If \xfile{pdftex.def} \textgreater= 2007/01/01 v0.04a is used with
%     \pdfTeX\ \textgreater= 1.40.0, then package \xpackage{pdfcolmk} is obsolete.
%   \end{Version}
%   \begin{Version}{2007/04/11 v1.1}
%   \item
%     Line ends sanitized.
%   \end{Version}
%   \begin{Version}{2008/08/11 v1.2}
%   \item
%     Code is not changed.
%   \item
%     URLs updated.
%   \end{Version}
% \end{History}
%
% \PrintIndex
%
% \Finale
\endinput
|
% \end{quote}
% Do not forget to quote the argument according to the demands
% of your shell.
%
% \paragraph{Generating the documentation.}
% You can use both the \xfile{.dtx} or the \xfile{.drv} to generate
% the documentation. The process can be configured by the
% configuration file \xfile{ltxdoc.cfg}. For instance, put this
% line into this file, if you want to have A4 as paper format:
% \begin{quote}
%   \verb|\PassOptionsToClass{a4paper}{article}|
% \end{quote}
% An example follows how to generate the
% documentation with pdf\LaTeX:
% \begin{quote}
%\begin{verbatim}
%pdflatex pdfcolmk.dtx
%makeindex -s gind.ist pdfcolmk.idx
%pdflatex pdfcolmk.dtx
%makeindex -s gind.ist pdfcolmk.idx
%pdflatex pdfcolmk.dtx
%\end{verbatim}
% \end{quote}
%
% \section{Catalogue}
%
% The following XML file can be used as source for the
% \href{http://mirror.ctan.org/help/Catalogue/catalogue.html}{\TeX\ Catalogue}.
% The elements \texttt{caption} and \texttt{description} are imported
% from the original XML file from the Catalogue.
% The name of the XML file in the Catalogue is \xfile{pdfcolmk.xml}.
%    \begin{macrocode}
%<*catalogue>
<?xml version='1.0' encoding='us-ascii'?>
<!DOCTYPE entry SYSTEM 'catalogue.dtd'>
<entry datestamp='$Date$' modifier='$Author$' id='pdfcolmk'>
  <name>pdfcolmk</name>
  <caption>Improving colour support under pdftex.</caption>
  <authorref id='auth:oberdiek'/>
  <copyright owner='Heiko Oberdiek' year='2000,2005-2008'/>
  <license type='lppl1.3'/>
  <version number='1.2'/>
  <description>
    The package provides macros that emulate the &#x2018;colour stack&#x2019;
    functionality of dvips.  The colour stack deals with colour
    manipulations when asynchronous events (like page-breaking) occur;
    pdftex does not (yet) have such a stack, but dvips does, and the
    <xref refid='color'>color</xref> package makes extensive use of
    it.
    <p/>
    This package is an experimental solution to the problem, and works
    best with pdf-e-tex.
    <p/>
    The package is part of the <xref refid='oberdiek'>oberdiek</xref> bundle.
  </description>
  <documentation details='Package documentation'
      href='ctan:/macros/latex/contrib/oberdiek/pdfcolmk.pdf'/>
  <ctan file='true' path='/macros/latex/contrib/oberdiek/pdfcolmk.dtx'/>
  <miktex location='oberdiek'/>
  <texlive location='oberdiek'/>
  <install path='/macros/latex/contrib/oberdiek/oberdiek.tds.zip'/>
</entry>
%</catalogue>
%    \end{macrocode}
%
% \begin{History}
%   \begin{Version}{2000/08/27 v0.1}
%   \item
%     First published version in newsgroup \xnewsgroup{comp.text.tex}:\\
%     \URL{``\link{pdftex: bug with colors?}''}^^A
%     {http://groups.google.com/group/comp.text.tex/msg/6f088e69e4085d2c}
%   \end{Version}
%   \begin{Version}{2000/09/02 v0.2}
%   \item
%     Next try.
%   \end{Version}
%   \begin{Version}{2000/09/02 v0.3}
%   \item
%     Solution without \eTeX\ added.
%   \end{Version}
%   \begin{Version}{2000/09/06 v0.4}
%   \item
%     Patch commands added.
%   \item
%     Patch for seminar.cls added.
%   \end{Version}
%   \begin{Version}{2000/09/06 v0.5}
%   \item
%     Bug fix: initialization of \cs{pec@value} added.
%   \end{Version}
%   \begin{Version}{2005/06/15 v0.6}
%   \item
%     Support for \cs{marginpar} added.
%     See thread in \xnewsgroup{comp.text.tex}:\\
%     \URL{``\link{Using \cs{textcolor} and \cs{marginpar} together}''}^^A
%     {http://groups.google.com/group/comp.text.tex/msg/38ed58f8845a2a4f}
%   \end{Version}
%   \begin{Version}{2005/07/09 v0.7}
%   \item
%     Output support added for \xpackage{memoir},
%     provided by Lars Madsen.
%   \end{Version}
%   \begin{Version}{2006/02/20 v0.8}
%   \item
%     Code is not changed.
%   \item
%     DTX framework.
%   \end{Version}
%   \begin{Version}{2007/01/01 v1.0}
%   \item
%     If \xfile{pdftex.def} \textgreater= 2007/01/01 v0.04a is used with
%     \pdfTeX\ \textgreater= 1.40.0, then package \xpackage{pdfcolmk} is obsolete.
%   \end{Version}
%   \begin{Version}{2007/04/11 v1.1}
%   \item
%     Line ends sanitized.
%   \end{Version}
%   \begin{Version}{2008/08/11 v1.2}
%   \item
%     Code is not changed.
%   \item
%     URLs updated.
%   \end{Version}
% \end{History}
%
% \PrintIndex
%
% \Finale
\endinput
