% \iffalse
%% skdoc documentation class
%%
%% Copyright (C) 2012-2013 by Simon Sigurdhsson <sigurdhsson@gmail.com>
%% 
%% This work may be distributed and/or modified under the
%% conditions of the LaTeX Project Public License, either version 1.3
%% of this license or (at your option) any later version.
%% The latest version of this license is in
%%   http://www.latex-project.org/lppl.txt
%% and version 1.3 or later is part of all distributions of LaTeX
%% version 2005/12/01 or later.
%% 
%% This work has the LPPL maintenance status `maintained'.
%% 
%% The Current Maintainer of this work is Simon Sigurdhsson.
%% 
%% This work consists of the files skdoc.dtx
%% and the derived filebase skdoc.cls.
%
%<*ignore>
\begingroup
  \catcode123=1 %
  \catcode125=2 %
  \def\x{LaTeX2e}%
\expandafter\endgroup
\ifcase 0\ifx\install y1\fi\expandafter
        \ifx\csname processbatchFile\endcsname\relax\else1\fi
        \ifx\fmtname\x\else 1\fi\relax
\else\csname fi\endcsname
%</ignore>
%<*install>
\input docstrip.tex

\preamble
\endpreamble

\keepsilent
\askforoverwritefalse

\generate{%
    \file{skdoc.cls}{\from{skdoc.dtx}{class}}%
}

\begingroup
\obeyspaces
\Msg{*************************************************************}%
\Msg{*                                                           *}%
\Msg{* To finish the installation you have to move the following *}%
\Msg{* file into a directory searched by TeX:                    *}%
\Msg{*                                                           *}%
\Msg{*     skdoc.cls                                             *}%
\Msg{*                                                           *}%
\Msg{* To produce the documentation run the file skdoc.dtx       *}%
\Msg{* through LaTeX.                                            *}%
\Msg{*                                                           *}%
\Msg{* Happy TeXing!                                             *}%
\Msg{*                                                           *}%
\Msg{*************************************************************}%
\endgroup%

\endbatchfile
%</install>
%<*ignore>
\fi
%</ignore>
%<*class>
\RequirePackage{expl3}
%</class>
%<*driver>
\RequirePackage{xparse}
\ProvidesExplFile{skdoc.dtx}
%</driver>
%<class>\ProvidesExplClass{skdoc}
%<*class>
    {2013/11/28}{1.4a}{skdoc documentation class}
%</class>
%
%<*driver>
\msg_new:nnn{skdoc-dtx}{not-installed}
    {Run~`tex'~on~skdoc.dtx~before~generating~the~documentation!}
\IfFileExists{skdoc.cls}{}{
    \msg_fatal:nn{skdoc-dtx}{not-installed}
}
\cs_set_protected_nopar:Npn\ExplHack{
    \char_set_catcode_letter:n{ 58 }
    \char_set_catcode_letter:n{ 95 }
}
\ExplSyntaxOff
\DeclareDocumentCommand\MakePercentIgnore{}{\catcode`\%9\relax}
\DeclareDocumentCommand\MakePercentComment{}{\catcode`\%14\relax}
\DeclareDocumentCommand\DocInput{m}{
    \MakePercentIgnore\input{#1}\MakePercentComment
}
\documentclass[highlight=false]{skdoc}
\usepackage{hologo}
\usepackage{booktabs}
\usepackage{csquotes}
\usepackage[style=authoryear,backend=biber]{biblatex}
%%\usepackage{chslacite}
\begin{filecontents}{skdoc.bib}
@online{Lazarides12,
    author = {Yannis Lazarides},
    title = {Different approach to literate programming for \LaTeX},
    year = {2012},
    url = {http://tex.stackexchange.com/q/47237/66}
}
@online{Talbot13,
    author = {Nicola Talbot},
    title = {Answer to \enquote{Distinct formatted page numbers with glossaries and Xindy}},
    year = {2013},
    url = {http://tex.stackexchange.com/a/89830/66}
}
@manual{Rahtz10,
    author = {Sebastian Rahtz and Herbert Vo\ss},
    title = {The \enquote{\textsf{fancyvrb}} package},
    subtitle = {Fancy Verbatims in \LaTeX},
    date = {2012-05-15},
    version = {2.8},
    url = {http://mirrors.ctan.org/macros/latex/contrib/fancyvrb/fancyvrb.pdf}
}
\end{filecontents}
\addbibresource{skdoc.bib}
\OnlyDescription
\begin{document}
    \DocInput{skdoc.dtx}
\end{document}
% \fi
%
% \deftripstyle{skdoc-class}{}{}{}%
% {\small The~\textbf{\thepkg}~document~class,~\theversion}%
% {}{\small\pagemark}
% \pagestyle{skdoc-class}
%
% \version{1.4a}
% \changes{1.0}{Initial version}
% \changes{1.1}{Added support for syntax highlighting using \pkg{minted}}
% \changes{1.1a}{Deprecate the use of \pkg{bibtex} in favour of \pkg{biblatex}}
% \changes{1.2}{Use \pkg{l3prg} booleans instead of toggles}
% \changes{1.2b}{Use \pkg{inconsolata}. Don't use \pkg{ascii}}
% \changes{1.3}{Allow multiple targets per \env{MacroCode}}
% \changes{1.3b}{Use \pkg{sourcecodepro} instead of \pkg{inconsolata}. Fix issue with index entries of different types with same name}
% \changes{1.4}{Added option to control \pkg{babel}. Allow optional default value arguments in \env{macro} and friends. Fix spacing issue in \env{option} and friends}
% \changes{1.4a}{Fix various compatibility issues with latest \pkg{glossaries}}
% \iffalse
%%% Don't forget to update the version number and release date of
%%% the package declaration on line 76!
% \fi
%
% \package[ctan=skdoc,vcs=https://github.com/urdh/skdoc]{skdoc}
% \repository{https://github.com/urdh/skdoc}
% \title{The \thepkg{} document class}
% \author{Simon Sigurdhsson}
% \email{sigurdhsson@gmail.com}
% \maketitle
%
% \begin{abstract}
% The \thepkg{} class provides macros to document the functionality
% and implementation of \LaTeX\ packages and document classes. It is
% loosely based on the \pkg{ydoc} and \pkg{ltxdoc} classes, but has
% a number of incompatible differences.
%
% The class defines a \env{MacroCode} environment which substitutes
% the usual \pkg{docstrip} method of installing packages. It has the
% ability to generate both documentation and code in a single run of
% a single file.
% \end{abstract}
%
% \tableofcontents
%
% \section{Introduction}
% This document class, inspired by a question on the \TeX\ Stack
% Exchange\footcite{Lazarides12}, aims to provide an alternative to
% the standard \pkg{docstrip} method of literate programming for
% \LaTeX\ packages. It also aims to provide a more modern, appealing
% style for \LaTeX\ package documentation.
%
% In order to achieve this, it builds upon already existing features
% of the \pkg{expl3}, \pkg{verbatim} and \pkg{ydoc} packages as well
% as the KOMA-script document classes.
%
% So far it is mainly intended to be an experiment to explore a less
% cumbersome way of writing \LaTeX\ packages, and as such I give no
% guarantee that this package will continue to exist in a working
% state (\emph{i.e.} future users may not be able to extract code from
% package documentation based on \thepkg) or that its public API
% (commands and environments described by this documentation; consider
% undocumented macros part of a private API) will be stable.
%
% The documentation of \thepkg\ is in fact typeset using the class
% itself. It does not, however, make use of the main feature of this
% class (the \env{MacroCode} environment), because bootstrapping the
% class to generate itself is more complicated than it is useful.
%
% \section{Documentation}
% Since \thepkg\ is based on \pkg{ydoc} many of the macros and
% environment present in that package are also available in \thepkg,
% in a possibly redefined incanation. However, any macros or
% environment present in \pkg{ydoc} but not described in this
% documentation should be considered part of the private API of
% \thepkg. In the future, the removal of the \pkg{ydoc} dependency
% may result in such macros being unavailable, and at present changes
% made by \thepkg\ may break such macros without notice.
%
% \subsection{Options}
% \Option{load}\WithValues{\meta{package}}\AndDefault{\cs{jobname}}
% \label{ssec:options:load} The \opt{load}
% option declares that if the package specified exists, it should
% be loaded. This is intended to load any package provided in the
% implementation, but requires that the documentation provides
% stub variants of the macros used in the documentation so that
% \LaTeX\ still completes its first run.
%
% \Option{highlight}\WithValues{true,false}\AndDefault{true}
% \label{ssec:option:highlight}
% The \opt{highlight} option enables or disabled syntax highlighting
% of the implementation code. Highlighting is performed using
% \pkg{minted}, and falls back to no highlighting if there is no
% \cs{write18} access, if \pkg{minted} is unavailable or if the
% \texttt{pygmentize} binary can't be found.
% \Notice{On non-unix platforms, the test for \texttt{pygmentize}
% will likely fail. Therefore, syntax highlighting is not supported
% on such platforms.}
%
% Generally, there should be no reason to disable syntax highlighting
% unless the documentation describes a very large package, and the
% repeated calls to \texttt{pygmentize} take too long.
%
% \Option{babel}
% The \opt{babel} option allows you to specify what languages are loaded
% by the \pkg{babel} package. It is a key-value option, and its content
% is passed as options to the \pkg{babel} \cs*{usepackage} declaration.
%
% \subsection{General macros}
% The document class defines a number of general macros intended for
% use in parts of the document not strictly considered
% \enquote{documentation} or \enquote{implementation}, in addition to
% being used in those parts. These \enquote{general} macros include
% macros that define metadata, generate the title page, typeset
% notices or warnings and those that refer to macros, environments,
% packages and such.
%
% \subsubsection{Metadata}
% Several macros for defining metadata (\emph{i.e.} information about
% the package and its documentation) are made available. These mostly
% set an internal variable which is used to typeset the title page,
% and to insert PDF metadata whenever \hologo{pdfLaTeX} is used to
% generate the documentation.
%
% \DescribeMacro\package['ctan='<identifier>',vcs='<url>]{<package name>}
% The \cs{package} macro defines the package name used by \cs{thepkg},
% \cs{maketitle} and similar macros. It also calls \cs{title} to set
% a sensible default title based on the package name. The optional
% key-value argument takes two keys: \texttt{ctan} and \texttt{vcs}.
% The first one accepts an optional value \meta{identifier}, which
% should be the identifier the package has on CTAN (the default is
% \cs{jobname}), while the other takes a mandatory argument \meta{url}
% specifying the URL of a VCS repository where development versions
% of the package are available. The two optional keys imply calls to
% the \cs{ctan} and \cs{repository} macros, respectively.
%
% \DescribeMacro\version{<version>}
% Sets the version number of the package the documentation describes.
% Here, \meta{version} should not include the initial \enquote{v},
% \emph{i.e.} the argument should be the same as that given to
% \emph{e.g.} the \LaTeX3 \cs{ProvidesExplPackage} or the standard
% \pkg{ltxdoc} \cs{changes}.
%
% \DescribeMacro\ctan{<identifier>}
% As detailed above, this macro defines the CTAN identifier of the
% package, which is (optionally) used in the \cs{maketitle} macro.
%
% \DescribeMacro\repository{<url>}
% Again, as detailed above this macro defines the URL of a source code
% repository containing a development version of the package, which
% is optionally used by \cs{maketitle}.
%
% \DescribeMacro\author{<name>}
% Defines the name of the package author. This is used by
% \cs{maketitle} and is mandatory if \cs{maketitle} is used.
%
% \DescribeMacro\email{<email>}
% Defines the email of the package author. This is used by
% \cs{maketitle} and is mandatory if \cs{maketitle} is used.
%
% \DescribeMacro\title{<title>}
% Defines the package title. By default, the \cs{package} macro sets
% a sensible title that should suit most packages, but using \cs{title}
% will override this title (useful for \emph{e.g.} document classes or
% \hologo{BibTeX} styles).
%
% Three macros retrieving the set metadata are also available. They
% can be used to typeset the current version of the package, and the
% package name, respectively.
%
% \DescribeMacro\theversion
% Returns the version as defined by \cs{version}, with a leading
% \enquote{v}. That is, issuing \Macro\version{1.0} makes 
% \cs{theversion} print \enquote{v1.0}.
%
% \DescribeMacro\thepackage
% \DescribeMacro\thepkg
% The \cs{thepackage} and \cs{thepkg} macros return the package name
% as defined by the \cs{package} macro, enclosed in \cs{pkg*}. That is,
% the package name is typeset as a package but not indexed.
%
% \subsubsection{The preamble}
% The preamble of any documentation most often consists of a title page
% containing an abstract, and possibly a table of contents. The \thepkg
% \ package provides macros and environments that typeset such things,
% and these should be fully compatible with most other document classes.
%
% \DescribeMacro\maketitle
% The \cs{maketitle} macro typesets a title page. This title page uses
% the metadata defined by the macros described earlier, and typesets
% them in a manner which is illustrated by the documentation of this
% class. This style is inspired by \pkg{skrapport}, which is in turn
% inspired by the title pages of the Prac\TeX\ Journal.
%
% \DescribeEnv[<package abstract>]{abstract}
% The \env{abstract} environment typesets an abstract of the package.
% Again, its style is illustrated by this document and it is inspired
% by the \pkg{skrapport} package as well as the Prac\TeX\ Journal.
%
% \DescribeMacro\tableofcontents
% Finally, a Table of Contents may be printed. The actual table of
% contents is provided by the \pkg{scrartcl} document class, but \thepkg
% \ redefines a couple of the internal macros to style the Table of
% Contents in a manner similar to that of the \pkg{microtype} manual.
%
% \subsubsection{The LPPL license}
% \DescribeMacro\PrintLPPL
% If the LPPL license is present in a directory where \LaTeX\ can find
% it, in a file called \file{lppl.tex}, then \cs{PrintLPPL} will
% include the entire LPPL license in the document, and typeset it in
% a fairly compact manner.
%
% \subsubsection{Notices and warnings}
% The document class provides macros to indicate information that may
% be of extra importance in the documentation. Such information is
% categorized as either notices or warnings, which are treated
% differently.
% 
% \DescribeMacro\Notice{<notice>}
% A notice is a short piece of text that contains information that may
% explain some unexpected but unharmful behaviour of a macro or similar.
% It is typeset inline, emphasized and in parantheses --- as such, the
% sequence \Macro\Notice{a notice} yields \Notice{a notice}.
%
% \DescribeMacro\Warning{<warning>}
% A warning is a short comment that conveys information that the user
% must be aware of. This includes unexpected potentially harmful
% behaviour, deprecation notices and so on. It is typeset in its own
% \cs{fbox} --- the sequence \Macro\Warning{a warning} yields the
% following: \Warning{a warning}
%
% \DescribeMacro\LongWarning{<warning>}
% The \cs{LongWarning} macro is a variant of \cs{Warning} that has been
% adapted for longer texts, possibly including paragraph breaks. Like
% \cs{Warning}, it is typeset in a box: \LongWarning{a long warning}
%
% \subsubsection{Referential macros}\label{sec:ref-macros}
% The family of macros originating from \cs{cs} are used to typeset
% various concepts in running text. In addition to adhering to the
% general format of the corresponding concept, they index their
% argument. Each of these macros have a starred variant which does
% not index its argument; use these when appropriate.
%
% \DescribeMacro\cs{<command sequence>}
% Typesets a command sequence, or macro. The argument should be
% provided without the leading backslash, and the command sequence
% will be typeset in a monospaced font.
%
% \DescribeMacro\env{<environment name>}
% Typesets an environment name, which will be typeset in a monospace
% font.
%
% \DescribeMacro\pkg{<package name>}
% Typesets a package, document class or bundle name. The name will
% be typeset in a sans-serif font.
%
% \DescribeMacro\opt{<option>}
% Typesets a package or document class option. As of \theversion,
% options are typeset using a monospace font.
%
% \DescribeMacro\bib{<\hologo{BibTeX} entry type>}
% Typesets a \hologo{BibTeX} entry type. The agument should be provided
% without the leading \texttt{@} sign. The entry type will be typeset
% in a monospace font.
% 
% \DescribeMacro\thm{<theme name>}
% Typesets a theme name. As of \theversion, the theme name will be
% typeset in an upright serif font.
%
% \DescribeMacro\file{<filename>}
% Typesets a filename. As of \theversion, the filename will be typeset
% in a monospace font.
%
% \subsection{Documenting the package}\label{sec:doc-macros}
% The documentation part of any \LaTeX\ manual is arguably the most
% important one, and to facilitate proper typesetting of the
% documentation \thepkg\ povides a number of different macros, all
% inspired by or inherited from \pkg{ydoc}. The first of these
% macros that will be discussed are the macros that typeset differen
% kinds of arguments in running text.
%
% \DescribeMacro\meta{<meta text>}
% The \cs{meta} macro typesets a placeholder to be placed in an
% argument. This can be used to refer to arguments and contents of
% macros and environments described by commands discussed later in
% this documentation. It is typeset in brackets: \meta{meta text}.
%
% \DescribeMacro\marg{<mandatory argument>}
% \DescribeMacro\oarg{<optional argument>}
% \DescribeMacro\parg{<picture-style argument>}
% \DescribeMacro\aarg{<beamer-style argument>}
% \DescribeMacro\sarg
% These macros typeset different kinds of arguments (mandatory,
% optional, picture-style, beamer-style and star arguments,
% respetively). These can be used to describe arguments, but
% are mostly used internally. See table~\ref{tab:args} for
% examples of how these macros are typeset.
%
% \begin{table}[tb]
%   \centering
%   \caption{Typesetting arguments}
%   \label{tab:args}
%   \begin{tabular}{ll}
%      \toprule
%      Invokation            & Result          \\
%      \midrule
%      \Macro\marg{argument} & \marg{argument} \\
%      \Macro\oarg{argument} & \oarg{argument} \\
%      \Macro\parg{argument} & \parg{argument} \\
%      \Macro\aarg{argument} & \aarg{argument} \\
%      \Macro\sarg           & \sarg           \\
%      \bottomrule
%   \end{tabular}
% \end{table}
%
% \subsubsection{Examples}
% \DescribeEnv[<example code>]{example}
% Perhaps the most powerful way to illustrate features of a package
% is to show their function by examples. This is made possible by
% the \env{example} environment. By enclosing example code in this
% environment, the actual code is typeset next to the result it would
% produce, as seen below\footnote{Note that the showcased \env{example}
% environment doesn't contain another \env{example} environment --- the
% environment is not intended to be nested inside itself.}\footnote{The
% percent characters in the example are caused by the \pkg{docstrip}
% requirement of prefixing the documentation with them.}:
% \begin{example}
% Simply typesetting a
% \emph{paragraph} may
% be simple enough, but
% it should showcase
% the utility of the
% environment well enough.
% \end{example}
%
% Note that for this to work the package obviously needs to be loaded.
% As such, it is probably a good idea to combine the use of \env{example}
% with the \opt{load} option, so be sure to read up on the caveats of
% using that option (see page \pageref{ssec:options:load}).
%
% Since the \env{example} environment is based on the same mechanisms
% as \env{MacroCode}, (mostly) the same typesetting properties apply.
% In particular, the code will be highlighted if \pkg{minted} is
% available. \Notice{Since the backend utilizes \cs{verbatim}, the
% usual caveats apply. In particular, leaving whitespace before
% \cs{end}\marg{example} will result in an extra newline at the end
% of the displayed code.}
%
% \subsubsection{Options}
% Package options are of course important to describe, and to this
% end four macros are provided. They aid in describing options of
% both regular boolean and the more modern key-value syntax. They
% are intended to be used in a sequence: \\ 
% \mbox{\Macro\Option{...}
%       \AlsoMacro\WithValues{...}
%       \AlsoMacro\AndDefault{...}}
%
% \DescribeMacro\Option{<option>}
% \DescribeMacro\Options{<option>,...}
% These macros typeset an option, and may be followed by the
% \cs{WithValues} macro \Notice{the with \cs{Options}, only the
% first option in the list will work with \cs{WithValues}}.
%
% \DescribeMacro\WithValues{<value>,...}
% This macro typesets a comma-separated list of values a specific
% option can take. It may be followed by the \cs{AndDefault} macro.
%
% \DescribeMacro\AndDefault{<default value>}
% This macro typesets the default value of an option. It may follow
% either \cs{Options} or \cs{WithValues}.
%
% Common constructs using these macros include:
% \begin{itemize}
% \item \mbox{\small
%       \Macro\Option{<option>}
%       \AlsoMacro\WithValues{<value>,...}
%       \AlsoMacro\AndDefault{<default>}}
% \item \mbox{\small
%       \Macro\Options{<option>,no\meta{option}}
%       \AlsoMacro\AndDefault{no\meta{option}}}
% \end{itemize}
% 
% \subsubsection{Macros}
% The \thepkg\ class inherits a number of macros for describing the
% package macros from the \pkg{ydoc} package. Only four of them are
% to be considered stable.
% \LongWarning{
% The macros \cs{MakeShortMacroArgs} and \cs{DeleteShortMacroArgs}
% and the environments \env{DescribeMacros} and \env{DescribeMacrosTab}
% provided by \pkg{ydoc} are unsupported as of \thepkg\ \theversion. 
% They may work, but this is
% not a guarantee and they are most likely broken or may break other
% features of \thepkg.
% }
%
% \DescribeMacro\DescribeMacro<\textbackslash macro><macro arguments>
% The \cs{DescribeMacro} macro documents a macro along with its
% arguments. Any number of \meta{macro arguments} may follow the macro,
% and \cs{DescribeMacro} will stop reading arguments on the first
% non-argument token. The macro will be indexed.
% \LongWarning{Although \meta{\textbackslash macro} can include
% \texttt{@} signs, it is not possible to document \LaTeX3-style
% macros (with underscores and colons) without the following hack:
% \par \medskip \begingroup
% \cs*{ExplSyntaxOn} \\
% \cs*{cs\_set\_protected\_nopar:Npn}\cs*{ExplHack}\texttt{\{} \\
% \hspace*{2ex}\cs*{char\_set\_catcode\_letter:n}\texttt{\{~58~\}} \\
% \hspace*{2ex}\cs*{char\_set\_catcode\_letter:n}\texttt{\{~95~\}} \\
% \texttt{\}} \\
% \cs*{ExplSyntaxOff} \\
% \cs*{ExplHack} \\
% \endgroup }
%
% \DescribeMacro\Macro<\textbackslash macro><macro arguments>
% This is simply a variant of \cs{DescribeMacro} for use in running
% text. It is equivalent to \cs{MacroArgs}\cs{AlsoMacro}.
%
% \DescribeMacro\MacroArgs<macro aguments>
% This macro formats \meta{macro arguments} the same way
% \cs{DescribeMacro} does. As with \cs{Macro}, it is used in
% running text.
%
% \DescribeMacro\AlsoMacro<\textbackslash macro><further arguments>
% This macro should be used inside \meta{macro arguments} of the macros
% described above, and typesets an additional macro as part of the
% syntax of the described macro. For instance, the \cs{csname} macro
% could be described with the sequence
% \cs*{Macro}\cs*{csname}\texttt{<command sequence name>}
% \cs*{AlsoMacro}\cs{endcsname}, which would be rendered as
% \Macro\csname<command sequence name>\AlsoMacro\endcsname\relax.
%
% \subsubsection{Environments}
% In addition to the macros describing macros, \thepkg\ also inherits
% one environment and one macro to describe environments. These are
% similar to the macros described previously in both form and function,
% but lack equivalents for running text.
%
% \DescribeMacro\DescribeEnv[<body content>]{<name>}<arguments>
% This macro describes an environment, in the same way
% \cs{DescribeMacro} does for macros. The \meta{body content}, which
% is optional, may be used to indicate what kind of content the
% environment is designed to contain. The \cs{MacroArgs}
% macro is automatically inserted before \meta{body content}.
%
% \subsubsection{Other entities}
% The document class also provides macros to describe \hologo{BibTeX}
% entries and generic themes. The \hologo{BibTeX} entries are described
% using the \cs{BibEntry} and \cs{WithFields} macros, while themes are
% described using the \cs{Theme} macro.
%
% \DescribeMacro\BibEntry{<entry name>}
% \AlsoMacro\WithFields[<optional fields>]{<mandatory fields>}
% These two macros describe a \hologo{BibTeX} entry named
% \meta{entry name} (\emph{i.e.}, \bib*{\meta{entry name}}) along with
% its optional and mandatory fields.
%
% \DescribeMacro\Theme{<theme name>}
% This macro describes a theme named \meta{theme name}. These could be
% used to describe any kind of theme, such as color themes of a
% document class.
%
% \DescribeMacro\DescribeFile{<filename>}
% This macro describes a special file named \meta{filename}. This
% could be a configuration file or similar that is either part of
% the package or something the package reads if available.
%
% \subsection{Describing the implementation}
% In true \TeX\ (and literal programming) fashion the document class
% also provides ways to describe, in detail, parts of the
% implementation. The most essential of the implementation environments,
% without which \thepkg\ doesn't generate any files, is the
% \env{MacroCode} environment. Other than that, the implementation
% environments should be compatible with or analogous to the standard
% \pkg{ltxdoc} document class.
% 
% \subsubsection{Implementation environments}\label{sec:impl-env}
% The environments described in this section indicate the
% implementation of different concepts including macros, environments
% and options. They each have a starred variant which doesn't print
% the concept name (only indexes it), and a non-starred variant which
% does \Notice{inside these environments, \cs{changes} will refer to
% the relevant entity instead of logging \enquote{general} changes}.
%
% Some of the following environment can typeset descriptions of the 
% internal arguments (\texttt{\#1}, \texttt{\#2} \emph{etc.}) to
% improve readability of the implementation code.
%
% \DescribeEnv[<description>]{macro}{<\textbackslash macro>}
%                     [<\# of args>]{<arg 1 description>}[<default value>]'...'
%                                   {<arg n description>}[<default value>]
% With this environment, the implementation of a macro is described.
% Note that as with \cs{DescribeMacro}, \LaTeX3-style macros can not
% be used in \meta{\textbackslash macro} without the catcode hack
% mentioned earlier.
%
% \DescribeEnv[<description>]{environment}{<environment>}
%                     [<\# of args>]{<arg 1 description>}[<default value>]'...'
%                                   {<arg n description>}[<default value>]
% This environment describes the implementation of an environment.
%
% \DescribeEnv[<description>]{option}{<option>}
% This environment describes the implementation of an option.
%
% \DescribeEnv[<description>]{bibentry}{<@entry>}
% This environment describes the implementation of a \hologo{BibTeX}
% entry type.
%
% \DescribeEnv[<description>]{theme}{<theme>}
% This environment describes the implementation of a theme.
%
% \subsubsection{The \env{MacroCode} environment}
% The \enquote{main event} of the \thepkg\ document class is the
% \env{MacroCode} environment. It has roughly the same role the 
% \env{macrocode} environment has in the \pkg{docstrip} system, except
% that it in addition to typesetting the implementation also saves it
% to the target files.
%
% The workflow is simple; before using \env{MacroCode} to export code
% to a file the file must be declared using \cs{DeclareFile}, which
% also assigns a key to the file (the default is the filename). This
% key is passed to the \env{MacroCode} environment, which saves the
% code to the specified file.
%
% \DescribeMacro\DeclareFile['key='<key>',preamble='<preamble>]{<filename>}
% The \cs{DeclareFile} macro declares a file for future use with
% \env{MacroCode}. The optional argument is a comma separated list of
% key-value options, where the possible keys are \texttt{key} and
% \texttt{preamble}. Here \meta{key} is a key that is used instead
% of the filename in \env{MacroCode}, and \meta{preamble} is a
% token or command sequence expanding to a preamble which will be 
% prepended to the file on output.
%
% \DescribeMacro\PreambleTo{<\textbackslash token>}{<filename>}
% Reads the preamble from \meta{filename}. Lines from the file are
% appended to \meta{\textbackslash token} until a line which does not
% begin with \texttt{\%\%} is encountered.
%
% \DescribeMacro\SelfPreambleTo{<\textbackslash token>}
% This reads the preamble from the curent file. It is equivalent to
% the sequence
% \Macro\PreambleTo{<\textbackslash token>}{\cs*{jobname}.tex}.
%
% \DescribeEnv[<implementation>]{MacroCode}{<key>',...'}
% The \env{MacroCode} environment typesets and exports
% \meta{implementation} verbatim to the file associated with \meta{key}.
% As such, it is the analogue of the \env{macrocode} environment from
% \pkg{ltxdoc}, but does not suffer from some of its drawbacks (the
% sensitivity to whitespace, for instance). As detailed by the
% description of the \opt{highlight} option (on page
% \pageref{ssec:option:highlight}), the environment will highlight
% the code using \pkg{minted} if possible. Multiple \meta{key}s are
% allowed, and the code will be written to all corresponding files.
%
% \subsubsection{Hiding the implementation}
% For lagre packages it may be of interest to hide the implementation
% from the documentation. This is accomplished using the two marker
% macros \cs{Implementation} and \cs{Finale} (which should be present
% even if not hiding the implementation), and the switch macro
% \cs{OnlyDescription}.
%
% \DescribeMacro\Implementation
% This macro indicates the start of the implementation. Normally, this
% would directly precede the \cs{section} under which the implementation
% is organized.
%
% \DescribeMacro\Finale
% This macro indicates the end of the implementation. Usually the only
% things happening after this is the printing of indices, the change
% log, bibliographies and the end of the \env{document} environment.
%
% \DescribeMacro\OnlyDescription
% This macro, which should be issued in the preamble, indicates that the
% implementation should be hidden. \Warning{this has the side effect
% that a page break is inserted where the implementation would normally
% reside.}
%
% \subsection{Documenting changes}
% One type of useful information you should provide in your
% documentation is a list of changes. The \thepkg\ document class
% provides a change list system based on the \pkg{glossaries} package.
% As such, including a change list in your documentation requires you
% to run \texttt{makeglossaries} between the first and second \LaTeX\
% run.
%
% \DescribeMacro\changes{<version>}{<description>}
% The \cs{changes} macro provides the main interface to the change
% list system, and adds changes to the change list. Each change is
% added with a \emph{context}; if the macro is issued inside one of
% the macros described in section~\ref{sec:impl-env}, the concept
% currenly being described will be the context. Outside these
% environment, the context is \enquote{general}. For every context
% and \meta{version}, only one change may be recorded, otherwise
% \pkg{glossaries} will issue a warning.
%
% \DescribeMacro\PrintChanges
% This macro prints the list of changes. As explained earlier, this
% requires you to run \texttt{makeglossaries} between the two \LaTeX\ 
% runs.
%
% \subsection{Producing an index}
% The macros previously discussed in sections~\ref{sec:ref-macros},
% \ref{sec:doc-macros} and \ref{sec:impl-env} automatically index their
% aguments using \pkg{glossaries}. By running \texttt{makeglossaries}
% you can include an index of all macros, environments, packages and 
% such that are discussed, documented or implemented in your package.
%
% \DescribeMacro\PrintIndex
% Much like the \cs{PrintChanges} macro, this prints the index. As with
% the list of changes, this requires that you run 
% \texttt{makeglossaries} between the two \LaTeX\ runs.
%
% \section{Known issues}
% A list of current issues is available in the Github repository of this
% package\footnote{\url{https://github.com/urdh/skdoc/issues}}, but as
% of the release of \theversion, there are no known issues.
% \iffalse
% \begin{description}
%   \item[\#?]  ???
% \end{description}
% \fi
% If you discover any bugs in this package, please report them to the 
% issue tracker in the \thepackage\ Github repository.
%
% \Implementation \ExplHack
% \section{Implementation}
% \iffalse
%</driver>
%<*class>
% \fi
% \subsection{Require packages}
% We begin with loading the \pkg{scrartcl} KOMA-script class and a
% few packages we'll be needing.
%    \begin{macrocode}
\PassOptionsToPackage{log-declarations=false}{xparse}
\LoadClass[ DIV7,
            headings=big,
            numbers=noenddot,
            abstracton,
            bibliography=totocnumbered,
            index=totoc
        ]{scrartcl}
%    \end{macrocode}
% These packages are basic low-level things that we use to declare
% commands, work with strings and so on.
%    \begin{macrocode}
\RequirePackage{etoolbox,xstring,xparse,atbegshi,kvoptions,pdftexcmds,everyhook}
%    \end{macrocode}
% Now, higher-level packages we use in our definitions.
%    \begin{macrocode}
\RequirePackage{verbatim,needspace,marginnote,calc,hyperref,multicol,hologo}
\RequirePackage[nomain,xindy,numberedsection,order=letter,
                sanitizesort=false]{glossaries}
%    \end{macrocode}
% We also include the \pkg{ydoc} packages we'll be extending.
%    \begin{macrocode}
\RequirePackage{ydoc-code,ydoc-desc}
%    \end{macrocode}
% The rest is basically just styling.
%    \begin{macrocode}
\RequirePackage{scrpage2}
\AtEndOfClass{%
    \RequirePackage[\skdoc@babel]{babel}
    \RequirePackage[english=british]{csquotes}
    \RequirePackage[font=small,labelfont=bf,textfont=it]{caption}
    \RequirePackage{PTSerif,sourcecodepro}
    \RequirePackage[defaultsans,osfigures,scale=0.95]{opensans}
    \RequirePackage[babel]{microtype}
}
%    \end{macrocode}
%
% \subsection{Error messages}
% Set up some error message texts for later use.
%    \begin{macrocode}
\msg_new:nnn{skdoc}{key-exists}{File~key~"#1"~already~declared!}
\msg_new:nnn{skdoc}{key-nexists}{File~key~"#1"~hasn't~been~declared!}
\msg_new:nnn{skdoc}{wrote-file}{Writing~things~to~file~"#1".}
\msg_new:nnn{skdoc}{read-preamble}{Reading~preamble~from~file~"#1".}
\msg_new:nnn{skdoc}{no-lppl}{Could~not~include~LPPL:~lppl.tex~does~not~exist!}
\msg_new:nnn{skdoc}{no-minted}{Syntax~highlighting~disabled:~couldn't~find~minted.sty!}
\msg_new:nnn{skdoc}{no-shell-escape}{Syntax~highlighting~disabled:~running~without~unrestricted~\string\write18.}
\msg_new:nnn{skdoc}{no-pygments}{Syntax~highlighting~disabled:~couldn't~find~"pygmentize".}
\msg_new:nnnn{skdoc}{bibtex-unsupported}{The `bibtex` package is unsupported: continue at own risk!}{Use a more modern bibliography system (i.e. `biblatex`) instead.}
%    \end{macrocode}
%
% \subsection{Booleans}
% Set up some booleans used throughout the code.
% \begin{macro}{\g__skdoc_use_minted_bool}
% The \texttt{use_minted} boolean keeps track of wether we're using
% \pkg{minted} or not. The default value is \emph{false}.
%    \begin{macrocode}
\bool_new:N\g__skdoc_use_minted_bool
\bool_gset_false:N\g__skdoc_use_minted_bool
%    \end{macrocode}
% \end{macro}
% \begin{macro}{\g__skdoc_no_index_bool}
% The \texttt{no_index} boolean keeps track of wether the indexing
% macros should write things to the index or not. The default value is
% \emph{false} (don't not write things to the index).
%    \begin{macrocode}
\bool_new:N\g__skdoc_no_index_bool
\bool_gset_false:N\g__skdoc_no_index_bool
%    \end{macrocode}
% \end{macro}
% \begin{macro}{\g__skdoc_in_example_bool}
% The \texttt{in_example} boolean keeps track of wether we are currently
% inside an example or not (used mainly by \cs{skdoc@verbatim}). The
% default value is of course \emph{false}.
%    \begin{macrocode}
\bool_new:N\g__skdoc_in_example_bool
\bool_gset_false:N\g__skdoc_in_example_bool
%    \end{macrocode}
% \end{macro}
% \begin{macro}{\g__skdoc_with_implementation_bool}
% The \texttt{with_implementation} boolean keeps track of wether we are
% going to typeset the implementation or not. The default value is
% \emph{true}, and this is only changed by \cs{OnlyDescription}.
%    \begin{macrocode}
\bool_new:N\g__skdoc_with_implementation_bool
\bool_gset_true:N\g__skdoc_with_implementation_bool
%    \end{macrocode}
% \end{macro}
% \begin{macro}{\g__skdoc_in_implementation_bool}
% The \texttt{in_implementation} boolean keeps track of wether we are
% in the implementation part of the documentation or not. It defaults
% to \emph{false}, and is changed by \cs{Implementation} and \cs{Finale}.
%    \begin{macrocode}
\bool_new:N\g__skdoc_in_implementation_bool
\bool_gset_false:N\g__skdoc_in_implementation_bool
%    \end{macrocode}
% \end{macro}
% \begin{macro}{\g__skdoc_negative_space_bool}
% This boolean keeps track of the negative spacing added by description
% environments, allowing \env{MacroCode} to offset it if it follows the
% start of these environments directly. Bit of a hack, really.
%    \begin{macrocode}
\bool_new:N\g__skdoc_negative_space_bool
\bool_gset_false:N\g__skdoc_negative_space_bool
%    \end{macrocode}
% \end{macro}
%
% Finally, we define a helpful conditional based on the last two
% booleans. It decides wether code in \cs{skdoc@verbatim} is typeset
% or just output to a file.
% \begin{macro*}{\__skdoc_if_print_code_p:}
% \begin{macro*}{\__skdoc_if_print_code:T}
% \begin{macro*}{\__skdoc_if_print_code:F}
% \begin{macro}{\__skdoc_if_print_code:TF}
%    \begin{macrocode}
\prg_new_conditional:Nnn\__skdoc_if_print_code:{p,T,F,TF}{
    \bool_if:nTF{
        \g__skdoc_in_implementation_bool &&
        !\g__skdoc_with_implementation_bool
    }{
        \prg_return_false:
    }{
        \prg_return_true:
    }
}
%    \end{macrocode}
% \end{macro}
% \end{macro*}
% \end{macro*}
% \end{macro*}
%
% \subsection{Options}
% Define the all options, and process them.
%    \begin{macrocode}
\SetupKeyvalOptions{
    family=skdoc,
    prefix=skdoc@
}
\DeclareStringOption{load}[\jobname]
\DeclareStringOption[british]{babel}
\DeclareBoolOption[true]{highlight}
\ProcessKeyvalOptions*
%    \end{macrocode}
% If the \opt{load} option was used, load the given package if it exists.
%    \begin{macrocode}
\IfStrEq{\skdoc@load}{}{}{%
    \IfFileExists{\skdoc@load.sty}{%
        \RequirePackage{\skdoc@load}
    }{}
}
%    \end{macrocode}
%
% \subsection{Special case: syntax highlighting}
% We support syntax highlighting through \pkg{minted}, but only if 
% we're running with unrestricted \cs{write18} access \emph{and}
% there exists a suitable executable (\texttt{pygmentize}). Since
% \pkg{minted} bails out without \cs{write18} access, we have to test
% that before even loading it.
%
% \subsubsection{A simple \cs{write18} test}
% Using the \cs{pdf@shellescape} macro from \pkg{pdftexcmds}, we
% can define a new conditional that decides if we have unrestricted
% \cs{write18} or not.
% \begin{macro*}{\__skdoc_if_shellescape:T}
% \begin{macro*}{\__skdoc_if_shellescape:F}
% \begin{macro}{\__skdoc_if_shellescape:TF}
%    \begin{macrocode}
\prg_new_conditional:Nnn\__skdoc_if_shellescape:{T,F,TF}{
    \if_cs_exist:N\pdf@shellescape
%    \end{macrocode}
% If the command sequence exists (it really should), we test to see
% if it is equal to one. The \cs{pdf@shellescape} macro will be zero
% if no \cs{write18} access is available, two if we have restricted
% access and one if access is unrestricted.
%    \begin{macrocode}
        \if_int_compare:w\pdf@shellescape=\c_one
            \prg_return_true:
        \else:
            \prg_return_false:
        \fi:
%    \end{macrocode}
% If the command sequence doesn't exist, we assume that we have
% unrestricted \cs{write18} access (we probably don't), and let
% \LaTeX\ complain about it later.
%    \begin{macrocode}
    \else:
        \prg_return_true:
    \fi:
}
%    \end{macrocode}
% \end{macro}
% \end{macro*}
% \end{macro*}
%
% \subsubsection{Testing for an application}
% In order to find out wether \texttt{pygmentize} is installed, we
% have to test for its binary. This is done by shelling out to
% \texttt{which} and reading a temporary file, which means we need
% a way to delete that file.
% \begin{macro}{\__skdoc_rm:n}[1]
%   {The file to be removed}
%    \begin{macrocode}
\cs_new:Npn\__skdoc_rm:n#1{
    \immediate\write18{rm~#1}
}
%    \end{macrocode}
% \end{macro}
% Now, let's define the actual test. Note that this test will fail
% on non-unixy platforms (\emph{i.e.} Windows). Deal with it for now.
% \begin{macro*}{\__skdoc_if_pygmentize:nT}
% \begin{macro*}{\__skdoc_if_pygmentize:nF}
% \begin{macro}{\__skdoc_if_pygmentize:nTF}[1]
%   {The name of the binary to check existence of}
%    \begin{macrocode}
\prg_new_conditional:Nnn\__skdoc_if_pygmentize:{T,F,TF}{
%    \end{macrocode}
% First, run \texttt{which} to create the temporary file.
%    \begin{macrocode}
    \immediate\write18{which~pygmentize~&&~touch~\jobname.aex}
%    \end{macrocode}
% A temporary boolean is defined to hold the result of the test (this
% is a bit of carco cult programming, any clues as to why placing the
% result inside \cs{file_if_exists:nTF} doesn't work are welcome), and
% if the test is successful we remove the temporary file.
%    \begin{macrocode}
    \bool_set:Nn\l_tmpa_bool{\c_false_bool}
    \file_if_exist:nT{\jobname.aex}{
        \__skdoc_rm:n{\jobname.aex}
        \bool_set:Nn\l_tmpa_bool{\c_true_bool}
    }
%    \end{macrocode}
% Finally, the temporary boolean is used to return a result.
%    \begin{macrocode}
    \if_bool:N\l_tmpa_bool
        \prg_return_true:
    \else:
        \prg_return_false:
    \fi:
}
%    \end{macrocode}
% \end{macro}
% \end{macro*}
% \end{macro*}
%
% Using the macros defined above, we can test for both \cs{write18}
% and \texttt{pygmentize}, loading \pkg{minted} and setting a flag
% if both exist. If the option \opt{highlight} was supplied and set
% to \texttt{false}, we do nothing.
%    \begin{macrocode}
\ifskdoc@highlight
    \IfFileExists{\skdoc@load.sty}{
        \__skdoc_if_shellescape:TF{
            \__skdoc_if_pygmentize:TF{
                \bool_gset_true:N\g__skdoc_use_minted_bool
                \RequirePackage{minted}
%    \end{macrocode}
% Now that we have \pkg{minted}, we redefine the style of the line
% numbers to match what we have further down for the non-highlighted
% mode.
%    \begin{macrocode}
                \renewcommand{\theFancyVerbLine}{
                    \sffamily\tiny\textcolor{lightgray}{
                    \oldstylenums{\arabic{FancyVerbLine}}}
                }
%    \end{macrocode}
% We also define our own Pygments style, to match the \thepackage%
% color theme. This is a bunch of code generated by running
% \texttt{pygmentize -S default -f latex}, with modifications. Note
% the abundancy of \verb|#| characters; this is somewhat black magic,
% but apparently \LaTeX\ expands them one more time than I'd expect
% in this context. Probably a \LaTeX3 artifact. Also, the macro names
% somehow correspond to Pygments entities (see \emph{e.g.}
% \texttt{pygmentize -S default -f html}), but it's not entirely
% apparent which macro corresponds to what entity.
%    \begin{macrocode}
                \renewcommand\minted@usedefaultstyle{}
                \def\PY@reset{\let\PY@it=\relax \let\PY@bf=\relax%
                    \let\PY@ul=\relax \let\PY@tc=\relax%
                    \let\PY@bc=\relax \let\PY@ff=\relax}
                \def\PY@tok##1{\csname PY@tok@##1\endcsname}
                \def\PY@toks##1+{\ifx\relax##1\empty\else%
                    \PY@tok{##1}\expandafter\PY@toks\fi}
                \def\PY@do##1{\PY@bc{\PY@tc{\PY@ul{%
                    \PY@it{\PY@bf{\PY@ff{##1}}}}}}}
                \def\PY##1##2{\PY@reset\PY@toks##1+\relax+\PY@do{##2}}
                \expandafter\def\csname PY@tok@gd\endcsname{%
                    \def\PY@tc####1{\textcolor{scriptcolor}{####1}}}
                \expandafter\def\csname PY@tok@gu\endcsname{%
                    \let\PY@bf=\textbf%
                    \def\PY@tc####1{\textcolor{optioncolor}{####1}}}
                \expandafter\def\csname PY@tok@gt\endcsname{%
                    \def\PY@tc####1{\textcolor{scriptcolor}{####1}}}
                \expandafter\def\csname PY@tok@gs\endcsname{%
                    \let\PY@bf=\textbf}
                \expandafter\def\csname PY@tok@gr\endcsname{%
                    \def\PY@tc####1{\textcolor{scriptcolor}{####1}}}
                \expandafter\def\csname PY@tok@cm\endcsname{%
                    \let\PY@it=\textit%
                    \def\PY@tc####1{\textcolor{lightgray}{####1}}}
                \expandafter\def\csname PY@tok@vg\endcsname{%
                    \def\PY@tc####1{\textcolor{scriptcolor}{####1}}}
                \expandafter\def\csname PY@tok@m\endcsname{%
                    \def\PY@tc####1{\textcolor{meta}{####1}}}
                \expandafter\def\csname PY@tok@mh\endcsname{%
                    \def\PY@tc####1{\textcolor{meta}{####1}}}
                \expandafter\def\csname PY@tok@cs\endcsname{%
                    \let\PY@it=\textit%
                    \def\PY@tc####1{\textcolor{macroimpl}{####1}}}
                \expandafter\def\csname PY@tok@ge\endcsname{%
                    \let\PY@it=\textit}
                \expandafter\def\csname PY@tok@vc\endcsname{%
                    \def\PY@tc####1{\textcolor{scriptcolor}{####1}}}
                \expandafter\def\csname PY@tok@il\endcsname{%
                    \def\PY@tc####1{\textcolor{meta}{####1}}}
                \expandafter\def\csname PY@tok@go\endcsname{%
                    \def\PY@tc####1{\textcolor{gray}{####1}}}
                \expandafter\def\csname PY@tok@cp\endcsname{%
                    \def\PY@tc####1{\textcolor{sharp!75}{####1}}}
                \expandafter\def\csname PY@tok@gi\endcsname{%
                    \def\PY@tc####1{\textcolor{bright}{####1}}}
                \expandafter\def\csname PY@tok@gh\endcsname{%
                    \let\PY@bf=\textbf%
                    \def\PY@tc####1{\textcolor{section}{####1}}}
                \expandafter\def\csname PY@tok@ni\endcsname{%
                    \let\PY@bf=\textbf%
                    \def\PY@tc####1{\textcolor{keydesc}{####1}}}
                \expandafter\def\csname PY@tok@nn\endcsname{%
                    \let\PY@ul=\underline%
                    \def\PY@tc####1{\textcolor{black}{####1}}}
                \expandafter\def\csname PY@tok@no\endcsname{%
                    \def\PY@tc####1{\textcolor{scriptcolor}{####1}}}
                \expandafter\def\csname PY@tok@na\endcsname{%
                    \def\PY@tc####1{\textcolor{meta!50}{####1}}}
                \expandafter\def\csname PY@tok@nb\endcsname{%
                    \def\PY@tc####1{\textcolor{black}{####1}}}
                \expandafter\def\csname PY@tok@nc\endcsname{%
                    \let\PY@ul=\underline%
                    \def\PY@tc####1{\textcolor{bright}{####1}}}
                \expandafter\def\csname PY@tok@nd\endcsname{%
                    \def\PY@tc####1{\textcolor{gray}{####1}}}
                \expandafter\def\csname PY@tok@si\endcsname{%
                    \def\PY@tc####1{\textcolor{sharp}{####1}}}
                \expandafter\def\csname PY@tok@nf\endcsname{%
                    \def\PY@tc####1{\textcolor{bright}{####1}}}
                \expandafter\def\csname PY@tok@s2\endcsname{%
                    \def\PY@tc####1{\textcolor{sharp}{####1}}}
                \expandafter\def\csname PY@tok@vi\endcsname{%
                    \def\PY@tc####1{\textcolor{scriptcolor}{####1}}}
                \expandafter\def\csname PY@tok@nt\endcsname{%
                    \let\PY@bf=\textbf%
                    \def\PY@tc####1{\textcolor{meta!50}{####1}}}
                \expandafter\def\csname PY@tok@nv\endcsname{%
                    \def\PY@tc####1{\textcolor{scriptcolor}{####1}}}
                \expandafter\def\csname PY@tok@s1\endcsname{%
                    \def\PY@tc####1{\textcolor{sharp}{####1}}}
                \expandafter\def\csname PY@tok@gp\endcsname{%
                    \def\PY@tc####1{\textcolor{gray}{####1}}}
                \expandafter\def\csname PY@tok@sh\endcsname{%
                    \def\PY@tc####1{\textcolor{sharp}{####1}}}
                \expandafter\def\csname PY@tok@ow\endcsname{%
                    \def\PY@tc####1{\textcolor{macroimpl}{####1}}}
                \expandafter\def\csname PY@tok@sx\endcsname{%
                    \def\PY@tc####1{\textcolor{sharp}{####1}}}
                \expandafter\def\csname PY@tok@bp\endcsname{%
                    \def\PY@tc####1{\textcolor{black}{####1}}}
                \expandafter\def\csname PY@tok@c1\endcsname{%
                    \let\PY@it=\textit%
                    \def\PY@tc####1{\textcolor{lightgray}{####1}}}
                \expandafter\def\csname PY@tok@kc\endcsname{%
                    \def\PY@tc####1{\textcolor{macroimpl}{####1}}}
                \expandafter\def\csname PY@tok@c\endcsname{%
                    \let\PY@it=\textit%
                    \def\PY@tc####1{\textcolor{lightgray}{####1}}}
                \expandafter\def\csname PY@tok@mf\endcsname{%
                    \def\PY@tc####1{\textcolor{meta}{####1}}}
                \expandafter\def\csname PY@tok@err\endcsname{%
                    \def\PY@tc####1{\textcolor{intlink}{####1}}%
                    \def\PY@bc####1{\setlength{\fboxsep}{0pt}%
                    \colorbox{bright}{\strut ####1}}}
                \expandafter\def\csname PY@tok@kd\endcsname{%
                    \def\PY@tc####1{\textcolor{macroimpl}{####1}}}
                \expandafter\def\csname PY@tok@ss\endcsname{%
                    \def\PY@tc####1{\textcolor{macroimpl}{####1}}}
                \expandafter\def\csname PY@tok@sr\endcsname{%
                    \def\PY@tc####1{\textcolor{meta}{####1}}}
                \expandafter\def\csname PY@tok@mo\endcsname{%
                    \def\PY@tc####1{\textcolor{meta}{####1}}}
                \expandafter\def\csname PY@tok@mi\endcsname{%
                    \def\PY@tc####1{\textcolor{meta}{####1}}}
                \expandafter\def\csname PY@tok@kn\endcsname{%
                    \def\PY@tc####1{\textcolor{macroimpl}{####1}}}
                \expandafter\def\csname PY@tok@kr\endcsname{%
                    \def\PY@tc####1{\textcolor{macroimpl}{####1}}}
                \expandafter\def\csname PY@tok@s\endcsname{%
                    \def\PY@tc####1{\textcolor{sharp}{####1}}}
                \expandafter\def\csname PY@tok@kp\endcsname{%
                    \def\PY@tc####1{\textcolor{macroimpl}{####1}}}
                \expandafter\def\csname PY@tok@w\endcsname{%
                    \def\PY@tc####1{\textcolor{lightgray}{####1}}}
                \expandafter\def\csname PY@tok@kt\endcsname{%
                    \def\PY@tc####1{\textcolor{black}{####1}}}
                \expandafter\def\csname PY@tok@sc\endcsname{%
                    \def\PY@tc####1{\textcolor{sharp}{####1}}}
                \expandafter\def\csname PY@tok@sb\endcsname{%
                    \def\PY@tc####1{\textcolor{sharp}{####1}}}
                \expandafter\def\csname PY@tok@k\endcsname{%
                    \def\PY@tc####1{\textcolor{macroimpl}{####1}}}
                \expandafter\def\csname PY@tok@se\endcsname{%
                    \def\PY@tc####1{\textcolor{sharp}{####1}}}
                \expandafter\def\csname PY@tok@sd\endcsname{%
                    \def\PY@tc####1{\textcolor{sharp}{####1}}}
                \expandafter\def\csname PY@tok@nl\endcsname{%
                    \def\PY@tc####1{\textcolor{bright}{####1}}}
                \expandafter\def\csname PY@tok@ne\endcsname{%
                    \def\PY@tc####1{\textcolor{bright}{####1}}}
                \expandafter\def\csname PY@tok@o\endcsname
                \def\PYZdl{\char`\$}
                \def\PYZhy{\char`\-}
                \def\PYZsq{\char`\'}
                \def\PYZdq{\char`\"}
                \def\PYZti{\char`\~}
                \def\PYZat{@}
                \def\PYZlb{[}
                \def\PYZrb{]}
%    \end{macrocode}
% If there's no \texttt{pygmentize}, no \cs{write18} or no \pkg{minted},
% we display a warning message and proceed without highlighting.
%    \begin{macrocode}
            }{
                \msg_warning:nn{skdoc}{no-pygments}
            }
        }{
            \msg_warning:nn{skdoc}{no-shell-escape}
        }
    }{
        \msg_warning:nn{skdoc}{no-minted}
    }
\fi
%    \end{macrocode}
% \subsection{The \env{MacroCode} environment}
% We need a token list and input/output.
%    \begin{macrocode}
\tl_new:N\skdoc@temptl
\ior_new:N\skdoc@input
\iow_new:N\skdoc@output
%    \end{macrocode}
% \begin{macro}{\DeclareFile}[2]
%   {A list of key-value parameters}
%   {Filename of the file being declared}
% This declares a file as part of the bundle, which means we will be
% writing things to it.
%    \changes{1.1a}{Don't create a new token list until we know the
%                   key isn't a duplicate}
%    \begin{macrocode}
\DeclareDocumentCommand\DeclareFile{om}{
    \group_begin:
    \keys_define:nn{skdoc@declarefile}{%
        preamble .value_required:,
        preamble .code:n = \edef\skdoc@preamble{##1},
        key .value_required:,
        key .code:n = \def\skdoc@key{##1}
    }%
    \def\skdoc@preamble{}%
    \def\skdoc@key{#2}%
    \IfNoValueTF{#1}{}{\keys_set:nn{skdoc@declarefile}{#1}}
    \int_if_exist:cTF{skdoc@output@\skdoc@key @line}{
        \msg_critical:nnx{skdoc}{key-exists}{\skdoc@key}
    }{
        \int_zero_new:c{skdoc@output@\skdoc@key @line}
    }
    \tl_new:c{skdoc@output@\skdoc@key}
    \IfStrEq{\skdoc@preamble}{}{
        \def\skdoc@preamble@extra{}
    }{
        \tl_set:Nx\l_tmpa_tl{\skdoc@preamble}
        \edef\skdoc@temp{\tl_head:N\l_tmpa_tl}
        \def\skdoc@preamble@extra{
            \skdoc@temp\skdoc@temp\space~This~is~file~`#2',~generated~from~`\c_job_name_tl.tex'~(key~`\skdoc@key').
        }
    }
    \expandafter\xdef\csname skdoc@write@#2\endcsname{%
        \noexpand\msg_log:nnn{skdoc}{wrote-file}{#2}
        \noexpand\iow_open:Nn\noexpand\skdoc@output{#2}
        \noexpand\IfStrEq{\skdoc@preamble}{}{}{
            \noexpand\iow_now:Nx\noexpand\skdoc@output{\skdoc@preamble@extra}
            \noexpand\iow_now:Nx\noexpand\skdoc@output{\skdoc@preamble}
        }
        \noexpand\iow_now:Nx\noexpand\skdoc@output{\noexpand\tl_to_str:c{skdoc@output@\skdoc@key}}
        \noexpand\iow_close:N\noexpand\skdoc@output
    }
    \AfterEndDocument{\csname skdoc@write@#2\endcsname}
    \group_end:
}
%    \end{macrocode}
% \end{macro}
% \begin{environment}{skdoc@verbatim}[1]
%   {The key of the file being described}
% \changes{1.1}{Introducing syntax highlighting}
% \changes{1.1a}{Test for key existence before bumping counter,
%                use \cs{Needspace*} to prevent code label from
%                staying on the wrong page in some situations}
% \changes{1.2}{Hide line numbers when inside \env{example}}
% \changes{1.3}{Allow multiple (comma-separated) targets}
% \changes{1.3a}{Fix critical bug}
% This environment does all the actual work for \env{MacroCode}.
%    \begin{macrocode}
\clist_new:N\l__skdoc_keys
\DeclareDocumentEnvironment{skdoc@verbatim}{m}{%
    \clist_set:Nn\l__skdoc_keys{#1}
%    \end{macrocode}
% If the file we're supposed to write to hasn't been initialized yet,
% we error out with a critical error.
%    \begin{macrocode}
    \clist_map_inline:Nn\l__skdoc_keys{
        \int_if_exist:cTF{skdoc@output@##1@line}{}{
            \msg_critical:nnn{skdoc}{key-nexists}{##1}
        }%
%    \end{macrocode}
% Before doing anything, set create or increment a counter keeping
% track of the line number of the file we're writing to.
%    \begin{macrocode}
        \int_compare:nNnT{\int_use:c{skdoc@output@##1@line}}=\c_zero%
            {\int_gincr:c{skdoc@output@##1@line}}%
    }
%    \end{macrocode}
% Now, if we're supposed to print code, we set a few things up.
%    \begin{macrocode}
    \__skdoc_if_print_code:T{
        \bool_if:NTF\g__skdoc_use_minted_bool{
%    \end{macrocode}
% If we're using \pkg{minted}, we set a few options ans open the
% output file.
%    \begin{macrocode}
            \minted@resetoptions%
            \exp_args:Nnx\setkeys{minted@opt}{
                \int_compare:nNnF{\clist_count:N\l__skdoc_keys}>\c_one{
                    \bool_if:NF\g__skdoc_in_example_bool{linenos,}
                    firstnumber=\int_use:c{skdoc@output@#1@line}
                }
            }%
            \iow_open:Nn\minted@code{\jobname.pyg}%
            \Needspace*{2\baselineskip}%
        }{
%    \end{macrocode}
% Otherwise, we hack spaces with \cs{@bsphack} (unless we're in an
% \env{example} environment).
%    \begin{macrocode}
            \bool_if:NF\g__skdoc_in_example_bool{\@bsphack}%
        }
        \bool_if:NF\g__skdoc_in_example_bool{
%    \end{macrocode}
% In all non-example code, \pkg{minted} or not, we output a small
% marker showing what file we are writing to.
%    \begin{macrocode}
            \marginnote{
                \leavevmode
                \llap{
                    \scriptsize\color{gray}
                    $\langle$#1$\rangle$
                    \makebox[2ex]{\strut}
                }
            }
        }
    }
%    \end{macrocode}
% Otherwise, we set up the \env{verbatim} line processor.
%    \begin{macrocode}
    \def\verbatim@processline{%
%    \end{macrocode}
% We alsays append the line to the appropriate token list, so that it
% is saved to the output file.
%    \begin{macrocode}
        \clist_map_inline:Nn\l__skdoc_keys{
            \tl_gput_right:cx{skdoc@output@####1}{\the\verbatim@line\iow_newline:}%
        }
%    \end{macrocode}
% If we're supposed to print code, we do a lot more...
%    \begin{macrocode}
        \__skdoc_if_print_code:T{
            \bool_if:NTF\g__skdoc_use_minted_bool{
%    \end{macrocode}
% ...but if we're using minted, \enquote{a lot more} consists of also
% writing the line to the file used by \pkg{minted}.
%    \begin{macrocode}
                \iow_now:Nx\minted@code{\the\verbatim@line}%
            }{
%    \end{macrocode}
% Otherwise, we actually do a lot of stuff. We typeset the line number
% (unless we're in an example):
%    \begin{macrocode}
                \noindent\leavevmode%
                \bool_if:NF\g__skdoc_in_example_bool{\hspace*{-5ex}}
                \begin{minipage}[c][1ex]{\textwidth}
                    \bool_if:nF{
                        \g__skdoc_in_example_bool &&
                        !\int_compare_p:nNn{\clist_count:N\l__skdoc_keys}>\c_one
                    }{
                        \makebox[4ex]{%
                            \leavevmode
                            \sffamily\tiny\color{lightgray}\hfill%
                            \clist_map_inline:Nn\l__skdoc_keys{
                                \oldstylenums{\int_use:c{skdoc@output@####1@line}}%
                            }
                        }%
                        \hspace*{1ex}%
                    }
%    \end{macrocode}
% We also typeset the actual line:
%    \begin{macrocode}
                    {
                        \verbatim@font
                        \the\verbatim@line
                    }
                \end{minipage}
%    \end{macrocode}
% Both of them in the same one-line-high minipage covering the page
% width. Note the use of \cs{llap} for line numbers in the margin.
% We end with a \cs{par} for the next line.
%    \begin{macrocode}
                \par
            }
        }
%    \end{macrocode}
% Finally, the line number counter is incremented.
%    \begin{macrocode}
        \clist_map_inline:Nn\l__skdoc_keys{
            \int_gincr:c{skdoc@output@####1@line}%
        }
    }%
%    \end{macrocode}
% Now the \env{verbatim} catcode magic begins.
%    \begin{macrocode}
    \group_begin:
    \let\do\@makeother\dospecials\catcode`\^^M\active%
    \bool_if:nT{
        \__skdoc_if_print_code_p: &&
        !\g__skdoc_use_minted_bool
    }{
        \frenchspacing\@vobeyspaces
    }
    \verbatim@start%
}{%
    \group_end:
%    \end{macrocode}
% The catcode magic is over! Now, if we're printing code, there are
% a few things left to do.
%    \begin{macrocode}
    \__skdoc_if_print_code:T{
        \bool_if:NTF\g__skdoc_use_minted_bool{
%    \end{macrocode}
% If we're using \pkg{minted}, we still have to actually print the code.
% We begin with closing the output file.
%    \begin{macrocode}
            \iow_close:N\minted@code%
%    \end{macrocode}
% A few spacing fixes are applied. Since \pkg{minted} uses
% \pkg{fancyvrb}, these negative \cs{vspace}s are derived from the
% \pkg{fancyvrb} documentation \parencite[pp.~46--47]{Rahtz10}. What
% we want to do is to offset the spacing produced by \pkg{minted}, so
% that we are in control.
%    \begin{macrocode}
            \bool_if:NF\g__skdoc_in_example_bool{
                \vspace*{-\topsep}
                \vspace*{-\partopsep}
                \vspace*{-\parskip}
            }
%    \end{macrocode}
% Now, the internal \pkg{minted} macro \cs{minted@pygmentize} is called
% to highlight and typeset the code, and the temporary file is removed.
%    \begin{macrocode}
            \minted@pygmentize{latex}%
            \DeleteFile{\jobname.pyg}%
%    \end{macrocode}
% This is followed by more space-offsetting.
%    \begin{macrocode}
            \vspace*{-\topsep}
            \vspace*{-\partopsep}
        }{
%    \end{macrocode}
% If we aren't using \pkg{minted}, we hack spaces with \cs{@esphack}
% instead.
%    \begin{macrocode}
            \bool_if:NF\g__skdoc_in_example_bool{\@esphack}%
        }
    }%
}
%    \end{macrocode}
% \end{environment}
% \begin{environment}{MacroCode}[1]
%   {The key of the file being described}
% \changes{1.1}{Minor changes due to syntax highlighting}
% \changes{1.3b}{Offset negative spacing from the description environments}
%    \begin{macrocode}
\DeclareDocumentEnvironment{MacroCode}{m}{
    \__skdoc_if_print_code:T{
        \bool_if:NT\g__skdoc_negative_space_bool{\vspace{\baselineskip}}
        \vspace{.2\baselineskip}
        \bool_if:NF\g__skdoc_use_minted_bool{\par\noindent}
    }
    \skdoc@verbatim{#1}
}{
    \endskdoc@verbatim
    \__skdoc_if_print_code:T{
        \vspace{.5\baselineskip}
    }
}
%    \end{macrocode}
% \end{environment}
%
% \subsubsection{Reading a preamble}
% \begin{macro}{\PreambleTo}[2]
%   {A token to which we will save the preamble}
%   {File to read the preamble from}
% Read preamble from a document and store in variable.
%    \begin{macrocode}
\DeclareDocumentCommand\PreambleTo{mm}{%
    \group_begin:
    \msg_info:nnn{skdoc}{read-preamble}{#2}
    \ior_open:Nn\skdoc@input{#2}
    \bool_until_do:nn{\ior_if_eof_p:N\skdoc@input}{%
        \tl_if_empty:NTF\skdoc@temptl{}{%
            \tl_put_right:Nx\skdoc@temptl{\iow_newline:}
        }
        \tl_clear:N\l_tmpb_tl
        \ior_get_str:NN\skdoc@input\l_tmpa_tl
        \tl_put_right:Nx\l_tmpb_tl{\tl_head:N\l_tmpa_tl}
        \IfStrEq{\tl_to_str:N\l_tmpb_tl}{\@percentchar}{%
            \tl_set_eq:NN\l_tmpb_tl\skdoc@temptl
            \tl_concat:NNN\skdoc@temptl\l_tmpb_tl\l_tmpa_tl
        }{%
            \ior_close:N\skdoc@input
        }
    }
    \xdef#1{\tl_to_str:N\skdoc@temptl}
    \group_end:
}
%    \end{macrocode}
% \end{macro}
% \begin{macro}{\SelfPreambleTo}[1]
%   {A token to which we will save the preamble}
% Shorthand to read preamble from current document.
%    \begin{macrocode}
\DeclareDocumentCommand\SelfPreambleTo{m}{%
    \PreambleTo{#1}{\c_job_name_tl}%
}
%    \end{macrocode}
% \end{macro}
%
% \subsection{Styling}
% \subsubsection{Colors}
% First, we redefine a couple of colors from \pkg{ydoc} as well as
% defining a couple for ourselves.
%    \begin{macrocode}
\definecolorset{RGB}{}{}{
    section,11,72,107;
    extlink,73,10,61;
    intlink,140,35,24;
    sharp,250,105,0;
    bright,198,229,217;
    macrodesc,73,10,61;
    keydesc,140,35,24;
    macroimpl,73,10,61;
    meta,11,72,107;
    scriptcolor,140,35,24;
    optioncolor,73,10,61;
    opt,73,10,61
}
%    \end{macrocode}
% \subsubsection{Fonts}
% Then we redefine a couple of the KOMA-script font commands to use
% our newly defined colors, along with other fixes.
%    \begin{macrocode}
\RenewDocumentCommand\descfont{}{\sffamily\fontseries{sb}}
\RenewDocumentCommand\sectfont{}{\sffamily\fontseries{sb}}
\addtokomafont{minisec}{\bfseries}
\addtokomafont{paragraph}{\color{section}}
\addtokomafont{section}{\color{section}}
\addtokomafont{subsection}{\color{section}}
\addtokomafont{subsubsection}{\color{section}}
\addtokomafont{descriptionlabel}{\color{section}}
\addtokomafont{sectionentry}{\rmfamily\bfseries}
\addtokomafont{sectionentrypagenumber}{\rmfamily\bfseries}
%    \end{macrocode}
% \subsubsection{Configuring \pkg{hyperref}}
% Finally, we set up \pkg{hyperref} to also use our colors.
%    \begin{macrocode}
\hypersetup{
    colorlinks=true,
    linkcolor=intlink,
    anchorcolor=intlink,
    citecolor=black,
    urlcolor=extlink
}
%    \end{macrocode}
%
% \subsection{Documentation macros}
% We can now start defining the documentation macros.
%
% \subsubsection{Inline referencing}
% We introduce a couple of macros for referencing various constructs
% in running text, \emph{i.e.} \cs{cs}-like macros. The starred
% variants will not index the use, the non-starred variants will.
% \begin{macro}{\cs}[2]
%   {True if this is the starred variant}
%   {The macro name to be typeset}
% The \cs{cs} macro typesets a macro.
%    \begin{macrocode}
\DeclareDocumentCommand\cs{sm}{
    \texttt{\char`\\#2}
    \bool_if:NF\g__skdoc_no_index_bool{
        \IfBooleanTF{#1}{}{\index@macro{#2}}
    }
}
%    \end{macrocode}
% \end{macro}
% \begin{macro}{\env}[2]
%   {True if this is the starred variant}
%   {The environment name to be typeset}
% The \cs{env} macro typesets an environment.
%    \begin{macrocode}
\DeclareDocumentCommand\env{sm}{
    \texttt{#2}
    \bool_if:NF\g__skdoc_no_index_bool{
        \IfBooleanTF{#1}{}{\index@environment{#2}}
    }
}
%    \end{macrocode}
% \end{macro}
% \begin{macro}{\pkg}[2]
%   {True if this is the starred variant}
%   {The package name to be typeset}
% The \cs{pkg} macro typesets a package.
%    \begin{macrocode}
\DeclareDocumentCommand\pkg{sm}{
    \textsf{#2}
    \bool_if:NF\g__skdoc_no_index_bool{
        \IfBooleanTF{#1}{}{\index@package{#2}}
    }
}
%    \end{macrocode}
% \end{macro}
% \begin{macro}{\opt}[2]
%   {True if this is the starred variant}
%   {The option name to be typeset}
% The \cs{opt} macro typesets an option
%    \begin{macrocode}
\DeclareDocumentCommand\opt{sm}{
    \texttt{#2}
    \bool_if:NF\g__skdoc_no_index_bool{
        \IfBooleanTF{#1}{}{\index@option{#2}}
    }
}
%    \end{macrocode}
% \end{macro}
% \begin{macro}{\bib}[2]
%   {True if this is the starred variant}
%   {The \hologo{BibTeX} entry type name to be typeset}
% The \cs{bib} macro typesets a \hologo{BibTeX} entry type.
%    \begin{macrocode}
\DeclareDocumentCommand\bib{sm}{
    \texttt{@#2}
    \bool_if:NF\g__skdoc_no_index_bool{
        \IfBooleanTF{#1}{}{\index@bibentry{#2}}
    }
}
%    \end{macrocode}
% \end{macro}
% \begin{macro}{\thm}[2]
%   {True if this is the starred variant}
%   {The theme name to be typeset}
% The \cs{thm} macro typesets a theme of some sort.
%    \begin{macrocode}
\DeclareDocumentCommand\thm{sm}{
    \textrm{#2}
    \bool_if:NF\g__skdoc_no_index_bool{
        \IfBooleanTF{#1}{}{\index@theme{#2}}
    }
}
%    \end{macrocode}
% \end{macro}
% \begin{macro}{\file}[2]
%   {True if this is the starred variant}
%   {The file name to be typeset}
% The \cs{file} macro typesets a file name.
%    \begin{macrocode}
\DeclareDocumentCommand\file{sm}{
    \texttt{#2}
    \bool_if:NF\g__skdoc_no_index_bool{
        \IfBooleanTF{#1}{}{\index@file{#2}}
    }
}
%    \end{macrocode}
% \end{macro}
%
% \subsubsection{Descriptional macros}
% A range of descriptional macros are made available by the \pkg{ydoc}
% package, but we need to redefine and extend them.
%
% We begin with extending.
% \begin{macro}{\Describe@Macro}[1]
%   {The macro name, including leading backslash}
% The \cs{Describe@Macro} macro is changed to typeset its agument in
% a \cs{marginnote} instead of an \cs{fbox}.
%    \begin{macrocode}
\def\Describe@Macro#1{%
    \endgroup
    \edef\name{\expandafter\@gobble\string#1}%
    \global\@namedef{href@desc@\name}{}%
    \immediate\write\@mainaux{%
        \global\noexpand\@namedef{href@desc@\name}{}%
    }%
    \hbox\y@bgroup
    \@ifundefined{href@impl@\name}{}{\hyperlink{impl:\name}}%
    {%
        \hbox{
            \vbox to 0pt{
                \vss\hbox{
                    \raisebox{4ex}{\hypertarget{desc:\name}{}
                }
            }
        }%
        \marginnote{\llap{\PrintMacroName{#1}}}
        }%
    }%
    \ydoc@macrocatcodes
    \macroargsstyle
    \read@Macro@arg
}
%    \end{macrocode}
% \end{macro}
% \begin{macro}{\descframe}[1]
%   {Contents to be framed}
% Similarly, \cs{descframe} is changed to produce an \cs{mbox}
% instead of an \cs{fbox}.
%    \begin{macrocode}
\def\descframe#1{%
    \mbox{\hspace*{1.5\descsep}#1\hspace*{2\descsep}}%
}
%    \end{macrocode}
% \end{macro}
% \begin{macro}{\PrintMacroName}[1]
%   {Macro name to be printed}
% \cs{PrintMacroName} is hooked to also index the macro name while
% printing it.
%    \begin{macrocode}
\let\old@PrintMacroName\PrintMacroName
\DeclareDocumentCommand\PrintMacroName{m}{%
    \index@macro*{\expandafter\@gobble\string#1}
    \old@PrintMacroName{#1}
}
%    \end{macrocode}
% \end{macro}
% \begin{macro}{\PrintEnvName}[2]
%   {Either \cs{end} or \cs{begin}}
%   {Name of the environment to be printed}
% Similarly to \cs{PrintMacroName}, the \cs{PrintEnvName} is hooked
% to index the environment when printing the \cs{begin} part of the
% environment.
%    \begin{macrocode}
\let\old@PrintEnvName\PrintEnvName
\def\PrintEnvName#1#2{
    \ifx#1\begin
        \edef\skdoc@temp{\noexpand\index@environment*{#2}}
        \skdoc@temp
    \fi
    \old@PrintEnvName{#1}{#2}
}
%    \end{macrocode}
% \end{macro}
% \begin{macro}{\DescribeEnv}[2]
%   {Environment body or \cs{NoValue}}
%   {Environment name}
%    \begin{macrocode}
\DeclareDocumentCommand\DescribeEnv{om}{
    \medskip\par\noindent
    \marginnote{
        \envcodestyle
        \hfill\llap{\PrintEnvName{\begin}{#2}}\mbox{}\\
        \IfNoValueTF{#1}{}{\hfill\llap{\MacroArgs#1}\mbox{}\\}
        \hfill\llap{\PrintEnvName{\end}{#2}}\mbox{}\\
    }
    \begingroup
    \def\after@Macro@args{\endDescribeEnv}
    \y@bgroup
    \ydoc@macrocatcodes
    \macroargsstyle
    \read@Macro@arg
}
\DeclareDocumentCommand\endDescribeEnv{}{
    \endgroup
    \smallskip\par\noindent
}
%    \end{macrocode}
% \end{macro}
%
% Then we add a few of our own. For instance, we add macros to
% typeset descriptions of options. We also undefine the \cs{optpar}
% macro defined by \pkg{ydoc}, since we supersede it with \cs{Option}.
%    \begin{macrocode}
\let\optpar\relax
%    \end{macrocode}
% \begin{macro}{\Options}[1]
%   {A comma-separated list of options}
%    \begin{macrocode}
\DeclareDocumentCommand\Options{m}{
    \clist_set:Nn\l_tmpa_clist{#1}
    \marginnote{
        \clist_map_inline:Nn\l_tmpa_clist{
            \index@option*{####1}
            \hfill
            \llap{\textcolor{opt}{\opt{####1}}}
            \mbox{}\\
        }
    }
    \nobreak
}
%    \end{macrocode}
% \end{macro}
% \begin{macro}{\Option}[1]
%   {And option}
%    \begin{macrocode}
\DeclareDocumentCommand\Option{m}{
    \Options{#1}
}
%    \end{macrocode}
% \end{macro}
% \begin{macro*}{\skdoc@WithValues@peek}
% \changes{1.2a}{Add \cs{ignorespaces} to fix spacing bug}
%    \begin{macrocode}
\def\skdoc@WithValues@peek{
    \ifx\@let@token\AndDefault\else\par\noindent\nobreak\ignorespaces\fi
}
%    \end{macrocode}
% \end{macro*}
% \begin{macro}{\WithValues}[1]
%   {Values of a key-value option}
% The \cs{WithValues} macro peeks ahead to see if there's an
% \cs{AndDefault} macro further down. It typesets the values of 
% a key-vaue option
%    \begin{macrocode}
\DeclareDocumentCommand\WithValues{m}{
    \noindent\makebox[\linewidth][l]{\texttt{#1}}
    \futurelet\@let@token\skdoc@WithValues@peek
}
%    \end{macrocode}
% \end{macro}
% \begin{macro}{\AndDefault}[1]
%   {The value of a key-value option}
% Typesets the default value of a key-value option. To the far
% right of the line.
% \changes{1.2a}{Add \cs{ignorespaces} to fix spacing bug}
%    \begin{macrocode}
\DeclareDocumentCommand\AndDefault{m}{
    \llap{\textcolor{gray}{\texttt{(#1)}}}\par\noindent\nobreak\ignorespaces
}
%    \end{macrocode}
% \end{macro}
%
% Similar macros are provided for \hologo{BibTeX} entries.
% \begin{macro}{\BibEntry}[1]
%   {The name of the \hologo{BibTeX} entry}
%    \begin{macrocode}
\DeclareDocumentCommand\BibEntry{m}{
    \marginnote{
        \index@bibentry*{#1}
        \hfill\llap{\textcolor{macrodesc}{\bib{#1}}}
    }
    \nobreak
}
%    \end{macrocode}
% \end{macro}
% \begin{macro}{\WithFields}[2]
%   {Optional fields}
%   {Mandatory fields}
% \changes{1.2a}{Add \cs{ignorespaces} to fix spacing bug}
%    \begin{macrocode}
\DeclareDocumentCommand\WithFields{om}{
    \noindent\makebox[\linewidth]{
        \texttt{#2}
        \IfNoValueTF{#1}{}{
            \textcolor{gray}{\texttt{,#2}}
        }
    }
    \par\noindent\nobreak\ignorespaces
}
%    \end{macrocode}
% \end{macro}
%
% A macro for describing themes is supplied.
% \begin{macro}{\Theme}[1]
%   {The theme name}
% \changes{1.2a}{Add \cs{ignorespaces} to fix spacing bug}
%    \begin{macrocode}
\DeclareDocumentCommand\Theme{m}{
    \marginnote{
        \index@theme*{#1}
        \hfill\llap{\textcolor{macrodesc}{\thm{#1}}}
    }
    \nobreak\par\noindent\nobreak\ignorespaces
}
%    \end{macrocode}
% \end{macro}
%
% And finally, a macro for describing files is provided.
% \begin{macro}{\DescribeFile}[1]
%   {The filename}
% \changes{1.2a}{Add \cs{ignorespaces} to fix spacing bug}
%    \begin{macrocode}
\DeclareDocumentCommand\DescribeFile{m}{
    \marginnote{
        \index@file*{#1}
        \hfill\llap{\textcolor{macrodesc}{\file{#1}}}
    }
    \nobreak\par\noindent\nobreak\ignorespaces
}
%    \end{macrocode}
% \end{macro}
%
% Lastly, we define an envionment for showing examples. It's quite
% complex, utilizing (and kind of hacking) \cs{skdoc@verbatim} and
% typesetting the content using \pkg{l3coffins}.
%
% First, three dimensions used in constructing the side-by-side boxes.
% \begin{macro}{\c__skdoc_example_margin_dim}
% This dimension, set to half of \cs{baselineskip}, dictates the margin
% between the content of the coffins and the dividing ruler.
%    \begin{macrocode}
\dim_const:Nn\c__skdoc_example_margin_dim{0.5\baselineskip}
%    \end{macrocode}
% \end{macro}
% \begin{macro}{\c__skdoc_example_linewidth_dim}
% This dimension sets the width of the dividing ruler.
%    \begin{macrocode}
\dim_const:Nn\c__skdoc_example_linewidth_dim{1pt}
%    \end{macrocode}
% \end{macro}
%
% Next, three coffins used by the \env{example} environment are defined.
% One for code, one for the divider and one for the result.
% \begin{macro}{\l__skdoc_example_code_coffin}
%    \begin{macrocode}
\coffin_new:N\l__skdoc_example_code_coffin
%    \end{macrocode}
% \end{macro}
% \begin{macro}{\l__skdoc_example_divider_coffin}
%    \begin{macrocode}
\coffin_new:N\l__skdoc_example_divider_coffin
%    \end{macrocode}
% \end{macro}
% \begin{macro}{\l__skdoc_example_result_coffin}
%    \begin{macrocode}
\coffin_new:N\l__skdoc_example_result_coffin
%    \end{macrocode}
% \end{macro}
%
% Finally, we can define the actual \env{example} environment!
% \begin{environment}{example}
% \changes{1.2}{Print example code next to the result}
%    \begin{macrocode}
\DeclareDocumentEnvironment{example}{}{
%    \end{macrocode}
% The environment starts by priting a small header indicating that
% the content is in fact an example. It also sets a couple of counters
% and IO variables to trick \cs{skdoc@verbatim} into thinking that it's
% actually writing to a defined output file.
%    \begin{macrocode}
    \bool_gset_true:N\g__skdoc_in_example_bool%
    \minisec{Example:}%
    \int_zero_new:c{skdoc@output@skdoc@private@example@line}%
    \tl_if_exist:cTF{skdoc@output@skdoc@private@example}{
        \tl_clear:c{skdoc@output@skdoc@private@example}
    }{
        \tl_new:c{skdoc@output@skdoc@private@example}
    }
%    \end{macrocode}
% Next, we calculate the coffin width. We do this locally, not in a
% global constant, since we may be somewhere where \cs{textwidth}
% might have changed.
%    \begin{macrocode}
    \dim_set:Nn\l_tmpa_dim{ \textwidth/2
                           -\c__skdoc_example_margin_dim
                           -\c__skdoc_example_linewidth_dim/2}
%    \end{macrocode}
% The code coffin is now filled, after clearing it. It'll get filled
% with the contents of the verbatim environment, typeset like other
% \env{MacroCode} environments in the document (\emph{i.e.} \pkg{minted}
% if possible).
%    \begin{macrocode}
    \coffin_clear:N\l__skdoc_example_code_coffin
    \vcoffin_set:Nnw\l__skdoc_example_code_coffin{\l_tmpa_dim}
    \skdoc@verbatim{skdoc@private@example}
}{
    \endskdoc@verbatim
    \vcoffin_set_end:
%    \end{macrocode}
% Next comes the example result coffin. After clearing it, it gets
% filled by writing the contents of the token list defined previously
% (the one \cs{skdoc@verbatim} is tricked into writing to) to a
% temporary file, then reading that temporary file using \cs{input}.
% We leave the temporary file behind; cleaning up requires \cs{write18}.
% Also: Meh.
%    \begin{macrocode}
    \coffin_clear:N\l__skdoc_example_result_coffin
    \vcoffin_set:Nnw\l__skdoc_example_result_coffin{\l_tmpa_dim}
    \iow_open:Nn\skdoc@output{\jobname.skdoc.tmp}
    \iow_now:Nx\skdoc@output{\tl_to_str:c{skdoc@output@skdoc@private@example}}
    \iow_close:N\skdoc@output
    \input{\jobname.skdoc.tmp}
    \vcoffin_set_end:
%    \end{macrocode}
% The divider coffin, being dependent on the height of the two previous
% coffins, is also reset every time. After clearing, we calculate the
% maximum height of the two coffins, and add two margins (top and
% bottom). This value is used to typeset a (gray) rule of the width
% specified earlier (by \cs{c__skdoc_example_linewidth_dim}). Margins
% are also added to either side of this rule.
%    \begin{macrocode}
    \coffin_clear:N\l__skdoc_example_divider_coffin
    \dim_set:Nn\l_tmpa_dim{
        \dim_max:nn{\coffin_ht:N\l__skdoc_example_code_coffin}%
                   {\coffin_ht:N\l__skdoc_example_result_coffin}
        + 2\c__skdoc_example_margin_dim}
    \hcoffin_set:Nn\l__skdoc_example_divider_coffin{
        \color{lightgray}
        \hspace*{\c__skdoc_example_margin_dim}
        \rule{\c__skdoc_example_linewidth_dim}{\l_tmpa_dim}
        \hspace*{\c__skdoc_example_margin_dim}
    }
%    \end{macrocode}
% It's finally time to join and typeset the coffins. We clear a temporary
% coffin, copy the divider into it, and proceed to attach the example
% result coffin to the left and the code to the right (vertical centers
% touching, so that they are \enquote{centered} vertically). The
% temporary coffin is then typeset.
%    \begin{macrocode}
    \coffin_clear:N\l_tmpa_coffin
    \coffin_set_eq:NN\l_tmpa_coffin\l__skdoc_example_divider_coffin
    \coffin_join:NnnNnnnn\l_tmpa_coffin{l}{vc}%
                         \l__skdoc_example_result_coffin{r}{vc}%
                         {0pt}{0pt}
    \coffin_join:NnnNnnnn\l_tmpa_coffin{r}{vc}%
                         \l__skdoc_example_code_coffin{l}{vc}%
                         {0pt}{0pt}
    \coffin_typeset:Nnnnn\l_tmpa_coffin{T}{l}{0pt}{0pt}
    \bool_gset_false:N\g__skdoc_in_example_bool%
    \vspace*{\c__skdoc_example_margin_dim}\par
}
%    \end{macrocode}
% \end{environment}
%
% \subsubsection{Implementation environment}
% We define environments that encase the implementation of macros,
% environments, options, \hologo{BibTeX} entry types and themes.
% Watch out---there's a lot of duplicate code here.
% \begin{environment}{macro}[3]
%   {True if this is the starred variant}
%   {Name of the macro being implemented}
%   {If given, the number of arguments that
%            \cs{macro@impl@args} will read}
%    \begin{macrocode}
\DeclareDocumentEnvironment{macro}{smo}{%
    \IfBooleanTF{#1}{}{\@bsphack}
    \index@macro!{\expandafter\@gobble\string#2}
    \@macroname{#2}%
    \IfBooleanTF{#1}{
        \IfNoValueTF{#3}{}{
            \int_compare:nNnTF{#3}>{0}{
                \use:c{use_none:\prg_replicate:nn{#3}{n}}
            }{}
        }
    }{
        \PrintMacroImpl{#2}
        \IfNoValueTF{#3}{
            \macro@impl@argline@noarg{(no~arguments)}
        }{\macro@impl@args[#3]}
    }%
}{
    \let\skdoc@macroname@key\@empty
    \IfBooleanTF{#1}{}{\par\@esphack}
}
%    \end{macrocode}
% \end{environment}
% \begin{environment}{environment}[3]
%   {True if this is the starred variant}
%   {Name of the environment being implemented}
%   {If given, the number of arguments that
%            \cs{macro@impl@args} will read}
%    \begin{macrocode}
\DeclareDocumentEnvironment{environment}{smo}{%
    \IfBooleanTF{#1}{}{\@bsphack}
    \index@environment!{#2}
    \@environmentname{#2}%
    \IfBooleanTF{#1}{
        \IfNoValueTF{#3}{}{
            \int_compare:nNnTF{#3}>{0}{
                \use:c{use_none:\prg_replicate:nn{#3}{n}}
            }{}
        }
    }{
        \PrintEnvImplName{#2}
        \IfNoValueTF{#3}{
            \macro@impl@argline@noarg{(no~arguments)}
        }{\macro@impl@args[#3]}
    }%
}{
    \let\skdoc@macroname@key\@empty
    \IfBooleanTF{#1}{}{\par\@esphack}
}
%    \end{macrocode}
% \end{environment}
% \begin{environment}{option}[3]
%   {True if this is the starred variant}
%   {Name of the option being implemented}
%   {Values this key-value option can take}
%    \begin{macrocode}
\DeclareDocumentEnvironment{option}{smg}{%
    \IfBooleanTF{#1}{}{\@bsphack}
    \index@option!{#2}
    \@optionname{#2}%
    \IfBooleanTF{#1}{}{
        \vspace{\baselineskip}
        \PrintEnvImplName{#2}
        \IfNoValueTF{#3}{
            \macro@impl@argline@noarg{(option)}
        }{
            \macro@impl@argline@noarg{
                Option~with~values~\texttt{\textcolor{gray}{#3}}
            }
        }
    }%
}{
    \let\skdoc@macroname@key\@empty
    \IfBooleanTF{#1}{}{\par\@esphack}
}
%    \end{macrocode}
% \end{environment}
% \begin{environment}{bibentry}[2]
%   {True if this is the starred variant}
%   {Name of the \hologo{BibTeX} entry type being implemented}
%    \begin{macrocode}
\DeclareDocumentEnvironment{bibentry}{sm}{%
    \IfBooleanTF{#1}{}{\@bsphack}
    \index@bibentry!{\expandafter\@gobble\string#2}
    \@bibentryname{#2}%
    \IfBooleanTF{#1}{}{
        \vspace{\baselineskip}
        \PrintEnvImplName{#2}
        \macro@impl@argline@noarg{(\hologoRobust{BibTeX}~entry~type)}
    }%
}{
    \let\skdoc@macroname@key\@empty
    \IfBooleanTF{#1}{}{\par\@esphack}
}
%    \end{macrocode}
% \end{environment}
% \begin{environment}{theme}[2]
%   {True if this is the starred variant}
%   {Name of the theme being implemented}
%    \begin{macrocode}
\DeclareDocumentEnvironment{theme}{sm}{%
    \IfBooleanTF{#1}{}{\@bsphack}
    \index@theme!{#2}
    \@themename{#2}%
    \IfBooleanTF{#1}{}{
        \vspace{\baselineskip}
        \PrintEnvImplName{#2}
        \macro@impl@argline@noarg{(theme)}
    }%
}{
    \let\skdoc@macroname@key\@empty
    \IfBooleanTF{#1}{}{\par\@esphack}
}
%    \end{macrocode}
% \end{environment}
% We also provide starred variants of the environments, which will
% add the implementation to the index but not print anything.
% \begin{environment}{macro*}[2]
%   {Name of the macro being implemented}
%   {If given, the number of arguments that
%            \cs{macro@impl@args} will read}
%    \begin{macrocode}
\DeclareDocumentEnvironment{macro*}{mo}%
    {\begin{macro}*{#1}[#2]}{\end{macro}}
%    \end{macrocode}
% \end{environment}
% \begin{environment}{environment*}[2]
%   {Name of the environment being implemented}
%   {If given, the number of arguments that
%            \cs{macro@impl@args} will read}
%    \begin{macrocode}
\DeclareDocumentEnvironment{environment*}{mo}%
    {\begin{environment}*{#1}[#2]}{\end{environment}}
%    \end{macrocode}
% \end{environment}
% \begin{environment}{option*}[2]
%   {Name of the option being implemented}
%   {Values this key-value option can take}
%    \begin{macrocode}
\DeclareDocumentEnvironment{option*}{mg}%
    {\begin{option}*{#1}{#2}}{\end{option}}
%    \end{macrocode}
% \end{environment}
% \begin{environment}{bibentry*}[1]
%   {Name of the \hologo{BibTeX} entry type being implemented}
%    \begin{macrocode}
\DeclareDocumentEnvironment{bibentry*}{m}%
    {\begin{bibentry}*{#1}}{\end{bibentry}}
%    \end{macrocode}
% \end{environment}
% \begin{environment}{theme*}[1]
%   {Name of the theme being implemented}
%    \begin{macrocode}
\DeclareDocumentEnvironment{theme*}{m}%
    {\begin{theme}*{#1}}{\end{theme}}
%    \end{macrocode}
% \end{environment}
% And finally, we redefine some of the underlying \pkg{ydoc} macros
% to behave the way we want them to.
% For instance, we redefine the commands that print environment and
% macro implementation names so that they typeset the name i a
% \cs{marginnote} rather than in an \cs{fbox}.
% \begin{macro}{\PrintEnvImplName}[1]
%   {The environment name to be printed}
% \changes{1.3}{Fixed incorrect spacing}
%    \begin{macrocode}
\def\PrintEnvImplName#1{%
    \par\leavevmode\null
    \hbox{%
        \marginnote{\llap{\implstyle{#1\strut}}}%
    }
    \null
}
%    \end{macrocode}
% \end{macro}
% \begin{macro}{\PrintMacroImpl}[1]
%   {The macro name to be printed}
%    \begin{macrocode}
\def\PrintMacroImpl#1{%
    \par
    \hbox{%
        \edef\name{\expandafter\@gobble\string#1}%
        \global\@namedef{href@impl@\name}{}%
        \immediate\write\@mainaux{%
            \global\noexpand\@namedef{href@impl@\name}{}%
        }%
        \raisebox{4ex}[4ex]{\hypertarget{impl:\name}{}}%
        \@ifundefined{href@desc@\name}{}{\hyperlink{desc:\name}}%
        \marginnote{\llap{\PrintMacroImplName{#1}}}%
    }%
    \par
}
%    \end{macrocode}
% \end{macro}
% We also redefine the utility macros belonging to \cs{macro@impl@arg}.
% \begin{macro*}{\macro@impl@argline}[2]
%   {The argument number}
%   {Description of the argument}
%    \begin{macrocode}
\def\macro@impl@argline#1#2{%
    \par\noindent{\texttt{\##1}:~#2\strut}%
}
%    \end{macrocode}
% \end{macro*}
% \begin{macro*}{\macro@impl@args}[1]
%   {The number of arguments to read}
%    \begin{macrocode}
\def\macro@impl@args[#1]{%
    \vspace*{-\baselineskip}
    \begingroup
    \let\macro@impl@argcnt\@tempcnta
    \let\macro@impl@curarg\@tempcntb
    \macro@impl@argcnt=#1\relax
    \macro@impl@curarg=0\relax
    \ifnum\macro@impl@curarg<\macro@impl@argcnt\relax
        \expandafter\macro@impl@arg
    \else
        \expandafter\macro@impl@endargs
    \fi
}
%    \end{macrocode}
% \end{macro*}
% \begin{macro*}{\macro@impl@arg@noopt}
%    \begin{macrocode}
\def\macro@impl@arg@noopt#1{%
    \macro@impl@argline{\the\macro@impl@curarg}{#1}
    \ifnum\macro@impl@curarg<\macro@impl@argcnt\relax
        \expandafter\macro@impl@arg
    \else
        \expandafter\macro@impl@endargs
    \fi
}
%    \end{macrocode}
% \end{macro*}
% \begin{macro*}{\macro@impl@arg@opt}
%    \begin{macrocode}
\def\macro@impl@arg@opt#1[#2]{%
    \macro@impl@argline{\the\macro@impl@curarg}
        {#1\hfill\AndDefault{#1}\vspace{-\baselineskip}}
    \ifnum\macro@impl@curarg<\macro@impl@argcnt\relax
        \expandafter\macro@impl@arg
    \else
        \expandafter\macro@impl@endargs
    \fi
}
%    \end{macrocode}
% \end{macro*}
% \begin{macro*}{\macro@impl@arg}
%    \begin{macrocode}
\def\macro@impl@arg#1{%
    \advance\macro@impl@curarg by\@ne\relax
    \@ifnextchar[%]
        {\macro@impl@arg@opt{#1}}%
        {\macro@impl@arg@noopt{#1}}%
}
%    \end{macrocode}
% \end{macro*}
% \begin{macro*}{\macro@impl@endargs}
% \changes{1.3b}{Conditionally don't add space if another environment follows}
%    \begin{macrocode}
\def\macro@impl@endargs{
    \endgroup
    \peek_meaning_ignore_spaces:NTF\begin
        {
            \vspace{-\baselineskip}
            \bool_gset_true:N\g__skdoc_negative_space_bool
            \PushPreHook{par}{
                \bool_gset_false:N\g__skdoc_negative_space_bool
                \PopPreHook{par}
            }
        }
        {\par\medskip}%
}
%    \end{macrocode}
% \end{macro*}
% \begin{macro*}{\macro@impl@argline@noarg}[1]
%   {The line to print instead of an argument line}
% \changes{1.3b}{Conditionally don't add space if another environment follows}
% This last macro is a replacement used when there are no arguments
% or if the implementation is an option or something like that. It
% behaves pretty much like \cs{macro@impl@args}, but with only one
% argument to read.
%    \begin{macrocode}
\def\macro@impl@argline@noarg#1{%
    \vspace*{-\baselineskip}
    \par\noindent{#1\strut}
    \peek_meaning_ignore_spaces:NTF\begin
        {
            \vspace{-\baselineskip}
            \bool_gset_true:N\g__skdoc_negative_space_bool
            \PushPreHook{par}{
                \bool_gset_false:N\g__skdoc_negative_space_bool
                \PopPreHook{par}
            }
        }
        {\par\medskip}%
}
%    \end{macrocode}
% \end{macro*}
%
% \subsection{The index}
% \begin{macro*}{\__skdoc_if_do_index:T}
% \begin{macro*}{\__skdoc_if_do_index:F}
% \begin{macro*}{\__skdoc_if_do_index:TF}
% \begin{macro}{\__skdoc_if_do_index_p:}
%    \begin{macrocode}
\prg_new_conditional:Nnn\__skdoc_if_do_index:{p,T,F,TF}{
    \bool_if:nTF{
        \__skdoc_if_print_code_p: &&
        !\g__skdoc_no_index_bool
    }{
        \prg_return_true:
    }{
        \prg_return_false:
    }
}
%    \end{macrocode}
% \end{macro}
% \end{macro*}
% \end{macro*}
% \end{macro*}
% The index is based on \pkg{glossaries}, and as such the whole
% process of adding entries to the index is reduced to adding
% glossary entries. This is done through two wrapper macros around
% the \cs{newglossaryentry} macro.
% \begin{macro}{\@index@}[1]
%   {The key of the index entry}
%   {The text of the index entry}
% What \cs{@index@} does is to decide wether we are hiding the
% implementation part of the documentation (discussed later), and
% wether we are in the actual implementation or not. If we are in
% the implementation and aren't printing it, we shouldn't add an
% index entry.
%    \begin{macrocode}
\DeclareDocumentCommand\@index@{mm}{
    \__skdoc_if_do_index:T{
        \@index@@{#1}{#2}
    }
}
%    \end{macrocode}
% \end{macro}
% \begin{macro}{\@index@@}[2]
%   {The key of the index entry}
%   {The text of the index entry}
% This macro does the actual adding to the glossary.
%    \begin{macrocode}
\DeclareDocumentCommand\@index@@{mm}{
    \newglossaryentry{index-#1}{
        type=index,
        name={#2},
        description={\nopostdesc},
        sort={#1}
    }
}
%    \end{macrocode}
% \end{macro}
%
% \subsubsection{Adding index entries}
% These macros add an index entry with different contents depending
% on the thing (macro, environment, etc.) that is being indexed. They
% all have non-starred variants which are used by the referring
% macros (\cs{cs} \emph{et. al}), and starred variants used by the 
% description macros (the star affects the style of the page number).
% Each environment first test wether the given entry key exists, and
% defines a new entry if it doesn't. Then, a usage of the entry is
% recorded.
% There is also a exclamation variant that is used by the implementation
% environments, that typesets a normal use of the entity.
%
% Note the duplicate use of \cs{ifglsentryexists} --- this is needed
% since \cs{@index@} doesn't always add the entity to the index
% \emph{i.e.} nothing in the implementation is added when we're hiding
% the implementation.
% \begin{macro}{\index@macro}[3]
%   {True if this is the starred variant}
%   {True if this is the exclamation variant}
%   {The name of the macro being indexed, without backslash}
%    \begin{macrocode}
\DeclareDocumentCommand\index@macro{st!m}{
    \def\skdoc@temp{#3-macro}
    \ifglsentryexists{index-\skdoc@temp}{}{
        \@index@{#3-macro}{\cs*{#3}}
    }
    \__skdoc_if_do_index:T{
        \IfBooleanTF{#2}{
            \glsadd[types=index,format=hyperul]{index-\skdoc@temp}
        }{
            \IfBooleanTF{#1}{
                \glsadd[types=index,format=hyperit]{index-\skdoc@temp}
            }{
                \glsadd[types=index]{index-\skdoc@temp}
            }
        }
    }
}
%    \end{macrocode}
% \end{macro}
% \begin{macro}{\index@environment}[3]
%   {True if this is the starred variant}
%   {True if this is the exclamation variant}
%   {The name of the environment being indexed}
%    \begin{macrocode}
\DeclareDocumentCommand\index@environment{st!m}{
    \def\skdoc@temp{\string#3-env}
    \ifglsentryexists{index-\skdoc@temp}{}{
        \@index@{\string#3-env}{\env*{\string#3}~(environment)}
    }
    \__skdoc_if_do_index:T{
        \IfBooleanTF{#2}{
            \glsadd[types=index,format=hyperul]{index-\skdoc@temp}
        }{
            \IfBooleanTF{#1}{
                \glsadd[types=index,format=hyperit]{index-\skdoc@temp}
            }{
                \glsadd[types=index]{index-\skdoc@temp}
            }
        }
    }
}
%    \end{macrocode}
% \end{macro}
% \begin{macro}{\index@option}[3]
%   {True if this is the starred variant}
%   {True if this is the exclamation variant}
%   {The name of the option being indexed}
%    \begin{macrocode}
\DeclareDocumentCommand\index@option{st!m}{
    \def\skdoc@temp{\string#3-opt}
    \ifglsentryexists{index-\skdoc@temp}{}{
        \@index@{\string#3-opt}{\opt*{\string#3}~(option)}
    }
    \__skdoc_if_do_index:T{
        \IfBooleanTF{#2}{
            \glsadd[types=index,format=hyperul]{index-\skdoc@temp}
        }{
            \IfBooleanTF{#1}{
                \glsadd[types=index,format=hyperit]{index-\skdoc@temp}
            }{
                \glsadd[types=index]{index-\skdoc@temp}
            }
        }
    }
}
%    \end{macrocode}
% \end{macro}
% \begin{macro}{\index@bibentry}[3]
%   {True if this is the starred variant}
%   {True if this is the exclamation variant}
%   {The name of the \hologo{BibTeX} entry type
%       being indexed, without initial \texttt{@} sign}
%    \begin{macrocode}
\DeclareDocumentCommand\index@bibentry{st!m}{
    \def\skdoc@temp{#3-bib}
    \ifglsentryexists{index-\skdoc@temp}{}{
        \@index@{#3-bib}{\bib*{#3}~(\hologoRobust{BibTeX}~entry~type)}
    }
    \__skdoc_if_do_index:T{
        \IfBooleanTF{#2}{
            \glsadd[types=index,format=hyperul]{index-\skdoc@temp}
        }{
            \IfBooleanTF{#1}{
                \glsadd[types=index,format=hyperit]{index-\skdoc@temp}
            }{
                \glsadd[types=index]{index-\skdoc@temp}
            }
        }
    }
}
%    \end{macrocode}
% \end{macro}
% \begin{macro}{\index@theme}[3]
%   {True if this is the starred variant}
%   {True if this is the exclamation variant}
%   {The name of the theme being indexed}
%    \begin{macrocode}
\DeclareDocumentCommand\index@theme{st!m}{
    \def\skdoc@temp{\string#3-theme}
    \ifglsentryexists{index-\skdoc@temp}{}{
        \@index@{\string#3-theme}{\thm*{\string#3}~(theme)}
    }
    \__skdoc_if_do_index:T{
        \IfBooleanTF{#2}{
            \glsadd[types=index,format=hyperul]{index-\skdoc@temp}
        }{
            \IfBooleanTF{#1}{
                \glsadd[types=index,format=hyperit]{index-\skdoc@temp}
            }{
                \glsadd[types=index]{index-\skdoc@temp}
            }
        }
    }
}
%    \end{macrocode}
% \end{macro}
% \begin{macro}{\index@package}[3]
%   {True if this is the starred variant}
%   {True if this is the exclamation variant}
%   {The name of the package being indexed}
%    \begin{macrocode}
\DeclareDocumentCommand\index@package{st!m}{
    \def\skdoc@temp{\string#3-pkg}
    \ifglsentryexists{index-\skdoc@temp}{}{
        \@index@{\string#3-pkg}{\pkg*{\string#3}~(package)}
    }
    \__skdoc_if_do_index:T{
        \IfBooleanTF{#2}{
            \glsadd[types=index,format=hyperul]{index-\skdoc@temp}
        }{
            \IfBooleanTF{#1}{
                \glsadd[types=index,format=hyperit]{index-\skdoc@temp}
            }{
                \glsadd[types=index]{index-\skdoc@temp}
            }
        }
    }
}
%    \end{macrocode}
% \end{macro}
% \begin{macro}{\index@file}[3]
%   {True if this is the starred variant}
%   {True if this is the exclamation variant}
%   {The name of the file being indexed}
%    \begin{macrocode}
\DeclareDocumentCommand\index@file{st!m}{
    \def\skdoc@temp{\string#3-file}
    \ifglsentryexists{index-\skdoc@temp}{}{
        \@index@{\string#3-file}{\file*{\string#3}~(file)}
    }
    \__skdoc_if_do_index:T{
        \IfBooleanTF{#2}{
            \glsadd[types=index,format=hyperul]{index-\skdoc@temp}
        }{
            \IfBooleanTF{#1}{
                \glsadd[types=index,format=hyperit]{index-\skdoc@temp}
            }{
                \glsadd[types=index]{index-\skdoc@temp}
            }
        }
    }
}
%    \end{macrocode}
% \end{macro}
%
% Notice the references to \cs{hyperul}? We need to define that as
% well. It's simple enough. Note that \cs{GlsAddXdyAttribute} isn't
% called until later.
% \begin{macro}{\hyperul}[1]
%   {Word to underline and link}
%    \begin{macrocode}
\ProvideDocumentCommand\hyperul{m}{
    \underline{\hyperup{#1}}
}
%    \end{macrocode}
% \end{macro}
%
% \subsubsection{Displaying the index}
% Displaying the index is very simple. We begin by defining
% our own \pkg{glossaries} style.
%    \begin{macrocode}
\newglossarystyle{docindex}{
    \glossarystyle{indexgroup}
    \renewcommand*{\glspostdescription}{\unskip\leaders\hbox to 2.9mm{\hss.}\hfill\strut}
    \renewenvironment{theglossary}{
        \bool_gset_true:N\g__skdoc_no_index_bool
        \begin{multicols}{2}
        \setlength{\parindent}{0pt}
        \setlength{\parskip}{0pt plus 0.3pt}
        \let\item\@idxitem
    }{
        \end{multicols}
        \bool_gset_false:N\g__skdoc_no_index_bool
    }
    \renewcommand*{\glossaryentryfield}[5]{
        \item\glsentryitem{##1}\glstarget{##1}{##2}
            \ifx\relax##4\relax\else\space(##4)\fi
            ##3\glspostdescription\space ##5}
    \renewcommand*{\glsgroupheading}[1]{
        \IfStrEq{##1}{default}{
            \item{\descfont\glssymbolsgroupname}
        }{
            \item{\descfont\glsgetgrouptitle{##1}}
        }
    }
    \renewcommand*{\glsgroupskip}{
        \par\vspace{15\p@}\relax
    }
}
%    \end{macrocode}
% We follow that up by defining the actual glossay, and making sure
% to run \cs{makeglossaries} when the preamble is complete.
%    \begin{macrocode}
\newglossary{index}{ids}{ido}{Index}
\AtBeginDocument{\makeglossaries}
%    \end{macrocode}
% \begin{macro}{\PrintIndex}
% \changes{1.2}{Fixed incorrect reference to boldfaced text}
% Finally, we define a command \cs{PrintIndex} that prints the index.
% Note the preamble that describes how the index page numbers work.
%    \begin{macrocode}
\providecommand*\PrintIndex{%
    \begingroup
    \renewcommand*{\glossarypreamble}{
        Numbers~written~in~italic~refer~to~the~page~where~the~
        corresponding~entry~is~described;~numbers~underlined~refer~
        to~the~page~were~the~implementation~of~the~corresponding~
        entry~is~discussed.~Numbers~in~roman~refer~to~other~
        mentions~of~the~entry.\par
    }
    \printglossary[type=index,style=docindex]
    \endgroup
}
%    \end{macrocode}
% \end{macro}
%
% \subsubsection{Hacking \pkg{glossaries}}
% The following redefinition of an internal \pkg{glossaries} macro,
% provided by \textcite{Talbot13}, makes sure that the underlined
% and italic page numbers in the index have precedence over the plain
% nubmer format. In the event that a macro is described and implemented
% on the same page, the description format (italic) is used.
% \begin{macro}{\@gls@addpredefinedattributes}
% \changes{1.4a}{Add Xdy attribute \texttt{@gobble}}
%    \begin{macrocode}
\RenewDocumentCommand\@gls@addpredefinedattributes{}{
    \GlsAddXdyAttribute{hyperit}
    \GlsAddXdyAttribute{hyperul}
    \GlsAddXdyAttribute{glsnumberformat}
    \GlsAddXdyAttribute{@gobble}
}
%    \end{macrocode}
% \end{macro}
%
% \subsection{The changelog}
% The changelog is implemented as a glossary using the
% \pkg{glossaries} package. We begin by defining a name
% for general changes, and commands that save the name of
% the current macro, environment or similar for use by the
% \cs{changes} macro.
% \begin{macro}{\generalname}
%    \begin{macrocode}
\DeclareDocumentCommand\generalname{}{General}
%    \end{macrocode}
% \end{macro}
% \begin{macro*}{\@macroname}[1]
%   {Name of the macro being described}
%    \begin{macrocode}
\DeclareDocumentCommand\@macroname{m}{
    \def\skdoc@macroname@stylized{\cs*{\expandafter\@gobble\string#1}}
    \def\skdoc@macroname@key{macro-\expandafter\@gobble\string#1}
}
%    \end{macrocode}
% \end{macro*}
% \begin{macro*}{\@environmentname}[1]
%   {Name of the environment being described}
%    \begin{macrocode}
\DeclareDocumentCommand\@environmentname{m}{
    \def\skdoc@macroname@stylized{\env*{\string#1}}
    \def\skdoc@macroname@key{env-#1}
}
%    \end{macrocode}
% \end{macro*}
% \begin{macro*}{\@optionname}[1]
%   {Name of the option being described}
%    \begin{macrocode}
\DeclareDocumentCommand\@optionname{m}{
    \def\skdoc@macroname@stylized{\opt*{\string#1}}
    \def\skdoc@macroname@key{opt-#1}
}
%    \end{macrocode}
% \end{macro*}
% \begin{macro*}{\@ebibentryname}[1]
%   {Name of the \hologo{BibTeX} entry being described}
%    \begin{macrocode}
\DeclareDocumentCommand\@bibentryname{m}{
    \def\skdoc@macroname@stylized{\bib*{\expandafter\@gobble\string#1}}
    \def\skdoc@macroname@key{bibentry-\expandafter\@gobble\string#1}
}
%    \end{macrocode}
% \end{macro*}
% \begin{macro*}{\@themename}[1]
%   {Name of the theme being described}
%    \begin{macrocode}
\DeclareDocumentCommand\@themename{m}{
    \def\skdoc@macroname@stylized{\thm*{\string#1}}
    \def\skdoc@macroname@key{thm-#1}
}
%    \end{macrocode}
% \end{macro*}
% Along with these we also define the variables they affect as empty.
%    \begin{macrocode}
\def\skdoc@macroname@stylized{}
\let\skdoc@macroname@key\@empty
%    \end{macrocode}
%
% \subsubsection{Adding changes}
% Since the changelog is based on \pkg{glossaries}, adding changes
% amounts to simply adding a glossary entry.
% \begin{macro}{\changes}[2]
%   {The version in which the changes were made}
%   {A short description of the changes}
% \changes{1.4a}{Unconditionally add version \enquote{parent} to
%   circumvent strange behaviour by \pkg{glossaries}. Also, gobble
%   page numbers for similar reasons}
%    \begin{macrocode}
\DeclareDocumentCommand\changes{mm}{%
    \@bsphack
    \newglossaryentry{#1}{
        type=changes,
        name={v#1},
        description={\nopostdesc},
        nonumberlist=true
    }
    \ifx\skdoc@macroname@key\@empty
        \newglossaryentry{#1-general}{
            type=changes,
            description={\generalname{}:~#2},
            parent={#1},
            sort={0},
            nonumberlist=true
        }
        \glsadd[types=changes,format=@gobble]{#1-general}
    \else
        \newglossaryentry{#1-\skdoc@macroname@key}{
            type=changes,
            description={\skdoc@macroname@stylized{}:~#2},
            parent={#1},
            sort={\skdoc@macroname@key},
            nonumberlist=true
        }
        \glsadd[types=changes,format=@gobble]{#1-\skdoc@macroname@key}
    \fi
    \@esphack
}
%    \end{macrocode}
% \end{macro}
%
% \subsubsection{Displaying the changelog}
% Displaying the changelog is equally simple. We begin by defining
% our own \pkg{glossaries} style.
%    \begin{macrocode}
\newglossarystyle{changelog}{
    \glossarystyle{altlist}
    \renewenvironment{theglossary}{
        \bool_gset_true:N\g__skdoc_no_index_bool
        \begin{multicols}{2}\begin{description}
    }{
        \end{description}\end{multicols}
        \bool_gset_false:N\g__skdoc_no_index_bool
    }
    \renewcommand*{\glossaryentryfield}[5]{
        \par\vspace{5\p@}\relax
        \item[\glsentryitem{##1}\glstarget{##1}{##2}]
                    \mbox{}\par\nobreak\@afterheading
    }
    \renewcommand{\glossarysubentryfield}[6]{%
        \par\hspace*{\itemindent}
        \glssubentryitem{##2}%
        \glstarget{##2}{\strut}##4\glspostdescription\space ##6
    }
}
%    \end{macrocode}
% We follow that up by defining the actual glossary, and making sure
% to run \cs{makeglossaries} when the preamble is complete.
%    \begin{macrocode}
\newglossary{changes}{gls}{glo}{Changes}
\AtBeginDocument{\makeglossaries}
%    \end{macrocode}
% \begin{macro}{\PrintChanges}
% Finally, we define a command \cs{PrintChanges} that prints the
% list of changes.
%    \begin{macrocode}
\DeclareDocumentCommand\PrintChanges{}{%
    \begingroup
    \printglossary[type=changes,style=changelog]
    \endgroup
}
%    \end{macrocode}
% \end{macro}
%
% \subsection{Hiding the implementation}
% We define commands to hide the implementation from the documentation.
% Here, the ``implementation'' is understood to be everything between
% the \cs{Implementation} and \cs{Finale} macros. What we do is disable
% and/or reset page and section counters for the duration of the
% implementation, and set a shipout hook that simply discards the pages
% while we are in the implementation. A consquence of this is that we
% must force a page break between what's before the implementation and
% what's after, which might look horrible.
%
% We define a counter in which we save the page number we had when
% the implementation started.
%    \begin{macrocode}
\newcounter{skdoc@impl@page}
%    \end{macrocode}
% Then we define the shipout hook. Fairly straight-forward.
%    \begin{macrocode}
\AtBeginShipout{
    \__skdoc_if_print_code:F{\AtBeginShipoutDiscard}
}
%    \end{macrocode}
% \begin{macro}{\OnlyDescription}
% The \cs{OnlyDescription} macro simply toggles the appropriate toggle.
%    \begin{macrocode}
\DeclareDocumentCommand\OnlyDescription{}{
    \bool_gset_false:N\g__skdoc_with_implementation_bool
}
%    \end{macrocode}
% \end{macro}
% \begin{macro}{\Implementation}
% \changes{1.1a}{Hide references used in the hidden implementation}
% The \cs{Implementation} macro defines all the sectioning commands
% to be empty (saving the old ones), clears the page, saves the page
% number and toggles the appropriate toggle. If \pkg{biblatex} is
% loaded, we start a new \env{refsection} so that we can hide
% references used in the implementation from the final bibliography.
%    \begin{macrocode}
\DeclareDocumentCommand\Implementation{}{
    \bool_if:NF\g__skdoc_with_implementation_bool{
        \clearpage
        \bool_gset_true:N\g__skdoc_in_implementation_bool
        \let\skdoc@old@part\part
        \DeclareDocumentCommand\part{som}{}
        \let\skdoc@old@section\section
        \DeclareDocumentCommand\section{som}{}
        \let\skdoc@old@subsection\subsection
        \DeclareDocumentCommand\subsection{som}{}
        \let\skdoc@old@subsubsection\subsubsection
        \DeclareDocumentCommand\subsubsection{som}{}
        \let\skdoc@old@paragraph\paragraph
        \DeclareDocumentCommand\paragraph{som}{}
        \let\skdoc@old@subparagraph\subparagraph
        \DeclareDocumentCommand\subparagraph{som}{}
        \cs_if_exist:NTF\refsection{\refsection}{}
        \setcounter{skdoc@impl@page}{\value{page}}
    }
}
%    \end{macrocode}
% \end{macro}
% \begin{macro}{\Finale}
% \changes{1.1a}{Hide references used in the hidden implementation}
% The \cs{Finale} macro basically just undoes what the 
% \cs{Implementation} macro did. If \pkg{biblatex} is used, the
% \env{refsection} environment is ended and the (local) bibliography
% is printed.
%    \begin{macrocode}
\DeclareDocumentCommand\Finale{}{
    \bool_if:NF\g__skdoc_with_implementation_bool{
        \cs_if_exist:NTF\refsection{\printbibliography\endrefsection}{}
        \clearpage
        \bool_gset_false:N\g__skdoc_in_implementation_bool
        \let\part\skdoc@old@part
        \let\section\skdoc@old@section
        \let\subsection\skdoc@old@subsection
        \let\subsubsection\skdoc@old@subsubsection
        \let\paragraph\skdoc@old@paragraph
        \let\subparagraph\skdoc@old@subparagraph
        \setcounter{page}{\value{skdoc@impl@page}}
    }
}
%    \end{macrocode}
% \end{macro}
%
% \subsection{Document metadata}
% \subsubsection{Setting metadata}
% We override a bunch of the general titlepage macros and add a few
% of our own. First, we initialize the underlying variables.
%    \begin{macrocode}
\let\@ctan\@empty%
\let\@repository\@empty%
\let\@plainemail\@empty%
\let\@email\@empty%
\let\@version\@empty%
%    \end{macrocode}
% Then, we define the actual macros.
% \begin{macro}{\package}[2]
%   {A list of key-value options}
%   {The package name}
% The \cs{package} macro sets the package name of the documentation.
% The key-value options are \opt{vcs} and \opt{ctan}.
%    \begin{macrocode}
\DeclareDocumentCommand\package{om}{%
    \keys_define:nn{skdoc@package}{%
        vcs .value_required:,%
        vcs .code:n = \repository{##1},%
        ctan .code:n = \ctan{##1},%
        ctan .default:n = #2%
    }%
    \IfNoValueTF{#1}{}{\keys_set:nn{skdoc@package}{#1}}%
    \def\@package{#2}%
    \title{The~\textbf{\pkg*{\@package}}~package}%
}
%    \end{macrocode}
% \end{macro}
% \begin{macro}{\ctan}[1]
%   {The name of a package or bundle on CTAN}
%    \begin{macrocode}
\DeclareDocumentCommand\ctan{m}{%
    \def\@ctan{\url{http://www.ctan.org/pkg/#1}}%
}
%    \end{macrocode}
% \end{macro}
% \begin{macro}{\repository}[1]
%   {The URI of an online repository}
%    \begin{macrocode}
\DeclareDocumentCommand\repository{m}{%
    \def\@repository{\url{#1}}%
}
%    \end{macrocode}
% \end{macro}
% \begin{macro}{\email}[1]
%   {The email address of the author}
%    \begin{macrocode}
\DeclareDocumentCommand\email{m}{%
    \def\@plainemail{#1}%
    \def\@email{\href{mailto:\@plainemail}{\@plainemail}}%
}
%    \end{macrocode}
% \end{macro}
% \begin{macro}{\version}[1]
%   {The version of the package, with no leading ``v''}
%    \begin{macrocode}
\DeclareDocumentCommand\version{m}{%
    \def\@version{#1}%
}
%    \end{macrocode}
% \end{macro}
% Finally, we set the default package name to \cs{jobname}.
%    \begin{macrocode}
\package{\jobname}
%    \end{macrocode}
%
% \subsubsection{Using metadata}
% We define two macros that read useful metadata; \cs{theversion}
% and \cs{thepackage}. These are used internally by \cs{maketitle}.
% \begin{macro}{\theversion}
%    \begin{macrocode}
\DeclareDocumentCommand\theversion{}{v\@version}
%    \end{macrocode}
% \end{macro}
% \begin{macro}{\thepackage}
%    \begin{macrocode}
\DeclareDocumentCommand\thepackage{}{\pkg*{\@package}}
%    \end{macrocode}
% \end{macro}
% \begin{macro}{\thepkg}
%    \begin{macrocode}
\DeclareDocumentCommand\thepkg{}{\thepackage}
%    \end{macrocode}
% \end{macro}
% Additionally we define \cs{skdocpdfsettings}, which is also used
% by \cs{maketitle}, to include PDF metadata if the documentation
% is being compiled using \hologo{pdfLaTeX}.
%    \begin{macrocode}
\ifpdf
    \def\skdocpdfsettings{%
        \hypersetup{%
            pdfauthor   = {\@author\space<\@plainemail>},
            pdftitle    = {\@title},
            pdfsubject  = {Documentation~of~LaTeX~package~\@package},
            pdfkeywords = {\@package,~LaTeX,~TeX}
        }%
    }%
\else
    \let\skdocpdfsettings\empty%
\fi
%    \end{macrocode}
%
% \subsection{General document commands}
% Most of the general document commands are defined by the
% \pkg{scrartcl} document class we base ourselves on, but a few
% of them have to be redefined.
%
% \subsubsection{Notices and warnings}
% We define macros to typeset notices and warnings in the documentation
% text. Notices are typeset as \Notice{this is a notice}, and warnings
% are typeset as follows: \Warning{this is a warning}
% A macro for longer warnings is also available:
% \LongWarning{This is a longer warning.}
%
% \begin{macro}{\Notice}[1]
%   {Notice text}
%    \begin{macrocode}
\DeclareDocumentCommand\Notice{m}{
    (\emph{\textbf{Note:}~#1})
}
%    \end{macrocode}
% \end{macro}
% \begin{macro}{\Warning}[1]
%   {Warning text}
%    \begin{macrocode}
\DeclareDocumentCommand\Warning{+m}{
    \vspace{\baselineskip}
    \par\noindent
    \fbox{\begin{minipage}[c]{\textwidth}
        \centering
        \textbf{Warning:}~#1
    \end{minipage}}
    \vspace{\baselineskip}
    \par
}
%    \end{macrocode}
% \end{macro}
% \begin{macro}{\LongWarning}[1]
%   {Warning text}
%    \begin{macrocode}
\DeclareDocumentCommand\LongWarning{+m}{
    \Warning{
        \par\noindent
        \begin{minipage}{\textwidth}
            #1
        \end{minipage}
    }
}
%    \end{macrocode}
% \end{macro}
%
% \subsubsection{The title page}
% The title page consists of the \cs{maketitle} and the \env{abstract}.
% We redefine both, inspired slightly by the Prac\TeX\ journal and the
% \pkg{skrapport} document class.
% \begin{macro}{\@maketitle}
%    \begin{macrocode}
\def\@maketitle{%
    \newpage
    \null
    \begin{flushleft}%
    {%
        \Huge\sectfont\@title%
        \ifx\@ctan\@empty\else%
            \footnote{Available~on~\@ctan.}%
        \fi
        \ifx\@repository\@empty\else%
            \footnote{Development~version~available~on~\@repository.}%
        \fi%
        \par%
    }%
    \vskip 1em
    {%
        \Large\@author
        \ifx\@email\@empty\else%
            \space
            \newlength\skdoc@minipage@ew%
            \settowidth{\skdoc@minipage@ew}{%
                \normalsize{$\lceil${\@email}$\rfloor$}}
            \begin{minipage}[b]{\skdoc@minipage@ew}
                \normalsize{$\lceil${\@email}$\rfloor$}
            \end{minipage}\par%
        \fi%
    }%
    \ifx\@version\@empty\else
        \vskip .5em
        {%
            \large Version~\@version\par%
        }%
    \fi
    \end{flushleft}%
    \par\bigskip%
}
%    \end{macrocode}
% \end{macro}
% \begin{macro}{\maketitle}
%    \begin{macrocode}
\def\maketitle{%
    \begingroup
    \skdocpdfsettings
    \renewcommand\thefootnote{\@fnsymbol\c@footnote}
    \@maketitle
    \setcounter{footnote}{0}
    \skdocpdfsettings
    \endgroup
}
%    \end{macrocode}
% \end{macro}
% \begin{environment}{abstract}
%    \begin{macrocode}
\DeclareDocumentEnvironment{abstract}{}{
    \newlength\skdoc@abstract@tw%
    \newlength\skdoc@abstract@aw%
    \settowidth{\skdoc@abstract@tw}{\descfont\abstractname}%
    \setlength{\skdoc@abstract@aw}{\the\textwidth-\the\skdoc@abstract@tw-2em}%
    \begin{minipage}{\textwidth}
        \begin{minipage}[t]{\skdoc@abstract@tw}%
            \begin{flushright}%
                \leavevmode\descfont\abstractname%
            \end{flushright}%
        \end{minipage}%
        \hspace{1em}%
        \begin{minipage}[t]{\skdoc@abstract@aw}%
}{
        \end{minipage}
    \end{minipage}
}
%    \end{macrocode}
% \end{environment}
%
% \subsubsection{Table of contents}
% The table of contents are redefined to imitate the excellent table
% of contents of the \pkg{microtype} manual.
% \begin{macro*}{\l@section}
%    \begin{macrocode}
\let\l@section@\l@section
\def\l@section{\vskip -.75ex\l@section@}
%    \end{macrocode}
% \end{macro*}
% \begin{macro*}{\l@subsection}
%    \begin{macrocode}
\def\l@subsection{\vskip.35ex\penalty\@secpenalty\@dottedtocline{2}{1.5em}{2.7em}}
%    \end{macrocode}
% \end{macro*}
% \begin{macro*}{l@subsubsection}
%    \begin{macrocode}
\def\l@subsubsection#1#2{
    \leftskip 4.2em
    \parindent 0pt
    {\let\numberline\@gobble{\small #1~[#2]}}
}
%    \end{macrocode}
% \end{macro*}
% \begin{macro*}{\l@table}
%    \begin{macrocode}
\def\l@table{\@dottedtocline{1}{0pt}{1.5em}}
%    \end{macrocode}
% \end{macro*}
% \begin{macro*}{\l@figure}
%    \begin{macrocode}
\def\l@figure{\@dottedtocline{1}{0pt}{1.5em}}
%    \end{macrocode}
% \end{macro*}
%    \begin{macrocode}
\def\@pnumwidth{1.7em}
\AtEndDocument{\addtocontents{toc}{\par}}
%    \end{macrocode}
% \begin{macro}{\tableofcontents}
%    \begin{macrocode}
\let\old@tableofcontents\tableofcontents
\DeclareDocumentCommand\tableofcontents{}{
    \bool_gset_true:N\g__skdoc_no_index_bool
    \microtypesetup{protrusion=false}
    \old@tableofcontents
    \microtypesetup{protrusion=true}
    \bool_gset_false:N\g__skdoc_no_index_bool
}
%    \end{macrocode}
% \end{macro}
%
% \subsubsection{Including the LPPL license}
% A helper macro that includes the LPPL license is also provided.
% \begin{macro}{\PrintLPPL}
%    \begin{macrocode}
\DeclareDocumentCommand\PrintLPPL{}{
    \begingroup
    \IfFileExists{lppl.tex}{
        \let\old@verbatim@font\verbatim@font
        \def\verbatim@font{
            \old@verbatim@font\tiny
        }
        \def\LPPLicense{\begingroup\small}
        \def\endLPPLicense{\endmulticols\endgroup}
        \DeclareDocumentCommand\LPPLsection{m}{
            \section{####1}
        }
        \DeclareDocumentCommand\skdoc@lppl@hack{m}{
            \end{multicols}
            \begin{multicols}{2}
                [\subsection*{####1}][6\baselineskip]
        }
        \DeclareDocumentCommand\LPPLsubsection{m}{
            \subsection*{####1}
            \let\LPPLsubsection\skdoc@lppl@hack
            \begin{multicols}{2}
        }
        \DeclareDocumentCommand\LPPLsubsubsection{m}{
            \subsubsection*{####1}
        }
        \DeclareDocumentCommand\LPPLparagraph{m}{\paragraph*{####1}}
        \DeclareDocumentCommand\LPPLfile{m}{\file{####1}}
        \let\oldmakeatletter\makeatletter
        \long\def\makeatletter####1\makeatother{
            \let\makeatletter\oldmakeatletter
        }
        \setlength\leftmargini{15pt}
        \setlength\leftmarginii{12.5pt}
        \setlength\leftmarginiii{10pt}
        \newenvironment{enum}[1][0]{
            \list\labelenumi{
                \usecounter{enumi}
                \setcounter{enumi}{####1}
                \addtocounter{enumi}{-1}
                \leftmargin 30pt
                \itemindent-15pt
                \labelwidth 15pt
                \labelsep 0pt
                \def\makelabel########1{########1\hss}}
        }{\endlist}
        %
% $Id: lppl.tex 5882 2008-05-04 17:28:59Z mittelba $
%
% Copyright 1999 2002-2011 LaTeX3 Project
%    Everyone is allowed to distribute verbatim copies of this
%    license document, but modification of it is not allowed.
%
%
% If you wish to load it as part of a ``doc'' source, you have to
% ensure that a) % is a comment character and b) that short verb
% characters are being turned off, i.e.,
%
%   \DeleteShortVerb{\'}   % or whatever was made a shorthand
%   \MakePercentComment
%   %
% $Id: lppl.tex 5882 2008-05-04 17:28:59Z mittelba $
%
% Copyright 1999 2002-2011 LaTeX3 Project
%    Everyone is allowed to distribute verbatim copies of this
%    license document, but modification of it is not allowed.
%
%
% If you wish to load it as part of a ``doc'' source, you have to
% ensure that a) % is a comment character and b) that short verb
% characters are being turned off, i.e.,
%
%   \DeleteShortVerb{\'}   % or whatever was made a shorthand
%   \MakePercentComment
%   %
% $Id: lppl.tex 5882 2008-05-04 17:28:59Z mittelba $
%
% Copyright 1999 2002-2011 LaTeX3 Project
%    Everyone is allowed to distribute verbatim copies of this
%    license document, but modification of it is not allowed.
%
%
% If you wish to load it as part of a ``doc'' source, you have to
% ensure that a) % is a comment character and b) that short verb
% characters are being turned off, i.e.,
%
%   \DeleteShortVerb{\'}   % or whatever was made a shorthand
%   \MakePercentComment
%   \input{lppl}
%   \MakePercentIgnore
%   \MakeShortVerb{\'}     % turn it on again if necessary
%
%
% By default the license is produced with \section* as the highest
% heading level. If this is not appropriate for the document in which
% it is included define the commands listed below before loading this
% document, e.g., for inclusion as a separate chapter define:
%
%  \providecommand{\LPPLsection}{\chapter*}
%  \providecommand{\LPPLsubsection}{\section*}
%  \providecommand{\LPPLsubsubsection}{\subsection*}
%  \providecommand{\LPPLparagraph}{\subsubsection*}
%
%
% To allow cross-referencing the headings \label's have been attached
% to them, all starting with ``LPPL:''. As by default headings without
% numbers are produced, this will only allow page references.
% However, you can use the titleref package to produce textual
% references or you change the definitions of \LPPLsection, and
% friends to generated numbered headings.
%
%
% We want it to be possible that this file can be processed by
% (pdf)LaTeX on its own, or that this file can be included in another
% LaTeX document without any modification whatsoever.
% Hence the little test below.
%
%
\makeatletter
\ifx\@preamblecmds\@notprerr
  % In this case the preamble has already been processed so this file
  % is loaded as part of another document; just enclose everything in
  % a group
  \let\LPPLicense\bgroup
  \let\endLPPLicense\egroup
\else
  % In this case the preamble has not been processed yet so this file
  % is processed by itself.
  \documentclass{article}
  \let\LPPLicense\document
  \let\endLPPLicense\enddocument
\fi
\makeatother


\begin{LPPLicense}
  \providecommand{\LPPLsection}{\section*}
  \providecommand{\LPPLsubsection}{\subsection*}
  \providecommand{\LPPLsubsubsection}{\subsubsection*}
  \providecommand{\LPPLparagraph}{\paragraph*}
  \providecommand*{\LPPLfile}[1]{\texttt{#1}}
  \providecommand*{\LPPLdocfile}[1]{`\LPPLfile{#1.tex}'}
  \providecommand*{\LPPL}{\textsc{lppl}}

  \LPPLsection{The \LaTeX\ Project Public License}
  \label{LPPL:LPPL}

  \emph{LPPL Version 1.3c  2008-05-04}

  \textbf{Copyright 1999, 2002--2008 \LaTeX3 Project}
  \begin{quotation}
    Everyone is allowed to distribute verbatim copies of this
    license document, but modification of it is not allowed.
  \end{quotation}

  \LPPLsubsection{Preamble}
  \label{LPPL:Preamble}

  The \LaTeX\ Project Public License (\LPPL) is the primary license
  under which the \LaTeX\ kernel and the base \LaTeX\ packages are
  distributed.

  You may use this license for any work of which you hold the
  copyright and which you wish to distribute.  This license may be
  particularly suitable if your work is \TeX-related (such as a
  \LaTeX\ package), but it is written in such a way that you can use
  it even if your work is unrelated to \TeX.

  The section `WHETHER AND HOW TO DISTRIBUTE WORKS UNDER THIS
  LICENSE', below, gives instructions, examples, and recommendations
  for authors who are considering distributing their works under this
  license.

  This license gives conditions under which a work may be distributed
  and modified, as well as conditions under which modified versions of
  that work may be distributed.

  We, the \LaTeX3 Project, believe that the conditions below give you
  the freedom to make and distribute modified versions of your work
  that conform with whatever technical specifications you wish while
  maintaining the availability, integrity, and reliability of that
  work.  If you do not see how to achieve your goal while meeting
  these conditions, then read the document \LPPLdocfile{cfgguide} and
  \LPPLdocfile{modguide} in the base \LaTeX\ distribution for suggestions.


  \LPPLsubsection{Definitions}
  \label{LPPL:Definitions}

  In this license document the following terms are used:

  \begin{description}
  \item[Work] Any work being distributed under this License.

  \item[Derived Work] Any work that under any applicable law is
    derived from the Work.

  \item[Modification] Any procedure that produces a Derived Work under
    any applicable law -- for example, the production of a file
    containing an original file associated with the Work or a
    significant portion of such a file, either verbatim or with
    modifications and/or translated into another language.

  \item[Modify] To apply any procedure that produces a Derived Work
    under any applicable law.

  \item[Distribution] Making copies of the Work available from one
    person to another, in whole or in part.  Distribution includes
    (but is not limited to) making any electronic components of the
    Work accessible by file transfer protocols such as \textsc{ftp} or
    \textsc{http} or by shared file systems such as Sun's Network File
    System (\textsc{nfs}).

  \item[Compiled Work] A version of the Work that has been processed
    into a form where it is directly usable on a computer system.
    This processing may include using installation facilities provided
    by the Work, transformations of the Work, copying of components of
    the Work, or other activities.  Note that modification of any
    installation facilities provided by the Work constitutes
    modification of the Work.

  \item[Current Maintainer] A person or persons nominated as such
    within the Work.  If there is no such explicit nomination then it
    is the `Copyright Holder' under any applicable law.

  \item[Base Interpreter] A program or process that is normally needed
    for running or interpreting a part or the whole of the Work.

    A Base Interpreter may depend on external components but these are
    not considered part of the Base Interpreter provided that each
    external component clearly identifies itself whenever it is used
    interactively.  Unless explicitly specified when applying the
    license to the Work, the only applicable Base Interpreter is a
    `\LaTeX-Format' or in the case of files belonging to the
    `\LaTeX-format' a program implementing the `\TeX{} language'.
  \end{description}

  \LPPLsubsection{Conditions on Distribution and Modification}
  \label{LPPL:Conditions}

  \begin{enumerate}
  \item Activities other than distribution and/or modification of the
    Work are not covered by this license; they are outside its scope.
    In particular, the act of running the Work is not restricted and
    no requirements are made concerning any offers of support for the
    Work.

  \item\label{LPPL:item:distribute} You may distribute a complete, unmodified
    copy of the Work as you received it.  Distribution of only part of
    the Work is considered modification of the Work, and no right to
    distribute such a Derived Work may be assumed under the terms of
    this clause.

  \item You may distribute a Compiled Work that has been generated
    from a complete, unmodified copy of the Work as distributed under
    Clause~\ref{LPPL:item:distribute} above, as long as that Compiled Work is
    distributed in such a way that the recipients may install the
    Compiled Work on their system exactly as it would have been
    installed if they generated a Compiled Work directly from the
    Work.

  \item\label{LPPL:item:currmaint} If you are the Current Maintainer of the
    Work, you may, without restriction, modify the Work, thus creating
    a Derived Work.  You may also distribute the Derived Work without
    restriction, including Compiled Works generated from the Derived
    Work.  Derived Works distributed in this manner by the Current
    Maintainer are considered to be updated versions of the Work.

  \item If you are not the Current Maintainer of the Work, you may
    modify your copy of the Work, thus creating a Derived Work based
    on the Work, and compile this Derived Work, thus creating a
    Compiled Work based on the Derived Work.

  \item\label{LPPL:item:conditions} If you are not the Current Maintainer
    of the
    Work, you may distribute a Derived Work provided the following
    conditions are met for every component of the Work unless that
    component clearly states in the copyright notice that it is exempt
    from that condition.  Only the Current Maintainer is allowed to
    add such statements of exemption to a component of the Work.
    \begin{enumerate}
    \item If a component of this Derived Work can be a direct
      replacement for a component of the Work when that component is
      used with the Base Interpreter, then, wherever this component of
      the Work identifies itself to the user when used interactively
      with that Base Interpreter, the replacement component of this
      Derived Work clearly and unambiguously identifies itself as a
      modified version of this component to the user when used
      interactively with that Base Interpreter.

    \item\label{LPPL:item:changelog} Every component of the Derived Work
      contains prominent
      notices detailing the nature of the changes to that component,
      or a prominent reference to another file that is distributed as
      part of the Derived Work and that contains a complete and
      accurate log of the changes.

    \item No information in the Derived Work implies that any persons,
      including (but not limited to) the authors of the original
      version of the Work, provide any support, including (but not
      limited to) the reporting and handling of errors, to recipients
      of the Derived Work unless those persons have stated explicitly
      that they do provide such support for the Derived Work.

    \item\label{LPPL:item:unmodifiedcopy} You distribute at least one of
      the following with the Derived Work:
      \begin{enumerate}
      \item A complete, unmodified copy of the Work; if your
        distribution of a modified component is made by offering
        access to copy the modified component from a designated place,
        then offering equivalent access to copy the Work from the same
        or some similar place meets this condition, even though third
        parties are not compelled to copy the Work along with the
        modified component;

      \item Information that is sufficient to obtain a complete,
        unmodified copy of the Work.
      \end{enumerate}
    \end{enumerate}
  \item If you are not the Current Maintainer of the Work, you may
    distribute a Compiled Work generated from a Derived Work, as long
    as the Derived Work is distributed to all recipients of the
    Compiled Work, and as long as the conditions of
    Clause~\ref{LPPL:item:conditions}, above, are met with regard to the
    Derived Work.

  \item The conditions above are not intended to prohibit, and hence
    do not apply to, the modification, by any method, of any component
    so that it becomes identical to an updated version of that
    component of the Work as it is distributed by the Current
    Maintainer under Clause~\ref{LPPL:item:currmaint}, above.

  \item Distribution of the Work or any Derived Work in an alternative
    format, where the Work or that Derived Work (in whole or in part)
    is then produced by applying some process to that format, does not
    relax or nullify any sections of this license as they pertain to
    the results of applying that process.

  \item
    \begin{enumerate}
    \item A Derived Work may be distributed under a different license
      provided that license itself honors the conditions listed in
      Clause~\ref{LPPL:item:conditions} above, in regard to the Work, though it
      does not have to honor the rest of the conditions in this
      license.

    \item If a Derived Work is distributed under a different license,
      that Derived Work must provide sufficient documentation as part
      of itself to allow each recipient of that Derived Work to honor
      the restrictions in Clause~\ref{LPPL:item:conditions} above, concerning
      changes from the Work.
    \end{enumerate}
  \item This license places no restrictions on works that are
    unrelated to the Work, nor does this license place any
    restrictions on aggregating such works with the Work by any means.

  \item Nothing in this license is intended to, or may be used to,
    prevent complete compliance by all parties with all applicable
    laws.
  \end{enumerate}

  \LPPLsubsection{No Warranty}
  \label{LPPL:Warranty}

  There is no warranty for the Work.  Except when otherwise stated in
  writing, the Copyright Holder provides the Work `as is', without
  warranty of any kind, either expressed or implied, including, but
  not limited to, the implied warranties of merchantability and
  fitness for a particular purpose.  The entire risk as to the quality
  and performance of the Work is with you.  Should the Work prove
  defective, you assume the cost of all necessary servicing, repair,
  or correction.

  In no event unless required by applicable law or agreed to in
  writing will The Copyright Holder, or any author named in the
  components of the Work, or any other party who may distribute and/or
  modify the Work as permitted above, be liable to you for damages,
  including any general, special, incidental or consequential damages
  arising out of any use of the Work or out of inability to use the
  Work (including, but not limited to, loss of data, data being
  rendered inaccurate, or losses sustained by anyone as a result of
  any failure of the Work to operate with any other programs), even if
  the Copyright Holder or said author or said other party has been
  advised of the possibility of such damages.

  \LPPLsubsection{Maintenance of The Work}
  \label{LPPL:Maintenance}

  The Work has the status `author-maintained' if the Copyright Holder
  explicitly and prominently states near the primary copyright notice
  in the Work that the Work can only be maintained by the Copyright
  Holder or simply that it is `author-maintained'.

  The Work has the status `maintained' if there is a Current
  Maintainer who has indicated in the Work that they are willing to
  receive error reports for the Work (for example, by supplying a
  valid e-mail address). It is not required for the Current Maintainer
  to acknowledge or act upon these error reports.

  The Work changes from status `maintained' to `unmaintained' if there
  is no Current Maintainer, or the person stated to be Current
  Maintainer of the work cannot be reached through the indicated means
  of communication for a period of six months, and there are no other
  significant signs of active maintenance.

  You can become the Current Maintainer of the Work by agreement with
  any existing Current Maintainer to take over this role.

  If the Work is unmaintained, you can become the Current Maintainer
  of the Work through the following steps:
  \begin{enumerate}
  \item Make a reasonable attempt to trace the Current Maintainer (and
    the Copyright Holder, if the two differ) through the means of an
    Internet or similar search.
  \item If this search is successful, then enquire whether the Work is
    still maintained.
    \begin{enumerate}
    \item If it is being maintained, then ask the Current Maintainer
      to update their communication data within one month.

    \item\label{LPPL:item:intention} If the search is unsuccessful or
      no action to resume active maintenance is taken by the Current
      Maintainer, then announce within the pertinent community your
      intention to take over maintenance.  (If the Work is a \LaTeX{}
      work, this could be done, for example, by posting to
      \texttt{comp.text.tex}.)
    \end{enumerate}
  \item {}
    \begin{enumerate}
    \item If the Current Maintainer is reachable and agrees to pass
      maintenance of the Work to you, then this takes effect
      immediately upon announcement.

    \item\label{LPPL:item:announce} If the Current Maintainer is not
      reachable and the Copyright Holder agrees that maintenance of
      the Work be passed to you, then this takes effect immediately
      upon announcement.
    \end{enumerate}
  \item\label{LPPL:item:change} If you make an `intention
    announcement' as described in~\ref{LPPL:item:intention} above and
    after three months your intention is challenged neither by the
    Current Maintainer nor by the Copyright Holder nor by other
    people, then you may arrange for the Work to be changed so as to
    name you as the (new) Current Maintainer.

  \item If the previously unreachable Current Maintainer becomes
    reachable once more within three months of a change completed
    under the terms of~\ref{LPPL:item:announce}
    or~\ref{LPPL:item:change}, then that Current Maintainer must
    become or remain the Current Maintainer upon request provided they
    then update their communication data within one month.
  \end{enumerate}
  A change in the Current Maintainer does not, of itself, alter the
  fact that the Work is distributed under the \LPPL\ license.

  If you become the Current Maintainer of the Work, you should
  immediately provide, within the Work, a prominent and unambiguous
  statement of your status as Current Maintainer.  You should also
  announce your new status to the same pertinent community as
  in~\ref{LPPL:item:intention} above.

  \LPPLsubsection{Whether and How to Distribute Works under This License}
  \label{LPPL:Distribute}

  This section contains important instructions, examples, and
  recommendations for authors who are considering distributing their
  works under this license.  These authors are addressed as `you' in
  this section.

  \LPPLsubsubsection{Choosing This License or Another License}
  \label{LPPL:Choosing}

  If for any part of your work you want or need to use
  \emph{distribution} conditions that differ significantly from those
  in this license, then do not refer to this license anywhere in your
  work but, instead, distribute your work under a different license.
  You may use the text of this license as a model for your own
  license, but your license should not refer to the \LPPL\ or
  otherwise give the impression that your work is distributed under
  the \LPPL.

  The document \LPPLdocfile{modguide} in the base \LaTeX\ distribution
  explains the motivation behind the conditions of this license.  It
  explains, for example, why distributing \LaTeX\ under the
  \textsc{gnu} General Public License (\textsc{gpl}) was considered
  inappropriate.  Even if your work is unrelated to \LaTeX, the
  discussion in \LPPLdocfile{modguide} may still be relevant, and authors
  intending to distribute their works under any license are encouraged
  to read it.

  \LPPLsubsubsection{A Recommendation on Modification Without Distribution}
  \label{LPPL:WithoutDistribution}

  It is wise never to modify a component of the Work, even for your
  own personal use, without also meeting the above conditions for
  distributing the modified component.  While you might intend that
  such modifications will never be distributed, often this will happen
  by accident -- you may forget that you have modified that component;
  or it may not occur to you when allowing others to access the
  modified version that you are thus distributing it and violating the
  conditions of this license in ways that could have legal
  implications and, worse, cause problems for the community.  It is
  therefore usually in your best interest to keep your copy of the
  Work identical with the public one.  Many works provide ways to
  control the behavior of that work without altering any of its
  licensed components.

  \LPPLsubsubsection{How to Use This License}
  \label{LPPL:HowTo}

  To use this license, place in each of the components of your work
  both an explicit copyright notice including your name and the year
  the work was authored and/or last substantially modified.  Include
  also a statement that the distribution and/or modification of that
  component is constrained by the conditions in this license.

  Here is an example of such a notice and statement:
\begin{verbatim}
  %% pig.dtx
  %% Copyright 2005 M. Y. Name
  %
  % This work may be distributed and/or modified under the
  % conditions of the LaTeX Project Public License, either version 1.3
  % of this license or (at your option) any later version.
  % The latest version of this license is in
  %   http://www.latex-project.org/lppl.txt
  % and version 1.3 or later is part of all distributions of LaTeX
  % version 2005/12/01 or later.
  %
  % This work has the LPPL maintenance status `maintained'.
  %
  % The Current Maintainer of this work is M. Y. Name.
  %
  % This work consists of the files pig.dtx and pig.ins
  % and the derived file pig.sty.
\end{verbatim}

  Given such a notice and statement in a file, the conditions given in
  this license document would apply, with the `Work' referring to the
  three files `\LPPLfile{pig.dtx}', `\LPPLfile{pig.ins}', and
  `\LPPLfile{pig.sty}' (the last being generated from
  `\LPPLfile{pig.dtx}' using `\LPPLfile{pig.ins}'), the `Base
  Interpreter' referring to any `\LaTeX-Format', and both `Copyright
  Holder' and `Current Maintainer' referring to the person `M. Y.
  Name'.

  If you do not want the Maintenance section of \LPPL\ to apply to
  your Work, change `maintained' above into `author-maintained'.
  However, we recommend that you use `maintained' as the Maintenance
  section was added in order to ensure that your Work remains useful
  to the community even when you can no longer maintain and support it
  yourself.

  \LPPLsubsubsection{Derived Works That Are Not Replacements}
  \label{LPPL:NotReplacements}

  Several clauses of the \LPPL\ specify means to provide reliability
  and stability for the user community. They therefore concern
  themselves with the case that a Derived Work is intended to be used
  as a (compatible or incompatible) replacement of the original
  Work. If this is not the case (e.g., if a few lines of code are
  reused for a completely different task), then clauses
  \ref{LPPL:item:changelog} and \ref{LPPL:item:unmodifiedcopy}
  shall not apply.

  \LPPLsubsubsection{Important Recommendations}
  \label{LPPL:Recommendations}

  \LPPLparagraph{Defining What Constitutes the Work}

  The \LPPL\ requires that distributions of the Work contain all the
  files of the Work.  It is therefore important that you provide a way
  for the licensee to determine which files constitute the Work.  This
  could, for example, be achieved by explicitly listing all the files
  of the Work near the copyright notice of each file or by using a
  line such as:
\begin{verbatim}
    % This work consists of all files listed in manifest.txt.
\end{verbatim}
  in that place.  In the absence of an unequivocal list it might be
  impossible for the licensee to determine what is considered by you
  to comprise the Work and, in such a case, the licensee would be
  entitled to make reasonable conjectures as to which files comprise
  the Work.

\end{LPPLicense}
\endinput

%   \MakePercentIgnore
%   \MakeShortVerb{\'}     % turn it on again if necessary
%
%
% By default the license is produced with \section* as the highest
% heading level. If this is not appropriate for the document in which
% it is included define the commands listed below before loading this
% document, e.g., for inclusion as a separate chapter define:
%
%  \providecommand{\LPPLsection}{\chapter*}
%  \providecommand{\LPPLsubsection}{\section*}
%  \providecommand{\LPPLsubsubsection}{\subsection*}
%  \providecommand{\LPPLparagraph}{\subsubsection*}
%
%
% To allow cross-referencing the headings \label's have been attached
% to them, all starting with ``LPPL:''. As by default headings without
% numbers are produced, this will only allow page references.
% However, you can use the titleref package to produce textual
% references or you change the definitions of \LPPLsection, and
% friends to generated numbered headings.
%
%
% We want it to be possible that this file can be processed by
% (pdf)LaTeX on its own, or that this file can be included in another
% LaTeX document without any modification whatsoever.
% Hence the little test below.
%
%
\makeatletter
\ifx\@preamblecmds\@notprerr
  % In this case the preamble has already been processed so this file
  % is loaded as part of another document; just enclose everything in
  % a group
  \let\LPPLicense\bgroup
  \let\endLPPLicense\egroup
\else
  % In this case the preamble has not been processed yet so this file
  % is processed by itself.
  \documentclass{article}
  \let\LPPLicense\document
  \let\endLPPLicense\enddocument
\fi
\makeatother


\begin{LPPLicense}
  \providecommand{\LPPLsection}{\section*}
  \providecommand{\LPPLsubsection}{\subsection*}
  \providecommand{\LPPLsubsubsection}{\subsubsection*}
  \providecommand{\LPPLparagraph}{\paragraph*}
  \providecommand*{\LPPLfile}[1]{\texttt{#1}}
  \providecommand*{\LPPLdocfile}[1]{`\LPPLfile{#1.tex}'}
  \providecommand*{\LPPL}{\textsc{lppl}}

  \LPPLsection{The \LaTeX\ Project Public License}
  \label{LPPL:LPPL}

  \emph{LPPL Version 1.3c  2008-05-04}

  \textbf{Copyright 1999, 2002--2008 \LaTeX3 Project}
  \begin{quotation}
    Everyone is allowed to distribute verbatim copies of this
    license document, but modification of it is not allowed.
  \end{quotation}

  \LPPLsubsection{Preamble}
  \label{LPPL:Preamble}

  The \LaTeX\ Project Public License (\LPPL) is the primary license
  under which the \LaTeX\ kernel and the base \LaTeX\ packages are
  distributed.

  You may use this license for any work of which you hold the
  copyright and which you wish to distribute.  This license may be
  particularly suitable if your work is \TeX-related (such as a
  \LaTeX\ package), but it is written in such a way that you can use
  it even if your work is unrelated to \TeX.

  The section `WHETHER AND HOW TO DISTRIBUTE WORKS UNDER THIS
  LICENSE', below, gives instructions, examples, and recommendations
  for authors who are considering distributing their works under this
  license.

  This license gives conditions under which a work may be distributed
  and modified, as well as conditions under which modified versions of
  that work may be distributed.

  We, the \LaTeX3 Project, believe that the conditions below give you
  the freedom to make and distribute modified versions of your work
  that conform with whatever technical specifications you wish while
  maintaining the availability, integrity, and reliability of that
  work.  If you do not see how to achieve your goal while meeting
  these conditions, then read the document \LPPLdocfile{cfgguide} and
  \LPPLdocfile{modguide} in the base \LaTeX\ distribution for suggestions.


  \LPPLsubsection{Definitions}
  \label{LPPL:Definitions}

  In this license document the following terms are used:

  \begin{description}
  \item[Work] Any work being distributed under this License.

  \item[Derived Work] Any work that under any applicable law is
    derived from the Work.

  \item[Modification] Any procedure that produces a Derived Work under
    any applicable law -- for example, the production of a file
    containing an original file associated with the Work or a
    significant portion of such a file, either verbatim or with
    modifications and/or translated into another language.

  \item[Modify] To apply any procedure that produces a Derived Work
    under any applicable law.

  \item[Distribution] Making copies of the Work available from one
    person to another, in whole or in part.  Distribution includes
    (but is not limited to) making any electronic components of the
    Work accessible by file transfer protocols such as \textsc{ftp} or
    \textsc{http} or by shared file systems such as Sun's Network File
    System (\textsc{nfs}).

  \item[Compiled Work] A version of the Work that has been processed
    into a form where it is directly usable on a computer system.
    This processing may include using installation facilities provided
    by the Work, transformations of the Work, copying of components of
    the Work, or other activities.  Note that modification of any
    installation facilities provided by the Work constitutes
    modification of the Work.

  \item[Current Maintainer] A person or persons nominated as such
    within the Work.  If there is no such explicit nomination then it
    is the `Copyright Holder' under any applicable law.

  \item[Base Interpreter] A program or process that is normally needed
    for running or interpreting a part or the whole of the Work.

    A Base Interpreter may depend on external components but these are
    not considered part of the Base Interpreter provided that each
    external component clearly identifies itself whenever it is used
    interactively.  Unless explicitly specified when applying the
    license to the Work, the only applicable Base Interpreter is a
    `\LaTeX-Format' or in the case of files belonging to the
    `\LaTeX-format' a program implementing the `\TeX{} language'.
  \end{description}

  \LPPLsubsection{Conditions on Distribution and Modification}
  \label{LPPL:Conditions}

  \begin{enumerate}
  \item Activities other than distribution and/or modification of the
    Work are not covered by this license; they are outside its scope.
    In particular, the act of running the Work is not restricted and
    no requirements are made concerning any offers of support for the
    Work.

  \item\label{LPPL:item:distribute} You may distribute a complete, unmodified
    copy of the Work as you received it.  Distribution of only part of
    the Work is considered modification of the Work, and no right to
    distribute such a Derived Work may be assumed under the terms of
    this clause.

  \item You may distribute a Compiled Work that has been generated
    from a complete, unmodified copy of the Work as distributed under
    Clause~\ref{LPPL:item:distribute} above, as long as that Compiled Work is
    distributed in such a way that the recipients may install the
    Compiled Work on their system exactly as it would have been
    installed if they generated a Compiled Work directly from the
    Work.

  \item\label{LPPL:item:currmaint} If you are the Current Maintainer of the
    Work, you may, without restriction, modify the Work, thus creating
    a Derived Work.  You may also distribute the Derived Work without
    restriction, including Compiled Works generated from the Derived
    Work.  Derived Works distributed in this manner by the Current
    Maintainer are considered to be updated versions of the Work.

  \item If you are not the Current Maintainer of the Work, you may
    modify your copy of the Work, thus creating a Derived Work based
    on the Work, and compile this Derived Work, thus creating a
    Compiled Work based on the Derived Work.

  \item\label{LPPL:item:conditions} If you are not the Current Maintainer
    of the
    Work, you may distribute a Derived Work provided the following
    conditions are met for every component of the Work unless that
    component clearly states in the copyright notice that it is exempt
    from that condition.  Only the Current Maintainer is allowed to
    add such statements of exemption to a component of the Work.
    \begin{enumerate}
    \item If a component of this Derived Work can be a direct
      replacement for a component of the Work when that component is
      used with the Base Interpreter, then, wherever this component of
      the Work identifies itself to the user when used interactively
      with that Base Interpreter, the replacement component of this
      Derived Work clearly and unambiguously identifies itself as a
      modified version of this component to the user when used
      interactively with that Base Interpreter.

    \item\label{LPPL:item:changelog} Every component of the Derived Work
      contains prominent
      notices detailing the nature of the changes to that component,
      or a prominent reference to another file that is distributed as
      part of the Derived Work and that contains a complete and
      accurate log of the changes.

    \item No information in the Derived Work implies that any persons,
      including (but not limited to) the authors of the original
      version of the Work, provide any support, including (but not
      limited to) the reporting and handling of errors, to recipients
      of the Derived Work unless those persons have stated explicitly
      that they do provide such support for the Derived Work.

    \item\label{LPPL:item:unmodifiedcopy} You distribute at least one of
      the following with the Derived Work:
      \begin{enumerate}
      \item A complete, unmodified copy of the Work; if your
        distribution of a modified component is made by offering
        access to copy the modified component from a designated place,
        then offering equivalent access to copy the Work from the same
        or some similar place meets this condition, even though third
        parties are not compelled to copy the Work along with the
        modified component;

      \item Information that is sufficient to obtain a complete,
        unmodified copy of the Work.
      \end{enumerate}
    \end{enumerate}
  \item If you are not the Current Maintainer of the Work, you may
    distribute a Compiled Work generated from a Derived Work, as long
    as the Derived Work is distributed to all recipients of the
    Compiled Work, and as long as the conditions of
    Clause~\ref{LPPL:item:conditions}, above, are met with regard to the
    Derived Work.

  \item The conditions above are not intended to prohibit, and hence
    do not apply to, the modification, by any method, of any component
    so that it becomes identical to an updated version of that
    component of the Work as it is distributed by the Current
    Maintainer under Clause~\ref{LPPL:item:currmaint}, above.

  \item Distribution of the Work or any Derived Work in an alternative
    format, where the Work or that Derived Work (in whole or in part)
    is then produced by applying some process to that format, does not
    relax or nullify any sections of this license as they pertain to
    the results of applying that process.

  \item
    \begin{enumerate}
    \item A Derived Work may be distributed under a different license
      provided that license itself honors the conditions listed in
      Clause~\ref{LPPL:item:conditions} above, in regard to the Work, though it
      does not have to honor the rest of the conditions in this
      license.

    \item If a Derived Work is distributed under a different license,
      that Derived Work must provide sufficient documentation as part
      of itself to allow each recipient of that Derived Work to honor
      the restrictions in Clause~\ref{LPPL:item:conditions} above, concerning
      changes from the Work.
    \end{enumerate}
  \item This license places no restrictions on works that are
    unrelated to the Work, nor does this license place any
    restrictions on aggregating such works with the Work by any means.

  \item Nothing in this license is intended to, or may be used to,
    prevent complete compliance by all parties with all applicable
    laws.
  \end{enumerate}

  \LPPLsubsection{No Warranty}
  \label{LPPL:Warranty}

  There is no warranty for the Work.  Except when otherwise stated in
  writing, the Copyright Holder provides the Work `as is', without
  warranty of any kind, either expressed or implied, including, but
  not limited to, the implied warranties of merchantability and
  fitness for a particular purpose.  The entire risk as to the quality
  and performance of the Work is with you.  Should the Work prove
  defective, you assume the cost of all necessary servicing, repair,
  or correction.

  In no event unless required by applicable law or agreed to in
  writing will The Copyright Holder, or any author named in the
  components of the Work, or any other party who may distribute and/or
  modify the Work as permitted above, be liable to you for damages,
  including any general, special, incidental or consequential damages
  arising out of any use of the Work or out of inability to use the
  Work (including, but not limited to, loss of data, data being
  rendered inaccurate, or losses sustained by anyone as a result of
  any failure of the Work to operate with any other programs), even if
  the Copyright Holder or said author or said other party has been
  advised of the possibility of such damages.

  \LPPLsubsection{Maintenance of The Work}
  \label{LPPL:Maintenance}

  The Work has the status `author-maintained' if the Copyright Holder
  explicitly and prominently states near the primary copyright notice
  in the Work that the Work can only be maintained by the Copyright
  Holder or simply that it is `author-maintained'.

  The Work has the status `maintained' if there is a Current
  Maintainer who has indicated in the Work that they are willing to
  receive error reports for the Work (for example, by supplying a
  valid e-mail address). It is not required for the Current Maintainer
  to acknowledge or act upon these error reports.

  The Work changes from status `maintained' to `unmaintained' if there
  is no Current Maintainer, or the person stated to be Current
  Maintainer of the work cannot be reached through the indicated means
  of communication for a period of six months, and there are no other
  significant signs of active maintenance.

  You can become the Current Maintainer of the Work by agreement with
  any existing Current Maintainer to take over this role.

  If the Work is unmaintained, you can become the Current Maintainer
  of the Work through the following steps:
  \begin{enumerate}
  \item Make a reasonable attempt to trace the Current Maintainer (and
    the Copyright Holder, if the two differ) through the means of an
    Internet or similar search.
  \item If this search is successful, then enquire whether the Work is
    still maintained.
    \begin{enumerate}
    \item If it is being maintained, then ask the Current Maintainer
      to update their communication data within one month.

    \item\label{LPPL:item:intention} If the search is unsuccessful or
      no action to resume active maintenance is taken by the Current
      Maintainer, then announce within the pertinent community your
      intention to take over maintenance.  (If the Work is a \LaTeX{}
      work, this could be done, for example, by posting to
      \texttt{comp.text.tex}.)
    \end{enumerate}
  \item {}
    \begin{enumerate}
    \item If the Current Maintainer is reachable and agrees to pass
      maintenance of the Work to you, then this takes effect
      immediately upon announcement.

    \item\label{LPPL:item:announce} If the Current Maintainer is not
      reachable and the Copyright Holder agrees that maintenance of
      the Work be passed to you, then this takes effect immediately
      upon announcement.
    \end{enumerate}
  \item\label{LPPL:item:change} If you make an `intention
    announcement' as described in~\ref{LPPL:item:intention} above and
    after three months your intention is challenged neither by the
    Current Maintainer nor by the Copyright Holder nor by other
    people, then you may arrange for the Work to be changed so as to
    name you as the (new) Current Maintainer.

  \item If the previously unreachable Current Maintainer becomes
    reachable once more within three months of a change completed
    under the terms of~\ref{LPPL:item:announce}
    or~\ref{LPPL:item:change}, then that Current Maintainer must
    become or remain the Current Maintainer upon request provided they
    then update their communication data within one month.
  \end{enumerate}
  A change in the Current Maintainer does not, of itself, alter the
  fact that the Work is distributed under the \LPPL\ license.

  If you become the Current Maintainer of the Work, you should
  immediately provide, within the Work, a prominent and unambiguous
  statement of your status as Current Maintainer.  You should also
  announce your new status to the same pertinent community as
  in~\ref{LPPL:item:intention} above.

  \LPPLsubsection{Whether and How to Distribute Works under This License}
  \label{LPPL:Distribute}

  This section contains important instructions, examples, and
  recommendations for authors who are considering distributing their
  works under this license.  These authors are addressed as `you' in
  this section.

  \LPPLsubsubsection{Choosing This License or Another License}
  \label{LPPL:Choosing}

  If for any part of your work you want or need to use
  \emph{distribution} conditions that differ significantly from those
  in this license, then do not refer to this license anywhere in your
  work but, instead, distribute your work under a different license.
  You may use the text of this license as a model for your own
  license, but your license should not refer to the \LPPL\ or
  otherwise give the impression that your work is distributed under
  the \LPPL.

  The document \LPPLdocfile{modguide} in the base \LaTeX\ distribution
  explains the motivation behind the conditions of this license.  It
  explains, for example, why distributing \LaTeX\ under the
  \textsc{gnu} General Public License (\textsc{gpl}) was considered
  inappropriate.  Even if your work is unrelated to \LaTeX, the
  discussion in \LPPLdocfile{modguide} may still be relevant, and authors
  intending to distribute their works under any license are encouraged
  to read it.

  \LPPLsubsubsection{A Recommendation on Modification Without Distribution}
  \label{LPPL:WithoutDistribution}

  It is wise never to modify a component of the Work, even for your
  own personal use, without also meeting the above conditions for
  distributing the modified component.  While you might intend that
  such modifications will never be distributed, often this will happen
  by accident -- you may forget that you have modified that component;
  or it may not occur to you when allowing others to access the
  modified version that you are thus distributing it and violating the
  conditions of this license in ways that could have legal
  implications and, worse, cause problems for the community.  It is
  therefore usually in your best interest to keep your copy of the
  Work identical with the public one.  Many works provide ways to
  control the behavior of that work without altering any of its
  licensed components.

  \LPPLsubsubsection{How to Use This License}
  \label{LPPL:HowTo}

  To use this license, place in each of the components of your work
  both an explicit copyright notice including your name and the year
  the work was authored and/or last substantially modified.  Include
  also a statement that the distribution and/or modification of that
  component is constrained by the conditions in this license.

  Here is an example of such a notice and statement:
\begin{verbatim}
  %% pig.dtx
  %% Copyright 2005 M. Y. Name
  %
  % This work may be distributed and/or modified under the
  % conditions of the LaTeX Project Public License, either version 1.3
  % of this license or (at your option) any later version.
  % The latest version of this license is in
  %   http://www.latex-project.org/lppl.txt
  % and version 1.3 or later is part of all distributions of LaTeX
  % version 2005/12/01 or later.
  %
  % This work has the LPPL maintenance status `maintained'.
  %
  % The Current Maintainer of this work is M. Y. Name.
  %
  % This work consists of the files pig.dtx and pig.ins
  % and the derived file pig.sty.
\end{verbatim}

  Given such a notice and statement in a file, the conditions given in
  this license document would apply, with the `Work' referring to the
  three files `\LPPLfile{pig.dtx}', `\LPPLfile{pig.ins}', and
  `\LPPLfile{pig.sty}' (the last being generated from
  `\LPPLfile{pig.dtx}' using `\LPPLfile{pig.ins}'), the `Base
  Interpreter' referring to any `\LaTeX-Format', and both `Copyright
  Holder' and `Current Maintainer' referring to the person `M. Y.
  Name'.

  If you do not want the Maintenance section of \LPPL\ to apply to
  your Work, change `maintained' above into `author-maintained'.
  However, we recommend that you use `maintained' as the Maintenance
  section was added in order to ensure that your Work remains useful
  to the community even when you can no longer maintain and support it
  yourself.

  \LPPLsubsubsection{Derived Works That Are Not Replacements}
  \label{LPPL:NotReplacements}

  Several clauses of the \LPPL\ specify means to provide reliability
  and stability for the user community. They therefore concern
  themselves with the case that a Derived Work is intended to be used
  as a (compatible or incompatible) replacement of the original
  Work. If this is not the case (e.g., if a few lines of code are
  reused for a completely different task), then clauses
  \ref{LPPL:item:changelog} and \ref{LPPL:item:unmodifiedcopy}
  shall not apply.

  \LPPLsubsubsection{Important Recommendations}
  \label{LPPL:Recommendations}

  \LPPLparagraph{Defining What Constitutes the Work}

  The \LPPL\ requires that distributions of the Work contain all the
  files of the Work.  It is therefore important that you provide a way
  for the licensee to determine which files constitute the Work.  This
  could, for example, be achieved by explicitly listing all the files
  of the Work near the copyright notice of each file or by using a
  line such as:
\begin{verbatim}
    % This work consists of all files listed in manifest.txt.
\end{verbatim}
  in that place.  In the absence of an unequivocal list it might be
  impossible for the licensee to determine what is considered by you
  to comprise the Work and, in such a case, the licensee would be
  entitled to make reasonable conjectures as to which files comprise
  the Work.

\end{LPPLicense}
\endinput

%   \MakePercentIgnore
%   \MakeShortVerb{\'}     % turn it on again if necessary
%
%
% By default the license is produced with \section* as the highest
% heading level. If this is not appropriate for the document in which
% it is included define the commands listed below before loading this
% document, e.g., for inclusion as a separate chapter define:
%
%  \providecommand{\LPPLsection}{\chapter*}
%  \providecommand{\LPPLsubsection}{\section*}
%  \providecommand{\LPPLsubsubsection}{\subsection*}
%  \providecommand{\LPPLparagraph}{\subsubsection*}
%
%
% To allow cross-referencing the headings \label's have been attached
% to them, all starting with ``LPPL:''. As by default headings without
% numbers are produced, this will only allow page references.
% However, you can use the titleref package to produce textual
% references or you change the definitions of \LPPLsection, and
% friends to generated numbered headings.
%
%
% We want it to be possible that this file can be processed by
% (pdf)LaTeX on its own, or that this file can be included in another
% LaTeX document without any modification whatsoever.
% Hence the little test below.
%
%
\makeatletter
\ifx\@preamblecmds\@notprerr
  % In this case the preamble has already been processed so this file
  % is loaded as part of another document; just enclose everything in
  % a group
  \let\LPPLicense\bgroup
  \let\endLPPLicense\egroup
\else
  % In this case the preamble has not been processed yet so this file
  % is processed by itself.
  \documentclass{article}
  \let\LPPLicense\document
  \let\endLPPLicense\enddocument
\fi
\makeatother


\begin{LPPLicense}
  \providecommand{\LPPLsection}{\section*}
  \providecommand{\LPPLsubsection}{\subsection*}
  \providecommand{\LPPLsubsubsection}{\subsubsection*}
  \providecommand{\LPPLparagraph}{\paragraph*}
  \providecommand*{\LPPLfile}[1]{\texttt{#1}}
  \providecommand*{\LPPLdocfile}[1]{`\LPPLfile{#1.tex}'}
  \providecommand*{\LPPL}{\textsc{lppl}}

  \LPPLsection{The \LaTeX\ Project Public License}
  \label{LPPL:LPPL}

  \emph{LPPL Version 1.3c  2008-05-04}

  \textbf{Copyright 1999, 2002--2008 \LaTeX3 Project}
  \begin{quotation}
    Everyone is allowed to distribute verbatim copies of this
    license document, but modification of it is not allowed.
  \end{quotation}

  \LPPLsubsection{Preamble}
  \label{LPPL:Preamble}

  The \LaTeX\ Project Public License (\LPPL) is the primary license
  under which the \LaTeX\ kernel and the base \LaTeX\ packages are
  distributed.

  You may use this license for any work of which you hold the
  copyright and which you wish to distribute.  This license may be
  particularly suitable if your work is \TeX-related (such as a
  \LaTeX\ package), but it is written in such a way that you can use
  it even if your work is unrelated to \TeX.

  The section `WHETHER AND HOW TO DISTRIBUTE WORKS UNDER THIS
  LICENSE', below, gives instructions, examples, and recommendations
  for authors who are considering distributing their works under this
  license.

  This license gives conditions under which a work may be distributed
  and modified, as well as conditions under which modified versions of
  that work may be distributed.

  We, the \LaTeX3 Project, believe that the conditions below give you
  the freedom to make and distribute modified versions of your work
  that conform with whatever technical specifications you wish while
  maintaining the availability, integrity, and reliability of that
  work.  If you do not see how to achieve your goal while meeting
  these conditions, then read the document \LPPLdocfile{cfgguide} and
  \LPPLdocfile{modguide} in the base \LaTeX\ distribution for suggestions.


  \LPPLsubsection{Definitions}
  \label{LPPL:Definitions}

  In this license document the following terms are used:

  \begin{description}
  \item[Work] Any work being distributed under this License.

  \item[Derived Work] Any work that under any applicable law is
    derived from the Work.

  \item[Modification] Any procedure that produces a Derived Work under
    any applicable law -- for example, the production of a file
    containing an original file associated with the Work or a
    significant portion of such a file, either verbatim or with
    modifications and/or translated into another language.

  \item[Modify] To apply any procedure that produces a Derived Work
    under any applicable law.

  \item[Distribution] Making copies of the Work available from one
    person to another, in whole or in part.  Distribution includes
    (but is not limited to) making any electronic components of the
    Work accessible by file transfer protocols such as \textsc{ftp} or
    \textsc{http} or by shared file systems such as Sun's Network File
    System (\textsc{nfs}).

  \item[Compiled Work] A version of the Work that has been processed
    into a form where it is directly usable on a computer system.
    This processing may include using installation facilities provided
    by the Work, transformations of the Work, copying of components of
    the Work, or other activities.  Note that modification of any
    installation facilities provided by the Work constitutes
    modification of the Work.

  \item[Current Maintainer] A person or persons nominated as such
    within the Work.  If there is no such explicit nomination then it
    is the `Copyright Holder' under any applicable law.

  \item[Base Interpreter] A program or process that is normally needed
    for running or interpreting a part or the whole of the Work.

    A Base Interpreter may depend on external components but these are
    not considered part of the Base Interpreter provided that each
    external component clearly identifies itself whenever it is used
    interactively.  Unless explicitly specified when applying the
    license to the Work, the only applicable Base Interpreter is a
    `\LaTeX-Format' or in the case of files belonging to the
    `\LaTeX-format' a program implementing the `\TeX{} language'.
  \end{description}

  \LPPLsubsection{Conditions on Distribution and Modification}
  \label{LPPL:Conditions}

  \begin{enumerate}
  \item Activities other than distribution and/or modification of the
    Work are not covered by this license; they are outside its scope.
    In particular, the act of running the Work is not restricted and
    no requirements are made concerning any offers of support for the
    Work.

  \item\label{LPPL:item:distribute} You may distribute a complete, unmodified
    copy of the Work as you received it.  Distribution of only part of
    the Work is considered modification of the Work, and no right to
    distribute such a Derived Work may be assumed under the terms of
    this clause.

  \item You may distribute a Compiled Work that has been generated
    from a complete, unmodified copy of the Work as distributed under
    Clause~\ref{LPPL:item:distribute} above, as long as that Compiled Work is
    distributed in such a way that the recipients may install the
    Compiled Work on their system exactly as it would have been
    installed if they generated a Compiled Work directly from the
    Work.

  \item\label{LPPL:item:currmaint} If you are the Current Maintainer of the
    Work, you may, without restriction, modify the Work, thus creating
    a Derived Work.  You may also distribute the Derived Work without
    restriction, including Compiled Works generated from the Derived
    Work.  Derived Works distributed in this manner by the Current
    Maintainer are considered to be updated versions of the Work.

  \item If you are not the Current Maintainer of the Work, you may
    modify your copy of the Work, thus creating a Derived Work based
    on the Work, and compile this Derived Work, thus creating a
    Compiled Work based on the Derived Work.

  \item\label{LPPL:item:conditions} If you are not the Current Maintainer
    of the
    Work, you may distribute a Derived Work provided the following
    conditions are met for every component of the Work unless that
    component clearly states in the copyright notice that it is exempt
    from that condition.  Only the Current Maintainer is allowed to
    add such statements of exemption to a component of the Work.
    \begin{enumerate}
    \item If a component of this Derived Work can be a direct
      replacement for a component of the Work when that component is
      used with the Base Interpreter, then, wherever this component of
      the Work identifies itself to the user when used interactively
      with that Base Interpreter, the replacement component of this
      Derived Work clearly and unambiguously identifies itself as a
      modified version of this component to the user when used
      interactively with that Base Interpreter.

    \item\label{LPPL:item:changelog} Every component of the Derived Work
      contains prominent
      notices detailing the nature of the changes to that component,
      or a prominent reference to another file that is distributed as
      part of the Derived Work and that contains a complete and
      accurate log of the changes.

    \item No information in the Derived Work implies that any persons,
      including (but not limited to) the authors of the original
      version of the Work, provide any support, including (but not
      limited to) the reporting and handling of errors, to recipients
      of the Derived Work unless those persons have stated explicitly
      that they do provide such support for the Derived Work.

    \item\label{LPPL:item:unmodifiedcopy} You distribute at least one of
      the following with the Derived Work:
      \begin{enumerate}
      \item A complete, unmodified copy of the Work; if your
        distribution of a modified component is made by offering
        access to copy the modified component from a designated place,
        then offering equivalent access to copy the Work from the same
        or some similar place meets this condition, even though third
        parties are not compelled to copy the Work along with the
        modified component;

      \item Information that is sufficient to obtain a complete,
        unmodified copy of the Work.
      \end{enumerate}
    \end{enumerate}
  \item If you are not the Current Maintainer of the Work, you may
    distribute a Compiled Work generated from a Derived Work, as long
    as the Derived Work is distributed to all recipients of the
    Compiled Work, and as long as the conditions of
    Clause~\ref{LPPL:item:conditions}, above, are met with regard to the
    Derived Work.

  \item The conditions above are not intended to prohibit, and hence
    do not apply to, the modification, by any method, of any component
    so that it becomes identical to an updated version of that
    component of the Work as it is distributed by the Current
    Maintainer under Clause~\ref{LPPL:item:currmaint}, above.

  \item Distribution of the Work or any Derived Work in an alternative
    format, where the Work or that Derived Work (in whole or in part)
    is then produced by applying some process to that format, does not
    relax or nullify any sections of this license as they pertain to
    the results of applying that process.

  \item
    \begin{enumerate}
    \item A Derived Work may be distributed under a different license
      provided that license itself honors the conditions listed in
      Clause~\ref{LPPL:item:conditions} above, in regard to the Work, though it
      does not have to honor the rest of the conditions in this
      license.

    \item If a Derived Work is distributed under a different license,
      that Derived Work must provide sufficient documentation as part
      of itself to allow each recipient of that Derived Work to honor
      the restrictions in Clause~\ref{LPPL:item:conditions} above, concerning
      changes from the Work.
    \end{enumerate}
  \item This license places no restrictions on works that are
    unrelated to the Work, nor does this license place any
    restrictions on aggregating such works with the Work by any means.

  \item Nothing in this license is intended to, or may be used to,
    prevent complete compliance by all parties with all applicable
    laws.
  \end{enumerate}

  \LPPLsubsection{No Warranty}
  \label{LPPL:Warranty}

  There is no warranty for the Work.  Except when otherwise stated in
  writing, the Copyright Holder provides the Work `as is', without
  warranty of any kind, either expressed or implied, including, but
  not limited to, the implied warranties of merchantability and
  fitness for a particular purpose.  The entire risk as to the quality
  and performance of the Work is with you.  Should the Work prove
  defective, you assume the cost of all necessary servicing, repair,
  or correction.

  In no event unless required by applicable law or agreed to in
  writing will The Copyright Holder, or any author named in the
  components of the Work, or any other party who may distribute and/or
  modify the Work as permitted above, be liable to you for damages,
  including any general, special, incidental or consequential damages
  arising out of any use of the Work or out of inability to use the
  Work (including, but not limited to, loss of data, data being
  rendered inaccurate, or losses sustained by anyone as a result of
  any failure of the Work to operate with any other programs), even if
  the Copyright Holder or said author or said other party has been
  advised of the possibility of such damages.

  \LPPLsubsection{Maintenance of The Work}
  \label{LPPL:Maintenance}

  The Work has the status `author-maintained' if the Copyright Holder
  explicitly and prominently states near the primary copyright notice
  in the Work that the Work can only be maintained by the Copyright
  Holder or simply that it is `author-maintained'.

  The Work has the status `maintained' if there is a Current
  Maintainer who has indicated in the Work that they are willing to
  receive error reports for the Work (for example, by supplying a
  valid e-mail address). It is not required for the Current Maintainer
  to acknowledge or act upon these error reports.

  The Work changes from status `maintained' to `unmaintained' if there
  is no Current Maintainer, or the person stated to be Current
  Maintainer of the work cannot be reached through the indicated means
  of communication for a period of six months, and there are no other
  significant signs of active maintenance.

  You can become the Current Maintainer of the Work by agreement with
  any existing Current Maintainer to take over this role.

  If the Work is unmaintained, you can become the Current Maintainer
  of the Work through the following steps:
  \begin{enumerate}
  \item Make a reasonable attempt to trace the Current Maintainer (and
    the Copyright Holder, if the two differ) through the means of an
    Internet or similar search.
  \item If this search is successful, then enquire whether the Work is
    still maintained.
    \begin{enumerate}
    \item If it is being maintained, then ask the Current Maintainer
      to update their communication data within one month.

    \item\label{LPPL:item:intention} If the search is unsuccessful or
      no action to resume active maintenance is taken by the Current
      Maintainer, then announce within the pertinent community your
      intention to take over maintenance.  (If the Work is a \LaTeX{}
      work, this could be done, for example, by posting to
      \texttt{comp.text.tex}.)
    \end{enumerate}
  \item {}
    \begin{enumerate}
    \item If the Current Maintainer is reachable and agrees to pass
      maintenance of the Work to you, then this takes effect
      immediately upon announcement.

    \item\label{LPPL:item:announce} If the Current Maintainer is not
      reachable and the Copyright Holder agrees that maintenance of
      the Work be passed to you, then this takes effect immediately
      upon announcement.
    \end{enumerate}
  \item\label{LPPL:item:change} If you make an `intention
    announcement' as described in~\ref{LPPL:item:intention} above and
    after three months your intention is challenged neither by the
    Current Maintainer nor by the Copyright Holder nor by other
    people, then you may arrange for the Work to be changed so as to
    name you as the (new) Current Maintainer.

  \item If the previously unreachable Current Maintainer becomes
    reachable once more within three months of a change completed
    under the terms of~\ref{LPPL:item:announce}
    or~\ref{LPPL:item:change}, then that Current Maintainer must
    become or remain the Current Maintainer upon request provided they
    then update their communication data within one month.
  \end{enumerate}
  A change in the Current Maintainer does not, of itself, alter the
  fact that the Work is distributed under the \LPPL\ license.

  If you become the Current Maintainer of the Work, you should
  immediately provide, within the Work, a prominent and unambiguous
  statement of your status as Current Maintainer.  You should also
  announce your new status to the same pertinent community as
  in~\ref{LPPL:item:intention} above.

  \LPPLsubsection{Whether and How to Distribute Works under This License}
  \label{LPPL:Distribute}

  This section contains important instructions, examples, and
  recommendations for authors who are considering distributing their
  works under this license.  These authors are addressed as `you' in
  this section.

  \LPPLsubsubsection{Choosing This License or Another License}
  \label{LPPL:Choosing}

  If for any part of your work you want or need to use
  \emph{distribution} conditions that differ significantly from those
  in this license, then do not refer to this license anywhere in your
  work but, instead, distribute your work under a different license.
  You may use the text of this license as a model for your own
  license, but your license should not refer to the \LPPL\ or
  otherwise give the impression that your work is distributed under
  the \LPPL.

  The document \LPPLdocfile{modguide} in the base \LaTeX\ distribution
  explains the motivation behind the conditions of this license.  It
  explains, for example, why distributing \LaTeX\ under the
  \textsc{gnu} General Public License (\textsc{gpl}) was considered
  inappropriate.  Even if your work is unrelated to \LaTeX, the
  discussion in \LPPLdocfile{modguide} may still be relevant, and authors
  intending to distribute their works under any license are encouraged
  to read it.

  \LPPLsubsubsection{A Recommendation on Modification Without Distribution}
  \label{LPPL:WithoutDistribution}

  It is wise never to modify a component of the Work, even for your
  own personal use, without also meeting the above conditions for
  distributing the modified component.  While you might intend that
  such modifications will never be distributed, often this will happen
  by accident -- you may forget that you have modified that component;
  or it may not occur to you when allowing others to access the
  modified version that you are thus distributing it and violating the
  conditions of this license in ways that could have legal
  implications and, worse, cause problems for the community.  It is
  therefore usually in your best interest to keep your copy of the
  Work identical with the public one.  Many works provide ways to
  control the behavior of that work without altering any of its
  licensed components.

  \LPPLsubsubsection{How to Use This License}
  \label{LPPL:HowTo}

  To use this license, place in each of the components of your work
  both an explicit copyright notice including your name and the year
  the work was authored and/or last substantially modified.  Include
  also a statement that the distribution and/or modification of that
  component is constrained by the conditions in this license.

  Here is an example of such a notice and statement:
\begin{verbatim}
  %% pig.dtx
  %% Copyright 2005 M. Y. Name
  %
  % This work may be distributed and/or modified under the
  % conditions of the LaTeX Project Public License, either version 1.3
  % of this license or (at your option) any later version.
  % The latest version of this license is in
  %   http://www.latex-project.org/lppl.txt
  % and version 1.3 or later is part of all distributions of LaTeX
  % version 2005/12/01 or later.
  %
  % This work has the LPPL maintenance status `maintained'.
  %
  % The Current Maintainer of this work is M. Y. Name.
  %
  % This work consists of the files pig.dtx and pig.ins
  % and the derived file pig.sty.
\end{verbatim}

  Given such a notice and statement in a file, the conditions given in
  this license document would apply, with the `Work' referring to the
  three files `\LPPLfile{pig.dtx}', `\LPPLfile{pig.ins}', and
  `\LPPLfile{pig.sty}' (the last being generated from
  `\LPPLfile{pig.dtx}' using `\LPPLfile{pig.ins}'), the `Base
  Interpreter' referring to any `\LaTeX-Format', and both `Copyright
  Holder' and `Current Maintainer' referring to the person `M. Y.
  Name'.

  If you do not want the Maintenance section of \LPPL\ to apply to
  your Work, change `maintained' above into `author-maintained'.
  However, we recommend that you use `maintained' as the Maintenance
  section was added in order to ensure that your Work remains useful
  to the community even when you can no longer maintain and support it
  yourself.

  \LPPLsubsubsection{Derived Works That Are Not Replacements}
  \label{LPPL:NotReplacements}

  Several clauses of the \LPPL\ specify means to provide reliability
  and stability for the user community. They therefore concern
  themselves with the case that a Derived Work is intended to be used
  as a (compatible or incompatible) replacement of the original
  Work. If this is not the case (e.g., if a few lines of code are
  reused for a completely different task), then clauses
  \ref{LPPL:item:changelog} and \ref{LPPL:item:unmodifiedcopy}
  shall not apply.

  \LPPLsubsubsection{Important Recommendations}
  \label{LPPL:Recommendations}

  \LPPLparagraph{Defining What Constitutes the Work}

  The \LPPL\ requires that distributions of the Work contain all the
  files of the Work.  It is therefore important that you provide a way
  for the licensee to determine which files constitute the Work.  This
  could, for example, be achieved by explicitly listing all the files
  of the Work near the copyright notice of each file or by using a
  line such as:
\begin{verbatim}
    % This work consists of all files listed in manifest.txt.
\end{verbatim}
  in that place.  In the absence of an unequivocal list it might be
  impossible for the licensee to determine what is considered by you
  to comprise the Work and, in such a case, the licensee would be
  entitled to make reasonable conjectures as to which files comprise
  the Work.

\end{LPPLicense}
\endinput

    }{
        \msg_warning:nn{skdoc}{no-lppl}
    }
    \endgroup
}
%    \end{macrocode}
% \end{macro}
%
% \subsection{Cosmetic changes}
% We perform a couple of cosmetic changes to existing features as
% well. First, we set a new header/footer style using the KOMA-script
% \cs{deftripstyle} macro.
%    \begin{macrocode}
\deftripstyle{skdoc}%
    {}{}{}%
    {\small The~\textbf{\pkg*{\@package}}~package,~v\@version}{}{\small\pagemark}
\AfterBeginDocument{\pagestyle{skdoc}}
%    \end{macrocode}
% We also redefine the section level format to set the section numbers
% in the margin, much like the \pkg{microtype} manual.
%    \begin{macrocode}
\RenewDocumentCommand{\othersectionlevelsformat}{m}{%
    \makebox[0pt][r]{%
    \fontfamily{fos}\mdseries\selectfont
    \csname the#1\endcsname\enskip}%
}
%    \end{macrocode}
% Finally, we actually use \pkg{microtype} in the document class, and
% make sure to disable it in the verbatim environments.
% Set up microtype properly
%    \begin{macrocode}
\g@addto@macro\@verbatim{\microtypesetup{activate=false}}
\AtEndOfClass{%
    \microtypesetup{expansion,kerning,spacing,tracking}%
    \DisableLigatures{family = tt*}%
}
%    \end{macrocode}
% We also want numbers on the bibliography headings, if we are loading
% \pkg{biblatex}. If we happen to be loading \pkg{bibtex}, we issue a
% warning instead.
%    \begin{macrocode}
\AtBeginDocument{
    \ifdefined\defbibheading
        \defbibheading{bibliography}[\bibname]{\section{#1}}
    \fi
    \@ifpackageloaded{bibtex}{\msg_warning:nn{skdoc}{bibtex-unsupported}}{}
}
%    \end{macrocode}
% Oh, and we want \cs{marginpar}s on the left, not on the right.
%    \begin{macrocode}
\AtBeginDocument{\reversemarginpar}
%    \end{macrocode}
%
% That's it, we're done!
%    \begin{macrocode}
\endinput
%    \end{macrocode}
% \iffalse
%</class>
% \fi
% \Finale
% \section{Installation}
% The easiest way to install this package is using the package
% manager provided by your \LaTeX\ installation if such a program
% is available. Failing that, provided you have obtained the package
% source (\file{skdoc.dtx} and \file{Makefile}) from either CTAN
% or Github, running \texttt{make install} inside the source directory
% works well. This will extract the documentation and code from
% \file{skdoc.dtx}, install all files into the TDS tree at
% \texttt{TEXMFHOME} and run \texttt{mktexlsr}.
%
% If you want to extract code and documentation without installing
% the package, run \texttt{make all} instead. If you insist on not
% using \texttt{make}, remember that \file{skdoc.cls} is generated
% by running \texttt{tex}, while the documentation is generated by
% running \texttt{pdflatex}.
%
% \PrintChanges
% \PrintIndex
% \printbibliography
% \endinput
