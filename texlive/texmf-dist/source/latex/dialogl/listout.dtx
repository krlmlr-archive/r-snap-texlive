%%% ====================================================================
%%%  @TeX-file{
%%%     filename        = "listout.dtx",
%%%     version         = "1.99a",
%%%     date            = "2013/01/24",
%%%     author          = "Michael Downes",
%%%     copyright       = "This file is part of the dialogl package, released
%%%                        under the LPPL; see dialogl.ins for details."
%%%     keywords        = "tex, verbatim",
%%%     abstract        = "The purpose of this macro file is to print
%%%       verbatim listings of arbitrary text files, fitting as much
%%%       text per sheet of paper as possible. The default settings
%%%       are for US letter size paper 8.5 x 11 inches, two `pages'
%%%       per sheet, landscape mode, but extensive customization
%%%       facilities are provided. To conserve even more paper, you
%%%       should take advantage of any two-sided capabilities your
%%%       printer may happen to have.",
%%%  }
%%% ====================================================================
%
% \iffalse
%<*driver>
\documentstyle[doc,dialogl-doc]{article}
\title{Verbatim file printing: {\tt listout.tex}}
\author{Michael Downes}
\def\PrintMacroName#1{}
\setlength{\MacroTopsep}{\MacrocodeTopsep}
\setlength{\MacroIndent}{0pt}
\def\MacroFont{\Vbtmfont}% doc.sty uses \MacroFont
\DontCheckModules
\multicoldefaultlayout
\begin{document}
\maketitle
\begin{multicols}{2}
\hDocInput{listout.dtx}
\end{multicols}
\ifdim\textwidth>\textheight
\typeout{^^JWarning: remember to print using LANDSCAPE orientation^^J}
\fi
\end{document}
%</driver>
%\fi
%
%^^A Actual code lines are broken up to be 50 characters or less in
%^^A order to fit within narrow column widths in two-column style.
%
% \section*{Introduction}
%
%    The tool \fn{listout.tex} has two purposes:
%
%    (1)~Verbatim printing of text files such as
%    e-\unpenalty mail, \tex/ log files, \tex/ macro files, or other
%    programming source code or, in general, any \ascii/ text, with
%    certain nice features that most printers and printer drivers
%    lack: reasonably good, and somewhat customizable, handling of tab
%    characters or other nonprinting characters; highly customizable
%    handling of overlong lines; absolute control over margins, number
%    of columns, intercolumn space, and line spacing; optional line
%    numbering; customizable running heads and running feet; not to
%    mention a few other things.
%
%    (2)~To serve as a proving ground for ideas and functions from
%    \fn{dialog.sty} and \fn{menus.sty}. \fn{Listout.tex} uses
%    \fn{menus.sty} to present an elaborate menu system for changing
%    options (like font size, line spacing, or how many spaces should
%    be printed for a tab character).
%
%    \fn{Listout} was originally written for \plaintex/. It seems to
%    work with \latex/ as well in the limited tests that I have done.
%
%^^A Use of \latex/ would require among other
%^^A things the following adjustments: Use \cs{c@page} instead of
%^^A \cs{pageno}; (nfss2) load math fonts and do other
%^^A \begin{document} processing.
%
%    \fn{Listout} is a part of a package that goes by the name of
%    {\bf dialogl}, which also includes the aforementioned
%    \fn{dialog.sty} and \fn{menus.sty}. The {\bf dialogl} package is
%    available on the Internet by anonymous ftp from {\sc ctan}
%    (Comprehensive \tex/ Archive Network), \eg., \fn{ftp.shsu.edu}
%    (USA), or \fn{ftp.dante.de} (Europe).
%
%    For maximum portability, \fn{listout} uses in its menus only
%    lowest-common-denominator ordinary printable \ascii/ characters
%    in the range 32\ndash 126.
%
%    Normal usage:
%
%^^V tex listout
%
%    You will be prompted for the names of the files to be printed. But
%    first, you will be offered a chance to enter the \fn{listout} menu
%    system, which allows you to change options interactively at
%    run-time. (If you want to save a set of option settings for later
%    re-use, see selection \qc{\E} in the main menu.)
%
%    \subsection*{Notation}
%
%    Double-hat notation such as \verb'^^J' is used in this article
%    for control characters, as in \texbook/. A couple of
%    abbreviations from \fn{grabhedr.sty} are used frequently in the
%    macro code: \cw{xp@} = \cw{expandafter}, and \cw{nx@} =
%    \cw{noexpand}. Standard abbreviations from \fn{plain.tex} such as
%    \cw{z@} or \cw{toks@} are used without special comment.
%
%    In the beginning \fn{listout.tex} was written without any private
%    control sequences. Eventually avoiding the use of private control
%    sequences became too inconvenient so I abandoned that restriction.
%    If it were redone from scratch I would privatize many of the
%    control sequences defined here to better support the future
%    possibility of using \fn{listout.tex} within arbitrary documents
%    where name conflicts might arise.
%
% \section{Preliminaries}
%    We begin by loading \fn{menus.sty}. As a matter of fact we won't
%    really input the file: The following few lines are for information
%    only, they will replaced by the contents of \fn{menus.sty} (and the
%    subordinate files \fn{dialog.sty} and \fn{grabhedr.sty} during the
%    generation of the executable file by docstrip. The reason is mainly
%    to make it possible to have a single version of \fn{listout.tex}
%    that runs under plain \tex/, \latex/ 2.09, and \latex/2e, but also
%    (a little bit) to avoid the speed cost of searching for input files
%    at run-time.
%    \begin{macrocode}
%<*ignore>
\inputfwh{menus.sty}
\fileversiondate{listout.tex}{1.4r}{1994/12/02}
\ProvidesFile{listout.tex}[1994/11/10 v1.4r]
%</ignore>
%    \end{macrocode}
%
%    \mjd{Check if more catcodes must be added here:}
%    \begin{macrocode}
\localcatcodes{\@{11}\~{13}\"{12}\:{12}}
%    \end{macrocode}
%
% \section{Miscellaneous utility functions}
%
%    The file \fn{dialog.sty} (called by \fn{menus.sty}) defines
%    \cw{actively}, \cw{afterfi}, \cw{controlchars}, and
%    \cw{otherchars}.
%
% \begin{macro}{\xmeaning}
%    \cw{xmeaning} is a useful tool to convert the contents of a macro
%    to all category 12 characters. This is applied to filenames when
%    printing them to prevent problems from special characters such as
%    \qc{\$} or \qc{\_} that might occur in a file name.
%
%    The argument \qarg{1} should be a macro.
%    \begin{macrocode}
\def\xmeaning#1{\xp@\xmeaningtrim\meaning#1}
%    \end{macrocode}
% \end{macro}
%
% \begin{macro}{\xmeaningtrim}
%    \begin{macrocode}
\def\xmeaningtrim#1->{}
%    \end{macrocode}
% \end{macro}
%
% \begin{macro}{\TRUE}
% \begin{macro}{\FALSE}
%    \cw{TRUE} and \cw{FALSE} are Boolean constants. After
%    \cw{let}\cw{sometest}\qc{\=}\cw{TRUE}, the variable \cw{sometest}
%    can be used with \cw{if}.
%    \begin{macrocode}
\def\TRUE{TT}
\def\FALSE{TF}
%    \end{macrocode}
% \end{macro}
% \end{macro}
%
% \begin{macro}{\@empty}
% \begin{macro}{\@iden}
%    Borrow a couple of things from \latex/.
%    \begin{macrocode}
\def\@empty{}
\def\@iden#1{#1}
%    \end{macrocode}
% \end{macro}
% \end{macro}
%
% %%%%%%%%%%%%%%%%%%%%%%%%%%%%%%%%%%%%%%%%%%%%%%%%%%%%%%%%%%%%%%%%%%%%%%
% \section{Date and time}
%
%    Some macros to help construct various date options.
% \begin{macro}{\datesepchar}
%    \begin{macrocode}
\def\datesepchar{ }
%    \end{macrocode}
% \end{macro}
%
% \begin{macro}{\keepleadingzeros}
%    \begin{macrocode}
\let\keepleadingzeros\FALSE
%    \end{macrocode}
% \end{macro}
%
% \begin{macro}{\twodigits}
%    \cw{twodigits} adds a leading zero to a number if the number is
%    less than 10. The \cw{expandafter} and \cw{ident} are to stop the
%    scanning of \cw{number} so that something like
%    \cw{twodigits}\verb'{35}' will not gobble a following space. (When
%    the argument of \cw{twodigits} is a count register or similar
%    complete integer according to \tex/'s scanning rules, this
%    precaution is superfluous.)
%    \begin{macrocode}
\def\twodigits#1{\ifnum#1<10 0\fi
  \xp@\@iden\xp@{\number#1}}
%    \end{macrocode}
% \end{macro}
%
% \begin{macro}{\themonth}
% \begin{macro}{\theday}
%    \cw{themonth}, \cw{theday}, and similar macros must be
%    reinitialized after the \fn{default.los} file is read, in case
%    the value of \cw{keepleadingzeros} changes.
%    \begin{macrocode}
\edef\themonth{\number\month}
\edef\theday{\number\day}
%    \end{macrocode}
% \end{macro}
% \end{macro}
%
% \begin{macro}{\theyear}
% \begin{macro}{\yearmodC}
%    \begin{macrocode}
\edef\theyear{\number\year}
\edef\yearmodC{\xp@\@gobbletwo\number\year}
%    \end{macrocode}
% \end{macro}
% \end{macro}
%
% \begin{macro}{\Jan}\begin{macro}{\Feb}\begin{macro}{\Mar}
% \begin{macro}{\Apr}\begin{macro}{\May}\begin{macro}{\Jun}
% \begin{macro}{\Jul}\begin{macro}{\Aug}\begin{macro}{\Sep}
% \begin{macro}{\Oct}\begin{macro}{\Nov}\begin{macro}{\Dec}
%    Month names. The default names are in English because that's this
%    author's native language. Month names for other languages are not
%    currently provided, perhaps in some future version. However
%    alternate names can be gotten by hand editing the \fn{.los} file
%    (`listout option settings') to change the definitions of
%    \cw{Jan}, \cw{Feb}, etc. \begin{macrocode}
\def\Jan{January}   \def\Feb{February}
\def\Mar{March}     \def\Apr{April}
\def\May{May}       \def\Jun{June}
\def\Jul{July}      \def\Aug{August}
\def\Sep{September} \def\Oct{October}
\def\Nov{November}  \def\Dec{December}
%    \end{macrocode}
% \end{macro}\end{macro}\end{macro}\end{macro}\end{macro}\end{macro}
% \end{macro}\end{macro}\end{macro}\end{macro}\end{macro}\end{macro}
%
% \begin{macro}{\datesetup}
%    Define \cw{monthname}, \cw{themon}, \cw{theMON} using the
%    value of \cw{month}.
%    \begin{macrocode}
\def\datesetup{%
  \edef\themonthname{%
    \ifcase\month ?% case 0---THIS CAN'T HAPPEN
    \or\Jan\or\Feb\or\Mar\or\Apr\or\May\or\Jun
    \or\Jul\or\Aug\or\Sep\or\Oct\or\Nov\or\Dec
    \else ?\fi}%
%    \end{macrocode}
%    Construct three-letter abbreviations.
%    \begin{macrocode}
  \def\do##1##2##3##4\od{##1##2##3}%
  \edef\themon{\xp@\do\themonthname\od}%
  \let\theMON\themon \changecase\uppercase\theMON
  \if\keepleadingzeros
    \edef\themonth{\twodigits\month}%
    \edef\theday{\twodigits\day}%
  \else
    \edef\themonth{\number\month}%
    \edef\theday{\number\day}%
  \fi
}
\datesetup
%    \end{macrocode}
% \end{macro}
%
% \begin{macro}{\dateA}
%    Date form A: 1992-09-09. The hyphen can be changed to some other
%    character by modifying \cw{datesepchar}.
%    \begin{macrocode}
\def\dateA{%
  \theyear\datesepchar\themonth\datesepchar\theday}
%    \end{macrocode}
% \end{macro}
%
% \begin{macro}{\dateB}
%    Date form B: 92-09-09
%    \begin{macrocode}
\def\dateB{%
  \yearmodC\datesepchar\themonth\datesepchar\theday}
%    \end{macrocode}
% \end{macro}
%
% \begin{macro}{\dateC}
%    Date form C: 09-09-92. This one is ambiguous. Since the author is
%    United-Statesian, let's go with the U.S. convention of month
%    first.
%    \begin{macrocode}
\def\dateC{%
  \themonth\datesepchar\theday\datesepchar\yearmodC}
%    \end{macrocode}
% \end{macro}
%
% \begin{macro}{\dateD}
%    Date form D: 09-09-1992
%    \begin{macrocode}
\def\dateD{%
  \themonth\datesepchar\theday\datesepchar\theyear}
%    \end{macrocode}
% \end{macro}
%
% \begin{macro}{\dateE}
%    Date form E: 09-Sep-92
%    \begin{macrocode}
\def\dateE{%
  \theday\datesepchar\themon\datesepchar\yearmodC}
%    \end{macrocode}
% \end{macro}
%
% \begin{macro}{\dateF}
%    Date form F: 09-Sep-1992
%    \begin{macrocode}
\def\dateF{%
  \theday\datesepchar\themon\datesepchar\theyear}
%    \end{macrocode}
% \end{macro}
%
% \begin{macro}{\dateG}
%    Date form G: 09-SEP-92
%    \begin{macrocode}
\def\dateG{%
  \theday\datesepchar\theMON\datesepchar\yearmodC}
%    \end{macrocode}
% \end{macro}
%
% \begin{macro}{\dateH}
%    Date form H: 09-SEP-1992
%    \begin{macrocode}
\def\dateH{%
  \theday\datesepchar\theMON\datesepchar\theyear}
%    \end{macrocode}
% \end{macro}
%
% \begin{macro}{\dateI}
%    Date form I: September 9, 1992 (\cw{datesepchar} not used)
%    \begin{macrocode}
\def\dateI{%
  \themonthname\space\number\day, \theyear}
%    \end{macrocode}
% \end{macro}
%
% \begin{macro}{\dateJ}
%    Date form J: 9 September 1992 (assuming \cw{datesepchar} set to
%    space).
%    \begin{macrocode}
\def\dateJ{%
  \theday\datesepchar\themonthname\datesepchar\theyear}
%    \end{macrocode}
% \end{macro}
%
% \begin{macro}{\todaysdate}
%    \cw{todaysdate} is assigned with \cw{def} rather than \cw{let} to
%    facilitate writing out the information to an \fn{.los} file.
%    \begin{macrocode}
\def\todaysdate{\dateJ}
%    \end{macrocode}
% \end{macro}
%
% \begin{macro}{\thehour}
%    Create \cw{thehour} holding two-digit number 00\ndash 23, number of
%    hours from \cw{time}.
%    \begin{macrocode}
\count@=\time \divide\count@ 60
\edef\thehour{\twodigits\count@}
%    \end{macrocode}
% \end{macro}
%
% \begin{macro}{\theminute}
%    Create \cw{theminute} holding two-digit number 00\ndash 59, number
%    of minutes from \cw{time} after subtracting $60*\cw{thehour}$.
%    \begin{macrocode}
\multiply\count@ -60 \advance\count@\time
\edef\theminute{\twodigits\count@}
%    \end{macrocode}
% \end{macro}
%
% \begin{macro}{\ampmpunct}
%    \begin{macrocode}
\def\ampmpunct{.}
%    \end{macrocode}
% \end{macro}
%
% \begin{macro}{\ampm}
% \begin{macro}{\thehourmodtwelve}
%    Define \cw{ampm} to be \verb"a" or \verb"p" (or
%    \verb"noon"\meta{gobble} or \verb"midnight"\meta{gobble}), and
%    define \cw{thehourmodtwelve} suitably. Notice that, unlike
%    \cw{thehour}, \cw{thehourmodtwelve} omits a leading zero.
%    \begin{macrocode}
\count@=\thehour\relax
\ifnum\count@>11 \def\ampm{p}%
  \ifnum\count@>12
%    \end{macrocode}
%    Convert hours in the range 13\ndash 23 to the range 1\ndash 11.
%    \begin{macrocode}
    \begingroup \advance\count@ -12
      \aftergroup\edef\aftergroup\thehourmodtwelve
      \aftergroup{\xp@
    \endgroup \number\count@}%
  \else
    \edef\thehourmodtwelve{12}%
  \fi
\else
  \def\ampm{a}%
%    \end{macrocode}
%    Hour 0 needs to be translated to 12:?? a.m.
%    \begin{macrocode}
  \ifnum\count@=0 \count@=12 \fi
  \edef\thehourmodtwelve{\number\count@}%
\fi
%    \end{macrocode}
% \end{macro}
% \end{macro}
%
% \begin{macro}{\AMPM}
%    Note that \cw{AMPM} should not be written to the \fn{.los} file:
%    it needs to be recomputed at run-time using the current value of
%    \cw{time}. The function \cw{changecase} is from \fn{dialog.sty}.
%    \begin{macrocode}
\let\AMPM\ampm \changecase\uppercase\AMPM
%    \end{macrocode}
% \end{macro}
%
% \begin{macro}{\midnoon}
%    If the current time is midnight or noon, \cw{midnoon} prints the
%    appropriate word and gobbles the `a.m.' or `p.m.' string.
%    \begin{macrocode}
\def\midnoon#1{\ifnum\time=0 midnight\else
  \ifnum\time=720 noon\else#1\fi\fi}
%    \end{macrocode}
% \end{macro}
%
% \begin{macro}{\timesepchar}
%    \begin{macrocode}
\def\timesepchar{:}
%    \end{macrocode}
% \end{macro}
%
% \begin{macro}{\timeA}
%    Time form A: 1947
%    \begin{macrocode}
\def\timeA{\thehour\theminute}
%    \end{macrocode}
% \end{macro}
%
% \begin{macro}{\timeB}
%    Time form B: 19:47
%    \begin{macrocode}
\def\timeB{\thehour\timesepchar\theminute}
%    \end{macrocode}
% \end{macro}
%
% \begin{macro}{\timeC}
%    Time form C: 7:47 pm
%    \begin{macrocode}
\def\timeC{\thehourmodtwelve\timesepchar\theminute
  \space\midnoon{\ampm m}}
%    \end{macrocode}
% \end{macro}
%
% \begin{macro}{\timeD}
%    Time form D: 7:47 p.m.
%    \begin{macrocode}
\def\timeD{\thehourmodtwelve\timesepchar\theminute
  \space\midnoon{\ampm\ampmpunct m\ampmpunct}}
%    \end{macrocode}
% \end{macro}
%
% \begin{macro}{\timeE}
%    Time form E: 7:47 PM
%    \begin{macrocode}
\def\timeE{\thehourmodtwelve\timesepchar\theminute
  \space\midnoon{\AMPM M}}
%    \end{macrocode}
% \end{macro}
%
% \begin{macro}{\timeF}
%    Time form F: 7:47 P.M.
%    \begin{macrocode}
\def\timeF{\thehourmodtwelve\timesepchar\theminute
  \space\midnoon{\AMPM\ampmpunct M\ampmpunct}}
%    \end{macrocode}
% \end{macro}
%
% \begin{macro}{\nowtime}
%    \cw{nowtime} is assigned with \cw{def} for the same reason as
%    \cw{todaysdate}.
%    \begin{macrocode}
\def\nowtime{\timeB}
%    \end{macrocode}
% \end{macro}
%
% %%%%%%%%%%%%%%%%%%%%%%%%%%%%%%%%%%%%%%%%%%%%%%%%%%%%%%%%%%%%%%%%%%%%%%
% \section{Output routine}
%
% \begin{macro}{\curcol}
% \begin{macro}{\totalcols}
%    \cw{curcol} is a counter to keep track of the current column
%    number; \cw{totalcols} is a variable that indicates how many
%    columns are desired.
%    \begin{macrocode}
\chardef\curcol=1
\chardef\totalcols=2
%    \end{macrocode}
% \end{macro}
% \end{macro}
%
% \begin{macro}{\firstpageno}
% \begin{macro}{\lastpageno}
%    For multi-column output, a `page' number is associated to each
%    column. At the time when a running head is attached to a column,
%    the current `page' number is always \cw{lastpageno}.
%    \begin{macrocode}
\countdef\firstpageno=1
\countdef\lastpageno=2
%    \end{macrocode}
% \end{macro}
% \end{macro}
%
% \begin{macro}{\prevcolsbox}
%    When the number of columns $n$ is greater than 1, columns $1$,
%    $\ldots$, $n-1$ are accumulated in \cw{prevcolsbox} before being
%    shipped out.
%    \begin{macrocode}
\newbox\prevcolsbox
%    \end{macrocode}
% \end{macro}
%
% \begin{macro}{\paht}
% \begin{macro}{\pawd}
%    Paper height and width.
%    \begin{macrocode}
\newdimen\paht \newdimen\pawd
%    \end{macrocode}
% \end{macro}
% \end{macro}
%
% \begin{macro}{\fullwd}
%    The `full width' is the width of all the $n$ columns plus $(n-1)
%    * \btext{intercolumn}\linebreak[0]\btext{space}$. Normally it is
%    computed at run time from paper size, number of columns and
%    intercolumn space.
%    \begin{macrocode}
\newdimen\fullwd
%    \end{macrocode}
% \end{macro}
%
% \begin{macro}{\colwd}
%    Column width is computed from number of columns, paper size,
%    margins, and intercolumn space, unless it is explicitly set to a
%    given width by the user, in which case \cw{fullwd} is computed from
%    number of columns, column width, and intercolumn space, and a
%    warning is given if \cw{fullwd} plus the margins exceeds paper
%    width. In other words, in the system of equations
% \xdef\restorearraycolsep{\arraycolsep \the\arraycolsep\relax}%
% \arraycolsep1pt\relax
% \begin{eqnarray}
%    \cw{rightmargin} &=& \cw{leftmargin}\label{rl}\\
%    \cw{fullwd} &=& \cw{paperwd} - \cw{leftmargin}\nonumber\\
%                     &&{} - \cw{rightmargin}\\
%    \cw{colwd} &=& \frac{\cw{fullwd}
%                     + \cw{intercolspace}}{\cw{totalcols}}\nonumber\\
%                    &&{} - \cw{intercolspace}
% \end{eqnarray}
% \restorearraycolsep
%    the variables on the left-hand side in each equation are the
%    dependent variables. If the user sets the right margin
%    explicitly, we delete equation (\ref{rl}) from the system; or if
%    the user sets column width explicitly, we change the status of
%    the \cw{colwd} variable from dependent to independent, which may
%    be thought of as switching to the equations
% \xdef\restorearraycolsep{\arraycolsep \the\arraycolsep\relax}%
% \arraycolsep0pt\relax
% \begin{eqnarray}
%    \cw{fullwd} &=& ((\cw{colwd} +\cw{intercolspace})\nonumber\\
%                 && {}\times \cw{totalcols})\nonumber\\
%                 && {}- \cw{intercolspace}\\
%    \cw{rightmargin} &=& \cw{paperwd} - \cw{leftmargin}\nonumber\\
%                      && {}- \cw{fullwd}
% \end{eqnarray}
% \restorearraycolsep
%
%    \begin{macrocode}
\newdimen\colwd
%    \end{macrocode}
% \end{macro}
%
% \begin{macro}{\dependentcolwd}
%    \begin{macrocode}
\let\dependentcolwd\TRUE
%    \end{macrocode}
% \end{macro}
%
% \begin{macro}{\leftmargin}
% \begin{macro}{\rightmargin}
%    \begin{macrocode}
\newdimen\leftmargin
\newdimen\rightmargin
%    \end{macrocode}
% \end{macro}
% \end{macro}
%
% \begin{macro}{\dependentrightmargin}
%    \begin{macrocode}
\let\dependentrightmargin\TRUE
%    \end{macrocode}
% \end{macro}
%
% \begin{macro}{\computerfc}
% \mjd{Maybe change this and related names to computeRFC (caps)?}
%    Function to find \cw{rightmargin}, \cw{fullwd}, and \cw{colwidth}.
%    \begin{macrocode}
\def\computerfc{\relax \rightmargin=\leftmargin
  \computefc }
%    \end{macrocode}
% \end{macro}
%
% \begin{macro}{\computefc}
%
%    Find \cw{fullwd} and \cw{colwd} but leave \cw{rightmargin} as is.
%    \begin{macrocode}
\def\computefc{\fullwd=\pawd
  \advance\fullwd-\leftmargin
  \advance\fullwd-\rightmargin
  \colwd\fullwd
  \advance\colwd \intercolspace
  \divide\colwd \totalcols
  \advance\colwd -\intercolspace
%    \end{macrocode}
%    \cw{columnwidth} is the version of \cw{colwd} reported to the user.
%    If it is set explicitly, the original form is retained, otherwise a
%    width value given by the user as `5cm' would be reported as
%    `142.26378pt'. But here we are computing the column width, so
%    we do not try to make the raw points value more presentable.
%    \begin{macrocode}
  \edef\columnwidth{\the\colwd}%
}
%    \end{macrocode}
% \end{macro}
%
% \begin{macro}{\computefr}
%
%    Find \cw{fullwd} and \cw{rightmargin} from a specified column
%    width.
%    \begin{macrocode}
\def\computefr{\fullwd\colwd
  \advance\fullwd \intercolspace
  \multiply\fullwd \totalcols
  \advance\fullwd -\intercolspace
  \rightmargin\pawd
  \advance\rightmargin-\leftmargin
  \advance\rightmargin-\fullwd
}
%    \end{macrocode}
% \end{macro}
%
% \begin{macro}{\colht}
%    Unless explicitly set to a given value by the user, bottom margin
%    and column height are computed from paper size and top margin,
%    starting with bottom margin = top margin and then rounding the
%    column height to an integral number of lines (with the usual
%    adjustment for \cw{topskip}).
%    \begin{macrocode}
\newdimen\colht
%    \end{macrocode}
% \end{macro}
%
% \begin{macro}{\intercolspace}
% \begin{macro}{\overrun}
%    \begin{macrocode}
\newdimen\intercolspace \newdimen\overrun
%    \end{macrocode}
% \end{macro}
% \end{macro}
%
% \begin{macro}{\runheadht}
% \begin{macro}{\runfootht}
%    \cw{runheadht} is the height of the running head, including space
%    below it. \cw{runfootht} is similar.
%    \begin{macrocode}
\newdimen\runheadht \newdimen\runfootht
%    \end{macrocode}
% \end{macro}
% \end{macro}
%
% \begin{macro}{\bottommargin}
% \begin{macro}{\topmargin}
%    \begin{macrocode}
\newdimen\bottommargin \newdimen\topmargin
%    \end{macrocode}
% \end{macro}
% \end{macro}
%
% \begin{macro}{\computebc}
%    \begin{macrocode}
\def\computebc{\bottommargin=\topmargin \computec}
%    \end{macrocode}
% \end{macro}
%
% \begin{macro}{\computec}
% \mjd{Maybe this should be named computeHB (height/bottom)}
%    This function is called if the user sets a new top or bottom
%    margin, because in that case we need to recompute \cw{colht} to
%    compensate.
%    \begin{macrocode}
\def\computec{%
%    \end{macrocode}
%    Find $\cw{colht} = \cw{paperht} - \cw{topmargin} -
%    \cw{bottommargin}$, rounded off to nearest multiple of the line
%    spacing.
%    \begin{macrocode}
  \colht\paht \advance\colht -\topmargin
  \advance\colht -\bottommargin
  \advance\colht -\runheadht
  \advance\colht -\runfootht
  \roundcolht
%    \end{macrocode}
%    Adjust \cw{bottommargin} to fit with the new \cw{colht}.
%    \begin{macrocode}
  \bottommargin\paht
  \advance\bottommargin -\topmargin
  \advance\bottommargin -\colht
}
%    \end{macrocode}
% \end{macro}
%
% \begin{macro}{\roundcolht}
%    \cw{roundcolht} rounds \cw{colht} to the nearest integer multiple
%    of the line spacing.
%    \begin{macrocode}
\def\roundcolht{%
  \advance\colht .5\baselineskip
  \divide\colht\baselineskip
  \edef\columnheight{\number\colht\space lines}%
  \multiply\colht\baselineskip
%    \end{macrocode}
%    Compensate for \cw{topskip}:
%    \begin{macrocode}
  \advance\colht\topskip
  \advance\colht -\baselineskip
}
%    \end{macrocode}
% \end{macro}
%
% \begin{macro}{\dependentcolht}
% \begin{macro}{\dependentbottommargin}
%    \begin{macrocode}
\let\dependentcolht\TRUE
\let\dependentbottommargin\TRUE
%    \end{macrocode}
% \end{macro}
% \end{macro}
%
% \begin{macro}{\computeb}
%    The function \cw{computeb} is called to recompute \cw{colht} and
%    \cw{bottommargin} if the user gives an explicit value for
%    \cw{colht}.
%    \begin{macrocode}
\def\computeb{%
%    \end{macrocode}
%    Round \cw{colht} off to nearest multiple of the line spacing.
%    \begin{macrocode}
  \roundcolht
%    \end{macrocode}
%    Adjust \cw{bottommargin} to fit with the new \cw{colht}.
%    \begin{macrocode}
  \bottommargin\paht
  \advance\bottommargin -\topmargin
  \advance\bottommargin -\colht
}
%    \end{macrocode}
% \end{macro}
%
% \begin{macro}{\pageno}
%    If \latex/ is being used, \cw{pageno} will be undefined; we then
%    provide a suitable definition.
%    \begin{macrocode}
\ifx\UndEFiNed\pageno \countdef\pageno=0 \fi
%    \end{macrocode}
% \end{macro}
%
% \begin{macro}{\pagenumber}
%    We retain \cw{pageno} as the number of shipouts (\ie., sheets
%    of paper, if printing normally). The logical `page' number within
%    the file being printed, at the time \cw{pagenumber} is called, is
%    always \cw{lastpageno}.
%    \begin{macrocode}
\def\pagenumber{[\number\pageno]\thinspace{%
  \csname\pagenumberfont
    \pagenumberfontsize\endcsname
  \number\lastpageno}}
%    \end{macrocode}
% \end{macro}
%
% \begin{macro}{\landscape}
%    Switch for portrait versus landscape.
%    \begin{macrocode}
\let\landscape=\TRUE
%    \end{macrocode}
% \end{macro}
%
% \begin{macro}{\pagenumbermessage}
%    A message to show at run-time how the pages are being put
%    together. In the simplest case, when there is only one text
%    column per sheet of paper, this message would be silly, so we
%    wrap it in a condition.
%    \begin{macrocode}
\def\pagenumbermessage{%
  \ifnum\totalcols>\@ne
    \message{%
Printing `pages' \the\firstpageno--\the\lastpageno
\space on \if\landscape landscape \fi sheet
\the\pageno}%
%    \end{macrocode}
%    Here I wanted to add the following line, but nay, let it be
%    omitted to support the unlikely, but not totally unreasonable,
%    possibility that different files in the same run might have
%    different number of columns per page.
%    \begin{macrocode}
%  \else \gdef\pagenumbermessage{}%
  \fi}
%    \end{macrocode}
% \end{macro}
%
% \begin{macro}{\Ncols}
%
%    \begin{macrocode}
\def\Ncols{\relax
  \ifnum\curcol=\totalcols
    \pagenumbermessage
    \shipout\hbox to\fullwd{%
      \unhbox\prevcolsbox\curcolbox}%
%    \end{macrocode}
%    \cw{pageno} is the number of the actual sheet of paper on which
%    we are currently printing.
%    \begin{macrocode}
    \global\advance\pageno\@ne
    \global\chardef\curcol \z@
    \ifnum\totalcols>\@ne
      \global\advance\lastpageno \@ne
      \global\firstpageno\lastpageno
    \fi
  \else
    \global\setbox\prevcolsbox\hbox{%
      \unhbox\prevcolsbox\curcolbox
      \hskip\intercolspace plus1fil\relax}%
    \ifnum\totalcols>\@ne
      \global\advance\lastpageno \@ne
    \fi
  \fi
%    \end{macrocode}
%    Maximum number of columns = 20. The preliminary \cw{edef}
%    step is necessary because \cw{mathchardef} makes \cw{curcol} =
%    \cw{relax} before scanning the number on the right-hand side of
%    the assignment, so that it would choke the \cw{ifcase}.
%    \begin{macrocode}
  \edef\next{\global\mathchardef\curcol=
    \ifcase\curcol
      1\or 2\or 3\or 4\or 5\or 6\or 7\or 8\or 9\or
      10\or 11\or 12\or 13\or 14\or 15\or 16\or
      17\or 18\or 19\or 20%
    \else "7FFF\fi\relax
  }%
  \next
}
%    \end{macrocode}
% \end{macro}
%
% \begin{macro}{\runhead}
% \begin{macro}{\runfoot}
%    We use \cw{runhead} and \cw{runfoot} instead of \cw{headline} and
%    \cw{footline}, to reduce the possibility of conflict if this
%    utility is used in conjunction with some other output routine.
%    (Future possibility, not yet implemented: allow calling of this
%    utility from within \latex/ or another macro package with its own
%    output routine.[mjd,20-Sep-1993])
%    \begin{macrocode}
\newtoks\runhead \newtoks\runfoot
%    \end{macrocode}
% \end{macro}
% \end{macro}
%
% \begin{macro}{\loheadline}
%    The test at the beginning here is to avoid problems when the
%    headline is wider than the column width; in that case only print
%    one headline over all the columns instead of a headline for each
%    column. If \qarg{1} = \cw{colwd}, it means that the headline fits
%    within \cw{colwd}: print a headline for each column; otherwise
%    print a headline only if \cw{curcol} = 1, and let it overlap over
%    all the columns.
%    \begin{macrocode}
\def\loheadline#1{\relax
  \if\ifdim#1=\colwd T\else
     \ifnum\curcol=\@ne T\else F\fi\fi T%
    \vbox to\runheadht{\rlap{\hbox to#1{%
      \csname\runheadfont\runheadfontsize\endcsname
      \the\runhead}}\vss}%
    \nointerlineskip
  \fi
}
%    \end{macrocode}
% \end{macro}
%
% \begin{macro}{\lofootline}
%    As for the headline, we print a single footline across all
%    columns if the footline width is greater than \cw{colwd}.
%    \begin{macrocode}
\def\lofootline#1{\relax
  \if \ifdim#1=\colwd T\else
      \ifnum\curcol=1 T\else F\fi\fi T%
    \nointerlineskip
%    \end{macrocode}
%    The \cw{vfil} pushes the footline to the bottom of the vbox; the
%    \cw{vfilneg}\cw{vss} at the end is just a compact way of writing
%    \cw{vskip} {\tt 0pt minus1fil}, meaning let the footline hang out
%    the bottom of the box if necessary, without giving an overfull
%    vbox message.
%    \begin{macrocode}
    \vbox to\runfootht{\vfil\rlap{\hbox to#1{%
      \csname\runheadfont\runheadfontsize\endcsname
      \the\runfoot}}\kern-\prevdepth
      \vfilneg\vss}%
  \fi
}
%    \end{macrocode}
% \end{macro}
%
% \begin{macro}{\setupheadlineandfootline}
%    Define how headline and footline are applied. If the running head
%    or running foot is too wide to fit within \cw{colwd}, then
%    \cw{hfwd} (headline/footline width) is defined to be \cw{fullwd}.
%    Cf.\ the comments for \cw{loheadline}.
%    \begin{macrocode}
\def\setupheadlineandfootline{%
  \gdef\hfwd{\colwd}%
  \if\runheads
    \setbox\z@\hbox{%
      \csname\runheadfont\runheadfontsize\endcsname
      \the\runhead}%
    \ifdim\wd\z@>\colwd \gdef\hfwd{\fullwd}\fi
  \fi
  \if\runfeet
    \setbox\z@\hbox{%
      \csname\runfootfont\runfootfontsize\endcsname
      \the\runfoot}%
    \ifdim\wd\z@>\colwd \gdef\hfwd{\fullwd}\fi
  \fi
  \if\runheads
    \if\runfeet
      \gdef\curcolbox{\vtop{\loheadline\hfwd
        \vbox to\vsize{\unvbox255 \vss}%
        \lofootline\hfwd}}%
    \else
      \gdef\curcolbox{%
        \vtop{\loheadline\hfwd \vtop{\unvbox255 }}}%
    \fi
  \else
    \if\runfeet
      \gdef\curcolbox{\vtop{\null\nointerlineskip
        \vbox to\vsize{\unvbox255 \vss}%
        \lofootline\hfwd}}%
    \else
%    \end{macrocode}
%    Print \cw{null} box for the headline rather than nothing, so that
%    the \cw{vtop} in \cw{curcolbox} will work.
%    \begin{macrocode}
      \gdef\curcolbox{%
        \vtop{\null\unvbox255 }}%
    \fi
  \fi
}
%    \end{macrocode}
% \end{macro}
%
% %%%%%%%%%%%%%%%%%%%%%%%%%%%%%%%%%%%%%%%%%%%%%%%%%%%%%%%%%%%%%%%%%%%%%%
% \section{Tab handling}
%
%    Well, at first I was only going to make tabs expand to a fixed
%    number of spaces regardless of where they happened to fall in the
%    line. This would work fine for tabs that fall at the beginning of a
%    line. But there are certain other reasonable possibilities in
%    program source code, e.g. assembly-language style comments that
%    start in column 40 or so, or tables of information.
%    The alignment in such cases can be preserved by setting a
%    box at the beginning of each line and then, when a tab comes
%    along, measuring the width of the line so far to figure out how
%    many characters it contains. Of course this is impossible unless
%    the font being used is monospace. But given that, and
%    given a specified width in characters of each tab, it's possible to
%    find the distance remaining to the next tab stop.
%
% \begin{macro}{\tableaderbox}
%    Box used with \cw{leaders} to represent tab characters.
%    \begin{macrocode}
\newbox\tableaderbox
%    \end{macrocode}
% \end{macro}
%
% \begin{macro}{\spacespertab}
%    The most common convention for printing tabs is eight spaces.
%    Some people may prefer printing fewer.
%    \begin{macrocode}
\chardef\spacespertab=8
%    \end{macrocode}
% \end{macro}
%
% \begin{macro}{\tabwidth}
%    \cw{tabwidth} is the width to be used for tab characters. Usually
%    a multiple of (monospaced-font) character width and
%    \cw{spacespertab}.
%    \begin{macrocode}
\newdimen\tabwidth
%    \end{macrocode}
% \end{macro}
%
% \begin{macro}{\printtaba}
%    The function \cw{printtaba} is called when we are in the middle
%    of typesetting an \cw{hbox} to \cw{hsize} (cf
%    \cw{newunnumberedline}).
%    \begin{macrocode}
\def\printtaba{%
%    \end{macrocode}
%    Finish the hbox. The \cw{hfil} is just to prevent an underfull
%    hbox message, it will be removed next. Using \cw{hfil} to supress
%    the message seems to be better than using \cw{hbadness}, because
%    the current box is being set to \cw{hsize} instead of natural
%    size so it will have to be reboxed anyway (therefore the
%    \cw{unskip} step costs little); and if \cw{hbadness} were set to
%    10000 then overfull messages would also be suppressed, which is
%    unnecessary and even slightly undesirable.
%    \begin{macrocode}
	  \hfil\egroup
  \setbox\z@\hbox{\unhbox\z@ \unskip}%
%    \end{macrocode}
%    Compute \cw{dimen@}, the distance remaining to the next tab stop
%    (the next integer multiple of \cw{tabwidth} that is greater than
%    the width of box 0).
%    \begin{macrocode}
  \dimen@\wd\z@ \divide\dimen@\tabwidth
  \multiply\dimen@\tabwidth
  \advance\dimen@\tabwidth
  \advance\dimen@ -\wd\z@
%    \end{macrocode}
%    Restart the line, adding leaders to the appropriate width.
%    \begin{macrocode}
  \setbox\z@\hbox to \hsize\bgroup \unhbox\z@
    \spacebreak\hbox to\dimen@{%
      \leaders\copy\tableaderbox\hfil}%
}
%    \end{macrocode}
% \end{macro}
%
% \begin{macro}{\newfileprinttab}
%    Print a message when a tab is encountered in the current file.
%    But after the first tab, don't send any more messages, until a
%    new file is started.
%    \begin{macrocode}
\def\newfileprinttab{\printtaba
  \message{%
Say, this file contains a tab character}%
  \global\let\printtab\printtaba}
%    \end{macrocode}
% \end{macro}
%
% \begin{macro}{\printtab}
%    \begin{macrocode}
\let\printtab=\newfileprinttab
%    \end{macrocode}
% \end{macro}
%
% \begin{macro}{\charwidth}
%    Variable to hold the width of one character in the current font.
%    (Monospace font assumed.)
%    \begin{macrocode}
\newdimen\charwidth
%    \end{macrocode}
% \end{macro}
%
% \begin{macro}{\inittabs}
%    \cw{inittabs} assumes that the proper font has already been
%    selected.
%    \begin{macrocode}
\def\inittabs{%
  \setbox\z@\hbox{m}\charwidth\wd\z@
  \tabwidth\spacespertab\charwidth
  \ifcase\tabselect % case 0: blank tabs option.
    \setbox\tableaderbox\hbox to\charwidth{}%
  \or % case 1: dots option.
    \setbox\tableaderbox\hbox to.2em{%
        \hskip\z@ plus2\p@ minus3\p@
        \mathhexbox 201% \cdot character
        \kern.1em }%
  \or % case 2: dashed option.
%    \end{macrocode}
%    The plus and minus for the first hskip are to avoid underfull
%    hbox messages from rounding errors in summing the em dimensions.
%    \begin{macrocode}
    \setbox\tableaderbox\hbox to.3em{%
        \hskip-.1em plus2sp minus2sp%
        \vrule width.25em height.25em\kern.15em }%
  \or % case 3: hrule option.
    \setbox\tableaderbox\hbox{%
      \vrule width\charwidth height2\p@ depth-1.5\p@}%
  \else % case 4: tiny TAB option.
%    \end{macrocode}
%    The font used for the small letters is always 5pt roman\mdash no
%    user choice here.
%    \begin{macrocode}
    \setbox\tableaderbox\hbox to.5em{\hss
        \vbox to6\p@{%
          \offinterlineskip\csname roman5\endcsname
          \hbox to7\p@{T\hss}\vss
            \hbox to7\p@{\hss\kern-.1em A\hss}\vss
          \hbox to7\p@{\hss B}\kern-2\p@}%
        \hss}%
  \fi
  \let\printtab\newfileprinttab
 }
%    \end{macrocode}
% \end{macro}
%
% \begin{macro}{\tabstyle}
%    \begin{macrocode}
\def\tabstyle{\ifcase\tabselect blank space\or
  dots\or dashed lines\or horizontal rules\else
  tiny TABs\fi}
%    \end{macrocode}
% \end{macro}
%
% %%%%%%%%%%%%%%%%%%%%%%%%%%%%%%%%%%%%%%%%%%%%%%%%%%%%%%%%%%%%%%%%%%%%%%
% \section{Printing non-printable characters}
%    The `invisible' control characters \ascii/ 0\ndash 31 and 127 need
%    special handling if they occur in the text being printed. For
%    \ascii/ 9 and 13, tab and carriage return, there is a generally
%    accepted way to print them, but for the others there are no
%    standards. So the chosen method here is to print a boxed letter:
%    boxed A for \ctrl{A}, boxed B for \ctrl{B}, and so forth. One
%    other exception seems to be convenient: \ctrl{L} (form feed) is
%    defined to act as a `newpage' command (`newcolumn' if
%    \cw{totalcols} is greater than 1).
%
%    For characters 128\ndash 255 there are generally accepted ways to
%    print them. The trouble is, there are too many different
%    generally accepted ways: the Macintosh way, the IBM PC codepage
%    ways (several different codepages), the DC font encoding way, the
%    DEC multinational character set way, the KOI-8 way, and many many
%    others. And most of these ways would require their own font, as
%    they include characters not found in the \fn{cmtt} character set.
%    Therefore I do not attempt here to do anything very useful with
%    characters $>$ 127, but only set them up to print a boxed
%    question mark and a warning about `unspecified character'.
%    If you're interested in helping me expand the capabilities of
%    \fn{listout} in this area, particularly regarding the font
%    complications, let me know.
%
% \begin{macro}{\charboxsidekern}
% \begin{macro}{\charboxstrut}
%    \cw{charboxstrut} is initialized in \cw{listout} after the font
%    is set.
%    \begin{macrocode}
\newdimen\charboxsidekern
\charboxsidekern=.3pt
\newbox\charboxstrut
%    \end{macrocode}
% \end{macro}
% \end{macro}
%
% \begin{macro}{\characterbox}
%    Function to print a box around a given character.
%    \begin{macrocode}
\def\characterbox#1{%
%    \end{macrocode}
%    The width of the hbox is set to preserve alignment of characters
%    across lines if we have a normal tt font with the width of all
%    characters  = .5\units{em}. In order for the enclosed character
%    to be slightly more readable, we allow the vertical rules to
%    partly overlap outside the .5\units{em} space instead of lapping
%    them entirely inward.
%    \begin{macrocode}
  \hbox to.5em{\hss
    \dimen@.22\p@ \dimen@ii -.11\p@
    \vrule width\dimen@ \kern\dimen@ii
    \vbox{\hrule height\dimen@
      \kern\charboxsidekern
      \hbox{\kern\charboxsidekern
        #1\kern\charboxsidekern
%    \end{macrocode}
%    A vertical strut is inserted to ensure a certain minimum height
%    of the box even if say the enclosed character has practically no
%    height (\eg. underscore).
%    \begin{macrocode}
        \copy\charboxstrut}%
      \dimen4 \prevdepth
      \advance\dimen4 \charboxsidekern
      \advance\dimen@ \dimen4
      \hrule depth\dimen@ height -\dimen4
    }%
    \kern\dimen@ii\vrule width\dimen@\hss}}
%    \end{macrocode}
% \end{macro}
%
% \begin{macro}{\controlchara}
%    Define the control characters \ascii/ 0\ndash 31 and 127 to print a
%    warning and a boxed letter. \ctrl{I}, \ctrl{L}, \ctrl{M} (tab,
%    formfeed, carriage return) will receive overriding definitions
%    later.
%    \begin{macrocode}
\def\controlchara#1{%
  \message{Warning:
Invisible control character, printing boxed letter}%
  \characterbox{#1}%
  \gdef\controlchar##1{%
    \message{@}\characterbox{##1}}%
}
%    \end{macrocode}
% \end{macro}
%
% \begin{macro}{\controlchar}
%    \begin{macrocode}
\def\controlchar{\controlchara}
%    \end{macrocode}
% \end{macro}
%
% \begin{macro}{\controlcharassignments}
%    \begin{macrocode}
\newtoks\controlcharassignments
%    \end{macrocode}
% \end{macro}
%
%    Start a group to localize the changes to lccodes, \cw{do}, etc.
%    \begin{macrocode}
\begingroup
%    \end{macrocode}
%    For \eg. \ctrl{A} we want to issue a warning message and print a
%    boxed capital A character. To get the cap A from the control
%    character \ctrl{A} which \cw{controlchars} provides, we turn it
%    into a number and add 64, then use the result with \cw{lccode} to
%    get a category-12 character inside the definition being
%    constructed.
%    \begin{macrocode}
\def\do#1{\count@=`#1 \next \do}
\def\next{\lccode`\~=\count@ \advance\count@ 64
  \ifnum\count@>127 \advance\count@ -128 \fi
  \lccode`\.=\count@
  \lowercase{\let~\relax
    \global\controlcharassignments\xp@{%
    \the\controlcharassignments
    \def~{\controlchar{.}}}}%
}
%    \end{macrocode}
%
%    \begin{macrocode}
\xp@\do\controlchars{a \@gobbletwo}
\endgroup
%    \end{macrocode}
%
% \begin{macro}{\unknownchar}
% \begin{macro}{\unknownchara}
%
%    Define characters in the range \ascii/ 128\ndash 255 to print as a
%    boxed question mark, with a warning. Those who have DC fonts or
%    other fonts with 256 characters instead of 128 might want to change
%    this behavior, but the default is chosen so as not to cause trouble
%    for those who are currently handicapped with 128-character fonts
%    (including myself \ldots). There is also the problem that in an
%    arbitrary file being printed, the original intended interpretation
%    of characters above \ascii/ 127 may vary widely, and more often
%    than not, perhaps, will not correspond properly to the characters
%    in the high 128 positions of the currently selected font. For
%    example, it's not clear how useful it would be to print out a file
%    coded for Russian characters in the high 128 with a font that uses
%    the DC character set.
%
%    \begin{macrocode}
\def\unknownchara{%
  \xmesj{Warning: Unspecified character
    (system-dependent interpretation);\
    printing question mark instead}%
  \characterbox{?}%
  \gdef\unknownchar{\message{?}\characterbox{?}}%
}
%
\def\unknownchar{\unknownchara}
%    \end{macrocode}
% \end{macro}
% \end{macro}
%
% \begin{macro}{\eightbitaction}
%    \begin{macrocode}
\def\eightbitaction{B}
%    \end{macrocode}
% \end{macro}
%
% \begin{macro}{\highASCIIassignments}
% \begin{macro}{\highBSCIIassignments}
%    \begin{macrocode}
\newtoks\highASCIIassignments
\newtoks\highBSCIIassignments
%
\highASCIIassignments{%
  \def\do{\catcode\count@=12 \advance\count@\@ne
    \ifnum\count@>\@cclv \expandafter\@gobble\fi\do}%
  \count@=128 \do}
%
\begingroup
\def\do{%
  \global\catcode\count@\active
  \lccode`\~\count@
  \lowercase{\global\highBSCIIassignments
    \xp@{\the\highBSCIIassignments
      \def~{\unknownchar}}}%
  \ifnum\count@<255\relax
    \advance\count@ 1\relax
    \xp@\do % tail recursion
  \fi}
%
\count@=128\relax \do \endgroup
%    \end{macrocode}
% \end{macro}
% \end{macro}
%
% %%%%%%%%%%%%%%%%%%%%%%%%%%%%%%%%%%%%%%%%%%%%%%%%%%%%%%%%%%%%%%%%%%%%%%
% \section{File name titles}
%
% \begin{macro}{\thisfilename}
%    Variable to hold the name of the current file being printed. To
%    be determined at run-time.
%    \begin{macrocode}
\def\thisfilename{??}
%    \end{macrocode}
% \end{macro}
%
% \begin{macro}{\printtitles}
%    If \cw{printtitles} is true then the name of each file is printed
%    at the beginning of each file as a centered title, with some
%    space above and below.
%    \begin{macrocode}
\let\printtitles=\TRUE
%    \end{macrocode}
% \end{macro}
%
% \begin{macro}{\@dogobble}
%    Auxiliary function to help in ending a recursive \cs{do}
%    operation.
%    \begin{macrocode}
\def\@dogobble#1\do{}
%    \end{macrocode}
% \end{macro}
%
% \begin{macro}{\nonttfont}
%    If \cw{titlefont} is not a tt font, then \qc{\_} will not print
%    properly; to get around this we print the title in math mode with
%    \cw{textfont}1 = \cw{titlefont}, just so that we can use
%    \cw{mathcode}"8000 to make the underscore pseudo-active. This is
%    convenient because all characters of \cw{thisfilename} were
%    converted to category 12 by \cw{xmeaning}, and mathcode "8000
%    allows us to use any underscore characters already present
%    instead of replacing them by active characters.  The alternative
%    would be to pick apart \cw{thisfilename} looking for \qc{\_}
%    characters and replace them by active \qc{\_} or by \cs{\_}; but
%    that seems like more work. For certain other characters (less
%    likely to occur in file names) there is the same problem.
%    \begin{macrocode}
\def\nonttfont#1{%
%    \end{macrocode}
%    Reset mathcodes of all nonalphanumeric visible characters, since
%    some of them (\eg., period) have mathcodes that point to
%    unexpected font positions. Yes, this is overkill, but it's
%    simpler to program and more likely to be robust against unknown
%    future changes in font layouts.
%    \begin{macrocode}
  \def\do##1{\mathcode`##1`##1 \do}%
  \xp@\do\otherchars{.`.\relax \@dogobble}%
%    \end{macrocode}
%    Furthermore, the characters \qc{\<} \qc{\>} \verb*" " (space)
%    \qc{\_} \qc{\{} \qc{\|} \qc{\}} \qc{\~} do not have adequate
%    representatives in their \ascii/ positions in normal \tex/ text
%    fonts (other than \fn{cmtt}). So we mathcode them to pull some
%    reasonable approximations out of \cw{textfont2} (\fn{cmsy10}),
%    except for underscore, which we mathcode to \verb/"8000/ (active)
%    and define to produce a horizontal rule, and space, which we
%    leave alone.
%    \begin{macrocode}
  \mathcode`\<"268 \mathcode`\>"269
  \mathcode`\\"26E \mathcode`\_"8000
  \mathcode`\{"266 \mathcode`\|"26A
  \mathcode`\}"267 \mathcode`\~"218
%    \end{macrocode}
%    Finally, we make the double-quote character active so that it
%    will produce a double-prime symbol rather than a directional
%    close-quotes character, which would be ambiguous with two single
%    apostrophes.
%    \begin{macrocode}
  \mathcode`\""8000
%    \end{macrocode}
%    Change the definition of active \qc{\_} to produce a thicker and
%    wider rule.
%    \begin{macrocode}
  \actively\def\_{%
    \leavevmode\kern.06em\vbox{%
      \hrule width.4em height.06em}}%
%    \end{macrocode}
%    Define active \qc{\"} character to produce double-prime.
%    \begin{macrocode}
  \actively\def\"{^{\prime\prime}}%
  \textfont\z@
    \csname\titlefont\titlefontsize\endcsname
%    \end{macrocode}
%    On general principles we initialize \cw{everymath} and
%    \cw{mathsurround} to make sure they don't do anything abnormal.
%    \begin{macrocode}
    \everymath{}\mathsurround\z@
  $\fam\z@#1$%
}
%    \end{macrocode}
% \end{macro}
%
%    The list of system types and associated directory separator
%    characters is determined arbitrarily as the systems that I happen
%    to have experience with. If you want to see your favorite system
%    added to the list write to me at the e-mail address given above
%    \ldots
%
% \begin{macro}{\DOSdirsepchar}
%    \begin{macrocode}
\escapechar=-1
\edef\DOSdirsepchar{\string\\}
\escapechar=`\\
%    \end{macrocode}
% \end{macro}
%
% \begin{macro}{\Macintoshdirsepchar}
%    \begin{macrocode}
\def\Macintoshdirsepchar{:}
%    \end{macrocode}
% \end{macro}
%
% \begin{macro}{\Unixdirsepchar}
%    \begin{macrocode}
\def\Unixdirsepchar{/}
\xp@\def\csname VAX/VMSdirsepchar\endcsname{]}
%    \end{macrocode}
% \end{macro}
%
% \begin{macro}{\chopname}
%
%    The function \cw{chopname} chops up a filename looking for the
%    characters that separate directory components from filename
%    components; this is in order to add an \cw{allowbreak} between
%    directory and filename if an explicit directory is included in
%    the given file name, so that when printing the file name in a
%    title, it will break into two lines if there isn't enough room
%    for the whole thing to fit on one line.
%
%    \cw{chopname} is intended to be called only after \cw{xmeaning} has
%    been applied to convert all constituent characters of \qarg{1} to
%    category 12.
%
%    Test cases:
%
%^^V \def\systemtype{Unix}
%^^V \def\test{goo/bar/flab/rtex}
%^^V \chopname\test \show\test
%^^V \def\test{rtex}
%^^V \chopname\test \show\test
%^^V
%^^V \def\systemtype{VAX/VMS}
%^^V \def\test{goo:[bar.flab]rtex}
%^^V \chopname\test\show\test
%^^V \def\test{rtex}
%^^V \chopname\test\show\test
%
%    \begin{macrocode}
\def\chopname#1{%
  \edef\dirsepchar{%
    \csname\systemtype dirsepchar\endcsname}%
  \edef\@tempa{%
    \def\nx@\do####1\dirsepchar}%
  \@tempa
%    \end{macrocode}
%    What we get here from \cw{@tempa} is something like
%    the following line.
%    \begin{macrocode}
% \def\do##1/%
  ##2{%
    \toks@\xp@{\next@}%
    \ifx\end##2% then just add ##1 at the end
      \edef\next@{\the\toks@##1}%
    \else
      \edef\next@{\the\toks@##1%
        \dirsepchar\nx@\allowbreak}%
      \afterfi\do##2%
    \fi
  }%
  \edef\@tempa{\toks@{}\def\nx@\next@{}%
    \nx@\do#1\dirsepchar\nx@\end}%
  \@tempa
  \let#1\next@
%    \end{macrocode}
%    And finally, if \cw{dirprefix} is not empty, add it.
%    \begin{macrocode}
  \ifx\@empty\dirprefix
  \else
    \edef#1{\xmeaning\dirprefix\allowbreak#1}%
  \fi
}
%    \end{macrocode}
% \end{macro}
%
% \begin{macro}{\printtitle}
%    Function to typeset the title of the current file.
%    \begin{macrocode}
\def\printtitle{%
%    \end{macrocode}
%    This \cw{bigskip} will be discarded following \tex/'s normal rules
%    if each file starts on a new page or column.
%    \begin{macrocode}
  \bigskip
  \centerline{%
    \csname\titlefont\titlefontsize\endcsname
%    \end{macrocode}
%    Add an \cw{allowbreak} between directory and filename if an
%    explicit directory is included in the given file name.
%    \begin{macrocode}
    \chopname\thisfilename
%    \end{macrocode}
%    If the interword stretch of \cw{titlefont} is 0, then it is
%    [almost surely] a tt font. But otherwise, special handling is
%    required in order to make sure that the characters
%    \verb/_ < > \ { | } ~/ print OK.
%    \begin{macrocode}
    \ifdim\fontdimen3\font=\z@
      \csname\titlecase\endcsname
        \xp@{\thisfilename}%
    \else
      \csname\titlecase\endcsname
        \xp@{\xp@
        \nonttfont\xp@{\thisfilename}}%
    \fi
  }% end \centerline
%    \end{macrocode}
%    If new file doesn't start a new column, use a smaller space below
%    the title than would otherwise be used.
%    \begin{macrocode}
  \if S\newfileoption \smallskip
  \else\bigskip\fi
}
%    \end{macrocode}
% \end{macro}
%
% \begin{macro}{\newfileoption}
%    Variable that selects the desired action at the beginning of a
%    new file.
%    \begin{macrocode}
\def\newfileoption{P}
%    \end{macrocode}
% \end{macro}
%
%    \mjd{Is \cw{eject} defined in LaTeX?}
% \begin{macro}{\newfileS}
%    Option S (space): Tack on the next file immediately after the
%    previous one, without even ending the current column (unless
%    there are only a couple of lines of space left on it). Assumes
%    that \cw{printtitle} will add some vertical space.
%    \begin{macrocode}
\def\newfileS{%
  \dimen@\pagegoal \advance\dimen@-\pagetotal
  \ifdim\dimen@<3\baselineskip
  \xp@\eject\fi
}
%    \end{macrocode}
% \end{macro}
%
% \begin{macro}{\newfileC}
%    Option C: Finish the current column.
%    \begin{macrocode}
\def\newfileC{\vfil\eject}
%    \end{macrocode}
% \end{macro}
%
% \begin{macro}{\newfileP}
%    Option P: Force a new page (= sheet of paper).
%    \begin{macrocode}
\def\newfileP{%
  \vfil\eject
  \loop
%    \end{macrocode}
%    Repeat until \cs{curcol} resets to 1.
%    \begin{macrocode}
  \ifnum\curcol>\@ne
    \hbox to\colwd{%
          \vrule width\z@ depth\vsize\hfil}%
    \eject
  \repeat
}
%    \end{macrocode}
% \end{macro}
%
% \begin{macro}{\newfileR}
%    Option R: Force right-hand column (\cw{totalcols} = 2 assumed).
%    \begin{macrocode}
\def\newfileR{\vfil\eject
  \ifnum\curcol=\@ne
    \hbox to\colwd{%
      \vrule width\z@ depth\vsize\hfil}%
  \xp@\eject
  \fi
}
%    \end{macrocode}
% \end{macro}
%
% %%%%%%%%%%%%%%%%%%%%%%%%%%%%%%%%%%%%%%%%%%%%%%%%%%%%%%%%%%%%%%%%%%%%%%
% \section{Line numbering}
%
% \begin{macro}{\linenumberfreq}
% \begin{macro}{\linecount}
% \begin{macro}{\linesubcount}
%    \begin{macrocode}
\newcount\linenumberfreq \linenumberfreq=1
\newcount\linecount
\newcount\linesubcount
%    \end{macrocode}
% \end{macro}
% \end{macro}
% \end{macro}
%
% \begin{macro}{\printlinenumber}
%
%    \begin{macrocode}
\def\printlinenumber{\llap{%
  \global\advance\linecount\@ne
  \global\advance\linesubcount\@ne
  \ifnum\linesubcount=\linenumberfreq
    \lnfnt \the\linecount\space
    \global\linesubcount\z@
  \fi
}}%
%    \end{macrocode}
% \end{macro}
%
% \begin{macro}{\spacebreak}
%    \begin{macrocode}
\def\spacebreak{}
%    \end{macrocode}
% \end{macro}
%
% \begin{macro}{\markedbreak}
%    \begin{macrocode}
\def\markedbreak{\penalty\z@}
%    \end{macrocode}
% \end{macro}
%
% \begin{macro}{\noextralinebreaks}
%    \begin{macrocode}
\let\noextralinebreaks=\TRUE
%    \end{macrocode}
% \end{macro}
%
% \begin{macro}{\newunnumberedline}
%    \begin{macrocode}
\def\newunnumberedline{\hskip\rightskip \egroup
  \box\z@ \controlLbreak
  \setbox\z@\hbox to\hsize\bgroup}%
%    \end{macrocode}
% \end{macro}
%
% \begin{macro}{\newnumberedline}
%    \begin{macrocode}
\def\newnumberedline{%
  \newunnumberedline\printlinenumber}%
%    \end{macrocode}
% \end{macro}
%
% \begin{macro}{\unnumberedbreakingline}
%    \begin{macrocode}
\def\unnumberedbreakingline{\relax
%    \end{macrocode}
%    \cw{noindent} \cw{unhbox}0 \cw{endgraf} produces only \cw{parskip}
%    if box 0 is empty, when what we really want is a full baselineskip
%    (a blank line) because box 0 empty means two consecutive \ctrl{M}
%    characters; so we check if box0 is empty by looking at
%    \cw{lastpenalty} to see if it is the same as the penalty that we
%    added at the beginning and if so throw in a \cw{null} to make sure
%    the line won't disappear.
%    \begin{macrocode}
  \ifnum\lastpenalty=\@ne \null\fi
  \hskip\rightskip\egroup
  \hfuzz \p@
  \noindent\unhbox\z@ %
%    \end{macrocode}
%    Notice that the normal \tex/ action at end of paragraph will
%    remove the \cw{hskip} that was added before the \cw{egroup} of the
%    \cw{setbox} operation. We also employ a hangindent of 1em to give
%    some indication of line breaks, if they occur.
%    \begin{macrocode}
  {\hangindent1em
  \parfillskip\z@ plus1fil \endgraf}%
%    \end{macrocode}
%    We need to prevent overfull rules from being added inside at the
%    end of the \cw{hbox}, because there's no way to break off the rule
%    from the end of the horizontal list once it gets in there.
%    \begin{macrocode}
  \hfuzz\maxdimen
  \controlLbreak
  \setbox\z@\hbox to\hsize\bgroup \penalty\@ne
}%
%    \end{macrocode}
% \end{macro}
%
% \begin{macro}{\numberedbreakingline}
%
%    \begin{macrocode}
\def\numberedbreakingline{%
  \unnumberedbreakingline\printlinenumber}
%    \end{macrocode}
% \end{macro}
%
% \begin{macro}{\@sptoken}
%
%    \begin{macrocode}
\lowercase{\let\@sptoken= } %
%    \end{macrocode}
% \end{macro}
%
% \begin{macro}{\CR}
%    Action performed at end of line. Can vary according to whether
%    line numbers are wanted and whether overlong lines are allowed to
%    break.
%    \begin{macrocode}
\let\CR\newunnumberedline
%    \end{macrocode}
% \end{macro}
%
% \begin{macro}{\CRSPsetup}
%
%    Setup for carriage returns and spaces. We need to prevent spaces
%    from disappearing at the beginning of a line, and also optionally
%    prevent line breaks.
%
%    \begin{macrocode}
\def\CRSPsetup{%
  \setupbreakingchars
  \if\noextralinebreaks
    \overrun\rightmargin
%    \end{macrocode}
%    NOTE: the space after the \qc{\=} is mandatory because of the way
%    \cw{@sptoken} is defined:
%    \begin{macrocode}
    \actively\let\ = \@sptoken\relax
    \if\linenumbers
      \let\CR\newnumberedline
    \else
      \let\CR\newunnumberedline
    \fi
  \else
    \actively\edef\ {\spacebreak\space}%
    \if\linenumbers
      \let\CR\numberedbreakingline
    \else
      \let\CR\unnumberedbreakingline
    \fi
  \fi
  \actively\let\^^M\CR
  \rightskip\z@ plus\hsize minus\overrun
%    \end{macrocode}
%    Parfillskip not needed, \cw{rightskip} suffices.
%    \begin{macrocode}
  \parfillskip\z@skip
}
%    \end{macrocode}
% \end{macro}
%
% %%%%%%%%%%%%%%%%%%%%%%%%%%%%%%%%%%%%%%%%%%%%%%%%%%%%%%%%%%%%%%%%%%%%%%
% \section{Main printing function}
%
% \begin{macro}{\listout}
%    The \cw{listout} function takes a file name argument, the name of
%    a file to be printed verbatim.
%    \begin{macrocode}
\def\listout#1{\endgraf
  \begingroup
%    \end{macrocode}
%    Setup that may not need to be done for each file. If
%    \cw{interactive} is \cw{TRUE}, this is reset to no-op since it's
%    not possible for the user to change any settings between files.
%    \begin{macrocode}
  \listoutsetup
%    \end{macrocode}
%    Convert \qarg{1} to all category 12 characters using \cw{meaning}.
%    This prevents problems when we print the file name if it contains
%    things like \qc{\$} or \qc{\_} characters.
%    \begin{macrocode}
  \gdef\thisfilename{#1}%
  \xdef\thisfilename{\xmeaning\thisfilename}%
  \if\printtitles \printtitle \fi
%    \end{macrocode}
%    Setup up needing to be redone for each file, or that only applies
%    to the body of the file, not to the title.
%    \begin{macrocode}
  \listoutmoresetup
  \CRSPsetup
  \hfuzz\maxdimen
  \setbox\z@\hbox to\hsize\bgroup \penalty\@ne
  \input#1 \hfil\egroup
%    \end{macrocode}
%    Ignore the final contents of box 0 since they are just the (null)
%    material between the file-ending \qc{\^^M} and the above
%    \cs{egroup}.
%
%    Reset \cw{hfuzz} to normal in case temporary value of
%    \cw{maxdimen} persists somehow.
%    \begin{macrocode}
  \hfuzz\p@
%    \end{macrocode}
%    Add a penalty to force \cw{pagetotal} to be updated with the
%    height of the preceding line.
%    \begin{macrocode}
  \penalty0 %
  \csname newfile\newfileoption\endcsname
%    \end{macrocode}
%    Ensure that everything gets shipped out even if this is the last
%    file in a noninteractive use of \fn{listout.tex} (where control may
%    return after the \cw{endgroup} to a different output routine).
%    This is a complicated question. The \cw{patchoutput} possibility
%    to make this work from inside a larger macro package is commented
%    out until such time as it can be done right.[mjd,21-Sep-1993]
%    \begin{macrocode}
  \ifvoid\prevcolsbox
  \else
    \savelistoutimage
%{\tracingonline=1 \showboxbreadth=\maxdimen
% \showboxdepth=9 \showbox\prevcolsbox}%
%    \afterfi\patchoutput
  \fi
  \endgroup }
%    \end{macrocode}
% \end{macro}
%
% \begin{macro}{\patchedoutputroutine}
%
%    \begin{macrocode}
%\let\patchedoutputroutine\FALSE
%    \end{macrocode}
% \end{macro}
%
% \begin{macro}{\patchoutput}
%
%    \begin{macrocode}
%\def\patchoutput#1{% #1 to remove the \endgroup
%  \endgroup
%  \relax\if\patchedoutputroutine
%  \else
%    \global\let\patchedoutputroutine\TRUE
%    \edef\next{%
%      \output{%
%        \box\prevcolsbox
%        \global\holdinginserts\the\holdinginserts\relax
%        \unvbox255 \relax
%        \global\let\nx@\patchedoutputroutine\nx@\FALSE
%%    Restore the previous output routine. Interestingly,
%%    \cw{the}\cw{output} returns an extra pair of braces
%%    (as hinted at in the \tex/book); so the extra step
%%    with \cw{toks@} here is slightly better than simply
%%    \cw{global}\cw{output}{\cw{the}\cw{output}}.
%        \toks@{\the\output}%
%        \global\nx@\xp@\output\nx@\the\toks@
%      }%
%%\showthe\output
%    }%
%    \global\holdinginserts1
%  \fi
%  \next}
%    \end{macrocode}
% \end{macro}
%
% \begin{macro}{\savelistoutimage}
%    For \cw{listoutimage} we only need to save those parameters that
%    are used when shipping out a page.
%    \begin{macrocode}
\def\savelistoutimage{%
  \xdef\listoutimage{%
    \let\nx@\landscape%
      \if\landscape\nx@\TRUE
      \else\nx@\FALSE\fi
    \pawd\the\pawd \paht\the\paht
    \topmargin\the\topmargin
    \bottommargin\the\bottommargin
    \leftmargin\the\leftmargin
    \rightmargin\the\rightmargin
    \normalbaselineskip\the\normalbaselineskip
    \normalbaselines
    \mathchardef\curcol\number\curcol\relax
    \chardef\totalcols\number\totalcols\relax
    \colwd\the\colwd \colht\the\colht
    \fullwd\the\fullwd
    \intercolspace\the\intercolspace
    \overrun\the\overrun
    \def\nx@\todaysdate{\todaysdate}%
    \def\nx@\nowtime{\nowtime}%
    \def\nx@\pagenumberfont{\pagenumberfont}%
    \def\nx@\pagenumberfontsize{%
      \pagenumberfontsize}
    \pageno\the\pageno
    \lastpageno\the\lastpageno
    \firstpageno\the\firstpageno
    \def\nx@\hfwd{\hfwd}%
    \def\nx@\thisfilename{\thisfilename}%
    \let\nx@\runheads%
      \if\runheads\nx@\TRUE
      \else\nx@\FALSE\fi
    \runhead{\the\runhead}%
    \def\nx@\runheadfont{\runheadfont}%
    \def\nx@\runheadfontsize{%
      \runheadfontsize}%
    \runheadht\the\runheadht
    \let\nx@\runfeet%
      \if\runfeet\nx@\TRUE
      \else\nx@\FALSE\fi
    \runfoot{\the\runfoot}%
    \def\nx@\runfootfont{\runfootfont}%
    \def\nx@\runfootfontsize{%
      \runfootfontsize}%
    \runfootht\the\runfootht
    \hoffset\the\hoffset \voffset\the\voffset
    \relax
  }%
}
%    \end{macrocode}
% \end{macro}
%
% \begin{macro}{\reportlayout}
%    Some of the line breaks in the following code are merely to
%    facilitate printing the documentation in narrow columns.
%    \begin{macrocode}
\fxmesj\reportlayout{
Using paper size &\the&\pawd x &\the&\paht%
 (&\paperwidth x &\paperheight),
&\number&\totalcols columns, intercolumnspace%
 = &\the&\intercolspace,
column width = &\the&\colwd, column height%
 = &\the&\colht,
top margin = &\the&\topmargin, left margin%
 = &\the&\leftmargin.
}
%    \end{macrocode}
% \end{macro}
%
% \begin{macro}{\listoutsetup}
%
%    \begin{macrocode}
\def\listoutsetup{%
  \output{\Ncols}%
  \loadfonts
  \csname\mainfont\mainfontsize\endcsname
%    \end{macrocode}
%    Preexpand line number font for faster execution, since it is
%    repeated on every line if line numbering is turned on.
%    \begin{macrocode}
  \xp@\let\xp@\lnfnt
     \csname\linenumberfont
       \linenumberfontsize\endcsname
  \normalbaselines
  \if\dependentcolht
    \if\dependentbottommargin \computebc
    \else \computec\fi
  \else
    \computeb
  \fi
  \vsize\colht
  \parskip \z@\relax
  \if\dependentcolwd
    \if\dependentrightmargin \computerfc
    \else \computefc \fi
  \else
    \computefr
  \fi
  \hsize\colwd
  \parindent\z@
%    \end{macrocode}
%    Reduce the size of \cw{overfullrule} so that if a too-long line
%    overprints, there is a better chance to read what lies underneath
%    at the end. But we still get visual identification of overlong
%    lines.
%    \begin{macrocode}
  \ifdim\overfullrule>\z@ \overfullrule 2\p@\fi
  \setupheadlineandfootline
  \datesetup
  \voffset\m@ne truein \hoffset\voffset
  \advance\voffset\topmargin \advance\hoffset\leftmargin
  \setbox\charboxstrut\hbox{%
    \vrule height.5emwidth\z@}%
  \inittabs
%    \end{macrocode}
%    If extra line breaks are allowed, we want to avoid page breaks in
%    the middle of something that was originally one line. Setting
%    \cw{clubpenalty} and \cw{widowpenalty} to these values means that a
%    line will have to break into four lines or more before a page
%    break will be allowed, and always the first two and last two line
%    pieces will remain together. But we don't prohibit page breaks
%    within a broken line entirely (\cw{interlinepenalty} = 10000)
%    because that might be undesirable in extreme cases.
%    \begin{macrocode}
  \clubpenalty\@M \widowpenalty\@M
  \count@12\relax
  \def\do##1{\catcode`##1\count@ \do}%
  \xp@\do\otherchars{a11 \@gobbletwo}%
  \count@\active
  \xp@\do\controlchars{a11 \@gobbletwo}%
  \the\controlcharassignments
  \global\let\controlchar\controlchara
  \if L\controlLaction
    \actively\let\^^L\linespaceL
  \else
    \if N\controlLaction
      \actively\let\^^L\ejectL
    \fi
  \fi
  \the\csname high\eightbitaction SCIIassignments\endcsname
  \global\let\unknownchar\unknownchara
%    \end{macrocode}
%    Standard disablement of \verb/?`/ and \verb/!`/ ligatures found in
%    \fn{cmtt} fonts:
%    \begin{macrocode}
  \actively\def\`{\kern\z@`}%
  \hfuzz\p@
}
%    \end{macrocode}
% \end{macro}
%
% \begin{macro}{\linespaceL}
%    \begin{macrocode}
\def\linespaceL{\message{Found a ^L character}\CR}
%    \end{macrocode}
% \end{macro}
%
% \begin{macro}{\ejectL}
%    \begin{macrocode}
\def\ejectL{%
  \message{Found a ^L character}%
  \gdef\controlLbreak{\eject}\CR\global\let\controlLbreak\relax
}
%    \end{macrocode}
% \end{macro}
%
% \begin{macro}{\controlLbreak}
%    \begin{macrocode}
\let\controlLbreak\relax
%    \end{macrocode}
% \end{macro}
%
% \begin{macro}{\controlLaction}
%    \begin{macrocode}
%    Define form-feed to do a `newcolumn' operation. Remove outerness
%    at the same time. Possible actions for \cw{controlLaction} are B,
%    L, N: print boxed character, print one-line space, or start new
%    column/page.
%    \begin{macrocode}
\def\controlLaction{N}
\actively\let\^^L=\ejectL
%    \end{macrocode}
% \end{macro}
%
% \begin{macro}{\listoutmoresetup}
%    \begin{macrocode}
\def\listoutmoresetup{%
  \global\linecount\z@ \global\linesubcount\z@
  \ifnum\totalcols>\@ne
    \global\firstpageno\@ne
    \global\lastpageno\@ne
  \fi
  \relax
  \frenchspacing
%    \end{macrocode}
%    Define tab to be visible.
%    \begin{macrocode}
  \global\let\printtab\newfileprinttab
  \actively\def\^^I{\printtab}%
  \langnohyphens
}
%    \end{macrocode}
% \end{macro}
%
% \begin{macro}{\DVImesj}
%    \cw{DVImesj} remains null unless the test at the
%    entry point for the menu system to see if \fn{listout} is being
%    run interactively returns true.
%    \begin{macrocode}
\def\DVImesj{}
%    \end{macrocode}
% \end{macro}
%
% \begin{macro}{\landscapereminder}
%    Here once again a couple of extra line breaks will be observed in
%    the printed documentation, introduced to avoid margin overrun in
%    narrow-column printing.
%    \begin{macrocode}
\fxmesj\landscapereminder{%
********************************************%
***********************
*     REMINDER: print in LANDSCAPE mode     %
                      *
********************************************%
***********************}
%    \end{macrocode}
% \end{macro}
%
% \begin{macro}{\listoutfinish}
%
%    \begin{macrocode}
\def\listoutfinish{%
%    \end{macrocode}
%    If part of the last file is pending in \cw{prevcolsbox}, we need
%    to make sure we print it here.
%    \begin{macrocode}
  \ifvoid\prevcolsbox
  \else
    \wlog{Clearing out previous file}%
    \begingroup
%    \end{macrocode}
%    Reset parameters that were turned off by the \cw{endgroup} in
%    \cw{listout}.
%    \begin{macrocode}
    \listoutimage \output{\Ncols}%
%    \end{macrocode}
%    Change newfile action to `new page', then run it, so that we fill
%    out the page with blank columns if necessary.
%    \begin{macrocode}
    \def\newfileoption{P}%
    \hbox to\colwd{}%
    \csname newfile\newfileoption\endcsname
%    \end{macrocode}
%    Message about the name of the DVI file
%    \begin{macrocode}
    \DVImesj
%    \end{macrocode}
%    If \cw{landscape} switch is true, reminder to use the `landscape'
%    switch of the DVI driver.
%    \begin{macrocode}
    \if\landscape
%    \end{macrocode}
%    For those drivers that can handle it (\fn{dvips}, \ldots ?), this
%    \cw{special} command is convenient to have in every job.
%    \begin{macrocode}
%     \cw{special}{landscape}%
      \landscapereminder
    \fi
    \endgroup
  \fi
}
%    \end{macrocode}
% \end{macro}
%
% %%%%%%%%%%%%%%%%%%%%%%%%%%%%%%%%%%%%%%%%%%%%%%%%%%%%%%%%%%%%%%%%%%%%%%
% \section{Fonts}
% \begin{macro}{\xfont}
%    \begin{macrocode}
\def\xfont#1={%
  \xp@\font\csname#1\endcsname=}
%    \end{macrocode}
% \end{macro}
%
% \begin{macro}{\ftypewriter}\begin{macro}{\fbold}
% \begin{macro}{\froman}\begin{macro}{\fitalic}
% \begin{macro}{\fsansserif}
%    \begin{macrocode}
\def\ftypewriter{cmtt}
\def\fbold{cmbx}
\def\froman{cmr}
\def\fitalic{cmti}
\def\fsansserif{cmss}
%    \end{macrocode}
% \end{macro}\end{macro}\end{macro}\end{macro}\end{macro}
%
% \begin{macro}{\loadfont}
%    Load the given font. Args \qarg{1} and \qarg{2} are the
%    descriptive name and the type size, respectively.
%    \begin{macrocode}
\def\loadfont#1#2{%
  \relax\ifnum#2>100 \loadscaledfont{#1}{#2}%
  \else
    \global\xfont#1#2=\csname f#1\endcsname
    \ifnum#2=14 10 scaled\magstep2
    \else\ifnum#2=12 10 scaled\magstep1
    \else\ifnum#2=11 10 scaled\magstephalf
    \else#2\fi\fi\fi
    \relax
  \fi
}
%    \end{macrocode}
% \end{macro}
%
% \begin{macro}{\loadscaledfont}
%    \begin{macrocode}
\def\loadscaledfont#1#2{\global\xfont#1#2=#1 scaled #2\relax}
%    \end{macrocode}
% \end{macro}
%
%    Default font is cmtt8. Note that cmtt7 and cmtt6 are not
%    standardly available to all users.
%
%    \begin{macrocode}
\loadfont{typewriter}{8} % for main text
\loadfont{bold}{10} % for titles
\loadfont{roman}{5} % for line numbers
\loadfont{roman}{6} % for running heads
%    \end{macrocode}
%
% \begin{macro}{\mainfont}
% \begin{macro}{\mainfontsize}
%    \begin{macrocode}
\def\mainfont{typewriter}
\def\mainfontsize{8}
\csname\mainfont\mainfontsize\endcsname
%    \end{macrocode}
% \end{macro}
% \end{macro}
%
% \begin{macro}{\titlefont}
% \begin{macro}{\titlefontsize}
%    \begin{macrocode}
\def\titlefont{bold}
\def\titlefontsize{10}
%    \end{macrocode}
% \end{macro}
% \end{macro}
%
% \begin{macro}{\linenumberfont}
% \begin{macro}{\linenumberfontsize}
%    \begin{macrocode}
\def\linenumberfont{typewriter}
\def\linenumberfontsize{8}
%    \end{macrocode}
% \end{macro}
% \end{macro}
%
% \begin{macro}{\pagenumberfont}
% \begin{macro}{\pagenumberfontsize}
%    \begin{macrocode}
\def\pagenumberfont{bold}
\def\pagenumberfontsize{7}
%    \end{macrocode}
% \end{macro}
% \end{macro}
%
% \begin{macro}{\runheadfont}
% \begin{macro}{\runheadfontsize}
%    \begin{macrocode}
\def\runheadfont{roman}
\def\runheadfontsize{7}
%    \end{macrocode}
% \end{macro}
% \end{macro}
%
% \begin{macro}{\runfootfont}
% \begin{macro}{\runfootfontsize}
%    \begin{macrocode}
\def\runfootfont{roman}
\def\runfootfontsize{7}
%    \end{macrocode}
% \end{macro}
% \end{macro}
%
% \begin{macro}{\loadfonts}
%    \cw{loadfonts} is called just before the start of typesetting, to
%    ensure that any needed fonts will be loaded. In fact, for maximum
%    flexibility, it is called at the beginning of each file that is
%    printed, so that if you use this utility in non-interactive mode
%    with a sequence of \cw{listout} commands you could conceivably
%    change fonts between files.
%
%    \begin{macrocode}
\def\loadfonts{\relax
  \xp@\ifx\csname\mainfont\mainfontsize\endcsname\relax
    \loadfont\mainfont\mainfontsize
  \fi
  \xp@\ifx\csname\pagenumberfont
        \pagenumberfontsize\endcsname\relax
    \loadfont\pagenumberfont\pagenumberfontsize
  \fi
  \if\printtitles
    \xp@\ifx\csname\titlefont
          \titlefontsize\endcsname\relax
      \loadfont\titlefont\titlefontsize
    \fi
  \fi
  \if\linenumbers
    \xp@\ifx\csname\linenumberfont
          \linenumberfontsize\endcsname\relax
      \loadfont\linenumberfont\linenumberfontsize
    \fi
  \fi
  \if\runheads
    \xp@\ifx\csname\runheadfont
          \runheadfontsize\endcsname\relax
      \loadfont\runheadfont\runheadfontsize
    \fi
  \fi
  \if\runfeet
    \xp@\ifx\csname\runfootfont
          \runfootfontsize\endcsname\relax
      \loadfont\runfootfont\runfootfontsize
    \fi
  \fi
}
%    \end{macrocode}
%
% %%%%%%%%%%%%%%%%%%%%%%%%%%%%%%%%%%%%%%%%%%%%%%%%%%%%%%%%%%%%%%%%%%%%%%
% \section{Miscellaneous}
%
%    Trace memory statistics to see how close to typical limits we are
%    coming.
%
%    \begin{macrocode}
\tracingstats=1
%    \end{macrocode}
%
%    Language with no hyphenation patterns, to inhibit hyphenation.
%    \begin{macrocode}
\ifx\undefined\language
  \def\langnohyphens{\hyphenchar\font\m@ne}
\else
  \xp@\ifx\csname newlanguage\endcsname\relax
    \def\langnohyphens{\language\@cclv}
  \else
    \csname newlanguage\endcsname
      \nohyphenslanguage
    \def\langnohyphens{%
      \language\nohyphenslanguage}%
  \fi
\fi
%    \end{macrocode}
% \end{macro}
%
% \begin{macro}{\setupbreakingchars}
%    Function to make the characters specified by \cw{breakingchars}
%    active. The break penalty goes after the character.
%    \begin{macrocode}
\def\setupbreakingchars{%
  \ifx\breakingchars\@empty
    \ifx\spacebreak\@empty
      \let\noextralinebreaks\TRUE
    \else
      \let\noextralinebreaks\FALSE
    \fi
  \else
    \let\noextralinebreaks\FALSE
    \def\do##1{\ifx\end##1\else
      \xp@\actively\xp@
      \def\csname##1\endcsname{##1\markedbreak}%
      \xp@\do\fi}%
    \xp@\do\breakingchars\end
    \let\spacebreak\markedbreak
  \fi
%    \end{macrocode}
%    No line breaks allowed after hyphens.
%    \begin{macrocode}
  \exhyphenpenalty\@M
}
%    \end{macrocode}
%
% \begin{macro}{\iwlos}
%    Function to write a line to the \fn{.los} file. The argument
%    \qarg{1} may vary in order to achieve special effects.
%    \begin{macrocode}
\def\iwlos#1{\immediate\write#1}
%    \end{macrocode}
%
%    \begin{macrocode}
\storexmesj\losfirstline{%
\% &\losname.los &\todaysdate &\nowtime
\% Option settings for listout.tex.}%
%    \end{macrocode}
% \end{macro}
%
% \begin{macro}{\addmonth}
%    An auxiliary function to add month names to \cw{toks2}.
%    \begin{macrocode}
\def\addmonth#1{%
  \toks4 \xp@{#1}%
  \edef\next{\the\toks2
    \string\def\string#1{\the\toks4 }}%
  \toks2 \xp@{\next}%
}
%    \end{macrocode}
% \end{macro}
%
%    \begin{macrocode}
\begingroup
\endlinechar13 \catcode\endlinechar13
%    \end{macrocode}
% \end{macro}
%
% \begin{macro}{\logoptionsettings}
%    \begin{macrocode}
\gdef\logoptionsettings#1{\begingroup%
  \newlinechar13\relax\actively\let\^^M\relax%
  \toks2 {}%
  \addmonth\Jan \addmonth\Feb \addmonth\Mar%
  \toks2 \xp@{\the\toks2\relax
  }%
  \addmonth\Apr \addmonth\May \addmonth\Jun%
  \toks2 \xp@{\the\toks2\relax
  }%
  \addmonth\Jul \addmonth\Aug \addmonth\Sep%
  \toks2 \xp@{\the\toks2\relax
  }%
  \addmonth\Oct \addmonth\Nov \addmonth\Dec%
  \iwlos{#1}{%
    \losfirstline
    \the\toks2\relax
  }%
  \iwlos{#1}{%
    \string\let\string\landscape=%
      \if\landscape\string\TRUE
      \else\string\FALSE\fi
    \string\pawd=\the\pawd
    \string\def\string\paperwidth{\paperwidth}
    \string\paht=\the\paht
    \string\def\string\paperheight{\paperheight}
    \string\topmargin=\the\topmargin
    \string\bottommargin=\the\bottommargin
    \string\leftmargin=\the\leftmargin
    \string\rightmargin=\the\rightmargin
    \string\normalbaselineskip=%
      \the\normalbaselineskip\string\relax
    \string\def\string\mainfontsize{\mainfontsize}
    \string\def\string\mainfont{\mainfont}
    \string\chardef\string\totalcols=%
      \number\totalcols\string\relax
    \string\colwd=\the\colwd
    \string\def\string\columnwidth{\columnwidth}
    \string\let\string\dependentcolwd=%
      \if\dependentcolwd\string\TRUE
      \else\string\FALSE\fi
    \string\let\string\dependentrightmargin=%
      \if\dependentrightmargin\string\TRUE
      \else\string\FALSE\fi
    \string\colht=\the\colht
    \string\def\string\columnheight{\columnheight}
    \string\let\string\dependentcolht=%
      \if\dependentcolht\string\TRUE
      \else\string\FALSE\fi
    \string\let\string\dependentbottommargin=%
      \if\dependentbottommargin\string\TRUE
      \else\string\FALSE\fi
    \string\intercolspace=\the\intercolspace
    \string\overrun=\the\overrun
    \string\fullwd=\the\fullwd
  }%
  \iwlos{#1}{%
    \string\def\string\newfileoption{%
      \newfileoption}
    \string\let\string\printtitles=%
      \if\printtitles\string\TRUE
      \else\string\FALSE\fi
    \string\def\string\titlefont{\titlefont}
    \string\def\string\titlefontsize{%
      \titlefontsize}
    \string\def\string\titlecase{\titlecase}
    \string\def\string\datesepchar{\datesepchar}
    \string\def\string\todaysdate{%
      \xp@\string\todaysdate}
    \string\def\string\nowtime{%
      \xp@\string\nowtime}
    \string\def\string\timesepchar{\timesepchar}
    \string\def\string\ampmpunct{\ampmpunct}
    \string\let\string\keepleadingzeros=%
      \if\keepleadingzeros\string\TRUE
      \else\string\FALSE\fi
    \string\def\string\linenumberfontsize{%
      \linenumberfontsize}
    \string\def\string\linenumberfont{%
      \linenumberfont}
    \string\let\string\linenumbers=%
      \if\linenumbers\string\TRUE
      \else\string\FALSE\fi
    \string\linenumberfreq=%
      \number\linenumberfreq\string\relax
    \string\def\string\pagenumberfont{%
      \pagenumberfont}
    \string\def\string\pagenumberfontsize{%
      \pagenumberfontsize}
    \string\let\string\runheads=%
      \if\runheads\string\TRUE
      \else\string\FALSE\fi
    \string\runhead={\the\runhead}
    \string\def\string\runheadfont{\runheadfont}
    \string\def\string\runheadfontsize{%
      \runheadfontsize}
    \string\runheadht=\the\runheadht
    \string\let\string\runfeet=%
      \if\runfeet\string\TRUE\else\string\FALSE\fi
    \string\runfoot={\the\runfoot}
    \string\def\string\runfootfont{\runfootfont}
    \string\def\string\runfootfontsize{%
      \runfootfontsize}
    \string\runfootht=\the\runfootht
    \string\def\string\spacebreak{\spacebreak}
    \string\def\string\breakingchars{%
      \breakingchars}
    \string\let\string\noextralinebreaks=%
      \if\noextralinebreaks\string\TRUE
      \else\string\FALSE\fi
    \string\def\string\controlLaction{\controlLaction}
    \string\def\string\eightbitaction{\eightbitaction}
    \string\def\string\systemtype{\systemtype}
    \string\def\string\losdir{\losdir}
%    \end{macrocode}
%    \cw{storemesj} is use here instead of \cw{def}, to prevent
%    potential problems from, \eg.,
%
%^^V \def\dirprefix{c:\test\}
%
%    \noindent with {\sc PC DOS} directory name ending with a backslash.
%    \begin{macrocode}
    \string\storemesj\string\dirprefix{\dirprefix}
    \string\chardef\string\spacespertab=%
      \number\spacespertab\string\relax
    \string\chardef\string\tabselect=%
      \number\tabselect\string\relax
}\endgroup}%
\endgroup% percent to avoid catcode 12 endlinechar
%    \end{macrocode}
% \end{macro}
%
% %%%%%%%%%%%%%%%%%%%%%%%%%%%%%%%%%%%%%%%%%%%%%%%%%%%%%%%%%%%%%%%%%%%%%%
% \section{Default option settings}
%    These defaults will be used only if \fn{default.los} cannot
%    be loaded for some reason.
%
%    \begin{macrocode}
% Option settings for listout.tex, 1994/11/4 15:45
\def\Jan{January}\def\Feb{February}\def\Mar{March}\relax
\def\Apr{April}\def\May{May}\def\Jun{June}\relax
\def\Jul{July}\def\Aug{August}\def\Sep{September}\relax
\def\Oct{October}\def\Nov{November}\def\Dec{December}\relax
\let\landscape=\TRUE
\pawd=794.96999pt
\def\paperwidth{11in}
\paht=614.295pt
\def\paperheight{8.5in}
\topmargin=18.06749pt
\bottommargin=19.22751pt
\leftmargin=21.68121pt
\rightmargin=21.68121pt
\normalbaselineskip=10.0pt\relax
\def\mainfontsize{10}
\def\mainfont{typewriter}
\chardef\totalcols=2\relax
\colwd=369.38373pt
\def\columnwidth{369.38373pt}
\let\dependentcolwd=\TRUE
\let\dependentrightmargin=\TRUE
\colht=577.0pt
\def\columnheight{64 lines}
\let\dependentcolht=\TRUE
\let\dependentbottommargin=\TRUE
\intercolspace=12.0pt
\overrun=21.68121pt
\fullwd=751.60756pt
\def\newfileoption{P}
\let\printtitles=\TRUE
\def\titlefont{bold}
\def\titlefontsize{10}
\def\titlecase{none}
\def\datesepchar{/}
\def\todaysdate{\dateA}
\def\nowtime{\timeB}
\def\timesepchar{:}
\def\ampmpunct{.}
\let\keepleadingzeros=\TRUE
\def\linenumberfontsize{5}
\def\linenumberfont{roman}
\let\linenumbers=\FALSE
\linenumberfreq=1\relax
\def\pagenumberfont{bold}
\def\pagenumberfontsize{7}
\let\runheads=\TRUE
\runhead={\todaysdate \space \nowtime \space \hfil
  \thisfilename \space \hfil \pagenumber }
\def\runheadfont{roman}
\def\runheadfontsize{7}
\runheadht=24.0pt
\let\runfeet=\FALSE
\runfoot={}
\def\runfootfont{roman}
\def\runfootfontsize{6}
\runfootht=24.0pt
\def\spacebreak{\penalty \z@ }
\def\breakingchars{}
\let\noextralinebreaks=\FALSE
\def\controlLaction{N}
\def\systemtype{DOS}
\def\losdir{}
\storemesj\dirprefix{}
\chardef\spacespertab=8\relax
\chardef\tabselect=1\relax
%    \end{macrocode}
%
%    Read in the option settings file.
%
%    In order to make do in narrow column width, we print a doctored
%    version of the following source code, with a few extra line
%    breaks thrown in. The comment at the end of the \cw{input} line,
%    seen here on two lines, is really all on one line in the source.
%
%^^V \immediate\write16{%
%^^V Reading option settings from `default.los' . . .}
%^^V \input default.los % try entering "nul" if
%^^V                    % file not found
%
%    \iffalse
%    \begin{macrocode}
\def\losname{default}
\edef\losfile{\losdir\losname.los }
\openin15=\losfile \relax
\ifeof15 \closein15 \expandafter\@gobbletwo
\else
\immediate\write16{Reading option settings from `\losname.los' . . .}
\closein15 \fi
\input\losfile % try entering "nul" if file not found
\datesetup
%    \end{macrocode}
%    \fi
%
%    Now transfer \cw{normalbaselineskip} value to \cw{baselineskip}.
%    \begin{macrocode}
\normalbaselines
%    \end{macrocode}
%
% %%%%%%%%%%%%%%%%%%%%%%%%%%%%%%%%%%%%%%%%%%%%%%%%%%%%%%%%%%%%%%%%%%%%%%
% \section{Test for interactive mode}
%    If interactive mode is detected, then the possibility of entering
%    the menu system will be offered to the user. Otherwise we will
%    skip all the loading of the menu system, to save time and main
%    memory.
%
%    If \cw{interactive} is undefined {\em and\/} \cw{jobname} does
%    not equal \verb"listout", then we set the \cw{interactive} switch
%    to \cw{FALSE}. Otherwise we set it to \cw{TRUE}.
%    \begin{macrocode}
\ifx\undefined\interactive
%    \end{macrocode}
%    If \cw{interactive} was not predefined, then check to see if
%    jobname = \verb"listout". In that case assume \cw{interactive}
%    mode is wanted.
%    \begin{macrocode}
  \def\next{listout}%
%    \end{macrocode}
%    Convert the letters inside \cw{next} to category 12, for
%    comparison to \cw{jobname}.
%    \begin{macrocode}
  \edef\next{\xmeaning\next}%
  \edef\jobname{\jobname}%
  \ifx\next\jobname
    \let\interactive\TRUE
  \else
    \let\interactive\FALSE
  \fi
\else
  \let\interactive\TRUE
\fi
%    \end{macrocode}
%
%    Note that the following \cw{fi} must be on the same line as the
%    \cw{endinput}.
%    \begin{macrocode}
\if\interactive
\else \restorecatcodes \endinput \fi
%    \end{macrocode}
%
% %%%%%%%%%%%%%%%%%%%%%%%%%%%%%%%%%%%%%%%%%%%%%%%%%%%%%%%%%%%%%%%%%%%%%%
% \newpage
% \section{Final preparations}
%
% \begin{macro}{\printfiles}
%    If the menu system is not needed, we want to skip all the loading
%    of it. So we define a function \cw{printfiles} which can be
%    executed either before or after loading the menu
%    system. If the user doesn't choose to enter the menu system, then
%    executing \cw{printfiles} will carry the current \tex/ run to
%    completion and the remainder of \fn{listout.tex}, comprising all
%    the menu system, will be ignored. Otherwise \cw{printfiles} will be
%    called later, after leaving the menu system.
%    \begin{macrocode}
\def\printfiles{%\reportlayout
  \mesj{OK, let's print some files.}%
  \begingroup
%    \end{macrocode}
%    Execute and clear \cw{listoutsetup} to avoid unnecessary
%    repetition in multiple-file runs.
%    \begin{macrocode}
  \listoutsetup \let\listoutsetup\relax
%    \end{macrocode}
%    Prompt for file names and print files until the user enters an
%    empty file name.
%    \begin{macrocode}
  \fileloop
  \endgroup
  \listoutfinish
%    \end{macrocode}
%    Execute \latex/ \cw{@@end} first; if it is not defined, then \tex/
%    will proceed to execute the other \cw{end}. This enables
%    the \fn{listout} utility to work both with \latex/ and plain \tex/.
%    \begin{macrocode}
  \csname \string @\string @end\endcsname
  \end
}%
%    \end{macrocode}
% \end{macro}
%
% \begin{macro}{\numberoffiles}
%    Keep track of the number of files printed so that an
%    informational message can be sent at the end of the run.
%    \begin{macrocode}
\newcount\numberoffiles
%    \end{macrocode}
% \end{macro}
%
% \begin{macro}{\fileloop}
%
%    \begin{macrocode}
\def\fileloop{%
  \promptmesj{!File name? (<return> to stop): }%
  \readline{}\reply
  \ifx\reply\@empty
  \else
    \global\advance\numberoffiles 1 \relax
  \afterfi
    \xp@\listout\xp@{\reply}%
    \fileloop
  \fi
}
%    \end{macrocode}
% \end{macro}
%
% \begin{macro}{\DVImesj}
%    If \cw{jobname} = \fn{listout} then the final star in the middle
%    line will line up properly.
%    \begin{macrocode}
\fxmesj\DVImesj{%
*******************************************************************
*     DVI file name is \jobname.dvi                                *
*******************************************************************}
%    \end{macrocode}
% \end{macro}
%
%    \begin{macrocode}
\mesj{
:---------------------------------------------------------------------
:  This is listout.tex: a tool for printing out `verbatim' listings of
:  text files with reasonably robust, and customizable, handling of
:  overlong lines, tab characters and other special characters, number
:  of columns, font size/line spacing, et cetera.
:---------------------------------------------------------------------
:  For additional information (e.g., about noninteractive use), or to
:  change option settings, enter
:    m   or   M
:  to enter the listout.tex menu system. Otherwise just press the
:  <return> key to continue:
:}
\message{:?: }
%    \end{macrocode}
%
%    Don't put up the full menu unless it's requested. Use
%    \cw{expandafter} to throw the \cw{printfiles} command beyond the
%    \cw{fi}, since it will end the job.
%    \begin{macrocode}
\readChar{X}\reply
\if M\reply \else\xp@\printfiles\fi
%    \end{macrocode}
% %%%%%%%%%%%%%%%%%%%%%%%%%%%%%%%%%%%%%%%%%%%%%%%%%%%%%%%%%%%%%%%%%%%%%%
%
%    \begin{macrocode}
\message{Starting up the listout.tex menu system . . .}
%    \end{macrocode}
%
% \section{Menu system startup}
%    Some of the following definitions merely reiterate defaults from
%    \fn{menus.sty}, for the record.
% \begin{macro}{\menuprefix}
%    \begin{macrocode}
\storexmesj\menuprefix{
======================================================================
}
%    \end{macrocode}
% \end{macro}
%
% \begin{macro}{\menusuffix}
%    \begin{macrocode}
\storexmesj\menusuffix{
         Q  Quit         X  Exit        ?  Help
======================================================================
}
%    \end{macrocode}
% \end{macro}
%
% \begin{macro}{\inmenuA}
% \begin{macro}{\inmenuB}
%    \begin{macrocode}
\storexmesj\inmenuA{
}
\storexmesj\inmenuB{
}
%    \end{macrocode}
% \end{macro}
% \end{macro}
%
% \begin{macro}{\menuhelpmesj}
%
%    Add more to the standard help message defined by \fn{menus.sty}.
%
%    \begin{macrocode}
\fxmesj\menuhelpmesj{%
======================================================================
--- <return> means the carriage return, or `Enter', key.
--- When selecting items from a menu with letter labels, you can enter
your choice in lowercase or uppercase.
--- A prompt that asks for `TeX units' requires in response a
measurement using the units recognized by TeX, for example

      cm  mm  in  pt  pc  dd  cc

--- A response of Q will usually send you back to the previous menu.
--- A response of X will get you entirely out of the menu system.
======================================================================
To continue, press the <return> key . . .}
%    \end{macrocode}
% \end{macro}
%
% \begin{macro}{\confirm}
%
%    \begin{macrocode}
\def\confirm#1{\xmesj{\
* #1}}
%    \end{macrocode}
% \end{macro}
%
% \begin{macro}{\mainmenu}
%
%    \begin{macrocode}
\fmenu\mainmenu{
                    MAIN  MENU
}{
A  Action for new files            L  Line numbers
B  Breaking lines                  M  Margins
C  Column width/height             N  Number of columns
D  Date/time form                  O  Orientation
E  Expectoration                   P  Paper size
F  Font, line spacing              R  Running heads/feet
G  General information             S  System info
H  Heading/title for each file     T  Tab treatment
I  Info about current settings     V  Various other things
}{}
%    \end{macrocode}
% \end{macro}
%
% \begin{macro}{\moption}
%    The command \cw{moption} without any menu-letter suffixes is the
%    main menu presentation function.
%    \begin{macrocode}
\def\moption{\mainmenu \readChar{Q}\reply
  \let\prevmenu\@empty \let\curmenu\@empty \optionexec\reply}
%    \end{macrocode}
% \end{macro}
%
% \write16{Page \thepage: full menu printout begins here; you might}
% \write16{want to omit these pages when printing, to save paper.}
%
% %%%%%%%%%%%%%%%%%%%%%%%%%%%%%%%%%%%%%%%%%%%%%%%%%%%%%%%%%%%%%%%%%%%%%%
% \subsection{Option A}
%
% \begin{macro}{\menuA}
%    Submenu for selecting various new file actions.
%    \begin{macrocode}
\fmenu\menuA#1{
The various actions possible at the beginning of a new file are:
}{
S  Space (one line of blank space, but not a new column,
     at the beginning of each new file)
C  New column
P  New page
R  New page and right-hand column
}{
These options are mutually exclusive. The currently selected option
is #1. See also option H in the main menu.
}
%    \end{macrocode}
% \end{macro}
%
% \begin{macro}{\moptionA}
%    \begin{macrocode}
\def\moptionA{\menuA{\newfileoption}%
  \readChar{Q}\reply \optionexec\reply}
%    \end{macrocode}
% \end{macro}
%
% \begin{macro}{\moptionAS}
%    \begin{macrocode}
\def\moptionAS{\def\newfileoption{S}%
  \confirm{%
  Each file will start immediately after the preceding file, with\
* some intervening blank space.}%
  \popoptions\moptionQ}
%    \end{macrocode}
% \end{macro}
%
% \begin{macro}{\moptionAC}
%    \begin{macrocode}
\def\moptionAC{\def\newfileoption{C}%
  \confirm{%
  Each file will start at the top of the next column.}%
  \popoptions\moptionQ}
%    \end{macrocode}
% \end{macro}
%
% \begin{macro}{\moptionAP}
%    \begin{macrocode}
\def\moptionAP{\def\newfileoption{P}%
  \confirm{%
  Each file will start on a new page.}%
  \popoptions\moptionQ}
%    \end{macrocode}
% \end{macro}
%
% \begin{macro}{\moptionAR}
%
%    \begin{macrocode}
\def\moptionAR{\def\newfileoption{R}%
  \confirm{%
  Each file will start on a new page, in the right-hand column.}
  \popoptions\moptionQ}
%    \end{macrocode}
% \end{macro}
%
% %%%%%%%%%%%%%%%%%%%%%%%%%%%%%%%%%%%%%%%%%%%%%%%%%%%%%%%%%%%%%%%%%%%%%%
% \subsection{Option B}
%
% \begin{macro}{\menuBtest}
%    Line breaking.
%    \begin{macrocode}
\def\menuBtest{%
  \ifx\spacebreak\@empty
    \ifx\breakingchars\@empty
Long lines will not be broken.%
    \else
Extra line breaks will be allowed after the characters\
\
\ \ \ \ \ \breakingchars\
      \ifdim\overrun=\z@
      \else\
(with overrun of \the\overrun\space allowed).%
      \fi
    \fi
  \else
    \ifx\breakingchars\@empty
Extra line breaks will be allowed at spaces.%
    \else
Extra line breaks will be allowed at spaces and\
after the characters\
\
\ \ \ \ \ \breakingchars\
    \fi
    \ifdim\overrun=\z@
    \else\
(with overrun of \the\overrun\space allowed).%
    \fi
  \fi
}
%    \end{macrocode}
% \end{macro}
%
% \begin{macro}{\menuB}
%    \begin{macrocode}
\fxmenu\menuB{
Files being printed sometimes contain unusually long lines.
You can choose the action that will be taken for overlong lines:
}{
N  No line breaking allowed
S  Allow breaks at spaces only
C  Allow line breaks at other characters
O  Change the overrun amount
}{
Current setting: &\menuBtest
}
%    \end{macrocode}
% \end{macro}
%
% \begin{macro}{\moptionB}
%    \begin{macrocode}
\def\moptionB{\lettermenu{B}}
%    \end{macrocode}
% \end{macro}
%
% \begin{macro}{\moptionBS}
%    \begin{macrocode}
\def\moptionBS{%
  \let\spacebreak\markedbreak \let\breakingchars\@empty
  \overrun\rightmargin \let\noextralinebreaks\FALSE
  \confirm{\menuBtest}\popoptions\moptionQ
}
%    \end{macrocode}
% \end{macro}
%
% \begin{macro}{\moptionBN}
%    \begin{macrocode}
\def\moptionBN{%
  \let\spacebreak\@empty \let\breakingchars\@empty
  \overrun=\rightmargin
  \confirm{\menuBtest}\popoptions\moptionQ
}
%    \end{macrocode}
% \end{macro}
%
% \begin{macro}{\moptionBO}
%    \begin{macrocode}
\def\moptionBO{%
  \promptmesj{Desired overrun value [TeX units] ? }%
  \readline{Q}\reply
  \if\xoptiontest\reply
  \else
    \checkdimen\reply\dimen@
    \ifdim\dimen@=-\maxdimen
    \else
      \overrun\reply\relax
      \confirm{New overrun value = \the\overrun.}%
      \def\reply{Q}%
    \fi
  \fi
  \optionexec\reply
}
%    \end{macrocode}
% \end{macro}
%
% \begin{macro}{\moptionBC}
%    One reason for automatically adding the space character to the
%    list of breaking chars is that otherwise we would have to caution
%    the user against typing the space at the end of the line when
%    responding to the prompt; since \tex/ removes character 32 at the
%    end of a line, we have no way of finding out whether one was
%    typed there or not.
%    \begin{macrocode}
\def\moptionBC{%
  \xmesj{%
Enter a list of characters. Line breaks will be allowed after any\
character in the list.\
Note: if this list is nonempty breaks will be allowed at spaces,\
even if you didn't include a space in the list.}%
  \readline{}\breakingchars
  \ifx\@empty\breakingchars
  \else \let\spacebreak\markedbreak\fi
  \let\noextralinebreaks\FALSE
  \overrun=\rightmargin
  \confirm{\menuBtest}%
  \popoptions\moptionQ
}
%    \end{macrocode}
% \end{macro}
%
% %%%%%%%%%%%%%%%%%%%%%%%%%%%%%%%%%%%%%%%%%%%%%%%%%%%%%%%%%%%%%%%%%%%%%%
% \subsection{Option C}
%
% \begin{macro}{\menuC}
%    \cw{columnwidth} is the value as originally given, including the
%    original units; may be a dimension variable or a character
%    string. The corresponding internal variable is \cw{colwd}. If we
%    did not maintain \cw{columnwidth} separately from \cw{colwd}, we
%    could not later echo the original units on-screen to the user,
%    since that information is lost in the act of storing the width in
%    \cw{colwd}. Similarly for \cw{columnheight}.
%    \begin{macrocode}
\fxmenu\menuC{
Column width is normally computed automatically, from the specified
paper width, number of columns, margins, and intercolumn space.
However, if you provide an explicit value for column width it will
be used and the right margin will be adjusted as necessary.
Similarly, column height is computed automatically from the
paper height and the top and bottom margins, unless you provide an
explicit value.
}{
W  Change column width
H  Change column height
I  Change intercolumn space
}{
Current column width: &\columnwidth.
Current column height: &\columnheight.
Current intercolumn space: &\the&\intercolspace.
}
%    \end{macrocode}
% \end{macro}
%
% \begin{macro}{\moptionC}
%    \begin{macrocode}
\def\moptionC{\lettermenu C}
%    \end{macrocode}
% \end{macro}
%
% \begin{macro}{\moptionCW}
%    \begin{macrocode}
\def\moptionCW{%
  \xmesj{%
For width in characters, enter just a number; otherwise enter a\
measurement using standard TeX units. To revert to automatic computing\
of the column width, give an empty reply (i.e., just press <return>).}%
  \message{Column width: }%
  \readline{}\reply
  \ifx\reply\@empty
    \let\dependentcolwd\TRUE
    \if\dependentrightmargin \computerfc \else \computefc \fi
    \def\reply{A}% report new value through \moptionCWA
  \fi
  \if\xoptiontest\reply
  \else
    \checkdimen\reply\dimen@
    \ifdim\dimen@=-\maxdimen
      \checkinteger\reply\count@
      \ifnum\count@=-\maxdimen
      \else
%    \end{macrocode}
%    A misspelled unit name, \eg. {\tt im} instead of {\tt in}, will
%    lead to this branch; maybe we should try to check whether the
%    answer contained letters after the digits and if so attempt an
%    explanation about possibly misspelled unit name. On the other
%    hand maybe that's too much work.
%    \begin{macrocode}
        \colwd.5em \multiply\colwd\count@\relax
        \edef\columnwidth{\the\count@\space characters}%
        \let\dependentcolwd\FALSE
        \def\reply{A}%
      \fi
    \else
      \colwd\dimen@ \let\columnwidth\reply \let\dependentcolwd\FALSE
      \def\reply{B}%
    \fi
  \fi
  \optionexec\reply
}
%    \end{macrocode}
% \end{macro}
%
% \begin{macro}{\moptionCWA}
%    \begin{macrocode}
\def\moptionCWA{%
  \confirm{Column width set to \columnwidth.}%
  \popoptions\moptionQ}
%    \end{macrocode}
% \end{macro}
%
% \begin{macro}{\moptionCWB}
%    \begin{macrocode}
\def\moptionCWB{%
  \confirm{Column width set to \columnwidth\space (= \the\colwd)}%
  \popoptions\moptionQ}
%    \end{macrocode}
% \end{macro}
%
% \begin{macro}{\moptionCH}
%    \begin{macrocode}
\def\moptionCH{%
  \xmesj{%
For height in lines, enter just a number; otherwise enter a\
measurement using standard TeX units. To revert to automatic computing\
of the column height, give an empty reply (i.e., just press <return>).}%
  \message{Column height: }%
  \readline{}\reply
  \ifx\reply\@empty
    \let\dependentcolht\TRUE
    \if\dependentbottommargin \computebc \else \computec \fi
    \def\reply{A}% report new value through \moptionCWA
  \fi
  \if\xoptiontest\reply
  \else
    \checkdimen\reply\dimen@
    \ifdim\dimen@=-\maxdimen
      \checkinteger\reply\count@
      \ifnum\count@=-\maxdimen
      \else
        \let\dependentcolht\FALSE
        \colht\baselineskip \multiply\colht\count@
        \advance\colht -\baselineskip \advance\colht\topskip
        \computeb
        \edef\columnheight{\the\count@\space lines}%
        \def\reply{A}%
      \fi
    \else
      \colht\dimen@ \let\columnheight\reply
      \let\dependentcolht\FALSE \computeb
      \def\reply{B}%
    \fi
  \fi
  \optionexec\reply
}
%    \end{macrocode}
% \end{macro}
%
% \begin{macro}{\moptionCHA}
%    \begin{macrocode}
\def\moptionCHA{%
  \confirm{Column height set to \columnheight.\
* (Bottom margin adjusted to \the\bottommargin.)}%
  \popoptions\moptionQ}
%    \end{macrocode}
% \end{macro}
%
% \begin{macro}{\moptionCHB}
%
%    \begin{macrocode}
\def\moptionCHB{%
  \confirm{Column height set to \columnheight\space (= \the\colht)\
* (Bottom margin adjusted to \the\bottommargin.)}%
  \popoptions\moptionQ}
%    \end{macrocode}
% \end{macro}
%
% \begin{macro}{\moptionCI}
%    \begin{macrocode}
\def\moptionCI{%
  \promptmesj{Enter desired intercolumn space [TeX units]: }%
  \readline{Q}\reply
  \ifx\optiontest\reply
  \else
    \checkdimen\reply\dimen@
    \ifdim\dimen@=-\maxdimen
    \else
      \intercolspace=\dimen@
      \confirm{%
Intercolumn space set to \reply\space (= \the\intercolspace).}%
      \if\linenumbers\ifdim\intercolspace<12\p@
      \confirm{\
* Warning: small intercolumn space, might not be big enough\
* for line numbers to fit ...}%
      \fi\fi
      \def\reply{Q}%
    \fi
  \fi
  \optionexec\reply
}
%    \end{macrocode}
% \end{macro}
%
% %%%%%%%%%%%%%%%%%%%%%%%%%%%%%%%%%%%%%%%%%%%%%%%%%%%%%%%%%%%%%%%%%%%%%%
% \subsection{Option D}
%
% \begin{macro}{\menuD}
%    \begin{macrocode}
\fxmenu\menuD{}{
D  Change date format
S  Change separator character
Z  &\if&\keepleadingzeros\.Omit&\else\.Add&\fi %
leading zeros in the day and month
T  Change time format
}{
Current date and time format is `&\todaysdate &\nowtime'.
}
%    \end{macrocode}
% \end{macro}
%
% \begin{macro}{\moptionD}
%
%    \begin{macrocode}
\def\moptionD{\lettermenu D}
%    \end{macrocode}
% \end{macro}
%
% \begin{macro}{\menuDD}
%
%    \begin{macrocode}
\fxmenu\menuDD{
Date options:
}{
A  &\dateA
B  &\dateB
C  &\dateC
D  &\dateD
E  &\dateE
F  &\dateF
G  &\dateG
H  &\dateH
I  &\dateI
J  &\dateJ
}{
Current date format is &\todaysdate.
}
%    \end{macrocode}
% \end{macro}
%
% \begin{macro}{\moptionDD}
%
%    Valid responses to this menu are \qc{\A}--\qc{\J}, \qc{\Q},
%    \qc{\X}, or \qc{\?}. Rather than define submenus \cw{moptionDDA},
%    \cw{moptionDDB}, \ldots, we use \tex/'s \verb'`'\meta{character}
%    feature to convert the character in \cw{reply} to a number, so
%    that a simple numeric test can be used to see if the reply is in
%    range.
%
%    \begin{macrocode}
\def\moptionDD{\menuDD \readChar{Q}\reply
  \count@=\xp@`\reply\relax
%    \end{macrocode}
%    A = 65 \ascii/, J = 74 \ascii/
%    \begin{macrocode}
  \ifnum\count@>64 \ifnum\count@<75
%    \end{macrocode}
%    We use \cw{def} rather than \cw{let} here so that we can record
%    the current date format in the \fn{.los} file with
%    \cw{expandafter}\cw{string} (see \cw{logoptionsettings}).
%    Otherwise we would probably have to use a separate variable just
%    to keep track of the date format letter.
%    \begin{macrocode}
    \xp@\def\xp@\todaysdate\xp@{\csname date\reply\endcsname}%
    \confirm{New date form: \todaysdate}%
    \def\reply{Q}%
  \fi\fi
  \optionexec\reply
}
%    \end{macrocode}
% \end{macro}
%
% \begin{macro}{\moptionDS}
%    \begin{macrocode}
\def\moptionDS{\promptmesj{!%
Current separator character between the parts of a date is
`\datesepchar'.!%
New separator: }%
  \readchar{}\reply
  \if Q\reply
    \confirm{Date separator char unchanged.}%
  \else
    \if ?\reply
    \else
      \if X\reply
        \confirm{Date separator char unchanged.}%
      \else
        \let\datesepchar\reply
        \confirm{New date format: \todaysdate.}%
        \def\reply{Q}%
      \fi
    \fi
  \fi
  \optionexec\reply
}
%    \end{macrocode}
% \end{macro}
%
% \begin{macro}{\moptionDZ}
%\mjd{Maybe date-time-reset doesn't need to do \themonth,\theday?}
%    \begin{macrocode}
\def\moptionDZ{%
  \if\keepleadingzeros
    \let\keepleadingzeros\FALSE
    \edef\themonth{\number\month}\edef\theday{\number\day}%
  \else
    \let\keepleadingzeros\TRUE
    \edef\themonth{\twodigits\month}\edef\theday{\twodigits\day}%
  \fi
  \confirm{Leading zeros will
    \if\keepleadingzeros NOT \else \fi
    be omitted in day and month numbers}%
  \def\reply{Q}\optionexec\reply
}
%    \end{macrocode}
% \end{macro}
%
% \begin{macro}{\menuDT}
%    \begin{macrocode}
\fxmenu\menuDT{
Time format options:
}{
A  &\timeA (HoursMinutes, 24-hour cycle)
B  &\timeB (Hours:Minutes, 24-hour cycle)
C  &\timeC (12-hour cycle)
D  &\timeD
E  &\timeE
F  &\timeF
}{
Current time format is &\nowtime.
}
%    \end{macrocode}
% \end{macro}
%
% \begin{macro}{\moptionDT}
%    If the user's reply to the time menu is one of the valid letters
%    (A\ndash F), we define \cw{nowtime} to be \cw{timeX}, where {\tt
%    X} represents the reply letter.
%    \begin{macrocode}
\def\moptionDT{\menuDT \readChar{Q}\reply
  \if A\if B\reply A\else\if C\reply A\else\if D\reply A\else
       \if E\reply A\else\if F\reply A\else\reply\fi\fi\fi\fi\fi
    \xp@\def\xp@\nowtime\xp@{\csname time\reply\endcsname}%
    \def\reply{Z}%
  \fi
  \optionexec\reply
}
%    \end{macrocode}
% \end{macro}
%
% \begin{macro}{\moptionDTZ}
%    \begin{macrocode}
\def\moptionDTZ{\confirm{New time format: \nowtime}%
  \popoptions\moptionQ}
%    \end{macrocode}
% \end{macro}
%
% %%%%%%%%%%%%%%%%%%%%%%%%%%%%%%%%%%%%%%%%%%%%%%%%%%%%%%%%%%%%%%%%%%%%%%
% \subsection{Option E}
%
% \begin{macro}{\menuE}
%    \begin{macrocode}
\fmenu\menuE{
This option allows you to save all the current option settings in a
file for later reuse. If you use the file name `default.los' the saved
settings will be used as defaults for future runs. Or you can use
different file names to have multiple saved sets of option settings.
}{
S  Save settings
L  Load saved settings from file
D  Specify a default directory/folder for .los files
}{
The last option allows you to specify a location if you want to
keep listout.tex and all its .los files in a particular directory or
folder. This works best if the given location is one searched
automatically by TeX (e.g., by being included in the `TeX inputs path').
}
%    \end{macrocode}
% \end{macro}
%
% \begin{macro}{\moptionE}
%    \begin{macrocode}
\def\moptionE{\lettermenu{E}}
%    \end{macrocode}
% \end{macro}
%
% \begin{macro}{\losname}
% \begin{macro}{\losdir}
%    \begin{macrocode}
\def\losname{default}
\def\losdir{}
%    \end{macrocode}
% \end{macro}
% \end{macro}

% \begin{macro}{\lossorrymessage}
%    \begin{macrocode}
\def\lossorrymessage{\message{%
? Could not create `\losdir\losname.los';
maybe there was a system problem.}}
%    \end{macrocode}
% \end{macro}
%
% \begin{macro}{\currentdir}
%    Default for \cw{currentdir} is empty, to avoid creating problems
%    for less-knowledgeable users.
%    \begin{macrocode}
\def\currentdir{}
%    \end{macrocode}
% \end{macro}
%
% \begin{macro}{\moptionES}
%    \begin{macrocode}
\def\moptionES{\xmesj{\
Enter the desired file name. The default name (if you just press the\
<return> key) is `default'. A file extension `.los' will be added\
automatically:}%
  \readline{default}\reply
  \edef\losname{\reply}%
  \def\next##1.los##2##3\next{\ifx\relax##2\else
    \def\losname{##1}\fi}
  \xp@\next\losname.los\relax\next
  \confirm{%
File name: `\losname.los' ---OK? If not, enter n or N to cancel:}%
  \readChar{Y}\reply
  \if Y\reply
    \edef\losfirstline{\xp@\@gobble\string
\% Option settings for listout.tex, \todaysdate\space\nowtime}%
    \immediate\openout15=\losdir\losname.los \relax
    \logoptionsettings{15}%
    \immediate\closeout15 \relax
%    \end{macrocode}
%    Check to make sure the file was created.
%    \begin{macrocode}
    \openin15=\losdir\losname.los \relax
    \ifeof15 \lossorrymessage
%    \end{macrocode}
%    If \fn{xxxxx.los} exists and is nonempty, it's still not certain
%    that we have not opened some other file named \fn{xxxxx.los}, such
%    as a default version located somewhere else in the tex-inputs
%    search path. We check this possibility by looking at the first
%    line, which has embedded the date and time of the \tex/ run that
%    produced the file.
%    \begin{macrocode}
    \else \testlocation \fi
    \closein15 \relax
  \fi
  \popoptions\moptionQ
}
%    \end{macrocode}
% \end{macro}
%
% \begin{macro}{\testlocation}
%    \begin{macrocode}
\def\testlocation{%
  \begingroup
    \catcode`\%=12 \catcode`\\=12 \catcode\endlinechar=9
    \read15 to\next
    \ifx\next\losfirstline
      \confirm{%
New `\losname.los' file successfully created}%
%    \end{macrocode}
%    There is one other possibility left to be checked: if the current
%    directory is excluded from the tex-inputs search path  (\eg.,
%    under Unix typically `\qc{\.}' must be explicitly included in the
%    search path to always have the current directory searched) then
%    it is possible that the option settings file will be created OK
%    but will not be accessible to \fn{listout.tex} through \cw{input}.
%    \begin{macrocode}
      \ifx\losdir\@empty
        \testinputtable
      \else
        \message{%
(in location `\losdir')}%
      \fi
    \else
      \lossorrymessage
    \fi
  \endgroup
}
%    \end{macrocode}
% \end{macro}
%
% \begin{macro}{\testinputtable}
%    This function tests if the \fn{.los} file written to the default
%    directory is inputtable. The method is to make the percent
%    character temporarily active (there is a
%    \cw{begingroup}\ldots\cw{endgroup} pair enclosing the
%    \cw{testinputtable} call) and define it to read the first line of
%    the \fn{.los} file, compare it to the expected string, and issue
%    a warning message if it doesn't match.
%    \begin{macrocode}
\def\testinputtable{%
  \actively\def\%##1@{%
    \edef\next{\xp@\@gobble\string\%##1}%
    \ifx\next\losfirstline
    \else
      \message{%
... but guess what: it seems to be inaccessible for \nx@\input by TeX%
      }%
    \fi
    \endinput}%
  \endlinechar`\@ \relax
  \input\losname.los \relax
}
%    \end{macrocode}
% \end{macro}
%
% \begin{macro}{\moptionEL}
%    \begin{macrocode}
\def\moptionEL{\xmesj{\
Enter the name of the desired option file:}%
  \readline{}\reply
  \ifx\@empty\reply
  \else
    \def\next##1.los##2##3\next{\ifx\relax##2\else
      \def\reply{##1}\fi}
    \xp@\next\reply.los\relax\next
    \xmesj{\
Attempting to load \reply.los . . .}%
    \input\reply.los\space\relax
  \fi
  \def\reply{Q}\optionexec\reply
}
%    \end{macrocode}
% \end{macro}
%
% \begin{macro}{\moptionED}
%    \begin{macrocode}
\def\moptionED{\xmesj{\
Current location name is `\losdir'.\
Enter a new location name (directory or folder or whatever, depending\
on your system). For Unix or DOS, make sure you include the final\
slash. (For DOS, use slashes instead of backslashes.) Just press\
<return> if you want to leave the current location unchanged:}%
  \readline{}\reply
  \ifx\@empty\reply
  \else
    \let\losdir\reply
    \confirm{Location name is now: `\losdir'}%
  \fi
  \def\reply{Q}\optionexec\reply
}
%    \end{macrocode}
% \end{macro}
%
% %%%%%%%%%%%%%%%%%%%%%%%%%%%%%%%%%%%%%%%%%%%%%%%%%%%%%%%%%%%%%%%%%%%%%%
% \subsection{Option F}
%
% \begin{macro}{\menuF}
%    \begin{macrocode}
\fxmenu\menuF{}{
F  Change font
S  Change font size
L  Change line spacing
T  Test font availability
N  Add new font
}{
Current settings: &\mainfont &\mainfontsize / &\the&\baselineskip.
}
%    \end{macrocode}
% \end{macro}
%
% \begin{macro}{\moptionF}
%
%    \begin{macrocode}
\def\moptionF{\lettermenu F}
%    \end{macrocode}
% \end{macro}
%
% \begin{macro}{\genfontmenu}
%
%    \begin{macrocode}
\fxmenu\genfontmenu{
&\firstpart
Font choices:
}{
R  Roman
B  Bold
I  Italic
T  Typewriter
S  Sans serif
}{
Current choice is `&\fonttoget'.
}
%    \end{macrocode}
% \end{macro}
%
% \begin{macro}{\getfont}
%    \cw{getfont} is a higher-level function to call \cw{genfontmenu}.
%    The arguments required by \cw{getfont} are \qarg{1} font style
%    control sequence, \ie. something like \cw{linenumberfont} whose
%    definition is `bold' or `italic' or `typewriter'; \qarg{2} text to
%    add above the basic list of font choices, \qarg{3} font description
%    such as `Main' or `Line number'.
%    \begin{macrocode}
\def\getfont#1#2#3{%
  \let\fonttoget#1\def\firstpart{#2}%
  \genfontmenu \readChar{Q}\reply
  \edef\next{\if R\reply roman%
             \else\if B\reply bold%
             \else\if I\reply italic%
             \else\if S\reply sansserif%
             \else\if T\reply typewriter%
             \fi\fi\fi\fi\fi}%
  \ifx\next\@empty
  \else
    \let#1\next \def\reply{Q}%
    \confirm{#3 font is now `#1'.}%
  \fi
  \optionexec\reply
}
%    \end{macrocode}
% \end{macro}
%
% \begin{macro}{\moptionFF}
%
%    \begin{macrocode}
\def\moptionFF{\getfont\mainfont{%
Note: Font style `typewriter' is recommended for printing computer\
program code verbatim because the character set of most non-typewriter\
fonts lacks certain characters such as { } | \string~ and \\ .}%
{Main}}
%    \end{macrocode}
% \end{macro}
%
% \begin{macro}{\wholepointsizes}
%
%    \begin{macrocode}
\fmesj\wholepointsizes{
Note: Currently only whole point sizes are supported; the fractional
part in a fractional point size will be ignored. If a font/size
combination is not available on your system there will be an error
message later on when we try to load the font.
}%
%    \end{macrocode}
% \end{macro}
%
% \begin{macro}{\getfontsize}
%
%    \begin{macrocode}
\def\getfontsize#1#2{\wholepointsizes
  \mesj{Current #2 font size is #1.}%
  \promptmesj{Desired font size? }%
  \readline{Q}\reply
  \if\xoptiontest\reply
  \else
    \checkinteger\reply\count@
    \ifnum\count@>\z@
      \edef#1{\the\count@}%
      \confirm{New font size for #2: #1}%
      \def\reply{Q}%
%    \end{macrocode}
%    No \cw{else} branch here: If $\cw{count@}=0$, leave \cw{reply}
%    unchanged. This will lead to a `don't understand' message and a
%    repeat of the same menu.
%    \begin{macrocode}
    \fi
  \fi
  \optionexec\reply
}
%    \end{macrocode}
% \end{macro}
%
% \begin{macro}{\moptionFS}
%    \begin{macrocode}
\def\moptionFS{\getfontsize\mainfontsize{main text}}
%    \end{macrocode}
% \end{macro}
%
% \begin{macro}{\moptionFL}
%    \begin{macrocode}
\def\moptionFL{\promptmesj{Desired line spacing [TeX units] ? }%
  \readline{Q}\reply
  \if\xoptiontest\reply
  \else
    \checkdimen\reply\dimen@
    \ifdim\dimen@>\z@
      \normalbaselineskip\dimen@\relax \normalbaselines
      \confirm{New line spacing: \the\normalbaselineskip}%
      \def\reply{Q}%
    \fi
  \fi
  \optionexec\reply
}
%    \end{macrocode}
% \end{macro}
%
% \begin{macro}{\menuFT}
%    \begin{macrocode}
\fxmenu\menuFT{
Font to test load?
}{
M  main font
T  title font
L  line number font
P  page number font
H  running head font
F  running foot font
A  all of the above
}{}
%    \end{macrocode}
% \end{macro}
%
% \begin{macro}{\moptionFT}
%    \begin{macrocode}
\def\moptionFT{\menuFT
  \readChar{Q}\reply
  \if M\reply
    \testload\mainfont\mainfontsize
    \def\reply{Q}%
  \else\if T\reply
    \testload\titlefont\titlefontsize
    \def\reply{Q}%
  \else\if L\reply
    \testload\linenumberfont\linenumberfontsize
    \def\reply{Q}%
  \else\if P\reply
    \testload\pagenumberfont\pagenumberfontsize
    \def\reply{Q}%
  \else\if H\reply
    \testload\runheadfont\runheadfontsize
    \def\reply{Q}%
  \else\if F\reply
    \testload\runfootfont\runfootfontsize
    \def\reply{Q}%
  \else\if A\reply
    \testload\mainfont\mainfontsize
    \testload\pagenumberfont\pagenumberfontsize
    \testload\titlefont\titlefontsize
    \testload\linenumberfont\linenumberfontsize
    \testload\runheadfont\runheadfontsize
    \testload\runfootfont\runfootfontsize
    \def\reply{Q}%
  \fi\fi\fi\fi\fi\fi\fi
  \optionexec\reply
}
%    \end{macrocode}
% \end{macro}
%
% \begin{macro}{\testload}
%    \begin{macrocode}
\def\testload#1#2{\loadfont#1#2\relax
  \xp@\ifx\csname#1#2\endcsname\nullfont
    \confirm{Unsuccessful font load...}%
  \else
    \confirm{Font successfully loaded.}%
  \fi
}
%    \end{macrocode}
% \end{macro}
%
% \begin{macro}{\menuFN}
%    \begin{macrocode}
\fxmesj\menuFN{
To use a custom font you must give two pieces of information: the
name of the font metrics file (.tfm file), not including the .tfm
extension, and a magnification (using TeX's standard notation: 1000 =
100\%, 1200 = 120\%, ...).
}
%    \end{macrocode}
% \end{macro}
%
% \begin{macro}{\moptionFN}
%    \begin{macrocode}
\def\moptionFN{\menuFN
  \promptmesj{OK, first give the font name: }%
  \readline{Q}\reply
  \if\xoptiontest\reply
  \else
    \let\mainfont\reply
    \def\reply{Q}%
    \promptmesj{Font magnification? [default=1000]: }%
    \readline{1000}\reply
    \if\xoptiontest\reply
    \else
      \checkinteger\reply\count@
      \ifnum\count@>100 %
        \let\mainfontsize\reply
        \def\reply{Q}%
      \fi
    \fi
    \xmesj{\
Attempting to load new font `\mainfont\ scaled \mainfontsize'; if you\
get an error message, press RETURN to continue, and try again.\
    }%
  \testload\mainfont\mainfontsize
  \fi
  \optionexec\reply
}
%    \end{macrocode}
% \end{macro}
%
% %%%%%%%%%%%%%%%%%%%%%%%%%%%%%%%%%%%%%%%%%%%%%%%%%%%%%%%%%%%%%%%%%%%%%%
% \subsection{Option G}
%
% \begin{macro}{\menuGa}
% \begin{macro}{\menuGb}
%    \begin{macrocode}
\fmesj\menuGa{
:  General information on a couple of subjects:
:
:  ---Non-interactive usage: Create a driver file containing
:
:   \input listout
:   %    Change options here, if desired
:   \listout{first.file}
:   \listout{second.file}
:   ...
:   \listoutfinish
:   \end
:
:  The \listoutfinish command is required to support the possibility
:  where multiple files are printed running together without intervening
:  page breaks. To find out how to change options, look at the file
:  default.los.
:
:  Press <return> to continue . . .}
%
\fmesj\menuGb{
:  ---Treatment of control characters and eight-bit characters: The
:  default behavior is to assume no knowledge about the meaning of
:  characters in the range 0--31 and 127--255, because depending on the
:  source, the intended meaning of one of these characters in a given
:  file can vary widely. And the default main font (cmtt) can't handle
:  characters above 127 anyway. In order to preserve vertical alignment
:  when printing with a monospace font (the usual case) control
:  characters aren't printed as multicharacter sequences (e.g. ^^F), but
:  as boxed letters. The first control character encountered in a file
:  will generate a warning/explanatory message on screen. Similarly,
:  characters above ASCII 126 are printed simply as boxed question
:  marks. This behavior can be overridden by changing the main font to a
:  256-character font and choosing the 8-bit option in the V menu.
:
:  Press <return> to continue . . .}

%    \end{macrocode}
% \end{macro}
% \end{macro}
%
% \begin{macro}{\moptionG}
%    \begin{macrocode}
\def\moptionG{\menuGa\readline{}\reply\menuGb\readline{}\reply \moptionQ}
%    \end{macrocode}
% \end{macro}
%
% %%%%%%%%%%%%%%%%%%%%%%%%%%%%%%%%%%%%%%%%%%%%%%%%%%%%%%%%%%%%%%%%%%%%%%
% \subsection{Option H}
%
% \begin{macro}{\menuH}
%    \begin{macrocode}
\fxmenu\menuH{}{
T  Turn filename titles&\if&\printtitles off&\else on&\fi
F  Change font of titles
S  Change font size
C  Change capitalization
}{
Current settings: %
&\titlefont &\titlefontsize, capitalization = &\titlecase.
}
%    \end{macrocode}
% \end{macro}
%
% \begin{macro}{\moptionH}
%    \begin{macrocode}
\def\moptionH{\lettermenu H}
%    \end{macrocode}
% \end{macro}
%
% \begin{macro}{\moptionHT}
%    \begin{macrocode}
\def\moptionHT{%
  \if\printtitles \let\printtitles\FALSE
  \else \let\printtitles\TRUE \fi
  \confirm{%
The name of each file \if\printtitles WILL \else will NOT \fi
be printed as a title at the\
* beginning of the file.
}%
  \if\printtitles\else\popoptions\fi
  \moptionQ
}%
%    \end{macrocode}
% \end{macro}
%
% \begin{macro}{\moptionHF}
%
%    \begin{macrocode}
\def\moptionHF{\getfont\titlefont{}{Title}}
%    \end{macrocode}
% \end{macro}
%
% \begin{macro}{\moptionHS}
%
%    \begin{macrocode}
\def\moptionHS{\getfontsize\titlefontsize{titles}}
%    \end{macrocode}
% \end{macro}
%
% \begin{macro}{\menuHC}
%    \begin{macrocode}
\fxmenu\menuHC{
Title capitalization choices:
}{
N  None
L  Lowercase
U  Uppercase
}{
Current choice is &\titlecase.
}
%    \end{macrocode}
% \end{macro}
%
% \begin{macro}{\moptionHC}
%
%    \begin{macrocode}
\def\moptionHC{\menuHC \readChar{Q}\reply
  \if L\reply
    \def\titlecase{lowercase}%
    \confirm{Capitalization is now: lowercase.}%
    \def\reply{Q}%
  \else\if U\reply
    \def\titlecase{uppercase}%
    \confirm{Capitalization is now: uppercase.}%
    \def\reply{Q}%
  \else\if N\reply
    \def\titlecase{none}%
    \confirm{Capitalization is now: none.}%
    \def\reply{Q}%
  \fi\fi\fi
  \optionexec\reply}
%    \end{macrocode}
% \end{macro}
%
% %%%%%%%%%%%%%%%%%%%%%%%%%%%%%%%%%%%%%%%%%%%%%%%%%%%%%%%%%%%%%%%%%%%%%%
% \subsection{Option I}
%
% \begin{macro}{\menuIa}
%    \begin{macrocode}
\fxmesj\menuIa{
Current option settings:
Orientation: &\if&\landscape landscape&\else portrait&\fi
Paper size: &\paperwidth x &\paperheight (&\the\pawd x &\the\paht)
Margins: Top &\the\topmargin, bottom &\the\bottommargin,
  left &\the\leftmargin, right &\the\rightmargin
&\number\totalcols columns, column width = &\columnwidth (&\the\colwd),
Column height: &\columnheight (&\the\colht)
Intercolumn space: &\the\intercolspace
Full width of text area: &\the\fullwd
Main font: &\mainfont &\mainfontsize
Linespacing: &\the\normalbaselineskip
Each new file starts%
&\if\.S&\newfileoption%
 immediately after the previous file.
&\else&\if\.C&\newfileoption%
 at the top of a new column.
&\else&\if\.P&\newfileoption%
 on a new page.
&\else&\if\.R&\newfileoption%
 on the right-hand-side of a new page.
&\else ??? unknown.
&\fi&\fi&\fi&\fi%
File names are &\if&\printtitles&\else\.not&\fi%
  printed as titles at the beginning of each file.
&\if&\printtitles%
File name font: &\titlefont &\titlefontsize
File name capitalization: &\titlecase
&\fi%
Date and time print in the form &\todaysdate and &\nowtime.
Line numbers are turned%
&\if&\linenumbers ON, font &\linenumberfont &\linenumberfontsize,
  every &\number\linenumberfreq line%
&\ifnum&\linenumberfreq=1 &\else s&\fi
&\else OFF&\fi
Press <return> to continue: }
% \end{macro}
%
% \begin{macro}{\menuIb}
%    \begin{macrocode}
\fxmesj\menuIb{
Page number font: &\pagenumberfont&\pagenumberfontsize
Running heads are turned &\if&\runheads ON:
  &\the\runhead,
  font &\runheadfont &\runheadfontsize,
  runhead takes up &\the\runheadht from column height.
&\else OFF &\fi
Running feet are turned &\if&\runfeet ON:
  &\the\runfoot,
  font &\runfootfont &\runfootfontsize,
  runfoot takes up &\the\runfootht from column height.
&\else OFF &\fi
Tabs are printed &\number\spacespertab characters wide, %
in style `&\tabstyle'.
&\if&\noextralinebreaks%
Line breaks are preserved exactly as in the original file.
&\else%
Extra line breaks are allowed at spaces and/or other characters.
Text is allowed to overrun right margin by &\the\overrun.
&\fi%
Control-L is defined to &\if\.N&\controlLaction%
end the current column.
&\else print as &\if\.L&\controlLaction%
a one-line vertical space.
&\else%
a boxed L.
&\fi&\fi%
System type is set to &\systemtype.
Directory for .los files: &\losdir
Directory prefix for file name titles: &\dirprefix
Press <return> to continue: }
%    \end{macrocode}
% \end{macro}
%
% \begin{macro}{\moptionI}
%    \begin{macrocode}
\def\moptionI{\menuIa \readline{}\reply \menuIb
  \readChar{Q}\reply \optionexec\reply}
%    \end{macrocode}
% \end{macro}
%
% %%%%%%%%%%%%%%%%%%%%%%%%%%%%%%%%%%%%%%%%%%%%%%%%%%%%%%%%%%%%%%%%%%%%%%
% \subsection{Option L}
%
% \begin{macro}{\menuL}
%    \begin{macrocode}
\fxmenu\menuL{}{
T  Turn line numbers&\if&\linenumbers off&\else on&\fi
F  Change font of line numbers
S  Change font size of line numbers
N  Change numbering frequency
}{&\if&\linenumbers
Current settings: &\linenumberfont &\linenumberfontsize, %
numbering every&\ifnum&\linenumberfreq=1 line%
&\else &\the&\linenumberfreq lines&\fi.%
&\fi%
}
%    \end{macrocode}
% \end{macro}
%
% \begin{macro}{\moptionL}
%
%    \begin{macrocode}
\def\moptionL{\lettermenu L}
%    \end{macrocode}
% \end{macro}
%
% \begin{macro}{\moptionLT}
%    When line numbering is toggled off, pop back up to the main menu.
%    But when line numbering is turned on, redisplay menu L in case
%    the user wants next to change the font or numbering frequency.
%    Also check \cw{leftmargin} (and \cw{intercolspace}, if
%    $\cw{totalcols}>1$) to see if they seem likely to provide enough
%    room.
%    \begin{macrocode}
\def\moptionLT{%
  \if\linenumbers
    \let\linenumbers\FALSE
  \else
    \let\linenumbers\TRUE
  \fi
  \confirm{Line numbers are switched \if\linenumbers on\else off\fi.}%
%    \end{macrocode}
%    This \cw{if} statement must be near the bogglitude record for \tex/
%    conditionals.
%    \begin{macrocode}
  \if\if\linenumbers\ifdim\leftmargin<12\p@ T
       \else\ifnum\totalcols>\@ne
          \ifdim\intercolspace<12\p@ T\else F\fi
        \else F\fi\fi\else F\fi T%
      \confirm{\
* You might want to check the left margin and/or intercolspace\
* to make sure there's enough room for the line numbers.}%
  \else\popoptions\fi
  \moptionQ
}
%    \end{macrocode}
% \end{macro}
%
% \begin{macro}{\moptionLF}
%
%    \begin{macrocode}
\def\moptionLF{\getfont\linenumberfont{}{Line number}}
%    \end{macrocode}
% \end{macro}
%
% \begin{macro}{\moptionLS}
%
%    \begin{macrocode}
\def\moptionLS{\getfontsize\linenumberfontsize{line numbers}}
%    \end{macrocode}
% \end{macro}
%
% \begin{macro}{\moptionLN}
%
%    \begin{macrocode}
\def\moptionLN{%
  \xmesj{%
Current line numbering frequency is every
\ifnum\linenumberfreq=1 line%
\else\ \number\linenumberfreq\ lines\fi.\
New line numbering frequency? (1 = every line, 2 = every 2 lines,\
and so forth):}%
  \readline{Q}\reply
  \if\xoptiontest\reply
  \else
    \checkinteger\reply\count@
    \ifnum\count@>\z@
      \linenumberfreq=\count@
      \confirm{New line numbering frequency: \the\linenumberfreq}%
      \def\reply{Q}%
    \fi
  \fi
  \optionexec\reply
}
%    \end{macrocode}
% \end{macro}
%
% %%%%%%%%%%%%%%%%%%%%%%%%%%%%%%%%%%%%%%%%%%%%%%%%%%%%%%%%%%%%%%%%%%%%%%
% \subsection{Option M}
%
% \begin{macro}{\menuM}
%    \begin{macrocode}
\fxmenu\menuM{
Current margin settings: top &\the&\topmargin, left &\the&\leftmargin,
bottom &\the&\bottommargin, right &\the&\rightmargin.
}{
L  Change left margin
T  Change top margin
B  Change bottom margin
R  Change right margin
}{}
%    \end{macrocode}
% \end{macro}
%
% \begin{macro}{\moptionM}
%
%    \begin{macrocode}
\def\moptionM{\lettermenu M}
%    \end{macrocode}
% \end{macro}
%
% \begin{macro}{\menuMB}
%    \begin{macrocode}
\fxmesj\menuMB#1{
Current bottom margin is #1.
%    \end{macrocode}
%
%    \begin{macrocode}
If you specify a new bottom margin, column height will be recomputed to
come as close as possible to the desired value (nearest integer multiple
of line spacing), taking the currently specified paper height and top
margin into consideration. To revert to automatic computing of bottom
margin, give an empty reply (i.e., just press the <return> key).
Otherwise enter a measurement using TeX units:}
%    \end{macrocode}
% \end{macro}
%
% \begin{macro}{\moptionMB}
%
%    \begin{macrocode}
\def\moptionMB{\menuMB{\the\bottommargin}%
  \readline{}\reply
  \ifx\@empty\reply
    \let\dependentbottommargin\TRUE
    \if\dependentcolht
      \computebc
%    \end{macrocode}
%    Otherwise \cw{bottommargin} should already have the correct value,
%    no extra computation needed.
%    \begin{macrocode}
    \fi
    \def\reply{Q}%
  \else
    \if\xoptiontest\reply
    \else
      \checkdimen\reply\dimen@
      \ifdim\dimen@=-\maxdimen % not a valid dimension
      \else
        \bottommargin\dimen@
        \computec
        \confirm{New column height: \the\colht\
* New bottom margin: \the\bottommargin}%
        \def\reply{Q}%
      \fi
    \fi
  \fi
  \optionexec\reply
}
%    \end{macrocode}
% \end{macro}
%
% \begin{macro}{\menuML}
%
%    \begin{macrocode}
\fxmesj\menuML#1{
Current left margin is #1 (measured from the left edge of the
paper to the left edge of the text).
<return> to keep the current value, or enter a new value:}
%    \end{macrocode}
% \end{macro}
%
% \begin{macro}{\moptionML}
%
%    \begin{macrocode}
\def\moptionML{\menuML{\the\leftmargin}%
  \readline{Q}\reply
  \if\xoptiontest\reply
  \else
    \checkdimen\reply\dimen@
    \ifdim\dimen@=-\maxdimen
    \else
      \leftmargin\dimen@
      \confirm{New left margin: \the\leftmargin}%
      \if\linenumbers\ifdim\leftmargin<12\p@
      \confirm{\
* Warning: small left margin, might not be big enough for line\
* numbers to fit ...}%
      \fi\fi
      \def\reply{Q}%
    \fi
  \fi
  \optionexec\reply
}
%    \end{macrocode}
% \end{macro}
%
% \begin{macro}{\menuMR}
%
%    \begin{macrocode}
\fxmesj\menuMR#1{
Current right margin is #1 (measured from the right edge of the
paper to the right edge of the text). To revert to automatic
computation of the right margin, give an empty reply; or enter a new
value using TeX units (or enter Q or X or ? to quit/exit/get-help).
New right margin:}
%    \end{macrocode}
% \end{macro}
%
% \begin{macro}{\moptionMR}
%
%    \begin{macrocode}
\def\moptionMR{\menuMR{\the\rightmargin}%
  \readline{}\reply
  \ifx\@empty\reply
    \let\dependentrightmargin\TRUE
    \rightmargin=\leftmargin
    \confirm{New right margin: \the\rightmargin}%
    \def\reply{Q}%
  \fi
  \if\xoptiontest\reply
  \else
    \checkdimen\reply\dimen@
    \ifdim\dimen@=-\maxdimen
    \else
      \rightmargin\dimen@
      \confirm{New right margin: \the\rightmargin}%
      \let\dependentrightmargin\FALSE
      \def\reply{Q}%
    \fi
  \fi
  \optionexec\reply
}
%    \end{macrocode}
% \end{macro}
%
% \begin{macro}{\menuMT}
%
%    \begin{macrocode}
\fxmesj\menuMT#1{
Current top margin is #1 (measured from the top edge of the
paper to the top of the running head or the text).
<return> to keep the current value, or enter a new value:}
%    \end{macrocode}
% \end{macro}
%
% \begin{macro}{\moptionMT}
%
%    \begin{macrocode}
\def\moptionMT{\menuMT{\the\topmargin}%
  \readline{Q}\reply
  \if\xoptiontest\reply
  \else
    \checkdimen\reply\dimen@
    \ifdim\dimen@=-\maxdimen
    \else
      \topmargin\dimen@
      \confirm{New top margin: \the\topmargin}%
      \def\reply{Q}%
    \fi
  \fi
  \optionexec\reply
}
%    \end{macrocode}
% \end{macro}
%
% %%%%%%%%%%%%%%%%%%%%%%%%%%%%%%%%%%%%%%%%%%%%%%%%%%%%%%%%%%%%%%%%%%%%%%
% \subsection{Option N}
%
% \begin{macro}{\menuN}
%    \begin{macrocode}
\fxmesj\menuN#1{
Number of columns: #1.
<return> to keep the current value, or enter a new value
(a number greater than 0):}
%    \end{macrocode}
% \end{macro}
%
% \begin{macro}{\moptionN}
%    \begin{macrocode}
\def\moptionN{\menuN{\number\totalcols}%
  \readline{Q}\reply
  \if\xoptiontest\reply
  \else
    \checkinteger\reply\count@
    \ifnum\count@=-\maxdimen
    \else
      \if\ifnum\count@>\z@
         \ifnum\count@<\sixt@@n T\else F\fi\else F\fi T%
        \chardef\totalcols=\count@
        \if\dependentcolwd
          \if\dependentrightmargin \computerfc \else \computefc \fi
        \else
          \computefr
        \fi
        \confirm{%
  Number of columns: \number\totalcols\space
  (column width: \the\colwd, %
\ifnum\totalcols>\@ne intercolumn space: \the\intercolspace,\
* \fi total width: \the\fullwd)}
        \def\reply{Q}%
      \else
        \ifnum\count@>20
%    \end{macrocode}
%    The \cw{specialhelp} function defines its first argument (in this
%    case, \cw{reply}) to be \qc{\?} and locally redefines
%    \cw{menuhelpmesj} to substitute the given text.
%    \begin{macrocode}
          \specialhelp\reply
            {Maximum number of columns is 20, sorry...}%
        \else
          \specialhelp\reply
            {Number of columns must be greater than 0}%
        \fi
      \fi
    \fi
  \fi
  \optionexec\reply
}
%    \end{macrocode}
% \end{macro}
%
% %%%%%%%%%%%%%%%%%%%%%%%%%%%%%%%%%%%%%%%%%%%%%%%%%%%%%%%%%%%%%%%%%%%%%%
% \subsection{Option O}
%
% \begin{macro}{\menuO}
%    \begin{macrocode}
\fxmenu\menuO{
Current orientation: &\if&\landscape\.landscape.&\else\.portrait&\fi
}{
P  Switch to portrait orientation
L  Switch to landscape orientation
}{}
%    \end{macrocode}
% \end{macro}
%
% \begin{macro}{\moptionO}
%
%    \begin{macrocode}
\def\moptionO{\menuO \readChar{Q}\reply
  \if P\reply
    \if\landscape \let\landscape\FALSE
      \dimen@\paht \paht\pawd \pawd\dimen@
    \fi
    \confirm{Orientation is now: portrait}%
    \def\reply{Q}%
  \else
    \if L\reply
      \if\landscape \else \let\landscape\TRUE
        \dimen@\paht \paht\pawd \pawd\dimen@
      \fi
      \confirm{Orientation is now: landscape}%
      \def\reply{Q}%
    \fi
  \fi
  \optionexec\reply
}
%    \end{macrocode}
% \end{macro}
%
% %%%%%%%%%%%%%%%%%%%%%%%%%%%%%%%%%%%%%%%%%%%%%%%%%%%%%%%%%%%%%%%%%%%%%%
% \subsection{Option P}
%
% \begin{macro}{\menuP}
%    \begin{macrocode}
\fxmenu\menuP{
Current paper size: &\paperwidth x &\paperheight
}{
W  Change paper width
H  Change paper height
U  U.S. letter size paper: 8.5in x 11in
A  A4 paper: 21cm x 29.7cm
}{}
%    \end{macrocode}
% \end{macro}
%
% \begin{macro}{\moptionP}
%
%    \begin{macrocode}
\def\moptionP{\lettermenu P}
%    \end{macrocode}
% \end{macro}
%
% \begin{macro}{\moptionPW}
%
%    \begin{macrocode}
\def\moptionPW{\promptmesj{New paper width: }%
  \readline{Q}\reply
  \if\xoptiontest\reply
  \else
    \checkdimen\reply\dimen@
    \ifdim\dimen@=-\maxdimen
    \else
      \ifdim\dimen@>1cm
        \let\paperwidth\reply
        \pawd\dimen@
        \confirm{New paper width: \paperwidth\space(\the\pawd)}%
        \def\reply{Q}%
      \else
        \specialhelp\reply{%
Sorry, I can't believe you really want a width of \reply!}%
      \fi
    \fi
  \fi
  \optionexec\reply
}
%    \end{macrocode}
% \end{macro}
%
% \begin{macro}{\moptionPH}
%
%    \begin{macrocode}
\def\moptionPH{\promptmesj{New paper height: }%
  \readline{Q}\reply
  \if\xoptiontest\reply
  \else
    \checkdimen\reply\dimen@
    \ifdim\dimen@=-\maxdimen
    \else
      \ifdim\dimen@>1cm
        \let\paperheight\reply
        \paht\dimen@
        \confirm{New paper height: \paperheight\space(\the\paht)}%
        \def\reply{Q}%
      \else
        \specialhelp\reply{%
Sorry, I can't believe you really want a height of \reply!}%
      \fi
    \fi
  \fi
  \optionexec\reply
}
%    \end{macrocode}
% \end{macro}
%
% \begin{macro}{\moptionPU}
%
%    \begin{macrocode}
\def\moptionPU{%
  \if\landscape
    \paht=8.5truein \pawd=11truein
    \def\paperheight{8.5in}\def\paperwidth{11in}%
  \else
    \paht=11truein \pawd=8.5truein
    \def\paperheight{11in}\def\paperwidth{8.5in}%
  \fi
  \confirm{New paper width: \paperwidth\space(\the\pawd)\
* New paper height: \paperheight\space(\the\paht)%
\if\landscape\
(landscape)\fi}%
  \moptionQ
}
%    \end{macrocode}
% \end{macro}
%
% \begin{macro}{\moptionPA}
%
%    \begin{macrocode}
\def\moptionPA{%
  \if\landscape
    \paht=21cm \pawd=29.7cm
    \def\paperheight{21cm}\def\paperwidth{29.7cm}%
  \else
    \paht=29.7cm \pawd=21cm
    \def\paperheight{29.7cm}\def\paperwidth{21cm}%
  \fi
  \confirm{New paper width: \paperwidth\space(\the\pawd)\
* New paper height: \paperheight\space(\the\paht)%
  \if\landscape\
  (landscape)\fi}%
  \moptionQ
}
%    \end{macrocode}
% \end{macro}
%
% %%%%%%%%%%%%%%%%%%%%%%%%%%%%%%%%%%%%%%%%%%%%%%%%%%%%%%%%%%%%%%%%%%%%%%
% \subsection{Option R}
%
% \begin{macro}{\menuR}
%    \begin{macrocode}
\fxmenu\menuR{
Running heads are turned &\if&\runheads\.on.
Running head contents:
&\the&\runhead
&\else\.off.
&\fi%
Running feet are turned &\if&\runfeet\.on.
Running foot contents:
&\the&\runfoot
&\else\.off.
&\fi%
}{
H  Change running heads
F  Change running feet
}{}
%    \end{macrocode}
% \end{macro}
%
% \begin{macro}{\moptionR}
%
%    \begin{macrocode}
\def\moptionR{\lettermenu R}
%    \end{macrocode}
% \end{macro}
%
% \begin{macro}{\menuRH}
%
%    \begin{macrocode}
\fmenu\menuRHorF#1{}{
C  Change running #1 contents
F  Change running #1 font
S  Change running #1 font size
}{}
%    \end{macrocode}
% \end{macro}
%
% \begin{macro}{\moptionRH}
%
%    \begin{macrocode}
\def\moptionRH{\menuRHorF{head}\readChar{Q}\reply \optionexec\reply}
%    \end{macrocode}
% \end{macro}
%
% \begin{macro}{\menuRHC}
%
%    \begin{macrocode}
\fxmesj\menuRHC{
Current running head contents:
&\the&\runhead
Enter new running head contents (to turn off running heads,
just press <return>):}
%    \end{macrocode}
% \end{macro}
%
% \begin{macro}{\moptionRHC}
%
%    Since the user response here can be arbitrary text, we need to do
%    some manual checking for the \qc{\?} \qc{\X} and \qc{\Q}
%    possibilities with \cw{xoptiontest} instead of calling
%    \cw{optionexec}.
%    \begin{macrocode}
\def\moptionRHC{\menuRHC
  \xreadline{}\reply
  \if\xoptiontest\reply
  \else
    \global\runhead\xp@{\reply}%
    \ifx\reply\@empty
      \global\let\runheads\FALSE
    \else
      \global\let\runheads\TRUE
      \confirm{New running head:\
               \the\runhead}%
    \fi
    \def\reply{Q}%
  \fi
  \optionexec\reply
}
%    \end{macrocode}
% \end{macro}
%
% \begin{macro}{\moptionRHF}
%
%    \begin{macrocode}
\def\moptionRHF{\getfont\runheadfont{}{Running head}}
%    \end{macrocode}
% \end{macro}
%
% \begin{macro}{\moptionRHS}
%    \begin{macrocode}
\def\moptionRHS{\getfontsize\runheadfontsize{running heads}}
%    \end{macrocode}
% \end{macro}
%
% \begin{macro}{\moptionRF}
%    \begin{macrocode}
\def\moptionRF{\menuRHorF{foot}\readChar{Q}\reply \optionexec\reply}
%    \end{macrocode}
% \end{macro}
%
% \begin{macro}{\menuRFC}
%    \begin{macrocode}
\fxmesj\menuRFC{
Current running foot contents:
&\the&\runfoot
Enter new running foot contents (to turn off running feet,
just press <return>):}
%    \end{macrocode}
% \end{macro}
%
% \begin{macro}{\moptionRFC}
%
%    Since the user response here can be arbitrary text, we need to do
%    some manual checking for the \qc{\?} \qc{\X} and \qc{\Q}
%    possibilities instead of calling \cw{optionexec}.
%    \begin{macrocode}
\def\moptionRFC{\menuRFC
  \xreadline{}\reply
  \if\xoptiontest\reply
  \else
    \global\runfoot\xp@{\reply}%
    \ifx\reply\@empty
      \global\let\runfeet\FALSE
    \else
      \global\let\runfeet\TRUE
      \confirm{New running foot:\
               \the\runfoot}%
    \fi
    \def\reply{Q}%
  \fi
  \optionexec\reply
}
%    \end{macrocode}
% \end{macro}
%
% \begin{macro}{\moptionRFF}
%
%    \begin{macrocode}
\def\moptionRFF{\getfont\runfootfont{}{Running foot}}
%    \end{macrocode}
% \end{macro}
%
% \begin{macro}{\moptionRFS}
%
%    \begin{macrocode}
\def\moptionRFS{\getfontsize\runfootfontsize{running feet}}
%    \end{macrocode}
% \end{macro}
%
% %%%%%%%%%%%%%%%%%%%%%%%%%%%%%%%%%%%%%%%%%%%%%%%%%%%%%%%%%%%%%%%%%%%%%%
% \subsection{Option S}
%
% \begin{macro}{\menuS}
%    \begin{macrocode}
\fxmenu\menuS{
System information. Current system type is &\systemtype;
directory prefix to be used for all files printed in this run is
&\ifx&\@empty&\dirprefix\.null.&\else
&\dirprefix&\fi
}{
S  Change system type
D  Change directory prefix
}{}
%    \end{macrocode}
% \end{macro}
%
% \begin{macro}{\moptionS}
%    \begin{macrocode}
\def\moptionS{\lettermenu{S}}
%    \end{macrocode}
% \end{macro}
%
% \begin{macro}{\menuSS}
%    \begin{macrocode}
\fxmenu\menuSS{
System type options:
}{
D  DOS
M  Macintosh
U  Unix
V  VAX/VMS
O  Other
}{
Currently this information is only used to help in printing directory
names that may occur in the titles printed at the beginning of each
file. Current system type is &\systemtype.
}
%    \end{macrocode}
% \end{macro}
%
% \begin{macro}{\moptionSS}
%    \begin{macrocode}
\def\moptionSS{\menuSS
  \readChar{Q}\reply
  \if D\reply \def\systemtype{DOS}\def\currentdir{./}\def\reply{Q}%
  \else\if M\reply
    \def\systemtype{Macintosh}\def\currentdir{}\def\reply{Q}%
  \else\if U\reply
    \def\systemtype{Unix}\def\currentdir{./}\def\reply{Q}%
  \else\if V\reply
    \def\systemtype{VAX/VMS}%
    \edef\currentdir{sys\string $disk:[]}\def\reply{Q}%
  \else \def\systemtype{other}\def\currentdir{}\def\reply{Q}%
  \fi\fi\fi\fi
  \optionexec\reply
}
%    \end{macrocode}
% \end{macro}
%
% \begin{macro}{\menuSD}
%    Directory prefix to be added in front of each filename title,
%    when printing multiple files. \tex/ version 3.x does not provide
%    a way to get at any part of a filename except \cw{jobname}.
%    Therefore in order for a directory name to be printed, it has to
%    be entered by the user somewhere along the line.
%    \begin{macrocode}
\fxmesj\menuSD{
Current directory/folder prefix for all file names is
&\ifx&\@empty&\dirprefix\.null.&\else
  `&\dirprefix'
&\fi
Enter the new directory prefix below.
(Enter Q to quit without changing the prefix.)}
%    \end{macrocode}
% \end{macro}
%
% \begin{macro}{\moptionSD}
%
%    \begin{macrocode}
\def\moptionSD{%
  \menuSD
  \readline{}\reply
  \let\next\reply \changecase\uppercase\next
  \if\xoptiontest\next
    \let\reply\next
  \else
    \let\dirprefix\reply
    \ifx\dirprefix\@empty
      \confirm{Directory prefix is now null.}%
    \else
      \confirm{New directory name:\
        \dirprefix\
      }%
    \fi
    \def\reply{Q}%
  \fi
  \optionexec\reply
}
%    \end{macrocode}
% \end{macro}
%
% %%%%%%%%%%%%%%%%%%%%%%%%%%%%%%%%%%%%%%%%%%%%%%%%%%%%%%%%%%%%%%%%%%%%%%
% \subsection{Option T}
%
% \begin{macro}{\menuT}
%    \begin{macrocode}
\fxmenu\menuT{
Currently tabs will print as &\tabstyle.
The horizontal extent of each tab will be %
&\number&\spacespertab spaces.
}{
T  Change tab representation
N  Change the number of spaces per tab
}{}
%    \end{macrocode}
% \end{macro}
%
% \begin{macro}{\moptionT}
%
%    \begin{macrocode}
\def\moptionT{\lettermenu T}
%    \end{macrocode}
% \end{macro}
%
% \begin{macro}{\menuTT}
%
%    \begin{macrocode}
\fmenu\menuTT#1{
Tab representation choices:
}{
B  Blank space
.  Dots
-  Dashed line
H  Hrulefill
T  Tiny`TAB's fill
}{
These are better understood by trying them out than by having them
explained in words. Current tab style is `#1'.
}
%    \end{macrocode}
% \end{macro}
%
% \begin{macro}{\reportnewtabstyle}
%
%    \begin{macrocode}
\def\reportnewtabstyle#1{%
  \chardef\tabselect=#1 \relax
  \confirm{Tab style is now `\tabstyle'}%
  \def\reply{Q}%
}
%    \end{macrocode}
% \end{macro}
%
% \begin{macro}{\moptionTT}
%
%    \begin{macrocode}
\def\moptionTT{\menuTT\tabstyle
  \readChar{Q}\reply
  \if B\reply
    \reportnewtabstyle{0}%
  \else\if .\reply
    \reportnewtabstyle{1}%
  \else\if -\reply
    \reportnewtabstyle{2}%
  \else\if H\reply
    \reportnewtabstyle{3}%
  \else\if T\reply
    \reportnewtabstyle{4}%
  \fi\fi\fi\fi\fi
  \optionexec\reply
}
%    \end{macrocode}
% \end{macro}
%
% \begin{macro}{\moptionTN}
%    \begin{macrocode}
\def\moptionTN{%
  \promptmesj{Enter the desired number of spaces per tab: }%
  \readline{Q}\reply
  \if\xoptiontest\reply
  \else
    \checkinteger\reply\count@
    \ifnum\count@=-\maxdimen
    \else
      \ifnum\count@<256
        \chardef\spacespertab=\count@
        \confirm{%
  The number of spaces per tab is now \number\spacespertab.}%
        \def\reply{Q}%
      \else
        \mesj{Sorry, spaces per tab cannot be more than 255.}%
        \def\reply{}%  repeat the prompt
      \fi
    \fi
  \fi
  \optionexec\reply
}
%    \end{macrocode}
% \end{macro}
%
% %%%%%%%%%%%%%%%%%%%%%%%%%%%%%%%%%%%%%%%%%%%%%%%%%%%%%%%%%%%%%%%%%%%%%%
% \subsection{Option V}
%
% \begin{macro}{\menuV}
%    \begin{macrocode}
\fmenu\menuV{}{
  8  Treatment of 8-bit characters
  L  Treatment of Control-L characters
}{}
%    \end{macrocode}
% \end{macro}
%
% \begin{macro}{\moptionV}
%    \begin{macrocode}
\def\moptionV{\lettermenu{V}}
%    \end{macrocode}
% \end{macro}
%
% \begin{macro}{\menuVL}
%    \begin{macrocode}
\fxmenu\menuVL{
Three choices are offered for the treatment of Control-L characters:
}{
  B  Print a boxed L (same treatment as other control characters)
  L  Print one line of space
  N  End current text column, start a new column
}{
Current setting is &\controlLaction.
}
%    \end{macrocode}
% \end{macro}
%
% \begin{macro}{\moptionVL}
%    \begin{macrocode}
\def\moptionVL{\menuVL
  \readChar{Q}\reply
  \if B\reply
    \let\controlLaction\reply
    \confirm{Control-L characters will print as boxed L's}%
  \else\if L\reply
    \let\controlLaction\reply
    \confirm{Control-L characters will print as a one-line space}%
  \else\if N\reply
    \let\controlLaction\reply
    \confirm{%
Control-L characters will perform a `formfeed' [new column] action}%
  \fi\fi\fi
  \def\reply{Q}\optionexec\reply
}
%    \end{macrocode}
% \end{macro}
%
% \begin{macro}{\menuV8}
%    \begin{macrocode}
\expandafter\fxmenu\csname menuV8\endcsname{
Two choices are offered for the treatment of 8-bit characters, that
is, characters in the range 128--255:
}{
  A  Print according to the current font
  B  Print a boxed question mark
}{
Current setting is &\eightbitaction.
For option A, you must be careful to also select a font that is capable
of printing 8-bit characters (see option F in the main menu); otherwise,
8-bit characters will disappear silently from the printed output with
nothing more than a warning in the TeX log.
}
%    \end{macrocode}
% \end{macro}
%
% \begin{macro}{\moptionV8}
%    \begin{macrocode}
\expandafter\def\csname moptionV8\endcsname{%
  \csname menuV8\endcsname
  \readchar{Q}\reply
  \if A\reply \def\eightbitaction{A}%
  \else\if B\reply \def\eightbitaction{B}%
  \fi\fi
  \def\reply{Q}\optionexec\reply
}
%    \end{macrocode}
% \end{macro}
%
% %%%%%%%%%%%%%%%%%%%%%%%%%%%%%%%%%%%%%%%%%%%%%%%%%%%%%%%%%%%%%%%%%%%%%%
%    Here perhaps I should clear out the memory used by the menus
%    since it is no longer needed. But so far I haven't had any pages
%    complex enough to exceed 64K main mem even after loading full
%    \fn{listout.tex} with the menu system. [mjd,25-Sep-1993]
%    \begin{macrocode}
\moption
\printfiles
%    \end{macrocode}
%
%    And a final \cw{endinput} to make docstrip skip the garbage below,
%    otherwise docstrip would choke on \ctrl{B} saying it's an invalid
%    character.
%    \begin{macrocode}
\endinput
%    \end{macrocode}
%
%^^A Undefine FORMFEED (control-L) and DEL (control-?) so they will
%^^A not cause the macrocode environment to choke.
%
%\catcode`\^^L=12 \catcode`\^^?=12
%%
\begin{macrocode}
Test of special characters (if listout.tex is used to print itself).
%
The NUL character is missing (Control-@) because apparently Unix Emacs
can't or won't preserve the character when saving the file. Put it
back in with your editor if you can. Control-J and Control-M are
printed at the end of their lines to avoid commenting out any
following characters (depending on the native file format of the
user's operating system). Control-Z is printed at the end of the file
to avoid premature file termination if the operating system is DOS.
The tab character (control-I) should show up as a dotted line 8
characters wide under the default settings of \spacespertab and
\tabstyle. And the formfeed character (control-L) should cause a page
break.
%
Control-@: ,
Control-A: , Control-B; , Control-C: ,
Control-D: , Control-E: , Control-F: ,
Control-G: , Control-H: ,
	Control-I: 	, Control-J:
, Control-K: , Control-L: , Control-M:
, Control-N: , Control-O: , Control-P: , Control-Q: ,
Control-R: , Control-S: , Control-T: , Control-U: ,
Control-V: , Control-W: , Control-X: , Control-Y: ,
Control-Z: (printed at end of file, below)
Control-[: , Control-\: , Control-]: , Control-^: ,
Control-_: , Control-?: .
ASCII 128: �, ASCII 131: �, ASCII 137: �, ASCII 139: �,
ASCII 149: �, ASCII 151: �, ASCII 157: �, ASCII 163: �,
ASCII 167: �, ASCII 173: �, ASCII 179: �, ASCII 181: �,
ASCII 191: �, ASCII 193: �, ASCII 197: �, ASCII 199: �,
ASCII 211: �, ASCII 223: �, ASCII 227: �, ASCII 229: �,
ASCII 233: �, ASCII 239: �, ASCII 241: �, ASCII 251: �,
ASCII 255: �.
%
Control-Z: .
%    \end{macrocode}
