% \iffalse
%% File: trfsigns.dtx Copyright (C) 1999 Kai Rascher
%
%<*dtx>
          \ProvidesFile{trfsigns.dtx}
%</dtx>
%<package>\NeedsTeXFormat{LaTeX2e}
%<package>\ProvidesPackage{trfsigns}
%<driver>\ProvidesFile{trfsigns.drv}
% \fi
%         \ProvidesFile{trfsigns.dtx}
%         [1999/08/07 1.01 Space after command names (DPC)]
%
% \iffalse
%<*driver>
\documentclass{ltxdoc}
\usepackage{german}
\usepackage{trfsigns}
\begin{document}
\DocInput{trfsigns.dtx}
\end{document}
%</driver>
% \fi
%
% \GetFileInfo{trfsigns.dtx}
% \title{Transformationszeichen mit \LaTeXe\\
%        --~trfsigns\thanks{Diese Datei
%        hat die Versionsnummer \fileversion\ --\
%        letzte "Uberarbeitung \filedate.}~~--}
% \author{Kai Rascher\\
%         Am Remenhof 17a\\
%         D--38104 Braunschweig\\
%         {\tt rascher@ifn.ing.tu-bs.de}}
% \date{Version~\fileversion}
% \maketitle
%
% %%%%%%%%%%%%%%%%%%%%%%%%%%%%%%%%%%%%%%%%%%%%%%%%%%%%%%%%%%%%%%%%%%%%
%
% \changes{v1.00}{1998/01/05}{Erste, unver"offentlichte Version}
% \changes{v1.01}{1999/08/07}{Erg"angung der Macros f"ur Eulersche Zahl
%                             und imagin"arer Einheit}
%
% \section{Kurzinformation}
%
% In den Natur- und Ingenieurswissenschaften spielen zahlreiche
% Transformationen eine wichtige Rolle. Insbesondere in der Elektrotechnik
% wird sehr h"aufig mit Zeitsignalen von Str"omen und Spannungen und den
% korrespondierenden Frequenzfunktionen bzw.\ Frequenzwerten gearbeitet.
%
% In Lehrb"uchern zum Thema \glqq lineare Systeme\grqq, \glqq Systemtheorie\grqq\
% u.\,"a.\ wird die Zuordnung zwischen den Zeit- und Frequenzfunktionen h"aufig
% mit sogenannten Transformations- oder Korrespondenzzeichen symbolisiert.
% Da diese Zeichen in den mathematischen Zeichen von \LaTeX\ nicht vorhanden
% sind, wurden im Paket trfsigns die Transformationszeichen, wie sie in
% \cite{UnbehauenR1} verwendet werden, definiert. Die angesprochenen Zeichen
% werden dabei aus \LaTeX-Zeichen zusammengesetzt.
%
% Zus"atzlich zu den Transformationszeichen sind in \verb|trfsigns.sty| das
% Zeichen f"ur die Eulersche Zahl \(\e\) als \verb|\e| und die imagin"are
% Einheit \(\im\)  als \verb|\im| definiert. Diese Zeichen sollten nicht als
% \(e\) bzw.\ \(j\) (Eingabe \verb|e| bzw.\ \verb|j|) gesetzt werden, da es
% sich dann um gew"ohnliche Variablen handeln w"urde.
%
% \section{Verwendung}
% \subsection{Laden der Style-Option}
% Um die Transformationszeichen und die beiden anderen Zeichen verwenden zu 
% k"onnen, mu"s die Style-Option mit einem Aufruf
%
% \verb|\usepackage{trfsigns}|
%
% nach der Deklaration der Dokumentenklasse mit \verb|\documentclass| geladen
% werden.
%
%
% \subsection{Befehle}
% Nach dem Laden stehen insgesamt 10~Befehle f"ur die Transformationszeichen
% zur Verf"ugung, die sowohl im Text-Modus als auch im mathematischen Modus 
% verwendet werden k"onnen. Die Transformationszeichen wurden in Abh"angigkeit
% der aktuellen Fontgr"o"se definiert.
% \begin{table}[htbp]
% \centering
% \renewcommand{\arraystretch}{1.25}
% \makeatletter
% \setlength{\@tempdima}{\abovecaptionskip}%
% \setlength{\abovecaptionskip}{\belowcaptionskip}%
% \setlength{\belowcaptionskip}{\@tempdima}%
% \makeatother
% \caption{Befehle zur Ausgabe der Transformationszeichen, der Eulerschen
%          Zahl und der imagin"aren Einheit }
% \label{tab:Befehle}
% \begin{tabular}{lll}
% (kontinuierliche)             & \verb!\fourier!  & \fourier\\
% Fourier-Transformation        & \verb!\Fourier!  & \Fourier\\[1.5ex]
% Laplace-Transformation        & \verb!\laplace!  & \laplace\\
%                               & \verb!\Laplace!  & \Laplace\\[1.5ex]
% diskontinuierliche            & \verb!\dfourier! & \dfourier\\
% Fourier-Transformation        & \verb!\Dfourier! & \Dfourier\\[1.5ex]
% Z-Transformation              & \verb!\ztransf!  & \ztransf\\
%                               & \verb!\Ztransf!  & \Ztransf\\[1.5ex]
% diskrete Fourier-             & \verb!\dft{N}!   & \dft{N}\\
% Transformation der L"ange $N$ & \verb!\DFT{N}!   & \DFT{N}\\[1.5ex]
% Eulersche Zahl                & \verb|\e|        & \e\\
% imagin"are Einheit            & \verb|\im|       & \im\\
% \end{tabular}
% \end{table}
%
% Die Korrespondenzzeichen k"onnen mit den Gr"o"senumschaltbefehlen 
% \verb!\small!, \verb!\large!, \verb!\Large! usw.\ verkleinert und 
% vergr"o"sert werden:
%
% {\small \fourier, \laplace, \dfourier, \ztransf, \dft{N};
%         \Fourier, \Laplace, \Dfourier, \Ztransf, \DFT{N}}
%
% {\large \fourier, \laplace, \dfourier, \ztransf, \dft{N};
%         \Fourier, \Laplace, \Dfourier, \Ztransf, \DFT{N}}
%
% {\Large \fourier, \laplace, \dfourier, \ztransf, \dft{N};
%         \Fourier, \Laplace, \Dfourier, \Ztransf, \DFT{N}}
%
%
% \subsubsection*{Bemerkungen}
% \begin{enumerate}
%   \item Die Transformationsl"ange $N$ wird standardm"a"sig mit einem Font der
%         Gr"o"se \verb!\footnotesize! gesetzt, dies ist nur f"ur die normale
%         Fontgr"o"se korrekt. Beim Umschalten auf eine andere
%         Fontgr"o"se mit \verb!\large! o.\,"a.\ mu"s der \glqq Abstand\grqq\
%         von zwei Gr"o"senstufen explizit angegeben werden:
%
%         \verb!\large\dft{N}! ergibt {\large\dft{N}}; falsch!
%
%         \verb!\large\dft{\mbox{\small $N$}}! ergibt
%         {\large\dft{\mbox{\small $N$}}}
%
%         \verb!\Large\dft{\mbox{\normalsize $N$}}! ergibt
%         {\Large\dft{\mbox{\normalsize $N$}}}
%
%         \verb!\LARGE\dft{\mbox{\large $N$}}! ergibt
%         {\LARGE\dft{\mbox{\large $N$}}}
%
%         \verb!\huge\dft{\mbox{\Large $N$}}! ergibt
%         {\huge\dft{\mbox{\Large $N$}}}
%
%         \verb!\Huge\dft{\mbox{\LARGE $N$}}! ergibt\rule[-3mm]{0mm}{10mm}
%         {\Huge\dft{\mbox{\LARGE $N$}}}
%   \item Die Transformationszeichen k"onnen auch in "Uberschriften verwendet
%         werden. Dann mu"s jedoch ein \verb!\protect! vorangestellt werden.
% \end{enumerate}
%
% \section{Implementierung}
%    \begin{macrocode}
%<*package>
%    \end{macrocode}
%
% \begin{macro}{\laplace}
% Alle Transformationszeichen sind aus \LaTeX-Zeichen zusammengesetzt,
% wobei die \verb|picture|-Umgebung in eine \verb|mbox| eingebettet wurde,
% damit eine Verwendung im Text-Modus und im mathematischen Modus m"oglich
% ist
%    \begin{macrocode}
\newcommand{\laplace}{\mbox{\setlength{\unitlength}{0.1em}%
                            \begin{picture}(20,10)%
                              \put(2,3){\circle{4}}%
                              \put(4,3){\line(1,0){13}}%
                              \put(18,3){\circle*{4}}%
                            \end{picture}%
                           }%
                     }%
%    \end{macrocode}
% \end{macro}
% \begin{macro}{\Laplace}
% Der ausgef"ullte Kreis ist stets der Laplace-Transformierten zugewandt, so
% da"s eine entsprechende Definition f"ur das umgedrehte Zeichen verwendet
% wird.
%    \begin{macrocode}
\newcommand{\Laplace}{\mbox{\setlength{\unitlength}{0.1em}%
                            \begin{picture}(20,10)%
                              \put(2,3){\circle*{4}}%
                              \put(3,3){\line(1,0){13}}%
                              \put(18,3){\circle{4}}%
                            \end{picture}%
                           }%
                     }%
%    \end{macrocode}
% \end{macro}
% \begin{macro}{\fourier}
% Die beiden Zeichen f"ur die kontinuierliche Fourier-Transformation
% besitzen die gleiche Ausdehung wie alle "ubrigen Zeichen, damit 
% Formeln gleichartig ausgerichet werden.
%    \begin{macrocode}
\newcommand{\fourier}{\mbox{\setlength{\unitlength}{0.1em}%
                            \begin{picture}(20,10)%
                              \put(2,3){\circle{4}}%
                              \put(4,3){\line(1,0){12}}%
                            \end{picture}%
                           }%
                      }%
%    \end{macrocode}
% \end{macro}
% \begin{macro}{\Fourier}
%    \begin{macrocode}
\newcommand{\Fourier}{\mbox{\setlength{\unitlength}{0.1em}%
                            \begin{picture}(20,10)%
                              \put(16,3){\line(-1,0){12}}%
                              \put(18,3){\circle{4}}%
                            \end{picture}%
                           }%
                      }%
%    \end{macrocode}
% \end{macro}
% \begin{macro}{\dfourier}
% Die schr"age Linie im Zeichen f"ur die diskontinuierliche 
% Fourier-Transformation wurde der Einfachheit halber aus kleinen Punkten
% zusammengesetzt. Dies ist eine suboptimale L"osung, der Befehl \verb|line|
% konnte jedoch nicht verwendet werden, da die Linie k"urzer als die minimal
% m"ogliche L"ange ist.
%    \begin{macrocode}
\newcommand{\dfourier}{\mbox{\setlength{\unitlength}{0.1em}%
                             \begin{picture}(20,10)%
                               \put(2,3){\circle{4}}%
                               \put(4,3){\line(1,0){4.75}}%
                               \multiput(8.625,3.15)(0.25,0.25){11}{%
                                 \makebox(0,0){\rm\tiny .}}%
                             \end{picture}%
                            }%
                      }%
%    \end{macrocode}
% \end{macro}
% \begin{macro}{\Dfourier}
%    \begin{macrocode}
\newcommand{\Dfourier}{\mbox{\setlength{\unitlength}{0.1em}%
                             \begin{picture}(20,10)%
                               \multiput(11.375,3.15)(-0.25,0.25){11}{%
                                 \makebox(0,0){\rm\tiny .}}%
                               \put(16,3){\line(-1,0){4.75}}%
                               \put(18,3){\circle{4}}%
                             \end{picture}%
                            }%
                      }%
%    \end{macrocode}
% \end{macro}
% \begin{macro}{\dft}
% Als einzige Macros mit optionalem Parameter kann bei den Zeichen f"ur die
% diskrete Fourier-Transformation die Transformationsl"ange mit angegeben
% werden.
%    \begin{macrocode}
\newcommand{\dft}[1]{\mbox{\setlength{\unitlength}{0.1em}%
                           \begin{picture}(20,10)%
                             \put(0,1){\line(0,1){4}}%
                             \put(0,3){\line(1,0){17}}%
                             \put(8.5,1){%
                               \makebox(0,0)[t]{\footnotesize $#1$}}%
                           \end{picture}%
                          }%
                    }%
%    \end{macrocode}
% \end{macro}
% \begin{macro}{\DFT}
% Man beachte, da"s der Macroname vollst"andig aus Gro"sbuchstaben besteht.
%    \begin{macrocode}
\newcommand{\DFT}[1]{\mbox{\setlength{\unitlength}{0.1em}%
                           \begin{picture}(20,10)%
                             \put(20,1){\line(0,1){4}}%
                             \put(20,3){\line(-1,0){17}}%
                             \put(11.5,1){%
                               \makebox(0,0)[t]{\footnotesize $#1$}}%
                           \end{picture}%
                          }%
                    }%
%    \end{macrocode}
% \end{macro}
% \begin{macro}{\ztransf}
% Schie"slich sind noch die Definitionen f"ur die Z-Transformation gegeben.
%    \begin{macrocode}
\newcommand{\ztransf}{\mbox{\setlength{\unitlength}{0.1em}%
                            \begin{picture}(20,10)%
                              \put(2,3){\circle{4}}%
                              \put(4,3){\line(1,0){4.75}}%
                              \multiput(8.625,3.15)(0.25,0.25){11}{%
                                \makebox(0,0){\rm\tiny .}}%
                              \put(17,3){\line(-1,0){5.75}}%
                              \put(18,3){\circle*{4}}%
                            \end{picture}%
                           }%
                      }%
%    \end{macrocode}
% \end{macro}
% \begin{macro}{\Ztransf}
%    \begin{macrocode}
\newcommand{\Ztransf}{\mbox{\setlength{\unitlength}{0.1em}%
                            \begin{picture}(20,10)%
                              \put(2,3){\circle*{4}}%
                              \put(3,3){\line(1,0){5.75}}%
                              \multiput(11.375,3.15)(-0.25,0.25){11}{%
                                \makebox(0,0){\rm\tiny .}}%
                              \put(16,3){\line(-1,0){4.75}}%
                              \put(18,3){\circle{4}}%
                            \end{picture}%
                           }%
                      }%
%    \end{macrocode}
% \end{macro}
% \begin{macro}{\e}
% Die Eulersche Zahl wird als Textzeichen im mathematischen Mode gesetzt,
% wobei ein horizontaler Abstand von 2/18 von einem quad eingef"ugt wird.
%    \begin{macrocode}
\newcommand{\e}{\ensuremath{\mathrm{e\;\!}}}
%    \end{macrocode}
% \end{macro}
% \begin{macro}{\im}
% Da in \LaTeXe\ die Befehle \verb|\i| (\i) und \verb|\j| (\j) bereits
% definiert sind, wird als Befehlsnahme \verb|\im| verwendet.
%    \begin{macrocode}
\newcommand{\im}{\ensuremath{\mathrm{j}}}
%    \end{macrocode}
% \end{macro}
%    \begin{macrocode}
%</package>
%    \end{macrocode}
%
% \section{Beispiel}
%    \begin{macrocode}
%<*sample>
\documentclass[11pt,a4paper]{article}
\usepackage{german}
\usepackage{trfsigns}

\begin{document}
\noindent
\textbf{\LARGE Transformationszeichen der}

\section*{Laplace-Transformation \protect\laplace, \protect\Laplace}
\begin{eqnarray}
  s(t)        &\laplace& \frac{1}{p}\\[1ex]
  \frac{1}{p} &\Laplace& s(t)
\end{eqnarray}


\section*{(kontinuierlichen) Fourier-Transformation\newline
          \protect\fourier, \protect\Fourier}
\begin{eqnarray}
  \cos(\omega_0 t) &\fourier& 
  \pi\left[\delta(\omega-\omega_0) +
           \delta(\omega+\omega_0)\right]\\[1ex]
  \pi\left[\delta(\omega-\omega_0) +
  \delta(\omega+\omega_0)\right] &
  \Fourier&
  \cos(\omega_0 t)
\end{eqnarray}


\section*{diskontinuierlichen Fourier-Transformation\newline
          \protect\dfourier, \protect\Dfourier}
\begin{eqnarray}
  a^n s[n]\quad\mbox{mit~}|a| < 1 &\dfourier& 
  \frac{1}{1-a \e^{\im\omega T}}\\[1ex]
  \frac{1}{1-a \e^{\im\omega T}}  &\Dfourier& 
  a^n s[n]\quad\mbox{mit~}|a| < 1
\end{eqnarray}


\section*{Z-Transformation
          \protect\ztransf, \protect\Ztransf}
\begin{eqnarray}
  s[n]          &\ztransf& \frac{z}{z-1}\\[1ex]
  \frac{z}{z-1} &\Ztransf& s[n]
\end{eqnarray}


\section*{diskreten Fourier-Transformation
          \protect\dft{}, \protect\DFT{}}
\begin{eqnarray}
  f[n] &\dft{N}& 
  F[m] = \sum_{n=0}^{N-1} f[n] \e^{-\im\frac{2 \pi}{N}nm}%
  \:\;(m = 0, 1, \ldots, N-1)\\[1ex]
  F[m] &\DFT{N}& 
  f[n] = \frac{1}{N}\sum_{m=0}^{N-1} F[m] \e^{\im\frac{2\pi}{N}mn}%
  \:\;(n = 0, 1, \ldots, N-1)
\end{eqnarray}

\end{document}
%</sample>
%    \end{macrocode}
%
% \begin{thebibliography}{99}
%   \bibitem{UnbehauenR1} Unbehauen, Rolf: \emph{Systemtheorie}, R.~Oldenbourg
%                         Verlag, M"unchen, 6.~Auf"|lage, 1993.
% \end{thebibliography}
%
% \Finale
%