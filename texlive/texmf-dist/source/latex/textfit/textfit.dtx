\def\fileversion{5}
\def\filedate{1994/04/15}
\def\docdate{1994/04/15}
% \iffalse
%% File: textfit.dtx Copyright (C) 1994 Sebastian Rahtz and Phil Taylor
% 
% This work may be distributed and/or modified under the
% conditions of the LaTeX Project Public License, either version 1.3
% of this license or (at your option) any later version.
% The latest version of this license is in
%   http://www.latex-project.org/lppl.txt
% and version 1.3 or later is part of all distributions of LaTeX
% version 2005/12/01 or later.
%
% This work has the LPPL maintenance status `maintained'.
% 
% The Current Maintainers of this work are Sebastian Rahtz and Phil Taylor.
%
% This work consists of all files listed in manifest.txt.
%
% This licence statement added 2010/11/26 by Clea F. Rees on behalf
% of Sebastian Rahtz and Phil Taylor.
%
%<*driver>
\documentclass{ltxdoc}
\usepackage{textfit}
\begin{document}
 \title{The \textsf{textfit} package for
scaling up text to a desired size\thanks{This file
        has version number \fileversion, last
        revised \filedate.}}
 \author{Sebastian Rahtz \and Phil Taylor}
 \date{\filedate}
 \maketitle
 \tableofcontents
 \DocInput{textfit.dtx}
 \end{document}
%</driver>
% \fi
% \CheckSum{281}
%
% \section{Introduction}
%
% The user needs to be able to specify the width or height for text, and
% use the largest size necessary to fill that width or height. You
% can choose whether to have the size increased by simple points, or
% by using magsteps.
% Usage:
% \verb|\scaletowidth{3in}{This is the way of the world}|
% \verb|\scaletoheight{7mm}{This is the way of the world}|
% Details are controlled by two options: `magstep' will
% make the system use magsteps, not points, and `noisy'
% will produce useful messages.
% \changes{1}{5/2/93}{Worked out point size by trial and error, increasing
%   it in increments of 1pt until result reached desired size}
% \changes{2}{9/2/93}{Final point size calculated by comparison of size at
%    10pt. faster, and takes less TeX memory for intermediate fonts}
% \changes{3}{10/1/94}{Revised to make it a \LaTeXe\ package}
% \changes{4}{6/2/94}{It only worked if `noisy' option was given ...}
% \changes{5}{15/4/94}{Checked again for \LaTeXe}
%
%Examples; note that the base font is taken from the current state when
%you enter the macros. so if you start off with a large font (eg
%cmr17), it will give a different result from that obtained by scaling
%up cmr5.
%
%\scaletowidth{1in}{This is the way of the world}
%
%{\Huge \scaletowidth{1in}{This is the way of the world}}
%
%{\Huge \scaletowidth{5in}{This is the way of the world}}
%
%\scaletowidth{5in}{This is the way of the world}
%
%{\Huge \scaletoheight{1.3in}{Cowabunga!} }
%
% \scaletowidth{5in}{Gloucestershire  Warwickshire Railway}
%
%\ScalebyMagstepstrue
%
%\scaletowidth{1in}{This is the way of the world}
%
%{\Huge \scaletowidth{1in}{This is the way of the world}}
%
%{\Huge \scaletowidth{5in}{This is the way of the world}}
%
%\scaletowidth{5in}{This is the way of the world}
%
%{\Huge \scaletoheight{1.3in}{Cowabunga!} }
% \StopEventually{}
%    \begin{macrocode}
%<*package>
\NeedsTeXFormat{LaTeX2e}
\ProvidesPackage{textfit}[\filedate]
\newif\ifScalebyMagsteps
\newif\ifNoisyFitting
\NoisyFittingfalse
\ScalebyMagstepsfalse
\DeclareOption{noisy}{\NoisyFittingtrue}
\DeclareOption{magstep}{\ScalebyMagstepstrue}
\ProcessOptions
\newcount \mags@f@r
\newdimen \desired@size
\newcount \r@mainder
\newcount \m@gstepcount
\newcount \m@gsteplimit
\newcount \m@gstepvalue
\newdimen \m@gstepdimen
%    \end{macrocode}
% Code for working out the right magstep (this is Phil's bit).
%    \begin{macrocode}
\def \magsteps {\afterassignment \m@gsteps \m@gsteplimit = }
\def \m@gsteps
    {\let \@r = \or
     \def \or {\noexpand \or}%
     \m@gstepdimen = 0.166667 \maxdimen
     \m@gstepvalue = 1000 
     \m@gstepcount = 0
     \def \magstep {}%
     \loop
     \ifnum     \m@gstepcount < \m@gsteplimit
                \advance \m@gstepcount by 1
                \ifnum \m@gstepvalue > \m@gstepdimen
                        \advance \m@gstepcount by -1
                        \message {Sorry --- integer overflow would occur if
                                        I went any further; stopped at 
                                                \the \m@gstepcount.}
                        \m@gsteplimit = \m@gstepcount 
                \else
                        \multiply \m@gstepvalue by 12
                        \multiply \r@mainder by 12
                        \divide \r@mainder by 10
                        \advance \m@gstepvalue by \r@mainder
                        \r@mainder = \m@gstepvalue
                        \divide \m@gstepvalue by 10
                        \multiply \m@gstepvalue by 10
                        \advance \r@mainder by -\m@gstepvalue
                        \divide \m@gstepvalue by 10
                        \ifnum \r@mainder < 5
                               \edef \magstep {\magstep \or \the \m@gstepvalue}%
                        \else
                               \advance \m@gstepvalue by 1
                               \edef \magstep {\magstep \or \the \m@gstepvalue}%
                               \advance \m@gstepvalue by -1
                        \fi
                \fi
     \repeat
     \edef \magstep ##1%
          {\noexpand \ifcase ##1
                        1000\magstep 
           \noexpand \else 0%
           \noexpand \message 
                {Sorry --- \string \magstep \space ##1
                                is not in range 0 .. \the \m@gsteplimit}%
           \noexpand \fi
           \noexpand \relax
          }%
     \let \or = \@r
    }
\magsteps 63
\def \Fontname #1{\expandafter \strip@size \fontname #1 }
\def \strip@size #1 #2#{#1\void@@@}
\def \void@@@ #1{}
\gdef\n@d#1.#2:{#1}
\def \@scaletofit[#1]#2#3{%
     \def\Hb@x{h}\def\H@rV{#1}%
     \ifx\H@rV\Hb@x\def\C@mpare{\wd}\else\def\C@mpare{\ht}\fi
     \desired@size #2 
     \def\curr@fontname{\Fontname \font {} }%
     \ifScalebyMagsteps
        \mags@f@r \z@
        \setbox 0 = \hbox{#3}%
        \def\@increment{1}%
	\def\@test{<}%
        \ifdim \C@mpare 0 > \desired@size
           \message{When scaling by Magsteps, you cannot go downwards!}%
        \fi
        \loop
        \font \temp@font = \curr@fontname  scaled \magstep \mags@f@r
	\setbox 0 = \hbox{{\temp@font #3}}%
        \ifNoisyFitting
          \message {Trying \noexpand \magstep \the \mags@f@r...}%
        \fi
        \ifdim \C@mpare 0 \@test \desired@size
           \advance \mags@f@r by \@increment
        \repeat
        \ifnum \mags@f@r=\z@
           \mags@f@r1
         \fi
         \advance \mags@f@r by -\@increment
     \else
        \font\temp@font=\curr@fontname at1pt
        \setbox 0 = \hbox{{\temp@font #3}}%
        \@tempdima\C@mpare0
        \multiply\@tempdima by 500 %\@m
        \def\foo@@{\expandafter\n@d\the\@tempdima:}%
        \@tempdimb\desired@size
        \divide\@tempdimb by \foo@@
        \multiply\@tempdimb by 500 %\@m
     \fi
     \ifScalebyMagsteps
        \ifNoisyFitting
	  \message {The calculated font is \curr@fontname 
           scaled \noexpand \magstep \the \mags@f@r}%
        \fi
        \font \temp@font = \curr@fontname scaled \magstep \mags@f@r
     \else
        \ifNoisyFitting
           \message {The calculated font is \curr@fontname ->\the\@tempdimb}%
        \fi
        \font\temp@font=\curr@fontname at \the\@tempdimb
     \fi
     \ifx\H@rV\Hb@x
        \hbox to \desired@size {\hss{\temp@font #3}\hss}%
     \else
        \hbox {{\temp@font #3}}%
     \fi     
}
% user interface
\def\scaletowidth{\@scaletofit[h]}
\def\scaletoheight{\@scaletofit[v]}
%</package>
%    \end{macrocode}
% \Finale
\endinput
%
%% \CharacterTable
%%  {Upper-case    \A\B\C\D\E\F\G\H\I\J\K\L\M\N\O\P\Q\R\S\T\U\V\W\X\Y\Z
%%   Lower-case    \a\b\c\d\e\f\g\h\i\j\k\l\m\n\o\p\q\r\s\t\u\v\w\x\y\z
%%   Digits        \0\1\2\3\4\5\6\7\8\9
%%   Exclamation   \!     Double quote  \"     Hash (number) \#
%%   Dollar        \$     Percent       \%     Ampersand     \&
%%   Acute accent  \'     Left paren    \(     Right paren   \)
%%   Asterisk      \*     Plus          \+     Comma         \,
%%   Minus         \-     Point         \.     Solidus       \/
%%   Colon         \:     Semicolon     \;     Less than     \<
%%   Equals        \=     Greater than  \>     Question mark \?
%%   Commercial at \@     Left bracket  \[     Backslash     \\
%%   Right bracket \]     Circumflex    \^     Underscore    \_
%%   Grave accent  \`     Left brace    \{     Vertical bar  \|
%%   Right brace   \}     Tilde         \~}

