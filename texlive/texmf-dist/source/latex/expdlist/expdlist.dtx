\def\filename{EXPDLIST}
\def\filedate{22.09.1999}
\def\fileversion{V 2.4}
\def\docdate {22.09.99}
% \iffalse
% !!! Bei Datumsaenderung: Auch "version", "date", "infdate" weiter
% !!!                     unten aendern.
% \fi
%% \CheckSum{214}
%% \CharacterTable
%%  {Upper-case    \A\B\C\D\E\F\G\H\I\J\K\L\M\N\O\P\Q\R\S\T\U\V\W\X\Y\Z
%%   Lower-case    \a\b\c\d\e\f\g\h\i\j\k\l\m\n\o\p\q\r\s\t\u\v\w\x\y\z
%%   Digits        \0\1\2\3\4\5\6\7\8\9
%%   Exclamation   \!     Double quote  \"     Hash (number) \#
%%   Dollar        \$     Percent       \%     Ampersand     \&
%%   Acute accent  \'     Left paren    \(     Right paren   \)
%%   Asterisk      \*     Plus          \+     Comma         \,
%%   Minus         \-     Point         \.     Solidus       \/
%%   Colon         \:     Semicolon     \;     Less than     \<
%%   Equals        \=     Greater than  \>     Question mark \?
%%   Commercial at \@     Left bracket  \[     Backslash     \\
%%   Right bracket \]     Circumflex    \^     Underscore    \_
%%   Grave accent  \`     Left brace    \{     Vertical bar  \|
%%   Right brace   \}     Tilde         \~}
%%
% \iffalse
% !!! Bei Anschriftaenderung: Auch die Anschrift vor
% !!!                         "\StopEventually{}" aendern
% @stylefile{Ltall
% shortpackagename = {expdlist},
% longpackagename  = {expdlist},
% baseformats      = {LaTeX2e $\langle$1997/12/01$\rangle$},
% version          = {2.4},
% date             = {22.09.1999},
% author           = {R. Huelse, W. Kaspar,
%                    kaspar@uni-muenster.de,
%                    Westf"alische Wilhelms-Universit"at M"unster,
%                    Zentrum f"ur Informationsverarbeitung,
%                    R"ontgenstra"se 9-13,
%                    48149 M"unster,
%                    02\,51/83-3\,16\,73},

% abstract         = {The expanded \texttt{description} environment
%                     will not replace the \LaTeX-\texttt{description}
%                     environment, but on request you will have some
%                     additional features. It supports an easy
%                     possibility of changing the left margin.
%                     Also there is with \verb+\listpart+ a new
%                     command available which is valid in all
%                     \texttt{list} environments. It gives the
%                     possibility to break a list for a comment
%                     without touching any counters.},
% support          = {yes},
% infauthor        = {\author},
% infdate          = {22.09.1999},
% comments         = {},
% requirements     = {},
% incompatibilities = {}}
% \fi
%
% \title{\bfseries \texttt{EXPDLIST}\thanks{This file has version number
%                            \fileversion, last revised \filedate.
%                            The documentation has been produced with
%                            Frank Mittelbach's \texttt{DOC.STY} (v1.7k).
%                            There is also a german documentation named
%                            \texttt{EXPDLISG.DRV}.}
%        -- an Expanded \texttt{description} Environment}
% \author{Rainer H\"ulse and Wolfgang Kaspar\\[2mm]
%         University of M\"unster (Germany)\\
%         Computing Center\\[2mm]
%          Internet:  $\langle$\texttt{kaspar@uni-muenster.de}$\rangle$}
% \date{\docdate}
%
% \maketitle
%
% \def\ttbackslash{\texttt{\symbol{92}}}        ^^A typewriter \
% \def\ttlbrace{\texttt{\symbol{123}}}          ^^A typewriter {
% \def\ttrbrace{\texttt{\symbol{125}}}          ^^A typewriter }
%
% \begin{abstract}
% \noindent
% The expanded \texttt{description} environment will not replace the
% \LaTeX-\texttt{description} environment, but on request you will have some
% additional features. It supports an easy possibility of changing the
% left margin. Also there is with \verb+\listpart+
% a new command available which is valid in all \texttt{list} environments.
% It gives the possibility to break a list for a comment without touching
% any counters.
%
% The required \texttt{STY}-file is \texttt{EXPDLIST} and
% will be enclosed in the \LaTeX-file as following:
% \begin{quote}
% \verb+\usepackage{expdlist}+
% \end{quote}
% \end{abstract}
%
% \section{The Expanded \texttt{description} Environment}
% The expanded \texttt{description} environment supports an easy
% possibility of changing the left margin in a \texttt{desciption list}.
% The text of the item begins at the left margin, either behind the label
% or in the following line.
% Another declaration eliminates the vertical space which is set by the
% \LaTeX-\texttt{STY}s.
% As well you can affect the appearance of the label.
% The syntax of the expanded \texttt{description} environment is:
% \begin{quote}
% \verb+\begin{description}[+\textit{declarations}\texttt{]}\\
% $\vdots$\\
% \verb+\end{description}+
% \end{quote}
% Without the optional \textit{declarations} this environment is equal to
% the original \LaTeX-\texttt{description} environment.
%
% \newpage
% \noindent
% The following declarations fix the left margin of the item:
% \begin{description}[\setlabelstyle{\ttfamily} \setleftmargin{3cm} \breaklabel
%                     \compact]
% \item[\ttbackslash setleftmargin\ttlbrace \textnormal{\textit{size}}\ttrbrace]
%                         gives the amount of horizontal
% \SpecialUsageIndex{\setleftmargin}
%                         space to be reserved
%                         for the left margin of the item,
%                         and defaults to the value of the original
%                         \LaTeX-\texttt{description} list if not entered.
% \item[\ttbackslash setlabelphantom\ttlbrace \textnormal{\textit{text}}\ttrbrace]
%                         calculates the left margin by the width of
% \SpecialUsageIndex{\setlabelphantom}
%                         \textit{text} and by the value of \verb+\labelsep+.
%                         The setting of \verb+\setlabelstyle+ is taken
%                         into account.
% \listpart{If you set \texttt{\ttbackslash setlabelphantom}
%                         as well as \texttt{\ttbackslash setleftmargin}, the
%                         horizontal space with the width defined by
%                         \texttt{\ttbackslash setlabelphantom} will be
%                         reserved.}
% \listpart{There are some other declarations affecting the layout of the
%           expanded \texttt{description} list:}
% \item[\ttbackslash breaklabel]
%                         causes the definition description to start on
% \SpecialUsageIndex{\breaklabel}
%                         the line following the label if the width of
%                         the label exceeds the width of the left
%                         margin. The default is to
%                         begin the description
%                         on the same line after the label.
% \item[\ttbackslash compact]
%                         indicates that items should not be
% \SpecialUsageIndex{\compact}
%                         separated from each other by vertical white
%                         space.
% \item[\ttbackslash setlabelstyle\ttlbrace \textnormal{\textit{typestyle}}\ttrbrace]
%                         identifies the style to be
% \SpecialUsageIndex{\setlabelstyle}
%                         used for labels, e.\,g. \verb+\bfseries+,
%                         \verb+\itshape+, \verb+\slshape+ or \verb+\sffamily+
%                         as well as \verb+\small+, \verb+\large+, etc.
%                         The default is \verb+\bfseries+ and \verb+\normalsize+.
% \end{description}
%
% \noindent
% The following examples demonstrate some features of the expanded
% \linebreak \texttt{description} environment.\\[4ex]
% The first example shows it without optional parameters being equal
% to the original \LaTeX\ environment.
% The command used is:\\[-4ex]
% \begin{center}
% \verb+\begin{description}+
% \end{center}
% \begin{quote}
% \begin{description}
% \item[First label] The first label is a normalsized label.
% \item[Here is a very long label] This is the text corresponding to the
%                                  very long label.
% \item[3rd] The 3rd label is a very short one.
% \item This item has no label and was produced by
%                                  \verb+\item+ \textit{text}.
% \end{description}
% \end{quote}
%
% \vspace{4ex}
% \noindent
% In the second example optional parameters are set with the following
% command:\\[-4ex]
% \begin{center}
% \verb+\begin{description}[\breaklabel\setleftmargin{80pt}+\\
% \verb+\setlabelstyle{\itshape}]+
% \end{center}
% \begin{quote}
% \begin{description}[\breaklabel\setleftmargin{80pt}\setlabelstyle{\itshape}]
% \item[First label] The first label is a normalsized label.
% \item[Here is a very long label] This is the text corresponding to the
%                                  very long label.
% \item[3rd] The 3rd label is a very short one.
% \item This item has no label and was produced
%                                  by \verb+\item+ \textit{text}.
% \end{description}
% \end{quote}
%
% \vspace{4ex}
% \noindent
% The last example shows the command with other optional parameters
% and their effects:\\[-4ex]
% \begin{center}
% \verb+\begin{description}[\compact\setlabelphantom{First label}]+
% \end{center}
% \begin{quote}
% \begin{description}[\compact\setlabelphantom{First label}]
% \item[First label] The first label is a normalsized label.
% \item[Here is a very long label] This is the text corresponding to the
%                                  very long label.
% \item[3rd] The 3rd label is a very short one.
% \item This item has no label and was produced by
%                                  \verb+\item+ \textit{text}.
% \end{description}
% \end{quote}
%
% \section{The \texttt{\char92 listpart} command}
% In the \texttt{EXPDLIST} style there are two new \LaTeX\ commands:
% \begin{description}[\setlabelstyle{\ttfamily} \setleftmargin{3cm} \breaklabel
%                     \compact]
% \item[\ttbackslash listpart\ttlbrace \textnormal{\textit{text}}\ttrbrace]
%                         identifies a comment or explanation that
% \SpecialUsageIndex{\listpart}
%                         applies to a part of a list.
%                         It can be placed anywhere
%                         within any \texttt{list} environment,
%                         immediately preceeding
%                         those items to which it applies.
%                         The width of \textit{text} depends on the width
%                         of the preceeding list. Therefore you are able
%                         to continue with the next item without
%                         closing and re-opening the list. The numbering
%                         of the \texttt{enumerate} environment is preserved.
% \item[\ttbackslash listpartsep]
%                         is the vertical space between the item and
% \SpecialUsageIndex{\listpartsep}
%                         the comment produced by
%                         \texttt{\ttbackslash listpart}.
%                         It defaults to \texttt{1ex}.
% \end{description}
% The following example demonstrates that you can use \verb+\listpart+
% also in multi-clause \texttt{list} environments:
% \begin{quote}
% \begin{itemize}
% \item You can use \verb+\listpart+ in multi-clause environments.
% \listpart{This is a \texttt{listpart} which puts in some text to
%           interrupt the list.}
% \item You can use \verb+\listpart+ in
% \begin{enumerate}
% \item{\texttt{itemize}-lists}
% \item{\texttt{enumerate}-lists}
% \listpart{This is a \texttt{listpart} which puts in some text to
%           interrupt the list.}
% \item{\texttt{description}-lists:}
% \begin{description}[\setleftmargin{60pt}]
% \item[1st Label] Description A
% \listpart{This is a \texttt{listpart} which puts in some text to
%           interrupt the list.}
% \item[2nd Label] Description B
% \end{description}
% \end{enumerate}
% \end{itemize}
% \end{quote}
%
%\StopEventually{}
%
% \newpage
% \changes{V 1.0}{02.03.90}{First published Version}
% \changes{V 2.0}{17.05.90}{Documentation with the \texttt{DOC.STY}}
% \section{The description of the \texttt{EXPDLIST.STY}-file}
% \subsection{The beginning}
% Here is a description of the macros used in the \texttt{EXPDLIST.STY}.
% We started by defining the current version and date of this file and
% documentation:
%\iffalse
%<*style>
%\fi
%    \begin{macrocode}
\typeout{Document Substyle `EXPDLIST'.
         Released \filedate \space (\fileversion)}
\typeout{English Documentation \space \docdate}
%    \end{macrocode}
% \subsection{The optional arguments}
% \begin{macro}{\compact}
% The first implemented macro is \verb+\compact+. Normally two items
% are separated by a blank line. This blank line is defined in \LaTeX\
% by \verb+\itemsep+ \texttt{+} \verb+\parsep+. To remove this blank line
% we defined
%    \begin{macrocode}
\def\compact%
   {\setlength{\itemsep}{-\parsep}}
%    \end{macrocode}
% \end{macro}
% \par
% \begin{macro}{\setleftmargin}
% \begin{macro}{\setlabelsize}
% To define \verb+\setleftmargin+ we assigned the new width
% to \verb+\leftmargin+:
%    \begin{macrocode}
\def\setleftmargin%
   #1%
   {\setlength{\leftmargin}{#1}}
%    \end{macrocode}
% The \texttt{\ttbackslash setleftmargin} command was named
% \texttt{\ttbackslash setlabelsize} in older versions. To be compatible with
% these versions the old command is also defined:
%    \begin{macrocode}
\let\setlabelsize = \setleftmargin
%    \end{macrocode}
% \end{macro}
% \end{macro}
% \changes{V 2.0}{17.05.90}{Change of \texttt{\ttbackslash setleftmargin}}
% \par
% \begin{macro}{\setlabelphantom}
% The \verb+\setlabelphantom+ command reserves the width of the argument
% as horizontal space for the label. We have to put a \verb+\hfil+ into
% \verb+\@tempboxa+ to avoid an \texttt{underful hbox} message because the
% box is wider than the argument by the value of \verb+\labelsep+.
% The width is stored in  \verb+\setleftmargin+.
%    \begin{macrocode}
\def\setlabelphantom%
   #1%
   {\def\set@labelphantom%
     {\setbox\@tempboxa=\hbox spread \labelsep {\@labelstyle #1\hfil}%
      \setleftmargin{\wd\@tempboxa}%
     }%
   }
\def\set@labelphantom{}
%    \end{macrocode}
% \end{macro}
% \changes{V 1.2}{09.05.90}{Change of \texttt{\ttbackslash hfil}}
% \par
% \begin{macro}{\setlabelstyle}
% The \verb+\setlabelstyle+ identifies the style to be used for labels.
% In \verb+\@labelstyle+ the default \verb+\bfseries+ is stored.
%    \begin{macrocode}
\def\@labelstyle%
   {\bfseries}
\def\setlabelstyle%
   #1%
   {\def\@labelstyle{#1}}
%    \end{macrocode}
% \end{macro}
% \par
% \begin{macro}{\breaklabel}
% To let work \verb+\breaklabel+ correctly as described before we need
% a rule with no dimension in the \verb+\item+ definition of \LaTeX.
% This rule is defined here:
%    \begin{macrocode}
\def\breaklabel%
   {\def\@breaklabel%
       {\rule{0mm}{0mm}%
        \\%
       }%
   }%
\def\@breaklabel%
   {}
%    \end{macrocode}
% \changes{V 1.1}{26.03.90}{Change of \texttt{\ttbackslash breaklabel}}
% The changed \verb+\item+ definition follows a little later.
% \end{macro}
% \subsection{The main macro}
% \begin{macro}{\description}
% \begin{macro}{\enddescription}
% Now we can begin with the new \verb+\description+ definition (which is
% in \LaTeX\ the same as \verb+\begin{description}+).
% First we had to rename \verb+\description+ to
% \verb+\@orgdlist+. It will be executed if no optional argument is set:
% \par
%    \begin{macrocode}
\let\@orgdlist\description
%    \end{macrocode}
% \changes{V 2.1}{13.08.92}{Change of \texttt{\ttbackslash @orgdlist}}
% We must look
% if there is an optional argument. If there is an optional
% argument the macro \verb+\@expdlist+ (our new macro) will be
% executed. Otherwise the original \LaTeX-macro will be executed which
% we have renamed to \verb+\@orgdlist+:
%    \begin{macrocode}
\def\description%
   {\@ifnextchar[%
       {\@expdlist}%
       {\@orgdlist}%
   }
\let\enddescription\endlist
%    \end{macrocode}
% \end{macro} ^^A of \enddescription
% \end{macro} ^^A of \description
% \par
% We had to rename \verb+\description+ to
% \verb+\@orgdlist+. It will be executed if no optional argument is set:
% \par
% If you have set any optional argument, the \verb+\@expdlist+
% definition will be executed.
%    \begin{macrocode}
\def\@expdlistlabel#1%
    {\@labelstyle
     #1%
     \hfil%
    }
\def\@expdlist[#1]%
    {\list{}%
      {\def\@breaklabel{}%
       \def\set@labelphantom{}%
       \def\@labelstyle{\bfseries}%
       #1%
       \set@labelphantom%
       \setlength{\labelwidth}{\leftmargin}%
       \addtolength{\labelwidth}{-\labelsep}%
       \let\makelabel\@expdlistlabel%
      }%
    }
%    \end{macrocode}
% \changes{V 1.1}{26.03.90}{Change of \texttt{\ttbackslash breaklabel}}
% \changes{V 1.2}{09.05.90}{Change of \texttt{\ttbackslash hfil}}
% \subsection{\texttt{\ttbackslash listpart} and
%                 \texttt{\ttbackslash listpartsep}}
% \begin{macro}{\listpartsep}
% Another feature of the \texttt{EXPDLIST.STY} is \verb+\listpart+. To
% adjust the vertical space between the item and the comment produced
% by \verb+\listpart+ we had to define a new measure named
% \verb+\listpartsep+.
%    \begin{macrocode}
\newlength{\listpartsep}
\listpartsep = 1ex
%    \end{macrocode}
% \end{macro}
% \begin{macro}{\listpart}
% Now we could define \verb+\listpart+ as a long definition, because
% its value can go over more than one paragraph. It is an item without
% label. So the text begins at the point where the label would begin.
% The width of the text is \verb+\linewidth+ \texttt{+} \verb+\rightmargin+
% \texttt{+} \verb+\leftmargin+. This value is registered in
% \verb+\@tempskipa+:
%    \begin{macrocode}
\long\def\listpart%
   #1%
   {\vspace{\listpartsep}%
    \item[]\hspace*{-\leftmargin}%
    \@tempskipa=\linewidth%
    \addtolength{\@tempskipa}{\rightmargin}%
    \addtolength{\@tempskipa}{\leftmargin}%
    \parbox{\@tempskipa}{#1}%
    \vspace{\listpartsep}%
   }
%    \end{macrocode}
% \end{macro}
% \iffalse    This is a METACOMMENT
%    *****************************************************************
%    *                                                               *
%    *                Original definition of \item.                  *
%    *                This is an excerpt from LATEX.TEX.             *
%    *                The marked lines are changed.                  *
%    *                                                               *
%    *****************************************************************
% \fi
% \subsection{The redefinition of \texttt{\ttbackslash \@item}}
% \begin{macro}{\@item}
% To let work \verb+\breaklabel+ correctly we had to redefine the
% original \LaTeX\ de\-fi\-ni\-tion of \verb+\@item+ in a few lines
% (see \texttt{RUM Change} marks). We had to
% define \verb+\set@break+ globally, because it is set within a
% \verb+\hbox+, but used outside.
% Depending on the width of the label text \verb+\set@break+ is set to
% \verb+\@breaklabel+ or to nothing.
% At the end of the \verb+\@item+ macro \texttt{\ttbackslash set\@break}
%  is called after the label is set.
% \changes{V 1.1}{26.03.90}{Change of \texttt{\ttbackslash breaklabel}}
%    \begin{macrocode}
\def\@item[#1]{%
  \if@noparitem
    \@donoparitem
  \else
    \if@inlabel
      \indent \par
    \fi
    \ifhmode
      \unskip\unskip \par
    \fi
    \if@newlist
      \if@nobreak
        \@nbitem
      \else
        \addpenalty\@beginparpenalty
        \addvspace\@topsep
        \addvspace{-\parskip}%
      \fi
    \else
      \addpenalty\@itempenalty
      \addvspace\itemsep
    \fi
    \global\@inlabeltrue
  \fi
  \everypar{%
    \@minipagefalse
    \global\@newlistfalse
    \if@inlabel
      \global\@inlabelfalse
      {\setbox\z@\lastbox
       \ifvoid\z@
         \kern-\itemindent
       \fi}%
      \box\@labels
      \penalty\z@
    \fi
    \if@nobreak
      \@nobreakfalse
      \clubpenalty \@M
    \else
      \clubpenalty \@clubpenalty
      \everypar{}%
    \fi}%
  \if@noitemarg
    \@noitemargfalse
    \if@nmbrlist
      \refstepcounter\@listctr
    \fi
  \fi
  \sbox\@tempboxa{\makelabel{#1}}%
  \global\setbox\@labels\hbox{%
    \unhbox\@labels
    \hskip \itemindent
    \hskip -\labelwidth
    \hskip -\labelsep
    \ifdim \wd\@tempboxa >\labelwidth
      \box\@tempboxa
      \gdef\set@break{\@breaklabel}     % RUM Change 2.3.90
    \else
      \hbox to\labelwidth {\unhbox\@tempboxa}%
      \gdef\set@break{}%                % RUM Change 2.3.90
    \fi
    \hskip \labelsep}%
    \set@break                          % RUM Change 2.3.90
  \ignorespaces}
%    \end{macrocode}
% \end{macro}
% \changes{V 2.1}{13.08.92}{\LaTeX\ Version~2.09 <25 March 1992>}
% \changes{V 2.3}{26.05.99}{\LaTeXe\ <1997/12/01>}
%
% \newpage
% \section{History of Changes}
% \begin{description}[\compact\setlabelphantom{V 8.8 (88.88.8888)M}]
% \item[V 1.0 (02.03.1990)] First published Version (H\"ulse and Kaspar)
% \item[V 1.1 (26.03.1990)] We had to change \texttt{\ttbackslash break} to
%                           \texttt{\ttbackslash breaklabel} and
%                           \texttt{\ttbackslash @break} to
%                           \texttt{\ttbackslash @breaklabel}
%                           because \texttt{\ttbackslash break}
%                           is a \TeX-primitive. This could cause
%                           difficulties with linebreaking. (H\"ulse)
% \item[V 1.2 (09.05.1990)] To be more flexible with the label, we changed
%                           \texttt{\ttbackslash hfill} to
%                           \texttt{\ttbackslash hfil}
%                            in \texttt{\ttbackslash @expdlistlabel} (H\"ulse)
% \item[V 2.0 (31.05.1990)] Documentation with the \texttt{DOC.STY} from Frank
%                           Mittelbach, University of Mainz,
%                           FRG.\newline
%                           \texttt{\ttbackslash setlabelsize}
%                            will be renamed to
%                           \texttt{\ttbackslash setleftmargin}
%                           (H\"ulse)
% \item[V 2.1 (13.08.1992)] \texttt{\ttbackslash @orgdlist} is defined by
%                           \texttt{\ttbackslash let}. Definition of
%                           \texttt{\ttbackslash item[]} out of \LaTeX\
%                           Version~2.09 $\langle$25 March 1992$\rangle$
%                           (Perske)
% \item[V 2.2 (23.09.1992)] Included Documentation Driver File and German
%                           Documentation File into this \texttt{.doc}-File.
%                           With Version 2.0 of \texttt{docstrip.tex} and the
%                           Batchfile \texttt{install.rum} you can extract
%                           them out of this \texttt{.doc}-File. (Perske)
% \item[V 2.3 (26.05.1999)] The Definition of \texttt{\ttbackslash item[]} is
%                           out of LaTeX2e $\langle$1997/12/01$\rangle$.
%                           The distribution is now supplied under the
%                           terms of the LPPL. The files were renamed
%                           to \texttt{expdlist.dtx}, \texttt{expdlist.ins}
%                           and \texttt{readme.txt}
%                           (Kaspar)
% \item[V 2.4 (22.09.1999)] Bugfix: percent added after
%                           \texttt{\ttbackslash gdef\ttbackslash
%                           set@break\ttlbrace\ttrbrace}.
%                           Thanks to Peter Karp, who drew my attention to that bug.
%                           \newline (Kaspar)
%
% \end{description}
%
% \PrintIndex
% \Finale
%\iffalse
%</style>
%\fi
%
%\iffalse
%<+driver>
%<+driver>% This is EXPDLIST.DRV 26.05.1999 Kr
%<+driver>%
%<+driver>\documentclass[a4paper]{article}
%<+driver>\usepackage{expdlist,doc}
%<+driver>
%<+driver>
%<+driver>\setcounter{IndexColumns}{3}
%<+driver>
%<+driver>\EnableCrossrefs
%<+driver>\CodelineIndex
%<+driver>
%<+driver>\pagestyle{headings}
%<+driver>
%<+driver>\begin{document}
%<+driver>  \DocInput{expdlist.dtx}
%<+driver>\end{document}
%<+driver>
%<+driver>\endinput
%\fi
%\iffalse
%<+german>\documentclass[a4paper]{article}
%<+german>\usepackage{german,expdlist,doc}
%<+german>\def\ttbackslash{\texttt{\symbol{92}}}        % typewriter \
%<+german>\def\ttlbrace{\texttt{\symbol{123}}}          % typewriter {
%<+german>\def\ttrbrace{\texttt{\symbol{125}}}          % typewriter }
%<+german>\def\thefootnote{\fnsymbol{footnote}}
%<+german>\begin{document}
%<+german>\pagestyle{headings}
%<+german>\title{\bfseries \texttt{EXPDLIST}%
%<+german>                \thanks{Derzeit g"ultige Version \fileversion\ vom
%<+german>                        \filedate.
%<+german>                        Mit Hilfe von Frank Mittelbachs \texttt{DOC.STY}
%<+german>                        (v1.7k) l"a"st sich aus dem \texttt{EXPDLIST.DTX}
%<+german>                        eine englische Dokumentation erstellen. Diese
%<+german>                        enth"alt zus"atzlich noch eine Beschreibung des
%<+german>                        Source-Codes.}
%<+german>       -- eine Erweiterung der \texttt{description}-Umgebung}
%<+german>\author{Rainer H\"ulse und Wolfgang Kaspar\\[2mm]
%<+german>Westf"alische Wilhelms-Universit"at M"unster\\
%<+german>Zentrum f"ur Informationsverarbeitung\\[2mm]
%<+german>Internet:  $\langle$\texttt{kaspar@uni-muenster.de}$\rangle$}
%<+german>\date{\docdate}
%<+german>\maketitle
%<+german>\noindent
%<+german>\begin{abstract}
%<+german>Die erweiterte \texttt{description}-Umgebung soll die
%<+german>\LaTeX-\texttt{description}-Umgebung
%<+german>nicht ersetzen, sondern bietet bei Bedarf
%<+german>einige zus"atzliche Merkmale. Sie unterst"utzt eine einfache
%<+german>M"oglichkeit, den linken Rand der Liste festzusetzen.  Daneben steht mit
%<+german>\verb+\listpart+ ein neuer, f"ur alle \texttt{list}-Umgebungen g"ultiger
%<+german>Befehl zur Verf"ugung. Dieses Kommando erm"oglicht es, eine Liste f"ur
%<+german>einen Kommentar zu unterbrechen, ohne irgendeinen Z"ahler dabei zu
%<+german>ver"andern.
%<+german>
%<+german>Der ben"otigte \texttt{STY}-File hei"st \texttt{EXPDLIST} und wird so in den
%<+german>\LaTeX-File eingebunden:
%<+german>\begin{quote}
%<+german>\verb+\usepackage{expdlist}+
%<+german>\end{quote}
%<+german>\end{abstract}
%<+german>\section{Die erweiterte \texttt{description}-Umgebung}
%<+german>Die erweiterte \texttt{description}-Umgebung unterst"utzt eine einfache
%<+german>M"oglichkeit, den linken Rand einer \texttt{description}-Liste zu
%<+german>ver"andern. Der Text des Erl"auter"-ungstextes beginnt am linken Rand,
%<+german>entweder hinter der Marke oder in der n"achsten Zeile. Eine andere
%<+german>Deklaration eliminiert den Freiraum zwischen den Listenpunkten, der von
%<+german>den \LaTeX-\texttt{STY}s gesetzt wird. Au"serdem kann noch das Aussehen der
%<+german>Marke beeinflu"st werden. Die Syntax der erweiterten
%<+german>\texttt{description}-Umgebung ist:
%<+german>\begin{quote}
%<+german>\verb+\begin{description}[+\textit{deklarationen}\texttt{]}\\
%<+german>$\vdots$\\
%<+german>\verb+\end{description}+
%<+german>\end{quote}
%<+german>Ohne die optionalen \texttt{[}\textit{deklarationen}\texttt{]}
%<+german>verh"alt sich diese
%<+german>Umgebung wie die originale \LaTeX\ \texttt{description}-Umgebung.
%<+german>
%<+german>\newpage
%<+german>\noindent
%<+german>Die folgenden Deklarationen legen den linken Rand des
%<+german>Erl"auterungstextes fest:
%<+german>\begin{description}[\setlabelstyle{\ttfamily} \setleftmargin{3cm} \breaklabel
%<+german>                    \compact]
%<+german>\item[\ttbackslash setleftmargin\ttlbrace \textnormal{\textit{l"ange}}\ttrbrace]
%<+german>                        gibt die L"ange des horizontalen Freiraums des
%<+german>                        linken Randes an.
%<+german>                        Die Voreinstellung entspricht dem Wert der
%<+german>                        originalen \texttt{description}-Liste in \LaTeX.
%<+german>\item[\ttbackslash setlabelphantom\ttlbrace \textnormal{\textit{text}}\ttrbrace]
%<+german>                        berechnet den linken Rand aus der L"ange von
%<+german>                        \textit{text} und aus dem Wert von
%<+german>                        \verb+\labelsep+. Dabei wird die Setzung von
%<+german>                        \verb+\setlabelstyle+ ber"ucksichtigt.
%<+german>\listpart{Wenn man sowohl \texttt{\ttbackslash setlabelphantom} als auch
%<+german>          \texttt{\ttbackslash setleftmargin} setzt, wird ein Freiraum der
%<+german>          L"ange, die durch \texttt{\ttbackslash setlabelphantom} definiert
%<+german>          ist, freigehalten.}
%<+german>\listpart{Es gibt noch einige andere Deklarationen, die das Layout der
%<+german>          erweiterten \texttt{description}-Liste beeinflussen:}
%<+german>\item[\ttbackslash breaklabel]
%<+german>                        l"a"st die Beschreibung in der n"achsten Zeile
%<+german>                        beginnen, wenn die L"ange der Marke die Breite
%<+german>                        des linken Randes "uberschreitet. In der
%<+german>                        Voreinstellung beginnt der Erl"auterungstext
%<+german>                        in gleichen Zeile, unmittelbar hinter der Marke.
%<+german>\item[\ttbackslash compact]
%<+german>                        zeigt an, da"s die Definitionen nicht durch
%<+german>                        Leerzeilen voneinander getrennt werden.
%<+german>\item[\ttbackslash setlabelstyle\ttlbrace \textnormal{\textit{schriftstil}}\ttrbrace]
%<+german>                        ist der Stil, der f"ur die Marken benutzt wird,
%<+german>                        z.~B. \verb+\bfseries+, \verb+\itshape+, \verb+\slshape+ oder
%<+german>                        \verb+\sffamily+ sowie \verb+\small+, \verb+\large+
%<+german>                        usw. Voreingestellt ist \verb+\bfseries+ und
%<+german>                        \verb+\normalsize+.
%<+german>\end{description}
%<+german>
%<+german>\noindent
%<+german>Die folgenden Beispiele zeigen einige Anwendungen der erweiterten
%<+german>\linebreak \texttt{description}-Umgebung.\\[4ex]
%<+german>Das erste Beispiel zeigt, da"s sie ohne optionalen Parametern der
%<+german>originalen \LaTeX-Umgebung entspricht. Die abgesetzte Markierung
%<+german>lautet:\\[-4ex]
%<+german>\begin{center}
%<+german>\verb+\begin{description}+
%<+german>\end{center}
%<+german>\begin{quote}
%<+german>\begin{description}
%<+german>\item[Erste Marke] Die erste Marke ist durchschnittlich lang.
%<+german>\item[Hier nun eine besonders lange Marke] Dies ist der Text, der zu der
%<+german>                                       besonders langen Marke geh"ort.
%<+german>\item[3.] Die 3. Marke ist sehr kurz.
%<+german>\item Dieser Eintrag hat keine Marke und wurde erzeugt mit
%<+german>                                       \verb+\item+ \textit{text}.
%<+german>\end{description}
%<+german>\end{quote}
%<+german>
%<+german>\vspace{4ex}
%<+german>\noindent
%<+german>Im zweiten Beispiel werden mit der folgenden Markierung optionale
%<+german>Parameter gesetzt:\\[-4ex]
%<+german>\begin{center}
%<+german>\verb+\begin{description}[\breaklabel\setleftmargin{80pt}+\\
%<+german>\verb+\setlabelstyle{\itshape}]+
%<+german>\end{center}
%<+german>\begin{quote}
%<+german>\begin{description}[\breaklabel\setleftmargin{80pt}\setlabelstyle{\itshape}]
%<+german>\item[Erste Marke] Die erste Marke ist durchschnittlich lang.
%<+german>\item[Hier nun eine besonders lange Marke] Dies ist der Text, der zu der
%<+german>                                       besonders langen Marke geh"ort.
%<+german>\item[3.] Die 3. Marke ist sehr kurz.
%<+german>\item Dieser Eintrag hat keine Marke und wurde erzeugt mit
%<+german>                                       \verb+\item+ \textit{text}.
%<+german>\end{description}
%<+german>\end{quote}
%<+german>
%<+german>\vspace{4ex}
%<+german>\noindent
%<+german>Das letze Beispiel zeigt die Markierung mit weiteren optionalen
%<+german>Parametern und ihre Wirkung:\\[-4ex]
%<+german>\begin{center}
%<+german>\verb+\begin{description}[\compact\setlabelphantom{Erste Marke}]+
%<+german>\end{center}
%<+german>\begin{quote}
%<+german>\begin{description}[\compact\setlabelphantom{Erste Marke}]
%<+german>\item[Erste Marke] Die erste Marke ist durchschnittlich lang.
%<+german>\item[Hier nun eine besonders lange Marke] Dies ist der Text, der zu der
%<+german>                                       besonders langen Marke geh"ort.
%<+german>\item[3.] Die 3. Marke ist sehr kurz.
%<+german>\item Dieser Eintrag hat keine Marke und wurde erzeugt mit
%<+german>                                       \verb+\item+ \textit{text}.
%<+german>\end{description}
%<+german>\end{quote}
%<+german>
%<+german>\section{Das \texttt{\ttbackslash listpart}-Kommando}
%<+german>Der \texttt{EXPDLIST}-Style enth"alt noch zwei weitere neue \LaTeX-Kommandos:
%<+german>\begin{description}[\setlabelstyle{\ttfamily} \setleftmargin{3cm} \breaklabel
%<+german>                    \compact]
%<+german>\item[\ttbackslash listpart\ttlbrace \textnormal{\textit{text}}\ttrbrace]
%<+german>                        ist ein Kommentar oder eine Erkl"arung, die als
%<+german>                        Teil einer Liste gilt. Er kann irgendwo in einer
%<+german>                        beliebigen
%<+german>                        \texttt{list}"=Umgebung stehen, direkt
%<+german>                        hinter dem Listeneintrag, zu dem er geh"ort. Die
%<+german>                        Zeilenbreite von \textit{text} richtet sich
%<+german>                        dabei nach der Breite der "ubergeordneten Liste.
%<+german>                        Man kann somit mit dem n"achsten Listenpunkt
%<+german>                        fortfahren, ohne die Liste beenden und
%<+german>                        an\-schlie"send wieder neu beginnen zu m"ussen.
%<+german>                        Die Numerierung in der \texttt{enumerate}-Umgebung
%<+german>                        bleibt dabei erhalten.
%<+german>\item[\ttbackslash listpartsep]
%<+german>                        ist der vertikale Abstand zwischen Listeneintrag
%<+german>                        und dem mit \texttt{\ttbackslash listpart} erzeugten
%<+german>                        Kommentar. Voreingestellt ist \linebreak \texttt{1ex}.
%<+german>\end{description}
%<+german>Das folgende Beispiel zeigt, da"s man \verb+\listpart+ auch in
%<+german>geschachtelten \texttt{list}-Umgebungen benutzen kann:
%<+german>\begin{quote}
%<+german>\begin{itemize}
%<+german>\item Man kann \verb+\listpart+ in einer geschachtelten Liste benutzen.
%<+german>\listpart{Dies ist ein \texttt{listpart}. Mit dieser Markierung wird
%<+german>          Text eingeschoben, der die Liste unterbricht.}
%<+german>\item Man kann \verb+\listpart+ benutzen in:
%<+german>\begin{enumerate}
%<+german>\item{\texttt{itemize}-Listen}
%<+german>\item{\texttt{enumerate}-Listen}
%<+german>\listpart{Dies ist ein \texttt{listpart}. Mit dieser Markierung wird
%<+german>          Text eingeschoben, der die Liste unterbricht.}
%<+german>\item{\texttt{description}-Listen:}
%<+german>\begin{description}[\setleftmargin{60pt}]
%<+german>\item[1. Marke] Beschreibung A
%<+german>\listpart{Dies ist ein \texttt{listpart}. Mit dieser Markierung wird
%<+german>          Text eingeschoben, der die Liste unterbricht.}
%<+german>\item[2. Marke] Beschreibung B
%<+german>\end{description}
%<+german>\end{enumerate}
%<+german>\end{itemize}
%<+german>\end{quote}
%<+german>\end{document}
%\fi
\endinput
