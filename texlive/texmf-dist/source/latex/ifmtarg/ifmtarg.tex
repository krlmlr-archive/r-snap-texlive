
%%%%%%%%%1%%%%%%%%%2%%%%%%%%%3%%%%%%%%%4%%%%%%%%%5
% Bundled source file for the IFMTARG package 
%--------1---------2---------3---------4---------5
% Please see the accompanying README for author,
% license, documentation and installation information
%%%%%%%%%1%%%%%%%%%2%%%%%%%%%3%%%%%%%%%4%%%%%%%%%5

\RequirePackage{filecontents}
\begin{filecontents}{ifmtarg.sty}
\NeedsTeXFormat{LaTeX2e}
\ProvidesPackage{ifmtarg}[2009/09/02 v1.2a check for an empty argument]
\begingroup
\catcode`\Q=3
\long\gdef\@ifmtarg#1{\@xifmtarg#1QQ\@secondoftwo\@firstoftwo\@nil}
\long\gdef\@xifmtarg#1#2Q#3#4#5\@nil{#4}
\long\gdef\@ifnotmtarg#1{\@xifmtarg#1QQ\@firstofone\@gobble\@nil}
\endgroup
\endinput
\end{filecontents}
%%%%%%%%%1%%%%%%%%%2%%%%%%%%%3%%%%%%%%%4%%%%%%%%%5


% Conditionally compile the documentation & generate the .ins file:
\providecommand\documentationCompile{Y}
\makeatletter
\if\documentationCompile N
  \expandafter\@@end
\fi


\begin{filecontents*}{ifmtarg.ins}
%&latex
\def\documentationCompile{N}
\input ifmtarg.tex
\csname@@end\endcsname
\end{filecontents*}




\makeatletter
\documentclass{article}

\usepackage[it,medium]{titlesec}

\usepackage{array,bigfoot,ifmtarg}
\usepackage[svgnames]{xcolor}
\usepackage[colorlinks,linktocpage]{hyperref}

\usepackage{geometry}
\geometry{margin=2cm}

\usepackage{gmdoc}
\usepackage{gmverb}
\dekclubs
\stanzaskip=\bigskipamount 
\CodeSpacesGrey

\usepackage{tocloft,varwidth}
\setcounter{tocdepth}{1}
\def\tocwidthA{0.55}
\def\tocwidthB{0.44}
\def\cftpartfont{\scshape}
\def\cftsecfont{\small}
\cftbeforesecskip=0pt
\def\cftpartleader{}
\def\cftpartafterpnum{\cftparfillskip}
\def\cftsecleader{}
\def\cftsecafterpnum{\cftparfillskip}

\DeclareRobustCommand\pkg{\textsf}
\def\pkgopt#1{\texttt{[#1]}}
\newcommand\chng[1]{\marginpar{\footnotesize\raggedright\textsf{#1}}}

\def\PDF{\textsc{pdf}}
\def\PS{\textsc{ps}}
\def\DVI{\textsc{dvi}}
\def\EPS{\textsc{eps}}

\usepackage{amsmath,listings}
\lstset{basicstyle=\ttfamily,columns=fullflexible}

\usepackage{changepage}
\usepackage[T1]{fontenc}
\usepackage{microtype}
\usepackage{lmodern}
\usepackage[sc,osf]{mathpazo}
\linespread{1.1}
\frenchspacing

\GetFileInfo{ifmtarg.sty}
\begin{document}
{\addtocontents{toc}{\protect\begin{varwidth}[t]{\tocwidthA\linewidth}}}

\title{The \pkg{ifmtarg} package}
\author{%
 Author: Donald Arseneau, and Peter Wilson, Herries Press\\
 Maintainer: Will Robertson\\
 \texttt{will dot robertson at latex-project dot org}%
}
\date{\fileversion \qquad \filedate}

\twocolumn[\maketitle]

\section{Documentation}

The \pkg{ifmtarg} package provides an if--then--else programmer's command \verb|\@ifmtarg| for testing for an empty macro argument.
`Empty' here refers to an argument of zero or more spaces only.\footnote{If you need a command to test for emptiness that doesn't include spaces, use the \verb|\tl_if_empty:nTF| conditional from the \pkg{expl3} package. \verb|\@ifmtarg| is equivalent to \pkg{expl3}'s \verb|\tl_if_blank:nTF|.}
The command is fully expandable; its syntax is:
\begin{quote}
\cmd{\@ifmtarg}\marg{arg}\marg{Code for arg empty}
\\\null\hfill
\marg{Code for arg not empty}
\end{quote}

A variation is provided, \cmd{\@ifnotmtarg}, to be used when only the `false' branch is required.
It is slightly more efficient when code is only required for a non-empty argument.
\begin{quote}
\cmd{\@ifnotmtarg}\marg{arg}\marg{Code for arg not empty}
\end{quote}

\section{Examples}

\begin{verbatim}
\newcommand{\isempty}{1]{%
  \@ifmtarg{#1}{YES}{NO}}
\end{verbatim}
\nobreak
\begin{tabular}{@{\hspace{1.8em}}>{\color{SeaGreen}}l@{$\quad\to\quad$}>{\ttfamily}l@{}}
  \verb+\isempty{}+    & YES \\
  \verb+\isempty{   }+ & YES \\
  \verb+\isempty{E}+   & NO  \\
  \verb+\isempty{ E }+ & NO
\end{tabular}

\begin{verbatim}
\newcommand{\isnotempty}[1]{%
   \@ifnotmtarg{#1}{YES}}
\end{verbatim}
\nobreak
\begin{tabular}{@{\hspace{1.8em}}>{\color{SeaGreen}}l@{$\quad\to\quad$}>{\ttfamily}l@{}}
  \verb+\isnotempty{}+    &     \\
  \verb+\isnotempty{   }+ &     \\
  \verb+\isnotempty{E}+   & YES \\
  \verb+\isnotempty{ E }+ & YES
\end{tabular}

\newpage
\section{History}

(\emph{Peter's comments follow.}) In an Email to me on 13 March 2000, Donald Arseneau pointed out some
failings with my original definition of the \cmd{\@ifmtarg} command:
%
\begin{verbatim}
\newcommand{\@ifmtarg}[3]{%
  \edef\@mtarg{\zap@space#1 \@empty}%  
  \ifx\@empty\@mtarg\relax #2\else #3\fi}
\end{verbatim}
%
It works most of the time correctly but Donald showed that it can 
give unexpected results 
under conditions that I had not thought of. He suggested the coding 
that now appears in the package above for the \cmd{\@ifmtarg} and 
\cmd{\@ifnotmtarg} commands. For a discussion on detecting empty arguments 
see \href{http://www.ctan.org/pub/tex-archive/info/aro-bend/answer.002}{\texttt{CTAN/info/aro-bend/answer.002}}


\section*{Change History}

\begin{itemize}
\item[v1.2a] New maintainer (Will Robertson)
\end{itemize}



\section*{Licence and copyright}
  
This work may be modified and/or distributed under the terms and
conditions of the \LaTeX\ Project Public License\footnote{\url{http://www.latex-project.org/lppl.txt}}, version~1.3c or later (your choice).
The current maintainer of this work is Will Robertson.

\bigskip
  \noindent
  Copyright Peter Wilson, 1996 \\
  Copyright Peter Wilson and Donald Arseneau, 2000

{\addtocontents{toc}{\protect\end{varwidth}\protect\hfill}}
{\addtocontents{toc}{\protect\begin{varwidth}[t]{\protect\tocwidthB\protect\linewidth}}}
\clearpage
\parindent=0pt

\section{Implementation}
\DocInput{ifmtarg.sty}

{\addtocontents{toc}{\protect\end{varwidth}}}

\end{document}
