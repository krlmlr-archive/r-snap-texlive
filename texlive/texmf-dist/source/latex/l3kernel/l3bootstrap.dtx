% \iffalse meta-comment
%
%% File: l3bootstrap.dtx Copyright (C) 2011-2014 The LaTeX3 project
%%
%% It may be distributed and/or modified under the conditions of the
%% LaTeX Project Public License (LPPL), either version 1.3c of this
%% license or (at your option) any later version.  The latest version
%% of this license is in the file
%%
%%    http://www.latex-project.org/lppl.txt
%%
%% This file is part of the "l3kernel bundle" (The Work in LPPL)
%% and all files in that bundle must be distributed together.
%%
%% The released version of this bundle is available from CTAN.
%%
%% -----------------------------------------------------------------------
%%
%% The development version of the bundle can be found at
%%
%%    http://www.latex-project.org/svnroot/experimental/trunk/
%%
%% for those people who are interested.
%%
%%%%%%%%%%%
%% NOTE: %%
%%%%%%%%%%%
%%
%%   Snapshots taken from the repository represent work in progress and may
%%   not work or may contain conflicting material!  We therefore ask
%%   people _not_ to put them into distributions, archives, etc. without
%%   prior consultation with the LaTeX3 Project.
%%
%% -----------------------------------------------------------------------
%
%<*driver|package>
% \begin{macro}{\GetIdInfo}
% \begin{macro}[aux]{\GetIdInfoAuxI, \GetIdInfoAuxII, \GetIdInfoAuxIII}
% \begin{macro}[int]{\GetIdInfoLog}
%   The idea here is to extract out the information needed from a standard
%   \textsc{svn} \texttt{Id} line, but without a line that will get
%   changed when the file is checked in. Hence the fact that there is
%   not a line which includes both a dollar sign and the \texttt{Id}
%   keyword together!
%
%   At this stage, no test has taken place for the \eTeX{} extensions, and
%   so using \tn{protected} could give an error. To avoid that, it is used
%   by csname: if it's not available, the bootstrap code will bail out at
%   the point of testing for the necessary primitives, while if it is available
%   then \cs{GetIdInfo} and so on will be properly protected. All of this is
%   then done in a group to avoid leaving \tn{protected} about as equivalent
%   to \tn{relax} if the extensions are unavailable.
%    \begin{macrocode}
\begingroup
  \csname protected\endcsname\gdef\GetIdInfo
    {%
      \begingroup
        \catcode 32 = 10 %
        \GetIdInfoAuxI
    }
%    \end{macrocode}
%   A first check for a completely empty \textsc{svn} field. If that is
%   not the case, there is a second case when a file created using
%   \texttt{svn cp} but has not been checked in. That leaves a special
%   marker \texttt{-1} version, which has no further data. Dealing correctly
%   with that is the reason for the space in the line to use
%   \cs{GetIdInforAuxII}.
%    \begin{macrocode}
  \csname protected\endcsname\gdef\GetIdInfoAuxI$#1$#2%
    {%
      \def\tempa{#1}%
      \def\tempb{Id}%
      \ifx\tempa\tempb
        \def\tempa
          {%
            \endgroup
            \def\ExplFileDate{0000/00/00}%
            \def\ExplFileDescription{#2}%
            \def\ExplFileName{[unknown name]}%
            \def\ExplFileExtension{[unknown extension]}%
            \def\ExplFileVersion{-1}%
          }%
      \else
        \def\tempa
          {%
            \endgroup
            \def\ExplFileDescription{#2}%
            \GetIdInfoAuxII$#1 $%
          }%
      \fi
      \tempa
      \GetIdInfoLog
    }
%    \end{macrocode}
%   Here, |#1| is |Id|, |#2| is the file name, |#3| is the extension,
%   |#4| is the version, |#5| is the check in date and |#6| is the check in
%   time and user, plus some trailing spaces. If |#4| is the marker |-1| value
%   then |#5| and |#6| will be empty.
%    \begin{macrocode}
  \csname protected\endcsname\gdef\GetIdInfoAuxII$#1 #2.#3 #4 #5 #6$%
    {%
      \def\ExplFileName{#2}%
      \def\ExplFileExtension{#3}%
      \def\ExplFileVersion{#4}%
      \begingroup
        \def\tempa{#4}%
        \def\tempb{-1}%
        \ifx\tempa\tempb
          \def\tempa
            {%
              \endgroup
              \def\ExplFileDate{0000/00/00}%
            }%
        \else
          \def\tempa
            {%
              \endgroup
              \GetIdInfoAuxIII$#5$%
            }%
        \fi
        \tempa
    }
%    \end{macrocode}
%   Convert an \textsc{svn}-style date into a \LaTeX{}-style one.
%    \begin{macrocode}
  \csname protected\endcsname\gdef\GetIdInfoAuxIII$#1-#2-#3$%
    {%
      \def\ExplFileDate{#1/#2/#3}%
    }
%    \end{macrocode}
%   During loading of \pkg{expl3}, module information is added to the log.
%   This function gets redefined once loading is complete.
%    \begin{macrocode}
  \csname protected\endcsname\gdef\GetIdInfoLog
    {%
      \immediate\write-1 %
        {%
          L3 Module:
            \ExplFileName\space
            \ExplFileDate\space v\ExplFileVersion\space \ExplFileDescription
        }
    }
\endgroup
%    \end{macrocode}
% \end{macro}
% \end{macro}
% \end{macro}
\GetIdInfo$Id: l3bootstrap.dtx 5354 2014-08-23 01:35:39Z bruno $
  {L3 Bootstrap code}
%</driver|package>
%<*driver>
\documentclass[full]{l3doc}
\begin{document}
  \DocInput{\jobname.dtx}
\end{document}
%</driver>
% \fi
%
% \title{^^A
%   The \pkg{l3bootstrap} package\\ Bootstrap code^^A
%   \thanks{This file describes v\ExplFileVersion,
%     last revised \ExplFileDate.}^^A
% }
%
% \author{^^A
%  The \LaTeX3 Project\thanks
%    {^^A
%      E-mail:
%        \href{mailto:latex-team@latex-project.org}
%          {latex-team@latex-project.org}^^A
%    }^^A
% }
%
% \date{Released \ExplFileDate}
%
% \maketitle
%
% \begin{documentation}
%
% \section{Using the \LaTeX3 modules}
%
% The modules documented in \file{source3} are designed to be used on top of
% \LaTeXe{} and are loaded all as one with the usual |\usepackage{expl3}| or
% |\RequirePackage{expl3}| instructions. These modules will also form the
% basis of the \LaTeX3 format, but work in this area is incomplete and not
% included in this documentation at present.
%
% As the modules use a coding syntax different from standard
% \LaTeXe{} it provides a few functions for setting it up.
%
% \begin{function}[updated = 2011-08-13]{\ExplSyntaxOn, \ExplSyntaxOff}
%   \begin{syntax}
%     \cs{ExplSyntaxOn} \meta{code} \cs{ExplSyntaxOff}
%   \end{syntax}
%   The \cs{ExplSyntaxOn} function switches to a category code
%   r\'{e}gime in which spaces are ignored and in which the colon (|:|)
%   and underscore (|_|) are treated as \enquote{letters}, thus allowing
%   access to the names of code functions and variables. Within this
%   environment, |~| is used to input a space. The \cs{ExplSyntaxOff}
%   reverts to the document category code r\'{e}gime.
% \end{function}
%
% \begin{function}{\ProvidesExplPackage, \ProvidesExplClass, \ProvidesExplFile}
%   \begin{syntax}
%     |\RequirePackage{expl3}| \\
%     \cs{ProvidesExplPackage} \Arg{package} \Arg{date} \Arg{version} \Arg{description}
%   \end{syntax}
%   These functions act broadly in the same way as the corresponding
%   \LaTeXe{} kernel
%   functions \tn{ProvidesPackage}, \tn{ProvidesClass} and
%   \tn{ProvidesFile}. However, they also implicitly switch
%   \cs{ExplSyntaxOn} for the remainder of the code with the file. At the
%   end of the file, \cs{ExplSyntaxOff} will be called to reverse this.
%   (This is the same concept as \LaTeXe{} provides in turning on
%   \tn{makeatletter} within package and class code.)
% \end{function}
%
% \begin{function}[updated = 2012-06-04]{\GetIdInfo}
%   \begin{syntax}
%     |\RequirePackage{l3bootstrap}|
%     \cs{GetIdInfo} |$Id:| \meta{SVN info field} |$| \Arg{description}
%   \end{syntax}
%   Extracts all information from a SVN field. Spaces are not
%   ignored in these fields. The information pieces are stored in
%   separate control sequences with \cs{ExplFileName} for the part of the
%   file name leading up to the period, \cs{ExplFileDate} for date,
%   \cs{ExplFileVersion} for version and \cs{ExplFileDescription} for the
%   description.
% \end{function}
%
% To summarize: Every single package using this syntax should identify
% itself using one of the above methods. Special care is taken so that
% every package or class file loaded with \tn{RequirePackage} or alike
% are loaded with usual \LaTeXe{} category codes and the \LaTeX3 category code
% scheme is reloaded when needed afterwards. See implementation for
% details. If you use the \cs{GetIdInfo} command you can use the
% information when loading a package with
% \begin{verbatim}
%   \ProvidesExplPackage{\ExplFileName}
%     {\ExplFileDate}{\ExplFileVersion}{\ExplFileDescription}
% \end{verbatim}
%
% \subsection{Internal functions and variables}
%
% \begin{variable}{\l__kernel_expl_bool}
%   A boolean which records the current code syntax status: \texttt{true} if
%   currently inside a code environment. This variable should only be
%   set by \cs{ExplSyntaxOn}/\cs{ExplSyntaxOff}.
% \end{variable}
%
% \end{documentation}
%
% \begin{implementation}
%
% \section{\pkg{l3bootstrap} implementation}
%
%    \begin{macrocode}
%<*initex|package>
%<@@=expl>
%    \end{macrocode}
%
% \subsection{Format-specific code}
%
% The very first thing to do is to bootstrap the \IniTeX{} system so
% that everything else will actually work. \TeX{} does not start with
% some pretty basic character codes set up.
%    \begin{macrocode}
%<*initex>
\catcode `\{ = 1 \relax
\catcode `\} = 2 \relax
\catcode `\# = 6 \relax
\catcode `\^ = 7 \relax
%</initex>
%    \end{macrocode}
%
% Tab characters should not show up in the code, but to be on the
% safe side.
%    \begin{macrocode}
%<*initex>
\catcode `\^^I = 10 \relax
%</initex>
%    \end{macrocode}
%
% For \LuaTeX{}, the extra primitives need to be enabled. This is not needed
% in package mode: plain \TeX{} and Con\TeX{}t have the primitives enabled
% while \LaTeXe{} has them with the prefix \texttt{luatex} (which is handled
% in \pkg{l3names}).
%    \begin{macrocode}
%<*initex>
\begingroup\expandafter\expandafter\expandafter\endgroup
\expandafter\ifx\csname directlua\endcsname\relax
\else
  \directlua{tex.enableprimitives ('', tex.extraprimitives ( ))}
\fi
%</initex>
%    \end{macrocode}
%
% \subsection{The \tn{pdfstrcmp} primitive with \XeTeX{} and \LuaTeX{}}
%
% Only \pdfTeX{} has a primitive called \tn{pdfstrcmp}. The \XeTeX{}
% version is just \tn{strcmp}, so there is some shuffling to do. As
% this is still a real primitive, using the \pdfTeX{} name is \enquote{safe}.
%    \begin{macrocode}
\begingroup\expandafter\expandafter\expandafter\endgroup
  \expandafter\ifx\csname pdfstrcmp\endcsname\relax
  \let\pdfstrcmp\strcmp
\fi
%    \end{macrocode}
%
%   If \LuaTeX{} is in use then no primitive \tn{pdfstrcmp} is available.
%   However, it can be emulated using some Lua code. In earlier versions of
%   the code, the \pkg{pdftexcmds} package was loaded to do this task. However,
%   that raises some issues in \enquote{generic} (it fails with Con\TeX{}t
%   MkIV), and also adds a hardly-needed dependency.  Note that \LuaTeX{}
%   prior to version $0.36$ is not supported by \pkg{expl3}: here that means
%   simply skipping the definition, which will then be picked up later. This
%   definition may need to be done twice: one \enquote{now} and once at the
%   start of every job. The latter can occur in package mode if for example a
%   custom format is being constructed. To achieve this while not requiring a
%   separate file, the Lua code is saved into a macro then used twice.
%   (In the long term, the Lua code here may be best moved to a separate
%   file.)
%
%   No macro definition is given just yet: that is left until \pkg{l3basics}.
%    \begin{macrocode}
\begingroup
  \expandafter\ifx\csname directlua\endcsname\relax
  \else
    \ifnum\luatexversion<36 %
    \else
      \catcode`\_=11 %
      \catcode`\:=11 %
      \def\tempa
        {%
          l3kernel = l3kernel or { }
          function l3kernel.strcmp (A, B)
            if A == B then
              tex.write ("0")
            elseif A < B then
              tex.write ("-1")
            else
              tex.write ("1")
            end
          end
        }
      \directlua{\tempa}
%    \end{macrocode}
%  A test for \LuaTeX{} in Ini\TeX{} mode.
%    \begin{macrocode}
      \ifnum 0%
        \directlua
          {%
            if status.ini_version then
              tex.write("1")
            end
          }>0 %
        \global\everyjob\expandafter
          {%
            \the\expandafter\everyjob
            \expandafter\luatex_directlua:D\expandafter{\tempa}%
          }
      \fi
    \fi
  \fi
\endgroup
%    \end{macrocode}
%
% \subsection{Engine requirements}
%
% The code currently requires functionality equivalent to \tn{pdfstrcmp}
% in addition to \eTeX{}. This is picked up by testing for the \tn{pdfstrcmp}
% primitive or a version of \LuaTeX{} capable of emulating it.
%    \begin{macrocode}
\begingroup
  \def\next{\endgroup}
  \def\ShortText{Required primitives not found}%
  \def\LongText%
    {%
      LaTeX3 requires the e-TeX primitives and \string\pdfstrcmp.\LineBreak
      \LineBreak
      These are available in engine versions:\LineBreak
      - pdfTeX 1.30\LineBreak
      - XeTeX 0.9994\LineBreak
      - LuaTeX 0.40\LineBreak
      or later.\LineBreak
      \LineBreak
    }%
  \expandafter\ifx\csname pdfstrcmp\endcsname\relax
    \expandafter\ifx\csname directlua\endcsname\relax
    \else
      \ifnum\luatexversion<36 %
        \newlinechar`\^^J\relax
%<*initex>
        \def\LineBreak{^^J}%
        \edef\next
          {%
            \errhelp
              {%
                \LongText
                For pdfTeX and XeTeX the '-etex' command-line switch is also
                needed.\LineBreak
                \LineBreak
                Format building will abort!\LineBreak
              }%
            \errmessage{\ShortText}%
            \endgroup
            \noexpand\end
          }%
%</initex>
%<*package>
        \def\LineBreak{\noexpand\MessageBreak}%
        \expandafter\ifx\csname PackageError\endcsname\relax
          \def\LineBreak{^^J}%
          \def\PackageError#1#2#3%
            {%
              \errhelp{#3}%
              \errmessage{#1 Error: #2!}
            }%
        \fi
        \edef\next
          {%
            \noexpand\PackageError{expl3}{\ShortText}
              {\LongText Loading of expl3 will abort!}%
            \endgroup
            \noexpand\endinput
          }%
%</package>
      \fi
    \fi
  \fi
\next
%    \end{macrocode}
%
% \subsection{Extending allocators}
%
% In format mode, allocating registers is handled by \pkg{l3alloc}. However, in
% package mode it's much safer to rely on more general code. For example,
% the ability to extend \TeX{}'s allocation routine to allow for \eTeX{} has
% been around since 1997 in the \pkg{etex} package.
%
% Loading this support is delayed until here as we are now sure that the
% \eTeX{} extensions and \tn{pdfstrcmp} or equivalent are available. Thus
% there is no danger of an \enquote{uncontrolled} error if the engine
% requirements are not met.
%
% For \LaTeX{} only, load \pkg{etex} as otherwise we are likely to get into
% trouble with registers. Some inserts are reserved also as these have to
% be from the standard pool. Note that \tn{reserveinserts} is \tn{outer}
% and so is accessed here by csname. In earlier versions, loading \pkg{etex}
% was done directly and so \tn{reserveinserts} appeared in the code: this
% then required a \tn{relax} after \tn{RequirePackage} to prevent an error
% with \enquote{unsafe} definitions as seen for example with \pkg{capoptions}.
% The optional loading here is done using a group and \tn{ifx} test as we
% are not quite in the position to have a single name for \tn{pdfstrcmp} just
% yet.
%    \begin{macrocode}
%<*package>
\begingroup
  \def\@tempa{LaTeX2e}
  \def\next{}
  \ifx\fmtname\@tempa
    \def\next
      {%
        \RequirePackage{etex}%
        \csname reserveinserts\endcsname{32}%
      }
  \fi
\expandafter\endgroup
\next
%</package>
%    \end{macrocode}
%
% \subsection{The \LaTeX3 code environment}
%
% The code environment is now set up.
%
% \begin{macro}{\ExplSyntaxOff}
%   Before changing any category codes, in package mode we need to save
%   the situation before loading. Note the set up here means that once applied
%   \cs{ExplSyntaxOff} will be a \enquote{do nothing} command until
%   \cs{ExplSyntaxOn} is used. For format mode, there is no need to save
%   category codes so that step is skipped.
%    \begin{macrocode}
\protected\def\ExplSyntaxOff{}
%<*package>
\protected\edef\ExplSyntaxOff
  {%
    \protected\def\ExplSyntaxOff{}%
    \catcode   9 = \the\catcode   9\relax
    \catcode  32 = \the\catcode  32\relax
    \catcode  34 = \the\catcode  34\relax
    \catcode  38 = \the\catcode  38\relax
    \catcode  58 = \the\catcode  58\relax
    \catcode  94 = \the\catcode  94\relax
    \catcode  95 = \the\catcode  95\relax
    \catcode 124 = \the\catcode 124\relax
    \catcode 126 = \the\catcode 126\relax
    \endlinechar = \the\endlinechar\relax
    \chardef\csname\detokenize{l__kernel_expl_bool}\endcsname = 0\relax
  }
%</package>
%    \end{macrocode}
% \end{macro}
%
% The code environment is now set up.
%    \begin{macrocode}
\catcode 9   = 9\relax
\catcode 32  = 9\relax
\catcode 34  = 12\relax
\catcode 58  = 11\relax
\catcode 94  = 7\relax
\catcode 95  = 11\relax
\catcode 124 = 12\relax
\catcode 126 = 10\relax
\endlinechar = 32\relax
%    \end{macrocode}
%
% \begin{variable}{\l__kernel_expl_bool}
%   The status for experimental code syntax: this is on at present.
%    \begin{macrocode}
\chardef\l__kernel_expl_bool = 1 ~
%    \end{macrocode}
%\end{variable}
%
% \begin{macro}{\ExplSyntaxOn}
%  The idea here is that multiple \cs{ExplSyntaxOn} calls are not
%  going to mess up category codes, and that multiple calls to
%  \cs{ExplSyntaxOff} are also not wasting time. Applying
%  \cs{ExplSyntaxOn} will alter the definition of \cs{ExplSyntaxOff}
%  and so in package mode this function should not be used until after
%  the end of the loading process!
%    \begin{macrocode}
\protected \def \ExplSyntaxOn
  {
    \bool_if:NF \l__kernel_expl_bool
      {
        \cs_set_protected_nopar:Npx \ExplSyntaxOff
          {
            \char_set_catcode:nn { 9 }   { \char_value_catcode:n { 9 } }
            \char_set_catcode:nn { 32 }  { \char_value_catcode:n { 32 } }
            \char_set_catcode:nn { 34 }  { \char_value_catcode:n { 34 } }
            \char_set_catcode:nn { 38 }  { \char_value_catcode:n { 38 } }
            \char_set_catcode:nn { 58 }  { \char_value_catcode:n { 58 } }
            \char_set_catcode:nn { 94 }  { \char_value_catcode:n { 94 } }
            \char_set_catcode:nn { 95 }  { \char_value_catcode:n { 95 } }
            \char_set_catcode:nn { 124 } { \char_value_catcode:n { 124 } }
            \char_set_catcode:nn { 126 } { \char_value_catcode:n { 126 } }
            \tex_endlinechar:D =
              \tex_the:D \tex_endlinechar:D \scan_stop:
            \bool_set_false:N \l__kernel_expl_bool
            \cs_set_protected_nopar:Npn \ExplSyntaxOff { }
          }
      }
    \char_set_catcode_ignore:n           { 9 }   % tab
    \char_set_catcode_ignore:n           { 32 }  % space
    \char_set_catcode_other:n            { 34 }  % double quote
    \char_set_catcode_alignment:n        { 38 }  % ampersand
    \char_set_catcode_letter:n           { 58 }  % colon
    \char_set_catcode_math_superscript:n { 94 }  % circumflex
    \char_set_catcode_letter:n           { 95 }  % underscore
    \char_set_catcode_other:n            { 124 } % pipe
    \char_set_catcode_space:n            { 126 } % tilde
    \tex_endlinechar:D = 32 \scan_stop:
    \bool_set_true:N \l__kernel_expl_bool
  }
%    \end{macrocode}
% \end{macro}
%
%    \begin{macrocode}
%</initex|package>
%    \end{macrocode}
%
% \end{implementation}
%
% \PrintIndex
