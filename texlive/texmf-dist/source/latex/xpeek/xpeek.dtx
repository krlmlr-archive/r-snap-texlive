% \iffalse meta-comment
% !TeX program = pdflatex
% !TeX encoding = utf-8
% !TeX spellcheck = en_US
%<*internal>
\begingroup
%</internal>
%<*install>
%%
%% This is file `l3docstrip.tex',
%% generated with the docstrip utility.
%%
%% The original source files were:
%%
%% l3docstrip.dtx  (with options: `program')
%% 
%% EXPERIMENTAL CODE
%% 
%% Do not distribute this file without also distributing the
%% source files specified above.
%% 
%% Do not distribute a modified version of this file.
%% 
%% File l3dosctrip.dtx Copyright (C) 2012 The LaTeX3 Project
%%
%% It may be distributed and/or modified under the conditions of the
%% LaTeX Project Public License (LPPL), either version 1.3c of this
%% license or (at your option) any later version.  The latest version
%% of this license is in the file
%%
%%    http://www.latex-project.org/lppl.txt
%%
%% This file is part of the "l3kernel bundle" (The Work in LPPL)
%% and all files in that bundle must be distributed together.
%%
%% The released version of this bundle is available from CTAN.
%%
%% -----------------------------------------------------------------------
%%
%% The development version of the bundle can be found at
%%
%%    http://www.latex-project.org/svnroot/experimental/trunk/
%%
%% for those people who are interested.
%%
%%%%%%%%%%%
%% NOTE: %%
%%%%%%%%%%%
%%
%%   Snapshots taken from the repository represent work in progress and may
%%   not work or may contain conflicting material!  We therefore ask
%%   people _not_ to put them into distributions, archives, etc. without
%%   prior consultation with the LaTeX Project Team.
%%
%% -----------------------------------------------------------------------
%%
\def\ExplFileDate{2012/06/08}
\def\ExplFileName{l3docstrip}
\def\ExplFileVersion{3787}
\input docstrip %
\def\checkOption<#1{%
  \ifcase
    \ifx*#10\else \ifx/#11\else
    \ifx+#12\else \ifx-#13\else
    \ifx<#14\else \ifx @#15\else 6\fi\fi\fi\fi\fi\fi\relax
  \expandafter\starOption\or
  \expandafter\slashOption\or
  \expandafter\plusOption\or
  \expandafter\minusOption\or
  \expandafter\verbOption\or
  \expandafter\moduleOption\or
  \expandafter\doOption\fi
  #1%
}
\def\moduleOption @@=#1>#2\endLine{%
  \maybeMsg{<@@=#1>}%
  \prepareActiveModule{#1}%
}
\begingroup
  \catcode`\_ = 12 %
  \long\gdef\prepareActiveModule#1{%
    \ifx\relax#1\relax
       \let\replaceModuleInLine\empty
    \else
      \def\replaceModuleInLine{%
        \replaceAllIn\inLine{__@@}{__#1}%
        \replaceAllIn\inLine{_@@}{__#1}%
        \replaceAllIn\inLine{@@}{__#1}%
      }%
    \fi
  }
\endgroup
\let\replaceModuleInLine\empty
\long\def\replaceAllIn#1#2#3{%
  \long\def\tempa##1##2#2{%
    ##2\qMark\replaceAllInAuxIII#3##1%
  }%
  \edef#1{\expandafter\replaceAllInAuxI#1\qMark#2\qStop}%
}
\def\replaceAllInAuxI{%
  \expandafter\replaceAllInAuxII\tempa\replaceAllInAuxI\empty
}
\long\def\replaceAllInAuxII#1\qMark#2{#1}
\long\def\replaceAllInAuxIII#1\qStop{}
\def\normalLine#1\endLine{%
  \maybeMsg{.}%
  \def\inLine{#1}%
  \replaceModuleInLine
  \let\do\putline@do
  \activefiles
}
\def\doOption#1>#2\endLine{%
  \maybeMsg{<#1 . >}%
  \Evaluate{#1}%
  \def\do##1##2##3{%
    \if1\Expr{##2}%
      \def\inLine{#2}%
      \replaceModuleInLine
      \StreamPut##1{\inLine}%
    \fi
  }%
  \activefiles
}
%% 
%%
%% End of file `l3docstrip.tex'.

\keepsilent
\askforoverwritefalse
\preamble
----------------------------------------------------------------

xpeek: Define commands that peek ahead in the input stream

Copyright © 2012 by Joel C. Salomon.

This work consists of the file  xpeek.dtx,
          and the derived files xpeek.ins,
                                xpeek.pdf, &
                                xpeek.sty.
It may be distributed and/or modified under the conditions of the
LaTeX Project Public License (LPPL), either version 1.3c of this
license or (at your option) any later version.  The latest version
of this license is at ‹http://www.latex-project.org/lppl.txt›.

This work is “maintained” (per LPPL maintenance status) by
Joel C. Salomon ‹joelcsalomon@gmail.com›.

----------------------------------------------------------------
\endpreamble
\declarepreamble\minimalpreamble
\endpreamble
\postamble
\endpostamble
\usedir{tex/latex/xpeek}
\generate{
  \file{\jobname.sty}{\from{\jobname.dtx}{package}}
}
\nopreamble\nopostamble
\generate{
  \file{\jobname.tests}{\from{\jobname.dtx}{tests}}
}
\generate{
  \file{\jobname-test.tex}{\from{\jobname.dtx}{testsuite,tests}}
}
%</install>
%<install>\endbatchfile
%<*internal>
\usepreamble\minimalpreamble
\usepostamble\defaultpostamble
\usedir{source/latex/xpeek}
\generate{
  \file{\jobname.ins}{\from{\jobname.dtx}{install}}
}
\endgroup
%</internal>
%<*driver>
\pdfobjcompresslevel=0
\RequirePackage[utf8]{inputenc}
\documentclass[full]{l3doc}
\usepackage{xpeek}
\usepackage{xspace}
\usepackage[all]{foreign}
\usepackage{qstest}

^^A Set up formatting options

^^A Format URLs surrounded by guillemets
\DeclareUrlCommand\url
  {\def\UrlLeft##1\UrlRight{\text{‹}\href{##1}{##1}\text{›}}}
\DeclareUrlCommand\email
  {\def\UrlLeft##1\UrlRight{\text{‹}\href{mailto:##1}{##1}\text{›}}}
\urlstyle{same}

^^A Make Unicode curly quotes (“”) behave correctly
\AtBeginDocument{
  \sfcode\csname\encodingdefault\string\textquotedblright\endcsname=0
}
\xspaceaddexceptions{”}

^^A Define indented verbatim environment
\DefineVerbatimEnvironment{IVerbatim}{Verbatim}%
  {xleftmargin=2.5\rmemwidth, gobble=4}
\newlength\rmemwidth
\AtBeginDocument{
  \setlength\rmemwidth{\fontdimen6\font}
}

^^A Enable line continuations in `syntax` environments
\NewDocumentCommand \fixsyntaxlines {}
  {\def\^^M{\unskip\space\ignorespaces}}

\begin{document}
  \DocInput{\jobname.dtx}
\end{document}
%</driver>
% \fi
%
% \changes{Version 0.1}{2012/08/09}{First useful version}
%
% \GetFileInfo{\jobname.sty}
%
% \DoNotIndex
%   {
%     \ProvidesExplPackage, \RequirePackage, \NewDocumentCommand,
%     \ExplSyntaxOn, \ExplSyntaxOff
%     \documentclass, \usepackage,
%     \begin, \end,
%     \xspace,
%     \bool_new:N, \bool_set_true:N, \bool_set_false:N, \bool_if:NTF,
%     \hbox_set:Nn, \box_wd:N, \l_tmpa_box, \l_tmpb_box,
%     \cs_new_protected:Npn,
%     \dim_use:N,
%     \peek_after:Nw,
%     \prg_new_protected_conditional:Npnn,
%     \prg_return_true:, \prg_return_false:,
%     \tl_new:N, \tl_clear:N, \tl_const:Nn, \tl_set:Nn,
%     \tl_put_right:Nn,
%     \tl_map_inline:Nn, \tl_map_break:,
%     \c_empty_tl,
%     \token_if_eq_charcode:NNT,
%     \xsp, \c_xsp_exceptions_tl,
%   }
%
% \NewDocumentCommand \releasenote {}
%   {This file describes release \fileversion, last revised \filedate.}
% \title{The \pkg{\jobname} package\thanks\releasenote}
%
% \urldef{\urlxthinsp}{\url}{http://tex.stackexchange.com/a/59542/2966}
% \urldef{\urlxflw}{\url}{http://tex.stackexchange.com/q/63568/2966}
% \urldef{\urlxign}{\url}{http://tex.stackexchange.com/q/63971/2966}
% \urldef{\urlltxl}
%   {\url}{http://thread.gmane.org/gmane.comp.tex.latex.latex3/2894}
% \NewDocumentCommand \acknote {}
%   {The original code is due to Enrico “egreg” Gregorio,
%    based on an answer he gave to a question of mine on
%    \href{http://tex.stackexchange.com}{\TeX.SX};
%    see \urlxthinsp.
%    Enrico, Joseph Wright, Bruno Le Floch, \& Clemens Niederberger
%    helped iron out the implementation;
%    Bruno also wrote the initial version of the documentation
%    and the near-final version of much of the code.
%    See \urlxflw, \urlxign, \& \urlltxl.}
% \author{Joel C. Salomon\\\email{joelcsalomon@gmail.com}\thanks\acknote}
% \date{Released \filedate}
%
% \maketitle
%
% \begin{abstract}
%   The \pkg{\jobname} package provides functions
%   for defining \& managing commands which,
%   like the familiar \cs{xspace},
%   “peek” ahead at what follows them in the input stream.
%   They compare this token against a stored list
%   and choose an action depending on whether there was a match.
% \end{abstract}
%
% \tableofcontents
%
% \begin{documentation}
%
% \section{Introduction}
%
% This package provides functions for defining \cs{xspace}-like commands.
% Such commands “peek” ahead into the input stream
% to see what follows their invocations;
% their behavior varies depending on what it is that comes next.
%
% Probably the most common \LaTeX\ command of this sort is \cs{textit}.
% It looks ahead at the token following its argument
% and decides whether to insert italic correction afterward.
% Table~\ref{tab:itcor} compares the use of \cs{textit}
% against using the font-switch command \cs{itshape}
% with \& without explicit italic correction.
% \begin{table}[hbp]\centering
%   \begin{tabular}{|lll|}
%   \hline
%     |{\itshape half}hearted| &
%      {\itshape half}hearted  &
%         (overlap)\\
%     |{\itshape half\/}hearted| &
%      {\itshape half\/}hearted  &
%         (correct)\\
%     |\textit{half}hearted| &
%       \textit{half}hearted &
%         (correctly including italic correction)\\
%     ^^A There’s a bug in the package `trace` which breaks suppression
%     ^^A of italic correction; see latex3/svn-mirror#96 for details.
%     ^^A As a temporary work-around, include `\nocorr` where needed.
%     |by {\itshape half}.| &
%      by {\itshape half}.  &
%         (correct)\\
%     |by {\itshape half\/}.| &
%      by {\itshape half\/}.  &
%         (too wide)\\
%     |by \textit{half}.| &
%      by \textit{half\nocorr}.  &
%         (correctly omitting italic correction)\\
%   \hline
%   \end{tabular}
%   \caption{Automatic italic correction with \cs{textit}}
%   \label{tab:itcor}
% \end{table}
% Notice how \cs{textit} seems to just “do the right thing”
% when it comes to inserting or omitting italic correction.
%
% It is, unfortunately, not hard to confuse \cs{textit}.
% As an example, consider the following code:
% \begin{IVerbatim}
%   \NewDocumentCommand \naif {} {\textit{na\"\i f}}
%   He was a true \naif.
% \end{IVerbatim}
% This yields
% \begin{quote}
%   \NewDocumentCommand \naif {} {\textit{na\"\i f\nocorr}}
%   He was a true \naif.
% \end{quote}
% So far, so good.
% But using this macro in middle of a sentence
% may produce results the naïve user does not expect.
% For example, try
% \begin{IVerbatim}
%   He was such a \naif he expected this example to work.
% \end{IVerbatim}
% yielding
% \begin{quote}
%   \NewDocumentCommand \naif {} {\textit{na\"\i f}}
%   He was such a \naif he expected this example to work.
% \end{quote}
% Oops.
% We forgot that command words eat the spaces that follow them.
%
% This is usually the point where \TeX perts explain
% about following command words with explicit spaces,
% \eg, writing “|such a \naif\ he |…”,
% or following such commands with an empty group,
% “|such a \naif{} he |…”, \eg.
%
% Others, more pragmatically-inclined,
% will mention the \pkg{xspace} package.
% This is the sort of use-case it was developed for, after all,
% so let’s try it:
% \begin{IVerbatim}
%   \NewDocumentCommand \naif {} {\textit{na\"\i f}\xspace}
%   He was such a \naif he expected this example to work.
% \end{IVerbatim}
% This produces
% \begin{quote}
%   \NewDocumentCommand \naif {} {\textit{na\"\i f}\xspace}
%   He was such a \naif he expected this example to work.
% \end{quote}
% Even “such a \foreign{naïf}” can get the right result sometimes.
%
% But you see, there’s a kinda hitch.^^A
% \footnote{Cue Malcolm Reynolds: “Don’t complicate things.”}
% Recall the first time we used the macro:
% \begin{IVerbatim}
%   He was a true \naif.
% \end{IVerbatim}
% Now it yields this:
% \begin{quote}
%   \NewDocumentCommand \naif {} {\textit{na\"\i f}\xspace}
%   He was a true \naif.
% \end{quote}
% Notice that ugly space between the \textit{f} and the period?
% That’s italic correction being added where it doesn’t belong.
%
% The trouble lies in how \LaTeX\ implements the \cs{textit} logic.
% The macro is clever enough to look ahead,
% seeking for short punctuation (\eg, the comma \& period)
% which should not be preceded by italic correction.
% What the macro finds first, though,
% is that invocation of \cs{xspace} we added.
% That’s not one of the “no italic correction” tokens,
% so the correction gets added and the sentence looks ugly.
%
% That’s where this package comes in.
%
% (As a side note, Phillip G. Ratcliffe’s \pkg{foreign} package,
%  bundled with both MiK\TeX\ and \TeX~Live
%  and available from CTAN at \url{http://ctan.org/pkg/foreign},
%  is an excellent tool for this sort of foreign-word macro
%  and was one of the inspirations for this package.)
%
% Commands generated via the tools the \pkg{xpeek} package provides
% can peek ahead in the input stream,
% like \cs{textit} and \cs{xspace},
% checking whether the next token is on their match-list.
% What \pkg{xpeek} adds is
% the ability to ignore certain tokens during that peek-ahead.
% These ignored tokens are not lost;
% the scan-ahead routine saves them
% and they are restored when the command executes.
%
% \section{Documentation}
%
% At this point in the development of \pkg{xpeek},
% only one set of follower-dependent functions is defined.
% Later versions will provide tools for defining such function sets.
%
% \subsection{Match- \& Ignore-Lists}
%
% \begin{variable}{\g_xpeek_matchlist_tl, \g_xpeek_ignorelist_tl}
%   \begin{syntax}
%     \cs{tl_gput_right:Nn} \cs{g_xpeek_\meta{list}_tl} \Arg{tokens}
%     \cs{tl_gremove_all:Nn} \cs{g_xpeek_\meta{list}_tl} \Arg{tokens}
%   \end{syntax}
%   These two token-lists are the heart of \pkg{xpeek}.
%
%   The \textit{match-list} is similar to \pkg{xspace}’s exceptions list
%   or \LaTeX’s \cs{nocorrlist} list for suppressing italic correction.
%
%   The functionality enabled by the \emph{ignore-list},
%   on the other hand,
%   is the \foreign{raison d’être} of this package.
%   For example,
%   if \LaTeX’s \cs{textit} had an ignore-list
%   and the token \cs{xspace} were in that list,
%   then the definition
%   \begin{IVerbatim}[gobble=6]
%     \NewDocumentCommand \naif {} {\textit{na\"\i f}\xspace}
%   \end{IVerbatim}
%   from the example above
%   would have worked properly:
%   \cs{textit} would have ignored the \cs{xspace},
%   looked past it \& seen the period,
%   and therefore omitted italic correction.
%   When that was done, \cs{xspace} would execute;
%   it too would see the period,
%   and therefore would not insert a space.
% \end{variable}
%
% \subsection{Peek-ahead}
%
% \begin{function}{\xpeek_after:nw}
%   \begin{syntax}
%     \cs{xpeek_after:nw} \Arg{code} \meta{token}
%   \end{syntax}
%   Similar to \pkg{expl3}’s \cs{peek_after:Nw},
%   this sets the test variable \cs{l_peek_token} to \meta{token}
%   then executes the \Arg{code}.
%   The \meta{token} remains in the input stream.
% \end{function}
%
% \subsection{Collecting Ignored Tokens}
%
% \begin{function}{\xpeek_collect_do:nn}
%   \begin{syntax}\fixsyntaxlines
%     \cs{xpeek_collect_do:nn} \Arg{ignore-list} \Arg{operation} \
%       \meta{the rest of the input-stream}
%   \end{syntax}
%   This collects all leading tokens in the input stream
%   which are in the ignore-list,
%   and passes them as an argument to the operation.
%   For example, the invocation
%   \begin{IVerbatim}[gobble=6]
%     \xpeek_collect_do:nn { abc } { \foo \bar } caada
%   \end{IVerbatim}
%   will collect the leading “\texttt{caa}” from the input stream
%   and pass those tokens as an argument to
%   the operation “|\foo \bar|”.
%   \Ie, the code above is equivalent to this:
%   \begin{IVerbatim}[gobble=6]
%     \foo \bar { caa } da
%   \end{IVerbatim}
%   (Note that it’s \cs{bar} \emph{within} the \Arg{operation}
%    that gets the collected tokens as an argument.)
% \end{function}
%
% \subsection{Searching Through Token-Lists}
%
% \begin{function}{\xpeek_if_in:NNTF}
%   \begin{syntax}\fixsyntaxlines
%     \cs{xpeek_if_in:NNTF} \cs{\meta{haystack}} \cs{\meta{needle}} \
%       \Arg{true-code} \Arg{false-code}
%   \end{syntax}
%   Among the variants of \cs{tl_if_in:Nn(TF)} defined in \pkg{expl3},
%   the form \cs{tl_if_in:NNTF} is missing.
%   But that’s the form needed in this package,
%   in constructions like this:
%   \begin{IVerbatim}[gobble=6, commandchars=+\[\]]
%     \xpeek_if_in:NNTF \g_xpeek_matchlist_tl \l_peek_token
%       +marg[true-code] +marg[false-code]
%   \end{IVerbatim}
%   So we define it ourselves.
% \end{function}
%
% \section{Example}
%
% As a demonstration,
% here’s an example of using the \pkg{xpeek} facilities
% to implement correct automatic italic correction \ala \cs{textit}.
%
% We set the match-list equal to \LaTeX’s default \cs{nocorrlist};
% as mentioned above, \cs{xspace} needs to be in the ignore-list:
% \begin{IVerbatim}
%   \tl_gset:Nn \g_xpeek_matchlist_tl  { , . }
%   \tl_gset:Nn \g_xpeek_ignorelist_tl { \xspace }
% \end{IVerbatim}
% (Note that this does not require \cs{xspace} to be defined;
%  it merely ensure that
%  the macro will work correctly
%  if it is defined.)
%
% Next, we declare the new italicization command \cs{xit}.
% It calls \cs{xpeek_collect_do:nn} to collect
% any instances of \cs{xspace} that might follow the invocation.
% These are saved, but ignored for the time being.
%
% It’s within the \Arg{code} argument to \cs{xpeek_collect_do:nn},
% between \cs{group_begin:} and \cs{group_end:},
% that the italicization happens.
% Before we return to the default font, though,
% the code checks whether \cs{l_peek_token} is in the ignore-list;
% it then decides whether to include italic correction.
%
% Recall next that the last command within the \Arg{code}
% is what will get the collected tokens as an argument.
% Since we want to keep \cs{xspace} if it’s there,
% we finish up the code with \cs{use:n},
% which simply puts its argument into the input stream.
%
% Here’s the code:
% \begin{IVerbatim}
%   \NewDocumentCommand \xit {m}
%     {
%       \xpeek_collect_do:nn \g_xpeek_ignorelist_tl
%         {
%           \group_begin:
%           \itshape #1
%           \xpeek_if_in:NNTF \g_xpeek_matchlist_tl \l_peek_token
%             { } { \/ }
%           \group_end:
%           \use:n
%         }
%     }
% \end{IVerbatim}
%
% Finally, define the command we tried to make work in the introduction,
% and see whether we’ve made it work:
% \begin{IVerbatim}
%   \NewDocumentCommand \naif {} {\xit{na\"\i f}\xspace}
%   He was a true \naif.
%   He was such a \naif he expected this example to work.
% \end{IVerbatim}
% The code above yields this result:
% \begin{quote}
%   \ExplSyntaxOn
%   \tl_gset:Nn \g_xpeek_matchlist_tl  { , . }
%   \tl_gset:Nn \g_xpeek_ignorelist_tl { \xspace }
%   \NewDocumentCommand \xit {m}
%     {
%       \xpeek_collect_do:nn \g_xpeek_ignorelist_tl
%         {
%           \group_begin:
%           \itshape #1
%           \xpeek_if_in:NNTF \g_xpeek_matchlist_tl \l_peek_token
%             { } { \/ }
%           \group_end:
%           \use:n
%         }
%     }
%   \ExplSyntaxOff
%   \NewDocumentCommand \naif {} {\xit{na\"\i f}\xspace}
%   He was a true \naif.
%   He was such a \naif he expected this example to work.
% \end{quote}
% And work it does.
%
% \end{documentation}
%
% \begin{implementation}
%
% \section{\pkg{\jobname} Implementation}
%
%    \begin{macrocode}
%<*package>
%    \end{macrocode}
%
%    \begin{macrocode}
%<@@=xpeek>
%    \end{macrocode}
%
%    \begin{macrocode}
\RequirePackage{expl3, xparse}
%    \end{macrocode}
%
%    \begin{macrocode}
\ProvidesExplPackage
  {xpeek} {2012/08/15} {0.2}
  {Define commands that peek ahead in the input stream}
%    \end{macrocode}
%
% \subsection{Match- \& Ignore-Lists}
%
% \begin{variable}{\g_xpeek_matchlist_tl, \g_xpeek_ignorelist_tl}
%   Define the lists.
%    \begin{macrocode}
\tl_new:N \g_xpeek_matchlist_tl
\tl_new:N \g_xpeek_ignorelist_tl
%    \end{macrocode}
% \end{variable}
%
% \subsection{Peek-ahead}
%
% \begin{macro}[internal]{\xpeek_after:nw}
% \begin{variable}[aux]{\l_@@_code_tl}
%   Define a function (AKA “token-list”) to hold the code passed in.
%    \begin{macrocode}
\tl_new:N \l_@@_code_tl
%    \end{macrocode}
%   Store the argument in \cs{l_@@_code_tl},
%   then defer to \cs{peek_after:Nw}
%    \begin{macrocode}
\cs_new_protected:Npn \xpeek_after:nw #1
  {
    \tl_set:Nn \l_@@_code_tl {#1}
    \peek_after:Nw \l_@@_code_tl
  }
%    \end{macrocode}
% \end{variable}
% \end{macro}
%
% \subsection{Collecting Ignored Tokens}
%
% \begin{macro}{\xpeek_collect_do:nn}
% \begin{macro}[aux]{\@@_collect_do:nnn}
% \begin{variable}[aux]{\l_@@_collected_tokens_tl}
%   \begin{arguments}
%     \item Tokens to collect from the input stream
%     \item The operation to be performed on the collected tokens
%     \item The next token on the input stream
%           (for \cs{@@_collect_do:nnn}).
%   \end{arguments}\noindent
%   Define a list to save collected tokens in.
%    \begin{macrocode}
\tl_new:N \l_@@_collected_tokens_tl
%    \end{macrocode}
%   Clear list and defer to auxiliary function.
%    \begin{macrocode}
\cs_new_protected:Npn \xpeek_collect_do:nn #1#2
  {
    \tl_clear:N \l_@@_collected_tokens_tl
    \@@_collect_do:nnn {#1} {#2} {}
  }
%    \end{macrocode}
%   Recursively consume tokens from the input stream
%   so long as they match the collect-list,
%   then apply the operation to the collected tokens.
%   The last matching token will not yet have been collected,
%   so pass it to the operation as well.
%   (If there were no matching tokens,
%    |#3| will be the empty argument
%    that was passed by \cs{xpeek_collect_do:nn}.)
%    \begin{macrocode}
\cs_new_protected:Npn \@@_collect_do:nnn #1#2#3
  {
    \xpeek_after:nw
      {
        \xpeek_if_in:NNTF #1 \l_peek_token
          {
            \tl_put_right:Nn \l_@@_collected_tokens_tl {#3}
            \@@_collect_do:nnn {#1} {#2}
          }
          {
            #2 { \l_@@_collected_tokens_tl #3 }
          }
      }
  }
%    \end{macrocode}
% \end{variable}
% \end{macro}
% \end{macro}
%
% \subsection{Searching Through Token-Lists}
%
% \begin{macro}{\xpeek_if_in:NNTF}
% \begin{variable}[aux]{\l_@@_bool}
%   \begin{arguments}
%     \item Token-list to search through (the “haystack”)
%     \item Token to search for (the “needle”)
%   \end{arguments}\noindent
%   Define an internal flag.
%    \begin{macrocode}
\bool_new:N \l_@@_bool
%    \end{macrocode}
%    Map the “haystack”, searching for the “needle”.
%    \begin{macrocode}
\prg_new_protected_conditional:Npnn \xpeek_if_in:NN #1#2 { TF }
  {
    \bool_set_false:N \l_@@_bool
    \tl_map_inline:Nn #1
      {
        \token_if_eq_charcode:NNT #2 ##1
          {
            \bool_set_true:N \l_@@_bool
            \tl_map_break:
          }
      }
    \bool_if:NTF \l_@@_bool
      { \prg_return_true: } { \prg_return_false: }
  }
%    \end{macrocode}
% \end{variable}
% \end{macro}
%
%    \begin{macrocode}
%</package>
%    \end{macrocode}
%
% \section{Test Suite}
%
% \changes{Version 0.2}{2012/08/15}{Added test-suite}
%
% ^^A Reset line numbers in code listings
% \makeatletter
% \c@CodelineNo 0
% \makeatother
%
% The test suite below is run automatically
% when this document is produced.
% It can also be run separately by executing
% \begin{IVerbatim}[commandchars=+\[\]]
%   latex +jobname-test
% \end{IVerbatim}
% at the command prompt.
%
% \input{\jobname.tests}
%
%    \begin{macrocode}
%<testsuite>\documentclass{article}
%<testsuite>\usepackage{xparse, expl3, xpeek, qstest}
%<testsuite>\begin{document}
%<*tests>
%    \end{macrocode}
%
% \subsection{Set-Up: Wrapping \cs{Expect}}
%
% \begin{macro}[internal]{\ExpectIdenticalWidths}
% Since the commands \pkg{xpeek} helps produce are not expandable,
% directly comparing their outputs is not feasible.
% Instead, typeset two versions into boxes
% and have \pkg{qstest} compare these boxes’ widths.
% (Thanks to Heiko Oberdiek for this technique;
%  see \url{http://tex.stackexchange.com/q/67192/2966}.)
%    \begin{macrocode}
\ExplSyntaxOn
\NewDocumentCommand \ExpectIdenticalWidths { m m }
  {
    \hbox_set:Nn \l_tmpa_box {#1}
    \hbox_set:Nn \l_tmpb_box {#2}
    \Expect
      * {\dim_use:N \box_wd:N \l_tmpa_box}
      * {\dim_use:N \box_wd:N \l_tmpb_box}
  }
\ExplSyntaxOff
%    \end{macrocode}
% \end{macro}
%
% \subsection{Emulating \cs{xspace}}
%
% Define a simple analogue to \cs{xspace}.
%    \begin{macrocode}
\ExplSyntaxOn
\tl_const:Nn \c_xsp_exceptions_tl { ,;:.!? }
\NewDocumentCommand \xsp {}
  {
    \xpeek_collect_do:nn \c_empty_tl
      {
        \xpeek_if_in:NNTF \c_xsp_exceptions_tl \l_peek_token
          { } { ~ }
      }
  }
\ExplSyntaxOff
%    \end{macrocode}
% Test \cs{xsp}, ensuring that it is space-factor–agnostic.
%    \begin{macrocode}
\begin{qstest}{Emulating \xspace}{xpeek}
  \ExpectIdenticalWidths{foo bar}{foo\xsp bar}
  \ExpectIdenticalWidths{foo. bar}{foo\xsp. bar}
  \ExpectIdenticalWidths{FOO. bar}{FOO\xsp. bar}
  \ExpectIdenticalWidths{foo. bar}{foo.\xsp bar}
  \ExpectIdenticalWidths{FOO. bar}{FOO.\xsp bar}
\end{qstest}
%    \end{macrocode}
%
%    \begin{macrocode}
%</tests>
%<testsuite>\end{document}
%    \end{macrocode}
% N.B\@. The stand-alone test-suite will not produce any output,
% only a log file.
%
% \end{implementation}
%
% \PrintChanges
%
% \PrintIndex
