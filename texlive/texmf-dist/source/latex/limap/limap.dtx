%%^^A%%%%%%%%%%%%%%%%%%%%%%%%%%%%%%%%%%%%%%%%%%%%%%%%%%%%%%%%%%%%%%%%%%%%%%%
%%^^A $Id: limap.dtx 1.2 2000/03/01 20:11:42 gene Exp gene $
%%^^A%%%%%%%%%%%%%%%%%%%%%%%%%%%%%%%%%%%%%%%%%%%%%%%%%%%%%%%%%%%%%%%%%%%%%%%
%\iffalse
%% Purpose:
%%      A package for typesetting Information Maps.
%%
%% Documentation:
%%      The documentation  can be generated  from  the  original  file
%%      limap.dtx with the doc style/package. LaTeX the file limap.tex
%%      to get the full documentation in \textsc{dvi} format.
%%
%% Author: Gerd Neugebauer
%%         Mainzer Str. 8
%%         56321 Rhens (Germany)
%% Mail:   gerd.neugebauer@sdm.de
%%	   gerd.neugebauer@gmx.de
%%
%% Copyright (C) 1999-2000 Gerd Neugebauer
%%
%% limap.dtx may be  distributed under the terms of  the LaTeX Project
%% Public  License,  as  described  in  lppl.txt  in  the  base  LaTeX
%% distribution.   Either version 1.0  or, at  your option,  any later
%% version.
%%
%% This class is still under development and  may be replaced with a
%% new version which provides an enhanced functionality.
%%
%\fi
%^^A%%%%%%%%%%%%%%%%%%%%%%%%%%%%%%%%%%%%%%%%%%%%%%%%%%%%%%%%%%%%%%%%%%%%%%%
%
% \title{A \LaTeX\ Package for Typesetting Information Maps\thanks{This file
%    documents \filename\ version \fileversion\ as of \filedate.}} 
% \author{Gerd Neugebauer\\
%	  Mainzer Str.\ 8\\
%	  56321 Rhens (Germany)\\
%	  Net: \texttt{gerd.neugebauer@sdm.de}}
%
% \date{{\footnotesize Documentation date: \docdate}}
%
% \maketitle
%
%^^A%%%%%%%%%%%%%%%%%%%%%%%%%%%%%%%%%%%%%%%%%%%%%%%%%%%%%%%%%%%%%%%%%%%%%%%
% \changes{1.0}{2000/03/01}{First public release.}
%^^A%%%%%%%%%%%%%%%%%%%%%%%%%%%%%%%%%%%%%%%%%%%%%%%%%%%%%%%%%%%%%%%%%%%%%%%
%
% \DoNotIndex{\ ,\",\',\.,\[,\\,\],\^,\`,\~,\@,\@author,\@auxout}
% \DoNotIndex{\@currentlabel,\@currenvir,\@date}
% \DoNotIndex{\@dottedtocline,\@gobble,\@gobbletwo,\@highpenalty}
% \DoNotIndex{\@ifnextchar,\@ifstar,\@ifundefined}
% \DoNotIndex{\@namedef,\@nameuse,\@pnumwidth,\@startsection,\@starttoc}
% \DoNotIndex{\@tempdima,\@title,\@thefnmark,\@undefined}
% \DoNotIndex{\@ixpt,\@vpt,\@vipt,\@viipt,\@viiipt}
% \DoNotIndex{\@xpt,\@xipt,\@xiipt,\@xivpt,\@xvii,\@xxpt,\@xxvpt}
% \DoNotIndex{\AA,\AE,\CodelineIndex}
% \DoNotIndex{\CurrentOption,\DeclareOption,\DeleteShortVerb,\DocInput}
% \DoNotIndex{\EnableCrossrefs,\H,\InputIfFileExists}
% \DoNotIndex{\L,\LARGE,\LaTeX,\Large}
% \DoNotIndex{\LoadClass,\NeedsTeXFormat,\O,\OE,\OptionNotUsed}
% \DoNotIndex{\PackageError,\PackageWarning,\PassOptionsToClass}
% \DoNotIndex{\ProcessOptions,\ProvidesClass}
% \DoNotIndex{\PrintChanges,\PrintIndex}
% \DoNotIndex{\ProvidesPackage,\RecordChanges,\RequirePackage,\TeX}
% \DoNotIndex{\aa,\addcontentsline,\addpenalty,\addtolength,\advance}
% \DoNotIndex{\addvspace,\ae,\ast,\arabic}
% \DoNotIndex{\b,\baselineskip,\begin,\begingroup,\bf,\bgroup,\egroup}
% \DoNotIndex{\bigskip,\box,\bullet}
% \DoNotIndex{\c,\cal,\catcode,\centering,\chapter,\chardef,\circ}
% \DoNotIndex{\clearpage,\closein,\closeout}
% \DoNotIndex{\clubpenalty,\csname}
% \DoNotIndex{\d,\def,\dimen,\diamond,\divide,\documentclass,\dots,\dp}
% \DoNotIndex{\edef,\else,\em,\emph,\empty,\end,\endcsname,\endgroup}
% \DoNotIndex{\endinput,\endlist}
% \DoNotIndex{\expandafter,\fbox,\fi,\footnotesize,\footskip,\framebox}
% \DoNotIndex{\frenchspacing,\futurelet,\gdef,\global,\gobble}
% \DoNotIndex{\hangafter,\hangindent,\hbox,\headheight,\headsep,\hfil}
% \DoNotIndex{\hfill,\hrule,\hskip,\hspace,\hss,\ht,\huge,\ifcat,\ifeof}
% \DoNotIndex{\ifdim,\ifnum,\iftrue,\ifx,\ignorespaces,\immediate,\index}
% \DoNotIndex{\input,\it}
% \DoNotIndex{\itemindent,\itemsep,\jobname,\kern,\l,\labelsep,\labelwidth}
% \DoNotIndex{\large,\leavevmode,\leftmark,\leftskip,\let,\list}
% \DoNotIndex{\llap,\long,\lower}
% \DoNotIndex{\m@th,\makebox,\magstep,\makeindex,\markboth,\mbox,\medskip}
% \DoNotIndex{\multicolumn}
% \DoNotIndex{\newblock,\newcommand,\newcount,\newenvironment,\newfont}
% \DoNotIndex{\newif,\newlength,\newline,\newpage,\newread,\newwrite}
% \DoNotIndex{\nobreak,\noindent,\normalsize,\null,\o,\oe}
% \DoNotIndex{\openin,\openout,\or,\pagestyle,\par,\paragraph,\parbox}
% \DoNotIndex{\parfillskip,\parindent,\parsep,\parskip,\part,\partopsep}
% \DoNotIndex{\penalty,\providecommand,\quad}
% \DoNotIndex{\raggedbottom,\raggedright,\raise,\raisebox,\refstepcounter}
% \DoNotIndex{\relax,\renewcommand,\renewenvironment,\rightskip,\rm,\rule}
% \DoNotIndex{\sbox,\sc,\section,\setcounter,\setlength,\settowidth}
% \DoNotIndex{\sf,\sfcode,\sl}
% \DoNotIndex{\sloppy,\small,\space,\ss,\string}
% \DoNotIndex{\subparagraph,\subsubparagraph}
% \DoNotIndex{\subsection,\subsubsection,\symbol}
% \DoNotIndex{\t,\tenex,\textheight,\textit,\textsf,\textstyle,\textwidth}
% \DoNotIndex{\the,\thepage,\thispagestyle}
% \DoNotIndex{\topmargin,\topsep,\tt,\typeout}
% \DoNotIndex{\u,\unitlength,\usecounter,\v,\varepsilon,\vbox}
% \DoNotIndex{\vfill,\vsize,\vskip,\vspace,\vss}
% \DoNotIndex{\wd,\widowpenalty,\write,\xdef,\z@}
%
%^^A%%%%%%%%%%%%%%%%%%%%%%%%%%%%%%%%%%%%%%%%%%%%%%%%%%%%%%%%%%%%%%%%%%%%%%%
%    \CheckSum{608}
%%%%%%%%%%%%%%%%%%%%%%%%%%%%%%%%%%%%%%%%%%%%%%%%%%%%%%%%%%%%%%%%%%%%%%%%
%%  \CharacterTable
%%  {Upper-case    \A\B\C\D\E\F\G\H\I\J\K\L\M\N\O\P\Q\R\S\T\U\V\W\X\Y\Z
%%   Lower-case    \a\b\c\d\e\f\g\h\i\j\k\l\m\n\o\p\q\r\s\t\u\v\w\x\y\z
%%   Digits        \0\1\2\3\4\5\6\7\8\9
%%   Exclamation   \!     Double quote  \"     Hash (number) \#
%%   Dollar        \$     Percent       \%     Ampersand     \&
%%   Acute accent  \'     Left paren    \(     Right paren   \)
%%   Asterisk      \*     Plus          \+     Comma         \,
%%   Minus         \-     Point         \.     Solidus       \/
%%   Colon         \:     Semicolon     \;     Less than     \<
%%   Equals        \=     Greater than  \>     Question mark \?
%%   Commercial at \@     Left bracket  \[     Backslash     \\
%%   Right bracket \]     Circumflex    \^     Underscore    \_
%%   Grave accent  \`     Left brace    \{     Vertical bar  \|
%%   Right brace   \}     Tilde         \~}
%%
%^^A%%%%%%%%%%%%%%%%%%%%%%%%%%%%%%%%%%%%%%%%%%%%%%%%%%%%%%%%%%%%%%%%%%%%%%%
%
%    \begin{abstract}
%      The Information
%      Mapping\(^{\textrm{\footnotesize\textregistered}}\) method
%      provides a different methodology for structuring and presenting
%      information. It claims to be useful for readers who are more
%      concerned about finding the right information than reading the
%      document as a whole. Thus short, highly structured, and context
%      free pieces of information are used.
%
%      \texttt{limap.dtx} provides a \LaTeX{} style and a \LaTeX{}
%      class.  The style contains definitions to typeset maps and
%      blocks according to the Information Mapping method. The
%      class provides all definitions to typeset a whole document.
%    \end{abstract}
%
%    \newpage
%    \tableofcontents
%    \newpage
%    
%    \section{Motivation}
%    
%    The information mapping method provides a methodology to
%    structure information in a special way. The aim is to help a
%    reader who uses the document to search for relevant information
%    instead of consuming  it from start to end. The information
%    mapping method also claims to raise the productivity of writers.
%
%    This document does not include an introduction to the information
%    mapping method. The reader is referred to other documents. Maybe
%    an acompanying document is distributed with this package.
%
%    To support the information mapping method several \LaTeX{} macros
%    and environments are needed which provide a logical description
%    of the relevant concepts. Those macros are provided in the
%    package and class file.
%    
%    This package provides both a class file as well as a package. The
%    package contains the definitions of maps, blocks, and
%    others. They can be used together with any base class. 
%
%    The class load a base class and does some other useful stuff. The
%    base class can be determined with two class options.
%    
%    
%    \section{The User Interface}
%    
%    The main part of the user interface is inherited from \LaTeX. The
%    major differences are the sectioning commands which are made
%    obsolete in parts by the information mapping method.
%
%    \subsection{Package and Class Options}
%    The package and the class can take several options to influence
%    the behaviour of the package or class. 
%    
%    First, we describe the settings influencing the langu�ge specific
%    settings. They do not make provisions to use the appropriate
%    hyphenation patterns. They just arrange things such that the
%    internaly used texts are displaeyed in the choosen language.
%    \begin{description}
%    \item [austrian] Activate the language specific text fragments
%    for the austrian language (in fact German with one minor modification).
%    \item [german] Activate the language specific text fragments
%    for the german language.
%    \item [french] Activate the language specific text fragments
%    for the french language.
%    \item [english] Activate the language specific text fragments
%    for the english language.
%    \item [USenglish] Activate the language specific text fragments
%    for the american language.
%    \end{description}
%
%    The class has two additional options to determine the base class
%    to be used. The first option is the variant. It can take the
%    following values:
%    
%    \begin{description}
%    \item [base] Use the base set of classes. This is the default.
%    \item [koma] Use the set of classes from koma-script.
%    \end{description}
%
%    The second option is the class type. It determines which kind of
%    document to typeset. It can talk the following values:
%    \begin{description}
%    \item [book] Typeset a book type document.
%    \item [report] Typeset a report type document.
%    \item [article] Typeset a article type document.
%    \item [letter] Typeset a letter type document.
%    \end{description}
%    
%    The following table shows which base classes are loaded according
%    to the given values:
%    
%    \begin{center}
%      \begin{tabular}{lcc}\hline
%      			& \emph{base}	& \emph{koma}	\\\hline
%      \emph{book}	& book		& scrbook	\\
%      \emph{report}	& report	& scrreprt	\\
%      \emph{article}	& article	& scrartcl	\\
%      \emph{letter}	& latter	& scrlettr	\\\hline
%      \end{tabular}
%    \end{center}
%    
%    Any option not processed by limap.cls is passed to the base class
%    used. Thus it is possible to customize those classes any further.
%    
%    
%    \subsection{Macros and Environments}
%
%    \DescribeEnv{Block}
%    The environment |Block| can be used to typeset an IMAP block. It takes
%    one argument which is the block title.
%
%    This environment is active inside a map only. It can't easily be
%    redefined.
%
%    \DescribeMacro{\Block}
%    The macro |\Block| can be used to typeset an IMAP block. It takes
%    one argument which is the block title.
%
%    This is a shorthand for denoting a block. The end mark can be
%    omitted if you use the macro instead of the environment.
%    Nevertheless this is depreciated. 
%
%    \DescribeMacro{\WideBlock}
%    The macro |\WideBlock| can be used to typeset a piece of
%    information on the whole page width. It is normally used after an
%    initiating block containng the title of the whole construction.
%
%    The macro |\WideBlock| takes one argument which contains the
%    material to span the whole page width.
%
%    \DescribeEnv{Map}
%    The environment |Map| can be used to typeset an IMAP map. 
%
%    \DescribeMacro{\MapTableOfContents}
%    The macro |\MapTableOfContents| can be used to typeset the table
%    of contents for a map. This table of contents includes all
%    submaps of the map it is contained in---not recursively but
%    only one level deeper.
%
%    \DescribeMacro{\maketitle}
%    The macro |\maketitle| is redefined by the class to show a new
%    appearance on a title page. It can be used as in the standard
%    classes. 
%
%    
%    \subsection{The Configuration Options}
%
%    The configuration options are settings which can be easily
%    adapted to suit your personal needs. Nevertheless it is strongly
%    recommended that you touch them only if you are really knowing
%    what you are doing and when you are willing to face the
%    consequences.
%
%    \DescribeMacro{\MapRuleWidth}
%    The macro |\MapRuleWidth| determines the width of the rules drawn
%    between blocks.
%
%    \DescribeMacro{\MapFont}
%    The macro |\MapFont| determines the font changing command to be
%    used when starting a new map. 
%
%    \DescribeMacro{\MapTitleSize}
%    The macro |\MapTitleSize| determines the size changing command to
%    be used when typesetting the title of a map.
%
%    \DescribeMacro{\MapTitleContinuedSize}
%    This macro determines the font changing command to be used for
%    typesetting the additional text after titles on followup pages of
%    multipage maps.
%
%    \DescribeMacro{\MapParskip}
%    The macro |\MapParskip| determines the distance of the text from
%    the separating rules.
%
%    \DescribeMacro{\MapTitlefraction}
%    The macro |\MapTitlefraction| determines the part of the page
%    width devoted to the title area. It is a fraction in the range
%    from 0 to 1. 
%
%    \DescribeMacro{\MapTextfraction}
%    This macro determines the part of the page width devoted to the
%    text area. It is a fraction in the range from 0 to 1.
%    |\MapTitlefraction| and |\MapTextfraction| should add up to
%    something less or equal to 1. Otherwise you will get some
%    ``overfull hbox'' messages.
%
%    \DescribeMacro{\MapNewpage}
%    The macro |\MapNewpage| is expanded whenever a new page is
%    required between maps. Thus it can be used to suppress the
%    newpages by |\let|ing it to |\relax|. Note that this is not in
%    the spirit of the Information Mapping method.
%
%    \DescribeMacro{\MapTOC} 
%    The macro |\MapTOC| is expanded to generate the entry in the
%    table of contents. It can be redefined to allow another
%    behaviour.
%
%    
%    \subsection{Changing or Adding Language Specific Settings}
%
%    Several strings are used automatically by the current class or
%    package. Default values for several languages are hardwired in
%    the implementation. Nevertheless it is possible to change those
%    language specific settings.
%
%    If you create settings for a new language it is highly
%    recommended to contact the author to integrate them into the
%    default distribution.
%
%    The following macros can be redefined in the preamble after the
%    package or class has been loaded to reset the language specific
%    text. 
%
%    \DescribeMacro{\MapContinued}
%    The macro |\MapContinued| contains the text appearing at the end
%    of map which are continued on the next page.
%
%    \DescribeMacro{\MapContinuing}
%    The macro |\MapContinuing| contains the text appearing at the
%    beginning of map which are continued from the previous page. It
%    is typeset after the map title.
%
%    \DescribeMacro{\MapTOCname}
%    The macro |\MapTOCname| contains the text of the heading in table
%    of contents of maps for the column of map titles.
%
%    \DescribeMacro{\MapTOCpage}
%    The macro |\MapTOCpage| contains the text of the heading in table
%    of contents of maps for the column of page numbers.
%
%    If you want to provide a new language \emph{lang} you can define
%    the macro |\IMAP@SelectLanguage@|\texttt{\textit{lang}} which
%    redefines the macros given above. This definition has to be
%    present before the package is loaded.
%
%    Note that the macro name contains a @ character. Thus the
%    definition should be made in a package of its own.
%
%    
%    \subsection{The Configuration File}
%
%    When the class or package is loaded as a last action a
%    configuration file is loaded if it can be found. The name of the
%    configuration file is \texttt{limap.cfg}. This file can contain
%    redefinitions of the several macros to adjust the behaviour of
%    limap on a per directory, per user or per installation base.
%
%    Note that some settings are activated before the configuration
%    file is loaded. Thus some settings may not have any effect at
%    all. 
%
%
%
%    \vfill\noindent\hrulefill
%
%    \emph{To be completed}
%
%    \noindent\hrulefill\vfill\null
%
%    \StopEventually{}
%    \newpage
%    \if@undefined\environment
%    \newenvironment{environment}[1]{}{}
%    \fi
%
%
%    \section{The Documentation Driver}
%
%    The documentation driver is necessary to provide a self
%    documenting dtx file. With this construction the dtx file can be
%    run through \LaTeX\ to produce the documentation.
%
%    \subsection{The Version Information}
%
%    \begin{macro}{\LIMAP@RCS}
%    The macro |\LIMAP@RCS| is used to parse rcs information. The
%    second word enclosed in spaces is preserved. The other parts are
%    ignored. 
%    \begin{macrocode}
\def\LIMAP@RCS$#1: #2 #3${#2}
%    \end{macrocode}
%    \end{macro}
%
%    Now the usual macros are filled. Some of the informations are
%    taken from strings automatically managed by rcs.
%
%    \begin{macro}{\filename}
%    |\filename| is the name of the dtx file.
%    \begin{macrocode}
\def\filename{limap.dtx}
%    \end{macrocode}
%    \end{macro}
%    \begin{macro}{\fileversion}
%    |\fileversion| is the version number of the dtx file.
%    \begin{macrocode}
\xdef\fileversion{\LIMAP@RCS$Revision: 1.2 $}
%    \end{macrocode}
%    \end{macro}
%    \begin{macro}{\filedate}
%    |\filedate| is the change date of the dtx file.
%    \begin{macrocode}
\xdef\filedate{\LIMAP@RCS$Date: 2000/03/01 20:11:42 $}
%    \end{macrocode}
%    \end{macro}
%    \begin{macro}{\docversion}
%    |\docversion| is version number of the documentation. It is
%    identical to the version number of the dtx file.
%    \begin{macrocode}
\let\docversion=\fileversion
%    \end{macrocode}
%    \end{macro}
%    \begin{macro}{\docdate}
%    |\docdate| is change date of the documentation. It is
%    identical to the change date of the dtx file.
%    \begin{macrocode}
\let\docdate=\filedate
%    \end{macrocode}
%    \end{macro}
%
%    \subsection{Producing the Documentation}
%
%    The driver section contains a complete \LaTeX\ document which
%    loads the dtx file. The special class |ltxdoc| is used and some
%    arraganements are made for this purpose.
%
%    \begin{macrocode}
%<*driver>
\documentclass{ltxdoc}
\RequirePackage{textcomp}
\InputIfFileExists{limap.dcf}{}{}
\RecordChanges
\EnableCrossrefs
\CodelineIndex
\begin{document}
\DeleteShortVerb{|}
\DocInput{\filename}
\newpage
\PrintChanges
\newpage
\setcounter{IndexColumns}{2}
\PrintIndex
\end{document}
%</driver>
%    \end{macrocode}
%
%    \section{The Implementation}
%    The rest of the document describes the implementation. Usually it
%    is not meant for the casual user. Nevertheless it might be
%    fruitful for those searching for inspiration or for tricks when
%    using this class or style.
%
%    \subsection{Preliminaries and Option Processing}
%
%    First of all we request a descent version of \LaTeX\ to be used.
%
%    \begin{macrocode}
\NeedsTeXFormat{LaTeX2e}
%    \end{macrocode}
%
%    \subsubsection{The Package/Class Declarations}
%
%    When the package is generated, the package identification is
%    included. 
%    \begin{macrocode}
%<*package>
\ProvidesPackage{limap}[\filedate\space  gene]
%</package>
%    \end{macrocode}
%
%    When the class is generated, the class identification is
%    included. 
%    \begin{macrocode}
%<*class>
\ProvidesClass{limap}[\filedate\space  gene]
%</class>
%    \end{macrocode}
%    
%    \subsubsection{Language Specific Declarations}
%
%    \begin{macro}{\LIMAP@Language}
%    The macro |\IMAP@Language| determines the language to be used for
%    several small text fragments to be inserted at certain places. It
%    is redefinded by package/class options and evaluated at the end
%    to activate the selected settings.
%    \begin{macrocode}
\providecommand\LIMAP@Language{english}
%    \end{macrocode}
%    \end{macro}
%    
%    \begin{macrocode}
\DeclareOption{austrian}{\renewcommand\LIMAP@Language{austrian}}
\DeclareOption{german}{\renewcommand\LIMAP@Language{german}}
\DeclareOption{french}{\renewcommand\LIMAP@Language{french}}
\DeclareOption{english}{\renewcommand\LIMAP@Language{english}}
\DeclareOption{USenglish}{\renewcommand\LIMAP@Language{USenglish}}
%    \end{macrocode}
%
%    \begin{macro}{\ifLIMAP@strict}
%    The boolean |\ifLIMAP@strict| determines if the lower sectioning
%    macros should be disabled in the class.
%    \begin{macrocode}
\newif\ifLIMAP@strict \LIMAP@stricttrue
%    \end{macrocode}
%    \end{macro}
%
%    \begin{macrocode}
\DeclareOption{nonstrict}{\LIMAP@strictfalse}
%    \end{macrocode}
%    
%    \subsubsection{Determining the Appropriate Base Class}
%
%    \begin{macrocode}
%<*class>
%    \end{macrocode}
%
%    \begin{macro}{\LIMAP@ClassType}
%    The macro |\LIMAP@ClassType| determines the type of the class to
%    be used. Usually it can take the values |book|, |report|,
%    |article|, and |letter| (for completeness). This macro is
%    redefined when the options of the class are evaluated. Finally
%    this macro helps to select the appropriate base class.
%    \begin{macrocode}
\providecommand\LIMAP@ClassType{report}
%    \end{macrocode}
%    \end{macro}
%    
%    \begin{macrocode}
\DeclareOption{book}{\renewcommand\LIMAP@ClassType{book}}
\DeclareOption{report}{\renewcommand\LIMAP@ClassType{report}}
\DeclareOption{article}{\renewcommand\LIMAP@ClassType{article}}
\DeclareOption{letter}{\renewcommand\LIMAP@ClassType{letter}}
%    \end{macrocode}
%    
%    \begin{macro}{\LIMAP@Variant}
%    The macro |\LIMAP@Variant| determines the variant of the class to
%    be used. Usually it can take the values |base| and |koma|. This
%    macro is redefined when the options of the class are evaluated.
%    Finally this macro helps to select the appropriate base class.
%    \begin{macrocode}
\providecommand\LIMAP@Variant{base}
%    \end{macrocode}
%    \end{macro}
%    
%    \begin{macrocode}
\DeclareOption{koma}{\renewcommand\LIMAP@Variant{koma}}
\DeclareOption{base}{\renewcommand\LIMAP@Variant{base}}
%    \end{macrocode}
%
%    Define a mapping between the variant and class type to the class
%    name to be used.
%    \begin{macrocode}
\newcommand\LIMAP@Class@base@article{article}
\newcommand\LIMAP@Class@base@report{report}
\newcommand\LIMAP@Class@base@book{book}
\newcommand\LIMAP@Class@base@letter{letter}
\newcommand\LIMAP@Class@koma@article{scrartcl}
\newcommand\LIMAP@Class@koma@report{scrreprt}
\newcommand\LIMAP@Class@koma@book{scrbook}
\newcommand\LIMAP@Class@koma@letter{scrlettr}
%    \end{macrocode}
%
%    \begin{macrocode}
\DeclareOption*{\PassOptionsToClass{\CurrentOption}{%
    \csname LIMAP@Class@\LIMAP@Variant @\LIMAP@ClassType\endcsname}%
  }
%    \end{macrocode}
%
%    Thus the class specific options are completed.
%    \begin{macrocode}
%</class>
%    \end{macrocode}
%    
%    Now we can process all options. 
%    \begin{macrocode}
\ProcessOptions
%    \end{macrocode}
%
%    \begin{macrocode}
%<*class>
%    \end{macrocode}
%    The requested class is loaded and the options remaining are
%    processed. 
%    \begin{macrocode}
\LoadClass{\csname LIMAP@Class@\LIMAP@Variant @\LIMAP@ClassType\endcsname}
%</class>
%    \end{macrocode}
%
%    \subsubsection{Loading Required Packages}
%
%    The package |longtable| is used internally to implement a part of
%    the required functionality. Thus we need to ensure that it is
%    loaded. 
%    \begin{macrocode}
\RequirePackage{longtable}
%    \end{macrocode}
%
%    The package |booktabs| is used internally to implement a part of
%    the required functionality. Thus we need to ensure that it is
%    loaded. 
%    \begin{macrocode}
\RequirePackage{booktabs}
%    \end{macrocode}
%    
%    
%    \begin{macrocode}
%<*class>
\RequirePackage{fancyhdr}
\addtolength{\headheight}{2ex}%
\pagestyle{fancy}%
\cfoot{}
\rhead{\small\thepage}
\lhead{\textit{\footnotesize\@title}}
\def\@title{}
%</class>
%    \end{macrocode}
%
%    Since the blocks are not supposed to line up at the end of the
%    page we declare |\raggedbottom|.
%    \begin{macrocode}
\raggedbottom
%    \end{macrocode}
%    
%    \subsection{Layout Parameters}
%
%    The layout can be influenced by a large number of
%    parameters. Thus the design decisions have been made transparent
%    (to a certain degree at least). These options are not meant to be
%    changed except when a new layout is beeing designed and implemented.
%
%    \begin{macro}{\MapRuleWidth}
%    The macro |\MapRuleWidth| determines the width of the rules drawn
%    between blocks. 
%    \begin{macrocode}
\newcommand\MapRuleWidth{.25pt}
%    \end{macrocode}
%    \end{macro}
%
%    \begin{macro}{\MapContinued}
%    This macro determines the text to be used in the title of
%    continued maps. This macro is reset when the language specific
%    initializations are performed.
%    \begin{macrocode}
\newcommand\MapContinued{}
%    \end{macrocode}
%    \end{macro}
%
%    \begin{macro}{\MapContinuing}
%    The macro |\MapContinuing| determines the  text to be used at the
%    bottom of the map which is continued. This macro is reset when
%    the language specific initializations are performed.
%    \begin{macrocode}
\newcommand\MapContinuing{}
%    \end{macrocode}
%    \end{macro}
%
%    \begin{macro}{\MapContinuingFormat}
%    This macro determines the format of the bottom line on continued
%    maps. I.e. it includes the text as well as font changing
%    commands. The text is passed to this command as argument 1.
%    \begin{macrocode}
\newcommand\MapContinuingFormat[1]{\textit{\footnotesize #1}}
%    \end{macrocode}
%    \end{macro}
%
%    \begin{macro}{\MapContinuedFormat}
%    This macro determines the format of the bottom line on continued
%    maps. I.e. it includes the text passed to it as argument 1 as
%    well as font changing commands. 
%    \begin{macrocode}
\newcommand\MapContinuedFormat[1]{, {\MapTitleContinuedSize #1}}
%    \end{macrocode}
%    \end{macro}
%
%    \begin{macro}{\MapFont}
%    The macro |\MapFont| determines the font changing command to be
%    used when starting a new map. 
%    \begin{macrocode}
\let\MapFont\textsf
%    \end{macrocode}
%    \end{macro}
%
%    \begin{macro}{\MapTitleSize}
%    The macro |\MapTitleSize| determines the size changing command to
%    be used when typesetting the title of a map.
%    \begin{macrocode}
\let\MapTitleSize\Large
%    \end{macrocode}
%    \end{macro}
%
%    \begin{macro}{\MapTitleContinuedSize}
%    This macro determines the font changing command to be used for
%    typesetting the additional text after titles on followup pages of
%    multipage maps.
%    \begin{macrocode}
\let\MapTitleContinuedSize\small
%    \end{macrocode}
%    \end{macro}
%
%    \begin{macro}{\MapParskip}
%    The macro |\MapParskip| determines the distance of the text from
%    the separating rules.
%    \begin{macrocode}
\newcommand\MapParskip{2ex}
%    \end{macrocode}
%    \end{macro}
%
%    \begin{macro}{\MapTitlefraction}
%    The macro |\MapTitlefraction| determines the part of the page
%    width devoted to the title area. It is a fraction in the range
%    from 0 to 1. 
%    \begin{macrocode}
\newcommand\MapTitlefraction{.2}
%    \end{macrocode}
%    \end{macro}
%
%    \begin{macro}{\MapTextfraction}
%    This macro determines the part of the page width devoted to the
%    text area. It is a fraction in the range from 0 to 1.
%    |\MapTitlefraction| and |\MapTextfraction| should add up to
%    something less or equal to 1. Otherwise you will get some
%    ``overfull hbox'' messages.
%    \begin{macrocode}
\newcommand\MapTextfraction{.75}
%    \end{macrocode}
%    \end{macro}
%
%    \subsection{Adaptable Macros}
%
%    \begin{macro}{\MapNewpage}
%    The macro |\MapNewpage| is expanded whenever a new page is
%    required between maps. Thus it can be used to suppress the
%    newpages by |\let|ing it to |\relax|.
%    \begin{macrocode}
\let\MapNewpage\newpage
%    \end{macrocode}
%    \end{macro}
%
%    \begin{macro}{\MapTOC}
%    The macro |\MapTOC| is expanded to generate the entry in the
%    table of contents. It can be redefined to allow another
%    behaviour.
%    \begin{macrocode}
\newcommand\MapTOC[1]{%
  \refstepcounter{\@nameuse{Map@TOC@name\the\Map@level}}%
  \addcontentsline{toc}{\@nameuse{Map@TOC@name\the\Map@level}}{#1}%
}
%    \end{macrocode}
%    \end{macro}
%
%    \begin{macro}{\MapTOCname}
%    The macro |\MapTOCname| contains the heading for the section title
%    in contents blocks. This macro is reset when the language specific
%    initializations are performed.
%    \begin{macrocode}
\newcommand\MapTOCname{}
%    \end{macrocode}
%    \end{macro}
%
%    \begin{macro}{\MapTOCpage}
%    The macro |\MapTOCpage| contains the heading for the page number
%    in contents blocks. This macro is reset when the language specific
%    initializations are performed.
%    \begin{macrocode}
\newcommand\MapTOCpage{}
%    \end{macrocode}
%    \end{macro}
%
%    \begin{macro}{\MapTOCemph}
%    The macro |\MapTOCemph| 
%    \begin{macrocode}
\let\MapTOCemph=\emph
%    \end{macrocode}
%    \end{macro}
%
%    \subsection{Language Specific Macros}
%
%    This section contains interanl macros used to implement the
%    functionality. New languages can be easily be added. For this
%    puropose only a new macro has to be defined and a package/class
%    option for the convenience of the user.
%
%    Consider you want to add a new language ``latin'' then you have
%    to provide the command |\LIMAP@SelectLanguage@latin|. This macro
%    should simply redefine the macros containing strings of the
%    language specific texts. Examples for other languages are
%    provided in this section.
%
%    To enable the language settings for ``latin'' the macro
%    |\LIMAP@Language| has to be defined to contain the value
%    ``latin''. Usually this is accomplished by providing a convenient
%    option to the package or class.
%
%    \begin{macro}{\LIMAP@SelectLanguage@austrian}
%    Provide the definition for the langauge ``austrian''.
%    \begin{macrocode}
\providecommand\LIMAP@SelectLanguage@austrian{%
  \renewcommand\MapContinued{ Fortsetzung}%
  \renewcommand\MapContinuing{Fortsetzung\dots}
  \renewcommand\MapTOCname{Titel}
  \renewcommand\MapTOCpage{Seite}
}
%    \end{macrocode}
%    \end{macro}
%
%    \begin{macro}{\LIMAP@SelectLanguage@german}
%    Provide the definitions for the laguage ``german''.
%    \begin{macrocode}
\providecommand\LIMAP@SelectLanguage@german{%
  \renewcommand\MapContinued{ Fortsetzung}%
  \renewcommand\MapContinuing{Fortsetzung\dots}
  \renewcommand\MapTOCname{Titel}
  \renewcommand\MapTOCpage{Seite}
}
%    \end{macrocode}
%    \end{macro}
%
%    \begin{macro}{\LIMAP@SelectLanguage@english}
%    Provide the definitions for the language ``english''.
%    \begin{macrocode}
\providecommand\LIMAP@SelectLanguage@english{%
  \renewcommand\MapContinued{ Continued}%
  \renewcommand\MapContinuing{Continuing\dots}
  \renewcommand\MapTOCname{Title}
  \renewcommand\MapTOCpage{Page}
}
%    \end{macrocode}
%    \end{macro}
%
%    \begin{macro}{\LIMAP@SelectLanguage@USenglish}
%    Provide the definitions for the language ``USenglish''.
%    \begin{macrocode}
\providecommand\LIMAP@SelectLanguage@USenglish{%
  \renewcommand\MapContinued{ Continued}%
  \renewcommand\MapContinuing{Continuing\dots}
  \renewcommand\MapTOCname{Title}
  \renewcommand\MapTOCpage{Page}
}
%    \end{macrocode}
%    \end{macro}
%
%
%    \subsection{Internal Macros, Lengths, and Counters}
%
%    This section contains internal macros used to implement the
%    functionality. 
%
%    \begin{macro}{\Map@length}
%    The length register |\Map@length| is allocated to store the width
%    of the space between the columns of a block.
%    \begin{macrocode}
\newlength{\Map@length}
%    \end{macrocode}
%    \end{macro}
%
%    \begin{macro}{\Map@level}
%    The macro |\Map@level| determines the level of inclusion of
%    maps. It is used to determine the appearence in the table of
%    contents. 
%    \begin{macrocode}
\newcount\Map@level
\Map@level=0
%    \end{macrocode}
%    \end{macro}
%
%    \begin{macro}{\Map@blockcount}
%    The macro |\Map@blockcount| is used to count the blocks per map
%    to issue a style warning if required.
%    \begin{macrocode}
\newcount\Map@blockcount
%    \end{macrocode}
%    \end{macro}
%
%    \begin{macro}{\ifMap@open@}
%    The conditional |\ifMap@open@| is used to record the opening and
%    closing of the |longtable| environment, since can not be used
%    inside itself. Thus it can be closed before a new instance is
%    opened. 
%    \begin{macrocode}
\newif\ifMap@open@
\Map@open@false
%    \end{macrocode}
%    \end{macro}
%
%    \begin{macro}{\Map@TOC@name}
%    The macros |\Map@TOC@name|\dots provide a mapping between a
%    number and a sectioning unit. This mapping is used when the
%    entry in the table of contents is generated.
%    \begin{macrocode}
\@namedef{Map@TOC@name0}{chapter}
\@namedef{Map@TOC@name1}{section}
\@namedef{Map@TOC@name2}{subsection}
\@namedef{Map@TOC@name3}{subsubsection}
\@namedef{Map@TOC@name4}{paragraph}
\@namedef{Map@TOC@name5}{subparagraph}
\@namedef{Map@TOC@name6}{subsubparagraph}
\@namedef{Map@TOC@name7}{subsubparagraph}
\@namedef{Map@TOC@name8}{subsubparagraph}
\@namedef{Map@TOC@name9}{subsubparagraph}
\@namedef{Map@TOC@name10}{subsubparagraph}
\@namedef{Map@TOC@name11}{subsubparagraph}
\@namedef{Map@TOC@name12}{subsubparagraph}
%    \end{macrocode}
%    \end{macro}
%
%    \begin{macro}{\Map@start}
%    The macro |\Map@start| is used to initiate the use of a map. It
%    uses the |longtable| environment to perform the page breaking and
%    marking of continued pages.
%    \begin{macrocode}
\newcommand\Map@start{%
%  \typeout{--- MAP START}%
  \setlength{\Map@length}{\textwidth}%
  \addtolength{\Map@length}{-\MapTitlefraction\textwidth}%
  \addtolength{\Map@length}{-\MapTextfraction\textwidth}%
  \MapTOC{\Map@TITLE}%
  \longtable
    {@{}p{\MapTitlefraction\textwidth}@{\hspace{\Map@length}}
        p{\MapTextfraction\textwidth}@{}}%
      \multicolumn{2}{@{}p{\textwidth}@{}}{%
        \MapFont{\MapTitleSize\rule{0pt}{3ex}%
          \Map@TITLE}}
    \endfirsthead
      \multicolumn{2}{@{}p{\textwidth}@{}}{%
        \MapFont{\MapTitleSize\rule{0pt}{3ex}%
          \Map@TITLE\MapContinuedFormat{\MapContinued}}}%
    \endhead
      \par\vspace*{-\parskip}\vspace*{-2ex}\\
      &\rule{\MapTextfraction\textwidth}{\MapRuleWidth}\newline
      \mbox{}\hfill\raisebox{3pt}{\MapContinuingFormat{\MapContinuing}}
    \endfoot
      &\rule{\MapTextfraction\textwidth}{\MapRuleWidth}%
       \vspace{\MapParskip}
    \endlastfoot
    \xdef\@currentlabel{\Map@TITLE}%
    \global\Map@open@true
}
%    \end{macrocode}
%    \end{macro}
%
%    \begin{macro}{\Map@end}
%    The macro |\Map@end| is expanded when the end of the end of the
%    |longtable| environment might be needed. The boolean |\ifMap@open@|
%    determines whether such an environment is really open.
%    \begin{macrocode}
\newcommand\Map@end{%
%  \typeout{--- MAP END}%
  \ifMap@open@
    \global\Map@open@false
    \endlongtable
    \MapNewpage
  \fi
  \iftrue
   \ifnum\Map@blockcount>9
    \PackageWarning{limap}%
    {*** The current map contains too much blocks: \the\Map@blockcount}%
   \else\ifnum\Map@blockcount>7
    \PackageWarning{limap}%
    {--- The current map contains \the\Map@blockcount blocks.}%
   \fi\fi
  \fi
}
%    \end{macrocode}
%    \end{macro}
%
%    \begin{macro}{\Map@UP}
%    The macro |\Map@UP| contains the number of the parent map or the
%    empty string.
%    \begin{macrocode}
\newcommand\Map@UP{}
%    \end{macrocode}
%    \end{macro}
%
%    \begin{macro}{\Map@no}
%    The counter |\Map@no| contains the sequence number for all
%    maps. This value is used internally to reference single maps.
%    \begin{macrocode}
\newcount\Map@no
%    \end{macrocode}
%    \end{macro}
%
%    \begin{macro}{\Map@parts@}
%    The macro |\Map@parts@| is used to store the parts of the
%    toplevel maps. This is the initialization of a feature otherwise
%    used in the aux file.
%    \begin{macrocode}
\@namedef{Map@parts@}{}
%    \end{macrocode}
%    \end{macro}
%
%    \subsection{Typesetting a Map}
%
%    \begin{environment}{Map}
%    This environment determines the apearance of a Map. It is
%    implemented as a longtable environment which takes care for the
%    page breaks and inserts material at the end of the page and the
%    beginning of the new page upon page break.
%    \begin{macrocode}
\newenvironment{Map}[1]{%
%    \end{macrocode}
%    First the messages of |longtable| are modified to show this
%    package name instead.
%    \begin{macrocode}
  \def\LT@err{\PackageError{limap}}%
  \def\LT@warn{\PackageWarning{limap}}%
%    \end{macrocode}
%    The map local macro |\Block| and the environment |Block| is
%    activated. The counter for blocks is reset.
%    \begin{macrocode}
  \let\Block\Map@Block
  \let\endBlock\Map@endBlock
  \Map@blockcount=0
%    \end{macrocode}
%    The number of the map in the internal counting is set by
%    incrementing the old value.
%    \begin{macrocode}
  \global\advance\Map@no1
%  \typeout{--- \the\Map@no: \Map@UP}%
%    \end{macrocode}
%    
%    \begin{macrocode}
  \ifx\Map@UP\empty\else
    \immediate\write\@auxout
      {\string\expandafter\string\xdef\string\csname\space
        Map@parts@\Map@UP\string\endcsname{\string\csname\space 
          Map@parts@\Map@UP\string\endcsname\the\Map@no:}}%
  \fi
%    \end{macrocode}
%    
%    \begin{macrocode}
  \edef\Map@UP{\the\Map@no}%
  \ifnum\Map@level>0
    \xdef\Map@@up{\Map@UP}% Just to save the value across blocks.
    \endgroup
%    \typeout{--- Closing Map \the\Map@level}%
    \Map@end
    \begingroup
    \edef\Map@UP{\Map@@up}%
    \def\@currenvir{Map}%
  \fi
  \edef\Map@this{\the\Map@no}%
%    \end{macrocode}
%    The entries for future use of submaps are written to the aux file.
%    \begin{macrocode}
  \immediate\write\@auxout
    {\string\global\string\@namedef{Map@title@\the\Map@no}{#1}}%
  \immediate\write\@auxout
    {\string\global\string\@namedef{Map@page@\the\Map@no}{\the\c@page}}%
  \immediate\write\@auxout
    {\string\global\string\@namedef{Map@parts@\the\Map@no}{}}%
%    \end{macrocode}
%    
%    \begin{macrocode}
%  \typeout{--- Opening Map \the\Map@level}%
  \global\advance\Map@level1
  \def\Map@TITLE{#1}%
  \Map@start
%    \end{macrocode}
%
%    \begin{macrocode}
  }{%
%    \end{macrocode}
%
%    \begin{macrocode}
  \Map@end
  \global\advance\Map@level-1
%  \typeout{--- At end of Map \Map@this  level \the\Map@level}%
}
%    \end{macrocode}
%    \end{environment}
%
%    \subsection{Typesetting a Block}
%
%    \begin{environment}{Map@Block}
%    This macro is used to typeset a block inside a Map. To avoid
%    abuse outside of a map it is activated within a Map only.
%    \begin{macrocode}
\newenvironment{Map@Block}[1]{\par\vspace*{-\parskip}\vspace*{-2ex}%
  \\\null\par
  \vspace*{\MapParskip}%
  \raggedright\hspace{0pt}\MapFont{#1}%
  \gdef\@currentlabel{#1}%
%  \global\advance\Map@blockcount1
  &\parskip=\MapParskip
  \rule{\MapTextfraction\textwidth}{\MapRuleWidth}\par
%    \end{macrocode}
%
%    The final action is empty. Thus the block can be used as a simple
%    macro as well. 
%    \begin{macrocode}
}{%
}
%    \end{macrocode}
%    \end{environment}
%
%    \begin{macro}{\WideBlock}
%    The macro |\WideBlock| takes one argument which is added to the
%    current block where the whole width of the table is used.
%    \begin{macrocode}
\newcommand\Wide@Block{\\\multicolumn2{@{}l@{}}}{}
%    \end{macrocode}
%    \end{macro}
%
%    \subsection{Typesetting a Table of Contents}
%
%    \begin{macro}{\MapTableOfContents}
%    The macro |\MapTableOfContents| produces the table of contents
%    for the current map. It produces a tabular containing the
%    titels and pages of all maps directly contained in the current
%    map. 
%    \begin{macrocode}
\newcommand\MapTableOfContents{%
  \medskip\par
  \xdef\Map@@{\csname Map@parts@\the\Map@no\endcsname}%
  \gdef\Map@@@{}%
  \centering
  \begin{tabular}{p{.6\textwidth}r}\toprule
    \MapTOCemph{\MapTOCname}&\MapTOCemph{\MapTOCpage}\\
    \midrule
    \expandafter\Map@toc@loop \Map@@:%
    \\\bottomrule
  \end{tabular}
}
%    \end{macrocode}
%    \end{macro}
%
%    \begin{macro}{\Map@toc@loop}
%    The macro |\Map@toc@loop| is a recursive solution to loop throup
%    all elements of a list of children.
%    \begin{macrocode}
\def\Map@toc@loop#1:{%
  \def\Map@@{#1}%
  \ifx\Map@@\empty
    \global\let\Map@@=\relax
  \else
    \gdef\Map@@{\Map@@@\@nameuse{Map@title@#1}&\@nameuse{Map@page@#1}%
      \global\let\Map@@@=\\%
      \Map@toc@loop}%
  \fi
  \Map@@
}
%    \end{macrocode}
%    \end{macro}
% 
%    \subsection{Typesetting a Title Page}
%
%    \begin{macro}{\MakeTitle}
%    The macro |\MakeTitle| can be used as a replacement for the
%    |\maketitle| macro.
%    \begin{macrocode}
\newcommand\MakeTitle{\thispagestyle{empty}
  \rule{0pt}{.25\textheight}\par
  \mbox{}\hfill
  \begin{minipage}{\MapTextfraction\textwidth}
    \raggedright
    \rule{\textwidth}{1pt}\par
    \vspace*{5ex}%
    \sf{\huge \@title\par}%
    \vspace*{5ex}%
    \rule{\textwidth}{1pt}\par
    \vspace*{5ex}%
    \MapFont{\large \@author}	\par
    \vspace*{10ex}%
    \MapFont{\footnotesize \@date}
    \vspace*{10ex}%
  \end{minipage}%
  \par
}
%    \end{macrocode}
%    \end{macro}
%
%    The new |\maketitle| macro is activated for the class.
%
%    \begin{macrocode}
%<*class>
\let\maketitle\MakeTitle
%</class>
%    \end{macrocode}
%    
%    \subsection{Final Actions}
%
%    Load the configuration file at the end if it can be found.
%    \begin{macrocode}
\InputIfFileExists{limap.cfg}{}{}
%    \end{macrocode}
%
%    Finally we have to activate the proper settings for the choosen
%    language.
%    \begin{macrocode}
\csname LIMAP@SelectLanguage@\LIMAP@Language\endcsname
%    \end{macrocode}
%
%    \begin{macrocode}
\ifLIMAP@strict
%  \def\section{\PackageWarning{limap}{The sectioning command 
%      `section' is not available.}}
  \def\subsection{\PackageWarning{limap}{The sectioning command
      `subsection' is not available.}}
  \def\subsubsection{\PackageWarning{limap}{The sectioning command
      `subsubsection' is not available.}}
  \def\paragraph{\PackageWarning{limap}{The sectioning command
      `paragraph' is not available.}}
  \def\subparagraph{\PackageWarning{limap}{The sectioning command
      `subparagraph' is not available.}}
  \def\subsubparagraph{\PackageWarning{limap}{The sectioning command
      `subsubparagraph' is not available.}}
\fi
%    \end{macrocode}
%
%    That's all.
%
%    \Finale
%
\endinput
%
% Local Variables: 
% mode: latex
% TeX-master: "limap.dtx"
% End: 
