% \iffalse meta-comment
%
% Copyright (C) 2014 by Walter Daems <walter.daems@uantwerpen.be>
%
% This work may be distributed and/or modified under the conditions of
% the LaTeX Project Public License, either version 1.3 of this license
% or (at your option) any later version.  The latest version of this
% license is in:
% 
%    http://www.latex-project.org/lppl.txt
% 
% and version 1.3 or later is part of all distributions of LaTeX version
% 2005/12/01 or later.
%
% This work has the LPPL maintenance status `maintained'.
% 
% The Current Maintainer of this work is Walter Daems.
%
% This work consists of the files exsol.dtx and exsol.ins and the derived 
% files:
%   - exsol.sty
%   - example.tex
%   - example-solutionbook.tex
%
% \fi
%
% \iffalse
%
%<package|driver>\NeedsTeXFormat{LaTeX2e}
%<driver>\ProvidesFile{exsol.dtx}
%<package>\ProvidesPackage{exsol}
%<package|driver>  [2014/08/31 v0.91 ExSol - Exercises and Solutions package (DMW)]
%<*driver> 
\documentclass[11pt]{ltxdoc}
\usepackage[english]{babel}
\usepackage[exercisesfontsize=small]{exsol}
\usepackage{metalogo}
\EnableCrossrefs
\CodelineIndex
\RecordChanges
\usepackage{makeidx}
\usepackage{alltt}
\IfFileExists{tocbibind.sty}{\usepackage{tocbibind}}{}
\IfFileExists{hyperref.sty}{\usepackage[bookmarksopen]{hyperref}}{}
\EnableCrossrefs         
\CodelineIndex
\RecordChanges
\newcommand{\exsol}{\textsf{ExSol}}
\StopEventually{\PrintChanges\PrintIndex}
\def\fileversion{0.91}
\def\filedate{2014/08/31}
\begin{document}
 \DocInput{exsol.dtx}
\end{document}
%</driver>
% \fi
%
% \CheckSum{0}
%
% \CharacterTable
%  {Upper-case    \A\B\C\D\E\F\G\H\I\J\K\L\M\N\O\P\Q\R\S\T\U\V\W\X\Y\Z
%   Lower-case    \a\b\c\d\e\f\g\h\i\j\k\l\m\n\o\p\q\r\s\t\u\v\w\x\y\z
%   Digits        \0\1\2\3\4\5\6\7\8\9
%   Exclamation   \!     Double quote  \"     Hash (number) \#
%   Dollar        \$     Percent       \%     Ampersand     \&
%   Acute accent  \'     Left paren    \(     Right paren   \)
%   Asterisk      \*     Plus          \+     Comma         \,
%   Minus         \-     Point         \.     Solidus       \/
%   Colon         \:     Semicolon     \;     Less than     \<
%   Equals        \=     Greater than  \>     Question mark \?
%   Commercial at \@     Left bracket  \[     Backslash     \\
%   Right bracket \]     Circumflex    \^     Underscore    \_
%   Grave accent  \`     Left brace    \{     Vertical bar  \|
%   Right brace   \}     Tilde         \~}
%
%
% \changes{v0.1}{2012/01/05}{. Initial version}
% \changes{v0.2}{2012/01/06}{. Minor bug fixes based on first use by
% Paul Levrie}
% \changes{v0.3}{2012/01/07}{. Minor bug fixes based on second use by
% Paul}
% \changes{v0.4}{2012/01/09}{. Allowed for non-list formatting of
% exercises (as default)}
% \changes{v0.5}{2012/01/15}{. Added option to also send exercises to
% solutions file}
% \changes{v0.6}{2013/05/12}{. Prepared for CTAN publication}
% \changes{v0.7}{2014/07/14}{. Fixed UTF8 compatibility issues}
% \changes{v0.8}{2014/07/15}{. Fixed missing babel tag and running out
% of write hanles}
% \changes{v0.9}{2014/07/28}{. Changed default behavior
% w.r.t. minipage-wraping of exercises} 
% \changes{v0.91}{2014/08/31}{. Corrected minipage dependence, made }
%
% \DoNotIndex{\newcommand,\newenvironment}
% \setlength{\parindent}{0em}
% \addtolength{\parskip}{0.5\baselineskip}
%
% \title{The \exsol{} package\thanks{This document
%   corresponds to exsol~\fileversion, dated \filedate.}}
% \author{Walter Daems (\texttt{walter.daems@uantwerpen.be})}
% \date{}
%
% \maketitle
%
% \section{Introduction}
% %%%%%%%%%%%%%%%%%%%%%%
% The package \exsol{} provides macros to allow
% embedding exercises and solutions in the \LaTeX{} source of an
% instructional text (e.g., a book or a course text) while generating
% the following separate documents:
% \begin{itemize}
% \item your original text that only contains the exercises, and
% \item a solution book that only contains the solutions to the
% exercises (a package option exists to also copy the exercises themselves to the solution book).
% \end{itemize}
% 
% The former is generated when running \LaTeX{} on your document. This
% run writes the solutions to a secondary file that can be included
% into a simple document harness, such that when running \LaTeX{} on
% the latter, you can generate a nice solution book.
% 
% Why use \exsol{}?
% \begin{itemize}
% \item It allows to keep the \LaTeX{} source of your exercises and their
% solutions in a single file. Away with the nightmare to keep your
% solutions in sync with the original text.
% \item It separates exercises and solutions, allowing you
%   \begin{itemize}
%   \item to only release the solution book to the instructors of the
%   course;
%   \item to encourage students that you provide with the solution
%   book to first try solving the exercises without opening the book;
%   this seems to be easier than not peeking into the solution of an
%   exercise that is typeset just below the exercise itself.
%   \end{itemize}
% \end{itemize}
%
% The code of the \exsol{} package was taken almost literally
% from \textsf{fancyvrb} \cite{fancyvrb}. Therefore, all credits go to the
% authors/maintainers of \textsf{fancyvrb}.
%
% Thanks to Pieter Pareit and Pekka Pere for signaling problems and
% making suggestions for the documentation.
%
% \section{Installation}
% %%%%%%%%%%%%%%%%%%%%%%
% Either you are a package manager and then you'll know how to
% prepare an installation package for \exsol{}.
%
% Either you are a normal user and then you have two options. First,
% check if there is a package that your favorite \LaTeX{}
% distributor has prepared for you. Second, grab the TDS package
% from CTAN \cite{CTAN} (\texttt{exsol.tds.zip}) and unzip it somewhere in your
% own TDS tree, regenerate your filename database and off you go.
% In any case, make sure that \LaTeX{} finds the \texttt{exsol.sty} file.
%
% The \exsol{} package uses some auxiliary packages: \textsf{fancyvrb},
% \textsf{ifthen}, \textsf{kvoptions} and, optionally,
% \textsf{babel}. Fetch them from CTAN \cite{CTAN} if your \TeX{}
% distributor does not provide them.
%
% \section{Usage}
% %%%%%%%%%%%%%%%
% 
% \subsection{Preparing your document source}
% %%%%%%%%%%%%%%%%%%%%%%%%%%%%%%%%%%%%%%%%%%%
% The macro package exsol can be loaded with:
% \begin{verbatim}
% \usepackage{exsol}
% \end{verbatim}
%
% Then, you are ready to add some exercises including their solution
% to your document source. To this end, embed them in a
% \texttt{exercise} and a corresponding \texttt{solution} environment.
% Optionally, you may embed several of them in a \texttt{exercises}
% environment, to make them stand out in your text.
% E.g.,
%
% \begin{VerbatimOut}{exsol.tmp}
% 
% \begin{exercises}
%
%   \begin{exercise}
%     Solve the following equation for $x \in C$, with $C$ the set of
%     complex numbers:
%     \begin{equation}
%       5 x^2 -3 x = 5
%     \end{equation}
%   \end{exercise}
%   \begin{solution}
%     Let's start by rearranging the equation, a bit:
%     \begin{eqnarray}
%       5.7 x^2 - 3.1 x &=& 5.3\\
%       5.7 x^2 - 3.1 x -5.3 &=& 0
%     \end{eqnarray}
%     The equation is now in the standard form:
%     \begin{equation}
%       a x^2 + b x + c = 0
%     \end{equation}
%     For quadratic equations in the standard form, we know that two
%     solutions exist:
%     \begin{equation}
%       x_{1,2} = \frac{ -b \pm \sqrt{d}}{2a}
%     \end{equation}
%     with
%     \begin{equation}
%       d = b^2 - 4 a c
%     \end{equation}
%     If we apply this to our case, we obtain:
%     \begin{equation}
%       d = (-3.1)^2 - 4 \cdot 5.7 \cdot (-5.3) = 130.45
%     \end{equation}
%     and
%     \begin{eqnarray}
%       x_1 &=& \frac{3.1 + \sqrt{130.45}}{11.4} = 1.27\\
%       x_2 &=& \frac{3.1 - \sqrt{130.45}}{11.4} = -0.73
%     \end{eqnarray}
%     The proposed values $x = x_1, x_2$ are solutions to the given equation.
%   \end{solution}
%   \begin{exercise}
%     Consider a 2-dimensional vector space equipped with a Euclidean
%     distance function. Given a right-angled triangle, with the sides
%     $A$ and $B$ adjacent to the right angle having lengths, $3$ and
%     $4$, calculate the length of the hypotenuse, labeled $C$.
%   \end{exercise}
%   \begin{solution}
%     This calls for application of Pythagoras' theorem, which 
%     tells us:
%     \begin{equation}
%       \left\|A\right\|^2 + \left\|B\right\|^2 = \left\|C\right\|^2
%     \end{equation}
%     and therefore:
%     \begin{eqnarray}
%       \left\|C\right\| 
%       &=& \sqrt{\left\|A\right\|^2 + \left\|B\right\|^2}\\
%       &=& \sqrt{3^2 + 4^2}\\
%       &=& \sqrt{25} = 5
%     \end{eqnarray}
%     Therefore, the length of the hypotenuse equals $5$.
%   \end{solution}
%
% \end{exercises}
% \end{VerbatimOut}
% \VerbatimInput[frame=lines,gobble=2,fontsize=\footnotesize]{exsol.tmp}
%
% The result in the original document, can be seen below. As you can
% see, there's no trace of the solution. 
%
% % \iffalse meta-comment
%
% Copyright (C) 2014 by Walter Daems <walter.daems@uantwerpen.be>
%
% This work may be distributed and/or modified under the conditions of
% the LaTeX Project Public License, either version 1.3 of this license
% or (at your option) any later version.  The latest version of this
% license is in:
% 
%    http://www.latex-project.org/lppl.txt
% 
% and version 1.3 or later is part of all distributions of LaTeX version
% 2005/12/01 or later.
%
% This work has the LPPL maintenance status `maintained'.
% 
% The Current Maintainer of this work is Walter Daems.
%
% This work consists of the files exsol.dtx and exsol.ins and the derived 
% files:
%   - exsol.sty
%   - example.tex
%   - example-solutionbook.tex
%
% \fi
%
% \iffalse
%
%<package|driver>\NeedsTeXFormat{LaTeX2e}
%<driver>\ProvidesFile{exsol.dtx}
%<package>\ProvidesPackage{exsol}
%<package|driver>  [2014/08/31 v0.91 ExSol - Exercises and Solutions package (DMW)]
%<*driver> 
\documentclass[11pt]{ltxdoc}
\usepackage[english]{babel}
\usepackage[exercisesfontsize=small]{exsol}
\usepackage{metalogo}
\EnableCrossrefs
\CodelineIndex
\RecordChanges
\usepackage{makeidx}
\usepackage{alltt}
\IfFileExists{tocbibind.sty}{\usepackage{tocbibind}}{}
\IfFileExists{hyperref.sty}{\usepackage[bookmarksopen]{hyperref}}{}
\EnableCrossrefs         
\CodelineIndex
\RecordChanges
\newcommand{\exsol}{\textsf{ExSol}}
\StopEventually{\PrintChanges\PrintIndex}
\def\fileversion{0.91}
\def\filedate{2014/08/31}
\begin{document}
 \DocInput{exsol.dtx}
\end{document}
%</driver>
% \fi
%
% \CheckSum{0}
%
% \CharacterTable
%  {Upper-case    \A\B\C\D\E\F\G\H\I\J\K\L\M\N\O\P\Q\R\S\T\U\V\W\X\Y\Z
%   Lower-case    \a\b\c\d\e\f\g\h\i\j\k\l\m\n\o\p\q\r\s\t\u\v\w\x\y\z
%   Digits        \0\1\2\3\4\5\6\7\8\9
%   Exclamation   \!     Double quote  \"     Hash (number) \#
%   Dollar        \$     Percent       \%     Ampersand     \&
%   Acute accent  \'     Left paren    \(     Right paren   \)
%   Asterisk      \*     Plus          \+     Comma         \,
%   Minus         \-     Point         \.     Solidus       \/
%   Colon         \:     Semicolon     \;     Less than     \<
%   Equals        \=     Greater than  \>     Question mark \?
%   Commercial at \@     Left bracket  \[     Backslash     \\
%   Right bracket \]     Circumflex    \^     Underscore    \_
%   Grave accent  \`     Left brace    \{     Vertical bar  \|
%   Right brace   \}     Tilde         \~}
%
%
% \changes{v0.1}{2012/01/05}{. Initial version}
% \changes{v0.2}{2012/01/06}{. Minor bug fixes based on first use by
% Paul Levrie}
% \changes{v0.3}{2012/01/07}{. Minor bug fixes based on second use by
% Paul}
% \changes{v0.4}{2012/01/09}{. Allowed for non-list formatting of
% exercises (as default)}
% \changes{v0.5}{2012/01/15}{. Added option to also send exercises to
% solutions file}
% \changes{v0.6}{2013/05/12}{. Prepared for CTAN publication}
% \changes{v0.7}{2014/07/14}{. Fixed UTF8 compatibility issues}
% \changes{v0.8}{2014/07/15}{. Fixed missing babel tag and running out
% of write hanles}
% \changes{v0.9}{2014/07/28}{. Changed default behavior
% w.r.t. minipage-wraping of exercises} 
% \changes{v0.91}{2014/08/31}{. Corrected minipage dependence, made }
%
% \DoNotIndex{\newcommand,\newenvironment}
% \setlength{\parindent}{0em}
% \addtolength{\parskip}{0.5\baselineskip}
%
% \title{The \exsol{} package\thanks{This document
%   corresponds to exsol~\fileversion, dated \filedate.}}
% \author{Walter Daems (\texttt{walter.daems@uantwerpen.be})}
% \date{}
%
% \maketitle
%
% \section{Introduction}
% %%%%%%%%%%%%%%%%%%%%%%
% The package \exsol{} provides macros to allow
% embedding exercises and solutions in the \LaTeX{} source of an
% instructional text (e.g., a book or a course text) while generating
% the following separate documents:
% \begin{itemize}
% \item your original text that only contains the exercises, and
% \item a solution book that only contains the solutions to the
% exercises (a package option exists to also copy the exercises themselves to the solution book).
% \end{itemize}
% 
% The former is generated when running \LaTeX{} on your document. This
% run writes the solutions to a secondary file that can be included
% into a simple document harness, such that when running \LaTeX{} on
% the latter, you can generate a nice solution book.
% 
% Why use \exsol{}?
% \begin{itemize}
% \item It allows to keep the \LaTeX{} source of your exercises and their
% solutions in a single file. Away with the nightmare to keep your
% solutions in sync with the original text.
% \item It separates exercises and solutions, allowing you
%   \begin{itemize}
%   \item to only release the solution book to the instructors of the
%   course;
%   \item to encourage students that you provide with the solution
%   book to first try solving the exercises without opening the book;
%   this seems to be easier than not peeking into the solution of an
%   exercise that is typeset just below the exercise itself.
%   \end{itemize}
% \end{itemize}
%
% The code of the \exsol{} package was taken almost literally
% from \textsf{fancyvrb} \cite{fancyvrb}. Therefore, all credits go to the
% authors/maintainers of \textsf{fancyvrb}.
%
% Thanks to Pieter Pareit and Pekka Pere for signaling problems and
% making suggestions for the documentation.
%
% \section{Installation}
% %%%%%%%%%%%%%%%%%%%%%%
% Either you are a package manager and then you'll know how to
% prepare an installation package for \exsol{}.
%
% Either you are a normal user and then you have two options. First,
% check if there is a package that your favorite \LaTeX{}
% distributor has prepared for you. Second, grab the TDS package
% from CTAN \cite{CTAN} (\texttt{exsol.tds.zip}) and unzip it somewhere in your
% own TDS tree, regenerate your filename database and off you go.
% In any case, make sure that \LaTeX{} finds the \texttt{exsol.sty} file.
%
% The \exsol{} package uses some auxiliary packages: \textsf{fancyvrb},
% \textsf{ifthen}, \textsf{kvoptions} and, optionally,
% \textsf{babel}. Fetch them from CTAN \cite{CTAN} if your \TeX{}
% distributor does not provide them.
%
% \section{Usage}
% %%%%%%%%%%%%%%%
% 
% \subsection{Preparing your document source}
% %%%%%%%%%%%%%%%%%%%%%%%%%%%%%%%%%%%%%%%%%%%
% The macro package exsol can be loaded with:
% \begin{verbatim}
% \usepackage{exsol}
% \end{verbatim}
%
% Then, you are ready to add some exercises including their solution
% to your document source. To this end, embed them in a
% \texttt{exercise} and a corresponding \texttt{solution} environment.
% Optionally, you may embed several of them in a \texttt{exercises}
% environment, to make them stand out in your text.
% E.g.,
%
% \begin{VerbatimOut}{exsol.tmp}
% 
% \begin{exercises}
%
%   \begin{exercise}
%     Solve the following equation for $x \in C$, with $C$ the set of
%     complex numbers:
%     \begin{equation}
%       5 x^2 -3 x = 5
%     \end{equation}
%   \end{exercise}
%   \begin{solution}
%     Let's start by rearranging the equation, a bit:
%     \begin{eqnarray}
%       5.7 x^2 - 3.1 x &=& 5.3\\
%       5.7 x^2 - 3.1 x -5.3 &=& 0
%     \end{eqnarray}
%     The equation is now in the standard form:
%     \begin{equation}
%       a x^2 + b x + c = 0
%     \end{equation}
%     For quadratic equations in the standard form, we know that two
%     solutions exist:
%     \begin{equation}
%       x_{1,2} = \frac{ -b \pm \sqrt{d}}{2a}
%     \end{equation}
%     with
%     \begin{equation}
%       d = b^2 - 4 a c
%     \end{equation}
%     If we apply this to our case, we obtain:
%     \begin{equation}
%       d = (-3.1)^2 - 4 \cdot 5.7 \cdot (-5.3) = 130.45
%     \end{equation}
%     and
%     \begin{eqnarray}
%       x_1 &=& \frac{3.1 + \sqrt{130.45}}{11.4} = 1.27\\
%       x_2 &=& \frac{3.1 - \sqrt{130.45}}{11.4} = -0.73
%     \end{eqnarray}
%     The proposed values $x = x_1, x_2$ are solutions to the given equation.
%   \end{solution}
%   \begin{exercise}
%     Consider a 2-dimensional vector space equipped with a Euclidean
%     distance function. Given a right-angled triangle, with the sides
%     $A$ and $B$ adjacent to the right angle having lengths, $3$ and
%     $4$, calculate the length of the hypotenuse, labeled $C$.
%   \end{exercise}
%   \begin{solution}
%     This calls for application of Pythagoras' theorem, which 
%     tells us:
%     \begin{equation}
%       \left\|A\right\|^2 + \left\|B\right\|^2 = \left\|C\right\|^2
%     \end{equation}
%     and therefore:
%     \begin{eqnarray}
%       \left\|C\right\| 
%       &=& \sqrt{\left\|A\right\|^2 + \left\|B\right\|^2}\\
%       &=& \sqrt{3^2 + 4^2}\\
%       &=& \sqrt{25} = 5
%     \end{eqnarray}
%     Therefore, the length of the hypotenuse equals $5$.
%   \end{solution}
%
% \end{exercises}
% \end{VerbatimOut}
% \VerbatimInput[frame=lines,gobble=2,fontsize=\footnotesize]{exsol.tmp}
%
% The result in the original document, can be seen below. As you can
% see, there's no trace of the solution. 
%
% % \iffalse meta-comment
%
% Copyright (C) 2014 by Walter Daems <walter.daems@uantwerpen.be>
%
% This work may be distributed and/or modified under the conditions of
% the LaTeX Project Public License, either version 1.3 of this license
% or (at your option) any later version.  The latest version of this
% license is in:
% 
%    http://www.latex-project.org/lppl.txt
% 
% and version 1.3 or later is part of all distributions of LaTeX version
% 2005/12/01 or later.
%
% This work has the LPPL maintenance status `maintained'.
% 
% The Current Maintainer of this work is Walter Daems.
%
% This work consists of the files exsol.dtx and exsol.ins and the derived 
% files:
%   - exsol.sty
%   - example.tex
%   - example-solutionbook.tex
%
% \fi
%
% \iffalse
%
%<package|driver>\NeedsTeXFormat{LaTeX2e}
%<driver>\ProvidesFile{exsol.dtx}
%<package>\ProvidesPackage{exsol}
%<package|driver>  [2014/08/31 v0.91 ExSol - Exercises and Solutions package (DMW)]
%<*driver> 
\documentclass[11pt]{ltxdoc}
\usepackage[english]{babel}
\usepackage[exercisesfontsize=small]{exsol}
\usepackage{metalogo}
\EnableCrossrefs
\CodelineIndex
\RecordChanges
\usepackage{makeidx}
\usepackage{alltt}
\IfFileExists{tocbibind.sty}{\usepackage{tocbibind}}{}
\IfFileExists{hyperref.sty}{\usepackage[bookmarksopen]{hyperref}}{}
\EnableCrossrefs         
\CodelineIndex
\RecordChanges
\newcommand{\exsol}{\textsf{ExSol}}
\StopEventually{\PrintChanges\PrintIndex}
\def\fileversion{0.91}
\def\filedate{2014/08/31}
\begin{document}
 \DocInput{exsol.dtx}
\end{document}
%</driver>
% \fi
%
% \CheckSum{0}
%
% \CharacterTable
%  {Upper-case    \A\B\C\D\E\F\G\H\I\J\K\L\M\N\O\P\Q\R\S\T\U\V\W\X\Y\Z
%   Lower-case    \a\b\c\d\e\f\g\h\i\j\k\l\m\n\o\p\q\r\s\t\u\v\w\x\y\z
%   Digits        \0\1\2\3\4\5\6\7\8\9
%   Exclamation   \!     Double quote  \"     Hash (number) \#
%   Dollar        \$     Percent       \%     Ampersand     \&
%   Acute accent  \'     Left paren    \(     Right paren   \)
%   Asterisk      \*     Plus          \+     Comma         \,
%   Minus         \-     Point         \.     Solidus       \/
%   Colon         \:     Semicolon     \;     Less than     \<
%   Equals        \=     Greater than  \>     Question mark \?
%   Commercial at \@     Left bracket  \[     Backslash     \\
%   Right bracket \]     Circumflex    \^     Underscore    \_
%   Grave accent  \`     Left brace    \{     Vertical bar  \|
%   Right brace   \}     Tilde         \~}
%
%
% \changes{v0.1}{2012/01/05}{. Initial version}
% \changes{v0.2}{2012/01/06}{. Minor bug fixes based on first use by
% Paul Levrie}
% \changes{v0.3}{2012/01/07}{. Minor bug fixes based on second use by
% Paul}
% \changes{v0.4}{2012/01/09}{. Allowed for non-list formatting of
% exercises (as default)}
% \changes{v0.5}{2012/01/15}{. Added option to also send exercises to
% solutions file}
% \changes{v0.6}{2013/05/12}{. Prepared for CTAN publication}
% \changes{v0.7}{2014/07/14}{. Fixed UTF8 compatibility issues}
% \changes{v0.8}{2014/07/15}{. Fixed missing babel tag and running out
% of write hanles}
% \changes{v0.9}{2014/07/28}{. Changed default behavior
% w.r.t. minipage-wraping of exercises} 
% \changes{v0.91}{2014/08/31}{. Corrected minipage dependence, made }
%
% \DoNotIndex{\newcommand,\newenvironment}
% \setlength{\parindent}{0em}
% \addtolength{\parskip}{0.5\baselineskip}
%
% \title{The \exsol{} package\thanks{This document
%   corresponds to exsol~\fileversion, dated \filedate.}}
% \author{Walter Daems (\texttt{walter.daems@uantwerpen.be})}
% \date{}
%
% \maketitle
%
% \section{Introduction}
% %%%%%%%%%%%%%%%%%%%%%%
% The package \exsol{} provides macros to allow
% embedding exercises and solutions in the \LaTeX{} source of an
% instructional text (e.g., a book or a course text) while generating
% the following separate documents:
% \begin{itemize}
% \item your original text that only contains the exercises, and
% \item a solution book that only contains the solutions to the
% exercises (a package option exists to also copy the exercises themselves to the solution book).
% \end{itemize}
% 
% The former is generated when running \LaTeX{} on your document. This
% run writes the solutions to a secondary file that can be included
% into a simple document harness, such that when running \LaTeX{} on
% the latter, you can generate a nice solution book.
% 
% Why use \exsol{}?
% \begin{itemize}
% \item It allows to keep the \LaTeX{} source of your exercises and their
% solutions in a single file. Away with the nightmare to keep your
% solutions in sync with the original text.
% \item It separates exercises and solutions, allowing you
%   \begin{itemize}
%   \item to only release the solution book to the instructors of the
%   course;
%   \item to encourage students that you provide with the solution
%   book to first try solving the exercises without opening the book;
%   this seems to be easier than not peeking into the solution of an
%   exercise that is typeset just below the exercise itself.
%   \end{itemize}
% \end{itemize}
%
% The code of the \exsol{} package was taken almost literally
% from \textsf{fancyvrb} \cite{fancyvrb}. Therefore, all credits go to the
% authors/maintainers of \textsf{fancyvrb}.
%
% Thanks to Pieter Pareit and Pekka Pere for signaling problems and
% making suggestions for the documentation.
%
% \section{Installation}
% %%%%%%%%%%%%%%%%%%%%%%
% Either you are a package manager and then you'll know how to
% prepare an installation package for \exsol{}.
%
% Either you are a normal user and then you have two options. First,
% check if there is a package that your favorite \LaTeX{}
% distributor has prepared for you. Second, grab the TDS package
% from CTAN \cite{CTAN} (\texttt{exsol.tds.zip}) and unzip it somewhere in your
% own TDS tree, regenerate your filename database and off you go.
% In any case, make sure that \LaTeX{} finds the \texttt{exsol.sty} file.
%
% The \exsol{} package uses some auxiliary packages: \textsf{fancyvrb},
% \textsf{ifthen}, \textsf{kvoptions} and, optionally,
% \textsf{babel}. Fetch them from CTAN \cite{CTAN} if your \TeX{}
% distributor does not provide them.
%
% \section{Usage}
% %%%%%%%%%%%%%%%
% 
% \subsection{Preparing your document source}
% %%%%%%%%%%%%%%%%%%%%%%%%%%%%%%%%%%%%%%%%%%%
% The macro package exsol can be loaded with:
% \begin{verbatim}
% \usepackage{exsol}
% \end{verbatim}
%
% Then, you are ready to add some exercises including their solution
% to your document source. To this end, embed them in a
% \texttt{exercise} and a corresponding \texttt{solution} environment.
% Optionally, you may embed several of them in a \texttt{exercises}
% environment, to make them stand out in your text.
% E.g.,
%
% \begin{VerbatimOut}{exsol.tmp}
% 
% \begin{exercises}
%
%   \begin{exercise}
%     Solve the following equation for $x \in C$, with $C$ the set of
%     complex numbers:
%     \begin{equation}
%       5 x^2 -3 x = 5
%     \end{equation}
%   \end{exercise}
%   \begin{solution}
%     Let's start by rearranging the equation, a bit:
%     \begin{eqnarray}
%       5.7 x^2 - 3.1 x &=& 5.3\\
%       5.7 x^2 - 3.1 x -5.3 &=& 0
%     \end{eqnarray}
%     The equation is now in the standard form:
%     \begin{equation}
%       a x^2 + b x + c = 0
%     \end{equation}
%     For quadratic equations in the standard form, we know that two
%     solutions exist:
%     \begin{equation}
%       x_{1,2} = \frac{ -b \pm \sqrt{d}}{2a}
%     \end{equation}
%     with
%     \begin{equation}
%       d = b^2 - 4 a c
%     \end{equation}
%     If we apply this to our case, we obtain:
%     \begin{equation}
%       d = (-3.1)^2 - 4 \cdot 5.7 \cdot (-5.3) = 130.45
%     \end{equation}
%     and
%     \begin{eqnarray}
%       x_1 &=& \frac{3.1 + \sqrt{130.45}}{11.4} = 1.27\\
%       x_2 &=& \frac{3.1 - \sqrt{130.45}}{11.4} = -0.73
%     \end{eqnarray}
%     The proposed values $x = x_1, x_2$ are solutions to the given equation.
%   \end{solution}
%   \begin{exercise}
%     Consider a 2-dimensional vector space equipped with a Euclidean
%     distance function. Given a right-angled triangle, with the sides
%     $A$ and $B$ adjacent to the right angle having lengths, $3$ and
%     $4$, calculate the length of the hypotenuse, labeled $C$.
%   \end{exercise}
%   \begin{solution}
%     This calls for application of Pythagoras' theorem, which 
%     tells us:
%     \begin{equation}
%       \left\|A\right\|^2 + \left\|B\right\|^2 = \left\|C\right\|^2
%     \end{equation}
%     and therefore:
%     \begin{eqnarray}
%       \left\|C\right\| 
%       &=& \sqrt{\left\|A\right\|^2 + \left\|B\right\|^2}\\
%       &=& \sqrt{3^2 + 4^2}\\
%       &=& \sqrt{25} = 5
%     \end{eqnarray}
%     Therefore, the length of the hypotenuse equals $5$.
%   \end{solution}
%
% \end{exercises}
% \end{VerbatimOut}
% \VerbatimInput[frame=lines,gobble=2,fontsize=\footnotesize]{exsol.tmp}
%
% The result in the original document, can be seen below. As you can
% see, there's no trace of the solution. 
%
% % \iffalse meta-comment
%
% Copyright (C) 2014 by Walter Daems <walter.daems@uantwerpen.be>
%
% This work may be distributed and/or modified under the conditions of
% the LaTeX Project Public License, either version 1.3 of this license
% or (at your option) any later version.  The latest version of this
% license is in:
% 
%    http://www.latex-project.org/lppl.txt
% 
% and version 1.3 or later is part of all distributions of LaTeX version
% 2005/12/01 or later.
%
% This work has the LPPL maintenance status `maintained'.
% 
% The Current Maintainer of this work is Walter Daems.
%
% This work consists of the files exsol.dtx and exsol.ins and the derived 
% files:
%   - exsol.sty
%   - example.tex
%   - example-solutionbook.tex
%
% \fi
%
% \iffalse
%
%<package|driver>\NeedsTeXFormat{LaTeX2e}
%<driver>\ProvidesFile{exsol.dtx}
%<package>\ProvidesPackage{exsol}
%<package|driver>  [2014/08/31 v0.91 ExSol - Exercises and Solutions package (DMW)]
%<*driver> 
\documentclass[11pt]{ltxdoc}
\usepackage[english]{babel}
\usepackage[exercisesfontsize=small]{exsol}
\usepackage{metalogo}
\EnableCrossrefs
\CodelineIndex
\RecordChanges
\usepackage{makeidx}
\usepackage{alltt}
\IfFileExists{tocbibind.sty}{\usepackage{tocbibind}}{}
\IfFileExists{hyperref.sty}{\usepackage[bookmarksopen]{hyperref}}{}
\EnableCrossrefs         
\CodelineIndex
\RecordChanges
\newcommand{\exsol}{\textsf{ExSol}}
\StopEventually{\PrintChanges\PrintIndex}
\def\fileversion{0.91}
\def\filedate{2014/08/31}
\begin{document}
 \DocInput{exsol.dtx}
\end{document}
%</driver>
% \fi
%
% \CheckSum{0}
%
% \CharacterTable
%  {Upper-case    \A\B\C\D\E\F\G\H\I\J\K\L\M\N\O\P\Q\R\S\T\U\V\W\X\Y\Z
%   Lower-case    \a\b\c\d\e\f\g\h\i\j\k\l\m\n\o\p\q\r\s\t\u\v\w\x\y\z
%   Digits        \0\1\2\3\4\5\6\7\8\9
%   Exclamation   \!     Double quote  \"     Hash (number) \#
%   Dollar        \$     Percent       \%     Ampersand     \&
%   Acute accent  \'     Left paren    \(     Right paren   \)
%   Asterisk      \*     Plus          \+     Comma         \,
%   Minus         \-     Point         \.     Solidus       \/
%   Colon         \:     Semicolon     \;     Less than     \<
%   Equals        \=     Greater than  \>     Question mark \?
%   Commercial at \@     Left bracket  \[     Backslash     \\
%   Right bracket \]     Circumflex    \^     Underscore    \_
%   Grave accent  \`     Left brace    \{     Vertical bar  \|
%   Right brace   \}     Tilde         \~}
%
%
% \changes{v0.1}{2012/01/05}{. Initial version}
% \changes{v0.2}{2012/01/06}{. Minor bug fixes based on first use by
% Paul Levrie}
% \changes{v0.3}{2012/01/07}{. Minor bug fixes based on second use by
% Paul}
% \changes{v0.4}{2012/01/09}{. Allowed for non-list formatting of
% exercises (as default)}
% \changes{v0.5}{2012/01/15}{. Added option to also send exercises to
% solutions file}
% \changes{v0.6}{2013/05/12}{. Prepared for CTAN publication}
% \changes{v0.7}{2014/07/14}{. Fixed UTF8 compatibility issues}
% \changes{v0.8}{2014/07/15}{. Fixed missing babel tag and running out
% of write hanles}
% \changes{v0.9}{2014/07/28}{. Changed default behavior
% w.r.t. minipage-wraping of exercises} 
% \changes{v0.91}{2014/08/31}{. Corrected minipage dependence, made }
%
% \DoNotIndex{\newcommand,\newenvironment}
% \setlength{\parindent}{0em}
% \addtolength{\parskip}{0.5\baselineskip}
%
% \title{The \exsol{} package\thanks{This document
%   corresponds to exsol~\fileversion, dated \filedate.}}
% \author{Walter Daems (\texttt{walter.daems@uantwerpen.be})}
% \date{}
%
% \maketitle
%
% \section{Introduction}
% %%%%%%%%%%%%%%%%%%%%%%
% The package \exsol{} provides macros to allow
% embedding exercises and solutions in the \LaTeX{} source of an
% instructional text (e.g., a book or a course text) while generating
% the following separate documents:
% \begin{itemize}
% \item your original text that only contains the exercises, and
% \item a solution book that only contains the solutions to the
% exercises (a package option exists to also copy the exercises themselves to the solution book).
% \end{itemize}
% 
% The former is generated when running \LaTeX{} on your document. This
% run writes the solutions to a secondary file that can be included
% into a simple document harness, such that when running \LaTeX{} on
% the latter, you can generate a nice solution book.
% 
% Why use \exsol{}?
% \begin{itemize}
% \item It allows to keep the \LaTeX{} source of your exercises and their
% solutions in a single file. Away with the nightmare to keep your
% solutions in sync with the original text.
% \item It separates exercises and solutions, allowing you
%   \begin{itemize}
%   \item to only release the solution book to the instructors of the
%   course;
%   \item to encourage students that you provide with the solution
%   book to first try solving the exercises without opening the book;
%   this seems to be easier than not peeking into the solution of an
%   exercise that is typeset just below the exercise itself.
%   \end{itemize}
% \end{itemize}
%
% The code of the \exsol{} package was taken almost literally
% from \textsf{fancyvrb} \cite{fancyvrb}. Therefore, all credits go to the
% authors/maintainers of \textsf{fancyvrb}.
%
% Thanks to Pieter Pareit and Pekka Pere for signaling problems and
% making suggestions for the documentation.
%
% \section{Installation}
% %%%%%%%%%%%%%%%%%%%%%%
% Either you are a package manager and then you'll know how to
% prepare an installation package for \exsol{}.
%
% Either you are a normal user and then you have two options. First,
% check if there is a package that your favorite \LaTeX{}
% distributor has prepared for you. Second, grab the TDS package
% from CTAN \cite{CTAN} (\texttt{exsol.tds.zip}) and unzip it somewhere in your
% own TDS tree, regenerate your filename database and off you go.
% In any case, make sure that \LaTeX{} finds the \texttt{exsol.sty} file.
%
% The \exsol{} package uses some auxiliary packages: \textsf{fancyvrb},
% \textsf{ifthen}, \textsf{kvoptions} and, optionally,
% \textsf{babel}. Fetch them from CTAN \cite{CTAN} if your \TeX{}
% distributor does not provide them.
%
% \section{Usage}
% %%%%%%%%%%%%%%%
% 
% \subsection{Preparing your document source}
% %%%%%%%%%%%%%%%%%%%%%%%%%%%%%%%%%%%%%%%%%%%
% The macro package exsol can be loaded with:
% \begin{verbatim}
% \usepackage{exsol}
% \end{verbatim}
%
% Then, you are ready to add some exercises including their solution
% to your document source. To this end, embed them in a
% \texttt{exercise} and a corresponding \texttt{solution} environment.
% Optionally, you may embed several of them in a \texttt{exercises}
% environment, to make them stand out in your text.
% E.g.,
%
% \begin{VerbatimOut}{exsol.tmp}
% 
% \begin{exercises}
%
%   \begin{exercise}
%     Solve the following equation for $x \in C$, with $C$ the set of
%     complex numbers:
%     \begin{equation}
%       5 x^2 -3 x = 5
%     \end{equation}
%   \end{exercise}
%   \begin{solution}
%     Let's start by rearranging the equation, a bit:
%     \begin{eqnarray}
%       5.7 x^2 - 3.1 x &=& 5.3\\
%       5.7 x^2 - 3.1 x -5.3 &=& 0
%     \end{eqnarray}
%     The equation is now in the standard form:
%     \begin{equation}
%       a x^2 + b x + c = 0
%     \end{equation}
%     For quadratic equations in the standard form, we know that two
%     solutions exist:
%     \begin{equation}
%       x_{1,2} = \frac{ -b \pm \sqrt{d}}{2a}
%     \end{equation}
%     with
%     \begin{equation}
%       d = b^2 - 4 a c
%     \end{equation}
%     If we apply this to our case, we obtain:
%     \begin{equation}
%       d = (-3.1)^2 - 4 \cdot 5.7 \cdot (-5.3) = 130.45
%     \end{equation}
%     and
%     \begin{eqnarray}
%       x_1 &=& \frac{3.1 + \sqrt{130.45}}{11.4} = 1.27\\
%       x_2 &=& \frac{3.1 - \sqrt{130.45}}{11.4} = -0.73
%     \end{eqnarray}
%     The proposed values $x = x_1, x_2$ are solutions to the given equation.
%   \end{solution}
%   \begin{exercise}
%     Consider a 2-dimensional vector space equipped with a Euclidean
%     distance function. Given a right-angled triangle, with the sides
%     $A$ and $B$ adjacent to the right angle having lengths, $3$ and
%     $4$, calculate the length of the hypotenuse, labeled $C$.
%   \end{exercise}
%   \begin{solution}
%     This calls for application of Pythagoras' theorem, which 
%     tells us:
%     \begin{equation}
%       \left\|A\right\|^2 + \left\|B\right\|^2 = \left\|C\right\|^2
%     \end{equation}
%     and therefore:
%     \begin{eqnarray}
%       \left\|C\right\| 
%       &=& \sqrt{\left\|A\right\|^2 + \left\|B\right\|^2}\\
%       &=& \sqrt{3^2 + 4^2}\\
%       &=& \sqrt{25} = 5
%     \end{eqnarray}
%     Therefore, the length of the hypotenuse equals $5$.
%   \end{solution}
%
% \end{exercises}
% \end{VerbatimOut}
% \VerbatimInput[frame=lines,gobble=2,fontsize=\footnotesize]{exsol.tmp}
%
% The result in the original document, can be seen below. As you can
% see, there's no trace of the solution. 
%
% \input{exsol.tmp}
%
% When running \LaTeX{} on your document (in our case on the
% \texttt{exsol.dtx} file, as a side effect a file with extension
% \texttt{.sol.tex} has been written to disk (in our case, the file
% \texttt{exsol.sol.tex}), containing all solutions in sequence.
%
% Generating a solution book is a simple as including the file into a
% simple \LaTeX{} harness, that allows you giving it a proper title page and to
% add other bells and whistles.
%
% E.g.,
% \begin{VerbatimOut}{exsol-solutionbook.tex}
% \documentclass{article}
% \usepackage[english]{babel}
% \title{Solutions to the exercises, specified in the \textsf{ExSol} package}
% \author{Walter Daems}
% \date{2013/05/12}
%
% \begin{document}
%
% \maketitle
%
% \input{exsol.sol}
%
% \end{document}
% \end{VerbatimOut}
% \VerbatimInput[frame=lines,gobble=2,fontsize=\footnotesize]{exsol-solutionbook.tex}
% 
% You may generate this solution book, by running \LaTeX{} on the
% file named \texttt{exsol-solutionbook.tex} that is generated when running
% \LaTeX{} on the \texttt{exsol.dtx} file.
%
% The result approximately looks like this:
%
% \setcounter{equation}{0}
% \rule{\linewidth}{.7pt}
% \begin{center}
% {\Large Solutions to the exercises, specified in the \textsf{ExSol} package}\\
% {\large Walter Daems}\\
% {\large 2013/05/12}
% \end{center}
% \par---\newline\textbf{Solution 3.1-1}
%     Let's start by rearranging the equation, a bit:
%     \begin{eqnarray}
%       5.7 x^2 - 3.1 x &=% 5.3\\
%       5.7 x^2 - 3.1 x -5.3 &=% 0
%     \end{eqnarray}
%     The equation is now in the standard form:
%     \begin{equation}
%       a x^2 + b x + c = 0
%     \end{equation}
%     For quadratic equations in the standard form, we know that two
%     solutions exist:
%     \begin{equation}
%       x_{1,2} = \frac{ -b \pm \sqrt{d}}{2a}
%     \end{equation}
%     with
%     \begin{equation}
%       d = b^2 - 4 a c
%     \end{equation}
%     If we apply this to our case, we obtain:
%     \begin{equation}
%       d = (-3.1)^2 - 4 \cdot 5.7 \cdot (-5.3) = 130.45
%     \end{equation}
%     and
%     \begin{eqnarray}
%       x_1 &=& \frac{3.1 + \sqrt{130.45}}{11.4} = 1.27\\
%       x_2 &=& \frac{3.1 + \sqrt{130.45}}{11.4} = -0.73
%     \end{eqnarray}
%     The proposed values $x = x_1, x_2$ are solutions to
%     the given equation.
% \par---\newline\textbf{Solution 3.1-2}
%       This calls for application of Pythagoras' theorem, which
%       tells us:
%       \begin{equation}
%         \left\|A\right\|^2 + \left\|B\right\|^2 = \left\|C\right\|^2
%       \end{equation}
%       and therefore:
%       \begin{eqnarray}
%         \left\|C\right\|
%         &=& \sqrt{\left\|A\right\|^2 + \left\|B\right\|^2}\\
%         &=& \sqrt{3^2 + 4^2}\\
%         &=& \sqrt{25} = 5
%       \end{eqnarray}
%       Therefore, the length of the hypotenuse equals $5$.
%
% \rule{\linewidth}{.7pt}
%
% \subsection{Fiddling with the spacing}
%
% The default spacing provided by the \textsf{ExSol} package should be
% fine for most users. However, if you like to tweak, below you can
% find the controls.
% \subsubsection{Spacing before and after the \texttt{exercises} environment}
%
% The lengths below control the spacing of the |exercises| environment:
% \begin{itemize}
% \item |exsolexerciseaboveskip|: rubber length controlling the
% vertical space after the top marker line of the environment
% \item |exsolexercisebelowskip|: rubber length controlling the
% vertical space before the bottom marker line of the environment
% \end{itemize}
%
% You can simply specify them like:
% \begin{verbatim}
% \setlength{\exsolexercisesaboveskip}{1ex plus 1pt minus 1pt}
% \setlength{\exsolexercisesbelowskip}{1ex plus 1pt minus 1pt}
% \end{verbatim}
% The spacings specified here are the package defaults.
%
% \subsubsection{Spacing of the individual exercises}
% Caution: the spacing can only be tuned, when one invokes the
% |exerciseaslist| package option!
%
% Then lengths below control the spacing of the |exercise| environment:
% \begin{itemize}
% \item |exercisetopbottomsep|: rubber length controlling the vertical
% space before and after individual exercises
% \item |exerciseleftmargin|: length controlling the horizontal
% space between the surrounding environment's left margin (most
% often the page margin) and the left edge of the exercise
% environment 
% \item |exerciseleftmargin|: length controlling the horizontal
% space between the surrounding environment's right margin (most
% often the page margin) and the right edge of the exercise
% environment
% \item |exerciseitemindent|: length controlling the first-line
% indentation of the first paragraph in the exercise environment
% (actually, the label is set w.r.t. this position, that we will
% conveniently call position 'x')
% \item |exerciseparindent|: length controlling the first-line
% indentation of the other paragraphs in the exercise environment.
% \item |exerciselabelsep|: length controlling the distance between
% the label and position 'x'
% \item |exerciselabelwidth|: minimal width of the (internally
% right-alligned) box to use for the exercises label; if the box is
% not sufficiently big, position 'x' is shifted to the right
% \item |exerciseparsep|: internal paragraph separation (vertically)
% \end{itemize}
% 
% You can simply specify them like:
% \begin{verbatim}
% \setlength{\exsolexercisetopbottomsep}{0pt plus 0pt minus 1pt}
% \setlength{\exsolexerciseleftmargin}{1em}
% \setlength{\exsolexerciserightmargin}{1em}
% \setlength{\exsolexerciseparindent}{0em}
% \setlength{\exsolexerciselabelsep}{0.5em}
% \setlength{\exsolexerciselabelwidth}{0pt}
% \setlength{\exsolexerciseitemindent}{0pt}
% \setlength{\exsolexerciseparsep}{\parskip}
% \end{verbatim}
% The spacings specified here are the package defaults.
%
% \subsection{Tips and tricks}
%
% If you want to include the solutions all at the
% end of the current document, you need to explicitly close the
% solution stream before including it:
% \begin{verbatim}
%   \closeout\solutionstream\input{\jobname.sol.tex}
% \end{verbatim}
%
% If you want to avoid exercises being split by a page boundary, then
% provide the package option 'minipage'. This causes the exercises to
% be wrapped in a minipage environment.
% 
% \clearpage
%
% \section{Implementation}
% %%%%%%%%%%%%%%%%%%%%%%%
%    \begin{macrocode}
%<*package>
%    \end{macrocode}
%
% \subsection{Auxiliary packages}
% %%%%%%%%%%%%%%%%%%%%%%%%%%%%%%%
% The package uses some auxiliary packages:
%    \begin{macrocode}
\RequirePackage{fancyvrb}
\RequirePackage{ifthen}
\RequirePackage{kvoptions}
%    \end{macrocode}
%
% \subsection{Package options}
% %%%%%%%%%%%%%%%%%%%%%%%%%%%%
% The package offers some options:
%
% \changes{v0.2}{2012/01/06}{Added option exercisesfont}
% \changes{v0.4}{2012/01/09}{Changed name of option to exercisesfontsize}
%
% \begin{macro}{exercisesfontsize}
%  This option allows setting the font of the \texttt{exercises}
%  environment. You may chopse one of tiny, scriptsize, footnotesize,
%  small, normalsize, large, etc.\\
%  E.g., \texttt{[exercisesfontsize=small]}.
%    \begin{macrocode}
\DeclareStringOption[normalsize]{exercisesfontsize}
%    \end{macrocode}
% \end{macro}
%
% \changes{v0.4}{2012/01/06}{Added option exercisesinlist}
% \changes{v0.5}{2012/01/09}{Changed option exercisesinlist to exerciseaslist}
%
% \begin{macro}{exerciseaslist}
%  This boolean option (true, false) allows setting the typesetting of
%  the \texttt{exercises} in a list environment. This causes the
%  exercises to be typeset in a more compact fashion, with indented
%  left and right margin. 
%    \begin{macrocode}
\DeclareBoolOption[false]{exerciseaslist}
%    \end{macrocode}
% \end{macro}
%
% \changes{v0.5}{2012/01/09}{Added option copyexercisesinsolutions}
% \begin{macro}{copyexercisesinsolutions}
%  This boolean option (true, false) allows copying the exercises in
%  the solutions file, to allow for making a complete stand-alone
%  exercises bundle.
%    \begin{macrocode}
\DeclareBoolOption[false]{copyexercisesinsolutions}
%    \end{macrocode}
% \end{macro}
%
% \changes{v0.9}{2014/07/28}{. Changed default behavior
% w.r.t. minipage-wraping of exercises}
% \begin{macro}{minipage}
%  This boolean option (true, false) causes the exercises to be
%  wrapped in minipages. This avoids them getting split by a page
%  boundary.
%    \begin{macrocode}
\DeclareBoolOption[false]{minipage}
%    \end{macrocode}
% \end{macro}
%
% The options are processed using:
%    \begin{macrocode}
\ProcessKeyvalOptions*
%    \end{macrocode}
%
% The options are subsequently handled
%    \begin{macrocode}
\newcommand{\exercisesfontsize}{\csname \exsol@exercisesfontsize\endcsname}
%    \end{macrocode}
%
%
% \subsection{Customization of lengths}
% %%%%%%%%%%%%%%%%%%%%%%%%%%%%%%%%%%%%%%%
% The commands below allow customizing many lengths that control the
% typesetting of the exercises.
%
% \changes{v0.91}{2014/08/31}{added user-accessible lengths}
% First some lengths to control the spacing before and after |exercises|.
%    \begin{macrocode}
\newlength{\exsolexercisesaboveskip}
\setlength{\exsolexercisesaboveskip}{1ex plus 1pt minus 1pt}
\newlength{\exsolexercisesbelowskip}
\setlength{\exsolexercisesbelowskip}{1ex plus 1pt minus 1pt}
%    \end{macrocode}
%
% Then some lengths to control the spacing for a single
% exercise. These lengths only work when the |exerciseaslist| package
% option has been specified. Sensible defaults have been set.
%    \begin{macrocode}
\newlength{\exsolexercisetopbottomsep}
\setlength{\exsolexercisetopbottomsep}{0pt plus 0pt minus 1pt}
\newlength{\exsolexerciseleftmargin}
\setlength{\exsolexerciseleftmargin}{1em}
\newlength{\exsolexerciserightmargin}
\setlength{\exsolexerciserightmargin}{1em}
\newlength{\exsolexerciseparindent}
\setlength{\exsolexerciseparindent}{0em}
\newlength{\exsolexerciselabelsep}
\setlength{\exsolexerciselabelsep}{0.5em}
\newlength{\exsolexerciselabelwidth}
\setlength{\exsolexerciselabelwidth}{0pt}
\newlength{\exsolexerciseitemindent}
\setlength{\exsolexerciseitemindent}{0pt}
\newlength{\exsolexerciseparsep}
\setlength{\exsolexerciseparsep}{\parskip}
%    \end{macrocode}
% 
% 
% \subsection{Con- and destruction of the auxiliary streams}
% %%%%%%%%%%%%%%%%%%%%%%%%%%%%%%%%%%%%%%%%%%%%%%%%%%%%%%%%%%
% At the beginning of your document, we start by opening a stream to a
% file that will be used to write the solutions to. At the end of your
% document, the package closes the stream.
% \changes{v0.8}{2014/07/15}{moved newwrite of exercise stream to this
% spot to avoid consuming all handles}
%    \begin{macrocode}
\AtBeginDocument{
  \newwrite\solutionstream
  \immediate\openout\solutionstream=\jobname.sol.tex
  \newwrite\exercisestream
}
\AtEndDocument{
  \immediate\closeout\solutionstream
}
%    \end{macrocode}
%
% \subsection{Exercises counter}
% %%%%%%%%%%%%%%%%%%%%%%%%%%%%%%
% By providing an exercise counter, proper numbering of the exercises
% is provided to allow for good cross referencing of the solutions to
% the exercises.
% \changes{v0.2}{2012/01/06}{Removed dash in counter when in document
% without sectioning commands}
%    \begin{macrocode}
\newcounter{exercise}[subsection]
\setcounter{exercise}{0}
\renewcommand{\theexercise}{%
  \@ifundefined{c@chapter}{}{\if0\arabic{chapter}\else\arabic{chapter}.\fi}%
  \if0\arabic{section}\else\arabic{section}\fi%
  \if0\arabic{subsection}\else.\arabic{subsection}\fi%
  \if0\arabic{subsubsection}\else.\arabic{subsubsection}\fi%
  \if0\arabic{exercise}\else%
    \@ifundefined{c@chapter}%
                 {\if0\arabic{section}\else-\fi}%
                 {-}%
    \arabic{exercise}%
  \fi
}
%    \end{macrocode}
%
%
%
% \subsection{Detokenization in order to cope with utf8}
%
% Combining old-school \LaTeX{} (before \XeTeX{} and \LuaTeX{}) and
% UTF-8 is a pain.
% Detokenization has been suggested by Geoffrey Poore to solve issues
% with UTF-8 characters messing up the |fancyvrb| internals.
% \changes{v0.7}{2014/07/14}{Added detokenized writing}
%    \begin{macrocode}
\newcommand{\GPES@write@detok}[1]{%
  \immediate\write\exercisestream{\detokenize{#1}}}%
\newcommand{\GPSS@write@detok}[1]{%
  \immediate\write\solutionstream{\detokenize{#1}}}%
\newcommand{\GPESS@write@detok}[1]{%
  \GPES@write@detok{#1}%
  \GPSS@write@detok{#1}}%
%    \end{macrocode}
%
%
% \section{The user environments}
%
% \begin{macro}{exercise}
%   The \texttt{exercise} environment is used to typeset your
%   exercises, provide them with a nice label and allow for copying
%   the exercise to the solutions file (if the package option
%   \texttt{copyexercisesinsolution}) is set. The label can be
%   set by redefining the \cs{exercisename} macro, or by relying on
%   the \textsf{Babel} provisions. The code is almost litteraly
%   taken from the \textsf{fancyvrb} package.
%    \begin{macrocode}
\def\exercise{\FV@Environment{}{exercise}}
\def\FVB@exercise{%
  \refstepcounter{exercise}%
  \immediate\openout\exercisestream=\jobname.exc.tex
  \ifexsol@copyexercisesinsolutions
    \typeout{Writing exercise to \jobname.sol.tex}
    \immediate\write\solutionstream{\string\par---\string\newline
      \string\textbf\string{\exercisename{} \theexercise \string}}
  \else
    \immediate\write\solutionstream{\string\par---\string\newline}
  \fi
  \immediate\write\exercisestream{\string\begin{exsol@exercise}}
  \@bsphack
  \begingroup
    \FV@UseKeyValues
    \FV@DefineWhiteSpace
    \def\FV@Space{\space}%
    \FV@DefineTabOut
    \ifexsol@copyexercisesinsolutions
      \let\FV@ProcessLine\GPESS@write@detok %
    \else
      \let\FV@ProcessLine\GPES@write@detok %
    \fi
    \relax
    \let\FV@FontScanPrep\relax
    \let\@noligs\relax
    \FV@Scan
  }
\def\FVE@exercise{
  \endgroup\@esphack
  \immediate\write\exercisestream{\string\end{exsol@exercise}}
  \ifexsol@copyexercisesinsolutions
    \immediate\write\solutionstream{\string~\string\newline}
  \fi
  \immediate\closeout\exercisestream
  \input{\jobname.exc.tex}
}
\DefineVerbatimEnvironment{exercise}{exercise}{}
%    \end{macrocode}
% \end{macro}
%
% \begin{macro}{exsol@exercise}
%   The \texttt{exsol@exercise} environment is an internal macro used
%   to typeset your exercises and provide them with a nice label and
%   number. Do not use it directly. Use the proper environment
%   \texttt{exercise} instead.
%   \changes{v0.2}{2012/01/06}{Attempted to fix MiKTeX formatting problems}
%   \changes{v0.3}{2012/01/08}{Fixed labelsep to avoid cluttered
%   itemize environments}
%   \changes{v0.4}{2012/01/06}{Added option exercisesinlist such that
%   default results in non list formatting of exercise}
%   \changes{v0.5}{2012/01/09}{Changed implementation to allow for
%   copying the exercises to the solutions file.}
%    \begin{macrocode}
\newenvironment{exsol@exercise}[0]
{%
  \ifthenelse{\boolean{exsol@minipage}}{\begin{minipage}[t]{\textwidth}}{}%
    \ifthenelse{\boolean{exsol@exerciseaslist}}
               {\begin{list}%
                   {%
                   }%
                   {%
                     \setlength{\topsep}{\exsolexercisetopbottomsep}%
                     \setlength{\leftmargin}{\exsolexerciseleftmargin}%
                     \setlength{\rightmargin}{\exsolexerciserightmargin}%
                     \setlength{\listparindent}{\exsolexerciseparindent}%
                     \setlength{\itemindent}{\exsolexerciseitemindent}%
                     \setlength{\parsep}{\exsolexerciseparsep}
                     \setlength{\labelsep}{\exsolexerciselabelsep}
                     \setlength{\labelwidth}{\exsolexerciselabelwidth}}
                 \item[\textit{~\exercisename{} \theexercise:~}]
               }%
               {\textit{\exercisename{} \theexercise:}}
}
{%
  \ifthenelse{\boolean{exsol@exerciseaslist}}%
             {\end{list}}{}%
  \ifthenelse{\boolean{exsol@minipage}}{\end{minipage}}{\par}%
}
%    \end{macrocode}
% \end{macro}
%
%
% \begin{macro}{solution}
%   The \texttt{solution} environment is used to typeset your solutions
%   and provide them with a nice label and number that corresponds to
%   the exercise that preceeded this solution. Theno label can be
%   set by redefining the \cs{solutionname} macro, or by relying on
%   the \textsf{Babel} provisions. The code is almost litteraly
%   taken from the \textsf{fancyvrb} package.
%    \begin{macrocode}
\def\solution{\FV@Environment{}{solution}}
\def\FVB@solution{%
  \typeout{Writing solution to \jobname.sol.tex}
  \immediate\write\solutionstream{\string\textbf\string{\solutionname{}\string}}
  \ifexsol@copyexercisesinsolutions
    \immediate\write\solutionstream{\string\newline}
  \else
    \immediate\write\solutionstream{\string\textbf\string{\theexercise\string}%
                                    \string\newline}
  \fi
  \@bsphack
  \begingroup
    \FV@UseKeyValues
    \FV@DefineWhiteSpace
    \def\FV@Space{\space}%
    \FV@DefineTabOut
    \let\FV@ProcessLine\GPSS@write@detok %
    \relax
    \let\FV@FontScanPrep\relax
    \let\@noligs\relax
    \FV@Scan
  }
\def\FVE@solution{\endgroup\@esphack}
\DefineVerbatimEnvironment{solution}{solution}{}
%    \end{macrocode}
% \end{macro}
%
% \begin{macro}{exercises}
%   The \texttt{exercises} environment helps typesetting your exercises to
%   stand out from the rest of the text. You may use it at the end of
%   a chapter, or just to group some exercises in the text.
%   \changes{v0.2}{2012/01/06}{Attempted to fix MiKTeX formatting problems}
%   \changes{v0.3}{2012/01/07}{Added some extra whitespace below exercisesname}
%    \begin{macrocode}
\newenvironment{exercises}
{\par\exercisesfontsize\rule{.25\linewidth}{0.15mm}\vspace*{\exsolexercisesaboveskip}\\*%
 \textbf{\normalsize \exercisesname}}
{\vspace*{-\baselineskip}\vspace*{\exsolexercisesbelowskip}\rule{.25\linewidth}{0.15mm}\par}
%    \end{macrocode}
% \end{macro}
%
% \subsection{Some Babel provisions}
% %%%%%%%%%%%%%%%%%%%%%%%%%%%%%%%%%%
% \changes{v0.2}{2012/01/06}{Fixed babel errors}
% \begin{macro}{\exercisename}
%   The exercise environment makes use of a label \texttt{\exercisename{}}
%   macro.
%    \begin{macrocode}
\newcommand{\exercisename}{Exercise}
%    \end{macrocode}
% \end{macro}
%
% \begin{macro}{\exercisesname}
%   The exercises environment makes use of a label \texttt{\exercisesname{}}
%   macro.
%    \begin{macrocode}
\newcommand{\exercisesname}{Exercises}
%    \end{macrocode}
% \end{macro}
% 
% \begin{macro}{\solutionname}
%   The solution environment makes use of a label \texttt{\solutionname{}}
%   macro.
%    \begin{macrocode}
\newcommand{\solutionname}{Solution}
%    \end{macrocode}
% \end{macro}
%
% \begin{macro}{\solutionname}
%   The solution environment makes use of a label \texttt{\solutionname{}}
%   macro.
% \changes{v0.8}{2014/07/15}{Added missing babel tag}
%    \begin{macrocode}
\newcommand{\solutionsname}{Solutions}
%    \end{macrocode}
% \end{macro}
% 
% 
% You may redefine these macros, but to help you out a little bit, we
% provide with some basic Babel auxiliaries. If you're a true polyglot
% and are willing to help me out by providing translations for other
% languages, I'm very willing to incorporate them into the code.
%
% \changes{v0.7}{2014/07/14}{Added Finnish language support}
%    \begin{macrocode}
\addto\captionsdutch{%
  \renewcommand{\exercisename}{Oefening}%
  \renewcommand{\exercisesname}{Oefeningen}%
  \renewcommand{\solutionname}{Oplossing}%
  \renewcommand{\solutionsname}{Oplossingen}%
}
\addto\captionsgerman{%
  \renewcommand{\exercisename}{Aufgabe}%
  \renewcommand{\exercisesname}{Aufgaben}%
  \renewcommand{\solutionname}{L\"osung}%
  \renewcommand{\solutionsname}{L\"osungen}%
}
\addto\captionsfrench{%
  \renewcommand{\exercisename}{Exercice}%
  \renewcommand{\exercisesname}{Exercices}%
  \renewcommand{\solutionname}{Solution}%
  \renewcommand{\solutionsname}{Solutions}%
}
\addto\captionsfinnish{
  \renewcommand{\exercisename}{Teht\"av\"a}%
  \renewcommand{\exercisesname}{Teht\"avi\"a}%
  \renewcommand{\solutionname}{Ratkaisu}%
  \renewcommand{\solutionsname}{Ratkaisut}%
}
%    \end{macrocode}
%
%
%
% Now the final hack overloads the basic sectioning commands to make
% sure that they are copied into your solution book.
%
%    \begin{macrocode}
\let\exsol@@makechapterhead\@makechapterhead
\def\@makechapterhead#1{%
  \immediate\write\solutionstream{\string\chapter{#1}}%
  \exsol@@makechapterhead{#1}
}
\ifdefined\frontmatter
  \let\exsol@@frontmatter\frontmatter
  \def\frontmatter{%
    \immediate\write\solutionstream{\string\frontmatter}%
    \exsol@@frontmatter
  }
\fi
\ifdefined\frontmatter
  \let\exsol@@mainmatter\mainmatter
  \def\mainmatter{%
    \immediate\write\solutionstream{\string\mainmatter}%
    \exsol@@mainmatter
  }
\fi
\ifdefined\backmatter
  \let\exsol@@backmatter\backmatter
  \def\backmatter{%
    \immediate\write\solutionstream{\string\backmatter}%
    \exsol@@backmatter
  }
\fi
%    \end{macrocode}
%
% \begin{macro}{\noexercisesinchapter}
%   If you have chapters without exercises, you may want to indicate
%   this clearly into your source. Otherwise empty chapters may appear
%   in your solution book.
%    \begin{macrocode}
\newcommand{\noexercisesinchapter}
{
  \immediate\write\solutionstream{No exercises in this chapter}
}
%    \end{macrocode}
% \end{macro}
%
%    \begin{macrocode}
%</package>
%    \end{macrocode}
%
% \bibliographystyle{alpha}
%
% \begin{thebibliography}{99}
%
% \bibitem{fancyvrb}
% Timothy Van Zandt, Herbert Vo\ss, Denis Girou, Sebastian Rahtz, Niall
% Mansfield 
% \newblock The \texttt{fancyvrb} package.
% \newblock \url{http://ctan.org/pkg/fancyvrb}.
% \newblock online, accessed in January 2012.
%
% \bibitem{CTAN} 
% The Comprehensive TeX Archive Network.
% \newblock \url{http://www.ctan.org}.
% \newblock online, accessed in January 2012.
%
% \end{thebibliography}
%
% \Finale
\endinput

%
% When running \LaTeX{} on your document (in our case on the
% \texttt{exsol.dtx} file, as a side effect a file with extension
% \texttt{.sol.tex} has been written to disk (in our case, the file
% \texttt{exsol.sol.tex}), containing all solutions in sequence.
%
% Generating a solution book is a simple as including the file into a
% simple \LaTeX{} harness, that allows you giving it a proper title page and to
% add other bells and whistles.
%
% E.g.,
% \begin{VerbatimOut}{exsol-solutionbook.tex}
% \documentclass{article}
% \usepackage[english]{babel}
% \title{Solutions to the exercises, specified in the \textsf{ExSol} package}
% \author{Walter Daems}
% \date{2013/05/12}
%
% \begin{document}
%
% \maketitle
%
% % \iffalse meta-comment
%
% Copyright (C) 2014 by Walter Daems <walter.daems@uantwerpen.be>
%
% This work may be distributed and/or modified under the conditions of
% the LaTeX Project Public License, either version 1.3 of this license
% or (at your option) any later version.  The latest version of this
% license is in:
% 
%    http://www.latex-project.org/lppl.txt
% 
% and version 1.3 or later is part of all distributions of LaTeX version
% 2005/12/01 or later.
%
% This work has the LPPL maintenance status `maintained'.
% 
% The Current Maintainer of this work is Walter Daems.
%
% This work consists of the files exsol.dtx and exsol.ins and the derived 
% files:
%   - exsol.sty
%   - example.tex
%   - example-solutionbook.tex
%
% \fi
%
% \iffalse
%
%<package|driver>\NeedsTeXFormat{LaTeX2e}
%<driver>\ProvidesFile{exsol.dtx}
%<package>\ProvidesPackage{exsol}
%<package|driver>  [2014/08/31 v0.91 ExSol - Exercises and Solutions package (DMW)]
%<*driver> 
\documentclass[11pt]{ltxdoc}
\usepackage[english]{babel}
\usepackage[exercisesfontsize=small]{exsol}
\usepackage{metalogo}
\EnableCrossrefs
\CodelineIndex
\RecordChanges
\usepackage{makeidx}
\usepackage{alltt}
\IfFileExists{tocbibind.sty}{\usepackage{tocbibind}}{}
\IfFileExists{hyperref.sty}{\usepackage[bookmarksopen]{hyperref}}{}
\EnableCrossrefs         
\CodelineIndex
\RecordChanges
\newcommand{\exsol}{\textsf{ExSol}}
\StopEventually{\PrintChanges\PrintIndex}
\def\fileversion{0.91}
\def\filedate{2014/08/31}
\begin{document}
 \DocInput{exsol.dtx}
\end{document}
%</driver>
% \fi
%
% \CheckSum{0}
%
% \CharacterTable
%  {Upper-case    \A\B\C\D\E\F\G\H\I\J\K\L\M\N\O\P\Q\R\S\T\U\V\W\X\Y\Z
%   Lower-case    \a\b\c\d\e\f\g\h\i\j\k\l\m\n\o\p\q\r\s\t\u\v\w\x\y\z
%   Digits        \0\1\2\3\4\5\6\7\8\9
%   Exclamation   \!     Double quote  \"     Hash (number) \#
%   Dollar        \$     Percent       \%     Ampersand     \&
%   Acute accent  \'     Left paren    \(     Right paren   \)
%   Asterisk      \*     Plus          \+     Comma         \,
%   Minus         \-     Point         \.     Solidus       \/
%   Colon         \:     Semicolon     \;     Less than     \<
%   Equals        \=     Greater than  \>     Question mark \?
%   Commercial at \@     Left bracket  \[     Backslash     \\
%   Right bracket \]     Circumflex    \^     Underscore    \_
%   Grave accent  \`     Left brace    \{     Vertical bar  \|
%   Right brace   \}     Tilde         \~}
%
%
% \changes{v0.1}{2012/01/05}{. Initial version}
% \changes{v0.2}{2012/01/06}{. Minor bug fixes based on first use by
% Paul Levrie}
% \changes{v0.3}{2012/01/07}{. Minor bug fixes based on second use by
% Paul}
% \changes{v0.4}{2012/01/09}{. Allowed for non-list formatting of
% exercises (as default)}
% \changes{v0.5}{2012/01/15}{. Added option to also send exercises to
% solutions file}
% \changes{v0.6}{2013/05/12}{. Prepared for CTAN publication}
% \changes{v0.7}{2014/07/14}{. Fixed UTF8 compatibility issues}
% \changes{v0.8}{2014/07/15}{. Fixed missing babel tag and running out
% of write hanles}
% \changes{v0.9}{2014/07/28}{. Changed default behavior
% w.r.t. minipage-wraping of exercises} 
% \changes{v0.91}{2014/08/31}{. Corrected minipage dependence, made }
%
% \DoNotIndex{\newcommand,\newenvironment}
% \setlength{\parindent}{0em}
% \addtolength{\parskip}{0.5\baselineskip}
%
% \title{The \exsol{} package\thanks{This document
%   corresponds to exsol~\fileversion, dated \filedate.}}
% \author{Walter Daems (\texttt{walter.daems@uantwerpen.be})}
% \date{}
%
% \maketitle
%
% \section{Introduction}
% %%%%%%%%%%%%%%%%%%%%%%
% The package \exsol{} provides macros to allow
% embedding exercises and solutions in the \LaTeX{} source of an
% instructional text (e.g., a book or a course text) while generating
% the following separate documents:
% \begin{itemize}
% \item your original text that only contains the exercises, and
% \item a solution book that only contains the solutions to the
% exercises (a package option exists to also copy the exercises themselves to the solution book).
% \end{itemize}
% 
% The former is generated when running \LaTeX{} on your document. This
% run writes the solutions to a secondary file that can be included
% into a simple document harness, such that when running \LaTeX{} on
% the latter, you can generate a nice solution book.
% 
% Why use \exsol{}?
% \begin{itemize}
% \item It allows to keep the \LaTeX{} source of your exercises and their
% solutions in a single file. Away with the nightmare to keep your
% solutions in sync with the original text.
% \item It separates exercises and solutions, allowing you
%   \begin{itemize}
%   \item to only release the solution book to the instructors of the
%   course;
%   \item to encourage students that you provide with the solution
%   book to first try solving the exercises without opening the book;
%   this seems to be easier than not peeking into the solution of an
%   exercise that is typeset just below the exercise itself.
%   \end{itemize}
% \end{itemize}
%
% The code of the \exsol{} package was taken almost literally
% from \textsf{fancyvrb} \cite{fancyvrb}. Therefore, all credits go to the
% authors/maintainers of \textsf{fancyvrb}.
%
% Thanks to Pieter Pareit and Pekka Pere for signaling problems and
% making suggestions for the documentation.
%
% \section{Installation}
% %%%%%%%%%%%%%%%%%%%%%%
% Either you are a package manager and then you'll know how to
% prepare an installation package for \exsol{}.
%
% Either you are a normal user and then you have two options. First,
% check if there is a package that your favorite \LaTeX{}
% distributor has prepared for you. Second, grab the TDS package
% from CTAN \cite{CTAN} (\texttt{exsol.tds.zip}) and unzip it somewhere in your
% own TDS tree, regenerate your filename database and off you go.
% In any case, make sure that \LaTeX{} finds the \texttt{exsol.sty} file.
%
% The \exsol{} package uses some auxiliary packages: \textsf{fancyvrb},
% \textsf{ifthen}, \textsf{kvoptions} and, optionally,
% \textsf{babel}. Fetch them from CTAN \cite{CTAN} if your \TeX{}
% distributor does not provide them.
%
% \section{Usage}
% %%%%%%%%%%%%%%%
% 
% \subsection{Preparing your document source}
% %%%%%%%%%%%%%%%%%%%%%%%%%%%%%%%%%%%%%%%%%%%
% The macro package exsol can be loaded with:
% \begin{verbatim}
% \usepackage{exsol}
% \end{verbatim}
%
% Then, you are ready to add some exercises including their solution
% to your document source. To this end, embed them in a
% \texttt{exercise} and a corresponding \texttt{solution} environment.
% Optionally, you may embed several of them in a \texttt{exercises}
% environment, to make them stand out in your text.
% E.g.,
%
% \begin{VerbatimOut}{exsol.tmp}
% 
% \begin{exercises}
%
%   \begin{exercise}
%     Solve the following equation for $x \in C$, with $C$ the set of
%     complex numbers:
%     \begin{equation}
%       5 x^2 -3 x = 5
%     \end{equation}
%   \end{exercise}
%   \begin{solution}
%     Let's start by rearranging the equation, a bit:
%     \begin{eqnarray}
%       5.7 x^2 - 3.1 x &=& 5.3\\
%       5.7 x^2 - 3.1 x -5.3 &=& 0
%     \end{eqnarray}
%     The equation is now in the standard form:
%     \begin{equation}
%       a x^2 + b x + c = 0
%     \end{equation}
%     For quadratic equations in the standard form, we know that two
%     solutions exist:
%     \begin{equation}
%       x_{1,2} = \frac{ -b \pm \sqrt{d}}{2a}
%     \end{equation}
%     with
%     \begin{equation}
%       d = b^2 - 4 a c
%     \end{equation}
%     If we apply this to our case, we obtain:
%     \begin{equation}
%       d = (-3.1)^2 - 4 \cdot 5.7 \cdot (-5.3) = 130.45
%     \end{equation}
%     and
%     \begin{eqnarray}
%       x_1 &=& \frac{3.1 + \sqrt{130.45}}{11.4} = 1.27\\
%       x_2 &=& \frac{3.1 - \sqrt{130.45}}{11.4} = -0.73
%     \end{eqnarray}
%     The proposed values $x = x_1, x_2$ are solutions to the given equation.
%   \end{solution}
%   \begin{exercise}
%     Consider a 2-dimensional vector space equipped with a Euclidean
%     distance function. Given a right-angled triangle, with the sides
%     $A$ and $B$ adjacent to the right angle having lengths, $3$ and
%     $4$, calculate the length of the hypotenuse, labeled $C$.
%   \end{exercise}
%   \begin{solution}
%     This calls for application of Pythagoras' theorem, which 
%     tells us:
%     \begin{equation}
%       \left\|A\right\|^2 + \left\|B\right\|^2 = \left\|C\right\|^2
%     \end{equation}
%     and therefore:
%     \begin{eqnarray}
%       \left\|C\right\| 
%       &=& \sqrt{\left\|A\right\|^2 + \left\|B\right\|^2}\\
%       &=& \sqrt{3^2 + 4^2}\\
%       &=& \sqrt{25} = 5
%     \end{eqnarray}
%     Therefore, the length of the hypotenuse equals $5$.
%   \end{solution}
%
% \end{exercises}
% \end{VerbatimOut}
% \VerbatimInput[frame=lines,gobble=2,fontsize=\footnotesize]{exsol.tmp}
%
% The result in the original document, can be seen below. As you can
% see, there's no trace of the solution. 
%
% \input{exsol.tmp}
%
% When running \LaTeX{} on your document (in our case on the
% \texttt{exsol.dtx} file, as a side effect a file with extension
% \texttt{.sol.tex} has been written to disk (in our case, the file
% \texttt{exsol.sol.tex}), containing all solutions in sequence.
%
% Generating a solution book is a simple as including the file into a
% simple \LaTeX{} harness, that allows you giving it a proper title page and to
% add other bells and whistles.
%
% E.g.,
% \begin{VerbatimOut}{exsol-solutionbook.tex}
% \documentclass{article}
% \usepackage[english]{babel}
% \title{Solutions to the exercises, specified in the \textsf{ExSol} package}
% \author{Walter Daems}
% \date{2013/05/12}
%
% \begin{document}
%
% \maketitle
%
% \input{exsol.sol}
%
% \end{document}
% \end{VerbatimOut}
% \VerbatimInput[frame=lines,gobble=2,fontsize=\footnotesize]{exsol-solutionbook.tex}
% 
% You may generate this solution book, by running \LaTeX{} on the
% file named \texttt{exsol-solutionbook.tex} that is generated when running
% \LaTeX{} on the \texttt{exsol.dtx} file.
%
% The result approximately looks like this:
%
% \setcounter{equation}{0}
% \rule{\linewidth}{.7pt}
% \begin{center}
% {\Large Solutions to the exercises, specified in the \textsf{ExSol} package}\\
% {\large Walter Daems}\\
% {\large 2013/05/12}
% \end{center}
% \par---\newline\textbf{Solution 3.1-1}
%     Let's start by rearranging the equation, a bit:
%     \begin{eqnarray}
%       5.7 x^2 - 3.1 x &=% 5.3\\
%       5.7 x^2 - 3.1 x -5.3 &=% 0
%     \end{eqnarray}
%     The equation is now in the standard form:
%     \begin{equation}
%       a x^2 + b x + c = 0
%     \end{equation}
%     For quadratic equations in the standard form, we know that two
%     solutions exist:
%     \begin{equation}
%       x_{1,2} = \frac{ -b \pm \sqrt{d}}{2a}
%     \end{equation}
%     with
%     \begin{equation}
%       d = b^2 - 4 a c
%     \end{equation}
%     If we apply this to our case, we obtain:
%     \begin{equation}
%       d = (-3.1)^2 - 4 \cdot 5.7 \cdot (-5.3) = 130.45
%     \end{equation}
%     and
%     \begin{eqnarray}
%       x_1 &=& \frac{3.1 + \sqrt{130.45}}{11.4} = 1.27\\
%       x_2 &=& \frac{3.1 + \sqrt{130.45}}{11.4} = -0.73
%     \end{eqnarray}
%     The proposed values $x = x_1, x_2$ are solutions to
%     the given equation.
% \par---\newline\textbf{Solution 3.1-2}
%       This calls for application of Pythagoras' theorem, which
%       tells us:
%       \begin{equation}
%         \left\|A\right\|^2 + \left\|B\right\|^2 = \left\|C\right\|^2
%       \end{equation}
%       and therefore:
%       \begin{eqnarray}
%         \left\|C\right\|
%         &=& \sqrt{\left\|A\right\|^2 + \left\|B\right\|^2}\\
%         &=& \sqrt{3^2 + 4^2}\\
%         &=& \sqrt{25} = 5
%       \end{eqnarray}
%       Therefore, the length of the hypotenuse equals $5$.
%
% \rule{\linewidth}{.7pt}
%
% \subsection{Fiddling with the spacing}
%
% The default spacing provided by the \textsf{ExSol} package should be
% fine for most users. However, if you like to tweak, below you can
% find the controls.
% \subsubsection{Spacing before and after the \texttt{exercises} environment}
%
% The lengths below control the spacing of the |exercises| environment:
% \begin{itemize}
% \item |exsolexerciseaboveskip|: rubber length controlling the
% vertical space after the top marker line of the environment
% \item |exsolexercisebelowskip|: rubber length controlling the
% vertical space before the bottom marker line of the environment
% \end{itemize}
%
% You can simply specify them like:
% \begin{verbatim}
% \setlength{\exsolexercisesaboveskip}{1ex plus 1pt minus 1pt}
% \setlength{\exsolexercisesbelowskip}{1ex plus 1pt minus 1pt}
% \end{verbatim}
% The spacings specified here are the package defaults.
%
% \subsubsection{Spacing of the individual exercises}
% Caution: the spacing can only be tuned, when one invokes the
% |exerciseaslist| package option!
%
% Then lengths below control the spacing of the |exercise| environment:
% \begin{itemize}
% \item |exercisetopbottomsep|: rubber length controlling the vertical
% space before and after individual exercises
% \item |exerciseleftmargin|: length controlling the horizontal
% space between the surrounding environment's left margin (most
% often the page margin) and the left edge of the exercise
% environment 
% \item |exerciseleftmargin|: length controlling the horizontal
% space between the surrounding environment's right margin (most
% often the page margin) and the right edge of the exercise
% environment
% \item |exerciseitemindent|: length controlling the first-line
% indentation of the first paragraph in the exercise environment
% (actually, the label is set w.r.t. this position, that we will
% conveniently call position 'x')
% \item |exerciseparindent|: length controlling the first-line
% indentation of the other paragraphs in the exercise environment.
% \item |exerciselabelsep|: length controlling the distance between
% the label and position 'x'
% \item |exerciselabelwidth|: minimal width of the (internally
% right-alligned) box to use for the exercises label; if the box is
% not sufficiently big, position 'x' is shifted to the right
% \item |exerciseparsep|: internal paragraph separation (vertically)
% \end{itemize}
% 
% You can simply specify them like:
% \begin{verbatim}
% \setlength{\exsolexercisetopbottomsep}{0pt plus 0pt minus 1pt}
% \setlength{\exsolexerciseleftmargin}{1em}
% \setlength{\exsolexerciserightmargin}{1em}
% \setlength{\exsolexerciseparindent}{0em}
% \setlength{\exsolexerciselabelsep}{0.5em}
% \setlength{\exsolexerciselabelwidth}{0pt}
% \setlength{\exsolexerciseitemindent}{0pt}
% \setlength{\exsolexerciseparsep}{\parskip}
% \end{verbatim}
% The spacings specified here are the package defaults.
%
% \subsection{Tips and tricks}
%
% If you want to include the solutions all at the
% end of the current document, you need to explicitly close the
% solution stream before including it:
% \begin{verbatim}
%   \closeout\solutionstream\input{\jobname.sol.tex}
% \end{verbatim}
%
% If you want to avoid exercises being split by a page boundary, then
% provide the package option 'minipage'. This causes the exercises to
% be wrapped in a minipage environment.
% 
% \clearpage
%
% \section{Implementation}
% %%%%%%%%%%%%%%%%%%%%%%%
%    \begin{macrocode}
%<*package>
%    \end{macrocode}
%
% \subsection{Auxiliary packages}
% %%%%%%%%%%%%%%%%%%%%%%%%%%%%%%%
% The package uses some auxiliary packages:
%    \begin{macrocode}
\RequirePackage{fancyvrb}
\RequirePackage{ifthen}
\RequirePackage{kvoptions}
%    \end{macrocode}
%
% \subsection{Package options}
% %%%%%%%%%%%%%%%%%%%%%%%%%%%%
% The package offers some options:
%
% \changes{v0.2}{2012/01/06}{Added option exercisesfont}
% \changes{v0.4}{2012/01/09}{Changed name of option to exercisesfontsize}
%
% \begin{macro}{exercisesfontsize}
%  This option allows setting the font of the \texttt{exercises}
%  environment. You may chopse one of tiny, scriptsize, footnotesize,
%  small, normalsize, large, etc.\\
%  E.g., \texttt{[exercisesfontsize=small]}.
%    \begin{macrocode}
\DeclareStringOption[normalsize]{exercisesfontsize}
%    \end{macrocode}
% \end{macro}
%
% \changes{v0.4}{2012/01/06}{Added option exercisesinlist}
% \changes{v0.5}{2012/01/09}{Changed option exercisesinlist to exerciseaslist}
%
% \begin{macro}{exerciseaslist}
%  This boolean option (true, false) allows setting the typesetting of
%  the \texttt{exercises} in a list environment. This causes the
%  exercises to be typeset in a more compact fashion, with indented
%  left and right margin. 
%    \begin{macrocode}
\DeclareBoolOption[false]{exerciseaslist}
%    \end{macrocode}
% \end{macro}
%
% \changes{v0.5}{2012/01/09}{Added option copyexercisesinsolutions}
% \begin{macro}{copyexercisesinsolutions}
%  This boolean option (true, false) allows copying the exercises in
%  the solutions file, to allow for making a complete stand-alone
%  exercises bundle.
%    \begin{macrocode}
\DeclareBoolOption[false]{copyexercisesinsolutions}
%    \end{macrocode}
% \end{macro}
%
% \changes{v0.9}{2014/07/28}{. Changed default behavior
% w.r.t. minipage-wraping of exercises}
% \begin{macro}{minipage}
%  This boolean option (true, false) causes the exercises to be
%  wrapped in minipages. This avoids them getting split by a page
%  boundary.
%    \begin{macrocode}
\DeclareBoolOption[false]{minipage}
%    \end{macrocode}
% \end{macro}
%
% The options are processed using:
%    \begin{macrocode}
\ProcessKeyvalOptions*
%    \end{macrocode}
%
% The options are subsequently handled
%    \begin{macrocode}
\newcommand{\exercisesfontsize}{\csname \exsol@exercisesfontsize\endcsname}
%    \end{macrocode}
%
%
% \subsection{Customization of lengths}
% %%%%%%%%%%%%%%%%%%%%%%%%%%%%%%%%%%%%%%%
% The commands below allow customizing many lengths that control the
% typesetting of the exercises.
%
% \changes{v0.91}{2014/08/31}{added user-accessible lengths}
% First some lengths to control the spacing before and after |exercises|.
%    \begin{macrocode}
\newlength{\exsolexercisesaboveskip}
\setlength{\exsolexercisesaboveskip}{1ex plus 1pt minus 1pt}
\newlength{\exsolexercisesbelowskip}
\setlength{\exsolexercisesbelowskip}{1ex plus 1pt minus 1pt}
%    \end{macrocode}
%
% Then some lengths to control the spacing for a single
% exercise. These lengths only work when the |exerciseaslist| package
% option has been specified. Sensible defaults have been set.
%    \begin{macrocode}
\newlength{\exsolexercisetopbottomsep}
\setlength{\exsolexercisetopbottomsep}{0pt plus 0pt minus 1pt}
\newlength{\exsolexerciseleftmargin}
\setlength{\exsolexerciseleftmargin}{1em}
\newlength{\exsolexerciserightmargin}
\setlength{\exsolexerciserightmargin}{1em}
\newlength{\exsolexerciseparindent}
\setlength{\exsolexerciseparindent}{0em}
\newlength{\exsolexerciselabelsep}
\setlength{\exsolexerciselabelsep}{0.5em}
\newlength{\exsolexerciselabelwidth}
\setlength{\exsolexerciselabelwidth}{0pt}
\newlength{\exsolexerciseitemindent}
\setlength{\exsolexerciseitemindent}{0pt}
\newlength{\exsolexerciseparsep}
\setlength{\exsolexerciseparsep}{\parskip}
%    \end{macrocode}
% 
% 
% \subsection{Con- and destruction of the auxiliary streams}
% %%%%%%%%%%%%%%%%%%%%%%%%%%%%%%%%%%%%%%%%%%%%%%%%%%%%%%%%%%
% At the beginning of your document, we start by opening a stream to a
% file that will be used to write the solutions to. At the end of your
% document, the package closes the stream.
% \changes{v0.8}{2014/07/15}{moved newwrite of exercise stream to this
% spot to avoid consuming all handles}
%    \begin{macrocode}
\AtBeginDocument{
  \newwrite\solutionstream
  \immediate\openout\solutionstream=\jobname.sol.tex
  \newwrite\exercisestream
}
\AtEndDocument{
  \immediate\closeout\solutionstream
}
%    \end{macrocode}
%
% \subsection{Exercises counter}
% %%%%%%%%%%%%%%%%%%%%%%%%%%%%%%
% By providing an exercise counter, proper numbering of the exercises
% is provided to allow for good cross referencing of the solutions to
% the exercises.
% \changes{v0.2}{2012/01/06}{Removed dash in counter when in document
% without sectioning commands}
%    \begin{macrocode}
\newcounter{exercise}[subsection]
\setcounter{exercise}{0}
\renewcommand{\theexercise}{%
  \@ifundefined{c@chapter}{}{\if0\arabic{chapter}\else\arabic{chapter}.\fi}%
  \if0\arabic{section}\else\arabic{section}\fi%
  \if0\arabic{subsection}\else.\arabic{subsection}\fi%
  \if0\arabic{subsubsection}\else.\arabic{subsubsection}\fi%
  \if0\arabic{exercise}\else%
    \@ifundefined{c@chapter}%
                 {\if0\arabic{section}\else-\fi}%
                 {-}%
    \arabic{exercise}%
  \fi
}
%    \end{macrocode}
%
%
%
% \subsection{Detokenization in order to cope with utf8}
%
% Combining old-school \LaTeX{} (before \XeTeX{} and \LuaTeX{}) and
% UTF-8 is a pain.
% Detokenization has been suggested by Geoffrey Poore to solve issues
% with UTF-8 characters messing up the |fancyvrb| internals.
% \changes{v0.7}{2014/07/14}{Added detokenized writing}
%    \begin{macrocode}
\newcommand{\GPES@write@detok}[1]{%
  \immediate\write\exercisestream{\detokenize{#1}}}%
\newcommand{\GPSS@write@detok}[1]{%
  \immediate\write\solutionstream{\detokenize{#1}}}%
\newcommand{\GPESS@write@detok}[1]{%
  \GPES@write@detok{#1}%
  \GPSS@write@detok{#1}}%
%    \end{macrocode}
%
%
% \section{The user environments}
%
% \begin{macro}{exercise}
%   The \texttt{exercise} environment is used to typeset your
%   exercises, provide them with a nice label and allow for copying
%   the exercise to the solutions file (if the package option
%   \texttt{copyexercisesinsolution}) is set. The label can be
%   set by redefining the \cs{exercisename} macro, or by relying on
%   the \textsf{Babel} provisions. The code is almost litteraly
%   taken from the \textsf{fancyvrb} package.
%    \begin{macrocode}
\def\exercise{\FV@Environment{}{exercise}}
\def\FVB@exercise{%
  \refstepcounter{exercise}%
  \immediate\openout\exercisestream=\jobname.exc.tex
  \ifexsol@copyexercisesinsolutions
    \typeout{Writing exercise to \jobname.sol.tex}
    \immediate\write\solutionstream{\string\par---\string\newline
      \string\textbf\string{\exercisename{} \theexercise \string}}
  \else
    \immediate\write\solutionstream{\string\par---\string\newline}
  \fi
  \immediate\write\exercisestream{\string\begin{exsol@exercise}}
  \@bsphack
  \begingroup
    \FV@UseKeyValues
    \FV@DefineWhiteSpace
    \def\FV@Space{\space}%
    \FV@DefineTabOut
    \ifexsol@copyexercisesinsolutions
      \let\FV@ProcessLine\GPESS@write@detok %
    \else
      \let\FV@ProcessLine\GPES@write@detok %
    \fi
    \relax
    \let\FV@FontScanPrep\relax
    \let\@noligs\relax
    \FV@Scan
  }
\def\FVE@exercise{
  \endgroup\@esphack
  \immediate\write\exercisestream{\string\end{exsol@exercise}}
  \ifexsol@copyexercisesinsolutions
    \immediate\write\solutionstream{\string~\string\newline}
  \fi
  \immediate\closeout\exercisestream
  \input{\jobname.exc.tex}
}
\DefineVerbatimEnvironment{exercise}{exercise}{}
%    \end{macrocode}
% \end{macro}
%
% \begin{macro}{exsol@exercise}
%   The \texttt{exsol@exercise} environment is an internal macro used
%   to typeset your exercises and provide them with a nice label and
%   number. Do not use it directly. Use the proper environment
%   \texttt{exercise} instead.
%   \changes{v0.2}{2012/01/06}{Attempted to fix MiKTeX formatting problems}
%   \changes{v0.3}{2012/01/08}{Fixed labelsep to avoid cluttered
%   itemize environments}
%   \changes{v0.4}{2012/01/06}{Added option exercisesinlist such that
%   default results in non list formatting of exercise}
%   \changes{v0.5}{2012/01/09}{Changed implementation to allow for
%   copying the exercises to the solutions file.}
%    \begin{macrocode}
\newenvironment{exsol@exercise}[0]
{%
  \ifthenelse{\boolean{exsol@minipage}}{\begin{minipage}[t]{\textwidth}}{}%
    \ifthenelse{\boolean{exsol@exerciseaslist}}
               {\begin{list}%
                   {%
                   }%
                   {%
                     \setlength{\topsep}{\exsolexercisetopbottomsep}%
                     \setlength{\leftmargin}{\exsolexerciseleftmargin}%
                     \setlength{\rightmargin}{\exsolexerciserightmargin}%
                     \setlength{\listparindent}{\exsolexerciseparindent}%
                     \setlength{\itemindent}{\exsolexerciseitemindent}%
                     \setlength{\parsep}{\exsolexerciseparsep}
                     \setlength{\labelsep}{\exsolexerciselabelsep}
                     \setlength{\labelwidth}{\exsolexerciselabelwidth}}
                 \item[\textit{~\exercisename{} \theexercise:~}]
               }%
               {\textit{\exercisename{} \theexercise:}}
}
{%
  \ifthenelse{\boolean{exsol@exerciseaslist}}%
             {\end{list}}{}%
  \ifthenelse{\boolean{exsol@minipage}}{\end{minipage}}{\par}%
}
%    \end{macrocode}
% \end{macro}
%
%
% \begin{macro}{solution}
%   The \texttt{solution} environment is used to typeset your solutions
%   and provide them with a nice label and number that corresponds to
%   the exercise that preceeded this solution. Theno label can be
%   set by redefining the \cs{solutionname} macro, or by relying on
%   the \textsf{Babel} provisions. The code is almost litteraly
%   taken from the \textsf{fancyvrb} package.
%    \begin{macrocode}
\def\solution{\FV@Environment{}{solution}}
\def\FVB@solution{%
  \typeout{Writing solution to \jobname.sol.tex}
  \immediate\write\solutionstream{\string\textbf\string{\solutionname{}\string}}
  \ifexsol@copyexercisesinsolutions
    \immediate\write\solutionstream{\string\newline}
  \else
    \immediate\write\solutionstream{\string\textbf\string{\theexercise\string}%
                                    \string\newline}
  \fi
  \@bsphack
  \begingroup
    \FV@UseKeyValues
    \FV@DefineWhiteSpace
    \def\FV@Space{\space}%
    \FV@DefineTabOut
    \let\FV@ProcessLine\GPSS@write@detok %
    \relax
    \let\FV@FontScanPrep\relax
    \let\@noligs\relax
    \FV@Scan
  }
\def\FVE@solution{\endgroup\@esphack}
\DefineVerbatimEnvironment{solution}{solution}{}
%    \end{macrocode}
% \end{macro}
%
% \begin{macro}{exercises}
%   The \texttt{exercises} environment helps typesetting your exercises to
%   stand out from the rest of the text. You may use it at the end of
%   a chapter, or just to group some exercises in the text.
%   \changes{v0.2}{2012/01/06}{Attempted to fix MiKTeX formatting problems}
%   \changes{v0.3}{2012/01/07}{Added some extra whitespace below exercisesname}
%    \begin{macrocode}
\newenvironment{exercises}
{\par\exercisesfontsize\rule{.25\linewidth}{0.15mm}\vspace*{\exsolexercisesaboveskip}\\*%
 \textbf{\normalsize \exercisesname}}
{\vspace*{-\baselineskip}\vspace*{\exsolexercisesbelowskip}\rule{.25\linewidth}{0.15mm}\par}
%    \end{macrocode}
% \end{macro}
%
% \subsection{Some Babel provisions}
% %%%%%%%%%%%%%%%%%%%%%%%%%%%%%%%%%%
% \changes{v0.2}{2012/01/06}{Fixed babel errors}
% \begin{macro}{\exercisename}
%   The exercise environment makes use of a label \texttt{\exercisename{}}
%   macro.
%    \begin{macrocode}
\newcommand{\exercisename}{Exercise}
%    \end{macrocode}
% \end{macro}
%
% \begin{macro}{\exercisesname}
%   The exercises environment makes use of a label \texttt{\exercisesname{}}
%   macro.
%    \begin{macrocode}
\newcommand{\exercisesname}{Exercises}
%    \end{macrocode}
% \end{macro}
% 
% \begin{macro}{\solutionname}
%   The solution environment makes use of a label \texttt{\solutionname{}}
%   macro.
%    \begin{macrocode}
\newcommand{\solutionname}{Solution}
%    \end{macrocode}
% \end{macro}
%
% \begin{macro}{\solutionname}
%   The solution environment makes use of a label \texttt{\solutionname{}}
%   macro.
% \changes{v0.8}{2014/07/15}{Added missing babel tag}
%    \begin{macrocode}
\newcommand{\solutionsname}{Solutions}
%    \end{macrocode}
% \end{macro}
% 
% 
% You may redefine these macros, but to help you out a little bit, we
% provide with some basic Babel auxiliaries. If you're a true polyglot
% and are willing to help me out by providing translations for other
% languages, I'm very willing to incorporate them into the code.
%
% \changes{v0.7}{2014/07/14}{Added Finnish language support}
%    \begin{macrocode}
\addto\captionsdutch{%
  \renewcommand{\exercisename}{Oefening}%
  \renewcommand{\exercisesname}{Oefeningen}%
  \renewcommand{\solutionname}{Oplossing}%
  \renewcommand{\solutionsname}{Oplossingen}%
}
\addto\captionsgerman{%
  \renewcommand{\exercisename}{Aufgabe}%
  \renewcommand{\exercisesname}{Aufgaben}%
  \renewcommand{\solutionname}{L\"osung}%
  \renewcommand{\solutionsname}{L\"osungen}%
}
\addto\captionsfrench{%
  \renewcommand{\exercisename}{Exercice}%
  \renewcommand{\exercisesname}{Exercices}%
  \renewcommand{\solutionname}{Solution}%
  \renewcommand{\solutionsname}{Solutions}%
}
\addto\captionsfinnish{
  \renewcommand{\exercisename}{Teht\"av\"a}%
  \renewcommand{\exercisesname}{Teht\"avi\"a}%
  \renewcommand{\solutionname}{Ratkaisu}%
  \renewcommand{\solutionsname}{Ratkaisut}%
}
%    \end{macrocode}
%
%
%
% Now the final hack overloads the basic sectioning commands to make
% sure that they are copied into your solution book.
%
%    \begin{macrocode}
\let\exsol@@makechapterhead\@makechapterhead
\def\@makechapterhead#1{%
  \immediate\write\solutionstream{\string\chapter{#1}}%
  \exsol@@makechapterhead{#1}
}
\ifdefined\frontmatter
  \let\exsol@@frontmatter\frontmatter
  \def\frontmatter{%
    \immediate\write\solutionstream{\string\frontmatter}%
    \exsol@@frontmatter
  }
\fi
\ifdefined\frontmatter
  \let\exsol@@mainmatter\mainmatter
  \def\mainmatter{%
    \immediate\write\solutionstream{\string\mainmatter}%
    \exsol@@mainmatter
  }
\fi
\ifdefined\backmatter
  \let\exsol@@backmatter\backmatter
  \def\backmatter{%
    \immediate\write\solutionstream{\string\backmatter}%
    \exsol@@backmatter
  }
\fi
%    \end{macrocode}
%
% \begin{macro}{\noexercisesinchapter}
%   If you have chapters without exercises, you may want to indicate
%   this clearly into your source. Otherwise empty chapters may appear
%   in your solution book.
%    \begin{macrocode}
\newcommand{\noexercisesinchapter}
{
  \immediate\write\solutionstream{No exercises in this chapter}
}
%    \end{macrocode}
% \end{macro}
%
%    \begin{macrocode}
%</package>
%    \end{macrocode}
%
% \bibliographystyle{alpha}
%
% \begin{thebibliography}{99}
%
% \bibitem{fancyvrb}
% Timothy Van Zandt, Herbert Vo\ss, Denis Girou, Sebastian Rahtz, Niall
% Mansfield 
% \newblock The \texttt{fancyvrb} package.
% \newblock \url{http://ctan.org/pkg/fancyvrb}.
% \newblock online, accessed in January 2012.
%
% \bibitem{CTAN} 
% The Comprehensive TeX Archive Network.
% \newblock \url{http://www.ctan.org}.
% \newblock online, accessed in January 2012.
%
% \end{thebibliography}
%
% \Finale
\endinput

%
% \end{document}
% \end{VerbatimOut}
% \VerbatimInput[frame=lines,gobble=2,fontsize=\footnotesize]{exsol-solutionbook.tex}
% 
% You may generate this solution book, by running \LaTeX{} on the
% file named \texttt{exsol-solutionbook.tex} that is generated when running
% \LaTeX{} on the \texttt{exsol.dtx} file.
%
% The result approximately looks like this:
%
% \setcounter{equation}{0}
% \rule{\linewidth}{.7pt}
% \begin{center}
% {\Large Solutions to the exercises, specified in the \textsf{ExSol} package}\\
% {\large Walter Daems}\\
% {\large 2013/05/12}
% \end{center}
% \par---\newline\textbf{Solution 3.1-1}
%     Let's start by rearranging the equation, a bit:
%     \begin{eqnarray}
%       5.7 x^2 - 3.1 x &=% 5.3\\
%       5.7 x^2 - 3.1 x -5.3 &=% 0
%     \end{eqnarray}
%     The equation is now in the standard form:
%     \begin{equation}
%       a x^2 + b x + c = 0
%     \end{equation}
%     For quadratic equations in the standard form, we know that two
%     solutions exist:
%     \begin{equation}
%       x_{1,2} = \frac{ -b \pm \sqrt{d}}{2a}
%     \end{equation}
%     with
%     \begin{equation}
%       d = b^2 - 4 a c
%     \end{equation}
%     If we apply this to our case, we obtain:
%     \begin{equation}
%       d = (-3.1)^2 - 4 \cdot 5.7 \cdot (-5.3) = 130.45
%     \end{equation}
%     and
%     \begin{eqnarray}
%       x_1 &=& \frac{3.1 + \sqrt{130.45}}{11.4} = 1.27\\
%       x_2 &=& \frac{3.1 + \sqrt{130.45}}{11.4} = -0.73
%     \end{eqnarray}
%     The proposed values $x = x_1, x_2$ are solutions to
%     the given equation.
% \par---\newline\textbf{Solution 3.1-2}
%       This calls for application of Pythagoras' theorem, which
%       tells us:
%       \begin{equation}
%         \left\|A\right\|^2 + \left\|B\right\|^2 = \left\|C\right\|^2
%       \end{equation}
%       and therefore:
%       \begin{eqnarray}
%         \left\|C\right\|
%         &=& \sqrt{\left\|A\right\|^2 + \left\|B\right\|^2}\\
%         &=& \sqrt{3^2 + 4^2}\\
%         &=& \sqrt{25} = 5
%       \end{eqnarray}
%       Therefore, the length of the hypotenuse equals $5$.
%
% \rule{\linewidth}{.7pt}
%
% \subsection{Fiddling with the spacing}
%
% The default spacing provided by the \textsf{ExSol} package should be
% fine for most users. However, if you like to tweak, below you can
% find the controls.
% \subsubsection{Spacing before and after the \texttt{exercises} environment}
%
% The lengths below control the spacing of the |exercises| environment:
% \begin{itemize}
% \item |exsolexerciseaboveskip|: rubber length controlling the
% vertical space after the top marker line of the environment
% \item |exsolexercisebelowskip|: rubber length controlling the
% vertical space before the bottom marker line of the environment
% \end{itemize}
%
% You can simply specify them like:
% \begin{verbatim}
% \setlength{\exsolexercisesaboveskip}{1ex plus 1pt minus 1pt}
% \setlength{\exsolexercisesbelowskip}{1ex plus 1pt minus 1pt}
% \end{verbatim}
% The spacings specified here are the package defaults.
%
% \subsubsection{Spacing of the individual exercises}
% Caution: the spacing can only be tuned, when one invokes the
% |exerciseaslist| package option!
%
% Then lengths below control the spacing of the |exercise| environment:
% \begin{itemize}
% \item |exercisetopbottomsep|: rubber length controlling the vertical
% space before and after individual exercises
% \item |exerciseleftmargin|: length controlling the horizontal
% space between the surrounding environment's left margin (most
% often the page margin) and the left edge of the exercise
% environment 
% \item |exerciseleftmargin|: length controlling the horizontal
% space between the surrounding environment's right margin (most
% often the page margin) and the right edge of the exercise
% environment
% \item |exerciseitemindent|: length controlling the first-line
% indentation of the first paragraph in the exercise environment
% (actually, the label is set w.r.t. this position, that we will
% conveniently call position 'x')
% \item |exerciseparindent|: length controlling the first-line
% indentation of the other paragraphs in the exercise environment.
% \item |exerciselabelsep|: length controlling the distance between
% the label and position 'x'
% \item |exerciselabelwidth|: minimal width of the (internally
% right-alligned) box to use for the exercises label; if the box is
% not sufficiently big, position 'x' is shifted to the right
% \item |exerciseparsep|: internal paragraph separation (vertically)
% \end{itemize}
% 
% You can simply specify them like:
% \begin{verbatim}
% \setlength{\exsolexercisetopbottomsep}{0pt plus 0pt minus 1pt}
% \setlength{\exsolexerciseleftmargin}{1em}
% \setlength{\exsolexerciserightmargin}{1em}
% \setlength{\exsolexerciseparindent}{0em}
% \setlength{\exsolexerciselabelsep}{0.5em}
% \setlength{\exsolexerciselabelwidth}{0pt}
% \setlength{\exsolexerciseitemindent}{0pt}
% \setlength{\exsolexerciseparsep}{\parskip}
% \end{verbatim}
% The spacings specified here are the package defaults.
%
% \subsection{Tips and tricks}
%
% If you want to include the solutions all at the
% end of the current document, you need to explicitly close the
% solution stream before including it:
% \begin{verbatim}
%   \closeout\solutionstream\input{\jobname.sol.tex}
% \end{verbatim}
%
% If you want to avoid exercises being split by a page boundary, then
% provide the package option 'minipage'. This causes the exercises to
% be wrapped in a minipage environment.
% 
% \clearpage
%
% \section{Implementation}
% %%%%%%%%%%%%%%%%%%%%%%%
%    \begin{macrocode}
%<*package>
%    \end{macrocode}
%
% \subsection{Auxiliary packages}
% %%%%%%%%%%%%%%%%%%%%%%%%%%%%%%%
% The package uses some auxiliary packages:
%    \begin{macrocode}
\RequirePackage{fancyvrb}
\RequirePackage{ifthen}
\RequirePackage{kvoptions}
%    \end{macrocode}
%
% \subsection{Package options}
% %%%%%%%%%%%%%%%%%%%%%%%%%%%%
% The package offers some options:
%
% \changes{v0.2}{2012/01/06}{Added option exercisesfont}
% \changes{v0.4}{2012/01/09}{Changed name of option to exercisesfontsize}
%
% \begin{macro}{exercisesfontsize}
%  This option allows setting the font of the \texttt{exercises}
%  environment. You may chopse one of tiny, scriptsize, footnotesize,
%  small, normalsize, large, etc.\\
%  E.g., \texttt{[exercisesfontsize=small]}.
%    \begin{macrocode}
\DeclareStringOption[normalsize]{exercisesfontsize}
%    \end{macrocode}
% \end{macro}
%
% \changes{v0.4}{2012/01/06}{Added option exercisesinlist}
% \changes{v0.5}{2012/01/09}{Changed option exercisesinlist to exerciseaslist}
%
% \begin{macro}{exerciseaslist}
%  This boolean option (true, false) allows setting the typesetting of
%  the \texttt{exercises} in a list environment. This causes the
%  exercises to be typeset in a more compact fashion, with indented
%  left and right margin. 
%    \begin{macrocode}
\DeclareBoolOption[false]{exerciseaslist}
%    \end{macrocode}
% \end{macro}
%
% \changes{v0.5}{2012/01/09}{Added option copyexercisesinsolutions}
% \begin{macro}{copyexercisesinsolutions}
%  This boolean option (true, false) allows copying the exercises in
%  the solutions file, to allow for making a complete stand-alone
%  exercises bundle.
%    \begin{macrocode}
\DeclareBoolOption[false]{copyexercisesinsolutions}
%    \end{macrocode}
% \end{macro}
%
% \changes{v0.9}{2014/07/28}{. Changed default behavior
% w.r.t. minipage-wraping of exercises}
% \begin{macro}{minipage}
%  This boolean option (true, false) causes the exercises to be
%  wrapped in minipages. This avoids them getting split by a page
%  boundary.
%    \begin{macrocode}
\DeclareBoolOption[false]{minipage}
%    \end{macrocode}
% \end{macro}
%
% The options are processed using:
%    \begin{macrocode}
\ProcessKeyvalOptions*
%    \end{macrocode}
%
% The options are subsequently handled
%    \begin{macrocode}
\newcommand{\exercisesfontsize}{\csname \exsol@exercisesfontsize\endcsname}
%    \end{macrocode}
%
%
% \subsection{Customization of lengths}
% %%%%%%%%%%%%%%%%%%%%%%%%%%%%%%%%%%%%%%%
% The commands below allow customizing many lengths that control the
% typesetting of the exercises.
%
% \changes{v0.91}{2014/08/31}{added user-accessible lengths}
% First some lengths to control the spacing before and after |exercises|.
%    \begin{macrocode}
\newlength{\exsolexercisesaboveskip}
\setlength{\exsolexercisesaboveskip}{1ex plus 1pt minus 1pt}
\newlength{\exsolexercisesbelowskip}
\setlength{\exsolexercisesbelowskip}{1ex plus 1pt minus 1pt}
%    \end{macrocode}
%
% Then some lengths to control the spacing for a single
% exercise. These lengths only work when the |exerciseaslist| package
% option has been specified. Sensible defaults have been set.
%    \begin{macrocode}
\newlength{\exsolexercisetopbottomsep}
\setlength{\exsolexercisetopbottomsep}{0pt plus 0pt minus 1pt}
\newlength{\exsolexerciseleftmargin}
\setlength{\exsolexerciseleftmargin}{1em}
\newlength{\exsolexerciserightmargin}
\setlength{\exsolexerciserightmargin}{1em}
\newlength{\exsolexerciseparindent}
\setlength{\exsolexerciseparindent}{0em}
\newlength{\exsolexerciselabelsep}
\setlength{\exsolexerciselabelsep}{0.5em}
\newlength{\exsolexerciselabelwidth}
\setlength{\exsolexerciselabelwidth}{0pt}
\newlength{\exsolexerciseitemindent}
\setlength{\exsolexerciseitemindent}{0pt}
\newlength{\exsolexerciseparsep}
\setlength{\exsolexerciseparsep}{\parskip}
%    \end{macrocode}
% 
% 
% \subsection{Con- and destruction of the auxiliary streams}
% %%%%%%%%%%%%%%%%%%%%%%%%%%%%%%%%%%%%%%%%%%%%%%%%%%%%%%%%%%
% At the beginning of your document, we start by opening a stream to a
% file that will be used to write the solutions to. At the end of your
% document, the package closes the stream.
% \changes{v0.8}{2014/07/15}{moved newwrite of exercise stream to this
% spot to avoid consuming all handles}
%    \begin{macrocode}
\AtBeginDocument{
  \newwrite\solutionstream
  \immediate\openout\solutionstream=\jobname.sol.tex
  \newwrite\exercisestream
}
\AtEndDocument{
  \immediate\closeout\solutionstream
}
%    \end{macrocode}
%
% \subsection{Exercises counter}
% %%%%%%%%%%%%%%%%%%%%%%%%%%%%%%
% By providing an exercise counter, proper numbering of the exercises
% is provided to allow for good cross referencing of the solutions to
% the exercises.
% \changes{v0.2}{2012/01/06}{Removed dash in counter when in document
% without sectioning commands}
%    \begin{macrocode}
\newcounter{exercise}[subsection]
\setcounter{exercise}{0}
\renewcommand{\theexercise}{%
  \@ifundefined{c@chapter}{}{\if0\arabic{chapter}\else\arabic{chapter}.\fi}%
  \if0\arabic{section}\else\arabic{section}\fi%
  \if0\arabic{subsection}\else.\arabic{subsection}\fi%
  \if0\arabic{subsubsection}\else.\arabic{subsubsection}\fi%
  \if0\arabic{exercise}\else%
    \@ifundefined{c@chapter}%
                 {\if0\arabic{section}\else-\fi}%
                 {-}%
    \arabic{exercise}%
  \fi
}
%    \end{macrocode}
%
%
%
% \subsection{Detokenization in order to cope with utf8}
%
% Combining old-school \LaTeX{} (before \XeTeX{} and \LuaTeX{}) and
% UTF-8 is a pain.
% Detokenization has been suggested by Geoffrey Poore to solve issues
% with UTF-8 characters messing up the |fancyvrb| internals.
% \changes{v0.7}{2014/07/14}{Added detokenized writing}
%    \begin{macrocode}
\newcommand{\GPES@write@detok}[1]{%
  \immediate\write\exercisestream{\detokenize{#1}}}%
\newcommand{\GPSS@write@detok}[1]{%
  \immediate\write\solutionstream{\detokenize{#1}}}%
\newcommand{\GPESS@write@detok}[1]{%
  \GPES@write@detok{#1}%
  \GPSS@write@detok{#1}}%
%    \end{macrocode}
%
%
% \section{The user environments}
%
% \begin{macro}{exercise}
%   The \texttt{exercise} environment is used to typeset your
%   exercises, provide them with a nice label and allow for copying
%   the exercise to the solutions file (if the package option
%   \texttt{copyexercisesinsolution}) is set. The label can be
%   set by redefining the \cs{exercisename} macro, or by relying on
%   the \textsf{Babel} provisions. The code is almost litteraly
%   taken from the \textsf{fancyvrb} package.
%    \begin{macrocode}
\def\exercise{\FV@Environment{}{exercise}}
\def\FVB@exercise{%
  \refstepcounter{exercise}%
  \immediate\openout\exercisestream=\jobname.exc.tex
  \ifexsol@copyexercisesinsolutions
    \typeout{Writing exercise to \jobname.sol.tex}
    \immediate\write\solutionstream{\string\par---\string\newline
      \string\textbf\string{\exercisename{} \theexercise \string}}
  \else
    \immediate\write\solutionstream{\string\par---\string\newline}
  \fi
  \immediate\write\exercisestream{\string\begin{exsol@exercise}}
  \@bsphack
  \begingroup
    \FV@UseKeyValues
    \FV@DefineWhiteSpace
    \def\FV@Space{\space}%
    \FV@DefineTabOut
    \ifexsol@copyexercisesinsolutions
      \let\FV@ProcessLine\GPESS@write@detok %
    \else
      \let\FV@ProcessLine\GPES@write@detok %
    \fi
    \relax
    \let\FV@FontScanPrep\relax
    \let\@noligs\relax
    \FV@Scan
  }
\def\FVE@exercise{
  \endgroup\@esphack
  \immediate\write\exercisestream{\string\end{exsol@exercise}}
  \ifexsol@copyexercisesinsolutions
    \immediate\write\solutionstream{\string~\string\newline}
  \fi
  \immediate\closeout\exercisestream
  \input{\jobname.exc.tex}
}
\DefineVerbatimEnvironment{exercise}{exercise}{}
%    \end{macrocode}
% \end{macro}
%
% \begin{macro}{exsol@exercise}
%   The \texttt{exsol@exercise} environment is an internal macro used
%   to typeset your exercises and provide them with a nice label and
%   number. Do not use it directly. Use the proper environment
%   \texttt{exercise} instead.
%   \changes{v0.2}{2012/01/06}{Attempted to fix MiKTeX formatting problems}
%   \changes{v0.3}{2012/01/08}{Fixed labelsep to avoid cluttered
%   itemize environments}
%   \changes{v0.4}{2012/01/06}{Added option exercisesinlist such that
%   default results in non list formatting of exercise}
%   \changes{v0.5}{2012/01/09}{Changed implementation to allow for
%   copying the exercises to the solutions file.}
%    \begin{macrocode}
\newenvironment{exsol@exercise}[0]
{%
  \ifthenelse{\boolean{exsol@minipage}}{\begin{minipage}[t]{\textwidth}}{}%
    \ifthenelse{\boolean{exsol@exerciseaslist}}
               {\begin{list}%
                   {%
                   }%
                   {%
                     \setlength{\topsep}{\exsolexercisetopbottomsep}%
                     \setlength{\leftmargin}{\exsolexerciseleftmargin}%
                     \setlength{\rightmargin}{\exsolexerciserightmargin}%
                     \setlength{\listparindent}{\exsolexerciseparindent}%
                     \setlength{\itemindent}{\exsolexerciseitemindent}%
                     \setlength{\parsep}{\exsolexerciseparsep}
                     \setlength{\labelsep}{\exsolexerciselabelsep}
                     \setlength{\labelwidth}{\exsolexerciselabelwidth}}
                 \item[\textit{~\exercisename{} \theexercise:~}]
               }%
               {\textit{\exercisename{} \theexercise:}}
}
{%
  \ifthenelse{\boolean{exsol@exerciseaslist}}%
             {\end{list}}{}%
  \ifthenelse{\boolean{exsol@minipage}}{\end{minipage}}{\par}%
}
%    \end{macrocode}
% \end{macro}
%
%
% \begin{macro}{solution}
%   The \texttt{solution} environment is used to typeset your solutions
%   and provide them with a nice label and number that corresponds to
%   the exercise that preceeded this solution. Theno label can be
%   set by redefining the \cs{solutionname} macro, or by relying on
%   the \textsf{Babel} provisions. The code is almost litteraly
%   taken from the \textsf{fancyvrb} package.
%    \begin{macrocode}
\def\solution{\FV@Environment{}{solution}}
\def\FVB@solution{%
  \typeout{Writing solution to \jobname.sol.tex}
  \immediate\write\solutionstream{\string\textbf\string{\solutionname{}\string}}
  \ifexsol@copyexercisesinsolutions
    \immediate\write\solutionstream{\string\newline}
  \else
    \immediate\write\solutionstream{\string\textbf\string{\theexercise\string}%
                                    \string\newline}
  \fi
  \@bsphack
  \begingroup
    \FV@UseKeyValues
    \FV@DefineWhiteSpace
    \def\FV@Space{\space}%
    \FV@DefineTabOut
    \let\FV@ProcessLine\GPSS@write@detok %
    \relax
    \let\FV@FontScanPrep\relax
    \let\@noligs\relax
    \FV@Scan
  }
\def\FVE@solution{\endgroup\@esphack}
\DefineVerbatimEnvironment{solution}{solution}{}
%    \end{macrocode}
% \end{macro}
%
% \begin{macro}{exercises}
%   The \texttt{exercises} environment helps typesetting your exercises to
%   stand out from the rest of the text. You may use it at the end of
%   a chapter, or just to group some exercises in the text.
%   \changes{v0.2}{2012/01/06}{Attempted to fix MiKTeX formatting problems}
%   \changes{v0.3}{2012/01/07}{Added some extra whitespace below exercisesname}
%    \begin{macrocode}
\newenvironment{exercises}
{\par\exercisesfontsize\rule{.25\linewidth}{0.15mm}\vspace*{\exsolexercisesaboveskip}\\*%
 \textbf{\normalsize \exercisesname}}
{\vspace*{-\baselineskip}\vspace*{\exsolexercisesbelowskip}\rule{.25\linewidth}{0.15mm}\par}
%    \end{macrocode}
% \end{macro}
%
% \subsection{Some Babel provisions}
% %%%%%%%%%%%%%%%%%%%%%%%%%%%%%%%%%%
% \changes{v0.2}{2012/01/06}{Fixed babel errors}
% \begin{macro}{\exercisename}
%   The exercise environment makes use of a label \texttt{\exercisename{}}
%   macro.
%    \begin{macrocode}
\newcommand{\exercisename}{Exercise}
%    \end{macrocode}
% \end{macro}
%
% \begin{macro}{\exercisesname}
%   The exercises environment makes use of a label \texttt{\exercisesname{}}
%   macro.
%    \begin{macrocode}
\newcommand{\exercisesname}{Exercises}
%    \end{macrocode}
% \end{macro}
% 
% \begin{macro}{\solutionname}
%   The solution environment makes use of a label \texttt{\solutionname{}}
%   macro.
%    \begin{macrocode}
\newcommand{\solutionname}{Solution}
%    \end{macrocode}
% \end{macro}
%
% \begin{macro}{\solutionname}
%   The solution environment makes use of a label \texttt{\solutionname{}}
%   macro.
% \changes{v0.8}{2014/07/15}{Added missing babel tag}
%    \begin{macrocode}
\newcommand{\solutionsname}{Solutions}
%    \end{macrocode}
% \end{macro}
% 
% 
% You may redefine these macros, but to help you out a little bit, we
% provide with some basic Babel auxiliaries. If you're a true polyglot
% and are willing to help me out by providing translations for other
% languages, I'm very willing to incorporate them into the code.
%
% \changes{v0.7}{2014/07/14}{Added Finnish language support}
%    \begin{macrocode}
\addto\captionsdutch{%
  \renewcommand{\exercisename}{Oefening}%
  \renewcommand{\exercisesname}{Oefeningen}%
  \renewcommand{\solutionname}{Oplossing}%
  \renewcommand{\solutionsname}{Oplossingen}%
}
\addto\captionsgerman{%
  \renewcommand{\exercisename}{Aufgabe}%
  \renewcommand{\exercisesname}{Aufgaben}%
  \renewcommand{\solutionname}{L\"osung}%
  \renewcommand{\solutionsname}{L\"osungen}%
}
\addto\captionsfrench{%
  \renewcommand{\exercisename}{Exercice}%
  \renewcommand{\exercisesname}{Exercices}%
  \renewcommand{\solutionname}{Solution}%
  \renewcommand{\solutionsname}{Solutions}%
}
\addto\captionsfinnish{
  \renewcommand{\exercisename}{Teht\"av\"a}%
  \renewcommand{\exercisesname}{Teht\"avi\"a}%
  \renewcommand{\solutionname}{Ratkaisu}%
  \renewcommand{\solutionsname}{Ratkaisut}%
}
%    \end{macrocode}
%
%
%
% Now the final hack overloads the basic sectioning commands to make
% sure that they are copied into your solution book.
%
%    \begin{macrocode}
\let\exsol@@makechapterhead\@makechapterhead
\def\@makechapterhead#1{%
  \immediate\write\solutionstream{\string\chapter{#1}}%
  \exsol@@makechapterhead{#1}
}
\ifdefined\frontmatter
  \let\exsol@@frontmatter\frontmatter
  \def\frontmatter{%
    \immediate\write\solutionstream{\string\frontmatter}%
    \exsol@@frontmatter
  }
\fi
\ifdefined\frontmatter
  \let\exsol@@mainmatter\mainmatter
  \def\mainmatter{%
    \immediate\write\solutionstream{\string\mainmatter}%
    \exsol@@mainmatter
  }
\fi
\ifdefined\backmatter
  \let\exsol@@backmatter\backmatter
  \def\backmatter{%
    \immediate\write\solutionstream{\string\backmatter}%
    \exsol@@backmatter
  }
\fi
%    \end{macrocode}
%
% \begin{macro}{\noexercisesinchapter}
%   If you have chapters without exercises, you may want to indicate
%   this clearly into your source. Otherwise empty chapters may appear
%   in your solution book.
%    \begin{macrocode}
\newcommand{\noexercisesinchapter}
{
  \immediate\write\solutionstream{No exercises in this chapter}
}
%    \end{macrocode}
% \end{macro}
%
%    \begin{macrocode}
%</package>
%    \end{macrocode}
%
% \bibliographystyle{alpha}
%
% \begin{thebibliography}{99}
%
% \bibitem{fancyvrb}
% Timothy Van Zandt, Herbert Vo\ss, Denis Girou, Sebastian Rahtz, Niall
% Mansfield 
% \newblock The \texttt{fancyvrb} package.
% \newblock \url{http://ctan.org/pkg/fancyvrb}.
% \newblock online, accessed in January 2012.
%
% \bibitem{CTAN} 
% The Comprehensive TeX Archive Network.
% \newblock \url{http://www.ctan.org}.
% \newblock online, accessed in January 2012.
%
% \end{thebibliography}
%
% \Finale
\endinput

%
% When running \LaTeX{} on your document (in our case on the
% \texttt{exsol.dtx} file, as a side effect a file with extension
% \texttt{.sol.tex} has been written to disk (in our case, the file
% \texttt{exsol.sol.tex}), containing all solutions in sequence.
%
% Generating a solution book is a simple as including the file into a
% simple \LaTeX{} harness, that allows you giving it a proper title page and to
% add other bells and whistles.
%
% E.g.,
% \begin{VerbatimOut}{exsol-solutionbook.tex}
% \documentclass{article}
% \usepackage[english]{babel}
% \title{Solutions to the exercises, specified in the \textsf{ExSol} package}
% \author{Walter Daems}
% \date{2013/05/12}
%
% \begin{document}
%
% \maketitle
%
% % \iffalse meta-comment
%
% Copyright (C) 2014 by Walter Daems <walter.daems@uantwerpen.be>
%
% This work may be distributed and/or modified under the conditions of
% the LaTeX Project Public License, either version 1.3 of this license
% or (at your option) any later version.  The latest version of this
% license is in:
% 
%    http://www.latex-project.org/lppl.txt
% 
% and version 1.3 or later is part of all distributions of LaTeX version
% 2005/12/01 or later.
%
% This work has the LPPL maintenance status `maintained'.
% 
% The Current Maintainer of this work is Walter Daems.
%
% This work consists of the files exsol.dtx and exsol.ins and the derived 
% files:
%   - exsol.sty
%   - example.tex
%   - example-solutionbook.tex
%
% \fi
%
% \iffalse
%
%<package|driver>\NeedsTeXFormat{LaTeX2e}
%<driver>\ProvidesFile{exsol.dtx}
%<package>\ProvidesPackage{exsol}
%<package|driver>  [2014/08/31 v0.91 ExSol - Exercises and Solutions package (DMW)]
%<*driver> 
\documentclass[11pt]{ltxdoc}
\usepackage[english]{babel}
\usepackage[exercisesfontsize=small]{exsol}
\usepackage{metalogo}
\EnableCrossrefs
\CodelineIndex
\RecordChanges
\usepackage{makeidx}
\usepackage{alltt}
\IfFileExists{tocbibind.sty}{\usepackage{tocbibind}}{}
\IfFileExists{hyperref.sty}{\usepackage[bookmarksopen]{hyperref}}{}
\EnableCrossrefs         
\CodelineIndex
\RecordChanges
\newcommand{\exsol}{\textsf{ExSol}}
\StopEventually{\PrintChanges\PrintIndex}
\def\fileversion{0.91}
\def\filedate{2014/08/31}
\begin{document}
 \DocInput{exsol.dtx}
\end{document}
%</driver>
% \fi
%
% \CheckSum{0}
%
% \CharacterTable
%  {Upper-case    \A\B\C\D\E\F\G\H\I\J\K\L\M\N\O\P\Q\R\S\T\U\V\W\X\Y\Z
%   Lower-case    \a\b\c\d\e\f\g\h\i\j\k\l\m\n\o\p\q\r\s\t\u\v\w\x\y\z
%   Digits        \0\1\2\3\4\5\6\7\8\9
%   Exclamation   \!     Double quote  \"     Hash (number) \#
%   Dollar        \$     Percent       \%     Ampersand     \&
%   Acute accent  \'     Left paren    \(     Right paren   \)
%   Asterisk      \*     Plus          \+     Comma         \,
%   Minus         \-     Point         \.     Solidus       \/
%   Colon         \:     Semicolon     \;     Less than     \<
%   Equals        \=     Greater than  \>     Question mark \?
%   Commercial at \@     Left bracket  \[     Backslash     \\
%   Right bracket \]     Circumflex    \^     Underscore    \_
%   Grave accent  \`     Left brace    \{     Vertical bar  \|
%   Right brace   \}     Tilde         \~}
%
%
% \changes{v0.1}{2012/01/05}{. Initial version}
% \changes{v0.2}{2012/01/06}{. Minor bug fixes based on first use by
% Paul Levrie}
% \changes{v0.3}{2012/01/07}{. Minor bug fixes based on second use by
% Paul}
% \changes{v0.4}{2012/01/09}{. Allowed for non-list formatting of
% exercises (as default)}
% \changes{v0.5}{2012/01/15}{. Added option to also send exercises to
% solutions file}
% \changes{v0.6}{2013/05/12}{. Prepared for CTAN publication}
% \changes{v0.7}{2014/07/14}{. Fixed UTF8 compatibility issues}
% \changes{v0.8}{2014/07/15}{. Fixed missing babel tag and running out
% of write hanles}
% \changes{v0.9}{2014/07/28}{. Changed default behavior
% w.r.t. minipage-wraping of exercises} 
% \changes{v0.91}{2014/08/31}{. Corrected minipage dependence, made }
%
% \DoNotIndex{\newcommand,\newenvironment}
% \setlength{\parindent}{0em}
% \addtolength{\parskip}{0.5\baselineskip}
%
% \title{The \exsol{} package\thanks{This document
%   corresponds to exsol~\fileversion, dated \filedate.}}
% \author{Walter Daems (\texttt{walter.daems@uantwerpen.be})}
% \date{}
%
% \maketitle
%
% \section{Introduction}
% %%%%%%%%%%%%%%%%%%%%%%
% The package \exsol{} provides macros to allow
% embedding exercises and solutions in the \LaTeX{} source of an
% instructional text (e.g., a book or a course text) while generating
% the following separate documents:
% \begin{itemize}
% \item your original text that only contains the exercises, and
% \item a solution book that only contains the solutions to the
% exercises (a package option exists to also copy the exercises themselves to the solution book).
% \end{itemize}
% 
% The former is generated when running \LaTeX{} on your document. This
% run writes the solutions to a secondary file that can be included
% into a simple document harness, such that when running \LaTeX{} on
% the latter, you can generate a nice solution book.
% 
% Why use \exsol{}?
% \begin{itemize}
% \item It allows to keep the \LaTeX{} source of your exercises and their
% solutions in a single file. Away with the nightmare to keep your
% solutions in sync with the original text.
% \item It separates exercises and solutions, allowing you
%   \begin{itemize}
%   \item to only release the solution book to the instructors of the
%   course;
%   \item to encourage students that you provide with the solution
%   book to first try solving the exercises without opening the book;
%   this seems to be easier than not peeking into the solution of an
%   exercise that is typeset just below the exercise itself.
%   \end{itemize}
% \end{itemize}
%
% The code of the \exsol{} package was taken almost literally
% from \textsf{fancyvrb} \cite{fancyvrb}. Therefore, all credits go to the
% authors/maintainers of \textsf{fancyvrb}.
%
% Thanks to Pieter Pareit and Pekka Pere for signaling problems and
% making suggestions for the documentation.
%
% \section{Installation}
% %%%%%%%%%%%%%%%%%%%%%%
% Either you are a package manager and then you'll know how to
% prepare an installation package for \exsol{}.
%
% Either you are a normal user and then you have two options. First,
% check if there is a package that your favorite \LaTeX{}
% distributor has prepared for you. Second, grab the TDS package
% from CTAN \cite{CTAN} (\texttt{exsol.tds.zip}) and unzip it somewhere in your
% own TDS tree, regenerate your filename database and off you go.
% In any case, make sure that \LaTeX{} finds the \texttt{exsol.sty} file.
%
% The \exsol{} package uses some auxiliary packages: \textsf{fancyvrb},
% \textsf{ifthen}, \textsf{kvoptions} and, optionally,
% \textsf{babel}. Fetch them from CTAN \cite{CTAN} if your \TeX{}
% distributor does not provide them.
%
% \section{Usage}
% %%%%%%%%%%%%%%%
% 
% \subsection{Preparing your document source}
% %%%%%%%%%%%%%%%%%%%%%%%%%%%%%%%%%%%%%%%%%%%
% The macro package exsol can be loaded with:
% \begin{verbatim}
% \usepackage{exsol}
% \end{verbatim}
%
% Then, you are ready to add some exercises including their solution
% to your document source. To this end, embed them in a
% \texttt{exercise} and a corresponding \texttt{solution} environment.
% Optionally, you may embed several of them in a \texttt{exercises}
% environment, to make them stand out in your text.
% E.g.,
%
% \begin{VerbatimOut}{exsol.tmp}
% 
% \begin{exercises}
%
%   \begin{exercise}
%     Solve the following equation for $x \in C$, with $C$ the set of
%     complex numbers:
%     \begin{equation}
%       5 x^2 -3 x = 5
%     \end{equation}
%   \end{exercise}
%   \begin{solution}
%     Let's start by rearranging the equation, a bit:
%     \begin{eqnarray}
%       5.7 x^2 - 3.1 x &=& 5.3\\
%       5.7 x^2 - 3.1 x -5.3 &=& 0
%     \end{eqnarray}
%     The equation is now in the standard form:
%     \begin{equation}
%       a x^2 + b x + c = 0
%     \end{equation}
%     For quadratic equations in the standard form, we know that two
%     solutions exist:
%     \begin{equation}
%       x_{1,2} = \frac{ -b \pm \sqrt{d}}{2a}
%     \end{equation}
%     with
%     \begin{equation}
%       d = b^2 - 4 a c
%     \end{equation}
%     If we apply this to our case, we obtain:
%     \begin{equation}
%       d = (-3.1)^2 - 4 \cdot 5.7 \cdot (-5.3) = 130.45
%     \end{equation}
%     and
%     \begin{eqnarray}
%       x_1 &=& \frac{3.1 + \sqrt{130.45}}{11.4} = 1.27\\
%       x_2 &=& \frac{3.1 - \sqrt{130.45}}{11.4} = -0.73
%     \end{eqnarray}
%     The proposed values $x = x_1, x_2$ are solutions to the given equation.
%   \end{solution}
%   \begin{exercise}
%     Consider a 2-dimensional vector space equipped with a Euclidean
%     distance function. Given a right-angled triangle, with the sides
%     $A$ and $B$ adjacent to the right angle having lengths, $3$ and
%     $4$, calculate the length of the hypotenuse, labeled $C$.
%   \end{exercise}
%   \begin{solution}
%     This calls for application of Pythagoras' theorem, which 
%     tells us:
%     \begin{equation}
%       \left\|A\right\|^2 + \left\|B\right\|^2 = \left\|C\right\|^2
%     \end{equation}
%     and therefore:
%     \begin{eqnarray}
%       \left\|C\right\| 
%       &=& \sqrt{\left\|A\right\|^2 + \left\|B\right\|^2}\\
%       &=& \sqrt{3^2 + 4^2}\\
%       &=& \sqrt{25} = 5
%     \end{eqnarray}
%     Therefore, the length of the hypotenuse equals $5$.
%   \end{solution}
%
% \end{exercises}
% \end{VerbatimOut}
% \VerbatimInput[frame=lines,gobble=2,fontsize=\footnotesize]{exsol.tmp}
%
% The result in the original document, can be seen below. As you can
% see, there's no trace of the solution. 
%
% % \iffalse meta-comment
%
% Copyright (C) 2014 by Walter Daems <walter.daems@uantwerpen.be>
%
% This work may be distributed and/or modified under the conditions of
% the LaTeX Project Public License, either version 1.3 of this license
% or (at your option) any later version.  The latest version of this
% license is in:
% 
%    http://www.latex-project.org/lppl.txt
% 
% and version 1.3 or later is part of all distributions of LaTeX version
% 2005/12/01 or later.
%
% This work has the LPPL maintenance status `maintained'.
% 
% The Current Maintainer of this work is Walter Daems.
%
% This work consists of the files exsol.dtx and exsol.ins and the derived 
% files:
%   - exsol.sty
%   - example.tex
%   - example-solutionbook.tex
%
% \fi
%
% \iffalse
%
%<package|driver>\NeedsTeXFormat{LaTeX2e}
%<driver>\ProvidesFile{exsol.dtx}
%<package>\ProvidesPackage{exsol}
%<package|driver>  [2014/08/31 v0.91 ExSol - Exercises and Solutions package (DMW)]
%<*driver> 
\documentclass[11pt]{ltxdoc}
\usepackage[english]{babel}
\usepackage[exercisesfontsize=small]{exsol}
\usepackage{metalogo}
\EnableCrossrefs
\CodelineIndex
\RecordChanges
\usepackage{makeidx}
\usepackage{alltt}
\IfFileExists{tocbibind.sty}{\usepackage{tocbibind}}{}
\IfFileExists{hyperref.sty}{\usepackage[bookmarksopen]{hyperref}}{}
\EnableCrossrefs         
\CodelineIndex
\RecordChanges
\newcommand{\exsol}{\textsf{ExSol}}
\StopEventually{\PrintChanges\PrintIndex}
\def\fileversion{0.91}
\def\filedate{2014/08/31}
\begin{document}
 \DocInput{exsol.dtx}
\end{document}
%</driver>
% \fi
%
% \CheckSum{0}
%
% \CharacterTable
%  {Upper-case    \A\B\C\D\E\F\G\H\I\J\K\L\M\N\O\P\Q\R\S\T\U\V\W\X\Y\Z
%   Lower-case    \a\b\c\d\e\f\g\h\i\j\k\l\m\n\o\p\q\r\s\t\u\v\w\x\y\z
%   Digits        \0\1\2\3\4\5\6\7\8\9
%   Exclamation   \!     Double quote  \"     Hash (number) \#
%   Dollar        \$     Percent       \%     Ampersand     \&
%   Acute accent  \'     Left paren    \(     Right paren   \)
%   Asterisk      \*     Plus          \+     Comma         \,
%   Minus         \-     Point         \.     Solidus       \/
%   Colon         \:     Semicolon     \;     Less than     \<
%   Equals        \=     Greater than  \>     Question mark \?
%   Commercial at \@     Left bracket  \[     Backslash     \\
%   Right bracket \]     Circumflex    \^     Underscore    \_
%   Grave accent  \`     Left brace    \{     Vertical bar  \|
%   Right brace   \}     Tilde         \~}
%
%
% \changes{v0.1}{2012/01/05}{. Initial version}
% \changes{v0.2}{2012/01/06}{. Minor bug fixes based on first use by
% Paul Levrie}
% \changes{v0.3}{2012/01/07}{. Minor bug fixes based on second use by
% Paul}
% \changes{v0.4}{2012/01/09}{. Allowed for non-list formatting of
% exercises (as default)}
% \changes{v0.5}{2012/01/15}{. Added option to also send exercises to
% solutions file}
% \changes{v0.6}{2013/05/12}{. Prepared for CTAN publication}
% \changes{v0.7}{2014/07/14}{. Fixed UTF8 compatibility issues}
% \changes{v0.8}{2014/07/15}{. Fixed missing babel tag and running out
% of write hanles}
% \changes{v0.9}{2014/07/28}{. Changed default behavior
% w.r.t. minipage-wraping of exercises} 
% \changes{v0.91}{2014/08/31}{. Corrected minipage dependence, made }
%
% \DoNotIndex{\newcommand,\newenvironment}
% \setlength{\parindent}{0em}
% \addtolength{\parskip}{0.5\baselineskip}
%
% \title{The \exsol{} package\thanks{This document
%   corresponds to exsol~\fileversion, dated \filedate.}}
% \author{Walter Daems (\texttt{walter.daems@uantwerpen.be})}
% \date{}
%
% \maketitle
%
% \section{Introduction}
% %%%%%%%%%%%%%%%%%%%%%%
% The package \exsol{} provides macros to allow
% embedding exercises and solutions in the \LaTeX{} source of an
% instructional text (e.g., a book or a course text) while generating
% the following separate documents:
% \begin{itemize}
% \item your original text that only contains the exercises, and
% \item a solution book that only contains the solutions to the
% exercises (a package option exists to also copy the exercises themselves to the solution book).
% \end{itemize}
% 
% The former is generated when running \LaTeX{} on your document. This
% run writes the solutions to a secondary file that can be included
% into a simple document harness, such that when running \LaTeX{} on
% the latter, you can generate a nice solution book.
% 
% Why use \exsol{}?
% \begin{itemize}
% \item It allows to keep the \LaTeX{} source of your exercises and their
% solutions in a single file. Away with the nightmare to keep your
% solutions in sync with the original text.
% \item It separates exercises and solutions, allowing you
%   \begin{itemize}
%   \item to only release the solution book to the instructors of the
%   course;
%   \item to encourage students that you provide with the solution
%   book to first try solving the exercises without opening the book;
%   this seems to be easier than not peeking into the solution of an
%   exercise that is typeset just below the exercise itself.
%   \end{itemize}
% \end{itemize}
%
% The code of the \exsol{} package was taken almost literally
% from \textsf{fancyvrb} \cite{fancyvrb}. Therefore, all credits go to the
% authors/maintainers of \textsf{fancyvrb}.
%
% Thanks to Pieter Pareit and Pekka Pere for signaling problems and
% making suggestions for the documentation.
%
% \section{Installation}
% %%%%%%%%%%%%%%%%%%%%%%
% Either you are a package manager and then you'll know how to
% prepare an installation package for \exsol{}.
%
% Either you are a normal user and then you have two options. First,
% check if there is a package that your favorite \LaTeX{}
% distributor has prepared for you. Second, grab the TDS package
% from CTAN \cite{CTAN} (\texttt{exsol.tds.zip}) and unzip it somewhere in your
% own TDS tree, regenerate your filename database and off you go.
% In any case, make sure that \LaTeX{} finds the \texttt{exsol.sty} file.
%
% The \exsol{} package uses some auxiliary packages: \textsf{fancyvrb},
% \textsf{ifthen}, \textsf{kvoptions} and, optionally,
% \textsf{babel}. Fetch them from CTAN \cite{CTAN} if your \TeX{}
% distributor does not provide them.
%
% \section{Usage}
% %%%%%%%%%%%%%%%
% 
% \subsection{Preparing your document source}
% %%%%%%%%%%%%%%%%%%%%%%%%%%%%%%%%%%%%%%%%%%%
% The macro package exsol can be loaded with:
% \begin{verbatim}
% \usepackage{exsol}
% \end{verbatim}
%
% Then, you are ready to add some exercises including their solution
% to your document source. To this end, embed them in a
% \texttt{exercise} and a corresponding \texttt{solution} environment.
% Optionally, you may embed several of them in a \texttt{exercises}
% environment, to make them stand out in your text.
% E.g.,
%
% \begin{VerbatimOut}{exsol.tmp}
% 
% \begin{exercises}
%
%   \begin{exercise}
%     Solve the following equation for $x \in C$, with $C$ the set of
%     complex numbers:
%     \begin{equation}
%       5 x^2 -3 x = 5
%     \end{equation}
%   \end{exercise}
%   \begin{solution}
%     Let's start by rearranging the equation, a bit:
%     \begin{eqnarray}
%       5.7 x^2 - 3.1 x &=& 5.3\\
%       5.7 x^2 - 3.1 x -5.3 &=& 0
%     \end{eqnarray}
%     The equation is now in the standard form:
%     \begin{equation}
%       a x^2 + b x + c = 0
%     \end{equation}
%     For quadratic equations in the standard form, we know that two
%     solutions exist:
%     \begin{equation}
%       x_{1,2} = \frac{ -b \pm \sqrt{d}}{2a}
%     \end{equation}
%     with
%     \begin{equation}
%       d = b^2 - 4 a c
%     \end{equation}
%     If we apply this to our case, we obtain:
%     \begin{equation}
%       d = (-3.1)^2 - 4 \cdot 5.7 \cdot (-5.3) = 130.45
%     \end{equation}
%     and
%     \begin{eqnarray}
%       x_1 &=& \frac{3.1 + \sqrt{130.45}}{11.4} = 1.27\\
%       x_2 &=& \frac{3.1 - \sqrt{130.45}}{11.4} = -0.73
%     \end{eqnarray}
%     The proposed values $x = x_1, x_2$ are solutions to the given equation.
%   \end{solution}
%   \begin{exercise}
%     Consider a 2-dimensional vector space equipped with a Euclidean
%     distance function. Given a right-angled triangle, with the sides
%     $A$ and $B$ adjacent to the right angle having lengths, $3$ and
%     $4$, calculate the length of the hypotenuse, labeled $C$.
%   \end{exercise}
%   \begin{solution}
%     This calls for application of Pythagoras' theorem, which 
%     tells us:
%     \begin{equation}
%       \left\|A\right\|^2 + \left\|B\right\|^2 = \left\|C\right\|^2
%     \end{equation}
%     and therefore:
%     \begin{eqnarray}
%       \left\|C\right\| 
%       &=& \sqrt{\left\|A\right\|^2 + \left\|B\right\|^2}\\
%       &=& \sqrt{3^2 + 4^2}\\
%       &=& \sqrt{25} = 5
%     \end{eqnarray}
%     Therefore, the length of the hypotenuse equals $5$.
%   \end{solution}
%
% \end{exercises}
% \end{VerbatimOut}
% \VerbatimInput[frame=lines,gobble=2,fontsize=\footnotesize]{exsol.tmp}
%
% The result in the original document, can be seen below. As you can
% see, there's no trace of the solution. 
%
% \input{exsol.tmp}
%
% When running \LaTeX{} on your document (in our case on the
% \texttt{exsol.dtx} file, as a side effect a file with extension
% \texttt{.sol.tex} has been written to disk (in our case, the file
% \texttt{exsol.sol.tex}), containing all solutions in sequence.
%
% Generating a solution book is a simple as including the file into a
% simple \LaTeX{} harness, that allows you giving it a proper title page and to
% add other bells and whistles.
%
% E.g.,
% \begin{VerbatimOut}{exsol-solutionbook.tex}
% \documentclass{article}
% \usepackage[english]{babel}
% \title{Solutions to the exercises, specified in the \textsf{ExSol} package}
% \author{Walter Daems}
% \date{2013/05/12}
%
% \begin{document}
%
% \maketitle
%
% \input{exsol.sol}
%
% \end{document}
% \end{VerbatimOut}
% \VerbatimInput[frame=lines,gobble=2,fontsize=\footnotesize]{exsol-solutionbook.tex}
% 
% You may generate this solution book, by running \LaTeX{} on the
% file named \texttt{exsol-solutionbook.tex} that is generated when running
% \LaTeX{} on the \texttt{exsol.dtx} file.
%
% The result approximately looks like this:
%
% \setcounter{equation}{0}
% \rule{\linewidth}{.7pt}
% \begin{center}
% {\Large Solutions to the exercises, specified in the \textsf{ExSol} package}\\
% {\large Walter Daems}\\
% {\large 2013/05/12}
% \end{center}
% \par---\newline\textbf{Solution 3.1-1}
%     Let's start by rearranging the equation, a bit:
%     \begin{eqnarray}
%       5.7 x^2 - 3.1 x &=% 5.3\\
%       5.7 x^2 - 3.1 x -5.3 &=% 0
%     \end{eqnarray}
%     The equation is now in the standard form:
%     \begin{equation}
%       a x^2 + b x + c = 0
%     \end{equation}
%     For quadratic equations in the standard form, we know that two
%     solutions exist:
%     \begin{equation}
%       x_{1,2} = \frac{ -b \pm \sqrt{d}}{2a}
%     \end{equation}
%     with
%     \begin{equation}
%       d = b^2 - 4 a c
%     \end{equation}
%     If we apply this to our case, we obtain:
%     \begin{equation}
%       d = (-3.1)^2 - 4 \cdot 5.7 \cdot (-5.3) = 130.45
%     \end{equation}
%     and
%     \begin{eqnarray}
%       x_1 &=& \frac{3.1 + \sqrt{130.45}}{11.4} = 1.27\\
%       x_2 &=& \frac{3.1 + \sqrt{130.45}}{11.4} = -0.73
%     \end{eqnarray}
%     The proposed values $x = x_1, x_2$ are solutions to
%     the given equation.
% \par---\newline\textbf{Solution 3.1-2}
%       This calls for application of Pythagoras' theorem, which
%       tells us:
%       \begin{equation}
%         \left\|A\right\|^2 + \left\|B\right\|^2 = \left\|C\right\|^2
%       \end{equation}
%       and therefore:
%       \begin{eqnarray}
%         \left\|C\right\|
%         &=& \sqrt{\left\|A\right\|^2 + \left\|B\right\|^2}\\
%         &=& \sqrt{3^2 + 4^2}\\
%         &=& \sqrt{25} = 5
%       \end{eqnarray}
%       Therefore, the length of the hypotenuse equals $5$.
%
% \rule{\linewidth}{.7pt}
%
% \subsection{Fiddling with the spacing}
%
% The default spacing provided by the \textsf{ExSol} package should be
% fine for most users. However, if you like to tweak, below you can
% find the controls.
% \subsubsection{Spacing before and after the \texttt{exercises} environment}
%
% The lengths below control the spacing of the |exercises| environment:
% \begin{itemize}
% \item |exsolexerciseaboveskip|: rubber length controlling the
% vertical space after the top marker line of the environment
% \item |exsolexercisebelowskip|: rubber length controlling the
% vertical space before the bottom marker line of the environment
% \end{itemize}
%
% You can simply specify them like:
% \begin{verbatim}
% \setlength{\exsolexercisesaboveskip}{1ex plus 1pt minus 1pt}
% \setlength{\exsolexercisesbelowskip}{1ex plus 1pt minus 1pt}
% \end{verbatim}
% The spacings specified here are the package defaults.
%
% \subsubsection{Spacing of the individual exercises}
% Caution: the spacing can only be tuned, when one invokes the
% |exerciseaslist| package option!
%
% Then lengths below control the spacing of the |exercise| environment:
% \begin{itemize}
% \item |exercisetopbottomsep|: rubber length controlling the vertical
% space before and after individual exercises
% \item |exerciseleftmargin|: length controlling the horizontal
% space between the surrounding environment's left margin (most
% often the page margin) and the left edge of the exercise
% environment 
% \item |exerciseleftmargin|: length controlling the horizontal
% space between the surrounding environment's right margin (most
% often the page margin) and the right edge of the exercise
% environment
% \item |exerciseitemindent|: length controlling the first-line
% indentation of the first paragraph in the exercise environment
% (actually, the label is set w.r.t. this position, that we will
% conveniently call position 'x')
% \item |exerciseparindent|: length controlling the first-line
% indentation of the other paragraphs in the exercise environment.
% \item |exerciselabelsep|: length controlling the distance between
% the label and position 'x'
% \item |exerciselabelwidth|: minimal width of the (internally
% right-alligned) box to use for the exercises label; if the box is
% not sufficiently big, position 'x' is shifted to the right
% \item |exerciseparsep|: internal paragraph separation (vertically)
% \end{itemize}
% 
% You can simply specify them like:
% \begin{verbatim}
% \setlength{\exsolexercisetopbottomsep}{0pt plus 0pt minus 1pt}
% \setlength{\exsolexerciseleftmargin}{1em}
% \setlength{\exsolexerciserightmargin}{1em}
% \setlength{\exsolexerciseparindent}{0em}
% \setlength{\exsolexerciselabelsep}{0.5em}
% \setlength{\exsolexerciselabelwidth}{0pt}
% \setlength{\exsolexerciseitemindent}{0pt}
% \setlength{\exsolexerciseparsep}{\parskip}
% \end{verbatim}
% The spacings specified here are the package defaults.
%
% \subsection{Tips and tricks}
%
% If you want to include the solutions all at the
% end of the current document, you need to explicitly close the
% solution stream before including it:
% \begin{verbatim}
%   \closeout\solutionstream\input{\jobname.sol.tex}
% \end{verbatim}
%
% If you want to avoid exercises being split by a page boundary, then
% provide the package option 'minipage'. This causes the exercises to
% be wrapped in a minipage environment.
% 
% \clearpage
%
% \section{Implementation}
% %%%%%%%%%%%%%%%%%%%%%%%
%    \begin{macrocode}
%<*package>
%    \end{macrocode}
%
% \subsection{Auxiliary packages}
% %%%%%%%%%%%%%%%%%%%%%%%%%%%%%%%
% The package uses some auxiliary packages:
%    \begin{macrocode}
\RequirePackage{fancyvrb}
\RequirePackage{ifthen}
\RequirePackage{kvoptions}
%    \end{macrocode}
%
% \subsection{Package options}
% %%%%%%%%%%%%%%%%%%%%%%%%%%%%
% The package offers some options:
%
% \changes{v0.2}{2012/01/06}{Added option exercisesfont}
% \changes{v0.4}{2012/01/09}{Changed name of option to exercisesfontsize}
%
% \begin{macro}{exercisesfontsize}
%  This option allows setting the font of the \texttt{exercises}
%  environment. You may chopse one of tiny, scriptsize, footnotesize,
%  small, normalsize, large, etc.\\
%  E.g., \texttt{[exercisesfontsize=small]}.
%    \begin{macrocode}
\DeclareStringOption[normalsize]{exercisesfontsize}
%    \end{macrocode}
% \end{macro}
%
% \changes{v0.4}{2012/01/06}{Added option exercisesinlist}
% \changes{v0.5}{2012/01/09}{Changed option exercisesinlist to exerciseaslist}
%
% \begin{macro}{exerciseaslist}
%  This boolean option (true, false) allows setting the typesetting of
%  the \texttt{exercises} in a list environment. This causes the
%  exercises to be typeset in a more compact fashion, with indented
%  left and right margin. 
%    \begin{macrocode}
\DeclareBoolOption[false]{exerciseaslist}
%    \end{macrocode}
% \end{macro}
%
% \changes{v0.5}{2012/01/09}{Added option copyexercisesinsolutions}
% \begin{macro}{copyexercisesinsolutions}
%  This boolean option (true, false) allows copying the exercises in
%  the solutions file, to allow for making a complete stand-alone
%  exercises bundle.
%    \begin{macrocode}
\DeclareBoolOption[false]{copyexercisesinsolutions}
%    \end{macrocode}
% \end{macro}
%
% \changes{v0.9}{2014/07/28}{. Changed default behavior
% w.r.t. minipage-wraping of exercises}
% \begin{macro}{minipage}
%  This boolean option (true, false) causes the exercises to be
%  wrapped in minipages. This avoids them getting split by a page
%  boundary.
%    \begin{macrocode}
\DeclareBoolOption[false]{minipage}
%    \end{macrocode}
% \end{macro}
%
% The options are processed using:
%    \begin{macrocode}
\ProcessKeyvalOptions*
%    \end{macrocode}
%
% The options are subsequently handled
%    \begin{macrocode}
\newcommand{\exercisesfontsize}{\csname \exsol@exercisesfontsize\endcsname}
%    \end{macrocode}
%
%
% \subsection{Customization of lengths}
% %%%%%%%%%%%%%%%%%%%%%%%%%%%%%%%%%%%%%%%
% The commands below allow customizing many lengths that control the
% typesetting of the exercises.
%
% \changes{v0.91}{2014/08/31}{added user-accessible lengths}
% First some lengths to control the spacing before and after |exercises|.
%    \begin{macrocode}
\newlength{\exsolexercisesaboveskip}
\setlength{\exsolexercisesaboveskip}{1ex plus 1pt minus 1pt}
\newlength{\exsolexercisesbelowskip}
\setlength{\exsolexercisesbelowskip}{1ex plus 1pt minus 1pt}
%    \end{macrocode}
%
% Then some lengths to control the spacing for a single
% exercise. These lengths only work when the |exerciseaslist| package
% option has been specified. Sensible defaults have been set.
%    \begin{macrocode}
\newlength{\exsolexercisetopbottomsep}
\setlength{\exsolexercisetopbottomsep}{0pt plus 0pt minus 1pt}
\newlength{\exsolexerciseleftmargin}
\setlength{\exsolexerciseleftmargin}{1em}
\newlength{\exsolexerciserightmargin}
\setlength{\exsolexerciserightmargin}{1em}
\newlength{\exsolexerciseparindent}
\setlength{\exsolexerciseparindent}{0em}
\newlength{\exsolexerciselabelsep}
\setlength{\exsolexerciselabelsep}{0.5em}
\newlength{\exsolexerciselabelwidth}
\setlength{\exsolexerciselabelwidth}{0pt}
\newlength{\exsolexerciseitemindent}
\setlength{\exsolexerciseitemindent}{0pt}
\newlength{\exsolexerciseparsep}
\setlength{\exsolexerciseparsep}{\parskip}
%    \end{macrocode}
% 
% 
% \subsection{Con- and destruction of the auxiliary streams}
% %%%%%%%%%%%%%%%%%%%%%%%%%%%%%%%%%%%%%%%%%%%%%%%%%%%%%%%%%%
% At the beginning of your document, we start by opening a stream to a
% file that will be used to write the solutions to. At the end of your
% document, the package closes the stream.
% \changes{v0.8}{2014/07/15}{moved newwrite of exercise stream to this
% spot to avoid consuming all handles}
%    \begin{macrocode}
\AtBeginDocument{
  \newwrite\solutionstream
  \immediate\openout\solutionstream=\jobname.sol.tex
  \newwrite\exercisestream
}
\AtEndDocument{
  \immediate\closeout\solutionstream
}
%    \end{macrocode}
%
% \subsection{Exercises counter}
% %%%%%%%%%%%%%%%%%%%%%%%%%%%%%%
% By providing an exercise counter, proper numbering of the exercises
% is provided to allow for good cross referencing of the solutions to
% the exercises.
% \changes{v0.2}{2012/01/06}{Removed dash in counter when in document
% without sectioning commands}
%    \begin{macrocode}
\newcounter{exercise}[subsection]
\setcounter{exercise}{0}
\renewcommand{\theexercise}{%
  \@ifundefined{c@chapter}{}{\if0\arabic{chapter}\else\arabic{chapter}.\fi}%
  \if0\arabic{section}\else\arabic{section}\fi%
  \if0\arabic{subsection}\else.\arabic{subsection}\fi%
  \if0\arabic{subsubsection}\else.\arabic{subsubsection}\fi%
  \if0\arabic{exercise}\else%
    \@ifundefined{c@chapter}%
                 {\if0\arabic{section}\else-\fi}%
                 {-}%
    \arabic{exercise}%
  \fi
}
%    \end{macrocode}
%
%
%
% \subsection{Detokenization in order to cope with utf8}
%
% Combining old-school \LaTeX{} (before \XeTeX{} and \LuaTeX{}) and
% UTF-8 is a pain.
% Detokenization has been suggested by Geoffrey Poore to solve issues
% with UTF-8 characters messing up the |fancyvrb| internals.
% \changes{v0.7}{2014/07/14}{Added detokenized writing}
%    \begin{macrocode}
\newcommand{\GPES@write@detok}[1]{%
  \immediate\write\exercisestream{\detokenize{#1}}}%
\newcommand{\GPSS@write@detok}[1]{%
  \immediate\write\solutionstream{\detokenize{#1}}}%
\newcommand{\GPESS@write@detok}[1]{%
  \GPES@write@detok{#1}%
  \GPSS@write@detok{#1}}%
%    \end{macrocode}
%
%
% \section{The user environments}
%
% \begin{macro}{exercise}
%   The \texttt{exercise} environment is used to typeset your
%   exercises, provide them with a nice label and allow for copying
%   the exercise to the solutions file (if the package option
%   \texttt{copyexercisesinsolution}) is set. The label can be
%   set by redefining the \cs{exercisename} macro, or by relying on
%   the \textsf{Babel} provisions. The code is almost litteraly
%   taken from the \textsf{fancyvrb} package.
%    \begin{macrocode}
\def\exercise{\FV@Environment{}{exercise}}
\def\FVB@exercise{%
  \refstepcounter{exercise}%
  \immediate\openout\exercisestream=\jobname.exc.tex
  \ifexsol@copyexercisesinsolutions
    \typeout{Writing exercise to \jobname.sol.tex}
    \immediate\write\solutionstream{\string\par---\string\newline
      \string\textbf\string{\exercisename{} \theexercise \string}}
  \else
    \immediate\write\solutionstream{\string\par---\string\newline}
  \fi
  \immediate\write\exercisestream{\string\begin{exsol@exercise}}
  \@bsphack
  \begingroup
    \FV@UseKeyValues
    \FV@DefineWhiteSpace
    \def\FV@Space{\space}%
    \FV@DefineTabOut
    \ifexsol@copyexercisesinsolutions
      \let\FV@ProcessLine\GPESS@write@detok %
    \else
      \let\FV@ProcessLine\GPES@write@detok %
    \fi
    \relax
    \let\FV@FontScanPrep\relax
    \let\@noligs\relax
    \FV@Scan
  }
\def\FVE@exercise{
  \endgroup\@esphack
  \immediate\write\exercisestream{\string\end{exsol@exercise}}
  \ifexsol@copyexercisesinsolutions
    \immediate\write\solutionstream{\string~\string\newline}
  \fi
  \immediate\closeout\exercisestream
  \input{\jobname.exc.tex}
}
\DefineVerbatimEnvironment{exercise}{exercise}{}
%    \end{macrocode}
% \end{macro}
%
% \begin{macro}{exsol@exercise}
%   The \texttt{exsol@exercise} environment is an internal macro used
%   to typeset your exercises and provide them with a nice label and
%   number. Do not use it directly. Use the proper environment
%   \texttt{exercise} instead.
%   \changes{v0.2}{2012/01/06}{Attempted to fix MiKTeX formatting problems}
%   \changes{v0.3}{2012/01/08}{Fixed labelsep to avoid cluttered
%   itemize environments}
%   \changes{v0.4}{2012/01/06}{Added option exercisesinlist such that
%   default results in non list formatting of exercise}
%   \changes{v0.5}{2012/01/09}{Changed implementation to allow for
%   copying the exercises to the solutions file.}
%    \begin{macrocode}
\newenvironment{exsol@exercise}[0]
{%
  \ifthenelse{\boolean{exsol@minipage}}{\begin{minipage}[t]{\textwidth}}{}%
    \ifthenelse{\boolean{exsol@exerciseaslist}}
               {\begin{list}%
                   {%
                   }%
                   {%
                     \setlength{\topsep}{\exsolexercisetopbottomsep}%
                     \setlength{\leftmargin}{\exsolexerciseleftmargin}%
                     \setlength{\rightmargin}{\exsolexerciserightmargin}%
                     \setlength{\listparindent}{\exsolexerciseparindent}%
                     \setlength{\itemindent}{\exsolexerciseitemindent}%
                     \setlength{\parsep}{\exsolexerciseparsep}
                     \setlength{\labelsep}{\exsolexerciselabelsep}
                     \setlength{\labelwidth}{\exsolexerciselabelwidth}}
                 \item[\textit{~\exercisename{} \theexercise:~}]
               }%
               {\textit{\exercisename{} \theexercise:}}
}
{%
  \ifthenelse{\boolean{exsol@exerciseaslist}}%
             {\end{list}}{}%
  \ifthenelse{\boolean{exsol@minipage}}{\end{minipage}}{\par}%
}
%    \end{macrocode}
% \end{macro}
%
%
% \begin{macro}{solution}
%   The \texttt{solution} environment is used to typeset your solutions
%   and provide them with a nice label and number that corresponds to
%   the exercise that preceeded this solution. Theno label can be
%   set by redefining the \cs{solutionname} macro, or by relying on
%   the \textsf{Babel} provisions. The code is almost litteraly
%   taken from the \textsf{fancyvrb} package.
%    \begin{macrocode}
\def\solution{\FV@Environment{}{solution}}
\def\FVB@solution{%
  \typeout{Writing solution to \jobname.sol.tex}
  \immediate\write\solutionstream{\string\textbf\string{\solutionname{}\string}}
  \ifexsol@copyexercisesinsolutions
    \immediate\write\solutionstream{\string\newline}
  \else
    \immediate\write\solutionstream{\string\textbf\string{\theexercise\string}%
                                    \string\newline}
  \fi
  \@bsphack
  \begingroup
    \FV@UseKeyValues
    \FV@DefineWhiteSpace
    \def\FV@Space{\space}%
    \FV@DefineTabOut
    \let\FV@ProcessLine\GPSS@write@detok %
    \relax
    \let\FV@FontScanPrep\relax
    \let\@noligs\relax
    \FV@Scan
  }
\def\FVE@solution{\endgroup\@esphack}
\DefineVerbatimEnvironment{solution}{solution}{}
%    \end{macrocode}
% \end{macro}
%
% \begin{macro}{exercises}
%   The \texttt{exercises} environment helps typesetting your exercises to
%   stand out from the rest of the text. You may use it at the end of
%   a chapter, or just to group some exercises in the text.
%   \changes{v0.2}{2012/01/06}{Attempted to fix MiKTeX formatting problems}
%   \changes{v0.3}{2012/01/07}{Added some extra whitespace below exercisesname}
%    \begin{macrocode}
\newenvironment{exercises}
{\par\exercisesfontsize\rule{.25\linewidth}{0.15mm}\vspace*{\exsolexercisesaboveskip}\\*%
 \textbf{\normalsize \exercisesname}}
{\vspace*{-\baselineskip}\vspace*{\exsolexercisesbelowskip}\rule{.25\linewidth}{0.15mm}\par}
%    \end{macrocode}
% \end{macro}
%
% \subsection{Some Babel provisions}
% %%%%%%%%%%%%%%%%%%%%%%%%%%%%%%%%%%
% \changes{v0.2}{2012/01/06}{Fixed babel errors}
% \begin{macro}{\exercisename}
%   The exercise environment makes use of a label \texttt{\exercisename{}}
%   macro.
%    \begin{macrocode}
\newcommand{\exercisename}{Exercise}
%    \end{macrocode}
% \end{macro}
%
% \begin{macro}{\exercisesname}
%   The exercises environment makes use of a label \texttt{\exercisesname{}}
%   macro.
%    \begin{macrocode}
\newcommand{\exercisesname}{Exercises}
%    \end{macrocode}
% \end{macro}
% 
% \begin{macro}{\solutionname}
%   The solution environment makes use of a label \texttt{\solutionname{}}
%   macro.
%    \begin{macrocode}
\newcommand{\solutionname}{Solution}
%    \end{macrocode}
% \end{macro}
%
% \begin{macro}{\solutionname}
%   The solution environment makes use of a label \texttt{\solutionname{}}
%   macro.
% \changes{v0.8}{2014/07/15}{Added missing babel tag}
%    \begin{macrocode}
\newcommand{\solutionsname}{Solutions}
%    \end{macrocode}
% \end{macro}
% 
% 
% You may redefine these macros, but to help you out a little bit, we
% provide with some basic Babel auxiliaries. If you're a true polyglot
% and are willing to help me out by providing translations for other
% languages, I'm very willing to incorporate them into the code.
%
% \changes{v0.7}{2014/07/14}{Added Finnish language support}
%    \begin{macrocode}
\addto\captionsdutch{%
  \renewcommand{\exercisename}{Oefening}%
  \renewcommand{\exercisesname}{Oefeningen}%
  \renewcommand{\solutionname}{Oplossing}%
  \renewcommand{\solutionsname}{Oplossingen}%
}
\addto\captionsgerman{%
  \renewcommand{\exercisename}{Aufgabe}%
  \renewcommand{\exercisesname}{Aufgaben}%
  \renewcommand{\solutionname}{L\"osung}%
  \renewcommand{\solutionsname}{L\"osungen}%
}
\addto\captionsfrench{%
  \renewcommand{\exercisename}{Exercice}%
  \renewcommand{\exercisesname}{Exercices}%
  \renewcommand{\solutionname}{Solution}%
  \renewcommand{\solutionsname}{Solutions}%
}
\addto\captionsfinnish{
  \renewcommand{\exercisename}{Teht\"av\"a}%
  \renewcommand{\exercisesname}{Teht\"avi\"a}%
  \renewcommand{\solutionname}{Ratkaisu}%
  \renewcommand{\solutionsname}{Ratkaisut}%
}
%    \end{macrocode}
%
%
%
% Now the final hack overloads the basic sectioning commands to make
% sure that they are copied into your solution book.
%
%    \begin{macrocode}
\let\exsol@@makechapterhead\@makechapterhead
\def\@makechapterhead#1{%
  \immediate\write\solutionstream{\string\chapter{#1}}%
  \exsol@@makechapterhead{#1}
}
\ifdefined\frontmatter
  \let\exsol@@frontmatter\frontmatter
  \def\frontmatter{%
    \immediate\write\solutionstream{\string\frontmatter}%
    \exsol@@frontmatter
  }
\fi
\ifdefined\frontmatter
  \let\exsol@@mainmatter\mainmatter
  \def\mainmatter{%
    \immediate\write\solutionstream{\string\mainmatter}%
    \exsol@@mainmatter
  }
\fi
\ifdefined\backmatter
  \let\exsol@@backmatter\backmatter
  \def\backmatter{%
    \immediate\write\solutionstream{\string\backmatter}%
    \exsol@@backmatter
  }
\fi
%    \end{macrocode}
%
% \begin{macro}{\noexercisesinchapter}
%   If you have chapters without exercises, you may want to indicate
%   this clearly into your source. Otherwise empty chapters may appear
%   in your solution book.
%    \begin{macrocode}
\newcommand{\noexercisesinchapter}
{
  \immediate\write\solutionstream{No exercises in this chapter}
}
%    \end{macrocode}
% \end{macro}
%
%    \begin{macrocode}
%</package>
%    \end{macrocode}
%
% \bibliographystyle{alpha}
%
% \begin{thebibliography}{99}
%
% \bibitem{fancyvrb}
% Timothy Van Zandt, Herbert Vo\ss, Denis Girou, Sebastian Rahtz, Niall
% Mansfield 
% \newblock The \texttt{fancyvrb} package.
% \newblock \url{http://ctan.org/pkg/fancyvrb}.
% \newblock online, accessed in January 2012.
%
% \bibitem{CTAN} 
% The Comprehensive TeX Archive Network.
% \newblock \url{http://www.ctan.org}.
% \newblock online, accessed in January 2012.
%
% \end{thebibliography}
%
% \Finale
\endinput

%
% When running \LaTeX{} on your document (in our case on the
% \texttt{exsol.dtx} file, as a side effect a file with extension
% \texttt{.sol.tex} has been written to disk (in our case, the file
% \texttt{exsol.sol.tex}), containing all solutions in sequence.
%
% Generating a solution book is a simple as including the file into a
% simple \LaTeX{} harness, that allows you giving it a proper title page and to
% add other bells and whistles.
%
% E.g.,
% \begin{VerbatimOut}{exsol-solutionbook.tex}
% \documentclass{article}
% \usepackage[english]{babel}
% \title{Solutions to the exercises, specified in the \textsf{ExSol} package}
% \author{Walter Daems}
% \date{2013/05/12}
%
% \begin{document}
%
% \maketitle
%
% % \iffalse meta-comment
%
% Copyright (C) 2014 by Walter Daems <walter.daems@uantwerpen.be>
%
% This work may be distributed and/or modified under the conditions of
% the LaTeX Project Public License, either version 1.3 of this license
% or (at your option) any later version.  The latest version of this
% license is in:
% 
%    http://www.latex-project.org/lppl.txt
% 
% and version 1.3 or later is part of all distributions of LaTeX version
% 2005/12/01 or later.
%
% This work has the LPPL maintenance status `maintained'.
% 
% The Current Maintainer of this work is Walter Daems.
%
% This work consists of the files exsol.dtx and exsol.ins and the derived 
% files:
%   - exsol.sty
%   - example.tex
%   - example-solutionbook.tex
%
% \fi
%
% \iffalse
%
%<package|driver>\NeedsTeXFormat{LaTeX2e}
%<driver>\ProvidesFile{exsol.dtx}
%<package>\ProvidesPackage{exsol}
%<package|driver>  [2014/08/31 v0.91 ExSol - Exercises and Solutions package (DMW)]
%<*driver> 
\documentclass[11pt]{ltxdoc}
\usepackage[english]{babel}
\usepackage[exercisesfontsize=small]{exsol}
\usepackage{metalogo}
\EnableCrossrefs
\CodelineIndex
\RecordChanges
\usepackage{makeidx}
\usepackage{alltt}
\IfFileExists{tocbibind.sty}{\usepackage{tocbibind}}{}
\IfFileExists{hyperref.sty}{\usepackage[bookmarksopen]{hyperref}}{}
\EnableCrossrefs         
\CodelineIndex
\RecordChanges
\newcommand{\exsol}{\textsf{ExSol}}
\StopEventually{\PrintChanges\PrintIndex}
\def\fileversion{0.91}
\def\filedate{2014/08/31}
\begin{document}
 \DocInput{exsol.dtx}
\end{document}
%</driver>
% \fi
%
% \CheckSum{0}
%
% \CharacterTable
%  {Upper-case    \A\B\C\D\E\F\G\H\I\J\K\L\M\N\O\P\Q\R\S\T\U\V\W\X\Y\Z
%   Lower-case    \a\b\c\d\e\f\g\h\i\j\k\l\m\n\o\p\q\r\s\t\u\v\w\x\y\z
%   Digits        \0\1\2\3\4\5\6\7\8\9
%   Exclamation   \!     Double quote  \"     Hash (number) \#
%   Dollar        \$     Percent       \%     Ampersand     \&
%   Acute accent  \'     Left paren    \(     Right paren   \)
%   Asterisk      \*     Plus          \+     Comma         \,
%   Minus         \-     Point         \.     Solidus       \/
%   Colon         \:     Semicolon     \;     Less than     \<
%   Equals        \=     Greater than  \>     Question mark \?
%   Commercial at \@     Left bracket  \[     Backslash     \\
%   Right bracket \]     Circumflex    \^     Underscore    \_
%   Grave accent  \`     Left brace    \{     Vertical bar  \|
%   Right brace   \}     Tilde         \~}
%
%
% \changes{v0.1}{2012/01/05}{. Initial version}
% \changes{v0.2}{2012/01/06}{. Minor bug fixes based on first use by
% Paul Levrie}
% \changes{v0.3}{2012/01/07}{. Minor bug fixes based on second use by
% Paul}
% \changes{v0.4}{2012/01/09}{. Allowed for non-list formatting of
% exercises (as default)}
% \changes{v0.5}{2012/01/15}{. Added option to also send exercises to
% solutions file}
% \changes{v0.6}{2013/05/12}{. Prepared for CTAN publication}
% \changes{v0.7}{2014/07/14}{. Fixed UTF8 compatibility issues}
% \changes{v0.8}{2014/07/15}{. Fixed missing babel tag and running out
% of write hanles}
% \changes{v0.9}{2014/07/28}{. Changed default behavior
% w.r.t. minipage-wraping of exercises} 
% \changes{v0.91}{2014/08/31}{. Corrected minipage dependence, made }
%
% \DoNotIndex{\newcommand,\newenvironment}
% \setlength{\parindent}{0em}
% \addtolength{\parskip}{0.5\baselineskip}
%
% \title{The \exsol{} package\thanks{This document
%   corresponds to exsol~\fileversion, dated \filedate.}}
% \author{Walter Daems (\texttt{walter.daems@uantwerpen.be})}
% \date{}
%
% \maketitle
%
% \section{Introduction}
% %%%%%%%%%%%%%%%%%%%%%%
% The package \exsol{} provides macros to allow
% embedding exercises and solutions in the \LaTeX{} source of an
% instructional text (e.g., a book or a course text) while generating
% the following separate documents:
% \begin{itemize}
% \item your original text that only contains the exercises, and
% \item a solution book that only contains the solutions to the
% exercises (a package option exists to also copy the exercises themselves to the solution book).
% \end{itemize}
% 
% The former is generated when running \LaTeX{} on your document. This
% run writes the solutions to a secondary file that can be included
% into a simple document harness, such that when running \LaTeX{} on
% the latter, you can generate a nice solution book.
% 
% Why use \exsol{}?
% \begin{itemize}
% \item It allows to keep the \LaTeX{} source of your exercises and their
% solutions in a single file. Away with the nightmare to keep your
% solutions in sync with the original text.
% \item It separates exercises and solutions, allowing you
%   \begin{itemize}
%   \item to only release the solution book to the instructors of the
%   course;
%   \item to encourage students that you provide with the solution
%   book to first try solving the exercises without opening the book;
%   this seems to be easier than not peeking into the solution of an
%   exercise that is typeset just below the exercise itself.
%   \end{itemize}
% \end{itemize}
%
% The code of the \exsol{} package was taken almost literally
% from \textsf{fancyvrb} \cite{fancyvrb}. Therefore, all credits go to the
% authors/maintainers of \textsf{fancyvrb}.
%
% Thanks to Pieter Pareit and Pekka Pere for signaling problems and
% making suggestions for the documentation.
%
% \section{Installation}
% %%%%%%%%%%%%%%%%%%%%%%
% Either you are a package manager and then you'll know how to
% prepare an installation package for \exsol{}.
%
% Either you are a normal user and then you have two options. First,
% check if there is a package that your favorite \LaTeX{}
% distributor has prepared for you. Second, grab the TDS package
% from CTAN \cite{CTAN} (\texttt{exsol.tds.zip}) and unzip it somewhere in your
% own TDS tree, regenerate your filename database and off you go.
% In any case, make sure that \LaTeX{} finds the \texttt{exsol.sty} file.
%
% The \exsol{} package uses some auxiliary packages: \textsf{fancyvrb},
% \textsf{ifthen}, \textsf{kvoptions} and, optionally,
% \textsf{babel}. Fetch them from CTAN \cite{CTAN} if your \TeX{}
% distributor does not provide them.
%
% \section{Usage}
% %%%%%%%%%%%%%%%
% 
% \subsection{Preparing your document source}
% %%%%%%%%%%%%%%%%%%%%%%%%%%%%%%%%%%%%%%%%%%%
% The macro package exsol can be loaded with:
% \begin{verbatim}
% \usepackage{exsol}
% \end{verbatim}
%
% Then, you are ready to add some exercises including their solution
% to your document source. To this end, embed them in a
% \texttt{exercise} and a corresponding \texttt{solution} environment.
% Optionally, you may embed several of them in a \texttt{exercises}
% environment, to make them stand out in your text.
% E.g.,
%
% \begin{VerbatimOut}{exsol.tmp}
% 
% \begin{exercises}
%
%   \begin{exercise}
%     Solve the following equation for $x \in C$, with $C$ the set of
%     complex numbers:
%     \begin{equation}
%       5 x^2 -3 x = 5
%     \end{equation}
%   \end{exercise}
%   \begin{solution}
%     Let's start by rearranging the equation, a bit:
%     \begin{eqnarray}
%       5.7 x^2 - 3.1 x &=& 5.3\\
%       5.7 x^2 - 3.1 x -5.3 &=& 0
%     \end{eqnarray}
%     The equation is now in the standard form:
%     \begin{equation}
%       a x^2 + b x + c = 0
%     \end{equation}
%     For quadratic equations in the standard form, we know that two
%     solutions exist:
%     \begin{equation}
%       x_{1,2} = \frac{ -b \pm \sqrt{d}}{2a}
%     \end{equation}
%     with
%     \begin{equation}
%       d = b^2 - 4 a c
%     \end{equation}
%     If we apply this to our case, we obtain:
%     \begin{equation}
%       d = (-3.1)^2 - 4 \cdot 5.7 \cdot (-5.3) = 130.45
%     \end{equation}
%     and
%     \begin{eqnarray}
%       x_1 &=& \frac{3.1 + \sqrt{130.45}}{11.4} = 1.27\\
%       x_2 &=& \frac{3.1 - \sqrt{130.45}}{11.4} = -0.73
%     \end{eqnarray}
%     The proposed values $x = x_1, x_2$ are solutions to the given equation.
%   \end{solution}
%   \begin{exercise}
%     Consider a 2-dimensional vector space equipped with a Euclidean
%     distance function. Given a right-angled triangle, with the sides
%     $A$ and $B$ adjacent to the right angle having lengths, $3$ and
%     $4$, calculate the length of the hypotenuse, labeled $C$.
%   \end{exercise}
%   \begin{solution}
%     This calls for application of Pythagoras' theorem, which 
%     tells us:
%     \begin{equation}
%       \left\|A\right\|^2 + \left\|B\right\|^2 = \left\|C\right\|^2
%     \end{equation}
%     and therefore:
%     \begin{eqnarray}
%       \left\|C\right\| 
%       &=& \sqrt{\left\|A\right\|^2 + \left\|B\right\|^2}\\
%       &=& \sqrt{3^2 + 4^2}\\
%       &=& \sqrt{25} = 5
%     \end{eqnarray}
%     Therefore, the length of the hypotenuse equals $5$.
%   \end{solution}
%
% \end{exercises}
% \end{VerbatimOut}
% \VerbatimInput[frame=lines,gobble=2,fontsize=\footnotesize]{exsol.tmp}
%
% The result in the original document, can be seen below. As you can
% see, there's no trace of the solution. 
%
% \input{exsol.tmp}
%
% When running \LaTeX{} on your document (in our case on the
% \texttt{exsol.dtx} file, as a side effect a file with extension
% \texttt{.sol.tex} has been written to disk (in our case, the file
% \texttt{exsol.sol.tex}), containing all solutions in sequence.
%
% Generating a solution book is a simple as including the file into a
% simple \LaTeX{} harness, that allows you giving it a proper title page and to
% add other bells and whistles.
%
% E.g.,
% \begin{VerbatimOut}{exsol-solutionbook.tex}
% \documentclass{article}
% \usepackage[english]{babel}
% \title{Solutions to the exercises, specified in the \textsf{ExSol} package}
% \author{Walter Daems}
% \date{2013/05/12}
%
% \begin{document}
%
% \maketitle
%
% \input{exsol.sol}
%
% \end{document}
% \end{VerbatimOut}
% \VerbatimInput[frame=lines,gobble=2,fontsize=\footnotesize]{exsol-solutionbook.tex}
% 
% You may generate this solution book, by running \LaTeX{} on the
% file named \texttt{exsol-solutionbook.tex} that is generated when running
% \LaTeX{} on the \texttt{exsol.dtx} file.
%
% The result approximately looks like this:
%
% \setcounter{equation}{0}
% \rule{\linewidth}{.7pt}
% \begin{center}
% {\Large Solutions to the exercises, specified in the \textsf{ExSol} package}\\
% {\large Walter Daems}\\
% {\large 2013/05/12}
% \end{center}
% \par---\newline\textbf{Solution 3.1-1}
%     Let's start by rearranging the equation, a bit:
%     \begin{eqnarray}
%       5.7 x^2 - 3.1 x &=% 5.3\\
%       5.7 x^2 - 3.1 x -5.3 &=% 0
%     \end{eqnarray}
%     The equation is now in the standard form:
%     \begin{equation}
%       a x^2 + b x + c = 0
%     \end{equation}
%     For quadratic equations in the standard form, we know that two
%     solutions exist:
%     \begin{equation}
%       x_{1,2} = \frac{ -b \pm \sqrt{d}}{2a}
%     \end{equation}
%     with
%     \begin{equation}
%       d = b^2 - 4 a c
%     \end{equation}
%     If we apply this to our case, we obtain:
%     \begin{equation}
%       d = (-3.1)^2 - 4 \cdot 5.7 \cdot (-5.3) = 130.45
%     \end{equation}
%     and
%     \begin{eqnarray}
%       x_1 &=& \frac{3.1 + \sqrt{130.45}}{11.4} = 1.27\\
%       x_2 &=& \frac{3.1 + \sqrt{130.45}}{11.4} = -0.73
%     \end{eqnarray}
%     The proposed values $x = x_1, x_2$ are solutions to
%     the given equation.
% \par---\newline\textbf{Solution 3.1-2}
%       This calls for application of Pythagoras' theorem, which
%       tells us:
%       \begin{equation}
%         \left\|A\right\|^2 + \left\|B\right\|^2 = \left\|C\right\|^2
%       \end{equation}
%       and therefore:
%       \begin{eqnarray}
%         \left\|C\right\|
%         &=& \sqrt{\left\|A\right\|^2 + \left\|B\right\|^2}\\
%         &=& \sqrt{3^2 + 4^2}\\
%         &=& \sqrt{25} = 5
%       \end{eqnarray}
%       Therefore, the length of the hypotenuse equals $5$.
%
% \rule{\linewidth}{.7pt}
%
% \subsection{Fiddling with the spacing}
%
% The default spacing provided by the \textsf{ExSol} package should be
% fine for most users. However, if you like to tweak, below you can
% find the controls.
% \subsubsection{Spacing before and after the \texttt{exercises} environment}
%
% The lengths below control the spacing of the |exercises| environment:
% \begin{itemize}
% \item |exsolexerciseaboveskip|: rubber length controlling the
% vertical space after the top marker line of the environment
% \item |exsolexercisebelowskip|: rubber length controlling the
% vertical space before the bottom marker line of the environment
% \end{itemize}
%
% You can simply specify them like:
% \begin{verbatim}
% \setlength{\exsolexercisesaboveskip}{1ex plus 1pt minus 1pt}
% \setlength{\exsolexercisesbelowskip}{1ex plus 1pt minus 1pt}
% \end{verbatim}
% The spacings specified here are the package defaults.
%
% \subsubsection{Spacing of the individual exercises}
% Caution: the spacing can only be tuned, when one invokes the
% |exerciseaslist| package option!
%
% Then lengths below control the spacing of the |exercise| environment:
% \begin{itemize}
% \item |exercisetopbottomsep|: rubber length controlling the vertical
% space before and after individual exercises
% \item |exerciseleftmargin|: length controlling the horizontal
% space between the surrounding environment's left margin (most
% often the page margin) and the left edge of the exercise
% environment 
% \item |exerciseleftmargin|: length controlling the horizontal
% space between the surrounding environment's right margin (most
% often the page margin) and the right edge of the exercise
% environment
% \item |exerciseitemindent|: length controlling the first-line
% indentation of the first paragraph in the exercise environment
% (actually, the label is set w.r.t. this position, that we will
% conveniently call position 'x')
% \item |exerciseparindent|: length controlling the first-line
% indentation of the other paragraphs in the exercise environment.
% \item |exerciselabelsep|: length controlling the distance between
% the label and position 'x'
% \item |exerciselabelwidth|: minimal width of the (internally
% right-alligned) box to use for the exercises label; if the box is
% not sufficiently big, position 'x' is shifted to the right
% \item |exerciseparsep|: internal paragraph separation (vertically)
% \end{itemize}
% 
% You can simply specify them like:
% \begin{verbatim}
% \setlength{\exsolexercisetopbottomsep}{0pt plus 0pt minus 1pt}
% \setlength{\exsolexerciseleftmargin}{1em}
% \setlength{\exsolexerciserightmargin}{1em}
% \setlength{\exsolexerciseparindent}{0em}
% \setlength{\exsolexerciselabelsep}{0.5em}
% \setlength{\exsolexerciselabelwidth}{0pt}
% \setlength{\exsolexerciseitemindent}{0pt}
% \setlength{\exsolexerciseparsep}{\parskip}
% \end{verbatim}
% The spacings specified here are the package defaults.
%
% \subsection{Tips and tricks}
%
% If you want to include the solutions all at the
% end of the current document, you need to explicitly close the
% solution stream before including it:
% \begin{verbatim}
%   \closeout\solutionstream\input{\jobname.sol.tex}
% \end{verbatim}
%
% If you want to avoid exercises being split by a page boundary, then
% provide the package option 'minipage'. This causes the exercises to
% be wrapped in a minipage environment.
% 
% \clearpage
%
% \section{Implementation}
% %%%%%%%%%%%%%%%%%%%%%%%
%    \begin{macrocode}
%<*package>
%    \end{macrocode}
%
% \subsection{Auxiliary packages}
% %%%%%%%%%%%%%%%%%%%%%%%%%%%%%%%
% The package uses some auxiliary packages:
%    \begin{macrocode}
\RequirePackage{fancyvrb}
\RequirePackage{ifthen}
\RequirePackage{kvoptions}
%    \end{macrocode}
%
% \subsection{Package options}
% %%%%%%%%%%%%%%%%%%%%%%%%%%%%
% The package offers some options:
%
% \changes{v0.2}{2012/01/06}{Added option exercisesfont}
% \changes{v0.4}{2012/01/09}{Changed name of option to exercisesfontsize}
%
% \begin{macro}{exercisesfontsize}
%  This option allows setting the font of the \texttt{exercises}
%  environment. You may chopse one of tiny, scriptsize, footnotesize,
%  small, normalsize, large, etc.\\
%  E.g., \texttt{[exercisesfontsize=small]}.
%    \begin{macrocode}
\DeclareStringOption[normalsize]{exercisesfontsize}
%    \end{macrocode}
% \end{macro}
%
% \changes{v0.4}{2012/01/06}{Added option exercisesinlist}
% \changes{v0.5}{2012/01/09}{Changed option exercisesinlist to exerciseaslist}
%
% \begin{macro}{exerciseaslist}
%  This boolean option (true, false) allows setting the typesetting of
%  the \texttt{exercises} in a list environment. This causes the
%  exercises to be typeset in a more compact fashion, with indented
%  left and right margin. 
%    \begin{macrocode}
\DeclareBoolOption[false]{exerciseaslist}
%    \end{macrocode}
% \end{macro}
%
% \changes{v0.5}{2012/01/09}{Added option copyexercisesinsolutions}
% \begin{macro}{copyexercisesinsolutions}
%  This boolean option (true, false) allows copying the exercises in
%  the solutions file, to allow for making a complete stand-alone
%  exercises bundle.
%    \begin{macrocode}
\DeclareBoolOption[false]{copyexercisesinsolutions}
%    \end{macrocode}
% \end{macro}
%
% \changes{v0.9}{2014/07/28}{. Changed default behavior
% w.r.t. minipage-wraping of exercises}
% \begin{macro}{minipage}
%  This boolean option (true, false) causes the exercises to be
%  wrapped in minipages. This avoids them getting split by a page
%  boundary.
%    \begin{macrocode}
\DeclareBoolOption[false]{minipage}
%    \end{macrocode}
% \end{macro}
%
% The options are processed using:
%    \begin{macrocode}
\ProcessKeyvalOptions*
%    \end{macrocode}
%
% The options are subsequently handled
%    \begin{macrocode}
\newcommand{\exercisesfontsize}{\csname \exsol@exercisesfontsize\endcsname}
%    \end{macrocode}
%
%
% \subsection{Customization of lengths}
% %%%%%%%%%%%%%%%%%%%%%%%%%%%%%%%%%%%%%%%
% The commands below allow customizing many lengths that control the
% typesetting of the exercises.
%
% \changes{v0.91}{2014/08/31}{added user-accessible lengths}
% First some lengths to control the spacing before and after |exercises|.
%    \begin{macrocode}
\newlength{\exsolexercisesaboveskip}
\setlength{\exsolexercisesaboveskip}{1ex plus 1pt minus 1pt}
\newlength{\exsolexercisesbelowskip}
\setlength{\exsolexercisesbelowskip}{1ex plus 1pt minus 1pt}
%    \end{macrocode}
%
% Then some lengths to control the spacing for a single
% exercise. These lengths only work when the |exerciseaslist| package
% option has been specified. Sensible defaults have been set.
%    \begin{macrocode}
\newlength{\exsolexercisetopbottomsep}
\setlength{\exsolexercisetopbottomsep}{0pt plus 0pt minus 1pt}
\newlength{\exsolexerciseleftmargin}
\setlength{\exsolexerciseleftmargin}{1em}
\newlength{\exsolexerciserightmargin}
\setlength{\exsolexerciserightmargin}{1em}
\newlength{\exsolexerciseparindent}
\setlength{\exsolexerciseparindent}{0em}
\newlength{\exsolexerciselabelsep}
\setlength{\exsolexerciselabelsep}{0.5em}
\newlength{\exsolexerciselabelwidth}
\setlength{\exsolexerciselabelwidth}{0pt}
\newlength{\exsolexerciseitemindent}
\setlength{\exsolexerciseitemindent}{0pt}
\newlength{\exsolexerciseparsep}
\setlength{\exsolexerciseparsep}{\parskip}
%    \end{macrocode}
% 
% 
% \subsection{Con- and destruction of the auxiliary streams}
% %%%%%%%%%%%%%%%%%%%%%%%%%%%%%%%%%%%%%%%%%%%%%%%%%%%%%%%%%%
% At the beginning of your document, we start by opening a stream to a
% file that will be used to write the solutions to. At the end of your
% document, the package closes the stream.
% \changes{v0.8}{2014/07/15}{moved newwrite of exercise stream to this
% spot to avoid consuming all handles}
%    \begin{macrocode}
\AtBeginDocument{
  \newwrite\solutionstream
  \immediate\openout\solutionstream=\jobname.sol.tex
  \newwrite\exercisestream
}
\AtEndDocument{
  \immediate\closeout\solutionstream
}
%    \end{macrocode}
%
% \subsection{Exercises counter}
% %%%%%%%%%%%%%%%%%%%%%%%%%%%%%%
% By providing an exercise counter, proper numbering of the exercises
% is provided to allow for good cross referencing of the solutions to
% the exercises.
% \changes{v0.2}{2012/01/06}{Removed dash in counter when in document
% without sectioning commands}
%    \begin{macrocode}
\newcounter{exercise}[subsection]
\setcounter{exercise}{0}
\renewcommand{\theexercise}{%
  \@ifundefined{c@chapter}{}{\if0\arabic{chapter}\else\arabic{chapter}.\fi}%
  \if0\arabic{section}\else\arabic{section}\fi%
  \if0\arabic{subsection}\else.\arabic{subsection}\fi%
  \if0\arabic{subsubsection}\else.\arabic{subsubsection}\fi%
  \if0\arabic{exercise}\else%
    \@ifundefined{c@chapter}%
                 {\if0\arabic{section}\else-\fi}%
                 {-}%
    \arabic{exercise}%
  \fi
}
%    \end{macrocode}
%
%
%
% \subsection{Detokenization in order to cope with utf8}
%
% Combining old-school \LaTeX{} (before \XeTeX{} and \LuaTeX{}) and
% UTF-8 is a pain.
% Detokenization has been suggested by Geoffrey Poore to solve issues
% with UTF-8 characters messing up the |fancyvrb| internals.
% \changes{v0.7}{2014/07/14}{Added detokenized writing}
%    \begin{macrocode}
\newcommand{\GPES@write@detok}[1]{%
  \immediate\write\exercisestream{\detokenize{#1}}}%
\newcommand{\GPSS@write@detok}[1]{%
  \immediate\write\solutionstream{\detokenize{#1}}}%
\newcommand{\GPESS@write@detok}[1]{%
  \GPES@write@detok{#1}%
  \GPSS@write@detok{#1}}%
%    \end{macrocode}
%
%
% \section{The user environments}
%
% \begin{macro}{exercise}
%   The \texttt{exercise} environment is used to typeset your
%   exercises, provide them with a nice label and allow for copying
%   the exercise to the solutions file (if the package option
%   \texttt{copyexercisesinsolution}) is set. The label can be
%   set by redefining the \cs{exercisename} macro, or by relying on
%   the \textsf{Babel} provisions. The code is almost litteraly
%   taken from the \textsf{fancyvrb} package.
%    \begin{macrocode}
\def\exercise{\FV@Environment{}{exercise}}
\def\FVB@exercise{%
  \refstepcounter{exercise}%
  \immediate\openout\exercisestream=\jobname.exc.tex
  \ifexsol@copyexercisesinsolutions
    \typeout{Writing exercise to \jobname.sol.tex}
    \immediate\write\solutionstream{\string\par---\string\newline
      \string\textbf\string{\exercisename{} \theexercise \string}}
  \else
    \immediate\write\solutionstream{\string\par---\string\newline}
  \fi
  \immediate\write\exercisestream{\string\begin{exsol@exercise}}
  \@bsphack
  \begingroup
    \FV@UseKeyValues
    \FV@DefineWhiteSpace
    \def\FV@Space{\space}%
    \FV@DefineTabOut
    \ifexsol@copyexercisesinsolutions
      \let\FV@ProcessLine\GPESS@write@detok %
    \else
      \let\FV@ProcessLine\GPES@write@detok %
    \fi
    \relax
    \let\FV@FontScanPrep\relax
    \let\@noligs\relax
    \FV@Scan
  }
\def\FVE@exercise{
  \endgroup\@esphack
  \immediate\write\exercisestream{\string\end{exsol@exercise}}
  \ifexsol@copyexercisesinsolutions
    \immediate\write\solutionstream{\string~\string\newline}
  \fi
  \immediate\closeout\exercisestream
  \input{\jobname.exc.tex}
}
\DefineVerbatimEnvironment{exercise}{exercise}{}
%    \end{macrocode}
% \end{macro}
%
% \begin{macro}{exsol@exercise}
%   The \texttt{exsol@exercise} environment is an internal macro used
%   to typeset your exercises and provide them with a nice label and
%   number. Do not use it directly. Use the proper environment
%   \texttt{exercise} instead.
%   \changes{v0.2}{2012/01/06}{Attempted to fix MiKTeX formatting problems}
%   \changes{v0.3}{2012/01/08}{Fixed labelsep to avoid cluttered
%   itemize environments}
%   \changes{v0.4}{2012/01/06}{Added option exercisesinlist such that
%   default results in non list formatting of exercise}
%   \changes{v0.5}{2012/01/09}{Changed implementation to allow for
%   copying the exercises to the solutions file.}
%    \begin{macrocode}
\newenvironment{exsol@exercise}[0]
{%
  \ifthenelse{\boolean{exsol@minipage}}{\begin{minipage}[t]{\textwidth}}{}%
    \ifthenelse{\boolean{exsol@exerciseaslist}}
               {\begin{list}%
                   {%
                   }%
                   {%
                     \setlength{\topsep}{\exsolexercisetopbottomsep}%
                     \setlength{\leftmargin}{\exsolexerciseleftmargin}%
                     \setlength{\rightmargin}{\exsolexerciserightmargin}%
                     \setlength{\listparindent}{\exsolexerciseparindent}%
                     \setlength{\itemindent}{\exsolexerciseitemindent}%
                     \setlength{\parsep}{\exsolexerciseparsep}
                     \setlength{\labelsep}{\exsolexerciselabelsep}
                     \setlength{\labelwidth}{\exsolexerciselabelwidth}}
                 \item[\textit{~\exercisename{} \theexercise:~}]
               }%
               {\textit{\exercisename{} \theexercise:}}
}
{%
  \ifthenelse{\boolean{exsol@exerciseaslist}}%
             {\end{list}}{}%
  \ifthenelse{\boolean{exsol@minipage}}{\end{minipage}}{\par}%
}
%    \end{macrocode}
% \end{macro}
%
%
% \begin{macro}{solution}
%   The \texttt{solution} environment is used to typeset your solutions
%   and provide them with a nice label and number that corresponds to
%   the exercise that preceeded this solution. Theno label can be
%   set by redefining the \cs{solutionname} macro, or by relying on
%   the \textsf{Babel} provisions. The code is almost litteraly
%   taken from the \textsf{fancyvrb} package.
%    \begin{macrocode}
\def\solution{\FV@Environment{}{solution}}
\def\FVB@solution{%
  \typeout{Writing solution to \jobname.sol.tex}
  \immediate\write\solutionstream{\string\textbf\string{\solutionname{}\string}}
  \ifexsol@copyexercisesinsolutions
    \immediate\write\solutionstream{\string\newline}
  \else
    \immediate\write\solutionstream{\string\textbf\string{\theexercise\string}%
                                    \string\newline}
  \fi
  \@bsphack
  \begingroup
    \FV@UseKeyValues
    \FV@DefineWhiteSpace
    \def\FV@Space{\space}%
    \FV@DefineTabOut
    \let\FV@ProcessLine\GPSS@write@detok %
    \relax
    \let\FV@FontScanPrep\relax
    \let\@noligs\relax
    \FV@Scan
  }
\def\FVE@solution{\endgroup\@esphack}
\DefineVerbatimEnvironment{solution}{solution}{}
%    \end{macrocode}
% \end{macro}
%
% \begin{macro}{exercises}
%   The \texttt{exercises} environment helps typesetting your exercises to
%   stand out from the rest of the text. You may use it at the end of
%   a chapter, or just to group some exercises in the text.
%   \changes{v0.2}{2012/01/06}{Attempted to fix MiKTeX formatting problems}
%   \changes{v0.3}{2012/01/07}{Added some extra whitespace below exercisesname}
%    \begin{macrocode}
\newenvironment{exercises}
{\par\exercisesfontsize\rule{.25\linewidth}{0.15mm}\vspace*{\exsolexercisesaboveskip}\\*%
 \textbf{\normalsize \exercisesname}}
{\vspace*{-\baselineskip}\vspace*{\exsolexercisesbelowskip}\rule{.25\linewidth}{0.15mm}\par}
%    \end{macrocode}
% \end{macro}
%
% \subsection{Some Babel provisions}
% %%%%%%%%%%%%%%%%%%%%%%%%%%%%%%%%%%
% \changes{v0.2}{2012/01/06}{Fixed babel errors}
% \begin{macro}{\exercisename}
%   The exercise environment makes use of a label \texttt{\exercisename{}}
%   macro.
%    \begin{macrocode}
\newcommand{\exercisename}{Exercise}
%    \end{macrocode}
% \end{macro}
%
% \begin{macro}{\exercisesname}
%   The exercises environment makes use of a label \texttt{\exercisesname{}}
%   macro.
%    \begin{macrocode}
\newcommand{\exercisesname}{Exercises}
%    \end{macrocode}
% \end{macro}
% 
% \begin{macro}{\solutionname}
%   The solution environment makes use of a label \texttt{\solutionname{}}
%   macro.
%    \begin{macrocode}
\newcommand{\solutionname}{Solution}
%    \end{macrocode}
% \end{macro}
%
% \begin{macro}{\solutionname}
%   The solution environment makes use of a label \texttt{\solutionname{}}
%   macro.
% \changes{v0.8}{2014/07/15}{Added missing babel tag}
%    \begin{macrocode}
\newcommand{\solutionsname}{Solutions}
%    \end{macrocode}
% \end{macro}
% 
% 
% You may redefine these macros, but to help you out a little bit, we
% provide with some basic Babel auxiliaries. If you're a true polyglot
% and are willing to help me out by providing translations for other
% languages, I'm very willing to incorporate them into the code.
%
% \changes{v0.7}{2014/07/14}{Added Finnish language support}
%    \begin{macrocode}
\addto\captionsdutch{%
  \renewcommand{\exercisename}{Oefening}%
  \renewcommand{\exercisesname}{Oefeningen}%
  \renewcommand{\solutionname}{Oplossing}%
  \renewcommand{\solutionsname}{Oplossingen}%
}
\addto\captionsgerman{%
  \renewcommand{\exercisename}{Aufgabe}%
  \renewcommand{\exercisesname}{Aufgaben}%
  \renewcommand{\solutionname}{L\"osung}%
  \renewcommand{\solutionsname}{L\"osungen}%
}
\addto\captionsfrench{%
  \renewcommand{\exercisename}{Exercice}%
  \renewcommand{\exercisesname}{Exercices}%
  \renewcommand{\solutionname}{Solution}%
  \renewcommand{\solutionsname}{Solutions}%
}
\addto\captionsfinnish{
  \renewcommand{\exercisename}{Teht\"av\"a}%
  \renewcommand{\exercisesname}{Teht\"avi\"a}%
  \renewcommand{\solutionname}{Ratkaisu}%
  \renewcommand{\solutionsname}{Ratkaisut}%
}
%    \end{macrocode}
%
%
%
% Now the final hack overloads the basic sectioning commands to make
% sure that they are copied into your solution book.
%
%    \begin{macrocode}
\let\exsol@@makechapterhead\@makechapterhead
\def\@makechapterhead#1{%
  \immediate\write\solutionstream{\string\chapter{#1}}%
  \exsol@@makechapterhead{#1}
}
\ifdefined\frontmatter
  \let\exsol@@frontmatter\frontmatter
  \def\frontmatter{%
    \immediate\write\solutionstream{\string\frontmatter}%
    \exsol@@frontmatter
  }
\fi
\ifdefined\frontmatter
  \let\exsol@@mainmatter\mainmatter
  \def\mainmatter{%
    \immediate\write\solutionstream{\string\mainmatter}%
    \exsol@@mainmatter
  }
\fi
\ifdefined\backmatter
  \let\exsol@@backmatter\backmatter
  \def\backmatter{%
    \immediate\write\solutionstream{\string\backmatter}%
    \exsol@@backmatter
  }
\fi
%    \end{macrocode}
%
% \begin{macro}{\noexercisesinchapter}
%   If you have chapters without exercises, you may want to indicate
%   this clearly into your source. Otherwise empty chapters may appear
%   in your solution book.
%    \begin{macrocode}
\newcommand{\noexercisesinchapter}
{
  \immediate\write\solutionstream{No exercises in this chapter}
}
%    \end{macrocode}
% \end{macro}
%
%    \begin{macrocode}
%</package>
%    \end{macrocode}
%
% \bibliographystyle{alpha}
%
% \begin{thebibliography}{99}
%
% \bibitem{fancyvrb}
% Timothy Van Zandt, Herbert Vo\ss, Denis Girou, Sebastian Rahtz, Niall
% Mansfield 
% \newblock The \texttt{fancyvrb} package.
% \newblock \url{http://ctan.org/pkg/fancyvrb}.
% \newblock online, accessed in January 2012.
%
% \bibitem{CTAN} 
% The Comprehensive TeX Archive Network.
% \newblock \url{http://www.ctan.org}.
% \newblock online, accessed in January 2012.
%
% \end{thebibliography}
%
% \Finale
\endinput

%
% \end{document}
% \end{VerbatimOut}
% \VerbatimInput[frame=lines,gobble=2,fontsize=\footnotesize]{exsol-solutionbook.tex}
% 
% You may generate this solution book, by running \LaTeX{} on the
% file named \texttt{exsol-solutionbook.tex} that is generated when running
% \LaTeX{} on the \texttt{exsol.dtx} file.
%
% The result approximately looks like this:
%
% \setcounter{equation}{0}
% \rule{\linewidth}{.7pt}
% \begin{center}
% {\Large Solutions to the exercises, specified in the \textsf{ExSol} package}\\
% {\large Walter Daems}\\
% {\large 2013/05/12}
% \end{center}
% \par---\newline\textbf{Solution 3.1-1}
%     Let's start by rearranging the equation, a bit:
%     \begin{eqnarray}
%       5.7 x^2 - 3.1 x &=% 5.3\\
%       5.7 x^2 - 3.1 x -5.3 &=% 0
%     \end{eqnarray}
%     The equation is now in the standard form:
%     \begin{equation}
%       a x^2 + b x + c = 0
%     \end{equation}
%     For quadratic equations in the standard form, we know that two
%     solutions exist:
%     \begin{equation}
%       x_{1,2} = \frac{ -b \pm \sqrt{d}}{2a}
%     \end{equation}
%     with
%     \begin{equation}
%       d = b^2 - 4 a c
%     \end{equation}
%     If we apply this to our case, we obtain:
%     \begin{equation}
%       d = (-3.1)^2 - 4 \cdot 5.7 \cdot (-5.3) = 130.45
%     \end{equation}
%     and
%     \begin{eqnarray}
%       x_1 &=& \frac{3.1 + \sqrt{130.45}}{11.4} = 1.27\\
%       x_2 &=& \frac{3.1 + \sqrt{130.45}}{11.4} = -0.73
%     \end{eqnarray}
%     The proposed values $x = x_1, x_2$ are solutions to
%     the given equation.
% \par---\newline\textbf{Solution 3.1-2}
%       This calls for application of Pythagoras' theorem, which
%       tells us:
%       \begin{equation}
%         \left\|A\right\|^2 + \left\|B\right\|^2 = \left\|C\right\|^2
%       \end{equation}
%       and therefore:
%       \begin{eqnarray}
%         \left\|C\right\|
%         &=& \sqrt{\left\|A\right\|^2 + \left\|B\right\|^2}\\
%         &=& \sqrt{3^2 + 4^2}\\
%         &=& \sqrt{25} = 5
%       \end{eqnarray}
%       Therefore, the length of the hypotenuse equals $5$.
%
% \rule{\linewidth}{.7pt}
%
% \subsection{Fiddling with the spacing}
%
% The default spacing provided by the \textsf{ExSol} package should be
% fine for most users. However, if you like to tweak, below you can
% find the controls.
% \subsubsection{Spacing before and after the \texttt{exercises} environment}
%
% The lengths below control the spacing of the |exercises| environment:
% \begin{itemize}
% \item |exsolexerciseaboveskip|: rubber length controlling the
% vertical space after the top marker line of the environment
% \item |exsolexercisebelowskip|: rubber length controlling the
% vertical space before the bottom marker line of the environment
% \end{itemize}
%
% You can simply specify them like:
% \begin{verbatim}
% \setlength{\exsolexercisesaboveskip}{1ex plus 1pt minus 1pt}
% \setlength{\exsolexercisesbelowskip}{1ex plus 1pt minus 1pt}
% \end{verbatim}
% The spacings specified here are the package defaults.
%
% \subsubsection{Spacing of the individual exercises}
% Caution: the spacing can only be tuned, when one invokes the
% |exerciseaslist| package option!
%
% Then lengths below control the spacing of the |exercise| environment:
% \begin{itemize}
% \item |exercisetopbottomsep|: rubber length controlling the vertical
% space before and after individual exercises
% \item |exerciseleftmargin|: length controlling the horizontal
% space between the surrounding environment's left margin (most
% often the page margin) and the left edge of the exercise
% environment 
% \item |exerciseleftmargin|: length controlling the horizontal
% space between the surrounding environment's right margin (most
% often the page margin) and the right edge of the exercise
% environment
% \item |exerciseitemindent|: length controlling the first-line
% indentation of the first paragraph in the exercise environment
% (actually, the label is set w.r.t. this position, that we will
% conveniently call position 'x')
% \item |exerciseparindent|: length controlling the first-line
% indentation of the other paragraphs in the exercise environment.
% \item |exerciselabelsep|: length controlling the distance between
% the label and position 'x'
% \item |exerciselabelwidth|: minimal width of the (internally
% right-alligned) box to use for the exercises label; if the box is
% not sufficiently big, position 'x' is shifted to the right
% \item |exerciseparsep|: internal paragraph separation (vertically)
% \end{itemize}
% 
% You can simply specify them like:
% \begin{verbatim}
% \setlength{\exsolexercisetopbottomsep}{0pt plus 0pt minus 1pt}
% \setlength{\exsolexerciseleftmargin}{1em}
% \setlength{\exsolexerciserightmargin}{1em}
% \setlength{\exsolexerciseparindent}{0em}
% \setlength{\exsolexerciselabelsep}{0.5em}
% \setlength{\exsolexerciselabelwidth}{0pt}
% \setlength{\exsolexerciseitemindent}{0pt}
% \setlength{\exsolexerciseparsep}{\parskip}
% \end{verbatim}
% The spacings specified here are the package defaults.
%
% \subsection{Tips and tricks}
%
% If you want to include the solutions all at the
% end of the current document, you need to explicitly close the
% solution stream before including it:
% \begin{verbatim}
%   \closeout\solutionstream\input{\jobname.sol.tex}
% \end{verbatim}
%
% If you want to avoid exercises being split by a page boundary, then
% provide the package option 'minipage'. This causes the exercises to
% be wrapped in a minipage environment.
% 
% \clearpage
%
% \section{Implementation}
% %%%%%%%%%%%%%%%%%%%%%%%
%    \begin{macrocode}
%<*package>
%    \end{macrocode}
%
% \subsection{Auxiliary packages}
% %%%%%%%%%%%%%%%%%%%%%%%%%%%%%%%
% The package uses some auxiliary packages:
%    \begin{macrocode}
\RequirePackage{fancyvrb}
\RequirePackage{ifthen}
\RequirePackage{kvoptions}
%    \end{macrocode}
%
% \subsection{Package options}
% %%%%%%%%%%%%%%%%%%%%%%%%%%%%
% The package offers some options:
%
% \changes{v0.2}{2012/01/06}{Added option exercisesfont}
% \changes{v0.4}{2012/01/09}{Changed name of option to exercisesfontsize}
%
% \begin{macro}{exercisesfontsize}
%  This option allows setting the font of the \texttt{exercises}
%  environment. You may chopse one of tiny, scriptsize, footnotesize,
%  small, normalsize, large, etc.\\
%  E.g., \texttt{[exercisesfontsize=small]}.
%    \begin{macrocode}
\DeclareStringOption[normalsize]{exercisesfontsize}
%    \end{macrocode}
% \end{macro}
%
% \changes{v0.4}{2012/01/06}{Added option exercisesinlist}
% \changes{v0.5}{2012/01/09}{Changed option exercisesinlist to exerciseaslist}
%
% \begin{macro}{exerciseaslist}
%  This boolean option (true, false) allows setting the typesetting of
%  the \texttt{exercises} in a list environment. This causes the
%  exercises to be typeset in a more compact fashion, with indented
%  left and right margin. 
%    \begin{macrocode}
\DeclareBoolOption[false]{exerciseaslist}
%    \end{macrocode}
% \end{macro}
%
% \changes{v0.5}{2012/01/09}{Added option copyexercisesinsolutions}
% \begin{macro}{copyexercisesinsolutions}
%  This boolean option (true, false) allows copying the exercises in
%  the solutions file, to allow for making a complete stand-alone
%  exercises bundle.
%    \begin{macrocode}
\DeclareBoolOption[false]{copyexercisesinsolutions}
%    \end{macrocode}
% \end{macro}
%
% \changes{v0.9}{2014/07/28}{. Changed default behavior
% w.r.t. minipage-wraping of exercises}
% \begin{macro}{minipage}
%  This boolean option (true, false) causes the exercises to be
%  wrapped in minipages. This avoids them getting split by a page
%  boundary.
%    \begin{macrocode}
\DeclareBoolOption[false]{minipage}
%    \end{macrocode}
% \end{macro}
%
% The options are processed using:
%    \begin{macrocode}
\ProcessKeyvalOptions*
%    \end{macrocode}
%
% The options are subsequently handled
%    \begin{macrocode}
\newcommand{\exercisesfontsize}{\csname \exsol@exercisesfontsize\endcsname}
%    \end{macrocode}
%
%
% \subsection{Customization of lengths}
% %%%%%%%%%%%%%%%%%%%%%%%%%%%%%%%%%%%%%%%
% The commands below allow customizing many lengths that control the
% typesetting of the exercises.
%
% \changes{v0.91}{2014/08/31}{added user-accessible lengths}
% First some lengths to control the spacing before and after |exercises|.
%    \begin{macrocode}
\newlength{\exsolexercisesaboveskip}
\setlength{\exsolexercisesaboveskip}{1ex plus 1pt minus 1pt}
\newlength{\exsolexercisesbelowskip}
\setlength{\exsolexercisesbelowskip}{1ex plus 1pt minus 1pt}
%    \end{macrocode}
%
% Then some lengths to control the spacing for a single
% exercise. These lengths only work when the |exerciseaslist| package
% option has been specified. Sensible defaults have been set.
%    \begin{macrocode}
\newlength{\exsolexercisetopbottomsep}
\setlength{\exsolexercisetopbottomsep}{0pt plus 0pt minus 1pt}
\newlength{\exsolexerciseleftmargin}
\setlength{\exsolexerciseleftmargin}{1em}
\newlength{\exsolexerciserightmargin}
\setlength{\exsolexerciserightmargin}{1em}
\newlength{\exsolexerciseparindent}
\setlength{\exsolexerciseparindent}{0em}
\newlength{\exsolexerciselabelsep}
\setlength{\exsolexerciselabelsep}{0.5em}
\newlength{\exsolexerciselabelwidth}
\setlength{\exsolexerciselabelwidth}{0pt}
\newlength{\exsolexerciseitemindent}
\setlength{\exsolexerciseitemindent}{0pt}
\newlength{\exsolexerciseparsep}
\setlength{\exsolexerciseparsep}{\parskip}
%    \end{macrocode}
% 
% 
% \subsection{Con- and destruction of the auxiliary streams}
% %%%%%%%%%%%%%%%%%%%%%%%%%%%%%%%%%%%%%%%%%%%%%%%%%%%%%%%%%%
% At the beginning of your document, we start by opening a stream to a
% file that will be used to write the solutions to. At the end of your
% document, the package closes the stream.
% \changes{v0.8}{2014/07/15}{moved newwrite of exercise stream to this
% spot to avoid consuming all handles}
%    \begin{macrocode}
\AtBeginDocument{
  \newwrite\solutionstream
  \immediate\openout\solutionstream=\jobname.sol.tex
  \newwrite\exercisestream
}
\AtEndDocument{
  \immediate\closeout\solutionstream
}
%    \end{macrocode}
%
% \subsection{Exercises counter}
% %%%%%%%%%%%%%%%%%%%%%%%%%%%%%%
% By providing an exercise counter, proper numbering of the exercises
% is provided to allow for good cross referencing of the solutions to
% the exercises.
% \changes{v0.2}{2012/01/06}{Removed dash in counter when in document
% without sectioning commands}
%    \begin{macrocode}
\newcounter{exercise}[subsection]
\setcounter{exercise}{0}
\renewcommand{\theexercise}{%
  \@ifundefined{c@chapter}{}{\if0\arabic{chapter}\else\arabic{chapter}.\fi}%
  \if0\arabic{section}\else\arabic{section}\fi%
  \if0\arabic{subsection}\else.\arabic{subsection}\fi%
  \if0\arabic{subsubsection}\else.\arabic{subsubsection}\fi%
  \if0\arabic{exercise}\else%
    \@ifundefined{c@chapter}%
                 {\if0\arabic{section}\else-\fi}%
                 {-}%
    \arabic{exercise}%
  \fi
}
%    \end{macrocode}
%
%
%
% \subsection{Detokenization in order to cope with utf8}
%
% Combining old-school \LaTeX{} (before \XeTeX{} and \LuaTeX{}) and
% UTF-8 is a pain.
% Detokenization has been suggested by Geoffrey Poore to solve issues
% with UTF-8 characters messing up the |fancyvrb| internals.
% \changes{v0.7}{2014/07/14}{Added detokenized writing}
%    \begin{macrocode}
\newcommand{\GPES@write@detok}[1]{%
  \immediate\write\exercisestream{\detokenize{#1}}}%
\newcommand{\GPSS@write@detok}[1]{%
  \immediate\write\solutionstream{\detokenize{#1}}}%
\newcommand{\GPESS@write@detok}[1]{%
  \GPES@write@detok{#1}%
  \GPSS@write@detok{#1}}%
%    \end{macrocode}
%
%
% \section{The user environments}
%
% \begin{macro}{exercise}
%   The \texttt{exercise} environment is used to typeset your
%   exercises, provide them with a nice label and allow for copying
%   the exercise to the solutions file (if the package option
%   \texttt{copyexercisesinsolution}) is set. The label can be
%   set by redefining the \cs{exercisename} macro, or by relying on
%   the \textsf{Babel} provisions. The code is almost litteraly
%   taken from the \textsf{fancyvrb} package.
%    \begin{macrocode}
\def\exercise{\FV@Environment{}{exercise}}
\def\FVB@exercise{%
  \refstepcounter{exercise}%
  \immediate\openout\exercisestream=\jobname.exc.tex
  \ifexsol@copyexercisesinsolutions
    \typeout{Writing exercise to \jobname.sol.tex}
    \immediate\write\solutionstream{\string\par---\string\newline
      \string\textbf\string{\exercisename{} \theexercise \string}}
  \else
    \immediate\write\solutionstream{\string\par---\string\newline}
  \fi
  \immediate\write\exercisestream{\string\begin{exsol@exercise}}
  \@bsphack
  \begingroup
    \FV@UseKeyValues
    \FV@DefineWhiteSpace
    \def\FV@Space{\space}%
    \FV@DefineTabOut
    \ifexsol@copyexercisesinsolutions
      \let\FV@ProcessLine\GPESS@write@detok %
    \else
      \let\FV@ProcessLine\GPES@write@detok %
    \fi
    \relax
    \let\FV@FontScanPrep\relax
    \let\@noligs\relax
    \FV@Scan
  }
\def\FVE@exercise{
  \endgroup\@esphack
  \immediate\write\exercisestream{\string\end{exsol@exercise}}
  \ifexsol@copyexercisesinsolutions
    \immediate\write\solutionstream{\string~\string\newline}
  \fi
  \immediate\closeout\exercisestream
  \input{\jobname.exc.tex}
}
\DefineVerbatimEnvironment{exercise}{exercise}{}
%    \end{macrocode}
% \end{macro}
%
% \begin{macro}{exsol@exercise}
%   The \texttt{exsol@exercise} environment is an internal macro used
%   to typeset your exercises and provide them with a nice label and
%   number. Do not use it directly. Use the proper environment
%   \texttt{exercise} instead.
%   \changes{v0.2}{2012/01/06}{Attempted to fix MiKTeX formatting problems}
%   \changes{v0.3}{2012/01/08}{Fixed labelsep to avoid cluttered
%   itemize environments}
%   \changes{v0.4}{2012/01/06}{Added option exercisesinlist such that
%   default results in non list formatting of exercise}
%   \changes{v0.5}{2012/01/09}{Changed implementation to allow for
%   copying the exercises to the solutions file.}
%    \begin{macrocode}
\newenvironment{exsol@exercise}[0]
{%
  \ifthenelse{\boolean{exsol@minipage}}{\begin{minipage}[t]{\textwidth}}{}%
    \ifthenelse{\boolean{exsol@exerciseaslist}}
               {\begin{list}%
                   {%
                   }%
                   {%
                     \setlength{\topsep}{\exsolexercisetopbottomsep}%
                     \setlength{\leftmargin}{\exsolexerciseleftmargin}%
                     \setlength{\rightmargin}{\exsolexerciserightmargin}%
                     \setlength{\listparindent}{\exsolexerciseparindent}%
                     \setlength{\itemindent}{\exsolexerciseitemindent}%
                     \setlength{\parsep}{\exsolexerciseparsep}
                     \setlength{\labelsep}{\exsolexerciselabelsep}
                     \setlength{\labelwidth}{\exsolexerciselabelwidth}}
                 \item[\textit{~\exercisename{} \theexercise:~}]
               }%
               {\textit{\exercisename{} \theexercise:}}
}
{%
  \ifthenelse{\boolean{exsol@exerciseaslist}}%
             {\end{list}}{}%
  \ifthenelse{\boolean{exsol@minipage}}{\end{minipage}}{\par}%
}
%    \end{macrocode}
% \end{macro}
%
%
% \begin{macro}{solution}
%   The \texttt{solution} environment is used to typeset your solutions
%   and provide them with a nice label and number that corresponds to
%   the exercise that preceeded this solution. Theno label can be
%   set by redefining the \cs{solutionname} macro, or by relying on
%   the \textsf{Babel} provisions. The code is almost litteraly
%   taken from the \textsf{fancyvrb} package.
%    \begin{macrocode}
\def\solution{\FV@Environment{}{solution}}
\def\FVB@solution{%
  \typeout{Writing solution to \jobname.sol.tex}
  \immediate\write\solutionstream{\string\textbf\string{\solutionname{}\string}}
  \ifexsol@copyexercisesinsolutions
    \immediate\write\solutionstream{\string\newline}
  \else
    \immediate\write\solutionstream{\string\textbf\string{\theexercise\string}%
                                    \string\newline}
  \fi
  \@bsphack
  \begingroup
    \FV@UseKeyValues
    \FV@DefineWhiteSpace
    \def\FV@Space{\space}%
    \FV@DefineTabOut
    \let\FV@ProcessLine\GPSS@write@detok %
    \relax
    \let\FV@FontScanPrep\relax
    \let\@noligs\relax
    \FV@Scan
  }
\def\FVE@solution{\endgroup\@esphack}
\DefineVerbatimEnvironment{solution}{solution}{}
%    \end{macrocode}
% \end{macro}
%
% \begin{macro}{exercises}
%   The \texttt{exercises} environment helps typesetting your exercises to
%   stand out from the rest of the text. You may use it at the end of
%   a chapter, or just to group some exercises in the text.
%   \changes{v0.2}{2012/01/06}{Attempted to fix MiKTeX formatting problems}
%   \changes{v0.3}{2012/01/07}{Added some extra whitespace below exercisesname}
%    \begin{macrocode}
\newenvironment{exercises}
{\par\exercisesfontsize\rule{.25\linewidth}{0.15mm}\vspace*{\exsolexercisesaboveskip}\\*%
 \textbf{\normalsize \exercisesname}}
{\vspace*{-\baselineskip}\vspace*{\exsolexercisesbelowskip}\rule{.25\linewidth}{0.15mm}\par}
%    \end{macrocode}
% \end{macro}
%
% \subsection{Some Babel provisions}
% %%%%%%%%%%%%%%%%%%%%%%%%%%%%%%%%%%
% \changes{v0.2}{2012/01/06}{Fixed babel errors}
% \begin{macro}{\exercisename}
%   The exercise environment makes use of a label \texttt{\exercisename{}}
%   macro.
%    \begin{macrocode}
\newcommand{\exercisename}{Exercise}
%    \end{macrocode}
% \end{macro}
%
% \begin{macro}{\exercisesname}
%   The exercises environment makes use of a label \texttt{\exercisesname{}}
%   macro.
%    \begin{macrocode}
\newcommand{\exercisesname}{Exercises}
%    \end{macrocode}
% \end{macro}
% 
% \begin{macro}{\solutionname}
%   The solution environment makes use of a label \texttt{\solutionname{}}
%   macro.
%    \begin{macrocode}
\newcommand{\solutionname}{Solution}
%    \end{macrocode}
% \end{macro}
%
% \begin{macro}{\solutionname}
%   The solution environment makes use of a label \texttt{\solutionname{}}
%   macro.
% \changes{v0.8}{2014/07/15}{Added missing babel tag}
%    \begin{macrocode}
\newcommand{\solutionsname}{Solutions}
%    \end{macrocode}
% \end{macro}
% 
% 
% You may redefine these macros, but to help you out a little bit, we
% provide with some basic Babel auxiliaries. If you're a true polyglot
% and are willing to help me out by providing translations for other
% languages, I'm very willing to incorporate them into the code.
%
% \changes{v0.7}{2014/07/14}{Added Finnish language support}
%    \begin{macrocode}
\addto\captionsdutch{%
  \renewcommand{\exercisename}{Oefening}%
  \renewcommand{\exercisesname}{Oefeningen}%
  \renewcommand{\solutionname}{Oplossing}%
  \renewcommand{\solutionsname}{Oplossingen}%
}
\addto\captionsgerman{%
  \renewcommand{\exercisename}{Aufgabe}%
  \renewcommand{\exercisesname}{Aufgaben}%
  \renewcommand{\solutionname}{L\"osung}%
  \renewcommand{\solutionsname}{L\"osungen}%
}
\addto\captionsfrench{%
  \renewcommand{\exercisename}{Exercice}%
  \renewcommand{\exercisesname}{Exercices}%
  \renewcommand{\solutionname}{Solution}%
  \renewcommand{\solutionsname}{Solutions}%
}
\addto\captionsfinnish{
  \renewcommand{\exercisename}{Teht\"av\"a}%
  \renewcommand{\exercisesname}{Teht\"avi\"a}%
  \renewcommand{\solutionname}{Ratkaisu}%
  \renewcommand{\solutionsname}{Ratkaisut}%
}
%    \end{macrocode}
%
%
%
% Now the final hack overloads the basic sectioning commands to make
% sure that they are copied into your solution book.
%
%    \begin{macrocode}
\let\exsol@@makechapterhead\@makechapterhead
\def\@makechapterhead#1{%
  \immediate\write\solutionstream{\string\chapter{#1}}%
  \exsol@@makechapterhead{#1}
}
\ifdefined\frontmatter
  \let\exsol@@frontmatter\frontmatter
  \def\frontmatter{%
    \immediate\write\solutionstream{\string\frontmatter}%
    \exsol@@frontmatter
  }
\fi
\ifdefined\frontmatter
  \let\exsol@@mainmatter\mainmatter
  \def\mainmatter{%
    \immediate\write\solutionstream{\string\mainmatter}%
    \exsol@@mainmatter
  }
\fi
\ifdefined\backmatter
  \let\exsol@@backmatter\backmatter
  \def\backmatter{%
    \immediate\write\solutionstream{\string\backmatter}%
    \exsol@@backmatter
  }
\fi
%    \end{macrocode}
%
% \begin{macro}{\noexercisesinchapter}
%   If you have chapters without exercises, you may want to indicate
%   this clearly into your source. Otherwise empty chapters may appear
%   in your solution book.
%    \begin{macrocode}
\newcommand{\noexercisesinchapter}
{
  \immediate\write\solutionstream{No exercises in this chapter}
}
%    \end{macrocode}
% \end{macro}
%
%    \begin{macrocode}
%</package>
%    \end{macrocode}
%
% \bibliographystyle{alpha}
%
% \begin{thebibliography}{99}
%
% \bibitem{fancyvrb}
% Timothy Van Zandt, Herbert Vo\ss, Denis Girou, Sebastian Rahtz, Niall
% Mansfield 
% \newblock The \texttt{fancyvrb} package.
% \newblock \url{http://ctan.org/pkg/fancyvrb}.
% \newblock online, accessed in January 2012.
%
% \bibitem{CTAN} 
% The Comprehensive TeX Archive Network.
% \newblock \url{http://www.ctan.org}.
% \newblock online, accessed in January 2012.
%
% \end{thebibliography}
%
% \Finale
\endinput

%
% \end{document}
% \end{VerbatimOut}
% \VerbatimInput[frame=lines,gobble=2,fontsize=\footnotesize]{exsol-solutionbook.tex}
% 
% You may generate this solution book, by running \LaTeX{} on the
% file named \texttt{exsol-solutionbook.tex} that is generated when running
% \LaTeX{} on the \texttt{exsol.dtx} file.
%
% The result approximately looks like this:
%
% \setcounter{equation}{0}
% \rule{\linewidth}{.7pt}
% \begin{center}
% {\Large Solutions to the exercises, specified in the \textsf{ExSol} package}\\
% {\large Walter Daems}\\
% {\large 2013/05/12}
% \end{center}
% \par---\newline\textbf{Solution 3.1-1}
%     Let's start by rearranging the equation, a bit:
%     \begin{eqnarray}
%       5.7 x^2 - 3.1 x &=% 5.3\\
%       5.7 x^2 - 3.1 x -5.3 &=% 0
%     \end{eqnarray}
%     The equation is now in the standard form:
%     \begin{equation}
%       a x^2 + b x + c = 0
%     \end{equation}
%     For quadratic equations in the standard form, we know that two
%     solutions exist:
%     \begin{equation}
%       x_{1,2} = \frac{ -b \pm \sqrt{d}}{2a}
%     \end{equation}
%     with
%     \begin{equation}
%       d = b^2 - 4 a c
%     \end{equation}
%     If we apply this to our case, we obtain:
%     \begin{equation}
%       d = (-3.1)^2 - 4 \cdot 5.7 \cdot (-5.3) = 130.45
%     \end{equation}
%     and
%     \begin{eqnarray}
%       x_1 &=& \frac{3.1 + \sqrt{130.45}}{11.4} = 1.27\\
%       x_2 &=& \frac{3.1 + \sqrt{130.45}}{11.4} = -0.73
%     \end{eqnarray}
%     The proposed values $x = x_1, x_2$ are solutions to
%     the given equation.
% \par---\newline\textbf{Solution 3.1-2}
%       This calls for application of Pythagoras' theorem, which
%       tells us:
%       \begin{equation}
%         \left\|A\right\|^2 + \left\|B\right\|^2 = \left\|C\right\|^2
%       \end{equation}
%       and therefore:
%       \begin{eqnarray}
%         \left\|C\right\|
%         &=& \sqrt{\left\|A\right\|^2 + \left\|B\right\|^2}\\
%         &=& \sqrt{3^2 + 4^2}\\
%         &=& \sqrt{25} = 5
%       \end{eqnarray}
%       Therefore, the length of the hypotenuse equals $5$.
%
% \rule{\linewidth}{.7pt}
%
% \subsection{Fiddling with the spacing}
%
% The default spacing provided by the \textsf{ExSol} package should be
% fine for most users. However, if you like to tweak, below you can
% find the controls.
% \subsubsection{Spacing before and after the \texttt{exercises} environment}
%
% The lengths below control the spacing of the |exercises| environment:
% \begin{itemize}
% \item |exsolexerciseaboveskip|: rubber length controlling the
% vertical space after the top marker line of the environment
% \item |exsolexercisebelowskip|: rubber length controlling the
% vertical space before the bottom marker line of the environment
% \end{itemize}
%
% You can simply specify them like:
% \begin{verbatim}
% \setlength{\exsolexercisesaboveskip}{1ex plus 1pt minus 1pt}
% \setlength{\exsolexercisesbelowskip}{1ex plus 1pt minus 1pt}
% \end{verbatim}
% The spacings specified here are the package defaults.
%
% \subsubsection{Spacing of the individual exercises}
% Caution: the spacing can only be tuned, when one invokes the
% |exerciseaslist| package option!
%
% Then lengths below control the spacing of the |exercise| environment:
% \begin{itemize}
% \item |exercisetopbottomsep|: rubber length controlling the vertical
% space before and after individual exercises
% \item |exerciseleftmargin|: length controlling the horizontal
% space between the surrounding environment's left margin (most
% often the page margin) and the left edge of the exercise
% environment 
% \item |exerciseleftmargin|: length controlling the horizontal
% space between the surrounding environment's right margin (most
% often the page margin) and the right edge of the exercise
% environment
% \item |exerciseitemindent|: length controlling the first-line
% indentation of the first paragraph in the exercise environment
% (actually, the label is set w.r.t. this position, that we will
% conveniently call position 'x')
% \item |exerciseparindent|: length controlling the first-line
% indentation of the other paragraphs in the exercise environment.
% \item |exerciselabelsep|: length controlling the distance between
% the label and position 'x'
% \item |exerciselabelwidth|: minimal width of the (internally
% right-alligned) box to use for the exercises label; if the box is
% not sufficiently big, position 'x' is shifted to the right
% \item |exerciseparsep|: internal paragraph separation (vertically)
% \end{itemize}
% 
% You can simply specify them like:
% \begin{verbatim}
% \setlength{\exsolexercisetopbottomsep}{0pt plus 0pt minus 1pt}
% \setlength{\exsolexerciseleftmargin}{1em}
% \setlength{\exsolexerciserightmargin}{1em}
% \setlength{\exsolexerciseparindent}{0em}
% \setlength{\exsolexerciselabelsep}{0.5em}
% \setlength{\exsolexerciselabelwidth}{0pt}
% \setlength{\exsolexerciseitemindent}{0pt}
% \setlength{\exsolexerciseparsep}{\parskip}
% \end{verbatim}
% The spacings specified here are the package defaults.
%
% \subsection{Tips and tricks}
%
% If you want to include the solutions all at the
% end of the current document, you need to explicitly close the
% solution stream before including it:
% \begin{verbatim}
%   \closeout\solutionstream\input{\jobname.sol.tex}
% \end{verbatim}
%
% If you want to avoid exercises being split by a page boundary, then
% provide the package option 'minipage'. This causes the exercises to
% be wrapped in a minipage environment.
% 
% \clearpage
%
% \section{Implementation}
% %%%%%%%%%%%%%%%%%%%%%%%
%    \begin{macrocode}
%<*package>
%    \end{macrocode}
%
% \subsection{Auxiliary packages}
% %%%%%%%%%%%%%%%%%%%%%%%%%%%%%%%
% The package uses some auxiliary packages:
%    \begin{macrocode}
\RequirePackage{fancyvrb}
\RequirePackage{ifthen}
\RequirePackage{kvoptions}
%    \end{macrocode}
%
% \subsection{Package options}
% %%%%%%%%%%%%%%%%%%%%%%%%%%%%
% The package offers some options:
%
% \changes{v0.2}{2012/01/06}{Added option exercisesfont}
% \changes{v0.4}{2012/01/09}{Changed name of option to exercisesfontsize}
%
% \begin{macro}{exercisesfontsize}
%  This option allows setting the font of the \texttt{exercises}
%  environment. You may chopse one of tiny, scriptsize, footnotesize,
%  small, normalsize, large, etc.\\
%  E.g., \texttt{[exercisesfontsize=small]}.
%    \begin{macrocode}
\DeclareStringOption[normalsize]{exercisesfontsize}
%    \end{macrocode}
% \end{macro}
%
% \changes{v0.4}{2012/01/06}{Added option exercisesinlist}
% \changes{v0.5}{2012/01/09}{Changed option exercisesinlist to exerciseaslist}
%
% \begin{macro}{exerciseaslist}
%  This boolean option (true, false) allows setting the typesetting of
%  the \texttt{exercises} in a list environment. This causes the
%  exercises to be typeset in a more compact fashion, with indented
%  left and right margin. 
%    \begin{macrocode}
\DeclareBoolOption[false]{exerciseaslist}
%    \end{macrocode}
% \end{macro}
%
% \changes{v0.5}{2012/01/09}{Added option copyexercisesinsolutions}
% \begin{macro}{copyexercisesinsolutions}
%  This boolean option (true, false) allows copying the exercises in
%  the solutions file, to allow for making a complete stand-alone
%  exercises bundle.
%    \begin{macrocode}
\DeclareBoolOption[false]{copyexercisesinsolutions}
%    \end{macrocode}
% \end{macro}
%
% \changes{v0.9}{2014/07/28}{. Changed default behavior
% w.r.t. minipage-wraping of exercises}
% \begin{macro}{minipage}
%  This boolean option (true, false) causes the exercises to be
%  wrapped in minipages. This avoids them getting split by a page
%  boundary.
%    \begin{macrocode}
\DeclareBoolOption[false]{minipage}
%    \end{macrocode}
% \end{macro}
%
% The options are processed using:
%    \begin{macrocode}
\ProcessKeyvalOptions*
%    \end{macrocode}
%
% The options are subsequently handled
%    \begin{macrocode}
\newcommand{\exercisesfontsize}{\csname \exsol@exercisesfontsize\endcsname}
%    \end{macrocode}
%
%
% \subsection{Customization of lengths}
% %%%%%%%%%%%%%%%%%%%%%%%%%%%%%%%%%%%%%%%
% The commands below allow customizing many lengths that control the
% typesetting of the exercises.
%
% \changes{v0.91}{2014/08/31}{added user-accessible lengths}
% First some lengths to control the spacing before and after |exercises|.
%    \begin{macrocode}
\newlength{\exsolexercisesaboveskip}
\setlength{\exsolexercisesaboveskip}{1ex plus 1pt minus 1pt}
\newlength{\exsolexercisesbelowskip}
\setlength{\exsolexercisesbelowskip}{1ex plus 1pt minus 1pt}
%    \end{macrocode}
%
% Then some lengths to control the spacing for a single
% exercise. These lengths only work when the |exerciseaslist| package
% option has been specified. Sensible defaults have been set.
%    \begin{macrocode}
\newlength{\exsolexercisetopbottomsep}
\setlength{\exsolexercisetopbottomsep}{0pt plus 0pt minus 1pt}
\newlength{\exsolexerciseleftmargin}
\setlength{\exsolexerciseleftmargin}{1em}
\newlength{\exsolexerciserightmargin}
\setlength{\exsolexerciserightmargin}{1em}
\newlength{\exsolexerciseparindent}
\setlength{\exsolexerciseparindent}{0em}
\newlength{\exsolexerciselabelsep}
\setlength{\exsolexerciselabelsep}{0.5em}
\newlength{\exsolexerciselabelwidth}
\setlength{\exsolexerciselabelwidth}{0pt}
\newlength{\exsolexerciseitemindent}
\setlength{\exsolexerciseitemindent}{0pt}
\newlength{\exsolexerciseparsep}
\setlength{\exsolexerciseparsep}{\parskip}
%    \end{macrocode}
% 
% 
% \subsection{Con- and destruction of the auxiliary streams}
% %%%%%%%%%%%%%%%%%%%%%%%%%%%%%%%%%%%%%%%%%%%%%%%%%%%%%%%%%%
% At the beginning of your document, we start by opening a stream to a
% file that will be used to write the solutions to. At the end of your
% document, the package closes the stream.
% \changes{v0.8}{2014/07/15}{moved newwrite of exercise stream to this
% spot to avoid consuming all handles}
%    \begin{macrocode}
\AtBeginDocument{
  \newwrite\solutionstream
  \immediate\openout\solutionstream=\jobname.sol.tex
  \newwrite\exercisestream
}
\AtEndDocument{
  \immediate\closeout\solutionstream
}
%    \end{macrocode}
%
% \subsection{Exercises counter}
% %%%%%%%%%%%%%%%%%%%%%%%%%%%%%%
% By providing an exercise counter, proper numbering of the exercises
% is provided to allow for good cross referencing of the solutions to
% the exercises.
% \changes{v0.2}{2012/01/06}{Removed dash in counter when in document
% without sectioning commands}
%    \begin{macrocode}
\newcounter{exercise}[subsection]
\setcounter{exercise}{0}
\renewcommand{\theexercise}{%
  \@ifundefined{c@chapter}{}{\if0\arabic{chapter}\else\arabic{chapter}.\fi}%
  \if0\arabic{section}\else\arabic{section}\fi%
  \if0\arabic{subsection}\else.\arabic{subsection}\fi%
  \if0\arabic{subsubsection}\else.\arabic{subsubsection}\fi%
  \if0\arabic{exercise}\else%
    \@ifundefined{c@chapter}%
                 {\if0\arabic{section}\else-\fi}%
                 {-}%
    \arabic{exercise}%
  \fi
}
%    \end{macrocode}
%
%
%
% \subsection{Detokenization in order to cope with utf8}
%
% Combining old-school \LaTeX{} (before \XeTeX{} and \LuaTeX{}) and
% UTF-8 is a pain.
% Detokenization has been suggested by Geoffrey Poore to solve issues
% with UTF-8 characters messing up the |fancyvrb| internals.
% \changes{v0.7}{2014/07/14}{Added detokenized writing}
%    \begin{macrocode}
\newcommand{\GPES@write@detok}[1]{%
  \immediate\write\exercisestream{\detokenize{#1}}}%
\newcommand{\GPSS@write@detok}[1]{%
  \immediate\write\solutionstream{\detokenize{#1}}}%
\newcommand{\GPESS@write@detok}[1]{%
  \GPES@write@detok{#1}%
  \GPSS@write@detok{#1}}%
%    \end{macrocode}
%
%
% \section{The user environments}
%
% \begin{macro}{exercise}
%   The \texttt{exercise} environment is used to typeset your
%   exercises, provide them with a nice label and allow for copying
%   the exercise to the solutions file (if the package option
%   \texttt{copyexercisesinsolution}) is set. The label can be
%   set by redefining the \cs{exercisename} macro, or by relying on
%   the \textsf{Babel} provisions. The code is almost litteraly
%   taken from the \textsf{fancyvrb} package.
%    \begin{macrocode}
\def\exercise{\FV@Environment{}{exercise}}
\def\FVB@exercise{%
  \refstepcounter{exercise}%
  \immediate\openout\exercisestream=\jobname.exc.tex
  \ifexsol@copyexercisesinsolutions
    \typeout{Writing exercise to \jobname.sol.tex}
    \immediate\write\solutionstream{\string\par---\string\newline
      \string\textbf\string{\exercisename{} \theexercise \string}}
  \else
    \immediate\write\solutionstream{\string\par---\string\newline}
  \fi
  \immediate\write\exercisestream{\string\begin{exsol@exercise}}
  \@bsphack
  \begingroup
    \FV@UseKeyValues
    \FV@DefineWhiteSpace
    \def\FV@Space{\space}%
    \FV@DefineTabOut
    \ifexsol@copyexercisesinsolutions
      \let\FV@ProcessLine\GPESS@write@detok %
    \else
      \let\FV@ProcessLine\GPES@write@detok %
    \fi
    \relax
    \let\FV@FontScanPrep\relax
    \let\@noligs\relax
    \FV@Scan
  }
\def\FVE@exercise{
  \endgroup\@esphack
  \immediate\write\exercisestream{\string\end{exsol@exercise}}
  \ifexsol@copyexercisesinsolutions
    \immediate\write\solutionstream{\string~\string\newline}
  \fi
  \immediate\closeout\exercisestream
  \input{\jobname.exc.tex}
}
\DefineVerbatimEnvironment{exercise}{exercise}{}
%    \end{macrocode}
% \end{macro}
%
% \begin{macro}{exsol@exercise}
%   The \texttt{exsol@exercise} environment is an internal macro used
%   to typeset your exercises and provide them with a nice label and
%   number. Do not use it directly. Use the proper environment
%   \texttt{exercise} instead.
%   \changes{v0.2}{2012/01/06}{Attempted to fix MiKTeX formatting problems}
%   \changes{v0.3}{2012/01/08}{Fixed labelsep to avoid cluttered
%   itemize environments}
%   \changes{v0.4}{2012/01/06}{Added option exercisesinlist such that
%   default results in non list formatting of exercise}
%   \changes{v0.5}{2012/01/09}{Changed implementation to allow for
%   copying the exercises to the solutions file.}
%    \begin{macrocode}
\newenvironment{exsol@exercise}[0]
{%
  \ifthenelse{\boolean{exsol@minipage}}{\begin{minipage}[t]{\textwidth}}{}%
    \ifthenelse{\boolean{exsol@exerciseaslist}}
               {\begin{list}%
                   {%
                   }%
                   {%
                     \setlength{\topsep}{\exsolexercisetopbottomsep}%
                     \setlength{\leftmargin}{\exsolexerciseleftmargin}%
                     \setlength{\rightmargin}{\exsolexerciserightmargin}%
                     \setlength{\listparindent}{\exsolexerciseparindent}%
                     \setlength{\itemindent}{\exsolexerciseitemindent}%
                     \setlength{\parsep}{\exsolexerciseparsep}
                     \setlength{\labelsep}{\exsolexerciselabelsep}
                     \setlength{\labelwidth}{\exsolexerciselabelwidth}}
                 \item[\textit{~\exercisename{} \theexercise:~}]
               }%
               {\textit{\exercisename{} \theexercise:}}
}
{%
  \ifthenelse{\boolean{exsol@exerciseaslist}}%
             {\end{list}}{}%
  \ifthenelse{\boolean{exsol@minipage}}{\end{minipage}}{\par}%
}
%    \end{macrocode}
% \end{macro}
%
%
% \begin{macro}{solution}
%   The \texttt{solution} environment is used to typeset your solutions
%   and provide them with a nice label and number that corresponds to
%   the exercise that preceeded this solution. Theno label can be
%   set by redefining the \cs{solutionname} macro, or by relying on
%   the \textsf{Babel} provisions. The code is almost litteraly
%   taken from the \textsf{fancyvrb} package.
%    \begin{macrocode}
\def\solution{\FV@Environment{}{solution}}
\def\FVB@solution{%
  \typeout{Writing solution to \jobname.sol.tex}
  \immediate\write\solutionstream{\string\textbf\string{\solutionname{}\string}}
  \ifexsol@copyexercisesinsolutions
    \immediate\write\solutionstream{\string\newline}
  \else
    \immediate\write\solutionstream{\string\textbf\string{\theexercise\string}%
                                    \string\newline}
  \fi
  \@bsphack
  \begingroup
    \FV@UseKeyValues
    \FV@DefineWhiteSpace
    \def\FV@Space{\space}%
    \FV@DefineTabOut
    \let\FV@ProcessLine\GPSS@write@detok %
    \relax
    \let\FV@FontScanPrep\relax
    \let\@noligs\relax
    \FV@Scan
  }
\def\FVE@solution{\endgroup\@esphack}
\DefineVerbatimEnvironment{solution}{solution}{}
%    \end{macrocode}
% \end{macro}
%
% \begin{macro}{exercises}
%   The \texttt{exercises} environment helps typesetting your exercises to
%   stand out from the rest of the text. You may use it at the end of
%   a chapter, or just to group some exercises in the text.
%   \changes{v0.2}{2012/01/06}{Attempted to fix MiKTeX formatting problems}
%   \changes{v0.3}{2012/01/07}{Added some extra whitespace below exercisesname}
%    \begin{macrocode}
\newenvironment{exercises}
{\par\exercisesfontsize\rule{.25\linewidth}{0.15mm}\vspace*{\exsolexercisesaboveskip}\\*%
 \textbf{\normalsize \exercisesname}}
{\vspace*{-\baselineskip}\vspace*{\exsolexercisesbelowskip}\rule{.25\linewidth}{0.15mm}\par}
%    \end{macrocode}
% \end{macro}
%
% \subsection{Some Babel provisions}
% %%%%%%%%%%%%%%%%%%%%%%%%%%%%%%%%%%
% \changes{v0.2}{2012/01/06}{Fixed babel errors}
% \begin{macro}{\exercisename}
%   The exercise environment makes use of a label \texttt{\exercisename{}}
%   macro.
%    \begin{macrocode}
\newcommand{\exercisename}{Exercise}
%    \end{macrocode}
% \end{macro}
%
% \begin{macro}{\exercisesname}
%   The exercises environment makes use of a label \texttt{\exercisesname{}}
%   macro.
%    \begin{macrocode}
\newcommand{\exercisesname}{Exercises}
%    \end{macrocode}
% \end{macro}
% 
% \begin{macro}{\solutionname}
%   The solution environment makes use of a label \texttt{\solutionname{}}
%   macro.
%    \begin{macrocode}
\newcommand{\solutionname}{Solution}
%    \end{macrocode}
% \end{macro}
%
% \begin{macro}{\solutionname}
%   The solution environment makes use of a label \texttt{\solutionname{}}
%   macro.
% \changes{v0.8}{2014/07/15}{Added missing babel tag}
%    \begin{macrocode}
\newcommand{\solutionsname}{Solutions}
%    \end{macrocode}
% \end{macro}
% 
% 
% You may redefine these macros, but to help you out a little bit, we
% provide with some basic Babel auxiliaries. If you're a true polyglot
% and are willing to help me out by providing translations for other
% languages, I'm very willing to incorporate them into the code.
%
% \changes{v0.7}{2014/07/14}{Added Finnish language support}
%    \begin{macrocode}
\addto\captionsdutch{%
  \renewcommand{\exercisename}{Oefening}%
  \renewcommand{\exercisesname}{Oefeningen}%
  \renewcommand{\solutionname}{Oplossing}%
  \renewcommand{\solutionsname}{Oplossingen}%
}
\addto\captionsgerman{%
  \renewcommand{\exercisename}{Aufgabe}%
  \renewcommand{\exercisesname}{Aufgaben}%
  \renewcommand{\solutionname}{L\"osung}%
  \renewcommand{\solutionsname}{L\"osungen}%
}
\addto\captionsfrench{%
  \renewcommand{\exercisename}{Exercice}%
  \renewcommand{\exercisesname}{Exercices}%
  \renewcommand{\solutionname}{Solution}%
  \renewcommand{\solutionsname}{Solutions}%
}
\addto\captionsfinnish{
  \renewcommand{\exercisename}{Teht\"av\"a}%
  \renewcommand{\exercisesname}{Teht\"avi\"a}%
  \renewcommand{\solutionname}{Ratkaisu}%
  \renewcommand{\solutionsname}{Ratkaisut}%
}
%    \end{macrocode}
%
%
%
% Now the final hack overloads the basic sectioning commands to make
% sure that they are copied into your solution book.
%
%    \begin{macrocode}
\let\exsol@@makechapterhead\@makechapterhead
\def\@makechapterhead#1{%
  \immediate\write\solutionstream{\string\chapter{#1}}%
  \exsol@@makechapterhead{#1}
}
\ifdefined\frontmatter
  \let\exsol@@frontmatter\frontmatter
  \def\frontmatter{%
    \immediate\write\solutionstream{\string\frontmatter}%
    \exsol@@frontmatter
  }
\fi
\ifdefined\frontmatter
  \let\exsol@@mainmatter\mainmatter
  \def\mainmatter{%
    \immediate\write\solutionstream{\string\mainmatter}%
    \exsol@@mainmatter
  }
\fi
\ifdefined\backmatter
  \let\exsol@@backmatter\backmatter
  \def\backmatter{%
    \immediate\write\solutionstream{\string\backmatter}%
    \exsol@@backmatter
  }
\fi
%    \end{macrocode}
%
% \begin{macro}{\noexercisesinchapter}
%   If you have chapters without exercises, you may want to indicate
%   this clearly into your source. Otherwise empty chapters may appear
%   in your solution book.
%    \begin{macrocode}
\newcommand{\noexercisesinchapter}
{
  \immediate\write\solutionstream{No exercises in this chapter}
}
%    \end{macrocode}
% \end{macro}
%
%    \begin{macrocode}
%</package>
%    \end{macrocode}
%
% \bibliographystyle{alpha}
%
% \begin{thebibliography}{99}
%
% \bibitem{fancyvrb}
% Timothy Van Zandt, Herbert Vo\ss, Denis Girou, Sebastian Rahtz, Niall
% Mansfield 
% \newblock The \texttt{fancyvrb} package.
% \newblock \url{http://ctan.org/pkg/fancyvrb}.
% \newblock online, accessed in January 2012.
%
% \bibitem{CTAN} 
% The Comprehensive TeX Archive Network.
% \newblock \url{http://www.ctan.org}.
% \newblock online, accessed in January 2012.
%
% \end{thebibliography}
%
% \Finale
\endinput

%
% When running \LaTeX{} on your document (in our case on the
% \texttt{exsol.dtx} file, as a side effect a file with extension
% \texttt{.sol.tex} has been written to disk (in our case, the file
% \texttt{exsol.sol.tex}), containing all solutions in sequence.
%
% Generating a solution book is a simple as including the file into a
% simple \LaTeX{} harness, that allows you giving it a proper title page and to
% add other bells and whistles.
%
% E.g.,
% \begin{VerbatimOut}{exsol-solutionbook.tex}
% \documentclass{article}
% \usepackage[english]{babel}
% \title{Solutions to the exercises, specified in the \textsf{ExSol} package}
% \author{Walter Daems}
% \date{2013/05/12}
%
% \begin{document}
%
% \maketitle
%
% % \iffalse meta-comment
%
% Copyright (C) 2014 by Walter Daems <walter.daems@uantwerpen.be>
%
% This work may be distributed and/or modified under the conditions of
% the LaTeX Project Public License, either version 1.3 of this license
% or (at your option) any later version.  The latest version of this
% license is in:
% 
%    http://www.latex-project.org/lppl.txt
% 
% and version 1.3 or later is part of all distributions of LaTeX version
% 2005/12/01 or later.
%
% This work has the LPPL maintenance status `maintained'.
% 
% The Current Maintainer of this work is Walter Daems.
%
% This work consists of the files exsol.dtx and exsol.ins and the derived 
% files:
%   - exsol.sty
%   - example.tex
%   - example-solutionbook.tex
%
% \fi
%
% \iffalse
%
%<package|driver>\NeedsTeXFormat{LaTeX2e}
%<driver>\ProvidesFile{exsol.dtx}
%<package>\ProvidesPackage{exsol}
%<package|driver>  [2014/08/31 v0.91 ExSol - Exercises and Solutions package (DMW)]
%<*driver> 
\documentclass[11pt]{ltxdoc}
\usepackage[english]{babel}
\usepackage[exercisesfontsize=small]{exsol}
\usepackage{metalogo}
\EnableCrossrefs
\CodelineIndex
\RecordChanges
\usepackage{makeidx}
\usepackage{alltt}
\IfFileExists{tocbibind.sty}{\usepackage{tocbibind}}{}
\IfFileExists{hyperref.sty}{\usepackage[bookmarksopen]{hyperref}}{}
\EnableCrossrefs         
\CodelineIndex
\RecordChanges
\newcommand{\exsol}{\textsf{ExSol}}
\StopEventually{\PrintChanges\PrintIndex}
\def\fileversion{0.91}
\def\filedate{2014/08/31}
\begin{document}
 \DocInput{exsol.dtx}
\end{document}
%</driver>
% \fi
%
% \CheckSum{0}
%
% \CharacterTable
%  {Upper-case    \A\B\C\D\E\F\G\H\I\J\K\L\M\N\O\P\Q\R\S\T\U\V\W\X\Y\Z
%   Lower-case    \a\b\c\d\e\f\g\h\i\j\k\l\m\n\o\p\q\r\s\t\u\v\w\x\y\z
%   Digits        \0\1\2\3\4\5\6\7\8\9
%   Exclamation   \!     Double quote  \"     Hash (number) \#
%   Dollar        \$     Percent       \%     Ampersand     \&
%   Acute accent  \'     Left paren    \(     Right paren   \)
%   Asterisk      \*     Plus          \+     Comma         \,
%   Minus         \-     Point         \.     Solidus       \/
%   Colon         \:     Semicolon     \;     Less than     \<
%   Equals        \=     Greater than  \>     Question mark \?
%   Commercial at \@     Left bracket  \[     Backslash     \\
%   Right bracket \]     Circumflex    \^     Underscore    \_
%   Grave accent  \`     Left brace    \{     Vertical bar  \|
%   Right brace   \}     Tilde         \~}
%
%
% \changes{v0.1}{2012/01/05}{. Initial version}
% \changes{v0.2}{2012/01/06}{. Minor bug fixes based on first use by
% Paul Levrie}
% \changes{v0.3}{2012/01/07}{. Minor bug fixes based on second use by
% Paul}
% \changes{v0.4}{2012/01/09}{. Allowed for non-list formatting of
% exercises (as default)}
% \changes{v0.5}{2012/01/15}{. Added option to also send exercises to
% solutions file}
% \changes{v0.6}{2013/05/12}{. Prepared for CTAN publication}
% \changes{v0.7}{2014/07/14}{. Fixed UTF8 compatibility issues}
% \changes{v0.8}{2014/07/15}{. Fixed missing babel tag and running out
% of write hanles}
% \changes{v0.9}{2014/07/28}{. Changed default behavior
% w.r.t. minipage-wraping of exercises} 
% \changes{v0.91}{2014/08/31}{. Corrected minipage dependence, made }
%
% \DoNotIndex{\newcommand,\newenvironment}
% \setlength{\parindent}{0em}
% \addtolength{\parskip}{0.5\baselineskip}
%
% \title{The \exsol{} package\thanks{This document
%   corresponds to exsol~\fileversion, dated \filedate.}}
% \author{Walter Daems (\texttt{walter.daems@uantwerpen.be})}
% \date{}
%
% \maketitle
%
% \section{Introduction}
% %%%%%%%%%%%%%%%%%%%%%%
% The package \exsol{} provides macros to allow
% embedding exercises and solutions in the \LaTeX{} source of an
% instructional text (e.g., a book or a course text) while generating
% the following separate documents:
% \begin{itemize}
% \item your original text that only contains the exercises, and
% \item a solution book that only contains the solutions to the
% exercises (a package option exists to also copy the exercises themselves to the solution book).
% \end{itemize}
% 
% The former is generated when running \LaTeX{} on your document. This
% run writes the solutions to a secondary file that can be included
% into a simple document harness, such that when running \LaTeX{} on
% the latter, you can generate a nice solution book.
% 
% Why use \exsol{}?
% \begin{itemize}
% \item It allows to keep the \LaTeX{} source of your exercises and their
% solutions in a single file. Away with the nightmare to keep your
% solutions in sync with the original text.
% \item It separates exercises and solutions, allowing you
%   \begin{itemize}
%   \item to only release the solution book to the instructors of the
%   course;
%   \item to encourage students that you provide with the solution
%   book to first try solving the exercises without opening the book;
%   this seems to be easier than not peeking into the solution of an
%   exercise that is typeset just below the exercise itself.
%   \end{itemize}
% \end{itemize}
%
% The code of the \exsol{} package was taken almost literally
% from \textsf{fancyvrb} \cite{fancyvrb}. Therefore, all credits go to the
% authors/maintainers of \textsf{fancyvrb}.
%
% Thanks to Pieter Pareit and Pekka Pere for signaling problems and
% making suggestions for the documentation.
%
% \section{Installation}
% %%%%%%%%%%%%%%%%%%%%%%
% Either you are a package manager and then you'll know how to
% prepare an installation package for \exsol{}.
%
% Either you are a normal user and then you have two options. First,
% check if there is a package that your favorite \LaTeX{}
% distributor has prepared for you. Second, grab the TDS package
% from CTAN \cite{CTAN} (\texttt{exsol.tds.zip}) and unzip it somewhere in your
% own TDS tree, regenerate your filename database and off you go.
% In any case, make sure that \LaTeX{} finds the \texttt{exsol.sty} file.
%
% The \exsol{} package uses some auxiliary packages: \textsf{fancyvrb},
% \textsf{ifthen}, \textsf{kvoptions} and, optionally,
% \textsf{babel}. Fetch them from CTAN \cite{CTAN} if your \TeX{}
% distributor does not provide them.
%
% \section{Usage}
% %%%%%%%%%%%%%%%
% 
% \subsection{Preparing your document source}
% %%%%%%%%%%%%%%%%%%%%%%%%%%%%%%%%%%%%%%%%%%%
% The macro package exsol can be loaded with:
% \begin{verbatim}
% \usepackage{exsol}
% \end{verbatim}
%
% Then, you are ready to add some exercises including their solution
% to your document source. To this end, embed them in a
% \texttt{exercise} and a corresponding \texttt{solution} environment.
% Optionally, you may embed several of them in a \texttt{exercises}
% environment, to make them stand out in your text.
% E.g.,
%
% \begin{VerbatimOut}{exsol.tmp}
% 
% \begin{exercises}
%
%   \begin{exercise}
%     Solve the following equation for $x \in C$, with $C$ the set of
%     complex numbers:
%     \begin{equation}
%       5 x^2 -3 x = 5
%     \end{equation}
%   \end{exercise}
%   \begin{solution}
%     Let's start by rearranging the equation, a bit:
%     \begin{eqnarray}
%       5.7 x^2 - 3.1 x &=& 5.3\\
%       5.7 x^2 - 3.1 x -5.3 &=& 0
%     \end{eqnarray}
%     The equation is now in the standard form:
%     \begin{equation}
%       a x^2 + b x + c = 0
%     \end{equation}
%     For quadratic equations in the standard form, we know that two
%     solutions exist:
%     \begin{equation}
%       x_{1,2} = \frac{ -b \pm \sqrt{d}}{2a}
%     \end{equation}
%     with
%     \begin{equation}
%       d = b^2 - 4 a c
%     \end{equation}
%     If we apply this to our case, we obtain:
%     \begin{equation}
%       d = (-3.1)^2 - 4 \cdot 5.7 \cdot (-5.3) = 130.45
%     \end{equation}
%     and
%     \begin{eqnarray}
%       x_1 &=& \frac{3.1 + \sqrt{130.45}}{11.4} = 1.27\\
%       x_2 &=& \frac{3.1 - \sqrt{130.45}}{11.4} = -0.73
%     \end{eqnarray}
%     The proposed values $x = x_1, x_2$ are solutions to the given equation.
%   \end{solution}
%   \begin{exercise}
%     Consider a 2-dimensional vector space equipped with a Euclidean
%     distance function. Given a right-angled triangle, with the sides
%     $A$ and $B$ adjacent to the right angle having lengths, $3$ and
%     $4$, calculate the length of the hypotenuse, labeled $C$.
%   \end{exercise}
%   \begin{solution}
%     This calls for application of Pythagoras' theorem, which 
%     tells us:
%     \begin{equation}
%       \left\|A\right\|^2 + \left\|B\right\|^2 = \left\|C\right\|^2
%     \end{equation}
%     and therefore:
%     \begin{eqnarray}
%       \left\|C\right\| 
%       &=& \sqrt{\left\|A\right\|^2 + \left\|B\right\|^2}\\
%       &=& \sqrt{3^2 + 4^2}\\
%       &=& \sqrt{25} = 5
%     \end{eqnarray}
%     Therefore, the length of the hypotenuse equals $5$.
%   \end{solution}
%
% \end{exercises}
% \end{VerbatimOut}
% \VerbatimInput[frame=lines,gobble=2,fontsize=\footnotesize]{exsol.tmp}
%
% The result in the original document, can be seen below. As you can
% see, there's no trace of the solution. 
%
% % \iffalse meta-comment
%
% Copyright (C) 2014 by Walter Daems <walter.daems@uantwerpen.be>
%
% This work may be distributed and/or modified under the conditions of
% the LaTeX Project Public License, either version 1.3 of this license
% or (at your option) any later version.  The latest version of this
% license is in:
% 
%    http://www.latex-project.org/lppl.txt
% 
% and version 1.3 or later is part of all distributions of LaTeX version
% 2005/12/01 or later.
%
% This work has the LPPL maintenance status `maintained'.
% 
% The Current Maintainer of this work is Walter Daems.
%
% This work consists of the files exsol.dtx and exsol.ins and the derived 
% files:
%   - exsol.sty
%   - example.tex
%   - example-solutionbook.tex
%
% \fi
%
% \iffalse
%
%<package|driver>\NeedsTeXFormat{LaTeX2e}
%<driver>\ProvidesFile{exsol.dtx}
%<package>\ProvidesPackage{exsol}
%<package|driver>  [2014/08/31 v0.91 ExSol - Exercises and Solutions package (DMW)]
%<*driver> 
\documentclass[11pt]{ltxdoc}
\usepackage[english]{babel}
\usepackage[exercisesfontsize=small]{exsol}
\usepackage{metalogo}
\EnableCrossrefs
\CodelineIndex
\RecordChanges
\usepackage{makeidx}
\usepackage{alltt}
\IfFileExists{tocbibind.sty}{\usepackage{tocbibind}}{}
\IfFileExists{hyperref.sty}{\usepackage[bookmarksopen]{hyperref}}{}
\EnableCrossrefs         
\CodelineIndex
\RecordChanges
\newcommand{\exsol}{\textsf{ExSol}}
\StopEventually{\PrintChanges\PrintIndex}
\def\fileversion{0.91}
\def\filedate{2014/08/31}
\begin{document}
 \DocInput{exsol.dtx}
\end{document}
%</driver>
% \fi
%
% \CheckSum{0}
%
% \CharacterTable
%  {Upper-case    \A\B\C\D\E\F\G\H\I\J\K\L\M\N\O\P\Q\R\S\T\U\V\W\X\Y\Z
%   Lower-case    \a\b\c\d\e\f\g\h\i\j\k\l\m\n\o\p\q\r\s\t\u\v\w\x\y\z
%   Digits        \0\1\2\3\4\5\6\7\8\9
%   Exclamation   \!     Double quote  \"     Hash (number) \#
%   Dollar        \$     Percent       \%     Ampersand     \&
%   Acute accent  \'     Left paren    \(     Right paren   \)
%   Asterisk      \*     Plus          \+     Comma         \,
%   Minus         \-     Point         \.     Solidus       \/
%   Colon         \:     Semicolon     \;     Less than     \<
%   Equals        \=     Greater than  \>     Question mark \?
%   Commercial at \@     Left bracket  \[     Backslash     \\
%   Right bracket \]     Circumflex    \^     Underscore    \_
%   Grave accent  \`     Left brace    \{     Vertical bar  \|
%   Right brace   \}     Tilde         \~}
%
%
% \changes{v0.1}{2012/01/05}{. Initial version}
% \changes{v0.2}{2012/01/06}{. Minor bug fixes based on first use by
% Paul Levrie}
% \changes{v0.3}{2012/01/07}{. Minor bug fixes based on second use by
% Paul}
% \changes{v0.4}{2012/01/09}{. Allowed for non-list formatting of
% exercises (as default)}
% \changes{v0.5}{2012/01/15}{. Added option to also send exercises to
% solutions file}
% \changes{v0.6}{2013/05/12}{. Prepared for CTAN publication}
% \changes{v0.7}{2014/07/14}{. Fixed UTF8 compatibility issues}
% \changes{v0.8}{2014/07/15}{. Fixed missing babel tag and running out
% of write hanles}
% \changes{v0.9}{2014/07/28}{. Changed default behavior
% w.r.t. minipage-wraping of exercises} 
% \changes{v0.91}{2014/08/31}{. Corrected minipage dependence, made }
%
% \DoNotIndex{\newcommand,\newenvironment}
% \setlength{\parindent}{0em}
% \addtolength{\parskip}{0.5\baselineskip}
%
% \title{The \exsol{} package\thanks{This document
%   corresponds to exsol~\fileversion, dated \filedate.}}
% \author{Walter Daems (\texttt{walter.daems@uantwerpen.be})}
% \date{}
%
% \maketitle
%
% \section{Introduction}
% %%%%%%%%%%%%%%%%%%%%%%
% The package \exsol{} provides macros to allow
% embedding exercises and solutions in the \LaTeX{} source of an
% instructional text (e.g., a book or a course text) while generating
% the following separate documents:
% \begin{itemize}
% \item your original text that only contains the exercises, and
% \item a solution book that only contains the solutions to the
% exercises (a package option exists to also copy the exercises themselves to the solution book).
% \end{itemize}
% 
% The former is generated when running \LaTeX{} on your document. This
% run writes the solutions to a secondary file that can be included
% into a simple document harness, such that when running \LaTeX{} on
% the latter, you can generate a nice solution book.
% 
% Why use \exsol{}?
% \begin{itemize}
% \item It allows to keep the \LaTeX{} source of your exercises and their
% solutions in a single file. Away with the nightmare to keep your
% solutions in sync with the original text.
% \item It separates exercises and solutions, allowing you
%   \begin{itemize}
%   \item to only release the solution book to the instructors of the
%   course;
%   \item to encourage students that you provide with the solution
%   book to first try solving the exercises without opening the book;
%   this seems to be easier than not peeking into the solution of an
%   exercise that is typeset just below the exercise itself.
%   \end{itemize}
% \end{itemize}
%
% The code of the \exsol{} package was taken almost literally
% from \textsf{fancyvrb} \cite{fancyvrb}. Therefore, all credits go to the
% authors/maintainers of \textsf{fancyvrb}.
%
% Thanks to Pieter Pareit and Pekka Pere for signaling problems and
% making suggestions for the documentation.
%
% \section{Installation}
% %%%%%%%%%%%%%%%%%%%%%%
% Either you are a package manager and then you'll know how to
% prepare an installation package for \exsol{}.
%
% Either you are a normal user and then you have two options. First,
% check if there is a package that your favorite \LaTeX{}
% distributor has prepared for you. Second, grab the TDS package
% from CTAN \cite{CTAN} (\texttt{exsol.tds.zip}) and unzip it somewhere in your
% own TDS tree, regenerate your filename database and off you go.
% In any case, make sure that \LaTeX{} finds the \texttt{exsol.sty} file.
%
% The \exsol{} package uses some auxiliary packages: \textsf{fancyvrb},
% \textsf{ifthen}, \textsf{kvoptions} and, optionally,
% \textsf{babel}. Fetch them from CTAN \cite{CTAN} if your \TeX{}
% distributor does not provide them.
%
% \section{Usage}
% %%%%%%%%%%%%%%%
% 
% \subsection{Preparing your document source}
% %%%%%%%%%%%%%%%%%%%%%%%%%%%%%%%%%%%%%%%%%%%
% The macro package exsol can be loaded with:
% \begin{verbatim}
% \usepackage{exsol}
% \end{verbatim}
%
% Then, you are ready to add some exercises including their solution
% to your document source. To this end, embed them in a
% \texttt{exercise} and a corresponding \texttt{solution} environment.
% Optionally, you may embed several of them in a \texttt{exercises}
% environment, to make them stand out in your text.
% E.g.,
%
% \begin{VerbatimOut}{exsol.tmp}
% 
% \begin{exercises}
%
%   \begin{exercise}
%     Solve the following equation for $x \in C$, with $C$ the set of
%     complex numbers:
%     \begin{equation}
%       5 x^2 -3 x = 5
%     \end{equation}
%   \end{exercise}
%   \begin{solution}
%     Let's start by rearranging the equation, a bit:
%     \begin{eqnarray}
%       5.7 x^2 - 3.1 x &=& 5.3\\
%       5.7 x^2 - 3.1 x -5.3 &=& 0
%     \end{eqnarray}
%     The equation is now in the standard form:
%     \begin{equation}
%       a x^2 + b x + c = 0
%     \end{equation}
%     For quadratic equations in the standard form, we know that two
%     solutions exist:
%     \begin{equation}
%       x_{1,2} = \frac{ -b \pm \sqrt{d}}{2a}
%     \end{equation}
%     with
%     \begin{equation}
%       d = b^2 - 4 a c
%     \end{equation}
%     If we apply this to our case, we obtain:
%     \begin{equation}
%       d = (-3.1)^2 - 4 \cdot 5.7 \cdot (-5.3) = 130.45
%     \end{equation}
%     and
%     \begin{eqnarray}
%       x_1 &=& \frac{3.1 + \sqrt{130.45}}{11.4} = 1.27\\
%       x_2 &=& \frac{3.1 - \sqrt{130.45}}{11.4} = -0.73
%     \end{eqnarray}
%     The proposed values $x = x_1, x_2$ are solutions to the given equation.
%   \end{solution}
%   \begin{exercise}
%     Consider a 2-dimensional vector space equipped with a Euclidean
%     distance function. Given a right-angled triangle, with the sides
%     $A$ and $B$ adjacent to the right angle having lengths, $3$ and
%     $4$, calculate the length of the hypotenuse, labeled $C$.
%   \end{exercise}
%   \begin{solution}
%     This calls for application of Pythagoras' theorem, which 
%     tells us:
%     \begin{equation}
%       \left\|A\right\|^2 + \left\|B\right\|^2 = \left\|C\right\|^2
%     \end{equation}
%     and therefore:
%     \begin{eqnarray}
%       \left\|C\right\| 
%       &=& \sqrt{\left\|A\right\|^2 + \left\|B\right\|^2}\\
%       &=& \sqrt{3^2 + 4^2}\\
%       &=& \sqrt{25} = 5
%     \end{eqnarray}
%     Therefore, the length of the hypotenuse equals $5$.
%   \end{solution}
%
% \end{exercises}
% \end{VerbatimOut}
% \VerbatimInput[frame=lines,gobble=2,fontsize=\footnotesize]{exsol.tmp}
%
% The result in the original document, can be seen below. As you can
% see, there's no trace of the solution. 
%
% % \iffalse meta-comment
%
% Copyright (C) 2014 by Walter Daems <walter.daems@uantwerpen.be>
%
% This work may be distributed and/or modified under the conditions of
% the LaTeX Project Public License, either version 1.3 of this license
% or (at your option) any later version.  The latest version of this
% license is in:
% 
%    http://www.latex-project.org/lppl.txt
% 
% and version 1.3 or later is part of all distributions of LaTeX version
% 2005/12/01 or later.
%
% This work has the LPPL maintenance status `maintained'.
% 
% The Current Maintainer of this work is Walter Daems.
%
% This work consists of the files exsol.dtx and exsol.ins and the derived 
% files:
%   - exsol.sty
%   - example.tex
%   - example-solutionbook.tex
%
% \fi
%
% \iffalse
%
%<package|driver>\NeedsTeXFormat{LaTeX2e}
%<driver>\ProvidesFile{exsol.dtx}
%<package>\ProvidesPackage{exsol}
%<package|driver>  [2014/08/31 v0.91 ExSol - Exercises and Solutions package (DMW)]
%<*driver> 
\documentclass[11pt]{ltxdoc}
\usepackage[english]{babel}
\usepackage[exercisesfontsize=small]{exsol}
\usepackage{metalogo}
\EnableCrossrefs
\CodelineIndex
\RecordChanges
\usepackage{makeidx}
\usepackage{alltt}
\IfFileExists{tocbibind.sty}{\usepackage{tocbibind}}{}
\IfFileExists{hyperref.sty}{\usepackage[bookmarksopen]{hyperref}}{}
\EnableCrossrefs         
\CodelineIndex
\RecordChanges
\newcommand{\exsol}{\textsf{ExSol}}
\StopEventually{\PrintChanges\PrintIndex}
\def\fileversion{0.91}
\def\filedate{2014/08/31}
\begin{document}
 \DocInput{exsol.dtx}
\end{document}
%</driver>
% \fi
%
% \CheckSum{0}
%
% \CharacterTable
%  {Upper-case    \A\B\C\D\E\F\G\H\I\J\K\L\M\N\O\P\Q\R\S\T\U\V\W\X\Y\Z
%   Lower-case    \a\b\c\d\e\f\g\h\i\j\k\l\m\n\o\p\q\r\s\t\u\v\w\x\y\z
%   Digits        \0\1\2\3\4\5\6\7\8\9
%   Exclamation   \!     Double quote  \"     Hash (number) \#
%   Dollar        \$     Percent       \%     Ampersand     \&
%   Acute accent  \'     Left paren    \(     Right paren   \)
%   Asterisk      \*     Plus          \+     Comma         \,
%   Minus         \-     Point         \.     Solidus       \/
%   Colon         \:     Semicolon     \;     Less than     \<
%   Equals        \=     Greater than  \>     Question mark \?
%   Commercial at \@     Left bracket  \[     Backslash     \\
%   Right bracket \]     Circumflex    \^     Underscore    \_
%   Grave accent  \`     Left brace    \{     Vertical bar  \|
%   Right brace   \}     Tilde         \~}
%
%
% \changes{v0.1}{2012/01/05}{. Initial version}
% \changes{v0.2}{2012/01/06}{. Minor bug fixes based on first use by
% Paul Levrie}
% \changes{v0.3}{2012/01/07}{. Minor bug fixes based on second use by
% Paul}
% \changes{v0.4}{2012/01/09}{. Allowed for non-list formatting of
% exercises (as default)}
% \changes{v0.5}{2012/01/15}{. Added option to also send exercises to
% solutions file}
% \changes{v0.6}{2013/05/12}{. Prepared for CTAN publication}
% \changes{v0.7}{2014/07/14}{. Fixed UTF8 compatibility issues}
% \changes{v0.8}{2014/07/15}{. Fixed missing babel tag and running out
% of write hanles}
% \changes{v0.9}{2014/07/28}{. Changed default behavior
% w.r.t. minipage-wraping of exercises} 
% \changes{v0.91}{2014/08/31}{. Corrected minipage dependence, made }
%
% \DoNotIndex{\newcommand,\newenvironment}
% \setlength{\parindent}{0em}
% \addtolength{\parskip}{0.5\baselineskip}
%
% \title{The \exsol{} package\thanks{This document
%   corresponds to exsol~\fileversion, dated \filedate.}}
% \author{Walter Daems (\texttt{walter.daems@uantwerpen.be})}
% \date{}
%
% \maketitle
%
% \section{Introduction}
% %%%%%%%%%%%%%%%%%%%%%%
% The package \exsol{} provides macros to allow
% embedding exercises and solutions in the \LaTeX{} source of an
% instructional text (e.g., a book or a course text) while generating
% the following separate documents:
% \begin{itemize}
% \item your original text that only contains the exercises, and
% \item a solution book that only contains the solutions to the
% exercises (a package option exists to also copy the exercises themselves to the solution book).
% \end{itemize}
% 
% The former is generated when running \LaTeX{} on your document. This
% run writes the solutions to a secondary file that can be included
% into a simple document harness, such that when running \LaTeX{} on
% the latter, you can generate a nice solution book.
% 
% Why use \exsol{}?
% \begin{itemize}
% \item It allows to keep the \LaTeX{} source of your exercises and their
% solutions in a single file. Away with the nightmare to keep your
% solutions in sync with the original text.
% \item It separates exercises and solutions, allowing you
%   \begin{itemize}
%   \item to only release the solution book to the instructors of the
%   course;
%   \item to encourage students that you provide with the solution
%   book to first try solving the exercises without opening the book;
%   this seems to be easier than not peeking into the solution of an
%   exercise that is typeset just below the exercise itself.
%   \end{itemize}
% \end{itemize}
%
% The code of the \exsol{} package was taken almost literally
% from \textsf{fancyvrb} \cite{fancyvrb}. Therefore, all credits go to the
% authors/maintainers of \textsf{fancyvrb}.
%
% Thanks to Pieter Pareit and Pekka Pere for signaling problems and
% making suggestions for the documentation.
%
% \section{Installation}
% %%%%%%%%%%%%%%%%%%%%%%
% Either you are a package manager and then you'll know how to
% prepare an installation package for \exsol{}.
%
% Either you are a normal user and then you have two options. First,
% check if there is a package that your favorite \LaTeX{}
% distributor has prepared for you. Second, grab the TDS package
% from CTAN \cite{CTAN} (\texttt{exsol.tds.zip}) and unzip it somewhere in your
% own TDS tree, regenerate your filename database and off you go.
% In any case, make sure that \LaTeX{} finds the \texttt{exsol.sty} file.
%
% The \exsol{} package uses some auxiliary packages: \textsf{fancyvrb},
% \textsf{ifthen}, \textsf{kvoptions} and, optionally,
% \textsf{babel}. Fetch them from CTAN \cite{CTAN} if your \TeX{}
% distributor does not provide them.
%
% \section{Usage}
% %%%%%%%%%%%%%%%
% 
% \subsection{Preparing your document source}
% %%%%%%%%%%%%%%%%%%%%%%%%%%%%%%%%%%%%%%%%%%%
% The macro package exsol can be loaded with:
% \begin{verbatim}
% \usepackage{exsol}
% \end{verbatim}
%
% Then, you are ready to add some exercises including their solution
% to your document source. To this end, embed them in a
% \texttt{exercise} and a corresponding \texttt{solution} environment.
% Optionally, you may embed several of them in a \texttt{exercises}
% environment, to make them stand out in your text.
% E.g.,
%
% \begin{VerbatimOut}{exsol.tmp}
% 
% \begin{exercises}
%
%   \begin{exercise}
%     Solve the following equation for $x \in C$, with $C$ the set of
%     complex numbers:
%     \begin{equation}
%       5 x^2 -3 x = 5
%     \end{equation}
%   \end{exercise}
%   \begin{solution}
%     Let's start by rearranging the equation, a bit:
%     \begin{eqnarray}
%       5.7 x^2 - 3.1 x &=& 5.3\\
%       5.7 x^2 - 3.1 x -5.3 &=& 0
%     \end{eqnarray}
%     The equation is now in the standard form:
%     \begin{equation}
%       a x^2 + b x + c = 0
%     \end{equation}
%     For quadratic equations in the standard form, we know that two
%     solutions exist:
%     \begin{equation}
%       x_{1,2} = \frac{ -b \pm \sqrt{d}}{2a}
%     \end{equation}
%     with
%     \begin{equation}
%       d = b^2 - 4 a c
%     \end{equation}
%     If we apply this to our case, we obtain:
%     \begin{equation}
%       d = (-3.1)^2 - 4 \cdot 5.7 \cdot (-5.3) = 130.45
%     \end{equation}
%     and
%     \begin{eqnarray}
%       x_1 &=& \frac{3.1 + \sqrt{130.45}}{11.4} = 1.27\\
%       x_2 &=& \frac{3.1 - \sqrt{130.45}}{11.4} = -0.73
%     \end{eqnarray}
%     The proposed values $x = x_1, x_2$ are solutions to the given equation.
%   \end{solution}
%   \begin{exercise}
%     Consider a 2-dimensional vector space equipped with a Euclidean
%     distance function. Given a right-angled triangle, with the sides
%     $A$ and $B$ adjacent to the right angle having lengths, $3$ and
%     $4$, calculate the length of the hypotenuse, labeled $C$.
%   \end{exercise}
%   \begin{solution}
%     This calls for application of Pythagoras' theorem, which 
%     tells us:
%     \begin{equation}
%       \left\|A\right\|^2 + \left\|B\right\|^2 = \left\|C\right\|^2
%     \end{equation}
%     and therefore:
%     \begin{eqnarray}
%       \left\|C\right\| 
%       &=& \sqrt{\left\|A\right\|^2 + \left\|B\right\|^2}\\
%       &=& \sqrt{3^2 + 4^2}\\
%       &=& \sqrt{25} = 5
%     \end{eqnarray}
%     Therefore, the length of the hypotenuse equals $5$.
%   \end{solution}
%
% \end{exercises}
% \end{VerbatimOut}
% \VerbatimInput[frame=lines,gobble=2,fontsize=\footnotesize]{exsol.tmp}
%
% The result in the original document, can be seen below. As you can
% see, there's no trace of the solution. 
%
% \input{exsol.tmp}
%
% When running \LaTeX{} on your document (in our case on the
% \texttt{exsol.dtx} file, as a side effect a file with extension
% \texttt{.sol.tex} has been written to disk (in our case, the file
% \texttt{exsol.sol.tex}), containing all solutions in sequence.
%
% Generating a solution book is a simple as including the file into a
% simple \LaTeX{} harness, that allows you giving it a proper title page and to
% add other bells and whistles.
%
% E.g.,
% \begin{VerbatimOut}{exsol-solutionbook.tex}
% \documentclass{article}
% \usepackage[english]{babel}
% \title{Solutions to the exercises, specified in the \textsf{ExSol} package}
% \author{Walter Daems}
% \date{2013/05/12}
%
% \begin{document}
%
% \maketitle
%
% \input{exsol.sol}
%
% \end{document}
% \end{VerbatimOut}
% \VerbatimInput[frame=lines,gobble=2,fontsize=\footnotesize]{exsol-solutionbook.tex}
% 
% You may generate this solution book, by running \LaTeX{} on the
% file named \texttt{exsol-solutionbook.tex} that is generated when running
% \LaTeX{} on the \texttt{exsol.dtx} file.
%
% The result approximately looks like this:
%
% \setcounter{equation}{0}
% \rule{\linewidth}{.7pt}
% \begin{center}
% {\Large Solutions to the exercises, specified in the \textsf{ExSol} package}\\
% {\large Walter Daems}\\
% {\large 2013/05/12}
% \end{center}
% \par---\newline\textbf{Solution 3.1-1}
%     Let's start by rearranging the equation, a bit:
%     \begin{eqnarray}
%       5.7 x^2 - 3.1 x &=% 5.3\\
%       5.7 x^2 - 3.1 x -5.3 &=% 0
%     \end{eqnarray}
%     The equation is now in the standard form:
%     \begin{equation}
%       a x^2 + b x + c = 0
%     \end{equation}
%     For quadratic equations in the standard form, we know that two
%     solutions exist:
%     \begin{equation}
%       x_{1,2} = \frac{ -b \pm \sqrt{d}}{2a}
%     \end{equation}
%     with
%     \begin{equation}
%       d = b^2 - 4 a c
%     \end{equation}
%     If we apply this to our case, we obtain:
%     \begin{equation}
%       d = (-3.1)^2 - 4 \cdot 5.7 \cdot (-5.3) = 130.45
%     \end{equation}
%     and
%     \begin{eqnarray}
%       x_1 &=& \frac{3.1 + \sqrt{130.45}}{11.4} = 1.27\\
%       x_2 &=& \frac{3.1 + \sqrt{130.45}}{11.4} = -0.73
%     \end{eqnarray}
%     The proposed values $x = x_1, x_2$ are solutions to
%     the given equation.
% \par---\newline\textbf{Solution 3.1-2}
%       This calls for application of Pythagoras' theorem, which
%       tells us:
%       \begin{equation}
%         \left\|A\right\|^2 + \left\|B\right\|^2 = \left\|C\right\|^2
%       \end{equation}
%       and therefore:
%       \begin{eqnarray}
%         \left\|C\right\|
%         &=& \sqrt{\left\|A\right\|^2 + \left\|B\right\|^2}\\
%         &=& \sqrt{3^2 + 4^2}\\
%         &=& \sqrt{25} = 5
%       \end{eqnarray}
%       Therefore, the length of the hypotenuse equals $5$.
%
% \rule{\linewidth}{.7pt}
%
% \subsection{Fiddling with the spacing}
%
% The default spacing provided by the \textsf{ExSol} package should be
% fine for most users. However, if you like to tweak, below you can
% find the controls.
% \subsubsection{Spacing before and after the \texttt{exercises} environment}
%
% The lengths below control the spacing of the |exercises| environment:
% \begin{itemize}
% \item |exsolexerciseaboveskip|: rubber length controlling the
% vertical space after the top marker line of the environment
% \item |exsolexercisebelowskip|: rubber length controlling the
% vertical space before the bottom marker line of the environment
% \end{itemize}
%
% You can simply specify them like:
% \begin{verbatim}
% \setlength{\exsolexercisesaboveskip}{1ex plus 1pt minus 1pt}
% \setlength{\exsolexercisesbelowskip}{1ex plus 1pt minus 1pt}
% \end{verbatim}
% The spacings specified here are the package defaults.
%
% \subsubsection{Spacing of the individual exercises}
% Caution: the spacing can only be tuned, when one invokes the
% |exerciseaslist| package option!
%
% Then lengths below control the spacing of the |exercise| environment:
% \begin{itemize}
% \item |exercisetopbottomsep|: rubber length controlling the vertical
% space before and after individual exercises
% \item |exerciseleftmargin|: length controlling the horizontal
% space between the surrounding environment's left margin (most
% often the page margin) and the left edge of the exercise
% environment 
% \item |exerciseleftmargin|: length controlling the horizontal
% space between the surrounding environment's right margin (most
% often the page margin) and the right edge of the exercise
% environment
% \item |exerciseitemindent|: length controlling the first-line
% indentation of the first paragraph in the exercise environment
% (actually, the label is set w.r.t. this position, that we will
% conveniently call position 'x')
% \item |exerciseparindent|: length controlling the first-line
% indentation of the other paragraphs in the exercise environment.
% \item |exerciselabelsep|: length controlling the distance between
% the label and position 'x'
% \item |exerciselabelwidth|: minimal width of the (internally
% right-alligned) box to use for the exercises label; if the box is
% not sufficiently big, position 'x' is shifted to the right
% \item |exerciseparsep|: internal paragraph separation (vertically)
% \end{itemize}
% 
% You can simply specify them like:
% \begin{verbatim}
% \setlength{\exsolexercisetopbottomsep}{0pt plus 0pt minus 1pt}
% \setlength{\exsolexerciseleftmargin}{1em}
% \setlength{\exsolexerciserightmargin}{1em}
% \setlength{\exsolexerciseparindent}{0em}
% \setlength{\exsolexerciselabelsep}{0.5em}
% \setlength{\exsolexerciselabelwidth}{0pt}
% \setlength{\exsolexerciseitemindent}{0pt}
% \setlength{\exsolexerciseparsep}{\parskip}
% \end{verbatim}
% The spacings specified here are the package defaults.
%
% \subsection{Tips and tricks}
%
% If you want to include the solutions all at the
% end of the current document, you need to explicitly close the
% solution stream before including it:
% \begin{verbatim}
%   \closeout\solutionstream\input{\jobname.sol.tex}
% \end{verbatim}
%
% If you want to avoid exercises being split by a page boundary, then
% provide the package option 'minipage'. This causes the exercises to
% be wrapped in a minipage environment.
% 
% \clearpage
%
% \section{Implementation}
% %%%%%%%%%%%%%%%%%%%%%%%
%    \begin{macrocode}
%<*package>
%    \end{macrocode}
%
% \subsection{Auxiliary packages}
% %%%%%%%%%%%%%%%%%%%%%%%%%%%%%%%
% The package uses some auxiliary packages:
%    \begin{macrocode}
\RequirePackage{fancyvrb}
\RequirePackage{ifthen}
\RequirePackage{kvoptions}
%    \end{macrocode}
%
% \subsection{Package options}
% %%%%%%%%%%%%%%%%%%%%%%%%%%%%
% The package offers some options:
%
% \changes{v0.2}{2012/01/06}{Added option exercisesfont}
% \changes{v0.4}{2012/01/09}{Changed name of option to exercisesfontsize}
%
% \begin{macro}{exercisesfontsize}
%  This option allows setting the font of the \texttt{exercises}
%  environment. You may chopse one of tiny, scriptsize, footnotesize,
%  small, normalsize, large, etc.\\
%  E.g., \texttt{[exercisesfontsize=small]}.
%    \begin{macrocode}
\DeclareStringOption[normalsize]{exercisesfontsize}
%    \end{macrocode}
% \end{macro}
%
% \changes{v0.4}{2012/01/06}{Added option exercisesinlist}
% \changes{v0.5}{2012/01/09}{Changed option exercisesinlist to exerciseaslist}
%
% \begin{macro}{exerciseaslist}
%  This boolean option (true, false) allows setting the typesetting of
%  the \texttt{exercises} in a list environment. This causes the
%  exercises to be typeset in a more compact fashion, with indented
%  left and right margin. 
%    \begin{macrocode}
\DeclareBoolOption[false]{exerciseaslist}
%    \end{macrocode}
% \end{macro}
%
% \changes{v0.5}{2012/01/09}{Added option copyexercisesinsolutions}
% \begin{macro}{copyexercisesinsolutions}
%  This boolean option (true, false) allows copying the exercises in
%  the solutions file, to allow for making a complete stand-alone
%  exercises bundle.
%    \begin{macrocode}
\DeclareBoolOption[false]{copyexercisesinsolutions}
%    \end{macrocode}
% \end{macro}
%
% \changes{v0.9}{2014/07/28}{. Changed default behavior
% w.r.t. minipage-wraping of exercises}
% \begin{macro}{minipage}
%  This boolean option (true, false) causes the exercises to be
%  wrapped in minipages. This avoids them getting split by a page
%  boundary.
%    \begin{macrocode}
\DeclareBoolOption[false]{minipage}
%    \end{macrocode}
% \end{macro}
%
% The options are processed using:
%    \begin{macrocode}
\ProcessKeyvalOptions*
%    \end{macrocode}
%
% The options are subsequently handled
%    \begin{macrocode}
\newcommand{\exercisesfontsize}{\csname \exsol@exercisesfontsize\endcsname}
%    \end{macrocode}
%
%
% \subsection{Customization of lengths}
% %%%%%%%%%%%%%%%%%%%%%%%%%%%%%%%%%%%%%%%
% The commands below allow customizing many lengths that control the
% typesetting of the exercises.
%
% \changes{v0.91}{2014/08/31}{added user-accessible lengths}
% First some lengths to control the spacing before and after |exercises|.
%    \begin{macrocode}
\newlength{\exsolexercisesaboveskip}
\setlength{\exsolexercisesaboveskip}{1ex plus 1pt minus 1pt}
\newlength{\exsolexercisesbelowskip}
\setlength{\exsolexercisesbelowskip}{1ex plus 1pt minus 1pt}
%    \end{macrocode}
%
% Then some lengths to control the spacing for a single
% exercise. These lengths only work when the |exerciseaslist| package
% option has been specified. Sensible defaults have been set.
%    \begin{macrocode}
\newlength{\exsolexercisetopbottomsep}
\setlength{\exsolexercisetopbottomsep}{0pt plus 0pt minus 1pt}
\newlength{\exsolexerciseleftmargin}
\setlength{\exsolexerciseleftmargin}{1em}
\newlength{\exsolexerciserightmargin}
\setlength{\exsolexerciserightmargin}{1em}
\newlength{\exsolexerciseparindent}
\setlength{\exsolexerciseparindent}{0em}
\newlength{\exsolexerciselabelsep}
\setlength{\exsolexerciselabelsep}{0.5em}
\newlength{\exsolexerciselabelwidth}
\setlength{\exsolexerciselabelwidth}{0pt}
\newlength{\exsolexerciseitemindent}
\setlength{\exsolexerciseitemindent}{0pt}
\newlength{\exsolexerciseparsep}
\setlength{\exsolexerciseparsep}{\parskip}
%    \end{macrocode}
% 
% 
% \subsection{Con- and destruction of the auxiliary streams}
% %%%%%%%%%%%%%%%%%%%%%%%%%%%%%%%%%%%%%%%%%%%%%%%%%%%%%%%%%%
% At the beginning of your document, we start by opening a stream to a
% file that will be used to write the solutions to. At the end of your
% document, the package closes the stream.
% \changes{v0.8}{2014/07/15}{moved newwrite of exercise stream to this
% spot to avoid consuming all handles}
%    \begin{macrocode}
\AtBeginDocument{
  \newwrite\solutionstream
  \immediate\openout\solutionstream=\jobname.sol.tex
  \newwrite\exercisestream
}
\AtEndDocument{
  \immediate\closeout\solutionstream
}
%    \end{macrocode}
%
% \subsection{Exercises counter}
% %%%%%%%%%%%%%%%%%%%%%%%%%%%%%%
% By providing an exercise counter, proper numbering of the exercises
% is provided to allow for good cross referencing of the solutions to
% the exercises.
% \changes{v0.2}{2012/01/06}{Removed dash in counter when in document
% without sectioning commands}
%    \begin{macrocode}
\newcounter{exercise}[subsection]
\setcounter{exercise}{0}
\renewcommand{\theexercise}{%
  \@ifundefined{c@chapter}{}{\if0\arabic{chapter}\else\arabic{chapter}.\fi}%
  \if0\arabic{section}\else\arabic{section}\fi%
  \if0\arabic{subsection}\else.\arabic{subsection}\fi%
  \if0\arabic{subsubsection}\else.\arabic{subsubsection}\fi%
  \if0\arabic{exercise}\else%
    \@ifundefined{c@chapter}%
                 {\if0\arabic{section}\else-\fi}%
                 {-}%
    \arabic{exercise}%
  \fi
}
%    \end{macrocode}
%
%
%
% \subsection{Detokenization in order to cope with utf8}
%
% Combining old-school \LaTeX{} (before \XeTeX{} and \LuaTeX{}) and
% UTF-8 is a pain.
% Detokenization has been suggested by Geoffrey Poore to solve issues
% with UTF-8 characters messing up the |fancyvrb| internals.
% \changes{v0.7}{2014/07/14}{Added detokenized writing}
%    \begin{macrocode}
\newcommand{\GPES@write@detok}[1]{%
  \immediate\write\exercisestream{\detokenize{#1}}}%
\newcommand{\GPSS@write@detok}[1]{%
  \immediate\write\solutionstream{\detokenize{#1}}}%
\newcommand{\GPESS@write@detok}[1]{%
  \GPES@write@detok{#1}%
  \GPSS@write@detok{#1}}%
%    \end{macrocode}
%
%
% \section{The user environments}
%
% \begin{macro}{exercise}
%   The \texttt{exercise} environment is used to typeset your
%   exercises, provide them with a nice label and allow for copying
%   the exercise to the solutions file (if the package option
%   \texttt{copyexercisesinsolution}) is set. The label can be
%   set by redefining the \cs{exercisename} macro, or by relying on
%   the \textsf{Babel} provisions. The code is almost litteraly
%   taken from the \textsf{fancyvrb} package.
%    \begin{macrocode}
\def\exercise{\FV@Environment{}{exercise}}
\def\FVB@exercise{%
  \refstepcounter{exercise}%
  \immediate\openout\exercisestream=\jobname.exc.tex
  \ifexsol@copyexercisesinsolutions
    \typeout{Writing exercise to \jobname.sol.tex}
    \immediate\write\solutionstream{\string\par---\string\newline
      \string\textbf\string{\exercisename{} \theexercise \string}}
  \else
    \immediate\write\solutionstream{\string\par---\string\newline}
  \fi
  \immediate\write\exercisestream{\string\begin{exsol@exercise}}
  \@bsphack
  \begingroup
    \FV@UseKeyValues
    \FV@DefineWhiteSpace
    \def\FV@Space{\space}%
    \FV@DefineTabOut
    \ifexsol@copyexercisesinsolutions
      \let\FV@ProcessLine\GPESS@write@detok %
    \else
      \let\FV@ProcessLine\GPES@write@detok %
    \fi
    \relax
    \let\FV@FontScanPrep\relax
    \let\@noligs\relax
    \FV@Scan
  }
\def\FVE@exercise{
  \endgroup\@esphack
  \immediate\write\exercisestream{\string\end{exsol@exercise}}
  \ifexsol@copyexercisesinsolutions
    \immediate\write\solutionstream{\string~\string\newline}
  \fi
  \immediate\closeout\exercisestream
  \input{\jobname.exc.tex}
}
\DefineVerbatimEnvironment{exercise}{exercise}{}
%    \end{macrocode}
% \end{macro}
%
% \begin{macro}{exsol@exercise}
%   The \texttt{exsol@exercise} environment is an internal macro used
%   to typeset your exercises and provide them with a nice label and
%   number. Do not use it directly. Use the proper environment
%   \texttt{exercise} instead.
%   \changes{v0.2}{2012/01/06}{Attempted to fix MiKTeX formatting problems}
%   \changes{v0.3}{2012/01/08}{Fixed labelsep to avoid cluttered
%   itemize environments}
%   \changes{v0.4}{2012/01/06}{Added option exercisesinlist such that
%   default results in non list formatting of exercise}
%   \changes{v0.5}{2012/01/09}{Changed implementation to allow for
%   copying the exercises to the solutions file.}
%    \begin{macrocode}
\newenvironment{exsol@exercise}[0]
{%
  \ifthenelse{\boolean{exsol@minipage}}{\begin{minipage}[t]{\textwidth}}{}%
    \ifthenelse{\boolean{exsol@exerciseaslist}}
               {\begin{list}%
                   {%
                   }%
                   {%
                     \setlength{\topsep}{\exsolexercisetopbottomsep}%
                     \setlength{\leftmargin}{\exsolexerciseleftmargin}%
                     \setlength{\rightmargin}{\exsolexerciserightmargin}%
                     \setlength{\listparindent}{\exsolexerciseparindent}%
                     \setlength{\itemindent}{\exsolexerciseitemindent}%
                     \setlength{\parsep}{\exsolexerciseparsep}
                     \setlength{\labelsep}{\exsolexerciselabelsep}
                     \setlength{\labelwidth}{\exsolexerciselabelwidth}}
                 \item[\textit{~\exercisename{} \theexercise:~}]
               }%
               {\textit{\exercisename{} \theexercise:}}
}
{%
  \ifthenelse{\boolean{exsol@exerciseaslist}}%
             {\end{list}}{}%
  \ifthenelse{\boolean{exsol@minipage}}{\end{minipage}}{\par}%
}
%    \end{macrocode}
% \end{macro}
%
%
% \begin{macro}{solution}
%   The \texttt{solution} environment is used to typeset your solutions
%   and provide them with a nice label and number that corresponds to
%   the exercise that preceeded this solution. Theno label can be
%   set by redefining the \cs{solutionname} macro, or by relying on
%   the \textsf{Babel} provisions. The code is almost litteraly
%   taken from the \textsf{fancyvrb} package.
%    \begin{macrocode}
\def\solution{\FV@Environment{}{solution}}
\def\FVB@solution{%
  \typeout{Writing solution to \jobname.sol.tex}
  \immediate\write\solutionstream{\string\textbf\string{\solutionname{}\string}}
  \ifexsol@copyexercisesinsolutions
    \immediate\write\solutionstream{\string\newline}
  \else
    \immediate\write\solutionstream{\string\textbf\string{\theexercise\string}%
                                    \string\newline}
  \fi
  \@bsphack
  \begingroup
    \FV@UseKeyValues
    \FV@DefineWhiteSpace
    \def\FV@Space{\space}%
    \FV@DefineTabOut
    \let\FV@ProcessLine\GPSS@write@detok %
    \relax
    \let\FV@FontScanPrep\relax
    \let\@noligs\relax
    \FV@Scan
  }
\def\FVE@solution{\endgroup\@esphack}
\DefineVerbatimEnvironment{solution}{solution}{}
%    \end{macrocode}
% \end{macro}
%
% \begin{macro}{exercises}
%   The \texttt{exercises} environment helps typesetting your exercises to
%   stand out from the rest of the text. You may use it at the end of
%   a chapter, or just to group some exercises in the text.
%   \changes{v0.2}{2012/01/06}{Attempted to fix MiKTeX formatting problems}
%   \changes{v0.3}{2012/01/07}{Added some extra whitespace below exercisesname}
%    \begin{macrocode}
\newenvironment{exercises}
{\par\exercisesfontsize\rule{.25\linewidth}{0.15mm}\vspace*{\exsolexercisesaboveskip}\\*%
 \textbf{\normalsize \exercisesname}}
{\vspace*{-\baselineskip}\vspace*{\exsolexercisesbelowskip}\rule{.25\linewidth}{0.15mm}\par}
%    \end{macrocode}
% \end{macro}
%
% \subsection{Some Babel provisions}
% %%%%%%%%%%%%%%%%%%%%%%%%%%%%%%%%%%
% \changes{v0.2}{2012/01/06}{Fixed babel errors}
% \begin{macro}{\exercisename}
%   The exercise environment makes use of a label \texttt{\exercisename{}}
%   macro.
%    \begin{macrocode}
\newcommand{\exercisename}{Exercise}
%    \end{macrocode}
% \end{macro}
%
% \begin{macro}{\exercisesname}
%   The exercises environment makes use of a label \texttt{\exercisesname{}}
%   macro.
%    \begin{macrocode}
\newcommand{\exercisesname}{Exercises}
%    \end{macrocode}
% \end{macro}
% 
% \begin{macro}{\solutionname}
%   The solution environment makes use of a label \texttt{\solutionname{}}
%   macro.
%    \begin{macrocode}
\newcommand{\solutionname}{Solution}
%    \end{macrocode}
% \end{macro}
%
% \begin{macro}{\solutionname}
%   The solution environment makes use of a label \texttt{\solutionname{}}
%   macro.
% \changes{v0.8}{2014/07/15}{Added missing babel tag}
%    \begin{macrocode}
\newcommand{\solutionsname}{Solutions}
%    \end{macrocode}
% \end{macro}
% 
% 
% You may redefine these macros, but to help you out a little bit, we
% provide with some basic Babel auxiliaries. If you're a true polyglot
% and are willing to help me out by providing translations for other
% languages, I'm very willing to incorporate them into the code.
%
% \changes{v0.7}{2014/07/14}{Added Finnish language support}
%    \begin{macrocode}
\addto\captionsdutch{%
  \renewcommand{\exercisename}{Oefening}%
  \renewcommand{\exercisesname}{Oefeningen}%
  \renewcommand{\solutionname}{Oplossing}%
  \renewcommand{\solutionsname}{Oplossingen}%
}
\addto\captionsgerman{%
  \renewcommand{\exercisename}{Aufgabe}%
  \renewcommand{\exercisesname}{Aufgaben}%
  \renewcommand{\solutionname}{L\"osung}%
  \renewcommand{\solutionsname}{L\"osungen}%
}
\addto\captionsfrench{%
  \renewcommand{\exercisename}{Exercice}%
  \renewcommand{\exercisesname}{Exercices}%
  \renewcommand{\solutionname}{Solution}%
  \renewcommand{\solutionsname}{Solutions}%
}
\addto\captionsfinnish{
  \renewcommand{\exercisename}{Teht\"av\"a}%
  \renewcommand{\exercisesname}{Teht\"avi\"a}%
  \renewcommand{\solutionname}{Ratkaisu}%
  \renewcommand{\solutionsname}{Ratkaisut}%
}
%    \end{macrocode}
%
%
%
% Now the final hack overloads the basic sectioning commands to make
% sure that they are copied into your solution book.
%
%    \begin{macrocode}
\let\exsol@@makechapterhead\@makechapterhead
\def\@makechapterhead#1{%
  \immediate\write\solutionstream{\string\chapter{#1}}%
  \exsol@@makechapterhead{#1}
}
\ifdefined\frontmatter
  \let\exsol@@frontmatter\frontmatter
  \def\frontmatter{%
    \immediate\write\solutionstream{\string\frontmatter}%
    \exsol@@frontmatter
  }
\fi
\ifdefined\frontmatter
  \let\exsol@@mainmatter\mainmatter
  \def\mainmatter{%
    \immediate\write\solutionstream{\string\mainmatter}%
    \exsol@@mainmatter
  }
\fi
\ifdefined\backmatter
  \let\exsol@@backmatter\backmatter
  \def\backmatter{%
    \immediate\write\solutionstream{\string\backmatter}%
    \exsol@@backmatter
  }
\fi
%    \end{macrocode}
%
% \begin{macro}{\noexercisesinchapter}
%   If you have chapters without exercises, you may want to indicate
%   this clearly into your source. Otherwise empty chapters may appear
%   in your solution book.
%    \begin{macrocode}
\newcommand{\noexercisesinchapter}
{
  \immediate\write\solutionstream{No exercises in this chapter}
}
%    \end{macrocode}
% \end{macro}
%
%    \begin{macrocode}
%</package>
%    \end{macrocode}
%
% \bibliographystyle{alpha}
%
% \begin{thebibliography}{99}
%
% \bibitem{fancyvrb}
% Timothy Van Zandt, Herbert Vo\ss, Denis Girou, Sebastian Rahtz, Niall
% Mansfield 
% \newblock The \texttt{fancyvrb} package.
% \newblock \url{http://ctan.org/pkg/fancyvrb}.
% \newblock online, accessed in January 2012.
%
% \bibitem{CTAN} 
% The Comprehensive TeX Archive Network.
% \newblock \url{http://www.ctan.org}.
% \newblock online, accessed in January 2012.
%
% \end{thebibliography}
%
% \Finale
\endinput

%
% When running \LaTeX{} on your document (in our case on the
% \texttt{exsol.dtx} file, as a side effect a file with extension
% \texttt{.sol.tex} has been written to disk (in our case, the file
% \texttt{exsol.sol.tex}), containing all solutions in sequence.
%
% Generating a solution book is a simple as including the file into a
% simple \LaTeX{} harness, that allows you giving it a proper title page and to
% add other bells and whistles.
%
% E.g.,
% \begin{VerbatimOut}{exsol-solutionbook.tex}
% \documentclass{article}
% \usepackage[english]{babel}
% \title{Solutions to the exercises, specified in the \textsf{ExSol} package}
% \author{Walter Daems}
% \date{2013/05/12}
%
% \begin{document}
%
% \maketitle
%
% % \iffalse meta-comment
%
% Copyright (C) 2014 by Walter Daems <walter.daems@uantwerpen.be>
%
% This work may be distributed and/or modified under the conditions of
% the LaTeX Project Public License, either version 1.3 of this license
% or (at your option) any later version.  The latest version of this
% license is in:
% 
%    http://www.latex-project.org/lppl.txt
% 
% and version 1.3 or later is part of all distributions of LaTeX version
% 2005/12/01 or later.
%
% This work has the LPPL maintenance status `maintained'.
% 
% The Current Maintainer of this work is Walter Daems.
%
% This work consists of the files exsol.dtx and exsol.ins and the derived 
% files:
%   - exsol.sty
%   - example.tex
%   - example-solutionbook.tex
%
% \fi
%
% \iffalse
%
%<package|driver>\NeedsTeXFormat{LaTeX2e}
%<driver>\ProvidesFile{exsol.dtx}
%<package>\ProvidesPackage{exsol}
%<package|driver>  [2014/08/31 v0.91 ExSol - Exercises and Solutions package (DMW)]
%<*driver> 
\documentclass[11pt]{ltxdoc}
\usepackage[english]{babel}
\usepackage[exercisesfontsize=small]{exsol}
\usepackage{metalogo}
\EnableCrossrefs
\CodelineIndex
\RecordChanges
\usepackage{makeidx}
\usepackage{alltt}
\IfFileExists{tocbibind.sty}{\usepackage{tocbibind}}{}
\IfFileExists{hyperref.sty}{\usepackage[bookmarksopen]{hyperref}}{}
\EnableCrossrefs         
\CodelineIndex
\RecordChanges
\newcommand{\exsol}{\textsf{ExSol}}
\StopEventually{\PrintChanges\PrintIndex}
\def\fileversion{0.91}
\def\filedate{2014/08/31}
\begin{document}
 \DocInput{exsol.dtx}
\end{document}
%</driver>
% \fi
%
% \CheckSum{0}
%
% \CharacterTable
%  {Upper-case    \A\B\C\D\E\F\G\H\I\J\K\L\M\N\O\P\Q\R\S\T\U\V\W\X\Y\Z
%   Lower-case    \a\b\c\d\e\f\g\h\i\j\k\l\m\n\o\p\q\r\s\t\u\v\w\x\y\z
%   Digits        \0\1\2\3\4\5\6\7\8\9
%   Exclamation   \!     Double quote  \"     Hash (number) \#
%   Dollar        \$     Percent       \%     Ampersand     \&
%   Acute accent  \'     Left paren    \(     Right paren   \)
%   Asterisk      \*     Plus          \+     Comma         \,
%   Minus         \-     Point         \.     Solidus       \/
%   Colon         \:     Semicolon     \;     Less than     \<
%   Equals        \=     Greater than  \>     Question mark \?
%   Commercial at \@     Left bracket  \[     Backslash     \\
%   Right bracket \]     Circumflex    \^     Underscore    \_
%   Grave accent  \`     Left brace    \{     Vertical bar  \|
%   Right brace   \}     Tilde         \~}
%
%
% \changes{v0.1}{2012/01/05}{. Initial version}
% \changes{v0.2}{2012/01/06}{. Minor bug fixes based on first use by
% Paul Levrie}
% \changes{v0.3}{2012/01/07}{. Minor bug fixes based on second use by
% Paul}
% \changes{v0.4}{2012/01/09}{. Allowed for non-list formatting of
% exercises (as default)}
% \changes{v0.5}{2012/01/15}{. Added option to also send exercises to
% solutions file}
% \changes{v0.6}{2013/05/12}{. Prepared for CTAN publication}
% \changes{v0.7}{2014/07/14}{. Fixed UTF8 compatibility issues}
% \changes{v0.8}{2014/07/15}{. Fixed missing babel tag and running out
% of write hanles}
% \changes{v0.9}{2014/07/28}{. Changed default behavior
% w.r.t. minipage-wraping of exercises} 
% \changes{v0.91}{2014/08/31}{. Corrected minipage dependence, made }
%
% \DoNotIndex{\newcommand,\newenvironment}
% \setlength{\parindent}{0em}
% \addtolength{\parskip}{0.5\baselineskip}
%
% \title{The \exsol{} package\thanks{This document
%   corresponds to exsol~\fileversion, dated \filedate.}}
% \author{Walter Daems (\texttt{walter.daems@uantwerpen.be})}
% \date{}
%
% \maketitle
%
% \section{Introduction}
% %%%%%%%%%%%%%%%%%%%%%%
% The package \exsol{} provides macros to allow
% embedding exercises and solutions in the \LaTeX{} source of an
% instructional text (e.g., a book or a course text) while generating
% the following separate documents:
% \begin{itemize}
% \item your original text that only contains the exercises, and
% \item a solution book that only contains the solutions to the
% exercises (a package option exists to also copy the exercises themselves to the solution book).
% \end{itemize}
% 
% The former is generated when running \LaTeX{} on your document. This
% run writes the solutions to a secondary file that can be included
% into a simple document harness, such that when running \LaTeX{} on
% the latter, you can generate a nice solution book.
% 
% Why use \exsol{}?
% \begin{itemize}
% \item It allows to keep the \LaTeX{} source of your exercises and their
% solutions in a single file. Away with the nightmare to keep your
% solutions in sync with the original text.
% \item It separates exercises and solutions, allowing you
%   \begin{itemize}
%   \item to only release the solution book to the instructors of the
%   course;
%   \item to encourage students that you provide with the solution
%   book to first try solving the exercises without opening the book;
%   this seems to be easier than not peeking into the solution of an
%   exercise that is typeset just below the exercise itself.
%   \end{itemize}
% \end{itemize}
%
% The code of the \exsol{} package was taken almost literally
% from \textsf{fancyvrb} \cite{fancyvrb}. Therefore, all credits go to the
% authors/maintainers of \textsf{fancyvrb}.
%
% Thanks to Pieter Pareit and Pekka Pere for signaling problems and
% making suggestions for the documentation.
%
% \section{Installation}
% %%%%%%%%%%%%%%%%%%%%%%
% Either you are a package manager and then you'll know how to
% prepare an installation package for \exsol{}.
%
% Either you are a normal user and then you have two options. First,
% check if there is a package that your favorite \LaTeX{}
% distributor has prepared for you. Second, grab the TDS package
% from CTAN \cite{CTAN} (\texttt{exsol.tds.zip}) and unzip it somewhere in your
% own TDS tree, regenerate your filename database and off you go.
% In any case, make sure that \LaTeX{} finds the \texttt{exsol.sty} file.
%
% The \exsol{} package uses some auxiliary packages: \textsf{fancyvrb},
% \textsf{ifthen}, \textsf{kvoptions} and, optionally,
% \textsf{babel}. Fetch them from CTAN \cite{CTAN} if your \TeX{}
% distributor does not provide them.
%
% \section{Usage}
% %%%%%%%%%%%%%%%
% 
% \subsection{Preparing your document source}
% %%%%%%%%%%%%%%%%%%%%%%%%%%%%%%%%%%%%%%%%%%%
% The macro package exsol can be loaded with:
% \begin{verbatim}
% \usepackage{exsol}
% \end{verbatim}
%
% Then, you are ready to add some exercises including their solution
% to your document source. To this end, embed them in a
% \texttt{exercise} and a corresponding \texttt{solution} environment.
% Optionally, you may embed several of them in a \texttt{exercises}
% environment, to make them stand out in your text.
% E.g.,
%
% \begin{VerbatimOut}{exsol.tmp}
% 
% \begin{exercises}
%
%   \begin{exercise}
%     Solve the following equation for $x \in C$, with $C$ the set of
%     complex numbers:
%     \begin{equation}
%       5 x^2 -3 x = 5
%     \end{equation}
%   \end{exercise}
%   \begin{solution}
%     Let's start by rearranging the equation, a bit:
%     \begin{eqnarray}
%       5.7 x^2 - 3.1 x &=& 5.3\\
%       5.7 x^2 - 3.1 x -5.3 &=& 0
%     \end{eqnarray}
%     The equation is now in the standard form:
%     \begin{equation}
%       a x^2 + b x + c = 0
%     \end{equation}
%     For quadratic equations in the standard form, we know that two
%     solutions exist:
%     \begin{equation}
%       x_{1,2} = \frac{ -b \pm \sqrt{d}}{2a}
%     \end{equation}
%     with
%     \begin{equation}
%       d = b^2 - 4 a c
%     \end{equation}
%     If we apply this to our case, we obtain:
%     \begin{equation}
%       d = (-3.1)^2 - 4 \cdot 5.7 \cdot (-5.3) = 130.45
%     \end{equation}
%     and
%     \begin{eqnarray}
%       x_1 &=& \frac{3.1 + \sqrt{130.45}}{11.4} = 1.27\\
%       x_2 &=& \frac{3.1 - \sqrt{130.45}}{11.4} = -0.73
%     \end{eqnarray}
%     The proposed values $x = x_1, x_2$ are solutions to the given equation.
%   \end{solution}
%   \begin{exercise}
%     Consider a 2-dimensional vector space equipped with a Euclidean
%     distance function. Given a right-angled triangle, with the sides
%     $A$ and $B$ adjacent to the right angle having lengths, $3$ and
%     $4$, calculate the length of the hypotenuse, labeled $C$.
%   \end{exercise}
%   \begin{solution}
%     This calls for application of Pythagoras' theorem, which 
%     tells us:
%     \begin{equation}
%       \left\|A\right\|^2 + \left\|B\right\|^2 = \left\|C\right\|^2
%     \end{equation}
%     and therefore:
%     \begin{eqnarray}
%       \left\|C\right\| 
%       &=& \sqrt{\left\|A\right\|^2 + \left\|B\right\|^2}\\
%       &=& \sqrt{3^2 + 4^2}\\
%       &=& \sqrt{25} = 5
%     \end{eqnarray}
%     Therefore, the length of the hypotenuse equals $5$.
%   \end{solution}
%
% \end{exercises}
% \end{VerbatimOut}
% \VerbatimInput[frame=lines,gobble=2,fontsize=\footnotesize]{exsol.tmp}
%
% The result in the original document, can be seen below. As you can
% see, there's no trace of the solution. 
%
% \input{exsol.tmp}
%
% When running \LaTeX{} on your document (in our case on the
% \texttt{exsol.dtx} file, as a side effect a file with extension
% \texttt{.sol.tex} has been written to disk (in our case, the file
% \texttt{exsol.sol.tex}), containing all solutions in sequence.
%
% Generating a solution book is a simple as including the file into a
% simple \LaTeX{} harness, that allows you giving it a proper title page and to
% add other bells and whistles.
%
% E.g.,
% \begin{VerbatimOut}{exsol-solutionbook.tex}
% \documentclass{article}
% \usepackage[english]{babel}
% \title{Solutions to the exercises, specified in the \textsf{ExSol} package}
% \author{Walter Daems}
% \date{2013/05/12}
%
% \begin{document}
%
% \maketitle
%
% \input{exsol.sol}
%
% \end{document}
% \end{VerbatimOut}
% \VerbatimInput[frame=lines,gobble=2,fontsize=\footnotesize]{exsol-solutionbook.tex}
% 
% You may generate this solution book, by running \LaTeX{} on the
% file named \texttt{exsol-solutionbook.tex} that is generated when running
% \LaTeX{} on the \texttt{exsol.dtx} file.
%
% The result approximately looks like this:
%
% \setcounter{equation}{0}
% \rule{\linewidth}{.7pt}
% \begin{center}
% {\Large Solutions to the exercises, specified in the \textsf{ExSol} package}\\
% {\large Walter Daems}\\
% {\large 2013/05/12}
% \end{center}
% \par---\newline\textbf{Solution 3.1-1}
%     Let's start by rearranging the equation, a bit:
%     \begin{eqnarray}
%       5.7 x^2 - 3.1 x &=% 5.3\\
%       5.7 x^2 - 3.1 x -5.3 &=% 0
%     \end{eqnarray}
%     The equation is now in the standard form:
%     \begin{equation}
%       a x^2 + b x + c = 0
%     \end{equation}
%     For quadratic equations in the standard form, we know that two
%     solutions exist:
%     \begin{equation}
%       x_{1,2} = \frac{ -b \pm \sqrt{d}}{2a}
%     \end{equation}
%     with
%     \begin{equation}
%       d = b^2 - 4 a c
%     \end{equation}
%     If we apply this to our case, we obtain:
%     \begin{equation}
%       d = (-3.1)^2 - 4 \cdot 5.7 \cdot (-5.3) = 130.45
%     \end{equation}
%     and
%     \begin{eqnarray}
%       x_1 &=& \frac{3.1 + \sqrt{130.45}}{11.4} = 1.27\\
%       x_2 &=& \frac{3.1 + \sqrt{130.45}}{11.4} = -0.73
%     \end{eqnarray}
%     The proposed values $x = x_1, x_2$ are solutions to
%     the given equation.
% \par---\newline\textbf{Solution 3.1-2}
%       This calls for application of Pythagoras' theorem, which
%       tells us:
%       \begin{equation}
%         \left\|A\right\|^2 + \left\|B\right\|^2 = \left\|C\right\|^2
%       \end{equation}
%       and therefore:
%       \begin{eqnarray}
%         \left\|C\right\|
%         &=& \sqrt{\left\|A\right\|^2 + \left\|B\right\|^2}\\
%         &=& \sqrt{3^2 + 4^2}\\
%         &=& \sqrt{25} = 5
%       \end{eqnarray}
%       Therefore, the length of the hypotenuse equals $5$.
%
% \rule{\linewidth}{.7pt}
%
% \subsection{Fiddling with the spacing}
%
% The default spacing provided by the \textsf{ExSol} package should be
% fine for most users. However, if you like to tweak, below you can
% find the controls.
% \subsubsection{Spacing before and after the \texttt{exercises} environment}
%
% The lengths below control the spacing of the |exercises| environment:
% \begin{itemize}
% \item |exsolexerciseaboveskip|: rubber length controlling the
% vertical space after the top marker line of the environment
% \item |exsolexercisebelowskip|: rubber length controlling the
% vertical space before the bottom marker line of the environment
% \end{itemize}
%
% You can simply specify them like:
% \begin{verbatim}
% \setlength{\exsolexercisesaboveskip}{1ex plus 1pt minus 1pt}
% \setlength{\exsolexercisesbelowskip}{1ex plus 1pt minus 1pt}
% \end{verbatim}
% The spacings specified here are the package defaults.
%
% \subsubsection{Spacing of the individual exercises}
% Caution: the spacing can only be tuned, when one invokes the
% |exerciseaslist| package option!
%
% Then lengths below control the spacing of the |exercise| environment:
% \begin{itemize}
% \item |exercisetopbottomsep|: rubber length controlling the vertical
% space before and after individual exercises
% \item |exerciseleftmargin|: length controlling the horizontal
% space between the surrounding environment's left margin (most
% often the page margin) and the left edge of the exercise
% environment 
% \item |exerciseleftmargin|: length controlling the horizontal
% space between the surrounding environment's right margin (most
% often the page margin) and the right edge of the exercise
% environment
% \item |exerciseitemindent|: length controlling the first-line
% indentation of the first paragraph in the exercise environment
% (actually, the label is set w.r.t. this position, that we will
% conveniently call position 'x')
% \item |exerciseparindent|: length controlling the first-line
% indentation of the other paragraphs in the exercise environment.
% \item |exerciselabelsep|: length controlling the distance between
% the label and position 'x'
% \item |exerciselabelwidth|: minimal width of the (internally
% right-alligned) box to use for the exercises label; if the box is
% not sufficiently big, position 'x' is shifted to the right
% \item |exerciseparsep|: internal paragraph separation (vertically)
% \end{itemize}
% 
% You can simply specify them like:
% \begin{verbatim}
% \setlength{\exsolexercisetopbottomsep}{0pt plus 0pt minus 1pt}
% \setlength{\exsolexerciseleftmargin}{1em}
% \setlength{\exsolexerciserightmargin}{1em}
% \setlength{\exsolexerciseparindent}{0em}
% \setlength{\exsolexerciselabelsep}{0.5em}
% \setlength{\exsolexerciselabelwidth}{0pt}
% \setlength{\exsolexerciseitemindent}{0pt}
% \setlength{\exsolexerciseparsep}{\parskip}
% \end{verbatim}
% The spacings specified here are the package defaults.
%
% \subsection{Tips and tricks}
%
% If you want to include the solutions all at the
% end of the current document, you need to explicitly close the
% solution stream before including it:
% \begin{verbatim}
%   \closeout\solutionstream\input{\jobname.sol.tex}
% \end{verbatim}
%
% If you want to avoid exercises being split by a page boundary, then
% provide the package option 'minipage'. This causes the exercises to
% be wrapped in a minipage environment.
% 
% \clearpage
%
% \section{Implementation}
% %%%%%%%%%%%%%%%%%%%%%%%
%    \begin{macrocode}
%<*package>
%    \end{macrocode}
%
% \subsection{Auxiliary packages}
% %%%%%%%%%%%%%%%%%%%%%%%%%%%%%%%
% The package uses some auxiliary packages:
%    \begin{macrocode}
\RequirePackage{fancyvrb}
\RequirePackage{ifthen}
\RequirePackage{kvoptions}
%    \end{macrocode}
%
% \subsection{Package options}
% %%%%%%%%%%%%%%%%%%%%%%%%%%%%
% The package offers some options:
%
% \changes{v0.2}{2012/01/06}{Added option exercisesfont}
% \changes{v0.4}{2012/01/09}{Changed name of option to exercisesfontsize}
%
% \begin{macro}{exercisesfontsize}
%  This option allows setting the font of the \texttt{exercises}
%  environment. You may chopse one of tiny, scriptsize, footnotesize,
%  small, normalsize, large, etc.\\
%  E.g., \texttt{[exercisesfontsize=small]}.
%    \begin{macrocode}
\DeclareStringOption[normalsize]{exercisesfontsize}
%    \end{macrocode}
% \end{macro}
%
% \changes{v0.4}{2012/01/06}{Added option exercisesinlist}
% \changes{v0.5}{2012/01/09}{Changed option exercisesinlist to exerciseaslist}
%
% \begin{macro}{exerciseaslist}
%  This boolean option (true, false) allows setting the typesetting of
%  the \texttt{exercises} in a list environment. This causes the
%  exercises to be typeset in a more compact fashion, with indented
%  left and right margin. 
%    \begin{macrocode}
\DeclareBoolOption[false]{exerciseaslist}
%    \end{macrocode}
% \end{macro}
%
% \changes{v0.5}{2012/01/09}{Added option copyexercisesinsolutions}
% \begin{macro}{copyexercisesinsolutions}
%  This boolean option (true, false) allows copying the exercises in
%  the solutions file, to allow for making a complete stand-alone
%  exercises bundle.
%    \begin{macrocode}
\DeclareBoolOption[false]{copyexercisesinsolutions}
%    \end{macrocode}
% \end{macro}
%
% \changes{v0.9}{2014/07/28}{. Changed default behavior
% w.r.t. minipage-wraping of exercises}
% \begin{macro}{minipage}
%  This boolean option (true, false) causes the exercises to be
%  wrapped in minipages. This avoids them getting split by a page
%  boundary.
%    \begin{macrocode}
\DeclareBoolOption[false]{minipage}
%    \end{macrocode}
% \end{macro}
%
% The options are processed using:
%    \begin{macrocode}
\ProcessKeyvalOptions*
%    \end{macrocode}
%
% The options are subsequently handled
%    \begin{macrocode}
\newcommand{\exercisesfontsize}{\csname \exsol@exercisesfontsize\endcsname}
%    \end{macrocode}
%
%
% \subsection{Customization of lengths}
% %%%%%%%%%%%%%%%%%%%%%%%%%%%%%%%%%%%%%%%
% The commands below allow customizing many lengths that control the
% typesetting of the exercises.
%
% \changes{v0.91}{2014/08/31}{added user-accessible lengths}
% First some lengths to control the spacing before and after |exercises|.
%    \begin{macrocode}
\newlength{\exsolexercisesaboveskip}
\setlength{\exsolexercisesaboveskip}{1ex plus 1pt minus 1pt}
\newlength{\exsolexercisesbelowskip}
\setlength{\exsolexercisesbelowskip}{1ex plus 1pt minus 1pt}
%    \end{macrocode}
%
% Then some lengths to control the spacing for a single
% exercise. These lengths only work when the |exerciseaslist| package
% option has been specified. Sensible defaults have been set.
%    \begin{macrocode}
\newlength{\exsolexercisetopbottomsep}
\setlength{\exsolexercisetopbottomsep}{0pt plus 0pt minus 1pt}
\newlength{\exsolexerciseleftmargin}
\setlength{\exsolexerciseleftmargin}{1em}
\newlength{\exsolexerciserightmargin}
\setlength{\exsolexerciserightmargin}{1em}
\newlength{\exsolexerciseparindent}
\setlength{\exsolexerciseparindent}{0em}
\newlength{\exsolexerciselabelsep}
\setlength{\exsolexerciselabelsep}{0.5em}
\newlength{\exsolexerciselabelwidth}
\setlength{\exsolexerciselabelwidth}{0pt}
\newlength{\exsolexerciseitemindent}
\setlength{\exsolexerciseitemindent}{0pt}
\newlength{\exsolexerciseparsep}
\setlength{\exsolexerciseparsep}{\parskip}
%    \end{macrocode}
% 
% 
% \subsection{Con- and destruction of the auxiliary streams}
% %%%%%%%%%%%%%%%%%%%%%%%%%%%%%%%%%%%%%%%%%%%%%%%%%%%%%%%%%%
% At the beginning of your document, we start by opening a stream to a
% file that will be used to write the solutions to. At the end of your
% document, the package closes the stream.
% \changes{v0.8}{2014/07/15}{moved newwrite of exercise stream to this
% spot to avoid consuming all handles}
%    \begin{macrocode}
\AtBeginDocument{
  \newwrite\solutionstream
  \immediate\openout\solutionstream=\jobname.sol.tex
  \newwrite\exercisestream
}
\AtEndDocument{
  \immediate\closeout\solutionstream
}
%    \end{macrocode}
%
% \subsection{Exercises counter}
% %%%%%%%%%%%%%%%%%%%%%%%%%%%%%%
% By providing an exercise counter, proper numbering of the exercises
% is provided to allow for good cross referencing of the solutions to
% the exercises.
% \changes{v0.2}{2012/01/06}{Removed dash in counter when in document
% without sectioning commands}
%    \begin{macrocode}
\newcounter{exercise}[subsection]
\setcounter{exercise}{0}
\renewcommand{\theexercise}{%
  \@ifundefined{c@chapter}{}{\if0\arabic{chapter}\else\arabic{chapter}.\fi}%
  \if0\arabic{section}\else\arabic{section}\fi%
  \if0\arabic{subsection}\else.\arabic{subsection}\fi%
  \if0\arabic{subsubsection}\else.\arabic{subsubsection}\fi%
  \if0\arabic{exercise}\else%
    \@ifundefined{c@chapter}%
                 {\if0\arabic{section}\else-\fi}%
                 {-}%
    \arabic{exercise}%
  \fi
}
%    \end{macrocode}
%
%
%
% \subsection{Detokenization in order to cope with utf8}
%
% Combining old-school \LaTeX{} (before \XeTeX{} and \LuaTeX{}) and
% UTF-8 is a pain.
% Detokenization has been suggested by Geoffrey Poore to solve issues
% with UTF-8 characters messing up the |fancyvrb| internals.
% \changes{v0.7}{2014/07/14}{Added detokenized writing}
%    \begin{macrocode}
\newcommand{\GPES@write@detok}[1]{%
  \immediate\write\exercisestream{\detokenize{#1}}}%
\newcommand{\GPSS@write@detok}[1]{%
  \immediate\write\solutionstream{\detokenize{#1}}}%
\newcommand{\GPESS@write@detok}[1]{%
  \GPES@write@detok{#1}%
  \GPSS@write@detok{#1}}%
%    \end{macrocode}
%
%
% \section{The user environments}
%
% \begin{macro}{exercise}
%   The \texttt{exercise} environment is used to typeset your
%   exercises, provide them with a nice label and allow for copying
%   the exercise to the solutions file (if the package option
%   \texttt{copyexercisesinsolution}) is set. The label can be
%   set by redefining the \cs{exercisename} macro, or by relying on
%   the \textsf{Babel} provisions. The code is almost litteraly
%   taken from the \textsf{fancyvrb} package.
%    \begin{macrocode}
\def\exercise{\FV@Environment{}{exercise}}
\def\FVB@exercise{%
  \refstepcounter{exercise}%
  \immediate\openout\exercisestream=\jobname.exc.tex
  \ifexsol@copyexercisesinsolutions
    \typeout{Writing exercise to \jobname.sol.tex}
    \immediate\write\solutionstream{\string\par---\string\newline
      \string\textbf\string{\exercisename{} \theexercise \string}}
  \else
    \immediate\write\solutionstream{\string\par---\string\newline}
  \fi
  \immediate\write\exercisestream{\string\begin{exsol@exercise}}
  \@bsphack
  \begingroup
    \FV@UseKeyValues
    \FV@DefineWhiteSpace
    \def\FV@Space{\space}%
    \FV@DefineTabOut
    \ifexsol@copyexercisesinsolutions
      \let\FV@ProcessLine\GPESS@write@detok %
    \else
      \let\FV@ProcessLine\GPES@write@detok %
    \fi
    \relax
    \let\FV@FontScanPrep\relax
    \let\@noligs\relax
    \FV@Scan
  }
\def\FVE@exercise{
  \endgroup\@esphack
  \immediate\write\exercisestream{\string\end{exsol@exercise}}
  \ifexsol@copyexercisesinsolutions
    \immediate\write\solutionstream{\string~\string\newline}
  \fi
  \immediate\closeout\exercisestream
  \input{\jobname.exc.tex}
}
\DefineVerbatimEnvironment{exercise}{exercise}{}
%    \end{macrocode}
% \end{macro}
%
% \begin{macro}{exsol@exercise}
%   The \texttt{exsol@exercise} environment is an internal macro used
%   to typeset your exercises and provide them with a nice label and
%   number. Do not use it directly. Use the proper environment
%   \texttt{exercise} instead.
%   \changes{v0.2}{2012/01/06}{Attempted to fix MiKTeX formatting problems}
%   \changes{v0.3}{2012/01/08}{Fixed labelsep to avoid cluttered
%   itemize environments}
%   \changes{v0.4}{2012/01/06}{Added option exercisesinlist such that
%   default results in non list formatting of exercise}
%   \changes{v0.5}{2012/01/09}{Changed implementation to allow for
%   copying the exercises to the solutions file.}
%    \begin{macrocode}
\newenvironment{exsol@exercise}[0]
{%
  \ifthenelse{\boolean{exsol@minipage}}{\begin{minipage}[t]{\textwidth}}{}%
    \ifthenelse{\boolean{exsol@exerciseaslist}}
               {\begin{list}%
                   {%
                   }%
                   {%
                     \setlength{\topsep}{\exsolexercisetopbottomsep}%
                     \setlength{\leftmargin}{\exsolexerciseleftmargin}%
                     \setlength{\rightmargin}{\exsolexerciserightmargin}%
                     \setlength{\listparindent}{\exsolexerciseparindent}%
                     \setlength{\itemindent}{\exsolexerciseitemindent}%
                     \setlength{\parsep}{\exsolexerciseparsep}
                     \setlength{\labelsep}{\exsolexerciselabelsep}
                     \setlength{\labelwidth}{\exsolexerciselabelwidth}}
                 \item[\textit{~\exercisename{} \theexercise:~}]
               }%
               {\textit{\exercisename{} \theexercise:}}
}
{%
  \ifthenelse{\boolean{exsol@exerciseaslist}}%
             {\end{list}}{}%
  \ifthenelse{\boolean{exsol@minipage}}{\end{minipage}}{\par}%
}
%    \end{macrocode}
% \end{macro}
%
%
% \begin{macro}{solution}
%   The \texttt{solution} environment is used to typeset your solutions
%   and provide them with a nice label and number that corresponds to
%   the exercise that preceeded this solution. Theno label can be
%   set by redefining the \cs{solutionname} macro, or by relying on
%   the \textsf{Babel} provisions. The code is almost litteraly
%   taken from the \textsf{fancyvrb} package.
%    \begin{macrocode}
\def\solution{\FV@Environment{}{solution}}
\def\FVB@solution{%
  \typeout{Writing solution to \jobname.sol.tex}
  \immediate\write\solutionstream{\string\textbf\string{\solutionname{}\string}}
  \ifexsol@copyexercisesinsolutions
    \immediate\write\solutionstream{\string\newline}
  \else
    \immediate\write\solutionstream{\string\textbf\string{\theexercise\string}%
                                    \string\newline}
  \fi
  \@bsphack
  \begingroup
    \FV@UseKeyValues
    \FV@DefineWhiteSpace
    \def\FV@Space{\space}%
    \FV@DefineTabOut
    \let\FV@ProcessLine\GPSS@write@detok %
    \relax
    \let\FV@FontScanPrep\relax
    \let\@noligs\relax
    \FV@Scan
  }
\def\FVE@solution{\endgroup\@esphack}
\DefineVerbatimEnvironment{solution}{solution}{}
%    \end{macrocode}
% \end{macro}
%
% \begin{macro}{exercises}
%   The \texttt{exercises} environment helps typesetting your exercises to
%   stand out from the rest of the text. You may use it at the end of
%   a chapter, or just to group some exercises in the text.
%   \changes{v0.2}{2012/01/06}{Attempted to fix MiKTeX formatting problems}
%   \changes{v0.3}{2012/01/07}{Added some extra whitespace below exercisesname}
%    \begin{macrocode}
\newenvironment{exercises}
{\par\exercisesfontsize\rule{.25\linewidth}{0.15mm}\vspace*{\exsolexercisesaboveskip}\\*%
 \textbf{\normalsize \exercisesname}}
{\vspace*{-\baselineskip}\vspace*{\exsolexercisesbelowskip}\rule{.25\linewidth}{0.15mm}\par}
%    \end{macrocode}
% \end{macro}
%
% \subsection{Some Babel provisions}
% %%%%%%%%%%%%%%%%%%%%%%%%%%%%%%%%%%
% \changes{v0.2}{2012/01/06}{Fixed babel errors}
% \begin{macro}{\exercisename}
%   The exercise environment makes use of a label \texttt{\exercisename{}}
%   macro.
%    \begin{macrocode}
\newcommand{\exercisename}{Exercise}
%    \end{macrocode}
% \end{macro}
%
% \begin{macro}{\exercisesname}
%   The exercises environment makes use of a label \texttt{\exercisesname{}}
%   macro.
%    \begin{macrocode}
\newcommand{\exercisesname}{Exercises}
%    \end{macrocode}
% \end{macro}
% 
% \begin{macro}{\solutionname}
%   The solution environment makes use of a label \texttt{\solutionname{}}
%   macro.
%    \begin{macrocode}
\newcommand{\solutionname}{Solution}
%    \end{macrocode}
% \end{macro}
%
% \begin{macro}{\solutionname}
%   The solution environment makes use of a label \texttt{\solutionname{}}
%   macro.
% \changes{v0.8}{2014/07/15}{Added missing babel tag}
%    \begin{macrocode}
\newcommand{\solutionsname}{Solutions}
%    \end{macrocode}
% \end{macro}
% 
% 
% You may redefine these macros, but to help you out a little bit, we
% provide with some basic Babel auxiliaries. If you're a true polyglot
% and are willing to help me out by providing translations for other
% languages, I'm very willing to incorporate them into the code.
%
% \changes{v0.7}{2014/07/14}{Added Finnish language support}
%    \begin{macrocode}
\addto\captionsdutch{%
  \renewcommand{\exercisename}{Oefening}%
  \renewcommand{\exercisesname}{Oefeningen}%
  \renewcommand{\solutionname}{Oplossing}%
  \renewcommand{\solutionsname}{Oplossingen}%
}
\addto\captionsgerman{%
  \renewcommand{\exercisename}{Aufgabe}%
  \renewcommand{\exercisesname}{Aufgaben}%
  \renewcommand{\solutionname}{L\"osung}%
  \renewcommand{\solutionsname}{L\"osungen}%
}
\addto\captionsfrench{%
  \renewcommand{\exercisename}{Exercice}%
  \renewcommand{\exercisesname}{Exercices}%
  \renewcommand{\solutionname}{Solution}%
  \renewcommand{\solutionsname}{Solutions}%
}
\addto\captionsfinnish{
  \renewcommand{\exercisename}{Teht\"av\"a}%
  \renewcommand{\exercisesname}{Teht\"avi\"a}%
  \renewcommand{\solutionname}{Ratkaisu}%
  \renewcommand{\solutionsname}{Ratkaisut}%
}
%    \end{macrocode}
%
%
%
% Now the final hack overloads the basic sectioning commands to make
% sure that they are copied into your solution book.
%
%    \begin{macrocode}
\let\exsol@@makechapterhead\@makechapterhead
\def\@makechapterhead#1{%
  \immediate\write\solutionstream{\string\chapter{#1}}%
  \exsol@@makechapterhead{#1}
}
\ifdefined\frontmatter
  \let\exsol@@frontmatter\frontmatter
  \def\frontmatter{%
    \immediate\write\solutionstream{\string\frontmatter}%
    \exsol@@frontmatter
  }
\fi
\ifdefined\frontmatter
  \let\exsol@@mainmatter\mainmatter
  \def\mainmatter{%
    \immediate\write\solutionstream{\string\mainmatter}%
    \exsol@@mainmatter
  }
\fi
\ifdefined\backmatter
  \let\exsol@@backmatter\backmatter
  \def\backmatter{%
    \immediate\write\solutionstream{\string\backmatter}%
    \exsol@@backmatter
  }
\fi
%    \end{macrocode}
%
% \begin{macro}{\noexercisesinchapter}
%   If you have chapters without exercises, you may want to indicate
%   this clearly into your source. Otherwise empty chapters may appear
%   in your solution book.
%    \begin{macrocode}
\newcommand{\noexercisesinchapter}
{
  \immediate\write\solutionstream{No exercises in this chapter}
}
%    \end{macrocode}
% \end{macro}
%
%    \begin{macrocode}
%</package>
%    \end{macrocode}
%
% \bibliographystyle{alpha}
%
% \begin{thebibliography}{99}
%
% \bibitem{fancyvrb}
% Timothy Van Zandt, Herbert Vo\ss, Denis Girou, Sebastian Rahtz, Niall
% Mansfield 
% \newblock The \texttt{fancyvrb} package.
% \newblock \url{http://ctan.org/pkg/fancyvrb}.
% \newblock online, accessed in January 2012.
%
% \bibitem{CTAN} 
% The Comprehensive TeX Archive Network.
% \newblock \url{http://www.ctan.org}.
% \newblock online, accessed in January 2012.
%
% \end{thebibliography}
%
% \Finale
\endinput

%
% \end{document}
% \end{VerbatimOut}
% \VerbatimInput[frame=lines,gobble=2,fontsize=\footnotesize]{exsol-solutionbook.tex}
% 
% You may generate this solution book, by running \LaTeX{} on the
% file named \texttt{exsol-solutionbook.tex} that is generated when running
% \LaTeX{} on the \texttt{exsol.dtx} file.
%
% The result approximately looks like this:
%
% \setcounter{equation}{0}
% \rule{\linewidth}{.7pt}
% \begin{center}
% {\Large Solutions to the exercises, specified in the \textsf{ExSol} package}\\
% {\large Walter Daems}\\
% {\large 2013/05/12}
% \end{center}
% \par---\newline\textbf{Solution 3.1-1}
%     Let's start by rearranging the equation, a bit:
%     \begin{eqnarray}
%       5.7 x^2 - 3.1 x &=% 5.3\\
%       5.7 x^2 - 3.1 x -5.3 &=% 0
%     \end{eqnarray}
%     The equation is now in the standard form:
%     \begin{equation}
%       a x^2 + b x + c = 0
%     \end{equation}
%     For quadratic equations in the standard form, we know that two
%     solutions exist:
%     \begin{equation}
%       x_{1,2} = \frac{ -b \pm \sqrt{d}}{2a}
%     \end{equation}
%     with
%     \begin{equation}
%       d = b^2 - 4 a c
%     \end{equation}
%     If we apply this to our case, we obtain:
%     \begin{equation}
%       d = (-3.1)^2 - 4 \cdot 5.7 \cdot (-5.3) = 130.45
%     \end{equation}
%     and
%     \begin{eqnarray}
%       x_1 &=& \frac{3.1 + \sqrt{130.45}}{11.4} = 1.27\\
%       x_2 &=& \frac{3.1 + \sqrt{130.45}}{11.4} = -0.73
%     \end{eqnarray}
%     The proposed values $x = x_1, x_2$ are solutions to
%     the given equation.
% \par---\newline\textbf{Solution 3.1-2}
%       This calls for application of Pythagoras' theorem, which
%       tells us:
%       \begin{equation}
%         \left\|A\right\|^2 + \left\|B\right\|^2 = \left\|C\right\|^2
%       \end{equation}
%       and therefore:
%       \begin{eqnarray}
%         \left\|C\right\|
%         &=& \sqrt{\left\|A\right\|^2 + \left\|B\right\|^2}\\
%         &=& \sqrt{3^2 + 4^2}\\
%         &=& \sqrt{25} = 5
%       \end{eqnarray}
%       Therefore, the length of the hypotenuse equals $5$.
%
% \rule{\linewidth}{.7pt}
%
% \subsection{Fiddling with the spacing}
%
% The default spacing provided by the \textsf{ExSol} package should be
% fine for most users. However, if you like to tweak, below you can
% find the controls.
% \subsubsection{Spacing before and after the \texttt{exercises} environment}
%
% The lengths below control the spacing of the |exercises| environment:
% \begin{itemize}
% \item |exsolexerciseaboveskip|: rubber length controlling the
% vertical space after the top marker line of the environment
% \item |exsolexercisebelowskip|: rubber length controlling the
% vertical space before the bottom marker line of the environment
% \end{itemize}
%
% You can simply specify them like:
% \begin{verbatim}
% \setlength{\exsolexercisesaboveskip}{1ex plus 1pt minus 1pt}
% \setlength{\exsolexercisesbelowskip}{1ex plus 1pt minus 1pt}
% \end{verbatim}
% The spacings specified here are the package defaults.
%
% \subsubsection{Spacing of the individual exercises}
% Caution: the spacing can only be tuned, when one invokes the
% |exerciseaslist| package option!
%
% Then lengths below control the spacing of the |exercise| environment:
% \begin{itemize}
% \item |exercisetopbottomsep|: rubber length controlling the vertical
% space before and after individual exercises
% \item |exerciseleftmargin|: length controlling the horizontal
% space between the surrounding environment's left margin (most
% often the page margin) and the left edge of the exercise
% environment 
% \item |exerciseleftmargin|: length controlling the horizontal
% space between the surrounding environment's right margin (most
% often the page margin) and the right edge of the exercise
% environment
% \item |exerciseitemindent|: length controlling the first-line
% indentation of the first paragraph in the exercise environment
% (actually, the label is set w.r.t. this position, that we will
% conveniently call position 'x')
% \item |exerciseparindent|: length controlling the first-line
% indentation of the other paragraphs in the exercise environment.
% \item |exerciselabelsep|: length controlling the distance between
% the label and position 'x'
% \item |exerciselabelwidth|: minimal width of the (internally
% right-alligned) box to use for the exercises label; if the box is
% not sufficiently big, position 'x' is shifted to the right
% \item |exerciseparsep|: internal paragraph separation (vertically)
% \end{itemize}
% 
% You can simply specify them like:
% \begin{verbatim}
% \setlength{\exsolexercisetopbottomsep}{0pt plus 0pt minus 1pt}
% \setlength{\exsolexerciseleftmargin}{1em}
% \setlength{\exsolexerciserightmargin}{1em}
% \setlength{\exsolexerciseparindent}{0em}
% \setlength{\exsolexerciselabelsep}{0.5em}
% \setlength{\exsolexerciselabelwidth}{0pt}
% \setlength{\exsolexerciseitemindent}{0pt}
% \setlength{\exsolexerciseparsep}{\parskip}
% \end{verbatim}
% The spacings specified here are the package defaults.
%
% \subsection{Tips and tricks}
%
% If you want to include the solutions all at the
% end of the current document, you need to explicitly close the
% solution stream before including it:
% \begin{verbatim}
%   \closeout\solutionstream\input{\jobname.sol.tex}
% \end{verbatim}
%
% If you want to avoid exercises being split by a page boundary, then
% provide the package option 'minipage'. This causes the exercises to
% be wrapped in a minipage environment.
% 
% \clearpage
%
% \section{Implementation}
% %%%%%%%%%%%%%%%%%%%%%%%
%    \begin{macrocode}
%<*package>
%    \end{macrocode}
%
% \subsection{Auxiliary packages}
% %%%%%%%%%%%%%%%%%%%%%%%%%%%%%%%
% The package uses some auxiliary packages:
%    \begin{macrocode}
\RequirePackage{fancyvrb}
\RequirePackage{ifthen}
\RequirePackage{kvoptions}
%    \end{macrocode}
%
% \subsection{Package options}
% %%%%%%%%%%%%%%%%%%%%%%%%%%%%
% The package offers some options:
%
% \changes{v0.2}{2012/01/06}{Added option exercisesfont}
% \changes{v0.4}{2012/01/09}{Changed name of option to exercisesfontsize}
%
% \begin{macro}{exercisesfontsize}
%  This option allows setting the font of the \texttt{exercises}
%  environment. You may chopse one of tiny, scriptsize, footnotesize,
%  small, normalsize, large, etc.\\
%  E.g., \texttt{[exercisesfontsize=small]}.
%    \begin{macrocode}
\DeclareStringOption[normalsize]{exercisesfontsize}
%    \end{macrocode}
% \end{macro}
%
% \changes{v0.4}{2012/01/06}{Added option exercisesinlist}
% \changes{v0.5}{2012/01/09}{Changed option exercisesinlist to exerciseaslist}
%
% \begin{macro}{exerciseaslist}
%  This boolean option (true, false) allows setting the typesetting of
%  the \texttt{exercises} in a list environment. This causes the
%  exercises to be typeset in a more compact fashion, with indented
%  left and right margin. 
%    \begin{macrocode}
\DeclareBoolOption[false]{exerciseaslist}
%    \end{macrocode}
% \end{macro}
%
% \changes{v0.5}{2012/01/09}{Added option copyexercisesinsolutions}
% \begin{macro}{copyexercisesinsolutions}
%  This boolean option (true, false) allows copying the exercises in
%  the solutions file, to allow for making a complete stand-alone
%  exercises bundle.
%    \begin{macrocode}
\DeclareBoolOption[false]{copyexercisesinsolutions}
%    \end{macrocode}
% \end{macro}
%
% \changes{v0.9}{2014/07/28}{. Changed default behavior
% w.r.t. minipage-wraping of exercises}
% \begin{macro}{minipage}
%  This boolean option (true, false) causes the exercises to be
%  wrapped in minipages. This avoids them getting split by a page
%  boundary.
%    \begin{macrocode}
\DeclareBoolOption[false]{minipage}
%    \end{macrocode}
% \end{macro}
%
% The options are processed using:
%    \begin{macrocode}
\ProcessKeyvalOptions*
%    \end{macrocode}
%
% The options are subsequently handled
%    \begin{macrocode}
\newcommand{\exercisesfontsize}{\csname \exsol@exercisesfontsize\endcsname}
%    \end{macrocode}
%
%
% \subsection{Customization of lengths}
% %%%%%%%%%%%%%%%%%%%%%%%%%%%%%%%%%%%%%%%
% The commands below allow customizing many lengths that control the
% typesetting of the exercises.
%
% \changes{v0.91}{2014/08/31}{added user-accessible lengths}
% First some lengths to control the spacing before and after |exercises|.
%    \begin{macrocode}
\newlength{\exsolexercisesaboveskip}
\setlength{\exsolexercisesaboveskip}{1ex plus 1pt minus 1pt}
\newlength{\exsolexercisesbelowskip}
\setlength{\exsolexercisesbelowskip}{1ex plus 1pt minus 1pt}
%    \end{macrocode}
%
% Then some lengths to control the spacing for a single
% exercise. These lengths only work when the |exerciseaslist| package
% option has been specified. Sensible defaults have been set.
%    \begin{macrocode}
\newlength{\exsolexercisetopbottomsep}
\setlength{\exsolexercisetopbottomsep}{0pt plus 0pt minus 1pt}
\newlength{\exsolexerciseleftmargin}
\setlength{\exsolexerciseleftmargin}{1em}
\newlength{\exsolexerciserightmargin}
\setlength{\exsolexerciserightmargin}{1em}
\newlength{\exsolexerciseparindent}
\setlength{\exsolexerciseparindent}{0em}
\newlength{\exsolexerciselabelsep}
\setlength{\exsolexerciselabelsep}{0.5em}
\newlength{\exsolexerciselabelwidth}
\setlength{\exsolexerciselabelwidth}{0pt}
\newlength{\exsolexerciseitemindent}
\setlength{\exsolexerciseitemindent}{0pt}
\newlength{\exsolexerciseparsep}
\setlength{\exsolexerciseparsep}{\parskip}
%    \end{macrocode}
% 
% 
% \subsection{Con- and destruction of the auxiliary streams}
% %%%%%%%%%%%%%%%%%%%%%%%%%%%%%%%%%%%%%%%%%%%%%%%%%%%%%%%%%%
% At the beginning of your document, we start by opening a stream to a
% file that will be used to write the solutions to. At the end of your
% document, the package closes the stream.
% \changes{v0.8}{2014/07/15}{moved newwrite of exercise stream to this
% spot to avoid consuming all handles}
%    \begin{macrocode}
\AtBeginDocument{
  \newwrite\solutionstream
  \immediate\openout\solutionstream=\jobname.sol.tex
  \newwrite\exercisestream
}
\AtEndDocument{
  \immediate\closeout\solutionstream
}
%    \end{macrocode}
%
% \subsection{Exercises counter}
% %%%%%%%%%%%%%%%%%%%%%%%%%%%%%%
% By providing an exercise counter, proper numbering of the exercises
% is provided to allow for good cross referencing of the solutions to
% the exercises.
% \changes{v0.2}{2012/01/06}{Removed dash in counter when in document
% without sectioning commands}
%    \begin{macrocode}
\newcounter{exercise}[subsection]
\setcounter{exercise}{0}
\renewcommand{\theexercise}{%
  \@ifundefined{c@chapter}{}{\if0\arabic{chapter}\else\arabic{chapter}.\fi}%
  \if0\arabic{section}\else\arabic{section}\fi%
  \if0\arabic{subsection}\else.\arabic{subsection}\fi%
  \if0\arabic{subsubsection}\else.\arabic{subsubsection}\fi%
  \if0\arabic{exercise}\else%
    \@ifundefined{c@chapter}%
                 {\if0\arabic{section}\else-\fi}%
                 {-}%
    \arabic{exercise}%
  \fi
}
%    \end{macrocode}
%
%
%
% \subsection{Detokenization in order to cope with utf8}
%
% Combining old-school \LaTeX{} (before \XeTeX{} and \LuaTeX{}) and
% UTF-8 is a pain.
% Detokenization has been suggested by Geoffrey Poore to solve issues
% with UTF-8 characters messing up the |fancyvrb| internals.
% \changes{v0.7}{2014/07/14}{Added detokenized writing}
%    \begin{macrocode}
\newcommand{\GPES@write@detok}[1]{%
  \immediate\write\exercisestream{\detokenize{#1}}}%
\newcommand{\GPSS@write@detok}[1]{%
  \immediate\write\solutionstream{\detokenize{#1}}}%
\newcommand{\GPESS@write@detok}[1]{%
  \GPES@write@detok{#1}%
  \GPSS@write@detok{#1}}%
%    \end{macrocode}
%
%
% \section{The user environments}
%
% \begin{macro}{exercise}
%   The \texttt{exercise} environment is used to typeset your
%   exercises, provide them with a nice label and allow for copying
%   the exercise to the solutions file (if the package option
%   \texttt{copyexercisesinsolution}) is set. The label can be
%   set by redefining the \cs{exercisename} macro, or by relying on
%   the \textsf{Babel} provisions. The code is almost litteraly
%   taken from the \textsf{fancyvrb} package.
%    \begin{macrocode}
\def\exercise{\FV@Environment{}{exercise}}
\def\FVB@exercise{%
  \refstepcounter{exercise}%
  \immediate\openout\exercisestream=\jobname.exc.tex
  \ifexsol@copyexercisesinsolutions
    \typeout{Writing exercise to \jobname.sol.tex}
    \immediate\write\solutionstream{\string\par---\string\newline
      \string\textbf\string{\exercisename{} \theexercise \string}}
  \else
    \immediate\write\solutionstream{\string\par---\string\newline}
  \fi
  \immediate\write\exercisestream{\string\begin{exsol@exercise}}
  \@bsphack
  \begingroup
    \FV@UseKeyValues
    \FV@DefineWhiteSpace
    \def\FV@Space{\space}%
    \FV@DefineTabOut
    \ifexsol@copyexercisesinsolutions
      \let\FV@ProcessLine\GPESS@write@detok %
    \else
      \let\FV@ProcessLine\GPES@write@detok %
    \fi
    \relax
    \let\FV@FontScanPrep\relax
    \let\@noligs\relax
    \FV@Scan
  }
\def\FVE@exercise{
  \endgroup\@esphack
  \immediate\write\exercisestream{\string\end{exsol@exercise}}
  \ifexsol@copyexercisesinsolutions
    \immediate\write\solutionstream{\string~\string\newline}
  \fi
  \immediate\closeout\exercisestream
  \input{\jobname.exc.tex}
}
\DefineVerbatimEnvironment{exercise}{exercise}{}
%    \end{macrocode}
% \end{macro}
%
% \begin{macro}{exsol@exercise}
%   The \texttt{exsol@exercise} environment is an internal macro used
%   to typeset your exercises and provide them with a nice label and
%   number. Do not use it directly. Use the proper environment
%   \texttt{exercise} instead.
%   \changes{v0.2}{2012/01/06}{Attempted to fix MiKTeX formatting problems}
%   \changes{v0.3}{2012/01/08}{Fixed labelsep to avoid cluttered
%   itemize environments}
%   \changes{v0.4}{2012/01/06}{Added option exercisesinlist such that
%   default results in non list formatting of exercise}
%   \changes{v0.5}{2012/01/09}{Changed implementation to allow for
%   copying the exercises to the solutions file.}
%    \begin{macrocode}
\newenvironment{exsol@exercise}[0]
{%
  \ifthenelse{\boolean{exsol@minipage}}{\begin{minipage}[t]{\textwidth}}{}%
    \ifthenelse{\boolean{exsol@exerciseaslist}}
               {\begin{list}%
                   {%
                   }%
                   {%
                     \setlength{\topsep}{\exsolexercisetopbottomsep}%
                     \setlength{\leftmargin}{\exsolexerciseleftmargin}%
                     \setlength{\rightmargin}{\exsolexerciserightmargin}%
                     \setlength{\listparindent}{\exsolexerciseparindent}%
                     \setlength{\itemindent}{\exsolexerciseitemindent}%
                     \setlength{\parsep}{\exsolexerciseparsep}
                     \setlength{\labelsep}{\exsolexerciselabelsep}
                     \setlength{\labelwidth}{\exsolexerciselabelwidth}}
                 \item[\textit{~\exercisename{} \theexercise:~}]
               }%
               {\textit{\exercisename{} \theexercise:}}
}
{%
  \ifthenelse{\boolean{exsol@exerciseaslist}}%
             {\end{list}}{}%
  \ifthenelse{\boolean{exsol@minipage}}{\end{minipage}}{\par}%
}
%    \end{macrocode}
% \end{macro}
%
%
% \begin{macro}{solution}
%   The \texttt{solution} environment is used to typeset your solutions
%   and provide them with a nice label and number that corresponds to
%   the exercise that preceeded this solution. Theno label can be
%   set by redefining the \cs{solutionname} macro, or by relying on
%   the \textsf{Babel} provisions. The code is almost litteraly
%   taken from the \textsf{fancyvrb} package.
%    \begin{macrocode}
\def\solution{\FV@Environment{}{solution}}
\def\FVB@solution{%
  \typeout{Writing solution to \jobname.sol.tex}
  \immediate\write\solutionstream{\string\textbf\string{\solutionname{}\string}}
  \ifexsol@copyexercisesinsolutions
    \immediate\write\solutionstream{\string\newline}
  \else
    \immediate\write\solutionstream{\string\textbf\string{\theexercise\string}%
                                    \string\newline}
  \fi
  \@bsphack
  \begingroup
    \FV@UseKeyValues
    \FV@DefineWhiteSpace
    \def\FV@Space{\space}%
    \FV@DefineTabOut
    \let\FV@ProcessLine\GPSS@write@detok %
    \relax
    \let\FV@FontScanPrep\relax
    \let\@noligs\relax
    \FV@Scan
  }
\def\FVE@solution{\endgroup\@esphack}
\DefineVerbatimEnvironment{solution}{solution}{}
%    \end{macrocode}
% \end{macro}
%
% \begin{macro}{exercises}
%   The \texttt{exercises} environment helps typesetting your exercises to
%   stand out from the rest of the text. You may use it at the end of
%   a chapter, or just to group some exercises in the text.
%   \changes{v0.2}{2012/01/06}{Attempted to fix MiKTeX formatting problems}
%   \changes{v0.3}{2012/01/07}{Added some extra whitespace below exercisesname}
%    \begin{macrocode}
\newenvironment{exercises}
{\par\exercisesfontsize\rule{.25\linewidth}{0.15mm}\vspace*{\exsolexercisesaboveskip}\\*%
 \textbf{\normalsize \exercisesname}}
{\vspace*{-\baselineskip}\vspace*{\exsolexercisesbelowskip}\rule{.25\linewidth}{0.15mm}\par}
%    \end{macrocode}
% \end{macro}
%
% \subsection{Some Babel provisions}
% %%%%%%%%%%%%%%%%%%%%%%%%%%%%%%%%%%
% \changes{v0.2}{2012/01/06}{Fixed babel errors}
% \begin{macro}{\exercisename}
%   The exercise environment makes use of a label \texttt{\exercisename{}}
%   macro.
%    \begin{macrocode}
\newcommand{\exercisename}{Exercise}
%    \end{macrocode}
% \end{macro}
%
% \begin{macro}{\exercisesname}
%   The exercises environment makes use of a label \texttt{\exercisesname{}}
%   macro.
%    \begin{macrocode}
\newcommand{\exercisesname}{Exercises}
%    \end{macrocode}
% \end{macro}
% 
% \begin{macro}{\solutionname}
%   The solution environment makes use of a label \texttt{\solutionname{}}
%   macro.
%    \begin{macrocode}
\newcommand{\solutionname}{Solution}
%    \end{macrocode}
% \end{macro}
%
% \begin{macro}{\solutionname}
%   The solution environment makes use of a label \texttt{\solutionname{}}
%   macro.
% \changes{v0.8}{2014/07/15}{Added missing babel tag}
%    \begin{macrocode}
\newcommand{\solutionsname}{Solutions}
%    \end{macrocode}
% \end{macro}
% 
% 
% You may redefine these macros, but to help you out a little bit, we
% provide with some basic Babel auxiliaries. If you're a true polyglot
% and are willing to help me out by providing translations for other
% languages, I'm very willing to incorporate them into the code.
%
% \changes{v0.7}{2014/07/14}{Added Finnish language support}
%    \begin{macrocode}
\addto\captionsdutch{%
  \renewcommand{\exercisename}{Oefening}%
  \renewcommand{\exercisesname}{Oefeningen}%
  \renewcommand{\solutionname}{Oplossing}%
  \renewcommand{\solutionsname}{Oplossingen}%
}
\addto\captionsgerman{%
  \renewcommand{\exercisename}{Aufgabe}%
  \renewcommand{\exercisesname}{Aufgaben}%
  \renewcommand{\solutionname}{L\"osung}%
  \renewcommand{\solutionsname}{L\"osungen}%
}
\addto\captionsfrench{%
  \renewcommand{\exercisename}{Exercice}%
  \renewcommand{\exercisesname}{Exercices}%
  \renewcommand{\solutionname}{Solution}%
  \renewcommand{\solutionsname}{Solutions}%
}
\addto\captionsfinnish{
  \renewcommand{\exercisename}{Teht\"av\"a}%
  \renewcommand{\exercisesname}{Teht\"avi\"a}%
  \renewcommand{\solutionname}{Ratkaisu}%
  \renewcommand{\solutionsname}{Ratkaisut}%
}
%    \end{macrocode}
%
%
%
% Now the final hack overloads the basic sectioning commands to make
% sure that they are copied into your solution book.
%
%    \begin{macrocode}
\let\exsol@@makechapterhead\@makechapterhead
\def\@makechapterhead#1{%
  \immediate\write\solutionstream{\string\chapter{#1}}%
  \exsol@@makechapterhead{#1}
}
\ifdefined\frontmatter
  \let\exsol@@frontmatter\frontmatter
  \def\frontmatter{%
    \immediate\write\solutionstream{\string\frontmatter}%
    \exsol@@frontmatter
  }
\fi
\ifdefined\frontmatter
  \let\exsol@@mainmatter\mainmatter
  \def\mainmatter{%
    \immediate\write\solutionstream{\string\mainmatter}%
    \exsol@@mainmatter
  }
\fi
\ifdefined\backmatter
  \let\exsol@@backmatter\backmatter
  \def\backmatter{%
    \immediate\write\solutionstream{\string\backmatter}%
    \exsol@@backmatter
  }
\fi
%    \end{macrocode}
%
% \begin{macro}{\noexercisesinchapter}
%   If you have chapters without exercises, you may want to indicate
%   this clearly into your source. Otherwise empty chapters may appear
%   in your solution book.
%    \begin{macrocode}
\newcommand{\noexercisesinchapter}
{
  \immediate\write\solutionstream{No exercises in this chapter}
}
%    \end{macrocode}
% \end{macro}
%
%    \begin{macrocode}
%</package>
%    \end{macrocode}
%
% \bibliographystyle{alpha}
%
% \begin{thebibliography}{99}
%
% \bibitem{fancyvrb}
% Timothy Van Zandt, Herbert Vo\ss, Denis Girou, Sebastian Rahtz, Niall
% Mansfield 
% \newblock The \texttt{fancyvrb} package.
% \newblock \url{http://ctan.org/pkg/fancyvrb}.
% \newblock online, accessed in January 2012.
%
% \bibitem{CTAN} 
% The Comprehensive TeX Archive Network.
% \newblock \url{http://www.ctan.org}.
% \newblock online, accessed in January 2012.
%
% \end{thebibliography}
%
% \Finale
\endinput

%
% When running \LaTeX{} on your document (in our case on the
% \texttt{exsol.dtx} file, as a side effect a file with extension
% \texttt{.sol.tex} has been written to disk (in our case, the file
% \texttt{exsol.sol.tex}), containing all solutions in sequence.
%
% Generating a solution book is a simple as including the file into a
% simple \LaTeX{} harness, that allows you giving it a proper title page and to
% add other bells and whistles.
%
% E.g.,
% \begin{VerbatimOut}{exsol-solutionbook.tex}
% \documentclass{article}
% \usepackage[english]{babel}
% \title{Solutions to the exercises, specified in the \textsf{ExSol} package}
% \author{Walter Daems}
% \date{2013/05/12}
%
% \begin{document}
%
% \maketitle
%
% % \iffalse meta-comment
%
% Copyright (C) 2014 by Walter Daems <walter.daems@uantwerpen.be>
%
% This work may be distributed and/or modified under the conditions of
% the LaTeX Project Public License, either version 1.3 of this license
% or (at your option) any later version.  The latest version of this
% license is in:
% 
%    http://www.latex-project.org/lppl.txt
% 
% and version 1.3 or later is part of all distributions of LaTeX version
% 2005/12/01 or later.
%
% This work has the LPPL maintenance status `maintained'.
% 
% The Current Maintainer of this work is Walter Daems.
%
% This work consists of the files exsol.dtx and exsol.ins and the derived 
% files:
%   - exsol.sty
%   - example.tex
%   - example-solutionbook.tex
%
% \fi
%
% \iffalse
%
%<package|driver>\NeedsTeXFormat{LaTeX2e}
%<driver>\ProvidesFile{exsol.dtx}
%<package>\ProvidesPackage{exsol}
%<package|driver>  [2014/08/31 v0.91 ExSol - Exercises and Solutions package (DMW)]
%<*driver> 
\documentclass[11pt]{ltxdoc}
\usepackage[english]{babel}
\usepackage[exercisesfontsize=small]{exsol}
\usepackage{metalogo}
\EnableCrossrefs
\CodelineIndex
\RecordChanges
\usepackage{makeidx}
\usepackage{alltt}
\IfFileExists{tocbibind.sty}{\usepackage{tocbibind}}{}
\IfFileExists{hyperref.sty}{\usepackage[bookmarksopen]{hyperref}}{}
\EnableCrossrefs         
\CodelineIndex
\RecordChanges
\newcommand{\exsol}{\textsf{ExSol}}
\StopEventually{\PrintChanges\PrintIndex}
\def\fileversion{0.91}
\def\filedate{2014/08/31}
\begin{document}
 \DocInput{exsol.dtx}
\end{document}
%</driver>
% \fi
%
% \CheckSum{0}
%
% \CharacterTable
%  {Upper-case    \A\B\C\D\E\F\G\H\I\J\K\L\M\N\O\P\Q\R\S\T\U\V\W\X\Y\Z
%   Lower-case    \a\b\c\d\e\f\g\h\i\j\k\l\m\n\o\p\q\r\s\t\u\v\w\x\y\z
%   Digits        \0\1\2\3\4\5\6\7\8\9
%   Exclamation   \!     Double quote  \"     Hash (number) \#
%   Dollar        \$     Percent       \%     Ampersand     \&
%   Acute accent  \'     Left paren    \(     Right paren   \)
%   Asterisk      \*     Plus          \+     Comma         \,
%   Minus         \-     Point         \.     Solidus       \/
%   Colon         \:     Semicolon     \;     Less than     \<
%   Equals        \=     Greater than  \>     Question mark \?
%   Commercial at \@     Left bracket  \[     Backslash     \\
%   Right bracket \]     Circumflex    \^     Underscore    \_
%   Grave accent  \`     Left brace    \{     Vertical bar  \|
%   Right brace   \}     Tilde         \~}
%
%
% \changes{v0.1}{2012/01/05}{. Initial version}
% \changes{v0.2}{2012/01/06}{. Minor bug fixes based on first use by
% Paul Levrie}
% \changes{v0.3}{2012/01/07}{. Minor bug fixes based on second use by
% Paul}
% \changes{v0.4}{2012/01/09}{. Allowed for non-list formatting of
% exercises (as default)}
% \changes{v0.5}{2012/01/15}{. Added option to also send exercises to
% solutions file}
% \changes{v0.6}{2013/05/12}{. Prepared for CTAN publication}
% \changes{v0.7}{2014/07/14}{. Fixed UTF8 compatibility issues}
% \changes{v0.8}{2014/07/15}{. Fixed missing babel tag and running out
% of write hanles}
% \changes{v0.9}{2014/07/28}{. Changed default behavior
% w.r.t. minipage-wraping of exercises} 
% \changes{v0.91}{2014/08/31}{. Corrected minipage dependence, made }
%
% \DoNotIndex{\newcommand,\newenvironment}
% \setlength{\parindent}{0em}
% \addtolength{\parskip}{0.5\baselineskip}
%
% \title{The \exsol{} package\thanks{This document
%   corresponds to exsol~\fileversion, dated \filedate.}}
% \author{Walter Daems (\texttt{walter.daems@uantwerpen.be})}
% \date{}
%
% \maketitle
%
% \section{Introduction}
% %%%%%%%%%%%%%%%%%%%%%%
% The package \exsol{} provides macros to allow
% embedding exercises and solutions in the \LaTeX{} source of an
% instructional text (e.g., a book or a course text) while generating
% the following separate documents:
% \begin{itemize}
% \item your original text that only contains the exercises, and
% \item a solution book that only contains the solutions to the
% exercises (a package option exists to also copy the exercises themselves to the solution book).
% \end{itemize}
% 
% The former is generated when running \LaTeX{} on your document. This
% run writes the solutions to a secondary file that can be included
% into a simple document harness, such that when running \LaTeX{} on
% the latter, you can generate a nice solution book.
% 
% Why use \exsol{}?
% \begin{itemize}
% \item It allows to keep the \LaTeX{} source of your exercises and their
% solutions in a single file. Away with the nightmare to keep your
% solutions in sync with the original text.
% \item It separates exercises and solutions, allowing you
%   \begin{itemize}
%   \item to only release the solution book to the instructors of the
%   course;
%   \item to encourage students that you provide with the solution
%   book to first try solving the exercises without opening the book;
%   this seems to be easier than not peeking into the solution of an
%   exercise that is typeset just below the exercise itself.
%   \end{itemize}
% \end{itemize}
%
% The code of the \exsol{} package was taken almost literally
% from \textsf{fancyvrb} \cite{fancyvrb}. Therefore, all credits go to the
% authors/maintainers of \textsf{fancyvrb}.
%
% Thanks to Pieter Pareit and Pekka Pere for signaling problems and
% making suggestions for the documentation.
%
% \section{Installation}
% %%%%%%%%%%%%%%%%%%%%%%
% Either you are a package manager and then you'll know how to
% prepare an installation package for \exsol{}.
%
% Either you are a normal user and then you have two options. First,
% check if there is a package that your favorite \LaTeX{}
% distributor has prepared for you. Second, grab the TDS package
% from CTAN \cite{CTAN} (\texttt{exsol.tds.zip}) and unzip it somewhere in your
% own TDS tree, regenerate your filename database and off you go.
% In any case, make sure that \LaTeX{} finds the \texttt{exsol.sty} file.
%
% The \exsol{} package uses some auxiliary packages: \textsf{fancyvrb},
% \textsf{ifthen}, \textsf{kvoptions} and, optionally,
% \textsf{babel}. Fetch them from CTAN \cite{CTAN} if your \TeX{}
% distributor does not provide them.
%
% \section{Usage}
% %%%%%%%%%%%%%%%
% 
% \subsection{Preparing your document source}
% %%%%%%%%%%%%%%%%%%%%%%%%%%%%%%%%%%%%%%%%%%%
% The macro package exsol can be loaded with:
% \begin{verbatim}
% \usepackage{exsol}
% \end{verbatim}
%
% Then, you are ready to add some exercises including their solution
% to your document source. To this end, embed them in a
% \texttt{exercise} and a corresponding \texttt{solution} environment.
% Optionally, you may embed several of them in a \texttt{exercises}
% environment, to make them stand out in your text.
% E.g.,
%
% \begin{VerbatimOut}{exsol.tmp}
% 
% \begin{exercises}
%
%   \begin{exercise}
%     Solve the following equation for $x \in C$, with $C$ the set of
%     complex numbers:
%     \begin{equation}
%       5 x^2 -3 x = 5
%     \end{equation}
%   \end{exercise}
%   \begin{solution}
%     Let's start by rearranging the equation, a bit:
%     \begin{eqnarray}
%       5.7 x^2 - 3.1 x &=& 5.3\\
%       5.7 x^2 - 3.1 x -5.3 &=& 0
%     \end{eqnarray}
%     The equation is now in the standard form:
%     \begin{equation}
%       a x^2 + b x + c = 0
%     \end{equation}
%     For quadratic equations in the standard form, we know that two
%     solutions exist:
%     \begin{equation}
%       x_{1,2} = \frac{ -b \pm \sqrt{d}}{2a}
%     \end{equation}
%     with
%     \begin{equation}
%       d = b^2 - 4 a c
%     \end{equation}
%     If we apply this to our case, we obtain:
%     \begin{equation}
%       d = (-3.1)^2 - 4 \cdot 5.7 \cdot (-5.3) = 130.45
%     \end{equation}
%     and
%     \begin{eqnarray}
%       x_1 &=& \frac{3.1 + \sqrt{130.45}}{11.4} = 1.27\\
%       x_2 &=& \frac{3.1 - \sqrt{130.45}}{11.4} = -0.73
%     \end{eqnarray}
%     The proposed values $x = x_1, x_2$ are solutions to the given equation.
%   \end{solution}
%   \begin{exercise}
%     Consider a 2-dimensional vector space equipped with a Euclidean
%     distance function. Given a right-angled triangle, with the sides
%     $A$ and $B$ adjacent to the right angle having lengths, $3$ and
%     $4$, calculate the length of the hypotenuse, labeled $C$.
%   \end{exercise}
%   \begin{solution}
%     This calls for application of Pythagoras' theorem, which 
%     tells us:
%     \begin{equation}
%       \left\|A\right\|^2 + \left\|B\right\|^2 = \left\|C\right\|^2
%     \end{equation}
%     and therefore:
%     \begin{eqnarray}
%       \left\|C\right\| 
%       &=& \sqrt{\left\|A\right\|^2 + \left\|B\right\|^2}\\
%       &=& \sqrt{3^2 + 4^2}\\
%       &=& \sqrt{25} = 5
%     \end{eqnarray}
%     Therefore, the length of the hypotenuse equals $5$.
%   \end{solution}
%
% \end{exercises}
% \end{VerbatimOut}
% \VerbatimInput[frame=lines,gobble=2,fontsize=\footnotesize]{exsol.tmp}
%
% The result in the original document, can be seen below. As you can
% see, there's no trace of the solution. 
%
% % \iffalse meta-comment
%
% Copyright (C) 2014 by Walter Daems <walter.daems@uantwerpen.be>
%
% This work may be distributed and/or modified under the conditions of
% the LaTeX Project Public License, either version 1.3 of this license
% or (at your option) any later version.  The latest version of this
% license is in:
% 
%    http://www.latex-project.org/lppl.txt
% 
% and version 1.3 or later is part of all distributions of LaTeX version
% 2005/12/01 or later.
%
% This work has the LPPL maintenance status `maintained'.
% 
% The Current Maintainer of this work is Walter Daems.
%
% This work consists of the files exsol.dtx and exsol.ins and the derived 
% files:
%   - exsol.sty
%   - example.tex
%   - example-solutionbook.tex
%
% \fi
%
% \iffalse
%
%<package|driver>\NeedsTeXFormat{LaTeX2e}
%<driver>\ProvidesFile{exsol.dtx}
%<package>\ProvidesPackage{exsol}
%<package|driver>  [2014/08/31 v0.91 ExSol - Exercises and Solutions package (DMW)]
%<*driver> 
\documentclass[11pt]{ltxdoc}
\usepackage[english]{babel}
\usepackage[exercisesfontsize=small]{exsol}
\usepackage{metalogo}
\EnableCrossrefs
\CodelineIndex
\RecordChanges
\usepackage{makeidx}
\usepackage{alltt}
\IfFileExists{tocbibind.sty}{\usepackage{tocbibind}}{}
\IfFileExists{hyperref.sty}{\usepackage[bookmarksopen]{hyperref}}{}
\EnableCrossrefs         
\CodelineIndex
\RecordChanges
\newcommand{\exsol}{\textsf{ExSol}}
\StopEventually{\PrintChanges\PrintIndex}
\def\fileversion{0.91}
\def\filedate{2014/08/31}
\begin{document}
 \DocInput{exsol.dtx}
\end{document}
%</driver>
% \fi
%
% \CheckSum{0}
%
% \CharacterTable
%  {Upper-case    \A\B\C\D\E\F\G\H\I\J\K\L\M\N\O\P\Q\R\S\T\U\V\W\X\Y\Z
%   Lower-case    \a\b\c\d\e\f\g\h\i\j\k\l\m\n\o\p\q\r\s\t\u\v\w\x\y\z
%   Digits        \0\1\2\3\4\5\6\7\8\9
%   Exclamation   \!     Double quote  \"     Hash (number) \#
%   Dollar        \$     Percent       \%     Ampersand     \&
%   Acute accent  \'     Left paren    \(     Right paren   \)
%   Asterisk      \*     Plus          \+     Comma         \,
%   Minus         \-     Point         \.     Solidus       \/
%   Colon         \:     Semicolon     \;     Less than     \<
%   Equals        \=     Greater than  \>     Question mark \?
%   Commercial at \@     Left bracket  \[     Backslash     \\
%   Right bracket \]     Circumflex    \^     Underscore    \_
%   Grave accent  \`     Left brace    \{     Vertical bar  \|
%   Right brace   \}     Tilde         \~}
%
%
% \changes{v0.1}{2012/01/05}{. Initial version}
% \changes{v0.2}{2012/01/06}{. Minor bug fixes based on first use by
% Paul Levrie}
% \changes{v0.3}{2012/01/07}{. Minor bug fixes based on second use by
% Paul}
% \changes{v0.4}{2012/01/09}{. Allowed for non-list formatting of
% exercises (as default)}
% \changes{v0.5}{2012/01/15}{. Added option to also send exercises to
% solutions file}
% \changes{v0.6}{2013/05/12}{. Prepared for CTAN publication}
% \changes{v0.7}{2014/07/14}{. Fixed UTF8 compatibility issues}
% \changes{v0.8}{2014/07/15}{. Fixed missing babel tag and running out
% of write hanles}
% \changes{v0.9}{2014/07/28}{. Changed default behavior
% w.r.t. minipage-wraping of exercises} 
% \changes{v0.91}{2014/08/31}{. Corrected minipage dependence, made }
%
% \DoNotIndex{\newcommand,\newenvironment}
% \setlength{\parindent}{0em}
% \addtolength{\parskip}{0.5\baselineskip}
%
% \title{The \exsol{} package\thanks{This document
%   corresponds to exsol~\fileversion, dated \filedate.}}
% \author{Walter Daems (\texttt{walter.daems@uantwerpen.be})}
% \date{}
%
% \maketitle
%
% \section{Introduction}
% %%%%%%%%%%%%%%%%%%%%%%
% The package \exsol{} provides macros to allow
% embedding exercises and solutions in the \LaTeX{} source of an
% instructional text (e.g., a book or a course text) while generating
% the following separate documents:
% \begin{itemize}
% \item your original text that only contains the exercises, and
% \item a solution book that only contains the solutions to the
% exercises (a package option exists to also copy the exercises themselves to the solution book).
% \end{itemize}
% 
% The former is generated when running \LaTeX{} on your document. This
% run writes the solutions to a secondary file that can be included
% into a simple document harness, such that when running \LaTeX{} on
% the latter, you can generate a nice solution book.
% 
% Why use \exsol{}?
% \begin{itemize}
% \item It allows to keep the \LaTeX{} source of your exercises and their
% solutions in a single file. Away with the nightmare to keep your
% solutions in sync with the original text.
% \item It separates exercises and solutions, allowing you
%   \begin{itemize}
%   \item to only release the solution book to the instructors of the
%   course;
%   \item to encourage students that you provide with the solution
%   book to first try solving the exercises without opening the book;
%   this seems to be easier than not peeking into the solution of an
%   exercise that is typeset just below the exercise itself.
%   \end{itemize}
% \end{itemize}
%
% The code of the \exsol{} package was taken almost literally
% from \textsf{fancyvrb} \cite{fancyvrb}. Therefore, all credits go to the
% authors/maintainers of \textsf{fancyvrb}.
%
% Thanks to Pieter Pareit and Pekka Pere for signaling problems and
% making suggestions for the documentation.
%
% \section{Installation}
% %%%%%%%%%%%%%%%%%%%%%%
% Either you are a package manager and then you'll know how to
% prepare an installation package for \exsol{}.
%
% Either you are a normal user and then you have two options. First,
% check if there is a package that your favorite \LaTeX{}
% distributor has prepared for you. Second, grab the TDS package
% from CTAN \cite{CTAN} (\texttt{exsol.tds.zip}) and unzip it somewhere in your
% own TDS tree, regenerate your filename database and off you go.
% In any case, make sure that \LaTeX{} finds the \texttt{exsol.sty} file.
%
% The \exsol{} package uses some auxiliary packages: \textsf{fancyvrb},
% \textsf{ifthen}, \textsf{kvoptions} and, optionally,
% \textsf{babel}. Fetch them from CTAN \cite{CTAN} if your \TeX{}
% distributor does not provide them.
%
% \section{Usage}
% %%%%%%%%%%%%%%%
% 
% \subsection{Preparing your document source}
% %%%%%%%%%%%%%%%%%%%%%%%%%%%%%%%%%%%%%%%%%%%
% The macro package exsol can be loaded with:
% \begin{verbatim}
% \usepackage{exsol}
% \end{verbatim}
%
% Then, you are ready to add some exercises including their solution
% to your document source. To this end, embed them in a
% \texttt{exercise} and a corresponding \texttt{solution} environment.
% Optionally, you may embed several of them in a \texttt{exercises}
% environment, to make them stand out in your text.
% E.g.,
%
% \begin{VerbatimOut}{exsol.tmp}
% 
% \begin{exercises}
%
%   \begin{exercise}
%     Solve the following equation for $x \in C$, with $C$ the set of
%     complex numbers:
%     \begin{equation}
%       5 x^2 -3 x = 5
%     \end{equation}
%   \end{exercise}
%   \begin{solution}
%     Let's start by rearranging the equation, a bit:
%     \begin{eqnarray}
%       5.7 x^2 - 3.1 x &=& 5.3\\
%       5.7 x^2 - 3.1 x -5.3 &=& 0
%     \end{eqnarray}
%     The equation is now in the standard form:
%     \begin{equation}
%       a x^2 + b x + c = 0
%     \end{equation}
%     For quadratic equations in the standard form, we know that two
%     solutions exist:
%     \begin{equation}
%       x_{1,2} = \frac{ -b \pm \sqrt{d}}{2a}
%     \end{equation}
%     with
%     \begin{equation}
%       d = b^2 - 4 a c
%     \end{equation}
%     If we apply this to our case, we obtain:
%     \begin{equation}
%       d = (-3.1)^2 - 4 \cdot 5.7 \cdot (-5.3) = 130.45
%     \end{equation}
%     and
%     \begin{eqnarray}
%       x_1 &=& \frac{3.1 + \sqrt{130.45}}{11.4} = 1.27\\
%       x_2 &=& \frac{3.1 - \sqrt{130.45}}{11.4} = -0.73
%     \end{eqnarray}
%     The proposed values $x = x_1, x_2$ are solutions to the given equation.
%   \end{solution}
%   \begin{exercise}
%     Consider a 2-dimensional vector space equipped with a Euclidean
%     distance function. Given a right-angled triangle, with the sides
%     $A$ and $B$ adjacent to the right angle having lengths, $3$ and
%     $4$, calculate the length of the hypotenuse, labeled $C$.
%   \end{exercise}
%   \begin{solution}
%     This calls for application of Pythagoras' theorem, which 
%     tells us:
%     \begin{equation}
%       \left\|A\right\|^2 + \left\|B\right\|^2 = \left\|C\right\|^2
%     \end{equation}
%     and therefore:
%     \begin{eqnarray}
%       \left\|C\right\| 
%       &=& \sqrt{\left\|A\right\|^2 + \left\|B\right\|^2}\\
%       &=& \sqrt{3^2 + 4^2}\\
%       &=& \sqrt{25} = 5
%     \end{eqnarray}
%     Therefore, the length of the hypotenuse equals $5$.
%   \end{solution}
%
% \end{exercises}
% \end{VerbatimOut}
% \VerbatimInput[frame=lines,gobble=2,fontsize=\footnotesize]{exsol.tmp}
%
% The result in the original document, can be seen below. As you can
% see, there's no trace of the solution. 
%
% \input{exsol.tmp}
%
% When running \LaTeX{} on your document (in our case on the
% \texttt{exsol.dtx} file, as a side effect a file with extension
% \texttt{.sol.tex} has been written to disk (in our case, the file
% \texttt{exsol.sol.tex}), containing all solutions in sequence.
%
% Generating a solution book is a simple as including the file into a
% simple \LaTeX{} harness, that allows you giving it a proper title page and to
% add other bells and whistles.
%
% E.g.,
% \begin{VerbatimOut}{exsol-solutionbook.tex}
% \documentclass{article}
% \usepackage[english]{babel}
% \title{Solutions to the exercises, specified in the \textsf{ExSol} package}
% \author{Walter Daems}
% \date{2013/05/12}
%
% \begin{document}
%
% \maketitle
%
% \input{exsol.sol}
%
% \end{document}
% \end{VerbatimOut}
% \VerbatimInput[frame=lines,gobble=2,fontsize=\footnotesize]{exsol-solutionbook.tex}
% 
% You may generate this solution book, by running \LaTeX{} on the
% file named \texttt{exsol-solutionbook.tex} that is generated when running
% \LaTeX{} on the \texttt{exsol.dtx} file.
%
% The result approximately looks like this:
%
% \setcounter{equation}{0}
% \rule{\linewidth}{.7pt}
% \begin{center}
% {\Large Solutions to the exercises, specified in the \textsf{ExSol} package}\\
% {\large Walter Daems}\\
% {\large 2013/05/12}
% \end{center}
% \par---\newline\textbf{Solution 3.1-1}
%     Let's start by rearranging the equation, a bit:
%     \begin{eqnarray}
%       5.7 x^2 - 3.1 x &=% 5.3\\
%       5.7 x^2 - 3.1 x -5.3 &=% 0
%     \end{eqnarray}
%     The equation is now in the standard form:
%     \begin{equation}
%       a x^2 + b x + c = 0
%     \end{equation}
%     For quadratic equations in the standard form, we know that two
%     solutions exist:
%     \begin{equation}
%       x_{1,2} = \frac{ -b \pm \sqrt{d}}{2a}
%     \end{equation}
%     with
%     \begin{equation}
%       d = b^2 - 4 a c
%     \end{equation}
%     If we apply this to our case, we obtain:
%     \begin{equation}
%       d = (-3.1)^2 - 4 \cdot 5.7 \cdot (-5.3) = 130.45
%     \end{equation}
%     and
%     \begin{eqnarray}
%       x_1 &=& \frac{3.1 + \sqrt{130.45}}{11.4} = 1.27\\
%       x_2 &=& \frac{3.1 + \sqrt{130.45}}{11.4} = -0.73
%     \end{eqnarray}
%     The proposed values $x = x_1, x_2$ are solutions to
%     the given equation.
% \par---\newline\textbf{Solution 3.1-2}
%       This calls for application of Pythagoras' theorem, which
%       tells us:
%       \begin{equation}
%         \left\|A\right\|^2 + \left\|B\right\|^2 = \left\|C\right\|^2
%       \end{equation}
%       and therefore:
%       \begin{eqnarray}
%         \left\|C\right\|
%         &=& \sqrt{\left\|A\right\|^2 + \left\|B\right\|^2}\\
%         &=& \sqrt{3^2 + 4^2}\\
%         &=& \sqrt{25} = 5
%       \end{eqnarray}
%       Therefore, the length of the hypotenuse equals $5$.
%
% \rule{\linewidth}{.7pt}
%
% \subsection{Fiddling with the spacing}
%
% The default spacing provided by the \textsf{ExSol} package should be
% fine for most users. However, if you like to tweak, below you can
% find the controls.
% \subsubsection{Spacing before and after the \texttt{exercises} environment}
%
% The lengths below control the spacing of the |exercises| environment:
% \begin{itemize}
% \item |exsolexerciseaboveskip|: rubber length controlling the
% vertical space after the top marker line of the environment
% \item |exsolexercisebelowskip|: rubber length controlling the
% vertical space before the bottom marker line of the environment
% \end{itemize}
%
% You can simply specify them like:
% \begin{verbatim}
% \setlength{\exsolexercisesaboveskip}{1ex plus 1pt minus 1pt}
% \setlength{\exsolexercisesbelowskip}{1ex plus 1pt minus 1pt}
% \end{verbatim}
% The spacings specified here are the package defaults.
%
% \subsubsection{Spacing of the individual exercises}
% Caution: the spacing can only be tuned, when one invokes the
% |exerciseaslist| package option!
%
% Then lengths below control the spacing of the |exercise| environment:
% \begin{itemize}
% \item |exercisetopbottomsep|: rubber length controlling the vertical
% space before and after individual exercises
% \item |exerciseleftmargin|: length controlling the horizontal
% space between the surrounding environment's left margin (most
% often the page margin) and the left edge of the exercise
% environment 
% \item |exerciseleftmargin|: length controlling the horizontal
% space between the surrounding environment's right margin (most
% often the page margin) and the right edge of the exercise
% environment
% \item |exerciseitemindent|: length controlling the first-line
% indentation of the first paragraph in the exercise environment
% (actually, the label is set w.r.t. this position, that we will
% conveniently call position 'x')
% \item |exerciseparindent|: length controlling the first-line
% indentation of the other paragraphs in the exercise environment.
% \item |exerciselabelsep|: length controlling the distance between
% the label and position 'x'
% \item |exerciselabelwidth|: minimal width of the (internally
% right-alligned) box to use for the exercises label; if the box is
% not sufficiently big, position 'x' is shifted to the right
% \item |exerciseparsep|: internal paragraph separation (vertically)
% \end{itemize}
% 
% You can simply specify them like:
% \begin{verbatim}
% \setlength{\exsolexercisetopbottomsep}{0pt plus 0pt minus 1pt}
% \setlength{\exsolexerciseleftmargin}{1em}
% \setlength{\exsolexerciserightmargin}{1em}
% \setlength{\exsolexerciseparindent}{0em}
% \setlength{\exsolexerciselabelsep}{0.5em}
% \setlength{\exsolexerciselabelwidth}{0pt}
% \setlength{\exsolexerciseitemindent}{0pt}
% \setlength{\exsolexerciseparsep}{\parskip}
% \end{verbatim}
% The spacings specified here are the package defaults.
%
% \subsection{Tips and tricks}
%
% If you want to include the solutions all at the
% end of the current document, you need to explicitly close the
% solution stream before including it:
% \begin{verbatim}
%   \closeout\solutionstream\input{\jobname.sol.tex}
% \end{verbatim}
%
% If you want to avoid exercises being split by a page boundary, then
% provide the package option 'minipage'. This causes the exercises to
% be wrapped in a minipage environment.
% 
% \clearpage
%
% \section{Implementation}
% %%%%%%%%%%%%%%%%%%%%%%%
%    \begin{macrocode}
%<*package>
%    \end{macrocode}
%
% \subsection{Auxiliary packages}
% %%%%%%%%%%%%%%%%%%%%%%%%%%%%%%%
% The package uses some auxiliary packages:
%    \begin{macrocode}
\RequirePackage{fancyvrb}
\RequirePackage{ifthen}
\RequirePackage{kvoptions}
%    \end{macrocode}
%
% \subsection{Package options}
% %%%%%%%%%%%%%%%%%%%%%%%%%%%%
% The package offers some options:
%
% \changes{v0.2}{2012/01/06}{Added option exercisesfont}
% \changes{v0.4}{2012/01/09}{Changed name of option to exercisesfontsize}
%
% \begin{macro}{exercisesfontsize}
%  This option allows setting the font of the \texttt{exercises}
%  environment. You may chopse one of tiny, scriptsize, footnotesize,
%  small, normalsize, large, etc.\\
%  E.g., \texttt{[exercisesfontsize=small]}.
%    \begin{macrocode}
\DeclareStringOption[normalsize]{exercisesfontsize}
%    \end{macrocode}
% \end{macro}
%
% \changes{v0.4}{2012/01/06}{Added option exercisesinlist}
% \changes{v0.5}{2012/01/09}{Changed option exercisesinlist to exerciseaslist}
%
% \begin{macro}{exerciseaslist}
%  This boolean option (true, false) allows setting the typesetting of
%  the \texttt{exercises} in a list environment. This causes the
%  exercises to be typeset in a more compact fashion, with indented
%  left and right margin. 
%    \begin{macrocode}
\DeclareBoolOption[false]{exerciseaslist}
%    \end{macrocode}
% \end{macro}
%
% \changes{v0.5}{2012/01/09}{Added option copyexercisesinsolutions}
% \begin{macro}{copyexercisesinsolutions}
%  This boolean option (true, false) allows copying the exercises in
%  the solutions file, to allow for making a complete stand-alone
%  exercises bundle.
%    \begin{macrocode}
\DeclareBoolOption[false]{copyexercisesinsolutions}
%    \end{macrocode}
% \end{macro}
%
% \changes{v0.9}{2014/07/28}{. Changed default behavior
% w.r.t. minipage-wraping of exercises}
% \begin{macro}{minipage}
%  This boolean option (true, false) causes the exercises to be
%  wrapped in minipages. This avoids them getting split by a page
%  boundary.
%    \begin{macrocode}
\DeclareBoolOption[false]{minipage}
%    \end{macrocode}
% \end{macro}
%
% The options are processed using:
%    \begin{macrocode}
\ProcessKeyvalOptions*
%    \end{macrocode}
%
% The options are subsequently handled
%    \begin{macrocode}
\newcommand{\exercisesfontsize}{\csname \exsol@exercisesfontsize\endcsname}
%    \end{macrocode}
%
%
% \subsection{Customization of lengths}
% %%%%%%%%%%%%%%%%%%%%%%%%%%%%%%%%%%%%%%%
% The commands below allow customizing many lengths that control the
% typesetting of the exercises.
%
% \changes{v0.91}{2014/08/31}{added user-accessible lengths}
% First some lengths to control the spacing before and after |exercises|.
%    \begin{macrocode}
\newlength{\exsolexercisesaboveskip}
\setlength{\exsolexercisesaboveskip}{1ex plus 1pt minus 1pt}
\newlength{\exsolexercisesbelowskip}
\setlength{\exsolexercisesbelowskip}{1ex plus 1pt minus 1pt}
%    \end{macrocode}
%
% Then some lengths to control the spacing for a single
% exercise. These lengths only work when the |exerciseaslist| package
% option has been specified. Sensible defaults have been set.
%    \begin{macrocode}
\newlength{\exsolexercisetopbottomsep}
\setlength{\exsolexercisetopbottomsep}{0pt plus 0pt minus 1pt}
\newlength{\exsolexerciseleftmargin}
\setlength{\exsolexerciseleftmargin}{1em}
\newlength{\exsolexerciserightmargin}
\setlength{\exsolexerciserightmargin}{1em}
\newlength{\exsolexerciseparindent}
\setlength{\exsolexerciseparindent}{0em}
\newlength{\exsolexerciselabelsep}
\setlength{\exsolexerciselabelsep}{0.5em}
\newlength{\exsolexerciselabelwidth}
\setlength{\exsolexerciselabelwidth}{0pt}
\newlength{\exsolexerciseitemindent}
\setlength{\exsolexerciseitemindent}{0pt}
\newlength{\exsolexerciseparsep}
\setlength{\exsolexerciseparsep}{\parskip}
%    \end{macrocode}
% 
% 
% \subsection{Con- and destruction of the auxiliary streams}
% %%%%%%%%%%%%%%%%%%%%%%%%%%%%%%%%%%%%%%%%%%%%%%%%%%%%%%%%%%
% At the beginning of your document, we start by opening a stream to a
% file that will be used to write the solutions to. At the end of your
% document, the package closes the stream.
% \changes{v0.8}{2014/07/15}{moved newwrite of exercise stream to this
% spot to avoid consuming all handles}
%    \begin{macrocode}
\AtBeginDocument{
  \newwrite\solutionstream
  \immediate\openout\solutionstream=\jobname.sol.tex
  \newwrite\exercisestream
}
\AtEndDocument{
  \immediate\closeout\solutionstream
}
%    \end{macrocode}
%
% \subsection{Exercises counter}
% %%%%%%%%%%%%%%%%%%%%%%%%%%%%%%
% By providing an exercise counter, proper numbering of the exercises
% is provided to allow for good cross referencing of the solutions to
% the exercises.
% \changes{v0.2}{2012/01/06}{Removed dash in counter when in document
% without sectioning commands}
%    \begin{macrocode}
\newcounter{exercise}[subsection]
\setcounter{exercise}{0}
\renewcommand{\theexercise}{%
  \@ifundefined{c@chapter}{}{\if0\arabic{chapter}\else\arabic{chapter}.\fi}%
  \if0\arabic{section}\else\arabic{section}\fi%
  \if0\arabic{subsection}\else.\arabic{subsection}\fi%
  \if0\arabic{subsubsection}\else.\arabic{subsubsection}\fi%
  \if0\arabic{exercise}\else%
    \@ifundefined{c@chapter}%
                 {\if0\arabic{section}\else-\fi}%
                 {-}%
    \arabic{exercise}%
  \fi
}
%    \end{macrocode}
%
%
%
% \subsection{Detokenization in order to cope with utf8}
%
% Combining old-school \LaTeX{} (before \XeTeX{} and \LuaTeX{}) and
% UTF-8 is a pain.
% Detokenization has been suggested by Geoffrey Poore to solve issues
% with UTF-8 characters messing up the |fancyvrb| internals.
% \changes{v0.7}{2014/07/14}{Added detokenized writing}
%    \begin{macrocode}
\newcommand{\GPES@write@detok}[1]{%
  \immediate\write\exercisestream{\detokenize{#1}}}%
\newcommand{\GPSS@write@detok}[1]{%
  \immediate\write\solutionstream{\detokenize{#1}}}%
\newcommand{\GPESS@write@detok}[1]{%
  \GPES@write@detok{#1}%
  \GPSS@write@detok{#1}}%
%    \end{macrocode}
%
%
% \section{The user environments}
%
% \begin{macro}{exercise}
%   The \texttt{exercise} environment is used to typeset your
%   exercises, provide them with a nice label and allow for copying
%   the exercise to the solutions file (if the package option
%   \texttt{copyexercisesinsolution}) is set. The label can be
%   set by redefining the \cs{exercisename} macro, or by relying on
%   the \textsf{Babel} provisions. The code is almost litteraly
%   taken from the \textsf{fancyvrb} package.
%    \begin{macrocode}
\def\exercise{\FV@Environment{}{exercise}}
\def\FVB@exercise{%
  \refstepcounter{exercise}%
  \immediate\openout\exercisestream=\jobname.exc.tex
  \ifexsol@copyexercisesinsolutions
    \typeout{Writing exercise to \jobname.sol.tex}
    \immediate\write\solutionstream{\string\par---\string\newline
      \string\textbf\string{\exercisename{} \theexercise \string}}
  \else
    \immediate\write\solutionstream{\string\par---\string\newline}
  \fi
  \immediate\write\exercisestream{\string\begin{exsol@exercise}}
  \@bsphack
  \begingroup
    \FV@UseKeyValues
    \FV@DefineWhiteSpace
    \def\FV@Space{\space}%
    \FV@DefineTabOut
    \ifexsol@copyexercisesinsolutions
      \let\FV@ProcessLine\GPESS@write@detok %
    \else
      \let\FV@ProcessLine\GPES@write@detok %
    \fi
    \relax
    \let\FV@FontScanPrep\relax
    \let\@noligs\relax
    \FV@Scan
  }
\def\FVE@exercise{
  \endgroup\@esphack
  \immediate\write\exercisestream{\string\end{exsol@exercise}}
  \ifexsol@copyexercisesinsolutions
    \immediate\write\solutionstream{\string~\string\newline}
  \fi
  \immediate\closeout\exercisestream
  \input{\jobname.exc.tex}
}
\DefineVerbatimEnvironment{exercise}{exercise}{}
%    \end{macrocode}
% \end{macro}
%
% \begin{macro}{exsol@exercise}
%   The \texttt{exsol@exercise} environment is an internal macro used
%   to typeset your exercises and provide them with a nice label and
%   number. Do not use it directly. Use the proper environment
%   \texttt{exercise} instead.
%   \changes{v0.2}{2012/01/06}{Attempted to fix MiKTeX formatting problems}
%   \changes{v0.3}{2012/01/08}{Fixed labelsep to avoid cluttered
%   itemize environments}
%   \changes{v0.4}{2012/01/06}{Added option exercisesinlist such that
%   default results in non list formatting of exercise}
%   \changes{v0.5}{2012/01/09}{Changed implementation to allow for
%   copying the exercises to the solutions file.}
%    \begin{macrocode}
\newenvironment{exsol@exercise}[0]
{%
  \ifthenelse{\boolean{exsol@minipage}}{\begin{minipage}[t]{\textwidth}}{}%
    \ifthenelse{\boolean{exsol@exerciseaslist}}
               {\begin{list}%
                   {%
                   }%
                   {%
                     \setlength{\topsep}{\exsolexercisetopbottomsep}%
                     \setlength{\leftmargin}{\exsolexerciseleftmargin}%
                     \setlength{\rightmargin}{\exsolexerciserightmargin}%
                     \setlength{\listparindent}{\exsolexerciseparindent}%
                     \setlength{\itemindent}{\exsolexerciseitemindent}%
                     \setlength{\parsep}{\exsolexerciseparsep}
                     \setlength{\labelsep}{\exsolexerciselabelsep}
                     \setlength{\labelwidth}{\exsolexerciselabelwidth}}
                 \item[\textit{~\exercisename{} \theexercise:~}]
               }%
               {\textit{\exercisename{} \theexercise:}}
}
{%
  \ifthenelse{\boolean{exsol@exerciseaslist}}%
             {\end{list}}{}%
  \ifthenelse{\boolean{exsol@minipage}}{\end{minipage}}{\par}%
}
%    \end{macrocode}
% \end{macro}
%
%
% \begin{macro}{solution}
%   The \texttt{solution} environment is used to typeset your solutions
%   and provide them with a nice label and number that corresponds to
%   the exercise that preceeded this solution. Theno label can be
%   set by redefining the \cs{solutionname} macro, or by relying on
%   the \textsf{Babel} provisions. The code is almost litteraly
%   taken from the \textsf{fancyvrb} package.
%    \begin{macrocode}
\def\solution{\FV@Environment{}{solution}}
\def\FVB@solution{%
  \typeout{Writing solution to \jobname.sol.tex}
  \immediate\write\solutionstream{\string\textbf\string{\solutionname{}\string}}
  \ifexsol@copyexercisesinsolutions
    \immediate\write\solutionstream{\string\newline}
  \else
    \immediate\write\solutionstream{\string\textbf\string{\theexercise\string}%
                                    \string\newline}
  \fi
  \@bsphack
  \begingroup
    \FV@UseKeyValues
    \FV@DefineWhiteSpace
    \def\FV@Space{\space}%
    \FV@DefineTabOut
    \let\FV@ProcessLine\GPSS@write@detok %
    \relax
    \let\FV@FontScanPrep\relax
    \let\@noligs\relax
    \FV@Scan
  }
\def\FVE@solution{\endgroup\@esphack}
\DefineVerbatimEnvironment{solution}{solution}{}
%    \end{macrocode}
% \end{macro}
%
% \begin{macro}{exercises}
%   The \texttt{exercises} environment helps typesetting your exercises to
%   stand out from the rest of the text. You may use it at the end of
%   a chapter, or just to group some exercises in the text.
%   \changes{v0.2}{2012/01/06}{Attempted to fix MiKTeX formatting problems}
%   \changes{v0.3}{2012/01/07}{Added some extra whitespace below exercisesname}
%    \begin{macrocode}
\newenvironment{exercises}
{\par\exercisesfontsize\rule{.25\linewidth}{0.15mm}\vspace*{\exsolexercisesaboveskip}\\*%
 \textbf{\normalsize \exercisesname}}
{\vspace*{-\baselineskip}\vspace*{\exsolexercisesbelowskip}\rule{.25\linewidth}{0.15mm}\par}
%    \end{macrocode}
% \end{macro}
%
% \subsection{Some Babel provisions}
% %%%%%%%%%%%%%%%%%%%%%%%%%%%%%%%%%%
% \changes{v0.2}{2012/01/06}{Fixed babel errors}
% \begin{macro}{\exercisename}
%   The exercise environment makes use of a label \texttt{\exercisename{}}
%   macro.
%    \begin{macrocode}
\newcommand{\exercisename}{Exercise}
%    \end{macrocode}
% \end{macro}
%
% \begin{macro}{\exercisesname}
%   The exercises environment makes use of a label \texttt{\exercisesname{}}
%   macro.
%    \begin{macrocode}
\newcommand{\exercisesname}{Exercises}
%    \end{macrocode}
% \end{macro}
% 
% \begin{macro}{\solutionname}
%   The solution environment makes use of a label \texttt{\solutionname{}}
%   macro.
%    \begin{macrocode}
\newcommand{\solutionname}{Solution}
%    \end{macrocode}
% \end{macro}
%
% \begin{macro}{\solutionname}
%   The solution environment makes use of a label \texttt{\solutionname{}}
%   macro.
% \changes{v0.8}{2014/07/15}{Added missing babel tag}
%    \begin{macrocode}
\newcommand{\solutionsname}{Solutions}
%    \end{macrocode}
% \end{macro}
% 
% 
% You may redefine these macros, but to help you out a little bit, we
% provide with some basic Babel auxiliaries. If you're a true polyglot
% and are willing to help me out by providing translations for other
% languages, I'm very willing to incorporate them into the code.
%
% \changes{v0.7}{2014/07/14}{Added Finnish language support}
%    \begin{macrocode}
\addto\captionsdutch{%
  \renewcommand{\exercisename}{Oefening}%
  \renewcommand{\exercisesname}{Oefeningen}%
  \renewcommand{\solutionname}{Oplossing}%
  \renewcommand{\solutionsname}{Oplossingen}%
}
\addto\captionsgerman{%
  \renewcommand{\exercisename}{Aufgabe}%
  \renewcommand{\exercisesname}{Aufgaben}%
  \renewcommand{\solutionname}{L\"osung}%
  \renewcommand{\solutionsname}{L\"osungen}%
}
\addto\captionsfrench{%
  \renewcommand{\exercisename}{Exercice}%
  \renewcommand{\exercisesname}{Exercices}%
  \renewcommand{\solutionname}{Solution}%
  \renewcommand{\solutionsname}{Solutions}%
}
\addto\captionsfinnish{
  \renewcommand{\exercisename}{Teht\"av\"a}%
  \renewcommand{\exercisesname}{Teht\"avi\"a}%
  \renewcommand{\solutionname}{Ratkaisu}%
  \renewcommand{\solutionsname}{Ratkaisut}%
}
%    \end{macrocode}
%
%
%
% Now the final hack overloads the basic sectioning commands to make
% sure that they are copied into your solution book.
%
%    \begin{macrocode}
\let\exsol@@makechapterhead\@makechapterhead
\def\@makechapterhead#1{%
  \immediate\write\solutionstream{\string\chapter{#1}}%
  \exsol@@makechapterhead{#1}
}
\ifdefined\frontmatter
  \let\exsol@@frontmatter\frontmatter
  \def\frontmatter{%
    \immediate\write\solutionstream{\string\frontmatter}%
    \exsol@@frontmatter
  }
\fi
\ifdefined\frontmatter
  \let\exsol@@mainmatter\mainmatter
  \def\mainmatter{%
    \immediate\write\solutionstream{\string\mainmatter}%
    \exsol@@mainmatter
  }
\fi
\ifdefined\backmatter
  \let\exsol@@backmatter\backmatter
  \def\backmatter{%
    \immediate\write\solutionstream{\string\backmatter}%
    \exsol@@backmatter
  }
\fi
%    \end{macrocode}
%
% \begin{macro}{\noexercisesinchapter}
%   If you have chapters without exercises, you may want to indicate
%   this clearly into your source. Otherwise empty chapters may appear
%   in your solution book.
%    \begin{macrocode}
\newcommand{\noexercisesinchapter}
{
  \immediate\write\solutionstream{No exercises in this chapter}
}
%    \end{macrocode}
% \end{macro}
%
%    \begin{macrocode}
%</package>
%    \end{macrocode}
%
% \bibliographystyle{alpha}
%
% \begin{thebibliography}{99}
%
% \bibitem{fancyvrb}
% Timothy Van Zandt, Herbert Vo\ss, Denis Girou, Sebastian Rahtz, Niall
% Mansfield 
% \newblock The \texttt{fancyvrb} package.
% \newblock \url{http://ctan.org/pkg/fancyvrb}.
% \newblock online, accessed in January 2012.
%
% \bibitem{CTAN} 
% The Comprehensive TeX Archive Network.
% \newblock \url{http://www.ctan.org}.
% \newblock online, accessed in January 2012.
%
% \end{thebibliography}
%
% \Finale
\endinput

%
% When running \LaTeX{} on your document (in our case on the
% \texttt{exsol.dtx} file, as a side effect a file with extension
% \texttt{.sol.tex} has been written to disk (in our case, the file
% \texttt{exsol.sol.tex}), containing all solutions in sequence.
%
% Generating a solution book is a simple as including the file into a
% simple \LaTeX{} harness, that allows you giving it a proper title page and to
% add other bells and whistles.
%
% E.g.,
% \begin{VerbatimOut}{exsol-solutionbook.tex}
% \documentclass{article}
% \usepackage[english]{babel}
% \title{Solutions to the exercises, specified in the \textsf{ExSol} package}
% \author{Walter Daems}
% \date{2013/05/12}
%
% \begin{document}
%
% \maketitle
%
% % \iffalse meta-comment
%
% Copyright (C) 2014 by Walter Daems <walter.daems@uantwerpen.be>
%
% This work may be distributed and/or modified under the conditions of
% the LaTeX Project Public License, either version 1.3 of this license
% or (at your option) any later version.  The latest version of this
% license is in:
% 
%    http://www.latex-project.org/lppl.txt
% 
% and version 1.3 or later is part of all distributions of LaTeX version
% 2005/12/01 or later.
%
% This work has the LPPL maintenance status `maintained'.
% 
% The Current Maintainer of this work is Walter Daems.
%
% This work consists of the files exsol.dtx and exsol.ins and the derived 
% files:
%   - exsol.sty
%   - example.tex
%   - example-solutionbook.tex
%
% \fi
%
% \iffalse
%
%<package|driver>\NeedsTeXFormat{LaTeX2e}
%<driver>\ProvidesFile{exsol.dtx}
%<package>\ProvidesPackage{exsol}
%<package|driver>  [2014/08/31 v0.91 ExSol - Exercises and Solutions package (DMW)]
%<*driver> 
\documentclass[11pt]{ltxdoc}
\usepackage[english]{babel}
\usepackage[exercisesfontsize=small]{exsol}
\usepackage{metalogo}
\EnableCrossrefs
\CodelineIndex
\RecordChanges
\usepackage{makeidx}
\usepackage{alltt}
\IfFileExists{tocbibind.sty}{\usepackage{tocbibind}}{}
\IfFileExists{hyperref.sty}{\usepackage[bookmarksopen]{hyperref}}{}
\EnableCrossrefs         
\CodelineIndex
\RecordChanges
\newcommand{\exsol}{\textsf{ExSol}}
\StopEventually{\PrintChanges\PrintIndex}
\def\fileversion{0.91}
\def\filedate{2014/08/31}
\begin{document}
 \DocInput{exsol.dtx}
\end{document}
%</driver>
% \fi
%
% \CheckSum{0}
%
% \CharacterTable
%  {Upper-case    \A\B\C\D\E\F\G\H\I\J\K\L\M\N\O\P\Q\R\S\T\U\V\W\X\Y\Z
%   Lower-case    \a\b\c\d\e\f\g\h\i\j\k\l\m\n\o\p\q\r\s\t\u\v\w\x\y\z
%   Digits        \0\1\2\3\4\5\6\7\8\9
%   Exclamation   \!     Double quote  \"     Hash (number) \#
%   Dollar        \$     Percent       \%     Ampersand     \&
%   Acute accent  \'     Left paren    \(     Right paren   \)
%   Asterisk      \*     Plus          \+     Comma         \,
%   Minus         \-     Point         \.     Solidus       \/
%   Colon         \:     Semicolon     \;     Less than     \<
%   Equals        \=     Greater than  \>     Question mark \?
%   Commercial at \@     Left bracket  \[     Backslash     \\
%   Right bracket \]     Circumflex    \^     Underscore    \_
%   Grave accent  \`     Left brace    \{     Vertical bar  \|
%   Right brace   \}     Tilde         \~}
%
%
% \changes{v0.1}{2012/01/05}{. Initial version}
% \changes{v0.2}{2012/01/06}{. Minor bug fixes based on first use by
% Paul Levrie}
% \changes{v0.3}{2012/01/07}{. Minor bug fixes based on second use by
% Paul}
% \changes{v0.4}{2012/01/09}{. Allowed for non-list formatting of
% exercises (as default)}
% \changes{v0.5}{2012/01/15}{. Added option to also send exercises to
% solutions file}
% \changes{v0.6}{2013/05/12}{. Prepared for CTAN publication}
% \changes{v0.7}{2014/07/14}{. Fixed UTF8 compatibility issues}
% \changes{v0.8}{2014/07/15}{. Fixed missing babel tag and running out
% of write hanles}
% \changes{v0.9}{2014/07/28}{. Changed default behavior
% w.r.t. minipage-wraping of exercises} 
% \changes{v0.91}{2014/08/31}{. Corrected minipage dependence, made }
%
% \DoNotIndex{\newcommand,\newenvironment}
% \setlength{\parindent}{0em}
% \addtolength{\parskip}{0.5\baselineskip}
%
% \title{The \exsol{} package\thanks{This document
%   corresponds to exsol~\fileversion, dated \filedate.}}
% \author{Walter Daems (\texttt{walter.daems@uantwerpen.be})}
% \date{}
%
% \maketitle
%
% \section{Introduction}
% %%%%%%%%%%%%%%%%%%%%%%
% The package \exsol{} provides macros to allow
% embedding exercises and solutions in the \LaTeX{} source of an
% instructional text (e.g., a book or a course text) while generating
% the following separate documents:
% \begin{itemize}
% \item your original text that only contains the exercises, and
% \item a solution book that only contains the solutions to the
% exercises (a package option exists to also copy the exercises themselves to the solution book).
% \end{itemize}
% 
% The former is generated when running \LaTeX{} on your document. This
% run writes the solutions to a secondary file that can be included
% into a simple document harness, such that when running \LaTeX{} on
% the latter, you can generate a nice solution book.
% 
% Why use \exsol{}?
% \begin{itemize}
% \item It allows to keep the \LaTeX{} source of your exercises and their
% solutions in a single file. Away with the nightmare to keep your
% solutions in sync with the original text.
% \item It separates exercises and solutions, allowing you
%   \begin{itemize}
%   \item to only release the solution book to the instructors of the
%   course;
%   \item to encourage students that you provide with the solution
%   book to first try solving the exercises without opening the book;
%   this seems to be easier than not peeking into the solution of an
%   exercise that is typeset just below the exercise itself.
%   \end{itemize}
% \end{itemize}
%
% The code of the \exsol{} package was taken almost literally
% from \textsf{fancyvrb} \cite{fancyvrb}. Therefore, all credits go to the
% authors/maintainers of \textsf{fancyvrb}.
%
% Thanks to Pieter Pareit and Pekka Pere for signaling problems and
% making suggestions for the documentation.
%
% \section{Installation}
% %%%%%%%%%%%%%%%%%%%%%%
% Either you are a package manager and then you'll know how to
% prepare an installation package for \exsol{}.
%
% Either you are a normal user and then you have two options. First,
% check if there is a package that your favorite \LaTeX{}
% distributor has prepared for you. Second, grab the TDS package
% from CTAN \cite{CTAN} (\texttt{exsol.tds.zip}) and unzip it somewhere in your
% own TDS tree, regenerate your filename database and off you go.
% In any case, make sure that \LaTeX{} finds the \texttt{exsol.sty} file.
%
% The \exsol{} package uses some auxiliary packages: \textsf{fancyvrb},
% \textsf{ifthen}, \textsf{kvoptions} and, optionally,
% \textsf{babel}. Fetch them from CTAN \cite{CTAN} if your \TeX{}
% distributor does not provide them.
%
% \section{Usage}
% %%%%%%%%%%%%%%%
% 
% \subsection{Preparing your document source}
% %%%%%%%%%%%%%%%%%%%%%%%%%%%%%%%%%%%%%%%%%%%
% The macro package exsol can be loaded with:
% \begin{verbatim}
% \usepackage{exsol}
% \end{verbatim}
%
% Then, you are ready to add some exercises including their solution
% to your document source. To this end, embed them in a
% \texttt{exercise} and a corresponding \texttt{solution} environment.
% Optionally, you may embed several of them in a \texttt{exercises}
% environment, to make them stand out in your text.
% E.g.,
%
% \begin{VerbatimOut}{exsol.tmp}
% 
% \begin{exercises}
%
%   \begin{exercise}
%     Solve the following equation for $x \in C$, with $C$ the set of
%     complex numbers:
%     \begin{equation}
%       5 x^2 -3 x = 5
%     \end{equation}
%   \end{exercise}
%   \begin{solution}
%     Let's start by rearranging the equation, a bit:
%     \begin{eqnarray}
%       5.7 x^2 - 3.1 x &=& 5.3\\
%       5.7 x^2 - 3.1 x -5.3 &=& 0
%     \end{eqnarray}
%     The equation is now in the standard form:
%     \begin{equation}
%       a x^2 + b x + c = 0
%     \end{equation}
%     For quadratic equations in the standard form, we know that two
%     solutions exist:
%     \begin{equation}
%       x_{1,2} = \frac{ -b \pm \sqrt{d}}{2a}
%     \end{equation}
%     with
%     \begin{equation}
%       d = b^2 - 4 a c
%     \end{equation}
%     If we apply this to our case, we obtain:
%     \begin{equation}
%       d = (-3.1)^2 - 4 \cdot 5.7 \cdot (-5.3) = 130.45
%     \end{equation}
%     and
%     \begin{eqnarray}
%       x_1 &=& \frac{3.1 + \sqrt{130.45}}{11.4} = 1.27\\
%       x_2 &=& \frac{3.1 - \sqrt{130.45}}{11.4} = -0.73
%     \end{eqnarray}
%     The proposed values $x = x_1, x_2$ are solutions to the given equation.
%   \end{solution}
%   \begin{exercise}
%     Consider a 2-dimensional vector space equipped with a Euclidean
%     distance function. Given a right-angled triangle, with the sides
%     $A$ and $B$ adjacent to the right angle having lengths, $3$ and
%     $4$, calculate the length of the hypotenuse, labeled $C$.
%   \end{exercise}
%   \begin{solution}
%     This calls for application of Pythagoras' theorem, which 
%     tells us:
%     \begin{equation}
%       \left\|A\right\|^2 + \left\|B\right\|^2 = \left\|C\right\|^2
%     \end{equation}
%     and therefore:
%     \begin{eqnarray}
%       \left\|C\right\| 
%       &=& \sqrt{\left\|A\right\|^2 + \left\|B\right\|^2}\\
%       &=& \sqrt{3^2 + 4^2}\\
%       &=& \sqrt{25} = 5
%     \end{eqnarray}
%     Therefore, the length of the hypotenuse equals $5$.
%   \end{solution}
%
% \end{exercises}
% \end{VerbatimOut}
% \VerbatimInput[frame=lines,gobble=2,fontsize=\footnotesize]{exsol.tmp}
%
% The result in the original document, can be seen below. As you can
% see, there's no trace of the solution. 
%
% \input{exsol.tmp}
%
% When running \LaTeX{} on your document (in our case on the
% \texttt{exsol.dtx} file, as a side effect a file with extension
% \texttt{.sol.tex} has been written to disk (in our case, the file
% \texttt{exsol.sol.tex}), containing all solutions in sequence.
%
% Generating a solution book is a simple as including the file into a
% simple \LaTeX{} harness, that allows you giving it a proper title page and to
% add other bells and whistles.
%
% E.g.,
% \begin{VerbatimOut}{exsol-solutionbook.tex}
% \documentclass{article}
% \usepackage[english]{babel}
% \title{Solutions to the exercises, specified in the \textsf{ExSol} package}
% \author{Walter Daems}
% \date{2013/05/12}
%
% \begin{document}
%
% \maketitle
%
% \input{exsol.sol}
%
% \end{document}
% \end{VerbatimOut}
% \VerbatimInput[frame=lines,gobble=2,fontsize=\footnotesize]{exsol-solutionbook.tex}
% 
% You may generate this solution book, by running \LaTeX{} on the
% file named \texttt{exsol-solutionbook.tex} that is generated when running
% \LaTeX{} on the \texttt{exsol.dtx} file.
%
% The result approximately looks like this:
%
% \setcounter{equation}{0}
% \rule{\linewidth}{.7pt}
% \begin{center}
% {\Large Solutions to the exercises, specified in the \textsf{ExSol} package}\\
% {\large Walter Daems}\\
% {\large 2013/05/12}
% \end{center}
% \par---\newline\textbf{Solution 3.1-1}
%     Let's start by rearranging the equation, a bit:
%     \begin{eqnarray}
%       5.7 x^2 - 3.1 x &=% 5.3\\
%       5.7 x^2 - 3.1 x -5.3 &=% 0
%     \end{eqnarray}
%     The equation is now in the standard form:
%     \begin{equation}
%       a x^2 + b x + c = 0
%     \end{equation}
%     For quadratic equations in the standard form, we know that two
%     solutions exist:
%     \begin{equation}
%       x_{1,2} = \frac{ -b \pm \sqrt{d}}{2a}
%     \end{equation}
%     with
%     \begin{equation}
%       d = b^2 - 4 a c
%     \end{equation}
%     If we apply this to our case, we obtain:
%     \begin{equation}
%       d = (-3.1)^2 - 4 \cdot 5.7 \cdot (-5.3) = 130.45
%     \end{equation}
%     and
%     \begin{eqnarray}
%       x_1 &=& \frac{3.1 + \sqrt{130.45}}{11.4} = 1.27\\
%       x_2 &=& \frac{3.1 + \sqrt{130.45}}{11.4} = -0.73
%     \end{eqnarray}
%     The proposed values $x = x_1, x_2$ are solutions to
%     the given equation.
% \par---\newline\textbf{Solution 3.1-2}
%       This calls for application of Pythagoras' theorem, which
%       tells us:
%       \begin{equation}
%         \left\|A\right\|^2 + \left\|B\right\|^2 = \left\|C\right\|^2
%       \end{equation}
%       and therefore:
%       \begin{eqnarray}
%         \left\|C\right\|
%         &=& \sqrt{\left\|A\right\|^2 + \left\|B\right\|^2}\\
%         &=& \sqrt{3^2 + 4^2}\\
%         &=& \sqrt{25} = 5
%       \end{eqnarray}
%       Therefore, the length of the hypotenuse equals $5$.
%
% \rule{\linewidth}{.7pt}
%
% \subsection{Fiddling with the spacing}
%
% The default spacing provided by the \textsf{ExSol} package should be
% fine for most users. However, if you like to tweak, below you can
% find the controls.
% \subsubsection{Spacing before and after the \texttt{exercises} environment}
%
% The lengths below control the spacing of the |exercises| environment:
% \begin{itemize}
% \item |exsolexerciseaboveskip|: rubber length controlling the
% vertical space after the top marker line of the environment
% \item |exsolexercisebelowskip|: rubber length controlling the
% vertical space before the bottom marker line of the environment
% \end{itemize}
%
% You can simply specify them like:
% \begin{verbatim}
% \setlength{\exsolexercisesaboveskip}{1ex plus 1pt minus 1pt}
% \setlength{\exsolexercisesbelowskip}{1ex plus 1pt minus 1pt}
% \end{verbatim}
% The spacings specified here are the package defaults.
%
% \subsubsection{Spacing of the individual exercises}
% Caution: the spacing can only be tuned, when one invokes the
% |exerciseaslist| package option!
%
% Then lengths below control the spacing of the |exercise| environment:
% \begin{itemize}
% \item |exercisetopbottomsep|: rubber length controlling the vertical
% space before and after individual exercises
% \item |exerciseleftmargin|: length controlling the horizontal
% space between the surrounding environment's left margin (most
% often the page margin) and the left edge of the exercise
% environment 
% \item |exerciseleftmargin|: length controlling the horizontal
% space between the surrounding environment's right margin (most
% often the page margin) and the right edge of the exercise
% environment
% \item |exerciseitemindent|: length controlling the first-line
% indentation of the first paragraph in the exercise environment
% (actually, the label is set w.r.t. this position, that we will
% conveniently call position 'x')
% \item |exerciseparindent|: length controlling the first-line
% indentation of the other paragraphs in the exercise environment.
% \item |exerciselabelsep|: length controlling the distance between
% the label and position 'x'
% \item |exerciselabelwidth|: minimal width of the (internally
% right-alligned) box to use for the exercises label; if the box is
% not sufficiently big, position 'x' is shifted to the right
% \item |exerciseparsep|: internal paragraph separation (vertically)
% \end{itemize}
% 
% You can simply specify them like:
% \begin{verbatim}
% \setlength{\exsolexercisetopbottomsep}{0pt plus 0pt minus 1pt}
% \setlength{\exsolexerciseleftmargin}{1em}
% \setlength{\exsolexerciserightmargin}{1em}
% \setlength{\exsolexerciseparindent}{0em}
% \setlength{\exsolexerciselabelsep}{0.5em}
% \setlength{\exsolexerciselabelwidth}{0pt}
% \setlength{\exsolexerciseitemindent}{0pt}
% \setlength{\exsolexerciseparsep}{\parskip}
% \end{verbatim}
% The spacings specified here are the package defaults.
%
% \subsection{Tips and tricks}
%
% If you want to include the solutions all at the
% end of the current document, you need to explicitly close the
% solution stream before including it:
% \begin{verbatim}
%   \closeout\solutionstream\input{\jobname.sol.tex}
% \end{verbatim}
%
% If you want to avoid exercises being split by a page boundary, then
% provide the package option 'minipage'. This causes the exercises to
% be wrapped in a minipage environment.
% 
% \clearpage
%
% \section{Implementation}
% %%%%%%%%%%%%%%%%%%%%%%%
%    \begin{macrocode}
%<*package>
%    \end{macrocode}
%
% \subsection{Auxiliary packages}
% %%%%%%%%%%%%%%%%%%%%%%%%%%%%%%%
% The package uses some auxiliary packages:
%    \begin{macrocode}
\RequirePackage{fancyvrb}
\RequirePackage{ifthen}
\RequirePackage{kvoptions}
%    \end{macrocode}
%
% \subsection{Package options}
% %%%%%%%%%%%%%%%%%%%%%%%%%%%%
% The package offers some options:
%
% \changes{v0.2}{2012/01/06}{Added option exercisesfont}
% \changes{v0.4}{2012/01/09}{Changed name of option to exercisesfontsize}
%
% \begin{macro}{exercisesfontsize}
%  This option allows setting the font of the \texttt{exercises}
%  environment. You may chopse one of tiny, scriptsize, footnotesize,
%  small, normalsize, large, etc.\\
%  E.g., \texttt{[exercisesfontsize=small]}.
%    \begin{macrocode}
\DeclareStringOption[normalsize]{exercisesfontsize}
%    \end{macrocode}
% \end{macro}
%
% \changes{v0.4}{2012/01/06}{Added option exercisesinlist}
% \changes{v0.5}{2012/01/09}{Changed option exercisesinlist to exerciseaslist}
%
% \begin{macro}{exerciseaslist}
%  This boolean option (true, false) allows setting the typesetting of
%  the \texttt{exercises} in a list environment. This causes the
%  exercises to be typeset in a more compact fashion, with indented
%  left and right margin. 
%    \begin{macrocode}
\DeclareBoolOption[false]{exerciseaslist}
%    \end{macrocode}
% \end{macro}
%
% \changes{v0.5}{2012/01/09}{Added option copyexercisesinsolutions}
% \begin{macro}{copyexercisesinsolutions}
%  This boolean option (true, false) allows copying the exercises in
%  the solutions file, to allow for making a complete stand-alone
%  exercises bundle.
%    \begin{macrocode}
\DeclareBoolOption[false]{copyexercisesinsolutions}
%    \end{macrocode}
% \end{macro}
%
% \changes{v0.9}{2014/07/28}{. Changed default behavior
% w.r.t. minipage-wraping of exercises}
% \begin{macro}{minipage}
%  This boolean option (true, false) causes the exercises to be
%  wrapped in minipages. This avoids them getting split by a page
%  boundary.
%    \begin{macrocode}
\DeclareBoolOption[false]{minipage}
%    \end{macrocode}
% \end{macro}
%
% The options are processed using:
%    \begin{macrocode}
\ProcessKeyvalOptions*
%    \end{macrocode}
%
% The options are subsequently handled
%    \begin{macrocode}
\newcommand{\exercisesfontsize}{\csname \exsol@exercisesfontsize\endcsname}
%    \end{macrocode}
%
%
% \subsection{Customization of lengths}
% %%%%%%%%%%%%%%%%%%%%%%%%%%%%%%%%%%%%%%%
% The commands below allow customizing many lengths that control the
% typesetting of the exercises.
%
% \changes{v0.91}{2014/08/31}{added user-accessible lengths}
% First some lengths to control the spacing before and after |exercises|.
%    \begin{macrocode}
\newlength{\exsolexercisesaboveskip}
\setlength{\exsolexercisesaboveskip}{1ex plus 1pt minus 1pt}
\newlength{\exsolexercisesbelowskip}
\setlength{\exsolexercisesbelowskip}{1ex plus 1pt minus 1pt}
%    \end{macrocode}
%
% Then some lengths to control the spacing for a single
% exercise. These lengths only work when the |exerciseaslist| package
% option has been specified. Sensible defaults have been set.
%    \begin{macrocode}
\newlength{\exsolexercisetopbottomsep}
\setlength{\exsolexercisetopbottomsep}{0pt plus 0pt minus 1pt}
\newlength{\exsolexerciseleftmargin}
\setlength{\exsolexerciseleftmargin}{1em}
\newlength{\exsolexerciserightmargin}
\setlength{\exsolexerciserightmargin}{1em}
\newlength{\exsolexerciseparindent}
\setlength{\exsolexerciseparindent}{0em}
\newlength{\exsolexerciselabelsep}
\setlength{\exsolexerciselabelsep}{0.5em}
\newlength{\exsolexerciselabelwidth}
\setlength{\exsolexerciselabelwidth}{0pt}
\newlength{\exsolexerciseitemindent}
\setlength{\exsolexerciseitemindent}{0pt}
\newlength{\exsolexerciseparsep}
\setlength{\exsolexerciseparsep}{\parskip}
%    \end{macrocode}
% 
% 
% \subsection{Con- and destruction of the auxiliary streams}
% %%%%%%%%%%%%%%%%%%%%%%%%%%%%%%%%%%%%%%%%%%%%%%%%%%%%%%%%%%
% At the beginning of your document, we start by opening a stream to a
% file that will be used to write the solutions to. At the end of your
% document, the package closes the stream.
% \changes{v0.8}{2014/07/15}{moved newwrite of exercise stream to this
% spot to avoid consuming all handles}
%    \begin{macrocode}
\AtBeginDocument{
  \newwrite\solutionstream
  \immediate\openout\solutionstream=\jobname.sol.tex
  \newwrite\exercisestream
}
\AtEndDocument{
  \immediate\closeout\solutionstream
}
%    \end{macrocode}
%
% \subsection{Exercises counter}
% %%%%%%%%%%%%%%%%%%%%%%%%%%%%%%
% By providing an exercise counter, proper numbering of the exercises
% is provided to allow for good cross referencing of the solutions to
% the exercises.
% \changes{v0.2}{2012/01/06}{Removed dash in counter when in document
% without sectioning commands}
%    \begin{macrocode}
\newcounter{exercise}[subsection]
\setcounter{exercise}{0}
\renewcommand{\theexercise}{%
  \@ifundefined{c@chapter}{}{\if0\arabic{chapter}\else\arabic{chapter}.\fi}%
  \if0\arabic{section}\else\arabic{section}\fi%
  \if0\arabic{subsection}\else.\arabic{subsection}\fi%
  \if0\arabic{subsubsection}\else.\arabic{subsubsection}\fi%
  \if0\arabic{exercise}\else%
    \@ifundefined{c@chapter}%
                 {\if0\arabic{section}\else-\fi}%
                 {-}%
    \arabic{exercise}%
  \fi
}
%    \end{macrocode}
%
%
%
% \subsection{Detokenization in order to cope with utf8}
%
% Combining old-school \LaTeX{} (before \XeTeX{} and \LuaTeX{}) and
% UTF-8 is a pain.
% Detokenization has been suggested by Geoffrey Poore to solve issues
% with UTF-8 characters messing up the |fancyvrb| internals.
% \changes{v0.7}{2014/07/14}{Added detokenized writing}
%    \begin{macrocode}
\newcommand{\GPES@write@detok}[1]{%
  \immediate\write\exercisestream{\detokenize{#1}}}%
\newcommand{\GPSS@write@detok}[1]{%
  \immediate\write\solutionstream{\detokenize{#1}}}%
\newcommand{\GPESS@write@detok}[1]{%
  \GPES@write@detok{#1}%
  \GPSS@write@detok{#1}}%
%    \end{macrocode}
%
%
% \section{The user environments}
%
% \begin{macro}{exercise}
%   The \texttt{exercise} environment is used to typeset your
%   exercises, provide them with a nice label and allow for copying
%   the exercise to the solutions file (if the package option
%   \texttt{copyexercisesinsolution}) is set. The label can be
%   set by redefining the \cs{exercisename} macro, or by relying on
%   the \textsf{Babel} provisions. The code is almost litteraly
%   taken from the \textsf{fancyvrb} package.
%    \begin{macrocode}
\def\exercise{\FV@Environment{}{exercise}}
\def\FVB@exercise{%
  \refstepcounter{exercise}%
  \immediate\openout\exercisestream=\jobname.exc.tex
  \ifexsol@copyexercisesinsolutions
    \typeout{Writing exercise to \jobname.sol.tex}
    \immediate\write\solutionstream{\string\par---\string\newline
      \string\textbf\string{\exercisename{} \theexercise \string}}
  \else
    \immediate\write\solutionstream{\string\par---\string\newline}
  \fi
  \immediate\write\exercisestream{\string\begin{exsol@exercise}}
  \@bsphack
  \begingroup
    \FV@UseKeyValues
    \FV@DefineWhiteSpace
    \def\FV@Space{\space}%
    \FV@DefineTabOut
    \ifexsol@copyexercisesinsolutions
      \let\FV@ProcessLine\GPESS@write@detok %
    \else
      \let\FV@ProcessLine\GPES@write@detok %
    \fi
    \relax
    \let\FV@FontScanPrep\relax
    \let\@noligs\relax
    \FV@Scan
  }
\def\FVE@exercise{
  \endgroup\@esphack
  \immediate\write\exercisestream{\string\end{exsol@exercise}}
  \ifexsol@copyexercisesinsolutions
    \immediate\write\solutionstream{\string~\string\newline}
  \fi
  \immediate\closeout\exercisestream
  \input{\jobname.exc.tex}
}
\DefineVerbatimEnvironment{exercise}{exercise}{}
%    \end{macrocode}
% \end{macro}
%
% \begin{macro}{exsol@exercise}
%   The \texttt{exsol@exercise} environment is an internal macro used
%   to typeset your exercises and provide them with a nice label and
%   number. Do not use it directly. Use the proper environment
%   \texttt{exercise} instead.
%   \changes{v0.2}{2012/01/06}{Attempted to fix MiKTeX formatting problems}
%   \changes{v0.3}{2012/01/08}{Fixed labelsep to avoid cluttered
%   itemize environments}
%   \changes{v0.4}{2012/01/06}{Added option exercisesinlist such that
%   default results in non list formatting of exercise}
%   \changes{v0.5}{2012/01/09}{Changed implementation to allow for
%   copying the exercises to the solutions file.}
%    \begin{macrocode}
\newenvironment{exsol@exercise}[0]
{%
  \ifthenelse{\boolean{exsol@minipage}}{\begin{minipage}[t]{\textwidth}}{}%
    \ifthenelse{\boolean{exsol@exerciseaslist}}
               {\begin{list}%
                   {%
                   }%
                   {%
                     \setlength{\topsep}{\exsolexercisetopbottomsep}%
                     \setlength{\leftmargin}{\exsolexerciseleftmargin}%
                     \setlength{\rightmargin}{\exsolexerciserightmargin}%
                     \setlength{\listparindent}{\exsolexerciseparindent}%
                     \setlength{\itemindent}{\exsolexerciseitemindent}%
                     \setlength{\parsep}{\exsolexerciseparsep}
                     \setlength{\labelsep}{\exsolexerciselabelsep}
                     \setlength{\labelwidth}{\exsolexerciselabelwidth}}
                 \item[\textit{~\exercisename{} \theexercise:~}]
               }%
               {\textit{\exercisename{} \theexercise:}}
}
{%
  \ifthenelse{\boolean{exsol@exerciseaslist}}%
             {\end{list}}{}%
  \ifthenelse{\boolean{exsol@minipage}}{\end{minipage}}{\par}%
}
%    \end{macrocode}
% \end{macro}
%
%
% \begin{macro}{solution}
%   The \texttt{solution} environment is used to typeset your solutions
%   and provide them with a nice label and number that corresponds to
%   the exercise that preceeded this solution. Theno label can be
%   set by redefining the \cs{solutionname} macro, or by relying on
%   the \textsf{Babel} provisions. The code is almost litteraly
%   taken from the \textsf{fancyvrb} package.
%    \begin{macrocode}
\def\solution{\FV@Environment{}{solution}}
\def\FVB@solution{%
  \typeout{Writing solution to \jobname.sol.tex}
  \immediate\write\solutionstream{\string\textbf\string{\solutionname{}\string}}
  \ifexsol@copyexercisesinsolutions
    \immediate\write\solutionstream{\string\newline}
  \else
    \immediate\write\solutionstream{\string\textbf\string{\theexercise\string}%
                                    \string\newline}
  \fi
  \@bsphack
  \begingroup
    \FV@UseKeyValues
    \FV@DefineWhiteSpace
    \def\FV@Space{\space}%
    \FV@DefineTabOut
    \let\FV@ProcessLine\GPSS@write@detok %
    \relax
    \let\FV@FontScanPrep\relax
    \let\@noligs\relax
    \FV@Scan
  }
\def\FVE@solution{\endgroup\@esphack}
\DefineVerbatimEnvironment{solution}{solution}{}
%    \end{macrocode}
% \end{macro}
%
% \begin{macro}{exercises}
%   The \texttt{exercises} environment helps typesetting your exercises to
%   stand out from the rest of the text. You may use it at the end of
%   a chapter, or just to group some exercises in the text.
%   \changes{v0.2}{2012/01/06}{Attempted to fix MiKTeX formatting problems}
%   \changes{v0.3}{2012/01/07}{Added some extra whitespace below exercisesname}
%    \begin{macrocode}
\newenvironment{exercises}
{\par\exercisesfontsize\rule{.25\linewidth}{0.15mm}\vspace*{\exsolexercisesaboveskip}\\*%
 \textbf{\normalsize \exercisesname}}
{\vspace*{-\baselineskip}\vspace*{\exsolexercisesbelowskip}\rule{.25\linewidth}{0.15mm}\par}
%    \end{macrocode}
% \end{macro}
%
% \subsection{Some Babel provisions}
% %%%%%%%%%%%%%%%%%%%%%%%%%%%%%%%%%%
% \changes{v0.2}{2012/01/06}{Fixed babel errors}
% \begin{macro}{\exercisename}
%   The exercise environment makes use of a label \texttt{\exercisename{}}
%   macro.
%    \begin{macrocode}
\newcommand{\exercisename}{Exercise}
%    \end{macrocode}
% \end{macro}
%
% \begin{macro}{\exercisesname}
%   The exercises environment makes use of a label \texttt{\exercisesname{}}
%   macro.
%    \begin{macrocode}
\newcommand{\exercisesname}{Exercises}
%    \end{macrocode}
% \end{macro}
% 
% \begin{macro}{\solutionname}
%   The solution environment makes use of a label \texttt{\solutionname{}}
%   macro.
%    \begin{macrocode}
\newcommand{\solutionname}{Solution}
%    \end{macrocode}
% \end{macro}
%
% \begin{macro}{\solutionname}
%   The solution environment makes use of a label \texttt{\solutionname{}}
%   macro.
% \changes{v0.8}{2014/07/15}{Added missing babel tag}
%    \begin{macrocode}
\newcommand{\solutionsname}{Solutions}
%    \end{macrocode}
% \end{macro}
% 
% 
% You may redefine these macros, but to help you out a little bit, we
% provide with some basic Babel auxiliaries. If you're a true polyglot
% and are willing to help me out by providing translations for other
% languages, I'm very willing to incorporate them into the code.
%
% \changes{v0.7}{2014/07/14}{Added Finnish language support}
%    \begin{macrocode}
\addto\captionsdutch{%
  \renewcommand{\exercisename}{Oefening}%
  \renewcommand{\exercisesname}{Oefeningen}%
  \renewcommand{\solutionname}{Oplossing}%
  \renewcommand{\solutionsname}{Oplossingen}%
}
\addto\captionsgerman{%
  \renewcommand{\exercisename}{Aufgabe}%
  \renewcommand{\exercisesname}{Aufgaben}%
  \renewcommand{\solutionname}{L\"osung}%
  \renewcommand{\solutionsname}{L\"osungen}%
}
\addto\captionsfrench{%
  \renewcommand{\exercisename}{Exercice}%
  \renewcommand{\exercisesname}{Exercices}%
  \renewcommand{\solutionname}{Solution}%
  \renewcommand{\solutionsname}{Solutions}%
}
\addto\captionsfinnish{
  \renewcommand{\exercisename}{Teht\"av\"a}%
  \renewcommand{\exercisesname}{Teht\"avi\"a}%
  \renewcommand{\solutionname}{Ratkaisu}%
  \renewcommand{\solutionsname}{Ratkaisut}%
}
%    \end{macrocode}
%
%
%
% Now the final hack overloads the basic sectioning commands to make
% sure that they are copied into your solution book.
%
%    \begin{macrocode}
\let\exsol@@makechapterhead\@makechapterhead
\def\@makechapterhead#1{%
  \immediate\write\solutionstream{\string\chapter{#1}}%
  \exsol@@makechapterhead{#1}
}
\ifdefined\frontmatter
  \let\exsol@@frontmatter\frontmatter
  \def\frontmatter{%
    \immediate\write\solutionstream{\string\frontmatter}%
    \exsol@@frontmatter
  }
\fi
\ifdefined\frontmatter
  \let\exsol@@mainmatter\mainmatter
  \def\mainmatter{%
    \immediate\write\solutionstream{\string\mainmatter}%
    \exsol@@mainmatter
  }
\fi
\ifdefined\backmatter
  \let\exsol@@backmatter\backmatter
  \def\backmatter{%
    \immediate\write\solutionstream{\string\backmatter}%
    \exsol@@backmatter
  }
\fi
%    \end{macrocode}
%
% \begin{macro}{\noexercisesinchapter}
%   If you have chapters without exercises, you may want to indicate
%   this clearly into your source. Otherwise empty chapters may appear
%   in your solution book.
%    \begin{macrocode}
\newcommand{\noexercisesinchapter}
{
  \immediate\write\solutionstream{No exercises in this chapter}
}
%    \end{macrocode}
% \end{macro}
%
%    \begin{macrocode}
%</package>
%    \end{macrocode}
%
% \bibliographystyle{alpha}
%
% \begin{thebibliography}{99}
%
% \bibitem{fancyvrb}
% Timothy Van Zandt, Herbert Vo\ss, Denis Girou, Sebastian Rahtz, Niall
% Mansfield 
% \newblock The \texttt{fancyvrb} package.
% \newblock \url{http://ctan.org/pkg/fancyvrb}.
% \newblock online, accessed in January 2012.
%
% \bibitem{CTAN} 
% The Comprehensive TeX Archive Network.
% \newblock \url{http://www.ctan.org}.
% \newblock online, accessed in January 2012.
%
% \end{thebibliography}
%
% \Finale
\endinput

%
% \end{document}
% \end{VerbatimOut}
% \VerbatimInput[frame=lines,gobble=2,fontsize=\footnotesize]{exsol-solutionbook.tex}
% 
% You may generate this solution book, by running \LaTeX{} on the
% file named \texttt{exsol-solutionbook.tex} that is generated when running
% \LaTeX{} on the \texttt{exsol.dtx} file.
%
% The result approximately looks like this:
%
% \setcounter{equation}{0}
% \rule{\linewidth}{.7pt}
% \begin{center}
% {\Large Solutions to the exercises, specified in the \textsf{ExSol} package}\\
% {\large Walter Daems}\\
% {\large 2013/05/12}
% \end{center}
% \par---\newline\textbf{Solution 3.1-1}
%     Let's start by rearranging the equation, a bit:
%     \begin{eqnarray}
%       5.7 x^2 - 3.1 x &=% 5.3\\
%       5.7 x^2 - 3.1 x -5.3 &=% 0
%     \end{eqnarray}
%     The equation is now in the standard form:
%     \begin{equation}
%       a x^2 + b x + c = 0
%     \end{equation}
%     For quadratic equations in the standard form, we know that two
%     solutions exist:
%     \begin{equation}
%       x_{1,2} = \frac{ -b \pm \sqrt{d}}{2a}
%     \end{equation}
%     with
%     \begin{equation}
%       d = b^2 - 4 a c
%     \end{equation}
%     If we apply this to our case, we obtain:
%     \begin{equation}
%       d = (-3.1)^2 - 4 \cdot 5.7 \cdot (-5.3) = 130.45
%     \end{equation}
%     and
%     \begin{eqnarray}
%       x_1 &=& \frac{3.1 + \sqrt{130.45}}{11.4} = 1.27\\
%       x_2 &=& \frac{3.1 + \sqrt{130.45}}{11.4} = -0.73
%     \end{eqnarray}
%     The proposed values $x = x_1, x_2$ are solutions to
%     the given equation.
% \par---\newline\textbf{Solution 3.1-2}
%       This calls for application of Pythagoras' theorem, which
%       tells us:
%       \begin{equation}
%         \left\|A\right\|^2 + \left\|B\right\|^2 = \left\|C\right\|^2
%       \end{equation}
%       and therefore:
%       \begin{eqnarray}
%         \left\|C\right\|
%         &=& \sqrt{\left\|A\right\|^2 + \left\|B\right\|^2}\\
%         &=& \sqrt{3^2 + 4^2}\\
%         &=& \sqrt{25} = 5
%       \end{eqnarray}
%       Therefore, the length of the hypotenuse equals $5$.
%
% \rule{\linewidth}{.7pt}
%
% \subsection{Fiddling with the spacing}
%
% The default spacing provided by the \textsf{ExSol} package should be
% fine for most users. However, if you like to tweak, below you can
% find the controls.
% \subsubsection{Spacing before and after the \texttt{exercises} environment}
%
% The lengths below control the spacing of the |exercises| environment:
% \begin{itemize}
% \item |exsolexerciseaboveskip|: rubber length controlling the
% vertical space after the top marker line of the environment
% \item |exsolexercisebelowskip|: rubber length controlling the
% vertical space before the bottom marker line of the environment
% \end{itemize}
%
% You can simply specify them like:
% \begin{verbatim}
% \setlength{\exsolexercisesaboveskip}{1ex plus 1pt minus 1pt}
% \setlength{\exsolexercisesbelowskip}{1ex plus 1pt minus 1pt}
% \end{verbatim}
% The spacings specified here are the package defaults.
%
% \subsubsection{Spacing of the individual exercises}
% Caution: the spacing can only be tuned, when one invokes the
% |exerciseaslist| package option!
%
% Then lengths below control the spacing of the |exercise| environment:
% \begin{itemize}
% \item |exercisetopbottomsep|: rubber length controlling the vertical
% space before and after individual exercises
% \item |exerciseleftmargin|: length controlling the horizontal
% space between the surrounding environment's left margin (most
% often the page margin) and the left edge of the exercise
% environment 
% \item |exerciseleftmargin|: length controlling the horizontal
% space between the surrounding environment's right margin (most
% often the page margin) and the right edge of the exercise
% environment
% \item |exerciseitemindent|: length controlling the first-line
% indentation of the first paragraph in the exercise environment
% (actually, the label is set w.r.t. this position, that we will
% conveniently call position 'x')
% \item |exerciseparindent|: length controlling the first-line
% indentation of the other paragraphs in the exercise environment.
% \item |exerciselabelsep|: length controlling the distance between
% the label and position 'x'
% \item |exerciselabelwidth|: minimal width of the (internally
% right-alligned) box to use for the exercises label; if the box is
% not sufficiently big, position 'x' is shifted to the right
% \item |exerciseparsep|: internal paragraph separation (vertically)
% \end{itemize}
% 
% You can simply specify them like:
% \begin{verbatim}
% \setlength{\exsolexercisetopbottomsep}{0pt plus 0pt minus 1pt}
% \setlength{\exsolexerciseleftmargin}{1em}
% \setlength{\exsolexerciserightmargin}{1em}
% \setlength{\exsolexerciseparindent}{0em}
% \setlength{\exsolexerciselabelsep}{0.5em}
% \setlength{\exsolexerciselabelwidth}{0pt}
% \setlength{\exsolexerciseitemindent}{0pt}
% \setlength{\exsolexerciseparsep}{\parskip}
% \end{verbatim}
% The spacings specified here are the package defaults.
%
% \subsection{Tips and tricks}
%
% If you want to include the solutions all at the
% end of the current document, you need to explicitly close the
% solution stream before including it:
% \begin{verbatim}
%   \closeout\solutionstream\input{\jobname.sol.tex}
% \end{verbatim}
%
% If you want to avoid exercises being split by a page boundary, then
% provide the package option 'minipage'. This causes the exercises to
% be wrapped in a minipage environment.
% 
% \clearpage
%
% \section{Implementation}
% %%%%%%%%%%%%%%%%%%%%%%%
%    \begin{macrocode}
%<*package>
%    \end{macrocode}
%
% \subsection{Auxiliary packages}
% %%%%%%%%%%%%%%%%%%%%%%%%%%%%%%%
% The package uses some auxiliary packages:
%    \begin{macrocode}
\RequirePackage{fancyvrb}
\RequirePackage{ifthen}
\RequirePackage{kvoptions}
%    \end{macrocode}
%
% \subsection{Package options}
% %%%%%%%%%%%%%%%%%%%%%%%%%%%%
% The package offers some options:
%
% \changes{v0.2}{2012/01/06}{Added option exercisesfont}
% \changes{v0.4}{2012/01/09}{Changed name of option to exercisesfontsize}
%
% \begin{macro}{exercisesfontsize}
%  This option allows setting the font of the \texttt{exercises}
%  environment. You may chopse one of tiny, scriptsize, footnotesize,
%  small, normalsize, large, etc.\\
%  E.g., \texttt{[exercisesfontsize=small]}.
%    \begin{macrocode}
\DeclareStringOption[normalsize]{exercisesfontsize}
%    \end{macrocode}
% \end{macro}
%
% \changes{v0.4}{2012/01/06}{Added option exercisesinlist}
% \changes{v0.5}{2012/01/09}{Changed option exercisesinlist to exerciseaslist}
%
% \begin{macro}{exerciseaslist}
%  This boolean option (true, false) allows setting the typesetting of
%  the \texttt{exercises} in a list environment. This causes the
%  exercises to be typeset in a more compact fashion, with indented
%  left and right margin. 
%    \begin{macrocode}
\DeclareBoolOption[false]{exerciseaslist}
%    \end{macrocode}
% \end{macro}
%
% \changes{v0.5}{2012/01/09}{Added option copyexercisesinsolutions}
% \begin{macro}{copyexercisesinsolutions}
%  This boolean option (true, false) allows copying the exercises in
%  the solutions file, to allow for making a complete stand-alone
%  exercises bundle.
%    \begin{macrocode}
\DeclareBoolOption[false]{copyexercisesinsolutions}
%    \end{macrocode}
% \end{macro}
%
% \changes{v0.9}{2014/07/28}{. Changed default behavior
% w.r.t. minipage-wraping of exercises}
% \begin{macro}{minipage}
%  This boolean option (true, false) causes the exercises to be
%  wrapped in minipages. This avoids them getting split by a page
%  boundary.
%    \begin{macrocode}
\DeclareBoolOption[false]{minipage}
%    \end{macrocode}
% \end{macro}
%
% The options are processed using:
%    \begin{macrocode}
\ProcessKeyvalOptions*
%    \end{macrocode}
%
% The options are subsequently handled
%    \begin{macrocode}
\newcommand{\exercisesfontsize}{\csname \exsol@exercisesfontsize\endcsname}
%    \end{macrocode}
%
%
% \subsection{Customization of lengths}
% %%%%%%%%%%%%%%%%%%%%%%%%%%%%%%%%%%%%%%%
% The commands below allow customizing many lengths that control the
% typesetting of the exercises.
%
% \changes{v0.91}{2014/08/31}{added user-accessible lengths}
% First some lengths to control the spacing before and after |exercises|.
%    \begin{macrocode}
\newlength{\exsolexercisesaboveskip}
\setlength{\exsolexercisesaboveskip}{1ex plus 1pt minus 1pt}
\newlength{\exsolexercisesbelowskip}
\setlength{\exsolexercisesbelowskip}{1ex plus 1pt minus 1pt}
%    \end{macrocode}
%
% Then some lengths to control the spacing for a single
% exercise. These lengths only work when the |exerciseaslist| package
% option has been specified. Sensible defaults have been set.
%    \begin{macrocode}
\newlength{\exsolexercisetopbottomsep}
\setlength{\exsolexercisetopbottomsep}{0pt plus 0pt minus 1pt}
\newlength{\exsolexerciseleftmargin}
\setlength{\exsolexerciseleftmargin}{1em}
\newlength{\exsolexerciserightmargin}
\setlength{\exsolexerciserightmargin}{1em}
\newlength{\exsolexerciseparindent}
\setlength{\exsolexerciseparindent}{0em}
\newlength{\exsolexerciselabelsep}
\setlength{\exsolexerciselabelsep}{0.5em}
\newlength{\exsolexerciselabelwidth}
\setlength{\exsolexerciselabelwidth}{0pt}
\newlength{\exsolexerciseitemindent}
\setlength{\exsolexerciseitemindent}{0pt}
\newlength{\exsolexerciseparsep}
\setlength{\exsolexerciseparsep}{\parskip}
%    \end{macrocode}
% 
% 
% \subsection{Con- and destruction of the auxiliary streams}
% %%%%%%%%%%%%%%%%%%%%%%%%%%%%%%%%%%%%%%%%%%%%%%%%%%%%%%%%%%
% At the beginning of your document, we start by opening a stream to a
% file that will be used to write the solutions to. At the end of your
% document, the package closes the stream.
% \changes{v0.8}{2014/07/15}{moved newwrite of exercise stream to this
% spot to avoid consuming all handles}
%    \begin{macrocode}
\AtBeginDocument{
  \newwrite\solutionstream
  \immediate\openout\solutionstream=\jobname.sol.tex
  \newwrite\exercisestream
}
\AtEndDocument{
  \immediate\closeout\solutionstream
}
%    \end{macrocode}
%
% \subsection{Exercises counter}
% %%%%%%%%%%%%%%%%%%%%%%%%%%%%%%
% By providing an exercise counter, proper numbering of the exercises
% is provided to allow for good cross referencing of the solutions to
% the exercises.
% \changes{v0.2}{2012/01/06}{Removed dash in counter when in document
% without sectioning commands}
%    \begin{macrocode}
\newcounter{exercise}[subsection]
\setcounter{exercise}{0}
\renewcommand{\theexercise}{%
  \@ifundefined{c@chapter}{}{\if0\arabic{chapter}\else\arabic{chapter}.\fi}%
  \if0\arabic{section}\else\arabic{section}\fi%
  \if0\arabic{subsection}\else.\arabic{subsection}\fi%
  \if0\arabic{subsubsection}\else.\arabic{subsubsection}\fi%
  \if0\arabic{exercise}\else%
    \@ifundefined{c@chapter}%
                 {\if0\arabic{section}\else-\fi}%
                 {-}%
    \arabic{exercise}%
  \fi
}
%    \end{macrocode}
%
%
%
% \subsection{Detokenization in order to cope with utf8}
%
% Combining old-school \LaTeX{} (before \XeTeX{} and \LuaTeX{}) and
% UTF-8 is a pain.
% Detokenization has been suggested by Geoffrey Poore to solve issues
% with UTF-8 characters messing up the |fancyvrb| internals.
% \changes{v0.7}{2014/07/14}{Added detokenized writing}
%    \begin{macrocode}
\newcommand{\GPES@write@detok}[1]{%
  \immediate\write\exercisestream{\detokenize{#1}}}%
\newcommand{\GPSS@write@detok}[1]{%
  \immediate\write\solutionstream{\detokenize{#1}}}%
\newcommand{\GPESS@write@detok}[1]{%
  \GPES@write@detok{#1}%
  \GPSS@write@detok{#1}}%
%    \end{macrocode}
%
%
% \section{The user environments}
%
% \begin{macro}{exercise}
%   The \texttt{exercise} environment is used to typeset your
%   exercises, provide them with a nice label and allow for copying
%   the exercise to the solutions file (if the package option
%   \texttt{copyexercisesinsolution}) is set. The label can be
%   set by redefining the \cs{exercisename} macro, or by relying on
%   the \textsf{Babel} provisions. The code is almost litteraly
%   taken from the \textsf{fancyvrb} package.
%    \begin{macrocode}
\def\exercise{\FV@Environment{}{exercise}}
\def\FVB@exercise{%
  \refstepcounter{exercise}%
  \immediate\openout\exercisestream=\jobname.exc.tex
  \ifexsol@copyexercisesinsolutions
    \typeout{Writing exercise to \jobname.sol.tex}
    \immediate\write\solutionstream{\string\par---\string\newline
      \string\textbf\string{\exercisename{} \theexercise \string}}
  \else
    \immediate\write\solutionstream{\string\par---\string\newline}
  \fi
  \immediate\write\exercisestream{\string\begin{exsol@exercise}}
  \@bsphack
  \begingroup
    \FV@UseKeyValues
    \FV@DefineWhiteSpace
    \def\FV@Space{\space}%
    \FV@DefineTabOut
    \ifexsol@copyexercisesinsolutions
      \let\FV@ProcessLine\GPESS@write@detok %
    \else
      \let\FV@ProcessLine\GPES@write@detok %
    \fi
    \relax
    \let\FV@FontScanPrep\relax
    \let\@noligs\relax
    \FV@Scan
  }
\def\FVE@exercise{
  \endgroup\@esphack
  \immediate\write\exercisestream{\string\end{exsol@exercise}}
  \ifexsol@copyexercisesinsolutions
    \immediate\write\solutionstream{\string~\string\newline}
  \fi
  \immediate\closeout\exercisestream
  \input{\jobname.exc.tex}
}
\DefineVerbatimEnvironment{exercise}{exercise}{}
%    \end{macrocode}
% \end{macro}
%
% \begin{macro}{exsol@exercise}
%   The \texttt{exsol@exercise} environment is an internal macro used
%   to typeset your exercises and provide them with a nice label and
%   number. Do not use it directly. Use the proper environment
%   \texttt{exercise} instead.
%   \changes{v0.2}{2012/01/06}{Attempted to fix MiKTeX formatting problems}
%   \changes{v0.3}{2012/01/08}{Fixed labelsep to avoid cluttered
%   itemize environments}
%   \changes{v0.4}{2012/01/06}{Added option exercisesinlist such that
%   default results in non list formatting of exercise}
%   \changes{v0.5}{2012/01/09}{Changed implementation to allow for
%   copying the exercises to the solutions file.}
%    \begin{macrocode}
\newenvironment{exsol@exercise}[0]
{%
  \ifthenelse{\boolean{exsol@minipage}}{\begin{minipage}[t]{\textwidth}}{}%
    \ifthenelse{\boolean{exsol@exerciseaslist}}
               {\begin{list}%
                   {%
                   }%
                   {%
                     \setlength{\topsep}{\exsolexercisetopbottomsep}%
                     \setlength{\leftmargin}{\exsolexerciseleftmargin}%
                     \setlength{\rightmargin}{\exsolexerciserightmargin}%
                     \setlength{\listparindent}{\exsolexerciseparindent}%
                     \setlength{\itemindent}{\exsolexerciseitemindent}%
                     \setlength{\parsep}{\exsolexerciseparsep}
                     \setlength{\labelsep}{\exsolexerciselabelsep}
                     \setlength{\labelwidth}{\exsolexerciselabelwidth}}
                 \item[\textit{~\exercisename{} \theexercise:~}]
               }%
               {\textit{\exercisename{} \theexercise:}}
}
{%
  \ifthenelse{\boolean{exsol@exerciseaslist}}%
             {\end{list}}{}%
  \ifthenelse{\boolean{exsol@minipage}}{\end{minipage}}{\par}%
}
%    \end{macrocode}
% \end{macro}
%
%
% \begin{macro}{solution}
%   The \texttt{solution} environment is used to typeset your solutions
%   and provide them with a nice label and number that corresponds to
%   the exercise that preceeded this solution. Theno label can be
%   set by redefining the \cs{solutionname} macro, or by relying on
%   the \textsf{Babel} provisions. The code is almost litteraly
%   taken from the \textsf{fancyvrb} package.
%    \begin{macrocode}
\def\solution{\FV@Environment{}{solution}}
\def\FVB@solution{%
  \typeout{Writing solution to \jobname.sol.tex}
  \immediate\write\solutionstream{\string\textbf\string{\solutionname{}\string}}
  \ifexsol@copyexercisesinsolutions
    \immediate\write\solutionstream{\string\newline}
  \else
    \immediate\write\solutionstream{\string\textbf\string{\theexercise\string}%
                                    \string\newline}
  \fi
  \@bsphack
  \begingroup
    \FV@UseKeyValues
    \FV@DefineWhiteSpace
    \def\FV@Space{\space}%
    \FV@DefineTabOut
    \let\FV@ProcessLine\GPSS@write@detok %
    \relax
    \let\FV@FontScanPrep\relax
    \let\@noligs\relax
    \FV@Scan
  }
\def\FVE@solution{\endgroup\@esphack}
\DefineVerbatimEnvironment{solution}{solution}{}
%    \end{macrocode}
% \end{macro}
%
% \begin{macro}{exercises}
%   The \texttt{exercises} environment helps typesetting your exercises to
%   stand out from the rest of the text. You may use it at the end of
%   a chapter, or just to group some exercises in the text.
%   \changes{v0.2}{2012/01/06}{Attempted to fix MiKTeX formatting problems}
%   \changes{v0.3}{2012/01/07}{Added some extra whitespace below exercisesname}
%    \begin{macrocode}
\newenvironment{exercises}
{\par\exercisesfontsize\rule{.25\linewidth}{0.15mm}\vspace*{\exsolexercisesaboveskip}\\*%
 \textbf{\normalsize \exercisesname}}
{\vspace*{-\baselineskip}\vspace*{\exsolexercisesbelowskip}\rule{.25\linewidth}{0.15mm}\par}
%    \end{macrocode}
% \end{macro}
%
% \subsection{Some Babel provisions}
% %%%%%%%%%%%%%%%%%%%%%%%%%%%%%%%%%%
% \changes{v0.2}{2012/01/06}{Fixed babel errors}
% \begin{macro}{\exercisename}
%   The exercise environment makes use of a label \texttt{\exercisename{}}
%   macro.
%    \begin{macrocode}
\newcommand{\exercisename}{Exercise}
%    \end{macrocode}
% \end{macro}
%
% \begin{macro}{\exercisesname}
%   The exercises environment makes use of a label \texttt{\exercisesname{}}
%   macro.
%    \begin{macrocode}
\newcommand{\exercisesname}{Exercises}
%    \end{macrocode}
% \end{macro}
% 
% \begin{macro}{\solutionname}
%   The solution environment makes use of a label \texttt{\solutionname{}}
%   macro.
%    \begin{macrocode}
\newcommand{\solutionname}{Solution}
%    \end{macrocode}
% \end{macro}
%
% \begin{macro}{\solutionname}
%   The solution environment makes use of a label \texttt{\solutionname{}}
%   macro.
% \changes{v0.8}{2014/07/15}{Added missing babel tag}
%    \begin{macrocode}
\newcommand{\solutionsname}{Solutions}
%    \end{macrocode}
% \end{macro}
% 
% 
% You may redefine these macros, but to help you out a little bit, we
% provide with some basic Babel auxiliaries. If you're a true polyglot
% and are willing to help me out by providing translations for other
% languages, I'm very willing to incorporate them into the code.
%
% \changes{v0.7}{2014/07/14}{Added Finnish language support}
%    \begin{macrocode}
\addto\captionsdutch{%
  \renewcommand{\exercisename}{Oefening}%
  \renewcommand{\exercisesname}{Oefeningen}%
  \renewcommand{\solutionname}{Oplossing}%
  \renewcommand{\solutionsname}{Oplossingen}%
}
\addto\captionsgerman{%
  \renewcommand{\exercisename}{Aufgabe}%
  \renewcommand{\exercisesname}{Aufgaben}%
  \renewcommand{\solutionname}{L\"osung}%
  \renewcommand{\solutionsname}{L\"osungen}%
}
\addto\captionsfrench{%
  \renewcommand{\exercisename}{Exercice}%
  \renewcommand{\exercisesname}{Exercices}%
  \renewcommand{\solutionname}{Solution}%
  \renewcommand{\solutionsname}{Solutions}%
}
\addto\captionsfinnish{
  \renewcommand{\exercisename}{Teht\"av\"a}%
  \renewcommand{\exercisesname}{Teht\"avi\"a}%
  \renewcommand{\solutionname}{Ratkaisu}%
  \renewcommand{\solutionsname}{Ratkaisut}%
}
%    \end{macrocode}
%
%
%
% Now the final hack overloads the basic sectioning commands to make
% sure that they are copied into your solution book.
%
%    \begin{macrocode}
\let\exsol@@makechapterhead\@makechapterhead
\def\@makechapterhead#1{%
  \immediate\write\solutionstream{\string\chapter{#1}}%
  \exsol@@makechapterhead{#1}
}
\ifdefined\frontmatter
  \let\exsol@@frontmatter\frontmatter
  \def\frontmatter{%
    \immediate\write\solutionstream{\string\frontmatter}%
    \exsol@@frontmatter
  }
\fi
\ifdefined\frontmatter
  \let\exsol@@mainmatter\mainmatter
  \def\mainmatter{%
    \immediate\write\solutionstream{\string\mainmatter}%
    \exsol@@mainmatter
  }
\fi
\ifdefined\backmatter
  \let\exsol@@backmatter\backmatter
  \def\backmatter{%
    \immediate\write\solutionstream{\string\backmatter}%
    \exsol@@backmatter
  }
\fi
%    \end{macrocode}
%
% \begin{macro}{\noexercisesinchapter}
%   If you have chapters without exercises, you may want to indicate
%   this clearly into your source. Otherwise empty chapters may appear
%   in your solution book.
%    \begin{macrocode}
\newcommand{\noexercisesinchapter}
{
  \immediate\write\solutionstream{No exercises in this chapter}
}
%    \end{macrocode}
% \end{macro}
%
%    \begin{macrocode}
%</package>
%    \end{macrocode}
%
% \bibliographystyle{alpha}
%
% \begin{thebibliography}{99}
%
% \bibitem{fancyvrb}
% Timothy Van Zandt, Herbert Vo\ss, Denis Girou, Sebastian Rahtz, Niall
% Mansfield 
% \newblock The \texttt{fancyvrb} package.
% \newblock \url{http://ctan.org/pkg/fancyvrb}.
% \newblock online, accessed in January 2012.
%
% \bibitem{CTAN} 
% The Comprehensive TeX Archive Network.
% \newblock \url{http://www.ctan.org}.
% \newblock online, accessed in January 2012.
%
% \end{thebibliography}
%
% \Finale
\endinput

%
% \end{document}
% \end{VerbatimOut}
% \VerbatimInput[frame=lines,gobble=2,fontsize=\footnotesize]{exsol-solutionbook.tex}
% 
% You may generate this solution book, by running \LaTeX{} on the
% file named \texttt{exsol-solutionbook.tex} that is generated when running
% \LaTeX{} on the \texttt{exsol.dtx} file.
%
% The result approximately looks like this:
%
% \setcounter{equation}{0}
% \rule{\linewidth}{.7pt}
% \begin{center}
% {\Large Solutions to the exercises, specified in the \textsf{ExSol} package}\\
% {\large Walter Daems}\\
% {\large 2013/05/12}
% \end{center}
% \par---\newline\textbf{Solution 3.1-1}
%     Let's start by rearranging the equation, a bit:
%     \begin{eqnarray}
%       5.7 x^2 - 3.1 x &=% 5.3\\
%       5.7 x^2 - 3.1 x -5.3 &=% 0
%     \end{eqnarray}
%     The equation is now in the standard form:
%     \begin{equation}
%       a x^2 + b x + c = 0
%     \end{equation}
%     For quadratic equations in the standard form, we know that two
%     solutions exist:
%     \begin{equation}
%       x_{1,2} = \frac{ -b \pm \sqrt{d}}{2a}
%     \end{equation}
%     with
%     \begin{equation}
%       d = b^2 - 4 a c
%     \end{equation}
%     If we apply this to our case, we obtain:
%     \begin{equation}
%       d = (-3.1)^2 - 4 \cdot 5.7 \cdot (-5.3) = 130.45
%     \end{equation}
%     and
%     \begin{eqnarray}
%       x_1 &=& \frac{3.1 + \sqrt{130.45}}{11.4} = 1.27\\
%       x_2 &=& \frac{3.1 + \sqrt{130.45}}{11.4} = -0.73
%     \end{eqnarray}
%     The proposed values $x = x_1, x_2$ are solutions to
%     the given equation.
% \par---\newline\textbf{Solution 3.1-2}
%       This calls for application of Pythagoras' theorem, which
%       tells us:
%       \begin{equation}
%         \left\|A\right\|^2 + \left\|B\right\|^2 = \left\|C\right\|^2
%       \end{equation}
%       and therefore:
%       \begin{eqnarray}
%         \left\|C\right\|
%         &=& \sqrt{\left\|A\right\|^2 + \left\|B\right\|^2}\\
%         &=& \sqrt{3^2 + 4^2}\\
%         &=& \sqrt{25} = 5
%       \end{eqnarray}
%       Therefore, the length of the hypotenuse equals $5$.
%
% \rule{\linewidth}{.7pt}
%
% \subsection{Fiddling with the spacing}
%
% The default spacing provided by the \textsf{ExSol} package should be
% fine for most users. However, if you like to tweak, below you can
% find the controls.
% \subsubsection{Spacing before and after the \texttt{exercises} environment}
%
% The lengths below control the spacing of the |exercises| environment:
% \begin{itemize}
% \item |exsolexerciseaboveskip|: rubber length controlling the
% vertical space after the top marker line of the environment
% \item |exsolexercisebelowskip|: rubber length controlling the
% vertical space before the bottom marker line of the environment
% \end{itemize}
%
% You can simply specify them like:
% \begin{verbatim}
% \setlength{\exsolexercisesaboveskip}{1ex plus 1pt minus 1pt}
% \setlength{\exsolexercisesbelowskip}{1ex plus 1pt minus 1pt}
% \end{verbatim}
% The spacings specified here are the package defaults.
%
% \subsubsection{Spacing of the individual exercises}
% Caution: the spacing can only be tuned, when one invokes the
% |exerciseaslist| package option!
%
% Then lengths below control the spacing of the |exercise| environment:
% \begin{itemize}
% \item |exercisetopbottomsep|: rubber length controlling the vertical
% space before and after individual exercises
% \item |exerciseleftmargin|: length controlling the horizontal
% space between the surrounding environment's left margin (most
% often the page margin) and the left edge of the exercise
% environment 
% \item |exerciseleftmargin|: length controlling the horizontal
% space between the surrounding environment's right margin (most
% often the page margin) and the right edge of the exercise
% environment
% \item |exerciseitemindent|: length controlling the first-line
% indentation of the first paragraph in the exercise environment
% (actually, the label is set w.r.t. this position, that we will
% conveniently call position 'x')
% \item |exerciseparindent|: length controlling the first-line
% indentation of the other paragraphs in the exercise environment.
% \item |exerciselabelsep|: length controlling the distance between
% the label and position 'x'
% \item |exerciselabelwidth|: minimal width of the (internally
% right-alligned) box to use for the exercises label; if the box is
% not sufficiently big, position 'x' is shifted to the right
% \item |exerciseparsep|: internal paragraph separation (vertically)
% \end{itemize}
% 
% You can simply specify them like:
% \begin{verbatim}
% \setlength{\exsolexercisetopbottomsep}{0pt plus 0pt minus 1pt}
% \setlength{\exsolexerciseleftmargin}{1em}
% \setlength{\exsolexerciserightmargin}{1em}
% \setlength{\exsolexerciseparindent}{0em}
% \setlength{\exsolexerciselabelsep}{0.5em}
% \setlength{\exsolexerciselabelwidth}{0pt}
% \setlength{\exsolexerciseitemindent}{0pt}
% \setlength{\exsolexerciseparsep}{\parskip}
% \end{verbatim}
% The spacings specified here are the package defaults.
%
% \subsection{Tips and tricks}
%
% If you want to include the solutions all at the
% end of the current document, you need to explicitly close the
% solution stream before including it:
% \begin{verbatim}
%   \closeout\solutionstream\input{\jobname.sol.tex}
% \end{verbatim}
%
% If you want to avoid exercises being split by a page boundary, then
% provide the package option 'minipage'. This causes the exercises to
% be wrapped in a minipage environment.
% 
% \clearpage
%
% \section{Implementation}
% %%%%%%%%%%%%%%%%%%%%%%%
%    \begin{macrocode}
%<*package>
%    \end{macrocode}
%
% \subsection{Auxiliary packages}
% %%%%%%%%%%%%%%%%%%%%%%%%%%%%%%%
% The package uses some auxiliary packages:
%    \begin{macrocode}
\RequirePackage{fancyvrb}
\RequirePackage{ifthen}
\RequirePackage{kvoptions}
%    \end{macrocode}
%
% \subsection{Package options}
% %%%%%%%%%%%%%%%%%%%%%%%%%%%%
% The package offers some options:
%
% \changes{v0.2}{2012/01/06}{Added option exercisesfont}
% \changes{v0.4}{2012/01/09}{Changed name of option to exercisesfontsize}
%
% \begin{macro}{exercisesfontsize}
%  This option allows setting the font of the \texttt{exercises}
%  environment. You may chopse one of tiny, scriptsize, footnotesize,
%  small, normalsize, large, etc.\\
%  E.g., \texttt{[exercisesfontsize=small]}.
%    \begin{macrocode}
\DeclareStringOption[normalsize]{exercisesfontsize}
%    \end{macrocode}
% \end{macro}
%
% \changes{v0.4}{2012/01/06}{Added option exercisesinlist}
% \changes{v0.5}{2012/01/09}{Changed option exercisesinlist to exerciseaslist}
%
% \begin{macro}{exerciseaslist}
%  This boolean option (true, false) allows setting the typesetting of
%  the \texttt{exercises} in a list environment. This causes the
%  exercises to be typeset in a more compact fashion, with indented
%  left and right margin. 
%    \begin{macrocode}
\DeclareBoolOption[false]{exerciseaslist}
%    \end{macrocode}
% \end{macro}
%
% \changes{v0.5}{2012/01/09}{Added option copyexercisesinsolutions}
% \begin{macro}{copyexercisesinsolutions}
%  This boolean option (true, false) allows copying the exercises in
%  the solutions file, to allow for making a complete stand-alone
%  exercises bundle.
%    \begin{macrocode}
\DeclareBoolOption[false]{copyexercisesinsolutions}
%    \end{macrocode}
% \end{macro}
%
% \changes{v0.9}{2014/07/28}{. Changed default behavior
% w.r.t. minipage-wraping of exercises}
% \begin{macro}{minipage}
%  This boolean option (true, false) causes the exercises to be
%  wrapped in minipages. This avoids them getting split by a page
%  boundary.
%    \begin{macrocode}
\DeclareBoolOption[false]{minipage}
%    \end{macrocode}
% \end{macro}
%
% The options are processed using:
%    \begin{macrocode}
\ProcessKeyvalOptions*
%    \end{macrocode}
%
% The options are subsequently handled
%    \begin{macrocode}
\newcommand{\exercisesfontsize}{\csname \exsol@exercisesfontsize\endcsname}
%    \end{macrocode}
%
%
% \subsection{Customization of lengths}
% %%%%%%%%%%%%%%%%%%%%%%%%%%%%%%%%%%%%%%%
% The commands below allow customizing many lengths that control the
% typesetting of the exercises.
%
% \changes{v0.91}{2014/08/31}{added user-accessible lengths}
% First some lengths to control the spacing before and after |exercises|.
%    \begin{macrocode}
\newlength{\exsolexercisesaboveskip}
\setlength{\exsolexercisesaboveskip}{1ex plus 1pt minus 1pt}
\newlength{\exsolexercisesbelowskip}
\setlength{\exsolexercisesbelowskip}{1ex plus 1pt minus 1pt}
%    \end{macrocode}
%
% Then some lengths to control the spacing for a single
% exercise. These lengths only work when the |exerciseaslist| package
% option has been specified. Sensible defaults have been set.
%    \begin{macrocode}
\newlength{\exsolexercisetopbottomsep}
\setlength{\exsolexercisetopbottomsep}{0pt plus 0pt minus 1pt}
\newlength{\exsolexerciseleftmargin}
\setlength{\exsolexerciseleftmargin}{1em}
\newlength{\exsolexerciserightmargin}
\setlength{\exsolexerciserightmargin}{1em}
\newlength{\exsolexerciseparindent}
\setlength{\exsolexerciseparindent}{0em}
\newlength{\exsolexerciselabelsep}
\setlength{\exsolexerciselabelsep}{0.5em}
\newlength{\exsolexerciselabelwidth}
\setlength{\exsolexerciselabelwidth}{0pt}
\newlength{\exsolexerciseitemindent}
\setlength{\exsolexerciseitemindent}{0pt}
\newlength{\exsolexerciseparsep}
\setlength{\exsolexerciseparsep}{\parskip}
%    \end{macrocode}
% 
% 
% \subsection{Con- and destruction of the auxiliary streams}
% %%%%%%%%%%%%%%%%%%%%%%%%%%%%%%%%%%%%%%%%%%%%%%%%%%%%%%%%%%
% At the beginning of your document, we start by opening a stream to a
% file that will be used to write the solutions to. At the end of your
% document, the package closes the stream.
% \changes{v0.8}{2014/07/15}{moved newwrite of exercise stream to this
% spot to avoid consuming all handles}
%    \begin{macrocode}
\AtBeginDocument{
  \newwrite\solutionstream
  \immediate\openout\solutionstream=\jobname.sol.tex
  \newwrite\exercisestream
}
\AtEndDocument{
  \immediate\closeout\solutionstream
}
%    \end{macrocode}
%
% \subsection{Exercises counter}
% %%%%%%%%%%%%%%%%%%%%%%%%%%%%%%
% By providing an exercise counter, proper numbering of the exercises
% is provided to allow for good cross referencing of the solutions to
% the exercises.
% \changes{v0.2}{2012/01/06}{Removed dash in counter when in document
% without sectioning commands}
%    \begin{macrocode}
\newcounter{exercise}[subsection]
\setcounter{exercise}{0}
\renewcommand{\theexercise}{%
  \@ifundefined{c@chapter}{}{\if0\arabic{chapter}\else\arabic{chapter}.\fi}%
  \if0\arabic{section}\else\arabic{section}\fi%
  \if0\arabic{subsection}\else.\arabic{subsection}\fi%
  \if0\arabic{subsubsection}\else.\arabic{subsubsection}\fi%
  \if0\arabic{exercise}\else%
    \@ifundefined{c@chapter}%
                 {\if0\arabic{section}\else-\fi}%
                 {-}%
    \arabic{exercise}%
  \fi
}
%    \end{macrocode}
%
%
%
% \subsection{Detokenization in order to cope with utf8}
%
% Combining old-school \LaTeX{} (before \XeTeX{} and \LuaTeX{}) and
% UTF-8 is a pain.
% Detokenization has been suggested by Geoffrey Poore to solve issues
% with UTF-8 characters messing up the |fancyvrb| internals.
% \changes{v0.7}{2014/07/14}{Added detokenized writing}
%    \begin{macrocode}
\newcommand{\GPES@write@detok}[1]{%
  \immediate\write\exercisestream{\detokenize{#1}}}%
\newcommand{\GPSS@write@detok}[1]{%
  \immediate\write\solutionstream{\detokenize{#1}}}%
\newcommand{\GPESS@write@detok}[1]{%
  \GPES@write@detok{#1}%
  \GPSS@write@detok{#1}}%
%    \end{macrocode}
%
%
% \section{The user environments}
%
% \begin{macro}{exercise}
%   The \texttt{exercise} environment is used to typeset your
%   exercises, provide them with a nice label and allow for copying
%   the exercise to the solutions file (if the package option
%   \texttt{copyexercisesinsolution}) is set. The label can be
%   set by redefining the \cs{exercisename} macro, or by relying on
%   the \textsf{Babel} provisions. The code is almost litteraly
%   taken from the \textsf{fancyvrb} package.
%    \begin{macrocode}
\def\exercise{\FV@Environment{}{exercise}}
\def\FVB@exercise{%
  \refstepcounter{exercise}%
  \immediate\openout\exercisestream=\jobname.exc.tex
  \ifexsol@copyexercisesinsolutions
    \typeout{Writing exercise to \jobname.sol.tex}
    \immediate\write\solutionstream{\string\par---\string\newline
      \string\textbf\string{\exercisename{} \theexercise \string}}
  \else
    \immediate\write\solutionstream{\string\par---\string\newline}
  \fi
  \immediate\write\exercisestream{\string\begin{exsol@exercise}}
  \@bsphack
  \begingroup
    \FV@UseKeyValues
    \FV@DefineWhiteSpace
    \def\FV@Space{\space}%
    \FV@DefineTabOut
    \ifexsol@copyexercisesinsolutions
      \let\FV@ProcessLine\GPESS@write@detok %
    \else
      \let\FV@ProcessLine\GPES@write@detok %
    \fi
    \relax
    \let\FV@FontScanPrep\relax
    \let\@noligs\relax
    \FV@Scan
  }
\def\FVE@exercise{
  \endgroup\@esphack
  \immediate\write\exercisestream{\string\end{exsol@exercise}}
  \ifexsol@copyexercisesinsolutions
    \immediate\write\solutionstream{\string~\string\newline}
  \fi
  \immediate\closeout\exercisestream
  \input{\jobname.exc.tex}
}
\DefineVerbatimEnvironment{exercise}{exercise}{}
%    \end{macrocode}
% \end{macro}
%
% \begin{macro}{exsol@exercise}
%   The \texttt{exsol@exercise} environment is an internal macro used
%   to typeset your exercises and provide them with a nice label and
%   number. Do not use it directly. Use the proper environment
%   \texttt{exercise} instead.
%   \changes{v0.2}{2012/01/06}{Attempted to fix MiKTeX formatting problems}
%   \changes{v0.3}{2012/01/08}{Fixed labelsep to avoid cluttered
%   itemize environments}
%   \changes{v0.4}{2012/01/06}{Added option exercisesinlist such that
%   default results in non list formatting of exercise}
%   \changes{v0.5}{2012/01/09}{Changed implementation to allow for
%   copying the exercises to the solutions file.}
%    \begin{macrocode}
\newenvironment{exsol@exercise}[0]
{%
  \ifthenelse{\boolean{exsol@minipage}}{\begin{minipage}[t]{\textwidth}}{}%
    \ifthenelse{\boolean{exsol@exerciseaslist}}
               {\begin{list}%
                   {%
                   }%
                   {%
                     \setlength{\topsep}{\exsolexercisetopbottomsep}%
                     \setlength{\leftmargin}{\exsolexerciseleftmargin}%
                     \setlength{\rightmargin}{\exsolexerciserightmargin}%
                     \setlength{\listparindent}{\exsolexerciseparindent}%
                     \setlength{\itemindent}{\exsolexerciseitemindent}%
                     \setlength{\parsep}{\exsolexerciseparsep}
                     \setlength{\labelsep}{\exsolexerciselabelsep}
                     \setlength{\labelwidth}{\exsolexerciselabelwidth}}
                 \item[\textit{~\exercisename{} \theexercise:~}]
               }%
               {\textit{\exercisename{} \theexercise:}}
}
{%
  \ifthenelse{\boolean{exsol@exerciseaslist}}%
             {\end{list}}{}%
  \ifthenelse{\boolean{exsol@minipage}}{\end{minipage}}{\par}%
}
%    \end{macrocode}
% \end{macro}
%
%
% \begin{macro}{solution}
%   The \texttt{solution} environment is used to typeset your solutions
%   and provide them with a nice label and number that corresponds to
%   the exercise that preceeded this solution. Theno label can be
%   set by redefining the \cs{solutionname} macro, or by relying on
%   the \textsf{Babel} provisions. The code is almost litteraly
%   taken from the \textsf{fancyvrb} package.
%    \begin{macrocode}
\def\solution{\FV@Environment{}{solution}}
\def\FVB@solution{%
  \typeout{Writing solution to \jobname.sol.tex}
  \immediate\write\solutionstream{\string\textbf\string{\solutionname{}\string}}
  \ifexsol@copyexercisesinsolutions
    \immediate\write\solutionstream{\string\newline}
  \else
    \immediate\write\solutionstream{\string\textbf\string{\theexercise\string}%
                                    \string\newline}
  \fi
  \@bsphack
  \begingroup
    \FV@UseKeyValues
    \FV@DefineWhiteSpace
    \def\FV@Space{\space}%
    \FV@DefineTabOut
    \let\FV@ProcessLine\GPSS@write@detok %
    \relax
    \let\FV@FontScanPrep\relax
    \let\@noligs\relax
    \FV@Scan
  }
\def\FVE@solution{\endgroup\@esphack}
\DefineVerbatimEnvironment{solution}{solution}{}
%    \end{macrocode}
% \end{macro}
%
% \begin{macro}{exercises}
%   The \texttt{exercises} environment helps typesetting your exercises to
%   stand out from the rest of the text. You may use it at the end of
%   a chapter, or just to group some exercises in the text.
%   \changes{v0.2}{2012/01/06}{Attempted to fix MiKTeX formatting problems}
%   \changes{v0.3}{2012/01/07}{Added some extra whitespace below exercisesname}
%    \begin{macrocode}
\newenvironment{exercises}
{\par\exercisesfontsize\rule{.25\linewidth}{0.15mm}\vspace*{\exsolexercisesaboveskip}\\*%
 \textbf{\normalsize \exercisesname}}
{\vspace*{-\baselineskip}\vspace*{\exsolexercisesbelowskip}\rule{.25\linewidth}{0.15mm}\par}
%    \end{macrocode}
% \end{macro}
%
% \subsection{Some Babel provisions}
% %%%%%%%%%%%%%%%%%%%%%%%%%%%%%%%%%%
% \changes{v0.2}{2012/01/06}{Fixed babel errors}
% \begin{macro}{\exercisename}
%   The exercise environment makes use of a label \texttt{\exercisename{}}
%   macro.
%    \begin{macrocode}
\newcommand{\exercisename}{Exercise}
%    \end{macrocode}
% \end{macro}
%
% \begin{macro}{\exercisesname}
%   The exercises environment makes use of a label \texttt{\exercisesname{}}
%   macro.
%    \begin{macrocode}
\newcommand{\exercisesname}{Exercises}
%    \end{macrocode}
% \end{macro}
% 
% \begin{macro}{\solutionname}
%   The solution environment makes use of a label \texttt{\solutionname{}}
%   macro.
%    \begin{macrocode}
\newcommand{\solutionname}{Solution}
%    \end{macrocode}
% \end{macro}
%
% \begin{macro}{\solutionname}
%   The solution environment makes use of a label \texttt{\solutionname{}}
%   macro.
% \changes{v0.8}{2014/07/15}{Added missing babel tag}
%    \begin{macrocode}
\newcommand{\solutionsname}{Solutions}
%    \end{macrocode}
% \end{macro}
% 
% 
% You may redefine these macros, but to help you out a little bit, we
% provide with some basic Babel auxiliaries. If you're a true polyglot
% and are willing to help me out by providing translations for other
% languages, I'm very willing to incorporate them into the code.
%
% \changes{v0.7}{2014/07/14}{Added Finnish language support}
%    \begin{macrocode}
\addto\captionsdutch{%
  \renewcommand{\exercisename}{Oefening}%
  \renewcommand{\exercisesname}{Oefeningen}%
  \renewcommand{\solutionname}{Oplossing}%
  \renewcommand{\solutionsname}{Oplossingen}%
}
\addto\captionsgerman{%
  \renewcommand{\exercisename}{Aufgabe}%
  \renewcommand{\exercisesname}{Aufgaben}%
  \renewcommand{\solutionname}{L\"osung}%
  \renewcommand{\solutionsname}{L\"osungen}%
}
\addto\captionsfrench{%
  \renewcommand{\exercisename}{Exercice}%
  \renewcommand{\exercisesname}{Exercices}%
  \renewcommand{\solutionname}{Solution}%
  \renewcommand{\solutionsname}{Solutions}%
}
\addto\captionsfinnish{
  \renewcommand{\exercisename}{Teht\"av\"a}%
  \renewcommand{\exercisesname}{Teht\"avi\"a}%
  \renewcommand{\solutionname}{Ratkaisu}%
  \renewcommand{\solutionsname}{Ratkaisut}%
}
%    \end{macrocode}
%
%
%
% Now the final hack overloads the basic sectioning commands to make
% sure that they are copied into your solution book.
%
%    \begin{macrocode}
\let\exsol@@makechapterhead\@makechapterhead
\def\@makechapterhead#1{%
  \immediate\write\solutionstream{\string\chapter{#1}}%
  \exsol@@makechapterhead{#1}
}
\ifdefined\frontmatter
  \let\exsol@@frontmatter\frontmatter
  \def\frontmatter{%
    \immediate\write\solutionstream{\string\frontmatter}%
    \exsol@@frontmatter
  }
\fi
\ifdefined\frontmatter
  \let\exsol@@mainmatter\mainmatter
  \def\mainmatter{%
    \immediate\write\solutionstream{\string\mainmatter}%
    \exsol@@mainmatter
  }
\fi
\ifdefined\backmatter
  \let\exsol@@backmatter\backmatter
  \def\backmatter{%
    \immediate\write\solutionstream{\string\backmatter}%
    \exsol@@backmatter
  }
\fi
%    \end{macrocode}
%
% \begin{macro}{\noexercisesinchapter}
%   If you have chapters without exercises, you may want to indicate
%   this clearly into your source. Otherwise empty chapters may appear
%   in your solution book.
%    \begin{macrocode}
\newcommand{\noexercisesinchapter}
{
  \immediate\write\solutionstream{No exercises in this chapter}
}
%    \end{macrocode}
% \end{macro}
%
%    \begin{macrocode}
%</package>
%    \end{macrocode}
%
% \bibliographystyle{alpha}
%
% \begin{thebibliography}{99}
%
% \bibitem{fancyvrb}
% Timothy Van Zandt, Herbert Vo\ss, Denis Girou, Sebastian Rahtz, Niall
% Mansfield 
% \newblock The \texttt{fancyvrb} package.
% \newblock \url{http://ctan.org/pkg/fancyvrb}.
% \newblock online, accessed in January 2012.
%
% \bibitem{CTAN} 
% The Comprehensive TeX Archive Network.
% \newblock \url{http://www.ctan.org}.
% \newblock online, accessed in January 2012.
%
% \end{thebibliography}
%
% \Finale
\endinput

%
% \end{document}
% \end{VerbatimOut}
% \VerbatimInput[frame=lines,gobble=2,fontsize=\footnotesize]{exsol-solutionbook.tex}
% 
% You may generate this solution book, by running \LaTeX{} on the
% file named \texttt{exsol-solutionbook.tex} that is generated when running
% \LaTeX{} on the \texttt{exsol.dtx} file.
%
% The result approximately looks like this:
%
% \setcounter{equation}{0}
% \rule{\linewidth}{.7pt}
% \begin{center}
% {\Large Solutions to the exercises, specified in the \textsf{ExSol} package}\\
% {\large Walter Daems}\\
% {\large 2013/05/12}
% \end{center}
% \par---\newline\textbf{Solution 3.1-1}
%     Let's start by rearranging the equation, a bit:
%     \begin{eqnarray}
%       5.7 x^2 - 3.1 x &=% 5.3\\
%       5.7 x^2 - 3.1 x -5.3 &=% 0
%     \end{eqnarray}
%     The equation is now in the standard form:
%     \begin{equation}
%       a x^2 + b x + c = 0
%     \end{equation}
%     For quadratic equations in the standard form, we know that two
%     solutions exist:
%     \begin{equation}
%       x_{1,2} = \frac{ -b \pm \sqrt{d}}{2a}
%     \end{equation}
%     with
%     \begin{equation}
%       d = b^2 - 4 a c
%     \end{equation}
%     If we apply this to our case, we obtain:
%     \begin{equation}
%       d = (-3.1)^2 - 4 \cdot 5.7 \cdot (-5.3) = 130.45
%     \end{equation}
%     and
%     \begin{eqnarray}
%       x_1 &=& \frac{3.1 + \sqrt{130.45}}{11.4} = 1.27\\
%       x_2 &=& \frac{3.1 + \sqrt{130.45}}{11.4} = -0.73
%     \end{eqnarray}
%     The proposed values $x = x_1, x_2$ are solutions to
%     the given equation.
% \par---\newline\textbf{Solution 3.1-2}
%       This calls for application of Pythagoras' theorem, which
%       tells us:
%       \begin{equation}
%         \left\|A\right\|^2 + \left\|B\right\|^2 = \left\|C\right\|^2
%       \end{equation}
%       and therefore:
%       \begin{eqnarray}
%         \left\|C\right\|
%         &=& \sqrt{\left\|A\right\|^2 + \left\|B\right\|^2}\\
%         &=& \sqrt{3^2 + 4^2}\\
%         &=& \sqrt{25} = 5
%       \end{eqnarray}
%       Therefore, the length of the hypotenuse equals $5$.
%
% \rule{\linewidth}{.7pt}
%
% \subsection{Fiddling with the spacing}
%
% The default spacing provided by the \textsf{ExSol} package should be
% fine for most users. However, if you like to tweak, below you can
% find the controls.
% \subsubsection{Spacing before and after the \texttt{exercises} environment}
%
% The lengths below control the spacing of the |exercises| environment:
% \begin{itemize}
% \item |exsolexerciseaboveskip|: rubber length controlling the
% vertical space after the top marker line of the environment
% \item |exsolexercisebelowskip|: rubber length controlling the
% vertical space before the bottom marker line of the environment
% \end{itemize}
%
% You can simply specify them like:
% \begin{verbatim}
% \setlength{\exsolexercisesaboveskip}{1ex plus 1pt minus 1pt}
% \setlength{\exsolexercisesbelowskip}{1ex plus 1pt minus 1pt}
% \end{verbatim}
% The spacings specified here are the package defaults.
%
% \subsubsection{Spacing of the individual exercises}
% Caution: the spacing can only be tuned, when one invokes the
% |exerciseaslist| package option!
%
% Then lengths below control the spacing of the |exercise| environment:
% \begin{itemize}
% \item |exercisetopbottomsep|: rubber length controlling the vertical
% space before and after individual exercises
% \item |exerciseleftmargin|: length controlling the horizontal
% space between the surrounding environment's left margin (most
% often the page margin) and the left edge of the exercise
% environment 
% \item |exerciseleftmargin|: length controlling the horizontal
% space between the surrounding environment's right margin (most
% often the page margin) and the right edge of the exercise
% environment
% \item |exerciseitemindent|: length controlling the first-line
% indentation of the first paragraph in the exercise environment
% (actually, the label is set w.r.t. this position, that we will
% conveniently call position 'x')
% \item |exerciseparindent|: length controlling the first-line
% indentation of the other paragraphs in the exercise environment.
% \item |exerciselabelsep|: length controlling the distance between
% the label and position 'x'
% \item |exerciselabelwidth|: minimal width of the (internally
% right-alligned) box to use for the exercises label; if the box is
% not sufficiently big, position 'x' is shifted to the right
% \item |exerciseparsep|: internal paragraph separation (vertically)
% \end{itemize}
% 
% You can simply specify them like:
% \begin{verbatim}
% \setlength{\exsolexercisetopbottomsep}{0pt plus 0pt minus 1pt}
% \setlength{\exsolexerciseleftmargin}{1em}
% \setlength{\exsolexerciserightmargin}{1em}
% \setlength{\exsolexerciseparindent}{0em}
% \setlength{\exsolexerciselabelsep}{0.5em}
% \setlength{\exsolexerciselabelwidth}{0pt}
% \setlength{\exsolexerciseitemindent}{0pt}
% \setlength{\exsolexerciseparsep}{\parskip}
% \end{verbatim}
% The spacings specified here are the package defaults.
%
% \subsection{Tips and tricks}
%
% If you want to include the solutions all at the
% end of the current document, you need to explicitly close the
% solution stream before including it:
% \begin{verbatim}
%   \closeout\solutionstream\input{\jobname.sol.tex}
% \end{verbatim}
%
% If you want to avoid exercises being split by a page boundary, then
% provide the package option 'minipage'. This causes the exercises to
% be wrapped in a minipage environment.
% 
% \clearpage
%
% \section{Implementation}
% %%%%%%%%%%%%%%%%%%%%%%%
%    \begin{macrocode}
%<*package>
%    \end{macrocode}
%
% \subsection{Auxiliary packages}
% %%%%%%%%%%%%%%%%%%%%%%%%%%%%%%%
% The package uses some auxiliary packages:
%    \begin{macrocode}
\RequirePackage{fancyvrb}
\RequirePackage{ifthen}
\RequirePackage{kvoptions}
%    \end{macrocode}
%
% \subsection{Package options}
% %%%%%%%%%%%%%%%%%%%%%%%%%%%%
% The package offers some options:
%
% \changes{v0.2}{2012/01/06}{Added option exercisesfont}
% \changes{v0.4}{2012/01/09}{Changed name of option to exercisesfontsize}
%
% \begin{macro}{exercisesfontsize}
%  This option allows setting the font of the \texttt{exercises}
%  environment. You may chopse one of tiny, scriptsize, footnotesize,
%  small, normalsize, large, etc.\\
%  E.g., \texttt{[exercisesfontsize=small]}.
%    \begin{macrocode}
\DeclareStringOption[normalsize]{exercisesfontsize}
%    \end{macrocode}
% \end{macro}
%
% \changes{v0.4}{2012/01/06}{Added option exercisesinlist}
% \changes{v0.5}{2012/01/09}{Changed option exercisesinlist to exerciseaslist}
%
% \begin{macro}{exerciseaslist}
%  This boolean option (true, false) allows setting the typesetting of
%  the \texttt{exercises} in a list environment. This causes the
%  exercises to be typeset in a more compact fashion, with indented
%  left and right margin. 
%    \begin{macrocode}
\DeclareBoolOption[false]{exerciseaslist}
%    \end{macrocode}
% \end{macro}
%
% \changes{v0.5}{2012/01/09}{Added option copyexercisesinsolutions}
% \begin{macro}{copyexercisesinsolutions}
%  This boolean option (true, false) allows copying the exercises in
%  the solutions file, to allow for making a complete stand-alone
%  exercises bundle.
%    \begin{macrocode}
\DeclareBoolOption[false]{copyexercisesinsolutions}
%    \end{macrocode}
% \end{macro}
%
% \changes{v0.9}{2014/07/28}{. Changed default behavior
% w.r.t. minipage-wraping of exercises}
% \begin{macro}{minipage}
%  This boolean option (true, false) causes the exercises to be
%  wrapped in minipages. This avoids them getting split by a page
%  boundary.
%    \begin{macrocode}
\DeclareBoolOption[false]{minipage}
%    \end{macrocode}
% \end{macro}
%
% The options are processed using:
%    \begin{macrocode}
\ProcessKeyvalOptions*
%    \end{macrocode}
%
% The options are subsequently handled
%    \begin{macrocode}
\newcommand{\exercisesfontsize}{\csname \exsol@exercisesfontsize\endcsname}
%    \end{macrocode}
%
%
% \subsection{Customization of lengths}
% %%%%%%%%%%%%%%%%%%%%%%%%%%%%%%%%%%%%%%%
% The commands below allow customizing many lengths that control the
% typesetting of the exercises.
%
% \changes{v0.91}{2014/08/31}{added user-accessible lengths}
% First some lengths to control the spacing before and after |exercises|.
%    \begin{macrocode}
\newlength{\exsolexercisesaboveskip}
\setlength{\exsolexercisesaboveskip}{1ex plus 1pt minus 1pt}
\newlength{\exsolexercisesbelowskip}
\setlength{\exsolexercisesbelowskip}{1ex plus 1pt minus 1pt}
%    \end{macrocode}
%
% Then some lengths to control the spacing for a single
% exercise. These lengths only work when the |exerciseaslist| package
% option has been specified. Sensible defaults have been set.
%    \begin{macrocode}
\newlength{\exsolexercisetopbottomsep}
\setlength{\exsolexercisetopbottomsep}{0pt plus 0pt minus 1pt}
\newlength{\exsolexerciseleftmargin}
\setlength{\exsolexerciseleftmargin}{1em}
\newlength{\exsolexerciserightmargin}
\setlength{\exsolexerciserightmargin}{1em}
\newlength{\exsolexerciseparindent}
\setlength{\exsolexerciseparindent}{0em}
\newlength{\exsolexerciselabelsep}
\setlength{\exsolexerciselabelsep}{0.5em}
\newlength{\exsolexerciselabelwidth}
\setlength{\exsolexerciselabelwidth}{0pt}
\newlength{\exsolexerciseitemindent}
\setlength{\exsolexerciseitemindent}{0pt}
\newlength{\exsolexerciseparsep}
\setlength{\exsolexerciseparsep}{\parskip}
%    \end{macrocode}
% 
% 
% \subsection{Con- and destruction of the auxiliary streams}
% %%%%%%%%%%%%%%%%%%%%%%%%%%%%%%%%%%%%%%%%%%%%%%%%%%%%%%%%%%
% At the beginning of your document, we start by opening a stream to a
% file that will be used to write the solutions to. At the end of your
% document, the package closes the stream.
% \changes{v0.8}{2014/07/15}{moved newwrite of exercise stream to this
% spot to avoid consuming all handles}
%    \begin{macrocode}
\AtBeginDocument{
  \newwrite\solutionstream
  \immediate\openout\solutionstream=\jobname.sol.tex
  \newwrite\exercisestream
}
\AtEndDocument{
  \immediate\closeout\solutionstream
}
%    \end{macrocode}
%
% \subsection{Exercises counter}
% %%%%%%%%%%%%%%%%%%%%%%%%%%%%%%
% By providing an exercise counter, proper numbering of the exercises
% is provided to allow for good cross referencing of the solutions to
% the exercises.
% \changes{v0.2}{2012/01/06}{Removed dash in counter when in document
% without sectioning commands}
%    \begin{macrocode}
\newcounter{exercise}[subsection]
\setcounter{exercise}{0}
\renewcommand{\theexercise}{%
  \@ifundefined{c@chapter}{}{\if0\arabic{chapter}\else\arabic{chapter}.\fi}%
  \if0\arabic{section}\else\arabic{section}\fi%
  \if0\arabic{subsection}\else.\arabic{subsection}\fi%
  \if0\arabic{subsubsection}\else.\arabic{subsubsection}\fi%
  \if0\arabic{exercise}\else%
    \@ifundefined{c@chapter}%
                 {\if0\arabic{section}\else-\fi}%
                 {-}%
    \arabic{exercise}%
  \fi
}
%    \end{macrocode}
%
%
%
% \subsection{Detokenization in order to cope with utf8}
%
% Combining old-school \LaTeX{} (before \XeTeX{} and \LuaTeX{}) and
% UTF-8 is a pain.
% Detokenization has been suggested by Geoffrey Poore to solve issues
% with UTF-8 characters messing up the |fancyvrb| internals.
% \changes{v0.7}{2014/07/14}{Added detokenized writing}
%    \begin{macrocode}
\newcommand{\GPES@write@detok}[1]{%
  \immediate\write\exercisestream{\detokenize{#1}}}%
\newcommand{\GPSS@write@detok}[1]{%
  \immediate\write\solutionstream{\detokenize{#1}}}%
\newcommand{\GPESS@write@detok}[1]{%
  \GPES@write@detok{#1}%
  \GPSS@write@detok{#1}}%
%    \end{macrocode}
%
%
% \section{The user environments}
%
% \begin{macro}{exercise}
%   The \texttt{exercise} environment is used to typeset your
%   exercises, provide them with a nice label and allow for copying
%   the exercise to the solutions file (if the package option
%   \texttt{copyexercisesinsolution}) is set. The label can be
%   set by redefining the \cs{exercisename} macro, or by relying on
%   the \textsf{Babel} provisions. The code is almost litteraly
%   taken from the \textsf{fancyvrb} package.
%    \begin{macrocode}
\def\exercise{\FV@Environment{}{exercise}}
\def\FVB@exercise{%
  \refstepcounter{exercise}%
  \immediate\openout\exercisestream=\jobname.exc.tex
  \ifexsol@copyexercisesinsolutions
    \typeout{Writing exercise to \jobname.sol.tex}
    \immediate\write\solutionstream{\string\par---\string\newline
      \string\textbf\string{\exercisename{} \theexercise \string}}
  \else
    \immediate\write\solutionstream{\string\par---\string\newline}
  \fi
  \immediate\write\exercisestream{\string\begin{exsol@exercise}}
  \@bsphack
  \begingroup
    \FV@UseKeyValues
    \FV@DefineWhiteSpace
    \def\FV@Space{\space}%
    \FV@DefineTabOut
    \ifexsol@copyexercisesinsolutions
      \let\FV@ProcessLine\GPESS@write@detok %
    \else
      \let\FV@ProcessLine\GPES@write@detok %
    \fi
    \relax
    \let\FV@FontScanPrep\relax
    \let\@noligs\relax
    \FV@Scan
  }
\def\FVE@exercise{
  \endgroup\@esphack
  \immediate\write\exercisestream{\string\end{exsol@exercise}}
  \ifexsol@copyexercisesinsolutions
    \immediate\write\solutionstream{\string~\string\newline}
  \fi
  \immediate\closeout\exercisestream
  \input{\jobname.exc.tex}
}
\DefineVerbatimEnvironment{exercise}{exercise}{}
%    \end{macrocode}
% \end{macro}
%
% \begin{macro}{exsol@exercise}
%   The \texttt{exsol@exercise} environment is an internal macro used
%   to typeset your exercises and provide them with a nice label and
%   number. Do not use it directly. Use the proper environment
%   \texttt{exercise} instead.
%   \changes{v0.2}{2012/01/06}{Attempted to fix MiKTeX formatting problems}
%   \changes{v0.3}{2012/01/08}{Fixed labelsep to avoid cluttered
%   itemize environments}
%   \changes{v0.4}{2012/01/06}{Added option exercisesinlist such that
%   default results in non list formatting of exercise}
%   \changes{v0.5}{2012/01/09}{Changed implementation to allow for
%   copying the exercises to the solutions file.}
%    \begin{macrocode}
\newenvironment{exsol@exercise}[0]
{%
  \ifthenelse{\boolean{exsol@minipage}}{\begin{minipage}[t]{\textwidth}}{}%
    \ifthenelse{\boolean{exsol@exerciseaslist}}
               {\begin{list}%
                   {%
                   }%
                   {%
                     \setlength{\topsep}{\exsolexercisetopbottomsep}%
                     \setlength{\leftmargin}{\exsolexerciseleftmargin}%
                     \setlength{\rightmargin}{\exsolexerciserightmargin}%
                     \setlength{\listparindent}{\exsolexerciseparindent}%
                     \setlength{\itemindent}{\exsolexerciseitemindent}%
                     \setlength{\parsep}{\exsolexerciseparsep}
                     \setlength{\labelsep}{\exsolexerciselabelsep}
                     \setlength{\labelwidth}{\exsolexerciselabelwidth}}
                 \item[\textit{~\exercisename{} \theexercise:~}]
               }%
               {\textit{\exercisename{} \theexercise:}}
}
{%
  \ifthenelse{\boolean{exsol@exerciseaslist}}%
             {\end{list}}{}%
  \ifthenelse{\boolean{exsol@minipage}}{\end{minipage}}{\par}%
}
%    \end{macrocode}
% \end{macro}
%
%
% \begin{macro}{solution}
%   The \texttt{solution} environment is used to typeset your solutions
%   and provide them with a nice label and number that corresponds to
%   the exercise that preceeded this solution. Theno label can be
%   set by redefining the \cs{solutionname} macro, or by relying on
%   the \textsf{Babel} provisions. The code is almost litteraly
%   taken from the \textsf{fancyvrb} package.
%    \begin{macrocode}
\def\solution{\FV@Environment{}{solution}}
\def\FVB@solution{%
  \typeout{Writing solution to \jobname.sol.tex}
  \immediate\write\solutionstream{\string\textbf\string{\solutionname{}\string}}
  \ifexsol@copyexercisesinsolutions
    \immediate\write\solutionstream{\string\newline}
  \else
    \immediate\write\solutionstream{\string\textbf\string{\theexercise\string}%
                                    \string\newline}
  \fi
  \@bsphack
  \begingroup
    \FV@UseKeyValues
    \FV@DefineWhiteSpace
    \def\FV@Space{\space}%
    \FV@DefineTabOut
    \let\FV@ProcessLine\GPSS@write@detok %
    \relax
    \let\FV@FontScanPrep\relax
    \let\@noligs\relax
    \FV@Scan
  }
\def\FVE@solution{\endgroup\@esphack}
\DefineVerbatimEnvironment{solution}{solution}{}
%    \end{macrocode}
% \end{macro}
%
% \begin{macro}{exercises}
%   The \texttt{exercises} environment helps typesetting your exercises to
%   stand out from the rest of the text. You may use it at the end of
%   a chapter, or just to group some exercises in the text.
%   \changes{v0.2}{2012/01/06}{Attempted to fix MiKTeX formatting problems}
%   \changes{v0.3}{2012/01/07}{Added some extra whitespace below exercisesname}
%    \begin{macrocode}
\newenvironment{exercises}
{\par\exercisesfontsize\rule{.25\linewidth}{0.15mm}\vspace*{\exsolexercisesaboveskip}\\*%
 \textbf{\normalsize \exercisesname}}
{\vspace*{-\baselineskip}\vspace*{\exsolexercisesbelowskip}\rule{.25\linewidth}{0.15mm}\par}
%    \end{macrocode}
% \end{macro}
%
% \subsection{Some Babel provisions}
% %%%%%%%%%%%%%%%%%%%%%%%%%%%%%%%%%%
% \changes{v0.2}{2012/01/06}{Fixed babel errors}
% \begin{macro}{\exercisename}
%   The exercise environment makes use of a label \texttt{\exercisename{}}
%   macro.
%    \begin{macrocode}
\newcommand{\exercisename}{Exercise}
%    \end{macrocode}
% \end{macro}
%
% \begin{macro}{\exercisesname}
%   The exercises environment makes use of a label \texttt{\exercisesname{}}
%   macro.
%    \begin{macrocode}
\newcommand{\exercisesname}{Exercises}
%    \end{macrocode}
% \end{macro}
% 
% \begin{macro}{\solutionname}
%   The solution environment makes use of a label \texttt{\solutionname{}}
%   macro.
%    \begin{macrocode}
\newcommand{\solutionname}{Solution}
%    \end{macrocode}
% \end{macro}
%
% \begin{macro}{\solutionname}
%   The solution environment makes use of a label \texttt{\solutionname{}}
%   macro.
% \changes{v0.8}{2014/07/15}{Added missing babel tag}
%    \begin{macrocode}
\newcommand{\solutionsname}{Solutions}
%    \end{macrocode}
% \end{macro}
% 
% 
% You may redefine these macros, but to help you out a little bit, we
% provide with some basic Babel auxiliaries. If you're a true polyglot
% and are willing to help me out by providing translations for other
% languages, I'm very willing to incorporate them into the code.
%
% \changes{v0.7}{2014/07/14}{Added Finnish language support}
%    \begin{macrocode}
\addto\captionsdutch{%
  \renewcommand{\exercisename}{Oefening}%
  \renewcommand{\exercisesname}{Oefeningen}%
  \renewcommand{\solutionname}{Oplossing}%
  \renewcommand{\solutionsname}{Oplossingen}%
}
\addto\captionsgerman{%
  \renewcommand{\exercisename}{Aufgabe}%
  \renewcommand{\exercisesname}{Aufgaben}%
  \renewcommand{\solutionname}{L\"osung}%
  \renewcommand{\solutionsname}{L\"osungen}%
}
\addto\captionsfrench{%
  \renewcommand{\exercisename}{Exercice}%
  \renewcommand{\exercisesname}{Exercices}%
  \renewcommand{\solutionname}{Solution}%
  \renewcommand{\solutionsname}{Solutions}%
}
\addto\captionsfinnish{
  \renewcommand{\exercisename}{Teht\"av\"a}%
  \renewcommand{\exercisesname}{Teht\"avi\"a}%
  \renewcommand{\solutionname}{Ratkaisu}%
  \renewcommand{\solutionsname}{Ratkaisut}%
}
%    \end{macrocode}
%
%
%
% Now the final hack overloads the basic sectioning commands to make
% sure that they are copied into your solution book.
%
%    \begin{macrocode}
\let\exsol@@makechapterhead\@makechapterhead
\def\@makechapterhead#1{%
  \immediate\write\solutionstream{\string\chapter{#1}}%
  \exsol@@makechapterhead{#1}
}
\ifdefined\frontmatter
  \let\exsol@@frontmatter\frontmatter
  \def\frontmatter{%
    \immediate\write\solutionstream{\string\frontmatter}%
    \exsol@@frontmatter
  }
\fi
\ifdefined\frontmatter
  \let\exsol@@mainmatter\mainmatter
  \def\mainmatter{%
    \immediate\write\solutionstream{\string\mainmatter}%
    \exsol@@mainmatter
  }
\fi
\ifdefined\backmatter
  \let\exsol@@backmatter\backmatter
  \def\backmatter{%
    \immediate\write\solutionstream{\string\backmatter}%
    \exsol@@backmatter
  }
\fi
%    \end{macrocode}
%
% \begin{macro}{\noexercisesinchapter}
%   If you have chapters without exercises, you may want to indicate
%   this clearly into your source. Otherwise empty chapters may appear
%   in your solution book.
%    \begin{macrocode}
\newcommand{\noexercisesinchapter}
{
  \immediate\write\solutionstream{No exercises in this chapter}
}
%    \end{macrocode}
% \end{macro}
%
%    \begin{macrocode}
%</package>
%    \end{macrocode}
%
% \bibliographystyle{alpha}
%
% \begin{thebibliography}{99}
%
% \bibitem{fancyvrb}
% Timothy Van Zandt, Herbert Vo\ss, Denis Girou, Sebastian Rahtz, Niall
% Mansfield 
% \newblock The \texttt{fancyvrb} package.
% \newblock \url{http://ctan.org/pkg/fancyvrb}.
% \newblock online, accessed in January 2012.
%
% \bibitem{CTAN} 
% The Comprehensive TeX Archive Network.
% \newblock \url{http://www.ctan.org}.
% \newblock online, accessed in January 2012.
%
% \end{thebibliography}
%
% \Finale
\endinput
