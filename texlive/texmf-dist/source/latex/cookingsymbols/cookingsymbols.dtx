% \iffalse meta-comment
%
% Copyright (C) 2011 by Sven Harder
% -----------------------------------
%
% This file may be distributed and/or modified under the
% conditions of the LaTeX Project Public License, either version 1.3
% of this license or (at your option) any later version.
% The latest version of this license is in:
%
% http://www.latex-project.org/lppl.txt
%
% and version 1.3 or later is part of all distributions of LaTeX
% version 2005/12/01 or later.
%
% This work has the LPPL maintenance status `maintained'.
% 
% The Current Maintainer of this work is Sven Harder.
%
% This work consists of the files cookingsymbols.dtx and cookingsymbols.ins
% and the derived files cookingsymbols.sty and cookingsymbols.mf.
%
% \fi
%
% \iffalse
%<package>\NeedsTeXFormat{LaTeX2e}
%<package>\ProvidesPackage{cookingsymbols}
%<package>[2011/11/02 v1.0 symbols for recipes by Sven Harder]
%
%<*driver>
\documentclass{ltxdoc}
\usepackage{cookingsymbols}
\usepackage[T1]{fontenc}
\usepackage{lmodern}
\usepackage{booktabs}
\usepackage{mflogo}
\EnableCrossrefs
\CodelineIndex
\RecordChanges
\addtolength{\oddsidemargin}{-1cm}
\begin{document}
    \DocInput{cookingsymbols.dtx}
\end{document}
%</driver>
% \fi
%
% \CheckSum{47}
%
% \CharacterTable
%  {Upper-case    \A\B\C\D\E\F\G\H\I\J\K\L\M\N\O\P\Q\R\S\T\U\V\W\X\Y\Z
%   Lower-case    \a\b\c\d\e\f\g\h\i\j\k\l\m\n\o\p\q\r\s\t\u\v\w\x\y\z
%   Digits        \0\1\2\3\4\5\6\7\8\9
%   Exclamation   \!     Double quote  \"     Hash (number) \#
%   Dollar        \$     Percent       \%     Ampersand     \&
%   Acute accent  \'     Left paren    \(     Right paren   \)
%   Asterisk      \*     Plus          \+     Comma         \,
%   Minus         \-     Point         \.     Solidus       \/
%   Colon         \:     Semicolon     \;     Less than     \<
%   Equals        \=     Greater than  \>     Question mark \?
%   Commercial at \@     Left bracket  \[     Backslash     \\
%   Right bracket \]     Circumflex    \^     Underscore    \_
%   Grave accent  \`     Left brace    \{     Vertical bar  \|
%   Right brace   \}     Tilde         \~}
%
%
% \changes{v1.0}{2011/11/02}{Initial version}
%
% \GetFileInfo{cookingsymbols.sty}
%
% \DoNotIndex{\#,\$,\%,\&,\@,\\,\{,\},\^,\_,\~,\ }
% \DoNotIndex{\@ne}
% \DoNotIndex{\advance,\begingroup,\catcode,\closein}
% \DoNotIndex{\closeout,\day,\def,\edef,\else,\empty,\endgroup}
% \DoNotIndex{\begin,\end,\equal,beginchar,endchar}
%
% \title{The \textsf{cookingsymbols} package\thanks{This document
% corresponds to \textsf{cookingsymbols}~\fileversion,
% dated \filedate.}}
% \author{Sven Harder \\ \texttt{sven\_one1@gmx.de}}
%
% \maketitle
%
% \section{Introduction}
%  This package includes 11 symbols. They are original created, when I was searching for symbols for typesetting recipes. However, I didn't find any symbols, so I decided to create my own ones. This is the result ;)
%
%  At the end of creating these symbols, I realized that \MF\ is out of date. But at this time, I had no time to create the same symbols as vector based fonts. This is still a task for the future.
%
%  \section{Symbols}
%  The following symbols are created by this package and are available by these macros. \textbf{Important:} The symbols are enhanced for a better view (by \textbackslash Large).
%  \begin{center}
%    \begin{tabular}[h]{rl}
%      \bf Macro & \bf Symbol \\
%      \toprule
%      \textbackslash Oven & \Large\Oven \\
%      \textbackslash Topbottomheat & \Large\Topbottomheat \\
%      \textbackslash Topheat & \Large\Topheat\\
%      \textbackslash Bottomheat & \Large\Bottomheat\\
%      \textbackslash Fanoven & \Large\Fanoven\\
%      \textbackslash Gasstove & \Large\Gasstove\\
%      \textbackslash Dish & \Large\Dish\\
%      \textbackslash Knife & \Large\Knife\\
%      \textbackslash Fork & \Large\Fork\\
%      \textbackslash Spoon & \Large\Spoon\\
%      \textbackslash Gloves & \Large\Gloves\\
%    \end{tabular}
%  \end{center}
%  And now have fun with these new symbols!
%
% \StopEventually{\PrintIndex}
%
% \section{Implementation}
%
% \subsection{cookingsymbols.sty}
% The following content can be found in the derived file \textsf{cookingsymbols.sty}. It includes the macros for an easy access and the definition of the (new) font.
%    \begin{macrocode}
%<*package>

\DeclareFontFamily{U}{cookingsymbols}{}

\DeclareFontShape{U}{cookingsymbols}{m}{n}{<-> cookingsymbols}{}
\DeclareFontShape{U}{cookingsymbols}{bx}{n}{<-> sub cookingsymbols/m/n }{}

\newcommand{\Oven}{{\usefont{U}{cookingsymbols}{m}{n}\symbol{0}}}
\newcommand{\Topbottomheat}{{\usefont{U}{cookingsymbols}{m}{n}\symbol{1}}}
\newcommand{\Topheat}{{\usefont{U}{cookingsymbols}{m}{n}\symbol{2}}}
\newcommand{\Bottomheat}{{\usefont{U}{cookingsymbols}{m}{n}\symbol{3}}}
\newcommand{\Fanoven}{{\usefont{U}{cookingsymbols}{m}{n}\symbol{4}}}
\newcommand{\Gasstove}{{\usefont{U}{cookingsymbols}{m}{n}\symbol{5}}}
\newcommand{\Dish}{{\usefont{U}{cookingsymbols}{m}{n}\symbol{6}}}
\newcommand{\Knife}{{\usefont{U}{cookingsymbols}{m}{n}\symbol{7}}}
\newcommand{\Fork}{{\usefont{U}{cookingsymbols}{m}{n}\symbol{8}}}
\newcommand{\Spoon}{{\usefont{U}{cookingsymbols}{m}{n}\symbol{9}}}
\newcommand{\Gloves}{{\usefont{U}{cookingsymbols}{m}{n}\symbol{10}}}

%</package>
%    \end{macrocode}
%
% \subsection{cookingsymbols.mf}
% The following content can be found in the derived file \textsf{cookingsymbols.mf}. In this file the symbols are defined by using \MF. 
%    \begin{macrocode}
%<*mf>

mode_setup;

font_coding_scheme = "cookingsymbols for recipes"; 
font_identifier = "cookingsymbols";

u#:=0.68pt#;
define_pixels(u);

font_size 10pt#;
%design_size=10pt#;
linewidth=0.4pt;

thinpen.w := 0.5linewidth;
pen normalpen, thinpen;
normalpen := pencircle scaled 1linewidth;
thinpen := pencircle scaled thinpen.w;

%    \end{macrocode}
%
% \subsubsection{Oven}
%    \begin{macrocode}
%% oven symbol
"Oven";
beginchar(0,10u#,10u#,0);
  pickup pencircle scaled 0.75 linewidth;

  % Umrandung
  y1=y2; y3=y4;
  x1=x4; x2=x3;
  x2-x1=w;
  y2-y3=h;
  z4=(0u, 0u);
  draw z1--z2--z3--z4--cycle;

  % Ofenklappe
  r:=1u; % Radius
  z5=z4+(r+0.7u, 0.7u);
  x5=x8; x6=x7;
  x9=x12; x10=x11;
  y5=y12; y6=y11;
  y7=y10; y8=y9;
  
  z6=z5+(-r, r);
  z8=z7+(r,r);
  z10=z9+(r,-r);
  x11-x6=w-1.4u; % Breite
  y8-y5=6.3u; % H�he
  
  fill z5{left}..z6{up}..z7{up}..z8{right}..z9{right}..z10{down}..z11{down}..z12{left}..cycle;
  
  % Ofengriff
  b:=3.2u; % Breite vom Griff
  h1:=0.5u; % H�he
  z14=z8+((x9-x8)/2-b/2, -0.8u);
  x14=x13; x15=x16;
  x16=x13+b;
  y14=y13+h1;
  y15=y14;
  y16=y13;
  
  unfill z13--z14--z15--z16--cycle;
  
  % Display
  b:=2.8u; % Breite vom Display
  h1:=1.05u;   % H�he
  z17=z1+(0.8u, -1u);
  x18=x19; x20=x17;
  y18=y17; y19=y20;
  x18-x17=b;
  y18-y19=h1;
  
  fill z17--z18--z19--z20--cycle;
  
  % Kn�pfe
  r:=0.48u; % Radius der Kn�pfe
  b:=1.3u; % Abstand der Kn�pfe untereinander
  h1:=y19+(y18-y19)/2;
  x:=x18+0.8u;
  x21=x23; y24=y22;
  z24=(x,h1);
  z23=(x+r,h1-r);
  x22=x24+2*r;
  y21=y23+2*r;  
  %fill z21..z22..z23..z24..cycle;
  
  for i=0 upto 3:
    fill (z21+(i*1b,0))..(z22+(i*b,0))..(z23+(i*b,0))..(z24+(i*b,0))..cycle;
  endfor
  
  %labels(range 1 thru 24);
endchar;

%    \end{macrocode}
%
% \subsubsection{Top and bottom heat}
%    \begin{macrocode}
%% top and bottomheat symbol
"Topbottomheat";
beginchar(1,11u#,10u#,0);
  pickup normalpen;
  % Umrandung
  ra:=1.0u; % Radius der Umrandung
  y1=y4; y2=y3;
  y5=y8; y6=y7;
  x1=x8; x2=x7;
  x3=x6; x4=x5;
  
  z2=z1+(ra,ra);
  z4=z3+(ra,-ra);
  z8=z7+(-ra,ra);
  x4-x1=w;
  y2-y7=h;
  z8=(0,ra);
  
  draw z1{up}..z2{right}..z3{right}..z4{down}..z5{down}..z6{left}..z7{left}..z8{up}..cycle;
  
  % St�be
  pickup pensquare scaled 1.15linewidth;
  % Oberhitze
  xa:=1.4u; % Abstand zum Rand
  ya:=1.85u;
  z9=(xa, (y2-y7)-ya);
  x10=(x4-x1)-xa;
  y10=y9;
  draw z9..z10;
  
  % Unterhitze
  x11=x9;x12=x10;
  y11=ya;
  y12=y11;
  draw z11..z12;
  
  %labels(range 1 thru 12);

endchar;

%    \end{macrocode}
%
% \subsubsection{Top heat}
% This is nearly the same definition as \textsf{Topbottomheat}, except there is only one rod.
%    \begin{macrocode}
%% top heat symbol
"Topheat";
beginchar(2,11u#,10u#,0);
  pickup normalpen;
  % Umrandung
  ra:=1.0u; % Radius der Umrandung
  y1=y4; y2=y3;
  y5=y8; y6=y7;
  x1=x8; x2=x7;
  x3=x6; x4=x5;
  
  z2=z1+(ra,ra);
  z4=z3+(ra,-ra);
  z8=z7+(-ra,ra);
  x4-x1=w;
  y2-y7=h;
  z8=(0,ra);
  
  draw z1{up}..z2{right}..z3{right}..z4{down}..z5{down}..z6{left}..z7{left}..z8{up}..cycle;
  
  % St�be
  pickup pensquare scaled 1.15linewidth;
  % Oberhitze
  xa:=1.4u; % Abstand zum Rand
  ya:=1.85u;
  z9=(xa, (y2-y7)-ya);
  x10=(x4-x1)-xa;
  y10=y9;
  draw z9..z10;
  
endchar;

%    \end{macrocode}
%
% \subsubsection{Bottom heat}
% This is nearly the same definition as \textsf{Topbottomheat}, except there is only one rod.
%    \begin{macrocode}
%% bottom heat symbol
"Bottomheat";
beginchar(3,11u#,10u#,0);
  pickup normalpen;
  % Umrandung
  ra:=1.0u; % Radius der Umrandung
  y1=y4; y2=y3;
  y5=y8; y6=y7;
  x1=x8; x2=x7;
  x3=x6; x4=x5;
  
  z2=z1+(ra,ra);
  z4=z3+(ra,-ra);
  z8=z7+(-ra,ra);
  x4-x1=w;
  y2-y7=h;
  z8=(0,ra);
  
  draw z1{up}..z2{right}..z3{right}..z4{down}..z5{down}..z6{left}..z7{left}..z8{up}..cycle;
  
  % St�be
  pickup pensquare scaled 1.15linewidth;
  % Oberhitze
  xa:=1.4u; % Abstand zum Rand
  ya:=1.85u;
  z9=(xa, (y2-y7)-ya);
  x10=(x4-x1)-xa;
  y10=y9;
  
  % Unterhitze
  x11=x9;x12=x10;
  y11=ya;
  y12=y11;
  draw z11..z12;
  
endchar;

%    \end{macrocode}
%
% \subsubsection{Fanoven}
%    \begin{macrocode}
%% fanoven symbol
"Fanoven";
beginchar(4,11u#,10u#,0);
  pickup normalpen;
  % Umrandung
  ra:=1.0u; % Radius der Umrandung
  y1=y4; y2=y3;
  y5=y8; y6=y7;
  x1=x8; x2=x7;
  x3=x6; x4=x5;
  
  z2=z1+(ra,ra);
  z4=z3+(ra,-ra);
  z8=z7+(-ra,ra);
  x4-x1=w;
  y2-y7=h;
  z8=(0,ra);
  
  draw z1{up}..z2{right}..z3{right}..z4{down}..z5{down}..z6{left}..z7{left}..z8{up}..cycle;
  
  % Propeller
  z15=(w/2,(h/2)-1.05u); % Rotationspunkt
  a:=1.3; % Skalierungsfaktor
  z9=z15+a*(0.95u,2.66u);
  %z9=z15+(0.8u,3u);
  y10=y9;
  x10=w-x9;
  z11=z9 rotatedaround(z15,120);
  z12=z10 rotatedaround(z15,120);
  z13=z9 rotatedaround(z15,-120);
  z14=z10 rotatedaround(z15,-120);
  
  z16=z15 + a*(0,4u);
  z17=z16 rotatedaround(z15,120);
  z18=z16 rotatedaround(z15,-120);
  
  %draw z15..z9{1,3}..z10{-1,-3}..z15..z13..z14..z15..z11..z12..cycle;
  %draw z9..z16{dir 180}..z10..z13..z18{dir 60}..z14..z11..z17{dir 300}..z12..cycle;
  %fill z9..z16{dir 180}..z10..z13..z18{dir 60}..z14..z11..z17{dir 300}..z12..cycle;
  fill z9{dir 90}..z16{dir 180}..z10{dir 270}..z13{dir -30}..z18{dir 60}..z14{dir 150}..z11{dir 210}..z17{dir 300}..z12{dir 30}..cycle;
  
  %labels(range 1 thru 18);
endchar;

%    \end{macrocode}
%
% \subsubsection{Gasstove}
%    \begin{macrocode}
%% gasstove symbol
"Gasstove";
beginchar(5,11u#,10u#,0);
  pickup normalpen;
  % Umrandung
  ra:=1.0u; % Radius der Umrandung
  y1=y4; y2=y3;
  y5=y8; y6=y7;
  x1=x8; x2=x7;
  x3=x6; x4=x5;
  
  z2=z1+(ra,ra);
  z4=z3+(ra,-ra);
  z8=z7+(-ra,ra);
  x4-x1=w;
  y2-y7=h;
  z8=(0,ra);
  
  draw z1{up}..z2{right}..z3{right}..z4{down}..z5{down}..z6{left}..z7{left}..z8{up}..cycle;
  
  % Flamme
  z9=(w/2,1.0u); % Ursprung
  z10=(w/2,9u);  % Endpunkt
  z11=z9+(-1.4u, 3.1u); 
  x12=w-x11;y12=y11;
  %z12=z9+(1.5u, 3.5u);
  
  fill z9{dir 135}..z11{up}..z10{dir 70}--cycle;
  %fill z9{dir 145}..z10{dir 65}--cycle;
  
  fill z9{dir 45}..z12{up}..z10{dir 110}--cycle;
  
  % Innere Flamme
  z13=z9+(0,0.7u); % Urspung
  z14=z13+(0, 3.7u); % Endpunkt
  z15=z9+(-0.5u,2.1u);
  z16=z9+(0.5u, 2.1u);
  
  unfill z13{dir 130}..z15{up}..z14{dir 65}--cycle;
  
  unfill z13{dir 60}..z16{up}..z14{dir 115}--cycle;
  
  %labels(range 1 thru 16);

endchar;

%    \end{macrocode}
%
% \subsubsection{Dish}
% This is the definition of two circles (with different radii).
%    \begin{macrocode}
%% dish symbol
"Dish";
beginchar(6,10u#,10u#,0);
  pickup normalpen;
  ra:=0.5h;
  x1=x3;y4=y2;
  z4=(0,ra);
  z3=(ra,0);
  x2=x4+2*ra;
  y1=y3+2*ra;
  draw z1..z2..z3..z4..cycle;
  
  ri:=3.7u;
  x5=x7;y8=y6;
  z8=(ra-ri,ra);
  z7=(ra,ra-ri);
  y6=0.5*(y5-y7) + (ra-ri);
  x5=0.5*(x6-x8) + (ra-ri);
  draw z5..z6..z7..z8..cycle;
  
%  labels(range 1 thru 8);
endchar;

%    \end{macrocode}
%
% \subsubsection{Knife}
%    \begin{macrocode}
%% knife symbol
"Knife";
beginchar(7,1.7u#,10u#,0);
  pickup normalpen;
  b:=0.8u; % Breite des Stiels
  z1=(w, 0u);
  x2=x1-b; y2=y1;
  
  z5=(x2,4.8u);  
  z6=z5+(-0.85u, 2u);
  x7=x1; y7=y1+10u;


  fill z1--z2--z5{dir 150}..z6{up}..z7--cycle;
  
  %labels(range 1 thru 11);

endchar;

%    \end{macrocode}
%
% \subsubsection{Fork}
%    \begin{macrocode}
%% fork symbol
"Fork";
beginchar(8,2u#,10u#,0);
  pickup normalpen;

  b:=0.8u;  % Breite des Stiels
  zb:=0.2u; % Zackenbreite
  za=0.4u;  % Zackenabstand
  z1=(w/2+b/2, 0u);
  x2=x1-b; y2=y1;
  x3=x2; y3=y2+6u;
  x4=x1; y4=y3;
  
  fill z1--z2--z3--z4--cycle;
  
  x5=x3-(4*zb+3*za)/2 +b/2;
  y5=y3+1.9u;
  x6=x5; y6=y5+2.1u;
  x7=x6+zb; y7=y6;
  x8=x7; y8=y5;
  
  x9=x8+za; y9=y8;
  x10=x9; y10=y6;
  x11=x10+zb; y11=y10;
  x12=x11; y12=y9;
  
  x13=x12+za; y13=y12;
  x14=x13; y14=y6;
  x15=x14+zb; y15=y14;
  x16=x15; y16=y5;
  
  x17=x16+za; y17=y16;
  x18=x17; y18=y6;
  x19=x18+zb; y19=y18;
  x20=x19; y20=y5; 
  
  fill z3{dir 130}..z5{up}--z6--z7--z8--z9--z10--z11--z12--z13--z14--z15--z16--z17--z18--z19--z20{down}..z4{dir 230}..cycle;
  
  %labels(range 1 thru 20);

endchar;

%    \end{macrocode}
%
% \subsubsection{Spoon}
%    \begin{macrocode}
%% spoon symbol
"Spoon";
beginchar(9, 3.4u#, 10u#,0);
  pickup normalpen;

  b:=0.8u;  % Breite des Stiels
  % Stiel
  z1=(w/2+b/2, 0u);
  x2=x1-b; y2=y1;
  x3=x2; y3=y2+5.74u;
  x4=x1; y4=y3;
  
  fill z1--z2--z3--z4--cycle;
  
  % Oberteil
  x5=x3-1.0u;
  x6=x4+(x3-x5);
  y5=y6=y3+(y7-y3)/2 - 0.2u; % halbe Breite
  z7=(x3+(x4-x3)/2, y3+4u); % y: maximale Ausdehnung
  
  draw z7{dir -170}..z5{down}..(x3+(x4-x3)/2, y3){dir -10};
  draw (x3+(x4-x3)/2, y3){dir 10}..z6{up}..z7{dir 170};%..(z7-(b/2, 0));
  %draw z3{dir 150}..z5{up}..z7{right}; umdrehen!
  %draw z7{right}..z6{down}..z4{dir -150}..z3;
  %draw z3{dir 170}..z5{dir 75}..z7{right}..z6{dir -75}..z4{dir -170}..z3;
  
  %labels(range 1 thru 12);
endchar;

%    \end{macrocode}
%
% \subsubsection{Gloves}
%    \begin{macrocode}
%% gloves symbol
"Gloves";
beginchar(10,10.4u#,10u#,0);
  pickup thinpen;

  a:=1.05; % Skalierungsfaktor
  breite:=a*4.3u; % Breite unten am Handschuh
  hoehe:=a*1.8u;
  
  z5=a*(3.3u-0.4u, 3u); % Rotationspunkt 1. Handschuh
  z9=z5 + a*(-0.55u, 0.1u); % Rotationspunkt 2. Handschuh
  alpha:=-37; % Rotationswinkel -37
  beta:=-26;  % Verschiebung des zweiten Handschuhs bzgl. des oberen
  
  % Oberer Handschuh
  z4=(0.8u, 1u);
  x1=x4; x2=x3;
  y1=y2; y3=y4;
  x2-x1=breite;
  y1-y4=hoehe;
  
  fill z1--z2--z3--z4--cycle rotatedaround(z5, alpha);
  
  % Daumen
  z7=z1 + a*(0.5thinpen.w+0.2u, 3.5u);
  z6=z7 + a*(-1.2u, 1.3u);
  
  draw ((z1+(0.5thinpen.w,0)){up}..z6{dir 75}..z7{dir -80}) rotatedaround(z5, alpha);
  
  % Oberer Rand/Begrenzung
  z8=z1 + ((x2-x1)/2 + a*0.2u, a*8.2u);
  
  draw (z7{dir 90}..z8{right}..(z2+(-0.5thinpen.w,0)){dir -90}) rotatedaround(z5, alpha);
  
  
  
  % Zweiter Handschuh
  fill z1--(z1-(-a*0.8u, (y1-y4)/2+a*0.6u))--z4--cycle rotatedaround(z9, alpha+beta);

  fill z4--z2--z3--cycle rotatedaround(z9, alpha+beta);
  
  %z10=z1 + a*(1u, 7.55u); % Schnittpunkt der beiden Handschuhe
  z10=z1 + a*(1u, 7.75u);
  
  draw (z10..z8{right}..z2{dir -90}) rotatedaround(z9, alpha+beta);
  
  
  
  % 'Wei�er' Trennstrich
  z11=z3 - (0, 0.5thinpen.w);
  x12=x1; y12=y11;
  unfill (z4--z3--z11--z12--cycle) rotatedaround(z5, alpha);
  
  
  %labels(range 1 thru 20);
endchar;

end
%</mf>
%    \end{macrocode}
%
% \Finale
\endinput