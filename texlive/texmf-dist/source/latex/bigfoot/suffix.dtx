% \iffalse
%%
%% perpage is part of the bigfoot bundle for critical typesetting
%% Copyright 2002, 03, 04, 05, 06  David Kastrup <dak@gnu.org>
%%
% This program is free software; you can redistribute it and/or modify
% it under the terms of the GNU General Public License as published by
% the Free Software Foundation; either version 2 of the License, or
% (at your option) any later version.
%
% This program is distributed in the hope that it will be useful,
% but WITHOUT ANY WARRANTY; without even the implied warranty of
% MERCHANTABILITY or FITNESS FOR A PARTICULAR PURPOSE.  See the
% GNU General Public License for more details.
%
% You should have received a copy of the GNU General Public License
% along with this program; if not, write to the Free Software
% Foundation, Inc., 51 Franklin St, Fifth Floor, Boston, MA  02110-1301  USA
% \fi
% \CheckSum{206}
% \GetFileInfo{suffix.sty}
% \author{David Kastrup\thanks{\href{mailto:dak@gnu.org}
%   {Email: \texttt{dak@gnu.org}}}}
% \title{The \texttt{suffix} Package\\Version \fileversion}
% \date{\filedate}
% \maketitle
% \section{Basics}
% The \texttt{suffix} package has the purpose of making it easy to
% define and maintain command variants like |\macro*| and even
% |\macro\/| or similar.  It requires e\TeX\ version~2 for its work.
% The suffixes are fetched with |\futurelet|, so things like |\bgroup|
% and |{| ^^A}
% can't be distinguished.  In addition, the efficiency depends on the
% complexity of the suffix' definition, so you should preferably use
% characters or short commands as a suffix.  A suffixed command itself
% counts as short for this purpose.  How does a suffix definition look
% like?
%
% \DescribeMacro{\WithSuffix} The general form is
% \begin{quote}
%  |\WithSuffix|\meta{prefixed definition}\meta{macro}\meta{suffix}\dots
% \end{quote}
% where \meta{prefixed definition} is something like |\xdef|,
% |\global\let| or similar.  Recognised prefixes are |\global|,
% |\long|, |\protected| (the latter is rarely useful, as the original
% definition already is robust), and |\expandafter| (with its
% `natural' meaning), specially recognized commands are |\gdef| and
% |\xdef|.  Other commands can be used as long as they are suitable as
% an undelimited macro argument.  This means they must either be a
% single token like |\newcommand| or brace-enclosed like
% |{\newcommand*}|.  \meta{macro} can be a macro or an active
% character.  It should be a single token suitable for assignment with
% |\let|.  \meta{suffix} can be something like a single letter such as
% |*| or |[|.
%
% For example, assume that a command |\snarf| already exists and we
% want to define a variant |\snarf|\oarg{optarg}.  Then we can do this
% with
% \begin{quote}
%   \cmd\WithSuffix|\long\def\snarf[#1]{|\meta{Definition using
%       \texttt{\#1}}|}|  
% \end{quote}
% That's pretty much it.  Oh, only when a command is recognised as
% having a prefix |\global| or being |\xdef| or |gdef| will the
% corresponding redefinitions be done globally.  This is rarely a
% concern.
%
% \DescribeMacro{\SuffixName} In case you need to refer to the control
% sequence name used to refer to the suffixed macro, you can access it
% as
% \begin{quote}
%   |\csname|\cmd{\SuffixName}\meta{macro}\meta{suffix}|\endcsname|
% \end{quote}
% \DescribeMacro{\NoSuffixName} and if you need to refer to the
% original unsuffixed macro, you can access it as
% \begin{quote}
%   |\csname|\cmd{\NoSuffixName}\meta{macro}|\endcsname|
% \end{quote}
% \StopEventually{}
% \section{The driver file for the documentation}
% Installation is done by |bigfoot.ins|, so look there for more
% information for that.  Here comes the documentation driver.
%    \begin{macrocode}
%<*driver>
\documentclass{ltxdoc}
\usepackage{hyperref}
\usepackage{suffix}
\begin{document}
\OnlyDescription
%<driver> \AlsoImplementation
\DocInput{suffix.dtx}
\end{document}
%</driver>
%    \end{macrocode}
% \section{Implementation}
% First we announce the package and check for e\TeX~2.
%   \begin{macrocode}
%<*style>
\def\next$#1: #2 #3${#2}
\edef\next{\noexpand
  \ProvidesPackage{suffix}[\next$Date: 2006/07/15 21:24:56 $
  \next$Revision: 1.5 $ Variant command support]}
\next
\ifcase\ifx\eTeXversion\@undefined \@ne\fi
  \ifnum\eTeXversion<\tw@ \@ne\fi\z@
\else
  \PackageError{suffix}{This package requires eTeX version 2}%
  {You might try to use the `elatex' command.}%
\fi
%    \end{macrocode}
% \begin{macro}{\WithSuffix}
%   Then we define the \cmd{\WithSuffix} command.  We use
%   \cmd{\@temptokena} to collect prefixes and let \cmd{\WSF@global}
%   to |\global| for global definitions.
%
%    \begin{macrocode}
\def\WithSuffix{\@temptokena{}\let\WSF@global\relax
  \WSF@sfx}
%    \end{macrocode}
% \end{macro}
% \begin{macro}{\WSF@sfx}
% \begin{macro}{\WSF@append}
% \begin{macro}{\WSF@gobblenext}
%   After checking all prefixes and stuff (we'll fill in this missing
%   link later), we add the defining command itself to the token list,
%   place \meta{macro} into \cmd{\reserved@a} and fetch \meta{suffix}
%   into \cmd{\reserved@b}.
%    \begin{macrocode}
\long\def\WSF@sfx#1#2{\WSF@append{#1}\def\reserved@a{#2}%
  \afterassignment\WSF@decsuff \WSF@gobblenext}

\def\WSF@append#1{\@temptokena\expandafter{\the\@temptokena#1}}

\def\WSF@gobblenext{\let\reserved@b= }
%    \end{macrocode}
% \end{macro}
% \end{macro}
% \end{macro}
% \begin{macro}{\SuffixName}
% \begin{macro}{\NoSuffixName}
%   While we are at it, let us define the macro names to use for
%   suffixed and unsuffixed \meta{macro}.
%    \begin{macrocode}
\long\def\SuffixName#1{WSF:\string#1 \meaning}
\def\NoSuffixName{WSF:\string}
%    \end{macrocode}
% \end{macro}
% \end{macro}
% \begin{macro}{\WSF@decsuff}
%   We first check whether the macro has already been suffixed.  If it
%   hasn't, we save a copy of it and redefine it in terms of
%   \cmd{\WSF@suffixcheck}.
%    \begin{macrocode}
\def\WSF@decsuff{\ifcsname
    \expandafter\NoSuffixName\reserved@a\endcsname
  \else
    \WSF@global\expandafter\let\csname
       \expandafter\NoSuffixName\reserved@a
      \expandafter\endcsname \reserved@a
    \long\def\reserved@c##1{\WSF@global\protected\def
      ##1{\WSF@suffixcheck##1}}%
    \expandafter\reserved@c\reserved@a
  \fi
%    \end{macrocode}
% Once we have done that, we are ready for calling the definition
% command on the suffixed \meta{macro}.
%    \begin{macrocode}
  \WSF@global
    \the\expandafter\@temptokena\csname
    \expandafter \SuffixName
    \reserved@a\reserved@b\endcsname}
%    \end{macrocode}
% \end{macro}
% \begin{macro}{\WSF@suffixcheck}
%   We now do the runtime code.  This is
%   executed in a group of its own in order not to interfere with any
%   other macros.
%    \begin{macrocode}
\def\WSF@suffixcheck#1{\begingroup\def\reserved@a{#1}%
  \futurelet\reserved@b\WSF@suffixcheckii}
%    \end{macrocode}
% \end{macro}
% \begin{macro}{\WSF@suffixcheckii}
%   After assigning the \meta{suffix} to \cmd{\reserved@b}, we split
%   into the case of known and unknown suffix.  We don't code this
%   inline, since \cmd{\reserved@} in a false conditional might
%   confuse \TeX\ if it happens to be something like |\iffalse|
%   itself.
%    \begin{macrocode}
\def\WSF@suffixcheckii{\ifcsname \expandafter\SuffixName
    \reserved@a\reserved@b\endcsname
      \expandafter
      \WSF@suffixcheckiii
    \else
      \expandafter
      \WSF@suffixcheckiv
    \fi}
%    \end{macrocode}
% \end{macro}
% \begin{macro}{\WSF@suffixcheckiii}
% \begin{macro}{\WSF@suffixcheckiv}
%   Calling the macros is reasonably straightforward, we just have to
%   take care not to close the group at the wrong time.
%    \begin{macrocode} 
\def\WSF@suffixcheckiii{%
  \afterassignment\endgroup
  \expandafter\aftergroup
    \csname \expandafter \SuffixName\reserved@a\reserved@b\endcsname
    \WSF@gobblenext}

\def\WSF@suffixcheckiv{%
    \expandafter\endgroup
    \csname \expandafter\NoSuffixName\reserved@a\endcsname}
%    \end{macrocode}
% \end{macro}
% \end{macro}
% \begin{macro}{\WSF@sfx}
%   Now we just augment \cmd{WSF@sfx} to recognize all prefixes and
%   global commands:
%    \begin{macrocode}
\WithSuffix\def\WSF@sfx\long{\WSF@append\long\WSF@sfx}
\WithSuffix\def\WSF@sfx\global{\let\WSF@global\global\WSF@sfx}
\WithSuffix\def\WSF@sfx\protected{\WSF@append\protected\WSF@sfx}
\WithSuffix\def\WSF@sfx\expandafter{\expandafter\WSF@sfx\expandafter}
\WithSuffix\edef\WSF@sfx\gdef{\let\WSF@global\global
  \expandafter\noexpand\csname\NoSuffixName\WSF@sfx\endcsname\def}
\WithSuffix\edef\WSF@sfx\xdef{\let\WSF@global\global
  \expandafter\noexpand\csname\NoSuffixName\WSF@sfx\endcsname\edef}
%</style>
%    \end{macrocode}
%\end{macro}
% \Finale
% \endinput
% Local Variables:
% mode: doctex
% TeX-master: "suffix.drv"
% End:
