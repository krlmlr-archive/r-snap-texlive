% \def\fileversion{0.1}
% \def\filedate{23. januar 2002}
% \def\docdate{23. januar 2002}
%
% \CheckSum{22}
%
% \MakeShortVerb{\|}
%
% \title{D�dningehovedafbildningen...\thanks{Pakken foreligger i
% �jeblikket i version \fileversion}}
% \author{Henrik Chr.~Grove}
% \date{Sidst rettet \filedate}
% \maketitle
%
% \begin{abstract}
% \changes{v0.1}{23. januar 2002}{F\o{}rste offentliggjorte version}
% En \MF-font med et d�dningehoved, og en \LaTeX-pakke til at
% g�re det tilg�ngeligt som et matematisk symbol.
% \end{abstract}
%
% \section{Introduktion}
% 
% Under en tavlegennemgang af et matematisk problem, kom en af mine
% medstuderende for skade at ben�vne en "`ond"' afbildning, med et
% d�dningehoved. Det kan man sagtens p� en tavle, men der fandtes
% tilsyneladende ikke nogen skrifttype til \LaTeX{} der indeholdt
% s�dan et.
%
% Nu (l�nge efter) er det lykkedes mig at �ndre p� denne situation. 
%
% \section{Brug af pakken}
%
% Skrifttype og pakke er s� simple at bruge som man kan �nske sig
% det. Man fort�ller \LaTeX{} at den skal bruge pakken med en
% |\usepackage{skull}|, og derefter har man i matematiktilstand
% (f.eks. mellem \$, eller mellem |\begin{equation}| og
% |\end{equation}|) adgang til kommandoen |\skull|, der producerer et
% $\skull$.
%
% \section{Installation}
%
% Pakken distribueres i tre filer |skull.mf|, |skull.dtx| og |skull.ins|. 
% Pakken installeres ved at k\o{}re |skull.ins| gennem \LaTeX{},
% hvilket genererer filen |skull.sty|, dette er den egentlige pakke
% der anvendes af \LaTeX. Dokumentationen kr�ver at du har installeret
% pakken f�rst, derefter skal du blot k�re |latex skull.dtx| to gange.
%
%\StopEventually{
% \tableofcontents
% 
% \begin{thebibliography}{9}
%  \bibitem{Companion} M.~Goosens, F.~Mittelbach and A.~Samarin:
%   \emph{The \LaTeX{} Companion}, Addison-Wesley 1994.
%  \bibitem{Metafont} Donald~E.~Knuth: \emph{The \MF book},
%  Addison-Wesley 1986.
% \end{thebibliography}
%}
% \section{Koden}
%
% \label{afs:kode}
% F\o{}rst s\o{}rger vi for, at man kan generere dokumentationen ved 
% at k\o{}re LaTeX{} p\aa{} denne fil. Dokumentationen skal 
% s\ae{}ttes p\aa{} A4-papir, og er p\aa{} dansk. Fjern procenttegnet
% (\%) foran |\OnlyDescription| for at generere dokumentationen uden 
% dette afsnit.
%    \begin{macrocode}
%<*driver>
\documentclass[12pt,a4paper]{ltxdoc}
\usepackage{a4,ifthen,mflogo,skull,amssymb,amsmath}
\usepackage[latin1]{inputenc}
\usepackage[danish]{babel} 
\usepackage[T1]{fontenc}
%\OnlyDescription
\EnableCrossrefs
\RecordChanges
\CodelineIndex
\OldMakeindexcd
\begin{document}
\DocInput{skull.dtx}
\end{document}
%</driver>
%    \end{macrocode}
%
% \noindent Flere trivialiteter.
%    \begin{macrocode}
\NeedsTeXFormat{LaTeX2e}[1995/12/01]
\ProvidesPackage{skull}
  [2002/01/23 v0.1 (c) Henrik Christian Grove <grove@math.ku.dk>]
%    \end{macrocode}
%
% \section{Selve koden}
% 
% Hvis man kender til \textsf{NFSS} skulle dette v�re let at forst�,
% ellers s� l�s om \textsf{NFSS} i \cite{Companion}.
%
%    \begin{macrocode}
\DeclareFontFamily{U}{skulls}{}
\DeclareFontShape{U}{skulls}{m}{n}{ <-> skull }{}
\DeclareSymbolFont{SKULL}{U}{skulls}{m}{n}
\DeclareMathSymbol{\skull}{\mathalpha}{SKULL}{'101}
%    \end{macrocode}
%
% \noindent Og det var det\ldots M�ske burde jeg se p� om det var
% muligt ogs� at f� \MF-koden ind her\ldots 
%    \begin{macrocode}
\endinput
%    \end{macrocode}
%
%%% \pagebreak[2]
%
% \IndexPrologue{{\section*{Indeks}%
%  \markboth{Indeks}{Indeks}%
%  Tal skrevet med kursiv henviser til siden hvor det
%  p\aa{}g\ae{}ldende opslag er beskrevet, de understregede til
%  den kodelinie hvor man kan finde definitionen, resten til
%  kodelinierne hvor opslaget bruges.}}
%
% \PrintChanges
%
% \PrintIndex
%
% \Finale
