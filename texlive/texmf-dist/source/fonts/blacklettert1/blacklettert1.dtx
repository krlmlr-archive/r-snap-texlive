% \iffalse meta-comment
%       $Id: blacklettert1.dtx,v 1.3 2002/05/30 18:16:53 torsten Exp torsten $    
%
%     blacklettert1.dtx -- Blackletter Typefaces in T1 Encoding
%     Copyright 2003 Torsten Bronger <torsten.bronger@gmx.de>
%
%   This program may be distributed and/or modified under the
%   conditions of the LaTeX Project Public License, either version 1.2
%   of this license or (at your option) any later version.
%   The latest version of this license is in
%     http://www.latex-project.org/lppl.txt
%   and version 1.2 or later is part of all distributions of LaTeX
%   version 1999/12/01 or later.
%
% \fi
%
% \CheckSum{5465}
% \CharacterTable
%  {Upper-case    \A\B\C\D\E\F\G\H\I\J\K\L\M\N\O\P\Q\R\S\T\U\V\W\X\Y\Z
%   Lower-case    \a\b\c\d\e\f\g\h\i\j\k\l\m\n\o\p\q\r\s\t\u\v\w\x\y\z
%   Digits        \0\1\2\3\4\5\6\7\8\9
%   Exclamation   \!     Double quote  \"     Hash (number) \#
%   Dollar        \$     Percent       \%     Ampersand     \&
%   Acute accent  \'     Left paren    \(     Right paren   \)
%   Asterisk      \*     Plus          \+     Comma         \,
%   Minus         \-     Point         \.     Solidus       \/
%   Colon         \:     Semicolon     \;     Less than     \<
%   Equals        \=     Greater than  \>     Question mark \?
%   Commercial at \@     Left bracket  \[     Backslash     \\
%   Right bracket \]     Circumflex    \^     Underscore    \_
%   Grave accent  \`     Left brace    \{     Vertical bar  \|
%   Right brace   \}     Tilde         \~}
% 
%
% \def\fileversion{v1.1}
% \def\filedate{2003/02/17}
%
% \title{Blackletter Typefaces in T1 Encoding\thanks{This file
%        has version number \fileversion, last
%        revised \filedate.}}
% \author{Torsten Bronger\\
%         \url{bronger@physik.rwth-aachen.de}}
%
% \maketitle
% 
% \begin{abstract}
% This package |blacklettert1| provides virtual fonts for T1-like variants of the
% blackletter typefaces published in 1990 by Yannis Haralambous, namely yfrak,
% yswab and ygoth. Their original encoding is unspecified.  Additionally, this
% package embeds these fonts nicely into \LaTeXe's font selection scheme.  The
% structure of this package allows for further blackletter typefaces from other
% sources to be made usable for \LaTeX\ in a similar way, too.
% \end{abstract}
% 
% \tableofcontents
%
% \section*{The documentation driver file}
%
% The next bit of code contains the documentation driver file for
% \TeX, i.\,e., the file that will produce the documentation you are
% currently reading. It will be extracted from this file by the
% \texttt{docstrip} program. Since it is the first code in the file
% one can alternatively process this file directly with \LaTeXe{} to
% obtain the documentation.
%
%    \begin{macrocode}
%<*driver>
\documentclass{ltxdoc}

%    \end{macrocode}
% If you comment out the next line, you get a \emph{full} description
% of this package.
%    \begin{macrocode}
\OnlyDescription

%    \end{macrocode}
% Not clean at all, but it works.
%    \begin{macrocode}
\newlabel{codingscheme}{{??}{??}}

\setcounter{StandardModuleDepth}{1}

\usepackage{url}
\providecommand*{\url}[1]{\texttt{\mbox{#1}}}

\usepackage{mathptmx}

\begin{document}
   \DocInput{blacklettert1.dtx}
   \PrintChanges
   \PrintIndex
\end{document}
%</driver>
%    \end{macrocode}
%
% \section{Introduction}
%
% In 1990, Yannis Haralambous published a set of three blackletter typefaces.
% (Very often, though inaccurately, such typefaces are also called ``Gothic''
% or ``old German'' typefaces.)
%
% These three fonts are yfrak (Fraktur), yswab (Schwabacher) and ygoth
% (Textur), and Haralambous made them freely available.  They were created
% using Metafont. Unfortunately, although all three fonts are very nicely
% drawn and well endowed with ligatures and special symbols, they are produced
% in a very awkward encoding, making it very inconvenient to use them.
% Therefore in 1994 Thierry Bouche wrote dcfrak, a Metafont font that by and
% large was a T1 (Cork) encoded re-implementation of Haralambous' yfrak.
% Bouche added blacker accents.
%
% His work has three flaws: It only covers yfrak, it is a new font (virtual
% fonts are prefered if possible) and it doesn't make use of \TeX's boundary
% character feature that was especially designed having blackletter typefaces and
% old Latin scripts in mind.
%
%
% \section{This package}
%
% \dots~provides virtual fonts and a (nearly) T1 encoded Fraktur family that
% can be used very easily with \LaTeX.  In fact, if you switch to the font by
% saying
%\begin{verbatim}
%\fontfamily{yfrak}\selectfont
%\end{verbatim}
% (no special package is needed, but T1 must be the current encoding) all you
% still have to do is to break up certain ligatures with \verb|\/| commands.
% You could even use the same \LaTeX\ source text for both Fraktur and, say,
% Times output.
%
% This approach has several advantages:
% \begin{itemize}
% \item Virtual fonts are easy to install and elegantly make use of already
%   installed fonts,
% \item you don't have to get adjusted to the quite strange input format
%   Haralambous suggested, actually the impact on the source text is (not zero,
%   but) relatively small,
% \item makes it possible to return to Latin typefaces without further changes,
% \item makes proper hyphenation in whatever language possible,
% \item makes spell checkers that can cope with \LaTeX\ work also for Fraktur
%   texts,
% \item allows for letterspaced version and
% \item embeds yfrak, yswab and ygoth nicely in the font selection scheme of
%   \LaTeX.
% \end{itemize}
%
% Additionally, it solves some design flaws of the original fonts:
% \begin{itemize}
%   \item ygoth is slightly letterspaced.
%   \item The letters i, l and fi of ygoth are now correctly positioned within 
%     their boxes.
%   \item Many too big and too small letter boxes are adjusted.
%   \item Missing dieresis accent added to ygoth and yfrak.
%   \item Inter-word space is reduced.
% \end{itemize}
%
% I have to admit one disadvantage of this virtual font approach: Some DVI
% viewers still can't handle them properly.  In this case, please complain to
% the DVI driver author.  At least dvips has no problem.
%
% \medskip The package bases on a modified version of the T1 encoding, but
% only unusual slots got a new contents.  This T1 variant can be used for
% other blackletter typefaces, too.  See the full documentation of this package for
% further details.
%
% \section{Installation}
%
% You can use the already compiled \mbox{|vf|-,} |tfm|- and |fd|-files.  Just
% copy them in directories where \LaTeX\ and your DVI driver can find them.
% (Typically where other files with the same respective extension already
% exist.)
%
% Alternatively you can use the Makefile and give the commands
%\begin{verbatim}
%make
%make install
%\end{verbatim}
% (At least the latter as root.)  This will create fresh versions of the
% virtual fonts and install them.  Please have a look at the Makefile, whether
% all paths are set as you need them.  If you want to compile the fonts, please
% notice that you need special PL files for yfrak, yswab, ygoth and cmu10.
% The problem is that they must have set a certain \texttt{CODINGSCHEME}. See
% page~\pageref{codingscheme} (this page is unknown if you have set
% \mbox{\texttt{\char'134OnlyDescription}}) for further details.
%
% Maybe it's necessary to run a program like te\TeX's |texhash| -- or
% something equivalent of your \TeX\ implementation -- after the installation
% process.
%
% \section{Using the fonts}
%
% \emph{Warning}: The original fonts (|.mf| or |.pfb|) must be available for
% your DVI driver before this package can be used!  They are not part of this
% package.
%
% \subsection{Switching to Fraktur}
%
% This package doesn't contain a |sty| file that you can include with a
% |\usepackage| command.  If you switch to yfrak in the T1 encoding with
% something like
%\begin{verbatim}
%\fontfamily{yfrak}\selectfont
%\end{verbatim}
% you get Fraktur.  The |\emph| version will be a neatly letterspaced Fraktur,
% |\textsl| is Schwabacher. In the preamble you can say
%\begin{verbatim}
%\renewcommand{\emph}[1]{\textsl{#1}}
%\end{verbatim}
% and Schwabacher will be the default |\emph| markup, if you prefer that.
% Bold face is Textur, so that Textur will be used for the headings (standard
% \LaTeX\ behaviour) which is common use.
%
%You can make the Fraktur font selection global for the whole document by
% saying
%\begin{verbatim}
%\renewcommand{\rmdefault}{yfrak}
%\end{verbatim}
% in the preamble.
%
% \subsection{Typographic difficulties}
%
% A very important point is breaking up ligatures with |\/| or -- in German
% texts -- with \verb!"|!.  Blackletter typefaces have many of them, but at subword
% boundaries you have to split them up.  The German word ``aus\/tragen'' is
% written in the source as ``\verb!aus"|tragen!'', because ``aus'' is a
% prefix.
%
% Blackletter typefaces have two lower case ``s'': A \emph{round} one, the one we
% know, and a \emph{long} one, similar of an ``f''.  In German, the latter has
% survived as the left part in the ligature ``\ss''.  One must choose the
% correct form for every occurrence.  As a rule of thumb, the round form at
% the end of the word, the long form everywhere else.  \TeX\ will do that
% automatically.  However, sometimes a round ``s'' is necessary within a word,
% at the end of a subword boundary.
%
% From \TeX's point of view, the letter of ``s'' within or at the beginning of
% words is a ligature, too.  A typical German case is ``aussetzen'' that must
% be input ``\verb!aus"|setzen!''.  Sometimes you will have to input a |\/| as
% in |kafkaes\/k|.  But there are also English examples: |news\/letter| --
% although Knuth insists on the ligature.~\texttt{\mbox{;-)}}
%
%
%
% \section{Remarks}
%
% The author of this package would like to encourage people to use Fraktur,
% for small bits of their texts, for special moods or just for aesthetical
% considerations.  Technically speaking, I try to lower the threshold for
% using these typefaces by making the access a lot easier.  They are of course
% not for everyday texts.
%
% But it's really a pity that many people think that such blackletter typefaces are
% old use, or connect them with German nationalism, especially the Nazi
% era. The first is not strictly true (at least not for German speaking
% regions), the latter is very unfair.  Only few people know that the Nazis
% themselves forbade them in 1941, because they were ``Jewish letters'' -- the
% ban being as silly as the given reason.
%
% One should consider blackletter typefaces another degree of freedom in the font
% space.  They are not only to revive old texts; in fact, many font foundries
% produce -- albeit only a few -- new Fraktur variants.
%
% \medskip To non-German authors: Although Fraktur and Schwabacher have been
% used almost entirely for German texts, there is really no reason to limit it
% to that.  Their parents and sisters, Gothic, Textura, Rotunda, Bastarda
% etc.\ have been heavily used for Latin, Italian, French and English.
% European and American designers have created beautiful Fraktur alphabets.
% A.~Kapr wrote, ``Schwabacher and Fraktur are just branches of the Latin
% script tree, tended by many nations.''
%
% 
% \section{Related work}
%
% The |fraktur| package by Matthias Muehlich (FixMe:~Link!)\ is another
% approach to bring blackletter typefaces to \LaTeX.  The main differences are:
% \begin{itemize}
%   \item He provides virtual font solutions for many commercial fonts (but
%     not for Haralambous' typefaces).
%   \item He uses another encoding (FixMe:~Name?)\ which is quite much different
%     from T1.  It covers some additional letters and ligatures.
% \end{itemize}
% If you want to have a ready-to-run solution for those typefaces (some of
% which are free for private use), this package will be helpful for you.
%
% \bigskip Walter Schmidt's |yfonts| package provides a more direct access to
% Haralambous' fonts, including the initials font |yinitas|.  This is especially
% useful if you want to typeset old texts, with all ligatures and Gutenberg
% feeling.
%
% \StopEventually{}
%
%
% \section{The \texttt{fontinst} input file}
%
% This file (|yutoyt1.tex|) is the driver file for the T1 representation
% of Yannis Haralambous' blackletter typefaces.  It it called by
%\begin{verbatim}
%tex yutoyt1
%\end{verbatim}
%    \begin{macrocode}
%<*yutoyt1tex>
\input fontinst.sty

\def\requiredversion{1.801}
\ifx\fontinstversion\requiredversion\else\errmessage{fontinst 1.801 needed.}\fi

%    \end{macrocode}
% The first step is quite unusual: I have to define some new encodings, 
% because Haralambous invented three different encodings for his three fonts,
% and, of course, also incompatible with all other encodings on this planet.
%
% \emph{Warning:}\label{codingscheme} That's the reason why you can't use this
% package with the \emph{original} TFM files.  You must use the PL files
% supplied with this package, because the PL files must contain the correct
% name for their encoding.  Although that's the only line that's been
% modified, it's crucial.
%    \begin{macrocode}
\declareencoding{YH GOTISCH}{ygoth}
\declareencoding{YH SCHWABACHER}{yswab}
\declareencoding{YH FRAKTUR}{yfrak}
%    \end{macrocode}
% The same is true for |cmu10.pl|:  It has a slightly different encoding
% (it contains a sterling sign instead of a dollar), but it doesn't say so
% in the tfm file header.  By the way, the |ot1i| encoding is part of the
% standard \texttt{fontinst} distribution.
%    \begin{macrocode}
\declareencoding{TEX ITALIC TEXT}{ot1i}
\installfonts
%    \end{macrocode}
% I didn't rename the family name because I don't know of another T1 version
% of yfrak.  I think it's not a lie when I claim that it's still yfrak, but
% in another encoding.
%
% I need another hyphen character because otherwise a long ``s'' may become
% round immediately before a hyphen.
%    \begin{macrocode}
\installfamily{T1}{yfrak}{\hyphenchar\font=127}

%    \end{macrocode}
% First, the normal version.  The tfm file will be called |tfrak.tfm|.
%    \begin{macrocode}
\installfont{tfrak}{yfrak,kernoff,yswab scaled 700,setglyph,%
                    cmr10,cmmi10,cmsy10,%
                    unsetacc,cmr7 scaled 700,ecrm1000,%
                    unsetcm,ecrm0700 scaled 700,kernon,%
                    unsetdia,yfrak,kernoff,unsetste,cmu10,kernon,%
                    fcleanup,blackletter}{T1frak}
  {T1}{yfrak}{m}{n}{}
%    \end{macrocode}
% Secondly, the letterspaced version.  By and large, it's a copy from above.
% Notice that I annihilate almost all kerning, in fact, only one pair 
% survives!  The latterspaced version is mapped to the italic variant.  This
% should transform legacy document automatically, and it helps to 
% ``re-antiqua'' Fraktur texts easily.
%    \begin{macrocode}
\installfont{tfrakls}{kernoff,yfrak,yswab scaled 700,setglyph,%
                    cmr10,cmmi10,cmsy10,%
                    unsetacc,cmr7 scaled 700,ecrm1000,%
                    unsetcm,ecrm0700 scaled 700,%
                    unsetdia,yfrak,unsetste,cmu10,%
                    fcleanup,kernon,lettersp,kernoff,blackletter}{T1frak}
  {T1}{yfrak}{m}{it}{}
%    \end{macrocode}
% The |\textsl| version is |tswab|, the T1 variant of |yswab|.  It is
% much simpler, largely because it doesn't need smaller accents.
%    \begin{macrocode}
\installfont{tswab}{yswab,setglyph,kernoff,cmbx10,cmmib10,cmbsy10,%
                    unsetcm,ecbx1000,kernon,%
                    unsetdia,yswab,scleanup,blackletter}{T1frak}
  {T1}{yfrak}{m}{sl}{}
%    \end{macrocode}
% Very closely related to the |tswab| solution is |tgoth|, which is mapped to
% the bold version of yfrak in the T1 encoding.  The only differences are in
% |gcleanup.mtx|.
%    \begin{macrocode}
\installfont{tgoth}{ygoth,setglyph,kernoff,cmbx10,cmmib10,cmbsy10,%
                    unsetcm,ecbx1000,kernon,%
                    unsetdia,ygoth,gcleanup,blackletter}{T1frak}
  {T1}{yfrak}{b}{n}{}
\endinstallfonts

\bye
%</yutoyt1tex>
%    \end{macrocode}
%
%
% \section{The \texttt{yfrak} text encoding vector}
%
% Here I describe the original encoding of the yfrak font by Yannis 
% Haralambous.  This file is read when |yfrak.pl| is converted into 
% |yfrak.mtx| in order to be digested by \texttt{fontinst}.
%
% FixMe: Probably this file could easily be simplified very
% much, especially the metric setting may be superfluous.  But I don't know.
%    \begin{macrocode}
%<*yfraketx>
\relax

\documentclass[twocolumn]{article}
\usepackage{fontdoc}

\title{The \texttt{yfrak} text encoding vector}
\author{Torsten Bronger}
\date{10 April 2002 \\
Version 1.0}

\begin{document}
\maketitle

\section{Introduction}

This document describes the encoding of yfrak (fraktur).

\encoding

\needsfontinstversion{1.801}


\comment{\section{Default values}}

\setcommand\lc#1#2{#2}
\setcommand\uc#1#2{#1}
\setcommand\lclig#1#2{#2}
\setcommand\uclig#1#2{#1}
\setcommand\digit#1{#1}
\setcommand\tty{n}
\setcommand\currency{dollar}

\setstr{codingscheme}{UNSPECIFIED}

\setint{italicslant}{0}

\ifisglyph{x}\then
   \setint{xheight}{\height{x}}
\else
   \setint{xheight}{500}
\fi

\ifisglyph{space}\then
   \setint{interword}{\width{space}}
\else\ifisglyph{i}\then
   \setint{interword}{\width{i}}
\else
   \setint{interword}{333}
\fi\fi


\comment{\section{Default font dimensions}}

\setint{fontdimen(1)}{\int{italicslant}}              % italic slant
\setint{fontdimen(2)}{\int{interword}}                % interword space
\ifisint{monowidth}\then
   \setint{fontdimen(3)}{0}                           % interword stretch
   \setint{fontdimen(4)}{0}                           % interword shrink
\else
   \setint{fontdimen(3)}{\scale{\int{interword}}{600}}% interword stretch
   \setint{fontdimen(4)}{\scale{\int{interword}}{240}}% interword shrink
\fi
\setint{fontdimen(5)}{\int{xheight}}                  % x-height
\setint{fontdimen(6)}{1000}                           % quad
\ifisint{monowidth}\then
   \setint{fontdimen(7)}{\int{interword}}             % extra space after .
\else
   \setint{fontdimen(7)}{\scale{\int{interword}}{240}}% extra space after .
\fi

\setint{boundarychar}{32}

\comment{\section{The encoding}}

\nextslot{"10}

\setslot{dotlessi}
   \comment{A dotless i like `\i'}
\endsetslot

\setslot{dotlessj}
   \comment{A dotless j like `\j'}
\endsetslot

\setslot{grave}
   \comment{The grave accent `\`a'.}
\endsetslot

\setslot{acute}
   \comment{The acute accent `\'a'.}
\endsetslot

\setslot{caron}
   \comment{The caron or h\'a\v cek accent `\v a'.}
\endsetslot

\setslot{breve}
   \comment{The breve accent `\u a'.}
\endsetslot

\setslot{macron}
   \comment{The macron accent `\=a'.}
\endsetslot

\setslot{ringfitted}
   \comment{The ring accent `\aa', fitted to be the same width as an
      upper case `A'.}
\endsetslot

\setslot{cedilla}
   \comment{The cedilla accent `\c a'.}
\endsetslot

\skipslots{1}

\setslot{\lc{SS}{germandbls}}
   \comment{The letter `\ss'.}
\endsetslot

\nextslot{"21}

\setslot{exclam}
   \comment{The exclamation mark `!'.}
\endsetslot

\setslot{quotedblleft}
  \comment{The English opening quote mark `\,\textquotedblleft\,'.}
\endsetslot

\setslot{numbersign}
   \comment{The hash sign `\#'.}
\endsetslot

\setslot{varJ}
   \comment{Variant of the letter `{J}'.}
\endsetslot

\setslot{percent}
   \comment{The percent sign `\%'.}
\endsetslot

\setslot{ampersand}
   \comment{The ampersand sign `\&'.}
\endsetslot

\setslot{quoteright}
   \ligature{LIG}{quoteright}{quotedblright}
   \comment{The English closing quotation mark `\,'\,'.}
\endsetslot

\setslot{parenleft}
   \comment{The opening parenthesis `('.}
\endsetslot

\setslot{parenright}
   \comment{The closing parenthesis `)'.}
\endsetslot

\setslot{asterisk}
   \comment{The raised asterisk `*'.}
\endsetslot

\setslot{plus}
   \comment{The addition sign `+'.}
\endsetslot

\setslot{comma}
   \comment{The comma `,'.}
\endsetslot

\setslot{hyphen}
   \ligature{LIG}{hyphen}{rangedash}
   \comment{The hyphen `-'.}
\endsetslot

\setslot{period}
   \comment{The full point `.'.}
\endsetslot

\setslot{slash}
   \comment{The forward oblique `/'.}
\endsetslot

\setslot{\digit{zero}}
   \comment{The number zero `0'.  This (and all the other numerals) may be
      old style or ranging digits.}
\endsetslot

\setslot{\digit{one}}
   \comment{The number one `1'.}
\endsetslot

\setslot{\digit{two}}
   \comment{The number two `2'.}
\endsetslot

\setslot{\digit{three}}
   \comment{The number three `3'.}
\endsetslot

\setslot{\digit{four}}
   \comment{The number four `4'.}
\endsetslot

\setslot{\digit{five}}
   \comment{The number five `5'.}
\endsetslot

\setslot{\digit{six}}
   \comment{The number six `6'.}
\endsetslot

\setslot{\digit{seven}}
   \comment{The number seven `7'.}
\endsetslot

\setslot{\digit{eight}}
   \comment{The number eight `8'.}
\endsetslot

\setslot{\digit{nine}}
   \comment{The number nine `9'.}
\endsetslot

\setslot{colon}
   \comment{The colon punctuation mark `:'.}
\endsetslot

\setslot{semicolon}
   \comment{The semi-colon punctuation mark `;'.}
\endsetslot

\setslot{section}
   \comment{The section sign `{\S}'.}
\endsetslot

\setslot{equal}
   \comment{The equals sign `='.}
\endsetslot

\skipslots{1}

\setslot{question}
   \comment{The question mark `?'.}
\endsetslot

\skipslots{1}

\setslot{\uc{A}{a}}
   \comment{The letter `{A}'.}
\endsetslot

\setslot{\uc{B}{b}}
   \comment{The letter `{B}'.}
\endsetslot

\setslot{\uc{C}{c}}
   \comment{The letter `{C}'.}
\endsetslot

\setslot{\uc{D}{d}}
   \comment{The letter `{D}'.}
\endsetslot

\setslot{\uc{E}{e}}
   \comment{The letter `{E}'.}
\endsetslot

\setslot{\uc{F}{f}}
   \comment{The letter `{F}'.}
\endsetslot

\setslot{\uc{G}{g}}
   \comment{The letter `{G}'.}
\endsetslot

\setslot{\uc{H}{h}}
   \comment{The letter `{H}'.}
\endsetslot

\setslot{\uc{I}{i}}
   \comment{The letter `{I}'.}
\endsetslot

\setslot{\uc{J}{j}}
   \comment{The letter `{J}'.}
\endsetslot

\setslot{\uc{K}{k}}
   \comment{The letter `{K}'.}
\endsetslot

\setslot{\uc{L}{l}}
   \comment{The letter `{L}'.}
\endsetslot

\setslot{\uc{M}{m}}
   \comment{The letter `{M}'.}
\endsetslot

\setslot{\uc{N}{n}}
   \comment{The letter `{N}'.}
\endsetslot

\setslot{\uc{O}{o}}
   \comment{The letter `{O}'.}
\endsetslot

\setslot{\uc{P}{p}}
   \comment{The letter `{P}'.}
\endsetslot

\setslot{\uc{Q}{q}}
   \comment{The letter `{Q}'.}
\endsetslot

\setslot{\uc{R}{r}}
   \comment{The letter `{R}'.}
\endsetslot

\setslot{\uc{S}{s}}
   \comment{The letter `{S}'.}
\endsetslot

\setslot{\uc{T}{t}}
   \comment{The letter `{T}'.}
\endsetslot

\setslot{\uc{U}{u}}
   \comment{The letter `{U}'.}
\endsetslot

\setslot{\uc{V}{v}}
   \comment{The letter `{V}'.}
\endsetslot

\setslot{\uc{W}{w}}
   \comment{The letter `{W}'.}
\endsetslot

\setslot{\uc{X}{x}}
   \comment{The letter `{X}'.}
\endsetslot

\setslot{\uc{Y}{y}}
   \comment{The letter `{Y}'.}
\endsetslot

\setslot{\uc{Z}{z}}
   \comment{The letter `{Z}'.}
\endsetslot

\setslot{bracketleft}
   \comment{The opening square bracket `['.}
\endsetslot

\setslot{quotedblbase}
  \comment{A German double quote mark `,\kern-0.1em,' similar to two commas,
      but with tighter letterspacing and different sidebearings.}
\endsetslot

\setslot{bracketright}
   \comment{The closing square bracket `]'.}
\endsetslot

\setslot{circumflex}
   \comment{The circumflex accent `\^ a'.}
\endsetslot

\setslot{dotaccent}
   \comment{The dot accent `\.a'.}
\endsetslot

\setslot{quoteleft}
   \ligature{LIG}{quoteleft}{quotedblleft}
   \comment{The English opening single quotation mark `\,`\,'.}
\endsetslot

\setslot{\lc{A}{a}}
   \comment{The letter `{a}'.}
\endsetslot

\setslot{\lc{B}{b}}
   \comment{The letter `{b}'.}
\endsetslot

\setslot{\lc{C}{c}}
   \comment{The letter `{c}'.}
\endsetslot

\setslot{\lc{D}{d}}
   \comment{The letter `{d}'.}
\endsetslot

\setslot{\lc{E}{e}}
   \comment{The letter `{e}'.}
\endsetslot

\setslot{\lc{F}{f}}
   \ligature{LIG}{\lc{I}{i}}{\lclig{FI}{fi}}
   \ligature{LIG}{\lc{F}{f}}{\lclig{FF}{ff}}
   \ligature{LIG}{\lc{L}{l}}{\lclig{FL}{fl}}
   \comment{The letter `{f}'.}
\endsetslot

\setslot{\lc{G}{g}}
   \comment{The letter `{g}'.}
\endsetslot

\setslot{\lc{H}{h}}
   \comment{The letter `{h}'.}
\endsetslot

\setslot{\lc{I}{i}}
   \comment{The letter `{i}'.}
\endsetslot

\setslot{\lc{J}{j}}
   \comment{The letter `{j}'.}
\endsetslot

\setslot{\lc{K}{k}}
   \comment{The letter `{k}'.}
\endsetslot

\setslot{\lc{L}{l}}
   \comment{The letter `{l}'.}
\endsetslot

\setslot{\lc{M}{m}}
   \comment{The letter `{m}'.}
\endsetslot

\setslot{\lc{N}{n}}
   \comment{The letter `{n}'.}
\endsetslot

\setslot{\lc{O}{o}}
   \comment{The letter `{o}'.}
\endsetslot

\setslot{\lc{P}{p}}
   \comment{The letter `{p}'.}
\endsetslot

\setslot{\lc{Q}{q}}
   \comment{The letter `{q}'.}
\endsetslot

\setslot{\lc{R}{r}}
   \comment{The letter `{r}'.}
\endsetslot

\setslot{\lc{S}{s}}
   \comment{The letter `{s}'.}
\endsetslot

\setslot{\lc{T}{t}}
   \comment{The letter `{t}'.}
\endsetslot

\setslot{\lc{U}{u}}
   \comment{The letter `{u}'.}
\endsetslot

\setslot{\lc{V}{v}}
   \comment{The letter `{v}'.}
\endsetslot

\setslot{\lc{W}{w}}
   \comment{The letter `{w}'.}
\endsetslot

\setslot{\lc{X}{x}}
   \comment{The letter `{x}'.}
\endsetslot

\setslot{\lc{Y}{y}}
   \comment{The letter `{y}'.}
\endsetslot

\setslot{\lc{Z}{z}}
   \comment{The letter `{z}'.}
\endsetslot

\setslot{rangedash}
   \ligature{LIG}{hyphen}{punctdash}
   \comment{The number range dash `1--9'.  In a monowidth font, this
      might be set as `{\tt 1{-}9}'.}
\endsetslot

\setslot{punctdash}
   \comment{The punctuation dash `Oh---boy.'  In a monowidth font, this
      might be set as `{\tt Oh{-}{-}boy.}'}
\endsetslot

\setslot{hungarumlaut}
   \comment{The long Hungarian umlaut `\H a'.}
\endsetslot

\setslot{tilde}
   \comment{The tilde accent `\~a'.}
\endsetslot

\comment{\section{Non-standard ligatures}}

\nextslot{"81}

\setslot{ss}
   \comment{The `ss' ligature.}
\endsetslot

\setslot{st}
   \comment{The `st' ligature.}
\endsetslot

\setslot{sf}
   \comment{The `sf' ligature.}
\endsetslot

\setslot{ff}
   \comment{The `ff' ligature.}
\endsetslot

\setslot{ch}
   \comment{The `ch' ligature.}
\endsetslot

\setslot{ck}
   \comment{The `ck' ligature.}
\endsetslot

\skipslots{2}

\setslot{asmalle}
   \comment{The letter `a' with little e at top.}
\endsetslot

\setslot{adieresis}
   \comment{the letter `\"a'}
\endsetslot

\setslot{tz}
   \comment{The `tz' ligature.}
\endsetslot

\skipslots{1}

\setslot{sround}
   \comment{The letter round `s'.}
\endsetslot

\skipslots{2}

\setslot{esmalle}
   \comment{The letter `e' with little e at top.}
\endsetslot

\setslot{edieresis}
   \comment{the letter `\"e'}
\endsetslot

\nextslot{"99}

\setslot{osmalle}
   \comment{The letter `o' with little e at top.}
\endsetslot

\setslot{odieresis}
   \comment{the letter `\"o'}
\endsetslot

\skipslots{3}

\setslot{usmalle}
   \comment{The letter `u' with little e at top.}
\endsetslot

\setslot{udieresis}
   \comment{the letter `\"u'}
\endsetslot

\nextslot{"A4}

\setslot{varsection}
   \comment{Variant of section sign `\S'.}
\endsetslot

\nextslot{"C9}

\setslot{etc}
   \comment{The abbreviation sign `etc'.}
\endsetslot


\endencoding
\end{document}
%</yfraketx>
%    \end{macrocode}
%
% 
% \section{The \texttt{yswab} text encoding vector}
%
% Here I describe the original encoding of the yswab font by Yannis 
% Haralambous.  This file is read when |yswab.pl| is converted into 
% |yswab.mtx| in order to be digested by \texttt{fontinst}.
%
% FixMe: Probably this file could easily be simplified very
% much, especially the metric setting may be superfluous.  But I don't know.
%    \begin{macrocode}
%<*yswabetx>
\relax

\documentclass[twocolumn]{article}
\usepackage{fontdoc}

\title{The \texttt{yswab} text encoding vector}
\author{Torsten Bronger}
\date{10 April 2002 \\
Version 1.0}

\begin{document}
\maketitle

\section{Introduction}

This document describes the encoding of yswab (schwabacher).

\encoding

\needsfontinstversion{1.801}


\comment{\section{Default values}}

\setcommand\lc#1#2{#2}
\setcommand\uc#1#2{#1}
\setcommand\lclig#1#2{#2}
\setcommand\uclig#1#2{#1}
\setcommand\digit#1{#1}
\setcommand\tty{n}
\setcommand\currency{dollar}

\setstr{codingscheme}{UNSPECIFIED}

\setint{italicslant}{0}

\ifisglyph{x}\then
   \setint{xheight}{\height{x}}
\else
   \setint{xheight}{500}
\fi

\ifisglyph{space}\then
   \setint{interword}{\width{space}}
\else\ifisglyph{i}\then
   \setint{interword}{\width{i}}
\else
   \setint{interword}{333}
\fi\fi


\comment{\section{Default font dimensions}}

\setint{fontdimen(1)}{\int{italicslant}}              % italic slant
\setint{fontdimen(2)}{\int{interword}}                % interword space
\ifisint{monowidth}\then
   \setint{fontdimen(3)}{0}                           % interword stretch
   \setint{fontdimen(4)}{0}                           % interword shrink
\else
   \setint{fontdimen(3)}{\scale{\int{interword}}{600}}% interword stretch
   \setint{fontdimen(4)}{\scale{\int{interword}}{240}}% interword shrink
\fi
\setint{fontdimen(5)}{\int{xheight}}                  % x-height
\setint{fontdimen(6)}{1000}                           % quad
\ifisint{monowidth}\then
   \setint{fontdimen(7)}{\int{interword}}             % extra space after .
\else
   \setint{fontdimen(7)}{\scale{\int{interword}}{240}}% extra space after .
\fi

\comment{\section{The encoding}}

\nextslot{"10}

\setslot{dotlessi}
   \comment{A dotless i like `\i'}
\endsetslot

\setslot{dotlessj}
   \comment{A dotless j like `\j'}
\endsetslot

\setslot{grave}
   \comment{The grave accent `\`a'.}
\endsetslot

\setslot{acute}
   \comment{The acute accent `\'a'.}
\endsetslot

\setslot{caron}
   \comment{The caron or h\'a\v cek accent `\v a'.}
\endsetslot

\setslot{breve}
   \comment{The breve accent `\u a'.}
\endsetslot

\setslot{macron}
   \comment{The macron accent `\=a'.}
\endsetslot

\setslot{ringfitted}
   \comment{The ring accent `\aa', fitted to be the same width as an
      upper case `A'.}
\endsetslot

\setslot{cedilla}
   \comment{The cedilla accent `\c a'.}
\endsetslot

\skipslots{1}

\setslot{\lc{SS}{germandbls}}
   \comment{The letter `\ss'.}
\endsetslot

\nextslot{"21}

\setslot{exclam}
   \comment{The exclamation mark `!'.}
\endsetslot

\setslot{quotedblleft}
  \comment{The English opening quote mark `\,\textquotedblleft\,'.}
\endsetslot

\setslot{numbersign}
   \comment{The hash sign `\#'.}
\endsetslot

\skipslots{1}

\setslot{percent}
   \comment{The percent sign `\%'.}
\endsetslot

\skipslots{1}

\setslot{quoteright}
   \ligature{LIG}{quoteright}{quotedblright}
   \comment{The English closing quotation mark `\,'\,'.}
\endsetslot

\setslot{parenleft}
   \comment{The opening parenthesis `('.}
\endsetslot

\setslot{parenright}
   \comment{The closing parenthesis `)'.}
\endsetslot

\setslot{asterisk}
   \comment{The raised asterisk `*'.}
\endsetslot

\setslot{plus}
   \comment{The addition sign `+'.}
\endsetslot

\setslot{comma}
   \comment{The comma `,'.}
\endsetslot

\setslot{hyphen}
   \ligature{LIG}{hyphen}{rangedash}
   \comment{The hyphen `-'.}
\endsetslot

\setslot{period}
   \comment{The full point `.'.}
\endsetslot

\setslot{slash}
   \comment{The forward oblique `/'.}
\endsetslot

\setslot{\digit{zero}}
   \comment{The number zero `0'.  This (and all the other numerals) may be
      old style or ranging digits.}
\endsetslot

\setslot{\digit{one}}
   \comment{The number one `1'.}
\endsetslot

\setslot{\digit{two}}
   \comment{The number two `2'.}
\endsetslot

\setslot{\digit{three}}
   \comment{The number three `3'.}
\endsetslot

\setslot{\digit{four}}
   \comment{The number four `4'.}
\endsetslot

\setslot{\digit{five}}
   \comment{The number five `5'.}
\endsetslot

\setslot{\digit{six}}
   \comment{The number six `6'.}
\endsetslot

\setslot{\digit{seven}}
   \comment{The number seven `7'.}
\endsetslot

\setslot{\digit{eight}}
   \comment{The number eight `8'.}
\endsetslot

\setslot{\digit{nine}}
   \comment{The number nine `9'.}
\endsetslot

\setslot{colon}
   \comment{The colon punctuation mark `:'.}
\endsetslot

\skipslots{1}

\setslot{section}
   \comment{The section sign `{\S}'.}
\endsetslot

\setslot{equal}
   \comment{The equals sign `='.}
\endsetslot

\skipslots{1}

\setslot{question}
   \comment{The question mark `?'.}
\endsetslot

\skipslots{1}

\setslot{\uc{A}{a}}
   \comment{The letter `{A}'.}
\endsetslot

\setslot{\uc{B}{b}}
   \comment{The letter `{B}'.}
\endsetslot

\setslot{\uc{C}{c}}
   \comment{The letter `{C}'.}
\endsetslot

\setslot{\uc{D}{d}}
   \comment{The letter `{D}'.}
\endsetslot

\setslot{\uc{E}{e}}
   \comment{The letter `{E}'.}
\endsetslot

\setslot{\uc{F}{f}}
   \comment{The letter `{F}'.}
\endsetslot

\setslot{\uc{G}{g}}
   \comment{The letter `{G}'.}
\endsetslot

\setslot{\uc{H}{h}}
   \comment{The letter `{H}'.}
\endsetslot

\setslot{\uc{I}{i}}
   \comment{The letter `{I}'.}
\endsetslot

\setslot{\uc{J}{j}}
   \comment{The letter `{J}'.}
\endsetslot

\setslot{\uc{K}{k}}
   \comment{The letter `{K}'.}
\endsetslot

\setslot{\uc{L}{l}}
   \comment{The letter `{L}'.}
\endsetslot

\setslot{\uc{M}{m}}
   \comment{The letter `{M}'.}
\endsetslot

\setslot{\uc{N}{n}}
   \comment{The letter `{N}'.}
\endsetslot

\setslot{\uc{O}{o}}
   \comment{The letter `{O}'.}
\endsetslot

\setslot{\uc{P}{p}}
   \comment{The letter `{P}'.}
\endsetslot

\setslot{\uc{Q}{q}}
   \comment{The letter `{Q}'.}
\endsetslot

\setslot{\uc{R}{r}}
   \comment{The letter `{R}'.}
\endsetslot

\setslot{\uc{S}{s}}
   \comment{The letter `{S}'.}
\endsetslot

\setslot{\uc{T}{t}}
   \comment{The letter `{T}'.}
\endsetslot

\setslot{\uc{U}{u}}
   \comment{The letter `{U}'.}
\endsetslot

\setslot{\uc{V}{v}}
   \comment{The letter `{V}'.}
\endsetslot

\setslot{\uc{W}{w}}
   \comment{The letter `{W}'.}
\endsetslot

\setslot{\uc{X}{x}}
   \comment{The letter `{X}'.}
\endsetslot

\setslot{\uc{Y}{y}}
   \comment{The letter `{Y}'.}
\endsetslot

\setslot{\uc{Z}{z}}
   \comment{The letter `{Z}'.}
\endsetslot

\setslot{bracketleft}
   \comment{The opening square bracket `['.}
\endsetslot

\setslot{quotedblbase}
  \comment{A German double quote mark `,\kern-0.1em,' similar to two commas,
      but with tighter letterspacing and different sidebearings.}
\endsetslot

\setslot{bracketright}
   \comment{The closing square bracket `]'.}
\endsetslot

\setslot{circumflex}
   \comment{The circumflex accent `\^ a'.}
\endsetslot

\setslot{dotaccent}
   \comment{The dot accent `\.a'.}
\endsetslot

\setslot{quoteleft}
   \ligature{LIG}{quoteleft}{quotedblleft}
   \comment{The English opening single quotation mark `\,`\,'.}
\endsetslot

\setslot{\lc{A}{a}}
   \comment{The letter `{a}'.}
\endsetslot

\setslot{\lc{B}{b}}
   \comment{The letter `{b}'.}
\endsetslot

\setslot{\lc{C}{c}}
   \comment{The letter `{c}'.}
\endsetslot

\setslot{\lc{D}{d}}
   \comment{The letter `{d}'.}
\endsetslot

\setslot{\lc{E}{e}}
   \comment{The letter `{e}'.}
\endsetslot

\setslot{\lc{F}{f}}
   \ligature{LIG}{\lc{I}{i}}{\lclig{FI}{fi}}
   \ligature{LIG}{\lc{F}{f}}{\lclig{FF}{ff}}
   \ligature{LIG}{\lc{L}{l}}{\lclig{FL}{fl}}
   \comment{The letter `{f}'.}
\endsetslot

\setslot{\lc{G}{g}}
   \comment{The letter `{g}'.}
\endsetslot

\setslot{\lc{H}{h}}
   \comment{The letter `{h}'.}
\endsetslot

\setslot{\lc{I}{i}}
   \comment{The letter `{i}'.}
\endsetslot

\setslot{\lc{J}{j}}
   \comment{The letter `{j}'.}
\endsetslot

\setslot{\lc{K}{k}}
   \comment{The letter `{k}'.}
\endsetslot

\setslot{\lc{L}{l}}
   \comment{The letter `{l}'.}
\endsetslot

\setslot{\lc{M}{m}}
   \comment{The letter `{m}'.}
\endsetslot

\setslot{\lc{N}{n}}
   \comment{The letter `{n}'.}
\endsetslot

\setslot{\lc{O}{o}}
   \comment{The letter `{o}'.}
\endsetslot

\setslot{\lc{P}{p}}
   \comment{The letter `{p}'.}
\endsetslot

\setslot{\lc{Q}{q}}
   \comment{The letter `{q}'.}
\endsetslot

\setslot{\lc{R}{r}}
   \comment{The letter `{r}'.}
\endsetslot

\setslot{\lc{S}{s}}
   \comment{The letter `{s}'.}
\endsetslot

\setslot{\lc{T}{t}}
   \comment{The letter `{t}'.}
\endsetslot

\setslot{\lc{U}{u}}
   \comment{The letter `{u}'.}
\endsetslot

\setslot{\lc{V}{v}}
   \comment{The letter `{v}'.}
\endsetslot

\setslot{\lc{W}{w}}
   \comment{The letter `{w}'.}
\endsetslot

\setslot{\lc{X}{x}}
   \comment{The letter `{x}'.}
\endsetslot

\setslot{\lc{Y}{y}}
   \comment{The letter `{y}'.}
\endsetslot

\setslot{\lc{Z}{z}}
   \comment{The letter `{z}'.}
\endsetslot

\setslot{rangedash}
   \ligature{LIG}{hyphen}{punctdash}
   \comment{The number range dash `1--9'.  In a monowidth font, this
      might be set as `{\tt 1{-}9}'.}
\endsetslot

\setslot{punctdash}
   \comment{The punctuation dash `Oh---boy.'  In a monowidth font, this
      might be set as `{\tt Oh{-}{-}boy.}'}
\endsetslot

\setslot{dieresis}
   \comment{The umlaut or dieresis accent `\"{}'.}
\endsetslot

\setslot{tilde}
   \comment{The tilde accent `\~a'.}
\endsetslot

\comment{\section{Non-standard ligatures}}

\nextslot{"81}

\setslot{ss}
   \comment{The `ss' ligature.}
\endsetslot

\setslot{st}
   \comment{The `st' ligature.}
\endsetslot

\setslot{sf}
   \comment{The `sf' ligature.}
\endsetslot

\setslot{ff}
   \comment{The `ff' ligature.}
\endsetslot

\setslot{ch}
   \comment{The `ch' ligature.}
\endsetslot

\setslot{ck}
   \comment{The `ck' ligature.}
\endsetslot

\skipslots{2}

\setslot{asmalle}
   \comment{The letter `a' with little e at top.}
\endsetslot

\setslot{adieresis}
   \comment{the letter `\"a'}
\endsetslot

\skipslots{2}

\setslot{sround}
   \comment{The letter round `s'.}
\endsetslot

\skipslots{2}

\setslot{esmalle}
   \comment{The letter `e' with little e at top.}
\endsetslot

\setslot{edieresis}
   \comment{the letter `\"e'}
\endsetslot

\nextslot{"99}

\setslot{osmalle}
   \comment{The letter `o' with little e at top.}
\endsetslot

\setslot{odieresis}
   \comment{the letter `\"o'}
\endsetslot

\skipslots{3}

\setslot{usmalle}
   \comment{The letter `u' with little e at top.}
\endsetslot

\setslot{udieresis}
   \comment{the letter `\"u'}
\endsetslot

\nextslot{"A4}

\setslot{varsection}
   \comment{Variant of section sign `\S'.}
\endsetslot

\skipslots{2}

\setslot{\lc{varSS}{vargermandbls}}
   \comment{A variant of the letter `\ss'.}
\endsetslot


\endencoding
\end{document}
%</yswabetx>
%    \end{macrocode}
%
%
% \section{The \texttt{ygoth} text encoding vector}
%
% Here I describe the original encoding of the ygoth font by Yannis 
% Haralambous.  This file is read when |ygoth.pl| is converted into 
% |ygoth.mtx| in order to be digested by \texttt{fontinst}.
%
% FixMe: Probably this file could easily be simplified very
% much, especially the metric setting may be superfluous.  But I don't know.
%    \begin{macrocode}
%<*ygothetx>
\relax

\documentclass[twocolumn]{article}
\usepackage{fontdoc}

\title{The \texttt{ygoth} text encoding vector}
\author{Torsten Bronger}
\date{10 April 2002 \\
Version 1.0}

\begin{document}
\maketitle

\section{Introduction}

This document describes the encoding of ygoth (textur).

\encoding

\needsfontinstversion{1.801}


\comment{\section{Default values}}

\setcommand\lc#1#2{#2}
\setcommand\uc#1#2{#1}
\setcommand\lclig#1#2{#2}
\setcommand\uclig#1#2{#1}
\setcommand\digit#1{#1}
\setcommand\tty{n}
\setcommand\currency{dollar}

\setstr{codingscheme}{UNSPECIFIED}

\setint{italicslant}{0}

\ifisglyph{x}\then
   \setint{xheight}{\height{x}}
\else
   \setint{xheight}{500}
\fi

\ifisglyph{space}\then
   \setint{interword}{\width{space}}
\else\ifisglyph{i}\then
   \setint{interword}{\width{i}}
\else
   \setint{interword}{333}
\fi\fi


\comment{\section{Default font dimensions}}

\setint{fontdimen(1)}{\int{italicslant}}              % italic slant
\setint{fontdimen(2)}{\int{interword}}                % interword space
\ifisint{monowidth}\then
   \setint{fontdimen(3)}{0}                           % interword stretch
   \setint{fontdimen(4)}{0}                           % interword shrink
\else
   \setint{fontdimen(3)}{\scale{\int{interword}}{600}}% interword stretch
   \setint{fontdimen(4)}{\scale{\int{interword}}{240}}% interword shrink
\fi
\setint{fontdimen(5)}{\int{xheight}}                  % x-height
\setint{fontdimen(6)}{1000}                           % quad
\ifisint{monowidth}\then
   \setint{fontdimen(7)}{\int{interword}}             % extra space after .
\else
   \setint{fontdimen(7)}{\scale{\int{interword}}{240}}% extra space after .
\fi

\comment{\section{The encoding}}

\nextslot{0}

\setslot{ba}
   \comment{the ligature `ba'}
\endsetslot

\setslot{be}
   \comment{the ligature `be'}
\endsetslot

\setslot{bo}
   \comment{the ligature `bo'}
\endsetslot

\setslot{ch}
   \comment{the ligature `ch'}
\endsetslot

\setslot{ck}
   \comment{the ligature `ck'}
\endsetslot

\setslot{ct}
   \comment{the ligature `ct'}
\endsetslot

\setslot{da}
   \comment{the ligature `da'}
\endsetslot

\setslot{de}
   \comment{the ligature `de'}
\endsetslot

\setslot{do}
   \comment{the ligature `do'}
\endsetslot

\setslot{ha}
   \comment{the ligature `ha'}
\endsetslot

\setslot{he}
   \comment{the ligature `he'}
\endsetslot

\setslot{ff}
   \comment{the ligature `ff'}
\endsetslot

\setslot{fi}
   \comment{the ligature `fi'}
\endsetslot

\setslot{fl}
   \comment{the ligature `fl'}
\endsetslot

\setslot{ffi}
   \comment{the ligature `ffi'}
\endsetslot

\setslot{ffl}
   \comment{the ligature `ffl'}
\endsetslot

\setslot{dotlessi}
   \comment{A dotless i like `\i'}
\endsetslot

\setslot{dotlessj}
   \comment{A dotless j like `\j'}
\endsetslot

\setslot{ho}
   \comment{the ligature `ho'}
\endsetslot

\setslot{pa}
   \comment{the ligature `pa'}
\endsetslot

\setslot{pe}
   \comment{the ligature `pe'}
\endsetslot

\setslot{po}
   \comment{the ligature `po'}
\endsetslot

\setslot{ij}
   \comment{the ligature `ij'}
\endsetslot

\setslot{qz}
   \comment{the ligature `qz'}
\endsetslot

\setslot{va}
   \comment{the ligature `va'}
\endsetslot

\setslot{\lc{SS}{germandbls}}
   \comment{The letter `\ss'.}
\endsetslot

\setslot{\lc{AE}{ae}}
   \comment{The letter `\ae'.  This is a single letter, and should not be
      faked with `ae'.}
\endsetslot

\setslot{\lc{OE}{oe}}
   \comment{The letter `\oe'.  This is a single letter, and should not be
      faked with `oe'.}
\endsetslot

\setslot{\lc{Oslash}{oslash}}
   \comment{The letter `\o'.}
\endsetslot

\setslot{ll}
   \comment{the ligature `ll'}
\endsetslot

\setslot{ve}
   \comment{the ligature `ve'}
\endsetslot

\setslot{vu}
   \comment{the ligature `vu'}
\endsetslot

\nextslot{"21}

\setslot{exclam}
   \comment{The exclamation mark `!'.}
\endsetslot

\setslot{quotedblright}
   \comment{An English double closing quote mark `\,''\,'.}
\endsetslot

\setslot{pp}
   \comment{the ligature `pp'}
\endsetslot

\setslot{qq}
   \comment{the ligature `qq'}
\endsetslot

\skipslots{1}

\setslot{ss}
   \comment{the ligature `ss'}
\endsetslot

\setslot{quoteright}
   \ligature{LIG}{quoteright}{quotedblright}
   \comment{The English closing quotation mark `\,'\,'.}
\endsetslot

\setslot{parenleft}
   \comment{The opening parenthesis `('.}
\endsetslot

\setslot{parenright}
   \comment{The closing parenthesis `)'.}
\endsetslot

\setslot{ssvar}
   \comment{the ligature `ss' (variant)}
\endsetslot

\setslot{ssi}
   \comment{the ligature `ssi'}
\endsetslot

\setslot{comma}
   \comment{The comma `,'.}
\endsetslot

\setslot{hyphen}
   \ligature{LIG}{hyphen}{rangedash}
   \comment{The hyphen `-'.}
\endsetslot

\setslot{period}
   \comment{The full point `.'.}
\endsetslot

\setslot{ssivar}
   \comment{the ligature `ssi' (variant)}
\endsetslot

\setslot{\digit{zero}}
   \comment{The number zero `0'.  This (and all the other numerals) may be
      old style or ranging digits.}
\endsetslot

\setslot{\digit{one}}
   \comment{The number one `1'.}
\endsetslot

\setslot{\digit{two}}
   \comment{The number two `2'.}
\endsetslot

\setslot{\digit{three}}
   \comment{The number three `3'.}
\endsetslot

\setslot{\digit{four}}
   \comment{The number four `4'.}
\endsetslot

\setslot{\digit{five}}
   \comment{The number five `5'.}
\endsetslot

\setslot{\digit{six}}
   \comment{The number six `6'.}
\endsetslot

\setslot{\digit{seven}}
   \comment{The number seven `7'.}
\endsetslot

\setslot{\digit{eight}}
   \comment{The number eight `8'.}
\endsetslot

\setslot{\digit{nine}}
   \comment{The number nine `9'.}
\endsetslot

\setslot{colon}
   \comment{The colon punctuation mark `:'.}
\endsetslot

\setslot{semicolon}
   \comment{The semi-colon punctuation mark `;'.}
\endsetslot

\setslot{st}
   \comment{the ligature `st'}
\endsetslot

\setslot{stvar}
   \comment{the ligature `st' (variant)}
\endsetslot

\setslot{tz}
   \comment{the ligature `tz'}
\endsetslot

\setslot{question}
   \comment{The question mark `?'.}
\endsetslot

\skipslots{1}

\setslot{\uc{A}{a}}
   \comment{The letter `{A}'.}
\endsetslot

\setslot{\uc{B}{b}}
   \comment{The letter `{B}'.}
\endsetslot

\setslot{\uc{C}{c}}
   \comment{The letter `{C}'.}
\endsetslot

\setslot{\uc{D}{d}}
   \comment{The letter `{D}'.}
\endsetslot

\setslot{\uc{E}{e}}
   \comment{The letter `{E}'.}
\endsetslot

\setslot{\uc{F}{f}}
   \comment{The letter `{F}'.}
\endsetslot

\setslot{\uc{G}{g}}
   \comment{The letter `{G}'.}
\endsetslot

\setslot{\uc{H}{h}}
   \comment{The letter `{H}'.}
\endsetslot

\setslot{\uc{I}{i}}
   \comment{The letter `{I}'.}
\endsetslot

\setslot{\uc{J}{j}}
   \comment{The letter `{J}'.}
\endsetslot

\setslot{\uc{K}{k}}
   \comment{The letter `{K}'.}
\endsetslot

\setslot{\uc{L}{l}}
   \comment{The letter `{L}'.}
\endsetslot

\setslot{\uc{M}{m}}
   \comment{The letter `{M}'.}
\endsetslot

\setslot{\uc{N}{n}}
   \comment{The letter `{N}'.}
\endsetslot

\setslot{\uc{O}{o}}
   \comment{The letter `{O}'.}
\endsetslot

\setslot{\uc{P}{p}}
   \comment{The letter `{P}'.}
\endsetslot

\setslot{\uc{Q}{q}}
   \comment{The letter `{Q}'.}
\endsetslot

\setslot{\uc{R}{r}}
   \comment{The letter `{R}'.}
\endsetslot

\setslot{\uc{S}{s}}
   \comment{The letter `{S}'.}
\endsetslot

\setslot{\uc{T}{t}}
   \comment{The letter `{T}'.}
\endsetslot

\setslot{\uc{U}{u}}
   \comment{The letter `{U}'.}
\endsetslot

\setslot{\uc{V}{v}}
   \comment{The letter `{V}'.}
\endsetslot

\setslot{\uc{W}{w}}
   \comment{The letter `{W}'.}
\endsetslot

\setslot{\uc{X}{x}}
   \comment{The letter `{X}'.}
\endsetslot

\setslot{\uc{Y}{y}}
   \comment{The letter `{Y}'.}
\endsetslot

\setslot{\uc{Z}{z}}
   \comment{The letter `{Z}'.}
\endsetslot

\setslot{adieresis}
   \comment{the letter `\"a'}
\endsetslot

\setslot{edieresis}
   \comment{the letter `\"e'}
\endsetslot

\setslot{odieresis}
   \comment{the letter `\"o'}
\endsetslot

\setslot{udieresis}
   \comment{the letter `\"u'}
\endsetslot

\setslot{quotedblleft}
  \comment{The English opening quote mark `\,\textquotedblleft\,'.}
\endsetslot

\setslot{sround}
   \comment{The letter round `s'.}
\endsetslot

\setslot{\lc{A}{a}}
   \comment{The letter `{a}'.}
\endsetslot

\setslot{\lc{B}{b}}
   \comment{The letter `{b}'.}
\endsetslot

\setslot{\lc{C}{c}}
   \comment{The letter `{c}'.}
\endsetslot

\setslot{\lc{D}{d}}
   \comment{The letter `{d}'.}
\endsetslot

\setslot{\lc{E}{e}}
   \comment{The letter `{e}'.}
\endsetslot

\setslot{\lc{F}{f}}
   \ligature{LIG}{\lc{I}{i}}{\lclig{FI}{fi}}
   \ligature{LIG}{\lc{F}{f}}{\lclig{FF}{ff}}
   \ligature{LIG}{\lc{L}{l}}{\lclig{FL}{fl}}
   \comment{The letter `{f}'.}
\endsetslot

\setslot{\lc{G}{g}}
   \comment{The letter `{g}'.}
\endsetslot

\setslot{\lc{H}{h}}
   \comment{The letter `{h}'.}
\endsetslot

\setslot{\lc{I}{i}}
   \comment{The letter `{i}'.}
\endsetslot

\setslot{\lc{J}{j}}
   \comment{The letter `{j}'.}
\endsetslot

\setslot{\lc{K}{k}}
   \comment{The letter `{k}'.}
\endsetslot

\setslot{\lc{L}{l}}
   \comment{The letter `{l}'.}
\endsetslot

\setslot{\lc{M}{m}}
   \comment{The letter `{m}'.}
\endsetslot

\setslot{\lc{N}{n}}
   \comment{The letter `{n}'.}
\endsetslot

\setslot{\lc{O}{o}}
   \comment{The letter `{o}'.}
\endsetslot

\setslot{\lc{P}{p}}
   \comment{The letter `{p}'.}
\endsetslot

\setslot{\lc{Q}{q}}
   \comment{The letter `{q}'.}
\endsetslot

\setslot{\lc{R}{r}}
   \comment{The letter `{r}'.}
\endsetslot

\setslot{\lc{S}{s}}
   \comment{The letter `{s}'.}
\endsetslot

\setslot{\lc{T}{t}}
   \comment{The letter `{t}'.}
\endsetslot

\setslot{\lc{U}{u}}
   \comment{The letter `{u}'.}
\endsetslot

\setslot{\lc{V}{v}}
   \comment{The letter `{v}'.}
\endsetslot

\setslot{\lc{W}{w}}
   \comment{The letter `{w}'.}
\endsetslot

\setslot{\lc{X}{x}}
   \comment{The letter `{x}'.}
\endsetslot

\setslot{\lc{Y}{y}}
   \comment{The letter `{y}'.}
\endsetslot

\setslot{\lc{Z}{z}}
   \comment{The letter `{z}'.}
\endsetslot

\setslot{rangedash}
   \ligature{LIG}{hyphen}{punctdash}
   \comment{The number range dash `1--9'.  In a monowidth font, this
      might be set as `{\tt 1{-}9}'.}
\endsetslot

\setslot{punctdash}
   \comment{The punctuation dash `Oh---boy.'  In a monowidth font, this
      might be set as `{\tt Oh{-}{-}boy.}'}
\endsetslot

\endencoding
\end{document}
%</ygothetx>
%    \end{macrocode}
%
%
% \section{First cleaning up}
%
% In this file (|setglyph.mtx|), we do two things: First, we move some
% ligatures from the poriginal encoding to the correct position in my 
% pseudo-T1, ans secondly we unset some glyphs that shouldn't be used from
% the Haralambous font.
% 
%    \begin{macrocode}
%<*setglyphmtx>
\relax

\metrics

%    \end{macrocode}  
%
% \subsection{Moving the ligatures}
%
% E.\,g., the ``ck'' ligature must end up in the T1 slot called
% ``ffl ligature''.  Here I do so.  Of cause, only if the respective ligature
% is available at all.
%    \begin{macrocode}
\setcommand\setglyphmaybe#1#2{%
\unsetglyph{#1}
\ifisglyph{#2}\then
\setglyph{#1}
  \glyph{#2}{1000}
\endsetglyph
\fi
}

\setglyphmaybe{dotlessj}{si}
\setglyphmaybe{ffi}{ch}
\setglyphmaybe{ffl}{ck}
\setglyphmaybe{backslash}{ft}
\setglyphmaybe{asciicircum}{ss}
\setglyphmaybe{underscore}{st}
\setglyphmaybe{bar}{tz}

%    \end{macrocode}
%
% \subsection{Taking accents from CM}
%
% All ligatures are taken from CM, only the Hungarian umlaut comes from EC
% (see below).  The Haralambous fonts don't contain all of them anyway.
%    \begin{macrocode}
\unsetglyph{circumflex}
\unsetglyph{breve}
\unsetglyph{ring}
\unsetglyph{macron}
\unsetglyph{caron}
\unsetglyph{tilde}
\unsetglyph{quotedblleft}
\unsetglyph{quotedblright}
\unsetglyph{quotedblbase}
\unsetglyph{tilde}
\unsetglyph{slash}


\endmetrics
%</setglyphmtx>
%    \end{macrocode}
%
%
% \section{Making the accents smaller}
%
% This part will eventually come to the file |unsetacc.mtx|.  I want to take
% the accents not from the original computer modern or their T1 variant, but
% from the seven point version, at least for the Fraktur font.  These smaller
% accents look nicer in my opinion, because the dainty Fraktur letters can't
% cope with usual ten point accents.
%    \begin{macrocode}
%<*unsetaccmtx>
\relax

\metrics

%    \end{macrocode}
% The following three lines save the old height of the tilde character in a
% variable called |oldtildeheight| in order to be able to shift the smaller
% accents upwards to the correct height later.  Probably this was superfluous,
% because I have to tweak the heights anyway.
%    \begin{macrocode}
\setint{oldtildeheight}{\height{tilde}}
\def\temp{\int{oldtildeheight}}
\xdef\oldtildeheight{\the\temp}

\unsetglyph{grave}
\unsetglyph{acute}
\unsetglyph{circumflex}
\unsetglyph{tilde}
\unsetglyph{ring}
\unsetglyph{caron}
\unsetglyph{breve}
\unsetglyph{dotaccent}

\endmetrics
%</unsetaccmtx>
%    \end{macrocode}
%
%
% \section{The Hungarian umlaut}
%
% This wanders eventually to the file |unsetcm.mtx| and does not more than
% just unsetting the Hungarian umlaut accent.  The reason for that is that
% this symbol looks much nicer in the EC fonts, especially because the normal
% dieresis accent looks wuite similar to the CM version of the Hungarian 
% umlaut, which is potentially dangerous.  (Hungarian uses both.)
%    \begin{macrocode}
%<*unsetcmmtx>
\relax

\metrics

\unsetglyph{hungarumlaut}

\endmetrics
%</unsetcmmtx>
%    \end{macrocode}
%
%
% \section{Unsetting the diacritics}
%
% In this section (eventually the file |unsetdia.mtx|), I make all
% pathologial latter, i.\,e.\ all letters with diacritic signs, undefined.
% After that, I can set some critical characters manually in the ``cleanup''
% files (see below), ar I can let them be constrcuted in |blackletter.mtx|.
%    \begin{macrocode}
%<*unsetdiamtx>
\relax

\metrics

\unsetglyph{hyphenchar}

\unsetglyph{Abreve}
\unsetglyph{Aogonek}
\unsetglyph{Cacute}
\unsetglyph{Ccaron}
\unsetglyph{Dcaron}
\unsetglyph{Ecaron}
\unsetglyph{Eogonek}
\unsetglyph{Gbreve}
\unsetglyph{Lacute}
\unsetglyph{Lcaron}
\unsetglyph{Lslash}
\unsetglyph{Nacute}
\unsetglyph{Ncaron}
\unsetglyph{Ng}
\unsetglyph{Ohungarumlaut}
\unsetglyph{Racute}
\unsetglyph{Rcaron}
\unsetglyph{Sacute}
\unsetglyph{Scaron}
\unsetglyph{Scedilla}
\unsetglyph{Tcaron}
\unsetglyph{Tcedilla}
\unsetglyph{Uhungarumlaut}
\unsetglyph{Uring}
\unsetglyph{Ydieresis}
\unsetglyph{Zacute}
\unsetglyph{Zcaron}
\unsetglyph{Zdotaccent}
\unsetglyph{IJ}
\unsetglyph{Idotaccent}
\unsetglyph{dbar}

\unsetglyph{abreve}
\unsetglyph{aogonek}
\unsetglyph{cacute}
\unsetglyph{ccaron}
\unsetglyph{dcaron}
\unsetglyph{ecaron}
\unsetglyph{eogonek}
\unsetglyph{gbreve}
\unsetglyph{lacute}
\unsetglyph{lcaron}
\unsetglyph{lslash}
\unsetglyph{nacute}
\unsetglyph{ncaron}
\unsetglyph{ng}
\unsetglyph{ohungarumlaut}
\unsetglyph{racute}
\unsetglyph{rcaron}
\unsetglyph{sacute}
\unsetglyph{scaron}
\unsetglyph{scedilla}
\unsetglyph{tcaron}
\unsetglyph{tcedilla}
\unsetglyph{uhungarumlaut}
\unsetglyph{uring}
\unsetglyph{ydieresis}
\unsetglyph{zacute}
\unsetglyph{zcaron}
\unsetglyph{zdotaccent}
\unsetglyph{ij}

\unsetglyph{Agrave}
\unsetglyph{Aacute}
\unsetglyph{Acircumflex}
\unsetglyph{Atilde}
\unsetglyph{Adieresis}
\unsetglyph{Aring}
\unsetglyph{AE}
\unsetglyph{Ccedilla}
\unsetglyph{Egrave}
\unsetglyph{Eacute}
\unsetglyph{Ecircumflex}
\unsetglyph{Edieresis}
\unsetglyph{Igrave}
\unsetglyph{Iacute}
\unsetglyph{Icircumflex}
\unsetglyph{Idieresis}
\unsetglyph{Eth}
\unsetglyph{Ntilde}
\unsetglyph{Ograve}
\unsetglyph{Oacute}
\unsetglyph{Ocircumflex}
\unsetglyph{Otilde}
\unsetglyph{Odieresis}
\unsetglyph{OE}
\unsetglyph{Oslash}
\unsetglyph{Ugrave}
\unsetglyph{Uacute}
\unsetglyph{Ucircumflex}
\unsetglyph{Udieresis}
\unsetglyph{Yacute}
\unsetglyph{Thorn}
\unsetglyph{SS}
\unsetglyph{agrave}
\unsetglyph{aacute}
\unsetglyph{acircumflex}
\unsetglyph{atilde}
\unsetglyph{adieresis}
\unsetglyph{aring}
\unsetglyph{ae}
\unsetglyph{ccedilla}
\unsetglyph{egrave}
\unsetglyph{eacute}
\unsetglyph{ecircumflex}
\unsetglyph{edieresis}
\unsetglyph{igrave}
\unsetglyph{iacute}
\unsetglyph{icircumflex}
\unsetglyph{idieresis}
\unsetglyph{eth}
\unsetglyph{ntilde}
\unsetglyph{ograve}
\unsetglyph{oacute}
\unsetglyph{ocircumflex}
\unsetglyph{otilde}
\unsetglyph{odieresis}
\unsetglyph{oe}
\unsetglyph{oslash}
\unsetglyph{ugrave}
\unsetglyph{uacute}
\unsetglyph{ucircumflex}
\unsetglyph{udieresis}
\unsetglyph{yacute}
\unsetglyph{thorn}
\unsetglyph{germandbls}

\endmetrics
%</unsetdiamtx>
%    \end{macrocode}
%
%
% \section{Getting the sterling symbol}
%
% This part will eventually come to the file |unsetste.mtx|.  It's quite 
% annoying that this is necessary, but I haven't found a better solution.
% Since I don't want to take the sterling symbol from the EC fonts, I have
% to unset it here.  Not more.
%    \begin{macrocode}
%<*unsetstemtx>
\relax

\metrics

\unsetglyph{sterling}

\endmetrics
%</unsetstemtx>
%    \end{macrocode}
%
%
% \section{Final touches to yfrak}
%
%    \begin{macrocode}
%<*fcleanupmtx>
\relax

\metrics

%    \end{macrocode}
%
% \subsection{Accent positioning}
%
% A top accent is set by placing the center of the accent at the given
% position along the width of the letter, raised up by the difference
% between the height of the letter and the xheight.
%    \begin{macrocode}
\setcommand\topaccent#1#2#3{
   \push
      \moveup{\max{0}{\sub{\height{#1}}{\int{xheight}}}}
      \movert{\add{\sub{\scale{\width{#1}}{#3}}{\scale{\width{#2}}{500}}}
         {\scale{\sub{\height{#1}}{\int{xheight}}}{\int{italicslant}}}}
      \glyph{#2}{1000}
   \pop
   \glyph{#1}{1000}
}

%    \end{macrocode}
% A bottom accent is set by placing the center of the accent at the given
% position along the width of the letter.
%    \begin{macrocode}
\setcommand\botaccent#1#2#3{
   \push
      \movert{\sub{\scale{\width{#1}}{#3}}{\scale{\width{#2}}{500}}}
      \glyph{#2}{1000}
   \pop
   \glyph{#1}{1000}
}

%    \end{macrocode}
% The following values are for the |#3| paramter in the above macros.
%    \begin{macrocode}
\setcommand\Askew{550}
\setcommand\Uskew{470}

%    \end{macrocode}
% |\adjustheight| is meant for characters with a too small height, which is
% true for all German umlauts, for example.
%    \begin{macrocode}
\setcommand\adjustheight#1{%
\resetglyph{#1}
  \glyph{#1}{1000}
  \resetheight{\height{k}}
\endsetglyph
}

\adjustheight{adieresis}
\adjustheight{edieresis}
\adjustheight{odieresis}
\adjustheight{udieresis}
\adjustheight{i}
\adjustheight{j}

%    \end{macrocode}
%
% \subsection{Setting up the accents}
%
% Here I copy the save value of the height of the tilde accent to
% |pushvalue|.
%    \begin{macrocode}
\setint{pushvalue}{\sub{\oldtildeheight}{\height{tilde}}}

%    \end{macrocode}
% This routine moves the accent up.  The accent is smaller than ten points
% (seven points), and this routine tries to ensure that it still has the
% correct height.  It doesn't do this perfectly, therefor I need the special
% tweak parameter |#2|, which is usually negative.
%    \begin{macrocode}
\setcommand\pushaccent#1#2{
  \resetglyph{#1}
    \moveup{\add{\int{pushvalue}}{#2}}
    \glyph{#1}{1000}
  \endsetglyph
}

%    \end{macrocode}
% Here now the concrete accents.
%    \begin{macrocode}
\pushaccent{grave}{-20}
\pushaccent{acute}{-20}
\pushaccent{circumflex}{-40}
\pushaccent{tilde}{-70}
\pushaccent{hungarumlaut}{-20}
%    \end{macrocode}
% I handle the dieresis separately because it comes from yswab, and not from
% CM.
%    \begin{macrocode}
\resetglyph{dieresis}
  \moveup{160}
  \glyph{dieresis}{1000}
\endsetglyph
\pushaccent{ring}{-50}
\pushaccent{caron}{-50}
\pushaccent{breve}{-20}
\pushaccent{dotaccent}{-30}

%    \end{macrocode}
% I want to have the original ring accent from yfrak.  It's quite nice.
%    \begin{macrocode}
\resetglyph{ring}
  \glyph{ringfitted}{1000}
\endsetglyph

%    \end{macrocode}
%
% \subsection{Correcting false characters boxes}
%
% The depth and height of the ``ch'' and ``ck'' ligature are wrong.  Here I
% correct that.
%    \begin{macrocode}
\resetglyph{ffi}
  \glyph{ffi}{1000}
  \resetheight{\height{h}}
  \resetdepth{\depth{h}}
\endsetglyph

\resetglyph{ffl}
  \glyph{ffl}{1000}
  \resetheight{\height{k}}
\endsetglyph

\resetint{interword}{250}

%    \end{macrocode}
% Similar correction for comma, period and semicolon.
%    \begin{macrocode}
\resetglyph{comma}
  \glyph{comma}{1000}
  \resetheight{\height{quotesinglbase}}
  \resetdepth{\depth{quotesinglbase}}
\endsetglyph

\resetglyph{period}
  \glyph{period}{1000}
  \resetheight{\height{quotesinglbase}}
\endsetglyph

\resetglyph{semicolon}
  \glyph{semicolon}{1000}
  \resetdepth{\depth{comma}}
\endsetglyph

%    \end{macrocode}
% The following ligatures are not available in yfrak and must be faked in
% |blackletter.mtx|.
%    \begin{macrocode}
\unsetglyph{dotlessj}
\unsetglyph{fi}
\unsetglyph{fl}
\unsetglyph{backslash}

%    \end{macrocode}
%
% \subsection{Setting up the quotes}
%
% What follows, are hopeless tricks to get nice double and single quotes.  I 
% move the glyphes arbitrarily around, in order to get nice spacing to 
% following or preceding letters and to avaiod EC characters here.
%    \begin{macrocode}
\resetglyph{quotedblleft}
  \movert{-100}
  \glyph{quotedblleft}{1000}
  \resetwidth{\add{\width{quotedblbase}}{70}}
\endsetglyph

\resetglyph{quotedblright}
  \movert{20}
  \glyph{quotedblright}{1000}
  \resetwidth{\add{\width{quotedblbase}}{20}}
\endsetglyph

\resetglyph{quotesinglbase}
  \moveup{\sub{\height{quotedblbase}}{\height{quoteright}}}
  \glyph{quoteright}{1000}
  \samesize{quotesinglbase}
\endsetglyph

\resetglyph{quotedblbase}
  \moveup{\sub{\height{quotedblbase}}{\height{quotedblright}}}
  \movert{-60}
  \glyph{quotedblright}{1000}
  \resetheight{\height{quotedblbase}}
  \resetdepth{\depth{quotedblbase}}
  \resetwidth{\add{\width{quotedblbase}}{40}}
\endsetglyph

%    \end{macrocode}
%
% \subsection{Faking the glyphs}
%
% Now I try to fake or improve the letters for the latin T1 version of yfrak.
% Most of this should be easy to understand.
%    \begin{macrocode}
\resetglyph{A}
  \glyph{A}{1000}
  \resetheight{\sub{\height{A}}{50}}
\endsetglyph

\setglyph{Abreve}
   \topaccent{A}{breve}{\Askew}
\endsetglyph

\setglyph{Aogonek}
   \botaccent{A}{ogonek}{610}
\endsetglyph

\setglyph{Eogonek}
   \botaccent{E}{ogonek}{590}
\endsetglyph

\setglyph{Lcaron}
   \glyph{L}{1000}
   \movert{-100}
   \glyph{quoteright}{1000}
\endsetglyph

\setglyph{Lslash}
   \glyph{lslashslash}{1000}
   \movert{-270}
   \glyph{L}{1000}
\endsetglyph

\setglyph{Ng}
  \glyph{S}{1000}
\endsetglyph

\setglyph{Uhungarumlaut}
   \topaccent{U}{hungarumlaut}{\Uskew}
\endsetglyph

\setglyph{Uring}
   \topaccent{U}{ring}{\Uskew}
\endsetglyph

\setglyph{dbar}
   \push
      \movert{-160}
      \moveup{
         \sub{\scale{\add{\height{d}}{\int{xheight}}}{470}}
             {\height{macron}}}
      \glyph{macron}{1000}
   \pop
   \glyph{d}{1000}
   \samesize{d}
\endsetglyph

\setglyph{lslash}
   \glyph{lslashslash}{1000}
   \movert{-210}
   \glyph{l}{1000}
\endsetglyph

\setglyph{ng}
  \glyph{sround}{1000}
\endsetglyph

\setglyph{sacute}
   \topaccent{sround}{acute}{500}
\endsetglyph

\setglyph{scaron}
   \topaccent{sround}{caron}{500}
\endsetglyph

\setglyph{scedilla}
   \botaccent{sround}{cedilla}{500}
\endsetglyph

\setglyph{Agrave}
   \topaccent{A}{grave}{\Askew}
\endsetglyph

\setglyph{Aacute}
   \topaccent{A}{acute}{\Askew}
\endsetglyph

\setglyph{Acircumflex}
   \topaccent{A}{circumflex}{\Askew}
\endsetglyph

\setglyph{Atilde}
   \topaccent{A}{tilde}{\Askew}
\endsetglyph

\setglyph{Adieresis}
   \topaccent{A}{dieresis}{\Askew}
\endsetglyph

\setglyph{Aring}
   \topaccent{A}{ring}{\Askew}
\endsetglyph

\resetglyph{AE}
   \glyph{A}{1000}
   \movert{-120}
   \glyph{E}{1000}
\endsetglyph

\setglyph{Eth}
   \push
      \movert{100}
      \moveup{
         \sub{\scale{\add{\height{d}}{\int{xheight}}}{300}}
             {\height{macron}}}
      \glyph{macron}{1000}
   \pop
   \glyph{D}{1000}
   \resetdepth{\depth{D}}
\endsetglyph

\resetglyph{OE}
   \glyph{O}{1000}
   \movert{-100}
   \glyph{E}{1000}
\endsetglyph

\resetglyph{Oslash}
   \push
     \moveup{130}
     \movert{170}
     \glyph{slash}{800}
   \pop
   \glyph{O}{1000}
\endsetglyph

\setglyph{Ugrave}
   \topaccent{U}{grave}{\Uskew}
\endsetglyph

\setglyph{Uacute}
   \topaccent{U}{acute}{\Uskew}
\endsetglyph

\setglyph{Ucircumflex}
   \topaccent{U}{circumflex}{\Uskew}
\endsetglyph

\setglyph{Udieresis}
   \topaccent{U}{dieresis}{\Uskew}
\endsetglyph

\setglyph{Thorn}
   \glyph{T}{1000}
   \movert{-100}
   \glyph{h}{1000}
\endsetglyph

\resetglyph{ae}
   \glyph{a}{1000}
   \movert{-150}
   \glyph{e}{1000}
\endsetglyph

\setglyph{eth}
   \push
      \movert{-100}
      \moveup{
         \sub{\scale{\add{\height{d}}{\int{xheight}}}{700}}
             {\height{macron}}}
      \glyph{lslashslash}{1000}
   \pop
   \glyph{d}{1000}
\endsetglyph

\resetglyph{oslash}
   \push
     \moveup{120}
     \movert{50}
     \glyph{slash}{500}
   \pop
   \glyph{o}{1000}
\endsetglyph

\resetglyph{oe}
   \glyph{o}{1000}
   \movert{-100}
   \glyph{e}{1000}
\endsetglyph

\setglyph{thorn}
   \glyph{t}{1000}
   \movert{-80}
   \glyph{h}{1000}
\endsetglyph

\endmetrics
%</fcleanupmtx>
%    \end{macrocode}
%
%
% \section{Final touches to yswab}
%
%    \begin{macrocode}
%<*scleanupmtx>
\relax

\metrics

%    \end{macrocode}
%
% \subsection{Accent positioning}
%
% A top accent is set by placing the center of the accent at the given
% position along the width of the letter, raised up by the difference
% between the height of the letter and the xheight.
%    \begin{macrocode}
\setcommand\topaccent#1#2#3{
   \push
      \moveup{\max{0}{\sub{\height{#1}}{\int{xheight}}}}
      \movert{\add{\sub{\scale{\width{#1}}{#3}}{\scale{\width{#2}}{500}}}
         {\scale{\sub{\height{#1}}{\int{xheight}}}{\int{italicslant}}}}
      \glyph{#2}{1000}
   \pop
   \glyph{#1}{1000}
}

%    \end{macrocode}
% A bottom accent is set by placing the center of the accent at the given
% position along the width of the letter.
%    \begin{macrocode}
\setcommand\botaccent#1#2#3{
   \push
      \movert{\sub{\scale{\width{#1}}{#3}}{\scale{\width{#2}}{500}}}
      \glyph{#2}{1000}
   \pop
   \glyph{#1}{1000}
}

%    \end{macrocode}
% The following value is for the |#3| paramter in the above macros.
%    \begin{macrocode}
\setcommand\Askew{580}

%    \end{macrocode}
% |\adjustheight| is meant for characters with a too small height, which is
% true for all German umlauts, for example.
%    \begin{macrocode}
\setcommand\adjustheight#1{%
\resetglyph{#1}
  \glyph{#1}{1000}
  \resetheight{\height{k}}
\endsetglyph
}

\adjustheight{adieresis}
\adjustheight{edieresis}
\adjustheight{odieresis}
\adjustheight{udieresis}

%    \end{macrocode}
%
% \subsection{Setting up the accents}
%
% Here I make all accents a little bit smaller to fit with ygoth.
% Additionally I push them upwards, which is accent dependent.
%    \begin{macrocode}
\resetglyph{tilde}
  \moveup{-40}
  \glyph{tilde}{1000}
\endsetglyph

\resetglyph{dieresis}
  \moveup{-40}
  \glyph{dieresis}{1000}
\endsetglyph

%    \end{macrocode}
% I want to have the original ring accent from yfrak.  It's quite nice.
%    \begin{macrocode}
\resetglyph{ring}
  \moveup{-30}
  \glyph{ringfitted}{1000}
\endsetglyph

%    \end{macrocode}
%
% \subsection{Setting up the quotes}
%
% What follows, are hopeless tricks to get nice double and single quotes.  I
% move the glyphes arbitrarily around, in order to get nice spacing to
% following or preceding letters and to avaiod EC characters here.
%    \begin{macrocode}
\resetglyph{quotedblleft}
  \movert{-90}
  \glyph{quotedblleft}{1000}
  \resetwidth{\sub{\width{quotedblleft}}{100}}
\endsetglyph

\resetglyph{quotedblright}
  \movert{20}
  \glyph{quotedblright}{1000}
  \resetwidth{\sub{\width{quotedblright}}{120}}
\endsetglyph

\resetglyph{quotesinglbase}
  \moveup{\sub{\height{quotedblbase}}{\height{quoteright}}}
  \glyph{quoteright}{1000}
  \samesize{quotesinglbase}
\endsetglyph

\resetglyph{quotedblbase}
  \moveup{\sub{\height{quotedblbase}}{\height{quotedblright}}}
  \movert{-45}
  \glyph{quotedblright}{1000}
  \movert{70}
  \resetheight{\height{quotedblbase}}
  \resetdepth{\depth{quotedblbase}}
\endsetglyph

%    \end{macrocode}
%
% \subsection{Correcting character boxes}
%
% I don't know why, but the original character boxes of Haralambous for
% yswab are sometimes wrong.
%    \begin{macrocode}
\resetglyph{ffi}
  \glyph{ffi}{1000}
  \resetheight{\height{h}}
  \resetdepth{\depth{h}}
\endsetglyph

\resetglyph{ffl}
  \glyph{ffl}{1000}
  \resetheight{\height{k}}
\endsetglyph

\resetglyph{comma}
  \glyph{comma}{1000}
  \resetheight{\height{quotesinglbase}}
  \resetdepth{\depth{quotesinglbase}}
\endsetglyph

\resetglyph{period}
  \glyph{period}{1000}
  \resetheight{\height{quotesinglbase}}
\endsetglyph

\resetglyph{m}
  \glyph{m}{1000}
  \resetheight{\height{u}}
\endsetglyph

\resetglyph{n}
  \glyph{n}{1000}
  \resetheight{\height{u}}
\endsetglyph

\resetint{interword}{250}

%    \end{macrocode}
%
% \subsection{Faking the glyphs}
%
% The following glyphs should be constructed in |blackletter.mtx|.
%    \begin{macrocode}
\unsetglyph{dotlessj}
\unsetglyph{fi}
\unsetglyph{fl}
\unsetglyph{backslash}
\unsetglyph{bar}

%    \end{macrocode}
% Now I try to fake or improve the letters for the latin T1 version of ygoth.
% Most of this should be easy to understand.
%    \begin{macrocode}
\setglyph{Abreve}
   \topaccent{A}{breve}{\Askew}
\endsetglyph

\setglyph{Lcaron}
   \glyph{L}{1000}
   \movert{-200}
   \glyph{quoteright}{1000}
\endsetglyph

\setglyph{Lslash}
   \glyph{lslashslash}{1000}
   \movert{-270}
   \glyph{L}{1000}
\endsetglyph

\setglyph{Ng}
  \glyph{S}{1000}
\endsetglyph

\setglyph{dbar}
   \push
      \movert{-160}
      \moveup{
         \sub{\scale{\add{\height{d}}{\int{xheight}}}{470}}
             {\height{macron}}}
      \glyph{macron}{1000}
   \pop
   \glyph{d}{1000}
   \samesize{d}
\endsetglyph

\setglyph{lslash}
   \glyph{lslashslash}{1000}
   \movert{-240}
   \glyph{l}{1000}
\endsetglyph

\setglyph{ng}
  \glyph{sround}{1000}
\endsetglyph

\setglyph{sacute}
   \topaccent{sround}{acute}{500}
\endsetglyph

\setglyph{scaron}
   \topaccent{sround}{caron}{500}
\endsetglyph

\setglyph{scedilla}
   \botaccent{sround}{cedilla}{500}
\endsetglyph

\setglyph{Agrave}
   \topaccent{A}{grave}{\Askew}
\endsetglyph

\setglyph{Aacute}
   \topaccent{A}{acute}{\Askew}
\endsetglyph

\setglyph{Acircumflex}
   \topaccent{A}{circumflex}{\Askew}
\endsetglyph

\setglyph{Atilde}
   \topaccent{A}{tilde}{\Askew}
\endsetglyph

\setglyph{Adieresis}
   \topaccent{A}{dieresis}{\Askew}
\endsetglyph

\setglyph{Aring}
   \topaccent{A}{ring}{\Askew}
\endsetglyph

\setglyph{AE}
   \glyph{A}{1000}
   \movert{-100}
   \glyph{E}{1000}
\endsetglyph

\setglyph{Eth}
   \push
      \movert{100}
      \moveup{
         \sub{\scale{\add{\height{d}}{\int{xheight}}}{300}}
             {\height{macron}}}
      \glyph{macron}{1000}
   \pop
   \glyph{D}{1000}
   \resetdepth{\depth{D}}
\endsetglyph

\setglyph{OE}
   \glyph{O}{1000}
   \movert{-100}
   \glyph{E}{1000}
\endsetglyph

\setglyph{Oslash}
   \push
     \moveup{130}
     \movert{170}
     \glyph{slash}{800}
   \pop
   \glyph{O}{1000}
\endsetglyph

\setglyph{Thorn}
   \glyph{T}{1000}
   \movert{-100}
   \glyph{h}{1000}
\endsetglyph

\setglyph{ae}
   \glyph{a}{1000}
   \movert{-150}
   \glyph{e}{1000}
\endsetglyph

\setglyph{eth}
   \push
      \movert{-30}
      \moveup{
         \sub{\scale{\add{\height{d}}{\int{xheight}}}{650}}
             {\height{macron}}}
      \glyph{lslashslash}{1000}
   \pop
   \glyph{d}{1000}
\endsetglyph

\setglyph{oslash}
   \push
     \moveup{120}
     \movert{70}
     \glyph{slash}{500}
   \pop
   \glyph{o}{1000}
\endsetglyph

\setglyph{oe}
   \glyph{o}{1000}
   \movert{-100}
   \glyph{e}{1000}
\endsetglyph

\setglyph{thorn}
   \glyph{t}{1000}
   \movert{-80}
   \glyph{h}{1000}
\endsetglyph

\endmetrics
%</scleanupmtx>
%    \end{macrocode}
%
%
% \section{Final touches to ygoth}
%
%    \begin{macrocode}
%<*gcleanupmtx>
\relax

\metrics

%    \end{macrocode}
%
% \subsection{Accent positioning}
%
% A top accent is set by placing the center of the accent at the given
% position along the width of the letter, raised up by the difference
% between the height of the letter and the xheight.
%    \begin{macrocode}
\setcommand\topaccent#1#2#3{
   \push
      \moveup{\max{0}{\sub{\height{#1}}{\int{xheight}}}}
      \movert{\add{\sub{\scale{\width{#1}}{#3}}{\scale{\width{#2}}{500}}}
         {\scale{\sub{\height{#1}}{\int{xheight}}}{\int{italicslant}}}}
      \glyph{#2}{1000}
   \pop
   \glyph{#1}{1000}
}

%    \end{macrocode}
% A bottom accent is set by placing the center of the accent at the given
% position along the width of the letter.
%    \begin{macrocode}
\setcommand\botaccent#1#2#3{
   \push
      \movert{\sub{\scale{\width{#1}}{#3}}{\scale{\width{#2}}{500}}}
      \glyph{#2}{1000}
   \pop
   \glyph{#1}{1000}
}

%    \end{macrocode}
% The following values are for the |#3| paramter in the above macros.
%    \begin{macrocode}
\setcommand\Askew{700}
\setcommand\Uskew{600}

%    \end{macrocode}
% |\adjustheight| is meant for characters with a too small height, which is
% true for all German umlauts, for example.
%    \begin{macrocode}
\setcommand\adjustheight#1{%
\resetglyph{#1}
  \glyph{#1}{1000}
  \resetheight{\height{k}}
\endsetglyph
}

\adjustheight{adieresis}
\adjustheight{edieresis}
\adjustheight{odieresis}
\adjustheight{udieresis}

%    \end{macrocode}
%
% \subsection{Setting up the accents}
%
% Here I make all accents a little bit smaller to fit with ygoth.
% Additionally I push them upwards, which is accent dependent.
%    \begin{macrocode}
\setcommand\SmallerAccent#1#2{
\resetglyph{#1}
  \moveup{\add{100}{#2}}
  \movert{\div{\width{#1}}{10}}
  \glyph{#1}{800}
  \samesize{#1}
\endsetglyph
}

\SmallerAccent{grave}{20}
\SmallerAccent{acute}{20}
\SmallerAccent{circumflex}{20}
\SmallerAccent{tilde}{0}
\SmallerAccent{hungarumlaut}{20}
\SmallerAccent{ring}{30}
\SmallerAccent{ringfitted}{20}
\SmallerAccent{caron}{20}
\SmallerAccent{breve}{20}

%    \end{macrocode}
% I construct the dieresis of two mere periods.  Don't say anything.
%    \begin{macrocode}
\resetglyph{dieresis}
  \moveup{560}
  \movert{80}
  \glyph{period}{1000}
  \movert{-140}
  \glyph{period}{1000}
  \samesize{dieresis}
\endsetglyph

%    \end{macrocode}
% Similar trick for the dot accent.
%    \begin{macrocode}
\resetglyph{dotaccent}
  \moveup{560}
  \movert{10}
  \glyph{period}{1000}
  \samesize{dotaccent}
\endsetglyph

\resetglyph{ring}
  \moveup{-30}
  \glyph{ringfitted}{1000}
\endsetglyph

%    \end{macrocode}
%
% \subsection{Setting up the quotes}
%
% What follows, are hopeless tricks to get nice double and single quotes.  I
% move the glyphes arbitrarily around, in order to get nice spacing to
% following or preceding letters and to avaiod EC characters here.
%    \begin{macrocode}
\resetglyph{quotesinglbase}
  \moveup{\sub{\height{quotedblbase}}{\height{quoteright}}}
  \glyph{quoteright}{1000}
  \samesize{quotesinglbase}
\endsetglyph

\resetglyph{quotedblleft}
  \movert{-90}
  \glyph{quotedblleft}{1000}
  \resetwidth{\sub{\width{quotedblleft}}{100}}
\endsetglyph

\resetglyph{quotedblright}
  \movert{20}
  \glyph{quotedblright}{1000}
  \resetwidth{\sub{\width{quotedblright}}{120}}
\endsetglyph

\resetglyph{quotedblbase}
  \moveup{\sub{\height{quotedblbase}}{\height{quotedblright}}}
  \movert{-80}
  \glyph{quotedblright}{1000}
  \resetheight{\height{quotedblbase}}
  \resetdepth{\depth{quotedblbase}}
  \resetwidth{\add{\width{quotedblbase}}{30}}
\endsetglyph

%    \end{macrocode}
%
% \subsection{Correcting character boxes}
%
% I don't know why, but the original character boxes of Haralambous for
% ygoth are sometimes wrong.  Here I shift the ``i'' a little bit to the
% left and the ``l'' a little bit to the right.  Similar changes to other
% characters.
%    \begin{macrocode}
\resetglyph{dotlessi}
  \movert{-30}
  \glyph{dotlessi}{1000}
  \resetwidth{\width{dotlessi}}
  \resetheight{\height{u}}
\endsetglyph

\resetglyph{fi}
  \glyph{fi}{1000}
  \movert{40}
\endsetglyph

\resetglyph{ffi}
  \glyph{ffi}{1000}
  \movert{20}
  \resetdepth{\depth{h}}
\endsetglyph

\resetglyph{ffl}
  \glyph{ffl}{1000}
  \movert{20}
\endsetglyph

\resetglyph{comma}
  \glyph{comma}{1000}
  \resetheight{\height{quotesinglbase}}
\endsetglyph

\resetglyph{period}
  \glyph{period}{1000}
  \resetheight{\height{quotesinglbase}}
\endsetglyph

\resetglyph{A}
  \glyph{A}{1000}
  \resetheight{\sub{\height{A}}{50}}
\endsetglyph

\resetglyph{i}
  \movert{-30}
  \glyph{i}{1000}
  \samesize{i}
\endsetglyph

\resetglyph{j}
  \movert{-30}
  \glyph{j}{1000}
  \movert{30}
\endsetglyph

\resetglyph{l}
  \movert{30}
  \glyph{l}{1000}
  \samesize{l}
\endsetglyph

\resetint{interword}{250}

%    \end{macrocode}
%
% \subsection{Kerning}
%
% ``F'' and ``V'' need some kerning.
%
%    \begin{macrocode}
\setkern{F}{a}{-140}
\setkern{F}{e}{-110}
\setkern{F}{o}{-140}
\setkern{F}{u}{-110}

\setkern{V}{a}{-140}
\setkern{V}{e}{-110}
\setkern{V}{o}{-140}
\setkern{V}{u}{-110}


%    \end{macrocode}
%
% \subsection{Faking the glyphs}
%
% The following glyphs should be constructed in |blackletter.mtx|.
%    \begin{macrocode}
\unsetglyph{dotlessj}
\unsetglyph{backslash}

%    \end{macrocode}
% Now I try to fake or improve the letters for the latin T1 version of ygoth.
% Most of this should be easy to understand.
%    \begin{macrocode}
\setglyph{Abreve}
   \topaccent{A}{breve}{\Askew}
\endsetglyph

\setglyph{Eogonek}
   \botaccent{E}{ogonek}{600}
\endsetglyph

\setglyph{Lacute}
   \topaccent{L}{acute}{700}
\endsetglyph

\setglyph{Lcaron}
   \glyph{L}{1000}
   \movert{-30}
   \glyph{quoteright}{1000}
\endsetglyph

\setglyph{Lslash}
   \moveup{-200}
   \glyph{lslashslash}{1800}
   \moveup{200}
   \movert{-640}
   \glyph{L}{1000}
   \samesize{L}
\endsetglyph

\setglyph{Ng}
  \glyph{S}{1000}
\endsetglyph

\setglyph{Uhungarumlaut}
   \topaccent{U}{hungarumlaut}{\Uskew}
\endsetglyph

\setglyph{Uring}
   \topaccent{U}{ring}{\Uskew}
\endsetglyph

\setglyph{dbar}
   \push
      \movert{-100}
      \moveup{
         \sub{\scale{\add{\height{d}}{\int{xheight}}}{390}}
             {\height{macron}}}
      \glyph{macron}{1000}
   \pop
   \glyph{d}{1000}
   \samesize{d}
\endsetglyph

\setglyph{lcaron}
   \glyph{l}{1000}
   \movert{-50}
   \glyph{quoteright}{1000}
\endsetglyph

\setglyph{lslash}
   \glyph{lslashslash}{1000}
   \movert{-220}
   \glyph{l}{1000}
\endsetglyph

\setglyph{ng}
  \glyph{sround}{1000}
\endsetglyph

\setglyph{sacute}
   \topaccent{sround}{acute}{500}
\endsetglyph

\setglyph{scaron}
   \topaccent{sround}{caron}{500}
\endsetglyph

\setglyph{scedilla}
   \botaccent{sround}{cedilla}{500}
\endsetglyph

\setglyph{Agrave}
   \topaccent{A}{grave}{\Askew}
\endsetglyph

\setglyph{Aacute}
   \topaccent{A}{acute}{\Askew}
\endsetglyph

\setglyph{Acircumflex}
   \topaccent{A}{circumflex}{\Askew}
\endsetglyph

\setglyph{Atilde}
   \topaccent{A}{tilde}{\Askew}
\endsetglyph

\setglyph{Adieresis}
   \topaccent{A}{dieresis}{\Askew}
\endsetglyph

\setglyph{Aring}
   \topaccent{A}{ring}{\Askew}
\endsetglyph

\setglyph{AE}
   \glyph{A}{1000}
   \movert{-190}
   \glyph{E}{1000}
\endsetglyph

\setglyph{Eth}
   \push
      \movert{-80}
      \moveup{
         \sub{\scale{\add{\height{d}}{\int{xheight}}}{380}}
             {\height{macron}}}
      \glyph{macron}{1000}
   \pop
   \glyph{D}{1000}
   \resetdepth{\depth{D}}
\endsetglyph

\setglyph{OE}
   \glyph{O}{1000}
   \movert{-100}
   \glyph{E}{1000}
\endsetglyph

\setglyph{Oslash}
   \push
     \moveup{130}
     \movert{190}
     \glyph{slash}{800}
   \pop
   \glyph{O}{1000}
\endsetglyph

\setglyph{Ugrave}
   \topaccent{U}{grave}{\Uskew}
\endsetglyph

\setglyph{Uacute}
   \topaccent{U}{acute}{\Uskew}
\endsetglyph

\setglyph{Ucircumflex}
   \topaccent{U}{circumflex}{\Uskew}
\endsetglyph

\setglyph{Udieresis}
   \topaccent{U}{dieresis}{\Uskew}
\endsetglyph

\setglyph{Thorn}
   \glyph{T}{1000}
   \movert{-100}
   \glyph{h}{1000}
\endsetglyph

\setglyph{ae}
   \glyph{a}{1000}
   \movert{-150}
   \glyph{e}{1000}
\endsetglyph

\setglyph{eth}
   \push
      \movert{-40}
      \moveup{
         \sub{\scale{\add{\height{d}}{\int{xheight}}}{620}}
             {\height{macron}}}
      \glyph{lslashslash}{1000}
   \pop
   \glyph{d}{1000}
\endsetglyph

\setglyph{oslash}
   \push
     \moveup{120}
     \movert{70}
     \glyph{slash}{500}
   \pop
   \glyph{o}{1000}
\endsetglyph

\setglyph{oe}
   \glyph{o}{1000}
   \movert{-100}
   \glyph{e}{1000}
\endsetglyph

\setglyph{thorn}
   \glyph{t}{1000}
   \movert{-80}
   \glyph{h}{1000}
\endsetglyph

%    \end{macrocode}
%
% \subsection{Letterspacing}
%
% ygoth runs too narrow.  I improve that here.
%    \begin{macrocode}
\setint{letterspacing}{50}

%    \end{macrocode}
% On the other hand, the German combination ``sch'' looks like ``s\,ch''.
% Therefore I set this kerning here.  FixMe: This induces wrong implicit
% kerning of diacritic letters in |blackletter.mtx|.
%    \begin{macrocode}
\setkern{s}{c}{-80}

\endmetrics
%</gcleanupmtx>
%    \end{macrocode}
%
%
% \section{Letterspacing}
%
% The following definitions form the file |lettersp.mtx| and define some
% thing for a letterspeced Fraktur version, used eventually as the italic
% counterpart.
%    \begin{macrocode}
%<*letterspmtx>
\relax

\metrics

%    \end{macrocode}
%
% \subsection{The ``sch'' case}
%
% First, I re-do the spacing for the ``sch'' combination.  This is unusual,
% but I find it more readable this way.
%
% This is now turned off.
%    \begin{macrocode}
\iffalse
\setkern{s}{c}{-150}
\fi

%    \end{macrocode}
% The following definitions turn off all kerning commands.  This prevents
% |blackletter.mtx| from copying the above kerning information to other 
% \mbox{``s--c''} combinations with diacritic marks.
%    \begin{macrocode}
\gdef\setleftrightkerning#1#2#3{}
\gdef\setleftkerning#1#2#3{}
\gdef\setrightkerning#1#2#3{}

%    \end{macrocode}
% Now I finally set the letterspacing.
%    \begin{macrocode}
\setint{letterspacing}{150}

\resetint{interword}{350}

%    \end{macrocode}
%
% \subsection{Breaking up of some ligatures}
%
% In the rest of the file I re-define some ligatures, namely all ligatures
% that have to be blackletter up in letterspacing.  Actually the ``st'' ligature
% should be blackletter up, too, according to Duden and other sources.  Curiously
% enough, I've found no real life book that did it that way, everybody keeps
% it together.  So I do that, too.  After all, it's a very nice ligature.
%    \begin{macrocode}
\resetglyph{dotlessj}
  \glyph{s}{1000}
  \movert{\int{letterspacing}}
  \glyph{i}{1000}
\endsetglyph

\resetglyph{ff}
  \glyph{f}{1000}
  \movert{\int{letterspacing}}
  \glyph{f}{1000}
\endsetglyph

\resetglyph{fi}
  \glyph{f}{1000}
  \movert{\int{letterspacing}}
  \glyph{i}{1000}
\endsetglyph

\resetglyph{fl}
  \glyph{f}{1000}
  \movert{\int{letterspacing}}
  \glyph{l}{1000}
\endsetglyph

\resetglyph{backslash}
  \glyph{f}{1000}
  \movert{\int{letterspacing}}
  \glyph{t}{1000}
\endsetglyph

\resetglyph{asciicircum}
  \glyph{s}{1000}
  \movert{\int{letterspacing}}
  \glyph{s}{1000}
\endsetglyph

\endmetrics
%</letterspmtx>
%    \end{macrocode}
% The following file |blackletter.mtx| is a slimmed and modified version of
% \texttt{fontinst}'s |latin.mtx|.  It makes last settings and tries to fake
% glyphs that are still missing.
%    \begin{macrocode}
%<*blacklettermtx>
\relax

\documentclass[twocolumn]{article}
\usepackage{fontdoc}

\title{The blackletter typefaces glyphs}
\author{Torsten Bronger}
\date{7.~May 2002 \\
Version 1.0}

\begin{document}
\maketitle

\section{Introduction}

This document describes the glyphs used by the {\tt fontinst} package when
generating blackletter typefaces fonts, especially those by Yannis Haralambous.  
It's a modified version of {\tt fontinst}'s \verb|latin.mtx|.

\metrics

\needsfontinstversion{1.801}

%    \end{macrocode}
%
% \subsection{Basic helper commands}
%
% |\unfakable| is a glyph which can't be faked.
%    \begin{macrocode}
\setcommand\unfakable#1{
   \setglyph{#1}
      \ifisglyph{#1-not}\then
         \moveup{\neg{\depth{#1-not}}}
         \glyphrule{
            \width{#1-not}
         }{
            \add{\depth{#1-not}}{\height{#1-not}}
         }
         \resetitalic{\italic{#1-not}}
         \moveup{\depth{#1-not}}
      \else
         \glyphrule{500}{500}
      \fi
      \glyphwarning{missing glyph `#1'}
   \endsetglyph
}

%    \end{macrocode}
% A top accent is set by placing the center of the accent at the given
% position along the width of the letter, raised up by the difference
% between the height of the letter and the xheight.
%    \begin{macrocode}
\setcommand\topaccent#1#2#3{
   \push
      \moveup{\max{0}{\sub{\height{#1}}{\int{xheight}}}}
      \movert{\add{\sub{\scale{\width{#1}}{#3}}{\scale{\width{#2}}{500}}}
         {\scale{\sub{\height{#1}}{\int{xheight}}}{\int{italicslant}}}}
      \glyph{#2}{1000}
   \pop
   \glyph{#1}{1000}
}

%    \end{macrocode}
% A bottom accent is set by placing the center of the accent at the given
% position along the width of the letter.
%    \begin{macrocode}
\setcommand\botaccent#1#2#3{
   \push
      \movert{\sub{\scale{\width{#1}}{#3}}{\scale{\width{#2}}{500}}}
      \glyph{#2}{1000}
   \pop
   \glyph{#1}{1000}
}

%    \end{macrocode}
%
% \subsection{Default font values}
%
%    \begin{macrocode}
\comment{\section{Default values}}

\setint{italicslant}{0}
\setint{xheight}{\height{x}}
\setint{capheight}{\height{A}}
\setint{ascender}{\height{k}}
\setint{descender}{\depth{g}}
\setint{underlinethickness}{40}
\setint{visiblespacedepth}{200}
\setint{visiblespacewidth}{400}
\setint{visiblespacesurround}{50}
\setint{smallcapsscale}{800}
\setint{smallcapskerning}{900}

\setint{capspacing}{50}
\setint{smallcapsextraspace}{0}

\setint{boundarychar}{32}


%    \end{macrocode}
%
% \subsection{Kerning adjustments}
%
%    \begin{macrocode}
\comment{\section{Kerning}}

%    \end{macrocode}
% The command |\typicalkerns| adds extra space before some punctuation.  This
% is common in typesetting with blackletter typefaces.
%    \begin{macrocode}
\setcommand\typicalkerns#1{
  \setkern{#1}{exclam}{100}
  \setkern{#1}{question}{100}
  \setkern{#1}{colon}{50}
  \setkern{#1}{semicolon}{50}
}

%    \end{macrocode}
% Now we call |\typicalkerns| for all letter-like characters.
%    \begin{macrocode}
\typicalkerns{dotlessj}
\typicalkerns{ff}
\typicalkerns{fi}
\typicalkerns{fl}
\typicalkerns{ffi}
\typicalkerns{ffl}

\typicalkerns{zero}
\typicalkerns{one}
\typicalkerns{two}
\typicalkerns{three}
\typicalkerns{four}
\typicalkerns{five}
\typicalkerns{six}
\typicalkerns{seven}
\typicalkerns{eight}
\typicalkerns{nine}

\typicalkerns{A} \typicalkerns{B}
\typicalkerns{C} \typicalkerns{D} \typicalkerns{E} \typicalkerns{F}
\typicalkerns{G} \typicalkerns{H} \typicalkerns{I} \typicalkerns{J}
\typicalkerns{K} \typicalkerns{L} \typicalkerns{M} \typicalkerns{N}
\typicalkerns{O} \typicalkerns{P} \typicalkerns{Q} \typicalkerns{R}
\typicalkerns{S} \typicalkerns{T} \typicalkerns{U} \typicalkerns{V}
\typicalkerns{W} \typicalkerns{X} \typicalkerns{Y} \typicalkerns{Z}

\typicalkerns{a} \typicalkerns{b}
\typicalkerns{c} \typicalkerns{d} \typicalkerns{e} \typicalkerns{f}
\typicalkerns{g} \typicalkerns{h} \typicalkerns{i} \typicalkerns{j}
\typicalkerns{k} \typicalkerns{l} \typicalkerns{m} \typicalkerns{n}
\typicalkerns{o} \typicalkerns{p} \typicalkerns{q} \typicalkerns{r}
\typicalkerns{s} \typicalkerns{t} \typicalkerns{u} \typicalkerns{v}
\typicalkerns{w} \typicalkerns{x} \typicalkerns{y} \typicalkerns{z}

\typicalkerns{backslash}
\typicalkerns{asciicircum}
\typicalkerns{underscore}
\typicalkerns{bar}

\typicalkerns{quoteleft}
\typicalkerns{quoteright}
\typicalkerns{quotedblleft}
\typicalkerns{quotedblright}

\typicalkerns{Abreve}
\typicalkerns{Aogonek} \typicalkerns{Cacute} \typicalkerns{Ccaron}
\typicalkerns{Dcaron} \typicalkerns{Ecaron} \typicalkerns{Eogonek}
\typicalkerns{Gbreve} \typicalkerns{Lacute} \typicalkerns{Lcaron}
\typicalkerns{Lslash} \typicalkerns{Nacute} \typicalkerns{Ncaron}
\typicalkerns{Ng} \typicalkerns{Ohungarumlaut} \typicalkerns{Racute}
\typicalkerns{Rcaron} \typicalkerns{Sacute} \typicalkerns{Scaron}
\typicalkerns{Scedilla} \typicalkerns{Tcaron} \typicalkerns{Tcedilla}
\typicalkerns{Uhungarumlaut} \typicalkerns{Uring} \typicalkerns{Ydieresis}
\typicalkerns{Zacute} \typicalkerns{Zcaron} \typicalkerns{Zdotaccent}
\typicalkerns{IJ} \typicalkerns{Idotaccent} \typicalkerns{dbar}

\typicalkerns{section}

\typicalkerns{abreve} \typicalkerns{aogonek}
\typicalkerns{cacute} \typicalkerns{ccaron} \typicalkerns{dcaron}
\typicalkerns{ecaron} \typicalkerns{eogonek} \typicalkerns{gbreve}
\typicalkerns{lacute} \typicalkerns{lcaron} \typicalkerns{lslash}
\typicalkerns{nacute} \typicalkerns{ncaron} \typicalkerns{ng}
\typicalkerns{ohungarumlaut} \typicalkerns{racute} \typicalkerns{rcaron}
\typicalkerns{sacute} \typicalkerns{scaron} \typicalkerns{scedilla}
\typicalkerns{tcaron} \typicalkerns{tcedilla} \typicalkerns{uhungarumlaut}
\typicalkerns{uring} \typicalkerns{ydieresis} \typicalkerns{zacute}
\typicalkerns{zcaron} \typicalkerns{zdotaccent} \typicalkerns{ij}

%\typicalkerns{exclamdown}
%\typicalkerns{questiondown}
\typicalkerns{sterling}

\typicalkerns{Agrave}
\typicalkerns{Aacute} \typicalkerns{Acircumflex} \typicalkerns{Atilde}
\typicalkerns{Adieresis} \typicalkerns{Aring} \typicalkerns{AE}
\typicalkerns{Ccedilla} \typicalkerns{Egrave} \typicalkerns{Eacute}
\typicalkerns{Ecircumflex} \typicalkerns{Edieresis} \typicalkerns{Igrave}
\typicalkerns{Iacute} \typicalkerns{Icircumflex} \typicalkerns{Idieresis}
\typicalkerns{Eth} \typicalkerns{Ntilde} \typicalkerns{Ograve}
\typicalkerns{Oacute} \typicalkerns{Ocircumflex} \typicalkerns{Otilde}
\typicalkerns{Odieresis} \typicalkerns{OE} \typicalkerns{Oslash}
\typicalkerns{Ugrave} \typicalkerns{Uacute} \typicalkerns{Ucircumflex}
\typicalkerns{Udieresis} \typicalkerns{Yacute} \typicalkerns{Thorn}
\typicalkerns{SS} \typicalkerns{agrave} \typicalkerns{aacute}
\typicalkerns{acircumflex} \typicalkerns{atilde} \typicalkerns{adieresis}
\typicalkerns{aring} \typicalkerns{ae} \typicalkerns{ccedilla}
\typicalkerns{egrave} \typicalkerns{eacute} \typicalkerns{ecircumflex}
\typicalkerns{edieresis} \typicalkerns{igrave} \typicalkerns{iacute}
\typicalkerns{icircumflex} \typicalkerns{idieresis} \typicalkerns{eth}
\typicalkerns{ntilde} \typicalkerns{ograve} \typicalkerns{oacute}
\typicalkerns{ocircumflex} \typicalkerns{otilde} \typicalkerns{odieresis}
\typicalkerns{oe} \typicalkerns{oslash} \typicalkerns{ugrave}
\typicalkerns{uacute} \typicalkerns{ucircumflex} \typicalkerns{udieresis}
\typicalkerns{yacute} \typicalkerns{thorn} \typicalkerns{germandbls}

%    \end{macrocode}
% Here we set the kernings of the new ligature characters.
%    \begin{macrocode}
\setleftkerning{dotlessj}{s}{1000}
\setrightkerning{dotlessj}{i}{1000}

\setleftkerning{backslash}{f}{1000}
\setrightkerning{backslash}{t}{1000}

\setleftkerning{asciicircum}{s}{1000}
\setrightkerning{asciicircum}{s}{1000}

\setleftkerning{underscore}{s}{1000}
\setrightkerning{unserscore}{t}{1000}

\setleftkerning{bar}{t}{1000}
\setrightkerning{bar}{z}{1000}


%    \end{macrocode}
% The following adjustments are again taken from |latin.mtx|.
%    \begin{macrocode}
\setleftrightkerning{visiblespace}{space}{1000}

\setleftkerning{hyphenchar}{hyphen}{1000}

\setleftrightkerning{Aacute}{A}{1000}
\setleftrightkerning{Abreve}{A}{1000}
\setleftrightkerning{Acircumflex}{A}{1000}
\setleftrightkerning{Adieresis}{A}{1000}
\setleftrightkerning{Agrave}{A}{1000}
\setleftrightkerning{Aogonek}{A}{1000}
\setleftrightkerning{Aring}{A}{1000}
\setleftrightkerning{Atilde}{A}{1000}

\setleftrightkerning{Cacute}{C}{1000}
\setleftrightkerning{Ccaron}{C}{1000}
\setleftrightkerning{Ccedilla}{C}{1000}

\setleftrightkerning{Dcaron}{D}{1000}

\setleftrightkerning{Eacute}{E}{1000}
\setleftrightkerning{Ecaron}{E}{1000}
\setleftrightkerning{Ecircumflex}{E}{1000}
\setleftrightkerning{Edieresis}{E}{1000}
\setleftrightkerning{Egrave}{E}{1000}
\setleftrightkerning{Eogonek}{E}{1000}

\setleftrightkerning{Gbreve}{G}{1000}

\setleftkerning{IJ}{I}{1000}

\setleftrightkerning{Iacute}{I}{1000}
\setleftrightkerning{Icircumflex}{I}{1000}
\setleftrightkerning{Idieresis}{I}{1000}
\setleftrightkerning{Idotaccent}{I}{1000}
\setleftrightkerning{Igrave}{I}{1000}

\setrightkerning{IJ}{J}{1000}

\setleftrightkerning{Lacute}{L}{1000}
\setleftrightkerning{Lslash}{L}{1000}
\setleftkerning{Lcaron}{L}{1000}

\setleftrightkerning{Nacute}{N}{1000}
\setleftrightkerning{Ncaron}{N}{1000}
\setleftrightkerning{Ntilde}{N}{1000}

\setleftrightkerning{Oacute}{O}{1000}
\setleftrightkerning{Ocircumflex}{O}{1000}
\setleftrightkerning{Odieresis}{O}{1000}
\setleftrightkerning{Ograve}{O}{1000}
\setleftrightkerning{Ohungarumlaut}{O}{1000}
\setleftrightkerning{Oslash}{O}{1000}
\setleftrightkerning{Otilde}{O}{1000}

\setleftrightkerning{Rcaron}{R}{1000}
\setleftrightkerning{Racute}{R}{1000}

\setleftrightkerning{SS}{S}{1000}
\setleftrightkerning{Sacute}{S}{1000}
\setleftrightkerning{Scaron}{S}{1000}
\setleftrightkerning{Scedilla}{S}{1000}

\setleftrightkerning{Tcaron}{T}{1000}
\setleftrightkerning{Tcedilla}{T}{1000}

\setleftrightkerning{Uacute}{U}{1000}
\setleftrightkerning{Ucircumflex}{U}{1000}
\setleftrightkerning{Udieresis}{U}{1000}
\setleftrightkerning{Ugrave}{U}{1000}
\setleftrightkerning{Uhungarumlaut}{U}{1000}
\setleftrightkerning{Uring}{U}{1000}

\setleftrightkerning{Yacute}{Y}{1000}
\setleftrightkerning{Ydieresis}{Y}{1000}

\setleftrightkerning{Zacute}{Z}{1000}
\setleftrightkerning{Zcaron}{Z}{1000}
\setleftrightkerning{Zdotaccent}{Z}{1000}

\setleftrightkerning{aogonek}{a}{1000}

\setleftrightkerning{ccedilla}{c}{1000}

\setleftrightkerning{eogonek}{e}{1000}

%    \end{macrocode}
% The following four lines are of course different from |latin.mtx|, because
% these positions contain different ligatures here.  ``ffi'' is ``ch'', and
% ``ffl'' is ``ck''.
%    \begin{macrocode}
\setleftkerning{ffi}{c}{1000}
\setleftkerning{ffl}{c}{1000}
\setrightkerning{ffi}{h}{1000}
\setrightkerning{ffl}{k}{1000}

\setleftrightkerning{ff}{f}{1000}
\setleftkerning{fi}{f}{1000}
\setleftkerning{fl}{f}{1000}

\setleftkerning{ij}{i}{1000}

\setrightkerning{fi}{i}{1000}

\setrightkerning{ij}{j}{1000}

\setrightkerning{fl}{l}{1000}

\setleftkerning{oe}{o}{1000}
\setrightkerning{oe}{e}{1000}
\setleftrightkerning{oslash}{o}{1000}

\setleftrightkerning{scedilla}{s}{1000}

\setleftrightkerning{tcedilla}{t}{1000}

%    \end{macrocode}
%
% \subsection{Faking of glyphs}
%
% \subsubsection{List of unfakable glyphs}
%
% At first, a list of all glyphs that can't be faked.  This list is shorter
% than in |latin.mtx|, because I'm naughty enough to fake more.
%    \begin{macrocode}
\comment{\section{Unfakable glyphs}}

\unfakable{Gamma}
\unfakable{Delta}
\unfakable{Theta}
\unfakable{Lambda}
\unfakable{Xi}
\unfakable{Pi}
\unfakable{Sigma}
\unfakable{Upsilon}
\unfakable{Upsilon1}
\unfakable{Phi}
\unfakable{Psi}
\unfakable{Omega}

\unfakable{grave}
\unfakable{acute}
\unfakable{circumflex}
\unfakable{tilde}
\unfakable{dieresis}
\unfakable{hungarumlaut}
\unfakable{ring}
\unfakable{caron}
\unfakable{breve}
\unfakable{macron}
\unfakable{dotaccent}
\unfakable{cedilla}
\unfakable{ogonek}
\unfakable{guilsinglleft}
\unfakable{guilsinglright}
\unfakable{quotedblleft}
\unfakable{quotedblright}
\unfakable{quotedblbase}
\unfakable{guillemotleft}
\unfakable{guillemotright}
\unfakable{endash}
\unfakable{emdash}
\unfakable{dotlessi}
\unfakable{perthousandzero}
\unfakable{exclam}
\unfakable{quotedbl}
\unfakable{numbersign}
\unfakable{dollar}
\unfakable{percent}
\unfakable{ampersand}
\unfakable{quoteright}
\unfakable{parenleft}
\unfakable{parenright}
\unfakable{asterisk}
\unfakable{plus}
\unfakable{comma}
\unfakable{hyphen}
\unfakable{period}
\unfakable{slash}
\unfakable{zero}
\unfakable{one}
\unfakable{two}
\unfakable{three}
\unfakable{four}
\unfakable{five}
\unfakable{six}
\unfakable{seven}
\unfakable{eight}
\unfakable{nine}
\unfakable{zerooldstyle}
\unfakable{oneoldstyle}
\unfakable{twooldstyle}
\unfakable{threeoldstyle}
\unfakable{fouroldstyle}
\unfakable{fiveoldstyle}
\unfakable{sixoldstyle}
\unfakable{sevenoldstyle}
\unfakable{eightoldstyle}
\unfakable{nineoldstyle}
\unfakable{colon}
\unfakable{semicolon}
\unfakable{less}
\unfakable{equal}
\unfakable{greater}
\unfakable{question}
\unfakable{at}
\unfakable{A}
\unfakable{B}
\unfakable{C}
\unfakable{D}
\unfakable{E}
\unfakable{F}
\unfakable{G}
\unfakable{H}
\unfakable{I}
\unfakable{J}
\unfakable{K}
\unfakable{L}
\unfakable{M}
\unfakable{N}
\unfakable{O}
\unfakable{P}
\unfakable{Q}
\unfakable{R}
\unfakable{S}
\unfakable{T}
\unfakable{U}
\unfakable{V}
\unfakable{W}
\unfakable{X}
\unfakable{Y}
\unfakable{Z}
\unfakable{bracketleft}
\unfakable{bracketright}
\unfakable{quoteleft}
\unfakable{a}
\unfakable{b}
\unfakable{c}
\unfakable{d}
\unfakable{e}
\unfakable{f}
\unfakable{g}
\unfakable{h}
\unfakable{i}
\unfakable{j}
\unfakable{k}
\unfakable{l}
\unfakable{m}
\unfakable{n}
\unfakable{o}
\unfakable{p}
\unfakable{q}
\unfakable{r}
\unfakable{s}
\unfakable{t}
\unfakable{u}
\unfakable{v}
\unfakable{w}
\unfakable{x}
\unfakable{y}
\unfakable{z}
\unfakable{braceleft}
\unfakable{braceright}
\unfakable{asciitilde}
\unfakable{Eng}
\unfakable{section}
\unfakable{eng}
\unfakable{exclamdown}
\unfakable{questiondown}
\unfakable{sterling}
\unfakable{AE}
\unfakable{Eth}
\unfakable{OE}
\unfakable{Oslash}
\unfakable{Thorn}
\unfakable{ae}
\unfakable{eth}
\unfakable{oe}
\unfakable{oslash}
\unfakable{thorn}
\unfakable{germandbls}
\unfakable{lslashslash}

%    \end{macrocode}
%
% \subsubsection{Glyph construction}
%
% Most of this here is taken from |latin.mtx|.
%    \begin{macrocode}
\comment{\section{Glyphs}}

\setglyph{space}
   \ifisglyph{space-not}\then
      \movert{\width{space-not}}
   \else
      \movert{\width{i}}
   \fi
\endsetglyph

\setglyph{compwordmark}
   \glyphrule{0}{\int{xheight}}
\endsetglyph
\setint{compwordmark-spacing}{0}

\setglyph{quotesinglbase}
   \glyph{comma}{1000}
\endsetglyph

%    \end{macrocode} 
% In the slot of the dotless ``j'', we actually have the ``si'' ligature.
%    \begin{macrocode}
\setglyph{dotlessj}
  \glyph{s}{1000}
  \glyph{i}{1000}
\endsetglyph

\setglyph{fi}
   \glyph{f}{1000}
   \movert{\kerning{f}{i}}
   \glyph{i}{1000}
\endsetglyph

\setglyph{ff}
   \glyph{f}{1000}
   \movert{\kerning{f}{f}}
   \glyph{f}{1000}
\endsetglyph

\setglyph{fl}
   \glyph{f}{1000}
   \movert{\kerning{f}{l}}
   \glyph{l}{1000}
\endsetglyph

%    \end{macrocode} 
% In the slot of the ``ffi'' ligature, we actually have the ``ch'' ligature.
%    \begin{macrocode}
\setglyph{ffi}
   \glyph{c}{1000}
   \movert{\kerning{c}{h}}
   \glyph{h}{1000}
\endsetglyph

%    \end{macrocode} 
% In the slot of the ``ffl'' ligature, we actually have the ``ck'' ligature.
%    \begin{macrocode}
\setglyph{ffl}
   \glyph{c}{1000}
   \movert{\kerning{c}{k}}
   \glyph{k}{1000}
\endsetglyph

\setglyph{visiblespace}
   \moveup{\neg{\int{visiblespacedepth}}}
   \movert{\int{visiblespacesurround}}
   \glyphrule
      {\int{underlinethickness}}
      {\int{visiblespacedepth}}
   \glyphrule
      {\int{visiblespacewidth}}
      {\int{underlinethickness}}
   \glyphrule
      {\int{underlinethickness}}
      {\int{visiblespacedepth}}
   \movert{\int{visiblespacesurround}}
   \moveup{\int{visiblespacedepth}}
\endsetglyph

\setglyph{rangedash}
   \ifisint{monowidth}\then
      \glyph{hyphen}{1000}
   \else
      \glyph{endash}{1000}
   \fi
\endsetglyph

\setglyph{punctdash}
   \ifisint{monowidth}\then
      \glyph{hyphen}{1000}
      \glyph{hyphen}{1000}
   \else
      \glyph{emdash}{1000}
   \fi
\endsetglyph

%    \end{macrocode} 
% In the slot of the backslash, we actually have the ``ft'' ligature.  This
% ligature is not available in Haralambous' fonts, but it's a common Fraktur
% ligature and other font that you may get will probably have it.
%    \begin{macrocode}
\setglyph{backslash}
  \glyph{f}{1000}
  \glyph{t}{1000}
\endsetglyph

%    \end{macrocode} 
% In the slot of the ascii circumflex, we actually have the ``ss'' ligature.
%    \begin{macrocode}
\setglyph{asciicircum}
  \glyph{s}{1000}
  \glyph{s}{1000}
\endsetglyph

%    \end{macrocode} 
% In the slot of the underscore, we actually have the ``st'' ligature.
%    \begin{macrocode}
\setglyph{underscore}
  \glyph{s}{1000}
  \glyph{t}{1000}
\endsetglyph

%    \end{macrocode} 
% In the slot of the vertical bar, we actually have the ``tz'' ligature.
%    \begin{macrocode}
\setglyph{bar}
  \glyph{t}{1000}
  \glyph{z}{1000}
\endsetglyph

%    \end{macrocode} 
% The hyphen character has width zero, which is different from |latin.mtx|.
%    \begin{macrocode}
\setglyph{hyphenchar}
  \glyph{hyphen}{1000}
%  \resetwidth{0}
\endsetglyph

\setglyph{ringfitted}
   \movert{\div{\sub{\width{A}}{\width{ring}}}{2}}
   \glyph{ring}{1000}
   \movert{\div{\sub{\width{A}}{\width{ring}}}{2}}
\endsetglyph

\setglyph{lslash}
   \glyph{lslashslash}{1000}
   \movert{\kerning{lslashslash}{l}}
   \glyph{l}{1000}
\endsetglyph

\setglyph{Lslash}
   \glyph{lslashslash}{1000}
   \movert{\kerning{lslashslash}{L}}
   \glyph{L}{1000}
\endsetglyph

\setglyph{Abreve}
   \topaccent{A}{breve}{500}
\endsetglyph

\setglyph{Aogonek}
   \botaccent{A}{ogonek}{900}
\endsetglyph

\setglyph{Cacute}
   \topaccent{C}{acute}{500}
\endsetglyph

\setglyph{Ccaron}
   \topaccent{C}{caron}{500}
\endsetglyph

\setglyph{Dcaron}
   \topaccent{D}{caron}{450}
\endsetglyph

\setglyph{Ecaron}
   \topaccent{E}{caron}{500}
\endsetglyph

\setglyph{Eogonek}
   \botaccent{E}{ogonek}{850}
\endsetglyph

\setglyph{Gbreve}
   \topaccent{G}{breve}{500}
\endsetglyph

\setglyph{Lacute}
   \topaccent{L}{acute}{250}
\endsetglyph

\setglyph{Lcaron}
   \glyph{L}{1000}
   \ifisint{monowidth}\then\else
      \movert{-200}
   \fi
   \glyph{quoteright}{1000}
\endsetglyph

\setglyph{Nacute}
   \topaccent{N}{acute}{500}
\endsetglyph

\setglyph{Ncaron}
   \topaccent{N}{caron}{500}
\endsetglyph

\setglyph{Ohungarumlaut}
   \topaccent{O}{hungarumlaut}{500}
\endsetglyph

\setglyph{Racute}
   \topaccent{R}{acute}{500}
\endsetglyph

\setglyph{Rcaron}
   \topaccent{R}{caron}{500}
\endsetglyph

\setglyph{Sacute}
   \topaccent{S}{acute}{500}
\endsetglyph

\setglyph{Scaron}
   \topaccent{S}{caron}{500}
\endsetglyph

\setglyph{Scedilla}
   \botaccent{S}{cedilla}{500}
\endsetglyph

\setglyph{Tcaron}
   \topaccent{T}{caron}{500}
\endsetglyph

\setglyph{Tcedilla}
   \botaccent{T}{cedilla}{500}
\endsetglyph

\setglyph{Uhungarumlaut}
   \topaccent{U}{hungarumlaut}{500}
\endsetglyph

\setglyph{Uring}
   \topaccent{U}{ring}{500}
\endsetglyph

\setglyph{Ydieresis}
   \topaccent{Y}{dieresis}{500}
\endsetglyph

\setglyph{Zacute}
   \topaccent{Z}{acute}{500}
\endsetglyph

\setglyph{Zcaron}
   \topaccent{Z}{caron}{500}
\endsetglyph

\setglyph{Zdotaccent}
   \topaccent{Z}{dotaccent}{500}
\endsetglyph

\setglyph{IJ}
   \glyph{I}{1000}
   \movert{\kerning{I}{J}}
   \glyph{J}{1000}
\endsetglyph

\setglyph{Idotaccent}
   \topaccent{I}{dotaccent}{500}
\endsetglyph

\setglyph{dbar}
   \push
      \movert{\sub{\width{d}}{\width{macron}}}
      \moveup{
         \sub{\scale{\add{\height{d}}{\int{xheight}}}{500}}
             {\height{macron}}}
      \glyph{macron}{1000}
   \pop
   \glyph{d}{1000}
\endsetglyph

\setglyph{abreve}
   \topaccent{a}{breve}{500}
\endsetglyph

\setglyph{aogonek}
   \botaccent{a}{ogonek}{850}
\endsetglyph

\setglyph{cacute}
   \topaccent{c}{acute}{500}
\endsetglyph

\setglyph{ccaron}
   \topaccent{c}{caron}{500}
\endsetglyph

\setglyph{dcaron}
   \glyph{d}{1000}
   \ifisint{monowidth}\then\else
      \movert{-75}
   \fi
   \glyph{quoteright}{1000}
\endsetglyph

\setglyph{ecaron}
   \topaccent{e}{caron}{500}
\endsetglyph

\setglyph{eogonek}
   \botaccent{e}{ogonek}{500}
\endsetglyph

\setglyph{gbreve}
   \topaccent{g}{breve}{500}
\endsetglyph

\setglyph{lacute}
   \topaccent{l}{acute}{500}
\endsetglyph

\setglyph{lcaron}
   \glyph{l}{1000}
   \ifisint{monowidth}\then\else
      \movert{-100}
   \fi
   \glyph{quoteright}{1000}
\endsetglyph

\setglyph{nacute}
   \topaccent{n}{acute}{500}
\endsetglyph

\setglyph{ncaron}
   \topaccent{n}{caron}{500}
\endsetglyph

\setglyph{ohungarumlaut}
   \topaccent{o}{hungarumlaut}{500}
\endsetglyph

\setglyph{racute}
   \topaccent{r}{acute}{500}
\endsetglyph

\setglyph{rcaron}
   \topaccent{r}{caron}{500}
\endsetglyph

\setglyph{sacute}
   \topaccent{s}{acute}{500}
\endsetglyph

\setglyph{scaron}
   \topaccent{s}{caron}{500}
\endsetglyph

\setglyph{scedilla}
   \botaccent{s}{cedilla}{500}
\endsetglyph

\setglyph{tcaron}
   \glyph{t}{1000}
   \ifisint{monowidth}\then\else
      \movert{-75}
   \fi
   \glyph{quoteright}{1000}
\endsetglyph

\setglyph{tcedilla}
   \botaccent{t}{cedilla}{500}
\endsetglyph

\setglyph{uhungarumlaut}
   \topaccent{u}{hungarumlaut}{500}
\endsetglyph

\setglyph{uring}
   \topaccent{u}{ring}{500}
\endsetglyph

\setglyph{ydieresis}
   \topaccent{y}{dieresis}{500}
\endsetglyph

\setglyph{zacute}
   \topaccent{z}{acute}{500}
\endsetglyph

\setglyph{zcaron}
   \topaccent{z}{caron}{500}
\endsetglyph

\setglyph{zdotaccent}
   \topaccent{z}{dotaccent}{500}
\endsetglyph

\setglyph{ij}
   \glyph{i}{1000}
   \movert{\kerning{i}{j}}
   \glyph{j}{1000}
\endsetglyph

\setglyph{Agrave}
   \topaccent{A}{grave}{500}
\endsetglyph

\setglyph{Aacute}
   \topaccent{A}{acute}{500}
\endsetglyph

\setglyph{Acircumflex}
   \topaccent{A}{circumflex}{500}
\endsetglyph

\setglyph{Atilde}
   \topaccent{A}{tilde}{500}
\endsetglyph

\setglyph{Adieresis}
   \topaccent{A}{dieresis}{500}
\endsetglyph

\setglyph{Aring}
   \topaccent{A}{ring}{500}
\endsetglyph

\setglyph{Ccedilla}
   \botaccent{C}{cedilla}{500}
\endsetglyph

\setglyph{Egrave}
   \topaccent{E}{grave}{500}
\endsetglyph

\setglyph{Eacute}
   \topaccent{E}{acute}{500}
\endsetglyph

\setglyph{Ecircumflex}
   \topaccent{E}{circumflex}{500}
\endsetglyph

\setglyph{Edieresis}
 \topaccent{E}{dieresis}{500}
\endsetglyph

\setglyph{Iacute}
   \topaccent{I}{acute}{500}
\endsetglyph

\setglyph{Igrave}
   \topaccent{I}{grave}{500}
\endsetglyph

\setglyph{Icircumflex}
   \topaccent{I}{circumflex}{500}
\endsetglyph

\setglyph{Idieresis}
   \topaccent{I}{dieresis}{500}
\endsetglyph

\setglyph{Ntilde}
   \topaccent{N}{tilde}{500}
\endsetglyph

\setglyph{Ograve}
   \topaccent{O}{grave}{500}
\endsetglyph

\setglyph{Oacute}
   \topaccent{O}{acute}{500}
\endsetglyph

\setglyph{Ocircumflex}
   \topaccent{O}{circumflex}{500}
\endsetglyph

\setglyph{Otilde}
   \topaccent{O}{tilde}{500}
\endsetglyph

\setglyph{Odieresis}
   \topaccent{O}{dieresis}{500}
\endsetglyph

\setglyph{Ugrave}
   \topaccent{U}{grave}{500}
\endsetglyph

\setglyph{Uacute}
   \topaccent{U}{acute}{500}
\endsetglyph

\setglyph{Ucircumflex}
   \topaccent{U}{circumflex}{500}
\endsetglyph

\setglyph{Udieresis}
   \topaccent{U}{dieresis}{500}
\endsetglyph

\setglyph{Yacute}
   \topaccent{Y}{acute}{500}
\endsetglyph

\setglyph{SS}
   \glyph{S}{1000}
   \movert{\kerning{S}{S}}
   \glyph{S}{1000}
\endsetglyph

\setglyph{agrave}
   \topaccent{a}{grave}{500}
\endsetglyph

\setglyph{aacute}
   \topaccent{a}{acute}{500}
\endsetglyph

\setglyph{acircumflex}
   \topaccent{a}{circumflex}{500}
\endsetglyph

\setglyph{atilde}
   \topaccent{a}{tilde}{500}
\endsetglyph

\setglyph{adieresis}
   \topaccent{a}{dieresis}{500}
\endsetglyph

\setglyph{aring}
   \topaccent{a}{ring}{500}
\endsetglyph

\setglyph{ccedilla}
   \botaccent{c}{cedilla}{500}
\endsetglyph

\setglyph{egrave}
   \topaccent{e}{grave}{500}
\endsetglyph

\setglyph{eacute}
   \topaccent{e}{acute}{500}
\endsetglyph

\setglyph{ecircumflex}
   \topaccent{e}{circumflex}{500}
\endsetglyph

\setglyph{edieresis}
   \topaccent{e}{dieresis}{500}
\endsetglyph

\setglyph{igrave}
   \topaccent{dotlessi}{grave}{500}
\endsetglyph

\setglyph{iacute}
   \topaccent{dotlessi}{acute}{500}
\endsetglyph

\setglyph{icircumflex}
   \topaccent{dotlessi}{circumflex}{500}
\endsetglyph

\setglyph{idieresis}
   \topaccent{dotlessi}{dieresis}{500}
\endsetglyph

\setglyph{ntilde}
   \topaccent{n}{tilde}{500}
\endsetglyph

\setglyph{ograve}
   \topaccent{o}{grave}{500}
\endsetglyph

\setglyph{oacute}
   \topaccent{o}{acute}{500}
\endsetglyph

\setglyph{ocircumflex}
   \topaccent{o}{circumflex}{500}
\endsetglyph

\setglyph{otilde}
   \topaccent{o}{tilde}{500}
\endsetglyph

\setglyph{odieresis}
   \topaccent{o}{dieresis}{500}
\endsetglyph

\setglyph{ugrave}
   \topaccent{u}{grave}{500}
\endsetglyph

\setglyph{uacute}
   \topaccent{u}{acute}{500}
\endsetglyph

\setglyph{ucircumflex}
   \topaccent{u}{circumflex}{500}
\endsetglyph

\setglyph{udieresis}
   \topaccent{u}{dieresis}{500}
\endsetglyph

\setglyph{yacute}
   \topaccent{y}{acute}{500}
\endsetglyph

%    \end{macrocode}
%
% \subsection{Deleting depths from accents}
%
%    \begin{macrocode}
\comment{\section{Accents}
   \TeX's math accent-positioning requires accents to have zero depth.}

\setcommand\zerodepth#1{
  \ifisglyph{#1}\then
   \resetglyph{#1}
        \glyph{#1}{1000}
        \resetdepth{0}
   \endresetglyph
  \fi
}
\zerodepth{grave}
\zerodepth{acute}
\zerodepth{circumflex}
\zerodepth{tilde}
\zerodepth{dieresis}
\zerodepth{hungarumlaut}
\zerodepth{ring}
\zerodepth{caron}
\zerodepth{breve}
\zerodepth{macron}
\zerodepth{dotaccent}

\endmetrics
\end{document}
%</blacklettermtx>
%    \end{macrocode}
%
%
% \section{The blackletter typefaces encoding vector, a T1 variant}
%
% This file |t1frak.etx| is a simplification and modification of |t1.etx| of
% the \texttt{fontinst} package.  It describes the output encoding vector of
% the blackletter typefaces.
%
% By and large, this is T1.  Only some not so frequently used characters have 
% been replaced with additional ligatures and the so called ``round~s''.
%
% However, the \emph{names} of the glyphs here are the same as in real T1.
% I wasn't brave enough to change them, but if it turns out that it's 
% completely undangerous to use other name, I may change them.  Both variants
% may cause confusion.
%
% The following substitutions are made from real T1:
% \begin{itemize}
% \item ``\j''~$\to$ ``si'' ligature
% \item ``ffi'' ligature~$\to$ ``ch'' ligature
% \item ``ffl''~$\to$ ``ck'' ligature
% \item ``$\backslash$''~$\to$ ``ft'' ligature
% \item ``$\hat{\ }$''~$\to$ ``ss'' ligature
% \item ``\underline{ }''~$\to$ ``st'' ligature
% \item ``$\vert$''~$\to$ ``tz'' ligature
% \item ``Eng''~$\to$ ``S''
% \item ``eng''~$\to$ ``round~s''
% \end{itemize}
%    \begin{macrocode}
%<*t1fraketx>
\relax

\documentclass[twocolumn]{article}
\usepackage[TS1,T1]{fontenc}
\usepackage{textcomp}
\usepackage{fontdoc}

\title{The Fraktur encoding vector, a T1 variant}
\author{Torsten Bronger}
\date{29 May 2002 \\
Version 1.0}

\begin{document}

\maketitle

\section{Introduction}

This document describes a modified Cork (T1) text encoding, modified
for the so called old German fonts.

\newcommand*{\actually}[1]{ \emph{\bfseries Actually, here
    in the Fraktur encoding, the #1}.}

\encoding

\needsfontinstversion{1.801}

%    \end{macrocode}
%
% \subsection{Default settings}
%
%    \begin{macrocode}
\comment{\section{Default values}}

\setstr{codingscheme}{EXTENDED TEX FONT ENCODING - LATIN}

\setcommand\lc#1#2{#2}
\setcommand\uc#1#2{#1}
\setcommand\lctop#1#2{#2}
\setcommand\uctop#1#2{#1}
\setcommand\lclig#1#2{#2}
\setcommand\uclig#1#2{#1}
\setcommand\digit#1{#1}

\setint{italicslant}{0}

\ifisglyph{x}\then
   \setint{xheight}{\height{x}}
\else
   \setint{xheight}{500}
\fi

\ifisglyph{space}\then
  \setint{interword}{\width{space}}
\else\ifisglyph{i}\then
  \setint{interword}{\width{i}}
\else
  \setint{interword}{333}
\fi\fi

% added by Thierry Bouche <Thierry.Bouche@ujf-grenoble.fr>
% 1997/02/07 to calculate values for extra EC fontdimens
% Amended by SPQR 1997/02/09
\ifisglyph{X}\then
   \setint{capheight}{\height{X}}
\else
   \setint{capheight}{750}
\fi

\ifisglyph{k}\then
   \setint{ascender}{\height{k}}
\else
   \ifisint{capheight}\then
        \setint{ascender}{\int{capheight}}
   \else
        \setint{ascender}{750}
\fi\fi

\ifisglyph{Aring}\then
   \setint{acccapheight}{\height{Aring}}
\else
   \setint{acccapheight}{999}
\fi

\ifisint{descender_neg}\then
  \setint{descender}{\neg{\int{descender_neg}}}
 \else
    \ifisglyph{p}\then
      \setint{descender}{\depth{p}}
   \else
      \setint{descender}{250}
   \fi
\fi

\ifisglyph{Aring}\then
   \setint{maxheight}{\height{Aring}}
\else
   \setint{maxheight}{1000}
\fi

\ifisint{maxdepth_neg}\then
  \setint{maxdepth}{\neg{\int{maxdepth_neg}}}
\else
   \ifisglyph{j}\then
    \setint{maxdepth}{\depth{j}}
  \else
    \setint{maxdepth}{250}
  \fi
\fi

\ifisglyph{six}\then
   \setint{digitwidth}{\width{six}}
\else
   \setint{digitwidth}{500}
\fi

\setint{capstem}{0} % not in AFM files
\setint{baselineskip}{1200}
% end changes by Thierry


\comment{\section{Default font dimensions}}

\setint{fontdimen(1)}{\int{italicslant}}              % italic slant
\setint{fontdimen(2)}{\int{interword}}                % interword space
\ifisint{monowidth}\then
   \setint{fontdimen(3)}{0}                           % interword stretch
   \setint{fontdimen(4)}{0}                           % interword shrink
\else
   \setint{fontdimen(3)}{\scale{\int{interword}}{600}}% interword stretch
%    \end{macrocode}
% The following was changed from 240 to 150.
%    \begin{macrocode}
   \setint{fontdimen(4)}{\scale{\int{interword}}{150}}% interword shrink
\fi
\setint{fontdimen(5)}{\int{xheight}}                  % x-height
\setint{fontdimen(6)}{1000}                           % quad
\ifisint{monowidth}\then
   \setint{fontdimen(7)}{\int{interword}}             % extra space after .
\else
%    \end{macrocode}
% The following was changed from 240 to 150.
%    \begin{macrocode}
   \setint{fontdimen(7)}{\scale{\int{interword}}{150}}% extra space after .
\fi
% added by Thierry Bouche <Thierry.Bouche@ujf-grenoble.fr> 1997/02/07
\setint{fontdimen(8)}{\int{capheight}}   % cap height
\setint{fontdimen(9)}{\int{ascender}}  % ascender
\setint{fontdimen(10)}{\int{acccapheight}} % accented cap height
\setint{fontdimen(11)}{\int{descender}} % descender's depth
\setint{fontdimen(12)}{\int{maxheight}} % max height
\setint{fontdimen(13)}{\int{maxdepth}} % max depth
\setint{fontdimen(14)}{\int{digitwidth}} % digit width
\setint{fontdimen(15)}{\int{capstem}} % cap_stem
\setint{fontdimen(16)}{\int{baselineskip}} % baselineskip

%    \end{macrocode}
%
% \subsection{The encoding vector}
%
%    \begin{macrocode}
\comment{\section{The encoding}
   There are 256 glyphs in this encoding.}

\setslot{\lc{Grave}{grave}}
   \comment{The grave accent `\`{}'.}
\endsetslot

\setslot{\lc{Acute}{acute}}
   \comment{The acute accent `\'{}'.}
\endsetslot

\setslot{\lc{Circumflex}{circumflex}}
   \comment{The circumflex accent `\^{}'.}
\endsetslot

\setslot{\lc{Tilde}{tilde}}
   \comment{The tilde accent `\~{}'.}
\endsetslot

\setslot{\lc{Dieresis}{dieresis}}
   \comment{The umlaut or dieresis accent `\"{}'.}
\endsetslot

\setslot{\lc{Hungarumlaut}{hungarumlaut}}
   \comment{The long Hungarian umlaut `\H{}'.}
\endsetslot

\setslot{\lc{Ring}{ring}}
   \comment{The ring accent `\r{}'.}
\endsetslot

\setslot{\lc{Caron}{caron}}
   \comment{The caron or h\'a\v cek accent `\v{}'.}
\endsetslot

\setslot{\lc{Breve}{breve}}
   \comment{The breve accent `\u{}'.}
\endsetslot

\setslot{\lc{Macron}{macron}}
   \comment{The macron accent `\={}'.}
\endsetslot

\setslot{\lc{Dotaccent}{dotaccent}}
   \comment{The dot accent `\.{}'.}
\endsetslot

\setslot{\lc{Cedilla}{cedilla}}
   \comment{The cedilla accent `\c {}'.}
\endsetslot

\setslot{\lc{Ogonek}{ogonek}}
   \comment{The ogonek accent `\k {}'.}
\endsetslot

\setslot{quotesinglbase}
  \comment{A German single quote mark `\quotesinglbase' similar to a comma,
      but with different sidebearings.}
\endsetslot

\setslot{guilsinglleft}
  \comment{A French single opening quote mark `\guilsinglleft',
      unavailable in \plain\ \TeX.}
\endsetslot

\setslot{guilsinglright}
  \comment{A French single closing quote mark `\guilsinglright',
      unavailable in \plain\ \TeX.}
\endsetslot

\setslot{quotedblleft}
  \comment{The English opening quote mark `\,\textquotedblleft\,'.}
\endsetslot

\setslot{quotedblright}
  \comment{The English closing quote mark `\,\textquotedblright\,'.}
\endsetslot

\setslot{quotedblbase}
  \comment{A German double quote mark `\quotedblbase' similar to two commas,
      but with tighter letterspacing and different sidebearings.}
\endsetslot

\setslot{guillemotleft}
  \comment{A French double opening quote mark `\guillemotleft',
      unavailable in \plain\ \TeX.}
\endsetslot

\setslot{guillemotright}
  \comment{A French closing opening quote mark `\guillemotright',
      unavailable in \plain\ \TeX.}
\endsetslot

\setslot{rangedash}
   \ligature{LIG}{hyphen}{punctdash}
   \comment{The number range dash `1--9'.  In a monowidth font, this
      might be set as `{\tt 1{-}9}'.}
\endsetslot

\setslot{punctdash}
   \comment{The punctuation dash `Oh---boy.'  In a monowidth font, this
      might be set as `{\tt Oh{-}{-}boy.}'}
\endsetslot

\setslot{compwordmark}
   \comment{An invisible glyph, with zero width and depth, but the
      height of lowercase letters without ascenders.
      It is used to stop ligaturing in words like `shelf{}ful'.}
\endsetslot

\setslot{perthousandzero}
   \comment{A glyph which is placed after `\%' to produce a
      `per-thousand', or twice to produce `per-ten-thousand'.
      Your guess is as good as mine as to what this glyph should look
      like in a monowidth font.}
\endsetslot

\setslot{\lc{dotlessI}{dotlessi}}
   \comment{A dotless i `\i', used to produce accented letters such as
      `\=\i'.}
\endsetslot

\setslot{\lc{dotlessJ}{dotlessj}}
   \comment{A dotless j `\j', used to produce accented letters such as
      `\=\j'.  Most non-\TeX\ fonts do not have this glyph.  \actually{`ssi'
      ligature}}
\endsetslot

\setslot{\lclig{FF}{ff}}
   \comment{The `ff' ligature.  It should be two characters wide in a
      monowidth font.}
\endsetslot

\setslot{\lclig{FI}{fi}}
   \comment{The `fi' ligature.  It should be two characters wide in a
      monowidth font.}
\endsetslot

\setslot{\lclig{FL}{fl}}
   \comment{The `fl' ligature.  It should be two characters wide in a
      monowidth font.}
\endsetslot

\setslot{\lclig{FFI}{ffi}}
   \comment{The `ffi' ligature.  It should be three characters wide in a
      monowidth font. \actually{`ch' ligature}}
\endsetslot

\setslot{\lclig{FFL}{ffl}}
   \comment{The `ffl' ligature.  It should be three characters wide in a
      monowidth font. \actually{`ck' ligature}}
\endsetslot

\setslot{visiblespace}
   \comment{A visible space glyph `\textvisiblespace'.}
\endsetslot

\setslot{exclam}
   \ligature{LIG}{quoteleft}{exclamdown}
   \comment{The exclamation mark `!'.}
\endsetslot

\setslot{quotedbl}
  \comment{The `neutral' double quotation mark `\,\textquotedbl\,',
      included for use in monowidth fonts, or for setting computer
      programs.  Note that the inclusion of this glyph in this slot
      means that \TeX\ documents which used `{\tt\char`\"}' as an
      input character will no longer work.}
\endsetslot

\setslot{numbersign}
   \comment{The hash sign `\#'.}
\endsetslot

\setslot{dollar}
   \comment{The dollar sign `\$'.}
\endsetslot

\setslot{percent}
   \comment{The percent sign `\%'.}
\endsetslot

\setslot{ampersand}
   \comment{The ampersand sign `\&'.}
\endsetslot

\setslot{quoteright}
   \ligature{LIG}{quoteright}{quotedblright}
   \comment{The English closing single quote mark `\,\textquoteright\,'.}
\endsetslot

\setslot{parenleft}
   \comment{The opening parenthesis `('.}
\endsetslot

\setslot{parenright}
   \comment{The closing parenthesis `)'.}
\endsetslot

\setslot{asterisk}
   \comment{The raised asterisk `*'.}
\endsetslot

\setslot{plus}
   \comment{The addition sign `+'.}
\endsetslot

\setslot{comma}
   \ligature{LIG}{comma}{quotedblbase}
   \comment{The comma `,'.}
\endsetslot

\setslot{hyphen}
   \ligature{LIG}{hyphen}{rangedash}
   \ligature{LIG}{hyphenchar}{hyphenchar}
   \comment{The hyphen `-'.}
\endsetslot

\setslot{period}
   \comment{The period `.'.}
\endsetslot

\setslot{slash}
   \comment{The forward oblique `/'.}
\endsetslot

\setslot{\digit{zero}}
   \comment{The number `0'.  This (and all the other numerals) may be
      old style or ranging digits.}
\endsetslot

\setslot{\digit{one}}
   \comment{The number `1'.}
\endsetslot

\setslot{\digit{two}}
   \comment{The number `2'.}
\endsetslot

\setslot{\digit{three}}
   \comment{The number `3'.}
\endsetslot

\setslot{\digit{four}}
   \comment{The number `4'.}
\endsetslot

\setslot{\digit{five}}
   \comment{The number `5'.}
\endsetslot

\setslot{\digit{six}}
   \comment{The number `6'.}
\endsetslot

\setslot{\digit{seven}}
   \comment{The number `7'.}
\endsetslot

\setslot{\digit{eight}}
   \comment{The number `8'.}
\endsetslot

\setslot{\digit{nine}}
   \comment{The number `9'.}
\endsetslot

\setslot{colon}
   \comment{The colon punctuation mark `:'.}
\endsetslot

\setslot{semicolon}
   \comment{The semi-colon punctuation mark `;'.}
\endsetslot

\setslot{less}
   \ligature{LIG}{less}{guillemotleft}
   \comment{The less-than sign `\textless'.}
\endsetslot

\setslot{equal}
   \comment{The equals sign `='.}
\endsetslot

\setslot{greater}
   \ligature{LIG}{greater}{guillemotright}
   \comment{The greater-than sign `\textgreater'.}
\endsetslot

\setslot{question}
   \ligature{LIG}{quoteleft}{questiondown}
   \comment{The question mark `?'.}
\endsetslot

\setslot{at}
   \comment{The at sign `@'.}
\endsetslot

\setslot{\uc{A}{a}}
   \comment{The letter `{A}'.}
\endsetslot

\setslot{\uc{B}{b}}
   \comment{The letter `{B}'.}
\endsetslot

\setslot{\uc{C}{c}}
   \comment{The letter `{C}'.}
\endsetslot

\setslot{\uc{D}{d}}
   \comment{The letter `{D}'.}
\endsetslot

\setslot{\uc{E}{e}}
   \comment{The letter `{E}'.}
\endsetslot

\setslot{\uc{F}{f}}
   \comment{The letter `{F}'.}
\endsetslot

\setslot{\uc{G}{g}}
   \comment{The letter `{G}'.}
\endsetslot

\setslot{\uc{H}{h}}
   \comment{The letter `{H}'.}
\endsetslot

\setslot{\uc{I}{i}}
   \comment{The letter `{I}'.}
\endsetslot

\setslot{\uc{J}{j}}
   \comment{The letter `{J}'.  But actually just an `I'\@.}
\endsetslot

\setslot{\uc{K}{k}}
   \comment{The letter `{K}'.}
\endsetslot

\setslot{\uc{L}{l}}
   \comment{The letter `{L}'.}
\endsetslot

\setslot{\uc{M}{m}}
   \comment{The letter `{M}'.}
\endsetslot

\setslot{\uc{N}{n}}
   \comment{The letter `{N}'.}
\endsetslot

\setslot{\uc{O}{o}}
   \comment{The letter `{O}'.}
\endsetslot

\setslot{\uc{P}{p}}
   \comment{The letter `{P}'.}
\endsetslot

\setslot{\uc{Q}{q}}
   \comment{The letter `{Q}'.}
\endsetslot

\setslot{\uc{R}{r}}
   \comment{The letter `{R}'.}
\endsetslot

\setslot{\uc{S}{s}}
   \comment{The letter `{S}'.}
\endsetslot

\setslot{\uc{T}{t}}
   \comment{The letter `{T}'.}
\endsetslot

\setslot{\uc{U}{u}}
   \comment{The letter `{U}'.}
\endsetslot

\setslot{\uc{V}{v}}
   \comment{The letter `{V}'.}
\endsetslot

\setslot{\uc{W}{w}}
   \comment{The letter `{W}'.}
\endsetslot

\setslot{\uc{X}{x}}
   \comment{The letter `{X}'.}
\endsetslot

\setslot{\uc{Y}{y}}
   \comment{The letter `{Y}'.}
\endsetslot

\setslot{\uc{Z}{z}}
   \comment{The letter `{Z}'.}
\endsetslot

\setslot{bracketleft}
   \comment{The opening square bracket `['.}
\endsetslot

\setslot{backslash}
   \comment{The backwards oblique `\textbackslash'.  \actually{`sf' ligature}}
\endsetslot

\setslot{bracketright}
   \comment{The closing square bracket `]'.}
\endsetslot

\setslot{asciicircum}
   \ligature{LIG}{visiblespace}{germandbls}
   \comment{The ASCII upward-pointing arrow head `\textasciicircum'.
      This is included for compatibility with typewriter fonts used
      for computer listings.  \actually{`ss' ligature}}
\endsetslot

\setslot{underscore}
   \comment{The ASCII underline character `\textunderscore', usually
      set on the baseline.
      This is included for compatibility with typewriter fonts used
      for computer listings.  \actually{`st' ligature}}
\endsetslot

\setslot{quoteleft}
   \ligature{LIG}{quoteleft}{quotedblleft}
   \comment{The English opening single quote mark `\,\textquoteleft\,'.}
\endsetslot

\setslot{\lc{A}{a}}
   \comment{The letter `{a}'.}
\endsetslot

\setslot{\lc{B}{b}}
   \comment{The letter `{b}'.}
\endsetslot

\setslot{\lc{C}{c}}
   \ligature{LIG}{h}{ffi}
   \ligature{LIG}{k}{ffl}
   \comment{The letter `{c}'.}
\endsetslot

\setslot{\lc{D}{d}}
   \comment{The letter `{d}'.}
\endsetslot

\setslot{\lc{E}{e}}
   \comment{The letter `{e}'.}
\endsetslot

\setslot{\lc{F}{f}}
   \ligature{LIG}{\lc{I}{i}}{\lclig{FI}{fi}}
   \ligature{LIG}{\lc{F}{f}}{\lclig{FF}{ff}}
   \ligature{LIG}{\lc{L}{l}}{\lclig{FL}{fl}}
   \ligature{LIG}{\lc{T}{t}}{\lclig{backslash}{backslash}}
   \comment{The letter `{f}'.}
\endsetslot

\setslot{\lc{G}{g}}
   \comment{The letter `{g}'.}
\endsetslot

\setslot{\lc{H}{h}}
   \comment{The letter `{h}'.}
\endsetslot

\setslot{\lc{I}{i}}
   \comment{The letter `{i}'.}
\endsetslot

\setslot{\lc{J}{j}}
   \comment{The letter `{j}'.}
\endsetslot

\setslot{\lc{K}{k}}
   \comment{The letter `{k}'.}
\endsetslot

\setslot{\lc{L}{l}}
   \comment{The letter `{l}'.}
\endsetslot

\setslot{\lc{M}{m}}
   \comment{The letter `{m}'.}
\endsetslot

\setslot{\lc{N}{n}}
   \comment{The letter `{n}'.}
\endsetslot

\setslot{\lc{O}{o}}
   \comment{The letter `{o}'.}
\endsetslot

\setslot{\lc{P}{p}}
   \comment{The letter `{p}'.}
\endsetslot

\setslot{\lc{Q}{q}}
   \comment{The letter `{q}'.}
\endsetslot

\setslot{\lc{R}{r}}
   \comment{The letter `{r}'.}
\endsetslot

\setslot{\lc{S}{s}}
%    \end{macrocode}
% The following ligature is with the boundary character and performs the
% transformation of the long~s to the round~s.
%    \begin{macrocode}
   \ligature{LIG}{visiblespace}{ng}
   \ligature{LIG}{t}{underscore}
   \ligature{LIG}{s}{asciicircum}
   \ligature{LIG}{f}{backslash}
   \ligature{LIG}{i}{dotlessj}
%    \end{macrocode}
% The following ligatures make the long~s to become a round~s if one of these
% characters immediately follows.  This prevents the user to insert |\/| and
% similar things at ubiquitous places.
%    \begin{macrocode}
   \ligature{LIG/}{quoteright}{ng}
   \ligature{LIG/}{guilsinglleft}{ng}
   \ligature{LIG/}{guilsinglright}{ng}
   \ligature{LIG/}{quotedblleft}{ng}
   \ligature{LIG/}{quotedblright}{ng}
   \ligature{LIG/}{guillemotleft}{ng}
   \ligature{LIG/}{guillemotright}{ng}
   \ligature{LIG/}{period}{ng}
   \ligature{LIG/}{comma}{ng}
   \ligature{LIG/}{colon}{ng}
   \ligature{LIG/}{semicolon}{ng}
   \ligature{LIG/}{exclam}{ng}
   \ligature{LIG/}{question}{ng}
   \ligature{LIG/}{slash}{ng}
   \ligature{LIG/}{hyphen}{ng}
   \ligature{LIG/}{parenleft}{ng}
   \ligature{LIG/}{parenright}{ng}
   \comment{The letter `{s}'.  \actually{letter long `s'}}
\endsetslot

\setslot{\lc{T}{t}}
   \ligature{LIG}{z}{bar}
   \comment{The letter `{t}'.}
\endsetslot

\setslot{\lc{U}{u}}
   \comment{The letter `{u}'.}
\endsetslot

\setslot{\lc{V}{v}}
   \comment{The letter `{v}'.}
\endsetslot

\setslot{\lc{W}{w}}
   \comment{The letter `{w}'.}
\endsetslot

\setslot{\lc{X}{x}}
   \comment{The letter `{x}'.}
\endsetslot

\setslot{\lc{Y}{y}}
   \comment{The letter `{y}'.}
\endsetslot

\setslot{\lc{Z}{z}}
   \comment{The letter `{z}'.}
\endsetslot

\setslot{braceleft}
   \comment{The opening curly brace `\textbraceleft'.}
\endsetslot

\setslot{bar}
   \comment{The ASCII vertical bar `\textbar'.
      This is included for compatibility with typewriter fonts used
      for computer listings.  \actually{`tz' ligature}}
\endsetslot

\setslot{braceright}
   \comment{The closing curly brace `\textbraceright'.}
\endsetslot

\setslot{asciitilde}
   \comment{The ASCII tilde `\textasciitilde'.
      This is included for compatibility with typewriter fonts used
      for computer listings.}
\endsetslot

\setslot{hyphenchar}
   \comment{The glyph used for hyphenation in this font, which will
      almost always be the same as `hyphen'.}
\endsetslot


\setslot{\uctop{Abreve}{abreve}}
   \comment{The letter `\u A'.}
\endsetslot

\setslot{\uc{Aogonek}{aogonek}}
   \comment{The letter `\k A'.}
\endsetslot

\setslot{\uctop{Cacute}{cacute}}
   \comment{The letter `\' C'.}
\endsetslot

\setslot{\uctop{Ccaron}{ccaron}}
   \comment{The letter `\v C'.}
\endsetslot

\setslot{\uctop{Dcaron}{dcaron}}
   \comment{The letter `\v D'.}
\endsetslot

\setslot{\uctop{Ecaron}{ecaron}}
   \comment{The letter `\v E'.}
\endsetslot

\setslot{\uc{Eogonek}{eogonek}}
   \comment{The letter `\k E'.}
\endsetslot

\setslot{\uctop{Gbreve}{gbreve}}
   \comment{The letter `\u G'.}
\endsetslot

\setslot{\uctop{Lacute}{lacute}}
   \comment{The letter `\' L'.}
\endsetslot

\setslot{\uc{Lcaron}{lcaron}}
   \comment{The letter `\v L'.}
\endsetslot

\setslot{\uc{Lslash}{lslash}}
   \comment{The letter `\L'.}
\endsetslot

\setslot{\uctop{Nacute}{nacute}}
   \comment{The letter `\' N'.}
\endsetslot

\setslot{\uctop{Ncaron}{ncaron}}
   \comment{The letter `\v N'.}
\endsetslot

\setslot{\uc{Ng}{ng}}
   \comment{The Sami letter `\NG'.  It is unavailable in \plain\ \TeX.
     \actually{letter `S'}}
\endsetslot

\setslot{\uctop{Ohungarumlaut}{ohungarumlaut}}
   \comment{The letter `\H O'.}
\endsetslot

\setslot{\uctop{Racute}{racute}}
   \comment{The letter `\' R'.}
\endsetslot

\setslot{\uctop{Rcaron}{rcaron}}
   \comment{The letter `\v R'.}
\endsetslot

\setslot{\uctop{Sacute}{sacute}}
   \comment{The letter `\' S'.}
\endsetslot

\setslot{\uctop{Scaron}{scaron}}
   \comment{The letter `\v S'.}
\endsetslot

\setslot{\uc{Scedilla}{scedilla}}
   \comment{The letter `\c S'.}
\endsetslot

\setslot{\uctop{Tcaron}{tcaron}}
   \comment{The letter `\v T'.}
\endsetslot

\setslot{\uc{Tcedilla}{tcedilla}}
   \comment{The letter `\c T'.}
\endsetslot

\setslot{\uctop{Uhungarumlaut}{uhungarumlaut}}
   \comment{The letter `\H U'.}
\endsetslot

\setslot{\uctop{Uring}{uring}}
   \comment{The letter `\r U'.}
\endsetslot

\setslot{\uctop{Ydieresis}{ydieresis}}
   \comment{The letter `\" Y'.}
\endsetslot

\setslot{\uctop{Zacute}{zacute}}
   \comment{The letter `\' Z'.}
\endsetslot

\setslot{\uctop{Zcaron}{zcaron}}
   \comment{The letter `\v Z'.}
\endsetslot

\setslot{\uctop{Zdotaccent}{zdotaccent}}
   \comment{The letter `\. Z'.}
\endsetslot

\setslot{\uclig{IJ}{ij}}
   \comment{The letter `IJ'.  This is a single letter, and in a monowidth
      font should ideally be one letter wide.}
\endsetslot

\setslot{\uctop{Idotaccent}{idotaccent}}
   \comment{The letter `\. I'.}
\endsetslot

\setslot{\lc{Dbar}{dbar}}
   \comment{The letter `\dj'.}
\endsetslot

\setslot{section}
   \comment{The section mark `\textsection'.}
\endsetslot

\setslot{\lctop{Abreve}{abreve}}
   \comment{The letter `\u a'.}
\endsetslot

\setslot{\lc{Aogonek}{aogonek}}
   \comment{The letter `\k a'.}
\endsetslot

\setslot{\lctop{Cacute}{cacute}}
   \comment{The letter `\' c'.}
\endsetslot

\setslot{\lctop{Ccaron}{ccaron}}
   \comment{The letter `\v c'.}
\endsetslot

\setslot{\lctop{Dcaron}{dcaron}}
   \comment{The letter `\v d'.}
\endsetslot

\setslot{\lctop{Ecaron}{ecaron}}
   \comment{The letter `\v e'.}
\endsetslot

\setslot{\lc{Eogonek}{eogonek}}
   \comment{The letter `\k e'.}
\endsetslot

\setslot{\lctop{Gbreve}{gbreve}}
   \comment{The letter `\u g'.}
\endsetslot

\setslot{\lctop{Lacute}{lacute}}
   \comment{The letter `\' l'.}
\endsetslot

\setslot{\lc{Lcaron}{lcaron}}
   \comment{The letter `\v l'.}
\endsetslot

\setslot{\lc{Lslash}{lslash}}
   \comment{The letter `\l'.}
\endsetslot

\setslot{\lctop{Nacute}{nacute}}
   \comment{The letter `\' n'.}
\endsetslot

\setslot{\lctop{Ncaron}{ncaron}}
   \comment{The letter `\v n'.}
\endsetslot

\setslot{\lc{Ng}{ng}}
%    \ligature{LIG/>}{hyphen}{s}
%    \ligature{LIG/>}{hyphenchar}{s}
   \comment{The Sami letter `\ng'.  It is unavailable in \plain\ \TeX.
     \actually{letter round `s'}}
\endsetslot

\setslot{\lctop{Ohungarumlaut}{ohungarumlaut}}
   \comment{The letter `\H o'.}
\endsetslot

\setslot{\lctop{Racute}{racute}}
   \comment{The letter `\' r'.}
\endsetslot

\setslot{\lctop{Rcaron}{rcaron}}
   \comment{The letter `\v r'.}
\endsetslot

\setslot{\lctop{Sacute}{sacute}}
   \comment{The letter `\' s'.}
\endsetslot

\setslot{\lctop{Scaron}{scaron}}
   \comment{The letter `\v s'.}
\endsetslot

\setslot{\lc{Scedilla}{scedilla}}
   \comment{The letter `\c s'.}
\endsetslot

\setslot{\lctop{Tcaron}{tcaron}}
   \comment{The letter `\v t'.}
\endsetslot

\setslot{\lc{Tcedilla}{tcedilla}}
   \comment{The letter `\c t'.}
\endsetslot

\setslot{\lctop{Uhungarumlaut}{uhungarumlaut}}
   \comment{The letter `\H u'.}
\endsetslot

\setslot{\lctop{Uring}{uring}}
   \comment{The letter `\r u'.}
\endsetslot

\setslot{\lctop{Ydieresis}{ydieresis}}
   \comment{The letter `\" y'.}
\endsetslot

\setslot{\lctop{Zacute}{zacute}}
   \comment{The letter `\' z'.}
\endsetslot

\setslot{\lctop{Zcaron}{zcaron}}
   \comment{The letter `\v z'.}
\endsetslot

\setslot{\lctop{Zdotaccent}{zdotaccent}}
   \comment{The letter `\. z'.}
\endsetslot

\setslot{\lclig{IJ}{ij}}
   \comment{The letter `ij'.  This is a single letter, and in a monowidth
      font should ideally be one letter wide.}
\endsetslot

\setslot{exclamdown}
   \comment{The Spanish punctuation mark `!`'.}
\endsetslot

\setslot{questiondown}
   \comment{The Spanish punctuation mark `?`'.}
\endsetslot

\setslot{sterling}
   \comment{The British currency mark `\textsterling'.}
\endsetslot

\setslot{\uctop{Agrave}{agrave}}
   \comment{The letter `\` A'.}
\endsetslot

\setslot{\uctop{Aacute}{aacute}}
   \comment{The letter `\' A'.}
\endsetslot

\setslot{\uctop{Acircumflex}{acircumflex}}
   \comment{The letter `\^ A'.}
\endsetslot

\setslot{\uctop{Atilde}{atilde}}
   \comment{The letter `\~ A'.}
\endsetslot

\setslot{\uctop{Adieresis}{adieresis}}
   \comment{The letter `\" A'.}
\endsetslot

\setslot{\uctop{Aring}{aring}}
   \comment{The letter `\r A'.}
\endsetslot

\setslot{\uc{AE}{ae}}
   \comment{The letter `\AE'.  This is a single letter, and should not be
      faked with `AE'.}
\endsetslot

\setslot{\uc{Ccedilla}{ccedilla}}
   \comment{The letter `\c C'.}
\endsetslot

\setslot{\uctop{Egrave}{egrave}}
   \comment{The letter `\` E'.}
\endsetslot

\setslot{\uctop{Eacute}{eacute}}
   \comment{The letter `\' E'.}
\endsetslot

\setslot{\uctop{Ecircumflex}{ecircumflex}}
   \comment{The letter `\^ E'.}
\endsetslot

\setslot{\uctop{Edieresis}{edieresis}}
   \comment{The letter `\" E'.}
\endsetslot

\setslot{\uctop{Igrave}{igrave}}
   \comment{The letter `\` I'.}
\endsetslot

\setslot{\uctop{Iacute}{iacute}}
   \comment{The letter `\' I'.}
\endsetslot

\setslot{\uctop{Icircumflex}{icircumflex}}
   \comment{The letter `\^ I'.}
\endsetslot

\setslot{\uctop{Idieresis}{idieresis}}
   \comment{The letter `\" I'.}
\endsetslot

\setslot{\uc{Eth}{eth}}
   \comment{The uppercase Icelandic letter `Eth' similar to a `D'
      with a horizontal bar through the stem.  It is unavailable
      in \plain\ \TeX.}
\endsetslot

\setslot{\uctop{Ntilde}{ntilde}}
   \comment{The letter `\~ N'.}
\endsetslot

\setslot{\uctop{Ograve}{ograve}}
   \comment{The letter `\` O'.}
\endsetslot

\setslot{\uctop{Oacute}{oacute}}
   \comment{The letter `\' O'.}
\endsetslot

\setslot{\uctop{Ocircumflex}{ocircumflex}}
   \comment{The letter `\^ O'.}
\endsetslot

\setslot{\uctop{Otilde}{otilde}}
   \comment{The letter `\~ O'.}
\endsetslot

\setslot{\uctop{Odieresis}{odieresis}}
   \comment{The letter `\" O'.}
\endsetslot

\setslot{\uc{OE}{oe}}
   \comment{The letter `\OE'.  This is a single letter, and should not be
      faked with `OE'.}
\endsetslot

\setslot{\uc{Oslash}{oslash}}
   \comment{The letter `\O'.}
\endsetslot

\setslot{\uctop{Ugrave}{ugrave}}
   \comment{The letter `\` U'.}
\endsetslot

\setslot{\uctop{Uacute}{uacute}}
   \comment{The letter `\' U'.}
\endsetslot

\setslot{\uctop{Ucircumflex}{ucircumflex}}
   \comment{The letter `\^ U'.}
\endsetslot

\setslot{\uctop{Udieresis}{udieresis}}
   \comment{The letter `\" U'.}
\endsetslot

\setslot{\uctop{Yacute}{yacute}}
   \comment{The letter `\' Y'.}
\endsetslot

\setslot{\uc{Thorn}{thorn}}
   \comment{The Icelandic capital letter Thorn, similar to a `P'
      with the bowl moved down.  It is unavailable in \plain\ \TeX.}
\endsetslot

\setslot{\uclig{SS}{germandbls}}
   \comment{The ligature `SS', used to give an upper case `\ss'.
      In a monowidth font it should be two letters wide.}
\endsetslot

\setslot{\lctop{Agrave}{agrave}}
   \comment{The letter `\` a'.}
\endsetslot

\setslot{\lctop{Aacute}{aacute}}
   \comment{The letter `\' a'.}
\endsetslot

\setslot{\lctop{Acircumflex}{acircumflex}}
   \comment{The letter `\^ a'.}
\endsetslot

\setslot{\lctop{Atilde}{atilde}}
   \comment{The letter `\~ a'.}
\endsetslot

\setslot{\lctop{Adieresis}{adieresis}}
   \comment{The letter `\" a'.}
\endsetslot

\setslot{\lctop{Aring}{aring}}
   \comment{The letter `\r a'.}
\endsetslot

\setslot{\lc{AE}{ae}}
   \comment{The letter `\ae'.  This is a single letter, and should not be
      faked with `ae'.}
\endsetslot

\setslot{\lc{Ccedilla}{ccedilla}}
   \comment{The letter `\c c'.}
\endsetslot

\setslot{\lctop{Egrave}{egrave}}
   \comment{The letter `\` e'.}
\endsetslot

\setslot{\lctop{Eacute}{eacute}}
   \comment{The letter `\' e'.}
\endsetslot

\setslot{\lctop{Ecircumflex}{ecircumflex}}
   \comment{The letter `\^ e'.}
\endsetslot

\setslot{\lctop{Edieresis}{edieresis}}
   \comment{The letter `\" e'.}
\endsetslot

\setslot{\lctop{Igrave}{igrave}}
   \comment{The letter `\`\i'.}
\endsetslot

\setslot{\lctop{Iacute}{iacute}}
   \comment{The letter `\'\i'.}
\endsetslot

\setslot{\lctop{Icircumflex}{icircumflex}}
   \comment{The letter `\^\i'.}
\endsetslot

\setslot{\lctop{Idieresis}{idieresis}}
   \comment{The letter `\"\i'.}
\endsetslot

\setslot{\lc{Eth}{eth}}
   \comment{The Icelandic lowercase letter `eth' similar to
     a `$\partial$' with an oblique bar through the stem.
     It is unavailable in \plain\ \TeX.}
\endsetslot

\setslot{\lctop{Ntilde}{ntilde}}
   \comment{The letter `\~ n'.}
\endsetslot

\setslot{\lctop{Ograve}{ograve}}
   \comment{The letter `\` o'.}
\endsetslot

\setslot{\lctop{Oacute}{oacute}}
   \comment{The letter `\' o'.}
\endsetslot

\setslot{\lctop{Ocircumflex}{ocircumflex}}
   \comment{The letter `\^ o'.}
\endsetslot

\setslot{\lctop{Otilde}{otilde}}
   \comment{The letter `\~ o'.}
\endsetslot

\setslot{\lctop{Odieresis}{odieresis}}
   \comment{The letter `\" o'.}
\endsetslot

\setslot{\lc{OE}{oe}}
   \comment{The letter `\oe'.  This is a single letter, and should not be
      faked with `oe'.}
\endsetslot

\setslot{\lc{Oslash}{oslash}}
   \comment{The letter `\o'.}
\endsetslot

\setslot{\lctop{Ugrave}{ugrave}}
   \comment{The letter `\` u'.}
\endsetslot

\setslot{\lctop{Uacute}{uacute}}
   \comment{The letter `\' u'.}
\endsetslot

\setslot{\lctop{Ucircumflex}{ucircumflex}}
   \comment{The letter `\^ u'.}
\endsetslot

\setslot{\lctop{Udieresis}{udieresis}}
   \comment{The letter `\" u'.}
\endsetslot

\setslot{\lctop{Yacute}{yacute}}
   \comment{The letter `\' y'.}
\endsetslot

\setslot{\lc{Thorn}{thorn}}
   \comment{The Icelandic lowercase letter `thorn', similar to a `p'
      with an ascender rising from the stem.  It is unavailable
      in \plain\ \TeX.}
\endsetslot

\setslot{\lc{SS}{germandbls}}
   \comment{The letter `\ss'.}
\endsetslot

\endencoding

\end{document}
%</t1fraketx>
%    \end{macrocode}
% \Finale

\endinput
%%% Local Variables: 
%%% mode: latex
%%% TeX-master: t
%%% End: 
