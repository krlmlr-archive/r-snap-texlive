% \iffalse meta-comment
%<*internal>
\begingroup
\input docstrip.tex
\keepsilent
\askforoverwritefalse

\preamble
  ______________________________________________________
  The LXfonts package 
  Copyright (C) 2008-2013 Claudio Beccari 
  All rights reserved
  
  Distributable under the LaTeX Project Public License,
  version 1.3c or higher (your choice). The latest version of
  this license is at: http://www.latex-project.org/lppl.txt
\endpreamble
\postamble
Read the README test file for further details about installation
\endpostamble


\generate{\file{lxfonts.sty}{\from{lxfonts.dtx}{lxsty}}}
\generate{\file{ot1llcmss.fd}{\from{lxfonts.dtx}{lxot1ssfd}}}
\generate{\file{ot1llcmtt.fd}{\from{lxfonts.dtx}{lxot1ttfd}}}
\generate{\file{t1llcmss.fd}{\from{lxfonts.dtx}{lxt1ssfd}}}
\generate{\file{t1llcmtt.fd}{\from{lxfonts.dtx}{lxt1ttfd}}}
\generate{\file{ts1llcmss.fd}{\from{lxfonts.dtx}{lxts1ssfd}}}
\generate{\file{omlllcmm.fd}{\from{lxfonts.dtx}{lxomlmmfd}}}
\generate{\file{omsllcmsy.fd}{\from{lxfonts.dtx}{lxomssyfd}}}
\generate{\file{omxllcmex.fd}{\from{lxfonts.dtx}{lxomxexfd}}}
\generate{\file{ulmsa.fd}{\from{lxfonts.dtx}{lxumsafd}}}
\generate{\file{ulmsb.fd}{\from{lxfonts.dtx}{lxumsbfd}}}
\generate{\file{ulllasy.fd}{\from{lxfonts.dtx}{lxultxfd}}}
\generate{\file{lxfonts.map}{\from{lxfonts.dtx}{lxmap}}}
\generate{\file{lgrllcmss.fd}{\from{lxfonts.dtx}{lxlgrfd}}}
\generate{\file{lgrllcmtt.fd}{\from{lxfonts.dtx}{lxlgrttfd}}}

\def\tmpa{plain}
\ifx\tmpa\fmtname\endgroup\expandafter\bye\fi
\endgroup
%</internal>
% \fi
%
% \iffalse
%<*driver>
\documentclass{ltxdoc}
\ProvidesFile{lxfonts.dtx}[2013/12/07 v.2.0b Documented TeX file for
the LXfonts bundle]
\GetFileInfo{lxfonts.dtx}
\title{The LXfonts bundle}
\date{\fileversion\space--- \filedate}
 \author{Claudio Beccari\thanks{\texttt{claudio dot beccari at gmail dot com}}}
 
\usepackage{mflogo}
\usepackage{textcomp}
\usepackage[LGR,OT1]{fontenc}

\def\prog#1{\textsf{#1}}
\def\pack#1{\textsf{\slshape#1}}
\DeclareRobustCommand\AMS{\ensuremath{\mathcal{A\!_{\textstyle M}\mkern-2mu S}}}

\begin{document}\errorcontextlines=9
\maketitle
 \setlength\hfuzz{20pt}
 \DocInput{lxfonts.dtx}
\end{document}
%</driver>
% \fi
%
% \CheckSum{0}
%
% This self extracting documented file |lxfonts.dtx|, besides the |lxfonts.sty|
% file,  produces the font description files necessary to use the LX fonts
% in any document, specifically presentations, but not only. It documents
% why they were made and the choices made in selecting which fonts to convert
% to this style.
%
% These fonts should be useful for typesetting documents with T1 encoding (OT1
% is also supported) and blends text and math fonts in the proper way; AMS fonts
% are  supported. For legacy reasons the \LaTeX\ special symbol set is also
% accordingly restyled. 
%
% \section{The LX fonts}
% In the first eighties, when \LaTeX\ was conceived, Leslie Lamport  created
% a stand alone program, or better, a mark-up \SliTeX\ format that
% could be used in place of \LaTeX\ when it was necessary to create transparencies
% for conferences, lectures, and presentations; when \LaTeXe\ was made available
% in the early nineties, that set of mark-up macros and format did not exist
% any more, and in its place there was the new class \pack{slides}.
%
% The processing of the source files was almost identical, although the new
% class \pack{slides} had more typesetting power thanks to the \LaTeXe\ more
% powerful macros.
% 
% Both processes produced printed documents that could be photocopied onto
% methacrylate transparencies and used with overhead projectors.
%
% Since the advent of beamer projectors, that can operate directly from a
% computer, and the availability of the \prog{pdfLaTeX} typesetting program,
% a set of packages was made available to the \TeX\ community so as to
% produce wonderful presentations, possibly together with their 
% printed handouts, that quickly replaced the old and honorable \SliTeX\ 
% format and the \pack{slides} class.
% 
% The fonts that Leslie Lamport had devised to use with \SliTeX\ and
% \pack{slides} had a great advantage over the default \TeX\ system fonts,
% namely the traditional Computer Modern ones, and, after the introduction
% of the  new encoding T1, of the European Computer fonts and, later on,
% of the Latin Modern vector fonts.
%
% It was \emph{legibility}.
%
% This advantage was obtained from the sans serif quotation fonts designed
% by D.E.~Knuth himself; he used them to typeset the witty quotations at each 
% chapter end of both his \TeX\ and \MF\ books.
%
% But these Knuthian fonts had a disadvantage: the capital `I' and the lower
% case `l' were hardly distinguishable from one another and from the absolute
% value vertical bar. Leslie Lamport solved in part this problem by changing
% the sans serif `I' with a serifed one; this was fine most of the time, but
% did not solve the similarity of the sans serif lower case `l' with the math
% absolute value bar.
%
% Moreover when typesetting math in those old fashioned slides, only the
% |operators| font was changed to a sans serif one, while the |letters| math
% italics font, together with the other |symbols| normal and |largesymbols| large
% symbol fonts remained the same; the result was that some math glyphs
% obtained by juxtaposition of symbols coming from the operator font and
% some other math font, resulted clearly inadequate.
%
% The solution of such problems consisted in redesigning the shape of the lower
% case `l', so as to be more similar to an upright italic one, with a curved
% bottom, and to restyle all the math fonts with the same graphic font settings
% of the Lamport fonts, both in OT1 and T1 encoding, and in the math encodings
% OML, OMS OMX, plus the Text Companion TS1 encoded fonts, the \AMS\ fonts and
% the \LaTeX\ symbol fonts.
% 
% This is all what concerns these extended slide LX fonts; while doing this
% extensions I had to change the metadescription of several glyphs, but the
% overall work was not that complicated; it was just heavy because the large
% number of fonts involved, and therefore the overall glyph number.
% 
% This package redefines both the text and math font families; checks if
% the \AMS\ fonts have been called for, avoids to redefine the nice glyphs of
% the \AMS\ collection with the \LaTeX\ symbol fonts, although it will
% redefine the latter ones in case the user wants to use them, but did not
% load the \AMS\ font collections.
%
% These settings are deferred at begin document time, so that the necessary
% checks may be used after other fonts are possibly loaded.
%
% As a bonus the typewriter fonts are loaded in a scaled up version so that
% their x-height matches the larger LX fonts. Such fonts are scaled 25\%
% up so that the 8pt design size typewriter font x-height matches
% the corresponding height of the LX fonts.
%
% These LX fonts are available in both vector and bitmapped form; I made the
% whole work by working on the \MF\ source files; when I was satisfied with
% the bitmapped fonts produced with \MF, I traced them by means of the
% \prog{mftrace} program, by Han-Wen Nien­huys, and produced the .pfb files
% containing the vector description of the glyphs. In order to use them
% with \prog{pdfLaTeX} it is necessary to have available the .map file,
% that I provided as part of this package. As a consequence of the method
% I followed, this bundle contains also the .tfm metric information
%
% \section{Usage}
% The only action needed to use these LX fonts is to call 
%\begin{verbatim}
%\usepackage{lxfonts}
%\end{verbatim}
% in the preamble; there are no options to set.
%
% If you use \pack{beamer} to create your presentation you might need to specify
%\begin{verbatim}
%\usefonttheme{professionalfonts}
%\end{verbatim}
% in order to avoid that beamer resets some math fonts the way it defaults to;
% for the rest you don't need to follow a particular loading order, although it
% might be clearer if you loaded your fonts after specifying the input and 
% font encoding(s).
%
% \section{Standalone usage of the LX fonts}\label{sec:standalone}
%
% The style file \texttt{lxfonts.sty} changes all the default font settings
% so that you cannot use any other font together with the LX ones.
%
% You can circumvent this rigidity by using the explicit declaration of the
% font family you want to use; for example, if you called this package and at a
% certain point you want to typeset something with a serifed font, say, the
% T1 encoded Latin Modern roman font, you can define a macro declaration or
% explicitly specify:
%\begin{verbatim}
%{\usefont{T1}{lmr}{m}{n} Words\ldots words}
%\end{verbatim}
% By so doing the ``Words\ldots words'' will be typeset in roman medium Latin
% Modern T1 encoded font at the current size
%
% On the opposite if you want to show a sample text written with the LX fonts while
% you are typesetting with other fonts, you don't use the above .sty file package,
% but use a macro declaration or the explicit command:
%\begin{verbatim}
%{\usefont{T1}{llcmss}{m}{n} Words\ldots words}
%\end{verbatim}
% and you get {\usefont{T1}{llcmss}{m}{n} Words\dots words} at the current type
% size; notice though, how larger the script appears thanks to the larger x-height,
% but uppercase letters are the same size as the serifed ones.
%
% With version 1.0 of this package, this functionality was not possible.
%
% \section{Integration of the  Latin and  the Greek scripts}  
% The Greek CB fonts contain also the  families and shapes for slides;
% the style is the same as these LX fonts, but the different script does
% not require any restyling of any glyph; so they can be used directly;
% but the script changing mechanism simply implies the change of the
% encoding; therefore in order to  use the above Greek CB fonts, it is
% simply necessary to have available new font description files whose
% name is the agglutination of the Greek encoding name  (in lower case)
% with the family name (identical to that of the Latin script); the contents
% of such files, of course, retrieves the glyphs from the Greek fonts files.
%
% Therefore it should not be difficult in a presentation to specify the Greek
% language in the preamble of the source file, and use the \pack{babel} language
% switching commands to as to write something like this:
% {\usefont{LGR}{llcmss}{m}{n} Kal'' hm'era!} means \emph{Good day!}.                                                                                                                                                                                                                                                                                                                    
%
% \section{ The LX fonts demo presentation}
% This bundle, besides this documentation, is shipped with a demo presentation
% where most features of the LX fonts are shown; the slides clearly show the
% features of these fonts, both the positive  and negative ones.  The
% \emph{legibility}, in my opinion, is excellent; if it was not for this
% point I would not have undertaken the work of restyling all these fonts.
% But there ale also some little glitches that are partly inherent to the
% chosen one-size continuously scalable font. Moreover the Greek fonts,
% which are not part in themselves of my LX project, requires some
% adjustments in several kerning values, but at the moment it is acceptable,
% even if its kerning is not perfect. 
% 
% While typesetting math it is clearly noticeable the lack of optical sizes:
% the super and subscripts appear definitely of the right size, but their
% scaling makes them appear lighter than they should be; this is common with
% all fonts that come in one size only, and this happens with the majority
% of the Type~1 fonts distributed with the \TeX\ system.
%
% The Greek font kerning  adjustment is on my TODO list.
%
% \section{Acknowledgemts}
% I would like to thank all the users who appreciated these fonts and
% gave me some feedback remarking some glitches; in particular I would
% like to acknowledge the interaction with G\"unter Milde, who gave me
% precious suggestions, among which that of detaching the font description
% files form the |lxfonts.sty| file.
%
% \StopEventually{}^^A This macro argument can be a bibliography, for example.
%
% \section{Documented code}
% \subsection{The package code}
% The settings relative to the LX fonts are deferred to the |\AtEndPreamble|
% hook (defined by package |etoolbox|) so that any previous font setting is
% replaced by the ones relative to the LX fonts; this is particularly useful
% when preparing source files for a presentation; in this way all the fonts
% connected to the the slide show will be homogeneously styled the same way.

% Of course this is not a serious drawback in other situations, since I have
% shown in section~\ref{sec:standalone} what to do in order to overcome this
% apparent limitation.
% 
% First I test if certain packages have been loaded, specifically packages
% \pack{latexsym} and \pack{amsfonts}; if so, certain switches are set |true|.
% At the proper point these switches will be used in order to load or to avoid
% to load certain font description files that will supersede the ones called
% by such named packages.
%
% At the same time the text and math defaults are set or reset so as to use only
% the LX fonts.
% \iffalse
%<*lxsty>
% \fi
%    \begin{macrocode}
\NeedsTeXFormat{LaTeX2e}[2001/06/01]
\ProvidesPackage{lxfonts}[2013/09/03 v,2.0 Macros for using LX fonts]
\newif\if@lasy \@lasyfalse
\newif\if@AMSfonts \@AMSfontsfalse
\RequirePackage{etoolbox}
\AtEndPreamble{%
	\@ifpackageloaded{latexsym}{\@lasytrue}{}
	\@ifpackageloaded{amsfonts}{\@AMSfontstrue}{}
	\def\rmdefault{llcmss}        % no roman
	\def\sfdefault{llcmss}
	\def\ttdefault{llcmtt}
	\def\itdefault{sl}
	\def\sldefault{sl}
	\def\bfdefault{bx}
	\SetSymbolFont{operators}{normal}{OT1}{llcmss}{m}{n}
	\SetSymbolFont{letters}{normal}{OML}{llcmm}{m}{it}
	\SetSymbolFont{symbols}{normal}{OMS}{llcmsy}{m}{n}
	\SetSymbolFont{largesymbols}{normal}{OMX}{llcmex}{m}{n}
	
	\SetSymbolFont{operators}{bold}{OT1}{llcmss} {bx}{n}
	\SetSymbolFont{letters}  {bold}{OML}{llcmm} {bx}{it}
	\SetSymbolFont{symbols}  {bold}{OMS}{llcmsy}{bx}{n}
	\SetSymbolFont{largesymbols}{bold}{OMX}{llcmex}{m}{n} % no bold!
	
	\DeclareSymbolFontAlphabet{\mathrm}    {operators}
	\DeclareSymbolFontAlphabet{\mathnormal}{letters}
	\DeclareSymbolFontAlphabet{\mathcal}   {symbols}
	
	\DeclareMathAlphabet      {\mathbf}{OT1}{llcmss}{bx}{n}
	\DeclareMathAlphabet      {\mathsf}{OT1}{llcmss}{m}{n}
	\DeclareMathAlphabet      {\mathit}{OT1}{llcmss}{m}{sl}
	\DeclareMathAlphabet      {\mathtt}{OT1}{llcmtt}{m}{n}
	
	\SetMathAlphabet\mathsf{bold}{OT1}{llcmss}{bx}{n}
	\SetMathAlphabet\mathit{bold}{OT1}{llcmss}{bx}{sl}
%    \end{macrocode}
% The following tests are needed to load the \AMS\ and the \LaTeX\ symbols
% fonts; in order to use them it is necessary to input the relevant font
% description files subject to the switches status; and then it is necessary
% to redeclare the math fonts accordingly.
%
% The |\Join| and |\leadsto| characters had to be redefined or declared
% to be aliases of some particular symbol command; the |\Join| macro had
% to be redefined because the dimensions of the new font don't agree with
% the kerning fixed in the |amssymb| package; of course, while I was at it,
% I might have designed a real glyph for |\Join|, but I thought that it
% was better to avoid conflicts with the existing \AMS\ symbol definitions.
% The |\leadsto| character has been let to |\rightsquigarrow| in order
% to name the same glyph also with the \LaTeX\ symbols font command.
% In this way it is possible to avoid loading the \LaTeX\ symbols font
% if the \AMS\ fonts have been already loaded.
%    \begin{macrocode}
\if@AMSfonts
	%%
%% This is file `ulmsa.fd',
%% generated with the docstrip utility.
%%
%% The original source files were:
%%
%% lxfonts.dtx  (with options: `lxumsafd')
%%   ______________________________________________________
%%   The LXfonts package
%%   Copyright (C) 2008-2013 Claudio Beccari
%%   All rights reserved
%% 
%%   Distributable under the LaTeX Project Public License,
%%   version 1.3c or higher (your choice). The latest version of
%%   this license is at: http://www.latex-project.org/lppl.txt


\DeclareFontFamily{U}{lmsa}{}
\DeclareFontShape{U}{lmsa}{m}{n}{<-> lmsam8}{}
\DeclareFontShape{U}{lmsa}{bx}{n}{<-> ssub* lmsa/m/n}{}
%% Read the README test file for further details about installation
%%
%% End of file `ulmsa.fd'.

	%%
%% This is file `ulmsb.fd',
%% generated with the docstrip utility.
%%
%% The original source files were:
%%
%% lxfonts.dtx  (with options: `lxumsbfd')
%%   ______________________________________________________
%%   The LXfonts package
%%   Copyright (C) 2008-2013 Claudio Beccari
%%   All rights reserved
%% 
%%   Distributable under the LaTeX Project Public License,
%%   version 1.3c or higher (your choice). The latest version of
%%   this license is at: http://www.latex-project.org/lppl.txt


\DeclareFontFamily{U}{lmsb}{}
\DeclareFontShape{U}{lmsb}{m}{n}{<-> lmsbm8}{}
\DeclareFontShape{U}{lmsb}{bx}{n}{<-> ssub* lmsb/m/n}{}
%% Read the README test file for further details about installation
%%
%% End of file `ulmsb.fd'.

	\SetSymbolFont{AMSa}{normal}{U}{lmsa}{m}{n}
	\SetSymbolFont{AMSb}{normal}{U}{lmsb}{m}{n}
	\xdef\Join{\mathrel{\mathchar"0\hexnumber@\symAMSb 6F%
	\mkern-14.2mu\mathchar"0\hexnumber@\symAMSb 6E}}
	\global\let\leadsto\rightsquigarrow
\fi
\if@lasy
	\if@AMSfonts%
		\PackageWarning{lxfonts}{%
		I did not load the LaTeX symbol fonts\MessageBreak
		because its glyphs are already provided by the AMS fonts\MessageBreak}
	\else	
		%%
%% This is file `ulllasy.fd',
%% generated with the docstrip utility.
%%
%% The original source files were:
%%
%% lxfonts.dtx  (with options: `lxultxfd')
%%   ______________________________________________________
%%   The LXfonts package
%%   Copyright (C) 2008-2013 Claudio Beccari
%%   All rights reserved
%% 
%%   Distributable under the LaTeX Project Public License,
%%   version 1.3c or higher (your choice). The latest version of
%%   this license is at: http://www.latex-project.org/lppl.txt


\DeclareFontFamily{U}{lllasy}{}
\DeclareFontShape{U}{lllasy}{m}{n}{<-> llasy8}{}
\DeclareFontShape{U}{lllasy}{b}{n}{<-> llasyb8}{}
%% Read the README test file for further details about installation
%%
%% End of file `ulllasy.fd'.

		\SetSymbolFont{lasy}{normal}{U}{lllasy}{m}{n}
		\SetSymbolFont{lasy}{bold}{U}{lllasy}{b}{n}
	\fi
\fi
%    \end{macrocode}
%
% The following commands are defined so as to implement part of the
% functionality of the old \LaTeX\,209 ones, with some significant modification:
% their effects are cumulative as they are in \LaTeXe; at the same time they
% can be used also in mathematics. These commands pamper those users who think
% that the \LaTeXe\ commands are too lengthy to write, but such users forgot
% that the old commands did not cumulate their effects, therefore they are
% not as flexible as the \LaTeXe\ font commands. On the opposite these
% definitions produce the same cumulative effects as the \LaTeXe\ commands;
% I do not think the habit of using the old commands is correct, but it does
% not harm anybody who has correct habits: it's not compulsory to use them
% even if they are available.
%    \begin{macrocode}
\DeclareOldFontCommand{\rm}{\rmfamily}{\mathrm}
\DeclareOldFontCommand{\sf}{\sffamily}{\mathsf}
\DeclareOldFontCommand{\tt}{\ttfamily}{\mathtt}
\DeclareOldFontCommand{\bf}{\bfseries}{\mathbf}
\DeclareOldFontCommand{\it}{\itshape}{\mathit}
\DeclareOldFontCommand{\sl}{\slshape}{\relax}
\DeclareOldFontCommand{\sc}{\scshape}{\relax}
}
%    \end{macrocode}
% \iffalse
%</lxsty>
% \fi
%
% \subsection{Fonts in OT1 encoding}
% As long as the maths fonts are arranged the way they are, it is necessary
% to have the OT1 encoded text fonts even if the user specifies the T1
% option to the \pack{fontenc} package; this is due to the fact that the
% |operators| font in math mode is always defined as the the OT1 encoded set.

% At the same time those who don't need to use the many diacritics taken
% care of by the T1 encoded fonts, can simply avoid to call the \pack{fontenc}
% package, and get along without any problem.
% \iffalse
%<*lxot1ssfd>
% \fi
%    \begin{macrocode}
\DeclareFontFamily{OT1}{llcmss}{\hyphenchar\font45}
\DeclareFontShape{OT1}{llcmss}{m}{n}{<-> llcmss8}{}
\DeclareFontShape{OT1}{llcmss}{m}{sl}{<-> llcmssi8}{}
\DeclareFontShape{OT1}{llcmss}{m}{it}{<->ssub* llcmss/m/sl}{}
\DeclareFontShape{OT1}{llcmss}{bx}{n}{<-> llcmssb8}{}
\DeclareFontShape{OT1}{llcmss}{bx}{sl}{<-> llcmsso8}{}
\DeclareFontShape{OT1}{llcmss}{bx}{it}{<->ssub* llcmss/bx/sl}{}
%
\DeclareFontShape{OT1}{llcmss}{m}{ui}{<-> ssub* llcmss/m/n}{}
\DeclareFontShape{OT1}{llcmss}{bx}{ui}{<->ssub* llcmss/bx/n}{}
%    \end{macrocode}
% \iffalse
%</lxot1ssfd>
% \fi
%
% The typewriter fonts are loaded as the usual OT1 encoded Computer Modern
% ones just scaled up by 25\% so as to have them with the same x-height as
% the LX fonts. Substitutions are provided for the missing shapes and series.
% \iffalse
%<*lxot1ttfd>
% \fi
%    \begin{macrocode}
\DeclareFontFamily{OT1}{llcmtt}{\hyphenchar\font\m@ne}
\DeclareFontShape{OT1}{llcmtt}{m}{n}{<-> [1.25]cmtt8}{}
\DeclareFontShape{OT1}{llcmtt}{m}{it}{<-> [1.25]cmti8}{}
\DeclareFontShape{OT1}{llcmtt}{m}{sl}{<-> ssub* llcmtt/m/it}{}
\DeclareFontShape{OT1}{llcmtt}{bx}{n}{<-> ssub* llcmtt/m/n}{}
\DeclareFontShape{OT1}{llcmtt}{bx}{it}{<-> ssub* llcmtt/m/it}{}
\DeclareFontShape{OT1}{llcmtt}{bx}{sl}{<-> ssub* llcmtt/m/it}{}
%    \end{macrocode}
% \iffalse
%</lxot1ttfd>
% \fi
% 
% \subsection{Fonts in T1 encoding}
% The T1 encoded fonts are now described by the suitable font description
% files; such fonts are essential for typesetting most languages that use
% the Latin script. Actually I know that there exist some languages that
% usually don't use diacritics; but even in such languages sometimes it is
% necessary to typeset a foreign name or to typeset some phrases or paragraphs
% in an ancient version of the same language and diacritics pop up again.
% Personally I believe that T1 encoded fonts should be the only ones to be
% used, unless the user has access to suitable UNICODE encoded OpenType
% fonts, but this is out of topic since these LX fonts can be used only
% by \prog{pdfLaTeX}, that cannot directly handle OpenType fonts.
% \iffalse
%<*lxt1ssfd>
% \fi
%    \begin{macrocode}
 \DeclareFontFamily{T1}{llcmss}{\hyphenchar\font45}
 \DeclareFontShape{T1}{llcmss}{m}{n}{<-> leclq8}{}
 \DeclareFontShape{T1}{llcmss}{m}{sl}{<-> lecli8}{}
 \DeclareFontShape{T1}{llcmss}{m}{it}{<->ssub* llcmss/m/sl}{}
 \DeclareFontShape{T1}{llcmss}{bx}{n}{<-> leclb8}{}
 \DeclareFontShape{T1}{llcmss}{bx}{sl}{<-> leclo8}{}
 \DeclareFontShape{T1}{llcmss}{bx}{it}{<->ssub*llcmss/bx/sl}{}
%
\DeclareFontShape{T1}{llcmss}{m}{ui}{<->ssub*llcmss/m/n}{}
\DeclareFontShape{T1}{llcmss}{bx}{ui}{<->ssub*llcmss/bx/n}{}
%    \end{macrocode}
% \iffalse
%</lxt1ssfd>
% \fi
%
% In this case of T1 encoded typewriter fonts I don't use any magnification,
% as it was done for the OT1 encoded ones, because the upright font exists
% at the design size of 8pt, but the italic one does not; therefore for
% homogeneity I thought it was better to use them at a design size of 10pt
% without any scaling. The differences are so tiny, that are invisible at
% naked eye.
% But since the |lcmtt| family is already defined as a standard family, we
% need a different family name in order to avoid confusion; this is because
% we use a single size to be enlarged or shrunk as it is done with the main
% text font.
% \iffalse
%<*lxt1ttfd>
% \fi
%    \begin{macrocode}
 \DeclareFontFamily{T1}{llcmtt}{\hyphenchar\font\m@ne}
 \DeclareFontShape{T1}{llcmtt}{m}{n}{<-> ec-lmtt10}{}
 \DeclareFontShape{T1}{llcmtt}{m}{it}{<-> ec-lmtti10}{}
 \DeclareFontShape{T1}{llcmtt}{m}{sl}{<-> ssub* llcmtt/m/it}{}
 \DeclareFontShape{T1}{llcmtt}{bx}{n}{<-> ssub* llcmtt/m/n}{}
 \DeclareFontShape{T1}{llcmtt}{bx}{it}{<-> ssub* llcmtt/m/it}{}
 \DeclareFontShape{T1}{llcmtt}{bx}{sl}{<-> ssub* llcmtt/m/it}{}
%    \end{macrocode}
% \iffalse
%</lxt1ttfd>
% \fi
%
% \subsection{Fonts in TS1 encoding}
% The Text Companion fonts are also restyled so that these font must be
% redeclared in case the user wants to use them.
% \iffalse
%<*lxts1ssfd>
% \fi
%    \begin{macrocode}
 \DeclareFontFamily{TS1}{llcmss}{\hyphenchar\font45}
 \DeclareFontShape{TS1}{llcmss}{m}{n}{<-> ltclq8}{}
 \DeclareFontShape{TS1}{llcmss}{m}{sl}{<-> ltcli8}{}
 \DeclareFontShape{TS1}{llcmss}{m}{it}{<-> ssub*llcmss/m/sl}{}
 \DeclareFontShape{TS1}{llcmss}{bx}{n}{<-> ltclb8}{}
 \DeclareFontShape{TS1}{llcmss}{bx}{sl}{<-> ltclo8}{}
 \DeclareFontShape{TS1}{llcmss}{bx}{it}{<-> ssub*llcmss/bx/sl}{}
 \DeclareFontShape{TS1}{llcmss}{m}{ui}{<-> ssub*llcmss/m/n}{}
 \DeclareFontShape{TS1}{llcmss}{bx}{ui}{<-> ssub*llcmss/bx/n}{}
%    \end{macrocode}
% \iffalse
%</lxts1ssfd>
% \fi
% \section {Math fonts and special fonts}
% \subsection{Math fonts in OML encoding}
% The math italic letters font has been restyled as the text fonts.
% \iffalse
%<*lxomlmmfd>
% \fi
%    \begin{macrocode}
\DeclareFontFamily{OML}{llcmm}{\skewchar\font'177}
\DeclareFontShape{OML}{llcmm}{m}{it}{<-> lcmmi8}{}
\DeclareFontShape{OML}{llcmm}{bx}{it}{<-> lcmmib8}{}

\DeclareFontFamily{OML}{llcmss}{}
\DeclareFontShape{OML}{llcmss}{m}{n}{<-> ssub*llcmm/m/it}{}
%    \end{macrocode}
% \iffalse
%</lxomlmmfd>
% \fi
%
% \subsection{Math fonts in OMS encoding}
% The math symbol font was restyled as the text fonts.
% \iffalse
%<*lxomssyfd>
% \fi
%    \begin{macrocode}
\DeclareFontFamily{OMS}{llcmsy}{\skewchar\font'60}
\DeclareFontShape{OMS}{llcmsy}{m}{n}{<-> lcmsy8}{}
\DeclareFontShape{OMS}{llcmsy}{bx}{n}{<-> lcmbsy8}{}

\DeclareFontFamily{OMS}{llcmss}{\skewchar\font'60}
\DeclareFontShape{OMS}{llcmss}{m}{n}{<->ssub*llcmsy/m/n}{}
%    \end{macrocode}
% \iffalse
%</lxomssyfd>
% \fi
%
% \subsection{Math fonts in OMX encoding}
% The large operators and delimiters font was restyled as the text fonts.
% \iffalse
%<*lxomxexfd>
% \fi
%    \begin{macrocode}
\DeclareFontFamily{OMX}{llcmex}{}
\DeclareFontShape{OMX}{llcmex}{m}{n}{<-> sfixed* lcmex8}{}
%    \end{macrocode}
% \iffalse
%</lxomxexfd>
% \fi
% \subsection {The \AMS\ fonts}
% Since the \AMS\ fonts are used so often, either directly or through
% the \pack{amssymb} package, I restyled both families |msam| and |msbm|.
% In the .sty file I kept the math group names the same so that the symbol
% definitions keep making reference to the same encoding and slot positions,
% but they take the glyphs from different files.
%
% In both cases the |\DeclareFontFamily| statement appears to be unnecessary
% because these font description files are loaded only if the \pack{amsfonts}
% package has been called. But we have seen above that these font description
% files are reloaded at the |\AtEndPreamble| hook; moreover if these fonts are
% used as standalone ones, the declaration is absolutely necessary.
% \iffalse
%<*lxumsafd>
% \fi
%    \begin{macrocode}
\DeclareFontFamily{U}{lmsa}{}
\DeclareFontShape{U}{lmsa}{m}{n}{<-> lmsam8}{}
\DeclareFontShape{U}{lmsa}{bx}{n}{<-> ssub* lmsa/m/n}{}
%    \end{macrocode}
% \iffalse
%</lxumsafd>
% \fi
%
% \iffalse
%<*lxumsbfd>
% \fi
%    \begin{macrocode}
\DeclareFontFamily{U}{lmsb}{}
\DeclareFontShape{U}{lmsb}{m}{n}{<-> lmsbm8}{}
\DeclareFontShape{U}{lmsb}{bx}{n}{<-> ssub* lmsb/m/n}{}
%    \end{macrocode}
% \iffalse
%</lxumsbfd>
% \fi
%
% \subsection{The \LaTeX\ symbol fonts}
% The same treatment is used for the \LaTeX\ symbol fonts as it was done with
% the teletype text font: a different family name, but one font to be enlarged
% or shrunk as the main text font..
% \iffalse
%<*lxultxfd>
% \fi
%    \begin{macrocode}
\DeclareFontFamily{U}{lllasy}{}
\DeclareFontShape{U}{lllasy}{m}{n}{<-> llasy8}{}
\DeclareFontShape{U}{lllasy}{b}{n}{<-> llasyb8}{}
%    \end{macrocode}
% \iffalse
%</lxultxfd>
% \fi
%
% \subsection{The Greek font for slides}
% The CB Greek font collection contains also the family and shape of the slides
% fonts; of course with the Greek script there is no problem with possible
% confusions of capital `I' and lower case `l', therefore they did not need
% any restyling. The only question related to Greek fonts is that the same
% font family names for the Latin Script must be associated with the Greek
% script with the Greek encoding LGR, instead of the Latin encoding T1. The
% Greek font description files, therefore, have a different prefix (LGR),
% the same family name (llcmss for proportional sans serif fonts, or lcmtt
% for monospaced ones), but the glyphs are taken from the relevant Greek fonts.
% \iffalse
%<*lxlgrfd>
% \fi
%    \begin{macrocode}
\DeclareFontFamily{LGR}{llcmss}{\hyphenchar\font45}
\DeclareFontShape{LGR}{llcmss}{m}{n}{<-> glmn0800}{}
\DeclareFontShape{LGR}{llcmss}{m}{sl}{<-> glmo0800}{}
\DeclareFontShape{LGR}{llcmss}{m}{it}{<-> ssub* llcmss/m/sl}{}
\DeclareFontShape{LGR}{llcmss}{bx}{n}{<-> glxn0800}{}
\DeclareFontShape{LGR}{llcmss}{bx}{sl}{<-> glxo0800}{}
\DeclareFontShape{LGR}{llcmss}{bx}{it}{<-> ssub* llcmss/bx/sl}{}
\DeclareFontShape{LGR}{llcmss}{m}{ui}{<-> ssub* llcmss/m/n}{}
\DeclareFontShape{LGR}{llcmss}{bx}{ui}{<-> ssub* llcmss/bx/n}{}
%    \end{macrocode}
% \iffalse
%</lxlgrfd>
% \fi
% The same action is taken for the teletype fonts, even if it is less probable
% that such fonts are used in a presentations, since the teletype font is
% normally used for typesetting programming code texts, very seldom written
% in Greek.
% \iffalse
%<*lxlgrttfd>
% \fi
%    \begin{macrocode}
 \DeclareFontFamily{LGR}{llcmtt}{\hyphenchar\font\m@ne}
 \DeclareFontShape{LGR}{llcmtt}{m}{n}{<-> gltn1000}{}
 \DeclareFontShape{LGR}{llcmtt}{m}{it}{<-> glto1000}{}
 \DeclareFontShape{LGR}{llcmtt}{m}{sl}{<-> ssub* lcmtt/m/it}{}
 \DeclareFontShape{LGR}{llcmtt}{bx}{n}{<-> ssub* lcmtt/m/n}{}
 \DeclareFontShape{LGR}{llcmtt}{bx}{it}{<-> ssub* lcmtt/m/it}{}
 \DeclareFontShape{LGR}{llcmtt}{bx}{sl}{<-> ssub* lcmtt/m/it}{}
%    \end{macrocode}
% \iffalse
%</lxlgrttfd>
% \fi
%
%\subsection{The map file}
% The map file is necessary in order to reconfigure the |updmap.cfg|
% so as to let \prog{pdftex} access these fonts. Notice that this
% reconfiguration is done by the system installation macros and the
% user should not play around with such delicate questions; it must be
% remarked also that the methods for reconfiguring that file
% are being updated in a significant way in certain distribution
% of the \TeX\ system, and it would be very dangerous to describe the
% 2013 procedure, when it possible that in 2014 it is a different one.
% In any case the user who really needs to install these fonts ``by hand''
% is advised to read the documentation of his/her current distribution
% of the \TeX\ system so as to perform  the update map correct configuration
% procedure.
% \iffalse
%<*lxmap>
% \fi
%    \begin{macrocode}
lcmbsy8   lcmbsy8   <lcmbsy8.pfb
lcmex8    lcmex8    <lcmex8.pfb
lcmmi8    lcmmi8    <lcmmi8.pfb
lcmmib8   lcmmib8   <lcmmib8.pfb
lcmsy8    lcmsy8    <lcmsy8.pfb
leclb8    leclb8    <leclb8.pfb
lecli8    lecli8    <lecli8.pfb
leclo8    leclo8    <leclo8.pfb
leclq8    leclq8    <leclq8.pfb
llasy8    llasy8    <llasy8.pfb
llasyb8   llasyb8   <llasyb8.pfb
llcmss8   llcmss8   <llcmss8.pfb
llcmssb8  llcmssb8  <llcmssb8.pfb
llcmssi8  llcmssi8  <llcmssi8.pfb
llcmsso8  llcmsso8  <llcmsso8.pfb
lmsam8    lmsam8    <lmsam8.pfb
lmsbm8    lmsbm8    <lmsbm8.pfb
ltclb8    ltclb8    <ltclb8.pfb
ltcli8    ltcli8    <ltcli8.pfb
ltclo8    ltclo8    <ltclo8.pfb
ltclq8    ltclq8    <ltclq8.pfb
%    \end{macrocode}
% \iffalse
%<*lxmap>
% \fi
% The Greek CB fonts need not to be listed in this map file, because they
% are already listed in the CB font map file that is being already used in
% the installation of that Greek font collection.
% \Finale
% \endinput





