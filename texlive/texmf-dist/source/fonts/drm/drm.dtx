% \iffalse
% +AMDG  This document was begun on 25 May 11EX, the feast
% of St. Mary Magdalen de Pazzi, and it is humbly dedicated
% to her and to the Immaculate Heart of Mary for their
% prayers, and to the Sacred Heart of Jesus for His mercy.
% 
% This document is copyright 2014 by Donald P. Goodman, and is
% released publicly under the LaTeX Project Public License.  The
% distribution and modification of this work is constrained by the
% conditions of that license.  See
% 	http://www.latex-project.org/lppl.txt
% for the text of the license.  This document is released
% under version 1.3 of that license, and this work may be distributed
% or modified under the terms of that license or, at your option, any
% later version.
% 
% This work has the LPPL maintenance status 'maintained'.
% 
% The Current Maintainer of this work is Donald P. Goodman
% (dgoodmaniii@gmail.com).
% 
% This work consists of all files listed in drmfilelist.txt.
% \fi

% \iffalse
%<package>\NeedsTeXFormat{LaTeX2e}[1999/12/01]
%<package>\ProvidesPackage{drm}[2015/01/01 v3.0 support for dozenal fonts]
%<*driver>
\documentclass{ltxdoc}

\usepackage[absolute]{textpos}
\usepackage{parcolumns}
\usepackage{doc}
\usepackage{array}
\usepackage{lettrine}
	\setcounter{DefaultLines}{3}
	\setlength{\DefaultFindent}{2pt}
	\renewcommand{\LettrineFontHook}{\color{red}}
\usepackage{url}
\usepackage{booktabs}
\usepackage{supertabular}
\usepackage{longtable}
\usepackage{fetamont}
\usepackage{spverbatim}
\usepackage[greek,english]{babel}
\languageattribute{greek}{polutoniko}
\usepackage[colorlinks]{hyperref}
\usepackage[typeone]{drm}
\usepackage{makeidx}
\EnableCrossrefs
\PageIndex
\CodelineNumbered
\RecordChanges
\makeindex
\DoNotIndex{\?,\{,\},\|,\DeclareFontFamily,\DeclareFontShape,
	\DeclareMathAccent,\DeclareMathAlphabet,\DeclareMathDelimiter,
	\DeclareMathSymbol,\DeclareMathVersion,\DeclareSymbolFont,\def,
	\drmsym,\encodingdefault,\familydefault,\fontencoding,\fontfamily,
	\fontseries,\fontshape,\hfil,\hbox,\mathalpha,\mathclose,
	\mathopen,\mathord,\mathversion,\mp,\nabla,\nbshortroman,
	\RedefineMRmdclxvij,\relax,\renewcomand,\RequirePackage,
	\selectfont,\SetMathAlphabet,\SetSymbolFont,\drmsymbolredef,
	\nodefaultfalse,\nodefaultmathfalse,\nodefaultmathtrue,\nodefaulttrue,
	\nodefaulttextfalse,\nodefaulttexttrue,\noindent,\numexpr,
	\acute,\addtolength\advance,\backslash,\baselineskip,
	\char,\counterA,\counterB,\DeclareOption,\ProcessOptions,
	\drmmathlets,\symbolsonlytrue,\if,\else,\fi,\ifnum,\fi,\ifdim,\fi,
	\symbolsonlyfalse,\fontsize,\newif,\newcount,
	\loop,\iter,\let,\renewcommand,\setbox,\setlength,\the,\vss,
	\vskip,\vbox,\ifnodefault,\fi,\ifnodefaulttext,\fi,
	\ifnodefaultmath,\fi,\ifsymbolsonly,\fi,\newlength,
	\counterA,\counterB,\newcount,\r@@t,\z@,\DeclareMathRadical,
	\DeclareRobustCommand,\bBigg@,\@ifnextchar,\@sqrt,\@makefnmark,
	\@thefnmark,\catcode,\active,\Q}

\begin{document}
\DocInput{drm.dtx}
\end{document}
%</driver>
% \fi
%
% \title{The \texttt{drm} Font Package, v3.0}
% \author{Donald P.\ Goodman III}
% \date{\today}
%
% \maketitle
% \begin{abstract}
% The |drm| package provides access to the DRM (Don's
% Revised Modern) family of fonts, which includes a variety
% of optical sizes in Roman, italic, and small caps, along
% with a set of symbols and ornaments.  It is intended to be
% a full-body text font, but its larger sizes can also be
% used for simple display purposes, and its significant body
% of symbols can stand on its own.  It comes complete with
% textual (``old-style'') and lining figures, and even has
% \emph{small-caps figures}, along with superior and
% inferior figures.  It also comes with extensible
% decorative rules to be used with ornaments from itself or
% other fonts, along with an extremely flexible ellipsis
% package.  Decorative initials are also provided by means
% of \MP\ macros and superimposed figures; these are
% flexible in color, size, and style.
% \end{abstract}
% 
% \tableofcontents
% 
% \section{Introduction}
% \label{sect:intro}
%
% \lettrine{A}{fter some time} of involvement with \TeX\ and \LaTeX\,
% your author finally bothered to go read \textit{The \TeX
% book} and \textit{The \MF book}.  This latter closes,
% of course, with Donald E.\ Knuth's famous exhortation to
% ``\textsc{Go forth} now and create \textsl{masterpieces of
% digital typography!''}  This call to arms stirred a longing to actually
% do so in my soul.
%
% I had some experience with the \MF\ language through
% my work with \MP, so I thought I might try my hand at
% it.  I started in fits and stops some years ago, and only
% a few months ago took up the cause again in earnest.  I
% found that, as Knuth also warned, 
%
% \begin{quote}\textsc{Warning}:  Type design can be
% hazardous to your other interests.  Once you get hooked,
% you will develop intense feelings about letterforms; the
% medium will intrude on the messages that you read.  And
% you will perpetually be thinking of improvements to the
% fonts that you see everywhere, especially those of your
% own design.\end{quote}
%
% Truer words were never spoken.
%
% This document is typeset in accordance with the
% \textsc{docstrip} utility for automatically extracting
% package code and documentation.
%
% \section{License(s)}
%
% \lettrine{T}{he \TeX\ and \LaTeX\ code} in this package is
% licensed under the \LaTeX\ Project Public License v1.3c,
% the details of which can be found in Appendix \ref{lppl}
% on page \pageref{lppl}.  It's a legal document, and bears
% all the concomitant complications of such.  The basic
% import is that you can use and distribute these files as
% you will, provided only that you do not restrict their use
% by their recipients; and that you can even modify them as
% you will, provided that if you distribute your
% modifications, you do so under a different name.
%
% The fonts themselves are licensed under the SIL Open Font
% License, v1.1, the details of which can be found in
% Appendix \ref{sil} on page \pageref{sil}.  It's a less
% complex legal document, but a legal document all the same.
% The basic import is that you can't sell the fonts all by
% themselves (why anyone would pay for them anyway is beyond
% me, but there it is); you can distribute original or
% modified versions of the fonts otherwise however you wish,
% as long as you keep the copyright notice and license with
% it; and if you distribute a modified version, that you do
% so under a different name; that you not use the name of
% the font designer to promote a modified version; and that
% any modified versions of the fonts must be kept under the
% same license.
%
% Finally, the software I used to build things, which is
% very simple and mostly uninteresting stuff, is also made
% available, under the GNU General Public License v3.  This
% is sufficiently well-known that it's not duplicated in
% this document; but the text is, of course, included in
% the distribution.
%
% That said, I'm pretty easy-going about this sort of thing;
% so if for some reason the above terms don't suit you, feel
% free to contact me and see if we can work something else
% out.  But honestly, the terms of these licenses are
% more than fair, and it's hard for me to see a reason to
% depart from them.
%
% \section{Usage}
% \label{sect:usage}
%
% \subsection{Basic Usage}
% \label{sub:basicusage}
%
% Using the DRM fonts is beyond easy; just include the
% following in your preamble:
%
% \begin{quote}
% |\usepackage{drm}|
% \end{quote}
%
% And you're done!  This makes the DRM fonts the default for
% your document, and defines appropriate commands for using
% them.  I've made every effort to make using DRM as
% unsurprising as possible, so the commands you'd use to
% change sizes, styles, shapes, and so forth should all work
% as expected.  That said, there are some unusual shapes and
% options available, and these are explained below.
%
% You do \emph{not} need to load |textcomp|; all those
% symbols and more are available from |drm|.
%
% \subsection{Package Options}
% \label{sub:packopt}
%
% DRM doesn't offer too many options, because too many
% aren't really needed; but it does allow some control over
% what defaults it resets.
%
% \begin{description}
% \item[typeone]  \DescribeMacro{typeone}The |typeone|
% option will probably be used most of the time that |drm|
% itself is used; it forces |drm| to use un-rasterized
% outlines rather than bitmapped pk files, letting the pdf
% viewer do the rendering.  Given that most pdf viewers have
% an awfully hard time decently displaying prerendered
% bitmap fonts, the |typeone| option will often be useful.
% \item[nodefault] \DescribeMacro{nodefault}The |nodefault|
% option means that |drm| will not change any of the
% defaults of the document; that is, loading |drm| with the
% |nodefault| option should have no effect whatsoever on the
% appearance of your document.  The fonts are defined,
% though, so you can use them if you decide you want to.
% Symbols will \emph{not} be redefined.
% \item[nodefaulttext] \DescribeMacro{nodefaulttext}The
% |nodefaulttext| option means that text fonts are
% \emph{not} redefined but math fonts \emph{are}.  Symbols
% will \emph{not} be redefined.
% \item[nodefaultmath] \DescribeMacro{nodefaultmath}The
% |nodefaultmath| option means that text fonts \emph{are}
% redefined but math fonts are \emph{not}.  Symbols
% \emph{will} be redefined.
% \item[symbolsonly] \DescribeMacro{symbolsonly}The
% |symbolsonly| option defines all the commands for the
% symbols (note that this may overwrite some command names,
% like |\textcopyright|; if you need these undefined, load
% |textcomp| \emph{after} |drm|).
% \end{description}
%
% The default is that none of these are selected; that is,
% the default is that both text and math fonts, along with
% symbols, are redefined to be DRM.  Commands which are
% font-independent, like |\tulipframe| and |\extrule|, are
% always defined when the package is loaded.  Also, the
% fonts themselves are always defined, so they can be
% accessed directly even if they are not the default.
%
% Note that even if symbols are not redefined, they are
% still available directly through the
% \DescribeMacro{\drmsym}|\drmsym| command.
% This command takes one argument, typically a |\char|
% directive, which will be the decimal, octal (if preceded
% by |'|), or hexadecimal (if preceded by |"|) position of
% the desired symbol in the font.  E.g.:  
%
% \hbox to\linewidth{%
%	\hfil|\drmsym{\char'117}|\hfil|\drmsym{\char"4F}|\hfil%
%	|\drmsym{\char79}|\hfil}
%
% \noindent yields
%
% \hbox to\linewidth{%
%	\hfil\drmsym{\char'117}\hfil\drmsym{\char"4F}\hfil%
%	\drmsym{\char79}\hfil}
%
% \subsection{Interaction with Other Packages}
% \label{sub:interpack}
%
% As far as your author has been able to tell, |drm| has no
% adverse reactions with any other packages.  A few notes
% are probably appropriate, however.
%
% \DescribeMacro{textcomp}You do \emph{not} have to load
% |textcomp| when you're loading |drm|; |drm| defines all
% the symbols in |textcomp|, and then some, allowing access
% to them with the same commands.  This is the |drmsym|
% font, which is encoded, like the |textcomp| font, as TS1.
% If you're not loading the symbols, though (e.g., you've
% loaded |drm| with options |nodefault| or |nodefaulttext|),
% you may still want to load |textcomp|.
%
% \DescribeMacro{lettrine}The |lettrine| package is used to
% typeset large dropped capitals at the beginning of
% paragraphs; it's an extraordinarily flexible and
% well-designed package.  |drm| works just fine with it;
% however, the proportions of the letters make a small
% tweak advisable.  If you'll be using lettrines larger than
% two lines high, the following will be helpful:
%
% \hbox to\linewidth{\hfil|\setlength{\DefaultFindent}{2pt}|\hfil}
%
% \noindent This will prevent your text from bumping into
% your lettrine.
%
% \subsection{Further Work Needed}
% \label{sub:further}
%
% While I'm quite happy with DRM right now, there are a few
% notable places where it needs some additional work.
%
% \begin{description}
% \item[Kerning]  The kerning is sometimes suboptimal.
% There really isn't much else to say about this.  For
% most of the fonts, the kerning is reasonably good (at
% least, in my opinion), but for upright italic, boldface,
% and occasionally small and titling caps, I do still find
% lacun\ae\ in my handling of certain kerning pairs.
% \item[Internal Code]  While the code is parameterized
% enough that, for example, boldfacing was a relatively
% simple process, it could use some improvement in this.
% Also, some code was repeated that would surely be better
% off included in macros, especially the placing of accents.
% \item[Decorative Initials]  I love, love, \emph{love}
% decorative initials, and want DRM to have them.  But
% writing them is a \emph{lot} of work.  I decided I wanted
% the fonts as they stand done before I get to work on
% those; but it's still further work that needs to happen.
% \item[More Ornaments]  I'm pretty happy with what
% ornaments I've designed for DRM, but it needs more of
% them.  A full, 8-bit ornamental font is in the works (the
% decorative initials will likely be A--Z in this font), but
% designing these is a similarly large amount of work, so
% it's still on the burners.
% \item[Greek Fonts]  DRM badly needs real Greek fonts.
% After I did the math fonts, Greek fonts seemed like a
% short step; but now I really need to add italic and
% boldface versions, and optically size the upright ones.
% \end{description}
%
% In addition to these specific needs, font metrics may
% still change, though only slightly, and shapes are subject
% to tweaking here and there.  But even now, the DRM fonts
% are usable, reasonably complete, and (in my view, at
% least) attractive.
%
% \section{About the DRM Fonts}
%
% So I've been plumbing the depths of alphabet design, and
% having a great time doing it.  The result is what you see
% before you, the DRM fonts.
%
% \subsection{About the Fonts}
% 
% They're not \textit{modern}, per se, but they do have
% modern characteristics, most especially the distinction
% between thick and thin strokes and the vertical
% orientation.  They have a number of old-style
% characteristics, as well, though, like the aforementioned
% ``Q'' tail, the relatively prominent serifs, and the
% slight but still present brackets.
%
% Overall, they're fairly dark fonts on average, as well as
% fairly wide.  To my eye, this makes them ideal for reading
% long passages.
%
% They have some unusual features.  For example, they have not
% only the standard run of f-ligatures (fi, fl, ff, ffi,
% ffl), but also some unusual f-ligatures (ft, fj), as well
% as a non-f-ligature (Th).  Some larger-size examples of
% the ligatures can be found in Table \ref{table:ligs}.  DRM 
% also contains some unusual shapes, such as \textui{upright 
% italic} and \texttc{titling caps}\index{titling caps}.
%
% \begin{table}
% \begin{center}\setlength{\extrarowheight}{9pt}
% \begin{tabular}{>{\Large}c>{\Large\itshape}c
%	>{\Large}c>{\Large\itshape}c
%	>{\Large}c>{\Large\itshape}c}
% \toprule
% Roman & Italic & Roman & Italic & Roman & Italic \\
% \midrule
% fi & fi & ff & ff & fl & fl \\
% fj & fj & ffi & ffi & ffl & ffl \\
% ft & ft & Th & Th & {} & {} \\
% \bottomrule
% \end{tabular}
% \caption{Ligatures in the DRM fonts.}
% \label{table:ligs}
% \end{center}
% \end{table}
%
% \subsection{Alternate Glyphs}
%
% While this section is titled in the plural, there is at
% present only one such:
% \DescribeMacro{\drmshortq}|\drmshortq|, which gives us
% ``\drmshortq'' rather than ``Q.''  This is mostly useful
% for situations in which the ``Q'' is followed by some
% character with a descender (say, ``\drmshortq p'' as
% opposed to ``Qp''), or when it is being used as a dropped
% capital and the extended tail would overwrite the text.
% (An enlarged dropped capital might have a tail
% underscoring the entire paragraph, which might actually
% look attractive.)
%
% If for some reason you'd like to use ``\drmshortq'' all
% the time, and consign the admittedly somewhat baroque
% ``Q'' to the dustbin, you can do so by issuing the
% following commands:
%
% \begin{center}
% |\catcode`\Q=\active\def Q{\drmshortq}|
% \end{center}
%
% Note that this involves some deep \TeX\ magic, and command
% names containing the character ``Q'' will be broken by
% this.  Fortunately, such commands are few and far between;
% |drm| does not contain any.
%
% \subsection{Font Families}
%
% DRM contains a full set of the normal font families you'd
% expect:  roman, bold, italic, small caps, and so forth.
% But it also contains some shapes that are rather unusual,
% as well as a wide variety of sizes, forms, and weights
% capable of filling most needs.
%
% \subsubsection{Optical Sizing}
%
% The advent of digital fonts made many typographers lazy.
% Previously, of course, a printer could only print fonts in
% sizes that he had; each size had to be separately cut and
% designed.  Digital fonts seemed to relieve this problem;
% now we can simply scale up or down, and only design a
% single size!  Experience has shown, however, that this
% produces suboptimal results, as Table \ref{tab:optsize}
% demonstrates.
%
% \begin{table}[htbp]
% \hbox to\linewidth{%
% 	\hfil%
% 	\Large Fourteen point font is different%
% 	\hfil%
% }%
% \hbox to\linewidth{%
% 	\hfil%
% 	\font\scalio=drm7 at14pt\scalio from scaled seven point font.%
% 	\hfil%
% }%
% \caption{Scaling and Optical Sizing Compared}
% \label{tab:optsize}
% \end{table}
%
% The human eye, as it turns out, does not perceive the
% world, least of all letterforms, as geometrically scaled
% versions of larger or smaller shapes.  For example, at
% small point sizes the eye tends to run adjacent strokes
% together, so proportionally wider letters and increased
% letterspacing are appropriate in smaller sizes but not in
% larger.  For another example, strokes often overlap the
% technical top or bottom lines because a curved line will
% appear to be lower than a straight line at the same
% height.  This effect diminishes at larger sizes; so this
% overshoot might be zero at double pica (twenty-four
% point), still significant at pica (twelve-point), and
% quite large at six-point.  If we merely scaled the six
% point to get our twenty-four point, this overshoot would
% make the curved strokes look comically larger than the
% straight ones; if we did the opposite, then our curved
% letters would seem noticeably shorter than our straight
% ones.
%
% The only real solution to this is to use \emph{optical
% sizes}; that is, have a reasonable set of sizes which are
% designed for use at that particular size.  \LaTeX\
% (largely transparently to the user, thanks to the magic of
% NFSS) will then select the closest optical size and scale
% as necessary from that.  This minimizes the effects of
% scaling on the appearance of the font, and gives vastly
% superior results.
%
% \begin{table}[htbp]
% \begin{center}\footnotesize
% \begin{tabular}{lp{0.2\textwidth}p{0.2\textwidth}p{0.2\textwidth}}
% \toprule
% Point & \multicolumn{2}{c}{Traditional Name} &
%	DRM \\
% {} & American & British & {} \\
% \midrule
% 3 & Excelsior & Minikin & |\excelsior|, |\minikin| \\
% 4 & Brilliant & {} & |\brilliant| \\
% 4.5 & Diamond & {} & |\diamond| \\
% 5 & Pearl & {} & |\pearl| \\
% 5.5 & Agate & Ruby & |\agate|, |\ruby| \\
% 6 & Nonpareille & {} & |\nonpareille| \\
% 6.5 & Minionette & Emerald & |\minionette|, |\emerald| \\
% 7 & Minion & {} & |\minion| \\
% 8 & Brevier, Petit, small text & {} & |\brevier|,
%	|\petit|, |\smalltext| \\
% 9 & Bourgeois; Galliard & {} & |\bourgeois|,
%	|\galliard| \\
% 10 & Long Primer; Corpus; Garamond & {} &
%	|\longprimer|, |\corpus|, |\garamond| \\
% 11 & Small Pica; Philosophy & {} & |\smallpica|,
% 	|\philosophy| \\
% 12 & Pica & {} & |\pica| \\
% 14 & English; Mittel; Augustin & {} & |\english|,
% |\mittel|, |\augustin| \\
% 16 & Columbian & Two-line Brevier & |\columbian|,
% |\twolinebrevier| \\
% 18 & Great Primer & {} & |\greatprimer| \\
% 20 & Paragon & {} & |\paragon| \\
% 21 & Double Small Pica & {} & |\doublesmallpica| \\
% 22 & Double Small Pica & Double Pica &
% |\doublesmallpicaus|, |\doublepicabrit| \\
% 24 & Double Pica & Two-line Pica & |\doublepica|,
% |\twolinepica| \\
% 28 & Double English & Two-line English & |\doubleenglish|, 
% |\twolineenglish| \\
% 30 & Five-line Nonpareil & {} & |\fivelinenonpareil| \\
% 32 & Four-line Brevier & {} & |\fourlinebrevier| \\
% 36 & Double Great Primer & Two-line Great Primer &
% |\doublegreatprimer|, |\twolinegreatprimer| \\
% 44 & Meridian & Two-line Double Pica; Trafalgar &
% |\meridian|, |\twolinedoublepica|, |\trafalgar| \\
% 48 & Canon; Four-line & {} & |\canon|, |\fourline| \\
% 60 & Five-line Pica & {} & |\fivelinepica| \\
% 72 & Inch & {} & |\inch| \\
% \bottomrule
% \end{tabular}
% \caption{Traditional size names, both American and
% British, with their corresponding point sizes and DRM
% command names.}
% \label{tab:fontsizes}
% \end{center}
% \end{table}
%
% DRM offers a reasonable selection of optical sizes, at 6-,
% 7-, 8-, 9-, 10-, 11-, 12-, 14-, 17-, and 24-point sizes in
% roman, italic, slanted, small caps, titling
% caps\index{titling caps}, and upright italic.  This
% variety should be sufficient for the vast majority of
% needs.
%
% However, traditional printing has developed a vast array
% of standard sizes, with the quaint, colorful names that
% always go with traditional crafts.  Setting text in
% Brevier Roman or Long Primer Italic means something very
% specific.  \LaTeX\ only offers a few default font
% size commands (e.g., |\small|, |\normalsize|, etc.), and
% DRM leaves those unchanged, as users expect specific
% things to happen when they issue those commands.  However,
% DRM does offer those traditional size names as commands,
% as well, giving quite a bit more breadth in font size
% choice than the default before one must resort to explicit
% |\fontsize| commands.  Table \ref{tab:fontsizes} on page
% \pageref{tab:fontsizes} lists these commands by name; where
% there is a name unique to British typesetting that differs
% from the American name, both are offered as equivalents.
%
% \subsubsection{Small Caps}
%
% In the first place, it's important to have what
% typographers call ``real'' small caps, not ``faked'' small
% caps.  There is a real and noticeable difference between
% the two.  Real small caps are designed for a particular
% size; the stroke widths match, the spacing is appropriate,
% and so forth.  Faked small caps are produced merely by
% scaling down normal capital letters for a given size,
% which produces inferior results.  The two types are
% compared in Table \ref{tab:smallcaps}.
%
% \begin{table}[htbp]
% \hbox to\linewidth{\hfil%
% \LARGE\textsc{These are real small caps.}
% \hfil}%
% \vskip2em%
% \hbox to\linewidth{\hfil%
% \LARGE{T\large HESE ARE FAKED SMALL CAPS.}
% \hfil}%
% \caption{Real and faked small caps compared.}
% \label{tab:smallcaps}
% \end{table}
%
% Plainly, the results of real small caps are far superior,
% and faked ones should only be employed when the
% typographer has no small caps available, and possibly
% not even then.
%
% \index{small caps}\index{small caps>petite small caps}
% \index{small caps>and titling caps, difference between}
% \index{titling caps>and small caps, difference between}
% DRM has, of course, a full set of real small cap fonts,
% which are appropriately scaled.  But DRM goes even further
% than this, offering both \emph{small caps} and
% \emph{petite small caps}.
% \index{petite small caps>small caps}
% \index{petite small caps>and titling caps}  DRM, though, considers petite
% small caps to be the normal type, and therefore refers to
% these as \emph{small caps} and \emph{titling
% caps}.\index{titling caps}\index{small caps>titling caps}  The
% distinction between these two probably bears some
% explanation.
%
% In Anglo-American typography small caps are
% typically a bit larger than the ex-height; in other
% countries, they are typically equal to the
% ex-height.\footnote{These are sometimes called
% \textit{petite small caps} among Anglo-American
% typographers.}  I see advantages in both approaches.
% So-called ``petite'' small caps look great in running text
% but seem rather squashed in titles and headings; larger
% small caps look better in titles and headings (they
% maintain the gravity of all-caps without the impression of
% shouting, a rather common impression here in the Internet
% age) but are simply too large to blend well with normal
% lowercase text.  So DRM offers both; \textsc{normal small
% caps}, accessed via the normal \LaTeX\ |\textsc| and
% |\scshape| commands, are ``petite'' small caps;
% Anglo-American large small caps are available as
% \texttc{titling small caps}, via the commands
% |\texttc|\DescribeMacro{\texttc}\ and
% |\tcshape|\DescribeMacro{\tcshape}.  (These stand,
% transparently enough, for ``titling caps.'')  An example
% of the difference, which may give further ideas for the
% appropriate uses for each, is in Table
% \ref{table:littlecaps}.
%
% \begin{table}
% \begin{center}
% \begin{tabular}{>{\centering\arraybackslash}p{0.48\textwidth}
%	>{\centering\arraybackslash}p{0.48\textwidth}}
% \LARGE\texttc{The Adventures of Robinson Crusoe} &
%	\LARGE\textsc{The Adventures of Robinson Crusoe} \\
% \texttc{Titling Caps} &
%	\textsc{Small Caps} \\
% \end{tabular}
% \caption{Small caps and titling caps compared.}
% \label{table:littlecaps}
% \end{center}
% \end{table}
%
% Both small caps and titling caps come in \emph{italic}
% (really simply \textsl{slanted}) varieties; these are
% accessed via 
% \DescribeMacro{\textitsc}|\textitsc| and 
% \DescribeMacro{\textittc}|\textittc| (or 
% \DescribeMacro{\itscshape}|\itscshape| and 
% \DescribeMacro{\ittcshape}|\ittcshape|).  However, it
% should be noted that both these font shapes are designed
% for emphasizing text, as is slanting; so while it's
% perfectly possible to \textitsc{slant your small
% caps} and \textittc{italicize your titling caps},
% this really should be done with great caution.
%
% \subsubsection{Slanted, Italic, and Upright Italic}
%
% DRM also offers the usual \emph{italic} type, typically
% used for titles of certain types of works, emphasis, and
% similar functions; and the \textsl{slanted} type, which is
% usually simply a poor-man's italic in fonts which don't
% have a real italic, but occasionally does find its uses in
% full-featured fonts.
%
% Non-typographers often think that italic type is simply
% normal type slanted to the right, but that's not
% really the case.  A comparison is given in Table
% \ref{tab:slantedtypes}.
%
% \begin{table}[htbp]
% \hbox to\linewidth{\hfil%
% \Large\textsl{This is slanted text.}
% \hfil}%
% \vskip2em%
% \hbox to\linewidth{\hfil%
% \Large\textit{This is italic text.}
% \hfil}%
% \caption{Slanted and italic text compared.}
% \label{tab:slantedtypes}
% \end{table}
%
% As Table \ref{tab:slantedtypes} shows, true italic is much
% more than simply slanted roman.  Many of the letterforms
% are quite different; the two-story ``a'' becomes a
% one-story ``\textit{a},'' the straight ``k'' becomes a
% curved or (in DRM's case) looped ``\textit{k},'' the
% double-looped ``g'' becomes a single-looped
% ``\textit{g},'' and so forth.
%
% So DRM offers true italic, often with some very ornate
% and, in the author's opinion, beautiful letterforms.  The
% ``\textit{Q}'' and ``\textit{J}'' and particular favorites
% of his, but the somewhat unusual looped ``\textit{k}'' and
% curled ``\textit{h},'' along with some other shapes like
% ``\textit{2}'' and ``\textit{3},'' are also interestingly
% different from most other fonts, as well as visually
% striking in their own right.
%
% DRM also offers \textui{upright italic}, a face with the
% italic letterforms but not slanted at all.  Some fonts
% have offered this as a difficult-to-access novelty, but as
% far the author knows DRM is the only one to offer it as a
% first-class citizen, accessed in the same way and just as
% easily as the more usual slanted italic shape.  It is
% accessed via the commands |\textui|\DescribeMacro{\textui}\ and
% |\uishape|\DescribeMacro{\uishape}.  This shape is not
% commonly used in running text; it remains to be seen
% whether this is due to its being not useful, or simply to
% its being rarely easily available.
%
% Your author can see certain uses for it; for example, when
% some font distinction is needed but no connotation of
% emphasis is desired, as in book titles.  In any case, DRM
% makes it easy to use in the event that it is wanted.
%
% \subsubsection{Weights}
% \label{subsub:weights}
%
% \textit{Weight} is the typographical term for what most
% folks call \emph{boldface} type; however, the dimension
% can go much deeper than that.  Fonts can be
% \textl{lighter than surrounding text} as well as
% \textbf{heavier}, and heavier weights can often be
% \textbf{extended in width as well as heavier in weight},
% or simply \textb{heavier in weight without increasing its
% width}.
%
% Some fonts take this to arguably absurd extremes, offering
% up to a dozen weights.  I've never seen much sense in
% this, and consequently haven't gone to these lengths.
% Instead, DRM offers three weights:  \textl{light}, normal,
% and \textb{bold}.  As noted, bold fonts are often wider
% than medium weights, and are referred to as \emph{bold
% extended}; DRM has a \textbf{bold extended}, as well.
% Lighter fonts are sometimes narrower, or \emph{condensed};
% \textl{DRM's light weight is not condensed, but rather
% normal width}.
%
% Light is achieved by |\textl|\DescribeMacro{\textl}\ and
% |\lseries|\DescribeMacro{\lseries}; boldface by
% |\textb|\DescribeMacro{\textb}\ and
% |\bseries|\DescribeMacro{\bseries}; bold extended by
% |\textbf|\DescribeMacro{\textbf}\ and
% |\bfseries|\DescribeMacro{\bfseries}.  Table
% \ref{tab:weights} shows the differences between these
% weights in twelve-point size.
%
% \begin{table}[htbp]\setlength{\extrarowheight}{8pt}
% \begin{tabular}{>{\large}c>{\large}p{0.7\linewidth}}
% |\lseries| & \textl{This sentence shows one of DRM's
%	weights.} \\
% |\mdseries| & \textmd{This sentence shows one of DRM's
%	weights.} \\
% |\bseries| & \textb{This sentence shows one of DRM's
%	weights.} \\
% |\bfseries| & \textbf{This sentence shows one of DRM's
%	weights.} \\
% \end{tabular}
% \caption{DRM's font weights compared.}
% \label{tab:weights}
% \end{table}
%
% One will probably note that bold extended is actually
% bolder than normal bold (that is, |\bfseries| is bolder
% than |\bseries|).  This is because the extra space means
% that it can be.  This is probably not the way it should
% be; but I think that people expect at least that much
% boldness when selecting |\bfseries|, so I decided to make
% it that way.
%
% Until v2.0, DRM did \emph{not} offer a bold italic or a
% bold small caps; this is because both italics and small
% caps are already meant to serve for emphasis, and bolding
% your already emphasized text is really a bad idea.  Bold
% italic I considered to be a particularly egregious
% typographical crime.
%
% However, after some conversations and one example in which
% bold italic was actually used well (a display; \emph{not}
% in running text), your author has tempered his aggressive
% stance and provided a bold italic, bold upright italic,
% and bold versions of both small and titling caps.  These
% are bold extended, and are available only in this one
% additional weight; this seemed appropriate given that
% their only appropriate use is displayed texts.  They are
% accessed simply by requesting either bold or italic, and
% then requesting the other, like so:
%
% \begin{center}
% \begin{tabular}{p{0.45\textwidth}p{0.45\textwidth}}
% \begin{spverbatim}\textbf{\textit{I hope you bold italic types are happy now}}\end{spverbatim} &
% \begin{spverbatim}\textui{\textbf{I hope you bold italic types are happy now}}\end{spverbatim} \\
% \textbf{\textit{I hope you bold italic types are happy now.}} &
% \textui{\textbf{I hope you bold italic types are happy now.}} \\
% \begin{spverbatim}\textsc{\textbf{Sometimes this might help with displays.}}\end{spverbatim} &
% \begin{spverbatim}\textbf{\texttc{Sometimes this might help with displays.}}\end{spverbatim} \\
% \textsc{\textbf{Sometimes this might help with displays.}} &
% \textbf{\texttc{Sometimes this might help with displays.}} \\
% \end{tabular}
% \end{center}
%
% There is still no bold italic small caps; I'll await an
% example of these being used appropriately before adding
% them.
%
% \subsection{Figures (Digits) (Numbers)}
% \label{sub:figures}
%
% Typographers typically call them ``figures'';
% mathematicians tend to prefer ``digits''; most folks
% simply call them ``numbers.''  Whatever we call them,
% there is a surprisingly large variety of ways to write
% them.
%
% There are, at the very least, seven separate kinds of
% figures:  textual, lining, tabular textual, tabular
% lining, small caps, superior, and inferior.  Small caps
% figures are for some reason rather rare, and strictly
% speaking there could be tabular and non-tabular versions
% of them, as well, but I've never encountered them.
%
% \emph{Textual figures}, also called \emph{old-style
% figures}, \emph{lowercase figures}, or even 
% \emph{medi\ae val figures}, are the ones that look sort 
% of\drmelip well, old-styled and lowercase.  They are
% centered on the ex-height, like lowercase letters, and
% some have ascenders, some descenders, and some neither,
% like lowercase letters.  They blend in with running text
% very well, whereas lining figures (which we'll get to in a
% moment) tend to stick out because they are all quite
% tall and often come in groups.  Most commonly, ``0,''
% ``1,'' and ``2'' have neither ascenders nor descenders;
% ``6'' and ``8'' have ascenders, and ``3,'' ``4,'' ``5,''
% ``7,'' and ``9'' have descenders; DRM follows this typical
% scheme in its roman types.  However, other systems have
% existed, particularly in France, where some famous fonts
% had an ascending rather than descending 3.  DRM has a
% non-typical set in its italic fonts, with a descending 3
% but an ascending 2:  \textit{0, 1, 2, 3,
% 4, 5, 6, 7, 8, 9}.
%
% Textual figures are the default in DRM in the roman and
% italic fonts.
%
% \emph{Lining figures}, also called \emph{titling figures}
% or \emph{modern figures}, line up at the baseline and all
% have a common height, typically something close to the
% height of capital letters or the ascenders of lowercase
% letters.  They look like this:  \liningnums{0123456789}.
% They're great when one wishes to draw extra attention to
% the figures, and practically mandatory when figures are
% being used with all caps; however, they throw off the
% color of the page and don't blend well with other running
% text.  In DRM, we get lining figures by using the
% \DescribeMacro{\liningnums}|\liningnums| command, which
% takes a single argument; namely, the number to be typeset
% in lining figures.  Each individual number can be accessed
% by command, as well, of the form |\liningzero|,
% |\liningone|, and so forth.
%
% Both of these types of figures can be \emph{tabular} or
% not.  This means, as a practical matter,
% \emph{monospaced}; that is, with tabular figures each
% digit takes up an identical horizontal space.  This is
% great for lining up numbers in columns, but produces
% rather bad spacing when used in running text.
%
% In DRM, the default textual (old-style) figures \emph{are
% not} tabular, while the lining figures \emph{are} tabular.
% It is possible to have tabular textual figures and
% proportional lining figures, but I've never seen much
% sense in either, as it seems that they defeat the purposes
% of their own particular form.
%
% DRM also has \emph{small-cap figures}, a relative rarity
% in the typographical world.  These are simply figures
% which match the style of the small caps fonts.  Neither
% textual nor lining figures work well with small caps;
% lining figure are too tall, and textual figures' ascenders
% and descenders don't fit with the relatively straight
% lines of small caps text.  So DRM has proportional
% (non-tabular), but short figures for small caps:
%
% \begin{center}
% \scshape\Huge Small caps 0123 figures.
% \end{center}
%
% Otherwise, it would like one of the following:
%
% \begin{center}
% \Huge\scshape Small caps \liningnums{0123} figures. \\
% Small caps \textup{0123} figures.
% \end{center}
%
% Neither of which looks very good.  There are similar
% digits for titling caps.  These are, of course, the
% defaults when using small or titling caps; if you need
% lining figures, you can still use |\liningnums|, and if
% you need textual figures, typeset them in normal roman
% text.
%
% Finally, DRM provides \emph{superior} and \emph{inferior}
% figures.  These are figures which are specially designed
% to appear in superscripted or subscripted text,
% respectively.  These avoid text color and spacing problems
% from forming superior figures merely from scaling and
% raising normal figures.  They look like the following:
%
% \begin{center}
% \LARGE\liningthree\textdrmsupfigs{0123456789} \\
% \liningthree\textdrminffigs{0123456789}
% \end{center}
%
% By default, in DRM (unless one of the |nodefault| options
% has been selected) footnote markers are made with superior
% figures.  Otherwise, superior figures must be selected
% with either \DescribeMacro{\drmsupfigs}|\drmsupfigs|, or
% the option with a single argument,
% \DescribeMacro{\textdrmsupfigs}|\textdrmsupfigs|.
% (Inferior numerals are selected with
% \DescribeMacro{\drminffigs}|\drminffigs| and the
% single-argument option,
% \DescribeMacro{\textdrminffigs}|\textdrminffigs|.)
% Inferior figures are typically useful for chemical
% formul\ae, but may conceivably find other uses, as well.
%
% These special superior figures do end up looking
% significantly better than merely superscripted-and-scaled
% footnote labels:
%
% \begin{center}
% \begin{tabular}{c|c}
% \LARGE\drmelip this fact.$^6$  As\drmelip & 
%	\LARGE\drmelip this fact.\textdrmsupfigs{6}  As\drmelip \\
% \end{tabular}
% \end{center}
%
% The superscripted and scaled version is too large, drawing
% more attention to itself than warranted (the purpose of a
% footnote is, after all, to provide citation without
% interrupting the flow of the text), not to mention that it
% protrudes above the height of the capitals and ascenders,
% making itself even more conspicuous; and the symbol itself
% is too thin, with lines almost spindly.  The superior
% figure, on the right, tops off at the height of the
% ascenders, and is specially designed to have lines of the
% same width as the body font.\footnote{Or as near as is
% possible and attractive, anyway; the conscious design is
% better than the automatic solution.}  This ensures an
% overall better appearance when these figures are used.
%
% \subsection{Symbols and Ornaments}
% \label{sub:symbols}
%
% One of DRM's strengths is its wide variety of symbols
% contained by default; rather than having to import
% separate fonts, or define macros to assemble common
% symbols out of their component parts, we can often simply
% use the symbols contained in DRM.
%
% Starting with the staples of traditional typography like
% the numero \DescribeMacro{\textnumero}(\textnumero) and
% the reference mark %
% \DescribeMacro{\textrefmark}(\textrefmark) to
% near-obsolete typesetting symbols like the asterism
% \DescribeMacro{\textasterism}(\textasterism) to more
% unique symbols like the international sign for radiation
% hazards \DescribeMacro{\textradiation}(\textradiation),
% DRM has something for most needs.
%
% \begin{center}
% \begin{longtable}{>{\Large}cp{0.4\textwidth}>{\raggedright\let\newline\\\arraybackslash}p{0.3\textwidth}}
% \toprule
% \multicolumn{3}{c}{Symbols of the DRM Font} \\
% \midrule
% \multicolumn{3}{c}{\itshape Religious Symbols} \\
% \midrule
% \textcrusadecross & |\textcrusadecross|\SpecialIndex{\textcrusadecross} & ``Crusader'' cross \\
% \textcrusadecrossoutline & |\textcrusadecrossoutline|\SpecialIndex{\textcrusadecrossoutline} &
% 	``Crusader'' cross in outline \\
% \textlatincross & |\textlatincross|\SpecialIndex{\textlatincross} & Latin cross \\
% \textlatincrossoutline & |\textlatincrossoutline|\SpecialIndex{\textlatincrossoutline} & Latin
% 	cross in outline \\
% \textgreekcross & |\textgreekcross|\SpecialIndex{\textgreekcross} & Greek cross \\
% \textgreekcrossoutline & |\textgreekcrossoutline|\SpecialIndex{\textgreekcrossoutline} & Greek
% 	cross in outline \\
% \textsaltirecross & |\textsaltirecross|\SpecialIndex{\textsaltirecross} & Saltire cross;
% 	cross of St.\ Andrew \\
% \textsaltirecrossoutline & |\textsaltirecrossoutline|\SpecialIndex{\textsaltirecrossoutline} &
% 	Saltire cross, cross of St.\ Andrew in outline \\
% \texteucharist & |\texteucharist|\SpecialIndex{\texteucharist} & Traditional
% 	representation of the Eucharist; chalice with Host and rays
% 	\\
% \textstardavid & |\textstardavid|\SpecialIndex{\textstardavid} & Traditional Star of
% 	David \\
% \textstardavidsolid & |\textstardavidsolid|\SpecialIndex{\textstardavidsolid} & Traditional
% 	Star of David, solid \\
% \textstardavidoutline & |\textstardavidoutline|\SpecialIndex{\textstardavidoutline} & 
% 	Traditional Star of David in outline \\
% \midrule
% \multicolumn{3}{c}{\itshape Genealogical Symbols} \\ 
% \midrule
% \textborn & |\textborn|\SpecialIndex{\textborn} & Symbol for born \\
% \textdied & |\textdied|\SpecialIndex{\textdied} & Symbol for died \\
% \textdivorced & |\textdivorced|\SpecialIndex{\textdivorced} & Symbol for divorced \\
% \textmarried & |\textmarried|\SpecialIndex{\textmarried} & Symbol for married \\
% \textleaf & |\textleaf|\SpecialIndex{\textleaf} & Leaf symbol \\
% \textmale & |\textmale|\SpecialIndex{\textmale} & Symbol for male \\
% \textfemale & |\textfemale|\SpecialIndex{\textfemale} & Symbol for female \\
% \midrule
% \multicolumn{3}{c}{\itshape Intellectual Property Symbols} \\ 
% \midrule
% \textregistered & |\textregistered|\SpecialIndex{\textregistered} & Registered mark \\
% \texttrademark & |\texttrademark|\SpecialIndex{\texttrademark} & Trademark sign \\
% \textservicemark & |\textservicemark|\SpecialIndex{\textservicemark} & Service mark sign \\
% \textsoundrecording & |\textsoundrecording|\SpecialIndex{\textsoundrecording} & Sound
% 	recording sign \\
% \textcopyright & |\textcopyright|\SpecialIndex{\textcopyright} & Copyright mark \\
% \textcopyleft & |\textcopyleft|\SpecialIndex{\textcopyleft} & Copyleft mark \\
% \midrule
% \multicolumn{3}{c}{\itshape Astronomical Symbols} \\ 
% \midrule
% \textsun & |\textsun|\SpecialIndex{\textsun} & Sun, Sol \\
% \textsunvar & |\textsunvar|\SpecialIndex{\textsunvar} & Variant Sun or Sol; with ray \\
% \textwaxcrescent & |\textwaxcrescent|\SpecialIndex{\textwaxcrescent} & Waxing crescent moon \\
% \textfullmoon & |\textfullmoon|\SpecialIndex{\textfullmoon} & Full moon \\
% \textwanecrescent & |\textwanecrescent|\SpecialIndex{\textwanecrescent} & Waning crescent moon \\
% \textnewmoon & |\textnewmoon|\SpecialIndex{\textnewmoon} & New moon \\
% \textmercury & |\textmercury|\SpecialIndex{\textmercury} & Mercury; Hermes \\
% \textearth & |\textearth|\SpecialIndex{\textearth}, |\textterra|\SpecialIndex{\textterra} & Earth, Terra \\
% \textearthvar & |\textearthvar|\SpecialIndex{\textearthvar},|\textterravar|\SpecialIndex{\textterravar} & Variant
% 	Earth, Terra \\
% \textmars & |\textmars|\SpecialIndex{\textmars} & Mars, Ares \\
% \textvenus & |\textvenus|\SpecialIndex{\textvenus} &  Venus, Aphrodite \\
% \textjupiter & |\textjupiter|\SpecialIndex{\textjupiter} & Jupiter, Jove, Zeus \\
% \textsaturn & |\textsaturn|\SpecialIndex{\textsaturn} & Saturn \\
% \texturanus & |\texturanus|\SpecialIndex{\texturanus} & Uranus \\
% \texturanusvar & |\texturanusvar|\SpecialIndex{\texturanusvar} & Variant Uranus \\
% \textneptune & |\textneptune|\SpecialIndex{\textneptune} & Neptune, Poseidon \\
% \textpluto & |\textpluto|\SpecialIndex{\textpluto} & Pluto \\
% \textplutovar & |\textplutovar|\SpecialIndex{\textplutovar} & Variant Pluto \\
% \textceres & |\textceres|\SpecialIndex{\textceres} & Ceres \\
% \textpallas & |\textpallas|\SpecialIndex{\textpallas} & Pallas \\
% \textjuno & |\textjuno|\SpecialIndex{\textjuno} &  Juno, Hera \\
% \textjunovar & |\textjunovar|\SpecialIndex{\textjunovar} & Variant Juno, Hera \\
% \textvesta & |\textvesta|\SpecialIndex{\textvesta} & Vesta \\
% \textvestavar & |\textvestavar|\SpecialIndex{\textvestavar} & Variant Vesta \\
% \textastraea & |\textastraea|\SpecialIndex{\textastraea} & Astr\ae a \\
% \textastraeavar & |\textastraeavar|\SpecialIndex{\textastraeavar} & Variant Astr\ae a \\
% \texthebe & |\texthebe|\SpecialIndex{\texthebe} & Hebe \\
% \textiris & |\textiris|\SpecialIndex{\textiris} & Iris \\
% \textaries & |\textaries|\SpecialIndex{\textaries}, |\textari|\SpecialIndex{\textari} & Aries \\
% \texttaurus & |\texttaurus|\SpecialIndex{\texttaurus}, |\texttau|\SpecialIndex{\texttau} & Taurus \\
% \textgemini & |\textgemini|\SpecialIndex{\textgemini}, |\textgem|\SpecialIndex{\textgem} & Gemini \\
% \textcancer & |\textcancer|\SpecialIndex{\textcancer}, |\textcnc|\SpecialIndex{\textcnc} & Cancer \\
% \textleo & |\textleo|\SpecialIndex{\textleo} & Leo \\
% \textvirgo & |\textvirgo|\SpecialIndex{\textvirgo}, |\textvir|\SpecialIndex{\textvir} & Virgo \\
% \textlibra & |\textlibra|\SpecialIndex{\textlibra}, |\textlib|\SpecialIndex{\textlib} & Libra \\
% \textscorpius & |\textscorpius|\SpecialIndex{\textscorpius}, |\textsco|\SpecialIndex{\textsco} & Scorpius \\
% \textsagittarius & |\textsagittarius|\SpecialIndex{\textsagittarius}, |\textsgr|\SpecialIndex{\textsgr} & Sagittarius \\
% \textcapricorn & |\textcapricorn|\SpecialIndex{\textcapricorn}, |\textcap|\SpecialIndex{\textcap} & Capricorn \\
% \textaquarius & |\textaquarius|\SpecialIndex{\textaquarius}, |\textaqr|\SpecialIndex{\textaqr} & Aquarius \\
% \textpisces & |\textpisces|\SpecialIndex{\textpisces}, |\textpsc|\SpecialIndex{\textpsc} & Pisces \\
% \textstar & |\textstar|\SpecialIndex{\textstar} & Star \\
% \textcomet & |\textcomet|\SpecialIndex{\textcomet} & Comet \\
% \textquadrature & |\textquadrature|\SpecialIndex{\textquadrature} & Quadrature \\
% \textopposition & |\textopposition|\SpecialIndex{\textopposition} & Opposition \\
% \textconjunction & |\textconjunction|\SpecialIndex{\textconjunction} & Conjunction \\
% \textascendingnode & |\textascendingnode|\SpecialIndex{\textascendingnode} & Ascending node \\
% \textdescendingnode & |\textdescendingnode|\SpecialIndex{\textdescendingnode} & Descending node \\
% \midrule
% \multicolumn{3}{c}{\itshape Currency Symbols} \\ 
% \midrule
% \textdollarsign & |\textdollarsign|\SpecialIndex{\textdollarsign} & Dollar sign \\
% \textolddollarsign & |\textolddollarsign|\SpecialIndex{\textolddollarsign} & Old-style dollar
% 	sign; double-slashed dollar sign \\
% \textcentsign & |\textcentsign|\SpecialIndex{\textcentsign} & Cent sign \\
% \textoldcentsign & |\textoldcentsign|\SpecialIndex{\textoldcentsign} & Old-style cent sign;
% 	diagonally slashed cent sign \\
% \textpoundsterling & |\textpoundsterling|\SpecialIndex{\textpoundsterling} & British pound
% 	sterling sign \\
% \textoldpoundsterling & |\textoldpoundsterling|\SpecialIndex{\textoldpoundsterling}, |\textlira|\SpecialIndex{\textlira} & Old-style
% 	British pound sterling sign; double-slashed British pound
% 	sterling sign; Italian lira sign \\
% \texteuro & |\texteuro|\SpecialIndex{\texteuro} & Euro sign \\
% \textyen & |\textyen|\SpecialIndex{\textyen} & Japanese yen sign \\
% \textbaht & |\textbaht|\SpecialIndex{\textbaht} & Thai baht sign \\
% \textcolon & |\textcolon|\SpecialIndex{\textcolon} & Costa Rican, Salvadoran colon sign \\
% \textdong & |\textdong|\SpecialIndex{\textdong} & Vietnamese dong sign \\
% \textflorin & |\textflorin|\SpecialIndex{\textflorin} & Florin sign \\
% \textguarani & |\textguarani|\SpecialIndex{\textguarani} & Uruguayan guarani sign \\
% \textnaira & |\textnaira|\SpecialIndex{\textnaira} & Nigerian naira sign \\
% \textpeso & |\textpeso|\SpecialIndex{\textpeso}, |\textruble|\SpecialIndex{\textruble} & Mexican peso sign; Russian ruble
% 	sign \\
% \textwon & |\textwon|\SpecialIndex{\textwon} & Won sign \\
% \textcurrency & |\textcurrency|\SpecialIndex{\textcurrency} & Generic currency \\
% \midrule
% \multicolumn{3}{c}{\itshape Roman Numerals} \\ 
% \midrule
% \romone & |\romone|\SpecialIndex{\romone} & Roman numeral one \\
% \romfive & |\romfive|\SpecialIndex{\romfive} & Roman numeral five \\
% \romten & |\romten|\SpecialIndex{\romten} & Roman numeral ten \\
% \romfifty & |\romfifty|\SpecialIndex{\romfifty} & Roman numeral fifty \\
% \romhundred & |\romhundred|\SpecialIndex{\romhundred} & Roman numeral hundred \\
% \romfivehundred & |\romfivehundred|\SpecialIndex{\romfivehundred} & Roman numeral hundred \\
% \romthousand & |\romthousand|\SpecialIndex{\romthousand} & Roman numeral thousand \\
% \romanize{1651} & |\romanize{1668}| & Convert Indo-arabic
% 	numeral to Roman numerals \\
% \midrule
% \multicolumn{3}{c}{\itshape Lining Numerals} \\ 
% \midrule
% \liningzero & |\liningzero|\SpecialIndex{\liningzero} & Lining numeral 0 \\
% \liningone & |\liningone|\SpecialIndex{\liningone} & Lining numeral 1 \\
% \liningtwo & |\liningtwo|\SpecialIndex{\liningtwo} & Lining numeral 2 \\
% \liningthree & |\liningthree|\SpecialIndex{\liningthree} & Lining numeral 3 \\
% \liningfour & |\liningfour|\SpecialIndex{\liningfour} & Lining numeral 4 \\
% \liningfive & |\liningfive|\SpecialIndex{\liningfive} & Lining numeral 5 \\
% \liningsix & |\liningsix|\SpecialIndex{\liningsix} & Lining numeral 6 \\
% \liningseven & |\liningseven|\SpecialIndex{\liningseven} & Lining numeral 7 \\
% \liningeight & |\liningeight|\SpecialIndex{\liningeight} & Lining numeral 8 \\
% \liningnine & |\liningnine|\SpecialIndex{\liningnine} & Lining numeral 9 \\
% \liningnums{3091} & |\liningnums{3091}| & Convert figures
%	into lining figures \\
% \midrule
% \multicolumn{3}{c}{\itshape Traditional and Innovative Typography} \\ 
% \midrule
% \textnumero & |\textnumero|\SpecialIndex{\textnumero} & Numero \\
% \textrefmark & |\textrefmark|\SpecialIndex{\textrefmark} & Reference mark \\
% \textasterism & |\textasterism|\SpecialIndex{\textasterism} & Asterism \\
% \textfeminineordinal & |\textfeminineordinal|\SpecialIndex{\textfeminineordinal} & Feminine
% 	Ordinal \\
% \textmasculineordinal & |\textmasculineordinal|\SpecialIndex{\textmasculineordinal} & Masculine
% 	Ordinal \\
% \textsupone & |\textsupone|\SpecialIndex{\textsupone} & Superscript 1; superior
% digit 1 \\
% \textsuptwo & |\textsuptwo|\SpecialIndex{\textsuptwo} & Superscript 2; superior
% digit 2 \\
% \textsupthree & |\textsupthree|\SpecialIndex{\textsupthree} & Superscript 3; superior
% digit 3\\
% \textpilcrowsolid & |\textpilcrowsolid|\SpecialIndex{\textpilcrowsolid} & Solid-lined
% 	pilcrow \\
% \textpilcrowoutline & |\textpilcrowoutline|\SpecialIndex{\textpilcrowoutline} & Outlined
% 	pilcrow \\
% \textsection & |\textsection|\SpecialIndex{\textsection} & Section mark \\
% \textdagger & |\textdagger|\SpecialIndex{\textdagger}, |\textdag|\SpecialIndex{\textdag}, |\dag|\SpecialIndex{\dag} & Dagger \\
% \textdbldagger & |\textdbldagger|\SpecialIndex{\textdbldagger}, |\textdbldag|\SpecialIndex{\textdbldag}, |\dbldag|\SpecialIndex{\dbldag} & 
% 	Double dagger \\
% \textpipe & |\textpipe|\SpecialIndex{\textpipe} & Pipe \\
% \textbrokenpipe & |\textbrokenpipe|\SpecialIndex{\textbrokenpipe} & Broken pipe \\
% \textrecipe & |\textrecipe|\SpecialIndex{\textrecipe} & Recipe mark \\
% \textintbang & |\textintbang|\SpecialIndex{\textintbang} & Interrobang \\
% \textopenintbang & |\textopenintbang|\SpecialIndex{\textopenintbang} & Opening interrobang \\
% \midrule
% \multicolumn{3}{c}{\itshape Text-mode Math Symbols} \\ 
% \midrule
% \textprime & |\textprime|\SpecialIndex{\textprime} & Single prime mark \\
% \textdoubleprime & |\textdoubleprime|\SpecialIndex{\textdoubleprime} & Double prime mark \\
% \texttripleprime & |\texttripleprime|\SpecialIndex{\texttripleprime} & Triple prime mark \\
% \textsqrt & |\textsqrt|\SpecialIndex{\textsqrt} & Square root sign; radical \\
% \textquarter & |\textquarter|\SpecialIndex{\textquarter} & One-quarter fraction,
% 	slanted \\
% \texthalf & |\texthalf|\SpecialIndex{\texthalf} & One-half fraction, slanted \\
% \textthreequarters & |\textthreequarters|\SpecialIndex{\textthreequarters} & Three-quarters
% 	fraction, slanted \\
% \textthird & |\textthird|\SpecialIndex{\textthird} & One-third fraction, slanted \\
% \texttwothirds & |\texttwothirds|\SpecialIndex{\texttwothirds} & Two-thirds fraction,
% 	slanted \\
% \textperbiqua & |\textperbiqua|\SpecialIndex{\textperbiqua}, |\textpermille|\SpecialIndex{\textpermille} & Perbiqua,
% 	permille, per thousand \\
% \textpertriqua & |\textpertriqua|\SpecialIndex{\textpertriqua}, |\textpertenmille|\SpecialIndex{\textpertenmille} &
% 	Pertriqua, per ten thousand \\
% \textequals & |\textequals|\SpecialIndex{\textequals} & Equals sign \\
% \textslash & |\textslash|\SpecialIndex{\textslash} & Forward slash \\
% \texttimes & |\texttimes|\SpecialIndex{\texttimes} & Multiplication; times \\
% \textdiv & |\textdiv|\SpecialIndex{\textdiv} & Division sign \\
% \textuparrow & |\textuparrow|\SpecialIndex{\textuparrow} & Upward-pointing arrow \\
% \textdownarrow & |\textdownarrow|\SpecialIndex{\textdownarrow} & Downward-pointing arrow \\
% \textleftarrow & |\textleftarrow|\SpecialIndex{\textleftarrow} & Left-pointing arrow \\
% \textrightarrow & |\textrightarrow|\SpecialIndex{\textrightarrow} & Right-pointing arrow \\
% \midrule
% \multicolumn{3}{c}{\itshape Warning Signs} \\ 
% \midrule
% \textradiation & |\textradiation|\SpecialIndex{\textradiation} & Radiation warning sign \\
% \textradiationnocircle & |\textradiationnocircle|\SpecialIndex{\textradiationnocircle} &
% 	Radiation warning sign, no enclosing circle \\
% \textbiohazard & |\textbiohazard|\SpecialIndex{\textbiohazard} & Biohazard warning sign \\
% \textbiohazardnocircle & |\textbiohazardnocircle|\SpecialIndex{\textbiohazardnocircle} &
% 	Biohazard warning sign, no enclosing circle \\
% \texthighvoltage & |\texthighvoltage|\SpecialIndex{\texthighvoltage} & High voltage warning
% 	sign \\
% \texthighvoltagenotriangle & |\texthighvoltagenotriangle|\SpecialIndex{\texthighvoltagenotriangle} &
% 	High voltage warning sign, no enclosing triangle \\
% \textgeneralwarning & |\textgeneralwarning|\SpecialIndex{\textgeneralwarning} & General
% 	warning sign \\
% \midrule
% \multicolumn{3}{c}{\itshape Bullets and Other Marks} \\ 
% \midrule
% \textbullet & |\textbullet|\SpecialIndex{\textbullet} & Solid circular bullet \\
% \textopenbullet & |\textopenbullet|\SpecialIndex{\textopenbullet} & Open circular bullet \\
% \textheart & |\textheart|\SpecialIndex{\textheart} & Solid heart \\
% \textopenheart & |\textopenheart|\SpecialIndex{\textopenheart} & Open heart \\
% \texteighthnote & |\texteighthnote|\SpecialIndex{\texteighthnote} & Eighth note \\
% \textdiamond & |\textdiamond|\SpecialIndex{\textdiamond} & Solid diamond; solid lozenge \\
% \textopendiamond & |\textopendiamond|\SpecialIndex{\textopendiamond}, |\textlozenge|\SpecialIndex{\textlozenge} & Open 
%	diamond; open lozenge \\
% \textdegree & |\textdegree|\SpecialIndex{\textdegree} & Degree symbol \\
% \textdegreec & |\textdegreec|\SpecialIndex{\textdegreec} & Degrees Celsius \\
% \texttilde & |\texttilde|\SpecialIndex{\texttilde}, |\tilde|\SpecialIndex{\tilde} & Tilde \\
% \midrule
% \multicolumn{3}{c}{\itshape Ornaments and Fleurons} \\ 
% \midrule
% \textrightupfleuron & |\textrightupfleuron|\SpecialIndex{\textrightupfleuron} &
% Rightward-pointing, upward fleuron \\
% \textrightdownfleuron & |\textrightdownfleuron|\SpecialIndex{\textrightdownfleuron} &
% Rightward-pointing, downward fleuron \\
% \textleftupfleuron & |\textleftupfleuron|\SpecialIndex{\textleftupfleuron} &
% Leftward-pointing, upward fleuron \\
% \textleftdownfleuron & |\textleftdownfleuron|\SpecialIndex{\textleftdownfleuron} &
% Leftward-pointing, downward fleuron \\
% \textupleftfleuron & |\textupleftfleuron|\SpecialIndex{\textupleftfleuron} &
% Upward-pointing, leftward fleuron \\
% \textuprightfleuron & |\textuprightfleuron|\SpecialIndex{\textuprightfleuron} &
% Upward-pointing, rightward fleuron \\
% \textdownrightfleuron & |\textdownrightfleuron|\SpecialIndex{\textdownrightfleuron} &
% Downward-pointing, rightward fleuron \\
% \textdownleftfleuron & |\textdownleftfleuron|\SpecialIndex{\textdownleftfleuron} &
% Downward-pointing, leftward fleuron \\
% \textsquaretulip & |\textsquaretulip|\SpecialIndex{\textsquaretulip} &
% Square of four tulips, facing up and down \\
% \textsquaretulipside &
% |\textsquaretulipside|\SpecialIndex{\textsquaretulipside} & Square of four tulips,
% facing left and right \\
% \textupdoubletulip & |\textupdoubletulip|\SpecialIndex{\textupdoubletulip} &
% Double tulips, facing upward \\
% \textdowndoubletulip &
% |\textdowndoubletulip|\SpecialIndex{\textdowndoubletulip} & Double tulips, facing
% downward \\
% \textrightdoubletulip &
% |\textrightdoubletulip|\SpecialIndex{\textrightdoubletulip} & Double tulips, facing
% righward \\
% \textleftdoubletulip &
% |\textleftdoubletulip|\SpecialIndex{\textleftdoubletulip} & Double tulips, facing
% leftward \\
% \textupleftcornertulip &
% |\textupleftcornertulip|\SpecialIndex{\textupleftcornertulip} & Single corner-facing
% tulip, for upper left corners \\
% \textuprightcornertulip &
% |\textuprightcornertulip|\SpecialIndex{\textuprightcornertulip} & Single corner-facing
% tulips, for  upper right corners \\
% \textlowleftcornertulip &
% |\textlowleftcornertulip|\SpecialIndex{\textlowleftcornertulip} & Single corner-facing
% tulips, for lower left corners \\
% \textlowrightcornertulip &
% |\textlowrightcornertulip|\SpecialIndex{\textlowrightcornertulip} & Single
% corner-facing tulip, for lower right corners  \\
% \textupsingletuliplong & |\textupsingletuliplong|\SpecialIndex{\textupsingletuliplong} & Single tulip,
% upward-facing \\
% \textdownsingletuliplong & |\textdownsingletuliplong|\SpecialIndex{\textdownsingletuliplong} & Single
% tulip, downward-facing \\
% \textleftsingletuliplong & |\textleftsingletuliplong|\SpecialIndex{\textleftsingletuliplong} & Single
% tulip, leftward-facing \\
% \textrightsingletuliplong & |\textrightsingletuliplong|\SpecialIndex{\textrightsingletuliplong} & Single
% tulip, rightward-facing \\
% \textupsingletulip & |\textupsingletulip|\SpecialIndex{\textupsingletulip} & Single tulip,
% upright \\
% \textdownsingletulip & |\textdownsingletulip|\SpecialIndex{\textdownsingletulip} & Single
% tulip, downward \\
% \textleftsingletulip & |\textleftsingletulip|\SpecialIndex{\textleftsingletulip} & Single
% tulip, leftward \\
% \textrightsingletulip & |\textrightsingletulip|\SpecialIndex{\textrightsingletulip} & Single
% tulip, rightward \\
% \spearright & |\spearright|\SpecialIndex{\spearright} &
% Rightward-pointing spear head \\
% \spearleft & |\spearleft|\SpecialIndex{\spearleft} &
% Leftward-pointing spear head \\
% \horizspearext &
% |\horizspearext|\SpecialIndex{\horizspearext} & Extension
% piece for horizontal shafts \\
% \spearup & |\spearup|\SpecialIndex{\spearup} &
% Upward-pointing spear head \\
% \speardown & |\speardown|\SpecialIndex{\speardown} &
% Downward-pointing spear head \\
% \vertspearext & |\vertspearext|\SpecialIndex{\vertspearext}
% & Extension piece for vertical shafts \\
% \fleurdelis & |\fleurdelis|, |\fleurdelys|\SpecialIndex{\fleurdelis}
%	\SpecialIndex{\fleurdelys} & Fleur-de-lis \\
% \fleurdelisdown &
% |\fleurdelisdown|, |\fleurdelysdown|\SpecialIndex{\fleurdelisdown}
%	\SpecialIndex{\fleurdelys} & Fleur-de-lis, downward \\
% \fleurdelisleft &
% |\fleurdelisleft|, |\fleurdelysleft|\SpecialIndex{\fleurdelisleft}
%	\SpecialIndex{\fleurdelys} & Fleur-de-lis, leftward \\
% \fleurdelisright &
% |\fleurdelisright|, |\fleurdelysright|\SpecialIndex{\fleurdelisright}
%\SpecialIndex{\fleurdelys} & Fleur-de-lis, rightward \\
% \LARGE\woundcordleftext & |\woundcordleftext|\SpecialIndex{\woundcordleftext} 
%	& Wound cord, leftward facing, extender \\
% \LARGE\woundcordrightext & 
% 	|\woundcordrightext|\SpecialIndex{\woundcordrightext} & 
%	Wound cord, rightward facing, extender \\
% \LARGE\woundcordleftend & |\woundcordleftend|\SpecialIndex{\woundcordleftend} 
%	& Wound cord, left end \\
% \LARGE\woundcordrightend & 
%	|\woundcordrightend|\SpecialIndex{\woundcordrightend} & 
%	Wound cord, right end \\
% \LARGE\woundcordleftendinv & 
%	|\woundcordleftendinv|\SpecialIndex{\woundcordleftendinv} & 
%	Wound cord, left end, inverted \\
% \LARGE\woundcordrightendinv & 
%	|\woundcordrightendinv|\SpecialIndex{\woundcordrightendinv} & 
%	Wound cord, right end, inverted \\
% \end{longtable}
% \end{center}
%
% \label{page:orncomment}
% These ornaments are often quite useful for decorative
% purposes, though textual ornaments are too often neglected
% these days.  (The sturdy |adforn| and intricate
% |psvectorian| packages for \LaTeX\ are notable and
% admirable exceptions.)  The possibilities with even just a
% few decorative shapes are endless.
% 
% \subsection{Special Symbol and Ornamental Commands} 
% \label{sub:ornamentals}
%
% \lettrine{B}{ecause typography} is an ancient art full of arcane
% knowledge, there are some things that simply won't fit
% into the general rules.  As a result, DRM offers a few
% interesting tidbits that your author hasn't found, or
% hasn't found useful, elsewhere.  We start with a few
% commands for using the textual ornaments DRM provides,
% followed by some more mundane but still useful
% typographical tools.
%
% \subsubsection{Ornamental Commands}
% \label{subsub:orncommands}
%
% Having just mentioned the great decorative utility of
% old-fashioned textual ornaments,\footnote{\textit{See
% supra} at \pageref{page:orncomment}.} it would be remiss not to
% offer some tools for actually using such ornaments short
% of entering them in and designing interlocking boxes by
% hand.  Ornaments, being inherently decorative rather than
% systematic, are not always subject to automation; but some
% limited applications can be, and DRM tries to offer some
% help with them.
%
% DRM offers |\tulipframe|\DescribeMacro{\tulipframe}, which
% frames a title in decorative tulip fleurons:
%
% \vskip1em%
% 
% \hbox to\linewidth{\hfil|\tulipframe{\texttc{Example}}|\hfil}%
% {\LARGE\tulipframe{\texttc{Example}}}
%
% \vskip1em%
% 
% The nature of the tulip fleurons in DRM's symbol font is
% such that these frames can be extended or shrunk as one
% wishes.  |\tulipframe|, alas, is not that intelligent; it
% doesn't grow or shrink with the text, but simply sits as
% it is.  Doing better than this will require box-fiddling
% by hand.  However, since DRM offers vertical \emph{and}
% horizontal tulip ornaments, it's possible to have
% ornamental frames of any height or width.
%
% DRM also offers an extremely flexible \emph{rule system},
% allowing the creation of vertical and horizontal rules of
% any length, out of any characters, in the beginning, the
% middle figures, and the end.  Meet
% \DescribeMacro{\extrule}|\extrule|, or \emph{extensible
% rule}, which can produce rules with whatever characters
% you'd like.
%
% |\extrule| requires five arguments, as shown below:
%
% \begin{center}
% \cmd{\extrule} \marg{orient} \marg{len}
% \marg{start} \marg{end} \marg{ext}
% \end{center}
%
% \begin{description}
% \item[orient] The rule's \emph{orientation}.  This can
% take the value \marg{h}, for \emph{horizontal}, or
% \marg{v}, for \emph{vertical}.
% \item[len] The rule's \emph{length}.  This will be the
% total length of the rule, including the start and end
% characters.  It can be passed in any form understood by
% \eTeX's |\numexpr| and |\dimexpr|, meaning that you can 
% give it formul\ae, such as |{0.2\linewidth}|.
% \item[start] The first character in the rule; this means
% either the left character in a horizontal rule, or the
% bottom character in a vertical rule.
% \item[end] The last character in the rule, either the
% right in a horizontal or the top in a vertical.
% \item[ext] The extension character; this is the character
% which will be repeated  until the rule is the appropriate
% length.
% \end{description}
%
% DRM offers several useful characters for producing such
% rules, which are designed to line up properly and thus
% produce attractive decorative rules.  Among these are the
% \emph{spear characters}, and we will demonstrate their use
% with a couple of sample rules:
%
% \begin{center}
% \begin{tabular}{m{0.5\linewidth}c}
% |\extrule{h}{\linewidth/2}{\spearleft}|
% |{\spearright}{\horizspearext}| &
% \extrule{h}{\linewidth/6}{\spearleft}
% {\spearright}{\horizspearext} \\
% |\extrule{v}{\linewidth/6}{\speardown}|
% |{\spearup}{\vertspearext}| &
% \extrule{v}{\linewidth/6}{\speardown}
% {\spearup}{\vertspearext} \\
% \end{tabular}
% \end{center}
%
% Of course, these are typically more useful when longer, as
% in the rule below, which is equal to the |\linewidth|:
%
% \extrule{h}{\linewidth}{\spearleft}{\spearright}{\horizspearext}
%
% As is evident, |\extrule| also suppresses indentation,
% which is almost certainly the right choice.  If you want
% an indent with it, it's easy enough to put one in
% explicitly.
%
% While characters like these, designed to line up
% correctly, are naturally the most likely candidates for
% such rules, you can use any characters you'd like, which
% can sometimes lead to some interesting choices:
%
% |\extrule{h}{\linewidth/2}{\textleftarrow}{\textrightarrow}{\dag}|
%
% \begin{center}
% \extrule{h}{\linewidth/2}{\textleftarrow}{\textrightarrow}{\dag}
% \end{center}
%
% This is a pretty absurd example, of course, but it's
% likely that better ones could and will be devised.
%
% And what about when we desire a special character in the
% \emph{middle} of the rule?  Use \emph{two} |\extrule|s and
% put the symbol you want in the middle between them; make
% sure you comment out the end of your first and second
% lines, so as not to introduce any extraneous spaces:
% 
% \begin{verbatim}
% \extrule{h}{\linewidth/2}{\spearleft}{}{\horizspearext}%
% \textbigcircle%
% \extrule{h}{\linewidth/2}{}{\spearright}{\horizspearext}%
% \end{verbatim}
% \begin{center}
% \extrule{h}{\linewidth/4}{\spearleft}{}{\horizspearext}\textbigcircle\extrule{h}{\linewidth/4}{}{\spearright}{\horizspearext}
% \end{center}
%
% There's no reason we can't put more than one character into
% these slots, as well, if we want to mix them with some
% different characters:
%
% \begin{verbatim}
% \extrule{h}{\linewidth/2}{\spearleft\raisebox{0.8pt}{\textpipe}}%
%		{\raisebox{0.8pt}{\textpipe}\spearright}{\horizspearext}%
% \end{verbatim}
% \begin{center}
% \extrule{h}{\linewidth/2}{\spearleft\raisebox{0.8pt}{\textpipe}}%
%	{\raisebox{0.8pt}{\textpipe}\spearright}{\horizspearext}%
% \end{center}
%
% Some other useful characters for decorative rules are the
% ``wound cord'' characters:
%
% \begin{center}
% \Huge
% \extrule{h}{\linewidth/2}{\drmsym{\char'323}}{\drmsym{\char'322}}{\drmsym{\char'324}}
% \extrule{h}{\linewidth/2}{\drmsym{\char'320}}{\drmsym{\char'321}}{\drmsym{\char'317}}
% \end{center}
% \vskip1em%
%
% Finally, these rules can often form very dignified page
% borders.  The border on this page, for example, was formed
% very simply by the following (using
% |\usepackage[absolute]{textpos}|):
% \setlength{\TPHorizModule}{\linewidth}
% \begin{textblock}{1}(0.22,2)
% \extrule{v}{\textheight/6*7}{\textbigcircle}{\spearup}{\vertspearext}%
% \hskip-1.3em%
% \extrule{h}{8\textwidth/6}{}{\spearright}{\horizspearext}%
% \end{textblock}
%
% \begin{verbatim}
% \setlength{\TPHorizModule}{\linewidth}
% \begin{textblock}{1}(0.22,2)
% \extrule{v}{\textheight/6*7}{\textbigcircle}{\spearup}{\vertspearext}%
% \hskip-1.3em%
% \extrule{h}{8*\textwidth/6}{}{\spearright}{\horizspearext}%
% \end{textblock}
% \end{verbatim}
%
% This takes a little hand-tuning (e.g., the |\hskip| prior
% to the horizontal |\extrule|, and the offset in the
% parentheses), but once done, it can look quite nice.
%
% \subsubsection{Ellipses}
% \label{subsub:ellipses}
%
% DRM also has some unreasonably configurable ellipses.
% Your author included these because he's often been
% displeased by the default ellipsis options.  (Of course,
% there is the excellent |ellipsis| package; but why not fix
% the problem here, when I've got the chance?)  DRM offers
% two ellipsis commands, \DescribeMacro{\drmelip}|\drmelip|,
% which gives a three-dot ellipsis, and
% \DescribeMacro{\drmfelip}|\drmfelip|, which gives a
% four-dot ellipsis.
%
% I was always taught then when an ellipsis occurs after a
% period, four dots should be used, the first dot being the
% period itself and the next three being the ellipsis.
% However, using |\ldots| and similar commands after a
% period always seems to result in spacing that was subtlely
% (or not-so-subtlely) off.  So DRM tries to fix that
% problem with these commands.
%
% The default behavior of the two:
%
% \begin{table}[htbp]
% \begin{center}\Large
% \begin{tabular}{ll}
% |\drmelip| & |Trying out\drmelip the ellipsis.| \\
% {} & Trying out\drmelip the ellipsis. \\
% |\drmfelip| & |\drmelip and so on\drmfelip| \\
% {} & \drmelip and so on\drmfelip \\
% \end{tabular}
% \caption{A demonstration of DRM's two types of ellipses.}
% \end{center}
% \end{table}
%
% It goes without saying, of course, that these ellipses
% won't break across lines.
%
% There are four parameters that govern how these ellipses
% actually appear:  the space before the ellipsis starts,
% the space in between the ellipsis characters, the space
% after the ellipsis ends, and the character used for the
% ellipsis.  Each of these parameters are configurable.
%
% \DescribeMacro{\drmelipgap}|\drmelipgap| is a \LaTeX\
% length which determines how much space is between each
% ellipsis character; reset it, if you like, with the
% standard |\setlength| command.  By default, it is just
% under three points (2.9, to be precise.)
%
% \DescribeMacro{\drmelipbef}|\drmelipbef| and
% \DescribeMacro{\drmelipaft}|\drmelipaft| are, as the names
% imply, the lengths which govern the amount of space
% before and after the ellipsis.  Reset them with the
% \LaTeX\ |\setlength| command.  By default, they are 2.4
% points and 1.4 points, respectively.
%
% Finally, the \DescribeMacro{\drmelipchar}|\drmelipchar|
% macro tells \LaTeX\ what character is used for the
% ellipsis.  By default, this is |.|, but it can be
% |\def|ed or |\renewcommand|ed to be anything you like.
% Always wanted an ellipse made out ampersands for some
% reason?  Or perhaps one made out of daggers?
%
% \vskip2em%
% \hbox to\linewidth{\hfil|\def\drmelipchar{\dag}\drmelip|\hfil}
% \hbox to\linewidth{\hfil\def\drmelipchar{\dag}\drmelip\hfil}
% \vskip2em%
%
% \def\drmelipchar{.}%
% It's probably wise not to abuse this, but it's good for a
% little fun sometimes, and it's easier to use (though
% obviously much less flexible) than \TeX's |\dotfill|
% incantations.
%
% It is occasionally useful, however; e.g., some legal
% writing makes ellipses out of asterisks:
%
% \vskip2em%
% \hbox % to\linewidth{\hfil|\def\drmelipchar{$^*$}The decision is hereby\drmelip reversed.|\hfil}
% \hbox to\linewidth{\hfil\def\drmelipchar{$^*$}The decision is
% hereby\drmelip reversed.\hfil}
% \vskip2em%
%
% \def\drmelipchar{.}%
% So once in a while, we might actually be able to use this
% feature for something other than its novelty value.
%
% \subsubsection{Decorative Initials}
% \label{subsub:decorinit}
%
% DRM, as of v3.0, provides for decorative initials.  These
% are not traditional decorative initials, however, with
% intricate patterns provided individually for each letter.
% They are, rather, formed with a single background pattern,
% with the necessary letter superimposed.  The goal is to
% make the background pattern interchangeable.  The color of
% that background pattern and the color of the foreground
% letter can be controlled separately.  Despite the single
% background pattern, therefore, this provides for a
% remarkable degree of flexibility.
%
% \DescribeMacro{\drmdecinit}|\drmdecinit| is the name of
% the game here, a command which takes five arguments, all
% of which are mandatory.  
%
% \begin{center}
% \cmd{\drmdecinit} \marg{width} \marg{height}
% \marg{bgcolor} \marg{fgcolor} \marg{fgchar}
% \end{center}
%
% These are largely self-explanatory, so a few examples will
% likely do.  Note that |drm| uses the excellent |gmp|
% package to get the \MP\ code to be part of the \LaTeX\
% code, allowing \LaTeX\ to control significant parts of the
% formatting.  This means that one will have to run a shell
% script along with compiling the document, similarly to
% |bibtex|, |makeindex|, or a host of others.
%
% \begin{parcolumns}{2}
% \colchunk[1]{
% \begin{spverbatim}\lettrine[lines=4,noindent=0pt, 
% findent=-1em]{\drmdecinit{40pt}
% {40pt}{blue}{(.625,0,0)}{L}}{orem
% ipsum}\end{spverbatim}}\colchunk[2]{
% \raggedright\begin{spverbatim}\lettrine[lines=4,noindent=0pt,
% findent=-1em]{\drmdecinit{40pt}
% {40pt}{blue}{red}{L}}{orem ipsum}\end{spverbatim}}\end{parcolumns}
% \begin{parcolumns}{2}
% \colchunk[1]{
% \lettrine[lines=4, findent=-1em,nindent=0pt]
%	{\drmdecinit{40pt}{40pt}{blue}{(.625,0,0)}{L}}{orem ipsum}
% dolor sit amet, consectetur adipiscing elit.  Ut porttitor
% libero lacus, a rhoncus dolor finibus vel.  Morbi
% volputate condimentum orna\-re.  In scelerisque 
% aliquam\drmfelip
% }\colchunk[2]{
% \lettrine[lines=4,nindent=0pt,findent=-1em]
%	{\drmdecinit{40pt}{40pt}{blue}{red}{L}}{orem ipsum}
% dolor sit amet, consectetur adipiscing elit.  Ut porttitor
% libero lacus, a rhoncus dolor finibus vel.  Morbi
% volputate condimentum orna\-re.  In scelerisque
% aliquam\drmfelip
% }\end{parcolumns}
%
% \vskip\baselineskip
% The colors are \MP\ colors; unfortunately, this means that
% we can use only ``black,'' ``white,'' ``red,'' ``green,''
% and ``blue'' by name.  However, any valid \MP\ color
% specification will work.  In the example above, for
% example, to get a darker red, one could use |.4red|, or
% one could specify colors in RGB notation, as shown above
% on the left.  Note that, when doing this latter, the
% parentheses are necessary.
%
% These decorative initials lend themselves to some other,
% sometimes unexpected, uses.  For example, decorative
% enumerates.  It is best to use lining figures rather than
% textual figures for this.
%
% DRM offers the command 
% \DescribeMacro{\drmdecinitfont}|\drmdecinitfont|, which is 
% the font which DRM uses for the decorative initials.  Because
% of the internals of the |gmp| package, the simple name of
% the font can't be inserted here; it must be defined in a
% particular way.  The default is, of course, to use DRM,
% and is defined thus:
%
% \begin{spverbatim}
% \def\drmdecinitfont{\unexpanded{\font\drminitfontcom=drm10}}%
% \end{spverbatim}
%
% In other words, one must define the fonts in the
% old-fashioned \TeX\ way.  The above is the default; so
% whenever you've changed it for some reason, you can get it
% back to the above by entering
% \DescribeMacro{\drmdecinitfontdefault}|\drmdecinitfontdefault|;
% this simply restores the default definition as given
% above.
%
% For the ornate enumerations, we can simply redefine
% |\drmdecinitfont| to use the lining figures from |drmsym|,
% which conveniently are located at precisely the code
% points that one would expect them.  Simply issue:
%
% \begin{quote}\begin{spverbatim}
% \def\drmdecinitfont{\unexpanded{\font\drminitfontcom=drmsym10} %}%
% \renewcommand{\labelenumi}{%
% 	\drmdecinit{14pt}{14pt}{blue}{red}{\theenumi}}
% \end{spverbatim}\end{quote}
%
% This redefines |\drmdecinitfont| to use |drmsym| rather
% than simply |drm|, then redefines the enumerate labels to
% be DRM decorative initials, resulting in the following:
%
%\def\drmdecinitfont{\unexpanded{\font\drminitfontcom=drmsym10} %}%
% \renewcommand{\labelenumi}{%
% 	\drmdecinit{14pt}{14pt}{blue}{red}{\theenumi}}
% \begin{quote}
% \begin{enumerate}
% \item The first item.
% \item The second item.
% \end{enumerate}
% \end{quote}
% \drmdecinitfontdefault
%
% \renewcommand{\labelenumi}{\theenumi.}
% Any character whatever can be used this way, provided that
% the font is correctly selected.  Doubtlessly many creative
% uses for this ability will be found.
%
% One trick, almost necessary when using these on any
% significant scale, is a macro to make them less typing.
% For example, to use them as four-line lettrines using
% Daniel Flipo's excellent |lettrine| package:
%
% \def\declettrine#1#2{%
% 	\lettrine[lines=4,nindent=0pt,findent=-1em]%
%	{\drmdecinit{40pt}{40pt}{blue}{red}{#1}}{#2}%
%	}%
% \begin{spverbatim}
% \def\declettrine#1#2{%
% 	\lettrine[lines=4,nindent=0pt,findent=-1em]%
%	{\drmdecinit{40pt}{40pt}{blue}{red}{#1}}{#2}%
%	}%
% \end{spverbatim}
%
% Rather than having to type the whole of the above each
% time now, one can do it in a more natural manner:
%
% \begin{center}
% \begin{tabular}{p{0.4\linewidth}p{0.4\linewidth}}
% \begin{spverbatim}
%\declettrine{L}{orem ipsum} dolor sit amet, consectetur
%adipiscing elit.  Ut porttitor libero lacus, a rhoncus dolor 
%finibus vel.  Morbi volputate condimentum ornare.  In 
%scelerisque aliquam\drmfelip\end{spverbatim} &
% \declettrine{L}{orem ipsum} dolor sit amet, consectetur
% adipiscing elit.  Ut porttitor libero lacus, a rhoncus
% dolor finibus vel.  Morbi volputate condimentum ornare.
% In scelerisque aliquam\drmfelip \\
% \end{tabular}
% \end{center}
% 
%
% \subsection{Math}
% \label{sub:math}
%
% Your author is far from a mathematician, so he's not
% really able to judge the quality of the following; but DRM
% does offer matching math fonts.  These are limited to the
% default \TeX\ math fonts, however; AMS extensions and the
% like are not available.  Perhaps one day (after finishing
% the ornaments and decorative initials) they will be, but
% for now one will have to pull in other fonts for anything
% that goes beyond plain \TeX.  Using them in bold goes a
% long way to making them match the rest of DRM.
%
% First, we have a full set of mathematical Greek letters.
% As seems to be the custom, the capitals are upright and
% the lowercase slanted.  These can all be accessed via the
% customary \TeX\ math character names.
%
% \begin{center}
% \begin{longtable}{llllllll}
% \toprule
% \multicolumn{8}{c}{Greek Letters} \\
% \midrule
% A & |A| & $\alpha$ & |$\alpha$| & B & |B| & $\beta$& |$\beta$| \\
% $\Gamma$& |$\Gamma$| & $\gamma$& |$\gamma$| & $\Delta$& |$\Delta$| &
% $\delta$& |$\delta$| \\ E & |E| & $\epsilon$& |$\epsilon$| & Z &
% 	|Z| &
% $\zeta$ & |$\zeta$| \\ H & |H| & $\eta$& |$\eta$| & $\Theta$& |$\Theta$| &
% $\theta$& |$\theta$| \\ I & |I| & $\iota$& |$\iota$| & K & |K| &
% $\kappa$& |$\kappa$| \\ $\Lambda$& |$\Lambda$| & $\lambda$&
% 	|$\lambda$| & M & |M| &
% $\mu$ & |$\mu$| \\ N & |N| & $\nu$& |$\nu$| & $\Xi$& |$\Xi$| &
% $\xi$& |$\xi$| \\ O & |O| & o & |o| & $\Pi$& |$\Pi$| &
% $\pi$& |$\pi$| \\ P & |P| & $\rho$& |$\rho$| & $\Sigma$& |$\Sigma$| &
% $\sigma$& |$\sigma$| \\ T & |T| & $\tau$& |$\tau$| & Y & |Y| &
% $\upsilon$& |$\upsilon$| \\ $\Phi$& |$\Phi$| & $\phi$& |$\phi$| & X & |X| &
% $\chi$& |$\chi$| \\ $\Psi$& |$\Psi$| & $\psi$& |$\psi$| & $\Omega$&
% 	|$\Omega$| & $\omega$& |$\omega$| \\ $\vartheta$& |$\vartheta$| & $\varpi$&
% 	|$\varpi$| & $\varsigma$& |$\varsigma$| & $\varphi$ &
%	|$\varphi$| \\
% \bottomrule
% \end{longtable}
% \end{center}
%
% This alphabet led directly to DRM's Greek font, which we
% discuss elsewhere.\footnote{\textit{See supra}, Section
% \ref{sub:greek}, at \pageref{sub:greek}.}
%
% \begin{center}
% \begin{longtable}{llllllll}
% \toprule
% \multicolumn{8}{c}{Math Calligraphic} \\
% \midrule
% $\mathcal{A}$ & $\mathcal{B}$ & $\mathcal{C}$ & $\mathcal{D}$ & $\mathcal{E}$ & $\mathcal{F}$ & $\mathcal{G}$ & $\mathcal{H}$ \\
% $\mathcal{I}$ & $\mathcal{J}$ & $\mathcal{K}$ & $\mathcal{L}$ & $\mathcal{M}$ & $\mathcal{N}$ & $\mathcal{O}$ & $\mathcal{P}$ \\
% $\mathcal{Q}$ & $\mathcal{R}$ & $\mathcal{S}$ & $\mathcal{T}$ & $\mathcal{U}$ & $\mathcal{V}$ & $\mathcal{W}$ & $\mathcal{X}$ \\
% $\mathcal{Y}$ & $\mathcal{Z}$ & {} & {} & {} & {} & {} & {} \\
% \bottomrule
% \end{longtable}
% \end{center}
%
% DRM also has its own extensible characters and
% variable-sized math characters; a few examples
% in various sizes are below.
%
% $$ \sum\limits_{i=1}^n i^2 = \frac{n(n+1)(2n+1)}{6} $$
% $$ \prod\limits_{i=1}^n i^2 = \left(\frac{n(n+1)(2n+1)}{6}\right) $$
% $$ \sum\nolimits_{P_i \in Paths(I)} Probes(P_{i}) $$
% $$ \underbrace{\overbrace{abcdefghijklmnop}} $$
%
% By default, using |\big| and friends doesn't work, a
% problem I haven't been able to resolve.  However, by
% requiring |amsmath|, |drm| provides a more directly
% flexible mechanism for this: \DescribeMacro{\bigd}|\bigd|, 
% which allows arbitrarily sized delimiters.  It takes a
% single argument, which is an integer describing the
% desired size:
%
% \begin{center}
% \begin{tabular}{llllll}
% |\bigd{2}\{| & \bigd{2}\{ &
% |\bigd{4}\{| & \bigd{4}\{ &
% |\bigd{8}\{| & \bigd{8}\{ \\
% \end{tabular}
% \end{center}
%
% |\left| and |\right| work as expected with DRM's
% delimiters.
%
% This symbols, of course, also work inline (as opposed to
% displayed, which is what we have above); you can take
% $\sqrt{2\over3}$ and have $(3\times\left(4\over3\right))$ just
% as easily in a paragraph as in a display, though you may
% want to take care that you're not using too much space for
% your lines.  (I didn't take care in this paragraph, and
% you can see how bad it looks.)
%
% \subsection{Greek}
% \label{sub:greek}
%
% Because DRM offers Greek characters in math, it was a
% short step to offer actual Greek text, and so I've done
% so, according to the standard LGR encoding.  I can just
% barely read the Greek alphabet and remember very little of
% the grammar, and what little I once knew was all ancient
% and koine, but here it is.  DRM's Greek support is
% limited; while it offers all the normal \emph{polutoniko}
% accents, subscripts, and breathings, along with some
% archaic characters like the digamma, there is no italic,
% small caps, or various weights.  DRM isn't, therefore,
% really suitable for typesetting whole Greek works; it
% will, however, offer attractive typesetting of Greek
% phrases and quotations within a text set otherwise in the
% Latin alphabet.
%
% The |\grktext|\DescribeMacro{\grktext}\ command changes the
% current font encoding to LGR, which for DRM's purposes
% means it's typesetting with Greek characters from then on.
% The macro |\textgrk|\DescribeMacro{\textgrk}\ is similar,
% but takes a single argument, which is typeset in Greek
% characters.  A few examples follow.
%
% \begin{center}
% \begin{tabular}{p{0.4\textwidth}p{0.4\textwidth}}
% \begin{spverbatim}We know that \textgrk{Aqilleuc} was one of the Greeks' greatest warriors.\end{spverbatim} &
% \begin{spverbatim}{\grktext >En {\>a}rq\char'254\ {\>~h}n {\<o} l{\'o}goc, ka{\'i} {\<o} l{\'o}goc {\~\>h}n pr{\'o}c t{\'o}n je{\'o}n, ka{\'i} je{\'o}c {\>\~h}n {\<o} l{\'o}goc.}\end{spverbatim} \\
% We know that \textgrk{Aqilleuc} was one 
% of the Greeks' greatest warriors. &
% {\grktext >En {\>a}rq\char'254\
% {\>~h}n {\<o} l{\'o}goc, ka{\'i} {\<o} l{\'o}goc {\~\>h}n
% pr{\'o}c t{\'o}n je{\'o}n, ka{\'i} je{\'o}c {\>\~h}n {\<o}
% l{\'o}goc.} \\
% \end{tabular}
% \end{center}
%
% I understand that |babel| has facilities for making the
% typesetting of all the \textgrk{polutoniko} accents much
% cleaner, but I don't write enough in Greek to have learned
% to use it, resulting in the mess you see above.  Note that
% |drm| (the package) does \emph{not} pull in |babel| or the
% |polutoniko| option, or any other Greek typesetting
% package; it simply provides the fonts.  If you're
% typesetting long enough passages that you need Greek
% hyphenation and the like, you'll have to invoke the
% appropriate package yourself.
%
% \section{Implementation}
% \label{sect:code}
% 
% Load the required packages.  DRM contains TS1, LGR, OML,
% OMS, and T1 encoded fonts, so we load |fontenc| with all
% these encodings as options.  We also load |modroman| for 
% the |\romanize| macro, defined below.  Finally, we load
% |gmp| for the decorative initials (this allows including
% \MP\ code in \LaTeX\ source).
%    \begin{macrocode}
\RequirePackage[LGR,OML,OMS,TS1,T1]{fontenc}
\RequirePackage{modroman}
\RequirePackage{amsmath}
\RequirePackage{gmp}
%    \end{macrocode}
% Now we declare our options.
%    \begin{macrocode}
\newif\ifnodefault\nodefaultfalse
\newif\ifnodefaultmath\nodefaultmathfalse
\newif\ifnodefaulttext\nodefaulttextfalse
\newif\ifsymbolsonly\symbolsonlyfalse
\newif\iftypeone\typeonefalse
\DeclareOption{nodefault}{\nodefaulttrue\nodefaultmathtrue%
	\nodefaulttexttrue}
\DeclareOption{nodefaultmath}{\nodefaultmathtrue}
\DeclareOption{nodefaulttext}{\nodefaulttexttrue}
\DeclareOption{symbolsonly}{\symbolsonlytrue\nodefaulttrue%
	\nodefaulttexttrue\nodefaultmathtrue}
\DeclareOption{typeone}{\typeonetrue}
\ProcessOptions
%    \end{macrocode}
% Begin defining the font families.  First, define the fonts
% with the file |drm.map| if the option |typeone| was
% requested; otherwise, load the \MF\ files directly.
%    \begin{macrocode}
\iftypeone
	\RequirePackage{ifpdf}
	\ifpdf
		\pdfmapfile{=drm.map}
	\fi
	\DeclareFontFamily{T1}{drm}{}
	\DeclareFontFamily{TS1}{drm}{}
	\DeclareFontFamily{LGR}{drm}{}
	\DeclareFontFamily{U}{drmsups}{}
	\DeclareFontFamily{U}{drminfs}{}
	\DeclareFontShape{U}{drminfs}{m}{n}{ <-7> drminf6 
		<7> drminf7 <8> drminf8 <9> drminf9 <10-12> drminf10 
		<12-13> drminf12 <14-17> drminf14 <17-24> drminf17 
		<24-> drminf24 }{}
	\DeclareFontShape{U}{drmsups}{m}{n}{ <-7> drmfigs6 
		<7> drmfigs7 <8> drmfigs8 <9> drmfigs9 <10-12> drmfigs10 
		<12-13> drmfigs12 <14-17> drmfigs14 <17-24> drmfigs17 
		<24-> drmfigs24 }{}
	\DeclareFontShape{T1}{drm}{m}{n}{ <-7> drm6 <7> drm7 <8> drm8
		<9> drm9 <10-12> drm10 <12-13> drm12 <14-17> drm14 
		<17-24> drm17 <24-> drm24 }{}
	\DeclareFontShape{T1}{drm}{m}{sc}{<-7> drmsc6 <7> drmsc7 
		<8> drmsc8 <9> drmsc9 <10-12> drmsc10 <12-14> drmsc12 
		<14-17> drmsc14 <17-24> drmsc17 <24-> drmsc24 }{}
	\DeclareFontShape{T1}{drm}{m}{tc}{<-7> drmtc6 <7> drmtc7 
		<8> drmtc8 <9> drmtc9 <10-12> drmtc10 <12-14> drmtc12 
		<14-17> drmtc14 <17-24> drmtc17 <24-> drmtc24 }{}
	\DeclareFontShape{T1}{drm}{bx}{sc}{<-7> drmscbx6 <7> drmscbx7 
		<8> drmscbx8 <9> drmscbx9 <10-12> drmscbx10 <12-14> drmscbx12 
		<14-17> drmscbx14 <17-24> drmscbx17 <24-> drmscbx24 }{}
	\DeclareFontShape{T1}{drm}{bx}{tc}{<-7> drmtcbx6 <7> drmtcbx7 
		<8> drmtcbx8 <9> drmtcbx9 <10-12> drmtcbx10 <12-14> drmtcbx12 
		<14-17> drmtcbx14 <17-24> drmtcbx17 <24-> drmtcbx24 }{}
	\DeclareFontShape{T1}{drm}{m}{itsc}{<-7> drmitsc6 <7> drmitsc7 
		<8> drmitsc8 <9> drmitsc9 <10-12> drmitsc10 <12-14> drmitsc12 
		<14-17> drmitsc14 <17-24> drmitsc17 <24-> drmitsc24 }{}
	\DeclareFontShape{T1}{drm}{m}{ittc}{<-7> drmittc6 <7> drmittc7 
		<8> drmittc8 <9> drmittc9 <10-12> drmittc10 <12-14> drmittc12 
		<14-17> drmittc14 <17-24> drmittc17 <24-> drmittc24 }{}
	\DeclareFontShape{T1}{drm}{m}{sl}{<-7> drmsl6 <7> drmsl7 
		<8> drmsl8 <9> drmsl9 <10-12> drmsl10 <12-14> drmsl12 
		<14-17> drmsl14 <17-24> drmsl17 <24-> drmsl24 }{}
	\DeclareFontShape{T1}{drm}{m}{it}{ <-7> drmit6 <7> drmit7 
		<8> drmit8 <9> drmit9 <10-12> drmit10 <12-14> drmit12 
		<14-17> drmit14 <17-24> drmit17 <24-> drmit24 }{}
	\DeclareFontShape{T1}{drm}{bx}{it}{ <-7> drmitbx6 <7> drmitbx7 
		<8> drmitbx8 <9> drmitbx9 <10-12> drmitbx10 <12-14> drmitbx12 
		<14-17> drmitbx14 <17-24> drmitbx17 <24-> drmitbx24 }{}
	\DeclareFontShape{T1}{drm}{m}{ui}{<-7> drmui6 <7> drmui7 
		<8> drmui8 <9> drmui9 <10-12> drmui10 <12-14> drmui12 
		<14-17> drmui14 <17-24> drmui17 <24-> drmui24 }{}
	\DeclareFontShape{T1}{drm}{bx}{ui}{<-7> drmuibx6 <7> drmuibx7 
		<8> drmuibx8 <9> drmuibx9 <10-12> drmuibx10 <12-14> drmuibx12 
		<14-17> drmuibx14 <17-24> drmuibx17 <24-> drmuibx24 }{}
	\DeclareFontShape{T1}{drm}{l}{n}{<-7> drml6 <7> drml7 
		<8> drml8 <9> drml9 <10-12> drml10 <12-14> drml12 
		<14-17> drml14 <17-24> drml17 <24-> drml24 }{}
	\DeclareFontShape{T1}{drm}{b}{n}{<-7> drmb6 <7> drmb7 
		<8> drmb8 <9> drmb9 <10-12> drmb10 <12-14> drmb12 
		<14-17> drmb14 <17-24> drmb17 <24-> drmb24 }{}
	\DeclareFontShape{T1}{drm}{bx}{n}{<-7> drmbx6 <7> drmbx7 
		<8> drmbx8 <9> drmbx9 <10-12> drmbx10 <12-14> drmbx12 
		<14-17> drmbx14 <17-24> drmbx17 <24-> drmbx24 }{}
	\DeclareFontShape{TS1}{drm}{m}{n}{<-7> drmsym7 
		<8> drmsym8 <9> drmsym9 <10-12> drmsym10 <12-14> drmsym12 
		<14-17> drmsym14 <17-24> drmsym17 <24-> drmsym24 }{}
	\DeclareFontShape{LGR}{drm}{m}{n}{<-> drmgrk10 }{}
\else
	\DeclareFontFamily{T1}{drm}{}
	\DeclareFontFamily{TS1}{drm}{}
	\DeclareFontFamily{LGR}{drm}{}
	\DeclareFontFamily{U}{drmsups}{}
	\DeclareFontFamily{U}{drminfs}{}
	\DeclareFontShape{U}{drminfs}{m}{n}{ <-7> drminf6 
		<7> drminf7 <8> drminf8 <9> drminf9 <10-12> drminf10 
		<12-13> drminf12 <14-17> drminf14 <17-24> drminf17 
		<24-> drminf24 }{}
	\DeclareFontShape{U}{drmsups}{m}{n}{ <-7> drmfigs6 
		<7> drmfigs7 <8> drmfigs8 <9> drmfigs9 <10-12> drmfigs10 
		<12-13> drmfigs12 <14-17> drmfigs14 <17-24> drmfigs17 
		<24-> drmfigs24 }{}
	\DeclareFontShape{T1}{drm}{m}{n}{ <-7> drm6 <7> drm7 <8> drm8
		<9> drm9 <10-12> drm10 <12-13> drm12 <14-17> drm14 
		<17-24> drm17 <24-> drm24 }{}
	\DeclareFontShape{T1}{drm}{m}{sc}{<-7> drmsc6 <7> drmsc7 
		<8> drmsc8 <9> drmsc9 <10-12> drmsc10 <12-14> drmsc12 
		<14-17> drmsc14 <17-24> drmsc17 <24-> drmsc24 }{}
	\DeclareFontShape{T1}{drm}{m}{tc}{<-7> drmtc6 <7> drmtc7 
		<8> drmtc8 <9> drmtc9 <10-12> drmtc10 <12-14> drmtc12 
		<14-17> drmtc14 <17-24> drmtc17 <24-> drmtc24 }{}
	\DeclareFontShape{T1}{drm}{bx}{sc}{<-7> drmscbx6 <7> drmscbx7 
		<8> drmscbx8 <9> drmscbx9 <10-12> drmscbx10 <12-14> drmscbx12 
		<14-17> drmscbx14 <17-24> drmscbx17 <24-> drmscbx24 }{}
	\DeclareFontShape{T1}{drm}{bx}{tc}{<-7> drmtcbx6 <7> drmtcbx7 
		<8> drmtcbx8 <9> drmtcbx9 <10-12> drmtcbx10 <12-14> drmtcbx12 
		<14-17> drmtcbx14 <17-24> drmtcbx17 <24-> drmtcbx24 }{}
	\DeclareFontShape{T1}{drm}{m}{itsc}{<-7> drmitsc6 <7> drmitsc7 
		<8> drmitsc8 <9> drmitsc9 <10-12> drmitsc10 <12-14> drmitsc12 
		<14-17> drmitsc14 <17-24> drmitsc17 <24-> drmitsc24 }{}
	\DeclareFontShape{T1}{drm}{m}{ittc}{<-7> drmittc6 <7> drmittc7 
		<8> drmittc8 <9> drmittc9 <10-12> drmittc10 <12-14> drmittc12 
		<14-17> drmittc14 <17-24> drmittc17 <24-> drmittc24 }{}
	\DeclareFontShape{T1}{drm}{m}{sl}{<-7> drmsl6 <7> drmsl7 
		<8> drmsl8 <9> drmsl9 <10-12> drmsl10 <12-14> drmsl12 
		<14-17> drmsl14 <17-24> drmsl17 <24-> drmsl24 }{}
	\DeclareFontShape{T1}{drm}{m}{it}{ <-7> drmit6 <7> drmit7 
		<8> drmit8 <9> drmit9 <10-12> drmit10 <12-14> drmit12 
		<14-17> drmit14 <17-24> drmit17 <24-> drmit24 }{}
	\DeclareFontShape{T1}{drm}{bx}{it}{ <-7> drmitbx6 <7> drmitbx7 
		<8> drmitbx8 <9> drmitbx9 <10-12> drmitbx10 <12-14> drmitbx12 
		<14-17> drmitbx14 <17-24> drmitbx17 <24-> drmitbx24 }{}
	\DeclareFontShape{T1}{drm}{m}{ui}{<-7> drmui6 <7> drmui7 
		<8> drmui8 <9> drmui9 <10-12> drmui10 <12-14> drmui12 
		<14-17> drmui14 <17-24> drmui17 <24-> drmui24 }{}
	\DeclareFontShape{T1}{drm}{bx}{ui}{<-7> drmuibx6 <7> drmuibx7 
		<8> drmuibx8 <9> drmuibx9 <10-12> drmuibx10 <12-14> drmuibx12 
		<14-17> drmuibx14 <17-24> drmuibx17 <24-> drmuibx24 }{}
	\DeclareFontShape{T1}{drm}{l}{n}{<-7> drml6 <7> drml7 
		<8> drml8 <9> drml9 <10-12> drml10 <12-14> drml12 
		<14-17> drml14 <17-24> drml17 <24-> drml24 }{}
	\DeclareFontShape{T1}{drm}{b}{n}{<-7> drmb6 <7> drmb7 
		<8> drmb8 <9> drmb9 <10-12> drmb10 <12-14> drmb12 
		<14-17> drmb14 <17-24> drmb17 <24-> drmb24 }{}
	\DeclareFontShape{T1}{drm}{bx}{n}{<-7> drmbx6 <7> drmbx7 
		<8> drmbx8 <9> drmbx9 <10-12> drmbx10 <12-14> drmbx12 
		<14-17> drmbx14 <17-24> drmbx17 <24-> drmbx24 }{}
	\DeclareFontShape{TS1}{drm}{m}{n}{<-7> drmsym7 
		<8> drmsym8 <9> drmsym9 <10-12> drmsym10 <12-14> drmsym12 
		<14-17> drmsym14 <17-24> drmsym17 <24-> drmsym24 }{}
	\DeclareFontShape{LGR}{drm}{m}{n}{<-> drmgrk10 }{}
\fi
%    \end{macrocode}
% Now, set the default text font as DRM unless |nodefault|
% or |nodefaulttext| has been specified.  Also redefine the
% default footnote counters to use superior figures rather
% than automatically scaled figures.
%    \begin{macrocode}
\ifnodefault\else\ifnodefaulttext\else
	\renewcommand\encodingdefault{T1}
	\renewcommand\familydefault{drm}
	\def\f@@n@te{footnote}
	\def\@makefnmark{%
		\hbox{\drmsupfigs\@thefnmark}
	}%
\fi\fi
%    \end{macrocode}
% Give ourselves a shortcut to access the short-tailed
% letter ``q,'' just in case we need it.
%    \begin{macrocode}
\def\drmshortq{{\usefont{T1}{drm}{m}{n}\char'137}}
%    \end{macrocode}
% That gives us the satisfying ``\drmshortq'' rather than
% the ``Q'' we would otherwise get.  Useful for circumstance
% when the ``Q'' is followed by characters which hang
% below the baseline, or in a dropped initial.
%
% Now we move on to define commands for the more unusual
% shapes, since \LaTeXe\ doesn't have them built in.  We
% start with \texttc{titling small caps}, then move on to
% \textui{upright italics}.  We also define |\textgrk| and
% |\grktext|, for typesetting in Greek characters.  Finally,
% we also define the commands to produce the superior and
% inferior figures.
%    \begin{macrocode}
\def\tcshape{\fontshape{tc}\selectfont}
\def\texttc#1{{\tcshape#1}}
\def\ittcshape{\fontshape{ittc}\selectfont}
\def\textittc#1{{\ittcshape#1}}
\def\itscshape{\fontshape{itsc}\selectfont}
\def\textitsc#1{{\itscshape#1}}
\def\uishape{\fontshape{ui}\selectfont}
\def\textui#1{{\uishape#1}}
\def\grktext{\fontencoding{LGR}\selectfont}
\def\textgrk#1{{\grktext#1}}
\def\drmsupfigs{\usefont{U}{drmsups}{m}{n}}
\def\textdrmsupfigs#1{{\drmsupfigs#1}}
\def\drminffigs{\usefont{U}{drminfs}{m}{n}}
\def\textdrminffigs#1{{\drminffigs#1}}
%    \end{macrocode}
% Next, we define the weights.  We know that |\textbf| will
% give us normal boldface, and that |\textmd| will return us
% to medium weight; but since DRM also has a light weight
% and a bold non-extended, we need to define commands for
% those, as well.
%    \begin{macrocode}
\def\lseries{\fontseries{l}\selectfont}
\def\textl#1{{\lseries#1}}
\def\bseries{\fontseries{b}\selectfont}
\def\textb#1{{\bseries#1}}
%    \end{macrocode}
% Moving on, we define size commands based on
% traditional English-language printers' names.  Why?
% \emph{Because we can}, that's why.
%    \begin{macrocode}
% \def\loosen{\addtolength{\baselineskip}{1pt}}
% \def\excelsior{\fontsize{3pt}{3.5pt}\selectfont}
% \def\minikin{\excelsior}
% \def\brilliant{\fontsize{4pt}{4.5pt}\selectfont}
% \def\diamond{\fontsize{4.5pt}{5pt}\selectfont}
% \def\pearl{\fontsize{5pt}{6pt}\selectfont}
% \def\agate{\fontsize{5.5pt}{6.5pt}\selectfont}
% \def\ruby{\agate}
% \def\nonpareille{\fontsize{6pt}{7pt}\selectfont}
% \def\minionette{\fontsize{6.5pt}{7.5pt}\selectfont}
% \def\emerald{\minionette}
% \def\minion{\fontsize{7pt}{8pt}\selectfont}
% \def\brevier{\fontsize{8pt}{9pt}\selectfont}
% \def\petit{\brevier}
% \def\smalltext{\brevier}
% \def\bourgeois{\fontsize{9pt}{10pt}\selectfont}
% \def\galliard{\bourgeois}
% \def\longprimer{\fontsize{10pt}{12pt}\selectfont}
% \def\corpus{\longprimer}
% \def\garamond{\longprimer}
% \def\smallpica{\fontsize{11pt}{13pt}\selectfont}
% \def\philosophy{\smallpica}
% \def\pica{\fontsize{12pt}{15pt}\selectfont}
% \def\english{\fontsize{14pt}{17pt}\selectfont}
% \def\mittel{\english}
% \def\augustin{\english}
% \def\columbian{\fontsize{16pt}{19pt}\selectfont}
% \def\twolinebrevier{\columbian}
% \def\greatprimer{\fontsize{18pt}{22pt}\selectfont}
% \def\paragon{\fontsize{20pt}{24pt}\selectfont}
% \def\doublesmallpica{\fontsize{21pt}{25pt}\selectfont}
% \def\doublesmallpicaus{\fontsize{22pt}{26pt}\selectfont}
% \def\doublepicabrit{\doublesmallpicaus}
% \def\doublepica{\fontsize{24pt}{28pt}\selectfont}
% \def\twolinepica{\doublepica}
% \def\doubleenglish{\fontsize{28pt}{33pt}\selectfont}
% \def\twolineenglish{\doubleenglish}
% \def\fivelinenonpareil{\fontsize{30pt}{35pt}\selectfont}
% \def\fourlinebrevier{\fontsize{32pt}{38pt}\selectfont}
% \def\doublegreatprimer{\fontsize{36pt}{42pt}\selectfont}
% \def\twolinegreatprimer{\doublegreatprimer}
% \def\meridian{\fontsize{44pt}{50pt}\selectfont}
% \def\twolinedoublepica{\meridian}
% \def\trafalgar{\meridian}
% \def\canon{\fontsize{48pt}{54pt}\selectfont}
% \def\fourline{\canon}
% \def\fivelinepica{\fontsize{60pt}{66pt}\selectfont}
% \def\inch{\fontsize{72pt}{78pt}\selectfont}
%    \end{macrocode}
% Now we move on to define the math fonts.  This turned out
% to be a surprisingly convoluted process, and I only
% marginally understand what's going on here; but it works,
% and I'll try to go through it as best as I can.
%
% First, we make \LaTeX\ aware of our math fonts:
%    \begin{macrocode}
\DeclareFontFamily{OML}{drm}{}
\DeclareFontShape{OML}{drm}{m}{n}{ <-> drmmi10 }{}
\DeclareFontFamily{OMS}{drm}{}
\DeclareFontShape{OMS}{drm}{m}{n}{ <-> drmsy10 }{}
\DeclareFontFamily{OMX}{drm}{}
\DeclareFontShape{OMX}{drm}{m}{n}{ <-> drmomx10 }{}
%    \end{macrocode}
% Next, we declare something called a math \emph{version};
% this way we can define new math shapes without clobbering
% the default settings.  I'm not sure why this is really
% necessary, but it appears to be; so we define a math
% version |drmmath|:
%    \begin{macrocode}
\DeclareMathVersion{drmmath}
%    \end{macrocode}
% Now we define our symbol fonts.  This lets \LaTeX\ know
% where to yank its symbols from when typesetting a math
% formula.
%    \begin{macrocode}
\ifnodefault\else\ifnodefaultmath\else
\SetSymbolFont{operators}{drmmath}{T1}{drm}{m}{n}
\SetSymbolFont{letters}{drmmath}{OML}{drm}{m}{n}
\DeclareSymbolFont{symbs}{TS1}{drm}{m}{n}
\SetSymbolFont{symbs}{drmmath}{TS1}{drm}{m}{n}
\DeclareSymbolFont{drmmathsy}{OMS}{drm}{m}{n}
\SetSymbolFont{drmmathsy}{drmmath}{OMS}{drm}{m}{n}
\DeclareSymbolFont{drmmathomx}{OMX}{drm}{m}{n}
\SetSymbolFont{drmmathomx}{drmmath}{OMX}{drm}{m}{n}
\fi\fi
%    \end{macrocode}
% Now we write in our \emph{math alphabets}, so that when we
% request |\mathcal| or something similar we'll get DRM and
% not Computer Modern.  We define |\mathcal|, of course, and
% also |\drmmathlets|, just in case we want to request DRM
% directly.
%    \begin{macrocode}
\ifnodefault\else\ifnodefaultmath\else
\DeclareMathAlphabet{\drmmathlets}{OML}{drm}{m}{n}
\SetMathAlphabet{\drmmathlets}{drmmath}{OML}{drm}{m}{n}
\DeclareMathAlphabet{\mathcal}{OMS}{drm}{m}{n}
\SetMathAlphabet{\mathcal}{drmmath}{OMS}{drm}{m}{n}
\fi\fi
%    \end{macrocode}
% Now ensure that we get lining figures in math mode.
%    \begin{macrocode}
\ifnodefault\else\ifnodefaultmath\else
\DeclareMathSymbol{0}{0}{symbs}{48}
\DeclareMathSymbol{1}{0}{symbs}{49}
\DeclareMathSymbol{2}{0}{symbs}{50}
\DeclareMathSymbol{3}{0}{symbs}{51}
\DeclareMathSymbol{4}{0}{symbs}{52}
\DeclareMathSymbol{5}{0}{symbs}{53}
\DeclareMathSymbol{6}{0}{symbs}{54}
\DeclareMathSymbol{7}{0}{symbs}{55}
\DeclareMathSymbol{8}{0}{symbs}{56}
\DeclareMathSymbol{9}{0}{symbs}{57}
\fi\fi
%    \end{macrocode}
% Now, it appears to be necessary to redefine all the math
% symbols, so we do that.  Beginning with the Greek letters:
%    \begin{macrocode}
\ifnodefault\else\ifnodefaultmath\else
\DeclareMathSymbol{\Gamma}{0}{letters}{0}
\DeclareMathSymbol{\Delta}{0}{letters}{1}
\DeclareMathSymbol{\Theta}{0}{letters}{2}
\DeclareMathSymbol{\Lambda}{0}{letters}{3}
\DeclareMathSymbol{\Xi}{0}{letters}{4}
\DeclareMathSymbol{\Pi}{0}{letters}{5}
\DeclareMathSymbol{\Sigma}{0}{letters}{6}
\DeclareMathSymbol{\Upsilon}{0}{letters}{7}
\DeclareMathSymbol{\Phi}{0}{letters}{8}
\DeclareMathSymbol{\Psi}{0}{letters}{9}
\DeclareMathSymbol{\Omega}{0}{letters}{10}
\DeclareMathSymbol{\alpha}{0}{letters}{11}
\DeclareMathSymbol{\beta}{0}{letters}{12}
\DeclareMathSymbol{\gamma}{0}{letters}{13}
\DeclareMathSymbol{\delta}{0}{letters}{14}
\DeclareMathSymbol{\epsilon}{0}{letters}{15}
\DeclareMathSymbol{\zeta}{0}{letters}{16}
\DeclareMathSymbol{\eta}{0}{letters}{17}
\DeclareMathSymbol{\theta}{0}{letters}{18}
\DeclareMathSymbol{\iota}{0}{letters}{19}
\DeclareMathSymbol{\kappa}{0}{letters}{20}
\DeclareMathSymbol{\lambda}{0}{letters}{21}
\DeclareMathSymbol{\mu}{0}{letters}{22}
\DeclareMathSymbol{\nu}{0}{letters}{23}
\DeclareMathSymbol{\xi}{0}{letters}{24}
\DeclareMathSymbol{\pi}{0}{letters}{25}
\DeclareMathSymbol{\rho}{0}{letters}{26}
\DeclareMathSymbol{\sigma}{0}{letters}{27}
\DeclareMathSymbol{\tau}{0}{letters}{28}
\DeclareMathSymbol{\upsilon}{0}{letters}{29}
\DeclareMathSymbol{\phi}{0}{letters}{30}
\DeclareMathSymbol{\chi}{0}{letters}{31}
\DeclareMathSymbol{\psi}{0}{letters}{32}
\DeclareMathSymbol{\omega}{0}{letters}{33}
\DeclareMathSymbol{\varepsilon}{0}{letters}{34}
\DeclareMathSymbol{\vartheta}{0}{letters}{35}
\DeclareMathSymbol{\varpi}{0}{letters}{36}
\DeclareMathSymbol{\varrho}{0}{letters}{37}
\DeclareMathSymbol{\varsigma}{0}{letters}{38}
\DeclareMathSymbol{\varphi}{0}{letters}{39}
\fi\fi
%    \end{macrocode}
% Now let's define some of the other symbols in the OML
% encoding.
%    \begin{macrocode}
\ifnodefault\else\ifnodefaultmath\else
\DeclareMathSymbol{\leftharpoonup}{0}{letters}{40}
\DeclareMathSymbol{\leftharpoondown}{0}{letters}{41}
\DeclareMathSymbol{\rightharpoonup}{0}{letters}{42}
\DeclareMathSymbol{\rightharpoondown}{0}{letters}{43}
\DeclareMathSymbol{\triangleright}{0}{letters}{46}
\DeclareMathSymbol{\triangleleft}{0}{letters}{47}
\DeclareMathSymbol{\flat}{0}{letters}{91}
\DeclareMathSymbol{\natural}{0}{letters}{92}
\DeclareMathSymbol{\sharp}{0}{letters}{93}
\DeclareMathSymbol{\smile}{0}{letters}{94}
\DeclareMathSymbol{\frown}{0}{letters}{95}
\DeclareMathSymbol{\ell}{0}{letters}{96}
\DeclareMathSymbol{\imath}{0}{letters}{123}
\DeclareMathSymbol{\jmath}{0}{letters}{124}
\DeclareMathSymbol{\wp}{0}{letters}{125}
\fi\fi
%    \end{macrocode}
% Now we go on to define the symbols from the OMS-encoded
% fonts.
%    \begin{macrocode}
\ifnodefault\else\ifnodefaultmath\else
\DeclareMathSymbol{-}{2}{drmmathsy}{'000}
\DeclareMathSymbol{\cdot}{2}{drmmathsy}{'001}
\DeclareMathSymbol{\times}{2}{drmmathsy}{'002}
\DeclareMathSymbol{\ast}{2}{drmmathsy}{'003}
\DeclareMathSymbol{\div}{2}{drmmathsy}{'004}
\DeclareMathSymbol{\diamond}{2}{drmmathsy}{'005}
\DeclareMathSymbol{\pm}{2}{drmmathsy}{'006}
\DeclareMathSymbol{\mp}{2}{drmmathsy}{'007}
\DeclareMathSymbol{\oplus}{2}{drmmathsy}{'010}
\DeclareMathSymbol{\ominus}{2}{drmmathsy}{'011}
\DeclareMathSymbol{\otimes}{2}{drmmathsy}{'012}
\DeclareMathSymbol{\oslash}{2}{drmmathsy}{'013}
\DeclareMathSymbol{\odot}{2}{drmmathsy}{'014}
\DeclareMathSymbol{\bigcirc}{2}{drmmathsy}{'015}
\DeclareMathSymbol{\circ}{2}{drmmathsy}{'016}
\DeclareMathSymbol{\bullet}{2}{drmmathsy}{'017}
\DeclareMathSymbol{\asymp}{3}{drmmathsy}{'020}
\DeclareMathSymbol{\equiv}{3}{drmmathsy}{'021}
\DeclareMathSymbol{\subseteq}{3}{drmmathsy}{'022}
\DeclareMathSymbol{\supseteq}{3}{drmmathsy}{'023}
\DeclareMathSymbol{\leq}{3}{drmmathsy}{'024}
\DeclareMathSymbol{\geq}{3}{drmmathsy}{'025}
\DeclareMathSymbol{\preceq}{3}{drmmathsy}{'026}
\DeclareMathSymbol{\succeq}{3}{drmmathsy}{'027}
\DeclareMathSymbol{\sim}{3}{drmmathsy}{'030}
\DeclareMathSymbol{\approx}{3}{drmmathsy}{'031}
\DeclareMathSymbol{\subset}{3}{drmmathsy}{'032}
\DeclareMathSymbol{\supset}{3}{drmmathsy}{'033}
\DeclareMathSymbol{\ll}{3}{drmmathsy}{'034}
\DeclareMathSymbol{\gg}{3}{drmmathsy}{'035}
\DeclareMathSymbol{\prec}{3}{drmmathsy}{'036}
\DeclareMathSymbol{\succ}{3}{drmmathsy}{'037}
\DeclareMathSymbol{\simeq}{3}{drmmathsy}{'047}
\DeclareMathSymbol{\propto}{3}{drmmathsy}{'057}
\DeclareMathSymbol{\prime}{0}{drmmathsy}{'060}
\DeclareMathSymbol{'}{0}{drmmathsy}{'060}
\DeclareMathSymbol{\infty}{0}{drmmathsy}{'061}
\DeclareMathSymbol{\in}{0}{drmmathsy}{'062}
\DeclareMathSymbol{\ni}{0}{drmmathsy}{'063}
\DeclareMathSymbol{\bigtriangleup}{2}{drmmathsy}{'064}
\DeclareMathSymbol{\bigtriangledown}{2}{drmmathsy}{'065}
\DeclareMathSymbol{/}{2}{drmmathsy}{'066}
%\DeclareMathSymbol{'}{2}{drmmathsy}{'067}
\DeclareMathSymbol{\forall}{2}{drmmathsy}{'070}
\DeclareMathSymbol{\exists}{2}{drmmathsy}{'071}
\DeclareMathSymbol{\neg}{2}{drmmathsy}{'072}
\DeclareMathSymbol{\emptyset}{2}{drmmathsy}{'073}
\DeclareMathSymbol{\Im}{0}{drmmathsy}{'074}
\DeclareMathSymbol{\Re}{0}{drmmathsy}{'075}
\DeclareMathSymbol{\top}{0}{drmmathsy}{'076}
\DeclareMathSymbol{\bot}{0}{drmmathsy}{'077}
%\DeclareMathSymbol{\aleph}{0}{drmmathsy}{'080}
\DeclareMathSymbol{\cup}{2}{drmmathsy}{'133}
\DeclareMathSymbol{\cap}{2}{drmmathsy}{'134}
\DeclareMathSymbol{\uplus}{2}{drmmathsy}{'135}
\DeclareMathSymbol{\wedge}{2}{drmmathsy}{'136}
\DeclareMathSymbol{\vee}{2}{drmmathsy}{'137}
\DeclareMathSymbol{\vdash}{3}{drmmathsy}{'140}
\DeclareMathSymbol{\dashv}{3}{drmmathsy}{'141}
\fi\fi
%    \end{macrocode}
% Now we define some arrow symbols; there is a surprisingly
% large variety of these.
%    \begin{macrocode}
\ifnodefault\else\ifnodefaultmath\else
\DeclareMathSymbol{\leftarrow}{0}{drmmathsy}{'040}
\DeclareMathSymbol{\rightarrow}{0}{drmmathsy}{'041}
\DeclareMathSymbol{\leftrightarrow}{0}{drmmathsy}{'044}
\DeclareMathSymbol{\nearrow}{0}{drmmathsy}{'045}
\DeclareMathSymbol{\searrow}{0}{drmmathsy}{'046}
\DeclareMathSymbol{\Leftarrow}{0}{drmmathsy}{'050}
\DeclareMathSymbol{\Rightarrow}{0}{drmmathsy}{'051}
\DeclareMathSymbol{\Leftrightarrow}{0}{drmmathsy}{'054}
\DeclareMathSymbol{\nwarrow}{0}{drmmathsy}{'055}
\DeclareMathSymbol{\swarrow}{0}{drmmathsy}{'056}
\DeclareMathSymbol{\wr}{2}{drmmathsy}{'157}
\DeclareMathSymbol{\surd}{0}{drmmathsy}{'160}
\DeclareMathSymbol{\amalg}{2}{drmmathsy}{'161}
\DeclareMathSymbol{\nabla}{0}{drmmathsy}{'162}
\DeclareMathSymbol{\sqcup}{2}{drmmathsy}{'164}
\DeclareMathSymbol{\sqcap}{2}{drmmathsy}{'165}
\DeclareMathSymbol{\sqsubseteq}{2}{drmmathsy}{'166}
\DeclareMathSymbol{\sqsupseteq}{2}{drmmathsy}{'167}
\DeclareMathSymbol{\dagger}{2}{drmmathsy}{'171}
\DeclareMathSymbol{\ddagger}{2}{drmmathsy}{'172}
\DeclareMathSymbol{:}{2}{operators}{'072}
\DeclareMathSymbol{;}{0}{operators}{'073}
\DeclareMathSymbol{.}{0}{letters}{'072}
\DeclareMathSymbol{,}{0}{operators}{'054}
\DeclareMathSymbol{\ldotp}{0}{letters}{'072}
\DeclareMathSymbol{\clubsuit}{0}{letters}{'174}
\DeclareMathSymbol{\diamondsuit}{0}{letters}{'175}
\DeclareMathSymbol{\heartsuit}{0}{letters}{'176}
\DeclareMathSymbol{\spadesuit}{0}{letters}{'177}
\DeclareMathSymbol{\partial}{0}{drmmathsy}{'100}
\fi\fi
%    \end{macrocode}
% Now we define some of the large/small symbols, like |\sum|
% and |\prod|.  It proved necessary to cancel out the
% previous definitions of these, or \LaTeX\ complained about
% them being already defined; it seems that it ought to be
% possible to redefine them only for a given math version,
% but I haven't figured it out yet.
%    \begin{macrocode}
\ifnodefault\else\ifnodefaultmath\else
\let\coprod\relax
\DeclareMathSymbol{\coprod}{\mathop}{drmmathomx}{"60}
\let\bigvee\relax
\DeclareMathSymbol{\bigvee}{\mathop}{drmmathomx}{"57}
\let\bigwedge\relax
\DeclareMathSymbol{\bigwedge}{\mathop}{drmmathomx}{"56}
\let\biguplus\relax
\DeclareMathSymbol{\biguplus}{\mathop}{drmmathomx}{"55}
\let\bigcap\relax
\DeclareMathSymbol{\bigcap}{\mathop}{drmmathomx}{"54}
\let\bigcup\relax
\DeclareMathSymbol{\bigcup}{\mathop}{drmmathomx}{"53}
\let\intop\relax
\DeclareMathSymbol{\intop}{\mathop}{drmmathomx}{"52}
    \def\int{\intop\nolimits}
\let\prod\relax
\DeclareMathSymbol{\prod}{\mathop}{drmmathomx}{"51}
\let\sum\relax
\DeclareMathSymbol{\sum}{\mathop}{drmmathomx}{"50}
\let\bigotimes\relax
\DeclareMathSymbol{\bigotimes}{\mathop}{drmmathomx}{"4E}
\let\bigoplus\relax
\DeclareMathSymbol{\bigoplus}{\mathop}{drmmathomx}{"4C}
\let\bigodot\relax
\DeclareMathSymbol{\bigodot}{\mathop}{drmmathomx}{"4A}
\let\ointcup\relax
\DeclareMathSymbol{\ointop}{\mathop}{drmmathomx}{"48}
    \def\oint{\ointop\nolimits}
\let\bigsqcup\relax
\DeclareMathSymbol{\bigsqcup}{\mathop}{drmmathomx}{"46}
\fi\fi
%    \end{macrocode}
% Moving on to delimiters.
%    \begin{macrocode}
\ifnodefault\else\ifnodefaultmath\else
\DeclareMathSymbol{|}{0}{drmmathsy}{'152}
\let\backslash\relax\DeclareMathSymbol{\backslash}{0}{drmmathsy}{'156}
\fi\fi
%    \end{macrocode}
% Declare the math accents.
%    \begin{macrocode}
\ifnodefault\else\ifnodefaultmath\else
\DeclareMathAccent{\vec}{\mathord}{letters}{126}
\DeclareMathAccent{\acute}{\mathalpha}{operators}{1}
\DeclareMathAccent{\hat}{\mathalpha}{operators}{2}
\DeclareMathAccent{\grave}{\mathalpha}{operators}{0}
\DeclareMathAccent{\check}{\mathalpha}{operators}{7}
\DeclareMathAccent{\bar}{\mathalpha}{operators}{9}
\DeclareMathAccent{\dot}{\mathalpha}{operators}{10}
\DeclareMathAccent{\ddot}{\mathalpha}{operators}{4}
\DeclareMathAccent{\breve}{\mathalpha}{operators}{8}
\DeclareMathAccent{\tilde}{\mathalpha}{operators}{3}
\fi\fi
%    \end{macrocode}
% Declare our math delimiters, so that \TeX's
% delimiter-expanding magic can work with our new
% characters.
%    \begin{macrocode}
\def\bigd#1{\bBigg@{#1}}
\ifnodefault\else\ifnodefaultmath\else
\DeclareMathDelimiter{(}{\mathopen}{operators}{"28}
	{drmmathomx}{"00}
\DeclareMathDelimiter{)}{\mathopen}{operators}{"29}
	{drmmathomx}{"01}
\let\{\relax\let\}\relax
\DeclareMathDelimiter{\{}{\mathopen}{operators}{"7B}
	{drmmathomx}{"08}
\DeclareMathDelimiter{\}}{\mathopen}{operators}{"7D}
	{drmmathomx}{"09}
\DeclareMathDelimiter{[}{\mathopen}{operators}{"5B}
	{drmmathomx}{"02}
\DeclareMathDelimiter{]}{\mathopen}{operators}{"5D}
	{drmmathomx}{"03}
\DeclareMathDelimiter{\lfloor}{\mathopen}{drmmathsy}{"62}
	{drmmathomx}{"04}
\DeclareMathDelimiter{\rfloor}{\mathclose}{drmmathsy}{"63}
	{drmmathomx}{"05}
\DeclareMathDelimiter{\lceil}{\mathopen}{drmmathsy}{"64}
	{drmmathomx}{"06}
\DeclareMathDelimiter{\rceil}{\mathclose}{drmmathsy}{"65}
	{drmmathomx}{"07}
\DeclareMathDelimiter{\langle}{\mathopen}{drmmathsy}{"68}
	{drmmathomx}{"0A}
\DeclareMathDelimiter{\rangle}{\mathclose}{drmmathsy}{"69}
	{drmmathomx}{"0B}
\DeclareMathDelimiter{|}{\mathclose}{drmmathomx}{"0C}
	{drmmathomx}{"0C}
\DeclareMathDelimiter{\vert}{\mathclose}{drmmathomx}{"0C}
	{drmmathomx}{"0C}
\DeclareMathDelimiter{\|}{\mathclose}{drmmathomx}{"0D}
	{drmmathomx}{"0D}
\DeclareMathDelimiter{\Vert}{\mathclose}{drmmathomx}{"0D}
	{drmmathomx}{"0D}
\DeclareMathDelimiter{\uparrow}{\mathrel}{drmmathomx}{"78}
	{drmmathomx}{"78}
\DeclareMathDelimiter{\downarrow}{\mathrel}{drmmathomx}{"79}
	{drmmathomx}{"79}
\DeclareMathDelimiter{\updownarrow}{\mathrel}{drmmathsy}{"6C}
	{drmmathomx}{"3F}
\DeclareMathDelimiter{\Uparrow}{\mathrel}{drmmathsy}{"2A}
	{drmmathomx}{"7E}
\DeclareMathDelimiter{\Downarrow}{\mathrel}{drmmathsy}{"2B}
	{drmmathomx}{"7F}
\DeclareMathDelimiter{\Updownarrow}{\mathrel}{drmmathsy}{"6D}
	{drmmathomx}{"77}
\fi\fi
%    \end{macrocode}
% Next, we define a \emph{math radical}, which essentially
% means a square root sign.  Curiously, the thickness of the
% rule enclosing the square root sign is governed by the
% \emph{height} of the square root character; this means
% that almost the entire character is \emph{depth}.  I had
% to jimmy a bit with the default \LaTeX\ |\sqrt| definition
% to make the root numbers (say, the 3 for the cube root)
% line up properly, as well, which is what all the |\r@@t|
% business here is.
%    \begin{macrocode}
\ifnodefault\else\ifnodefaultmath\else
\DeclareMathRadical{\sqrtsign}{drmmathsy}{"70}{drmmathomx}{"70}
\DeclareRobustCommand\sqrt{\@ifnextchar[\@sqrt\sqrtsign}
\def\r@@t#1#2{
	\setbox\z@\hbox{$\m@th#1\sqrtsign{#2}$}
	\dimen@\ht\z@ \advance\dimen@-\dp\z@
	\mkern5mu\raise.8\dimen@\copy\rootbox
	\mkern-7mu\box\z@}
\fi\fi
%    \end{macrocode}
% Now, finally, we declare |drmmath| to be the default math
% version, so that all this will become the norm in a
% document declaring the |drm| package.  Unless, of course,
% either |nodefault| or |nodefaultmath| has been specified
% as an option.
%    \begin{macrocode}
\ifnodefault\else\ifnodefaultmath\else
	\mathversion{drmmath}
\fi\fi
%    \end{macrocode}
% Now we define the special symbols.  First, we define
% |\drmsym|, which takes a single argument to be typeset
% from the |drmsym| font.   Then we define a (rather huge)
% macro for redefining all the symbols.  This macro will be
% called only if |nodefault| or |nodefaulttext| have not
% been selected, or if |symbolsonly| has been selected.
%    \begin{macrocode}
\def\drmsym#1{{\fontencoding{TS1}\selectfont\fontfamily{drm}\selectfont#1}}
\def\drmsymbolredef{
	\def\textbigcircle{\drmsym{\char'117}}
	\def\textregistered{\drmsym{\char'256}}
	\def\texttrademark{\drmsym{\char'227}}
	\def\textservicemark{\drmsym{\char'237}}
	\def\textsoundrecording{\drmsym{\char'255}}
	\def\textcopyright{\drmsym{\char'251}}
	\def\textcopyleft{\drmsym{\char'253}}
	\def\textborn{\drmsym{\char'142}}
	\def\textdied{\drmsym{\char'144}}
	\def\textdivorced{\drmsym{\char'143}}
	\def\textmarried{\drmsym{\char'155}}
	\def\textleaf{\drmsym{\char'154}}
	\def\textmale{\drmsym{\char'153}}
	\def\textfemale{\drmsym{\char'145}}
	\def\textcrusadecross{\drmsym{\char'130}}
	\def\textcrusadecrossoutline{\drmsym{\char'131}}
	\def\textlatincross{\drmsym{\char'144}}
	\def\textlatincrossoutline{\drmsym{\char'134}}
	\def\textgreekcross{\drmsym{\char'170}}
	\def\textgreekcrossoutline{\drmsym{\char'171}}
	\def\textsaltirecross{\drmsym{\char'172}}
	\def\textsaltirecrossoutline{\drmsym{\char'173}}
	\def\texteucharist{\drmsym{\char'120}}
	\def\textstardavid{\drmsym{\char'140}}
	\def\textstardavidsolid{\drmsym{\char'141}}
	\def\textstardavidoutline{\drmsym{\char'151}}
	\def\textsun{\drmsym{\char'330}}
	\def\textsunvar{\drmsym{\char'331}}
	\def\textwaxcrescent{\drmsym{\char'332}}
	\def\textfullmoon{\drmsym{\char'333}}
	\def\textwanecrescent{\drmsym{\char'334}}
	\def\textnewmoon{\drmsym{\char'335}}
	\def\textmercury{\drmsym{\char'336}}
	\def\textearth{\drmsym{\char'337}}
	\def\textterra{\drmsym{\char'337}}
	\def\textearthvar{\drmsym{\char'340}}
	\def\textterravar{\drmsym{\char'340}}
	\def\textmars{\drmsym{\char'153}}
	\def\textvenus{\drmsym{\char'145}}
	\def\textjupiter{\drmsym{\char'341}}
	\def\textsaturn{\drmsym{\char'342}}
	\def\texturanus{\drmsym{\char'343}}
	\def\texturanusvar{\drmsym{\char'344}}
	\def\textneptune{\drmsym{\char'345}}
	\def\textceres{\drmsym{\char'346}}
	\def\textpallas{\drmsym{\char'347}}
	\def\textjuno{\drmsym{\char'350}}
	\def\textjunovar{\drmsym{\char'351}}
	\def\textvesta{\drmsym{\char'352}}
	\def\textvestavar{\drmsym{\char'353}}
	\def\textastraea{\drmsym{\char'354}}
	\def\textastraeavar{\drmsym{\char'355}}
	\def\texthebe{\drmsym{\char'356}}
	\def\textiris{\drmsym{\char'357}}
	\def\textaries{\drmsym{\char'360}}
	\def\textari{\drmsym{\char'360}}
	\def\texttaurus{\drmsym{\char'361}}
	\def\texttau{\drmsym{\char'361}}
	\def\textgemini{\drmsym{\char'362}}
	\def\textgem{\drmsym{\char'362}}
	\def\textcancer{\drmsym{\char'363}}
	\def\textcnc{\drmsym{\char'363}}
	\def\textleo{\drmsym{\char'364}}
	\def\textvirgo{\drmsym{\char'365}}
	\def\textvir{\drmsym{\char'365}}
	\def\textlibra{\drmsym{\char'367}}
	\def\textlib{\drmsym{\char'367}}
	\def\textscorpius{\drmsym{\char'370}}
	\def\textsco{\drmsym{\char'370}}
	\def\textsagittarius{\drmsym{\char'371}}
	\def\textsgr{\drmsym{\char'371}}
	\def\textcapricorn{\drmsym{\char'372}}
	\def\textcap{\drmsym{\char'372}}
	\def\textaquarius{\drmsym{\char'373}}
	\def\textaqr{\drmsym{\char'373}}
	\def\textpisces{\drmsym{\char'374}}
	\def\textpsc{\drmsym{\char'374}}
	\def\textpluto{\drmsym{\char'375}}
	\def\textplutovar{\drmsym{\char'376}}
	\def\textstar{\drmsym{\char'142}}
	\def\textcomet{\drmsym{\char'377}}
	\def\textquadrature{\drmsym{\char'310}}
	\def\textopposition{\drmsym{\char'311}}
	\def\textconjunction{\drmsym{\char'312}}
	\def\textascendingnode{\drmsym{\char'315}}
	\def\textdescendingnode{\drmsym{\char'314}}
	\def\textdollarsign{\drmsym{\char'044}}
	\def\textolddollarsign{\drmsym{\char'212}}
	\def\textcentsign{\drmsym{\char'242}}
	\def\textoldcentsign{\drmsym{\char'213}}
	\def\textpoundsterling{\drmsym{\char'243}}
	\def\textoldpoundsterling{\drmsym{\char'222}}
	\def\textlira{\drmsym{\char'222}}
	\def\texteuro{\drmsym{\char'277}}
	\def\textyen{\drmsym{\char'245}}
	\def\textbaht{\drmsym{\char'232}}
	\def\textcolon{\drmsym{\char'215}}
	\def\textdong{\drmsym{\char'226}}
	\def\textflorin{\drmsym{\char'214}}
	\def\textguarani{\drmsym{\char'220}}
	\def\textnaira{\drmsym{\char'217}}
	\def\textpeso{\drmsym{\char'221}}
	\def\textruble{\drmsym{\char'221}}
	\def\textwon{\drmsym{\char'216}}
	\def\textcurrency{\drmsym{\char'244}}
	\def\romone{\drmsym{\char'100}}
	\def\romfive{\drmsym{\char'101}}
	\def\romten{\drmsym{\char'102}}
	\def\romfifty{\drmsym{\char'103}}
	\def\romhundred{\drmsym{\char'104}}
	\def\romfivehundred{\drmsym{\char'105}}
	\def\romthousand{\drmsym{\char'106}}
	\def\liningzero{\drmsym{\char'060}}
	\def\liningone{\drmsym{\char'061}}
	\def\liningtwo{\drmsym{\char'062}}
	\def\liningthree{\drmsym{\char'063}}
	\def\liningfour{\drmsym{\char'064}}
	\def\liningfive{\drmsym{\char'065}}
	\def\liningsix{\drmsym{\char'066}}
	\def\liningseven{\drmsym{\char'067}}
	\def\liningeight{\drmsym{\char'070}}
	\def\liningnine{\drmsym{\char'071}}
	\def\textnumero{\drmsym{\char'233}}
	\def\textrefmark{\drmsym{\char'270}}
	\def\textasterism{\drmsym{\char'302}}
	\def\textfeminineordinal{\drmsym{\char'252}}
	\def\textmasculineordinal{\drmsym{\char'272}}
	\def\textsupone{\drmsym{\char'271}}
	\def\textsuptwo{\drmsym{\char'262}}
	\def\textsupthree{\drmsym{\char'263}}
	\def\textpilcrowsolid{\drmsym{\char'231}}
	\def\textpilcrowoutline{\drmsym{\char'266}}
	\def\textsection{\drmsym{\char'247}}
	\def\textdagger{\drmsym{\char'204}}
	\def\textdag{\drmsym{\char'204}}
	\def\dag{\drmsym{\char'204}}
	\def\textdbldagger{\drmsym{\char'205}}
	\def\textdbldag{\drmsym{\char'205}}
	\def\dbldag{\drmsym{\char'205}}
	\def\textpipe{\drmsym{\char'206}}
	\def\textbrokenpipe{\drmsym{\char'246}}
	\def\textprime{\drmsym{\char'264}}
	\def\textdoubleprime{\drmsym{\char'303}}
	\def\texttripleprime{\drmsym{\char'313}}
	\def\textsqrt{\drmsym{\char'273}}
	\def\textquarter{\drmsym{\char'274}}
	\def\texthalf{\drmsym{\char'275}}
	\def\textthreequarters{\drmsym{\char'276}}
	\def\textthird{\drmsym{\char'304}}
	\def\texttwothirds{\drmsym{\char'305}}
	\def\textpermille{\drmsym{\char'207}}
	\def\textperbiqua{\drmsym{\char'207}}
	\def\textpertenmille{\drmsym{\char'230}}
	\def\textpertriqua{\drmsym{\char'230}}
	\def\textequals{\drmsym{\char'055}}
	\def\textslash{\drmsym{\char'057}}
	\def\texttimes{\drmsym{\char'326}}
	\def\textdiv{\drmsym{\char'366}}
	\def\textradiation{\drmsym{\char'161}}
	\def\textradiationnocircle{\drmsym{\char'160}}
	\def\textbiohazard{\drmsym{\char'163}}
	\def\textbiohazardnocircle{\drmsym{\char'162}}
	\def\texthighvoltage{\drmsym{\char'166}}
	\def\texthighvoltagenotriangle{\drmsym{\char'165}}
	\def\textgeneralwarning{\drmsym{\char'164}}
	\def\textuparrow{\drmsym{\char'136}}
	\def\textdownarrow{\drmsym{\char'137}}
	\def\textleftarrow{\drmsym{\char'030}}
	\def\textrightarrow{\drmsym{\char'031}}
	\def\textrecipe{\drmsym{\char'223}}
	\def\textintbang{\drmsym{\char'224}}
	\def\textopenintbang{\drmsym{\char'225}}
	\def\textbullet{\drmsym{\char'210}}
	\def\textopenbullet{\drmsym{\char'236}}
	\def\textheart{\drmsym{\char'174}}
	\def\textopenheart{\drmsym{\char'175}}
	\def\texteighthnote{\drmsym{\char'156}}
	\def\textdiamond{\drmsym{\char'306}}
	\def\textopendiamond{\drmsym{\char'307}}
	\def\textlozenge{\drmsym{\char'307}}
	\def\textdegree{\drmsym{\char'260}}
	\def\texttilde{\drmsym{\char'176}}
	\def\textasciitilde{\texttilde}
	\def\tilde{\texttilde}
	\def\textasciicircum{\drmsym{\char'002}}
	\def\textdegreec{\drmsym{\char'211}}
	\def\textrightupfleuron{\drmsym{\char'016}}
	\def\textrightdownfleuron{\drmsym{\char'017}}
	\def\textleftupfleuron{\drmsym{\char'020}}
	\def\textleftdownfleuron{\drmsym{\char'021}}
	\def\textupleftfleuron{\drmsym{\char'050}}
	\def\textuprightfleuron{\drmsym{\char'051}}
	\def\textdownrightfleuron{\drmsym{\char'077}}
	\def\textdownleftfleuron{\drmsym{\char'107}}
	\def\textsquaretulip{\drmsym{\char'023}}
	\def\textsquaretulipside{\drmsym{\char'046}}
	\def\textupdoubletulip{\drmsym{\char'024}}
	\def\textdowndoubletulip{\drmsym{\char'027}}
	\def\textrightdoubletulip{\drmsym{\char'036}}
	\def\textleftdoubletulip{\drmsym{\char'037}}
	\def\textupleftcornertulip{\drmsym{\char'053}}
	\def\textuprightcornertulip{\drmsym{\char'072}}
	\def\textlowleftcornertulip{\drmsym{\char'073}}
	\def\textlowrightcornertulip{\drmsym{\char'110}}
	\def\textupsingletuliplong{\drmsym{\char'111}}
	\def\textdownsingletuliplong{\drmsym{\char'112}}
	\def\textleftsingletuliplong{\drmsym{\char'113}}
	\def\textrightsingletuliplong{\drmsym{\char'114}}
	\def\textupsingletulip{\drmsym{\char'116}}
	\def\textdownsingletulip{\drmsym{\char'121}}
	\def\textleftsingletulip{\drmsym{\char'122}}
	\def\textrightsingletulip{\drmsym{\char'123}}
	\def\spearright{\drmsym{\char'124}}
	\def\spearleft{\drmsym{\char'125}}
	\def\horizspearext{\drmsym{\char'126}}
	\def\spearup{\drmsym{\char'132}}
	\def\speardown{\drmsym{\char'146}}
	\def\vertspearext{\drmsym{\char'147}}
	\def\fleurdelis{\drmsym{\char'157}}
	\def\fleurdelys{\drmsym{\char'157}}
	\def\fleurdelisdown{\drmsym{\char'167}}
	\def\fleurdelysdown{\drmsym{\char'167}}
	\def\fleurdelisleft{\drmsym{\char'316}}
	\def\fleurdelysleft{\drmsym{\char'316}}
	\def\fleurdelisright{\drmsym{\char'177}}
	\def\fleurdelysright{\drmsym{\char'177}}
	\def\woundcordleftext{\drmsym{\char'317}}
	\def\woundcordrightext{\drmsym{\char'324}}
	\def\woundcordleftend{\drmsym{\char'320}}
	\def\woundcordrightend{\drmsym{\char'321}}
	\def\woundcordleftendinv{\drmsym{\char'323}}
	\def\woundcordrightendinv{\drmsym{\char'322}}
}
\ifnodefault\else\ifnodefaulttext\else
	\drmsymbolredef
\fi\fi
\ifsymbolsonly\drmsymbolredef\fi
%    \end{macrocode}
% Now, rather than require people to enter the Roman numeral
% macros by hand, we provide a command, |\romanize|, which
% takes as its only argument an Indo-Arabic numeral and
% converts it into a Roman numeral.  This macro is a thin
% wrapper around one from the |modroman| package, and in
% fact requires |modroman| to work.
%    \begin{macrocode}
\def\romanize#1{%
	\RedefineMRmdclxvij{\romthousand}{\romfivehundred}
		{\romhundred}{\romfifty}{\romten}{\romfive}{\romone}{\romone}%
	\nbshortroman{#1}%
}%
%    \end{macrocode}
% We also define a command for producing lining numerals
% rather than old-style figures, so that these long-winded
% command names don't need to be typed if lining numerals
% will be used frequently.  It takes the number to be output
% as lining as its only argument.
%    \begin{macrocode}
\def\liningnums#1{%
	\drmsym{#1}%
}%
%    \end{macrocode}
% Now we define the decorative tulip frame macro, as an
% example of the beautiful constructions which are possible
% with fleurons and other textual ornaments.
%    \begin{macrocode}
\def\tulipframe#1{%
	\vbox{%
		\hbox to\linewidth{\hfil%
			{\drmsym{\char'053}}%
				{\drmsym{\char'111}}%
				{\drmsym{\char'024}}%
				{\drmsym{\char'111}}%
				{\drmsym{\char'072}}\hfil}%
		\vskip-0.5\baselineskip%
		\hbox to\linewidth{\hfil%
			#1%
			\hfil%
		}%
		\vskip-0.5\baselineskip%
		\hbox to\linewidth{\hfil%
			{\drmsym{\char'073}}%
				{\drmsym{\char'112}}%
				{\drmsym{\char'027}}%
				{\drmsym{\char'112}}%
				{\drmsym{\char'110}}\hfil}%
	}%
}%
%    \end{macrocode}
% Next, we define the macros for the extensible rules.  Lots
% of down-and-dirty \TeX\ stuff here.
%    \begin{macrocode}
\newcount\counterA
\newcount\counterB
\newcount\iter
\newlength{\extcharwid}
\newlength{\leftcharwid}
\newlength{\rightcharwid}
\newlength{\greaterwid}
\def\extrule#1#2#3#4#5{%
	\if#1h%
		\settowidth{\extcharwid}{#5}%
		\settowidth{\leftcharwid}{#3}%
		\settowidth{\rightcharwid}{#4}%
		\counterB=\numexpr\dimexpr#2\relax\relax%
		\advance\counterB by-\leftcharwid%
		\advance\counterB by-\rightcharwid%
		\counterA=\dimexpr\extcharwid\relax%
		\divide\counterB by\counterA%
		\iter=0%
		\noindent#3%
		\loop%
			\ifnum\iter<\counterB%
				\advance\iter by 1%
				#5%
				\repeat%
		#4%
	\fi%
	\if#1v%
		\setbox0=\hbox{#3}%
		\leftcharwid=\ht0\advance\leftcharwid by\dp0%
		\setbox0=\hbox{#4}%
		\rightcharwid=\ht0\advance\rightcharwid by\dp0%
		\setbox0=\hbox{#5}%
		\extcharwid=\ht0\advance\extcharwid by\dp0%
		\ifdim\leftcharwid>\rightcharwid%
			\greaterwid=\leftcharwid%
		\else%
			\greaterwid=\rightcharwid%
		\fi\if\extcharwid>\greaterwid%
			\greaterwid=\extcharwid%
		\fi%
		\counterB=\numexpr\dimexpr#2\relax\relax%
		\advance\counterB by-\leftcharwid%
		\advance\counterB by-\rightcharwid%
		\counterA=\dimexpr\extcharwid\relax%
		\divide\counterB by\counterA%
		\iter=0%
		\noindent\vbox to\dimexpr#2{\baselineskip=0pt%
			\hbox to\greaterwid{\hfil#4\hfil}%
			\loop%
				\ifnum\iter<\counterB%
					\advance\iter by 1%
					\vss\hbox to\greaterwid{\hfil#5\hfil}%
					\repeat%
			\vss\hbox to\greaterwid{\hfil#3\hfil}}%
	\fi%
}%
%    \end{macrocode}
% Next, we move on to define the unreasonably complex and
% configurable ellipsis commands.  First we define the
% |\drmelip|, then the four-dotted |\drmfelip|.
%    \begin{macrocode}
\newlength{\drmelipgap}\setlength{\drmelipgap}{2.9pt}
\newlength{\drmelipbef}\setlength{\drmelipbef}{2.4pt}
\newlength{\drmelipaft}\setlength{\drmelipaft}{1.4pt}
\def\drmelipchar{.}
\def\drmelip{%
	\hbox{%
		\hbox to\the\drmelipbef{\hfil}%
		\drmelipchar%
		\hbox to\drmelipgap{\hfil}%
		\drmelipchar%
		\hbox to\drmelipgap{\hfil}%
		\drmelipchar%
		\hbox to\drmelipaft{\hfil}%
	}%
}%
\newlength{\drmfelipbef}\setlength{\drmfelipbef}{0pt}
\newlength{\drmfelipaft}\setlength{\drmfelipaft}{\the\drmelipaft}
\newlength{\drmfelipwid}
\def\drmfelip{%
	\hbox{%
		\hbox to\the\drmfelipbef{\hfil}%
		\drmelipchar%
		\hbox to\drmelipgap{\hfil}%
		\drmelipchar%
		\hbox to\drmelipgap{\hfil}%
		\drmelipchar%
		\hbox to\drmelipgap{\hfil}%
		\drmelipchar%
		\hbox to\drmfelipaft{\hfil}%
	}%
}%
%    \end{macrocode}
% Now, we begin the decorative initials.  These are designed
% using a common background written in \MP\ with a DRM
% figure superimposed, so a great deal of the code in this
% section is, in fact, \MP\ rather than \TeX\ or \MF.
%
% We begin by defining |\drmdecinit|, which takes five
% arguments:  the width, the height, the color of the
% background, the color of the letter, and the letter
% itself.  It includes a \MP\ macro, |along|, derived from
% \url{http://tex.stackexchange.com/questions/176665/define-a-pair-point-along-a-path-length-metapost}.
%    \begin{macrocode}
\def\drmdecinitfontdefault{%
	\def\drmdecinitfont{%
		\unexpanded{\font\drminitfontcom=drm10} 
	}%
}%
\def\drmdecinitfont{%
	\unexpanded{\font\drminitfontcom=drm10}  %
}%
\def\drmdecinit#1#2#3#4#5{%
	\begin{mpost}
	primarydef pct along pat =
		(arctime (pct * (arclength pat)) of pat) of pat
	enddef;
	w=#1; h=#2;
	pen thinpen; thinpen = pencircle scaled (w/288);
	pen medpen; medpen = pencircle scaled (w/144);
	pen thickpen; thickpen = pencircle scaled (w/144 + w/288);
	pen ththickpen; ththickpen = pencircle scaled (w/72);
	pen thththickpen; thththickpen = pencircle scaled (w/36);
	leaflen = w/6.5;
	leafletlen = w/20;
	leafletwid = w/40;
	leafletgap = w/40;
	pen leafpen; leafpen = pencircle xscaled leafletlen
		yscaled leafletwid;
	pen sideleafpen; sideleafpen = pencircle yscaled leafletlen
		xscaled leafletwid rotated -35;
	def border =
		pickup thththickpen;
		draw top lft (0,h)--top rt(w,h)--bot rt(w,0)--bot
			lft(0,0)--cycle withcolor #3;
		pickup ththickpen;
		draw (top lft (0,h)--top rt(w,h)--bot rt(w,0)--bot
			lft(0,0)--cycle) scaled 0.95 shifted (0.025w,0.025h) 
			withcolor #3;
	enddef;
	def leaf(expr p,s,r,t) =
		path leafpath;
		leafpath = ((p shifted (leafletlen/2,0))..
			(p shifted (0,leafletwid/2))..
			(p shifted (-leafletlen/2,0))..
			(p shifted (0,-leafletwid/2))..cycle) 
			rotatedaround (p,s);
		if t = 0:
			fill leafpath withcolor r;
		elseif t = 1:
			fill leafpath reflectedabout
				((w/2,h),(w/2,0)) withcolor r;
		elseif t = 2:
			fill leafpath reflectedabout
				((0,h/2),(w,h/2)) withcolor r;
		elseif t = 3:
			fill leafpath reflectedabout
				((0,h/2),(w,h/2)) reflectedabout ((w/2,h),(w/2,0))
				withcolor r;
		fi
	enddef;
	def branch(expr p,s,b) =
		pickup thickpen;
		pair t; t = p rotatedaround (p,s);
		pair u; u = point 1.0 along (t{dir (s+90)}..
			t shifted (0.3leaflen,leaflen) rotatedaround (t,s));
		pair q; q = (t shifted (1.4leafletwid,0)) rotatedaround (t,s);
		pair v; v = (q shifted (0.3leaflen,leaflen)) rotatedaround (q,s);
		pair r; r = (t shifted (-1.4leafletwid,0)) rotatedaround (t,s);
		pair a; a = (r shifted (0.3leaflen,leaflen)) rotatedaround (r,s);
		if b = 0:
			draw (t{dir (s+90)}..u) withcolor #3;
		elseif b = 1:
			draw (t{dir (s+90)}..u) reflectedabout ((w/2,h),(w/2,0))
				withcolor #3;
		elseif b = 2:
			draw (t{dir (s+90)}..u) reflectedabout ((0,h/2),(w,h/2))
				withcolor #3;
		elseif b = 3:
			draw (t{dir (s+90)}..u) reflectedabout ((0,h/2),(w,h/2))
				reflectedabout ((w/2,h),(w/2,0)) withcolor #3;
		fi
		leaf((point 0.15 along (q{dir (s+90)}..v)),s,#3,b);
		leaf((point 0.45 along (q{dir (s+90)}..v)),s,#3,b);
		leaf((point 0.75 along (q{dir (s+90)}..v)),s,#3,b);
		leaf((point 0.15 along (r{dir (s+90)}..a)),s,#3,b);
		leaf((point 0.45 along (r{dir (s+90)}..a)),s,#3,b);
		leaf((point 0.75 along (r{dir (s+90)}..a)),s,#3,b);
		leaf((point 0.98 along (t{dir (s+90)}..u)),s+60,white,b);
		leaf((point 1.00 along (t{dir (s+90)}..u)),s+60,#3,b);
	enddef;
	def football(expr p) =
		pickup thinpen;
		draw (z26..z30..z31..z26..z32..z33..cycle)
			rotatedaround ((w/2,h/2),p) withcolor #3;
		pickup ththickpen;
		draw (z20..z24..z21) rotatedaround ((w/2,h/2),p)
			withcolor white;
		draw (z21..z25..z20) rotatedaround ((w/2,h/2),p)
			withcolor white;
		draw (z20..tension 1.4..z22..z23..tension 1.5..z20)
			rotatedaround ((w/2,h/2),p) withcolor white;
		draw (z21..tension 1.4..z22..z23..tension 1.5..z21)
			rotatedaround ((w/2,h/2),p) withcolor white;
		pickup medpen;
		draw (z20..z24..z21) rotatedaround ((w/2,h/2),p)
			withcolor #3;
		draw (z21..z25..z20) rotatedaround ((w/2,h/2),p)
			withcolor #3;
		draw (z20..tension 1.4..z22..z23..tension 1.5..z20)
			rotatedaround ((w/2,h/2),p) withcolor #3;
		draw (z21..tension 1.4..z22..z23..tension 1.5..z21)
			rotatedaround ((w/2,h/2),p) withcolor #3;
		fill (z34..z36..z35..z37..cycle) rotatedaround
			((w/2,h/2),p) withcolor #3;
		fill (z38..z40..z39..z41..cycle) rotatedaround
			((w/2,h/2),p) withcolor #3;
		fill (z42..z44..z43..z45..cycle) rotatedaround
			((w/2,h/2),p) withcolor #3;
	enddef;
	border;
	z0 = (w-3.3leafletwid-(w/11),h-leaflen-(w/144));
	z1 = (w/2,2h/3);%h-leafletwid-2pt);
	z2 = z0 reflectedabout ((w/2,h),(w/2,0));
	z3 = (2w/3,h/2);%w-leafletwid-2pt,h/2);
	z4 = z0 reflectedabout ((w,h/2),(0,h/2));
	z5 = z1 reflectedabout ((0,h/2),(w,h/2));
	z6 = z4 reflectedabout ((w/2,h),(w/2,0));
	z7 = z3 reflectedabout ((w/2,h),(w/2,0));
	pickup thickpen;
	draw z0{dir -120}..{left}z1{left}..{dir 120}z2 withcolor #3;
	draw z0{dir -120}..{down}z3{down}..{dir -60}z4 withcolor #3;
	draw z4{dir 120}..{left}z5{left}..{dir -120}z6 withcolor #3;
	draw z6{dir 60}..{up}z7{up}..{dir 120}z2 withcolor #3;
	branch(z0,-30,0);
	branch(z0,-30,1);
	branch(z0,-30,2);
	branch(z0,-30,3);
	z10 = (w/2,h-leafletwid-(w/72));
	z11 = z10 rotatedaround ((w/2,h/2),90);
	z12 = z10 rotatedaround ((w/2,h/2),180);
	z13 = z10 rotatedaround ((w/2,h/2),270);
	path greatcirc; greatcirc = z10..z11..z12..z13..cycle;
	pickup thththickpen;
	draw greatcirc withcolor white;
	pickup ththickpen;
	draw greatcirc withcolor #3;
	z20 = z2 shifted (0,-leafletlen);
	z21 = z6 shifted (0,leafletlen);
	z22 = z11 shifted (leafletlen,0);
	z23 = z7 shifted (-leafletlen,0);
	z24 = 0.25[z11,z7];
	z25 = 0.75[z11,z7];
	z26 = 0.5[z11,z7];
	z27 = 0.25[z2,z6];
	z28 = 0.5[z2,z6];
	z29 = 0.75[z2,z6];
	z30 = 0.5[z11,z2];
	z31 = 0.5[z7,z2];
	z32 = 0.5[z6,z11];
	z33 = 0.5[z6,z7];
	z34 = z26 shifted (-leafletlen,0);
	z35 = z26 shifted (leafletlen,0);
	z36 = z26 shifted (0,leafletwid);
	z37 = z26 shifted (0,-leafletwid);
	z38 = z27 shifted (-0.8leafletwid,0);
	z39 = z27 shifted (0.8leafletwid,0);
	z40 = z27 shifted (0,0.8leafletlen);
	z41 = z27 shifted (0,-0.8leafletlen);
	z42 = z29 shifted (-0.8leafletwid,0);
	z43 = z29 shifted (0.8leafletwid,0);
	z44 = z29 shifted (0,0.8leafletlen);
	z45 = z29 shifted (0,-0.8leafletlen);
	football(0);
	football(90);
	football(180);
	football(270);
	z50 = z1 shifted (0,-leafletwid);
	z51 = z3 shifted (-leafletwid,0);
	z52 = z5 shifted (0,leafletwid);
	z53 = z7 shifted (leafletwid,0);
	z54 = 0.4[(w/2,h/2),(0,h)];
	z55 = 0.4[(w/2,h/2),(w,h)];
	z56 = 0.4[(w/2,h/2),(w,0)];
	z57 = 0.4[(w/2,h/2),(0,0)];
	pickup thickpen;
	draw z50..z51..z52..z53..cycle withcolor #3;
	pickup medpen;
	draw z50{left}..z54 withcolor #3;
	draw z50{right}..z55 withcolor #3;
	draw z51{up}..z55 withcolor #3;
	draw z51{down}..z56 withcolor #3;
	draw z52{right}..z56 withcolor #3;
	draw z52{left}..z57 withcolor #3;
	draw z53{down}..z57 withcolor #3;
	draw z53{up}..z54 withcolor #3;
	z60 = z50 shifted (0,-leafletwid);
	z61 = z51 shifted (-leafletwid,0);
	z62 = z52 shifted (0,leafletwid);
	z63 = z53 shifted (leafletwid,0);
	z64 = (w/2,h/2) shifted (0,leafletlen);
	z65 = (w/2,h/2) shifted (leafletlen,0);
	z66 = (w/2,h/2) shifted (0,-leafletlen);
	z67 = (w/2,h/2) shifted (-leafletlen,0);
	z68 = 0.5[z64,z65];
	z69 = 0.5[z65,z66];
	z70 = 0.5[z66,z67];
	z71 = 0.5[z67,z64];
	z72 = 0.5[z60,z61];
	z73 = 0.5[z61,z62];
	z74 = 0.5[z62,z63];
	z75 = 0.5[z63,z60];
	z76 = 0.2[z71,z75];
	z77 = point 0.4 along (z63{up}..{right}z60);
	z78 = point 0.6 along (z63{up}..{right}z60);
	z79 = 0.6[z63,z76];
	pickup thinpen;
	path innerbord; innerbord = 
		z60{z64-z60}..z71..{z63-z67}z63{up}..{right}z60;
	draw innerbord withcolor #3;
	draw innerbord rotatedaround ((w/2,h/2),90) withcolor #3;
	draw innerbord rotatedaround ((w/2,h/2),180) withcolor #3;
	draw innerbord rotatedaround ((w/2,h/2),270) withcolor #3;
	label(btex {\drmdecinitfont at#2\unexpanded{\drminitfontcom} #5} etex,(w/2,h/2)) 
		withcolor #4;
	\end{mpost}
}
%    \end{macrocode}
% And that's the end.  Thanks for reading, 
% folks; please email me with any suggestions or improvements.
%
% \appendix
%
% \section{The Secret History:  Building DRM}
%
% \lettrine{H}{ere you can get all} the answers to questions about DRM
% that you didn't have and never asked.  This appendix is
% essentially an exercise in self-gratification, to explain
% a few things about the fonts and why I made some of the
% choices that I did.  As such, it'll probably be
% interesting to few, if any; but here it all is anyway.
%
% \subsection{About the Name}
% \label{sub:aboutname}
%
% When I started this font, I was trying to ape an old-style
% Caslon specimen that I'd found on the Internet.  (If you
% search for one, you'll doubtlessly find the one I was
% going for; it's littered all over the place.)  You can
% still see certain traces of this, particularly in the long
% tail of the Q (there it is!), and in the serifs on the E
% and F.  On the other hand, even at the very beginning,
% before the font had taken on a character of its own, I was
% doing a pretty poor job of imitating this other one.  My
% serifs were fairly prominent, but only slightly bracketed;
% there was a pretty drastic distinction between thick and
% thin strokes; it had a vertical orientation.  Before long,
% it was clear that I had a very different font.
%
% So the name was ``DRM,'' for ``Day Roman Modern.''  But
% this didn't really accurately describe the font, and it
% didn't keep this meaning for long.  (Maybe a few days; the
% original files were titled ``dayroman,'' an even more
% inaccurate appellation.)  I've since backronymed this to
% ``Don's Revised Modern,'' which still isn't strictly
% correct, but it's pretty well ensconced at the moment.
% There are a limited number of descriptors with those
% initials, and I've grown pretty fond of those initials; it
% would be difficult for me to think of the font with any
% other name.  But if somebody has a better backronym, I'd
% love to hear it.
%
% \subsection{Why \MF?}
% \label{sub:whymf}
%
% So why \MF?  Isn't \MF\ horribly out of date,
% unconscionably producing nasty, decrepit bitmapped glyphs
% instead of shiny, futuristic outlines?  Doesn't it somehow
% involve hatred of mom's apple pie, summertime barbecues,
% and the girl next door?
%
% Well, in some ways \MF\ is certainly out of date.  It's
% limited to eight bits (\liningnums{2}$^8$ characters), for example, and
% that limit can't be circumvented by any trivial means.
% Due to the brilliance of its author, it has several
% similar limits which, while seeming arbitrary and
% capricious to us today, were absolutely necessary for
% allowing \MF\ to run on the machines available at the time
% it was produced.  There's really no denying this, and I'd
% be the last to try.
%
% On the other hand, \MF\ is not out of date for the reasons
% most people who eloquently pronounce its obsolescence believe
% it is.  The bitmapped glyph issue, for example; there
% really is nothing wrong with this.  In fact, in some ways
% it's a benefit.  Scaling is not really an issue,
% particularly in this age when \MF\ can be run
% automatically when \TeX\ encounters a size that it doesn't
% already have on hand; we can easily acquire fonts of
% whatever size we need.  And, much like \MF's eight-bit
% stricture, whatever memory benefits come from storing
% fonts as outlines rather than bitmaps is surely irrelevant
% in this day and age.
%
% Bitmaps are beneficial in that they remind us that optical
% sizing is still important; outline fonts have made us
% lazy, preventing the development of real font families
% with many optical sizes.  Too many amateurs (a term I use
% without derision, and proudly apply to myself) think they
% can avoid designing optical sizes because their outlines
% can be automatically scaled.  This leads to poor results.
%
% But most importantly, \MF\ makes writing
% parameter-based fonts easy.  The \textb{bold} and
% \textl{light} versions of the DRM fonts, for example, were
% produced by modifying only a few parameters of the base
% DRM roman font; the actual letterforms remain the same.
% This is a powerful tool that assists greatly in the
% creation of \emph{families} of fonts.
%
% \MF\ also lends itself quite nicely to customization.  As
% a command-line program, I can easily script it to produce
% proofs, or to produce real fonts for inclusion in test
% documents, or to produce font charts, or all of the above.
% For example, your author used scripts to compile proofs as
% well as working fonts to produce this documentation; it
% was relatively trivial to produce a script which would,
% inelegantly but quite effectively, output proof sheets and
% sample texts along with charts of each individual font,
% from simple roman text to quite complex math.  Below is an
% example of the (rather messy, but functional) script I
% used to produce font charts and sample texts for the
% fonts:
%
% \begin{verbatim}
% #!/bin/bash
% # +AMDG
% 
% re="^drm([m|b|bx|c|sym]*)([n|it|sl|sc]*)([0-9]{1,2})$";
% ifmath="mmi";
% ifmathsym="sy[0-9]";
% font=$1;
% fname="drm";
% fenc="T1";
% commands="\\sample\\bye";
% if [[ $font =~ $ifmath ]]; then
% 	fenc="OML";
% 	commands="\\table\\math\\bye";
% fi
% if [[ $font =~ $ifmathsym ]]; then
% 	fenc="OMS";
% 	commands="\\table\\math\\bye";
% fi
% [[ $font =~ $re ]] && fweight="${BASH_REMATCH[1]}" &&
% 	fshape="${BASH_REMATCH[2]}" && fsize="${BASH_REMATCH[3]}";
% if [[ $fshape == "" ]]; then
% 	fshape="m";
% fi
% if [[ $fweight == "" ]]; then
% 	fweight="n";
% elif [[ $fweight == "sym" ]]; then
% 	fweight="n";
% 	fname="drmsym";
% fi
% #echo "fweight = $fweight; fshape = $fshape; fsize = $fsize";
% nfssfontin=$(cat <<ENDFONTIN
% $font
% $commands
% $fenc
% $fname
% $fweight
% $fshape
% $fsize
% $commands
% ENDFONTIN
% )
% echo $nfssfontin;
% 
% rm $1*pk; rm $1*gf;
% mf "\mode=localfont; input $1";
% gftopk $1.600gf $1.pk;
% echo "$nfssfontin" | pdflatex nfssfont;
% \end{verbatim}
%
% This little gem took a single argument, the name of the
% font that I wanted compiled; it then determined the
% appropriate parameters to hand over to |nfssfont|, including
% what type of sample was needed (text or math), and
% compiled it for me, which meant that with a single command
% (|./allcomp fontname|) I could get a complete chart of
% the font I was working on, along with a sample text to
% help judge kerning and general appearance.  Doing the same
% with more ``modern'' font programs, particularly GUI ones,
% is doubtlessly more difficult.
%
% Finally, pens.  Pens are \emph{endlessly} superior to
% defining points along outlines.  The degree to which
% grokking and employing \MF's pen metaphor simplified the
% task of drawing these characters, particularly the more
% calligraphic varieties thereof, simply cannot be
% adequately expressed.  Defining points along outlines and
% connecting them with zero-width lines did fine for
% \emph{most} of the roman characters, but would have been
% painfully sluggish with, for example, the italic fonts.
%
% Your author emphasized ``most'' above for good reason:
% while the points-and-outlines approach worked quite
% effectively for the stately forms of roman characters, 
% \emph{modifying} those characters was sometimes much more
% difficult.  Take, for example, the very basic different
% between ``o'' and ``\o.''  Visually, of course, these are
% almost identical shapes, the latter simply having a line
% drawn through it; however, by outlines these shapes are so
% extremely different that drawing the latter would more
% easily be done from scratch than by a simple modification
% of the former.  Using \MF's pen metaphor, though, the
% shape of ``\o'' could be drawn exactly as we would draw it
% on paper:  by forming an ``o,'' and then drawing a slash
% through it.  And so your author accomplished it.
%
% And though your author put off the development of italic
% until he'd become really proficient with \MF's pens,
% knowing that such intricate shapes as ``\textit{f}'' and
% ``\textit{Q}'' would be quite challenging with
% points-and-outlines, he was able to race through drawing
% the italics with ease, and wound up using \MF's pens much
% more frequently in the development of the remaining fonts
% than points-and-outlines, as drawing shapes with this
% metaphor is much more intuitive and easily visualized, at
% least to him, than the alternatives.
%
% \section{The \LaTeX\ Project Public License, v1.3c}
% \label{lppl}
% \MakePercentComment%
% $Id: lppl-1-3c.tex 160 2009-12-06 23:08:41Z lotze $
%
% Copyright 1999 2002-2008 LaTeX3 Project
%    Everyone is allowed to distribute verbatim copies of this
%    license document, but modification of it is not allowed.
%
%
% If you wish to load it as part of a ``doc'' source, you have to
% ensure that a) % is a comment character and b) that short verb
% characters are being turned off, i.e.,
%
%   \DeleteShortVerb{\'}   % or whatever was made a shorthand
%   \MakePercentComment
%   %
% $Id: lppl.tex 5882 2008-05-04 17:28:59Z mittelba $
%
% Copyright 1999 2002-2011 LaTeX3 Project
%    Everyone is allowed to distribute verbatim copies of this
%    license document, but modification of it is not allowed.
%
%
% If you wish to load it as part of a ``doc'' source, you have to
% ensure that a) % is a comment character and b) that short verb
% characters are being turned off, i.e.,
%
%   \DeleteShortVerb{\'}   % or whatever was made a shorthand
%   \MakePercentComment
%   %
% $Id: lppl.tex 5882 2008-05-04 17:28:59Z mittelba $
%
% Copyright 1999 2002-2011 LaTeX3 Project
%    Everyone is allowed to distribute verbatim copies of this
%    license document, but modification of it is not allowed.
%
%
% If you wish to load it as part of a ``doc'' source, you have to
% ensure that a) % is a comment character and b) that short verb
% characters are being turned off, i.e.,
%
%   \DeleteShortVerb{\'}   % or whatever was made a shorthand
%   \MakePercentComment
%   \input{lppl}
%   \MakePercentIgnore
%   \MakeShortVerb{\'}     % turn it on again if necessary
%
%
% By default the license is produced with \section* as the highest
% heading level. If this is not appropriate for the document in which
% it is included define the commands listed below before loading this
% document, e.g., for inclusion as a separate chapter define:
%
%  \providecommand{\LPPLsection}{\chapter*}
%  \providecommand{\LPPLsubsection}{\section*}
%  \providecommand{\LPPLsubsubsection}{\subsection*}
%  \providecommand{\LPPLparagraph}{\subsubsection*}
%
%
% To allow cross-referencing the headings \label's have been attached
% to them, all starting with ``LPPL:''. As by default headings without
% numbers are produced, this will only allow page references.
% However, you can use the titleref package to produce textual
% references or you change the definitions of \LPPLsection, and
% friends to generated numbered headings.
%
%
% We want it to be possible that this file can be processed by
% (pdf)LaTeX on its own, or that this file can be included in another
% LaTeX document without any modification whatsoever.
% Hence the little test below.
%
%
\makeatletter
\ifx\@preamblecmds\@notprerr
  % In this case the preamble has already been processed so this file
  % is loaded as part of another document; just enclose everything in
  % a group
  \let\LPPLicense\bgroup
  \let\endLPPLicense\egroup
\else
  % In this case the preamble has not been processed yet so this file
  % is processed by itself.
  \documentclass{article}
  \let\LPPLicense\document
  \let\endLPPLicense\enddocument
\fi
\makeatother


\begin{LPPLicense}
  \providecommand{\LPPLsection}{\section*}
  \providecommand{\LPPLsubsection}{\subsection*}
  \providecommand{\LPPLsubsubsection}{\subsubsection*}
  \providecommand{\LPPLparagraph}{\paragraph*}
  \providecommand*{\LPPLfile}[1]{\texttt{#1}}
  \providecommand*{\LPPLdocfile}[1]{`\LPPLfile{#1.tex}'}
  \providecommand*{\LPPL}{\textsc{lppl}}

  \LPPLsection{The \LaTeX\ Project Public License}
  \label{LPPL:LPPL}

  \emph{LPPL Version 1.3c  2008-05-04}

  \textbf{Copyright 1999, 2002--2008 \LaTeX3 Project}
  \begin{quotation}
    Everyone is allowed to distribute verbatim copies of this
    license document, but modification of it is not allowed.
  \end{quotation}

  \LPPLsubsection{Preamble}
  \label{LPPL:Preamble}

  The \LaTeX\ Project Public License (\LPPL) is the primary license
  under which the \LaTeX\ kernel and the base \LaTeX\ packages are
  distributed.

  You may use this license for any work of which you hold the
  copyright and which you wish to distribute.  This license may be
  particularly suitable if your work is \TeX-related (such as a
  \LaTeX\ package), but it is written in such a way that you can use
  it even if your work is unrelated to \TeX.

  The section `WHETHER AND HOW TO DISTRIBUTE WORKS UNDER THIS
  LICENSE', below, gives instructions, examples, and recommendations
  for authors who are considering distributing their works under this
  license.

  This license gives conditions under which a work may be distributed
  and modified, as well as conditions under which modified versions of
  that work may be distributed.

  We, the \LaTeX3 Project, believe that the conditions below give you
  the freedom to make and distribute modified versions of your work
  that conform with whatever technical specifications you wish while
  maintaining the availability, integrity, and reliability of that
  work.  If you do not see how to achieve your goal while meeting
  these conditions, then read the document \LPPLdocfile{cfgguide} and
  \LPPLdocfile{modguide} in the base \LaTeX\ distribution for suggestions.


  \LPPLsubsection{Definitions}
  \label{LPPL:Definitions}

  In this license document the following terms are used:

  \begin{description}
  \item[Work] Any work being distributed under this License.

  \item[Derived Work] Any work that under any applicable law is
    derived from the Work.

  \item[Modification] Any procedure that produces a Derived Work under
    any applicable law -- for example, the production of a file
    containing an original file associated with the Work or a
    significant portion of such a file, either verbatim or with
    modifications and/or translated into another language.

  \item[Modify] To apply any procedure that produces a Derived Work
    under any applicable law.

  \item[Distribution] Making copies of the Work available from one
    person to another, in whole or in part.  Distribution includes
    (but is not limited to) making any electronic components of the
    Work accessible by file transfer protocols such as \textsc{ftp} or
    \textsc{http} or by shared file systems such as Sun's Network File
    System (\textsc{nfs}).

  \item[Compiled Work] A version of the Work that has been processed
    into a form where it is directly usable on a computer system.
    This processing may include using installation facilities provided
    by the Work, transformations of the Work, copying of components of
    the Work, or other activities.  Note that modification of any
    installation facilities provided by the Work constitutes
    modification of the Work.

  \item[Current Maintainer] A person or persons nominated as such
    within the Work.  If there is no such explicit nomination then it
    is the `Copyright Holder' under any applicable law.

  \item[Base Interpreter] A program or process that is normally needed
    for running or interpreting a part or the whole of the Work.

    A Base Interpreter may depend on external components but these are
    not considered part of the Base Interpreter provided that each
    external component clearly identifies itself whenever it is used
    interactively.  Unless explicitly specified when applying the
    license to the Work, the only applicable Base Interpreter is a
    `\LaTeX-Format' or in the case of files belonging to the
    `\LaTeX-format' a program implementing the `\TeX{} language'.
  \end{description}

  \LPPLsubsection{Conditions on Distribution and Modification}
  \label{LPPL:Conditions}

  \begin{enumerate}
  \item Activities other than distribution and/or modification of the
    Work are not covered by this license; they are outside its scope.
    In particular, the act of running the Work is not restricted and
    no requirements are made concerning any offers of support for the
    Work.

  \item\label{LPPL:item:distribute} You may distribute a complete, unmodified
    copy of the Work as you received it.  Distribution of only part of
    the Work is considered modification of the Work, and no right to
    distribute such a Derived Work may be assumed under the terms of
    this clause.

  \item You may distribute a Compiled Work that has been generated
    from a complete, unmodified copy of the Work as distributed under
    Clause~\ref{LPPL:item:distribute} above, as long as that Compiled Work is
    distributed in such a way that the recipients may install the
    Compiled Work on their system exactly as it would have been
    installed if they generated a Compiled Work directly from the
    Work.

  \item\label{LPPL:item:currmaint} If you are the Current Maintainer of the
    Work, you may, without restriction, modify the Work, thus creating
    a Derived Work.  You may also distribute the Derived Work without
    restriction, including Compiled Works generated from the Derived
    Work.  Derived Works distributed in this manner by the Current
    Maintainer are considered to be updated versions of the Work.

  \item If you are not the Current Maintainer of the Work, you may
    modify your copy of the Work, thus creating a Derived Work based
    on the Work, and compile this Derived Work, thus creating a
    Compiled Work based on the Derived Work.

  \item\label{LPPL:item:conditions} If you are not the Current Maintainer
    of the
    Work, you may distribute a Derived Work provided the following
    conditions are met for every component of the Work unless that
    component clearly states in the copyright notice that it is exempt
    from that condition.  Only the Current Maintainer is allowed to
    add such statements of exemption to a component of the Work.
    \begin{enumerate}
    \item If a component of this Derived Work can be a direct
      replacement for a component of the Work when that component is
      used with the Base Interpreter, then, wherever this component of
      the Work identifies itself to the user when used interactively
      with that Base Interpreter, the replacement component of this
      Derived Work clearly and unambiguously identifies itself as a
      modified version of this component to the user when used
      interactively with that Base Interpreter.

    \item\label{LPPL:item:changelog} Every component of the Derived Work
      contains prominent
      notices detailing the nature of the changes to that component,
      or a prominent reference to another file that is distributed as
      part of the Derived Work and that contains a complete and
      accurate log of the changes.

    \item No information in the Derived Work implies that any persons,
      including (but not limited to) the authors of the original
      version of the Work, provide any support, including (but not
      limited to) the reporting and handling of errors, to recipients
      of the Derived Work unless those persons have stated explicitly
      that they do provide such support for the Derived Work.

    \item\label{LPPL:item:unmodifiedcopy} You distribute at least one of
      the following with the Derived Work:
      \begin{enumerate}
      \item A complete, unmodified copy of the Work; if your
        distribution of a modified component is made by offering
        access to copy the modified component from a designated place,
        then offering equivalent access to copy the Work from the same
        or some similar place meets this condition, even though third
        parties are not compelled to copy the Work along with the
        modified component;

      \item Information that is sufficient to obtain a complete,
        unmodified copy of the Work.
      \end{enumerate}
    \end{enumerate}
  \item If you are not the Current Maintainer of the Work, you may
    distribute a Compiled Work generated from a Derived Work, as long
    as the Derived Work is distributed to all recipients of the
    Compiled Work, and as long as the conditions of
    Clause~\ref{LPPL:item:conditions}, above, are met with regard to the
    Derived Work.

  \item The conditions above are not intended to prohibit, and hence
    do not apply to, the modification, by any method, of any component
    so that it becomes identical to an updated version of that
    component of the Work as it is distributed by the Current
    Maintainer under Clause~\ref{LPPL:item:currmaint}, above.

  \item Distribution of the Work or any Derived Work in an alternative
    format, where the Work or that Derived Work (in whole or in part)
    is then produced by applying some process to that format, does not
    relax or nullify any sections of this license as they pertain to
    the results of applying that process.

  \item
    \begin{enumerate}
    \item A Derived Work may be distributed under a different license
      provided that license itself honors the conditions listed in
      Clause~\ref{LPPL:item:conditions} above, in regard to the Work, though it
      does not have to honor the rest of the conditions in this
      license.

    \item If a Derived Work is distributed under a different license,
      that Derived Work must provide sufficient documentation as part
      of itself to allow each recipient of that Derived Work to honor
      the restrictions in Clause~\ref{LPPL:item:conditions} above, concerning
      changes from the Work.
    \end{enumerate}
  \item This license places no restrictions on works that are
    unrelated to the Work, nor does this license place any
    restrictions on aggregating such works with the Work by any means.

  \item Nothing in this license is intended to, or may be used to,
    prevent complete compliance by all parties with all applicable
    laws.
  \end{enumerate}

  \LPPLsubsection{No Warranty}
  \label{LPPL:Warranty}

  There is no warranty for the Work.  Except when otherwise stated in
  writing, the Copyright Holder provides the Work `as is', without
  warranty of any kind, either expressed or implied, including, but
  not limited to, the implied warranties of merchantability and
  fitness for a particular purpose.  The entire risk as to the quality
  and performance of the Work is with you.  Should the Work prove
  defective, you assume the cost of all necessary servicing, repair,
  or correction.

  In no event unless required by applicable law or agreed to in
  writing will The Copyright Holder, or any author named in the
  components of the Work, or any other party who may distribute and/or
  modify the Work as permitted above, be liable to you for damages,
  including any general, special, incidental or consequential damages
  arising out of any use of the Work or out of inability to use the
  Work (including, but not limited to, loss of data, data being
  rendered inaccurate, or losses sustained by anyone as a result of
  any failure of the Work to operate with any other programs), even if
  the Copyright Holder or said author or said other party has been
  advised of the possibility of such damages.

  \LPPLsubsection{Maintenance of The Work}
  \label{LPPL:Maintenance}

  The Work has the status `author-maintained' if the Copyright Holder
  explicitly and prominently states near the primary copyright notice
  in the Work that the Work can only be maintained by the Copyright
  Holder or simply that it is `author-maintained'.

  The Work has the status `maintained' if there is a Current
  Maintainer who has indicated in the Work that they are willing to
  receive error reports for the Work (for example, by supplying a
  valid e-mail address). It is not required for the Current Maintainer
  to acknowledge or act upon these error reports.

  The Work changes from status `maintained' to `unmaintained' if there
  is no Current Maintainer, or the person stated to be Current
  Maintainer of the work cannot be reached through the indicated means
  of communication for a period of six months, and there are no other
  significant signs of active maintenance.

  You can become the Current Maintainer of the Work by agreement with
  any existing Current Maintainer to take over this role.

  If the Work is unmaintained, you can become the Current Maintainer
  of the Work through the following steps:
  \begin{enumerate}
  \item Make a reasonable attempt to trace the Current Maintainer (and
    the Copyright Holder, if the two differ) through the means of an
    Internet or similar search.
  \item If this search is successful, then enquire whether the Work is
    still maintained.
    \begin{enumerate}
    \item If it is being maintained, then ask the Current Maintainer
      to update their communication data within one month.

    \item\label{LPPL:item:intention} If the search is unsuccessful or
      no action to resume active maintenance is taken by the Current
      Maintainer, then announce within the pertinent community your
      intention to take over maintenance.  (If the Work is a \LaTeX{}
      work, this could be done, for example, by posting to
      \texttt{comp.text.tex}.)
    \end{enumerate}
  \item {}
    \begin{enumerate}
    \item If the Current Maintainer is reachable and agrees to pass
      maintenance of the Work to you, then this takes effect
      immediately upon announcement.

    \item\label{LPPL:item:announce} If the Current Maintainer is not
      reachable and the Copyright Holder agrees that maintenance of
      the Work be passed to you, then this takes effect immediately
      upon announcement.
    \end{enumerate}
  \item\label{LPPL:item:change} If you make an `intention
    announcement' as described in~\ref{LPPL:item:intention} above and
    after three months your intention is challenged neither by the
    Current Maintainer nor by the Copyright Holder nor by other
    people, then you may arrange for the Work to be changed so as to
    name you as the (new) Current Maintainer.

  \item If the previously unreachable Current Maintainer becomes
    reachable once more within three months of a change completed
    under the terms of~\ref{LPPL:item:announce}
    or~\ref{LPPL:item:change}, then that Current Maintainer must
    become or remain the Current Maintainer upon request provided they
    then update their communication data within one month.
  \end{enumerate}
  A change in the Current Maintainer does not, of itself, alter the
  fact that the Work is distributed under the \LPPL\ license.

  If you become the Current Maintainer of the Work, you should
  immediately provide, within the Work, a prominent and unambiguous
  statement of your status as Current Maintainer.  You should also
  announce your new status to the same pertinent community as
  in~\ref{LPPL:item:intention} above.

  \LPPLsubsection{Whether and How to Distribute Works under This License}
  \label{LPPL:Distribute}

  This section contains important instructions, examples, and
  recommendations for authors who are considering distributing their
  works under this license.  These authors are addressed as `you' in
  this section.

  \LPPLsubsubsection{Choosing This License or Another License}
  \label{LPPL:Choosing}

  If for any part of your work you want or need to use
  \emph{distribution} conditions that differ significantly from those
  in this license, then do not refer to this license anywhere in your
  work but, instead, distribute your work under a different license.
  You may use the text of this license as a model for your own
  license, but your license should not refer to the \LPPL\ or
  otherwise give the impression that your work is distributed under
  the \LPPL.

  The document \LPPLdocfile{modguide} in the base \LaTeX\ distribution
  explains the motivation behind the conditions of this license.  It
  explains, for example, why distributing \LaTeX\ under the
  \textsc{gnu} General Public License (\textsc{gpl}) was considered
  inappropriate.  Even if your work is unrelated to \LaTeX, the
  discussion in \LPPLdocfile{modguide} may still be relevant, and authors
  intending to distribute their works under any license are encouraged
  to read it.

  \LPPLsubsubsection{A Recommendation on Modification Without Distribution}
  \label{LPPL:WithoutDistribution}

  It is wise never to modify a component of the Work, even for your
  own personal use, without also meeting the above conditions for
  distributing the modified component.  While you might intend that
  such modifications will never be distributed, often this will happen
  by accident -- you may forget that you have modified that component;
  or it may not occur to you when allowing others to access the
  modified version that you are thus distributing it and violating the
  conditions of this license in ways that could have legal
  implications and, worse, cause problems for the community.  It is
  therefore usually in your best interest to keep your copy of the
  Work identical with the public one.  Many works provide ways to
  control the behavior of that work without altering any of its
  licensed components.

  \LPPLsubsubsection{How to Use This License}
  \label{LPPL:HowTo}

  To use this license, place in each of the components of your work
  both an explicit copyright notice including your name and the year
  the work was authored and/or last substantially modified.  Include
  also a statement that the distribution and/or modification of that
  component is constrained by the conditions in this license.

  Here is an example of such a notice and statement:
\begin{verbatim}
  %% pig.dtx
  %% Copyright 2005 M. Y. Name
  %
  % This work may be distributed and/or modified under the
  % conditions of the LaTeX Project Public License, either version 1.3
  % of this license or (at your option) any later version.
  % The latest version of this license is in
  %   http://www.latex-project.org/lppl.txt
  % and version 1.3 or later is part of all distributions of LaTeX
  % version 2005/12/01 or later.
  %
  % This work has the LPPL maintenance status `maintained'.
  %
  % The Current Maintainer of this work is M. Y. Name.
  %
  % This work consists of the files pig.dtx and pig.ins
  % and the derived file pig.sty.
\end{verbatim}

  Given such a notice and statement in a file, the conditions given in
  this license document would apply, with the `Work' referring to the
  three files `\LPPLfile{pig.dtx}', `\LPPLfile{pig.ins}', and
  `\LPPLfile{pig.sty}' (the last being generated from
  `\LPPLfile{pig.dtx}' using `\LPPLfile{pig.ins}'), the `Base
  Interpreter' referring to any `\LaTeX-Format', and both `Copyright
  Holder' and `Current Maintainer' referring to the person `M. Y.
  Name'.

  If you do not want the Maintenance section of \LPPL\ to apply to
  your Work, change `maintained' above into `author-maintained'.
  However, we recommend that you use `maintained' as the Maintenance
  section was added in order to ensure that your Work remains useful
  to the community even when you can no longer maintain and support it
  yourself.

  \LPPLsubsubsection{Derived Works That Are Not Replacements}
  \label{LPPL:NotReplacements}

  Several clauses of the \LPPL\ specify means to provide reliability
  and stability for the user community. They therefore concern
  themselves with the case that a Derived Work is intended to be used
  as a (compatible or incompatible) replacement of the original
  Work. If this is not the case (e.g., if a few lines of code are
  reused for a completely different task), then clauses
  \ref{LPPL:item:changelog} and \ref{LPPL:item:unmodifiedcopy}
  shall not apply.

  \LPPLsubsubsection{Important Recommendations}
  \label{LPPL:Recommendations}

  \LPPLparagraph{Defining What Constitutes the Work}

  The \LPPL\ requires that distributions of the Work contain all the
  files of the Work.  It is therefore important that you provide a way
  for the licensee to determine which files constitute the Work.  This
  could, for example, be achieved by explicitly listing all the files
  of the Work near the copyright notice of each file or by using a
  line such as:
\begin{verbatim}
    % This work consists of all files listed in manifest.txt.
\end{verbatim}
  in that place.  In the absence of an unequivocal list it might be
  impossible for the licensee to determine what is considered by you
  to comprise the Work and, in such a case, the licensee would be
  entitled to make reasonable conjectures as to which files comprise
  the Work.

\end{LPPLicense}
\endinput

%   \MakePercentIgnore
%   \MakeShortVerb{\'}     % turn it on again if necessary
%
%
% By default the license is produced with \section* as the highest
% heading level. If this is not appropriate for the document in which
% it is included define the commands listed below before loading this
% document, e.g., for inclusion as a separate chapter define:
%
%  \providecommand{\LPPLsection}{\chapter*}
%  \providecommand{\LPPLsubsection}{\section*}
%  \providecommand{\LPPLsubsubsection}{\subsection*}
%  \providecommand{\LPPLparagraph}{\subsubsection*}
%
%
% To allow cross-referencing the headings \label's have been attached
% to them, all starting with ``LPPL:''. As by default headings without
% numbers are produced, this will only allow page references.
% However, you can use the titleref package to produce textual
% references or you change the definitions of \LPPLsection, and
% friends to generated numbered headings.
%
%
% We want it to be possible that this file can be processed by
% (pdf)LaTeX on its own, or that this file can be included in another
% LaTeX document without any modification whatsoever.
% Hence the little test below.
%
%
\makeatletter
\ifx\@preamblecmds\@notprerr
  % In this case the preamble has already been processed so this file
  % is loaded as part of another document; just enclose everything in
  % a group
  \let\LPPLicense\bgroup
  \let\endLPPLicense\egroup
\else
  % In this case the preamble has not been processed yet so this file
  % is processed by itself.
  \documentclass{article}
  \let\LPPLicense\document
  \let\endLPPLicense\enddocument
\fi
\makeatother


\begin{LPPLicense}
  \providecommand{\LPPLsection}{\section*}
  \providecommand{\LPPLsubsection}{\subsection*}
  \providecommand{\LPPLsubsubsection}{\subsubsection*}
  \providecommand{\LPPLparagraph}{\paragraph*}
  \providecommand*{\LPPLfile}[1]{\texttt{#1}}
  \providecommand*{\LPPLdocfile}[1]{`\LPPLfile{#1.tex}'}
  \providecommand*{\LPPL}{\textsc{lppl}}

  \LPPLsection{The \LaTeX\ Project Public License}
  \label{LPPL:LPPL}

  \emph{LPPL Version 1.3c  2008-05-04}

  \textbf{Copyright 1999, 2002--2008 \LaTeX3 Project}
  \begin{quotation}
    Everyone is allowed to distribute verbatim copies of this
    license document, but modification of it is not allowed.
  \end{quotation}

  \LPPLsubsection{Preamble}
  \label{LPPL:Preamble}

  The \LaTeX\ Project Public License (\LPPL) is the primary license
  under which the \LaTeX\ kernel and the base \LaTeX\ packages are
  distributed.

  You may use this license for any work of which you hold the
  copyright and which you wish to distribute.  This license may be
  particularly suitable if your work is \TeX-related (such as a
  \LaTeX\ package), but it is written in such a way that you can use
  it even if your work is unrelated to \TeX.

  The section `WHETHER AND HOW TO DISTRIBUTE WORKS UNDER THIS
  LICENSE', below, gives instructions, examples, and recommendations
  for authors who are considering distributing their works under this
  license.

  This license gives conditions under which a work may be distributed
  and modified, as well as conditions under which modified versions of
  that work may be distributed.

  We, the \LaTeX3 Project, believe that the conditions below give you
  the freedom to make and distribute modified versions of your work
  that conform with whatever technical specifications you wish while
  maintaining the availability, integrity, and reliability of that
  work.  If you do not see how to achieve your goal while meeting
  these conditions, then read the document \LPPLdocfile{cfgguide} and
  \LPPLdocfile{modguide} in the base \LaTeX\ distribution for suggestions.


  \LPPLsubsection{Definitions}
  \label{LPPL:Definitions}

  In this license document the following terms are used:

  \begin{description}
  \item[Work] Any work being distributed under this License.

  \item[Derived Work] Any work that under any applicable law is
    derived from the Work.

  \item[Modification] Any procedure that produces a Derived Work under
    any applicable law -- for example, the production of a file
    containing an original file associated with the Work or a
    significant portion of such a file, either verbatim or with
    modifications and/or translated into another language.

  \item[Modify] To apply any procedure that produces a Derived Work
    under any applicable law.

  \item[Distribution] Making copies of the Work available from one
    person to another, in whole or in part.  Distribution includes
    (but is not limited to) making any electronic components of the
    Work accessible by file transfer protocols such as \textsc{ftp} or
    \textsc{http} or by shared file systems such as Sun's Network File
    System (\textsc{nfs}).

  \item[Compiled Work] A version of the Work that has been processed
    into a form where it is directly usable on a computer system.
    This processing may include using installation facilities provided
    by the Work, transformations of the Work, copying of components of
    the Work, or other activities.  Note that modification of any
    installation facilities provided by the Work constitutes
    modification of the Work.

  \item[Current Maintainer] A person or persons nominated as such
    within the Work.  If there is no such explicit nomination then it
    is the `Copyright Holder' under any applicable law.

  \item[Base Interpreter] A program or process that is normally needed
    for running or interpreting a part or the whole of the Work.

    A Base Interpreter may depend on external components but these are
    not considered part of the Base Interpreter provided that each
    external component clearly identifies itself whenever it is used
    interactively.  Unless explicitly specified when applying the
    license to the Work, the only applicable Base Interpreter is a
    `\LaTeX-Format' or in the case of files belonging to the
    `\LaTeX-format' a program implementing the `\TeX{} language'.
  \end{description}

  \LPPLsubsection{Conditions on Distribution and Modification}
  \label{LPPL:Conditions}

  \begin{enumerate}
  \item Activities other than distribution and/or modification of the
    Work are not covered by this license; they are outside its scope.
    In particular, the act of running the Work is not restricted and
    no requirements are made concerning any offers of support for the
    Work.

  \item\label{LPPL:item:distribute} You may distribute a complete, unmodified
    copy of the Work as you received it.  Distribution of only part of
    the Work is considered modification of the Work, and no right to
    distribute such a Derived Work may be assumed under the terms of
    this clause.

  \item You may distribute a Compiled Work that has been generated
    from a complete, unmodified copy of the Work as distributed under
    Clause~\ref{LPPL:item:distribute} above, as long as that Compiled Work is
    distributed in such a way that the recipients may install the
    Compiled Work on their system exactly as it would have been
    installed if they generated a Compiled Work directly from the
    Work.

  \item\label{LPPL:item:currmaint} If you are the Current Maintainer of the
    Work, you may, without restriction, modify the Work, thus creating
    a Derived Work.  You may also distribute the Derived Work without
    restriction, including Compiled Works generated from the Derived
    Work.  Derived Works distributed in this manner by the Current
    Maintainer are considered to be updated versions of the Work.

  \item If you are not the Current Maintainer of the Work, you may
    modify your copy of the Work, thus creating a Derived Work based
    on the Work, and compile this Derived Work, thus creating a
    Compiled Work based on the Derived Work.

  \item\label{LPPL:item:conditions} If you are not the Current Maintainer
    of the
    Work, you may distribute a Derived Work provided the following
    conditions are met for every component of the Work unless that
    component clearly states in the copyright notice that it is exempt
    from that condition.  Only the Current Maintainer is allowed to
    add such statements of exemption to a component of the Work.
    \begin{enumerate}
    \item If a component of this Derived Work can be a direct
      replacement for a component of the Work when that component is
      used with the Base Interpreter, then, wherever this component of
      the Work identifies itself to the user when used interactively
      with that Base Interpreter, the replacement component of this
      Derived Work clearly and unambiguously identifies itself as a
      modified version of this component to the user when used
      interactively with that Base Interpreter.

    \item\label{LPPL:item:changelog} Every component of the Derived Work
      contains prominent
      notices detailing the nature of the changes to that component,
      or a prominent reference to another file that is distributed as
      part of the Derived Work and that contains a complete and
      accurate log of the changes.

    \item No information in the Derived Work implies that any persons,
      including (but not limited to) the authors of the original
      version of the Work, provide any support, including (but not
      limited to) the reporting and handling of errors, to recipients
      of the Derived Work unless those persons have stated explicitly
      that they do provide such support for the Derived Work.

    \item\label{LPPL:item:unmodifiedcopy} You distribute at least one of
      the following with the Derived Work:
      \begin{enumerate}
      \item A complete, unmodified copy of the Work; if your
        distribution of a modified component is made by offering
        access to copy the modified component from a designated place,
        then offering equivalent access to copy the Work from the same
        or some similar place meets this condition, even though third
        parties are not compelled to copy the Work along with the
        modified component;

      \item Information that is sufficient to obtain a complete,
        unmodified copy of the Work.
      \end{enumerate}
    \end{enumerate}
  \item If you are not the Current Maintainer of the Work, you may
    distribute a Compiled Work generated from a Derived Work, as long
    as the Derived Work is distributed to all recipients of the
    Compiled Work, and as long as the conditions of
    Clause~\ref{LPPL:item:conditions}, above, are met with regard to the
    Derived Work.

  \item The conditions above are not intended to prohibit, and hence
    do not apply to, the modification, by any method, of any component
    so that it becomes identical to an updated version of that
    component of the Work as it is distributed by the Current
    Maintainer under Clause~\ref{LPPL:item:currmaint}, above.

  \item Distribution of the Work or any Derived Work in an alternative
    format, where the Work or that Derived Work (in whole or in part)
    is then produced by applying some process to that format, does not
    relax or nullify any sections of this license as they pertain to
    the results of applying that process.

  \item
    \begin{enumerate}
    \item A Derived Work may be distributed under a different license
      provided that license itself honors the conditions listed in
      Clause~\ref{LPPL:item:conditions} above, in regard to the Work, though it
      does not have to honor the rest of the conditions in this
      license.

    \item If a Derived Work is distributed under a different license,
      that Derived Work must provide sufficient documentation as part
      of itself to allow each recipient of that Derived Work to honor
      the restrictions in Clause~\ref{LPPL:item:conditions} above, concerning
      changes from the Work.
    \end{enumerate}
  \item This license places no restrictions on works that are
    unrelated to the Work, nor does this license place any
    restrictions on aggregating such works with the Work by any means.

  \item Nothing in this license is intended to, or may be used to,
    prevent complete compliance by all parties with all applicable
    laws.
  \end{enumerate}

  \LPPLsubsection{No Warranty}
  \label{LPPL:Warranty}

  There is no warranty for the Work.  Except when otherwise stated in
  writing, the Copyright Holder provides the Work `as is', without
  warranty of any kind, either expressed or implied, including, but
  not limited to, the implied warranties of merchantability and
  fitness for a particular purpose.  The entire risk as to the quality
  and performance of the Work is with you.  Should the Work prove
  defective, you assume the cost of all necessary servicing, repair,
  or correction.

  In no event unless required by applicable law or agreed to in
  writing will The Copyright Holder, or any author named in the
  components of the Work, or any other party who may distribute and/or
  modify the Work as permitted above, be liable to you for damages,
  including any general, special, incidental or consequential damages
  arising out of any use of the Work or out of inability to use the
  Work (including, but not limited to, loss of data, data being
  rendered inaccurate, or losses sustained by anyone as a result of
  any failure of the Work to operate with any other programs), even if
  the Copyright Holder or said author or said other party has been
  advised of the possibility of such damages.

  \LPPLsubsection{Maintenance of The Work}
  \label{LPPL:Maintenance}

  The Work has the status `author-maintained' if the Copyright Holder
  explicitly and prominently states near the primary copyright notice
  in the Work that the Work can only be maintained by the Copyright
  Holder or simply that it is `author-maintained'.

  The Work has the status `maintained' if there is a Current
  Maintainer who has indicated in the Work that they are willing to
  receive error reports for the Work (for example, by supplying a
  valid e-mail address). It is not required for the Current Maintainer
  to acknowledge or act upon these error reports.

  The Work changes from status `maintained' to `unmaintained' if there
  is no Current Maintainer, or the person stated to be Current
  Maintainer of the work cannot be reached through the indicated means
  of communication for a period of six months, and there are no other
  significant signs of active maintenance.

  You can become the Current Maintainer of the Work by agreement with
  any existing Current Maintainer to take over this role.

  If the Work is unmaintained, you can become the Current Maintainer
  of the Work through the following steps:
  \begin{enumerate}
  \item Make a reasonable attempt to trace the Current Maintainer (and
    the Copyright Holder, if the two differ) through the means of an
    Internet or similar search.
  \item If this search is successful, then enquire whether the Work is
    still maintained.
    \begin{enumerate}
    \item If it is being maintained, then ask the Current Maintainer
      to update their communication data within one month.

    \item\label{LPPL:item:intention} If the search is unsuccessful or
      no action to resume active maintenance is taken by the Current
      Maintainer, then announce within the pertinent community your
      intention to take over maintenance.  (If the Work is a \LaTeX{}
      work, this could be done, for example, by posting to
      \texttt{comp.text.tex}.)
    \end{enumerate}
  \item {}
    \begin{enumerate}
    \item If the Current Maintainer is reachable and agrees to pass
      maintenance of the Work to you, then this takes effect
      immediately upon announcement.

    \item\label{LPPL:item:announce} If the Current Maintainer is not
      reachable and the Copyright Holder agrees that maintenance of
      the Work be passed to you, then this takes effect immediately
      upon announcement.
    \end{enumerate}
  \item\label{LPPL:item:change} If you make an `intention
    announcement' as described in~\ref{LPPL:item:intention} above and
    after three months your intention is challenged neither by the
    Current Maintainer nor by the Copyright Holder nor by other
    people, then you may arrange for the Work to be changed so as to
    name you as the (new) Current Maintainer.

  \item If the previously unreachable Current Maintainer becomes
    reachable once more within three months of a change completed
    under the terms of~\ref{LPPL:item:announce}
    or~\ref{LPPL:item:change}, then that Current Maintainer must
    become or remain the Current Maintainer upon request provided they
    then update their communication data within one month.
  \end{enumerate}
  A change in the Current Maintainer does not, of itself, alter the
  fact that the Work is distributed under the \LPPL\ license.

  If you become the Current Maintainer of the Work, you should
  immediately provide, within the Work, a prominent and unambiguous
  statement of your status as Current Maintainer.  You should also
  announce your new status to the same pertinent community as
  in~\ref{LPPL:item:intention} above.

  \LPPLsubsection{Whether and How to Distribute Works under This License}
  \label{LPPL:Distribute}

  This section contains important instructions, examples, and
  recommendations for authors who are considering distributing their
  works under this license.  These authors are addressed as `you' in
  this section.

  \LPPLsubsubsection{Choosing This License or Another License}
  \label{LPPL:Choosing}

  If for any part of your work you want or need to use
  \emph{distribution} conditions that differ significantly from those
  in this license, then do not refer to this license anywhere in your
  work but, instead, distribute your work under a different license.
  You may use the text of this license as a model for your own
  license, but your license should not refer to the \LPPL\ or
  otherwise give the impression that your work is distributed under
  the \LPPL.

  The document \LPPLdocfile{modguide} in the base \LaTeX\ distribution
  explains the motivation behind the conditions of this license.  It
  explains, for example, why distributing \LaTeX\ under the
  \textsc{gnu} General Public License (\textsc{gpl}) was considered
  inappropriate.  Even if your work is unrelated to \LaTeX, the
  discussion in \LPPLdocfile{modguide} may still be relevant, and authors
  intending to distribute their works under any license are encouraged
  to read it.

  \LPPLsubsubsection{A Recommendation on Modification Without Distribution}
  \label{LPPL:WithoutDistribution}

  It is wise never to modify a component of the Work, even for your
  own personal use, without also meeting the above conditions for
  distributing the modified component.  While you might intend that
  such modifications will never be distributed, often this will happen
  by accident -- you may forget that you have modified that component;
  or it may not occur to you when allowing others to access the
  modified version that you are thus distributing it and violating the
  conditions of this license in ways that could have legal
  implications and, worse, cause problems for the community.  It is
  therefore usually in your best interest to keep your copy of the
  Work identical with the public one.  Many works provide ways to
  control the behavior of that work without altering any of its
  licensed components.

  \LPPLsubsubsection{How to Use This License}
  \label{LPPL:HowTo}

  To use this license, place in each of the components of your work
  both an explicit copyright notice including your name and the year
  the work was authored and/or last substantially modified.  Include
  also a statement that the distribution and/or modification of that
  component is constrained by the conditions in this license.

  Here is an example of such a notice and statement:
\begin{verbatim}
  %% pig.dtx
  %% Copyright 2005 M. Y. Name
  %
  % This work may be distributed and/or modified under the
  % conditions of the LaTeX Project Public License, either version 1.3
  % of this license or (at your option) any later version.
  % The latest version of this license is in
  %   http://www.latex-project.org/lppl.txt
  % and version 1.3 or later is part of all distributions of LaTeX
  % version 2005/12/01 or later.
  %
  % This work has the LPPL maintenance status `maintained'.
  %
  % The Current Maintainer of this work is M. Y. Name.
  %
  % This work consists of the files pig.dtx and pig.ins
  % and the derived file pig.sty.
\end{verbatim}

  Given such a notice and statement in a file, the conditions given in
  this license document would apply, with the `Work' referring to the
  three files `\LPPLfile{pig.dtx}', `\LPPLfile{pig.ins}', and
  `\LPPLfile{pig.sty}' (the last being generated from
  `\LPPLfile{pig.dtx}' using `\LPPLfile{pig.ins}'), the `Base
  Interpreter' referring to any `\LaTeX-Format', and both `Copyright
  Holder' and `Current Maintainer' referring to the person `M. Y.
  Name'.

  If you do not want the Maintenance section of \LPPL\ to apply to
  your Work, change `maintained' above into `author-maintained'.
  However, we recommend that you use `maintained' as the Maintenance
  section was added in order to ensure that your Work remains useful
  to the community even when you can no longer maintain and support it
  yourself.

  \LPPLsubsubsection{Derived Works That Are Not Replacements}
  \label{LPPL:NotReplacements}

  Several clauses of the \LPPL\ specify means to provide reliability
  and stability for the user community. They therefore concern
  themselves with the case that a Derived Work is intended to be used
  as a (compatible or incompatible) replacement of the original
  Work. If this is not the case (e.g., if a few lines of code are
  reused for a completely different task), then clauses
  \ref{LPPL:item:changelog} and \ref{LPPL:item:unmodifiedcopy}
  shall not apply.

  \LPPLsubsubsection{Important Recommendations}
  \label{LPPL:Recommendations}

  \LPPLparagraph{Defining What Constitutes the Work}

  The \LPPL\ requires that distributions of the Work contain all the
  files of the Work.  It is therefore important that you provide a way
  for the licensee to determine which files constitute the Work.  This
  could, for example, be achieved by explicitly listing all the files
  of the Work near the copyright notice of each file or by using a
  line such as:
\begin{verbatim}
    % This work consists of all files listed in manifest.txt.
\end{verbatim}
  in that place.  In the absence of an unequivocal list it might be
  impossible for the licensee to determine what is considered by you
  to comprise the Work and, in such a case, the licensee would be
  entitled to make reasonable conjectures as to which files comprise
  the Work.

\end{LPPLicense}
\endinput

%   \MakePercentIgnore
%   \MakeShortVerb{\'}     % turn it on again if necessary
%
%
% By default the license is produced with \section* as the highest
% heading level. If this is not appropriate for the document in which
% it is included define the commands listed below before loading this
% document, e.g., for inclusion as a separate chapter define:
%
%  \providecommand{\LPPLsection}{\chapter*}
%  \providecommand{\LPPLsubsection}{\section*}
%  \providecommand{\LPPLsubsubsection}{\subsection*}
%  \providecommand{\LPPLparagraph}{\subsubsection*}
%
% 
% To allow cross-referencing the headings \label's have been attached
% to them, all starting with ``LPPL:''. As by default headings without
% numbers are produced, this will only allow page references.
% However, you can use the titleref package to produce textual
% references or you change the definitions of \LPPLsection, and
% friends to generated numbered headings.
%
%
% We want it to be possible that this file can be processed by
% (pdf)LaTeX on its own, or that this file can be included in another
% LaTeX document without any modification whatsoever.
% Hence the little test below.
%
%
\makeatletter
\ifx\@preamblecmds\@notprerr
  % In this case the preamble has already been processed so this file
  % is loaded as part of another document; just enclose everything in
  % a group
  \let\LPPLicense\bgroup
  \let\endLPPLicense\egroup
\else
  % In this case the preamble has not been processed yet so this file
  % is processed by itself.
  \documentclass{article}
  \let\LPPLicense\document
  \let\endLPPLicense\enddocument
\fi
\makeatother


\begin{LPPLicense}
  \providecommand{\LPPLsection}{\section*}
  \providecommand{\LPPLsubsection}{\subsection*}
  \providecommand{\LPPLsubsubsection}{\subsubsection*}
  \providecommand{\LPPLparagraph}{\paragraph*}
  \providecommand*{\LPPLfile}[1]{\texttt{#1}}
  \providecommand*{\LPPLdocfile}[1]{`\LPPLfile{#1.tex}'}
  \providecommand*{\LPPL}{\textsc{lppl}}

  \LPPLsection{The \LaTeX\ Project Public License}
  \label{LPPL:LPPL}

  \emph{LPPL Version 1.3c  2008-05-04}

  \textbf{Copyright 1999, 2002--2008 \LaTeX3 Project}
  \begin{quotation}
    Everyone is allowed to distribute verbatim copies of this
    license document, but modification of it is not allowed.
  \end{quotation}

  \LPPLsubsection{Preamble}
  \label{LPPL:Preamble}
  
  The \LaTeX\ Project Public License (\LPPL) is the primary license
  under which the \LaTeX\ kernel and the base \LaTeX\ packages are
  distributed.

  You may use this license for any work of which you hold the
  copyright and which you wish to distribute.  This license may be
  particularly suitable if your work is \TeX-related (such as a
  \LaTeX\ package), but it is written in such a way that you can use 
  it even if your work is unrelated to \TeX.

  The section `WHETHER AND HOW TO DISTRIBUTE WORKS UNDER THIS
  LICENSE', below, gives instructions, examples, and recommendations
  for authors who are considering distributing their works under this
  license.

  This license gives conditions under which a work may be distributed
  and modified, as well as conditions under which modified versions of
  that work may be distributed.

  We, the \LaTeX3 Project, believe that the conditions below give you
  the freedom to make and distribute modified versions of your work
  that conform with whatever technical specifications you wish while
  maintaining the availability, integrity, and reliability of that
  work.  If you do not see how to achieve your goal while meeting
  these conditions, then read the document \LPPLdocfile{cfgguide} and
  \LPPLdocfile{modguide} in the base \LaTeX\ distribution for suggestions.


  \LPPLsubsection{Definitions}
  \label{LPPL:Definitions}

  In this license document the following terms are used:

  \begin{description}
  \item[Work] Any work being distributed under this License.

  \item[Derived Work] Any work that under any applicable law is
    derived from the Work.

  \item[Modification] Any procedure that produces a Derived Work under
    any applicable law -- for example, the production of a file
    containing an original file associated with the Work or a
    significant portion of such a file, either verbatim or with
    modifications and/or translated into another language.

  \item[Modify] To apply any procedure that produces a Derived Work
    under any applicable law.
    
  \item[Distribution] Making copies of the Work available from one
    person to another, in whole or in part.  Distribution includes
    (but is not limited to) making any electronic components of the
    Work accessible by file transfer protocols such as \textsc{ftp} or
    \textsc{http} or by shared file systems such as Sun's Network File
    System (\textsc{nfs}).

  \item[Compiled Work] A version of the Work that has been processed
    into a form where it is directly usable on a computer system.
    This processing may include using installation facilities provided
    by the Work, transformations of the Work, copying of components of
    the Work, or other activities.  Note that modification of any
    installation facilities provided by the Work constitutes
    modification of the Work.

  \item[Current Maintainer] A person or persons nominated as such
    within the Work.  If there is no such explicit nomination then it
    is the `Copyright Holder' under any applicable law.

  \item[Base Interpreter] A program or process that is normally needed
    for running or interpreting a part or the whole of the Work.
    
    A Base Interpreter may depend on external components but these are
    not considered part of the Base Interpreter provided that each
    external component clearly identifies itself whenever it is used
    interactively.  Unless explicitly specified when applying the
    license to the Work, the only applicable Base Interpreter is a
    `\LaTeX-Format' or in the case of files belonging to the
    `\LaTeX-format' a program implementing the `\TeX{} language'.
  \end{description}

  \LPPLsubsection{Conditions on Distribution and Modification}
  \label{LPPL:Conditions}

  \begin{enumerate}
  \item Activities other than distribution and/or modification of the
    Work are not covered by this license; they are outside its scope.
    In particular, the act of running the Work is not restricted and
    no requirements are made concerning any offers of support for the
    Work.

  \item\label{LPPL:item:distribute} You may distribute a complete, unmodified
    copy of the Work as you received it.  Distribution of only part of
    the Work is considered modification of the Work, and no right to
    distribute such a Derived Work may be assumed under the terms of
    this clause.

  \item You may distribute a Compiled Work that has been generated
    from a complete, unmodified copy of the Work as distributed under
    Clause~\ref{LPPL:item:distribute} above, as long as that Compiled Work is
    distributed in such a way that the recipients may install the
    Compiled Work on their system exactly as it would have been
    installed if they generated a Compiled Work directly from the
    Work.

  \item\label{LPPL:item:currmaint} If you are the Current Maintainer of the
    Work, you may, without restriction, modify the Work, thus creating
    a Derived Work.  You may also distribute the Derived Work without
    restriction, including Compiled Works generated from the Derived
    Work.  Derived Works distributed in this manner by the Current
    Maintainer are considered to be updated versions of the Work.

  \item If you are not the Current Maintainer of the Work, you may
    modify your copy of the Work, thus creating a Derived Work based
    on the Work, and compile this Derived Work, thus creating a
    Compiled Work based on the Derived Work.

  \item\label{LPPL:item:conditions} If you are not the Current Maintainer of the
    Work, you may distribute a Derived Work provided the following
    conditions are met for every component of the Work unless that
    component clearly states in the copyright notice that it is exempt
    from that condition.  Only the Current Maintainer is allowed to
    add such statements of exemption to a component of the Work.
    \begin{enumerate}
    \item If a component of this Derived Work can be a direct
      replacement for a component of the Work when that component is
      used with the Base Interpreter, then, wherever this component of
      the Work identifies itself to the user when used interactively
      with that Base Interpreter, the replacement component of this
      Derived Work clearly and unambiguously identifies itself as a
      modified version of this component to the user when used
      interactively with that Base Interpreter.
     
    \item Every component of the Derived Work contains prominent
      notices detailing the nature of the changes to that component,
      or a prominent reference to another file that is distributed as
      part of the Derived Work and that contains a complete and
      accurate log of the changes.
  
    \item No information in the Derived Work implies that any persons,
      including (but not limited to) the authors of the original
      version of the Work, provide any support, including (but not
      limited to) the reporting and handling of errors, to recipients
      of the Derived Work unless those persons have stated explicitly
      that they do provide such support for the Derived Work.

    \item You distribute at least one of the following with the Derived Work:
      \begin{enumerate}
      \item A complete, unmodified copy of the Work; if your
        distribution of a modified component is made by offering
        access to copy the modified component from a designated place,
        then offering equivalent access to copy the Work from the same
        or some similar place meets this condition, even though third
        parties are not compelled to copy the Work along with the
        modified component;

      \item Information that is sufficient to obtain a complete,
        unmodified copy of the Work.
      \end{enumerate}
    \end{enumerate}
  \item If you are not the Current Maintainer of the Work, you may
    distribute a Compiled Work generated from a Derived Work, as long
    as the Derived Work is distributed to all recipients of the
    Compiled Work, and as long as the conditions of
    Clause~\ref{LPPL:item:conditions}, above, are met with regard to the Derived
    Work.

  \item The conditions above are not intended to prohibit, and hence
    do not apply to, the modification, by any method, of any component
    so that it becomes identical to an updated version of that
    component of the Work as it is distributed by the Current
    Maintainer under Clause~\ref{LPPL:item:currmaint}, above.

  \item Distribution of the Work or any Derived Work in an alternative
    format, where the Work or that Derived Work (in whole or in part)
    is then produced by applying some process to that format, does not
    relax or nullify any sections of this license as they pertain to
    the results of applying that process.
     
  \item \null
    \begin{enumerate}
    \item A Derived Work may be distributed under a different license
      provided that license itself honors the conditions listed in
      Clause~\ref{LPPL:item:conditions} above, in regard to the Work, though it
      does not have to honor the rest of the conditions in this
      license.
      
    \item If a Derived Work is distributed under a different license,
      that Derived Work must provide sufficient documentation as part
      of itself to allow each recipient of that Derived Work to honor
      the restrictions in Clause~\ref{LPPL:item:conditions} above, concerning
      changes from the Work.
    \end{enumerate}
  \item This license places no restrictions on works that are
    unrelated to the Work, nor does this license place any
    restrictions on aggregating such works with the Work by any means.

  \item Nothing in this license is intended to, or may be used to,
    prevent complete compliance by all parties with all applicable
    laws.
  \end{enumerate}

  \LPPLsubsection{No Warranty}
  \label{LPPL:Warranty}

  There is no warranty for the Work.  Except when otherwise stated in
  writing, the Copyright Holder provides the Work `as is', without
  warranty of any kind, either expressed or implied, including, but
  not limited to, the implied warranties of merchantability and
  fitness for a particular purpose.  The entire risk as to the quality
  and performance of the Work is with you.  Should the Work prove
  defective, you assume the cost of all necessary servicing, repair,
  or correction.

  In no event unless required by applicable law or agreed to in
  writing will The Copyright Holder, or any author named in the
  components of the Work, or any other party who may distribute and/or
  modify the Work as permitted above, be liable to you for damages,
  including any general, special, incidental or consequential damages
  arising out of any use of the Work or out of inability to use the
  Work (including, but not limited to, loss of data, data being
  rendered inaccurate, or losses sustained by anyone as a result of
  any failure of the Work to operate with any other programs), even if
  the Copyright Holder or said author or said other party has been
  advised of the possibility of such damages.

  \LPPLsubsection{Maintenance of The Work}
  \label{LPPL:Maintenance}

  The Work has the status `author-maintained' if the Copyright Holder
  explicitly and prominently states near the primary copyright notice
  in the Work that the Work can only be maintained by the Copyright
  Holder or simply that it is `author-maintained'.

  The Work has the status `maintained' if there is a Current
  Maintainer who has indicated in the Work that they are willing to
  receive error reports for the Work (for example, by supplying a
  valid e-mail address). It is not required for the Current Maintainer
  to acknowledge or act upon these error reports.

  The Work changes from status `maintained' to `unmaintained' if there
  is no Current Maintainer, or the person stated to be Current
  Maintainer of the work cannot be reached through the indicated means
  of communication for a period of six months, and there are no other
  significant signs of active maintenance.

  You can become the Current Maintainer of the Work by agreement with
  any existing Current Maintainer to take over this role.

  If the Work is unmaintained, you can become the Current Maintainer
  of the Work through the following steps:
  \begin{enumerate}
  \item Make a reasonable attempt to trace the Current Maintainer (and
    the Copyright Holder, if the two differ) through the means of an
    Internet or similar search.
  \item If this search is successful, then enquire whether the Work is
    still maintained.
    \begin{enumerate}
    \item If it is being maintained, then ask the Current Maintainer
      to update their communication data within one month.
     
    \item\label{LPPL:item:intention} If the search is unsuccessful or
      no action to resume active maintenance is taken by the Current
      Maintainer, then announce within the pertinent community your
      intention to take over maintenance.  (If the Work is a \LaTeX{}
      work, this could be done, for example, by posting to
      \texttt{comp.text.tex}.)
    \end{enumerate}
  \item {}
    \begin{enumerate}
    \item If the Current Maintainer is reachable and agrees to pass
      maintenance of the Work to you, then this takes effect
      immediately upon announcement.
     
    \item\label{LPPL:item:announce} If the Current Maintainer is not
      reachable and the Copyright Holder agrees that maintenance of
      the Work be passed to you, then this takes effect immediately
      upon announcement.
    \end{enumerate}
  \item\label{LPPL:item:change} If you make an `intention
    announcement' as described in~\ref{LPPL:item:intention} above and
    after three months your intention is challenged neither by the
    Current Maintainer nor by the Copyright Holder nor by other
    people, then you may arrange for the Work to be changed so as to
    name you as the (new) Current Maintainer.
     
  \item If the previously unreachable Current Maintainer becomes
    reachable once more within three months of a change completed
    under the terms of~\ref{LPPL:item:announce}
    or~\ref{LPPL:item:change}, then that Current Maintainer must
    become or remain the Current Maintainer upon request provided they
    then update their communication data within one month.
  \end{enumerate}
  A change in the Current Maintainer does not, of itself, alter the
  fact that the Work is distributed under the \LPPL\ license.

  If you become the Current Maintainer of the Work, you should
  immediately provide, within the Work, a prominent and unambiguous
  statement of your status as Current Maintainer.  You should also
  announce your new status to the same pertinent community as
  in~\ref{LPPL:item:intention} above.

  \LPPLsubsection{Whether and How to Distribute Works under This License}
  \label{LPPL:Distribute}

  This section contains important instructions, examples, and
  recommendations for authors who are considering distributing their
  works under this license.  These authors are addressed as `you' in
  this section.

  \LPPLsubsubsection{Choosing This License or Another License}
  \label{LPPL:Choosing}

  If for any part of your work you want or need to use
  \emph{distribution} conditions that differ significantly from those
  in this license, then do not refer to this license anywhere in your
  work but, instead, distribute your work under a different license.
  You may use the text of this license as a model for your own
  license, but your license should not refer to the \LPPL\ or
  otherwise give the impression that your work is distributed under
  the \LPPL.

  The document \LPPLdocfile{modguide} in the base \LaTeX\ distribution
  explains the motivation behind the conditions of this license.  It
  explains, for example, why distributing \LaTeX\ under the
  \textsc{gnu} General Public License (\textsc{gpl}) was considered
  inappropriate.  Even if your work is unrelated to \LaTeX, the
  discussion in \LPPLdocfile{modguide} may still be relevant, and authors
  intending to distribute their works under any license are encouraged
  to read it.

  \LPPLsubsubsection{A Recommendation on Modification Without Distribution}
  \label{LPPL:WithoutDistribution}

  It is wise never to modify a component of the Work, even for your
  own personal use, without also meeting the above conditions for
  distributing the modified component.  While you might intend that
  such modifications will never be distributed, often this will happen
  by accident -- you may forget that you have modified that component;
  or it may not occur to you when allowing others to access the
  modified version that you are thus distributing it and violating the
  conditions of this license in ways that could have legal
  implications and, worse, cause problems for the community.  It is
  therefore usually in your best interest to keep your copy of the
  Work identical with the public one.  Many works provide ways to
  control the behavior of that work without altering any of its
  licensed components.

  \LPPLsubsubsection{How to Use This License}
  \label{LPPL:HowTo}

  To use this license, place in each of the components of your work
  both an explicit copyright notice including your name and the year
  the work was authored and/or last substantially modified.  Include
  also a statement that the distribution and/or modification of that
  component is constrained by the conditions in this license.

  Here is an example of such a notice and statement:
\begin{verbatim}
  %% pig.dtx
  %% Copyright 2005 M. Y. Name
  %
  % This work may be distributed and/or modified under the
  % conditions of the LaTeX Project Public License, either version 1.3
  % of this license or (at your option) any later version.
  % The latest version of this license is in
  %   http://www.latex-project.org/lppl.txt
  % and version 1.3 or later is part of all distributions of LaTeX
  % version 2005/12/01 or later.
  %
  % This work has the LPPL maintenance status `maintained'.
  % 
  % The Current Maintainer of this work is M. Y. Name.
  %
  % This work consists of the files pig.dtx and pig.ins
  % and the derived file pig.sty.
\end{verbatim}
  
  Given such a notice and statement in a file, the conditions given in
  this license document would apply, with the `Work' referring to the
  three files `\LPPLfile{pig.dtx}', `\LPPLfile{pig.ins}', and
  `\LPPLfile{pig.sty}' (the last being generated from
  `\LPPLfile{pig.dtx}' using `\LPPLfile{pig.ins}'), the `Base
  Interpreter' referring to any `\LaTeX-Format', and both `Copyright
  Holder' and `Current Maintainer' referring to the person `M. Y.
  Name'.

  If you do not want the Maintenance section of \LPPL\ to apply to
  your Work, change `maintained' above into `author-maintained'.
  However, we recommend that you use `maintained' as the Maintenance
  section was added in order to ensure that your Work remains useful
  to the community even when you can no longer maintain and support it
  yourself.

  \LPPLsubsubsection{Derived Works That Are Not Replacements}
  \label{LPPL:NotReplacements}

  Several clauses of the \LPPL\ specify means to provide reliability
  and stability for the user community. They therefore concern
  themselves with the case that a Derived Work is intended to be used
  as a (compatible or incompatible) replacement of the original
  Work. If this is not the case (e.g., if a few lines of code are
  reused for a completely different task), then clauses 6b and 6d
  shall not apply.

  \LPPLsubsubsection{Important Recommendations}
  \label{LPPL:Recommendations}

  \LPPLparagraph{Defining What Constitutes the Work}

  The \LPPL\ requires that distributions of the Work contain all the
  files of the Work.  It is therefore important that you provide a way
  for the licensee to determine which files constitute the Work.  This
  could, for example, be achieved by explicitly listing all the files
  of the Work near the copyright notice of each file or by using a
  line such as:
\begin{verbatim}
    % This work consists of all files listed in manifest.txt.
\end{verbatim}
  in that place.  In the absence of an unequivocal list it might be
  impossible for the licensee to determine what is considered by you
  to comprise the Work and, in such a case, the licensee would be
  entitled to make reasonable conjectures as to which files comprise
  the Work.

\end{LPPLicense}
\endinput
\MakePercentIgnore
%
% \section{The SIL Open Font License, v1.1}
% \label{sil}
% 
% Copyright \textcopyright\ 2014, Donald P. Goodman III
% (dgoodmaniii@gmail.com), with Reserved Font Name Don's
% Revised Modern (DRM).
% 
% This Font Software is licensed under the SIL Open Font
License, Version 1.1.  This license is copied below, and is
also available with a FAQ at:
\url{http://scripts.sil.org/OFL}.

\def\ofl{\textsc{ofl}}

\subsection*{Preamble}
\label{sil:preamble}

The goals of the Open Font License (\ofl) are to stimulate
worldwide development of collaborative font projects, to
support the font creation efforts of academic and linguistic
communities, and to provide a free and open framework in
which fonts may be shared and improved in partnership with
others.

The \ofl\ allows the licensed fonts to be used, studied,
modified and redistributed freely as long as they are not
sold by themselves. The fonts, including any derivative
works, can be bundled, embedded, redistributed and/or sold
with any software provided that any reserved names are not
used by derivative works. The fonts and derivatives,
however, cannot be released under any other type of license.
The requirement for fonts to remain under this license does
not apply to any document created using the fonts or their
derivatives.

\subsection*{Definitions}
\label{sil:definitions}

\def\silterm#1{\noindent``#1''}

\begin{description}
\item[\silterm{Font Software}] refers to the set of files released
by the Copyright Holder(s) under this license and clearly
marked as such. This may include source files, build scripts
and documentation.

\item[\silterm{Reserved Font Name}] refers to any names specified
as such after the copyright statement(s).

\item[\silterm{Original Version}] refers to the collection of Font
Software components as distributed by the Copyright
Holder(s).

\item[\silterm{Modified Version}] refers to any derivative made by
adding to, deleting, or substituting --- in part or in whole
--- any of the components of the Original Version, by
changing formats or by porting the Font Software to a new
environment.

\item[\silterm{Author}] refers to any designer, engineer,
programmer, technical writer or other person who contributed
to the Font Software.
\end{description}

\subsection*{Permission \& Conditions}
\label{sil:permission}

Permission is hereby granted, free of charge, to any person
obtaining a copy of the Font Software, to use, study, copy,
merge, embed, modify, redistribute, and sell modified and
unmodified copies of the Font Software, subject to the
following conditions:

\begin{enumerate}
\item Neither the Font Software nor any of its individual
components, in Original or Modified Versions, may be sold by
itself.

\item Original or Modified Versions of the Font Software may
be bundled, redistributed and/or sold with any software,
provided that each copy contains the above copyright notice
and this license. These can be included either as
stand-alone text files, human-readable headers or in the
appropriate machine-readable metadata fields within text or
binary files as long as those fields can be easily viewed by
the user.

\item No Modified Version of the Font Software may use the
Reserved Font Name(s) unless explicit written permission is
granted by the corresponding Copyright Holder. This
restriction only applies to the primary font name as
presented to the users.

\item The name(s) of the Copyright Holder(s) or the
Author(s) of the Font Software shall not be used to promote,
endorse or advertise any Modified Version, except to
acknowledge the contribution(s) of the Copyright Holder(s)
and the Author(s) or with their explicit written permission.

\item The Font Software, modified or unmodified, in part or
in whole, must be distributed entirely under this license,
and must not be distributed under any other license. The
requirement for fonts to remain under this license does not
apply to any document created using the Font Software.
\end{enumerate}

\subsection*{Termination}
\label{sil:termination}

This license becomes null and void if any of the above
conditions are not met.

\subsection*{Disclaimer}
\label{sil:disclaimer}

\textsc{The Font Software is provided ``as is'', without
warranty of any kind, express or implied, including but not
limited to any warranties of merchantability, fitness for a
particular purpose and noninfringement of copyright, patent,
trademark, or other right. In no event shall the copyright
holder be liable for any claim, damages or other liability,
including any general, special, indirect, incidental, or
consequential damages, whether in an action of contract,
tort or otherwise, arising from, out of the use or inability
to use the Font Software or from other dealings in the Font
Software.}

%
% \PrintIndex
