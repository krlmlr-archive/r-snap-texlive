\documentclass{article}
\usepackage{fourier,hacm,color,graphicx}
\title{hacm.sty --- typesetting \textalblant{hacm}.}
\author{Kazuaki Miyatani\footnote{For any comments, discussions or invitations for a drink (especially japanese sake),
        please feel free to contact me at: \texttt{miyatanium[at]gmail[dot]com}}}
\date{\today}
\renewcommand{\thesection}{\arabic{section} (\textalblant{\arabic{section}}).}
\renewcommand{\thepage}{\textalblant{\arabic{page}}}
\renewcommand{\theenumi}{\textlantia{\arabic{enumi}}}
\begin{document}
    \maketitle

    \section{Introduction}

    The Arka\footnote{Official website in Japanese: \texttt{http://constructed-language.org/arka/} and in English:
    \texttt{http://constructed-language.org/arka/e\_index.html}} is a constructed language,
    which is extraordinarily elaborate, made largely by Seren Arbazard.
    One of the key points of this language is that we have an (also elaborately constructed) imaginary world, Kaldia,
    in which it is spoken, and by which it is affected.

    This package supports your typesetting hacm, the Arka alphabet.
    The following nine official fonts\footnote{The information on the official fonts are available on the official website above.
        You can also find some funny official/non-official fonts in \texttt{http://www9.atwiki.jp/hrain/} (Japanese).}
    are available in this package:

    \begin{center}
    \renewcommand{\arraystretch}{1.5}
        \begin{tabular}{ll}
            kardinal & \textkardinal{tu et kardinal.}\\
            alblant & \textalblant{tu et alblant.}\\
            fenlil & \textfenlil{tu et fenlil.}\\
            nalnia & \textnalnia{tu et nalnia.}\\
            olivia & \textolivia{tu et olivia.}\\
            lantia & \textlantia{tu et lantia.}\\
            inje & \textinje{tu et inje.}\\
            defans & \textdefans{tu et defans.}\\
            fialis & \textfialis{tu et fialis.}
        \end{tabular}
    \end{center}

    \section{License}

    This package is subject to the \LaTeX{} Project Public Licence\footnote{\texttt{http://www.latex-project.org/lppl/lppl-1-3c.pdf}}.
    In particular, you can freely use, copy, distribute this package and/or works made by using this package.
    If you want to distribute this package after some modifications, you must do it in a way that
    users do not confuse your work and the original one, e.g. by changing the name of the package.

    \section{Usage}
    
    In order to use this package, just write \texttt{\textbackslash usepackage\{hacm\}} in the preamble.

    With this package, you have two ways to typewrite Arka words with the font $X$ ( $X = $ kardinal, alblant,... ).
    The first one is to use \textbackslash\texttt{textX} command;

    \begin{center}
    \renewcommand{\arraystretch}{1.5}
        \begin{tabular}{lll}
            \small \textbackslash\texttt{textkardinal\{tu et to?\}} & $\Longrightarrow$ & \textkardinal{tu et to?}\\
            \small \texttt{\textbackslash textalblant\{an axt\} \textbackslash textfenlil\{fenlil.\}} & $\Longrightarrow$ &
            \small \textalblant{an axt} \textfenlil{fenlil.}
        \end{tabular}
    \end{center}

    The second one is to use \textbackslash\texttt{X} command.
    This command changes the font of all letters in the group, after the command;
    for example, \texttt{\textbackslash textalblant\{12345\}} and \texttt{\{\textbackslash alblant 12345\}} both result in
    {\alblant 12345}.

    The default commands for enlarging/shrinking letters , like \texttt{\textbackslash Huge} and \texttt{\textbackslash{tiny}},
    also work for hecm's.

    \begin{center}
    \renewcommand{\arraystretch}{1.5}
        \begin{tabular}{lll}
            \small \texttt{\{\textbackslash Huge  \textbackslash lantia kai tiina\}} & $\Longrightarrow$ & 
            {\Huge \lantia kai tiina}\\
            \small \texttt{\{\textbackslash tiny  \textbackslash fialis  lis tiina\}} & $\Longrightarrow$ &
            {\tiny \fialis lis tiina}
        \end{tabular}
    \end{center}

    Of course, we can rotate \rotatebox{90}{\inje hacm} (with \texttt{graphicx.sty}), or color \textcolor{red}{\defans hacm} (with \texttt{color.sty}).

    \texttt{\textbackslash textit}, \texttt{\textbackslash itshape} is only supported for alblant; it makes nalnia.
    Examples are:

    \begin{center}
    \renewcommand{\arraystretch}{1.5}
        \begin{tabular}{lll}
            \small \texttt{\{\textbackslash alblant skol!\}} & $\Longrightarrow$ & \alblant skol!\\
            \small \texttt{\{\textbackslash alblant \textbackslash itshape tik!\}} & $\Longrightarrow$ & \alblant \itshape tik!\\
       \end{tabular}
    \end{center}

    \section{Things to be considered.}
        \begin{enumerate}
            \item This package should include Lunar Letter, and some special commands to indicate single letters.
            \item This package should support writing boards of xelt.
            \item Is it useful to provide separate commands to write the page number, chapter number etc. with the digits in Arka
                (just as in this list) ?
            \item On using dvipdfm(x), we get too many warnings ``invalid glyph name in ..."
                I have no idea how to fix it.
        \end{enumerate}

    \section{History}
        \begin{tabular}{lll}
            Ver 0.1 &  Sep. 6, 2012, & The first version.
        \end{tabular}

\end{document}
