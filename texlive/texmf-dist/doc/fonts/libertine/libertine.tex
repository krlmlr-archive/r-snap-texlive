\documentclass[11pt]{article}
\usepackage[textwidth=5.5in,textheight=8.5in]{geometry}
\usepackage[T1]{fontenc}
\usepackage[osf,proportional]{libertine}
\usepackage{lettrine}
\renewcommand{\ttfamily}{\fontencoding{OT1}\fontfamily{cmtt}\selectfont}
\PassOptionsToPackage{urlcolor=black,colorlinks}{hyperref}
\RequirePackage{hyperref}
\usepackage{xcolor}
\usepackage{longtable}
\usepackage{multicol}
\newcommand{\typeone}{Type~\liningnums{1}}
\begin{document}
\title{\LaTeX\ Support for Linux Libertine and Biolinum Fonts}
\author{Bob Tennent\\
\small\url{rdt@cs.queensu.ca}}
\date{\today}
\maketitle 
\thispagestyle{empty}
\begin{small}
\tableofcontents
\end{small}
\sloppy
\clearpage
\section{Introduction}

This package provides support for use of the Linux Libertine and Linux
Biolinum families of fonts in \LaTeX.
Most features 
are usable
with \LaTeX\ and \verb|dvips|, pdf\LaTeX, xe\LaTeX\ and lua\LaTeX;
the features in Section~\ref{OpenType} are only usable with xe\LaTeX\ or lua\LaTeX. 
This package compatibly replaces several earlier packages (\texttt{libertine-type1},
\texttt{biolinum-type1}, \texttt{libertine}) and should provide partial compatibility with
the obsolete \texttt{libertineotf} and \texttt{libertine-legacy} packages.

\section{Installation}

To install this package on a TDS-compliant \TeX\ system, download the
file 
\begin{list}{}{}\item \verb\tex-archive/install/fonts/libertine.tds.zip\ 
\end{list}
and unzip at the root of an appropriate
\verb\texmf\ tree, likely a personal or local tree. If necessary, update
the file-name database (e.g., \verb\texhash texmf\). Update the font-map files by
enabling the Map file \verb\libertine.map\.


\section{Basic Usage}

For most purposes, simply add

\begin{list}{}{}\item
\verb|\usepackage{libertine}|
\end{list}
to the preamble of your document. This will activate Libertine as
the main (seriffed) text font, Biolinum as the sans font,
and (from January~2013) LibertineMono as the monospaced font. 
It is
recommended that the font encoding be set to \verb\T1\ or \verb\LY1\ but the default
\verb\OT1\  encoding is also supported. Available shapes in all series (except \texttt{tt}, which
only has \texttt{it}) include:
\begin{list}{}{}\item
\begin{tabular}{ll}
\texttt{it} &               italic\\
\texttt{sc} &            small caps\\
\texttt{scit} &            italic small caps
\end{tabular}
\end{list}
Slanted variants are not supported; the designed italic variants will be
automatically substituted. The exceptions are the monospaced font and the bold series of Biolinum,
for which designed italics are not currently available. Artificially
slanted variants have been generated and treated as if they were italic.

To activate Libertine (without Biolinum), use the \texttt{libertine} (or \texttt{rm})
option. Similarly, to activate Biolinum (without Libertine) use the
\texttt{biolinum} (or \texttt{sf} or \texttt{ss}) option. To use Biolinum as the main text font (as
well as the sans font), use the option \texttt{sfdefault}.
Use the \verb|mono=false| (or \verb|tt=false|) option to suppress
activating LibertineMono. 
To activate single font families,
use one or more of
\begin{list}{}{}
\item \verb|\usepackage{libertineRoman}|
\item \verb|\usepackage{libertineMono}|
\item \verb|\usepackage{biolinum}|
\end{list}


\section{Advanced Usage}

Lua\LaTeX\ and xe\LaTeX\ users who might prefer to use Type~1 fonts or who
wish to avoid \texttt{fontspec} may use the \texttt{type1} (or \texttt{nofontspec}) option.
The \verb\libertine-type1.sty\,
\verb\biolinum-type1.sty\ and \verb\libertineMono-type1.sty\ packages provide
compatibility
with older packages. For legacy documents that use only basic
facilities of \verb\libertineotf\, a wrapper package \verb\libertineotf.sty\ is provided.
The following features of the original \verb|libertine| or \verb|libertineotf|
packages are
\emph{not} supported:
\begin{itemize}

 \item font-features such as \texttt{Ligatures} or \texttt{Scale} as option parameters

 \item the Outline or Shadow fonts 

 \item commands \verb|\Lnnum|, \verb|\Lpnum|, \verb|\Lcnum|, etc.

 \item environments \texttt{Ltable} and \texttt{libertineenumerate} 
\end{itemize}
If your documents use any of the features listed above,
you may have to continue to use the \verb\libertineotf\ package (which is still available from
CTAN) or access the OpenType fonts directly using \texttt{fontspec}.


The following options are available in all styles (except monospaced):
\begin{list}{}{}\item
\begin{tabular}{ll}
\texttt{oldstyle} (\texttt{osf}) &  old-style figures\\
\texttt{lining} (\texttt{nf}, \texttt{lf}) &  lining figures\\
\texttt{proportional} (\texttt{p})&  varying-width figures\\
\texttt{tabular} (\texttt{t}) &     fixed-width figures\\
\end{tabular}
\end{list}
The defaults (from January~2013) are \texttt{lining} and \texttt{tabular}. These apply to both Libertine
and Biolinum;
to change the default figure style of just the Biolinum (sans)
fonts, use options
\begin{list}{}{}\item
\texttt{sflining} (\texttt{sflf}) or \texttt{sfoldstyle} (\texttt{sfosf}, \texttt{osfss})
\item
\texttt{sftabular} (\texttt{sft}) or \texttt{sfproportional} (\texttt{sfp}) 
\end{list}

The \texttt{semibold} (\texttt{sb}) option will enable use of the
semi-bold series of Libertine; this has no effect on the Biolinum fonts,
for which there is no semi-bold variant. The options \verb|scale=|<\emph{number}> (or
\verb|scaled=|<\emph{number}>) will scale the Biolinum fonts but have no effect on the
Libertine fonts. Similarly, the options \verb|ttscale=|<\emph{number}> (or  \verb|ttscaled=|<\emph{number}>)
will scale the LibertineMono font.  Any of the ``Boolean'' options, such as \texttt{osf}, may 
also be used
in the form \verb|osf=true| or \verb|osf=false|.

The option \verb\defaultfeatures=...\ allows the user to add default OpenType
features; for example, \verb\defaultfeatures={Variant=01}\ will force use of the Stylistic~Set~1
variant glyphs.

Commands \verb|\oldstylenums{|\ldots\verb|}| and \verb|\oldstylenumsf{|\ldots\verb|}| are defined to
allow for local use of old-style figures in Libertine and Biolinum,
respectively, if lining figures is the default, and similarly
\verb|\liningnums{|\ldots\verb|}| and \verb|\liningnumsf{|\ldots\verb|}|.

Similarly, commands \verb|\tabularnums{|\ldots\verb|}| and \verb|\tabularnumsf{|\ldots\verb|}| are defined
to allow local use of monospaced figures in Libertine or Biolinum,
respectively, if proportional figures is the default, and similarly
\verb|\proportionalnums{|\ldots\verb|}| and \verb|\proportionalnumsf{|\ldots\verb|}|.

Superior numbers (for footnote markers) are available using \verb|\sufigures|
or \verb|\textsu{|\ldots\verb|}|.


Command \verb|\useosf| switches the default figure style for Libertine and Biolinum to old-style figures; this is
primarily for use \emph{after} calling a math package (such as \verb|newtxmath| with the
\verb|libertine| option) with lining figures as the default.

The following macros select the font family indicated:
\begin{center}
\begin{tabular}{ll}
\verb|\libertine| & Libertine \\
\verb|\libertineSB|& Libertine with semibold \\
\verb|\libertineOsF| & Libertine with oldstyle figures \\
\verb|\libertineLF| & Libertine with lining figures \\
\verb|\libertineDisplay| & Libertine Display \\
\verb|\libmono| & Libertine Monospaced \\
\verb|\libertineInitial| & Libertine Initials \\
\verb|\biolinum| & Biolinum \\
\verb|\biolinumOsF|& Biolinum with oldstyle figures \\
\verb|\biolinumLF| & Biolinum with lining figures \\
\end{tabular}
\end{center}
Macro \verb|\libertineInitialGlyph{|\ldots\verb|}| produces a glyph in the Libertine Initial font;
Appendix~\ref{InitialGlyphs} has a table of some of the glyphs.

\section{OpenType Fonts}
\label{OpenType}

The features in this section are only available to xe\LaTeX\ and lua\LaTeX\ users.

Macros \verb|\libertineGlyph{|\ldots\verb|}| and \verb|\biolinumGlyph{|\ldots\verb|}| produce the
glyph named in the argument in the Libertine or Biolinum font,
respectively; for example, in regular-weight and upright-shape,
\verb|\libertineGlyph{seven.cap}| and \verb|\libertineGlyph{uniE10F}| both produce a
lining~7 that matches the height of capital letters, as in
\begin{list}{}{}\item
K\libertineGlyph{seven.cap}L~\libertineGlyph{three.cap}N\libertineGlyph{six.cap}
\end{list} 
Similarly, \verb|\biolinumKeyGlyph{|\ldots\verb|}| produces the named glyph
in the Biolinum Keyboard font; for example: \verb|\biolinumKeyGlyph{seven}| produces \biolinumKeyGlyph{seven}.
A large number of macros of the form \verb|\LKey|\ldots or \verb|\LMouse|\ldots
are provided to simplify production of glyphs in the Biolinum Keyboard font;
see Appendix~\ref{LKey}. Appendix~\ref{KeyboardGlyphs} has a table of the entire
Linux Biolinum Keyboard font, with corresponding glyph name and codepoint.


The directory
\verb|/fonts/opentype/public/libertine| 
has the fonts used for these features, as follows:
\begin{list}{}{}\item\small
\begin{tabular}{lll}
\multicolumn{1}{c}{\bf File name} & \multicolumn{1}{c}{\bf Internal name} & \multicolumn{1}{c}{\bf Description} \\
\hline
\verb|LinBiolinum_RBO.otf|  & \verb|LinBiolinumOBO|  & sans serif bold italic (oblique) \\
\verb|LinBiolinum_RB.otf|  & \verb|LinBiolinumOB| & sans serif bold \\
\verb|LinBiolinum_RI.otf|  & \verb|LinBiolinumOI| & sans serif italic \\
\verb|LinBiolinum_R.otf|  & \verb|LinBiolinumO| & sans serif regular \\
\verb|LinLibertine_RBI.otf|  & \verb|LinLibertineOBI| & bold italic \\
\verb|LinLibertine_RB.otf|  & \verb|LinLibertineOB| & bold \\
\verb|LinLibertine_RI.otf|  & \verb|LinLibertineOI| & italic \\
\verb|LinLibertine_R.otf|  & \verb|LinLibertineO| & regular \\
\verb|LinLibertine_RZI.otf|  & \verb|LinLibertineOZI| & semibold italic \\
\verb|LinLibertine_RZ.otf|  & \verb|LinLibertineOZ| & semibold \\
\verb|LinLibertine_MBO.otf| & \verb|LinLibertineMOBO| & mono bold italic (oblique) \\
\verb|LinLibertine_MB.otf| & \verb|LinLibertineMOB| & mono bold \\
\verb|LinLibertine_MO.otf| & \verb|LinLibertineMOO| & mono italic (oblique) \\
\verb|LinLibertine_M.otf| & \verb|LinLibertineMO| & mono \\
\verb|LinBiolinum_K.otf|  & \verb|LinBiolinumOKb| & keyboard \\
\verb|LinLibertine_I.otf|  & \verb|LinLibertineIO| &   decorative capitals \\
\verb|LinLibertine_DR.otf| & \verb|LinLibertineDisplayO| &   a display (titling) font  \\
\end{tabular}
\end{list}

\section{Concluding Remarks}


For compatible mathematics, it is recommended to use
\begin{verbatim}
  \usepackage[libertine]{newtxmath}
\end{verbatim}
with pdf\LaTeX\ and
\begin{verbatim}
  \usepackage{unicode-math}
  \setmathfont{texgyrepagellamath-regular.otf}
\end{verbatim}
with xe\LaTeX\ or lua\LaTeX.

The original OpenType fonts were created by Philipp H. Poll 
(\url{gillian@linuxlibertine.org}) and are licensed under the terms of the GNU General
Public License (Version~2, with font exception) and under the terms of
the Open Font License. For details look into the \verb|doc| directory of the
distribution or at
\begin{list}{}{}\item
\url{http://www.linuxlibertine.org/}
\end{list}
The Glyph and KeyCap support was adapted from the original \verb\libertine\
package by Michael Niedermair.

Three of the Libertine fonts were modified by Michael Sharpe (\url{msharpe@ucsd.edu}) using 
\texttt{fontforge} to correct minor problems, including adding
three missing ligatures (\emph{\bfseries fl, ffl, ffi}) to the bold-italic font.

The \typeone\ fonts were created using \verb|cfftot1| or \verb|fontforge|. 
The internal font-family names of the \typeone\ 
fonts have been changed to \verb|Linux Libertine T| and \verb|Linux Biolinum T| to avoid
interfering with xe\LaTeX\ users who access system fonts. 

The support
files were created using \verb|autoinst|. The support files are licensed under
the terms of the LaTeX Project Public License.  
See Appendix~\ref{impl} for more detailed discussion of the implementation.

Thanks to Herbert Voss, Patrick Gundlach, Silke Hofstra, Marc Penninga, Michael Sharpe,
Denis Bitouz\'{e}, and Khaled Hosny for their assistance.
The maintainer of this
package is Bob Tennent (\url{rdt@cs.queensu.ca})

\clearpage

\appendix

\setlength{\fboxrule}{0.1pt}
\section{Biolinum KeyCap Macros}\small\tt
\label{LKey}

\subsection{Special Keys}
\begin{longtable}[l]{lll}
Tux         & \verb|\LKeyTux|      & \LARGE\strut\fbox{\LKeyTux} \\
Win         & \verb|\LKeyWin|      & \LARGE\strut\fbox{\LKeyWin} \\
Menu        & \verb|\LKeyMenu|     & \LARGE\strut\fbox{\LKeyMenu} \\
Strg        & \verb|\LKeyStrg|     & \LARGE\strut\fbox{\LKeyStrg} \\
Ctrl        & \verb|\LKeyCtrl|     & \LARGE\strut\fbox{\LKeyCtrl} \\
Alt         & \verb|\LKeyAlt|      & \LARGE\strut\fbox{\LKeyAlt} \\
AltGr       & \verb|\LKeyAltGr|    & \LARGE\strut\fbox{\LKeyAltGr} \\
Shift       & \verb|\LKeyShift|    & \LARGE\strut\fbox{\LKeyShift} \\
Enter       & \verb|\LKeyEnter|    & \LARGE\strut\fbox{\LKeyEnter} \\
Tab         & \verb|\LKeyTab|      & \LARGE\strut\fbox{\LKeyTab} \\
CapsLock    & \verb|\LKeyCapsLock| & \LARGE\strut\fbox{\LKeyCapsLock} \\
Pos         & \verb|\LKeyPos|      & \LARGE\strut\fbox{\LKeyPos} \\
Entf        & \verb|\LKeyEntf|     & \LARGE\strut\fbox{\LKeyEntf} \\
Einf        & \verb|\LKeyEinf|     & \LARGE\strut\fbox{\LKeyEinf} \\
Leer        & \verb|\LKeyLeer|     & \LARGE\strut\fbox{\LKeyLeer} \\
Esc         & \verb|\LKeyEsc|      & \LARGE\strut\fbox{\LKeyEsc} \\
Ende        & \verb|\LKeyEnde|     & \LARGE\strut\fbox{\LKeyEnde} \\
Back        & \verb|\LKeyBack|     & \LARGE\strut\fbox{\LKeyBack} \\
Up          & \verb|\LKeyUp|       & \LARGE\strut\fbox{\LKeyUp} \\
Dwon        & \verb|\LKeyDown|     & \LARGE\strut\fbox{\LKeyDown} \\
Left        & \verb|\LKeyLeft|     & \LARGE\strut\fbox{\LKeyLeft} \\
Right       & \verb|\LKeyRight|    & \LARGE\strut\fbox{\LKeyRight} \\
PgUp        & \verb|\LKeyPgUp|     & \LARGE\strut\fbox{\LKeyPgUp} \\
PgDown      & \verb|\LKeyPgDown|   & \LARGE\strut\fbox{\LKeyPgDown} \\
At          & \verb|\LKeyAt|       & \LARGE\strut\fbox{\LKeyAt} \\
Fn          & \verb|\LKeyFn|       & \LARGE\strut\fbox{\LKeyFn} \\
Home        & \verb|\LKeyHome|     & \LARGE\strut\fbox{\LKeyHome} \\
Del         & \verb|\LKeyDel|      & \LARGE\strut\fbox{\LKeyDel} \\
Space       & \verb|\LKeySpace|    & \LARGE\strut\fbox{\LKeySpace} \\
ScreenUp    & \verb|\LKeyScreenUp| & \LARGE\strut\fbox{\LKeyScreenUp} \\
ScreenDown  & \verb|\LKeyScreenDown| & \LARGE\strut\fbox{\LKeyScreenDown} \\
Ins         & \verb|\LKeyIns|      & \LARGE\strut\fbox{\LKeyIns} \\
End         & \verb|\LKeyEnd|      & \LARGE\strut\fbox{\LKeyEnd} \\
GNU         & \verb|\LKeyGNU|      & \LARGE\strut\fbox{\LKeyGNU} \\
PageUp      & \verb|\LKeyPageUp|   & \LARGE\strut\fbox{\LKeyPageUp} \\
PageDown    & \verb|\LKeyPageDown| & \LARGE\strut\fbox{\LKeyPageDown} \\
Command     & \verb|\LKeyCommand| & \LARGE\strut\fbox{\LKeyCommand} \\
OptionKey   & \verb|\LKeyOptionKey| & \LARGE\strut\fbox{\LKeyOptionKey} \\
%
F1          & \verb|\LKeyF{1}|     & \LARGE\strut\fbox{\LKeyF{1}} \\
F2          & \verb|\LKeyF{2}|     & \LARGE\strut\fbox{\LKeyF{2}} \\
F3          & \verb|\LKeyF{3}|     & \LARGE\strut\fbox{\LKeyF{3}} \\
F4          & \verb|\LKeyF{4}|     & \LARGE\strut\fbox{\LKeyF{4}} \\
F5          & \verb|\LKeyF{5}|     & \LARGE\strut\fbox{\LKeyF{5}} \\
F6          & \verb|\LKeyF{6}|     & \LARGE\strut\fbox{\LKeyF{6}} \\
F7          & \verb|\LKeyF{7}|     & \LARGE\strut\fbox{\LKeyF{7}} \\
F8          & \verb|\LKeyF{8}|     & \LARGE\strut\fbox{\LKeyF{8}} \\
F9          & \verb|\LKeyF{9}|     & \LARGE\strut\fbox{\LKeyF{9}} \\
F10         & \verb|\LKeyF{10}|    & \LARGE\strut\fbox{\LKeyF{10}} \\
F11         & \verb|\LKeyF{11}|    & \LARGE\strut\fbox{\LKeyF{11}} \\
F12         & \verb|\LKeyF{12}|    & \LARGE\strut\fbox{\LKeyF{12}} \\
F13         & \verb|\LKeyF{13}|    & \LARGE\strut\fbox{\LKeyF{13}} \\
F14         & \verb|\LKeyF{14}|    & \LARGE\strut\fbox{\LKeyF{14}} \\
F15         & \verb|\LKeyF{15}|    & \LARGE\strut\fbox{\LKeyF{15}} \\
F16         & \verb|\LKeyF{16}|    & \LARGE\strut\fbox{\LKeyF{16}} \\
%
PAD0        & \verb|\LKeyPad{1}|     & \LARGE\strut\fbox{\LKeyPad{0}} \\
PAD1        & \verb|\LKeyPad{1}|     & \LARGE\strut\fbox{\LKeyPad{1}} \\
PAD2        & \verb|\LKeyPad{2}|     & \LARGE\strut\fbox{\LKeyPad{2}} \\
PAD3        & \verb|\LKeyPad{3}|     & \LARGE\strut\fbox{\LKeyPad{3}} \\
PAD4        & \verb|\LKeyPad{4}|     & \LARGE\strut\fbox{\LKeyPad{4}} \\
PAD5        & \verb|\LKeyPad{5}|     & \LARGE\strut\fbox{\LKeyPad{5}} \\
PAD6        & \verb|\LKeyPad{6}|     & \LARGE\strut\fbox{\LKeyPad{6}} \\
PAD7        & \verb|\LKeyPad{7}|     & \LARGE\strut\fbox{\LKeyPad{7}} \\
PAD8        & \verb|\LKeyPad{8}|     & \LARGE\strut\fbox{\LKeyPad{8}} \\
PAD9        & \verb|\LKeyPad{9}|     & \LARGE\strut\fbox{\LKeyPad{9}} \\
PAD10       & \verb|\LKeyPad{10}|    & \LARGE\strut\fbox{\LKeyPad{10}} \\
PAD11       & \verb|\LKeyPad{11}|    & \LARGE\strut\fbox{\LKeyPad{11}} \\
PAD12       & \verb|\LKeyPad{12}|    & \LARGE\strut\fbox{\LKeyPad{12}} \\
PAD13       & \verb|\LKeyPad{13}|    & \LARGE\strut\fbox{\LKeyPad{13}} \\
PAD14       & \verb|\LKeyPad{14}|    & \LARGE\strut\fbox{\LKeyPad{14}} \\
\end{longtable}

\clearpage
\subsection{General Keyboard}
\begin{longtable}[l]{lll}
0           & \verb|\LKey{zero}|,\verb|\LKey{0}|    & \LARGE\strut\fbox{\LKey{zero}} \\
9           & \verb|\LKey{nine}|,\verb|\LKey{9}|   & \LARGE\strut\fbox{\LKey{nine}} \\
A           & \verb|\LKey{A}|      & \LARGE\strut\fbox{\LKey{A}} \\
Z           & \verb|\LKey{Z}|      & \LARGE\strut\fbox{\LKey{Z}} \\
\end{longtable}

\subsection{Keyboard Shortcuts}
\begin{longtable}[l]{lll}
Strg-A           & \verb|\LKeyStrgX{A}|       & \LARGE\strut\fbox{\LKeyStrgX{A}} \\
Ctrl-A           & \verb|\LKeyCtrlX{A}|       & \LARGE\strut\fbox{\LKeyCtrlX{A}} \\
Shift-A          & \verb|\LKeyShiftX{A}|      & \LARGE\strut\fbox{\LKeyShiftX{A}} \\
Alt-A            & \verb|\LKeyAltX{A}|        & \LARGE\strut\fbox{\LKeyAltX{A}} \\
AltGr-A          & \verb|\LKeyAltGrX{A}|      & \LARGE\strut\fbox{\LKeyAltGrX{A}} \\
Shift-Strg-A     & \verb|\LKeyShiftStrgX{A}|  & \LARGE\strut\fbox{\LKeyShiftStrgX{A}} \\
Shift-Ctrl-A     & \verb|\LKeyShiftCtrlX{A}|  & \LARGE\strut\fbox{\LKeyShiftCtrlX{A}} \\
Shift-Alt-A      & \verb|\LKeyShiftAltX{A}|   & \LARGE\strut\fbox{\LKeyShiftAltX{A}} \\
Shift-AltGr-A    & \verb|\LKeyShiftAltGrX{A}| & \LARGE\strut\fbox{\LKeyShiftAltGrX{A}} \\
Strg-Alt-A       & \verb|\LKeyStrgAltX{A}|    & \LARGE\strut\fbox{\LKeyStrgAltX{A}} \\
Strg-Alt-Entf    & \verb|\LKeyStrgAltEnt|     & \LARGE\strut\fbox{\LKeyStrgAltEnt} \\
Strg-Alt-Entf    & \verb|\LKeyReset|          & \LARGE\strut\fbox{\LKeyReset} \\
Ctrl-Alt-A       & \verb|\LKeyCtrlAltX{A}|    & \LARGE\strut\fbox{\LKeyCtrlAltX{A}} \\
Ctrl-Alt-Entf    & \verb|\LKeyCtrlAltEnt|     & \LARGE\strut\fbox{\LKeyCtrlAltEnt} \\
%
Alt-F1           & \verb|\LKeyAltF{1}|        & \LARGE\strut\fbox{\LKeyAltF{1}} \\
Alt-F2           & \verb|\LKeyAltF{2}|        & \LARGE\strut\fbox{\LKeyAltF{2}} \\
Alt-F3           & \verb|\LKeyAltF{3}|        & \LARGE\strut\fbox{\LKeyAltF{3}} \\
Alt-F4           & \verb|\LKeyAltF{4}|        & \LARGE\strut\fbox{\LKeyAltF{4}} \\
Alt-F5           & \verb|\LKeyAltF{5}|        & \LARGE\strut\fbox{\LKeyAltF{5}} \\
Alt-F6           & \verb|\LKeyAltF{6}|        & \LARGE\strut\fbox{\LKeyAltF{6}} \\
Alt-F7           & \verb|\LKeyAltF{7}|        & \LARGE\strut\fbox{\LKeyAltF{7}} \\
Alt-F8           & \verb|\LKeyAltF{8}|        & \LARGE\strut\fbox{\LKeyAltF{8}} \\
Alt-F9           & \verb|\LKeyAltF{9}|        & \LARGE\strut\fbox{\LKeyAltF{9}} \\
Alt-F10          & \verb|\LKeyAltF{10}|       & \LARGE\strut\fbox{\LKeyAltF{10}} \\
Alt-F11          & \verb|\LKeyAltF{11}|       & \LARGE\strut\fbox{\LKeyAltF{11}} \\
Alt-F12          & \verb|\LKeyAltF{12}|       & \LARGE\strut\fbox{\LKeyAltF{12}} \\
Alt-F13          & \verb|\LKeyAltF{13}|       & \LARGE\strut\fbox{\LKeyAltF{13}} \\
Alt-F14          & \verb|\LKeyAltF{14}|       & \LARGE\strut\fbox{\LKeyAltF{14}} \\
Alt-F15          & \verb|\LKeyAltF{15}|       & \LARGE\strut\fbox{\LKeyAltF{15}} \\
Alt-F16          & \verb|\LKeyAltF{16}|       & \LARGE\strut\fbox{\LKeyAltF{16}} \\
%
Strg-Alt-F1           & \verb|\LKeyStrgAltF{1}|        & \LARGE\strut\fbox{\LKeyStrgAltF{1}} \\
Strg-Alt-F2           & \verb|\LKeyStrgAltF{2}|        & \LARGE\strut\fbox{\LKeyStrgAltF{2}} \\
Strg-Alt-F3           & \verb|\LKeyStrgAltF{3}|        & \LARGE\strut\fbox{\LKeyStrgAltF{3}} \\
Strg-Alt-F4           & \verb|\LKeyStrgAltF{4}|        & \LARGE\strut\fbox{\LKeyStrgAltF{4}} \\
Strg-Alt-F5           & \verb|\LKeyStrgAltF{5}|        & \LARGE\strut\fbox{\LKeyStrgAltF{5}} \\
Strg-Alt-F6           & \verb|\LKeyStrgAltF{6}|        & \LARGE\strut\fbox{\LKeyStrgAltF{6}} \\
Strg-Alt-F7           & \verb|\LKeyStrgAltF{7}|        & \LARGE\strut\fbox{\LKeyStrgAltF{7}} \\
Strg-Alt-F8           & \verb|\LKeyStrgAltF{8}|        & \LARGE\strut\fbox{\LKeyStrgAltF{8}} \\
Strg-Alt-F9           & \verb|\LKeyStrgAltF{9}|        & \LARGE\strut\fbox{\LKeyStrgAltF{9}} \\
Strg-Alt-F10          & \verb|\LKeyStrgAltF{10}|       & \LARGE\strut\fbox{\LKeyStrgAltF{10}} \\
Strg-Alt-F11          & \verb|\LKeyStrgAltF{11}|       & \LARGE\strut\fbox{\LKeyStrgAltF{11}} \\
Strg-Alt-F12          & \verb|\LKeyStrgAltF{12}|       & \LARGE\strut\fbox{\LKeyStrgAltF{12}} \\
Strg-Alt-F13          & \verb|\LKeyStrgAltF{13}|       & \LARGE\strut\fbox{\LKeyStrgAltF{13}} \\
Strg-Alt-F14          & \verb|\LKeyStrgAltF{14}|       & \LARGE\strut\fbox{\LKeyStrgAltF{14}} \\
Strg-Alt-F15          & \verb|\LKeyStrgAltF{15}|       & \LARGE\strut\fbox{\LKeyStrgAltF{15}} \\
Strg-Alt-F16          & \verb|\LKeyStrgAltF{16}|       & \LARGE\strut\fbox{\LKeyStrgAltF{16}} \\
Ctrl-Alt-F1           & \verb|\LKeyCtrlAltF{1}|        & \LARGE\strut\fbox{\LKeyCtrlAltF{1}} \\
Ctrl-Alt-F2           & \verb|\LKeyCtrlAltF{2}|        & \LARGE\strut\fbox{\LKeyCtrlAltF{2}} \\
Ctrl-Alt-F3           & \verb|\LKeyCtrlAltF{3}|        & \LARGE\strut\fbox{\LKeyCtrlAltF{3}} \\
Ctrl-Alt-F4           & \verb|\LKeyCtrlAltF{4}|        & \LARGE\strut\fbox{\LKeyCtrlAltF{4}} \\
Ctrl-Alt-F5           & \verb|\LKeyCtrlAltF{5}|        & \LARGE\strut\fbox{\LKeyCtrlAltF{5}} \\
Ctrl-Alt-F6           & \verb|\LKeyCtrlAltF{6}|        & \LARGE\strut\fbox{\LKeyCtrlAltF{6}} \\
Ctrl-Alt-F7           & \verb|\LKeyCtrlAltF{7}|        & \LARGE\strut\fbox{\LKeyCtrlAltF{7}} \\
Ctrl-Alt-F8           & \verb|\LKeyCtrlAltF{8}|        & \LARGE\strut\fbox{\LKeyCtrlAltF{8}} \\
Ctrl-Alt-F9           & \verb|\LKeyCtrlAltF{9}|        & \LARGE\strut\fbox{\LKeyCtrlAltF{9}} \\
Ctrl-Alt-F10          & \verb|\LKeyCtrlAltF{10}|       & \LARGE\strut\fbox{\LKeyCtrlAltF{10}} \\
Ctrl-Alt-F11          & \verb|\LKeyCtrlAltF{11}|       & \LARGE\strut\fbox{\LKeyCtrlAltF{11}} \\
Ctrl-Alt-F12          & \verb|\LKeyCtrlAltF{12}|       & \LARGE\strut\fbox{\LKeyCtrlAltF{12}} \\
Ctrl-Alt-F13          & \verb|\LKeyCtrlAltF{13}|       & \LARGE\strut\fbox{\LKeyCtrlAltF{13}} \\
Ctrl-Alt-F14          & \verb|\LKeyCtrlAltF{14}|       & \LARGE\strut\fbox{\LKeyCtrlAltF{14}} \\
Ctrl-Alt-F15          & \verb|\LKeyCtrlAltF{15}|       & \LARGE\strut\fbox{\LKeyCtrlAltF{15}} \\
Ctrl-Alt-F16          & \verb|\LKeyCtrlAltF{16}|       & \LARGE\strut\fbox{\LKeyCtrlAltF{16}} \\
\end{longtable}

\clearpage
\subsection{Mouse Buttons (Three-Button Mice)}
\begin{longtable}[l]{lll}
Empty        & \verb|\LMouseEmpty|   & \LARGE\strut\fbox{\LMouseEmpty} \\
No           & \verb|\LMouseN|       & \LARGE\strut\fbox{\LMouseN} \\
Left         & \verb|\LMouseL|       & \LARGE\strut\fbox{\LMouseL} \\
Middle       & \verb|\LMouseM|       & \LARGE\strut\fbox{\LMouseM} \\
Right        & \verb|\LMouseR|      & \LARGE\strut\fbox{\LMouseR} \\
LeftRight    & \verb|\LMouseLR|      & \LARGE\strut\fbox{\LMouseLR} \\
\end{longtable}

\subsection{Mouse Buttons (Two-Button Mice)}
\begin{longtable}[l]{lll}
Empty        & \verb|\LMouseIIEmpty|   & \LARGE\strut\fbox{\LMouseIIEmpty} \\
No           & \verb|\LMouseIIN|       & \LARGE\strut\fbox{\LMouseIIN} \\
Left         & \verb|\LMouseIIL|       & \LARGE\strut\fbox{\LMouseIIL} \\
Right        & \verb|\LMouseIIR|       & \LARGE\strut\fbox{\LMouseIIR} \\
LeftRight    & \verb|\LMouseIILR|      & \LARGE\strut\fbox{\LMouseIILR} \\
\end{longtable}


\clearpage
\section{Linux Biolinum Keyboard Glyphs}\small\tt
\label{KeyboardGlyphs}
\renewcommand\DeclareTextGlyphY[3]{\makebox[2.5cm]{\LARGE\strut\fbox{\biolinumKeyGlyph{#2}}} #2\\}%
\catcode`\_=12%
\begin{multicols}{2}
\par\noindent
\DeclareTextGlyphY{LinBiolinum_K}{space}{32}
\DeclareTextGlyphY{LinBiolinum_K}{uni0020}{32}
\DeclareTextGlyphY{LinBiolinum_K}{exclam}{33}
\DeclareTextGlyphY{LinBiolinum_K}{uni0021}{33}
\DeclareTextGlyphY{LinBiolinum_K}{quotedbl}{34}
\DeclareTextGlyphY{LinBiolinum_K}{uni0022}{34}
\DeclareTextGlyphY{LinBiolinum_K}{numbersign}{35}
\DeclareTextGlyphY{LinBiolinum_K}{uni0023}{35}
\DeclareTextGlyphY{LinBiolinum_K}{dollar}{36}
\DeclareTextGlyphY{LinBiolinum_K}{uni0024}{36}
\DeclareTextGlyphY{LinBiolinum_K}{percent}{37}
\DeclareTextGlyphY{LinBiolinum_K}{uni0025}{37}
\DeclareTextGlyphY{LinBiolinum_K}{ampersand}{38}
\DeclareTextGlyphY{LinBiolinum_K}{uni0026}{38}
\DeclareTextGlyphY{LinBiolinum_K}{quotesingle}{39}
\DeclareTextGlyphY{LinBiolinum_K}{uni0027}{39}
\DeclareTextGlyphY{LinBiolinum_K}{parenleft}{40}
\DeclareTextGlyphY{LinBiolinum_K}{uni0028}{40}
\DeclareTextGlyphY{LinBiolinum_K}{parenright}{41}
\DeclareTextGlyphY{LinBiolinum_K}{uni0029}{41}
\DeclareTextGlyphY{LinBiolinum_K}{asterisk}{42}
\DeclareTextGlyphY{LinBiolinum_K}{uni002A}{42}
\DeclareTextGlyphY{LinBiolinum_K}{plus}{43}
\DeclareTextGlyphY{LinBiolinum_K}{uni002B}{43}
\DeclareTextGlyphY{LinBiolinum_K}{comma}{44}
\DeclareTextGlyphY{LinBiolinum_K}{uni002C}{44}
\DeclareTextGlyphY{LinBiolinum_K}{hyphen}{45}
\DeclareTextGlyphY{LinBiolinum_K}{uni002D}{45}
\DeclareTextGlyphY{LinBiolinum_K}{period}{46}
\DeclareTextGlyphY{LinBiolinum_K}{uni002E}{46}
\DeclareTextGlyphY{LinBiolinum_K}{slash}{47}
\DeclareTextGlyphY{LinBiolinum_K}{uni002F}{47}
\DeclareTextGlyphY{LinBiolinum_K}{zero}{48}
\DeclareTextGlyphY{LinBiolinum_K}{uni0030}{48}
\DeclareTextGlyphY{LinBiolinum_K}{one}{49}
\DeclareTextGlyphY{LinBiolinum_K}{uni0031}{49}
\DeclareTextGlyphY{LinBiolinum_K}{two}{50}
\DeclareTextGlyphY{LinBiolinum_K}{uni0032}{50}
\DeclareTextGlyphY{LinBiolinum_K}{three}{51}
\DeclareTextGlyphY{LinBiolinum_K}{uni0033}{51}
\DeclareTextGlyphY{LinBiolinum_K}{four}{52}
\DeclareTextGlyphY{LinBiolinum_K}{uni0034}{52}
\DeclareTextGlyphY{LinBiolinum_K}{five}{53}
\DeclareTextGlyphY{LinBiolinum_K}{uni0035}{53}
\DeclareTextGlyphY{LinBiolinum_K}{six}{54}
\DeclareTextGlyphY{LinBiolinum_K}{uni0036}{54}
\DeclareTextGlyphY{LinBiolinum_K}{seven}{55}
\DeclareTextGlyphY{LinBiolinum_K}{uni0037}{55}
\DeclareTextGlyphY{LinBiolinum_K}{eight}{56}
\DeclareTextGlyphY{LinBiolinum_K}{uni0038}{56}
\DeclareTextGlyphY{LinBiolinum_K}{nine}{57}
\DeclareTextGlyphY{LinBiolinum_K}{uni0039}{57}
\DeclareTextGlyphY{LinBiolinum_K}{colon}{58}
\DeclareTextGlyphY{LinBiolinum_K}{uni003A}{58}
\DeclareTextGlyphY{LinBiolinum_K}{semicolon}{59}
\DeclareTextGlyphY{LinBiolinum_K}{uni003B}{59}
\DeclareTextGlyphY{LinBiolinum_K}{less}{60}
\DeclareTextGlyphY{LinBiolinum_K}{uni003C}{60}
\DeclareTextGlyphY{LinBiolinum_K}{equal}{61}
\DeclareTextGlyphY{LinBiolinum_K}{uni003D}{61}
\DeclareTextGlyphY{LinBiolinum_K}{greater}{62}
\DeclareTextGlyphY{LinBiolinum_K}{uni003E}{62}
\DeclareTextGlyphY{LinBiolinum_K}{question}{63}
\DeclareTextGlyphY{LinBiolinum_K}{uni003F}{63}
\DeclareTextGlyphY{LinBiolinum_K}{at}{64}
\DeclareTextGlyphY{LinBiolinum_K}{uni0040}{64}
\DeclareTextGlyphY{LinBiolinum_K}{A}{65}
\DeclareTextGlyphY{LinBiolinum_K}{uni0041}{65}
\DeclareTextGlyphY{LinBiolinum_K}{B}{66}
\DeclareTextGlyphY{LinBiolinum_K}{uni0042}{66}
\DeclareTextGlyphY{LinBiolinum_K}{C}{67}
\DeclareTextGlyphY{LinBiolinum_K}{uni0043}{67}
\DeclareTextGlyphY{LinBiolinum_K}{D}{68}
\DeclareTextGlyphY{LinBiolinum_K}{uni0044}{68}
\DeclareTextGlyphY{LinBiolinum_K}{E}{69}
\DeclareTextGlyphY{LinBiolinum_K}{uni0045}{69}
\DeclareTextGlyphY{LinBiolinum_K}{F}{70}
\DeclareTextGlyphY{LinBiolinum_K}{uni0046}{70}
\DeclareTextGlyphY{LinBiolinum_K}{G}{71}
\DeclareTextGlyphY{LinBiolinum_K}{uni0047}{71}
\DeclareTextGlyphY{LinBiolinum_K}{H}{72}
\DeclareTextGlyphY{LinBiolinum_K}{uni0048}{72}
\DeclareTextGlyphY{LinBiolinum_K}{I}{73}
\DeclareTextGlyphY{LinBiolinum_K}{uni0049}{73}
\DeclareTextGlyphY{LinBiolinum_K}{J}{74}
\DeclareTextGlyphY{LinBiolinum_K}{uni004A}{74}
\DeclareTextGlyphY{LinBiolinum_K}{K}{75}
\DeclareTextGlyphY{LinBiolinum_K}{uni004B}{75}
\DeclareTextGlyphY{LinBiolinum_K}{L}{76}
\DeclareTextGlyphY{LinBiolinum_K}{uni004C}{76}
\DeclareTextGlyphY{LinBiolinum_K}{M}{77}
\DeclareTextGlyphY{LinBiolinum_K}{uni004D}{77}
\DeclareTextGlyphY{LinBiolinum_K}{N}{78}
\DeclareTextGlyphY{LinBiolinum_K}{uni004E}{78}
\DeclareTextGlyphY{LinBiolinum_K}{O}{79}
\DeclareTextGlyphY{LinBiolinum_K}{uni004F}{79}
\DeclareTextGlyphY{LinBiolinum_K}{P}{80}
\DeclareTextGlyphY{LinBiolinum_K}{uni0050}{80}
\DeclareTextGlyphY{LinBiolinum_K}{Q}{81}
\DeclareTextGlyphY{LinBiolinum_K}{uni0051}{81}
\DeclareTextGlyphY{LinBiolinum_K}{R}{82}
\DeclareTextGlyphY{LinBiolinum_K}{uni0052}{82}
\DeclareTextGlyphY{LinBiolinum_K}{S}{83}
\DeclareTextGlyphY{LinBiolinum_K}{uni0053}{83}
\DeclareTextGlyphY{LinBiolinum_K}{T}{84}
\DeclareTextGlyphY{LinBiolinum_K}{uni0054}{84}
\DeclareTextGlyphY{LinBiolinum_K}{U}{85}
\DeclareTextGlyphY{LinBiolinum_K}{uni0055}{85}
\DeclareTextGlyphY{LinBiolinum_K}{V}{86}
\DeclareTextGlyphY{LinBiolinum_K}{uni0056}{86}
\DeclareTextGlyphY{LinBiolinum_K}{W}{87}
\DeclareTextGlyphY{LinBiolinum_K}{uni0057}{87}
\DeclareTextGlyphY{LinBiolinum_K}{X}{88}
\DeclareTextGlyphY{LinBiolinum_K}{uni0058}{88}
\DeclareTextGlyphY{LinBiolinum_K}{Y}{89}
\DeclareTextGlyphY{LinBiolinum_K}{uni0059}{89}
\DeclareTextGlyphY{LinBiolinum_K}{Z}{90}
\DeclareTextGlyphY{LinBiolinum_K}{uni005A}{90}
\DeclareTextGlyphY{LinBiolinum_K}{bracketleft}{91}
\DeclareTextGlyphY{LinBiolinum_K}{uni005B}{91}
\DeclareTextGlyphY{LinBiolinum_K}{backslash}{92}
\DeclareTextGlyphY{LinBiolinum_K}{uni005C}{92}
\DeclareTextGlyphY{LinBiolinum_K}{bracketright}{93}
\DeclareTextGlyphY{LinBiolinum_K}{uni005D}{93}
\DeclareTextGlyphY{LinBiolinum_K}{asciicircum}{94}
\DeclareTextGlyphY{LinBiolinum_K}{uni005E}{94}
\DeclareTextGlyphY{LinBiolinum_K}{underscore}{95}
\DeclareTextGlyphY{LinBiolinum_K}{uni005F}{95}
\DeclareTextGlyphY{LinBiolinum_K}{grave}{96}
\DeclareTextGlyphY{LinBiolinum_K}{uni0060}{96}
\DeclareTextGlyphY{LinBiolinum_K}{a}{97}
\DeclareTextGlyphY{LinBiolinum_K}{uni0061}{97}
\DeclareTextGlyphY{LinBiolinum_K}{b}{98}
\DeclareTextGlyphY{LinBiolinum_K}{uni0062}{98}
\DeclareTextGlyphY{LinBiolinum_K}{c}{99}
\DeclareTextGlyphY{LinBiolinum_K}{uni0063}{99}
\DeclareTextGlyphY{LinBiolinum_K}{d}{100}
\DeclareTextGlyphY{LinBiolinum_K}{uni0064}{100}
\DeclareTextGlyphY{LinBiolinum_K}{e}{101}
\DeclareTextGlyphY{LinBiolinum_K}{uni0065}{101}
\DeclareTextGlyphY{LinBiolinum_K}{f}{102}
\DeclareTextGlyphY{LinBiolinum_K}{uni0066}{102}
\DeclareTextGlyphY{LinBiolinum_K}{g}{103}
\DeclareTextGlyphY{LinBiolinum_K}{uni0067}{103}
\DeclareTextGlyphY{LinBiolinum_K}{h}{104}
\DeclareTextGlyphY{LinBiolinum_K}{uni0068}{104}
\DeclareTextGlyphY{LinBiolinum_K}{i}{105}
\DeclareTextGlyphY{LinBiolinum_K}{uni0069}{105}
\DeclareTextGlyphY{LinBiolinum_K}{j}{106}
\DeclareTextGlyphY{LinBiolinum_K}{uni006A}{106}
\DeclareTextGlyphY{LinBiolinum_K}{k}{107}
\DeclareTextGlyphY{LinBiolinum_K}{uni006B}{107}
\DeclareTextGlyphY{LinBiolinum_K}{l}{108}
\DeclareTextGlyphY{LinBiolinum_K}{uni006C}{108}
\DeclareTextGlyphY{LinBiolinum_K}{m}{109}
\DeclareTextGlyphY{LinBiolinum_K}{uni006D}{109}
\DeclareTextGlyphY{LinBiolinum_K}{n}{110}
\DeclareTextGlyphY{LinBiolinum_K}{uni006E}{110}
\DeclareTextGlyphY{LinBiolinum_K}{o}{111}
\DeclareTextGlyphY{LinBiolinum_K}{uni006F}{111}
\DeclareTextGlyphY{LinBiolinum_K}{p}{112}
\DeclareTextGlyphY{LinBiolinum_K}{uni0070}{112}
\DeclareTextGlyphY{LinBiolinum_K}{q}{113}
\DeclareTextGlyphY{LinBiolinum_K}{uni0071}{113}
\DeclareTextGlyphY{LinBiolinum_K}{r}{114}
\DeclareTextGlyphY{LinBiolinum_K}{uni0072}{114}
\DeclareTextGlyphY{LinBiolinum_K}{s}{115}
\DeclareTextGlyphY{LinBiolinum_K}{uni0073}{115}
\DeclareTextGlyphY{LinBiolinum_K}{t}{116}
\DeclareTextGlyphY{LinBiolinum_K}{uni0074}{116}
\DeclareTextGlyphY{LinBiolinum_K}{u}{117}
\DeclareTextGlyphY{LinBiolinum_K}{uni0075}{117}
\DeclareTextGlyphY{LinBiolinum_K}{v}{118}
\DeclareTextGlyphY{LinBiolinum_K}{uni0076}{118}
\DeclareTextGlyphY{LinBiolinum_K}{w}{119}
\DeclareTextGlyphY{LinBiolinum_K}{uni0077}{119}
\DeclareTextGlyphY{LinBiolinum_K}{x}{120}
\DeclareTextGlyphY{LinBiolinum_K}{uni0078}{120}
\DeclareTextGlyphY{LinBiolinum_K}{y}{121}
\DeclareTextGlyphY{LinBiolinum_K}{uni0079}{121}
\DeclareTextGlyphY{LinBiolinum_K}{z}{122}
\DeclareTextGlyphY{LinBiolinum_K}{uni007A}{122}
\DeclareTextGlyphY{LinBiolinum_K}{braceleft}{123}
\DeclareTextGlyphY{LinBiolinum_K}{uni007B}{123}
\DeclareTextGlyphY{LinBiolinum_K}{bar}{124}
\DeclareTextGlyphY{LinBiolinum_K}{uni007C}{124}
\DeclareTextGlyphY{LinBiolinum_K}{braceright}{125}
\DeclareTextGlyphY{LinBiolinum_K}{uni007D}{125}
\DeclareTextGlyphY{LinBiolinum_K}{asciitilde}{126}
\DeclareTextGlyphY{LinBiolinum_K}{uni007E}{126}
\DeclareTextGlyphY{LinBiolinum_K}{exclamdown}{161}
\DeclareTextGlyphY{LinBiolinum_K}{uni00A1}{161}
\DeclareTextGlyphY{LinBiolinum_K}{cent}{162}
\DeclareTextGlyphY{LinBiolinum_K}{uni00A2}{162}
\DeclareTextGlyphY{LinBiolinum_K}{sterling}{163}
\DeclareTextGlyphY{LinBiolinum_K}{uni00A3}{163}
\DeclareTextGlyphY{LinBiolinum_K}{currency}{164}
\DeclareTextGlyphY{LinBiolinum_K}{uni00A4}{164}
\DeclareTextGlyphY{LinBiolinum_K}{yen}{165}
\DeclareTextGlyphY{LinBiolinum_K}{uni00A5}{165}
\DeclareTextGlyphY{LinBiolinum_K}{brokenbar}{166}
\DeclareTextGlyphY{LinBiolinum_K}{uni00A6}{166}
\DeclareTextGlyphY{LinBiolinum_K}{section}{167}
\DeclareTextGlyphY{LinBiolinum_K}{uni00A7}{167}
\DeclareTextGlyphY{LinBiolinum_K}{dieresis}{168}
\DeclareTextGlyphY{LinBiolinum_K}{uni00A8}{168}
\DeclareTextGlyphY{LinBiolinum_K}{guillemotleft}{171}
\DeclareTextGlyphY{LinBiolinum_K}{uni00AB}{171}
\DeclareTextGlyphY{LinBiolinum_K}{uni00AD}{173}
\DeclareTextGlyphY{LinBiolinum_K}{degree}{176}
\DeclareTextGlyphY{LinBiolinum_K}{uni00B0}{176}
\DeclareTextGlyphY{LinBiolinum_K}{plusminus}{177}
\DeclareTextGlyphY{LinBiolinum_K}{uni00B1}{177}
\DeclareTextGlyphY{LinBiolinum_K}{acute}{180}
\DeclareTextGlyphY{LinBiolinum_K}{uni00B4}{180}
\DeclareTextGlyphY{LinBiolinum_K}{uni00B5}{181}
\DeclareTextGlyphY{LinBiolinum_K}{periodcentered}{183}
\DeclareTextGlyphY{LinBiolinum_K}{uni00B7}{183}
\DeclareTextGlyphY{LinBiolinum_K}{cedilla}{184}
\DeclareTextGlyphY{LinBiolinum_K}{uni00B8}{184}
\DeclareTextGlyphY{LinBiolinum_K}{guillemotright}{187}
\DeclareTextGlyphY{LinBiolinum_K}{uni00BB}{187}
\DeclareTextGlyphY{LinBiolinum_K}{Agrave}{192}
\DeclareTextGlyphY{LinBiolinum_K}{uni00C0}{192}
\DeclareTextGlyphY{LinBiolinum_K}{Aacute}{193}
\DeclareTextGlyphY{LinBiolinum_K}{uni00C1}{193}
\DeclareTextGlyphY{LinBiolinum_K}{Acircumflex}{194}
\DeclareTextGlyphY{LinBiolinum_K}{uni00C2}{194}
\DeclareTextGlyphY{LinBiolinum_K}{Atilde}{195}
\DeclareTextGlyphY{LinBiolinum_K}{uni00C3}{195}
\DeclareTextGlyphY{LinBiolinum_K}{Adieresis}{196}
\DeclareTextGlyphY{LinBiolinum_K}{uni00C4}{196}
\DeclareTextGlyphY{LinBiolinum_K}{Aring}{197}
\DeclareTextGlyphY{LinBiolinum_K}{uni00C5}{197}
\DeclareTextGlyphY{LinBiolinum_K}{Ccedilla}{199}
\DeclareTextGlyphY{LinBiolinum_K}{uni00C7}{199}
\DeclareTextGlyphY{LinBiolinum_K}{Egrave}{200}
\DeclareTextGlyphY{LinBiolinum_K}{uni00C8}{200}
\DeclareTextGlyphY{LinBiolinum_K}{Eacute}{201}
\DeclareTextGlyphY{LinBiolinum_K}{uni00C9}{201}
\DeclareTextGlyphY{LinBiolinum_K}{Ecircumflex}{202}
\DeclareTextGlyphY{LinBiolinum_K}{uni00CA}{202}
\DeclareTextGlyphY{LinBiolinum_K}{Edieresis}{203}
\DeclareTextGlyphY{LinBiolinum_K}{uni00CB}{203}
\DeclareTextGlyphY{LinBiolinum_K}{Igrave}{204}
\DeclareTextGlyphY{LinBiolinum_K}{uni00CC}{204}
\DeclareTextGlyphY{LinBiolinum_K}{Iacute}{205}
\DeclareTextGlyphY{LinBiolinum_K}{uni00CD}{205}
\DeclareTextGlyphY{LinBiolinum_K}{Icircumflex}{206}
\DeclareTextGlyphY{LinBiolinum_K}{uni00CE}{206}
\DeclareTextGlyphY{LinBiolinum_K}{Idieresis}{207}
\DeclareTextGlyphY{LinBiolinum_K}{uni00CF}{207}
\DeclareTextGlyphY{LinBiolinum_K}{Eth}{208}
\DeclareTextGlyphY{LinBiolinum_K}{uni00D0}{208}
\DeclareTextGlyphY{LinBiolinum_K}{Ntilde}{209}
\DeclareTextGlyphY{LinBiolinum_K}{uni00D1}{209}
\DeclareTextGlyphY{LinBiolinum_K}{Ograve}{210}
\DeclareTextGlyphY{LinBiolinum_K}{uni00D2}{210}
\DeclareTextGlyphY{LinBiolinum_K}{Oacute}{211}
\DeclareTextGlyphY{LinBiolinum_K}{uni00D3}{211}
\DeclareTextGlyphY{LinBiolinum_K}{Ocircumflex}{212}
\DeclareTextGlyphY{LinBiolinum_K}{uni00D4}{212}
\DeclareTextGlyphY{LinBiolinum_K}{Otilde}{213}
\DeclareTextGlyphY{LinBiolinum_K}{uni00D5}{213}
\DeclareTextGlyphY{LinBiolinum_K}{Odieresis}{214}
\DeclareTextGlyphY{LinBiolinum_K}{uni00D6}{214}
\DeclareTextGlyphY{LinBiolinum_K}{multiply}{215}
\DeclareTextGlyphY{LinBiolinum_K}{uni00D7}{215}
\DeclareTextGlyphY{LinBiolinum_K}{Oslash}{216}
\DeclareTextGlyphY{LinBiolinum_K}{uni00D8}{216}
\DeclareTextGlyphY{LinBiolinum_K}{Ugrave}{217}
\DeclareTextGlyphY{LinBiolinum_K}{uni00D9}{217}
\DeclareTextGlyphY{LinBiolinum_K}{Uacute}{218}
\DeclareTextGlyphY{LinBiolinum_K}{uni00DA}{218}
\DeclareTextGlyphY{LinBiolinum_K}{Ucircumflex}{219}
\DeclareTextGlyphY{LinBiolinum_K}{uni00DB}{219}
\DeclareTextGlyphY{LinBiolinum_K}{Udieresis}{220}
\DeclareTextGlyphY{LinBiolinum_K}{uni00DC}{220}
\DeclareTextGlyphY{LinBiolinum_K}{Yacute}{221}
\DeclareTextGlyphY{LinBiolinum_K}{uni00DD}{221}
\DeclareTextGlyphY{LinBiolinum_K}{Thorn}{222}
\DeclareTextGlyphY{LinBiolinum_K}{uni00DE}{222}
\DeclareTextGlyphY{LinBiolinum_K}{germandbls}{223}
\DeclareTextGlyphY{LinBiolinum_K}{uni00DF}{223}
\DeclareTextGlyphY{LinBiolinum_K}{agrave}{224}
\DeclareTextGlyphY{LinBiolinum_K}{uni00E0}{224}
\DeclareTextGlyphY{LinBiolinum_K}{aacute}{225}
\DeclareTextGlyphY{LinBiolinum_K}{uni00E1}{225}
\DeclareTextGlyphY{LinBiolinum_K}{acircumflex}{226}
\DeclareTextGlyphY{LinBiolinum_K}{uni00E2}{226}
\DeclareTextGlyphY{LinBiolinum_K}{atilde}{227}
\DeclareTextGlyphY{LinBiolinum_K}{uni00E3}{227}
\DeclareTextGlyphY{LinBiolinum_K}{adieresis}{228}
\DeclareTextGlyphY{LinBiolinum_K}{uni00E4}{228}
\DeclareTextGlyphY{LinBiolinum_K}{aring}{229}
\DeclareTextGlyphY{LinBiolinum_K}{uni00E5}{229}
\DeclareTextGlyphY{LinBiolinum_K}{ae}{230}
\DeclareTextGlyphY{LinBiolinum_K}{uni00E6}{230}
\DeclareTextGlyphY{LinBiolinum_K}{ccedilla}{231}
\DeclareTextGlyphY{LinBiolinum_K}{uni00E7}{231}
\DeclareTextGlyphY{LinBiolinum_K}{egrave}{232}
\DeclareTextGlyphY{LinBiolinum_K}{uni00E8}{232}
\DeclareTextGlyphY{LinBiolinum_K}{eacute}{233}
\DeclareTextGlyphY{LinBiolinum_K}{uni00E9}{233}
\DeclareTextGlyphY{LinBiolinum_K}{ecircumflex}{234}
\DeclareTextGlyphY{LinBiolinum_K}{uni00EA}{234}
\DeclareTextGlyphY{LinBiolinum_K}{edieresis}{235}
\DeclareTextGlyphY{LinBiolinum_K}{uni00EB}{235}
\DeclareTextGlyphY{LinBiolinum_K}{igrave}{236}
\DeclareTextGlyphY{LinBiolinum_K}{uni00EC}{236}
\DeclareTextGlyphY{LinBiolinum_K}{iacute}{237}
\DeclareTextGlyphY{LinBiolinum_K}{uni00ED}{237}
\DeclareTextGlyphY{LinBiolinum_K}{icircumflex}{238}
\DeclareTextGlyphY{LinBiolinum_K}{uni00EE}{238}
\DeclareTextGlyphY{LinBiolinum_K}{idieresis}{239}
\DeclareTextGlyphY{LinBiolinum_K}{uni00EF}{239}
\DeclareTextGlyphY{LinBiolinum_K}{eth}{240}
\DeclareTextGlyphY{LinBiolinum_K}{uni00F0}{240}
\DeclareTextGlyphY{LinBiolinum_K}{ntilde}{241}
\DeclareTextGlyphY{LinBiolinum_K}{uni00F1}{241}
\DeclareTextGlyphY{LinBiolinum_K}{ograve}{242}
\DeclareTextGlyphY{LinBiolinum_K}{uni00F2}{242}
\DeclareTextGlyphY{LinBiolinum_K}{oacute}{243}
\DeclareTextGlyphY{LinBiolinum_K}{uni00F3}{243}
\DeclareTextGlyphY{LinBiolinum_K}{ocircumflex}{244}
\DeclareTextGlyphY{LinBiolinum_K}{uni00F4}{244}
\DeclareTextGlyphY{LinBiolinum_K}{otilde}{245}
\DeclareTextGlyphY{LinBiolinum_K}{uni00F5}{245}
\DeclareTextGlyphY{LinBiolinum_K}{odieresis}{246}
\DeclareTextGlyphY{LinBiolinum_K}{uni00F6}{246}
\DeclareTextGlyphY{LinBiolinum_K}{divide}{247}
\DeclareTextGlyphY{LinBiolinum_K}{uni00F7}{247}
\DeclareTextGlyphY{LinBiolinum_K}{oslash}{248}
\DeclareTextGlyphY{LinBiolinum_K}{uni00F8}{248}
\DeclareTextGlyphY{LinBiolinum_K}{ugrave}{249}
\DeclareTextGlyphY{LinBiolinum_K}{uni00F9}{249}
\DeclareTextGlyphY{LinBiolinum_K}{uacute}{250}
\DeclareTextGlyphY{LinBiolinum_K}{uni00FA}{250}
\DeclareTextGlyphY{LinBiolinum_K}{ucircumflex}{251}
\DeclareTextGlyphY{LinBiolinum_K}{uni00FB}{251}
\DeclareTextGlyphY{LinBiolinum_K}{udieresis}{252}
\DeclareTextGlyphY{LinBiolinum_K}{uni00FC}{252}
\DeclareTextGlyphY{LinBiolinum_K}{yacute}{253}
\DeclareTextGlyphY{LinBiolinum_K}{uni00FD}{253}
\DeclareTextGlyphY{LinBiolinum_K}{thorn}{254}
\DeclareTextGlyphY{LinBiolinum_K}{uni00FE}{254}
\DeclareTextGlyphY{LinBiolinum_K}{ydieresis}{255}
\DeclareTextGlyphY{LinBiolinum_K}{uni00FF}{255}
\DeclareTextGlyphY{LinBiolinum_K}{Amacron}{256}
\DeclareTextGlyphY{LinBiolinum_K}{uni0100}{256}
\DeclareTextGlyphY{LinBiolinum_K}{amacron}{257}
\DeclareTextGlyphY{LinBiolinum_K}{uni0101}{257}
\DeclareTextGlyphY{LinBiolinum_K}{Abreve}{258}
\DeclareTextGlyphY{LinBiolinum_K}{uni0102}{258}
\DeclareTextGlyphY{LinBiolinum_K}{abreve}{259}
\DeclareTextGlyphY{LinBiolinum_K}{uni0103}{259}
\DeclareTextGlyphY{LinBiolinum_K}{Aogonek}{260}
\DeclareTextGlyphY{LinBiolinum_K}{uni0104}{260}
\DeclareTextGlyphY{LinBiolinum_K}{aogonek}{261}
\DeclareTextGlyphY{LinBiolinum_K}{uni0105}{261}
\DeclareTextGlyphY{LinBiolinum_K}{Cacute}{262}
\DeclareTextGlyphY{LinBiolinum_K}{uni0106}{262}
\DeclareTextGlyphY{LinBiolinum_K}{cacute}{263}
\DeclareTextGlyphY{LinBiolinum_K}{uni0107}{263}
\DeclareTextGlyphY{LinBiolinum_K}{Ccircumflex}{264}
\DeclareTextGlyphY{LinBiolinum_K}{uni0108}{264}
\DeclareTextGlyphY{LinBiolinum_K}{ccircumflex}{265}
\DeclareTextGlyphY{LinBiolinum_K}{uni0109}{265}
\DeclareTextGlyphY{LinBiolinum_K}{Cdotaccent}{266}
\DeclareTextGlyphY{LinBiolinum_K}{uni010A}{266}
\DeclareTextGlyphY{LinBiolinum_K}{cdotaccent}{267}
\DeclareTextGlyphY{LinBiolinum_K}{uni010B}{267}
\DeclareTextGlyphY{LinBiolinum_K}{Ccaron}{268}
\DeclareTextGlyphY{LinBiolinum_K}{uni010C}{268}
\DeclareTextGlyphY{LinBiolinum_K}{ccaron}{269}
\DeclareTextGlyphY{LinBiolinum_K}{uni010D}{269}
\DeclareTextGlyphY{LinBiolinum_K}{Dcaron}{270}
\DeclareTextGlyphY{LinBiolinum_K}{uni010E}{270}
\DeclareTextGlyphY{LinBiolinum_K}{dcaron}{271}
\DeclareTextGlyphY{LinBiolinum_K}{uni010F}{271}
\DeclareTextGlyphY{LinBiolinum_K}{Dcroat}{272}
\DeclareTextGlyphY{LinBiolinum_K}{uni0110}{272}
\DeclareTextGlyphY{LinBiolinum_K}{dcroat}{273}
\DeclareTextGlyphY{LinBiolinum_K}{uni0111}{273}
\DeclareTextGlyphY{LinBiolinum_K}{Emacron}{274}
\DeclareTextGlyphY{LinBiolinum_K}{uni0112}{274}
\DeclareTextGlyphY{LinBiolinum_K}{emacron}{275}
\DeclareTextGlyphY{LinBiolinum_K}{uni0113}{275}
\DeclareTextGlyphY{LinBiolinum_K}{Ebreve}{276}
\DeclareTextGlyphY{LinBiolinum_K}{uni0114}{276}
\DeclareTextGlyphY{LinBiolinum_K}{ebreve}{277}
\DeclareTextGlyphY{LinBiolinum_K}{uni0115}{277}
\DeclareTextGlyphY{LinBiolinum_K}{Edotaccent}{278}
\DeclareTextGlyphY{LinBiolinum_K}{uni0116}{278}
\DeclareTextGlyphY{LinBiolinum_K}{edotaccent}{279}
\DeclareTextGlyphY{LinBiolinum_K}{uni0117}{279}
\DeclareTextGlyphY{LinBiolinum_K}{Eogonek}{280}
\DeclareTextGlyphY{LinBiolinum_K}{uni0118}{280}
\DeclareTextGlyphY{LinBiolinum_K}{eogonek}{281}
\DeclareTextGlyphY{LinBiolinum_K}{uni0119}{281}
\DeclareTextGlyphY{LinBiolinum_K}{Ecaron}{282}
\DeclareTextGlyphY{LinBiolinum_K}{uni011A}{282}
\DeclareTextGlyphY{LinBiolinum_K}{ecaron}{283}
\DeclareTextGlyphY{LinBiolinum_K}{uni011B}{283}
\DeclareTextGlyphY{LinBiolinum_K}{Gcircumflex}{284}
\DeclareTextGlyphY{LinBiolinum_K}{uni011C}{284}
\DeclareTextGlyphY{LinBiolinum_K}{gcircumflex}{285}
\DeclareTextGlyphY{LinBiolinum_K}{uni011D}{285}
\DeclareTextGlyphY{LinBiolinum_K}{Gbreve}{286}
\DeclareTextGlyphY{LinBiolinum_K}{uni011E}{286}
\DeclareTextGlyphY{LinBiolinum_K}{gbreve}{287}
\DeclareTextGlyphY{LinBiolinum_K}{uni011F}{287}
\DeclareTextGlyphY{LinBiolinum_K}{Gdotaccent}{288}
\DeclareTextGlyphY{LinBiolinum_K}{uni0120}{288}
\DeclareTextGlyphY{LinBiolinum_K}{gdotaccent}{289}
\DeclareTextGlyphY{LinBiolinum_K}{uni0121}{289}
\DeclareTextGlyphY{LinBiolinum_K}{Gcommaaccent}{290}
\DeclareTextGlyphY{LinBiolinum_K}{uni0122}{290}
\DeclareTextGlyphY{LinBiolinum_K}{gcommaaccent}{291}
\DeclareTextGlyphY{LinBiolinum_K}{uni0123}{291}
\DeclareTextGlyphY{LinBiolinum_K}{Hcircumflex}{292}
\DeclareTextGlyphY{LinBiolinum_K}{uni0124}{292}
\DeclareTextGlyphY{LinBiolinum_K}{hcircumflex}{293}
\DeclareTextGlyphY{LinBiolinum_K}{uni0125}{293}
\DeclareTextGlyphY{LinBiolinum_K}{Hbar}{294}
\DeclareTextGlyphY{LinBiolinum_K}{uni0126}{294}
\DeclareTextGlyphY{LinBiolinum_K}{hbar}{295}
\DeclareTextGlyphY{LinBiolinum_K}{uni0127}{295}
\DeclareTextGlyphY{LinBiolinum_K}{Itilde}{296}
\DeclareTextGlyphY{LinBiolinum_K}{uni0128}{296}
\DeclareTextGlyphY{LinBiolinum_K}{itilde}{297}
\DeclareTextGlyphY{LinBiolinum_K}{uni0129}{297}
\DeclareTextGlyphY{LinBiolinum_K}{Imacron}{298}
\DeclareTextGlyphY{LinBiolinum_K}{uni012A}{298}
\DeclareTextGlyphY{LinBiolinum_K}{imacron}{299}
\DeclareTextGlyphY{LinBiolinum_K}{uni012B}{299}
\DeclareTextGlyphY{LinBiolinum_K}{Ibreve}{300}
\DeclareTextGlyphY{LinBiolinum_K}{uni012C}{300}
\DeclareTextGlyphY{LinBiolinum_K}{ibreve}{301}
\DeclareTextGlyphY{LinBiolinum_K}{uni012D}{301}
\DeclareTextGlyphY{LinBiolinum_K}{Iogonek}{302}
\DeclareTextGlyphY{LinBiolinum_K}{uni012E}{302}
\DeclareTextGlyphY{LinBiolinum_K}{iogonek}{303}
\DeclareTextGlyphY{LinBiolinum_K}{uni012F}{303}
\DeclareTextGlyphY{LinBiolinum_K}{Idotaccent}{304}
\DeclareTextGlyphY{LinBiolinum_K}{uni0130}{304}
\DeclareTextGlyphY{LinBiolinum_K}{dotlessi}{305}
\DeclareTextGlyphY{LinBiolinum_K}{uni0131}{305}
\DeclareTextGlyphY{LinBiolinum_K}{IJ}{306}
\DeclareTextGlyphY{LinBiolinum_K}{uni0132}{306}
\DeclareTextGlyphY{LinBiolinum_K}{ij}{307}
\DeclareTextGlyphY{LinBiolinum_K}{uni0133}{307}
\DeclareTextGlyphY{LinBiolinum_K}{Jcircumflex}{308}
\DeclareTextGlyphY{LinBiolinum_K}{uni0134}{308}
\DeclareTextGlyphY{LinBiolinum_K}{jcircumflex}{309}
\DeclareTextGlyphY{LinBiolinum_K}{uni0135}{309}
\DeclareTextGlyphY{LinBiolinum_K}{Kcommaaccent}{310}
\DeclareTextGlyphY{LinBiolinum_K}{uni0136}{310}
\DeclareTextGlyphY{LinBiolinum_K}{kcommaaccent}{311}
\DeclareTextGlyphY{LinBiolinum_K}{uni0137}{311}
\DeclareTextGlyphY{LinBiolinum_K}{kgreenlandic}{312}
\DeclareTextGlyphY{LinBiolinum_K}{uni0138}{312}
\DeclareTextGlyphY{LinBiolinum_K}{Lacute}{313}
\DeclareTextGlyphY{LinBiolinum_K}{uni0139}{313}
\DeclareTextGlyphY{LinBiolinum_K}{lacute}{314}
\DeclareTextGlyphY{LinBiolinum_K}{uni013A}{314}
\DeclareTextGlyphY{LinBiolinum_K}{Lcommaaccent}{315}
\DeclareTextGlyphY{LinBiolinum_K}{uni013B}{315}
\DeclareTextGlyphY{LinBiolinum_K}{lcommaaccent}{316}
\DeclareTextGlyphY{LinBiolinum_K}{uni013C}{316}
\DeclareTextGlyphY{LinBiolinum_K}{Lcaron}{317}
\DeclareTextGlyphY{LinBiolinum_K}{uni013D}{317}
\DeclareTextGlyphY{LinBiolinum_K}{lcaron}{318}
\DeclareTextGlyphY{LinBiolinum_K}{uni013E}{318}
\DeclareTextGlyphY{LinBiolinum_K}{Ldot}{319}
\DeclareTextGlyphY{LinBiolinum_K}{uni013F}{319}
\DeclareTextGlyphY{LinBiolinum_K}{ldot}{320}
\DeclareTextGlyphY{LinBiolinum_K}{uni0140}{320}
\DeclareTextGlyphY{LinBiolinum_K}{Lslash}{321}
\DeclareTextGlyphY{LinBiolinum_K}{uni0141}{321}
\DeclareTextGlyphY{LinBiolinum_K}{lslash}{322}
\DeclareTextGlyphY{LinBiolinum_K}{uni0142}{322}
\DeclareTextGlyphY{LinBiolinum_K}{Nacute}{323}
\DeclareTextGlyphY{LinBiolinum_K}{uni0143}{323}
\DeclareTextGlyphY{LinBiolinum_K}{nacute}{324}
\DeclareTextGlyphY{LinBiolinum_K}{uni0144}{324}
\DeclareTextGlyphY{LinBiolinum_K}{Ncommaaccent}{325}
\DeclareTextGlyphY{LinBiolinum_K}{uni0145}{325}
\DeclareTextGlyphY{LinBiolinum_K}{ncommaaccent}{326}
\DeclareTextGlyphY{LinBiolinum_K}{uni0146}{326}
\DeclareTextGlyphY{LinBiolinum_K}{Ncaron}{327}
\DeclareTextGlyphY{LinBiolinum_K}{uni0147}{327}
\DeclareTextGlyphY{LinBiolinum_K}{ncaron}{328}
\DeclareTextGlyphY{LinBiolinum_K}{uni0148}{328}
\DeclareTextGlyphY{LinBiolinum_K}{napostrophe}{329}
\DeclareTextGlyphY{LinBiolinum_K}{uni0149}{329}
\DeclareTextGlyphY{LinBiolinum_K}{Omacron}{332}
\DeclareTextGlyphY{LinBiolinum_K}{uni014C}{332}
\DeclareTextGlyphY{LinBiolinum_K}{omacron}{333}
\DeclareTextGlyphY{LinBiolinum_K}{uni014D}{333}
\DeclareTextGlyphY{LinBiolinum_K}{Obreve}{334}
\DeclareTextGlyphY{LinBiolinum_K}{uni014E}{334}
\DeclareTextGlyphY{LinBiolinum_K}{obreve}{335}
\DeclareTextGlyphY{LinBiolinum_K}{uni014F}{335}
\DeclareTextGlyphY{LinBiolinum_K}{Ohungarumlaut}{336}
\DeclareTextGlyphY{LinBiolinum_K}{uni0150}{336}
\DeclareTextGlyphY{LinBiolinum_K}{ohungarumlaut}{337}
\DeclareTextGlyphY{LinBiolinum_K}{uni0151}{337}
\DeclareTextGlyphY{LinBiolinum_K}{Racute}{340}
\DeclareTextGlyphY{LinBiolinum_K}{uni0154}{340}
\DeclareTextGlyphY{LinBiolinum_K}{racute}{341}
\DeclareTextGlyphY{LinBiolinum_K}{uni0155}{341}
\DeclareTextGlyphY{LinBiolinum_K}{Rcommaaccent}{342}
\DeclareTextGlyphY{LinBiolinum_K}{uni0156}{342}
\DeclareTextGlyphY{LinBiolinum_K}{rcommaaccent}{343}
\DeclareTextGlyphY{LinBiolinum_K}{uni0157}{343}
\DeclareTextGlyphY{LinBiolinum_K}{Rcaron}{344}
\DeclareTextGlyphY{LinBiolinum_K}{uni0158}{344}
\DeclareTextGlyphY{LinBiolinum_K}{rcaron}{345}
\DeclareTextGlyphY{LinBiolinum_K}{uni0159}{345}
\DeclareTextGlyphY{LinBiolinum_K}{Sacute}{346}
\DeclareTextGlyphY{LinBiolinum_K}{uni015A}{346}
\DeclareTextGlyphY{LinBiolinum_K}{sacute}{347}
\DeclareTextGlyphY{LinBiolinum_K}{uni015B}{347}
\DeclareTextGlyphY{LinBiolinum_K}{Scircumflex}{348}
\DeclareTextGlyphY{LinBiolinum_K}{uni015C}{348}
\DeclareTextGlyphY{LinBiolinum_K}{scircumflex}{349}
\DeclareTextGlyphY{LinBiolinum_K}{uni015D}{349}
\DeclareTextGlyphY{LinBiolinum_K}{Scedilla}{350}
\DeclareTextGlyphY{LinBiolinum_K}{uni015E}{350}
\DeclareTextGlyphY{LinBiolinum_K}{scedilla}{351}
\DeclareTextGlyphY{LinBiolinum_K}{uni015F}{351}
\DeclareTextGlyphY{LinBiolinum_K}{Scaron}{352}
\DeclareTextGlyphY{LinBiolinum_K}{uni0160}{352}
\DeclareTextGlyphY{LinBiolinum_K}{scaron}{353}
\DeclareTextGlyphY{LinBiolinum_K}{uni0161}{353}
\DeclareTextGlyphY{LinBiolinum_K}{Tcedilla}{354}
\DeclareTextGlyphY{LinBiolinum_K}{uni0162}{354}
\DeclareTextGlyphY{LinBiolinum_K}{tcedilla}{355}
\DeclareTextGlyphY{LinBiolinum_K}{uni0163}{355}
\DeclareTextGlyphY{LinBiolinum_K}{Tcaron}{356}
\DeclareTextGlyphY{LinBiolinum_K}{uni0164}{356}
\DeclareTextGlyphY{LinBiolinum_K}{tcaron}{357}
\DeclareTextGlyphY{LinBiolinum_K}{uni0165}{357}
\DeclareTextGlyphY{LinBiolinum_K}{Tbar}{358}
\DeclareTextGlyphY{LinBiolinum_K}{uni0166}{358}
\DeclareTextGlyphY{LinBiolinum_K}{tbar}{359}
\DeclareTextGlyphY{LinBiolinum_K}{uni0167}{359}
\DeclareTextGlyphY{LinBiolinum_K}{Utilde}{360}
\DeclareTextGlyphY{LinBiolinum_K}{uni0168}{360}
\DeclareTextGlyphY{LinBiolinum_K}{utilde}{361}
\DeclareTextGlyphY{LinBiolinum_K}{uni0169}{361}
\DeclareTextGlyphY{LinBiolinum_K}{Umacron}{362}
\DeclareTextGlyphY{LinBiolinum_K}{uni016A}{362}
\DeclareTextGlyphY{LinBiolinum_K}{umacron}{363}
\DeclareTextGlyphY{LinBiolinum_K}{uni016B}{363}
\DeclareTextGlyphY{LinBiolinum_K}{Ubreve}{364}
\DeclareTextGlyphY{LinBiolinum_K}{uni016C}{364}
\DeclareTextGlyphY{LinBiolinum_K}{ubreve}{365}
\DeclareTextGlyphY{LinBiolinum_K}{uni016D}{365}
\DeclareTextGlyphY{LinBiolinum_K}{Uring}{366}
\DeclareTextGlyphY{LinBiolinum_K}{uni016E}{366}
\DeclareTextGlyphY{LinBiolinum_K}{uring}{367}
\DeclareTextGlyphY{LinBiolinum_K}{uni016F}{367}
\DeclareTextGlyphY{LinBiolinum_K}{Uhungarumlaut}{368}
\DeclareTextGlyphY{LinBiolinum_K}{uni0170}{368}
\DeclareTextGlyphY{LinBiolinum_K}{uhungarumlaut}{369}
\DeclareTextGlyphY{LinBiolinum_K}{uni0171}{369}
\DeclareTextGlyphY{LinBiolinum_K}{Uogonek}{370}
\DeclareTextGlyphY{LinBiolinum_K}{uni0172}{370}
\DeclareTextGlyphY{LinBiolinum_K}{uogonek}{371}
\DeclareTextGlyphY{LinBiolinum_K}{uni0173}{371}
\DeclareTextGlyphY{LinBiolinum_K}{Wcircumflex}{372}
\DeclareTextGlyphY{LinBiolinum_K}{uni0174}{372}
\DeclareTextGlyphY{LinBiolinum_K}{wcircumflex}{373}
\DeclareTextGlyphY{LinBiolinum_K}{uni0175}{373}
\DeclareTextGlyphY{LinBiolinum_K}{Ycircumflex}{374}
\DeclareTextGlyphY{LinBiolinum_K}{uni0176}{374}
\DeclareTextGlyphY{LinBiolinum_K}{ycircumflex}{375}
\DeclareTextGlyphY{LinBiolinum_K}{uni0177}{375}
\DeclareTextGlyphY{LinBiolinum_K}{Ydieresis}{376}
\DeclareTextGlyphY{LinBiolinum_K}{uni0178}{376}
\DeclareTextGlyphY{LinBiolinum_K}{Zacute}{377}
\DeclareTextGlyphY{LinBiolinum_K}{uni0179}{377}
\DeclareTextGlyphY{LinBiolinum_K}{zacute}{378}
\DeclareTextGlyphY{LinBiolinum_K}{uni017A}{378}
\DeclareTextGlyphY{LinBiolinum_K}{Zdotaccent}{379}
\DeclareTextGlyphY{LinBiolinum_K}{uni017B}{379}
\DeclareTextGlyphY{LinBiolinum_K}{zdotaccent}{380}
\DeclareTextGlyphY{LinBiolinum_K}{uni017C}{380}
\DeclareTextGlyphY{LinBiolinum_K}{Zcaron}{381}
\DeclareTextGlyphY{LinBiolinum_K}{uni017D}{381}
\DeclareTextGlyphY{LinBiolinum_K}{zcaron}{382}
\DeclareTextGlyphY{LinBiolinum_K}{uni017E}{382}
\DeclareTextGlyphY{LinBiolinum_K}{h.superior}{688}
\DeclareTextGlyphY{LinBiolinum_K}{uni02B0}{688}
\DeclareTextGlyphY{LinBiolinum_K}{hhook.superior}{689}
\DeclareTextGlyphY{LinBiolinum_K}{uni02B1}{689}
\DeclareTextGlyphY{LinBiolinum_K}{j.superior}{690}
\DeclareTextGlyphY{LinBiolinum_K}{uni02B2}{690}
\DeclareTextGlyphY{LinBiolinum_K}{r.superior}{691}
\DeclareTextGlyphY{LinBiolinum_K}{uni02B3}{691}
\DeclareTextGlyphY{LinBiolinum_K}{rturned.superior}{692}
\DeclareTextGlyphY{LinBiolinum_K}{uni02B4}{692}
\DeclareTextGlyphY{LinBiolinum_K}{rhookturned.superior}{693}
\DeclareTextGlyphY{LinBiolinum_K}{uni02B5}{693}
\DeclareTextGlyphY{LinBiolinum_K}{Rsmallinverted.superior}{694}
\DeclareTextGlyphY{LinBiolinum_K}{uni02B6}{694}
\DeclareTextGlyphY{LinBiolinum_K}{w.superior}{695}
\DeclareTextGlyphY{LinBiolinum_K}{uni02B7}{695}
\DeclareTextGlyphY{LinBiolinum_K}{y.superior}{696}
\DeclareTextGlyphY{LinBiolinum_K}{uni02B8}{696}
\DeclareTextGlyphY{LinBiolinum_K}{uni02B9}{697}
\DeclareTextGlyphY{LinBiolinum_K}{uni02BA}{698}
\DeclareTextGlyphY{LinBiolinum_K}{uni02BB}{699}
\DeclareTextGlyphY{LinBiolinum_K}{afii57929}{700}
\DeclareTextGlyphY{LinBiolinum_K}{uni02BC}{700}
\DeclareTextGlyphY{LinBiolinum_K}{afii64937}{701}
\DeclareTextGlyphY{LinBiolinum_K}{uni02BD}{701}
\DeclareTextGlyphY{LinBiolinum_K}{uni02BE}{702}
\DeclareTextGlyphY{LinBiolinum_K}{uni02BF}{703}
\DeclareTextGlyphY{LinBiolinum_K}{uni02C0}{704}
\DeclareTextGlyphY{LinBiolinum_K}{uni02C1}{705}
\DeclareTextGlyphY{LinBiolinum_K}{uni02C2}{706}
\DeclareTextGlyphY{LinBiolinum_K}{uni02C3}{707}
\DeclareTextGlyphY{LinBiolinum_K}{uni02C4}{708}
\DeclareTextGlyphY{LinBiolinum_K}{uni02C5}{709}
\DeclareTextGlyphY{LinBiolinum_K}{circumflex}{710}
\DeclareTextGlyphY{LinBiolinum_K}{uni02C6}{710}
\DeclareTextGlyphY{LinBiolinum_K}{caron}{711}
\DeclareTextGlyphY{LinBiolinum_K}{uni02C7}{711}
\DeclareTextGlyphY{LinBiolinum_K}{uni02C8}{712}
\DeclareTextGlyphY{LinBiolinum_K}{uni02C9}{713}
\DeclareTextGlyphY{LinBiolinum_K}{uni02CA}{714}
\DeclareTextGlyphY{LinBiolinum_K}{uni02CB}{715}
\DeclareTextGlyphY{LinBiolinum_K}{uni02CC}{716}
\DeclareTextGlyphY{LinBiolinum_K}{uni02CD}{717}
\DeclareTextGlyphY{LinBiolinum_K}{uni02CE}{718}
\DeclareTextGlyphY{LinBiolinum_K}{uni02CF}{719}
\DeclareTextGlyphY{LinBiolinum_K}{uni02D0}{720}
\DeclareTextGlyphY{LinBiolinum_K}{uni02D1}{721}
\DeclareTextGlyphY{LinBiolinum_K}{uni02D2}{722}
\DeclareTextGlyphY{LinBiolinum_K}{uni02D3}{723}
\DeclareTextGlyphY{LinBiolinum_K}{uni02D4}{724}
\DeclareTextGlyphY{LinBiolinum_K}{uni02D5}{725}
\DeclareTextGlyphY{LinBiolinum_K}{uni02D6}{726}
\DeclareTextGlyphY{LinBiolinum_K}{uni02D7}{727}
\DeclareTextGlyphY{LinBiolinum_K}{breve}{728}
\DeclareTextGlyphY{LinBiolinum_K}{uni02D8}{728}
\DeclareTextGlyphY{LinBiolinum_K}{dotaccent}{729}
\DeclareTextGlyphY{LinBiolinum_K}{uni02D9}{729}
\DeclareTextGlyphY{LinBiolinum_K}{ring}{730}
\DeclareTextGlyphY{LinBiolinum_K}{uni02DA}{730}
\DeclareTextGlyphY{LinBiolinum_K}{ogonek}{731}
\DeclareTextGlyphY{LinBiolinum_K}{uni02DB}{731}
\DeclareTextGlyphY{LinBiolinum_K}{tilde}{732}
\DeclareTextGlyphY{LinBiolinum_K}{uni02DC}{732}
\DeclareTextGlyphY{LinBiolinum_K}{hungarumlaut}{733}
\DeclareTextGlyphY{LinBiolinum_K}{uni02DD}{733}
\DeclareTextGlyphY{LinBiolinum_K}{uni02DE}{734}
\DeclareTextGlyphY{LinBiolinum_K}{uni02DF}{735}
\DeclareTextGlyphY{LinBiolinum_K}{gammalatin.superior}{736}
\DeclareTextGlyphY{LinBiolinum_K}{uni02E0}{736}
\DeclareTextGlyphY{LinBiolinum_K}{l.superior}{737}
\DeclareTextGlyphY{LinBiolinum_K}{uni02E1}{737}
\DeclareTextGlyphY{LinBiolinum_K}{s.superior}{738}
\DeclareTextGlyphY{LinBiolinum_K}{uni02E2}{738}
\DeclareTextGlyphY{LinBiolinum_K}{x.superior}{739}
\DeclareTextGlyphY{LinBiolinum_K}{uni02E3}{739}
\DeclareTextGlyphY{LinBiolinum_K}{glottalstopreversed.superior}{740}
\DeclareTextGlyphY{LinBiolinum_K}{uni02E4}{740}
\DeclareTextGlyphY{LinBiolinum_K}{uni02EC}{748}
\DeclareTextGlyphY{LinBiolinum_K}{uni02ED}{749}
\DeclareTextGlyphY{LinBiolinum_K}{uni02EE}{750}
\DeclareTextGlyphY{LinBiolinum_K}{gravecomb}{768}
\DeclareTextGlyphY{LinBiolinum_K}{uni0300}{768}
\DeclareTextGlyphY{LinBiolinum_K}{acutecomb}{769}
\DeclareTextGlyphY{LinBiolinum_K}{uni0301}{769}
\DeclareTextGlyphY{LinBiolinum_K}{uni0302}{770}
\DeclareTextGlyphY{LinBiolinum_K}{tildecomb}{771}
\DeclareTextGlyphY{LinBiolinum_K}{uni0303}{771}
\DeclareTextGlyphY{LinBiolinum_K}{uni0304}{772}
\DeclareTextGlyphY{LinBiolinum_K}{uni0305}{773}
\DeclareTextGlyphY{LinBiolinum_K}{uni0306}{774}
\DeclareTextGlyphY{LinBiolinum_K}{uni0307}{775}
\DeclareTextGlyphY{LinBiolinum_K}{uni0308}{776}
\DeclareTextGlyphY{LinBiolinum_K}{hookabovecomb}{777}
\DeclareTextGlyphY{LinBiolinum_K}{uni0309}{777}
\DeclareTextGlyphY{LinBiolinum_K}{uni030A}{778}
\DeclareTextGlyphY{LinBiolinum_K}{uni030B}{779}
\DeclareTextGlyphY{LinBiolinum_K}{uni030C}{780}
\DeclareTextGlyphY{LinBiolinum_K}{uni030D}{781}
\DeclareTextGlyphY{LinBiolinum_K}{uni030E}{782}
\DeclareTextGlyphY{LinBiolinum_K}{uni030F}{783}
\DeclareTextGlyphY{LinBiolinum_K}{uni0310}{784}
\DeclareTextGlyphY{LinBiolinum_K}{uni0311}{785}
\DeclareTextGlyphY{LinBiolinum_K}{uni0312}{786}
\DeclareTextGlyphY{LinBiolinum_K}{uni0313}{787}
\DeclareTextGlyphY{LinBiolinum_K}{uni0314}{788}
\DeclareTextGlyphY{LinBiolinum_K}{uni0315}{789}
\DeclareTextGlyphY{LinBiolinum_K}{uni0316}{790}
\DeclareTextGlyphY{LinBiolinum_K}{uni0317}{791}
\DeclareTextGlyphY{LinBiolinum_K}{uni0318}{792}
\DeclareTextGlyphY{LinBiolinum_K}{uni0319}{793}
\DeclareTextGlyphY{LinBiolinum_K}{uni031A}{794}
\DeclareTextGlyphY{LinBiolinum_K}{uni031B}{795}
\DeclareTextGlyphY{LinBiolinum_K}{uni031C}{796}
\DeclareTextGlyphY{LinBiolinum_K}{uni031D}{797}
\DeclareTextGlyphY{LinBiolinum_K}{uni031E}{798}
\DeclareTextGlyphY{LinBiolinum_K}{uni031F}{799}
\DeclareTextGlyphY{LinBiolinum_K}{uni0320}{800}
\DeclareTextGlyphY{LinBiolinum_K}{uni0321}{801}
\DeclareTextGlyphY{LinBiolinum_K}{uni0322}{802}
\DeclareTextGlyphY{LinBiolinum_K}{dotbelowcomb}{803}
\DeclareTextGlyphY{LinBiolinum_K}{uni0323}{803}
\DeclareTextGlyphY{LinBiolinum_K}{uni0324}{804}
\DeclareTextGlyphY{LinBiolinum_K}{uni0325}{805}
\DeclareTextGlyphY{LinBiolinum_K}{uni0326}{806}
\DeclareTextGlyphY{LinBiolinum_K}{uni0327}{807}
\DeclareTextGlyphY{LinBiolinum_K}{uni0328}{808}
\DeclareTextGlyphY{LinBiolinum_K}{uni0329}{809}
\DeclareTextGlyphY{LinBiolinum_K}{uni032A}{810}
\DeclareTextGlyphY{LinBiolinum_K}{uni032B}{811}
\DeclareTextGlyphY{LinBiolinum_K}{uni032C}{812}
\DeclareTextGlyphY{LinBiolinum_K}{uni032D}{813}
\DeclareTextGlyphY{LinBiolinum_K}{uni032E}{814}
\DeclareTextGlyphY{LinBiolinum_K}{uni032F}{815}
\DeclareTextGlyphY{LinBiolinum_K}{uni0330}{816}
\DeclareTextGlyphY{LinBiolinum_K}{uni0331}{817}
\DeclareTextGlyphY{LinBiolinum_K}{uni0338}{824}
\DeclareTextGlyphY{LinBiolinum_K}{uni0342}{834}
\DeclareTextGlyphY{LinBiolinum_K}{uni0343}{835}
\DeclareTextGlyphY{LinBiolinum_K}{uni0351}{849}
\DeclareTextGlyphY{LinBiolinum_K}{uni0357}{855}
\DeclareTextGlyphY{LinBiolinum_K}{uni0358}{856}
\DeclareTextGlyphY{LinBiolinum_K}{uni0359}{857}
\DeclareTextGlyphY{LinBiolinum_K}{uni035A}{858}
\DeclareTextGlyphY{LinBiolinum_K}{uni035B}{859}
\DeclareTextGlyphY{LinBiolinum_K}{uni035C}{860}
\DeclareTextGlyphY{LinBiolinum_K}{uni035D}{861}
\DeclareTextGlyphY{LinBiolinum_K}{uni035E}{862}
\DeclareTextGlyphY{LinBiolinum_K}{uni035F}{863}
\DeclareTextGlyphY{LinBiolinum_K}{uni0360}{864}
\DeclareTextGlyphY{LinBiolinum_K}{uni0361}{865}
\DeclareTextGlyphY{LinBiolinum_K}{uni0362}{866}
\DeclareTextGlyphY{LinBiolinum_K}{uni0363}{867}
\DeclareTextGlyphY{LinBiolinum_K}{uni0374}{884}
\DeclareTextGlyphY{LinBiolinum_K}{uni0375}{885}
\DeclareTextGlyphY{LinBiolinum_K}{afii57799}{1456}
\DeclareTextGlyphY{LinBiolinum_K}{uni05B0}{1456}
\DeclareTextGlyphY{LinBiolinum_K}{afii57801}{1457}
\DeclareTextGlyphY{LinBiolinum_K}{uni05B1}{1457}
\DeclareTextGlyphY{LinBiolinum_K}{afii57800}{1458}
\DeclareTextGlyphY{LinBiolinum_K}{uni05B2}{1458}
\DeclareTextGlyphY{LinBiolinum_K}{afii57802}{1459}
\DeclareTextGlyphY{LinBiolinum_K}{uni05B3}{1459}
\DeclareTextGlyphY{LinBiolinum_K}{afii57793}{1460}
\DeclareTextGlyphY{LinBiolinum_K}{uni05B4}{1460}
\DeclareTextGlyphY{LinBiolinum_K}{afii57794}{1461}
\DeclareTextGlyphY{LinBiolinum_K}{uni05B5}{1461}
\DeclareTextGlyphY{LinBiolinum_K}{afii57795}{1462}
\DeclareTextGlyphY{LinBiolinum_K}{uni05B6}{1462}
\DeclareTextGlyphY{LinBiolinum_K}{afii57798}{1463}
\DeclareTextGlyphY{LinBiolinum_K}{uni05B7}{1463}
\DeclareTextGlyphY{LinBiolinum_K}{afii57797}{1464}
\DeclareTextGlyphY{LinBiolinum_K}{uni05B8}{1464}
\DeclareTextGlyphY{LinBiolinum_K}{afii57806}{1465}
\DeclareTextGlyphY{LinBiolinum_K}{uni05B9}{1465}
\DeclareTextGlyphY{LinBiolinum_K}{uni05BA}{1466}
\DeclareTextGlyphY{LinBiolinum_K}{afii57796}{1467}
\DeclareTextGlyphY{LinBiolinum_K}{uni05BB}{1467}
\DeclareTextGlyphY{LinBiolinum_K}{afii57807}{1468}
\DeclareTextGlyphY{LinBiolinum_K}{uni05BC}{1468}
\DeclareTextGlyphY{LinBiolinum_K}{afii57839}{1469}
\DeclareTextGlyphY{LinBiolinum_K}{uni05BD}{1469}
\DeclareTextGlyphY{LinBiolinum_K}{afii57645}{1470}
\DeclareTextGlyphY{LinBiolinum_K}{uni05BE}{1470}
\DeclareTextGlyphY{LinBiolinum_K}{afii57841}{1471}
\DeclareTextGlyphY{LinBiolinum_K}{uni05BF}{1471}
\DeclareTextGlyphY{LinBiolinum_K}{afii57842}{1472}
\DeclareTextGlyphY{LinBiolinum_K}{uni05C0}{1472}
\DeclareTextGlyphY{LinBiolinum_K}{afii57804}{1473}
\DeclareTextGlyphY{LinBiolinum_K}{uni05C1}{1473}
\DeclareTextGlyphY{LinBiolinum_K}{afii57803}{1474}
\DeclareTextGlyphY{LinBiolinum_K}{uni05C2}{1474}
\DeclareTextGlyphY{LinBiolinum_K}{afii57658}{1475}
\DeclareTextGlyphY{LinBiolinum_K}{uni05C3}{1475}
\DeclareTextGlyphY{LinBiolinum_K}{uni05C6}{1478}
\DeclareTextGlyphY{LinBiolinum_K}{afii57664}{1488}
\DeclareTextGlyphY{LinBiolinum_K}{uni05D0}{1488}
\DeclareTextGlyphY{LinBiolinum_K}{afii57665}{1489}
\DeclareTextGlyphY{LinBiolinum_K}{uni05D1}{1489}
\DeclareTextGlyphY{LinBiolinum_K}{afii57666}{1490}
\DeclareTextGlyphY{LinBiolinum_K}{uni05D2}{1490}
\DeclareTextGlyphY{LinBiolinum_K}{afii57667}{1491}
\DeclareTextGlyphY{LinBiolinum_K}{uni05D3}{1491}
\DeclareTextGlyphY{LinBiolinum_K}{afii57668}{1492}
\DeclareTextGlyphY{LinBiolinum_K}{uni05D4}{1492}
\DeclareTextGlyphY{LinBiolinum_K}{afii57669}{1493}
\DeclareTextGlyphY{LinBiolinum_K}{uni05D5}{1493}
\DeclareTextGlyphY{LinBiolinum_K}{afii57670}{1494}
\DeclareTextGlyphY{LinBiolinum_K}{uni05D6}{1494}
\DeclareTextGlyphY{LinBiolinum_K}{afii57671}{1495}
\DeclareTextGlyphY{LinBiolinum_K}{uni05D7}{1495}
\DeclareTextGlyphY{LinBiolinum_K}{afii57672}{1496}
\DeclareTextGlyphY{LinBiolinum_K}{uni05D8}{1496}
\DeclareTextGlyphY{LinBiolinum_K}{afii57673}{1497}
\DeclareTextGlyphY{LinBiolinum_K}{uni05D9}{1497}
\DeclareTextGlyphY{LinBiolinum_K}{afii57674}{1498}
\DeclareTextGlyphY{LinBiolinum_K}{uni05DA}{1498}
\DeclareTextGlyphY{LinBiolinum_K}{afii57675}{1499}
\DeclareTextGlyphY{LinBiolinum_K}{uni05DB}{1499}
\DeclareTextGlyphY{LinBiolinum_K}{afii57676}{1500}
\DeclareTextGlyphY{LinBiolinum_K}{uni05DC}{1500}
\DeclareTextGlyphY{LinBiolinum_K}{afii57677}{1501}
\DeclareTextGlyphY{LinBiolinum_K}{uni05DD}{1501}
\DeclareTextGlyphY{LinBiolinum_K}{afii57678}{1502}
\DeclareTextGlyphY{LinBiolinum_K}{uni05DE}{1502}
\DeclareTextGlyphY{LinBiolinum_K}{afii57679}{1503}
\DeclareTextGlyphY{LinBiolinum_K}{uni05DF}{1503}
\DeclareTextGlyphY{LinBiolinum_K}{afii57680}{1504}
\DeclareTextGlyphY{LinBiolinum_K}{uni05E0}{1504}
\DeclareTextGlyphY{LinBiolinum_K}{afii57681}{1505}
\DeclareTextGlyphY{LinBiolinum_K}{uni05E1}{1505}
\DeclareTextGlyphY{LinBiolinum_K}{afii57682}{1506}
\DeclareTextGlyphY{LinBiolinum_K}{uni05E2}{1506}
\DeclareTextGlyphY{LinBiolinum_K}{afii57683}{1507}
\DeclareTextGlyphY{LinBiolinum_K}{uni05E3}{1507}
\DeclareTextGlyphY{LinBiolinum_K}{afii57684}{1508}
\DeclareTextGlyphY{LinBiolinum_K}{uni05E4}{1508}
\DeclareTextGlyphY{LinBiolinum_K}{afii57685}{1509}
\DeclareTextGlyphY{LinBiolinum_K}{uni05E5}{1509}
\DeclareTextGlyphY{LinBiolinum_K}{afii57686}{1510}
\DeclareTextGlyphY{LinBiolinum_K}{uni05E6}{1510}
\DeclareTextGlyphY{LinBiolinum_K}{afii57687}{1511}
\DeclareTextGlyphY{LinBiolinum_K}{uni05E7}{1511}
\DeclareTextGlyphY{LinBiolinum_K}{afii57688}{1512}
\DeclareTextGlyphY{LinBiolinum_K}{uni05E8}{1512}
\DeclareTextGlyphY{LinBiolinum_K}{afii57689}{1513}
\DeclareTextGlyphY{LinBiolinum_K}{uni05E9}{1513}
\DeclareTextGlyphY{LinBiolinum_K}{afii57690}{1514}
\DeclareTextGlyphY{LinBiolinum_K}{uni05EA}{1514}
\DeclareTextGlyphY{LinBiolinum_K}{afii57716}{1520}
\DeclareTextGlyphY{LinBiolinum_K}{uni05F0}{1520}
\DeclareTextGlyphY{LinBiolinum_K}{afii57717}{1521}
\DeclareTextGlyphY{LinBiolinum_K}{uni05F1}{1521}
\DeclareTextGlyphY{LinBiolinum_K}{afii57718}{1522}
\DeclareTextGlyphY{LinBiolinum_K}{uni05F2}{1522}
\DeclareTextGlyphY{LinBiolinum_K}{arrowleft}{8592}
\DeclareTextGlyphY{LinBiolinum_K}{uni2190}{8592}
\DeclareTextGlyphY{LinBiolinum_K}{arrowup}{8593}
\DeclareTextGlyphY{LinBiolinum_K}{uni2191}{8593}
\DeclareTextGlyphY{LinBiolinum_K}{arrowright}{8594}
\DeclareTextGlyphY{LinBiolinum_K}{uni2192}{8594}
\DeclareTextGlyphY{LinBiolinum_K}{arrowdown}{8595}
\DeclareTextGlyphY{LinBiolinum_K}{uni2193}{8595}
\DeclareTextGlyphY{LinBiolinum_K}{arrowboth}{8596}
\DeclareTextGlyphY{LinBiolinum_K}{uni2194}{8596}
\DeclareTextGlyphY{LinBiolinum_K}{arrowupdn}{8597}
\DeclareTextGlyphY{LinBiolinum_K}{uni2195}{8597}
\DeclareTextGlyphY{LinBiolinum_K}{uni2196}{8598}
\DeclareTextGlyphY{LinBiolinum_K}{uni2197}{8599}
\DeclareTextGlyphY{LinBiolinum_K}{uni2198}{8600}
\DeclareTextGlyphY{LinBiolinum_K}{uni2199}{8601}
\DeclareTextGlyphY{LinBiolinum_K}{uni219A}{8602}
\DeclareTextGlyphY{LinBiolinum_K}{uni219B}{8603}
\DeclareTextGlyphY{LinBiolinum_K}{uni21A5}{8613}
\DeclareTextGlyphY{LinBiolinum_K}{uni21A7}{8615}
\DeclareTextGlyphY{LinBiolinum_K}{uni21BC}{8636}
\DeclareTextGlyphY{LinBiolinum_K}{uni21BD}{8637}
\DeclareTextGlyphY{LinBiolinum_K}{uni21C0}{8640}
\DeclareTextGlyphY{LinBiolinum_K}{uni21C1}{8641}
\DeclareTextGlyphY{LinBiolinum_K}{arrowdblleft}{8656}
\DeclareTextGlyphY{LinBiolinum_K}{uni21D0}{8656}
\DeclareTextGlyphY{LinBiolinum_K}{arrowdblup}{8657}
\DeclareTextGlyphY{LinBiolinum_K}{uni21D1}{8657}
\DeclareTextGlyphY{LinBiolinum_K}{arrowdblright}{8658}
\DeclareTextGlyphY{LinBiolinum_K}{uni21D2}{8658}
\DeclareTextGlyphY{LinBiolinum_K}{arrowdbldown}{8659}
\DeclareTextGlyphY{LinBiolinum_K}{uni21D3}{8659}
\DeclareTextGlyphY{LinBiolinum_K}{arrowdblboth}{8660}
\DeclareTextGlyphY{LinBiolinum_K}{uni21D4}{8660}
\DeclareTextGlyphY{LinBiolinum_K}{uni21D5}{8661}
\DeclareTextGlyphY{LinBiolinum_K}{Nwarrow}{8662}
\DeclareTextGlyphY{LinBiolinum_K}{uni21D6}{8662}
\DeclareTextGlyphY{LinBiolinum_K}{Nearrow}{8663}
\DeclareTextGlyphY{LinBiolinum_K}{uni21D7}{8663}
\DeclareTextGlyphY{LinBiolinum_K}{Searrow}{8664}
\DeclareTextGlyphY{LinBiolinum_K}{uni21D8}{8664}
\DeclareTextGlyphY{LinBiolinum_K}{Swarrow}{8665}
\DeclareTextGlyphY{LinBiolinum_K}{uni21D9}{8665}
\DeclareTextGlyphY{LinBiolinum_K}{uni2318}{8984}
\DeclareTextGlyphY{LinBiolinum_K}{uni2325}{8997}
\DeclareTextGlyphY{LinBiolinum_K}{uni2326}{8998}
\DeclareTextGlyphY{LinBiolinum_K}{uni2327}{8999}
\DeclareTextGlyphY{LinBiolinum_K}{uni232B}{9003}
\DeclareTextGlyphY{LinBiolinum_K}{uni237D}{9085}
\DeclareTextGlyphY{LinBiolinum_K}{uni2380}{9088}
\DeclareTextGlyphY{LinBiolinum_K}{uni2423}{9251}
\DeclareTextGlyphY{LinBiolinum_K}{filledbox}{9632}
\DeclareTextGlyphY{LinBiolinum_K}{uni25A0}{9632}
\DeclareTextGlyphY{LinBiolinum_K}{H22073}{9633}
\DeclareTextGlyphY{LinBiolinum_K}{uni25A1}{9633}
\DeclareTextGlyphY{LinBiolinum_K}{triagup}{9650}
\DeclareTextGlyphY{LinBiolinum_K}{uni25B2}{9650}
\DeclareTextGlyphY{LinBiolinum_K}{uni25B3}{9651}
\DeclareTextGlyphY{LinBiolinum_K}{uni25B6}{9654}
\DeclareTextGlyphY{LinBiolinum_K}{uni25B7}{9655}
\DeclareTextGlyphY{LinBiolinum_K}{triagdn}{9660}
\DeclareTextGlyphY{LinBiolinum_K}{uni25BC}{9660}
\DeclareTextGlyphY{LinBiolinum_K}{uni25BD}{9661}
\DeclareTextGlyphY{LinBiolinum_K}{uni25C0}{9664}
\DeclareTextGlyphY{LinBiolinum_K}{uni25C1}{9665}
\DeclareTextGlyphY{LinBiolinum_K}{uni25C6}{9670}
\DeclareTextGlyphY{LinBiolinum_K}{uni25C7}{9671}
\DeclareTextGlyphY{LinBiolinum_K}{uni25C9}{9673}
\DeclareTextGlyphY{LinBiolinum_K}{lozenge}{9674}
\DeclareTextGlyphY{LinBiolinum_K}{uni25CA}{9674}
\DeclareTextGlyphY{LinBiolinum_K}{circle}{9675}
\DeclareTextGlyphY{LinBiolinum_K}{uni25CB}{9675}
\DeclareTextGlyphY{LinBiolinum_K}{uni25CE}{9678}
\DeclareTextGlyphY{LinBiolinum_K}{H18533}{9679}
\DeclareTextGlyphY{LinBiolinum_K}{uni25CF}{9679}
\DeclareTextGlyphY{LinBiolinum_K}{uni25D0}{9680}
\DeclareTextGlyphY{LinBiolinum_K}{uni25D1}{9681}
\DeclareTextGlyphY{LinBiolinum_K}{uni25D2}{9682}
\DeclareTextGlyphY{LinBiolinum_K}{uni25D3}{9683}
\DeclareTextGlyphY{LinBiolinum_K}{uni25D4}{9684}
\DeclareTextGlyphY{LinBiolinum_K}{uni25D5}{9685}
\DeclareTextGlyphY{LinBiolinum_K}{uni25D6}{9686}
\DeclareTextGlyphY{LinBiolinum_K}{uni25D7}{9687}
\DeclareTextGlyphY{LinBiolinum_K}{openbullet}{9702}
\DeclareTextGlyphY{LinBiolinum_K}{uni25E6}{9702}
\DeclareTextGlyphY{LinBiolinum_K}{uni2605}{9733}
\DeclareTextGlyphY{LinBiolinum_K}{uni2619}{9753}
\DeclareTextGlyphY{LinBiolinum_K}{uni261B}{9755}
\DeclareTextGlyphY{LinBiolinum_K}{uni261E}{9758}
\DeclareTextGlyphY{LinBiolinum_K}{uni2627}{9767}
\DeclareTextGlyphY{LinBiolinum_K}{uni262F}{9775}
\DeclareTextGlyphY{LinBiolinum_K}{uni2639}{9785}
\DeclareTextGlyphY{LinBiolinum_K}{uni263A}{9786}
\DeclareTextGlyphY{LinBiolinum_K}{uni263B}{9787}
\DeclareTextGlyphY{LinBiolinum_K}{uni263C}{9788}
\DeclareTextGlyphY{LinBiolinum_K}{uni263D}{9789}
\DeclareTextGlyphY{LinBiolinum_K}{uni263E}{9790}
\DeclareTextGlyphY{LinBiolinum_K}{uni263F}{9791}
\DeclareTextGlyphY{LinBiolinum_K}{female}{9792}
\DeclareTextGlyphY{LinBiolinum_K}{uni2640}{9792}
\DeclareTextGlyphY{LinBiolinum_K}{uni2641}{9793}
\DeclareTextGlyphY{LinBiolinum_K}{male}{9794}
\DeclareTextGlyphY{LinBiolinum_K}{uni2642}{9794}
\DeclareTextGlyphY{LinBiolinum_K}{uni2643}{9795}
\DeclareTextGlyphY{LinBiolinum_K}{uni2644}{9796}
\DeclareTextGlyphY{LinBiolinum_K}{uni2645}{9797}
\DeclareTextGlyphY{LinBiolinum_K}{uni2646}{9798}
\DeclareTextGlyphY{LinBiolinum_K}{uni2647}{9799}
\DeclareTextGlyphY{LinBiolinum_K}{uni2648}{9800}
\DeclareTextGlyphY{LinBiolinum_K}{uni2649}{9801}
\DeclareTextGlyphY{LinBiolinum_K}{uni264A}{9802}
\DeclareTextGlyphY{LinBiolinum_K}{uni264B}{9803}
\DeclareTextGlyphY{LinBiolinum_K}{uni264C}{9804}
\DeclareTextGlyphY{LinBiolinum_K}{uni264D}{9805}
\DeclareTextGlyphY{LinBiolinum_K}{uni264E}{9806}
\DeclareTextGlyphY{LinBiolinum_K}{uni264F}{9807}
\DeclareTextGlyphY{LinBiolinum_K}{uni2650}{9808}
\DeclareTextGlyphY{LinBiolinum_K}{uni2651}{9809}
\DeclareTextGlyphY{LinBiolinum_K}{uni2652}{9810}
\DeclareTextGlyphY{LinBiolinum_K}{uni2653}{9811}
\DeclareTextGlyphY{LinBiolinum_K}{uni2660}{9824}
\DeclareTextGlyphY{LinBiolinum_K}{uni2663}{9827}
\DeclareTextGlyphY{LinBiolinum_K}{uni2665}{9829}
\DeclareTextGlyphY{LinBiolinum_K}{uni2666}{9830}
\DeclareTextGlyphY{LinBiolinum_K}{uni2669}{9833}
\DeclareTextGlyphY{LinBiolinum_K}{musicalnote}{9834}
\DeclareTextGlyphY{LinBiolinum_K}{uni266A}{9834}
\DeclareTextGlyphY{LinBiolinum_K}{musicalnotedbl}{9835}
\DeclareTextGlyphY{LinBiolinum_K}{uni266B}{9835}
\DeclareTextGlyphY{LinBiolinum_K}{uni266C}{9836}
\DeclareTextGlyphY{LinBiolinum_K}{uni2695}{9877}
\DeclareTextGlyphY{LinBiolinum_K}{uni2698}{9880}
\DeclareTextGlyphY{LinBiolinum_K}{uni26A2}{9890}
\DeclareTextGlyphY{LinBiolinum_K}{uni26A3}{9891}
\DeclareTextGlyphY{LinBiolinum_K}{uni26A4}{9892}
\DeclareTextGlyphY{LinBiolinum_K}{uni26A5}{9893}
\DeclareTextGlyphY{LinBiolinum_K}{uni26AD}{9901}
\DeclareTextGlyphY{LinBiolinum_K}{uni2767}{10087}
\DeclareTextGlyphY{LinBiolinum_K}{uni2776}{10102}
\DeclareTextGlyphY{LinBiolinum_K}{uni2777}{10103}
\DeclareTextGlyphY{LinBiolinum_K}{uni2778}{10104}
\DeclareTextGlyphY{LinBiolinum_K}{uni2779}{10105}
\DeclareTextGlyphY{LinBiolinum_K}{uni277A}{10106}
\DeclareTextGlyphY{LinBiolinum_K}{uni277B}{10107}
\DeclareTextGlyphY{LinBiolinum_K}{uni277C}{10108}
\DeclareTextGlyphY{LinBiolinum_K}{uni277D}{10109}
\DeclareTextGlyphY{LinBiolinum_K}{uni277E}{10110}
\DeclareTextGlyphY{LinBiolinum_K}{uni277F}{10111}
\DeclareTextGlyphY{LinBiolinum_K}{T_u_x}{57344}
\DeclareTextGlyphY{LinBiolinum_K}{uniE000}{57344}
\DeclareTextGlyphY{LinBiolinum_K}{uniE104}{57604}
\DeclareTextGlyphY{LinBiolinum_K}{uniE128}{57640}
\DeclareTextGlyphY{LinBiolinum_K}{uniE129}{57641}
\DeclareTextGlyphY{LinBiolinum_K}{uniE12A}{57642}
\DeclareTextGlyphY{LinBiolinum_K}{uniE130}{57648}
\DeclareTextGlyphY{LinBiolinum_K}{uniE131}{57649}
\DeclareTextGlyphY{LinBiolinum_K}{uniE132}{57650}
\DeclareTextGlyphY{LinBiolinum_K}{uniE133}{57651}
\DeclareTextGlyphY{LinBiolinum_K}{uniE134}{57652}
\DeclareTextGlyphY{LinBiolinum_K}{uniE135}{57653}
\DeclareTextGlyphY{LinBiolinum_K}{uniE138}{57656}
\DeclareTextGlyphY{LinBiolinum_K}{uniE139}{57657}
\DeclareTextGlyphY{LinBiolinum_K}{uniE13A}{57658}
\DeclareTextGlyphY{LinBiolinum_K}{uniE13C}{57660}
\DeclareTextGlyphY{LinBiolinum_K}{uniE13D}{57661}
\DeclareTextGlyphY{LinBiolinum_K}{uniE168}{57704}
\DeclareTextGlyphY{LinBiolinum_K}{B_a_c_k}{57710}
\DeclareTextGlyphY{LinBiolinum_K}{uniE16E}{57710}
\DeclareTextGlyphY{LinBiolinum_K}{S_t_r_g}{57712}
\DeclareTextGlyphY{LinBiolinum_K}{uniE170}{57712}
\DeclareTextGlyphY{LinBiolinum_K}{A_l_t}{57713}
\DeclareTextGlyphY{LinBiolinum_K}{uniE171}{57713}
\DeclareTextGlyphY{LinBiolinum_K}{A_l_t_G_r}{57714}
\DeclareTextGlyphY{LinBiolinum_K}{uniE172}{57714}
\DeclareTextGlyphY{LinBiolinum_K}{C_t_r_l}{57715}
\DeclareTextGlyphY{LinBiolinum_K}{uniE173}{57715}
\DeclareTextGlyphY{LinBiolinum_K}{S_h_i_f_t}{57716}
\DeclareTextGlyphY{LinBiolinum_K}{uniE174}{57716}
\DeclareTextGlyphY{LinBiolinum_K}{T_a_b}{57717}
\DeclareTextGlyphY{LinBiolinum_K}{uniE175}{57717}
\DeclareTextGlyphY{LinBiolinum_K}{E_n_t_e_r}{57718}
\DeclareTextGlyphY{LinBiolinum_K}{uniE176}{57718}
\DeclareTextGlyphY{LinBiolinum_K}{C_a_p_s_l_o_c_k}{57719}
\DeclareTextGlyphY{LinBiolinum_K}{uniE177}{57719}
\DeclareTextGlyphY{LinBiolinum_K}{F_1}{57720}
\DeclareTextGlyphY{LinBiolinum_K}{uniE178}{57720}
\DeclareTextGlyphY{LinBiolinum_K}{F_2}{57721}
\DeclareTextGlyphY{LinBiolinum_K}{uniE179}{57721}
\DeclareTextGlyphY{LinBiolinum_K}{F_3}{57722}
\DeclareTextGlyphY{LinBiolinum_K}{uniE17A}{57722}
\DeclareTextGlyphY{LinBiolinum_K}{F_4}{57723}
\DeclareTextGlyphY{LinBiolinum_K}{uniE17B}{57723}
\DeclareTextGlyphY{LinBiolinum_K}{F_5}{57724}
\DeclareTextGlyphY{LinBiolinum_K}{uniE17C}{57724}
\DeclareTextGlyphY{LinBiolinum_K}{F_6}{57725}
\DeclareTextGlyphY{LinBiolinum_K}{uniE17D}{57725}
\DeclareTextGlyphY{LinBiolinum_K}{F_7}{57726}
\DeclareTextGlyphY{LinBiolinum_K}{uniE17E}{57726}
\DeclareTextGlyphY{LinBiolinum_K}{F_8}{57727}
\DeclareTextGlyphY{LinBiolinum_K}{uniE17F}{57727}
\DeclareTextGlyphY{LinBiolinum_K}{F_9}{57728}
\DeclareTextGlyphY{LinBiolinum_K}{uniE180}{57728}
\DeclareTextGlyphY{LinBiolinum_K}{F_1_0}{57729}
\DeclareTextGlyphY{LinBiolinum_K}{uniE181}{57729}
\DeclareTextGlyphY{LinBiolinum_K}{F_1_1}{57730}
\DeclareTextGlyphY{LinBiolinum_K}{uniE182}{57730}
\DeclareTextGlyphY{LinBiolinum_K}{F_1_2}{57731}
\DeclareTextGlyphY{LinBiolinum_K}{uniE183}{57731}
\DeclareTextGlyphY{LinBiolinum_K}{F_1_3}{57732}
\DeclareTextGlyphY{LinBiolinum_K}{uniE184}{57732}
\DeclareTextGlyphY{LinBiolinum_K}{F_1_4}{57733}
\DeclareTextGlyphY{LinBiolinum_K}{uniE185}{57733}
\DeclareTextGlyphY{LinBiolinum_K}{F_1_5}{57734}
\DeclareTextGlyphY{LinBiolinum_K}{uniE186}{57734}
\DeclareTextGlyphY{LinBiolinum_K}{F_1_6}{57735}
\DeclareTextGlyphY{LinBiolinum_K}{uniE187}{57735}
\DeclareTextGlyphY{LinBiolinum_K}{uniE188}{57736}
\DeclareTextGlyphY{LinBiolinum_K}{H_o_m_e}{57737}
\DeclareTextGlyphY{LinBiolinum_K}{uniE189}{57737}
\DeclareTextGlyphY{LinBiolinum_K}{D_e_l}{57738}
\DeclareTextGlyphY{LinBiolinum_K}{uniE18A}{57738}
\DeclareTextGlyphY{LinBiolinum_K}{I_n_s}{57739}
\DeclareTextGlyphY{LinBiolinum_K}{uniE18B}{57739}
\DeclareTextGlyphY{LinBiolinum_K}{uniE18C}{57740}
\DeclareTextGlyphY{LinBiolinum_K}{E_n_d}{57742}
\DeclareTextGlyphY{LinBiolinum_K}{uniE18E}{57742}
\DeclareTextGlyphY{LinBiolinum_K}{G_N_U}{57744}
\DeclareTextGlyphY{LinBiolinum_K}{uniE190}{57744}
\DeclareTextGlyphY{LinBiolinum_K}{P_o_s_1}{57745}
\DeclareTextGlyphY{LinBiolinum_K}{uniE191}{57745}
\DeclareTextGlyphY{LinBiolinum_K}{E_n_t_f}{57746}
\DeclareTextGlyphY{LinBiolinum_K}{uniE192}{57746}
\DeclareTextGlyphY{LinBiolinum_K}{E_i_n_f}{57747}
\DeclareTextGlyphY{LinBiolinum_K}{uniE193}{57747}
\DeclareTextGlyphY{LinBiolinum_K}{L_e_e_r}{57748}
\DeclareTextGlyphY{LinBiolinum_K}{uniE194}{57748}
\DeclareTextGlyphY{LinBiolinum_K}{E_s_c}{57749}
\DeclareTextGlyphY{LinBiolinum_K}{uniE195}{57749}
\DeclareTextGlyphY{LinBiolinum_K}{E_n_d_e}{57750}
\DeclareTextGlyphY{LinBiolinum_K}{uniE196}{57750}
\DeclareTextGlyphY{LinBiolinum_K}{uniE198}{57752}
\DeclareTextGlyphY{LinBiolinum_K}{uniE199}{57753}
\DeclareTextGlyphY{LinBiolinum_K}{uniE19A}{57754}
\DeclareTextGlyphY{LinBiolinum_K}{uniE19B}{57755}
\DeclareTextGlyphY{LinBiolinum_K}{uniE1A0}{57760}
\DeclareTextGlyphY{LinBiolinum_K}{uniE1A1}{57761}
\DeclareTextGlyphY{LinBiolinum_K}{uniE1A2}{57762}
\DeclareTextGlyphY{LinBiolinum_K}{uniE1A3}{57763}
\DeclareTextGlyphY{LinBiolinum_K}{uniE1A4}{57764}
\DeclareTextGlyphY{LinBiolinum_K}{uniE1A5}{57765}
\DeclareTextGlyphY{LinBiolinum_K}{uniE1A6}{57766}
\DeclareTextGlyphY{LinBiolinum_K}{uniE1A7}{57767}
\DeclareTextGlyphY{LinBiolinum_K}{uniE1A8}{57768}
\DeclareTextGlyphY{LinBiolinum_K}{uniE1A9}{57769}
\DeclareTextGlyphY{LinBiolinum_K}{uniE1AA}{57770}
\DeclareTextGlyphY{LinBiolinum_K}{uniE1AB}{57771}
\DeclareTextGlyphY{LinBiolinum_K}{uniE1AC}{57772}
\DeclareTextGlyphY{LinBiolinum_K}{uniE1AD}{57773}
\DeclareTextGlyphY{LinBiolinum_K}{uniE1AE}{57774}
\DeclareTextGlyphY{LinBiolinum_K}{uniE1B0}{57776}
\DeclareTextGlyphY{LinBiolinum_K}{uniE1B1}{57777}
\DeclareTextGlyphY{LinBiolinum_K}{grave.cap}{58200}
\DeclareTextGlyphY{LinBiolinum_K}{uniE358}{58200}
\DeclareTextGlyphY{LinBiolinum_K}{acute.cap}{58201}
\DeclareTextGlyphY{LinBiolinum_K}{uniE359}{58201}
\DeclareTextGlyphY{LinBiolinum_K}{circumflex.cap}{58202}
\DeclareTextGlyphY{LinBiolinum_K}{uniE35A}{58202}
\DeclareTextGlyphY{LinBiolinum_K}{caron.cap}{58203}
\DeclareTextGlyphY{LinBiolinum_K}{uniE35B}{58203}
\DeclareTextGlyphY{LinBiolinum_K}{breve.cap}{58204}
\DeclareTextGlyphY{LinBiolinum_K}{uniE35C}{58204}
\DeclareTextGlyphY{LinBiolinum_K}{hungarumlaut.cap}{58205}
\DeclareTextGlyphY{LinBiolinum_K}{uniE35D}{58205}
\DeclareTextGlyphY{LinBiolinum_K}{space_uni030F.cap}{58206}
\DeclareTextGlyphY{LinBiolinum_K}{uniE35E}{58206}
\DeclareTextGlyphY{LinBiolinum_K}{breveinvertedcmb.cap}{58207}
\DeclareTextGlyphY{LinBiolinum_K}{uniE35F}{58207}
\DeclareTextGlyphY{LinBiolinum_K}{breve.cyrcap}{58208}
\DeclareTextGlyphY{LinBiolinum_K}{uniE360}{58208}
\DeclareTextGlyphY{LinBiolinum_K}{breve.cyr}{58209}
\DeclareTextGlyphY{LinBiolinum_K}{uniE361}{58209}
\DeclareTextGlyphY{LinBiolinum_K}{dieresis.cap}{58210}
\DeclareTextGlyphY{LinBiolinum_K}{uniE362}{58210}
\DeclareTextGlyphY{LinBiolinum_K}{hookabovecomb.cap}{58211}
\DeclareTextGlyphY{LinBiolinum_K}{uniE363}{58211}
\DeclareTextGlyphY{LinBiolinum_K}{uniFFFD}{65533}

\end{multicols}


\clearpage
\section{Selected Libertine Initials}\tt
\label{InitialGlyphs}
\renewcommand\DeclareTextGlyphY[3]{\makebox[2.0cm]{\LARGE\strut\fbox{\libertineInitialGlyph{#2}}} #2\\}%
\catcode`\_=12%
\begin{multicols}{4}
\par\noindent
\DeclareTextGlyphY{LinLibertine_I}{zero}{48}
\DeclareTextGlyphY{LinLibertine_I}{one}{49}
\DeclareTextGlyphY{LinLibertine_I}{two}{50}
\DeclareTextGlyphY{LinLibertine_I}{three}{51}
\DeclareTextGlyphY{LinLibertine_I}{four}{52}
\DeclareTextGlyphY{LinLibertine_I}{five}{53}
\DeclareTextGlyphY{LinLibertine_I}{six}{54}
\DeclareTextGlyphY{LinLibertine_I}{seven}{55}
\DeclareTextGlyphY{LinLibertine_I}{eight}{56}
\DeclareTextGlyphY{LinLibertine_I}{nine}{57}
\DeclareTextGlyphY{LinLibertine_I}{A}{65}
\DeclareTextGlyphY{LinLibertine_I}{B}{66}
\DeclareTextGlyphY{LinLibertine_I}{C}{67}
\DeclareTextGlyphY{LinLibertine_I}{D}{68}
\DeclareTextGlyphY{LinLibertine_I}{E}{69}
\DeclareTextGlyphY{LinLibertine_I}{F}{70}
\DeclareTextGlyphY{LinLibertine_I}{G}{71}
\DeclareTextGlyphY{LinLibertine_I}{H}{72}
\DeclareTextGlyphY{LinLibertine_I}{I}{73}
\DeclareTextGlyphY{LinLibertine_I}{J}{74}
\DeclareTextGlyphY{LinLibertine_I}{K}{75}
\DeclareTextGlyphY{LinLibertine_I}{L}{76}
\DeclareTextGlyphY{LinLibertine_I}{M}{77}
\DeclareTextGlyphY{LinLibertine_I}{N}{78}
\DeclareTextGlyphY{LinLibertine_I}{O}{79}
\DeclareTextGlyphY{LinLibertine_I}{P}{80}
\DeclareTextGlyphY{LinLibertine_I}{Q}{81}
\DeclareTextGlyphY{LinLibertine_I}{R}{82}
\DeclareTextGlyphY{LinLibertine_I}{S}{83}
\DeclareTextGlyphY{LinLibertine_I}{T}{84}
\DeclareTextGlyphY{LinLibertine_I}{U}{85}
\DeclareTextGlyphY{LinLibertine_I}{V}{86}
\DeclareTextGlyphY{LinLibertine_I}{W}{87}
\DeclareTextGlyphY{LinLibertine_I}{X}{88}
\DeclareTextGlyphY{LinLibertine_I}{Y}{89}
\DeclareTextGlyphY{LinLibertine_I}{Z}{90}
\end{multicols}


% To generate a table of all the glyphs, uncomment the following lines:
%
%\begin{multicols}{2}
%\par\noindent
%\DeclareTextGlyphY{LinLibertine_I}{space}{32}
\DeclareTextGlyphY{LinLibertine_I}{uni0020}{32}
\DeclareTextGlyphY{LinLibertine_I}{zero}{48}
\DeclareTextGlyphY{LinLibertine_I}{uni0030}{48}
\DeclareTextGlyphY{LinLibertine_I}{one}{49}
\DeclareTextGlyphY{LinLibertine_I}{uni0031}{49}
\DeclareTextGlyphY{LinLibertine_I}{two}{50}
\DeclareTextGlyphY{LinLibertine_I}{uni0032}{50}
\DeclareTextGlyphY{LinLibertine_I}{three}{51}
\DeclareTextGlyphY{LinLibertine_I}{uni0033}{51}
\DeclareTextGlyphY{LinLibertine_I}{four}{52}
\DeclareTextGlyphY{LinLibertine_I}{uni0034}{52}
\DeclareTextGlyphY{LinLibertine_I}{five}{53}
\DeclareTextGlyphY{LinLibertine_I}{uni0035}{53}
\DeclareTextGlyphY{LinLibertine_I}{six}{54}
\DeclareTextGlyphY{LinLibertine_I}{uni0036}{54}
\DeclareTextGlyphY{LinLibertine_I}{seven}{55}
\DeclareTextGlyphY{LinLibertine_I}{uni0037}{55}
\DeclareTextGlyphY{LinLibertine_I}{eight}{56}
\DeclareTextGlyphY{LinLibertine_I}{uni0038}{56}
\DeclareTextGlyphY{LinLibertine_I}{nine}{57}
\DeclareTextGlyphY{LinLibertine_I}{uni0039}{57}
\DeclareTextGlyphY{LinLibertine_I}{A}{65}
\DeclareTextGlyphY{LinLibertine_I}{uni0041}{65}
\DeclareTextGlyphY{LinLibertine_I}{B}{66}
\DeclareTextGlyphY{LinLibertine_I}{uni0042}{66}
\DeclareTextGlyphY{LinLibertine_I}{C}{67}
\DeclareTextGlyphY{LinLibertine_I}{uni0043}{67}
\DeclareTextGlyphY{LinLibertine_I}{D}{68}
\DeclareTextGlyphY{LinLibertine_I}{uni0044}{68}
\DeclareTextGlyphY{LinLibertine_I}{E}{69}
\DeclareTextGlyphY{LinLibertine_I}{uni0045}{69}
\DeclareTextGlyphY{LinLibertine_I}{F}{70}
\DeclareTextGlyphY{LinLibertine_I}{uni0046}{70}
\DeclareTextGlyphY{LinLibertine_I}{G}{71}
\DeclareTextGlyphY{LinLibertine_I}{uni0047}{71}
\DeclareTextGlyphY{LinLibertine_I}{H}{72}
\DeclareTextGlyphY{LinLibertine_I}{uni0048}{72}
\DeclareTextGlyphY{LinLibertine_I}{I}{73}
\DeclareTextGlyphY{LinLibertine_I}{uni0049}{73}
\DeclareTextGlyphY{LinLibertine_I}{J}{74}
\DeclareTextGlyphY{LinLibertine_I}{uni004A}{74}
\DeclareTextGlyphY{LinLibertine_I}{K}{75}
\DeclareTextGlyphY{LinLibertine_I}{uni004B}{75}
\DeclareTextGlyphY{LinLibertine_I}{L}{76}
\DeclareTextGlyphY{LinLibertine_I}{uni004C}{76}
\DeclareTextGlyphY{LinLibertine_I}{M}{77}
\DeclareTextGlyphY{LinLibertine_I}{uni004D}{77}
\DeclareTextGlyphY{LinLibertine_I}{N}{78}
\DeclareTextGlyphY{LinLibertine_I}{uni004E}{78}
\DeclareTextGlyphY{LinLibertine_I}{O}{79}
\DeclareTextGlyphY{LinLibertine_I}{uni004F}{79}
\DeclareTextGlyphY{LinLibertine_I}{P}{80}
\DeclareTextGlyphY{LinLibertine_I}{uni0050}{80}
\DeclareTextGlyphY{LinLibertine_I}{Q}{81}
\DeclareTextGlyphY{LinLibertine_I}{uni0051}{81}
\DeclareTextGlyphY{LinLibertine_I}{R}{82}
\DeclareTextGlyphY{LinLibertine_I}{uni0052}{82}
\DeclareTextGlyphY{LinLibertine_I}{S}{83}
\DeclareTextGlyphY{LinLibertine_I}{uni0053}{83}
\DeclareTextGlyphY{LinLibertine_I}{T}{84}
\DeclareTextGlyphY{LinLibertine_I}{uni0054}{84}
\DeclareTextGlyphY{LinLibertine_I}{U}{85}
\DeclareTextGlyphY{LinLibertine_I}{uni0055}{85}
\DeclareTextGlyphY{LinLibertine_I}{V}{86}
\DeclareTextGlyphY{LinLibertine_I}{uni0056}{86}
\DeclareTextGlyphY{LinLibertine_I}{W}{87}
\DeclareTextGlyphY{LinLibertine_I}{uni0057}{87}
\DeclareTextGlyphY{LinLibertine_I}{X}{88}
\DeclareTextGlyphY{LinLibertine_I}{uni0058}{88}
\DeclareTextGlyphY{LinLibertine_I}{Y}{89}
\DeclareTextGlyphY{LinLibertine_I}{uni0059}{89}
\DeclareTextGlyphY{LinLibertine_I}{Z}{90}
\DeclareTextGlyphY{LinLibertine_I}{uni005A}{90}
\DeclareTextGlyphY{LinLibertine_I}{asciicircum}{94}
\DeclareTextGlyphY{LinLibertine_I}{uni005E}{94}
\DeclareTextGlyphY{LinLibertine_I}{Adieresis}{196}
\DeclareTextGlyphY{LinLibertine_I}{uni00C4}{196}
\DeclareTextGlyphY{LinLibertine_I}{Aring}{197}
\DeclareTextGlyphY{LinLibertine_I}{uni00C5}{197}
\DeclareTextGlyphY{LinLibertine_I}{AE}{198}
\DeclareTextGlyphY{LinLibertine_I}{uni00C6}{198}
\DeclareTextGlyphY{LinLibertine_I}{Ccedilla}{199}
\DeclareTextGlyphY{LinLibertine_I}{uni00C7}{199}
\DeclareTextGlyphY{LinLibertine_I}{Odieresis}{214}
\DeclareTextGlyphY{LinLibertine_I}{uni00D6}{214}
\DeclareTextGlyphY{LinLibertine_I}{Oslash}{216}
\DeclareTextGlyphY{LinLibertine_I}{uni00D8}{216}
\DeclareTextGlyphY{LinLibertine_I}{Udieresis}{220}
\DeclareTextGlyphY{LinLibertine_I}{uni00DC}{220}
\DeclareTextGlyphY{LinLibertine_I}{Thorn}{222}
\DeclareTextGlyphY{LinLibertine_I}{uni00DE}{222}
\DeclareTextGlyphY{LinLibertine_I}{iogonek}{303}
\DeclareTextGlyphY{LinLibertine_I}{uni012F}{303}
\DeclareTextGlyphY{LinLibertine_I}{IJ}{306}
\DeclareTextGlyphY{LinLibertine_I}{uni0132}{306}
\DeclareTextGlyphY{LinLibertine_I}{Eng}{330}
\DeclareTextGlyphY{LinLibertine_I}{uni014A}{330}
\DeclareTextGlyphY{LinLibertine_I}{Ohungarumlaut}{336}
\DeclareTextGlyphY{LinLibertine_I}{uni0150}{336}
\DeclareTextGlyphY{LinLibertine_I}{OE}{338}
\DeclareTextGlyphY{LinLibertine_I}{uni0152}{338}
\DeclareTextGlyphY{LinLibertine_I}{Uhungarumlaut}{368}
\DeclareTextGlyphY{LinLibertine_I}{uni0170}{368}
\DeclareTextGlyphY{LinLibertine_I}{Alpha}{913}
\DeclareTextGlyphY{LinLibertine_I}{uni0391}{913}
\DeclareTextGlyphY{LinLibertine_I}{Beta}{914}
\DeclareTextGlyphY{LinLibertine_I}{uni0392}{914}
\DeclareTextGlyphY{LinLibertine_I}{Gamma}{915}
\DeclareTextGlyphY{LinLibertine_I}{uni0393}{915}
\DeclareTextGlyphY{LinLibertine_I}{Delta}{916}
\DeclareTextGlyphY{LinLibertine_I}{uni0394}{916}
\DeclareTextGlyphY{LinLibertine_I}{Epsilon}{917}
\DeclareTextGlyphY{LinLibertine_I}{uni0395}{917}
\DeclareTextGlyphY{LinLibertine_I}{Zeta}{918}
\DeclareTextGlyphY{LinLibertine_I}{uni0396}{918}
\DeclareTextGlyphY{LinLibertine_I}{Eta}{919}
\DeclareTextGlyphY{LinLibertine_I}{uni0397}{919}
\DeclareTextGlyphY{LinLibertine_I}{Theta}{920}
\DeclareTextGlyphY{LinLibertine_I}{uni0398}{920}
\DeclareTextGlyphY{LinLibertine_I}{Iota}{921}
\DeclareTextGlyphY{LinLibertine_I}{uni0399}{921}
\DeclareTextGlyphY{LinLibertine_I}{Kappa}{922}
\DeclareTextGlyphY{LinLibertine_I}{uni039A}{922}
\DeclareTextGlyphY{LinLibertine_I}{Lambda}{923}
\DeclareTextGlyphY{LinLibertine_I}{uni039B}{923}
\DeclareTextGlyphY{LinLibertine_I}{Mu}{924}
\DeclareTextGlyphY{LinLibertine_I}{uni039C}{924}
\DeclareTextGlyphY{LinLibertine_I}{Nu}{925}
\DeclareTextGlyphY{LinLibertine_I}{uni039D}{925}
\DeclareTextGlyphY{LinLibertine_I}{Xi}{926}
\DeclareTextGlyphY{LinLibertine_I}{uni039E}{926}
\DeclareTextGlyphY{LinLibertine_I}{Omicron}{927}
\DeclareTextGlyphY{LinLibertine_I}{uni039F}{927}
\DeclareTextGlyphY{LinLibertine_I}{Pi}{928}
\DeclareTextGlyphY{LinLibertine_I}{uni03A0}{928}
\DeclareTextGlyphY{LinLibertine_I}{Rho}{929}
\DeclareTextGlyphY{LinLibertine_I}{uni03A1}{929}
\DeclareTextGlyphY{LinLibertine_I}{Sigma}{931}
\DeclareTextGlyphY{LinLibertine_I}{uni03A3}{931}
\DeclareTextGlyphY{LinLibertine_I}{Tau}{932}
\DeclareTextGlyphY{LinLibertine_I}{uni03A4}{932}
\DeclareTextGlyphY{LinLibertine_I}{Upsilon}{933}
\DeclareTextGlyphY{LinLibertine_I}{uni03A5}{933}
\DeclareTextGlyphY{LinLibertine_I}{Phi}{934}
\DeclareTextGlyphY{LinLibertine_I}{uni03A6}{934}
\DeclareTextGlyphY{LinLibertine_I}{Chi}{935}
\DeclareTextGlyphY{LinLibertine_I}{uni03A7}{935}
\DeclareTextGlyphY{LinLibertine_I}{Psi}{936}
\DeclareTextGlyphY{LinLibertine_I}{uni03A8}{936}
\DeclareTextGlyphY{LinLibertine_I}{Omega}{937}
\DeclareTextGlyphY{LinLibertine_I}{uni03A9}{937}
\DeclareTextGlyphY{LinLibertine_I}{Iotadieresis}{938}
\DeclareTextGlyphY{LinLibertine_I}{uni03AA}{938}
\DeclareTextGlyphY{LinLibertine_I}{Upsilon1}{978}
\DeclareTextGlyphY{LinLibertine_I}{uni03D2}{978}
\DeclareTextGlyphY{LinLibertine_I}{afii10051}{1026}
\DeclareTextGlyphY{LinLibertine_I}{uni0402}{1026}
\DeclareTextGlyphY{LinLibertine_I}{afii10053}{1028}
\DeclareTextGlyphY{LinLibertine_I}{uni0404}{1028}
\DeclareTextGlyphY{LinLibertine_I}{afii10054}{1029}
\DeclareTextGlyphY{LinLibertine_I}{uni0405}{1029}
\DeclareTextGlyphY{LinLibertine_I}{afii10055}{1030}
\DeclareTextGlyphY{LinLibertine_I}{uni0406}{1030}
\DeclareTextGlyphY{LinLibertine_I}{afii10056}{1031}
\DeclareTextGlyphY{LinLibertine_I}{uni0407}{1031}
\DeclareTextGlyphY{LinLibertine_I}{afii10057}{1032}
\DeclareTextGlyphY{LinLibertine_I}{uni0408}{1032}
\DeclareTextGlyphY{LinLibertine_I}{afii10058}{1033}
\DeclareTextGlyphY{LinLibertine_I}{uni0409}{1033}
\DeclareTextGlyphY{LinLibertine_I}{afii10059}{1034}
\DeclareTextGlyphY{LinLibertine_I}{uni040A}{1034}
\DeclareTextGlyphY{LinLibertine_I}{afii10060}{1035}
\DeclareTextGlyphY{LinLibertine_I}{uni040B}{1035}
\DeclareTextGlyphY{LinLibertine_I}{afii10061}{1036}
\DeclareTextGlyphY{LinLibertine_I}{uni040C}{1036}
\DeclareTextGlyphY{LinLibertine_I}{uni040D}{1037}
\DeclareTextGlyphY{LinLibertine_I}{afii10062}{1038}
\DeclareTextGlyphY{LinLibertine_I}{uni040E}{1038}
\DeclareTextGlyphY{LinLibertine_I}{afii10145}{1039}
\DeclareTextGlyphY{LinLibertine_I}{uni040F}{1039}
\DeclareTextGlyphY{LinLibertine_I}{afii10017}{1040}
\DeclareTextGlyphY{LinLibertine_I}{uni0410}{1040}
\DeclareTextGlyphY{LinLibertine_I}{afii10018}{1041}
\DeclareTextGlyphY{LinLibertine_I}{uni0411}{1041}
\DeclareTextGlyphY{LinLibertine_I}{afii10019}{1042}
\DeclareTextGlyphY{LinLibertine_I}{uni0412}{1042}
\DeclareTextGlyphY{LinLibertine_I}{afii10020}{1043}
\DeclareTextGlyphY{LinLibertine_I}{uni0413}{1043}
\DeclareTextGlyphY{LinLibertine_I}{afii10021}{1044}
\DeclareTextGlyphY{LinLibertine_I}{uni0414}{1044}
\DeclareTextGlyphY{LinLibertine_I}{afii10022}{1045}
\DeclareTextGlyphY{LinLibertine_I}{uni0415}{1045}
\DeclareTextGlyphY{LinLibertine_I}{afii10024}{1046}
\DeclareTextGlyphY{LinLibertine_I}{uni0416}{1046}
\DeclareTextGlyphY{LinLibertine_I}{afii10025}{1047}
\DeclareTextGlyphY{LinLibertine_I}{uni0417}{1047}
\DeclareTextGlyphY{LinLibertine_I}{afii10026}{1048}
\DeclareTextGlyphY{LinLibertine_I}{uni0418}{1048}
\DeclareTextGlyphY{LinLibertine_I}{afii10027}{1049}
\DeclareTextGlyphY{LinLibertine_I}{uni0419}{1049}
\DeclareTextGlyphY{LinLibertine_I}{afii10028}{1050}
\DeclareTextGlyphY{LinLibertine_I}{uni041A}{1050}
\DeclareTextGlyphY{LinLibertine_I}{afii10029}{1051}
\DeclareTextGlyphY{LinLibertine_I}{uni041B}{1051}
\DeclareTextGlyphY{LinLibertine_I}{afii10030}{1052}
\DeclareTextGlyphY{LinLibertine_I}{uni041C}{1052}
\DeclareTextGlyphY{LinLibertine_I}{afii10031}{1053}
\DeclareTextGlyphY{LinLibertine_I}{uni041D}{1053}
\DeclareTextGlyphY{LinLibertine_I}{afii10032}{1054}
\DeclareTextGlyphY{LinLibertine_I}{uni041E}{1054}
\DeclareTextGlyphY{LinLibertine_I}{afii10033}{1055}
\DeclareTextGlyphY{LinLibertine_I}{uni041F}{1055}
\DeclareTextGlyphY{LinLibertine_I}{afii10034}{1056}
\DeclareTextGlyphY{LinLibertine_I}{uni0420}{1056}
\DeclareTextGlyphY{LinLibertine_I}{afii10035}{1057}
\DeclareTextGlyphY{LinLibertine_I}{uni0421}{1057}
\DeclareTextGlyphY{LinLibertine_I}{afii10036}{1058}
\DeclareTextGlyphY{LinLibertine_I}{uni0422}{1058}
\DeclareTextGlyphY{LinLibertine_I}{afii10037}{1059}
\DeclareTextGlyphY{LinLibertine_I}{uni0423}{1059}
\DeclareTextGlyphY{LinLibertine_I}{afii10038}{1060}
\DeclareTextGlyphY{LinLibertine_I}{uni0424}{1060}
\DeclareTextGlyphY{LinLibertine_I}{afii10039}{1061}
\DeclareTextGlyphY{LinLibertine_I}{uni0425}{1061}
\DeclareTextGlyphY{LinLibertine_I}{afii10040}{1062}
\DeclareTextGlyphY{LinLibertine_I}{uni0426}{1062}
\DeclareTextGlyphY{LinLibertine_I}{afii10041}{1063}
\DeclareTextGlyphY{LinLibertine_I}{uni0427}{1063}
\DeclareTextGlyphY{LinLibertine_I}{afii10042}{1064}
\DeclareTextGlyphY{LinLibertine_I}{uni0428}{1064}
\DeclareTextGlyphY{LinLibertine_I}{afii10043}{1065}
\DeclareTextGlyphY{LinLibertine_I}{uni0429}{1065}
\DeclareTextGlyphY{LinLibertine_I}{afii10044}{1066}
\DeclareTextGlyphY{LinLibertine_I}{uni042A}{1066}
\DeclareTextGlyphY{LinLibertine_I}{afii10045}{1067}
\DeclareTextGlyphY{LinLibertine_I}{uni042B}{1067}
\DeclareTextGlyphY{LinLibertine_I}{afii10046}{1068}
\DeclareTextGlyphY{LinLibertine_I}{uni042C}{1068}
\DeclareTextGlyphY{LinLibertine_I}{afii10047}{1069}
\DeclareTextGlyphY{LinLibertine_I}{uni042D}{1069}
\DeclareTextGlyphY{LinLibertine_I}{afii10048}{1070}
\DeclareTextGlyphY{LinLibertine_I}{uni042E}{1070}
\DeclareTextGlyphY{LinLibertine_I}{afii10049}{1071}
\DeclareTextGlyphY{LinLibertine_I}{uni042F}{1071}
\DeclareTextGlyphY{LinLibertine_I}{uni05C6}{1478}
\DeclareTextGlyphY{LinLibertine_I}{afii57664}{1488}
\DeclareTextGlyphY{LinLibertine_I}{uni05D0}{1488}
\DeclareTextGlyphY{LinLibertine_I}{afii57665}{1489}
\DeclareTextGlyphY{LinLibertine_I}{uni05D1}{1489}
\DeclareTextGlyphY{LinLibertine_I}{afii57666}{1490}
\DeclareTextGlyphY{LinLibertine_I}{uni05D2}{1490}
\DeclareTextGlyphY{LinLibertine_I}{afii57667}{1491}
\DeclareTextGlyphY{LinLibertine_I}{uni05D3}{1491}
\DeclareTextGlyphY{LinLibertine_I}{afii57668}{1492}
\DeclareTextGlyphY{LinLibertine_I}{uni05D4}{1492}
\DeclareTextGlyphY{LinLibertine_I}{afii57669}{1493}
\DeclareTextGlyphY{LinLibertine_I}{uni05D5}{1493}
\DeclareTextGlyphY{LinLibertine_I}{afii57670}{1494}
\DeclareTextGlyphY{LinLibertine_I}{uni05D6}{1494}
\DeclareTextGlyphY{LinLibertine_I}{afii57671}{1495}
\DeclareTextGlyphY{LinLibertine_I}{uni05D7}{1495}
\DeclareTextGlyphY{LinLibertine_I}{afii57672}{1496}
\DeclareTextGlyphY{LinLibertine_I}{uni05D8}{1496}
\DeclareTextGlyphY{LinLibertine_I}{afii57673}{1497}
\DeclareTextGlyphY{LinLibertine_I}{uni05D9}{1497}
\DeclareTextGlyphY{LinLibertine_I}{afii57674}{1498}
\DeclareTextGlyphY{LinLibertine_I}{uni05DA}{1498}
\DeclareTextGlyphY{LinLibertine_I}{afii57675}{1499}
\DeclareTextGlyphY{LinLibertine_I}{uni05DB}{1499}
\DeclareTextGlyphY{LinLibertine_I}{afii57676}{1500}
\DeclareTextGlyphY{LinLibertine_I}{uni05DC}{1500}
\DeclareTextGlyphY{LinLibertine_I}{afii57677}{1501}
\DeclareTextGlyphY{LinLibertine_I}{uni05DD}{1501}
\DeclareTextGlyphY{LinLibertine_I}{afii57678}{1502}
\DeclareTextGlyphY{LinLibertine_I}{uni05DE}{1502}
\DeclareTextGlyphY{LinLibertine_I}{afii57679}{1503}
\DeclareTextGlyphY{LinLibertine_I}{uni05DF}{1503}
\DeclareTextGlyphY{LinLibertine_I}{afii57680}{1504}
\DeclareTextGlyphY{LinLibertine_I}{uni05E0}{1504}
\DeclareTextGlyphY{LinLibertine_I}{afii57681}{1505}
\DeclareTextGlyphY{LinLibertine_I}{uni05E1}{1505}
\DeclareTextGlyphY{LinLibertine_I}{afii57682}{1506}
\DeclareTextGlyphY{LinLibertine_I}{uni05E2}{1506}
\DeclareTextGlyphY{LinLibertine_I}{afii57683}{1507}
\DeclareTextGlyphY{LinLibertine_I}{uni05E3}{1507}
\DeclareTextGlyphY{LinLibertine_I}{afii57684}{1508}
\DeclareTextGlyphY{LinLibertine_I}{uni05E4}{1508}
\DeclareTextGlyphY{LinLibertine_I}{afii57685}{1509}
\DeclareTextGlyphY{LinLibertine_I}{uni05E5}{1509}
\DeclareTextGlyphY{LinLibertine_I}{afii57686}{1510}
\DeclareTextGlyphY{LinLibertine_I}{uni05E6}{1510}
\DeclareTextGlyphY{LinLibertine_I}{afii57687}{1511}
\DeclareTextGlyphY{LinLibertine_I}{uni05E7}{1511}
\DeclareTextGlyphY{LinLibertine_I}{afii57688}{1512}
\DeclareTextGlyphY{LinLibertine_I}{uni05E8}{1512}
\DeclareTextGlyphY{LinLibertine_I}{afii57689}{1513}
\DeclareTextGlyphY{LinLibertine_I}{uni05E9}{1513}
\DeclareTextGlyphY{LinLibertine_I}{afii57690}{1514}
\DeclareTextGlyphY{LinLibertine_I}{uni05EA}{1514}
\DeclareTextGlyphY{LinLibertine_I}{afii57716}{1520}
\DeclareTextGlyphY{LinLibertine_I}{uni05F0}{1520}
\DeclareTextGlyphY{LinLibertine_I}{afii57717}{1521}
\DeclareTextGlyphY{LinLibertine_I}{uni05F1}{1521}
\DeclareTextGlyphY{LinLibertine_I}{afii57718}{1522}
\DeclareTextGlyphY{LinLibertine_I}{uni05F2}{1522}
\DeclareTextGlyphY{LinLibertine_I}{uni05F3}{1523}
\DeclareTextGlyphY{LinLibertine_I}{uni05F4}{1524}
\DeclareTextGlyphY{LinLibertine_I}{Germandbls}{7838}
\DeclareTextGlyphY{LinLibertine_I}{uni1E9E}{7838}
\DeclareTextGlyphY{LinLibertine_I}{Oneroman}{8544}
\DeclareTextGlyphY{LinLibertine_I}{uni2160}{8544}
\DeclareTextGlyphY{LinLibertine_I}{Tworoman}{8545}
\DeclareTextGlyphY{LinLibertine_I}{uni2161}{8545}
\DeclareTextGlyphY{LinLibertine_I}{Threeroman}{8546}
\DeclareTextGlyphY{LinLibertine_I}{uni2162}{8546}
\DeclareTextGlyphY{LinLibertine_I}{Fourroman}{8547}
\DeclareTextGlyphY{LinLibertine_I}{uni2163}{8547}
\DeclareTextGlyphY{LinLibertine_I}{Fiveroman}{8548}
\DeclareTextGlyphY{LinLibertine_I}{uni2164}{8548}
\DeclareTextGlyphY{LinLibertine_I}{Sixroman}{8549}
\DeclareTextGlyphY{LinLibertine_I}{uni2165}{8549}
\DeclareTextGlyphY{LinLibertine_I}{Sevenroman}{8550}
\DeclareTextGlyphY{LinLibertine_I}{uni2166}{8550}
\DeclareTextGlyphY{LinLibertine_I}{Eightroman}{8551}
\DeclareTextGlyphY{LinLibertine_I}{uni2167}{8551}
\DeclareTextGlyphY{LinLibertine_I}{Nineroman}{8552}
\DeclareTextGlyphY{LinLibertine_I}{uni2168}{8552}
\DeclareTextGlyphY{LinLibertine_I}{Tenroman}{8553}
\DeclareTextGlyphY{LinLibertine_I}{uni2169}{8553}
\DeclareTextGlyphY{LinLibertine_I}{Elevenroman}{8554}
\DeclareTextGlyphY{LinLibertine_I}{uni216A}{8554}
\DeclareTextGlyphY{LinLibertine_I}{Twelveroman}{8555}
\DeclareTextGlyphY{LinLibertine_I}{uni216B}{8555}
\DeclareTextGlyphY{LinLibertine_I}{uni216C}{8556}
\DeclareTextGlyphY{LinLibertine_I}{uni216D}{8557}
\DeclareTextGlyphY{LinLibertine_I}{uni216E}{8558}
\DeclareTextGlyphY{LinLibertine_I}{uni216F}{8559}
\DeclareTextGlyphY{LinLibertine_I}{uni24B6}{9398}
\DeclareTextGlyphY{LinLibertine_I}{uni24B7}{9399}
\DeclareTextGlyphY{LinLibertine_I}{uni24B8}{9400}
\DeclareTextGlyphY{LinLibertine_I}{uni24B9}{9401}
\DeclareTextGlyphY{LinLibertine_I}{uni24BA}{9402}
\DeclareTextGlyphY{LinLibertine_I}{uni24BB}{9403}
\DeclareTextGlyphY{LinLibertine_I}{uni24BC}{9404}
\DeclareTextGlyphY{LinLibertine_I}{uni24BE}{9406}
\DeclareTextGlyphY{LinLibertine_I}{uni24BF}{9407}
\DeclareTextGlyphY{LinLibertine_I}{uni24C0}{9408}
\DeclareTextGlyphY{LinLibertine_I}{uni24C1}{9409}
\DeclareTextGlyphY{LinLibertine_I}{uni24C2}{9410}
\DeclareTextGlyphY{LinLibertine_I}{uni24C3}{9411}
\DeclareTextGlyphY{LinLibertine_I}{uni24C4}{9412}
\DeclareTextGlyphY{LinLibertine_I}{uni24C5}{9413}
\DeclareTextGlyphY{LinLibertine_I}{uni24C6}{9414}
\DeclareTextGlyphY{LinLibertine_I}{uni24C7}{9415}
\DeclareTextGlyphY{LinLibertine_I}{uni24C8}{9416}
\DeclareTextGlyphY{LinLibertine_I}{uni24CA}{9418}
\DeclareTextGlyphY{LinLibertine_I}{uni24CB}{9419}
\DeclareTextGlyphY{LinLibertine_I}{uni24CC}{9420}
\DeclareTextGlyphY{LinLibertine_I}{uni24CD}{9421}
\DeclareTextGlyphY{LinLibertine_I}{uni24CE}{9422}
\DeclareTextGlyphY{LinLibertine_I}{uni24CF}{9423}
\DeclareTextGlyphY{LinLibertine_I}{Tux}{57344}
\DeclareTextGlyphY{LinLibertine_I}{uniE000}{57344}
\DeclareTextGlyphY{LinLibertine_I}{uniE001}{57345}
\DeclareTextGlyphY{LinLibertine_I}{uniE002}{57346}
\DeclareTextGlyphY{LinLibertine_I}{uniE003}{57347}
\DeclareTextGlyphY{LinLibertine_I}{uniE004}{57348}
\DeclareTextGlyphY{LinLibertine_I}{uniE005}{57349}
\DeclareTextGlyphY{LinLibertine_I}{uniE006}{57350}
\DeclareTextGlyphY{LinLibertine_I}{uniE007}{57351}
\DeclareTextGlyphY{LinLibertine_I}{uniE008}{57352}
\DeclareTextGlyphY{LinLibertine_I}{uniE009}{57353}
\DeclareTextGlyphY{LinLibertine_I}{uniE00A}{57354}
\DeclareTextGlyphY{LinLibertine_I}{uniE00B}{57355}
\DeclareTextGlyphY{LinLibertine_I}{copyleft}{57356}
\DeclareTextGlyphY{LinLibertine_I}{uniE00C}{57356}
\DeclareTextGlyphY{LinLibertine_I}{publicdomain}{57357}
\DeclareTextGlyphY{LinLibertine_I}{uniE00D}{57357}
\DeclareTextGlyphY{LinLibertine_I}{creativecommons}{57358}
\DeclareTextGlyphY{LinLibertine_I}{uniE00E}{57358}
\DeclareTextGlyphY{LinLibertine_I}{zero.fitted}{57360}
\DeclareTextGlyphY{LinLibertine_I}{uniE010}{57360}
\DeclareTextGlyphY{LinLibertine_I}{one.fitted}{57361}
\DeclareTextGlyphY{LinLibertine_I}{uniE011}{57361}
\DeclareTextGlyphY{LinLibertine_I}{two.fitted}{57362}
\DeclareTextGlyphY{LinLibertine_I}{uniE012}{57362}
\DeclareTextGlyphY{LinLibertine_I}{three.fitted}{57363}
\DeclareTextGlyphY{LinLibertine_I}{uniE013}{57363}
\DeclareTextGlyphY{LinLibertine_I}{four.fitted}{57364}
\DeclareTextGlyphY{LinLibertine_I}{uniE014}{57364}
\DeclareTextGlyphY{LinLibertine_I}{five.fitted}{57365}
\DeclareTextGlyphY{LinLibertine_I}{uniE015}{57365}
\DeclareTextGlyphY{LinLibertine_I}{six.fitted}{57366}
\DeclareTextGlyphY{LinLibertine_I}{uniE016}{57366}
\DeclareTextGlyphY{LinLibertine_I}{seven.fitted}{57367}
\DeclareTextGlyphY{LinLibertine_I}{uniE017}{57367}
\DeclareTextGlyphY{LinLibertine_I}{eight.fitted}{57368}
\DeclareTextGlyphY{LinLibertine_I}{uniE018}{57368}
\DeclareTextGlyphY{LinLibertine_I}{nine.fitted}{57369}
\DeclareTextGlyphY{LinLibertine_I}{uniE019}{57369}
\DeclareTextGlyphY{LinLibertine_I}{W.alt}{57391}
\DeclareTextGlyphY{LinLibertine_I}{uniE02F}{57391}
\DeclareTextGlyphY{LinLibertine_I}{uniE040}{57408}
\DeclareTextGlyphY{LinLibertine_I}{uniE041}{57409}
\DeclareTextGlyphY{LinLibertine_I}{uniE042}{57410}
\DeclareTextGlyphY{LinLibertine_I}{ampersand.alt}{57424}
\DeclareTextGlyphY{LinLibertine_I}{uniE050}{57424}
\DeclareTextGlyphY{LinLibertine_I}{a.sc}{57425}
\DeclareTextGlyphY{LinLibertine_I}{uniE051}{57425}
\DeclareTextGlyphY{LinLibertine_I}{b.sc}{57426}
\DeclareTextGlyphY{LinLibertine_I}{uniE052}{57426}
\DeclareTextGlyphY{LinLibertine_I}{c.sc}{57427}
\DeclareTextGlyphY{LinLibertine_I}{uniE053}{57427}
\DeclareTextGlyphY{LinLibertine_I}{d.sc}{57428}
\DeclareTextGlyphY{LinLibertine_I}{uniE054}{57428}
\DeclareTextGlyphY{LinLibertine_I}{e.sc}{57429}
\DeclareTextGlyphY{LinLibertine_I}{uniE055}{57429}
\DeclareTextGlyphY{LinLibertine_I}{f.sc}{57430}
\DeclareTextGlyphY{LinLibertine_I}{uniE056}{57430}
\DeclareTextGlyphY{LinLibertine_I}{g.sc}{57431}
\DeclareTextGlyphY{LinLibertine_I}{uniE057}{57431}
\DeclareTextGlyphY{LinLibertine_I}{h.sc}{57432}
\DeclareTextGlyphY{LinLibertine_I}{uniE058}{57432}
\DeclareTextGlyphY{LinLibertine_I}{i.sc}{57433}
\DeclareTextGlyphY{LinLibertine_I}{uniE059}{57433}
\DeclareTextGlyphY{LinLibertine_I}{j.sc}{57434}
\DeclareTextGlyphY{LinLibertine_I}{uniE05A}{57434}
\DeclareTextGlyphY{LinLibertine_I}{k.sc}{57435}
\DeclareTextGlyphY{LinLibertine_I}{uniE05B}{57435}
\DeclareTextGlyphY{LinLibertine_I}{l.sc}{57436}
\DeclareTextGlyphY{LinLibertine_I}{uniE05C}{57436}
\DeclareTextGlyphY{LinLibertine_I}{m.sc}{57437}
\DeclareTextGlyphY{LinLibertine_I}{uniE05D}{57437}
\DeclareTextGlyphY{LinLibertine_I}{n.sc}{57438}
\DeclareTextGlyphY{LinLibertine_I}{uniE05E}{57438}
\DeclareTextGlyphY{LinLibertine_I}{o.sc}{57439}
\DeclareTextGlyphY{LinLibertine_I}{uniE05F}{57439}
\DeclareTextGlyphY{LinLibertine_I}{p.sc}{57440}
\DeclareTextGlyphY{LinLibertine_I}{uniE060}{57440}
\DeclareTextGlyphY{LinLibertine_I}{q.sc}{57441}
\DeclareTextGlyphY{LinLibertine_I}{uniE061}{57441}
\DeclareTextGlyphY{LinLibertine_I}{r.sc}{57442}
\DeclareTextGlyphY{LinLibertine_I}{uniE062}{57442}
\DeclareTextGlyphY{LinLibertine_I}{s.sc}{57443}
\DeclareTextGlyphY{LinLibertine_I}{uniE063}{57443}
\DeclareTextGlyphY{LinLibertine_I}{t.sc}{57444}
\DeclareTextGlyphY{LinLibertine_I}{uniE064}{57444}
\DeclareTextGlyphY{LinLibertine_I}{u.sc}{57445}
\DeclareTextGlyphY{LinLibertine_I}{uniE065}{57445}
\DeclareTextGlyphY{LinLibertine_I}{v.sc}{57446}
\DeclareTextGlyphY{LinLibertine_I}{uniE066}{57446}
\DeclareTextGlyphY{LinLibertine_I}{w.sc}{57447}
\DeclareTextGlyphY{LinLibertine_I}{uniE067}{57447}
\DeclareTextGlyphY{LinLibertine_I}{x.sc}{57448}
\DeclareTextGlyphY{LinLibertine_I}{uniE068}{57448}
\DeclareTextGlyphY{LinLibertine_I}{y.sc}{57449}
\DeclareTextGlyphY{LinLibertine_I}{uniE069}{57449}
\DeclareTextGlyphY{LinLibertine_I}{z.sc}{57450}
\DeclareTextGlyphY{LinLibertine_I}{uniE06A}{57450}
\DeclareTextGlyphY{LinLibertine_I}{uniE06B}{57451}
\DeclareTextGlyphY{LinLibertine_I}{hyphen.sc}{57453}
\DeclareTextGlyphY{LinLibertine_I}{uniE06D}{57453}
\DeclareTextGlyphY{LinLibertine_I}{adieresis.sc}{57460}
\DeclareTextGlyphY{LinLibertine_I}{uniE074}{57460}
\DeclareTextGlyphY{LinLibertine_I}{ae.sc}{57462}
\DeclareTextGlyphY{LinLibertine_I}{uniE076}{57462}
\DeclareTextGlyphY{LinLibertine_I}{odieresis.sc}{57478}
\DeclareTextGlyphY{LinLibertine_I}{uniE086}{57478}
\DeclareTextGlyphY{LinLibertine_I}{oe.sc}{57479}
\DeclareTextGlyphY{LinLibertine_I}{uniE087}{57479}
\DeclareTextGlyphY{LinLibertine_I}{thorn.sc}{57486}
\DeclareTextGlyphY{LinLibertine_I}{uniE08E}{57486}
\DeclareTextGlyphY{LinLibertine_I}{ij.sc}{57488}
\DeclareTextGlyphY{LinLibertine_I}{uniE090}{57488}
\DeclareTextGlyphY{LinLibertine_I}{germandbls.sc}{57490}
\DeclareTextGlyphY{LinLibertine_I}{uniE092}{57490}
\DeclareTextGlyphY{LinLibertine_I}{Q_u.sc}{57491}
\DeclareTextGlyphY{LinLibertine_I}{uniE093}{57491}
\DeclareTextGlyphY{LinLibertine_I}{eng.sc}{57508}
\DeclareTextGlyphY{LinLibertine_I}{uniE0A4}{57508}
\DeclareTextGlyphY{LinLibertine_I}{ohungarumlaut.sc}{57509}
\DeclareTextGlyphY{LinLibertine_I}{uniE0A5}{57509}
\DeclareTextGlyphY{LinLibertine_I}{uhungarumlaut.sc}{57518}
\DeclareTextGlyphY{LinLibertine_I}{uniE0AE}{57518}
\DeclareTextGlyphY{LinLibertine_I}{uniE104}{57604}
\DeclareTextGlyphY{LinLibertine_I}{uniE105}{57605}
\DeclareTextGlyphY{LinLibertine_I}{uniE106}{57606}
\DeclareTextGlyphY{LinLibertine_I}{uniE107}{57607}
\DeclareTextGlyphY{LinLibertine_I}{leaf}{57656}
\DeclareTextGlyphY{LinLibertine_I}{uniE138}{57656}
\DeclareTextGlyphY{LinLibertine_I}{hungarumlaut.cap}{58205}
\DeclareTextGlyphY{LinLibertine_I}{uniE35D}{58205}
\DeclareTextGlyphY{LinLibertine_I}{breve.cyrcap}{58208}
\DeclareTextGlyphY{LinLibertine_I}{uniE360}{58208}
\DeclareTextGlyphY{LinLibertine_I}{breve.cyr}{58209}
\DeclareTextGlyphY{LinLibertine_I}{uniE361}{58209}
\DeclareTextGlyphY{LinLibertine_I}{dieresis.cap}{58210}
\DeclareTextGlyphY{LinLibertine_I}{uniE362}{58210}
\DeclareTextGlyphY{LinLibertine_I}{metric}{58400}
\DeclareTextGlyphY{LinLibertine_I}{uniE420}{58400}
\DeclareTextGlyphY{LinLibertine_I}{uniFFFD}{65533}

%\end{multicols}

\clearpage
\section{Implementation Notes}\rm
\label{impl}

\subsection{Aims}

Modern OpenType and TrueType fonts are not directly usable with traditional
typesetting engines such as \LaTeX\ or pdf\LaTeX.  On the other hand, many documents
that use traditional font-selection mechanisms cannot be processed by
emerging new technologies such as xe\LaTeX\ and lua\LaTeX.  The primary aim
of the \texttt{libertine} package is, as much as possible, to allow documents
to use Linux Libertine and Biolinum fonts compatibly with \emph{all} current \LaTeX\
engines.  Another aim is maintainability: it should be possible to update
the package easily when updated fonts become available.

\subsection{The Fonts}

OpenType Linux Libertine and Biolinum fonts (with \texttt{otf} extensions) may be downloaded from 
\url{http://sourceforge.net/projects/linuxlibertine/files/linuxlibertine/}.
There are a few problems with the current versions of the fonts (5.3.0).
\begin{itemize}
\item 
Currently, there is no bold-italic variant of the Biolinum family; an \emph{ad hoc}
solution is to use \texttt{fontforge} to generate an artificially slanted
version of the bold variant.  Note that the most recent version of \texttt{fontforge}
must be used on Biolinum fonts; an earlier version will generate fonts with incorrect \texttt{ex}-height.
\item Slanted (oblique) variants are not available
from the upstream site.  These could be
generated easily but we have decided not to attempt to support slanted
variants for the fonts; the italic (or fake-italic) variants will be silently substituted.
\item
The bold-italic variant of the Libertine family is missing
several ligatures; the ligatures would be taken from the
regular-weight italic variant, which is unacceptable.
Michael Sharpe (\url{msharpe@ucsd.edu}) has generated
the missing glyphs  (\emph{\bfseries fl, ffl, ffi})  and added them to the \texttt{otf}
file.

\item Currently, Libertine Monospaced does not have bold, italic or bold-italic
variants; \texttt{fontforge} has been used to generate artificially emboldened and/or slanted
variants.

\item
When several of the fonts are opened in \texttt{fontforge}, warning
messages are generated about errors in the glyph programs.  Some of
these are sufficient to cause failures or even crashes when conversion
to \typeone\ format is attempted using \texttt{cfftot1}. Michael
Sharpe has corrected the most serious of these.  In some cases, 
\texttt{fontforge}
has been used to convert the format, as it is less sensitive than \texttt{cfftot1}
to faulty glyph programs.
\end{itemize}

In some \TeX\ distributions, the OpenType and \typeone\ fonts are installed as system fonts,
and xe\LaTeX\ or lua\LaTeX\ users may attempt to select the OpenType fonts directly by 
their Postscript FontName.  If \typeone\ versions with the \emph{same} FontName
have been installed, the latter may be selected by the system font-selection
mechanism.  To avoid this, it is appropriate to modify the
FontNames of the \texttt{otf} fonts before converting to \typeone\ 
format (but not \emph{distribute} these re-named \texttt{otf} fonts).
The \typeone\  Libertine and Biolinum fonts distributed in this
package have had the \texttt{O} (for Opentype) in their FontNames replaced by 
\texttt{T} 
(for Type~\liningnums{1}) using \texttt{fontforge}.  This font-renaming
must be done \emph{before} generating the \LaTeX-support files, or else
\texttt{dvi2ps} will fail.


\subsection{Generation of Support Files}

The \texttt{otftotfm} tool of the \texttt{lcdftypetools}
package and the \texttt{autoinst} script of the \texttt{fontools} package are convenient tools
for generating \LaTeX\ support files for OpenType families. 
To generate a \texttt{texmf} tree for the \texttt{libertine}
package on a Unix-like system, one puts all the \texttt{otf} files to be supported for \LaTeX\ or
pdf\LaTeX\footnote{Currently, all of the OpenType fonts except the Keyboard font are supported
for \LaTeX\ and pdf\LaTeX.} into a directory, creates a \texttt{texmf} sub-directory  and executes
\begin{verbatim}
    autoinst -target=./texmf -encoding=OT1,T1,LY1,TS1 \
      -vendor=public -typeface=libertine -noupdmap    \
      -noswash -notitling -noornaments                \
      *.otf
\end{verbatim}
Then move to the \texttt{texmf} directory and do
\begin{verbatim}
    rm -rf fonts/pl fonts/vpl fonts/truetype fonts/type42
    mv fonts/enc/dvips/public fonts/enc/dvips/libertine
    mv fonts/map/dvips/public fonts/map/dvips/libertine
\end{verbatim}
to delete irrelevant sub-directories and re-name directories as required by
TeXLive.

A few additional steps are needed.

\subsubsection{Renaming of the Encoding Files}
\texttt{otftotfm} generates encoding files with filenames of the form \verb|a_xxxxxx|;
to avoid filename conflicts with other packages, the files have been re-named
to have a distinctive prefix using the
command
\begin{verbatim}
    rename_enc libertine lbtn
\end{verbatim}
executed in the \texttt{texmf} directory, where
\verb|rename_enc| is a PERL script in 
\begin{verbatim}
    doc/fonts/libertine
\end{verbatim}
Then in \verb|fonts/map/dvips/libertine|, 
the map files can be concatenated into a single file \verb|libertine.map| and
all instances of \verb|a_| changed to \verb|lbtn_|; the 
original \verb|map| files have been deleted.

\subsubsection{Installation of the Fonts}

The \texttt{otf} files after corrections (but before re-naming) are
installed into the \texttt{texmf} tree in the following sub-directory:
\begin{verbatim}
    fonts/OpenType/public/libertine/
\end{verbatim}
The \texttt{autoinst} script will normally use \texttt{cfftot1}
to create \verb|pfb| files with appropriate internal names
and filenames; but if more than one font family has been processed
or if \texttt{cfftot1} runs into trouble, this may not happen.  In that
case, one must do the conversion font-by-font using either \texttt{cfftot1}
or \texttt{fontforge}; the appropriate internal names and filenames are as specified in 
\verb|libertine.map|. 
The \texttt{pfb} files are installed into the \texttt{texmf} tree in the following
sub-directory:
\begin{verbatim}
    fonts/type1/public/libertine/
\end{verbatim}

\subsubsection{The \texttt{fd} Files}
The \texttt{autoinst} script generates a large number of files with
\verb|.fd| extensions in the \verb|tex/latex/libertine/| directory.
Recent versions will generate ``silent substitution''
rules for mapping \texttt{sl} to \texttt{it} and \texttt{bx} to \texttt{b};
if not, these have to be added by hand.

\subsubsection{The \texttt{sty} Files}
The \texttt{autoinst} script generates files with \verb|.sty| extensions 
in the \verb|tex/latex/libertine/| directory for each of the font families;
but these are useless for xe\LaTeX\ and lua\LaTeX\ users and have been deleted.
A \verb|libertine.sty| file has been generated ``by hand'' and is discussed in Section~\ref{sty}.

\subsection{\texttt{libertine.sty}}
\label{sty}
This file implements the support for both \typeone\ and OpenType usage; the choice
is initially determined by the processing engine, but as some xe\LaTeX\  and lua\LaTeX\ 
users may prefer to avoid \texttt{fontspec}, a \texttt{type1} (or \texttt{nofontspec}) option
is provided to change this.

The \verb\...@scale\ commands are invoked in the \verb\fd\ files or when
specifying fonts with \texttt{fontspec}; only the scale factors for Biolinum and
Libertine Mono are adjustable using option parameters.

If the \texttt{sfdefault} option has been used, the \verb|\familydefault|
is set to the \emph{current} value of \verb|\sfdefault| (with no change to
\verb|\rmdefault|).

The use of \verb|\newfontfamily| rather than \verb|\addfontfeatures| avoids
problems in the implementation of the latter for some fonts (including, unfortunately,
Libertine).

For the Mono and Keyboard font families, the Ligature and SmallCap features must be turned off.

Commands to switch locally to oldstyle/lining/proportional/tabular numbers 
are defined; the definitions of \verb|\oldstylenums| must deal with possible
pre-existing definitions.

To implement the \verb|\...Glyph| commands, it is necessary to, essentially, iterate
through all the \emph{defined} glyphs in the relevant OpenType font.  This is implemented
by creating files \verb|LinLibertine_R.tex|, \verb|\LinBiolinum_R.tex|,
\verb|LinBiolinum_K.tex| and \verb|LinLibertine_I.tex| which declare the glyph name (when available),
unicode code point, and glyph index for every defined glyph.
These files are created by using \texttt{fontforge} to generate
a ``glyph map'' file (extension \verb|.g2n|) for the relevant font and then the small \texttt{C} program
\verb|doc/fonts/libertine/g2ntotex.c| will convert this into the required \verb|.tex| file.

The final step in \verb|libertine.sty| is to remove all default font features
in \texttt{fontspec} in case other fonts will be activated by the user.

\subsection{Additional \texttt{sty} Files}
The \verb|tex/latex/libertine/| directory also contains three ``front-end'' files
\verb|libertineotf.sty|, \verb|libertine-type1.sty|, and \verb|biolinum-type1.sty|,
which provide partial compatibility with obsolete packages, primarily for legacy
documents.  

\end{document}
