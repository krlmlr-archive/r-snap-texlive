% !TEX TS-program = pdflatex
% !TEX encoding = UTF-8 Unicode
%% venturisadf.tex
%% Copyright 2010 Clea F. Rees
%
% This work may be distributed and/or modified under the
% conditions of the LaTeX Project Public License, either version 1.3
% of this license or (at your option) any later version.
% The latest version of this license is in
%   http://www.latex-project.org/lppl.txt
% and version 1.3 or later is part of all distributions of LaTeX
% version 2005/12/01 or later.
%
% This work has the LPPL maintenance status `maintained'.
% 
% The Current Maintainer of this work is Clea F. Rees.
%
% This work consists of all files listed in manifest.txt.
\listfiles
\documentclass[11pt,british]{article}
\usepackage{babel}
% The next ten lines may be commented out if either 
% the current or no version of the fonts is installed.
\pdfmapfile{- yvt.map}
\pdfmapfile{- yv1.map}
\pdfmapfile{- yv2.map}
\pdfmapfile{- yv3.map}
\pdfmapfile{- yvo.map}
\pdfmapfile{+ yvt.map}
\pdfmapfile{+ yv1.map}
\pdfmapfile{+ yv2.map}
\pdfmapfile{+ yv3.map}
\pdfmapfile{+ yvo.map}
\usepackage{venturis}
% The following commands are only necessary to enable access to all the fonts in the collection in the same document. Normally, one would load venturis2 or venturisold rather than defining things explicitly. But those packages will overwrite the declarations in venturis so aren't an option in this peculiar context. (It is assumed that in ordinary usage, only one of the serif/sans-serif combinations provided by venturis and venturis2 or the serif offered by venturisold will be wanted in a single document.)
	\DeclareRobustCommand{\venturistworm}{%
		\fontencoding{T1}%
		\fontfamily{yv2}%
		\selectfont}
	\DeclareRobustCommand{\venturistwosf}{%
		\fontencoding{T1}%
		\fontfamily{yv3}%
		\selectfont}
	\DeclareRobustCommand{\venturisold}{%
		\fontencoding{T1}%
		\fontfamily{yvo}%
		\selectfont}
	\DeclareTextFontCommand{\vtworm}{\venturistworm}
	\DeclareTextFontCommand{\vtwosf}{\venturistwosf}
	\DeclareTextFontCommand{\vo}{\venturisold}
	\makeatletter
		\DeclareRobustCommand{\vostyle}[1][]{%
		\not@math@alphabet\vostyle\relax
		\fontfamily{yvod}\selectfont}
	\makeatother
	\DeclareTextFontCommand{\textvo}{\vostyle}
	\DeclareTextFontCommand{\textvol}{\vostyle}
\DeclareRobustCommand{\lmrmfamily}{%
	\fontencoding{T1}%
	\fontfamily{lmr}%
	\selectfont}
\DeclareTextFontCommand{\lmrm}{\lmrmfamily}
\renewcommand{\ttdefault}{lmvtt}
\usepackage{fancyhdr,lastpage,fancyref}
\usepackage{array,longtable,verbatim}
\usepackage{booktabs}
\usepackage{url}
	\urlstyle{sf}
\usepackage{metalogo}
	\setlogokern{Te}{-0.065em}% default: -0.1667em
	\setlogokern{eX}{-0.06em}% default: -0.125em
	\setlogokern{La}{-0.265em}% default: -0.36em
	\setlogokern{aT}{-.055em}% default: -0.15em
%	\setlogokern{X2}{}% default: 0.15em
	\setlogodrop[TeX]{0.355ex}% default: 0.5ex
	\setLaTeXa{\scshape a}
%	\setLaTeXee{<arg>}	
\usepackage{microtype}
\usepackage[a4paper,headheight=14pt]{geometry}	% use 14pt for 11pt text, 15pt for 12pt text

\title{venturisadf}
\author{Clea F.\ Rees\footnote{cfrees <at> imapmail <dot> org}}
\newcommand*{\dyddiad}{6\textsuperscript{th} July, 2010}
\date{\dyddiad}
\pagestyle{fancy}
	\fancyhf[lh]{\itshape venturisadf}
	\fancyhf[rh]{\itshape \dyddiad}
	\fancyhf[ch]{}
	\fancyhf[lf]{}
	\fancyhf[rf]{}
	\fancyhf[cf]{\itshape --- \thepage~\ofname~\pageref{LastPage} ---}
	

\begin{document}
\maketitle\thispagestyle{empty}
\pdfinfo{%
	/Creator		(TeX)
	/Producer		(pdfTeX)
	/Author			(Clea F.\ Rees)
	/Title				(venturisadf)
	/Subject		(TeX)
	/Keywords		(TeX,LaTeX,font,fonts,tex,latex,Venturis,venturis,venturisadf,VventurisADF,ADF,adf,Arkandis,Digital,Foundry,arkandis,digital,foundry,Hirwen,Harendal,Clea,Rees)}
\pdfcatalog{%
	/URL				()
	/PageMode	/UseOutlines}	% other values: /UseNone, /UseOutlines, /UseThumbs, /FullScreen
	%[openaction <actionspec>]
%	\pagestyle{empty}	% if you want this, you probably want to comment out \maketitle as well...?
\setlength{\parindent}{0pt}
\setlength{\parskip}{0.5em}
	
	
\newcommand*{\adf}{\textsc{adf}}
\newcommand*{\lpack}[1]{\textsf{#1}}
\newcommand*{\fgroup}[1]{\textsf{#1}}
\newcommand*{\fname}[1]{\textsf{#1}}

\begin{abstract}
	\hspace*{-\parindent}Hirwen Harendal, Arkandis Digital Foundry (\adf) has produced the Venturis \adf\ font collection. This guide outlines the \TeX/\LaTeX\ support provided with version 1.005 of the fonts in postscript type 1 format.
\end{abstract}

\tableofcontents

\section{Introduction}

This document explains how to use the \TeX/\LaTeX\ support included with version 1.005 of the Venturis \adf\ font collection in postscript type 1 format. The fonts were developed by Hirwen Harendal of the Arkandis Digital Foundry (\adf), and information about the fonts themselves, together with copies of the fonts in opentype format, can be found at \url{http://pagesperso-orange.fr/arkandis/ADF/tugfonts.htm}. The fonts are based on Adobe's Utopia and released under the same licence. For details, see \textsc{readme} and \textsc{license-}utopia.txt. \textsc{list-v}enturis.txt includes a list of the fonts included in the collection.

The \TeX/\LaTeX\ support package consists of all files listed in \lpack{manifest.txt}\ and these files are released under the \LaTeX\ Project Public Licence as explained in the included licensing notices and \textsc{readme}. Please let me know of any problems so that I can solve them if I can. If you can correct the problems and send me the fix, that would be even better. Unlike the fonts themselves, the \TeX/\LaTeX\ support is still somewhat experimental.

\section{The collection}

Venturis \adf\ currently includes 69 fonts in six groups. The support package rearranges these into five basic groups and renames them according to the Karl Berry fontname scheme.

\begin{longtable}{llll}
	\toprule
	\textbf{Original grouping}		&	\textbf{Original} \textbf{name}	&	\textbf{\TeX\ directory/group}	& \textbf{\TeX\ name}\\\midrule\endhead
		\bottomrule\endfoot
	Venturis							&	VenturisADF--						&	venturis				&	yvt--\\
	Venturis No2				&	VenturisADFNo2--				&	venturis2			&	yv2--\\
	Venturis Old					&	VenturisOldADF--				&	venturisold		&	yvo--\\
	Venturis Sans				&	VenturisSansADF--				&	venturissans		&	yv1--\\
	Venturis Sans No2		&	VenturisSansADFNo2--		&	venturissans2	&	yv3--\\
	Venturis Titling No1	&	VenturisADFTitlingNo1		&	venturissans		&	yv1dd8au\\
	Venturis Titling No2	&	VenturisADFTitlingNo2		&	venturis				&	yvtbd8ac\\
	Venturis Titling No3	&	VenturisADFTitlingNo3		&	venturissans		&	yv1bd8a\\
	Venturis Titling No4	&	VenturisADFTitlingNo4		& venturis 				&	yvtrdl8a\\
	Venturis Goth Titling &	VenturisADFGothTitling	&	venturisold		&	yvodd8a\\
\end{longtable}

Venturis Titling is thus split according to style and intended usage. The first four fonts in this group are intended for use with Venturis, Venturis No2, Venturis Sans and Venturis Sans No2. For the purposes of installation for use with \TeX\ and \LaTeX\ they have been grouped with Venturis and Venturis Sans. This is an arbitrary decision and users may also wish to use them with Venturis No2 and Venturis Sans No2. To this end, the special titling commands provided by \lpack{venturis} are also available when \lpack{venturis2} is loaded. (See \fref{sec:titling}.)

Apart from Venturis Titling which is not designed as a general text font, all fonts come in at least an upright and either an italic or oblique shape. Most fonts also include a variety of widths and weights. In addition, condensed and standard-width Venturis regular and bold provide a choice of oldstyle and lining figures, upright and italic small capitals and alternate swash characters while the regular upright face also offers end-of-word swashes. Venturis Old offers two additional ligatures and the long s while the bold upright face also includes two alternate characters. All of these features are accessible in \LaTeX\ through the included package files as explained in \fref{sec:support} and \fref{sec:commands}. 

\section{Installation}

Apart from such obvious requirements as \LaTeXe, the \LaTeX\ packages \lpack{venturis}, \lpack{venturis2}\ and \lpack{venturisold}\ require \lpack{xkeyval} and \lpack{nfssext-cfr}. \lpack{xkeyval} is required for option processing. \lpack{nfssext-cfr} provides additional font selection commands and without it you will get errors complaining that the package cannot be found and you will not be able to use any of the additional font commands described in \fref{sec:commands}. Both packages are included in \TeX~Live\footnote{This includes Mac\TeX\ for OS X users.}, as is \lpack{venturisadf}\ itself. If you have a recent full install of \TeX~Live, you can almost certainly skip this section entirely.

Installation varies with \TeX\ distribution so you should consult the documentation which came with your system for details. In most cases, you will need to perform three steps:
		\begin{enumerate}
			\item move or copy the package files to appropriate locations on your system;
			\item refresh the \TeX\ database;
			\item incorporate the included map file fragments for the different engines your distribution supports.
		\end{enumerate}

The following instructions assume you are using \TeX~Live. They should not be too difficult to adapt if you are using a different distribution.

\subsection{Install the files}

The files should be installed in one of two locations: \emph{either} the local system-wide \TeX\ tree \emph{or} your personal tree. If the package is installed system-wide, all users will have access to it. On the other hand, you may need privileges you do not have to do this in which case you must use your personal tree.

For \TeX~Live, \verb|kpsewhich -var-value TEXMFLOCAL| will return the path to the local tree and \verb|kpsewhich -var-value TEXMFHOME| the path to your personal tree. The package already includes a hierarchy of files to help you install them correctly. Ignoring any symbolic link in the top directory, move or copy the files in \path{doc}, \path{fonts} and \path{tex} into the appropriate locations. If the tree is initially empty, you can simply move or copy the directories in as they are. If the tree already contains other packages, you may need to merge the package hierarchy with the pre-existing one. For example, if you already have a \path{doc/fonts} directory, move or copy \path{doc/fonts/venturis*} into \path{doc/fonts/}. If you have a \path{doc} directory but not a \path{doc/fonts}, move \path{doc/fonts} into \path{doc/}.

\subsection{Refresh the database}

Again, this depends on your distribution. For \TeX~Live, \verb|mktexlsr <path to directory>| for the directory you used in the first step should do the trick. Note that you \emph{may} be able to skip this step if you install into your personal tree. Whether this is so depends on the details of your set-up. As a test, move to a directory containing none of the package files and try \verb|kpsewhich venturis.sty|. If the file is found, you don't need to refresh the database; otherwise use \verb|mktexlsr| and then try again.

\subsection{Install the map fragments}

For \TeX~Live, there are at least two ways of doing this. The second method varies according to the version of \TeX~Live and instructions are provided accordingly. Both methods depend on whether you installed into \verb|TEXMFLOCAL| or \verb|TEXMFHOME|. If you installed system-wide, the choice is relatively straightforward --- it obviously makes sense in that case to update the font maps system-wide as well. If, on the other hand, you installed into your personal tree, the matter is more complex. On the one hand, updating the system-wide maps may create difficulties or confusion for other users because while the map files will list the fonts as available, they will not be able to access them. On the other hand, maintaining personal font map files can produce difficulties and confusions of its own. Whether it is to be preferred or not is a complex issue and depends on the details of your \TeX\ distribution, local configuration and personal preference. The one clear case is that in which you install into your personal tree because you lack the privileges needed to install system-wide. In that case, you have no choice but to maintain personal font map files or forgo the use of all fonts not provided by your administrator. Other cases are thankfully beyond the scope of this document.

\subsubsection{Method 1}

If you installed the package system-wide, use these commands:
\begin{verbatim}
	updmap-sys --enable Map=yvt.map
	updmap-sys --enable Map=yv1.map
	updmap-sys --enable Map=yv2.map
	updmap-sys --enable Map=yv3.map
	updmap-sys --enable Map=yvo.map
\end{verbatim}
If you installed the package in your personal tree, you \emph{may} prefer to use:
\begin{verbatim}
	updmap --enable Map=yvt.map
	updmap --enable Map=yv1.map
	updmap --enable Map=yv2.map
	updmap --enable Map=yv3.map
	updmap --enable Map=yvo.map
\end{verbatim}

Either way, \verb|updmap| will output a good deal of information after each incantation. This is normal. Just check that it does not end with an error and that it found each of the new map files.

\subsubsection{Method 2: \TeX~Live 2008 (and probably earlier)}

If you installed the package system-wide, use \verb|updmap-sys --edit|.

If you installed into your personal tree, you \emph{may} prefer to use	\verb|updmap --edit|.

Either way, a configuration file will be opened which you can edit. Move to the end of the file and add the following:
\begin{verbatim}
	# Maps for using PS version of VenturisADF fonts
	Map yvt.map
	Map yv2.map
	Map yv1.map
	Map yv3.map
	Map yvo.map
	# end maps for VenturisADF
\end{verbatim}
When you are done, save the file. \verb|updmap| will produce a great deal of output if all is well. Just check that it does not end with an error and that the map files are found.

\subsubsection{Method 2: \TeX~Live 2009 (and possibly later)}

If you installed the package system-wide, edit or create a local configuration file, \path{TEXMFLOCAL/web2c/updmap-local.cfg}, and add the following lines to the end of the file:
\begin{verbatim}
	# Maps for using PS version of VenturisADF fonts
	Map yvt.map
	Map yv2.map
	Map yv1.map
	Map yv3.map
	Map yvo.map
	# end maps for VenturisADF
\end{verbatim}
Save the file and tell \verb|tlmgr| to merge in your additions using the command:
\begin{verbatim}
	tlmgr generate updmap
\end{verbatim}
\verb|tlmgr| will then tell you that you need to ensure the changes are propagated correctly by calling \verb|updmap-sys|. This should produce a great deal of output. Check that it finds the new map files and does not end with an error.

If you installed into your personal tree, you \emph{may} prefer to use \verb|updmap --edit| as described above for \TeX~Live 2008.

To test your installation and that the package works on your system, you may wish to try typesetting the source for the documentation \path{venturisadf.tex}. The console output and/or log should tell you whether any fonts were not found. If you are careful not to overwrite it, you may also compare your output with \path{venturisadf.pdf}.

\section{The support package}\label{sec:support}

\subsection{Encodings}

The package supports the \textsc{ec}/\textsc{t1} and Text Companion (\textsc{ts1}) encodings. Most characters in the \textsc{ec} encoding are available\footnote{The exceptions are the perthousandzero and the Sami characters Eng/Ng (\lmrm{\NG}) and eng/ng (\lmrm{\ng}) which the fonts do not provide and which fontinst cannot construct from other glyphs.} and the fonts provide a small number of characters from the \textsc{ts1} encoding as well, including the \texteuro. The exceptions are: the virtual font which provides access to the end-of-word swashes available in the upright, regular Venturis font; and the virtual fonts for Venturis Old. In former case, because additional slots are required to accommodate the end-of-word swashes, a number of characters normally available in the \textsc{ec} encoding are unavailable. Attempting to access these characters while using this font will result in errors of various kinds and/or unexpected output and should be avoided. For details, see the list of changes described at the top of \path{t1-venturis.etx}\ which lists the slots reassigned to end-of-word swashes. In the latter case, the setup for Venturis Old uses two encodings: \path{t1-venturisold.etx} and \path{t1-venturisold-longs.etx}. The first of these uses slots normally assigned to characters unavailable in this font\footnote{That is, the perthousand zero and the Sami characters.} for the two alternate characters and one of the extra ligatures. In a sense, nothing is lost by this since the characters are unavailable anyway but attempting to typeset those characters will produce rather odd results. The second encoding uses two additional slots for the long s and the second additional ligature\footnote{The ``lost'' characters are dbar (\dj) and the ij ligature.}. Although the slots used are for characters not provided by the font, these characters can be faked by \path{fontinst} so that the loss in this case is genuine. Details of all reassignments are provided in the encoding files themselves.

\subsection{Package files}

Three \LaTeX\ packages are provided. Each package provides a number of additional font selection commands for use in \LaTeX\footnote{Note that the provision of a command does not guarantee that it will have any effect. If you are using Venturis Sans and try to switch to oldstyle figures, for example, nothing will happen except for a warning message output to the console to inform you that no suitable font is available.}.

Note that all packages substitute italic for oblique and \emph{vice-versa}. If italic small-caps is unavailable, small-caps is substituted where possible. Upright is substituted for small-caps when the latter are unavailable. The titling families typically substitute a titling font in another weight or width if no font in the current weight or width is available, and small-caps is usually substituted for other shapes. This means that if you switch to titling while using italic, extended bold, you should generally get what you expect i.e.\ a titling font even though this may be an upright small-caps font in another weight and width.

\subsubsection{venturis}\label{sec:venturis}
	
	To load this package, use \verb|\usepackage{venturis}| in your document preamble. The package enables the fonts in \fgroup{venturis}\ and \fgroup{venturissans}. By default, the package will use oldstyle figures where possible. If you would prefer lining figures by default, use \verb|\usepackage[lf]{venturis}| instead. \verb|\usepackage[osf]{venturis}| requests oldstyle figures explicitly .
	
	The package will select \fname{yvtj}\ or \fname{yvt}\ as the default serif/roman family for the document, \fname{yv1}\ as the default sans-serif family, \fname{yvtd}\ as the default serif titling family and \fname{yv1d}\ as the default sans-serif titling family.  In addition, the package will enable access to the various alternate and swash characters available in \fname{yvtjw}, \fname{yvtajw}, \fname{yvtw}\ and \fname{yvtaw}. The difference between the \fgroup{alternate swash}\ and \fgroup{swash}\ families is that the former enables end-of-word swashes in the regular-width regular-weight upright font whereas the latter does not.
	
	\begin{longtable}{l>{\raggedright}p{.15\textwidth}l>{\raggedright}p{.125\textwidth}>{\raggedright}p{.15\textwidth}>{\raggedright}p{.2\textwidth}}
		\toprule
		\textbf{family}	&	\textbf{text}	& \textbf{figures}	&	\textbf{widths}	&	\textbf{weights}	&	 \textbf{shapes}\tabularnewline\midrule\endhead
		\bottomrule\endfoot
		yvt		&	serif			&	lining		&	standard	&	regular, bold		&	upright, italic,\\small-caps,\\italic small-caps\tabularnewline\cmidrule{5-6}
					&						&					&						&	heavy					&	upright, italic\tabularnewline\cmidrule{4-6}
					&						&					&	condensed	&	regular, bold		&	upright, italic,\\small-caps,\\italic small-caps\tabularnewline\midrule
		yvtj		&	serif	& oldstyle	&	standard,\\condensed	&	regular, bold	&	upright, italic,\\small-caps,\\italic small-caps\tabularnewline\midrule
		yvtw	&	serif, swash	&	lining	&	standard,\\condensed	&	regular, bold		&	upright, italic,\\small-caps,\\italic small-caps\tabularnewline\midrule
		yvtjw	&	serif, swash	&	oldstyle	&	standard,\\condensed&	regular, bold	&	upright, italic,\\small-caps,\\italic small-caps\tabularnewline\midrule
		yvtaw\footnote{yvtw substituted where applicable.}	&	serif, alternate swash	&	lining	&	standard	&	regular	&	upright\tabularnewline\midrule
		yvtajw\footnote{yvtjw substituted where applicable.}&	serif, alternate swash	&	oldstyle	&	standard	&	regular	&	upright\tabularnewline\midrule
		yvtd	&	serif, titling	&	lining	&	condensed				&	bold	&	small-caps\footnote{Substituted for other weights and widths as applicable. See \fref{sec:titling}.}\tabularnewline\cmidrule{4-6}
					&							&							&	standard				&	regular				&	outline small-caps\tabularnewline\midrule
		yv1		&	sans					&	lining				&	standard				&	regular, bold, light, demibold	&	upright, italic\tabularnewline\cmidrule{5-6}
					&							&				&									&	heavy					&	upright, oblique\tabularnewline\cmidrule{4-6}
					&							&				&	condensed				&	regular, bold		&	upright, italic\tabularnewline\cmidrule{4-6}
					&							&				&	extended				&	regular, bold		&	upright, italic\tabularnewline\midrule
		yv1d	&	sans, titling	& lining	&	standard				&	bold					& small-caps\footnote{Substituted for other weights and widths as applicable. See \fref{sec:titling}.}\tabularnewline\cmidrule{4-6}
					&							&				&	ultra-condensed	&	demibold			&	small-caps\footnote{Substituted for other weights and widths as applicable. See \fref{sec:titling}.}\tabularnewline
	\end{longtable}
	
\subsubsection{venturis2}

	Use \verb|\usepackage{venturis2}| in your document preamble to load this package, which enables the fonts in \fgroup{venturis2}\ and \fgroup{venturissans2} and the titling fonts from \fgroup{venturis} and \fgroup{venturissans}. The package will select \fname{yv2}\ as the default roman/serif family for the document and \fname{yv3}\ as the default sans-serif family. Although no titling family is enabled by default, the special titling commands explained in \fref{sec:titling} provide access to the serif titling family \fname{yvtd}\ and the sans-serif titling family\fname{yv1d} (see \fref{sec:venturis} for details of these families). 
	
	\begin{longtable}{l>{\raggedright}p{.15\textwidth}l>{\raggedright}p{.125\textwidth}>{\raggedright}p{.15\textwidth}>{\raggedright}p{.2\textwidth}}
		\toprule
		\textbf{family}	&	\textbf{text}	& \textbf{figures}	&	\textbf{widths}	&	\textbf{weights}	&	 \textbf{shapes}\tabularnewline\midrule\endhead
		\bottomrule\endfoot
		yv2	&	serif	&	lining	&	standard	&	regular, bold, medium, heavy	&	upright, italic\tabularnewline\cmidrule{4-6}
				&				&				&	condensed	&	regular, bold									&	upright, italic\tabularnewline\midrule
		yv3	&	sans		&	lining	&	standard	&	regular, bold 								&	upright, italic\tabularnewline\cmidrule{4-6}
				&				&				&	condensed	&	regular, bold 								&	upright, italic\tabularnewline\cmidrule{4-6}
				&				&				&	extended	&	regular, bold 								&	upright, italic\tabularnewline
	\end{longtable}
	
\subsubsection{venturisold}

	To load this package, which enables the fonts in \fgroup{venturisold}, use \verb|\usepackage{venturisold}| in your document preamble. Note that this package does not affect the default sans-serif font so that \verb|\textsf|, \verb|\sffamily| etc.\ will still use Computer Modern by default. The package will select \fname{yvo}\ as the default roman/serif family for the document and \fname{yvod}\ as the default titling family.

	\begin{longtable}{l>{\raggedright}p{.15\textwidth}l>{\raggedright}p{.125\textwidth}>{\raggedright}p{.15\textwidth}>{\raggedright}p{.2\textwidth}}
		\toprule
		\textbf{family}	&	\textbf{text}	& \textbf{figures}	&	\textbf{widths}	&	\textbf{weights}	&	 \textbf{shapes}\tabularnewline\midrule\endhead
		\bottomrule\endfoot
		yvo		&	serif															&	oldstyle		&	standard	&	regular, bold		&	upright, italic\tabularnewline\midrule
		yvod	&	serif, titling												& lining			&	standard	&	demibold			&	small-caps\tabularnewline\midrule
		yvoa	&	serif, alternate (includes the long s)	&	oldstyle		&	standard	&	regular, bold		&	upright, italic\tabularnewline\midrule
		yvoad%
			\footnote{yvoad is identical to yvod and is installed under this name as a convenience.}%
					&	serif, titling												& lining			&	standard	&	demibold			&	small-caps\tabularnewline
	\end{longtable}
	
	\fname{yvo} provides two alternate characters and one additional ligature. The additional ligature will automatically replace `\vo{s{}t}' with `\vo{st}'. The two alternate characters, `\vo{\textbf{d*}}' and `\vo{\textbf{s*}}', may be accessed by typing \verb|d*| and \verb|s*|. These only exist in the bold upright face, but the encoding tries to approximate the expected output for the regular weight and italic shape by substituting the ordinary `\vo{d}' or `\vo{s}' for \verb|d*| and \verb|s*| when no alternates are available.
	
	\fname{yvoa} provides in addition the long s, `\vo{\textalt{s+}}\,', which may be accessed using \verb|s+|, and another ligature, `\vo{\textalt{s+t}}', to replace `\vo{\textalt{s+{}t}}'.
	
	Note that these characters are absent from the titling font, so you should not type a \verb|d| or an \verb|s| followed by an \verb|*| or \verb|+| when using this unless you literally want to see the sequence in your output. Likewise, you should not type \verb|s+| except within the scope of a suitable command (\fref{sec:styles}) unless you actually want to see the plus-sign in your output.
	
\section{Additional font selection commands}\label{sec:commands}

	The \LaTeX\ packages \lpack{venturis}, \lpack{venturis2}\ and \lpack{venturisold}\ load \lpack{nfssext-cfr}\ which is an extension of the package \lpack{nfssext}\ supplied by Philipp Lehman as part of The Font Installation Guide. The file extends the font selection commands to facilitate access to various font features. Both the original and the extension are designed for use with a wide range of fonts. For this reason, only a subset of the additional commands are relevant to any particular font support package. Those relevant to \lpack{venturisadf}\ are described below.
	
	I consider my additions to \lpack{nfssext-cfr}\ to be \emph{highly experimental}. If things don't work as advertised, apart from letting me know about the problem, you may be able to access the features you need by issuing a \verb|\normalfont| and then selecting features from there. This command will return you to the default document text font --- typically the relevant serif in regular weight, standard width and upright shape with oldstyle or lining figures as determined by the package and options loaded. 
	
	In addition, \lpack{venturis}\ and \lpack{venturisold}\ provide custom titling commands as explained later. These provide alternatives to the titling commands provided by \lpack{nfssext} (see \fref{sec:styles} and \fref{sec:titling}).

\subsection{nfssext-cfr}

These commands are available when \lpack{venturis}, \lpack{venturis2}\ or \lpack{venturisold} is loaded. If for some reason you wish to use them at other times, include \verb|\usepackage{nfssext-cfr}| in your document preamble.
	
\subsubsection{Widths}

	\begin{longtable}{lll}
		\toprule
		\textbf{width}	&	\textbf{width command}	&	\textbf{text command}\\\midrule\endhead
		\bottomrule\endfoot
		standard				&	\verb|\regwidth|					&	\verb|\textrw{}|\\
		condensed				&	\verb|\cdwidth|					&	\verb|\textcd{}|\\
		extended\footnote{Note that because \LaTeX\ defines the default bold series as bold, extended by default, it makes no difference whether extended or standard width is used if the active weight is bold.}				&	\verb|\etwidth| 					&	\verb|\textet{}|\\
		ultra-condensed	&	\verb|\ucwidth|					&	\verb|\textuc{}|\\
	\end{longtable}

	To switch to an extended width until further notice, for example, you could use \verb|\etwidth|. Or use \verb|\textsf{\textcd{Hello, world!}}| to typeset just the text \textsf{\textcd{Hello, world!}} in a condensed width sans.
	
\subsubsection{Weights}

	\begin{longtable}{lll}
		\toprule
		\textbf{weight}		&	\textbf{weight command}	&	\textbf{text command}\\\midrule\endhead
		\bottomrule\endfoot
		light							&	\verb|\lgweight|						&	\verb|\textlg{}|\\
		medium						&	\verb|\mbweight|					&	\verb|\textmb{}|\\
		demibold					&	\verb|\dbweight|					&	\verb|\textdb{}|\\
		heavy							&	\verb|\ebweight|					&	\verb|\texteb{}|\\
	\end{longtable}
	
	These work in the same way as the standard \LaTeX\ commands for switching to bold text, for example, and as the commands for selecting widths described above. Note that if you are using \lpack{venturis}\ with oldstyle figures and you wish to use a heavy serif, for example, you need to switch temporarily to lining figures. \verb|\textl{\texteb{Hello, world!}}| will produce \textl{\texteb{Hello, world!}}, for example, but \verb|\texteb{Hello, world!}| will produce only \texteb{Hello, world!}
	\begin{verbatim}
		\textl{\texteb{Heavy and \textsl{heavy oblique} serif}}\\
		\textsf{\textlg{Light sans}}\\
		\textsf{\textdb{Demibold sans}}
	\end{verbatim}
	produces:
	\begin{center}
		\textl{\texteb{Heavy and \textsl{heavy oblique} serif}}\\
		\textsf{\textlg{Light sans}}\\
		\textsf{\textdb{Demibold sans}}
	\end{center}

\subsubsection{Shapes}

	\begin{longtable}{lll}
		\toprule
		\textbf{shape}		&	\textbf{shape command}	&	\textbf{text command}\\\midrule\endhead
		\bottomrule\endfoot
		italic small-caps		&	\verb|\sishape|					&	\verb|\textsi{}|\\
		outline						&	\verb|\olshape|					&	\verb|\textol{}|\\		
	\end{longtable}
		
	For example, \verb|\textsi{I always avoid a kangaroo.}| produces:
		\begin{center}
			\textsi{I always avoid a kangaroo.}
		\end{center}
	if italic small-caps is available for the active font. If it is not, the small-caps upright shape or, failing that, the ordinary upright shape will be substituted.
	
	For example, \verb|\textsf{\textsi{The last example was from Lewis Carroll.}}| produces:
		\begin{center}
			\textsf{\textsi{The last example was from Lewis Carroll.}}
		\end{center}
		
\subsubsection{Styles}\label{sec:styles}

	\begin{longtable}{lll}
		\toprule
		\textbf{style}			&	\textbf{style command}	&	\textbf{text command}\\\midrule\endhead
		\bottomrule\endfoot
		swash							&	\verb|\swashstyle|				&	\verb|\textswash{}|\\
		titling%
			\footnote{Note that these commands \emph{\textbf{cannot}} be used to access titling fonts when \lpack{venturis2} is loaded. Use the commands explained in \fref{sec:titling} instead.}%
											&	\verb|\tistyle|						&	\verb|\textti{}|\\
		alternate/long s		&	\verb|\altsytle|					&	\verb|\textalt{}|\\
	\end{longtable}
	
	For example, if \lpack{venturis}\ was loaded and the current font is of standard width and weight then:
		\begin{verbatim}
			\textswash{\textsc{Kinky Querulous Rhinos X-Ray Exultant Risque Zebras}}\\
			\textswash{\textsi{Kinky Querulous Rhinos X-Ray Exultant Risque Zebras}}\\
			\textswash{Kinky Querulous Rhinos X-Ray Exultant Risque Zebras}\\
			\textswash{\textit{Kinky Querulous Rhinos X-Ray Exultant Risque Zebras}}
		\end{verbatim}
		produces:
		\begin{center}
		 	\textswash{\textsc{Kinky Querulous Rhinos X-Ray Exultant Risque Zebras}}\\
			\textswash{\textsi{Kinky Querulous Rhinos X-Ray Exultant Risque Zebras}}\\
			\textswash{Kinky Querulous Rhinos X-Ray Exultant Risque Zebras}\\
			\textswash{\textit{Kinky Querulous Rhinos X-Ray Exultant Risque Zebras}}
		\end{center}
		and:
		\begin{verbatim}
			\textswash{\textalt{Kinky Querulous Rhinos X-Ray Exultant Risque Zebras\\
			agoraphobia doddled essence giggling hashish\\
			 loll momentum nationalisation ontological\\orzo rarer tritest unanimous\\
			 mu zigzag oz}}
		\end{verbatim}
		produces:
		\begin{center}
			 \textswash{\textalt{Kinky Querulous Rhinos X-Ray Exultant Risque Zebras\\
			 agoraphobia doddled essence giggling hashish\\
			 loll momentum nationalisation ontological\\orzo rarer tritest unanimous\\
			 mu zigzag oz}}
		\end{center}
	If, \lpack{venturisold}\ was loaded, then:
		\begin{verbatim}
			Pandas on scooters s*treaked* past fast.\\
			\textalt{Pandas+ on s+cooters s*treaked* pas+t fast.}
		\end{verbatim}
	produces:
		\begin{center}	
			\vo{Pandas on scooters s*treaked* past fast.\\
			\textalt{Pandas+ on s+cooters s*treaked* pas+t fast.}}
		\end{center}					
	
\subsubsection{Figures}

	\begin{longtable}{lll}
		\toprule
		\textbf{figure style}			&	\textbf{style command}	&	\textbf{text command}\\\midrule\endhead
		\bottomrule\endfoot
		lining										&	\verb|\lstyle|						&	\verb|\textl{}|\\
		oldstyle									&	\verb|\ostyle|						&	\verb|\texto{}|\\
	\end{longtable}
	
	In this document, oldstyle figures are used when available by default:
		\begin{center}
			0123456789
		\end{center}
	but lining figures are also accessible. For example, \verb|\textl{0123456789}| produces:
		\begin{center}
			\textl{0123456789}
		\end{center}

\subsection{Special titling commands}\label{sec:titling}

	These commands are available when \lpack{venturis}, \lpack{venturis2}\ or \lpack{venturisold}\ is loaded although their affects are package and context dependent.
	
	\begin{longtable}{llll}
		\toprule
		\textbf{style}			&	\textbf{shape}		&	\textbf{style command}	&	\textbf{text command}\\\midrule\endhead
		\bottomrule\endfoot
		titling							&	small-caps					&	\verb|\vtstyle|					&	\verb|\textvt{}|\\
											&	outline small-caps	&	\verb|\vtstyle[ol]|				&	\verb|\textvtl{}|\\
	\end{longtable}
	
	When \lpack{venturisold}\ is loaded, \verb|\textvtl{}| and \verb|\textvt{}| are equivalent and the optional argument to \verb|\vtstyle| has no effect. The single titling font available will be selected regardless of the current font shape, weight and width. For example:
	\begin{verbatim}
		\textbf{\textit{\textvtl{Venturis Old Titling}}}\\
		\vtstyle[yddraiggoch]{\textsc{Venturis Old Titling}}
	\end{verbatim}
	produces:
	\begin{center}
		\textbf{\textit{\textvol{Venturis Old Titling}}}\\
		\vostyle[yddraiggoch]{\textsc{Venturis Old Titling}}
	\end{center}
	Note that the titling font does not include any provision for the alternate characters, additional ligatures or long s. Unless you wish to produce a `d' or `s' followed by an asterisk or plus-sign, you should avoid these sequences in titling text.
	
	When \lpack{venturis}\ or \lpack{venturis2}\ is loaded, \verb|\textvtl{}| and \verb|\vtstyle[ol]| will always select the outline small-caps font in regular width and weight. The effect of \verb|\textvt{}| and \verb|\vtstyle|, however, will depend on the active font. If the current font is the serif, the small-caps serif font in demibold will be loaded regardless of the current weight and width. If the current font is the sans, then either the bold small-caps sans in regular width or the demibold small-caps sans in the ultra-condensed width will be loaded. The selection will depend on the active font series i.e.\ the current weight and width. For example:
	\begin{verbatim}
		\textvtl{Outline Small-Caps Serif Titling}\\
		\textvt{Small-Caps Serif Titling Condensed}\\
		\textsf{\textvt{Small-Caps Sans Titling Demibold Ultra-Condensed}}\\
		\textsf{\textbf{\textvt{Small-Caps Sans Titling Bold}}}
	\end{verbatim}
	produces:
	\begin{center}
		\textvtl{Outline Small-Caps Serif Titling}\\
		\textvt{Small-Caps Serif Titling Condensed}\\
		\textsf{\textvt{Small-Caps Sans Titling Demibold Ultra-Condensed}}\\
		\textsf{\textbf{\textvt{Small-Caps Sans Titling Bold}}}
	\end{center}
	
\section{Mathematics}

The Venturis \adf\ Collection does not include fonts for mathematics and \lpack{venturis}, \lpack{venturis2}\ and \lpack{venturisold}\ do not touch the default settings for mathematical fonts. This means that you will probably find mathematics typeset in Computer Modern by default and this is likely not what you want.

Assuming that you are using \lpack{venturis}\ for text, Hirwen Harendal, who designed the collection, advises that the best choice for mathematics is \fgroup{fourier}\ and the second best is \fgroup{mathdesign}.

I am grateful to G\"unter Milde, who tested an earlier version of the packages, was responsible for the correction of many errors on my part and offered advise regarding the use of fonts in the collection for documents containing mathematics. He noted that combining \lpack{venturis}\ with Fourier Maths works quite nicely since \fgroup{venturis}, \fgroup{venturissans}\ and \fgroup{fourier}\ are all derived from Adobe's Utopia, but that neither \lpack{venturis2}\ (i.e.~\fgroup{venturis2} and \fgroup{venturissans2}) nor \lpack{venturisold}\ lend themselves to mathematical texts since no suitably complementary fonts for mathematics are available.

Milde suggests one of the following combinations:
\begin{itemize}
	\item Venturis, Venturis Sans \& Fourier Maths:
		\begin{verbatim}
			\usepackage{fourier}			% Fourier (sets text and math fonts)
			\usepackage[lf]{venturis}	% Venturis with lining figures
		\end{verbatim}
	\item Venturis, Venturis Sans \& Mathdesign:
		\begin{verbatim}
			\usepackage[utopia,expert]{mathdesign}
			\usepackage[lf]{venturis}  % Venturis with lining figures
		\end{verbatim}
\end{itemize}

Milde offers the following notes and suggestions:
\begin{itemize}
	\item If oldstyle figures are used (the default when \lpack{venturis}\ is loaded without options), these are also used in equation numbering, witch looks out of place.
	\item There are slight differences in both letters and figures, so you should take care to use text style figures and math figures as well as \verb|mathrm| and \verb|textrm| (for text-in-math) consistently.
\end{itemize}

Milde also suggested that the best match is achieved by choosing the extended width with the \verb|\etwidth| command but that even Venturis extended is narrower than Fourier normal. If you are considering this, however, you should note that the command will only affect text in sans-serif as Venturis, unlike Venturis Sans, does not include an extended width.

\end{document}  

