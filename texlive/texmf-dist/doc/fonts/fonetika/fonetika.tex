\documentclass[a4paper,11pt,british,DIVcalc]{scrartcl}
\usepackage[T1]{fontenc,url}
\usepackage[utf8]{inputenc}
\usepackage{babel,fonetika,listings}
\usepackage[sc,osf]{mathpazo}
\usepackage[scaled]{luximono}
\author{Palle Jørgensen}
\renewcommand*\sfdefault{iwonac}
\title{The fonetika package}
\renewcommand*\textsl[1]{{\fontfamily{ppl}\slshape #1}}
\urlstyle{rm}
\newcommand*\nooldstyle[1]{{\fontfamily{ppl}\selectfont #1}}
\newcommand*\TI{\nooldstyle{T1}}
\lstset{breaklines=true,language=[latex]tex}

\begin{document}

\maketitle  
\section{Introduction}
\label{sec:english-summary}

The Fonetika fonts is a set of fonts designed to be used for the
danish \emph{Dania}\,\cite{dania} phonetic alphabet. 

There are two fonts; a serif font \emph{Fonetika Dania Pallae} based
on \textsl{URW Palladio} (Palatino), and a sans serif font
\emph{Fonetika Dania Iwonae} based on \textsl{Iwona Condendsed.} Both
fonts exist in a regular and a bold weight.\footnote{The fonts
  Palladio and Iwona Condensed are the fonts used in this document}

The fonts are made with FontForge\,\cite{fontforge}.

For easy typing, the special characters are placed on the places of
uppercase letters in the \TI{} encoding. Hence the fonetika package
requires the fontenc package with option \TI{}. This is loaded if not
loaded by the user. The encoding of the fonetika fonts is a sort of
``fake'' \TI{} encoding.

\section{User interface}
\label{sec:user-interface}

\subsection{Loading the package}
\label{sec:loading-package}

The fonetika package has two options, \texttt{serif} and \texttt{sans}
which determines whether the \string\fonetika\ command chooses the
serif or the the sans serif font. The option \texttt{serif} is default.

\subsection{Commands}
\label{sec:commands}

The basic command of the fonetika package is (surprise)
\string\fonetika\ which takes an argument, and typesets it with the
current fonetika-font.

\begin{tabular}{lll}
  chokolade & \verb+\fonetika{Sogo|laDe}+ & \fonetika{Sogo|laDe}
\end{tabular}

Furthermore there are two commands, \verb+\fonetikaserif+ and
\verb+\fonetikasans+, which works just like the \verb+\fonetika+
command, but selects the serif and sans serif fonts specificly. This
is mainly for use in section titles if they are set with another font
than the body text.

\subsection{Input and output}
\label{sec:input-output}

The tables below show the typed input and the phonetic characters it
produces.

\begingroup
\tabcolsep1pt
\parindent0pt
\catcode`!\active
\let!\fonetikafamily
\catcode`?\active
\let?\ttfamily
\begin{tabular}{@{}lllllllllllllllllllllllllllll}
  \noalign{\subsubsection*{Uppercase letters}}
  ?A&?B&?C&?D&?E&?F&?G&?H&?I&?J&?K&?L&?M&?N&?O&?P&?Q&?R&?S&?T&?U&?V&?W&?X&?Y&?Z&?Æ&?Ø&?Å\\
  !A&!B&!C&!D&!E&!F&!G&!H&!I&!J&!K&!L&!M&!N&!O&!P&!Q&!R&!S&!T&!U&!V&!W&!X&!Y&!Z&!Æ&!Ø&!Å\bigskip\\
  \noalign{\subsubsection*{Lowercase letters}}
  ?a&?b&?c&?d&?e&?f&?g&?h&?i&?j&?k&?l&?m&?n&?o&?p&?q&?r&?s&?t&?u&?v&?w&?x&?y&?z&?æ&?ø&?å\\
  !a&!b&!c&!d&!e&!f&!g&!h&!i&!j&!k&!l&!m&!n&!o&!p&!q&!r&!s&!t&!u&!v&!w&!x&!y&!z&!æ&!ø&!å\bigskip\\
  \noalign{\subsubsection*{Other characters}}
  ?|&?'&?[&?]&?,&?.&?0\\
  !|&!'&![&!]&!,&!.&!0\bigskip\\
\end{tabular}
\endgroup

\section{License}
\label{sec:license}

The license of the fonetika fonts is Gnu General Public License, GPL\cite{gpl},
exept the fonts and metrics of the Fonetika Dania Iwonae. The license
of those is GUST Font Nosource License, \cite{gust}.

\begin{thebibliography}{9}
\bibitem{dania} \url{http://da.wikipedia.org/wiki/Dania}
\bibitem{fontforge} \url{http://sourceforge.net/projects/fontforge/}
\bibitem{gpl} \url{http://www.gnu.org/licenses/gpl.txt}
\bibitem{gust} \url{http://tug.org/fonts/licenses/GUST-FONT-NOSOURCE-LICENSE.txt}
\end{thebibliography}
\appendix
\clearpage
\section{Source of the fonetika files}
\label{sec:source-fonet-files}

\subsection{Source of fonetika.sty}
\label{sec:source-fonetika.sty}
\lstinputlisting{fonetika.sty}

\subsection{Source of t1fonetika.fd}
\label{sec:source-t1fonetika.fd}
\lstinputlisting{t1fonetika.fd}

\subsection{Source of fonetika.map}
\label{sec:source-fonetika.map}
\lstinputlisting{fonetika.map}


\end{document}

%%% Local Variables: 
%%% mode: latex
%%% TeX-master: t
%%% End: 
