\documentclass[10pt]{article}

\usepackage{a4wide,amsmath,pxfonts}
\normalfont
%\usepackage[T1]{fontenc}
%\usepackage{textcomp}
\let\orgnonumber=\nonumber\usepackage{mathenv}\let\nonumb=\nonumber\let\nonumber=\orgnonumber

\allowdisplaybreaks

\newcommand{\bs}{\symbol{'134}}


\def\Ent#1{\csname #1\endcsname & \texttt{\bs #1}}
\def\EEnt#1#2{\csname #1\endcsname & \texttt{\bs #1},\,\texttt{\bs #2}}

\makeatletter
\newcount\curchar \newcount\currow \newcount\curcol
\newdimen\indexwd \newdimen\tempcellwd
\setbox0\hbox{\ttfamily0\kern.2em}
\indexwd=\wd0

\def\ident#1{#1}
\def\hexnumber#1{\ifcase\expandafter\ident\expandafter{\number#1} 0\or
1\or 2\or 3\or 4\or 5\or 6\or 7\or 8\or 9\or A\or B\or C\or D\or E\or
F\else ?\fi}

\def\rownumber{\ttfamily\hexnumber\currow}
\def\colnumber{\ttfamily\hexnumber\curcol \global\advance\curcol 1 }

\def\charnumber{\setbox0=\hbox{\char\curchar}%
  \ifdim\ht0>7.5pt\reposition
  \else\ifdim\dp0>2.5pt\reposition\fi\fi
  \box0 \global\advance\curchar1 }
\def\reposition{\setbox0=\hbox{$\vcenter{\kern1.5pt\box0\kern1.5pt}$}}

\def\dochart#1{%
  \begingroup
  \global\curchar=0 \global\currow=0 \global\curcol=0
  \def\hline{\kern2pt\hrule\kern3pt }%
  \setbox0\vbox{#1%
    \def\0{\hbox to\cellwd{\curcol}{\hss\charnumber\hss}}%
    \colnumbers
    \hline
    \setrow\setrow\setrow\setrow
    \hline
    \setrow\setrow\setrow\setrow
    \hline
    \colnumbers
  }%
  \vbox{%
    \hbox to\hsize{\kern\indexwd
      \def\fullrule{\hfil\vrule height\ht0 depth\dp0\hfil}%
      \fullrule\kern\cellwd{0}\kern\cellwd{1}\kern\cellwd{2}\kern\cellwd{3}%
      \fullrule\kern\cellwd{4}\kern\cellwd{5}\kern\cellwd{6}\kern\cellwd{7}%
      \fullrule\kern\cellwd{8}\kern\cellwd{9}\kern\cellwd{10}\kern\cellwd{11}%
      \fullrule\kern\cellwd{12}\kern\cellwd{13}\kern\cellwd{14}\kern\cellwd{15}%
      \fullrule\kern\indexwd}%
    \kern-\ht0 \kern-\dp0 \unvbox0}%
  \endgroup
}

\def\dochartA#1{%
  \begingroup
  \global\curchar=0 \global\currow=0 \global\curcol=0
  \def\hline{\kern2pt\hrule\kern3pt }%
  \setbox0\vbox{#1%
    \def\0{\hbox to\cellwd{\curcol}{\hss\charnumber\hss}}%
    \colnumbers
    \hline
    \setrow\setrow\setrow\setrow
    \hline
    \setrow\setrow\setrow\setrow
    \hline
    \setrow\setrowX\setrow\setrowX %
%    \hline %
    \setrow\setrowX\setrow\setrowX %
    \hline %
    \colnumbers
  }%
  \vbox{%
    \hbox to\hsize{\kern\indexwd
      \def\fullrule{\hfil\vrule height\ht0 depth\dp0\hfil}%
      \fullrule\kern\cellwd{0}\kern\cellwd{1}\kern\cellwd{2}\kern\cellwd{3}%
      \fullrule\kern\cellwd{4}\kern\cellwd{5}\kern\cellwd{6}\kern\cellwd{7}%
      \fullrule\kern\cellwd{8}\kern\cellwd{9}\kern\cellwd{10}\kern\cellwd{11}%
      \fullrule\kern\cellwd{12}\kern\cellwd{13}\kern\cellwd{14}\kern\cellwd{15}%
      \fullrule\kern\indexwd}%
    \kern-\ht0 \kern-\dp0 \unvbox0}%
  \endgroup
}

\def\dochartB#1{%
  \begingroup
  \global\curchar=0 \global\currow=0 \global\curcol=0
  \def\hline{\kern2pt\hrule\kern3pt }%
  \setbox0\vbox{#1%
    \def\0{\hbox to\cellwd{\curcol}{\hss\charnumber\hss}}%
    \colnumbers
    \hline
    \setrow\setrow\setrow\setrow
    \hline
    \setrow\setrow%\setrow\setrow
    \hline
    \colnumbers
  }%
  \vbox{%
    \hbox to\hsize{\kern\indexwd
      \def\fullrule{\hfil\vrule height\ht0 depth\dp0\hfil}%
      \fullrule\kern\cellwd{0}\kern\cellwd{1}\kern\cellwd{2}\kern\cellwd{3}%
      \fullrule\kern\cellwd{4}\kern\cellwd{5}\kern\cellwd{6}\kern\cellwd{7}%
      \fullrule\kern\cellwd{8}\kern\cellwd{9}\kern\cellwd{10}\kern\cellwd{11}%
      \fullrule\kern\cellwd{12}\kern\cellwd{13}\kern\cellwd{14}\kern\cellwd{15}%
      \fullrule\kern\indexwd}%
    \kern-\ht0 \kern-\dp0 \unvbox0}%
  \endgroup
}

\def\dochartC#1{%
  \begingroup
  \global\curchar=0 \global\currow=0 \global\curcol=0
  \def\hline{\kern2pt\hrule\kern3pt }%
  \setbox0\vbox{#1%
    \def\0{\hbox to\cellwd{\curcol}{\hss\charnumber\hss}}%
    \colnumbers
    \hline
    \setrow\setrow\setrow\setrow
    \hline
    \setrow\setrow\setrow\setrow
    \hline
    \setrow\setrow
    \hline
    \colnumbers
  }%
  \vbox{%
    \hbox to\hsize{\kern\indexwd
      \def\fullrule{\hfil\vrule height\ht0 depth\dp0\hfil}%
      \fullrule\kern\cellwd{0}\kern\cellwd{1}\kern\cellwd{2}\kern\cellwd{3}%
      \fullrule\kern\cellwd{4}\kern\cellwd{5}\kern\cellwd{6}\kern\cellwd{7}%
      \fullrule\kern\cellwd{8}\kern\cellwd{9}\kern\cellwd{10}\kern\cellwd{11}%
      \fullrule\kern\cellwd{12}\kern\cellwd{13}\kern\cellwd{14}\kern\cellwd{15}%
      \fullrule\kern\indexwd}%
    \kern-\ht0 \kern-\dp0 \unvbox0}%
  \endgroup
}

\def\dochartD#1{%
  \begingroup
  \global\curchar=0 \global\currow=0 \global\curcol=0
  \def\hline{\kern2pt\hrule\kern3pt }%
  \setbox0\vbox{#1%
    \def\0{\hbox to\cellwd{\curcol}{\hss\charnumber\hss}}%
    \colnumbers
    \hline
    \setrow\setrow\setrow\setrow
    \hline
    \setrow\setrow\setrow\setrow
    \hline
    \setrow\setrow\setrow\setrow
    \hline
    \setrow\setrow\setrow\setrow
    \hline
    \colnumbers
  }%
  \vbox{%
    \hbox to\hsize{\kern\indexwd
      \def\fullrule{\hfil\vrule height\ht0 depth\dp0\hfil}%
      \fullrule\kern\cellwd{0}\kern\cellwd{1}\kern\cellwd{2}\kern\cellwd{3}%
      \fullrule\kern\cellwd{4}\kern\cellwd{5}\kern\cellwd{6}\kern\cellwd{7}%
      \fullrule\kern\cellwd{8}\kern\cellwd{9}\kern\cellwd{10}\kern\cellwd{11}%
      \fullrule\kern\cellwd{12}\kern\cellwd{13}\kern\cellwd{14}\kern\cellwd{15}%
      \fullrule\kern\indexwd}%
    \kern-\ht0 \kern-\dp0 \unvbox0}%
  \endgroup
}

\def\dochartE#1{%
  \begingroup
  \global\curchar=0 \global\currow=0 \global\curcol=0
  \def\hline{\kern2pt\hrule\kern3pt }%
  \setbox0\vbox{#1%
    \def\0{\hbox to\cellwd{\curcol}{\hss\charnumber\hss}}%
    \colnumbers
    \hline
    \setrow\setrow\setrow\setrow
    \hline
    \setrow\setrow\setrow\setrow
    \hline
    \setrow\setrow\setrow\setrow
    \hline
    \setrowX\setrow\setrowX\setrow
    \hline
    \colnumbers
  }%
  \vbox{%
    \hbox to\hsize{\kern\indexwd
      \def\fullrule{\hfil\vrule height\ht0 depth\dp0\hfil}%
      \fullrule\kern\cellwd{0}\kern\cellwd{1}\kern\cellwd{2}\kern\cellwd{3}%
      \fullrule\kern\cellwd{4}\kern\cellwd{5}\kern\cellwd{6}\kern\cellwd{7}%
      \fullrule\kern\cellwd{8}\kern\cellwd{9}\kern\cellwd{10}\kern\cellwd{11}%
      \fullrule\kern\cellwd{12}\kern\cellwd{13}\kern\cellwd{14}\kern\cellwd{15}%
      \fullrule\kern\indexwd}%
    \kern-\ht0 \kern-\dp0 \unvbox0}%
  \endgroup
}
\def\colnumbers{\hbox to\hsize{\global\curcol 0
  \def\1{\hbox to\cellwd{\curcol}{\hfil\colnumber\hfil}}%
  \kern\indexwd\hfil\hfil
  \1\1\1\1\hfil\hfil \1\1\1\1\hfil\hfil
  \1\1\1\1\hfil\hfil \1\1\1\1\hfil\hfil
  \kern\indexwd}%
}

\def\dochartF#1{%
  \begingroup
  \global\curchar=0 \global\currow=0 \global\curcol=0
  \def\hline{\kern2pt\hrule\kern3pt }%
  \setbox0\vbox{#1%
    \def\0{\hbox to\cellwd{\curcol}{\hss\charnumber\hss}}%
    \colnumbers
    \hline
    \setrow\setrow\setrow\setrow
    \hline
    \setrow\setrow\setrow\setrow
    \hline
    \setrow\setrow\setrow
    \hline
    \colnumbers
  }%
  \vbox{%
    \hbox to\hsize{\kern\indexwd
      \def\fullrule{\hfil\vrule height\ht0 depth\dp0\hfil}%
      \fullrule\kern\cellwd{0}\kern\cellwd{1}\kern\cellwd{2}\kern\cellwd{3}%
      \fullrule\kern\cellwd{4}\kern\cellwd{5}\kern\cellwd{6}\kern\cellwd{7}%
      \fullrule\kern\cellwd{8}\kern\cellwd{9}\kern\cellwd{10}\kern\cellwd{11}%
      \fullrule\kern\cellwd{12}\kern\cellwd{13}\kern\cellwd{14}\kern\cellwd{15}%
      \fullrule\kern\indexwd}%
    \kern-\ht0 \kern-\dp0 \unvbox0}%
  \endgroup
}


\def\setrow{\hbox to\hsize{%
  \hbox to\indexwd{\hfil\rownumber\kern.2em}\hfil\hfil
  \0\0\0\0\hfil\hfil \0\0\0\0\hfil\hfil
  \0\0\0\0\hfil\hfil \0\0\0\0\hfil\hfil
  \hbox to\indexwd{\ttfamily\kern.2em \rownumber\hfil}}%
  \global\advance\currow 1 }%

\def\setrowX{\global\advance\curchar16\global\advance\currow 1\relax}

\def\cellwd#1{20pt}% initialize

\def\measurecolwidths#1{%
  \tempcellwd\hsize \advance\tempcellwd-2\indexwd
  \advance\tempcellwd -12pt
  \divide\tempcellwd 16
  \xdef\cellwd##1{\the\tempcellwd}%
}

\def \table #1#2#3{\par\penalty-200 \bigskip
  \font #1=#2 \relax
  \vbox{\hsize=29pc
    \measurecolwidths{#1}%
    \centerline{#3 -- {\tt#2}}%
    \medskip
    \dochart{#1}%
}}


\def \tableA #1#2#3{\par\penalty-200 \bigskip
  \font #1=#2 \relax
  \vbox{\hsize=29pc
    \measurecolwidths{#1}%
    \centerline{#3 -- {\tt#2}}%
    \medskip
    \dochartA{#1}%
}}

\def \tableB #1#2#3{\par\penalty-200 \bigskip
  \font #1=#2 \relax
  \vbox{\hsize=29pc
    \measurecolwidths{#1}%
    \centerline{#3 -- {\tt#2}}%
    \medskip
    \dochartB{#1}%
}}

\def \tableC #1#2#3{\par\penalty-200 \bigskip
  \font #1=#2 \relax
  \vbox{\hsize=29pc
    \measurecolwidths{#1}%
    \centerline{#3 -- {\tt#2}}%
    \medskip
    \dochartC{#1}%
}}

\def \tableD #1#2#3{\par\penalty-200 \bigskip
  \font #1=#2 \relax
  \vbox{\hsize=29pc
    \measurecolwidths{#1}%
    \centerline{#3 -- {\tt#2}}%
    \medskip
    \dochartD{#1}%
}}

\def \tableE #1#2#3{\par\penalty-200 \bigskip
  \font #1=#2 \relax
  \vbox{\hsize=29pc
    \measurecolwidths{#1}%
    \centerline{#3 -- {\tt#2}}%
    \medskip
    \dochartE{#1}%
}}

\def \tableF #1#2#3{\par\penalty-200 \bigskip
  \font #1=#2 \relax
  \vbox{\hsize=29pc
    \measurecolwidths{#1}%
    \centerline{#3 -- {\tt#2}}%
    \medskip
    \dochartF{#1}%
}}

\makeatother

\begin{document}

\title{The \texttt{PX} Fonts}

\author{Young Ryu}

\date{December 14, 2000}

\maketitle

\tableofcontents

\clearpage
\section{Introduction}

The \texttt{PX} fonts consist of
\begin{enumerate}\itemsep=0pt
\item virtual text roman fonts using Adobe Palatino (or URWPalladioL) with
      some modified and additional text symbols in the OT1, T1, and TS1 encoding
      %, T1, TS1, and LY1 encodings
\item \textsf{virtual text sans serif fonts using Adobe Helvetica (or URW NimbusSanL) with
      additional text symbols in OT1, T1, TS1, and LY1 encodings}
      (Provided in the \texttt{TX} fonts distribution)
\item \texttt{monospaced typewriter fonts in OT1, T1, TS1, and LY1 encodings}
      (Provided in the \texttt{TX} fonts distribution)
\item math alphabets using Adobe Palatino (or URWPalladioL)
      with modified metrics
\item math fonts of all symbols corresponding to those of Computer Modern
      math fonts (CMSY, CMMI, CMEX, and Greek letters of CMR)
\item math fonts of all symbols corresponding to those of \AmS\ fonts
      (MSAM and MSBM)
\item additional math fonts of various symbols
\end{enumerate}
%
All fonts are in the Type 1 format (in \texttt{afm} and \texttt{pfb} files).
Necessary \texttt{tfm} and \texttt{vf} files together with
\LaTeXe\ package files and font map files for \texttt{dvips} are
provided.

\begin{bfseries}%\itshape
The \texttt{PX} fonts and related files are distributed 
without any guaranty or warranty.
I do not assume responsibility for any actual or possible
damages or losses, directly or indirectly caused by the
distributed files.
\end{bfseries}
The \texttt{PX} fonts are distributed under the GNU public license (GPL)\@.
The fonts will be improved and additional glyphs will be added
in the future.

\section{Requirements}

Since \textsf{sans serif fonts based on Adobe Helvetica (or URW NimbusSanL)}
and \texttt{monospaced typewriter fonts} of the \texttt{TX} fonts are to be
used with the \texttt{PX} fonts, one must get and properly install
the \texttt{TX} fonts, which are available from CTAN
(or \verb|www.utdallas.edu/~ryoung/txfonts|).

\section{Changes}

\begin{description}
\item[0.1] (November 30, 2000) 1st public release
\item[0.2] (Decemver 4, 2000)
    \begin{itemize}
    \item Redesign of various math symbols to be more consistent with Palatino text fonts.
    \item Improved Metrics
    \end{itemize}
\item[0.3] (Decemver 7, 2000)
    \begin{itemize}
    \item More large operators symbols
    \item Now \verb|\lbag| ($\lbag$) and \verb|\rbag| ($\rbag$) are
          delimiters.
    \item An alternative math italic $\varg$ (produced by \verb|$\varg$|)
    \end{itemize}
\item[0.4] (Decemver 12, 2000)
    \begin{itemize}
    \item T1 and TS1 encodings supported
    \item Various bugs fixed
    \end{itemize}
\item[1.0] (November 14, 2000)
     \begin{itemize}
     \item Minor problem fixes.
     \item Hopefully, this is the final version ...
     \end{itemize}
\end{description}

\section{A Problem: \texttt{DVIPS} Partial Font Downloading}

It was reported that when \texttt{PX} fonts
are partially downloaded with \texttt{dvips},
some HP Laserprinters (with Postscript) cannot
print documents. To resolve this problem,
turn the partial font downloading off.
See the \texttt{dvips} document for various ways to
turn off partial font downloading.

\textbf{\itshape Even though one does not observe such a problem,
I would like to strongly recommend to turn off \texttt{dvips}
partial font downloading.}

\section{Installation}

Put all files in \texttt{afm}, \texttt{tfm}, \texttt{vf},
and \texttt{pfb} files in proper locations of your \TeX\ system.
For Mik\TeX, they may go
\begin{verbatim}
     \localtexmf\fonts\afm\pxr\
     \localtexmf\fonts\tfm\pxr\
     \localtexmf\fonts\vf\pxr\
     \localtexmf\fonts\type1\pxr\
\end{verbatim}
The all files of the \texttt{input} directory must
be placed where \LaTeX\ finds its package files.
For Mik\TeX, they may go
\begin{verbatim}
     \localtexmf\tex\latex\pxr\
\end{verbatim}
Put the \texttt{pxr.map}, \texttt{pxr1.map}, and \texttt{pxr2.map} %, and \texttt{tx8r.enc}
files of the \texttt{dvips}
directory in a proper place that \texttt{dvips} refers to.
For Mik\TeX, they may go
\begin{verbatim}
     \localtexmf\dvips\config\
\end{verbatim}
Also add the reference to \texttt{pxr2.map} in
the \texttt{dvips} configuration file (\texttt{config.ps})
\begin{verbatim}
     . . .
     % Configuration of postscript type 1 fonts:
     p psfonts.map
     p +pxr2.map
     . . .
\end{verbatim}
and in the PDF\TeX\ configuration file (\texttt{pdftex.cfg})
\begin{verbatim}
     . . .
     % pdftex.map is set up by texmf/dvips/config/updmap
     map pdftex.map
     map +pxr2.map
     . . .
\end{verbatim}
(The \texttt{pxr.map} file has only named references to the Adobe Palatino fonts;
the \texttt{pxr1.map} file makes \texttt{dvips} load Adobe Palatino font files;
and the \texttt{pxr2.map}  file makes \texttt{dvips} load URWPalladioL font files.)
Be sure to get URWPalladioL fonts included in the recent Ghostscript
distribution and properly install them in your \texttt{texmf} tree.
If you have the real Adobe Palatino font files, put
\texttt{pxr1.map} instead of \texttt{pxr2.map} in
\texttt{dvips} and PDF\TeX\ configuration files.

\section{Using the \texttt{PX} Fonts with \LaTeX}

It is as simple as
\begin{verbatim}
     \documentclass{article}
     \usepackage{pxfonts}

     \begin{document}

     This is a very short article.

     \end{document}
\end{verbatim}

%The standard \LaTeX\ distribution does not include
%files supporting the LY1 encoding.
%One needs at least \texttt{ly1enc.def}, which is available
%from both CTAN and Y\&Y (\texttt{www.yandy.com}).
%At the time this document was written, CTAN had
%an old version (1997/03/21 v0.3); \texttt{ly1enc.def}
%available from Y\&Y's downloads sites was dated on 1998/04/21 v0.4.

\section{Additional Symbols in the \texttt{PX} Math Fonts}

\emph{All} CM symbols are included in the \texttt{PX} math fonts.
In addition, the \texttt{PX} math fonts provide or modify
the following symbols, including all of \AmS\ and most of \LaTeX\ symbols.

\subsubsection*{Binary Operator Symbols}
\begin{eqnarray*}[c@{\enskip}l@{\qquad\qquad\qquad}c@{\enskip}l@{\qquad\qquad\qquad}c@{\enskip}l]
\Ent{medcirc}&
\Ent{medbullet}&
\Ent{invamp}\\
\Ent{circledwedge}&
\Ent{circledvee}&
\Ent{circledbar}\\
\Ent{circledbslash}&
\Ent{nplus}&
\Ent{boxast}\\
\Ent{boxbslash}&
\Ent{boxbar}&
\Ent{boxslash}\\
\Ent{Wr}&
\Ent{sqcupplus}&
\Ent{sqcapplus}\\
\Ent{rhd}&
\Ent{lhd}&
\Ent{unrhd}\\
\Ent{unlhd}
\end{eqnarray*}

\subsubsection*{Binary Relation Symbols}
\begin{eqnarray*}[c@{\enskip}l@{\quad}c@{\enskip}l@{\quad}c@{\enskip}l]
\Ent{mappedfrom}&
\Ent{longmappedfrom}&
\Ent{Mapsto}\\
\Ent{Longmapsto}&
\Ent{Mappedfrom}&
\Ent{Longmappedfrom}\\
\Ent{mmapsto}&
\Ent{longmmapsto}&
\Ent{mmappedfrom}\\
\Ent{longmmappedfrom}&
\Ent{Mmapsto}&
\Ent{Longmmapsto}\\
\Ent{Mmappedfrom}&
\Ent{Longmmappedfrom}&
\Ent{varparallel}\\
\Ent{varparallelinv}&
\Ent{nvarparallel}&
\Ent{nvarparallelinv}\\
\Ent{colonapprox}&
\Ent{colonsim}&
\Ent{Colonapprox}\\
\Ent{Colonsim}&
\Ent{doteq}&
\Ent{multimapinv}\\
\Ent{multimapboth}&
\Ent{multimapdot}&
\Ent{multimapdotinv}\\
\Ent{multimapdotboth}&
\Ent{multimapdotbothA}&
\Ent{multimapdotbothB}\\
\Ent{VDash}&
\Ent{VvDash}&
\Ent{cong}\\
\Ent{preceqq}&
\Ent{succeqq}&
\Ent{nprecsim}\\
\Ent{nsuccsim}&
\Ent{nlesssim}&
\Ent{ngtrsim}\\
\Ent{nlessapprox}&
\Ent{ngtrapprox}&
\Ent{npreccurlyeq}\\
\Ent{nsucccurlyeq}&
\Ent{ngtrless}&
\Ent{nlessgtr}\\
\Ent{nbumpeq}&
\Ent{nBumpeq}&
\Ent{nbacksim}\\
\Ent{nbacksimeq}&
\EEnt{neq}{ne}&
\Ent{nasymp}\\
\Ent{nequiv}&
\Ent{nsim}&
\Ent{napprox}\\
\Ent{nsubset}&
\Ent{nsupset}&
\Ent{nll}\\
\Ent{ngg}&
\Ent{nthickapprox}&
\Ent{napproxeq}\\
\Ent{nprecapprox}&
\Ent{nsuccapprox}&
\Ent{npreceqq}\\
\Ent{nsucceqq}&
\Ent{nsimeq}&
\Ent{notin}\\
\EEnt{notni}{notowns}&
\Ent{nSubset}&
\Ent{nSupset}\\
\Ent{nsqsubseteq}&
\Ent{nsqsupseteq}&
\Ent{coloneqq}\\
\Ent{eqqcolon}&
\Ent{coloneq}&
\Ent{eqcolon}\\
\Ent{Coloneqq}&
\Ent{Eqqcolon}&
\Ent{Coloneq}\\
\Ent{Eqcolon}&
\Ent{strictif}&
\Ent{strictfi}\\
\Ent{strictiff}&
\Ent{circledless}&
\Ent{circledgtr}\\
\Ent{lJoin}&
\Ent{rJoin}&
\EEnt{Join}{lrJoin}\\
\Ent{openJoin}&
\Ent{lrtimes}&
\Ent{opentimes}\\
\Ent{nsqsubset}&
\Ent{nsqsupset}&
\Ent{dashleftarrow}\\
%\EEnt{dashrightarrow}{dasharrow}&
\Ent{dashrightarrow}&
\Ent{dashleftrightarrow}&
\Ent{leftsquigarrow}\\
\Ent{ntwoheadrightarrow}&
\Ent{ntwoheadleftarrow}&
\Ent{Nearrow}\\
\Ent{Searrow}&
\Ent{Nwarrow}&
\Ent{Swarrow}\\
\Ent{Perp}&
\Ent{leadstoext}&
\Ent{leadsto}\\
\Ent{boxright}&
\Ent{boxleft}&
\Ent{boxdotright}\\
\Ent{boxdotleft}&
\Ent{Diamondright}&
\Ent{Diamondleft}\\
\Ent{Diamonddotright}&
\Ent{Diamonddotleft}&
\Ent{boxRight}\\
\Ent{boxLeft}&
\Ent{boxdotRight}&
\Ent{boxdotLeft}\\
\Ent{DiamondRight}&
\Ent{DiamondLeft}&
\Ent{DiamonddotRight}\\
\Ent{DiamonddotLeft}&
\Ent{circleright}&
\Ent{circleleft}\\
\Ent{circleddotright}&
\Ent{circleddotleft}&
\Ent{multimapbothvert}\\
\Ent{multimapdotbothvert}&
\Ent{multimapdotbothAvert}&
\Ent{multimapdotbothBvert}
\end{eqnarray*}

\subsubsection*{Ordinary Symbols}
\begin{eqnarray*}[c@{\enskip}l@{\qquad\qquad\qquad}c@{\enskip}l@{\qquad\qquad\qquad}c@{\enskip}l]
\Ent{alphaup}&
\Ent{betaup}&
\Ent{gammaup}\\
\Ent{deltaup}&
\Ent{epsilonup}&
\Ent{varepsilonup}\\
\Ent{zetaup}&
\Ent{etaup}&
\Ent{thetaup}\\
\Ent{varthetaup}&
\Ent{iotaup}&
\Ent{kappaup}\\
\Ent{lambdaup}&
\Ent{muup}&
\Ent{nuup}\\
\Ent{xiup}&
\Ent{piup}&
\Ent{varpiup}\\
\Ent{rhoup}&
\Ent{varrhoup}&
\Ent{sigmaup}\\
\Ent{varsigmaup}&
\Ent{tauup}&
\Ent{upsilonup}\\
\Ent{phiup}&
\Ent{varphiup}&
\Ent{chiup}\\
\Ent{psiup}&
\Ent{omegaup}&
\Ent{Diamond}\\
\Ent{Diamonddot}&
\Ent{Diamondblack}&
\Ent{lambdaslash}\\
\Ent{lambdabar}&
\Ent{varclubsuit}&
\Ent{vardiamondsuit}\\
\Ent{varheartsuit}&
\Ent{varspadesuit}&
\Ent{Top}\\
\Ent{Bot}
\end{eqnarray*}

\subsubsection*{Math Alphabets}

\begin{eqnarray*}[c@{\enskip}l]
\Ent{varg}
\end{eqnarray*}
In order to replace math alphabet $g$ by this alternative,
use the \texttt{varg} option with the \texttt{pxfonts} package:
\begin{verbatim}
     \usepackage[varg]{pxfonts}
\end{verbatim}
Then, \verb|$g$| will produce $\varg$ (instead of $g$).

\subsubsection*{Large Operator Symbols}

\begin{eqnarray*}[c@{\enskip}l@{\quad}c@{\enskip}l@{\quad}c@{\enskip}l]
\Ent{bignplus}&
\Ent{bigsqcupplus}&
\Ent{bigsqcapplus}\\
\Ent{bigsqcap}&
\Ent{bigsqcap}&
\Ent{varprod}\\
\Ent{oiint}&
\Ent{oiiint}&
\Ent{ointctrclockwise}\\
\Ent{ointclockwise}&
\Ent{varointctrclockwise}&
\Ent{varointclockwise}\\
\Ent{sqint}&
\Ent{sqiintop}&
\Ent{sqiiintop}\\
\Ent{fint}&
\Ent{iint}&
\Ent{iiint}\\
\Ent{iiiint}&
\Ent{idotsint}&
\Ent{oiintctrclockwise}\\
\Ent{oiintclockwise}&
\Ent{varoiintctrclockwise}&
\Ent{varoiintclockwise}\\
\Ent{oiiintctrclockwise}&
\Ent{oiiintclockwise}&
\Ent{varoiiintctrclockwise}\\
\Ent{varoiiintclockwise}&
\end{eqnarray*}

\subsubsection*{Delimiters}
\begin{eqnarray*}[c@{\enskip}l@{\qquad\qquad\qquad}c@{\enskip}l@{\qquad\qquad\qquad}c@{\enskip}l@{\qquad\qquad\qquad}c@{\enskip}l]
\Big\llbracket&\texttt{\bs llbracket}&
\Big\rrbracket&\texttt{\bs rrbracket}&
\Big\lbag&\texttt{\bs lbag}&
\Big\rbag&\texttt{\bs rbag}
\end{eqnarray*}

%\subsubsection*{Parentheses}
%\begin{eqnarray*}[c@{\enskip}l@{\qquad\qquad\qquad}c@{\enskip}l@{\qquad\qquad\qquad}c@{\enskip}l@{\qquad\qquad\qquad}c@{\enskip}l]
%\Ent{lbag}&
%\Ent{rbag}&
%\Ent{Lbag}&
%\Ent{Rbag}
%\end{eqnarray*}

\subsubsection*{Miscellaneous}

\verb|$\mathfrak{...}$| produces
$\mathfrak{A} \ldots \mathfrak{Z}$ and $\mathfrak{a} \ldots \mathfrak{z}$.
%
\verb|$\mathbb{...}$| produces $\mathbb{A} \ldots \mathbb{Z}$;
\verb|$\Bbbk$| produces $\Bbbk$.


\section{Remarks}

\subsection{Some Font Design Issues}

For negated relation symbols, the CM fonts composes
relation symbols with the negation slash (\texttt{"36} in CMSY).
Even though the CM fonts were very carefully designed
to look reasonable when negated relation symbols are composed
(except `$\notin$' \verb|\notin|, which is composed of
`$\in$' and the normal slash `$/$'),
the \AmS\ font set includes many negated relation symbols,
mainly because the vertical placement and
height\slash depth of the negation slash are not optimal
when composed with certain relation symbols, I guess.
The \texttt{PX} fonts include the negation slash symbol
(\texttt{"36} in pxsy), which could be composed with
relation symbols to give reasonably looking negated related symbols.
I believe, however, explicitly designed negated relation symbols
are looking better than composed relation symbols.
Thus, in addition to negated relation symbols matching those of
the \AmS\ fonts, many negated symbols such as `$\neq$' are introduced
in the \texttt{PX} fonts.

Further, in order to maintain editing compatibility with
vanilla \LaTeXe\ typesetting, \verb|\not| is redefined in \texttt{pxfonts.sty} 
so that when \verb|\not\XYZ| is processed,
if \verb|\notXYZ| or \verb|\nXYZ| is defined, it will be used
in place of \verb|\not\XYZ|; otherwise,
\verb|\XYZ| is composed with the negation slash.
For instance, `$\nprecsim$' is available as \verb|\nprecsim| in the \texttt{PX} fonts.
Thus, if \verb|\not\precsim| is typed in the document,
the \verb|\nprecsim| symbol, instead of \verb|\precsim| composed
with the negation slash, is printed.

\subsection{Glyph Hinting}

The hinting of the \texttt{PX} fonts is far from ideal.
As a result, when documents with the \texttt{PX} fonts
are \emph{viewed} with Gsview (or Ghostview), you might notice
some display quality problem. When they are \emph{viewed}
with Acrobat, they look much better.
However, when they are \emph{printed} in laser printers,
there will be no quality problem.
(Note, hinting is to improve display quality on low resolution devices such as
display screens.)

\subsection{Glyphs in Low Positions}

It is known that Acrobat often does not properly handle
CM font glyphs placed between \texttt{"00} and \texttt{"1F}.
Thus, most Type 1 versions of CM fonts publicly available
have these glyphs in higher positions above \texttt{"7F}.
When the \texttt{-G} flag is used with \texttt{dvips},
those glyphs in low positions are shifted to higher positions.
The \texttt{PX} text fonts have
glyphs in the low positions between \texttt{"00} and \texttt{"1F}.
As of now, these glyphs are not available in higher positions above \texttt{"7F}.
Thus, when run \texttt{dvips}, do not use the \texttt{-G}
flag (or remove \texttt{G} in the \texttt{dvips} configuration file).
Especially, do not use \texttt{config.pdf}.
In my computer systems, Acrobat correctly handles glyphs in low positions.
However, if this known Acrobat problem occurs in other computer systems,
I will modify the \texttt{PX} fonts so that glyphs in low positions
are also available in higher positions.


\section{Font Charts}

The original Computer Modern (CM) text fonts (aka \TeX\ text fonts)
have the OT1 encoding. The OT1 \texttt{PX} text fonts follow
the CM fonts' encoding as much as possible, but have some
variations and additions:
\begin{itemize}\parskip=0pt\itemsep=0pt
\item The position \texttt{"24} of text italic fonts has
      the dollar symbol (\textit{\textdollar}), not the sterling symbol (\textit{\textsterling}).
\item The uppercase and lowercase lslash (\L, \l) and aring (\AA, \aa) letters are added.
\item The cent (\ifx\textcentoldstyle\undefined\textcent\else\textcentoldstyle\fi)
      and sterling (\textsterling) symbols are added.
\end{itemize}
The original CM text fonts have somewhat different encodings in
\textsc{cap \& small cap} and \texttt{typewriter} fonts.
\texttt{PX} fonts corresponding to them have the original CM encodings,
not the strict OT1 encoding.

The T1 encoding text fonts (known as EC fonts) are designed to
replace the CM text fonts in the OT1 encoding.
The LY1 encoding is another text font encoding, which is based
on both \TeX\ and ANSI encodings.
Both T1 and LY1 encoding fonts are especially useful to typeset
European languages with proper hyphenation.
The TS1 encoding text companion fonts (known as TC fonts) have
additional text symbols.
All corresponding \texttt{PX} fonts are implemented.

The Computer Modern (CM) math fonts (aka \TeX\ math fonts)
consist of three fonts: math italic (CMMI), math symbols (CMSY), and
math extension (CMEX). The American Mathematical Society provided
two additional math symbol fonts (MSAM and MSBM).
The \texttt{PX} math fonts include those exactly corresponding to them.
In addition, the \texttt{PX} math fonts include math italic A,
math symbols C, and math extension A fonts.


\subsection{OT1 (CM) Encoding Text Fonts}

These fonts' encodings are identical to those of corresponding CM fonts,
except 6~additional glyphs.

\begin{center}
\centering
\leavevmode\hbox{\tableA \fonttab{pxr}{Text Roman Upright}}

\bigskip\bigskip
\leavevmode\hbox{\tableA \fonttab{pxi}{\textit{Text Roman Italic}}}

\bigskip\bigskip
\leavevmode\hbox{\tableA \fonttab{pxsl}{\textsl{Text Roman Slanted}}}

\bigskip\bigskip
\leavevmode\hbox{\tableA \fonttab{pxsc}{\textsc{Text Roman Cap \& Small Cap}}}

%\bigskip\bigskip
%\leavevmode\hbox{\tableA \fonttab{txss}{\textsf{Text Sans Serif Upright}}}

%\bigskip\bigskip
%\leavevmode\hbox{\tableA \fonttab{txsssl}{\textsf{\slshape Text Sans Serif Slanted}}}

%\bigskip\bigskip
%\leavevmode\hbox{\tableA \fonttab{txsssc}{\textsf{\scshape Text Sans Serif Cap \& Small Cap}}}

%\bigskip\bigskip
%\leavevmode\hbox{\tableA \fonttab{txtt}{\texttt{Text Typewriter Upright}}}

%\bigskip\bigskip
%\leavevmode\hbox{\tableA \fonttab{txttsl}{\texttt{\slshape Text Typewriter Slanted}}}

%\bigskip\bigskip
%\leavevmode\hbox{\tableA \fonttab{txttsc}{\texttt{\scshape Text Typewriter Cap \& Small Cap}}}
\end{center}

\subsection{T1 (EC) Cork Encoding Text Fonts}

These fonts' encodings are identical to those of corresponding EC fonts.

\begin{center}
\centering
\leavevmode\hbox{\tableD \fonttab{p1xr}{Text Roman Upright}}

\bigskip\bigskip
\leavevmode\hbox{\tableD \fonttab{p1xi}{\textit{Text Roman Italic}}}

\bigskip\bigskip
\leavevmode\hbox{\tableD \fonttab{p1xsl}{\textsl{Text Roman Slanted}}}

\bigskip\bigskip
\leavevmode\hbox{\tableD \fonttab{p1xsc}{\textsc{Text Roman Cap \& Small Cap}}}

%\bigskip\bigskip
%\leavevmode\hbox{\tableD \fonttab{t1xss}{\textsf{Text Sans Serif Upright}}}

%\bigskip\bigskip
%\leavevmode\hbox{\tableD \fonttab{t1xsssl}{\textsf{\slshape Text Sans Serif Slanted}}}

%\bigskip\bigskip
%\leavevmode\hbox{\tableD \fonttab{t1xsssc}{\textsf{\scshape Text Sans Serif  Cap \& Small Cap}}}

%\bigskip\bigskip
%\leavevmode\hbox{\tableD \fonttab{t1xtt}{\texttt{Text Typewriter Upright}}}

%\bigskip\bigskip
%\leavevmode\hbox{\tableD \fonttab{t1xttsl}{\texttt{\slshape Text Typewriter Slanted}}}

%\bigskip\bigskip
%\leavevmode\hbox{\tableD \fonttab{t1xttsc}{\texttt{\scshape Text Typewriter Cap \& Small Cap}}}
\end{center}

%\subsection{LY1 \TeX\ and ANSI Encoding Text Fonts}

%\begin{center}
%\centering
%\leavevmode\hbox{\tableD \fonttab{tyxr}{Text Roman Upright}}

%\bigskip\bigskip
%\leavevmode\hbox{\tableD \fonttab{tyxi}{\textit{Text Roman Italic}}}

%\bigskip\bigskip
%\leavevmode\hbox{\tableD \fonttab{tyxsl}{\textsl{Text Roman Slanted}}}

%\bigskip\bigskip
%\leavevmode\hbox{\tableD \fonttab{tyxsc}{\textsc{Text Roman Cap \& Small Cap}}}

%\bigskip\bigskip
%\leavevmode\hbox{\tableD \fonttab{tyxss}{\textsf{Text Sans Serif Upright}}}

%\bigskip\bigskip
%\leavevmode\hbox{\tableD \fonttab{tyxsssl}{\textsf{\slshape Text Sans Serif Slanted}}}

%\bigskip\bigskip
%\leavevmode\hbox{\tableD \fonttab{tyxsssc}{\textsf{\scshape Text Sans Serif Cap \& Small Cap}}}

%\bigskip\bigskip
%\leavevmode\hbox{\tableD \fonttab{tyxtt}{\texttt{Text Typewriter Upright}}}

%\bigskip\bigskip
%\leavevmode\hbox{\tableD \fonttab{tyxttsl}{\texttt{\slshape Text Typewriter Slanted}}}

%\bigskip\bigskip
%\leavevmode\hbox{\tableD \fonttab{tyxttsc}{\texttt{\scshape Text Typewriter Cap \& Small Cap}}}
%\end{center}

\subsection{TS1 (TC) Encoding Text Companion Fonts}

These fonts' encodings are identical to those of corresponding TC fonts.

\begin{center}
\centering
\leavevmode\hbox{\tableE \fonttab{pcxr}{Text Companion Roman Upright}}

\bigskip\bigskip
\leavevmode\hbox{\tableE \fonttab{pcxi}{\textit{Text Companion Roman Italic}}}

\bigskip\bigskip
\leavevmode\hbox{\tableE \fonttab{pcxsl}{\textsl{Text Companion Roman Slanted}}}

%\bigskip\bigskip
%\leavevmode\hbox{\tableE \fonttab{tcxss}{\textsf{Text Companion Sans Serif Upright}}}

%\bigskip\bigskip
%\leavevmode\hbox{\tableE \fonttab{tcxsssl}{\textsf{\slshape Text Companion Sans Serif Slanted}}}

%\bigskip\bigskip
%\leavevmode\hbox{\tableE \fonttab{tcxtt}{\texttt{Text Companion Typewriter Upright}}}

%\bigskip\bigskip
%\leavevmode\hbox{\tableE \fonttab{tcxttsl}{\texttt{\slshape Text Companion Typewriter Slanted}}}
\end{center}

\subsection{Math Fonts}

These fonts' encodings are identical to those of corresponding CM 
and \AmS\ Math fonts.
Additional math fonts are provided.

\begin{center}
\centering
\leavevmode\hbox{\table \fonttab{pxmi}{Math Italic (Corresponding to CMMI)}}

\bigskip\bigskip
\leavevmode\hbox{\table \fonttab{pxmi1}{Math Italic (Corresponding to CMMI) used with the \texttt{varg} option}}

\bigskip\bigskip
%\leavevmode\hbox{\tableF \fonttab{pxmia}{Math Italic A}}
\leavevmode\hbox{\table \fonttab{pxmia}{Math Italic A}}

\bigskip\bigskip
\leavevmode\hbox{\table \fonttab{pxsy}{Math Symbols (Corresponding to CMSY)}}

\bigskip\bigskip
\leavevmode\hbox{\table \fonttab{pxsya}{Math Symbols A (Corresponding to MSAM)}}

\bigskip\bigskip
\leavevmode\hbox{\table \fonttab{pxsyb}{Math Symbols B (Corresponding to MSBM)}}

\bigskip\bigskip
\leavevmode\hbox{\tableC \fonttab{pxsyc}{Math Symbols C}}

\bigskip\bigskip
\leavevmode\hbox{\table \fonttab{pxex}{Math Extension (Corresponding to CMEX)}}

\bigskip\bigskip
\leavevmode\hbox{\tableB \fonttab{pxexa}{Math Extension A}}
\end{center}

Bold versions of all fonts are available.

\end{document}
