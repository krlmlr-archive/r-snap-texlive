% !TEX TS-program = pdflatex
% !TEX encoding = UTF-8 Unicode

% This file is part of the Gentium package for TeX.
% It is licensed under the Expat License, see doc//README for details.

\documentclass[11pt, welsh, british]{article}
\usepackage{babel}
\usepackage[utf8x]{inputenc}
\usepackage[T1]{fontenc}
\usepackage{gentium}
\renewcommand{\ttdefault}{lmvtt}
\usepackage{fancyhdr,lastpage,fancyref}
\usepackage{array,longtable,verbatim}
\usepackage{booktabs}
\usepackage{multirow}
\usepackage{url}
	\urlstyle{tt}
\usepackage[breaklinks,colorlinks,
            linkcolor=black,citecolor=black,urlcolor=black]
           {hyperref}
%\usepackage{microtype}
\usepackage[a4paper,headheight=14pt,scale=0.8]{geometry}
  % use 14pt for 11pt text, 15pt for 12pt text

\title{Gentium for \TeX}
\author{Mojca Miklavec \and Pavel Farář \and Thomas A. Schmitz}
\newcommand*{\dyddiad}{\today}
\newcommand*{\ofname}{of}
\date{\dyddiad}
\pagestyle{fancy}
	\fancyhf[lh]{\itshape gentium}
	\fancyhf[rh]{\itshape\dyddiad}
	\fancyhf[ch]{}
	\fancyhf[lf]{}
	\fancyhf[rf]{}
	\fancyhf[cf]{\itshape --- \thepage~\ofname~\pageref{LastPage} ---}

\usepackage{textcomp}

\def\ConTeXt{Con{\TeX}t}
\def\ConTeXtMKII{Con{\TeX}t {\sc MkII}}
\def\ConTeXtMKIV{Con{\TeX}t {\sc MkIV}}
\def\XeTeX{Xe{\TeX}} % TODO: improve the first "e"--reversed and lowered
\def\pdfTeX{pdf{\TeX}}
\def\pdfLaTeX{pdf{\LaTeX}}
\def\TeXLive{\TeX~Live}
\def\MiKTeX{MiK\TeX}
\def\MacTeX{Mac\TeX}

\begin{document}
\maketitle
\thispagestyle{empty}
\pdfinfo{%
	/Creator	(TeX)
	/Producer	(pdfTeX)
	/Author		(Pavel Farář, Mojca Miklavec, Clea F.\ Rees)
	/Title		(Gentium for TeX)
	/Subject	(TeX)
	/Keywords	(TeX,LaTeX,font,fonts,tex,latex,gentium,GentiumPlus,GentiumBasic,Gentium,SIL,sil)}
\pdfcatalog{%
	/URL		()
	/PageMode	/UseOutlines}
    % other values: /UseNone, /UseOutlines, /UseThumbs, /FullScreen
	%[openaction <actionspec>]
%	\pagestyle{empty}
    % if you want this, you probably want to comment out \maketitle as well...?
\setlength{\parindent}{0pt}
\setlength{\parskip}{0.5em}
	
	
\newcommand*{\sil}{\textsc{sil}}
\newcommand*{\pkgname}[1]{\textsf{#1}}
\newcommand*{\fname}[1]{\textsf{#1}}

\begin{abstract}
	\hspace*{-\parindent}This document outlines the
	\TeX/\LaTeX/\ConTeXt\ support provided by this
	package for the Gentium font collection released by \sil.
\end{abstract}

\tableofcontents


\section{Introduction}

This document explains how to use the \TeX/\LaTeX/\ConTeXt\ support in
the present package for the Gentium font collection developed by \sil.
This package includes fonts in PostScript Type~1 format converted from
the original TrueType files released by \sil\ (using the FontForge
routines found in this package).  These Type~1 fonts use the name
`Gentium' by permission of \sil\ to the \TeX\ Users Group.  Therefore
the name of this \TeX\ package, by request of \sil, is
\pkgname{gentium-tug}.  Its home page is \url{http://tug.org/gentium}.

Further information about the fonts themselves can be found in the
included documentation and at \url{http://scripts.sil.org/gentium}. The
fonts are released under the \textsc{sil} Open Font License. For
details, see \url{ofl.txt} and \url{ofl-faq.txt}.  (In the event of
releasing modified versions of the fonts, either TrueType or Type~1,
it's required to use a name that doesn't include ``Gentium'', per the
\textsc{ofl}.)

This \TeX/\LaTeX/\ConTeXt\ support package consists of metrics, map
files, style files, documentation, and so on.  These files are released
under the Expat license.  The text for both licenses is included at the
end of this document, and in files in the distribution.

If you have the package installed and just want to know how to use
Gentium in your \TeX\ documents, feel free to skip to
section~\ref{sec:latex-package} (\LaTeX\ usage) or
section~\ref{sec:context-package} (\ConTeXt\ usage).

Please report any problems (contact info on the web page).  If you can
also send a fix, so much the better.


\section{Gentium collection background}

This package uses the original fonts GentiumPlus to make the regular and
italic styles and GentiumBasic to make the bold and bold italic styles,
and combines them into one \TeX-world family.

GentiumPlus is a serif family designed to support a wide range of Latin,
Greek and Cyrillic characters.  It currently includes just the regular
and italic style, as well as smallcaps.

GentiumBasic has just the Latin characters and it even lacks some Latin
characters found in GentiumPlus.  Therefore, the bold and bold italic
styles of this package support fewer languages than the regular and
italic styles (e.g.\ Czech and Slovak are not fully
supported). Moreover, GentiumBasic has no smallcaps and no kerning
pairs.  When and if \sil\ releases bold and bold italic GentiumPlus
fonts, we will switch to those, of course.

This package does not use the Berry fontname scheme, but longer names
similar to those of Latin Modern, etc.  One disadvantage of this choice
is that you cannot use the additional font selection commands provided
by the package \pkgname{nfssext-cfr}.


\section{Installation}

If you are using a reasonably recent \MiKTeX\ or \TeXLive\ or distro
installation it should be enough to install the \pkgname{gentium-tug}
package (if it's not already present).

Otherwise, or if you want to install the font manually, you will in
general need to perform these steps:
\begin{enumerate}
\item choose whether to install the font system-wide or in a personal
      directory;
\item move or copy the package files to the appropriate location;
\item refresh the \TeX\ database;
\item incorporate the included map file fragments for the different
      engines.
\end{enumerate}

However, that's all that it make sense to say here.  It's beyond the
scope of this Gentium documentation to explain all the myriad ways in
which \TeX-world map files can be installed and used; there are
differences between \TeXLive\ and \MiKTeX, too.

Instead, we refer you to an explanation of the system-wide installation,
with information for both \MiKTeX\ and \TeXLive, at
\url{http://tug.org/fonts/fontinstall.html}.  A corresponding
explanation for personal installation is at
\url{http://tug.org/fonts/fontinstall-personal.html}. 

Those pages describe using \path{testfont.tex} for a basic test of the
font; a tfm file name to use with that is \url{ec-gentiumplus-regular}.

To further test your installation and that the package works on your
system, run \LaTeX\ on this \path{gentium.tex} source file.  (You'll
need some commonly-available \LaTeX\ packages too, or comment them at.)
The console output and/or log should tell you whether any fonts were not
found. You can also compare your output with the original
\path{gentium.pdf}.


\section{Gentium \TeX\ support packages}\label{sec:support}

In short, for \LaTeX\ it suffices to include \verb|\usepackage{gentium}|
in your document preamble, and for \ConTeXtMKIV,
\verb|\setupbodyfont[gentium]|.  Details follow.

The only prerequisite is that the \LaTeX\ package (\path{gentium.sty})
requires \pkgname{xkeyval}, which you almost certainly already have.


\subsection{Encodings}\label{sec:encs}

The package supports not only the most common Latin encodings such as
\textsc{ot1}, TeXnANSI/\textsc{ly1}, Cork/\textsc{ec}/\textsc{t1} and
Text Companion/\textsc{ts1} encodings, but also (in regular and italic)
other Latin, Greek and Cyrillic encodings. Most characters in the text
encodings and some of those in the Text Companion encoding are
available, including the~\texteuro.  You can see the available encodings
in table~\ref{encodings}. The Greek encoding LGR is supported only in
\LaTeX, AGR only in \ConTeXt.

\begin{table}[h]
\centering
  \begin{tabular}{lll}
    \toprule
    script & available encodings & styles \\
    \midrule
    Latin    & OT1, T1, LY1, L7x, QX, T5, TS1 & regular, italic, bold,
    bold italic \\
    Cyrillic & T2A, T2B, T2C, X2 & regular, italic\\
    Greek    & LGR (\LaTeX), AGR (\ConTeXt) & regular, italic \\
    \bottomrule
  \end{tabular}
  \caption{Available encodings in \pkgname{gentium-tug}.}
  \label{encodings}
\end{table}

The regular and italic styles support all these Latin, Greek and Cyrillic
encodings. They support also small caps for the Latin and Cyrillic encodings,
but there are no small caps for Greek.

The bold and bold italic styles support only the Latin encodings and they
have no small caps.

Cork/T1 encoding lacks visible space, cwm (compound work mark),
SS and the character for composing permille sign.

Missing characters in bold styles for Latin encodings:

T1: Aogonek / aogonek, Eogonek / eogonek, Lcaron / lcaron,
    Scedilla / scedilla, uni021A (Tcommaaccent) / uni021B (tcommaaccent),
    dcaron, tcaron

L7x: Rcommaaccent, Gcommaaccent, Kcommaaccent, Lcommaaccent, Ncommaaccent,
    Aogonek, Eogonek, Iogonek, Uogonek

QX: Aogonek, Eogonek, Iogonek, uni021A (Tcommaaccent)


\subsection{\LaTeX}
\label{sec:latex-package}

To use Gentium fonts in a \LaTeX\ document, add
\verb|\usepackage{gentium}| to your document preamble. This will set
the default serif/roman family to \fname{gentium}.

If you want to use Gentium together with another font (sans or
typewriter) with a different x-height, you should consider using the
option \verb|scaled|. This scales Gentium font and if you choose the
right scaling factor, you will get the same x-height of both fonts.
Here is an example of this option:
\begin{verbatim}
\usepackage[scaled=0.9]{gentium}
\end{verbatim}


\subsection{\ConTeXt}
\label{sec:context-package}

In \ConTeXtMKIV, to switch to the Gentium typeface you only need:
\begin{verbatim}
\setupbodyfont[gentium]
\end{verbatim}

In \ConTeXtMKII, before doing that you need either (for \pdfTeX):
\begin{verbatim}
\usetypescriptfile[type-gentium]
\usetypescript[gentium][ec] % or whatever encoding
\end{verbatim}
or (for \XeTeX):
\begin{verbatim}
\usetypescriptfile[type-gentium]
\usetypescript[gentium]
\end{verbatim}


\subsection{Changes in the Type~1 fonts vs.\ the TrueType originals}

The shapes of all characters in the Type~1 fonts are the same as in the
original TrueType fonts, except for the unavoidable changes are induced
by the format conversion (that is, the spline representations are
necessarily different).

However, a few small changes---hopefully improvements---were made to the
metrics of the GentiumPlus family (that is, regular and italic; bold and
bold italic are untouched). The Type~1 fonts are used to generate the
\TeX\ font metric files (tfm), so these changes propagate to the 8-bit
engines like \pdfTeX.

The first change concerns Greek.  8-bit Greek encodings (LGR and AGR) do
not use precomposed accented capital letters. These letters are composed
as a sequence of two glyphs: accent + capital letter. The problem is
that this sequence does not look like the precomposed letter---there is
often big space between the accent and the letter. Therefore, there are
some extra kerning pairs between accents and capital Greek
letters. These kerning pairs are created automatically (the script is
included in the sources) and the goal is to have the same relative
position between the accent and the letter as in the precomposed
letter. Thanks to these extra kerning pairs you should get better
results for 8-bit engines. These changes are irrelevant for Unicode
engines---they use the precomposed letters.

The second change is in the letters dcaron (ď) and lcaron
(ľ) that are used in the Czech and Slovak languages. (There was
no need to change tcaron (ť) and Lcaron (Ľ) with the same
accent.) There is no change of their shapes, and their advance widths
are also untouched---the change is to add several kerning pairs with
quite big negative values. Without these changes there was often a large
space between dcaron or lcaron and the following letter, so the changes
make the words containing these letters much more compact. You get these
changes automatically if you use 8-bit engines.  If you use Unicode
engines with the original TrueType fonts, you get the original
metrics. However, you can tell the Unicode engines to use the Type~1
fonts (which append \verb|PS| to the family name) like this:
\begin{verbatim}
\usepackage{fontspec}
\setmainfont{GentiumPlusPS}
\end{verbatim}
Then you get the additional kern pairs for dcaron and lcaron. The
unfortunately disadvantage is that you cannot use small caps.

The last (similar) change is that additional kerning pairs were added to
the Type~1 fonts for accented Latin letters and small caps. The original
TrueType fonts have no such kerning pairs. As with the Czech/Slovak
changes, you get these changes automatically if you use 8-bit
engines. If you use Unicode engines with the original TrueType fonts,
you get the original metrics, but you can override as above (but since
you cannot use small caps with that method, the kerning pairs for small
caps become irrelevant).


\section{Known bugs}

There are problems in older versions of \pdfTeX\ with small caps when
using TrueType fonts.  Especially the Latin encoding \textsc{t5} and all
Cyrillic encodings are unusable.  You can use the Type~1 version of the
fonts or at least \pdfTeX\ version 1.40.13 to avoid these problems.

The Gentium fonts are a work in progress and as such they still miss
some features like kerning pairs for some letters. Currently, there are
no kerning pairs in the GentiumBasic family at all and the GentiumPlus
family has kerning pairs just for Latin letters without accents and for
Greek letters; there are no kerning pairs for small caps, accented Latin
letters or Cyrillic letters. The Type~1 fonts in this package have some
additional kerning pairs for accented Latin letters and for small
caps. Kerning pairs for Cyrillic are under consideration.


\section{License}

The fonts in this page, both the \sil\ originals and the derived Type~1
versions, are released under \textsc{ofl}.  The \TeX\ support files
are licensed under the Expat License.  Here are the full license texts.


\subsection{SIL Open Font License}

Copyright (c) 2003-2011 SIL International (http://www.sil.org/),
with Reserved Font Names ``Gentium'' and ``SIL''.

This Font Software is licensed under the SIL Open Font License, Version 1.1.
This license is copied below, and is also available with a FAQ at:
http://scripts.sil.org/OFL


% TODO: Try to have the similar look as the text version
%-----------------------------------------------------------
SIL OPEN FONT LICENSE Version 1.1 --- 26 February 2007
%-----------------------------------------------------------


PREAMBLE

The goals of the Open Font License (OFL) are to stimulate worldwide
development of collaborative font projects, to support the font creation
efforts of academic and linguistic communities, and to provide a free and
open framework in which fonts may be shared and improved in partnership
with others.

The OFL allows the licensed fonts to be used, studied, modified and
redistributed freely as long as they are not sold by themselves. The
fonts, including any derivative works, can be bundled, embedded, 
redistributed and/or sold with any software provided that any reserved
names are not used by derivative works. The fonts and derivatives,
however, cannot be released under any other type of license. The
requirement for fonts to remain under this license does not apply
to any document created using the fonts or their derivatives.


DEFINITIONS

``Font Software'' refers to the set of files released by the Copyright
Holder(s) under this license and clearly marked as such. This may
include source files, build scripts and documentation.

``Reserved Font Name'' refers to any names specified as such after the
copyright statement(s).

``Original Version'' refers to the collection of Font Software components as
distributed by the Copyright Holder(s).

``Modified Version'' refers to any derivative made by adding to, deleting,
or substituting---in part or in whole---any of the components of the
Original Version, by changing formats or by porting the Font Software to a
new environment.

``Author'' refers to any designer, engineer, programmer, technical
writer or other person who contributed to the Font Software.


PERMISSION \& CONDITIONS

Permission is hereby granted, free of charge, to any person obtaining
a copy of the Font Software, to use, study, copy, merge, embed, modify,
redistribute, and sell modified and unmodified copies of the Font
Software, subject to the following conditions:

1) Neither the Font Software nor any of its individual components,
in Original or Modified Versions, may be sold by itself.

2) Original or Modified Versions of the Font Software may be bundled,
redistributed and/or sold with any software, provided that each copy
contains the above copyright notice and this license. These can be
included either as stand-alone text files, human-readable headers or
in the appropriate machine-readable metadata fields within text or
binary files as long as those fields can be easily viewed by the user.

3) No Modified Version of the Font Software may use the Reserved Font
Name(s) unless explicit written permission is granted by the corresponding
Copyright Holder. This restriction only applies to the primary font name as
presented to the users.

4) The name(s) of the Copyright Holder(s) or the Author(s) of the Font
Software shall not be used to promote, endorse or advertise any
Modified Version, except to acknowledge the contribution(s) of the
Copyright Holder(s) and the Author(s) or with their explicit written
permission.

5) The Font Software, modified or unmodified, in part or in whole,
must be distributed entirely under this license, and must not be
distributed under any other license. The requirement for fonts to
remain under this license does not apply to any document created
using the Font Software.


TERMINATION

This license becomes null and void if any of the above conditions are
not met.


DISCLAIMER

\textsc{The font software is provided ``as is'', without warranty of any kind,
express or implied, including but not limited to any warranties of
merchantability, fitness for a particular purpose and noninfringement
of copyright, patent, trademark, or other right. In no event shall the
copyright holder be liable for any claim, damages or other liability,
including any general, special, indirect, incidental, or consequential
damages, whether in an action of contract, tort or otherwise, arising
from, out of the use or inability to use the font software or from
other dealings in the font software.}


\subsection{Expat License}

Copyright (c) 2008-2013 TeX Users Group

Permission is hereby granted, free of charge, to any person obtaining
a copy of this software and associated documentation files (the
``Software''), to deal in the Software without restriction, including
without limitation the rights to use, copy, modify, merge, publish,
distribute, sublicense, and/or sell copies of the Software, and to
permit persons to whom the Software is furnished to do so, subject to
the following conditions:

The above copyright notice and this permission notice shall be included
in all copies or substantial portions of the Software.

\textsc{The software is provided ``as is'', without warranty of any kind,
express or implied, including but not limited to the warranties of
merchantability, fitness for a particular purpose and noninfringement.
In no event shall the authors or copyright holders be liable for any
claim, damages or other liability, whether in an action of contract,
tort or otherwise, arising from, out of or in connection with the
software or the use or other dealings in the software.}



% OT1 is the same for regular and italic style (like TeX Gyre, unlike
% Computer Modern)---this might go to the bugs section.

% Cyrillic encodings use some letters with tail instead of with descender.


% TODO

% Show the available encodings in a better way (probably something
% similar to README), probably use small caps. Look at Clea's tables
% if they are better than what I wrote.

% Improve the LaTeX logo in Gentium (not so much related to this
% documentation).

% Search for gentiumplus and vgx---the latter shouldn't be present.

% Scale the monospaced font to match the x-height.

% Change the license and other information in the header.


% NOTES TO CLEA

% Current (unlike Clea's) LY1 doesn't contain Wcircumflex and others.
% Look at it!
% \textsc{ly1} is
% necessary for access to Wcircumflex (Ŵ), wcircumflex (ŵ),
% Ycircumflex (Ŷ) and ycircumflex (ŷ) as pre-composed
% glyphs.

% Look what's different from Clea's package.


% Some Clea's texts that might be used:

% The final option controls whether Latin Modern is used for sans and
% typewriter text. Because Computer Modern does not support the
% TeXnANSI/\textsc{ly1} encoding, you will likely get strange results
% unless you redefine \path{\sfdefault} and \path{\ttdefault}. Latin
% Modern is used because it is close to the default Computer Modern fonts
% and is widely available. If you would prefer that the package not
% redefine the default sans and typewriter families, use \verb|lm=false|
% when loading \pkgname{gentiumplus}. To explicitly request the default
% behaviour, which does redefine these families, use \verb|lm| or
% \verb|lm=true|.

% Loading \path{gentium.sty} does not affect the setup for
% mathematics.

\end{document}  
