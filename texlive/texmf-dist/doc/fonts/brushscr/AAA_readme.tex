\documentclass[a4paper]{article}
\usepackage{shortvrb}
\pagestyle{plain}
\MakeShortVerb{\|}

\begin{document}
\title{BrushScriptX-Italic PostScript\thanks{PostScript is a
    registered trademark of Adobe Systems Inc.} \\
  Type-1 Fonts}
\author{\Large Maurizio Loreti}
\date{Rev.\ 2.0 \\
  September 21, 2001} \maketitle

The files included in this package allow to use in your
\LaTeX\ documents a PostScript Type-1 font named
``BrushScriptX-Italic''; this font, similar to Adobe
BrushScript, simulates hand-written characters.  In detail,
these files are:
\begin{enumerate}
\item AAA\_readme.tex: this file.
\item README: a shorter ASCII version of this file.
\item Makefile: contains the commends needed, on my Unix
  system, to generate all the auxiliary files needed in
  order to use this fonts with \LaTeX\ and |dvips|.  In
  particular, the included file generate.tex is used as
  input to the \TeX\ package |fontinst|.
\item config.pbsi and pbsi.map: the auxiliary files for
  |dvips|.  pbsi.map maps BrushScript to the ASCII version
  of the font (the BrushScriptX-Italic.pfa file).
\item t1pbsi.fd: the \TeX\ font definition file, for the T1
  encoding.
\item pbsi8t.vf: the \TeX\ virtual font file, for the T1
  encoding.
\item pbsi8t.tfm: the \TeX\ font metric file, for the T1
  encoding.
\item BrushScriptX-Italic.pfa and BrushScriptX-Italic.afm:
  the font, in ASCII format, and the corresponding font
  metric file.  The afm file lacks the part about the
  kerning, that has been separately tuned by hand and is
  defined in another file, kern.txt (merged with
  BrushScriptX-Italic.afm at the generation time from the
  Makefile).
\item sample.tex: a first example file (a real page printed
  out in the past for my students; in Italian, sorry).
\item pbsi.sty and example.tex: the style defines a command
  |\textbsi|, that print its argument using the
  BrushScriptX-Italic fonts; the second is another example
  file showing how to use it.
\end{enumerate}

For the installation, you should copy
BrushScriptX-Italic.pfa, pbsi.sty, config.pbsi, pbsi.map,
and the .fd, .tfm and .vf files, in appropriate directories
of the \TeX\ tree.  Some installations (e.g.\ teTeX) require
to update the file database, running (as root) the command
|texhash|; check your documentation.  Done that, use the
fonts as shown in the sample.tex example (for long texts),
or (as outlined in example.tex) with |\textbsi{...}| (for
short texts); and \emph{remember} to run |dvips| \emph{with
  the command option} |-P pbsi|.  Some historical notes:
\begin{itemize}
\item I have found the original BrushScriptX-Italic font (a
  public domain NeXT screen font) on the web; and I have
  heavily modified the original file on my Amiga personal
  computer, with a commercial font drawing package named
  TypeSmith.  This happened in the first years of 1990's.
\item I then submitted to CTAN the modified font; my version
  is also public domain: you are free to modify my files as
  you want, as long as you don't make any profit from them.
\item Later, when a new version of |dvips| appeared that
  made more tight controls on the handled Type-1 fonts, the
  modified BrushScript was refused by |dvips|.  At the time
  I did not have any Amiga computer, nor any tool to use in
  order to edit and/or modify a Type-1 font; Louis Vosloo of
  Y\&Y has pointed to me that the original tool (TypeSmith)
  did not generate a correct font, and (with enormous
  courtesy) edited the font himself.  That modified font was
  resubmitted to CTAN.
\item Still later, Mr.\ Rolf Niepraschk of
  Physikalisch-Technische Bundesanstalt Berlin (Germany) has
  both contributed a file pbsi.sty and, in 2001, found that
  the last |dvips| (5.86e) still refused to handle
  BrushScript.  He has also contributed a fix for that,
  using the ``type1fix'' script part of the \TeX{}trace
  package by P\'eter Szab\'o
  (|http://www.inf.bme.hu/~pts/textrace/|).
\item A kind reader of comp.text.tex pointed to me the
  existence of a free font editor for Unix
  (|http://pfaedit.sourceforge.net|); so I decided to clean
  up the font shapes, to specify better kernings and to
  build a brand new release (v2.0, released at the end of
  September 2001).  Mr.\ Thierry Bouche helped several times
  with |fontinst|, whose documentation is, err, quite
  obscure.
\end{itemize}

Be careful, do not use \TeX\ commands that use math
characters, or that change the typeface (like |\bfseries| or
|\sffamily|); Brush-Script-Italic has no defined variants!


\par\vspace*{1cm}\noindent Thanks to:
\begin{itemize}
\item Louis Vosloo |<support@yandy.com>|, who helped with
  the font.
\item Rolf Niepraschk |<niepraschk@ptb.de>|, who helped with
  the font.
\item Thierry Bouche |<bouche@mozart.ujf-grenoble.Fr>|, who
  helped with |fontinst|.
\item George Williams, the author of |pfaedit|.
\end{itemize}

\par\vspace*{1cm}\noindent Happy \TeX{}ing!\hfill
Maurizio Loreti\vspace*{2\baselineskip}
\par\hfill EMail: |loreti@pd.infn.it|
\par\hfill WWW: |http://www.pd.infn.it/~loreti/mlo.html|
\end{document}
