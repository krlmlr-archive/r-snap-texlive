\documentclass[a4paper]{article}
\hyphenation{di-spo-si-zio-ne i-ni-zio}

\pagestyle{empty}
\setlength{\parskip}{0.7\baselineskip}
\begin{document}

\input t1pbsi.fd
\newcommand{\bsi}[2]{\fontsize{#1}{#2}\usefont{T1}{pbsi}{xl}{n}}

\bsi{16pt}{20pt}
\begin{center}
\vspace*{8mm}
  Avviso per gli studenti del primo anno \\
  del Corso di Laurea in Fisica \\[2ex]
  Corso di Esperimentazioni di Fisica 1
\end{center}
\vspace*{8mm}

\bsi{12pt}{15pt}
Il corso di Esperimentazioni di Fisica 1 \`e composto
sia da una parte sperimentale che da una parte teorica;
la parte sperimentale viene svolta nei locali del
Laboratorio di Fisica per gli studenti del primo anno
(che si trovano nell'edificio dei Laboratori Didattici
di Fisica in via Loredan 10) in quattro turni
indipendenti tenuti nei pomeriggi dei primi quattro
giorni della settimana, con inizio per tutti alle ore
15, affidati al Prof.\ Loreti (turni del lunedi e del
martedi) e al Prof.\ Ciampolillo (turni del mercoledi
e del giovedi).

Gli studenti sono invitati ad iscriversi ad uno (ed
uno solo) dei quattro turni di laboratorio, ponendo
il loro nome ed il numero di matricola nelle apposite
liste che sono a loro disposizione nella portineria
dell'ingresso del Dipartimento di Fisica in via
Paolotti.

La parte teorica viene svolta nell'aula D del
Dipartimento di Fisica, secondo l'orario esposto
sempre in portineria; gli studenti devono frequentare
solo le due ore settimanali di lezione tenute
da quel docente a cui \`e affidato il loro turno
pomeridiano.

Le lezioni sperimentali inizieranno a partire dal
pomeriggio di lunedi 10 ottobre; le lezioni teoriche
invece avranno inizio lunedi 3 ottobre, e la prima
di esse avr\`a carattere introduttivo generale: per
cui tutti gli studenti (e non solo quelli
iscritti col Prof.\ Loreti) sono invitati ad essere
presenti.
\end{document}
