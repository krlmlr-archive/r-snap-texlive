% !TEX TS-program = pdflatex

% sansmathfonts.tex
%
% Documentation for the sansmathfonts package
%
% author: Ariel Barton
%
% Copyright Ariel Barton, 2013
%
% This work may be distributed and/or modified under the
% conditions of the LaTeX Project Public License, either
% version 1.3 of this license or (at your option) any
% later version.
% The latest version of the license is in
%    http://www.latex-project.org/lppl.txt
% and version 1.3 or later is part of all distributions of
% LaTeX version 2003/06/01 or later.
%
% This work has the LPPL maintenance status "author-maintained".
%
% The complete list of files considered part of this work is in
% the file `sansmathfonts.pdf' and its source code `sansmathfonts.tex'.
%
% Date: 2013/03/22

\documentclass{amsart}

\raggedbottom

\usepackage{sansmathfonts}
\usepackage{hyperref, amsfonts, esint, slantsc, bm}

\def\sectionautorefname{Section}

\title{The \textsf{sansmathfonts} package}
\author{Ariel Barton}

\DeclareTextFontCommand{\cmsmf}
{\fontencoding{OT1}\fontfamily{cmsmf}\selectfont}

\begin{document}

\maketitle

The Computer Modern font family has a sans serif typeface. However, compared to the serif typeface, it is incomplete: there are no sans serif small caps or math fonts. Furthermore, the bold slanted font is not available as an outline font.
This leads to highly unsatisfactory typography of documents that use sans serif for the body text.


The \textsf{sansmathfonts} package provides these ``missing'' fonts. Most of the usefulness of the package is in the fonts; \texttt{sansmathfonts.sty} is a small package providing \LaTeX\ support. To use it, simply say \texttt{\char`\\usepackage\char`\{sansmathfonts\char`\}} in the document preamble.

This will redefine the document's default sans serif font family from \textsf{cmss} to \textsf{xcmss}; this will make the \textsf{\textbf{\textit{bold slanted}}} and \textsf{\textsc{Caps and small caps}} fonts available via normal \LaTeX\ font commands (\verb|\textbf|, \verb|\textit| and \verb|\textsc|). If you additionally load Harald Harder's \textsf{slantsc} package, this will make \textsf{\textit{\textsc{slanted caps and small caps}}} available.

This will also switch the math fonts to sans serif:
\[\Im \mathop{\mathrm{exp}}(i\omega)=\sin(\omega) \]
If you use symbols from the \textsf{amsfonts} or \textsf{esint} packages, they will also be replaced by appropriate sans serif versions:
\[\ointclockwise \quad \mho \quad \hbar\]
By default, the commands \verb|\mathrm| and \verb|\mathsf| both produce sans serif text. To get serifed roman text, use the command \verb|\mathserif|:
\[\mathrm{mathrm}\quad\mathsf{mathsf}\quad\mathserif{mathserif}\]
\textsf{sansmathfonts} knows about the \textsf{beamer} document class and will automatically use \textsf{beamer}'s \texttt{professionalfonts} theme.


If you prefer roman math fonts, either use the \texttt{[notmath]} package option, or don't use the package at all; instead use the line \texttt{\char`\\renewcommand\char`\{\char`\\sfdefault\char`\}\char `\{xcmss\char`\}}
in the document preamble.

The math fonts differ slightly from Knuth's standard sans serif fonts. Specifically, for ease of reading I have chosen to put the serifs back on the uppercase I, Pi and~Xi:
\[I\quad\mathrm{I}\quad\mathbf{I}\quad\bm{I}\quad
\mathit{\Pi}\quad\Pi\quad\mathbf{\Pi}\quad
\mathit{\Xi}\quad\Xi\quad\mathbf{\Xi}\quad
\text{and not\quad \textsf{I\quad \char5\quad\char4}}\]
\textsf{Sans serif Is outside of math mode} still have no serifs unless the package option \texttt{[I]} is used; note that this option is incompatible with the T1 encoding.


Feedback is appreciated and may be sent to \texttt{origamist@gmail.com}.




\section{List of new fonts}\label{sec:fonts}

All of the Type 1 fonts in this package were generated using mftrace 1.2.18 and Fontforge.

The following fonts are based mainly on Donald Knuth's Computer Modern fonts.

\textsf{\fontshape{ui}\selectfont Unslanted italic} (needed for the pounds symbol \textsf{\pounds}):
\begin{itemize}
\item \textsf{cmssu10}
\end{itemize}

Text \textsf{\textsc{caps and small caps}}, OT1 encoding:

\nobreak

\noindent\parbox[t]{0.25\textwidth}{
\begin{itemize}
\item \textsf{cmssbxcsc10}
\item \textsf{cmssxicsc10}
\end{itemize}}\parbox[t]{0.25\textwidth}{\begin{itemize}
\item \textsf{cmsscsc8}
\item \textsf{cmsscsci8}
\end{itemize}}\parbox[t]{0.25\textwidth}{\begin{itemize}
\item \textsf{cmsscsc9}
\item \textsf{cmsscsci9}
\end{itemize}}\parbox[t]{0.25\textwidth}{\begin{itemize}
\item \textsf{cmsscsc10}
\item \textsf{cmsscsci10}
\end{itemize}}

\bigskip

Math italic ($\alpha \beta \gamma abc \ell \wp$):

\nobreak

\noindent\parbox[t]{0.25\textwidth}{
\begin{itemize}
\item \textsf{cmssmi5}
\item \textsf{cmssmi6}
\item \textsf{cmssmi7}
\end{itemize}}\parbox[t]{0.25\textwidth}{\begin{itemize}
\item \textsf{cmssmi8}
\item \textsf{cmssmi9}
\item \textsf{cmssmi10}
\end{itemize}}\parbox[t]{0.25\textwidth}{\begin{itemize}
\item \textsf{cmssmib5}
\item \textsf{cmssmib6}
\item \textsf{cmssmib7}
\end{itemize}}\parbox[t]{0.25\textwidth}{\begin{itemize}
\item \textsf{cmssmib8}
\item \textsf{cmssmib9}
\item \textsf{cmssmib10}
\end{itemize}}

\bigskip

Math symbols ($\Re\oplus \Im  $):

\nobreak

\noindent\parbox[t]{0.25\textwidth}{
\begin{itemize}
\item \textsf{cmsssy5}
\item \textsf{cmsssy6}
\item \textsf{cmsssy7}
\end{itemize}}\parbox[t]{0.25\textwidth}{\begin{itemize}
\item \textsf{cmsssy8}
\item \textsf{cmsssy9}
\item \textsf{cmsssy10}
\end{itemize}}\parbox[t]{0.25\textwidth}{\begin{itemize}
\item \textsf{cmssbsy5}
\item \textsf{cmssbsy6}
\item \textsf{cmssbsy7}
\end{itemize}}\parbox[t]{0.25\textwidth}{\begin{itemize}
\item \textsf{cmssbsy8}
\item \textsf{cmssbsy9}
\item \textsf{cmssbsy10}
\end{itemize}}

\bigskip


Math extended fonts ($\int \sum \prod$):

\nobreak

\noindent\parbox[t]{0.25\textwidth}{
\begin{itemize}
\item \textsf{cmssex7}
\end{itemize}}\parbox[t]{0.25\textwidth}{\begin{itemize}
\item \textsf{cmssex8}
\end{itemize}}\parbox[t]{0.25\textwidth}{\begin{itemize}
\item \textsf{cmssex9}
\end{itemize}}\parbox[t]{0.25\textwidth}{\begin{itemize}
\item \textsf{cmssex10}
\end{itemize}}

\bigskip

\cmsmf{Sans serif text fonts with serifed capital~I:}

\noindent\parbox[t]{0.25\textwidth}{
\begin{itemize}
\item \textsf{cmsmf8}
\item \textsf{cmsmf9}
\item \textsf{cmsmf10}
\item \textsf{cmsmf12}
\item \textsf{cmsmf17}
\item \textsf{cmsmfcsc8}
\item \textsf{cmsmfcsc9}
\item \textsf{cmsmfcsc10}
\end{itemize}}\parbox[t]{0.25\textwidth}{\begin{itemize}
\item \textsf{cmsmfbx8}
\item \textsf{cmsmfbx9}
\item \textsf{cmsmfbx10}
\item \textsf{cmsmfbx12}
\item \textsf{cmsmfbx17}
\item \noindent\rlap{\textsf{cmsmfbxcsc10}}
\end{itemize}}\parbox[t]{0.25\textwidth}{\begin{itemize}
\item \textsf{cmsmfi8}
\item \textsf{cmsmfi9}
\item \textsf{cmsmfi10}
\item \textsf{cmsmfi12}
\item \textsf{cmsmfi17}
\item \textsf{cmsmfcsci8}
\item \textsf{cmsmfcsci9}
\item \textsf{cmsmfcsci10}
\end{itemize}}\parbox[t]{0.25\textwidth}{\begin{itemize}
\item \textsf{cmsmfxi8}
\item \textsf{cmsmfxi9}
\item \textsf{cmsmfxi10}
\item \textsf{cmsmfxi12}
\item \textsf{cmsmfxi17}
\item \noindent\rlap{\textsf{cmsmfxicsc10}}
\end{itemize}}


\bigskip


The following fonts are based on fonts by other authors.

\nobreak

\noindent\parbox[t]{0.33\textwidth}{
\raggedright
Eddie Saudrais's \textsf{esint} package
\begin{itemize}
\item \textsf{ssesint7}
\item \textsf{ssesint8}
\item \textsf{ssesint9}
\item \textsf{ssesint10}
\end{itemize}}\parbox[t]{0.33\textwidth}{
\raggedright
AMS symbols (\textsf{amsfonts} package)
\begin{itemize}
\item \textsf{ssmsam5}
\item \textsf{ssmsam6}
\item \textsf{ssmsam7}
\item \textsf{ssmsam8}
\item \textsf{ssmsam9}
\item \textsf{ssmsam10}
\end{itemize}}\parbox[t]{0.33\textwidth}{
\raggedright
AMS symbols (\textsf{amsfonts} package)
\begin{itemize}
\item \textsf{ssmsbm5}
\item \textsf{ssmsbm6}
\item \textsf{ssmsbm7}
\item \textsf{ssmsbm8}
\item \textsf{ssmsbm9}
\item \textsf{ssmsbm10}
\end{itemize}}

\bigskip

The following fonts are based on J\"org Knappen's European Computer Modern fonts.

\nobreak

\bigskip
\noindent
\parbox[t]{0.25\textwidth}{
\font\temp = eczz1000 \temp
Normal
\begin{itemize}
\item \textsf{eczz0500}
\item \textsf{eczz0600}
\item \textsf{eczz0700}
\item \textsf{eczz0800}
\item \textsf{eczz0900}
\item \textsf{eczz1000}
\item \textsf{eczz1095}
\item \textsf{eczz1200}
\item \textsf{eczz1440}
\item \textsf{eczz1728}
\item \textsf{eczz2074}
\item \textsf{eczz2488}
\item \textsf{eczz2986}
\item \textsf{eczz3583}
\end{itemize}
}\parbox[t]{0.25\textwidth}{
\font\temp = eczi1000 \temp
Slanted
\begin{itemize}
\item \textsf{eczi0500}
\item \textsf{eczi0600}
\item \textsf{eczi0700}
\item \textsf{eczi0800}
\item \textsf{eczi0900}
\item \textsf{eczi1000}
\item \textsf{eczi1095}
\item \textsf{eczi1200}
\item \textsf{eczi1440}
\item \textsf{eczi1728}
\item \textsf{eczi2074}
\item \textsf{eczi2488}
\item \textsf{eczi2986}
\item \textsf{eczi3583}
\end{itemize}
}\parbox[t]{0.25\textwidth}{
\font\temp = eczx1000 \temp
Bold
\begin{itemize}
\item \textsf{eczx0500}
\item \textsf{eczx0600}
\item \textsf{eczx0700}
\item \textsf{eczx0800}
\item \textsf{eczx0900}
\item \textsf{eczx1000}
\item \textsf{eczx1095}
\item \textsf{eczx1200}
\item \textsf{eczx1440}
\item \textsf{eczx1728}
\item \textsf{eczx2074}
\item \textsf{eczx2488}
\item \textsf{eczx2986}
\item \textsf{eczx3583}
\end{itemize}
}\parbox[t]{0.25\textwidth}{
\font\temp = eczo1000 \temp
Bold slanted
\begin{itemize}
\item \textsf{eczo0500}
\item \textsf{eczo0600}
\item \textsf{eczo0700}
\item \textsf{eczo0800}
\item \textsf{eczo0900}
\item \textsf{eczo1000}
\item \textsf{eczo1095}
\item \textsf{eczo1200}
\item \textsf{eczo1440}
\item \textsf{eczo1728}
\item \textsf{eczo2074}
\item \textsf{eczo2488}
\item \textsf{eczo2986}
\item \textsf{eczo3583}
\end{itemize}
}

\bigskip

The \textsf{sansmathfonts} also provides outline versions of the following fonts (supplied with Mac\TeX\ 2012 as Metafont fonts only). These provide \textsf{\textbf{bold}} and \textsf{\textsl{\textbf{bold slanted}}} fonts at varying sizes.

\noindent\parbox[t]{0.25\textwidth}{
\font\temp = eczo1000 \temp
\begin{itemize}
\item \textsf{cmssxi8}
\item \textsf{cmssxi9}
\item \textsf{cmssxi10}
\end{itemize}
}\parbox[t]{0.25\textwidth}{
\begin{itemize}
\item \textsf{cmssxi12}
\item \textsf{cmssxi17}
\end{itemize}
}\parbox[t]{0.25\textwidth}{
\begin{itemize}
\item \textsf{cmssbx8}
\item \textsf{cmssbx9}
\end{itemize}
}\parbox[t]{0.25\textwidth}{
\begin{itemize}
\item \textsf{cmssbx12}
\item \textsf{cmssbx17}
\end{itemize}
}


\section{Files in this package}\label{sec:files}

109 of the new fonts listed in \autoref{sec:fonts} come in three files each: the \TeX\ Font Metric files (extension \texttt{.tfm}), the Type 1 font file (extension \texttt{.pfb}), and Metafont source file (extension \texttt{.mf}).

The 9 \textsf{cmssxi} and \textsf{cmssbx} fonts come as \texttt{.pfb} files only, as the MetaFont sources are already part of the \TeX\ Live distribution (see also the \textsf{sauter} package at \texttt{\href{http://www.ctan.org/tex-archive/fonts/cm/sauter} {http:/\slash www.ctan.org\slash tex-archive\slash fonts\slash cm\slash sauter}}).

The 28 \textsf{cmsmf} fonts are almost identical to their \textsf{cmss} counterparts. Thus, these fonts are provided as \emph{virtual} fonts, and so come in five parts: the virtual font file (\texttt{cmsmf.vf}), the \TeX\ Font Metric file (\texttt{cmsmf.tfm}), and the font \textsf{cmsmfIPiXi} containing only the altered letters \cmsmf{I}, \cmsmf{\char4} and \cmsmf{\char5} (and \cmsmf{\textsc{i}}, in the small caps fonts); this font comes as MetaFont source (\texttt{cmsmfIPiXi.mf}), \TeX\ font metric (\texttt{cmsmfIPiXi.tfm}) and Type 1 font (\texttt{cmsmfIPiXi.pfb}).

In addition, this package should come with the following 29 supplementary Metafont source files:

\begin{itemize}
\item \texttt{eczi.mf}
\item \texttt{eczo.mf}
\item \texttt{eczx.mf}
\item \texttt{eczz.mf}
\item \texttt{sans-amsya.mf}
\item \texttt{sans-amsyb.mf}
\item \texttt{sans-asymbols.mf}
\item \texttt{sans-bigdel.mf}
\item \texttt{sans-bigint.mf}
\item \texttt{sans-bigop.mf}
\item \texttt{sans-bsymbols.mf}
\item \texttt{sans-calu.mf}
\item \texttt{sans-csc.mf}
\item \texttt{sans-greekl.mf}
\item \texttt{sans-greeku.mf}
\item \texttt{sans-IPiXi.mf}
\item \texttt{sans-IPiXicsc.mf}
\item \texttt{sans-mathex.mf}
\item \texttt{sans-mathint.mf}
\item \texttt{sans-mathsl.mf}
\item \texttt{sans-mathsy.mf}
\item \texttt{sans-roman.mf}
\item \texttt{sans-romanu.mf}
\item \texttt{sans-romms.mf}
\item \texttt{sans-slantms.mf}
\item \texttt{sans-sym.mf}
\item \texttt{sans-symbol.mf}
\item \texttt{sans-xbbold.mf}
\item \texttt{sansfontbase.mf}
\end{itemize}

This package should also come with the following 11 \LaTeX\ Font Definition files:

\begin{itemize}
\item \texttt{omlcmssm.fd}
\item \texttt{omscmsssy.fd}
\item \texttt{omxcmssex.fd}

\item \texttt{ot1cmsmf.fd}
\item \texttt{ot1xcmss.fd}
\item \texttt{t1xcmss.fd}

\item \texttt{ucmsmf.fd}
\item \texttt{ussesint.fd}
\item \texttt{ussmsa.fd}
\item \texttt{ussmsb.fd}
\item \texttt{uxcmss.fd}
\end{itemize}

Finally, it should come with the font map file, LaTeX package, and documentation:

\begin{itemize}
\item \texttt{sansmathfonts.map}
\item \texttt{sansmathfonts.sty}
\item \texttt{sansmathfonts.tex}
\item \texttt{sansmathfonts.pdf}
\end{itemize}

% The subfolders should contain:

% doc: 2 files
% map: 1 file
% tex: 12 files

% New Type 1 fonts: 	137
% Old fonts as Type 1: 	9
% Virtual fonts:		28
% Supplementary .mf:	29

% source: 166 files (=137+29)
% tfm: 165 files	(=137+28)
% type1: 146 files  (=137+9)
% vf: 28  			(=28)

\section{License}

This work (the \textsf{sansmathfonts} package) consists of the
files listed in \autoref{sec:files}.

This work may be distributed and/or modified under the
conditions of the \LaTeX\ Project Public License, either
version 1.3 of this license or (at your option) any
later version.

The latest version of the license is in
\begin{quote}
\href {http://www.latex-project.org/lppl.txt} 
{\texttt{http://www.latex-project.org/lppl.txt}}
\end{quote}
and version 1.3 or later is part of all distributions of
\LaTeX\ version 2003/06/01 or later.

This work has the LPPL maintenance status ``author-maintained''.

Almost all of the Metafont files in this package are very closely based on existing files in the 2011 \TeX\ Live distribution; see comments near the start of the individual files for notes on their sources. Also, note that the 
files
\begin{itemize}
\item \texttt{cmssxi8.pfb}
\item \texttt{cmssxi9.pfb}
\item \texttt{cmssxi10.pfb}
\item \texttt{cmssxi12.pfb}
\item \texttt{cmssxi17.pfb}
\item \texttt{cmssbx8.pfb}
\item \texttt{cmssbx9.pfb}
\item \texttt{cmssbx12.pfb}
\item \texttt{cmssbx17.pfb}
\end{itemize}
were derived from unedited MetaFont source files in the \textsf{sauter} package using mftrace 1.2.18 and Fontforge. 



\end{document}