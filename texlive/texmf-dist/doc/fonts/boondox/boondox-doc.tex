\documentclass[11pt]{article}
\usepackage[margin=1.4in]{geometry} 
%\geometry{landscape}                % Activate for for rotated page geometry
\usepackage[parfill]{parskip}    % Activate to begin paragraphs with an empty line rather than an indent
\usepackage{url}\usepackage{graphicx}

\usepackage{hyperref}
\title{BOONDOX math alphabets}
\author{Michael Sharpe\\
msharpe at ucsd dot edu}
\date{}

\begin{document}
\maketitle
The BOONDOX fonts are PostScript versions of subsets of the STIX fonts corresponding to regular and bold weights of three alphabets---calligraphic, fraktur and double struck, \textsc{aka} blackboard bold. Support files are provided so that they can be called up from \LaTeX\ math mode using the commands \verb|\mathcal|, \verb|\mathbcal|, \verb|\mathfrak|, \verb|\mathbfrak|, \verb|\mathbb| and \verb|\mathbbb|. The font family name derives from the fact that, at least in the US, the phrase ``in the boondox'' implies ``in the stix.''

The base PostScript fonts were constructed from \textsf{STIXGeneral.otf} and \textsf{STIXGeneralBol.otf} using a \textsf{FontForge} script, resulting in 
\begin{verbatim}
zxxrl8a.pfb % BOONDOXDoubleStruck-Regular
zxxbl8a.pfb % BOONDOXDoubleStruck-Bold
zxxrw8a.pfb % BOONDOXCalligraphic-Regular
zxxbw8a.pfb % BOONDOXCalligraphic-Bold
zxxrf8a.pfb % BOONDOXFraktur-Regular
zxxbf8a.pfb % BOONDOXFraktur-Bold
\end{verbatim}
together with the corresponding {\tt.afm} files. (The names are almost Berry conformant: the initial {\tt z} warns that they break the rules, and the font id {\tt xx} is completely unblessed by any authority. The remaining parts are nearly OK, except that the font lack many glyphs normally in {\tt 8a} encoding, but all glyphs are in the correct slots.)

Using \textsf{afm2tfm}, the afm files were transformed to raw tfm files (kern information discarded)
\begin{verbatim}
zxxrl7z.tfm
zxxbl7z.tfm
zxxrw7z.tfm
zxxbw7z.tfm
zxxrf7z.tfm
zxxbf7z.tfm
zxxrow7z.tfm % same as zxxrw7z, less oblique
zxxbow7z.tfm % same as zxxbw7z, less oblique
\end{verbatim}
which serve as the basis for further virtual math fonts. Finally, using \textsf{FontForge} scripts and manual adjustments to the metrics to suit my personal taste, produces (no pretense of using Berry names):
\begin{verbatim}
BOONDOX-r-cal.tfm
BOONDOX-b-cal.tfm
BOONDOX-r-calo.tfm
BOONDOX-b-calo.tfm
BOONDOX-r-frak.tfm
BOONDOX-b-frak.tfm
BOONDOX-r-ds.tfm
BOONDOX-b-ds.tfm
\end{verbatim}
and the corresponding {\tt.vf} files. 

There are two ways to use these. The traditional method is based on  {\tt.sty} and {\tt.fd} files for each pair: {\tt BOONDOX-cal}, {\tt BOONDOX-calo}, {\tt BOONDOX-frak} and {\tt BOONDOX-ds}. For example,
\begin{verbatim}
\usepackage[scaled=.95]{BOONDOX-ds}
\end{verbatim}
defines the output from \verb|\mathbb| and \verb|\mathbbb| to come from {\tt BOONDOX-r-ds} and {\tt BOONDOX-b-ds} respectively, scaled to 95\% of normal size, and 
\begin{verbatim}
\usepackage{BOONDOX-calo}
\end{verbatim}
defines the output from \verb|\mathcal| and \verb|\mathbcal| to come from {\tt BOONDOX-r-calo} and {\tt BOONDOX-b-calo} respectively. (These are \emph{less} oblique versions of the {\tt BOONDOX} calligraphic fonts.)

The second method uses a different interface not depending on the {\tt.sty} and {\tt.fd} files at all. The package \textsf{mathalfa	} permits you to say
\begin{verbatim}
\usepackage[bb=boondox,bbscaled=.95,cal=boondoxo]{mathalfa}
\end{verbatim}
to accomplish the same effect as the above.

See the \textsf{mathalfa} documentation for font samples of these and many other math alphabets.

\end{document}