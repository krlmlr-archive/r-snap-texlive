\documentclass[12pt,a4paper,openany]{book}
\usepackage[EU1]{fontenc}
\newcommand{\UTFencname}{EU1}
\newcommand{\cyrillicencoding}{EU1}

\usepackage[russian,english]{babel}

\usepackage{fontspec}
\usepackage{xunicode}
\defaultfontfeatures{Mapping=tex-text}

\setmainfont{OldStandard-Regular}

\hoffset=-1in
\voffset=-1in
\oddsidemargin=30mm
\evensidemargin=20mm
\textwidth=160mm
\textheight=240mm

\catcode"2019=12
\lccode"2019="2019

\usepackage{array}
\usepackage[dvipdfm,colorlinks=true]{hyperref}

\providecommand{\XeTeX}{X\kern-.125em\lower.5ex\hbox{Ǝ}\kern-.1667em\TeX}

\makeatletter
\renewcommand\tableofcontents{%
\chapter*{\contentsname}%
\thispagestyle{empty}
\markboth{Table of contents}{Table of contents}
\@starttoc{toc}%
}
\makeatother
\sloppy

\begin{document}

\pagestyle{empty}

\vspace*{\stretch{0.4}}

\begin{center}

{\fontsize{48}{56}\selectfont OLD STANDARD}

\bigskip

{\huge A Unicode Font\\ for Classical and Medieval Studies

}

\bigskip

\rule{\textwidth}{0.5pt}

\bigskip

{\Large\itshape User’s manual\\
Version 2.0
}

\bigskip

\rule{\textwidth}{0.5pt}

\vspace{\stretch{1}}

{\Large Alexey Kryukov

}

\vspace{\stretch{0.6}}

\end{center}

\clearpage

\vspace*{\stretch{1}}

This manual is set in Old Standard with Latin Modern fonts used for missing 
styles (e. g. typewriter fonts).

\vspace{\stretch{0.1}}

Copyright © 2006--2008 Alexey Kryukov.

Permission is granted to copy, distribute and/or modify this document
under the terms of the GNU Free Documentation License, Version 1.2
or any later version published by the Free Software Foundation;
with no Invariant Sections, no Front-Cover Texts, and no Back-Cover Texts.
A copy of the license is included in the section entitled “GNU
Free Documentation License”.

\vspace{\stretch{0.2}}

\clearpage

\setcounter{page}{1}

\tableofcontents

\pagestyle{headings}

\chapter{About Old Standard}
\thispagestyle{empty}

Everybody who has ever thumbed through any old books printed in the late
19\textsuperscript{th} or early 20\textsuperscript{th} century may have
noted a specific typeface style most commonly used at that time: basically,
a variation of the modern (classicist) antiqua, but less contrast and more
legible. This group of typefaces also had an accompanying style of italics
with some specific shapes: \textit{k} with the upper leg terminating with a
rounded ball, open bowl on \textit{g} (again, with a rounded ball at its end),
curved bowl on \textit{y} and so on. May be, you was wandering, why it is so
difficult to find a digital typeface of similar style, despite of the vast
amount of computer fonts currently available. In general, the Modern style
was almost completely abandoned in the middle 20\textsuperscript{th}
century, as it no longer corresponded to the tastes of the time; moreover,
contemporary typographers often consider this lettertype obsolete and
out-of-fashion due to its “unnaturality”.

Nevertheless, the classicist style in general, and its modification used in
early 20\textsuperscript{th} century in particular, has at least one
advantage: it is still very suitable for typesetting scientific papers,
especially on social and humanitarian sciences, as its specific features
are closely associated in the people’s eyes with old books they learned on.
However, it would be even more important to stress the fact that the book
printing in many non-Western languages first appeared or was greatly
improved in 19\textsuperscript{th} century, and thus many classical
typefaces for non-Latin scripts (the most beautiful examples of Greek and
Cyrillic lettertypes in particular) were designed to be harmonizable with
the Modern faces~— the standard Roman printing style of the time.

That’s why the Modern style should be considered an extremely good choice
for typesetting multilingual texts, and so I am really surprised that still
nobody has attempted to implement a multilingual typeface on this basis.
Instead, multilingual typesetting is usually done with Times-styled fonts,
which eliminate specific features of each script instead of stressing them.
This is the main reason for which I have designed Old Standard, a
multilingual font which attempts to revive the most common printing style
of early 20\textsuperscript{th} century. Old Standard has two main
purposes: it is intended to be used as a specialized font for philologists
(mainly classicists and slavists) and also as a general-purpose font for
typesetting various editions in languages which use Greek or Cyrillic
script. For this reason Old Standard provides glyphs for a wide range of
Latin, Greek and Cyrillic characters.

\section{Origin and Design}

Old Standard was first intended as a digital version of
\textit{Обыкновенная} (Standard) typeface found in the following font
catalogues printed in the Soviet Union: 

\begin{figure}

\centerline{\XeTeXpicfile "stand-su.png" width 160mm}

\caption{The regular version of the Russian “Standard” typeface from the
1966 font catalogue}

\label{fig:stand-su}

\end{figure}

\begin{otherlanguage}{russian}

\begin{itemize}

\item Каталог ручных и машинных шрифтов. М.: Книга, 1966.

\item Каталог ручных шрифтов и наборных украшений. Харьков: Прапор, 1973.

\end{itemize}

\end{otherlanguage}

That’s where the name originates from: I have have only added “Old” to
stress the difference from \textit{Обыкновенная Новая} (“New Standard”)~—
another, a bit similar and yet quite different typeface, much more
popular in the Soviet typography. Currently there is a good digital version
of New Standard, available from \href{http://www.paratype.ru}{Paratype},
so I was not planning to reproduce it.

Later, however, I realized that the \textit{Обыкновенная} typeface, as it
was used in Soviet printing of the second half of the
20\textsuperscript{th} century, is not an independent family, but rather a
bunch of various sets inherited from pre-1917 Russian typography. So I had
to improve the initial design basing mainly on various Russian and German
editions of the late 19\textsuperscript{th} and early
20\textsuperscript{th} centuries, mainly manuals of ancient languages and
editions of classical (Greek and Latin) authors, where I could find good
examples of Latin, Greek, and, in case of Russian books, also Cyrillic
letters, used alongside. I have also brought the following font catalogue,
which, unlike later Soviet catalogues, contains examples of several
“Standard” typefaces, so that I could compare the letterforms and select
those I considered the most elegant:
\foreignlanguage{russian}{Государственный трест ВСНХ «Полиграф». Образцы
шрифтов. М., 1927}.

Thus the current version of Old Standard doesn’t reproduce any particular
typeface, but rather attempts to revive the general style of the early
20\textsuperscript{th} century typography (mostly Russian and German).
Nevertheless, I have decided to keep the initial name: of course, it
doesn’t look very original, but seems to be a good choice for a lettertype
which was once so common that no special name was associated with it
(typefaces of this style are usually called just “Standard” or “Modern” in
old font catalogues).

\section{Greek font design}

The Greek characters in Old Standard require a separate note. The upright
letters follow the style first introduced by famous French typecutter
Firmin Didot and then widely used in various editions both in Greece itself
and many other European countries. It would be no exaggeration to state that
the most part of Greek editions printed in continental Europe for more than
100 yers was set with Didot faces. So it is no wonder that digital versions
of this design have already been created by several type foundries. However
almost all these fonts either cover just the Greek script and provide no
support for Latin (not to say Cyrillic) characters, or combine Didot's Greek
design with a stylistically incompatible (usually Times-styled) Latin face.
Most of them (even some hightly overpriced commercial products) also 
don't meet my quality standards.

\begin{figure}[htb]

\centerline{\XeTeXpicfile "didot-fr.png" width 160mm}

\caption{An excerpt from a French edition typeset with a Didot face. The 
example is taken from: Les hanrangues de Démosthène. Text grec publié 
d’après les travaux les plus récents de la philologie avec un commentaire 
critique et explicatif, une introduction générale et des notices sur chaque 
discours par Henri Weil. Deuxieme édition entèrement revue et corrigée. 
Paris, 1881.}

\hypertarget{fig:didot-fr}{}\label{fig:didot-fr}

\end{figure}

A notable exception is
\href{http://www.greekfontsociety.org/pages/en\_typefaces.html}{GFS Didot},
now available for free from \href{http://www.greekfontsociety.gr/}{Greek
Font Society}. Unlike many others, the designers of this font did care about
a matching Latin face, but, surprisingly, their choice has nothing to do
with the classicist style: instead, they implemented their font as an
accompanying Greek family for Adobe Palatino. For this reason the proportions
and metrics of GFS Didot are quite different from those of original Greek
Didot; in particular ascenders and descenders are significantly shorter.
The Unicode version now comes with its own Latin alphabet, but, again, it
is based mostly on the Palatino design, although some glyph features are
adapted to the geometrical shapes of Greek capitals. The resulting font
may be very elegant, but, again, it is not suitable to reproduce the authentic
look of old editions, and essentally should have not been called Didot
due to a different style of its Latin part.

\begin{figure}[htb]

\centerline{\XeTeXpicfile "didot-de.png" width 160mm}

\caption{A modification of the Didot style, used in German editions. The
example is taken from: Herodoti Historiae. Recensuit Henricus Stein. Tomus
II. Berolini, 1871. P.~318.}

\hypertarget{fig:didot-de}{}\label{fig:didot-de}

\end{figure}

It should also be noted that the historic Didot style had several variations;
in particular its \hyperlink{fig:didot-de}{German version} (popular also in
Russia) is slightly different from the \hyperlink{fig:didot-fr}{font used in 
French editions of the same time}. Old Standard seems to be the only digital
typeface which follows mostly the German and Russian understanding of the
Didot style, although for some characters (e.~g. Greek circumflex) I have 
preferred French forms, considering them more elegant.

\begin{figure}[htb]

\centerline{\XeTeXpicfile "teubner.png" width 160mm}

\caption{An example of the Teubner Greek font, taken from: Herodotus für
den Schulgebrauch erklärt. Von Dr. K.~Abicht, Direktor des Gymnasiums zu
Ols. Vierter Band. Buch~VIII. Dritte verbesserte Auflage. Leipzig, 1882.
S.~192.}

\hypertarget{fig:teubner}{}\label{fig:teubner}

\end{figure}

Designing an italic style for a Greek typeface represents a separate
problem. Most modern implementations of Greek Didot are accompanied with
italic versions obtained by applying a slant to the upright glyphs. I have
chosen a different solution: instead of creating a slanted version of
the Didot family (completely unknown to the traditional typography), I have
based my italics on various cursive Greek fonts actually used in the German
typography of the early 20\textsuperscript{th} century. The most elegant of
those fonts was the face used by the famous Teubner publishing house in
Leipzig for their editions of classical authors.

Surprisingly, until recently nobody has attempted to implement a digital version
of the \hyperlink{fig:teubner}{Teubner Greek font}, and this is a pity,
because Teubner editions are still considered a model of fine Greek
printing in Germany, Russia and, I think, many other European countries,
exactly like the Loeb classical library in the Anglo-American world. It
should be noted here that the actual Teubner typeface is sometimes
confused with another cursive Greek font, also called “λιπσιακό” in Greece,
which does have some digital implementations, in particular Monotype Greek
91 and the grml/grbl fonts which Claudio Beccari has designed to provide a
matching italic font for his CB Greek package. Indeed, a similar font was
sometimes used in Leipzig editions (mainly for headings), but it is quite
different from the standard text face these editions are set with.

I should admit however, that even Old Standard Italic doesn’t provide an
authentic reproduction of the Teubner font. The problem is that the Greek
letters used in Leipzig editions are a bit bolder than their accompanying
Latin face, so that it was really difficult to bring them into a better
correspondence with Latin and Cyrillic glyphs. That’s why I had to consider
also some less elegant, but lighter Greek typefaces used by other printing
houses in Germany at the same time. I hope however that the general style
of the Teubner font is preserved, so that anybody who likes Leipzig
editions of classical authors will like Old Standard as well.

\chapter{Installation and Usage}
\thispagestyle{empty}

\section{Obtaining Old Standard}

If you are reading this document, then you probably have already downloaded
Old Standard. You may check if you have the most recent version by visiting
the following page at the Thessalonica web site: 

\href{http://www.thessalonica.org.ru/en/fonts.html}
{\texttt{http://www.thessalonica.org.ru/en/fonts.html}}

This page contains information about all font projects I am currently
developing and download links.

%You may download Old Standard either as a standalone package, or with
%Thessalonica for Microsoft Word 97/2000/XP/2003~— my utility designed to
%simplify the multilingual input in this word processor.

\section{Which format to prefer?}

The Old Standard font family is currently available in two formats, so that
before downloading fonts you should consider with which software you are
planning to use them:

\begin{description}

\item[\XeTeXpicfile "truetype.png"] TrueType fonts, or, more precisely,
\hyperlink{OT}{OpenType} fonts with TrueType outlines. These fonts use the
file extension *.ttf, and under Windows they appear in a folder or on a
disk using a dog-eared page icon with two overlapping “T’s” (for
TrueType);

\item[\XeTeXpicfile "opentype.png"] \hyperlink{OT}{OpenType} fonts with
PostScript outlines (also called OpenType-CFF), with an *.otf extension.
The Windows icon for a PostScript-flavored OpenType font is a dog-eared
page icon with an “O” (for OpenType).

\end{description}

It is worth pointing out, that, despite of the old “TT” icon used by
Windows to represent them, the TrueType fonts actually support the
same set of advanced \hyperlink{OT}{OpenType} features as their
OpenType-CFF counterparts (see \autoref{i18n} for information on
how to take advantage of those features).

Note that you can install both TrueType and OpenType-CFF versions
alongside, as they use different naming conventions (the TrueType fonts
have a “TT” suffix in their names). However, in most cases you will
probably prefer the TrueType fonts, as this format was introduced long
time ago and now is widely supported on various platforms and operational
systems.

OpenType-CFF is a relatively new font format, which is sometimes preferred
over TrueType for the purposes of the desktop publishing. A system-level
support for this font format was first introduced in Windows 2000, but in
practice it was poorly supported by Microsoft software until Office 2003
was released. In particular, before Office 2003 one could not use
OpenType-CFF fonts to input a non-Latin text in Microsoft Word, although it
was possible to apply them to an existing text. Even now some problems
still persist: the most important one is that in most applications
\textit{kerning}\footnote{Kerning is the adjustment of space between pairs
of letters, especially by placing two characters closer together than
normal. Kerning makes certain combinations of letters, such as WA, MW, TA
or VA, look better. Kerning data is specific for each particular font and
for this reason is normally specified in the font file; carefully designed
fonts normally have a large number of kerning pairs.} will work only for
the first 256 characters in the font. Of course this means that you can get
kerning working neither for Greek nor for Cyrillic letters. The only
exception are Adobe’s desktop publishing applications (InDesign,
Illustrator, Photoshop), which don’t have this problem and provide a really
good support for OpenType-CFF fonts. I hope the situation will change in
future, but now I can recommend installing the OpenType-CFF versions under
Windows only if you a planning to use them with the Adobe software.

You can install OpenType-CFF fonts also in Unix-like systems (since this
format is supported by the FreeType library), but be aware that here you
can face even more problems than under Windows. The worst thing is that
OpenOffice.org currently can’t access OpenType-CFF fonts at all, although
this issue was already several times reported to the developers.  Some
other applications have no such problem, but they are still unable to
access OpenType kerning information. The exceptions are rare: the only GUI
application which allows to take a full advantage of the OpenType-CFF fonts
is Scribus, an open source desktop publishing system. You can also use
OpenType fonts with the Linux port of
\href{http://scripts.sil.org/xetex}{\XeTeX}~— the Unicode version of the
\TeX{} typesetting system. In particular this manual was set with \XeTeX{}
and the OpenType-CFF versions of the fonts.

\section{Source Package}

You also can download the FontForge sources of the Old Standard font family.
Of course this package may be useful for you only if you have the 
\href{http://fontforge.sourceforge.net}{FontForge} font editor, as well
as some other font editing utilities, and know how to use them.
Note that downloading the source package may make a sense for you only
if you are going to apply some modifications to the original files, i.~e.
to prepare your own version of the fonts. Please consult the 
\hyperlink{license}{Terms of Use} section of this document to see which
license conditions should be met when distributing such derivative works.

Sometimes I am getting e-mails from packagers of Linux distributions
asking if they could build Old Standard from sources just like they used to
do for application executables. Well, I can't prohibit this (as the fonts
are available under a free license and even the name itself is not reserved,
as explained \hyperlink{license}{below}) but \textbf{I strongly discourage
doing so}. The reason is that, despite the common name, font sources aren't
very much like application sources, and similarly TTF or OTF fonts have very
few common with compiled programs. When an application is built from sources,
the resulting files are usually suitable only for a particular platform or
system and cannot be used in other environments. Fonts represent just an
opposite case: font sources are specific for a particular font editing
application, while the output files are suitable for various platforms and
can be easily disassembled/opened/edited. 

This means rebuilding fonts from sources will not give you any productivity 
improvements, but you can easily lose some functionality (e. g. because 
your FontForge version doesn't work exactly like one I used to build the
original font files). That's why I can recommend this approach only of you
know what are you doing and your intent is to apply some real changes/
improvements to the font sources.

\section{System Requirements}

\subsection{Windows}

Old Standard is a large Unicode font.

For Windows, you need at least Windows 95 (or at least Windows 2000 for the
PostScript-flavored OpenType fonts) and a word processor that can handle
Unicode-based documents: either Microsoft Word 97~/ 2000~/ 2002~(=~XP)~/
2003~/ 2007 or OpenOffice.org 1.0 or above. For more information about
OpenOffice.org, a full-featured, open-source cross-platform suite
comparable to Microsoft Office that is attracting considerable interest
these days, see
\href{http://www.openoffice.org}{\texttt{http://www.openoffice.org}}.

You will also need a way to enter the Unicode characters that are not
directly accessible from standard keyboards. Remember that you can browse
the contents of any font and copy characters to the clipboard by using the
Character Map utility that comes with Windows. Character Map does not
support Unicode values beyond the Basic Multilingual Plane; an excellent
alternative is Andrew West’s
\href{http://www.babelstone.co.uk/Software/BabelMap.html}{BabelMap}
(free). Some applications also provide their own mechanisms for entering
characters, such as Insert→Symbol in MS Word or Insert→Special Character
in OpenOffice.org. In Microsoft Office applications you can also enter 
a Unicode character by typing its hexadecimal number followed by
\texttt{ALT-x}.

Of course inputting Unicode characters via a character table or accessing
them directly by their hexadecimal codes has some significant
disadvantages: first, it is relatively slow and so may be used only for
characters which you need relatively rare, and second, it may be
recommended only for experienced users, since Unicode includes a lot of
similar characters, which, however, are intended for different purposes, so
that sometimes it is difficult to make the correct choice without
consulting the documentation. So normally you will need a special keyboard
utility allowing to input characters needed for the language of your
choice. Some custom keyboard layouts for such languages as Classical Greek
are provided by my \href{http://www.thessalonica.org.ru}{Thessalonica}
package. Alternatively, you may use
\href{http://www.tavultesoft.com/keyman/}{Tavultesoft Keyman}~— the
leading keyboard mapping utility, providing an extensive range of features.
There is a large number of keyboard layouts already designed for
Tavultesoft Keyman, so you probably just have to check
\href{http://www.tavultesoft.com/keyman/downloads/keyboards/}{the list of
available keyboard} to select one or more which are suitable for your
needs.

\subsection{Linux and X11 Windowing Environment}

Most Unix-like systems now use the same basic framework, called X Window
System (commonly X or X11) to build graphical user interfaces. This means
that all issues related with font installation and usage are basically the
same, no matter, if you use Linux, BSD, Solaris or some other system. In
order to be able to handle TrueType or OpenType fonts your system should
have the \href{http://freetype.sourceforge.net}{freetype} library installed
and enabled; this is normally done by default in all modern distributions.
As under Windows, you will need a Unicode-aware word processor. Presumable
you will do most of your work in OpenOffice.org; other, less powerful word
processors, like AbiWord or KWord, support Unicode as well.

As under Windows, you may input Unicode characters using either a character
map utility (both the most full-featured X11-based desktop environments,
KDE and Gnome, include such utilities, comparable with the Windows
Character Map), or a special keyboard driver. Again, you can try
\href{http://www.thessalonica.org.ru}{Thessalonica} for OpenOffice.org.
Another good choice is \href{href://kmfl.sourceforge.net}{kmfl}~— a
keyboarding input method which aims to bring Tavultesoft Keyman
functionality to *nix operating systems. KMFL is being jointly developed by
\href{http://www.sil.org}{SIL International} and
\href{http://www.tavultesoft.com}{Tavultesoft}. Note that KMFL is not
available by default in some popular Linux distributions, so that
you may have to compile, install and configure it yourself. This task
is a bit difficult for an average user, but the result surely worth
efforts.

\section{Installation Instructions}

\subsection{Windows}

Font installation under Windows is simple. You can install Old Standard as
you would any TrueType or OpenType-CFF font by placing the font files to
the Windows \texttt{fonts} folder. To do that:

\begin{enumerate}

\item Go to the Windows Control Panel and open the “Fonts” applet;

\item On the File menu, select “Install New Font\ldots”;

\item Switch to the drive and directory that contain the fonts you want to
add;

\item To select more than one font to add, press and hold down the CTRL
key, click the fonts you want, then click on OK.

\end{enumerate}

You may need to restart some applications before they can access the fonts
you have just installed.

\subsection{Linux and X11}

Currently there are no prepackaged RPM or DEB files for Old Standard, but,
of course, you can always install the fonts manually, which is actually not
so complex task with modern Linux distributions. A tricky part is related
with the fact that there are actually two engines responsible for font
installation and handling in X11 environment:
\href{http://www.fontconfig.org}{fontconfig} and an older X11 engine. Since
fontconfig is used by almost all recent applications (including those
based on GTK2 and QT4), in most cases it is sufficient to install fonts
via fontconfig (this is the only option in case of OpenType-CFF fonts). On
most distributions you can do that just by placing the font files to your
\texttt{~/.fonts} directory. After that you may need to run

\texttt{\$ fc-cache}

\noindent from your command line to update your fontconfig configuration.
You can also use a graphical font installation tool provided by KDE (the
most powerful graphical desktop environment for X11), but be aware that
this tool actually does just the things described above, i.~e. copies the
fonts to the appropriate directory and runs \texttt{fc-cache}.

However, if you want to make TrueType fonts accessible to some older X11
applications, then additional steps are required:

\begin{enumerate}

\item Find the place in your directory tree where your X stores TTF fonts.
The usual place is \texttt{/usr/X11R6/lib/X11/fonts/truetype} and the
subdirectories therein;

\item create under that location a subdirectory for the fonts you are going
to install, for example:

\texttt{\$ mkdir /usr/X11R6/lib/X11/fonts/truetype/oldstand}.

You should become root to do that. Then copy the *.ttf files there:

\texttt{\$ cp *.ttf /usr/X11R6/lib/X11/fonts/oldstand/};

\item switch to the directory where you have just copied the font files and
run the following commands:

\texttt{\$ ttmkfdir > fonts.scale}
\texttt{\$ mkfontdir}

\item Now the hardest part: we have to inform your X server about the path
where the recently installed fonts are placed. This can be done by
two ways:

\begin{enumerate}

\item in most distributions fonts are managed directly by the X11 system.
In this case the information about font paths is stored in the main X11
server configuration file, which is located under \texttt{/etc/X11} and
may be called \texttt{xorg.conf}, \texttt{XF66Config} or
\texttt{XF86Config-4} depending from your distribution and the version of
the X11 server it uses. So open that file in your favorite text editor,
and add the following line to the “Files” section:

\texttt{FontPath "/usr/X11R6/lib/X11/fonts/oldstand/"};

\item some Linux distributions (\href{http://www.altlinux.ru}{Alt Linux} in
particular) handle fonts using a special X Font Server (xfs). You can
easily determine if your distribution belongs to this second group, as in
this case the only “FontPath” element in your \texttt{xorg.conf} or
\texttt{XF86Config} will look as follows:

\texttt{FontPath "unix/:-1"}

If you have noticed such a line in your main X11 configuration file, you
should keep it untouched and instead edit the \texttt{/etc/X11/xfs/config}
file and add the new font path there.

\end{enumerate}

\item Finally, if everything is done correctly, the fonts will be
accessible for X11 applications when you restart your X Server. However,
you can also activate your new fonts immediately. Again, this can be done
by two ways:

\begin{enumerate}

\item if your system doesn’t use xfs, then you should execute the following
commands:

\texttt{\$ xset fp+ /usr/X11R6/lib/X11/fonts/oldstand/}

\texttt{\$ xset rehash}

\item otherwise you have to restart your X Font Server. Usually this
can be done by executing

\texttt{\$ service xfs restart}

\end{enumerate}

\end{enumerate}

\subsection{OpenOffice.org}

Under MS Windows OpenOffice.org just uses system-wide installed fonts, but
Unix versions have their own font administration utility, inherited from
the dark times when no suitable engine that would be able to properly
handle scalable fonts existed at the X11 level. Normally OpenOffice.org can
automatically detect X11 fonts and add them to its configuration (so no
additional steps are required), but sometimes it fails to find them. In this 
case you should let OpenOffice.org know about your new fonts using the
\texttt{spadmin} utility. You can either run this tool manually from your 
OpenOffice.org directory, or select the “OpenOffice.org printer administration”
GUI menu item in KDE or Gnome (you should close any open OpenOffice.org 
instances before you can do this). When the \texttt{spadmin} window appears, 
do the following:

\begin{figure}[htb]

\centerline{\XeTeXpicfile "spadmin.png" width 160mm}

\caption{The OpenOffice.org printer administration utility: main window}

\hypertarget{fig:spadmin}{}\label{fig:spadmin}

\end{figure}

\begin{enumerate}

\item click on the “Fonts...” button;

\item click on "Add...;

\item look for the directory where the fonts are installed\\ (e.~g.
\texttt{/usr/share/fonts/truetype/oldstand/}), as \autoref{fig:spadmin-add}
shows;

\begin{figure}[htb]

\centerline{\XeTeXpicfile "spadmin-add.png" width 140mm}

\caption{Adding new fonts to OpenOffice.org via spadmin}

\hypertarget{fig:spadmin-add}{}\label{fig:spadmin-add}

\end{figure}

\item Click on “Select all”;

\item Click on OK.

\end{enumerate}

When you restart OpenOffice.org, the fonts should be available to its
applications.

\subsection{TeX systems}

Adding new fonts to a \TeX{} installation is always difficult for an
average user, as in order to use a font with \TeX{} typesetting system one
has to generate many additional files, \TeX{} font metrics files (TFM) in
particular. Yet I still haven’t provided a \TeX{} support package for Old
Standard, mainly because Old Standard currently has only three shapes
(regular, italic and bold), and thus such a package would have very limited
functionality from the \TeX{} point of view. However, you can easily use
Old Standard (as well as any other TrueType or OpenType-CFF font) in your
\TeX{} documents without any additional steps if you install
\href{http://scripts.sil.org/xetex}{\XeTeX}~— a Unicode enabled version of
the \TeX{} compiler, currently actively developed by
\href{http://www.sil.org}{SIL international}.

\XeTeX{} has many other advantages over traditional \TeX{} compilers, as it
combines the full Unicode support with a very good support of advanced
\hyperlink{OT}{OpenType} features. In particular, this manual (including
all examples demonstrating smart font rendering features available in Old
Standard) was typeset with \XeTeX.

\section{Terms of use}
\hypertarget{license}{}

The current version of Old Standard is distributed under the
\href{http://scripts.sil.org/OFL}{SIL Open Font License} (OFL) v.~1.1. I have
selected OFL for my typeface because it is the only known license developed
specially for fonts, which meets the standards of the FLOSS (Free/Libre and
Open Source Software) community, in particular the Debian Free Software
Guidelines. Both the text of the license itself and the OFL FAQ are
included into the fonts package, so I don’t reproduce them here. Basically
licensing under OFL means that you can freely use, copy, modify and
distribute the fonts, as long as the terms of the license are not violated.
In particular you are not allowed to remove the original copyright notices
from the font software and to change licensing conditions (i.~e. distribute
either original or modified versions under a different license). One more
significant restriction is that you can’t sell the fonts alone (however
OFL allows to bundle and sell them together with any other software, either
free or commercial).

A large part of OFL is devoted to so-called Reserved Font Names which can't
be used in derivative works without a written permission of the original
author. However there are no Reserved Font Names specified for Old Standard.
This is because I think I can't prohibit anybody from using such common
words as “Old” or “Standard” in their font names. In fact I even encourage
you to base names of your modified versions on the original one, so that 
the user can easily determine where the main design comes from. For example,
if you have modified Old Standard in order to get Serbian Cyrillic glyphs
displayed by default instead of Russian ones, it might be logical to call
your version “Old Standard Serbian”. It is still desired however that you
don't take the original name as is, but add some suffix specific for your
version.

Note that this manual is NOT covered by SIL OFL, but distributed under the
\href{http://gnu.mirror.fr/licenses/fdl.html}{GNU Free Documentation
license}. See \autoref{FDL} for more information.

\section{Acknowledgments}

I would like to thank:

\begin{itemize}

\item George Williams for his excellent
\href{http://fontforge.sourceforge.net}{FontForge} program, and
especially for his responsiveness in fixing bugs and adding new features.
Without his assistance this package would never be released!

\item Peter Baker for his \href{http://xgridfit.sourceforge.net}{xgridfit}
utility, which provides a good Open Source solution for adding TrueType
instructions to a font, and also for valuable information on the design of
the Middle English letter yogh he provided;

\item Tavmjong Bah (Tav), who kindly granted me his Perl scripts (originally
written for his \href{http://tavmjong.free.fr/FONTS/}{Arev fonts}) used to
convert separate kerning pairs defined in a FontForge source file into
kerning classes;

\item \href{http://canopus.iacp.dvo.ru/~panov/}{Andrew Panov} for valuable
remarks on the design of mathematical characters and scanned images he
provided.

\end{itemize}

\chapter{Multilingual Support, Unicode and OpenType}
\hypertarget{i18n}{}\label{i18n}
\thispagestyle{empty}

\section{Unicode coverage}

\subsection{General principles}

Since Old Standard is a multilingual font family, I will always do my best
to extend the range of supported characters, thus providing support for
more languages. Nevertheless, I would like to protect my typeface from some
problems shared by many similar free font projects. The developers of those
fonts are often attempting to cover the widest possible number of scripts
and Unicode blocks, even if the Unicode
\href{http://www.unicode.org/charts}{code charts} is the only source of
their knowledge about the design of a specific character. Of course, the
resulting glyphs not always look really acceptable for actual typesetting.
Moreover, due to the lack of time and resources the designers are often
unable to keep all glyphs at the same quality level: for example, we often
can see autogenerated accented characters with mispositioned diacritics. In
particular, there are so many fonts which are claimed to support the
extended Greek range, but actually are not suitable for typesetting
classical Greek\ldots{} Another common problem is that only the regular
version of each particular font is really actively developed, while all
additional weights and shapes fall far behind it (e.~g. support much less
Unicode characters).

That’s why I have formulated for myself several principles which I am
always trying to follow when designing additional glyphs:

\begin{itemize}

\item I shall never add any new characters just for completeness, i.~e. to
get a specific Unicode range fully covered. Before drawing a new glyph I
must ensure that I really understand its intended purpose and the
principles of its design;

\item since Old Standard is supposed to reproduce the actual printing style
of the early 20\textsuperscript{th} century, I shall avoid implementing new
characters basing just on general considerations. Ideally, all glyphs
should be based on real examples taken from some old editions. Of course,
exceptions from this rule are sometimes necessary, as many characters were
first introduced only in 20\textsuperscript{th} century, or even never
existed in traditional typography before they were adopted by the Unicode
standard;

\item I shall try to develop all font styles (currently regular,
italic and bold) only simultaneously, i.~e. if a specific character is added 
to the regular font, it should also be designed for italics and bold. The 
exceptions are allowed for glyphs which don't have dedicated codepoints
and supposed to be accessible via smart font features, as well as for those 
characters which have no corresponding italic or slanted style (this is the case 
of many mathematical symbols).

\end{itemize}

\subsection{Character repertoire}

Currently the following Unicode ranges are fully or partially covered by
Old Standard:

\begin{description}

\item[Basic Latin (0000–007F)] Fully supported.

\item[Latin 1 Supplement (0080–00FF)] Fully supported.

\item[Latin Extended-A (0100–017F)] Fully supported.

\item[Latin Extended-B (0180–024F)] Old Standard implements two groups of
characters from this block, namely several letters needed for various Old
Germanic languages and Croatian accented characters and digraphs.

\item[IPA Extensions (0250–02AF)] From this range Old Standard currently
implements a few characters which can be used in other contexts, except IPA.
One such example is U+0280 LATIN LETTER SMALL CAPITAL R, needed for the
transliteration of Old Norse runic inscriptions.

\item[Spacing Modifier Letters (02B0–02FF)] Old Standard implements spacing
versions of some combining diacritical marks, available in the next block.

\item[Combining Diacritical Marks (0300–036F)] Most standard accents,
commonly used in various European languages, are supported.

\item[Greek and Coptic (0370–03FF)] Almost fully covered, except Coptic
letters, editorial signs and some archaic characters, recently added to
Unicode. These characters are rarely used and there is no stable tradition
of their typographic representation. However, if you really need them,
write me and I’ll add them for you.

\item[Cyrillic (0400–04FF)] Old Standard implements all modern Slavic
(i.~e. Russian, Uk\-rai\-nian, Byelorussian, Serbian and Macedonian)
characters, as well as historical characters and extensions for Old
Slavonic.

\item[Phonetic Extensions (1D00–1D7F)] Only one character (U+1D79 LATIN
SMALL LETTER INSULAR G) is implemented. Note that the uppercase version 
of this letterform is now encoded in Latin Extended-D range.

\item[Latin Extended Additional (1E00–1EFF)] Old Standard implements 
accented combinations with W for Welsh and LATIN CAPITAL LETTER SHARP S
(U+1E9E) for German.

\item[Greek Extended (1F00–1FFF)] Fully supported.

\item[General Punctuation (2000–206F)] Almost fully supported, although
some characters, which might be of some interest for philologists, are
still missing. Report me if you need them.

\item[Superscripts and Subscripts (2070–209F)] Subscript and superscript
forms of digits and math operators (but not letters), available in this
block, are covered.

\item[Currency Symbols (20A0–20CF)] The EURO SIGN U+20AC.

\item[Letterlike Symbols (2100–214F)] In this block Old Standard implements
a few characters, belonging to the following two categories: first, a few
standard symbols, present in most Western or Cyrillic fonts (in particular
NUMERO SIGN U+2116, TRADE MARK SIGN U+2122 and OHM SIGN U+2126), and second,
some characters which may be useful for textual criticism (such as Fraktur ℭ
and ℌ).

\item[Number Forms (2150–218F)] Fully covered.

\item[Mathematical Operators (2200–22FF)] This block is far from being
finished, and yet it already includes (I hope) all symbols which are most
commonly used in mathematical typesetting.

\item[Miscellaneous Technical (2300–23FF)] In this block Old Standard
implements angle brackets U+2329 and U+232A (these characters should
probably be avoided: use “mathematical” angle brackets at U+27E8/U+27E9
instead) and ancient metrical symbols (23D1—23D9).

\item[Geometric Shapes (25A0–25FF)] Old Standard implements only a few of
these symbols, for compatibility with legacy fonts and charsets.

\item[Miscellaneous Mathematical Symbols-A (27C0–27EF)] Old Standard
implements mathematical angle, square and double angle brackets (useful
also for critical text editions).

\item[Supplemental Mathematical Operators (2A00–2AFF)] In this block I have
implemented only a few characters, in particular alternate “less than” and
“greater than” symbols with a slanted bar, which actually where preferred
forms in the traditional European typesetting before the arrive of modern
standards.

\item[Cyrillic Extended-A (2DE0–2DFF)] Fully covered.

\item[CJK Symbols and Punctuation (3000–303F)] Again, Old Standard includes
angle and square brackets at U+3008/U+3009 and U+301A/U+301B
correspondingly, as some people have used to use them for textual
criticism. Nevertheless “mathematical” versions of those characters (see
above) should probably be preferred for their purposes.

\item[Cyrillic Extended-B (A640–A69F)] Old Standard implements letters and
signs for Old Cyrillic (but not letters for old Abkhasian orthography).

\item[Latin Extended-D (A720–A7FF)] LATIN CAPITAL LETTER INSULAR G (U+A77D)
and Ancient Roman epigraphic letters.

\item[Private Use Area] This block includes many different glyphs, but it
is not recommended to use them directly. Instead, you should access them
by applying various OpenType features (see \autoref{OT} for more
information), if only your application allows this.

\item[Alphabetic Presentation Forms (FB00–FB4F)] In this block the standard
Latin f-ligatures are available.

\item[Math Alphanumeric Symbols (1D400–1D7FF)] Old Standard includes a few
Fraktur letters, useful for critical editions of ancient/biblical texts.
This block is far from being complete (and I am not planning to implement
the whole alphabet anyway); however, it already includes all characters
which appear in the Nestle---Aland New Testament.

\end{description}

\subsection{TODO}

As you can see, still lots of characters are waiting to be implemented.
Since Old Standard is oriented mainly to historians and philologists, I am
especially interested in adding those characters which might be useful for
textual studies and studying various ancient languages. Here are some
priorities:

\begin{itemize}

\item Some characters useful for medievalists are still missing from the
Latin Extended-B range;

\item Some accented characters from the Latin Extended Additional range
(the first part, 1E00-1E9F), useful for Indic and Semitic transliterations.
However, currently you already can type almost anyone of those characters
using combining diacritical marks, if only your application supports smart
accent positioning (as Microsoft Word 2003 for example);

\item Some IPA characters (at least those needed for English phonetic
transcription);

\item Supplemental Punctuation (2E00—2E0D) and New Testament editorial
symbols in particular. Again, I need some good examples showing these
characters used in editions of the pre-computer era;

\item A large group of medievalist additions has been adopted in Unicode~5.1.
Of course it would be nice to implement them in Old Standard.

\end{itemize}

Currently I have no plans of providing support for any other scripts except
Latin, Greek and Cyrillic, as there are other needs that are much higher
priority. It is also very unlikely that I can implement small capitals and
some other nice typographic features in the near future, although it would
be really nice to have them supported.

\subsection{How you can help}

If you would like to get a specific character available in Old Standard,
then probably the best help you can offer is to provide some high
resolution (normally 600dpi) scans showing you character used in an old
book, where the rest of text is set with a Modern typeface (this condition
is especially important for additional Latin letters). If it is impossible
to find such examples (e.~g. because your character had not yet been
introduced at the time when Modern typefaces were popular), then at least 
provide a clear description on how it should be designed (or point me to a
such description). Also remember that, except the upright character, I will
have to implement also an italic version, and the design of italic glyphs
may often require additional notes.

Of course you can also design the desired character(s) yourself and then
contribute them to Old Standard. Such contributions are always very
welcome, but be aware that I will review the submissions carefully in order
to be able to guarantee a high level of quality for the fonts. Please
don’t be discouraged if I do not include a submission for this reason, or
ask you to make some specific revisions.

\section{Smart Font Capabilities and Language Support}

This section is intended to demonstrate, how Old Standard can be used for
typesetting texts in various languages. This assumes discussing two types
of issues: “smart” font rendering features intended to provide a better
support for each specific language and some glyph design peculiarities. As
the implementation of “smart” features in Old Standard is based on the
OpenType technology, I had to start this section from a special paragraph
about OpenType. After that the manual describes various language-specific
details, sorting them by scripts: Latin, Greek and Cyrillic.

\subsection{What is OpenType?}
\hypertarget{OT}{}\label{OT}

OpenType is a smart font rendering technology, that allows proper
typographic treatment of complex scripts and advanced typographic effects
for simpler scripts. This is achieved by applying various
\textit{features}, or \textit{tags}, described in the OpenType
specification. Some of those features are supposed to be enabled by
default, while others are considered optional. In order to get advantage of
all those advanced typographic features, you need two basic components: a
“smart” font including certain extra tables, where the features applicable
to this font are specified, and an OpenType-aware application. Not all
applications currently support OpenType, although their number is growing.
So before relying on any smart features provided by Old Standard or another
typeface you should carefully examine which of those features are expected
to work in your application, and which are not.

The most popular OpenType rendering engine for Windows platform is the
\textit{Uniscribe} library, developed by Microsoft. This library is used
not only by own Microsoft software, but also by many other Windows
applications, for example, the Windows version of
\href{http://www.openoffice.org}{OpenOffice.org}. Initially Uniscribe
supported only complex scripts (like Arabic or Devanagari), but the most
recent versions, supplied with Microsoft Windows XP SP2 and Microsoft
Office 2003 (note that MS Office uses its own version of Uniscribe rather
than the system library) also perform some processing for Latin, Greek and
Cyrillic. The Uniscribe support for Western scripts is still limited:
Microsoft Word 2003 performs only \hyperlink{mark}{accent positioning} and
\hyperlink{ccmp}{character composition/decomposition}. On the other hand,
the supported features are actually the most important ones, and they are
really sufficient for proper text rendering, although without additional
typographic niceties.

Adobe’s applications (such as InDesign) use another shaping engine, called
\textit{CoolType}, which provides access to many optional features offered 
by OpenType, such as small caps, stylistic sets and various types of ligatures. 
Old Standard currently supports some of those optional features, such as 
stylistic sets. To tell the truth, this functionality is very important from
the point of view of a fine typography, but in most cases almost useless for 
a linguist. However beginning from the CS3 version Adobe Creativity Suite
applications are said to support a wider range of OT features, including 
mark positioning and glyph composition/decomposition, which makes them much 
more suitable for typesetting linguistic texts when previously.

In the Unix world, there are at least two free OpenType rendering
libraries. One such library is \textit{Pango}, used in applications based
on the GTK2 toolkit. This library currently has nearly the same
capabilities as MS Uniscribe (although still there are some glitches).
Another, even more powerful rendering engine is
\href{http://icu.sourceforge.net}{\textit{ICU}}, used by \XeTeX. ICU
properly handles virtually all features provided by Old Standard, even those 
not supported by most other rendering engines (language-dependent substitutions 
for example). Unix versions of \href{http://www.openoffice.org}{OpenOffice.org}
also use ICU, but, unfortunately, this is not very useful for our purposes,
as they enable complex text processing only for complex scripts.

I know very little about Mac, but I have to mention that many applications for
this platform also have a very good level of OpenType support. One such
application is \href{http://www.redlers.com}{Mellel}, the leading word
processor for Mac OS X, designed to serve scholars, creative and technical
writing and multilingual word processing.

\subsection{Latin Script}

\subsubsection{Standard Ligatures}

Old Standard currently includes 5 standard f-ligatures (namely \textit{ff,
fi, fl, ffi} and \textit{ffl}) present in most OpenType fonts and also
\textit{fj} and \textit{ffj} ligatures which are required for proper
typesetting in Nordic languages. All these ligatures are accessible via the
\texttt{liga} feature, enabled by default in most applications which
support it (such as Adobe InDesign). Two language-dependent exceptions have
been made from this rule, according to the common convention usually
applied to OpenType fonts:

\begin{itemize}

\item Turkish, Azerbaijani and Crimean Tatar alphabets have two distinct
versions of the letter \textit{i}, one dotted and the other dotless. For
this reason the \textit{fi} and \textit{ffi} ligatures are not applied for
those language systems to avoid the confusion which would be possible
otherwise.

\item No ligatures are enabled by default for German, since this language
has very complex rules of ligature processing. You still can get them if
you enable the \texttt{dlig} feature tag in addition to \texttt{liga}.

\end{itemize}

Note that the exceptions described above will work as expected only if your
application can perform OpenType processing depending from the current
language.

\subsubsection{Combining Mark Positioning}
\hypertarget{mark}{}

One of the most attractive possibilities offered by OpenType is smart
diacritic positioning: if you type a letter followed by a diacritic from
the Unicode “Combining Diacritics” range, the diacritic will be placed
exactly above or below the letter. To achieve this effect, an OpenType
font should support the \texttt{mark} feature tag. This feature allows to
add \textit{anchor points} both to base letters and diacritics, so that,
when an accent mark is typed after a base character, the glyphs are
positioned by such a way that their anchor points are coincident. Another
type of anchor points, specified by the \texttt{mkmk} feature, is used to
position two marks with respect to each other, so that an additional
diacritic can be stacked properly above the first. 

Old Standard provides proper \texttt{mark} and \texttt{mkmk} anchor points
for most Latin letters and combining marks, so that you can type them in
almost any combination and the result will be visually identical with the
corresponding precomposed accented characters (in case they are available
in the font). Most OpenType renderers (except older versions of Adobe’s 
Cooltype library) support the corresponding feature tags, and so you can
safely use these features in most OpenType-aware applications (MS Word 2003
for example).

\subsubsection{Unicode Composition and Decomposition}
\hypertarget{ccmp}{}

Another important OpenType feature is \texttt{ccmp}. This feature allows
to decompose a character into two glyphs or, on the contrary, to compose
two characters into a single glyph for better glyph processing. Often such
substitutions correspond to canonical (de)compositions specified in the
Unicode character database, but this is not a required condition. So if we
would like to replace a specific glyph or a group of glyphs with another
glyph or a group of glyphs, such replacement can almost always be
implemented via \texttt{ccmp}: the only important limitation here is that
this feature is not supposed to (an often just cannot) be turned off, and
thus it should not be used for optional typographic refinements, such as
Latin ligatures.

Old Standard uses \texttt{ccmp} mainly to compose accented glyphs from an
accent and a base character in those cases where a simple accent positioning
would not produce the desired result. For example, the Czech alphabet has
some accented characters (\textit{ď}, \textit{ľ}, \textit{ť}) where the
accent is identified with the haček (caron), but actually looks like an
apostrophe. So when you type \textit{d}, \textit{l} or \textit{t} followed
by combining haček, Old Standard just substitutes the corresponding Czech
character for you.

There are also some situations where \texttt{mark} and \texttt{ccmp} should
be used together to produce a better result. For example, before you can
place an accent above letters like \textit{i} or \textit{j} you have to
replace the base letter with a dotless variant first, and this can be done
only with \texttt{ccmp}. For this reason all OpenType renderers which
support accent positioning support also this feature (Word 2003 does).

\subsubsection{Stylistic Sets}

Stylistic sets are used to enable a group of stylistic variant glyphs,
designed to harmonize visually, and make them automatically substituted
instead of the default forms. OpenType allows to specify up to 20 stylistic
sets, marking them \texttt{ss01}, \texttt{ss02}\ldots{} \texttt{ss20}. The
following stylistic sets, currently available in Old Standard, are relevant
for the Latin script:

\begin{description}

\item[ss01] This set allows to automatically substitute small and capital
\textit{s} and \textit{t} with commaaccent (U+0218, U+0219, U+021A, U+021B)
instead of the corresponding letters with cedilla (U+015E, U+015F, U+0162,
U+0163), as required by Romanian typographic rules. The same substitution
can be done automatically for Romanian and Moldavian languages, if only
your application supports the \texttt{local} feature tag; otherwise you can
use \texttt{ss01} instead. Of course this is important only if the glyph
variants with commaaccent are not typed directly (which is also possible,
as now those letterforms have separate Unicode codepoints).

\begin{center}
\LARGE
\begin{tabular}[c]{ccc}

\fontspec[Script=Latin,Color=696969]{OldStandard-Regular}
raţiune şi conştiinţă & ⇒ &
\fontspec[Script=Latin,Language=Romanian]{OldStandard-Regular}
raţiune şi conştiinţă \\
\fontspec[Script=Latin,Color=696969]{OldStandard-Regular}
\itshape raţiune şi conştiinţă & ⇒ &\itshape 
\fontspec[Script=Latin,Language=Romanian]{OldStandard-Regular}
raţiune şi conştiinţă \\

\end{tabular}
\end{center}

\item[ss02] By enabling this feature tag you can get all occurrences of
small and capital Latin \textit{g} automatically replaced with “insular”
forms, sometimes preferred for typesetting Old English:

\begin{center}
\LARGE
\begin{tabular}[c]{ccc}

\fontspec[Script=Latin,Color=696969]{OldStandard-Regular}
Gosfregð & ⇒ &
\fontspec[Script=Latin,RawFeature=+ss02]{OldStandard-Regular}
Gosfregð \\
\fontspec[Script=Latin,Color=696969]{OldStandard-Regular}
\itshape Gosfregð & ⇒ &\itshape 
\fontspec[Script=Latin,RawFeature=+ss02]{OldStandard-Regular}
Gosfregð \\

\end{tabular}
\end{center}

This stylistic set is preserved for backwards compatibility: I no longer 
recommend using it, as both capital and small insular \textit{g} now 
have dedicated Unicode codepoints, and it is probably better to type them
directly.

\end{description}

\subsubsection{Sample Text Fragments in Old and Classical Languages}

\paragraph{Classical Latin}

Of course classical Latin is supported. Just an example:

\begin{quote}
\large
\fontspec[Script=Latin,Language=Latin]{OldStandard-Regular}
Gallia est omnis divīsa in partes tres, quārum unam incŏlunt Belgae, aliam
Aquitāni, tertiam qui ipsōrum lingua Celtae, nostra Galli appellantur. Hi
omnes lingua, institūtis, legĭbus inter se diffĕrunt. Gallos ab Aquitānis
Garumna flumen, a Belgis Matrŏna et Sequăna divĭdit. Horum omnium
fortissimi sunt Belgae, propterea quod a cultu atque humanitāte provinciae
longissime absunt, minimeque ad eos mercatōres saepe commeant atque ea, quae
ad effeminandos anĭmos pertĭnent, important, proximique sunt Germānis, qui
trans Rhenum incŏlunt, quibuscum continenter bellum gerunt. Qua de causa
Helvetii quoque relĭquos Gallos virtūte praecēdunt, quod fere cotidiānis
proeliis cum Germānis contendunt, cum aut suis finĭbus eos prohĭbent aut
ipsi in eōrum finĭbus bellum gerunt.

\smallskip

\itshape
Gallia est omnis divīsa in partes tres, quārum unam incŏlunt Belgae, aliam
Aquitāni, tertiam qui ipsōrum lingua Celtae, nostra Galli appellantur. Hi
omnes lingua, institūtis, legĭbus inter se diffĕrunt. Gallos ab Aquitānis
Garumna flumen, a Belgis Matrŏna et Sequăna divĭdit. Horum omnium
fortissimi sunt Belgae, propterea quod a cultu atque humanitāte provinciae
longissime absunt, minimeque ad eos mercatōres saepe commeant atque ea, quae
ad effeminandos anĭmos pertĭnent, important, proximique sunt Germānis, qui
trans Rhenum incŏlunt, quibuscum continenter bellum gerunt. Qua de causa
Helvetii quoque relĭquos Gallos virtūte praecēdunt, quod fere cotidiānis
proeliis cum Germānis contendunt, cum aut suis finĭbus eos prohĭbent aut
ipsi in eōrum finĭbus bellum gerunt.

\end{quote}

\paragraph{Old English}

The following text (a writ from William the Conqueror to the citizens of
London, 1066) demonstrates several specific characters used in Old English.
Note the insular “G” automatically substituted instead of the regular Latin
“G” by applying the stylistic set 02.

\begin{quote}
\large

\fontspec[Script=Latin,Language=English,RawFeature=+ss02]{OldStandard-Regular}
Will(el)m kyng gret Will(el)m bisceop and Gosfregð portirēfan and ealle þā
burhwaru binnan Londone, Frencisce and Englisce, frēond\-līce. And ic kȳðe ēow
þæt ic wylle þæt get bēon eallre þǣra laga weorðe þē gyt wǣran on Eadwerdes
dæge kynges. And ic wylle þæt ǣlc cyld bēo his fæder yrfnume æfter his
fæder dæge. And ic nelle geþolian þæt ǣnig man ēow ǣnig wrang bēode. God
ēow gehealde!

\smallskip

\itshape
Will(el)m kyng gret Will(el)m bisceop and Gosfregð portirēfan and ealle þā
burhwaru binnan Londone, Frencisce and Englisce, frēond\-līce. And ic kȳðe ēow
þæt ic wylle þæt get bēon eallre þǣra laga weorðe þē gyt wǣran on Eadwerdes
dæge kynges. And ic wylle þæt ǣlc cyld bēo his fæder yrfnume æfter his
fæder dæge. And ic nelle geþolian þæt ǣnig man ēow ǣnig wrang bēode. God
ēow gehealde!

\end{quote}

\paragraph{Middle English}

No special typographic features are required for typesetting Middle
English, so the following example just demonstrates some characters,
specific for this language, in particular the \textit{ȝ} (yogh):

\begin{quote}
\large

\fontspec[Script=Latin,Language=English]{OldStandard-Regular}
Our Lord, which ich shal douten, is my liȝtyng and my helpe. Our Lord is
defendour of my lif; for what þyng shal ich drede? To þat noiand comen
neȝe vp me, þat hij etand my flesshes: Myn enemys, þat trubleden me, ben
made sike, and hij fellen. Ȝif hij setten manaces oȝains me, myn hert ne
shal nouȝt drede. Ȝyf myn enemy arere bataile oȝains me, y shal hopen in
þat. Ich asked þe lif þat euer shal last of our Lord; ich shal bisechen
þat, þat ich mai wonne in þe hous of our Lord alle þe daies of my lif; Þat
ich se þe wille of our Lord and uisite his temple.

\smallskip

\itshape
Our Lord, which ich shal douten, is my liȝtyng and my helpe. Our Lord is
defendour of my lif; for what þyng shal ich drede? To þat noiand comen
neȝe vp me, þat hij etand my flesshes: Myn enemys, þat trubleden me, ben
made sike, and hij fellen. Ȝif hij setten manaces oȝains me, myn hert ne
shal nouȝt drede. Ȝyf myn enemy arere bataile oȝains me, y shal hopen in
þat. Ich asked þe lif þat euer shal last of our Lord; ich shal bisechen
þat, þat ich mai wonne in þe hous of our Lord alle þe daies of my lif; Þat
ich se þe wille of our Lord and uisite his temple.

\end{quote}

\paragraph{Gothic Transliteration}

Two additional letters are used in Gothic transliteration: \textit{þ}
(þiuþ, thorn) and \textit{ƕ} (hwair). Both of them are available in Old
Standard:

\begin{quote}
\large

\fontspec[Script=Latin]{OldStandard-Regular}
Akei ik sunja izwis qiþa: batizo ist izwis ei ik galeiþau; unte jabai ik ni
galeiþa, parakletus ni qimiþ at izwis; aþþan jabai gagga, sandja ina du
izwis. Jah qimands is gasakiþ þo manaseþ bi frawaurht jah bi garaihtiþa
jah bi staua; bi frawaurht raihtis, þata þatei ni galaubjand du mis; iþ bi
garaihtiþa, þatei du attin meinamma gagga, jah ni þanaseiþs saiƕiþ mik;
iþ bi staua, þatei sa reiks þis fairƕaus afdomiþs warþ.

\smallskip

\itshape
Akei ik sunja izwis qiþa: batizo ist izwis ei ik galeiþau; unte jabai ik ni
galeiþa, parakletus ni qimiþ at izwis; aþþan jabai gagga, sandja ina du
izwis. Jah qimands is gasakiþ þo manaseþ bi frawaurht jah bi garaihtiþa
jah bi staua; bi frawaurht raihtis, þata þatei ni galaubjand du mis; iþ bi
garaihtiþa, þatei du attin meinamma gagga, jah ni þanaseiþs saiƕiþ mik;
iþ bi staua, þatei sa reiks þis fairƕaus afdomiþs warþ.

\end{quote}
 
\paragraph{Old Icelandic}

A fragment of text in Old Icelandic. Note some specific letters used in
that language, as well as the \textit{fj} ligature.

\begin{quote}
\large

\fontspec[Script=Latin,Language=Icelandic]{OldStandard-Regular}
Kømr nú þessi fregn fyrir Hrólf konung ok kappa hans upp í kastalann, at
maðr mikilúðligr sé kominn til hallarinnar ok hafi drepit einn hirðmann
hans, ok vildu þeir láta drepa manninn. Hrólfr konungr spurðisk eptir,
hvárt hirðmaðrinn hefði verit saklauss drepinn. „Því var næsta“, sǫgðu
þeir. Kómusk þá fyrir Hrólf konung ǫll sannindi hér um. Hrólfr konungr
sagði þat skyldu fjarri, at drepa skyldi manninn~— „hafi þit hér illan
vanda upp tekit, at berja saklausa menn beinum; er mér í því óvirðing, en
yðr stór skǫmm, at gøra slíkt. Hefi ek jafnan rœtt um þetta áðr, ok hafi
þit at þessu engan gaum gefit, ok hygg ek at þessi maðr muni ekki alllítill
fyrir sér, er þér hafið nú á leitat; ok kallið hann til mín, svá at ek viti
hverr hann er“.

\smallskip

\itshape
Kømr nú þessi fregn fyrir Hrólf konung ok kappa hans upp í kastalann, at
maðr mikilúðligr sé kominn til hallarinnar ok hafi drepit einn hirðmann
hans, ok vildu þeir láta drepa manninn. Hrólfr konungr spurðisk eptir,
hvárt hirðmaðrinn hefði verit saklauss drepinn. „Því var næsta“, sǫgðu
þeir. Kómusk þá fyrir Hrólf konung ǫll sannindi hér um. Hrólfr konungr
sagði þat skyldu fjarri, at drepa skyldi manninn~— „hafi þit hér illan
vanda upp tekit, at berja saklausa menn beinum; er mér í því óvirðing, en
yðr stór skǫmm, at gøra slíkt. Hefi ek jafnan rœtt um þetta áðr, ok hafi
þit at þessu engan gaum gefit, ok hygg ek at þessi maðr muni ekki alllítill
fyrir sér, er þér hafið nú á leitat; ok kallið hann til mín, svá at ek viti
hverr hann er“.

\end{quote}

A special note is required on the shape of the Icelandic letter \textit{þ}
(thorn). In modern fonts this character’s design is almost always based on
the lowercase \textit{p} with an ascender added. This design is also the
only mentioned by Icelandic type designer Gunnlaugur SE Briem in his
article \href{http://briem.ismennt.is/2/2.11/index.htm}{Thorn and eth: how
to get them right}. And yet this letterform doesn’t look characteristic for
the traditional typography. Generally speaking, there were two styles of
\textit{thorn} most commonly used in the late 19\textsuperscript{th} and
early 20\textsuperscript{th} century printing:

\begin{itemize}

\item a glyph based on the lowercase \textit{p}, but with a double sided
serif at the top of the ascender;

\item a glyph with its top and bottom serifs positioned under an angle to
the vertical stem and the bowl stretched upwards.

\end{itemize}

In both cases the upper element often doesn’t reach the full ascender
height, which makes a significant advantage over the modern letterform
where the glyph often looks unbalanced due to the fact that the ascender is
significantly longer than the descender.

I have preferred the second form for the upright font, as it looks more
elegant and seems to be preferable for Old English and the Gothic
transliteration. However, it is important to stress the fact, that it is
also perfectly suitable for Norse languages. In particular it was actively
used for this purpose in the German printing, as for example the “Sammlung
kurzer Grammatiken Germanischer Dialekte” series, published in Halle a.S.
in early 20\textsuperscript{th} century and now, thanks to the
\href{http://www.ling.upenn.edu/~kurisuto/germanic/language\_resources.html}{Germanic
Lexicon Project}, available on the web in the form high resolution scans, can
demonstrate.

In the same books, however, the italic thorn already has the contemporary
style. So I have implemented this letterform too in the italic font
(where, indeed, it looks more appropriate than in the regular version).

\subsection{Greek Script}

\subsubsection{Alternate Forms}

In addition to the basic Greek alphabet the Unicode standard includes
alternate forms for several letters, such as script \textit{theta}, stroked
\textit{phi} and so on. These characters were included mainly for
compatibility with legacy character sets (Symbol for example), and using
them anywhere except mathematical contexts is strongly discouraged.
Nevertheless, the fact these characters are encoded causes a great mess by
itself, since it convinces font designers to think that any Greek typeface
can and should include two basic forms for several Greek letters, and that
some of these forms are always preferred for a Greek text, while others are
intended only for mathematical usage. Of course this assumption is wrong:
in fact all such letterforms are font-specific, so that normally only one
of them is stylistically compatible with each particular typeface.

That’s why, although OldStandard implements several alternate forms for
Greek letters, only a few of them can be considered really useful. The most
important of such exceptions is curly \textit{beta} U+03D0: this character,
indeed, should be available in any correct Greek font, since according to
the French typographic rules it is used instead of the regular
\textit{beta} with descender as a medial and final form (the same rule was
sometimes applied also in Greece itself). For this reason French
classicists often type U+03D0 directly in their documents, and particularly
I see nothing wrong in this practice, although it is not recommended by
Unicode. However, in a “smart” font it is also possible to implement
a contextual substitution rule, allowing initial/medial forms to be
automatically substituted at the correct places.

In Old Standard v.~1.0 I used contextual alternates (the \texttt{calt}
feature tag) for this purpose, but later I realized this feature is
normally enabled by default in most applications which support it, and,
since contextual forms are not very common in contemporary Greek publishing
outside France, most classicists would probably be discouraged if they appear
automatically in their texts. So now a stylistic set (\texttt{ss06}) 
is used instead.

\begin{figure}[htb]

\centerline{\XeTeXpicfile "greek-contextual.png" width 160mm}

\caption{Contextual forms of \textit{beta} and \textit{theta} in
traditional Greek typesetting. This example has been taken from: Ὡρολόγιον
τὸ μέγα, περιέχον ἁπάσαν τὴν ἀνήκουσαν αὐτῷ ἀκολουθίαν, κατὰ τὴν τάξιν τῆς
ἀνατολικῆς τοῦ Χριστοῦ ἐκκλησίας, καὶ ἐξαιρέτως τῶν ὑποκειμένων αὐτῇ εὐαγῶν
μοναστηρίων. Ἔκδοσις ἑβδόμη. Ἐν Βενετία, 1851. Σ.~32.}

\label{fig:greek-contextual}

\end{figure}

\textit{Theta} is another letter, which can have two different forms, both
of which are stylistically compatible with Didot faces. The Unicode
code chart displays the closed \textit{theta} \textit{\char"EB02} at U+03B8
(thus making it the default letterform), while the open, or script variant
form \textit{ϑ} is mapped to U+03D1 and intended only for mathematical
usage.  Most fonts currently follow this convention. Historically, however,
selection of one or another form has been made depending from national
typographic traditions. In particular, French and Greek publishers
certainly preferred the closed letterform, although in some
19\textsuperscript{th} century editions the open theta is used at the
beginning of words, i.~e. a rule, similar to one of \textit{beta}, is
applied (see \autoref{fig:greek-contextual} for example). On the other
hand, in German and Russian typography the open \textit{theta} was normally
used; this is also the only style of this letter found in the Teubner font
and other cursive Greek typefaces of a German origin.

Since my sources contained good examples of both open and closed
\textit{theta} in Didot-styled Greek fonts, I have implemented them both,
and have added a closed letterform even to the italic font for better
compatibility with the regular version. However, since Old Standard mainly
follows German typographic conventions, it seemed inappropriate to map this
form to U+03B8 and thus make it the only accessible glyph for the case
advanced Open Type features are not supported by user’s application.
Instead the following solution has been preferred: the open \textit{theta}
is mapped both to U+03B8 (GREEK SMALL LETTER THETA) and U+03D1 (GREEK THETA
SYMBOL), while the closed glyph is placed to a slot in the PUA and can be
automatically substituted instead of U+03B8 in one of the following
situations:

\begin{itemize}

\item in any postition, if the \texttt{mgrk} (Mathematical Greek) feature
tag is applied. Thus in mathematical contexts you can get a glyph mapping
which exactly corresponds to one defined by Unicode;

\item at the middle and the end of words, if stylistic set 06 (the
\texttt{ss06} feature tag) is applied.

\end{itemize}

Thus enabling \texttt{ss06} will turn on contextual forms both for
\textit{beta} and \textit{theta}, as demonstrated below:

\begin{center}
\LARGE
\begin{tabular}[c]{ccc}

\fontspec[Script=Greek,Color=696969]{OldStandard-Regular}
θαυμασθεὶς βάρβαρος & ⇒ &
\fontspec[Script=Greek,RawFeature=+ss06]{OldStandard-Regular}
θαυμασθεὶς βάρβαρος \\
\fontspec[Script=Greek,Color=696969]{OldStandard-Regular}
\itshape θαυμασθεὶς βάρβαρος & ⇒ & \itshape 
\fontspec[Script=Greek,RawFeature=+ss06]{OldStandard-Regular}
θαυμασθεὶς βάρβαρος \\

\end{tabular}
\end{center}

Note that the U+03D1 character will always be displayed as a script theta,
no matter, which feature tags you have applied.

The following example shows a fragment of Greek text with contextual
alternates (note the medial \textit{beta} and the closed \textit{theta}
substituted in the appropriate places):

\begin{quote}
\large

\fontspec[Script=Greek,RawFeature=+ss06]{OldStandard-Regular}
Κῦρος δὲ συγκαλέσας τοὺς στρατηγοὺς καὶ λοχαγοὺς τῶν Ἑλλήνων συνεβουλεύετό
τε πῶς ἂν τὴν μάχην ποιοῖτο καὶ αὐτὸς παρῄνει θαρ\-ρύνων τοιάδε· «ὦ ἄνδρες
Ἕλληνες, οὐκ ἀνθρώπων ἀπορῶν [βαρβά\-ρων] συμ\-μάχους ὑμᾶς ἄγω, ἀλλὰ νομίζων
ἀμείνονας καὶ κρείττους πολλῶν βαρβάρων ὑμᾶς εἶναι, διὰ τοῦτο προσέλαβον.
ὅπως οὖν ἔσεσθε ἄνδρες ἄξιοι τῆς ἐλευθερίας ἧς κέκτησθε καὶ ἧς ὑμᾶς ἐγὼ
εὐδαιμονίζω. εὖ γὰρ ἴστε ὅτι τὴν ἐλευθερίαν ἑλοίμην ἂν ἀντὶ ὧν ἔχω πάντων
καὶ ἄλλων πολλαπλασίων».

\itshape
Κῦρος δὲ συγκαλέσας τοὺς στρατηγοὺς καὶ λοχαγοὺς τῶν Ἑλλήνων συνεβουλεύετό
τε πῶς ἂν τὴν μάχην ποιοῖτο καὶ αὐτὸς παρῄνει θαρ\-ρύνων τοιάδε· «ὦ ἄνδρες
Ἕλληνες, οὐκ ἀνθρώπων ἀπορῶν [βαρβά\-ρων] συμ\-μάχους ὑμᾶς ἄγω, ἀλλὰ νομίζων
ἀμείνονας καὶ κρείττους πολλῶν βαρβάρων ὑμᾶς εἶναι, διὰ τοῦτο προσέλαβον.
ὅπως οὖν ἔσεσθε ἄνδρες ἄξιοι τῆς ἐλευθερίας ἧς κέκτησθε καὶ ἧς ὑμᾶς ἐγὼ
εὐδαιμονίζω. εὖ γὰρ ἴστε ὅτι τὴν ἐλευθερίαν ἑλοίμην ἂν ἀντὶ ὧν ἔχω πάντων
καὶ ἄλλων πολλαπλασίων».

\end{quote}

Old Standard also implements stroked \textit{phi} (U+03D5), omega-like
\textit{pi} (U+03D6) and lunate epsilon (U+03F5), but these glyphs are not
accessible via any OpenType features, and using them anywhere except
mathematical contexts is not recommended. In other cases I had just to put
the same glyph image to more than one Unicode slot.  For example, U+03F0
(GREEK KAPPA SYMBOL) just contains a reference to U+03BA (the regular Greek
kappa), as the only possible form for this letter both in the Didot and
Teubner styles is the x-shaped one. All such “alternate” codepoints are
supported only for compatibility with other existing Greek fonts and for
the case you have occasionally used them in some texts.

The same statement is true for the lunate \textit{sigma}, both small and
capital: although it is sometimes reasonable to use this form e. g. for 
typesetting papyrological texts (where word breaks and thus the usage of
final sigmas are sometimes not obvious), it is probably impossible to 
implement a lunate sigma fully conforming the Didot style. So I don't
recommend using this letterform and have implemented it mainly in order
to make existing documents which use this character (such as some texts 
from the \href{http://www.tlg.uci.edu}{Thesaurus Linguae Graecae} corpus)
readable.

\subsubsection{Combining Mark Positioning}

Unicode provides codepoints for all accented characters needed for the
standard Greek orthography, and yet this set is often insufficient for
classicists. The most common problem is combining a breathing and/or an
accent with a macron or a breve mark. Also one often has to put a macron,
a breve or a circumflex above \textit{epsilon} or \textit{omicron} when
publishing epigraphical documents, although such combinations make no sense
for literary Greek. For this reason some Unicode Greek fonts include a huge
number of additional accented characters in the Unicode Private Use Area.
The most important problem here is that each vendor uses its own
arrangement of PUA slots, so that fonts are often incompatible with each
other, especially because very few of them have more or less correct OT
layouts allowing to access those glyphs without typing them directly.

Old Standard uses a different approach: it has a carefully adjusted set of
anchor points and \texttt{ccmp} rules, which allow to correctly position
accent marks relatively to each other and combine breathings with accents
to specially designed combinations, when necessary. Moreover, when you
type a capital letter folowed by one or more accents, these accents are
placed \textit{before} the letter, and the letter itself is shifted right
to the necessary amount of space. Thus you can type any possible accented
combination using combining marks, if only your application supports smart
accent positioning (but this is not a problem at least with Microsoft Word
2003 and above). Note that you should observe the following order of typing
diacritics: first a macron or a breve, then a breathing and finally an
accent. For example, combining marks were used to type the following
fragment of the Mantinea inscription:

\begin{quote}
\large

\fontspec[Script=Greek]{OldStandard-Regular}
ὀσ̱έοι ἂν χρε̄στε̄́ριον κακρίνε̄ || ε̄̓̀ γνο̄σίαι κακριθε̄́ε̄ το̄͂ν χρε̄μάτο̄ν, | πὲ τοῖς
ϝοικιάται(ς) τᾶς θεο̄͂ ε̄̓͂ναι, κὰ ϝοικίας δάσασθαι τὰς ἂν ο̄̓͂δ᾽ ἐάσας. εἰ τοῖς
ϝο̄φλε̄κόσι ἐπὶ τοῖδ’ ἐδικάσαμε[ν], | ἄ τε θεὸς κὰς οἰ δικασσταὶ,
ἀπυσ̱εδομίν[ος] || το̄͂ν χρε̄μάτο̄ν τὸ λάχος, ἀπεχομίνος | κὰ το̄̓ρρέντερον γένος
ε̄̓͂ναι | ἄματα πάντα ἀπὺ τοῖ ἰεροῖ, ἴλαον ε̄̓͂ναι.

\smallskip

\itshape
ὀσ̱έοι ἂν χρε̄στε̄́ριον κακρίνε̄ || ε̄̓̀ γνο̄σίαι κακριθε̄́ε̄ το̄͂ν χρε̄μάτο̄ν, | πὲ τοῖς
ϝοικιάται(ς) τᾶς θεο̄͂ ε̄̓͂ναι, κὰ ϝοικίας δάσασθαι τὰς ἂν ο̄̓͂δ᾽ ἐάσας. εἰ τοῖς
ϝο̄φλε̄κόσι ἐπὶ τοῖδ’ ἐδικάσαμε[ν], | ἄ τε θεὸς κὰς οἰ δικασσταὶ,
ἀπυσ̱εδομίν[ος] || το̄͂ν χρε̄μάτο̄ν τὸ λάχος, ἀπεχομίνος | κὰ το̄̓ρρέντερον γένος
ε̄̓͂ναι | ἄματα πάντα ἀπὺ τοῖ ἰεροῖ, ἴλαον ε̄̓͂ναι.

\end{quote}

Old Standard includes also several precomposed accented Greek characters in
the PUA, added for compatibility with
\href{http://www.users.dircon.co.uk/~hancock/index.htm}{Ralph Hancock}’s
fonts. However, you should use those characters with a caution and only if
your application doesn’t support combining mark positioning.

\subsubsection{Tilda-Shaped Circumflex vs. Lunate Circumflex}

Greek circumflex (perispomeni) often becomes a matter of discussions. I
know, that some (mostly English and American) classicists prefer porsonic
(lunate) circumflex, similar to an inverted breve, mainly because this form is
characteristic for most Greek fonts traditionally used in English and
American typography. However, in fact the preferred design of this accent
is a purely font specific question. For most typefaces of the continental
European origin (such as Didot or Teubner) only the tilde-shaped form is
acceptable, as inverted breve just cannot be harmonized with most letters.
So, don’t ask me to implement a version with “porsonic” circumflex.

\subsubsection{Iota Adscript vs. Iota Subscript}

Combinations of Greek vowels with “mute” iota, defined in Unicode, is one
more important group of glyphs, which may be designed by various ways,
depending from the designer’s preferences. Most ancient Greek language
manuals state that mute iota (called \textit{iota subscript}) is written
below lowercase letters, but after capital vowels a regular small iota,
written inline and so called \textit{iota adscript}, should be used
instead.  Currently most Unicode Greek fonts follow this convention, and
many classicists even suppose any over implementations of uppercase
combinations with mute iota to be illegal.

However, iota subscript below capital letters also may occur in some
editions. In particular, this orthography is very common for liturgical
books of the Greek Orthodox church. Particularly I prefer this typographic
tradition, not only because it is inherited from fine Greek typography of
the past centuries, but also for some technical reasons. The problem here
is that, if a mute iota is designed as a regular iota and printed inline,
it should behave as a separate character. This means that, when
letterspacing for the surrounding text is changed, the distance between the
iota and the preceding vowel should be scaled too. Of course this is
impossible if both characters are implemented as a single glyph.

That’s why in Old Standard mute iota is implemented as iota subscript in
all accented combinations with capital vowels. Note that \textit{unaccented}
capital vowels with mute iota represent a special case: unlike their
accented counterparts, they are used in upper case only, i.~e. may occur
only in a fully capitalized text. So for these glyphs (namely, Unicode
characters U+1FBC, U+1FCC, U+1FFC) I have designed a special version of
iota adscript, which looks like a \textit{capital} Iota, decreased in size.
To my mind, this shape will better match to the design of surrounding
capital glyphs.

Such an implementation of capital vowels with mute iota has nothing wrong
by itself, but, of course, it would be nice to allow replacing each of
affected Unicode codepoints with a pair of glyphs: the vowel itself and a
regular iota. Theoretically, this can be done by applying an OpenType
feature, but, unfortunately, I am not aware of any suitable OpenType
feature, which:

\begin{itemize}

\item can be used for replacing a single glyph with two or more glyphs, as in
our case;

\item can be disabled if a user doesn’t like it.

\end{itemize}

So for now I can’t provide an alternative for the implementation which
looks preferred for me. If you absolutely don’t like capital vowels with
iota subscript, at least you can always type regular iota as a separate
character.

\subsection{Cyrillic Script}

\subsubsection{Combining Mark Positioning}

Smart combining mark positioning is often necessary for Cyrillic.
Although the stress is usually not indicated in modern languages which use
the Cyrillic script, accentuation is still mandatory for textbooks,
dictionaries and books for children. This is especially important for
Serbian, which has long and short vowels and four types of accent.
Nevertheless, there are virtually no precomposed Cyrillic accented
characters in Unicode, so that using combining marks remains the only
option. So Old Standard provides all necessary anchor points allowing to
place accents above Cyrillic vowels (see the following sections about
Serbian and Old Slavonic for examples).

\subsubsection{Serbian Alternate Forms}

It is a well known fact, that, except several specific letters, Serbian
Cyrillic alphabet also has different preferred shapes for some letters
common for most languages which use the Cyrillic script. According to the
most common opinion, four Serbian variant forms are specific for the
italic style, while one can occur both in roman and italic styles, as
\autoref{fig:serbian} demonstrates. This practice was adopted by many font
designers, and Adobe even included Serbian variant forms into its Cyrillic
specification, although they have not been accepted by Unicode.

\begin{figure}

\centerline{\XeTeXpicfile "serbian.png" width 80mm}

\caption{Serbian and Macedonian variant forms. Russian norms on the left,
Serbian and Macedonian norms on the right}
\label{fig:serbian}

\end{figure}

However, after studying several examples of old Serbian printing (a small
\href{http://alas.matf.bg.ac.yu/biblioteka/home.jsp}{collection of such
examples} is available at the site of the faculty of Mathematics at the
Belgrade university) I have an impression that the modern practice is not
fully justified by the preceding tradition.

As far as I can see, there are only two letters (namely Cyrillic \textit{п}
and \textit{т}), which always have typically “Serbian” forms, clearly
distinct from their Russian counterparts. However, the late
19\textsuperscript{th} and early 20\textsuperscript{th} century editions,
set with Modern typefaces, also show a significant difference from the
contemporary “Serbian” style, as the horizontal bar (the most
characteristic feature of “Serbian” \textit{п} and \textit{т}) is attached
to the base glyph rather than positioned above it (like a diacritical
mark). The \textit{т} also may look like a slanted upright glyph, but I
have preferred to draw both \textit{п} and \textit{т} in the same
“historical” style.

It is especially important to stress that “Serbian” \textit{д} with a hook
below seems to never occur in the traditional Serbian printing, although
there was absolutely no problem to reproduce this form, if somebody
considered it correct, as Latin italic \textit{g} has exactly the same
shape in Modern typefaces of the early 20\textsuperscript{th} century. My
own opinion is that the contemporary Serbian letterform first appeared as a
result of uncritically reproducing the handwritten shape, erroneously
considered typically Serbian (actually it is not, as the same style is
preferred also in the Russian handwriting, which doesn’t mean this practice
should necessarily be reflected in printing). On the other hand, I have an
impression that the “Russian” italic \textit{д} (with an ascender) is also
not so common in Serbian printing: often it is replaced with a slanted
version of the upright letter. This glyph seemed a good compromise for me:
based on the historical tradition and at the same time certainly acceptable
for those Serbs who absolutely don’t like the Russian form.

In Old Standard only three italic letters, listed above (\textit{д},
\textit{п} and \textit{т}) form the default set of Serbian alternate forms,
which are automatically enabled when you mark a text with Serbian language.
Alternatively, if your application doesn’t support the \texttt{locl}
feature tag (which is probably the case) you can achieve the same result by
enabling the stylistic set 11. Here an example of a fully accentuated 
Serbian text, which demonstrates both the combining mark positioning and
the localized forms in the italic style:

\begin{quote}
\large

\fontspec[Script=Cyrillic,Language=Serbian]{OldStandard-Regular}
На но̀ве̄мбарско̄м су̑нцу прѐврће̄ се пр̏љава̄ у̏троба на̏ше̄ ку̏ће̄. Чу̏дно сам ту̑жан.
И док но̏сӣм с ма̑јко̄м си̑вӯ о̀тр̄ца̄нӯ сла̏марицу ту̑по за̀гледан у јѐдан мр̑твӣ
о̏бла̄к над цр̑нӣм, ни̏скӣм кро̀во\-вима на̏ше̄г прѐдгра̄ђа~— са̀плиће̄м се о пра̏г. О̀на
ми ка̑же̄: „Па̏зи“. О̏нда̄ бри̑жно: „Шта̏ ти је да̀нас?“ О̀на је ве̏лика гла̑дна̄ жѐна,
си́во̄ст ѝзбӣја̄ ѝз ње̄. У̀опште, све̏ је да̀нас си̑во. И не̏бо, и на̏ша а̀влија, и
шу́паљ, гра̀нат о̀рах сред ње̑, и о̀ве̄ на̏ше ства̑ри ко̀је̄, јѐдна по јѐдна, ѝзлазе̄
на да̑н.

\smallskip

\itshape
На но̀ве̄мбарско̄м су̑нцу прѐврће̄ се пр̏љава̄ у̏троба на̏ше̄ ку̏ће̄. Чу̏дно сам ту̑жан.
И док но̏сӣм с ма̑јко̄м си̑вӯ о̀тр̄ца̄нӯ сла̏ма\-рицу ту̑по за̀гледан у јѐдан мр̑твӣ
о̏бла̄к над цр̑нӣм, ни̏скӣм кро̀вовима на̏ше̄г прѐдгра̄ђа~— са̀плиће̄м се о пра̏г. О̀на
ми ка̑же̄: „Па̏зи“. О̏нда̄ бри̑жно: „Шта̏ ти је да̀нас?“ О̀на је ве̏лика гла̑д\-на̄ жѐна,
си́во̄ст ѝзбӣја̄ ѝз ње̄. У̀опште, све̏ је да̀нас си̑во. И не̏бо, и на̏ша а̀влија, и
шу́паљ, гра̀нат о̀рах сред ње̑, и о̀ве̄ на̏ше ства̑ри ко̀је̄, јѐдна по јѐдна, ѝзлазе̄
на да̑н.

\end{quote}

The case of the letter \textit{б} is basically the same as one of the
\textit{д}. The only difference here is that the “script” form actually
seems to be more common for Russian, than for Serbian printing, although in
the Russian tradition it is applicable only for the italic style. At least
it was used in the italic version of one particular “Standard” typeface of
early 20\textsuperscript{th} century. That’s why I have implemented this
letterform in Old Standard, although the italic version of the glyph is
actually based on a Russian source, and the upright shape has been added
just for completeness. These glyphs are not automatically applied for
Serbian text by default, but you can enable the stylistic set 12 to get
them substituted, as in the following example:

\begin{center}
\LARGE

\begin{tabular}[c]{ccc}

\fontspec[Script=Cyrillic,Language=Serbian,Color=696969]
{OldStandard-Regular}мртви облак & ⇒ &
\fontspec[Script=Cyrillic,Language=Serbian,RawFeature=+ss12]
{OldStandard-Regular}мртви облак \\
\fontspec[Script=Cyrillic,Language=Serbian,Color=696969]
{OldStandard-Regular}\itshape мртви облак & ⇒ &\itshape 
\fontspec[Script=Cyrillic,Language=Serbian,RawFeature=+ss12]
{OldStandard-Regular}мртви облак \\

\end{tabular}
\end{center}

Finally, the case of “Serbian” \textit{г} is a bit special: here the
specific shape is really justified by the peculiarities of the Serbian
handwriting tradition, and yet the letterform normally used in pre-computer
Serbian printing was typically Russian, i.~e. had no horizontal bar above.
Particularly I think the modern “Serbian” variant has nothing wrong by
itself, but, of course, it is difficult to implement it, if both
\textit{п} and \textit{т} are designed in the historical style, so that
there is no gap between the bar and the base glyph. Nevertheless I have
attempted to implement a Serbian \textit{г} in the same style as \textit{п}
and \textit{т}, basing on
\href{http://cirilica.com/cirilica/Strane/Slova/Azbuka.html}{the
recommendations by Nikola Kovanovich}, but this glyph is purely
experimental, and thus currently it is not accessible via any OpenType
features.

\subsubsection{Old Slavonic and Church Slavonic}

Until 2008, Unicode included only a subset of historical Cyrillic
characters, which was not sufficient for typesetting any actual texts.
Thus legacy encodings or PUA-based solutions were the only solution for 
representing historical documents in old Slavic languages which used
the Cyrillic script. In Unicode 5.1 the range of supported early 
Cyrillic characters was greatly extended and now includes all letters
and signs normally used in scientific publication and Orthodox liturgical 
books (including even combining letters). Beginning from the version
2.0 Old Standard fully supports historical Cyrillic, including Unicode
5.1 extensions.

However, except just having all necessary characters available in a font,
typesetting Old Slavonic or Church Slavonic also requires some complex
text rendering. So the following smart font features necessary for this
purpose are implemented in Old Standard:

\begin{description}

\item[Combining mark positioning] Old Slavonic (and especially modern
Church Slavonic) has a wide range of combining characters, such as accents,
breathings, titlos and superscript letters. Basically the accentuation
system is very similar to Greek one, but, unlike for Greek, there are no
precomposed accented characters available in Unicode, so that using
combining marks is the only option.

\item[Enclosing combining marks] Church Slavonic inherited from Greek
its numeric system, where numbers are denoted with letters. However, special
enclosing marks, shown in the following table, have been invented to denote
large numbers beginning from 10000:

\begin{center}
\begin{tabular}[c]
{|>{\fontspec[Script=Cyrillic,Language=Church Slavonic]{OldStandard-Regular}}
p{4em}|p{6em}|p{6em}|}
\hline
\textit{Notation}& \textit{Numerical meaning}& \textit{Old Slavonic name}\\
\hline
 \Large\hfil{а҃⃝}\hfil & 10\,000 & тьма\\
\hline
 \Large\hfil{а҃҈}\hfil & 100\,000 & легион\\
\hline
 \Large\hfil{а҃҉}\hfil & 1\,000\,000 & леодр\\
\hline
 \Large\hfil{а҃꙰ }\hfil & 10\,000\,000 & ворон\\
\hline
 \Large\hfil{а҃꙱ }\hfil & 100\,000\,000 & колода\\
\hline
\end{tabular}
\end{center}

Old Standard implements two types of OT lookups to achieve proper positioning
for this kind of marks: first, standard anchor points (the \texttt{mark}
feature) used to attach a mark to a base character, and second, contextual
positioning lookups allowing to increase the base character bearings and
advance width when it is followed by enclosing marks. Unfortunately, this
technique is not guaranteed to work in all OpenType-aware applications: in 
particular at the time this manual was written contextual positioning did 
not properly work in \XeTeX.

\item[Historic letterforms] Although the modern Cyrillic script (so called
“civil” style) is often used to typeset medieval texts, some of the modern
letterforms are especially closely associated with the typographic reform under
Peter the Great, and thus would look out of place in a historical context. 
That's why Old Standard provides some stylistic alternates, specially intended
for Old Russian and Old (Church) Slavonic:

\begin{center}
\LARGE
\begin{tabular}[c]{cccc}

\fontspec[Script=Cyrillic,Color=696969]{OldStandard-Regular}
не вѣдыи будущаго & 
⇒ & 
\fontspec[Script=Cyrillic,Language=Church Slavonic]{OldStandard-Regular}
не вѣдыи будущаго &\\

\fontspec[Script=Cyrillic,Color=696969]{OldStandard-Regular}
\itshape не вѣдыи будущаго & 
⇒ & 
\fontspec[Script=Cyrillic,Language=Church Slavonic]{OldStandard-Regular}
\itshape не вѣдыи будущаго &\\

\end{tabular}
\end{center}

These alternates are automatically enabled for Old Church Slavonic, if
your application understands the \texttt{locl} feature tag and allows
to mark a text with this language. Alternatively you can get the same
substitutions by applying the stylistic set 14 (\texttt{ss14}).

\item[Contextual letterforms] Some Cyrillic letters have tall ascenders,
while in medieval manuscripts the same letters normally did not extend above
x-height, so that it was possible to put an accent or a combining letter
above them. Old Standard includes special low forms for some of such letters
(namely \textit{б} and \textit{ѣ}) and can automatically subsitute them
when the letter is followed by an accent:

\begin{center}
\LARGE
\fontspec[Script=Cyrillic,Language=Church Slavonic]{OldStandard-Regular}

\begin{tabular}[c]{ccc}

\fontspec[Script=Cyrillic,Language=Church Slavonic,Color=696969]{OldStandard-Regular}
прѣдъ богомъ & ⇒ & прѣⷣ҇ бⷢ҇омъ \\
\fontspec[Script=Cyrillic,Language=Church Slavonic,Color=696969]{OldStandard-Regular}
\itshape прѣдъ богомъ & ⇒ &\itshape прѣⷣ҇ бⷢ҇омъ \\

\end{tabular}
\end{center}

\end{description} 

Finally an example of Old Slavonic text with combining marks and historic
letters and letterforms:

\begin{quote}
\large

\fontspec[Script=Cyrillic,Language=Church Slavonic]{OldStandard-Regular}
А҆ ѡ҆ сеⷨ Иракліи и҆ною̀ притчею рекоша сеⷤ Феѡ҆́фиⷧ҇ мⷣрыи хронограⷴ написа.
Є҆рміи же раⷥумѣвь на нь творѧщꙋся братію ѿиде, ꙁлато многѡ̀ вꙁемъ, и҆ и҆де 
въ Є҆гѷпеⷮ҇ къ коленꙋ Хамову сн҃а Ноєва. иⷤ и҆ пріаша є҆́го с̾ честію. и҆ живе
тѹ во̑ вѣлице чьсти, носѧ риꙁу ꙁлаⷮу и҆ мⷣрствꙋꙗ҆́ше па Є҆гѷпетскыⷯ влъхвоⷯ҇,
влъх̾вѹѧ и҆ повѣда и҆мъ хотѧщаѧ̀ быти. бѣ̏ же и҆ хїтръ бесѣдаⷨ. и҆ кланѧхꙋсѧ
є҆́му гл҃ще б҃ъ Є҆рміин ꙗ҆ко повѣдающа иⷨ хотѧщаѧ̀ быти и҆мъ и повѣдающа 
иⷨ и҆мѣніє є҆гоⷤ и дателѧ богаⷮствѹ нарицахѹ ꙗ҆ко ꙁлат̾наго б҃а мнѧще.

\smallskip\itshape

А҆ ѡ҆ сеⷨ Иракліи и҆ною̀ притчею рекоша сеⷤ Феѡ҆́фиⷧ҇ мⷣрыи хронограⷴ написа.
Є҆рміи же раⷥумѣвь на нь творѧщꙋся братію ѿиде, ꙁла\-то многѡ̀ вꙁемъ, и҆ и҆де 
въ Є҆гѷпеⷮ҇ къ коленꙋ Хамову сн҃а Ноєва. иⷤ и҆ пріаша є҆́го с̾ честію. и҆ живе
тѹ во̑ вѣлице чьсти, носѧ риꙁу ꙁлаⷮу и҆ мⷣрствꙋꙗ҆́ше па Є҆гѷпетскыⷯ влъхвоⷯ҇,
влъх̾вѹѧ и҆ повѣда и҆мъ хотѧщаѧ̀ быти. бѣ̏ же и҆ хїтръ бесѣдаⷨ. и҆ кланѧхꙋсѧ
є҆́му гл҃ще б҃ъ Є҆рміин ꙗ҆ко повѣдающа иⷨ хотѧщаѧ̀ быти и҆мъ и повѣдающа 
иⷨ и҆мѣніє є҆гоⷤ и дателѧ богаⷮствѹ нарицахѹ ꙗ҆ко ꙁлат̾наго б҃а мнѧще.

\end{quote}

\chapter{GNU Free Documentation License}
\label{FDL}

\begin{center}

Version 1.2, November 2002

Copyright \copyright{} 2000,2001,2002  Free Software Foundation, Inc.
 
\bigskip
 
51 Franklin St, Fifth Floor, Boston, MA  02110-1301  USA
  
\bigskip
 
Everyone is permitted to copy and distribute verbatim copies of this
license document, but changing it is not allowed.

\end{center}

\subsection*{Preamble}

The purpose of this License is to make a manual, textbook, or other
functional and useful document “free” in the sense of freedom: to assure
everyone the effective freedom to copy and redistribute it, with or without
modifying it, either commercially or noncommercially.  Secondarily, this
License preserves for the author and publisher a way to get credit for
their work, while not being considered responsible for modifications made
by others.

This License is a kind of “copyleft”, which means that derivative works of
the document must themselves be free in the same sense.  It complements the
GNU General Public License, which is a copyleft license designed for free
software.

We have designed this License in order to use it for manuals for free
software, because free software needs free documentation: a free program
should come with manuals providing the same freedoms that the software
does.  But this License is not limited to software manuals; it can be used
for any textual work, regardless of subject matter or whether it is
published as a printed book.  We recommend this License principally for
works whose purpose is instruction or reference.

\section{Applicability and Definitions}
\label{FDL:sec1}

This License applies to any manual or other work, in any medium, that
contains a notice placed by the copyright holder saying it can be
distributed under the terms of this License.  Such a notice grants a
world-wide, royalty-free license, unlimited in duration, to use that work
under the conditions stated herein.  The “\textbf{Document}”, below,
refers to any such manual or work.  Any member of the public is a licensee,
and is addressed as “\textbf{you}”.  You accept the license if you copy,
modify or distribute the work in a way requiring permission under copyright
law.

A “\textbf{Modified Version}” of the Document means any work containing the
Document or a portion of it, either copied verbatim, or with modifications
and/or translated into another language.

A “\textbf{Secondary Section}” is a named appendix or a front-matter
section of the Document that deals exclusively with the relationship of the
publishers or authors of the Document to the Document’s overall subject (or
to related matters) and contains nothing that could fall directly within
that overall subject.  (Thus, if the Document is in part a textbook of
mathematics, a Secondary Section may not explain any mathematics.)  The
relationship could be a matter of historical connection with the subject or
with related matters, or of legal, commercial, philosophical, ethical or
political position regarding them.

The “\textbf{Invariant Sections}” are certain Secondary Sections whose
titles are designated, as being those of Invariant Sections, in the notice
that says that the Document is released under this License.  If a section
does not fit the above definition of Secondary then it is not allowed to be
designated as Invariant.  The Document may contain zero Invariant Sections.
If the Document does not identify any Invariant Sections then there are
none.

The “\textbf{Cover Texts}” are certain short passages of text that are
listed, as Front-Cover Texts or Back-Cover Texts, in the notice that says
that the Document is released under this License.  A Front-Cover Text may
be at most 5 words, and a Back-Cover Text may be at most 25 words.

A “\textbf{Transparent}” copy of the Document means a machine-readable
copy, represented in a format whose specification is available to the
general public, that is suitable for revising the document
straightforwardly with generic text editors or (for images composed of
pixels) generic paint programs or (for drawings) some widely available
drawing editor, and that is suitable for input to text formatters or for
automatic translation to a variety of formats suitable for input to text
formatters.  A copy made in an otherwise Transparent file format whose
markup, or absence of markup, has been arranged to thwart or discourage
subsequent modification by readers is not Transparent. An image format is
not Transparent if used for any substantial amount of text. A copy that is
not “Transparent” is called “\textbf{Opaque}”.

Examples of suitable formats for Transparent copies include plain ASCII
without markup, Texinfo input format, LaTeX input format, SGML or XML using
a publicly available DTD, and standard-conforming simple HTML, PostScript
or PDF designed for human modification.  Examples of transparent image
formats include PNG, XCF and JPG.  Opaque formats include proprietary
formats that can be read and edited only by proprietary word processors,
SGML or XML for which the DTD and/or processing tools are not generally
available, and the machine-generated HTML, PostScript or PDF produced by
some word processors for output purposes only.

The “\textbf{Title Page}” means, for a printed book, the title page itself,
plus such following pages as are needed to hold, legibly, the material this
License requires to appear in the title page. For works in formats which
do not have any title page as such, “Title Page” means the text near the
most prominent appearance of the work’s title, preceding the beginning of
the body of the text.

A section “\textbf{Entitled XYZ}” means a named subunit of the Document
whose title either is precisely XYZ or contains XYZ in parentheses
following text that translates XYZ in another language.  (Here XYZ stands
for a specific section name mentioned below, such as
“\textbf{Acknowledgements}”, “\textbf{Dedications}”,
“\textbf{Endorsements}”, or “\textbf{History}”.) To “\textbf{Preserve the
Title}” of such a section when you modify the Document means that it
remains a section “Entitled XYZ” according to this definition.

The Document may include Warranty Disclaimers next to the notice which
states that this License applies to the Document.  These Warranty
Disclaimers are considered to be included by reference in this
License, but only as regards disclaiming warranties: any other
implication that these Warranty Disclaimers may have is void and has
no effect on the meaning of this License.

\section{Verbatim Copying}
\label{FDL:sec2}

You may copy and distribute the Document in any medium, either commercially
or noncommercially, provided that this License, the copyright notices, and
the license notice saying this License applies to the Document are
reproduced in all copies, and that you add no other conditions whatsoever
to those of this License.  You may not use technical measures to obstruct
or control the reading or further copying of the copies you make or
distribute.  However, you may accept compensation in exchange for copies.
If you distribute a large enough number of copies you must also follow the
conditions in \autoref{FDL:sec3}.

You may also lend copies, under the same conditions stated above, and
you may publicly display copies.

\section{Copying in Quantity}
\label{FDL:sec3}

If you publish printed copies (or copies in media that commonly have
printed covers) of the Document, numbering more than 100, and the
Document’s license notice requires Cover Texts, you must enclose the
copies in covers that carry, clearly and legibly, all these Cover
Texts: Front-Cover Texts on the front cover, and Back-Cover Texts on
the back cover.  Both covers must also clearly and legibly identify
you as the publisher of these copies.  The front cover must present
the full title with all words of the title equally prominent and
visible.  You may add other material on the covers in addition.
Copying with changes limited to the covers, as long as they preserve
the title of the Document and satisfy these conditions, can be treated
as verbatim copying in other respects.

If the required texts for either cover are too voluminous to fit
legibly, you should put the first ones listed (as many as fit
reasonably) on the actual cover, and continue the rest onto adjacent
pages.

If you publish or distribute Opaque copies of the Document numbering
more than 100, you must either include a machine-readable Transparent
copy along with each Opaque copy, or state in or with each Opaque copy
a computer-network location from which the general network-using
public has access to download using public-standard network protocols
a complete Transparent copy of the Document, free of added material.
If you use the latter option, you must take reasonably prudent steps,
when you begin distribution of Opaque copies in quantity, to ensure
that this Transparent copy will remain thus accessible at the stated
location until at least one year after the last time you distribute an
Opaque copy (directly or through your agents or retailers) of that
edition to the public.

It is requested, but not required, that you contact the authors of the
Document well before redistributing any large number of copies, to give
them a chance to provide you with an updated version of the Document.

\section{Modifications}
\label{FDL:sec4}

You may copy and distribute a Modified Version of the Document under
the conditions of sections 2 and 3 above, provided that you release
the Modified Version under precisely this License, with the Modified
Version filling the role of the Document, thus licensing distribution
and modification of the Modified Version to whoever possesses a copy
of it.  In addition, you must do these things in the Modified Version:

\begin{itemize}

\item[A.] Use in the Title Page (and on the covers, if any) a title
distinct from that of the Document, and from those of previous versions
(which should, if there were any, be listed in the History section of the
Document).  You may use the same title as a previous version if the
original publisher of that version gives permission.
 
\item[B.] List on the Title Page, as authors, one or more persons or
entities responsible for authorship of the modifications in the Modified
Version, together with at least five of the principal authors of the
Document (all of its principal authors, if it has fewer than five), unless
they release you from this requirement.

\item[C.] State on the Title page the name of the publisher of the Modified
Version, as the publisher.

\item[D.] Preserve all the copyright notices of the Document.

\item[E.] Add an appropriate copyright notice for your modifications
adjacent to the other copyright notices.

\item[F.] Include, immediately after the copyright notices, a license
notice giving the public permission to use the Modified Version under the
terms of this License, in the form shown in the Addendum below.

\item[G.] Preserve in that license notice the full lists of Invariant
Sections and required Cover Texts given in the Document’s license notice.
 
\item[H.] Include an unaltered copy of this License.
   
\item[I.] Preserve the section Entitled “History”, Preserve its Title, and
add to it an item stating at least the title, year, new authors, and
publisher of the Modified Version as given on the Title Page.  If there is
no section Entitled “History” in the Document, create one stating the
title, year, authors, and publisher of the Document as given on its Title
Page, then add an item describing the Modified Version as stated in the
previous sentence.
 
\item[J.] Preserve the network location, if any, given in the Document for
public access to a Transparent copy of the Document, and likewise the
network locations given in the Document for previous versions it was based
on.  These may be placed in the “History” section.  You may omit a network
location for a work that was published at least four years before the
Document itself, or if the original publisher of the version it refers to
gives permission.
 
\item[K.] For any section Entitled “Acknowledgements” or “Dedications”,
Preserve the Title of the section, and preserve in the section all the
substance and tone of each of the contributor acknowledgements and/or
dedications given therein.
 
\item[L.] Preserve all the Invariant Sections of the Document, unaltered in
their text and in their titles. Section numbers or the equivalent are not
considered part of the section titles.
 
\item[M.] Delete any section Entitled “Endorsements”.  Such a section may
not be included in the Modified Version.
 
\item[N.] Do not retitle any existing section to be Entitled
“Endorsements” or to conflict in title with any Invariant Section.
 
\item[O.] Preserve any Warranty Disclaimers.

\end{itemize}

If the Modified Version includes new front-matter sections or appendices
that qualify as Secondary Sections and contain no material copied from the
Document, you may at your option designate some or all of these sections as
invariant.  To do this, add their titles to the list of Invariant Sections
in the Modified Version’s license notice.  These titles must be distinct
from any other section titles.

You may add a section Entitled “Endorsements”, provided it contains
nothing but endorsements of your Modified Version by various parties~— for
example, statements of peer review or that the text has been approved by an
organization as the authoritative definition of a standard.

You may add a passage of up to five words as a Front-Cover Text, and a
passage of up to 25 words as a Back-Cover Text, to the end of the list of
Cover Texts in the Modified Version.  Only one passage of Front-Cover Text
and one of Back-Cover Text may be added by (or through arrangements made
by) any one entity.  If the Document already includes a cover text for the
same cover, previously added by you or by arrangement made by the same
entity you are acting on behalf of, you may not add another; but you may
replace the old one, on explicit permission from the previous publisher
that added the old one.

The author(s) and publisher(s) of the Document do not by this License give
permission to use their names for publicity for or to assert or imply
endorsement of any Modified Version.

\section{Combining Documents}
\label{FDL:sec5}

You may combine the Document with other documents released under this
License, under the terms defined in \autoref{FDL:sec4} above for modified
versions, provided that you include in the combination all of the Invariant
Sections of all of the original documents, unmodified, and list them all as
Invariant Sections of your combined work in its license notice, and that
you preserve all their Warranty Disclaimers.

The combined work need only contain one copy of this License, and multiple
identical Invariant Sections may be replaced with a single copy.  If there
are multiple Invariant Sections with the same name but different contents,
make the title of each such section unique by adding at the end of it, in
parentheses, the name of the original author or publisher of that section
if known, or else a unique number.  Make the same adjustment to the section
titles in the list of Invariant Sections in the license notice of the
combined work.

In the combination, you must combine any sections Entitled “History” in the
various original documents, forming one section Entitled “History”;
likewise combine any sections Entitled “Acknowledgements”, and any sections
Entitled “Dedications”.  You must delete all sections Entitled
“Endorsements”.

\section{Collection of Documents}
\label{FDL:sec6}

You may make a collection consisting of the Document and other documents
released under this License, and replace the individual copies of this
License in the various documents with a single copy that is included in
the collection, provided that you follow the rules of this License for
verbatim copying of each of the documents in all other respects.

You may extract a single document from such a collection, and distribute
it individually under this License, provided you insert a copy of this
License into the extracted document, and follow this License in all
other respects regarding verbatim copying of that document.

\section{Aggregation with Independent Works}
\label{FDL:sec7}

A compilation of the Document or its derivatives with other separate and
independent documents or works, in or on a volume of a storage or
distribution medium, is called an “aggregate” if the copyright resulting
from the compilation is not used to limit the legal rights of the
compilation’s users beyond what the individual works permit.  When the
Document is included in an aggregate, this License does not apply to the
other works in the aggregate which are not themselves derivative works of
the Document.

If the Cover Text requirement of \autoref{FDL:sec3} is applicable to these
copies of the Document, then if the Document is less than one half of the
entire aggregate, the Document’s Cover Texts may be placed on covers that
bracket the Document within the aggregate, or the electronic equivalent of
covers if the Document is in electronic form.  Otherwise they must appear
on printed covers that bracket the whole aggregate.

\section{Translation}
\label{FDL:sec8}

Translation is considered a kind of modification, so you may distribute
translations of the Document under the terms of \autoref{FDL:sec4}.
Replacing Invariant Sections with translations requires special permission
from their copyright holders, but you may include translations of some or
all Invariant Sections in addition to the original versions of these
Invariant Sections.  You may include a translation of this License, and all
the license notices in the Document, and any Warranty Disclaimers,
provided that you also include the original English version of this License
and the original versions of those notices and disclaimers.  In case of a
disagreement between the translation and the original version of this
License or a notice or disclaimer, the original version will prevail.

If a section in the Document is Entitled “Acknowledgements”, “Dedications”,
or “History”, the requirement (\autoref{FDL:sec4}) to Preserve its Title
(section~1) will typically require changing the actual title.

\section{Termination}
\label{FDL:sec9}

You may not copy, modify, sublicense, or distribute the Document except
as expressly provided for under this License.  Any other attempt to
copy, modify, sublicense or distribute the Document is void, and will
automatically terminate your rights under this License.  However,
parties who have received copies, or rights, from you under this
License will not have their licenses terminated so long as such
parties remain in full compliance.

\section{Future Revisions of this License}
\label{FDL:sec10}

The Free Software Foundation may publish new, revised versions
of the GNU Free Documentation License from time to time.  Such new
versions will be similar in spirit to the present version, but may
differ in detail to address new problems or concerns. See
\href{http://www.gnu.org/copyleft/}{http://www.gnu.org/copyleft/}.

Each version of the License is given a distinguishing version number.
If the Document specifies that a particular numbered version of this
License “or any later version” applies to it, you have the option of
following the terms and conditions either of that specified version or
of any later version that has been published (not as a draft) by the
Free Software Foundation.  If the Document does not specify a version
number of this License, you may choose any version ever published (not
as a draft) by the Free Software Foundation.

\section*{ADDENDUM: How to use this License for your documents}
\phantomsection
\addcontentsline{toc}{section}{ADDENDUM: How to use this License for 
your documents}

To use this License in a document you have written, include a copy of
the License in the document and put the following copyright and
license notices just after the title page:

\bigskip

\begin{quote}
Copyright © YEAR YOUR NAME.
Permission is granted to copy, distribute and/or modify this document
under the terms of the GNU Free Documentation License, Version 1.2
or any later version published by the Free Software Foundation;
with no Invariant Sections, no Front-Cover Texts, and no Back-Cover Texts.
A copy of the license is included in the section entitled “GNU
Free Documentation License”.
\end{quote}

\bigskip
    
If you have Invariant Sections, Front-Cover Texts and Back-Cover Texts,
replace the “with \dots\ Texts.” line with this:

\bigskip

\begin{quote}
with the Invariant Sections being LIST THEIR TITLES, with the
Front-Cover Texts being LIST, and with the Back-Cover Texts being LIST.
\end{quote}

\bigskip
    
If you have Invariant Sections without Cover Texts, or some other
combination of the three, merge those two alternatives to suit the
situation.

If your document contains nontrivial examples of program code, we
recommend releasing these examples in parallel under your choice of
free software license, such as the GNU General Public License,
to permit their use in free software.

\end{document}

