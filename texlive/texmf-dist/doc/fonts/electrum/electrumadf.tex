% !TEX TS-program = pdflatex
% !TEX encoding = UTF-8 Unicode
%% electrumadf.tex
%% Copyright 2010 Clea F. Rees
%
% This work may be distributed and/or modified under the
% conditions of the LaTeX Project Public License, either version 1.3
% of this license or (at your option) any later version.
% The latest version of this license is in
%   http://www.latex-project.org/lppl.txt
% and version 1.3 or later is part of all distributions of LaTeX
% version 2005/12/01 or later.
%
% This work has the LPPL maintenance status `maintained'.
% 
% The Current Maintainer of this work is Clea F. Rees.
%
% This work consists of all files listed in manifest.txt.
\listfiles
\documentclass[11pt,british]{article}
\usepackage{babel}
\usepackage{lmodern}
\renewcommand{\ttdefault}{lmvtt}
\let\origrmdefault\rmdefault
\DeclareRobustCommand{\origrmfamily}{%
	\fontencoding{T1}%
	\fontfamily{\origrmdefault}%
	\selectfont}
\DeclareTextFontCommand{\textorigrm}{\origrmfamily}
\usepackage{electrum}
	\pdfmapfile{+yes.map}	% not necessary for installed package
\usepackage{fancyhdr,lastpage,fancyref}
\usepackage{array,longtable,verbatim}
\usepackage{booktabs}
\usepackage{multirow}
\usepackage{url}
	\urlstyle{tt}
\usepackage{hyperref}
\usepackage{microtype}
\usepackage[a4paper,headheight=14pt]{geometry}	% use 14pt for 11pt text, 15pt for 12pt text

\title{electrumadf}
\author{Clea F.\ Rees\footnote{cfrees <at> imapmail <dot> org}}
\newcommand*{\dyddiad}{18\textsuperscript{th} July, 2010}
\date{\dyddiad}
\pagestyle{fancy}
	\fancyhf[lh]{\itshape electrumadf}
	\fancyhf[rh]{\itshape\dyddiad}
	\fancyhf[ch]{}
	\fancyhf[lf]{}
	\fancyhf[rf]{}
	\fancyhf[cf]{\itshape --- \thepage~\ofname~\pageref{LastPage} ---}
	
\begin{document}
\maketitle\thispagestyle{empty}
\pdfinfo{%
	/Creator		(TeX)
	/Producer		(pdfTeX)
	/Author			(Clea F.\ Rees)
	/Title				(electrumadf)
	/Subject		(TeX)
	/Keywords		(TeX,LaTeX,font,fonts,tex,latex,Electrum,electrum,electrumadf,ElectrumADF,ADF,adf,Arkandis,Digital,Foundry,arkandis,digital,foundry,Hirwen,Harendal,Clea,Rees)}
\pdfcatalog{%
	/URL				()
	/PageMode	/UseOutlines}	% other values: /UseNone, /UseOutlines, /UseThumbs, /FullScreen
	%[openaction <actionspec>]
%	\pagestyle{empty}	% if you want this, you probably want to comment out \maketitle as well...?
\setlength{\parindent}{0pt}
\setlength{\parskip}{0.5em}
	
	
\newcommand*{\adf}{\textsc{adf}}
\newcommand*{\lpack}[1]{\textsf{#1}}
\newcommand*{\fgroup}[1]{\textsf{#1}}
\newcommand*{\fname}[1]{\textsf{#1}}

\begin{abstract}
	\hspace*{-\parindent}Hirwen Harendal, Arkandis Digital Foundry (\adf) has produced the Electrum \adf\ font collection. This guide outlines the \emph{experimental} \TeX/\LaTeX\ support provided by \lpack{electrumadf} for version 1.005 of the fonts.
\end{abstract}

\tableofcontents

\section{Introduction}

This document explains how to use the \TeX/\LaTeX\ support provided for version 1.005 of the Electrum \adf\ font collection developed by Hirwen Harendal of the Arkandis Digital Foundry (\adf). \lpack{electrumadf} includes copies of the fonts in postscript type 1 format. Further  information about the fonts themselves and alternative font formats for use with other programmes can be found at \url{http://arkandis.tuxfamily.org/adffonts.html}. The fonts are released under the \textsc{gnu} General Public License as published by the Free Software Foundation; either version 2 of the License, or any later version, with font exception. For details, see \textsc{notice}.txt and \textsc{copying}.

The \TeX/\LaTeX\ support package consists of all files listed in \path{manifest.txt}\ and these files are released under the \LaTeX\ Project Public Licence as explained in the included licensing notices. Please let me know of any problems so that I can solve them if I can. If you can correct the problems and send me the fix, that would be even better. Unlike the fonts themselves, the \TeX/\LaTeX\ support is somewhat experimental.

\section{The collection}

Electrum \adf\ is a slab serif family designed as a substitute for Eurostyle or URWCity. The family currently includes upright, oblique, small-caps and oblique small-caps shapes in each of light, regular, semi-bold and bold weights. Four sets of digits are provided: oldstyle, lining, inferior and superior\footnote{In fact, the fonts also include denominator and numerator figures. Since there is currently no use for these in TeX, however, the support package ignores them.}. The support package renames the fonts according to the Karl Berry fontname scheme and defines six families. Two of these primarily provide access to the ``standard'' or default characters while the two ``ligature'' families support additional non-standard ligatures\footnote{\Fref{sec:encs} describes the encodings used to create these families. The fifth and sixth families include the inferior and superior figures, together with any other complementary characters included in the fonts. For further details see the encoding (\path{.etx}) files.}. The included package files provide access to these features in \LaTeX\ as explained in \fref{sec:support} and \fref{sec:commands}.

\clearpage

\begin{longtable}{llll}
	\toprule
	\textbf{\TeX\ directory}	&	\textbf{font families}	&	\textbf{Original name}	& \textbf{\TeX\ name}\\\midrule\endhead
		\bottomrule\endfoot
	\multirow{8}*{electrum}	& \multirow{8}{.2\textwidth}{yes, yesj, yesjw, yesw, \textl{yes0, yes1}	}	&	ElectrumADFExp-Light							&	yesl8a\\
						& 																		&	ElectrumADFExp-LightOblique			&	yeslo8a\\
						&																		&	ElectrumADFExp-Regular						&	yesr8a\\
						&																		&	ElectrumADFExp-Oblique						&	yesro8a\\
						&																		&	ElectrumADFExp-SemiBold					&	yess8a\\						
						& 																		&	ElectrumADFExp-SemiBoldOblique	&	yesso8a\\
						&																		&	ElectrumADFExp-Bold							&	yesb8a\\
						&																		&	ElectrumADFExp-BoldOblique				&	yesbo8a\\
\end{longtable}

\section{Requirements}

Apart from such obvious requirements as \LaTeXe, the \LaTeX\ support provided by \path{electrum.sty} requires \lpack{nfssext-cfr} and \lpack{xkeyval}. Without this, you will get errors complaining that the package cannot be found and you will not be able to use any of the additional font commands described in \fref{sec:commands}.

The documentation requires in additional packages. These are all standard and available from \textsc{ctan} but you can always comment out the relevant lines in \path{electrumadf.tex} if you wish.

\section{Installation}

Installation varies with \TeX\ distribution so you should consult the documentation which came with your system for details. In most cases, you will need to perform three steps:
		\begin{enumerate}
			\item move or copy the package files to appropriate locations on your system;
			\item refresh the \TeX\ database;
			\item incorporate the included map file fragments for the different engines your distribution supports.
		\end{enumerate}

The following instructions assume you are using \TeX~Live\footnote{This includes Mac\TeX\ for OS X users.}. They should not be too difficult to adapt if you are using a different distribution.

\subsection{Install the files}

The files should be installed in one of two locations: \emph{either} the local system-wide \TeX\ tree \emph{or} your personal tree. If the package is installed system-wide, all users will have access to it. On the other hand, you may need privileges you do not have to do this in which case you must use your personal tree.

For \TeX~Live, \verb|kpsewhich -var-value TEXMFLOCAL| will return the path to the local tree and \verb|kpsewhich -var-value TEXMFHOME| the path to your personal tree. The package already includes a hierarchy of files to help you install them correctly. Ignoring any symbolic link in the top directory, move or copy the files in \path{doc}, \path{fonts} and \path{tex} into the appropriate locations. If the tree is initially empty, you can simply move or copy the directories in as they are. If the tree already contains other packages, you may need to merge the package hierarchy with the pre-existing one. For example, if you already have a \path{doc/fonts} directory, move or copy \path{doc/fonts/electrum} into \path{doc/fonts/}. If you have a \path{doc} directory but not a \path{doc/fonts}, move \path{doc/fonts} into \path{doc/}.

\subsection{Refresh the database}

Again, this depends on your distribution. For \TeX~Live, \verb|mktexlsr <path to directory>| for the directory you used in the first step should do the trick. Note that you \emph{may} be able to skip this step if you install into your personal tree. Whether this is so depends on the details of your set-up. As a test, move to a directory containing none of the package files and try \verb|kpsewhich electrum.sty|. If the file is found, you don't need to refresh the database; otherwise use \verb|mktexlsr| and then try again.

\subsection{Install the map fragments}

For \TeX~Live, there are at least two ways of doing this. The second method varies according to the version of \TeX~Live and instructions are provided accordingly. Both methods depend on whether you installed into \verb|TEXMFLOCAL| or \verb|TEXMFHOME|. If you installed system-wide, the choice is relatively straightforward --- it obviously makes sense in that case to update the font maps system-wide as well. If, on the other hand, you installed into your personal tree, the matter is more complex. On the one hand, updating the system-wide maps may create difficulties or confusion for other users because while the map files will list the fonts as available, they will not be able to access them. On the other hand, maintaining personal font map files can produce difficulties and confusions of its own. Whether it is to be preferred or not is a complex issue and depends on the details of your \TeX\ distribution, local configuration and personal preference. The one clear case is that in which you install into your personal tree because you lack the privileges needed to install system-wide. In that case, you have no choice but to maintain personal font map files or forgo the use of all fonts not provided by your administrator. Other cases are thankfully beyond the scope of this document.

\subsubsection{Method 1}

If you installed the package system-wide, use the command:
\begin{verbatim}
	updmap-sys --enable Map=yes.map
\end{verbatim}
If you installed the package in your personal tree, you \emph{may} prefer to use:
\begin{verbatim}
	updmap --enable Map=yes.map
\end{verbatim}

Either way, \verb|updmap| will output a good deal of information after each incantation. This is normal. Just check that it does not end with an error and that it found the new map file.

\subsubsection{Method 2: \TeX~Live 2008 (and probably earlier)}

If you installed the package system-wide, use \verb|updmap-sys --edit|.

If you installed into your personal tree, you \emph{may} prefer to use	\verb|updmap --edit|.

Either way, a configuration file will be opened which you can edit. Move to the end of the file and add the following line:
\begin{verbatim}
	Map yes.map
\end{verbatim}
When you are done, save the file. \verb|updmap| or \verb|updmap-sys| will produce a great deal of output if all is well. Just check that it does not end with an error and that \path{yes.map} is found.

\subsubsection{Method 2: \TeX~Live 2009 (and possibly later)}

If you installed the package system-wide, edit or or create \path{TEXMFLOCAL/web2c/updmap-local.cfg} and add the following line to the end of the file:
\begin{verbatim}
	Map yes.map
\end{verbatim}
Save the file and tell \verb|tlmgr| to merge in your addition using the command:
\begin{verbatim}
	tlmgr generate updmap
\end{verbatim}
\verb|tlmgr| will then tell you that you need to ensure the changes are propagated correctly by calling \verb|updmap-sys|. This should produce a great deal of output. Check that it finds the new map file and does not end with an error.

If you installed into your personal tree, you \emph{may} prefer to use \verb|	updmap --edit| as described above for \TeX~Live 2008.

To test your installation and that the package works on your system, latex this file (\path{electrumadf.tex}). The console output and/or log should tell you whether any fonts were not found. If you are careful not to overwrite it, you may also compare your output with \path{electrumadf.pdf}.

\section{The support package}\label{sec:support}

\subsection{Encodings}\label{sec:encs}

The package supports modified \textsc{ec}/\textsc{t1} and Text Companion (\textsc{ts1}) encodings. Most characters in the \textsc{ec} encoding are available and the fonts provide a small number of characters from the \textsc{ts1} encoding as well, including the \texteuro. The regular versions of the \textsc{ec}/\textsc{t1} encoding (\path{t1-yes.etx}/\path{t1j-yes.etx}) reassign three slots which would otherwise be empty due to missing glyphs which \path{fontinst} cannot fake. In the \textsc{t1} encoding, these slot are standardly used for the per thousand zero and the unfakable Sami Eng/eng characters (\textorigrm{\NG}/\textorigrm{\ng}). The modified encodings use \verb|zero.denominator| in place of the per thousand zero which should provide a reasonable substitute (\textl{\textperthousand}/\textl{\textpertenthousand}) when lining figures are in use and a substitute which is at least intelligible (\textperthousand/\textpertenthousand) for oldstyle digits. The two further slots are used for the alternate `Q' (Q*) and the \verb|t_t| ligature (tt).

The ``ligature'' versions of the \textsc{ec}/\textsc{t1} encoding (\path{t1-yesw.etx}/\path{t1j-yesw.etx}/\path{t1-yesw-sc.etx}/\path{t1j-yesw-sc.etx}) provide access to the full range of ligatures available --- including `\textswash{ct}', `\textswash{it}', `\textswash{sp}' and `\textswash{st}'. In addition the alternate `Q' (\textswash{Q}) becomes the default `Q' and the standard `Q' is installed as an alternate (\textswash{Q*}). Because further slots are required to accommodate the additional ligatures, a number of characters normally available in the \textsc{ec} encoding are unavailable in upright and oblique shapes. These are the \textsc{ascii} upward-pointing arrowhead (\textasciicircum), the \textsc{ascii} tilde (\textasciitilde) and the \verb|dbar| (\dj).  Attempting to access these characters while using the ligature versions of the fonts may result in errors of various kinds and will certainly produce unexpected output even though the characters are provided by the fonts, as the previous sentence demonstrates. To access these glyphs, ensure that the regular version of the fonts is active.

The difference between the \verb|t1-| and \verb|t1j-| encodings is that the latter use oldstyle rather than lining figures and the corresponding symbols designed to complement them. For example, \verb|0123456789 \& \$ \pounds\ \%| produces \textl{0123456789 \& \$ \pounds\ \%} when an encoding of the former kind is active, but 0123456789 \& \$ \pounds\ \% when an encoding of the latter sort is used.

Finally, \path{t1-dotinferiors.etx} and \path{t1-dotsuperiors.etx} support the inferiors and superiors provided by the fonts. This amounts to the digits (\textin{0123456789}/\textsu{0123456789}), some basic punctuation (\textin{(,)-.}/\textsu{(,)-.}) and symbols (\textin{+\pounds\$}/\textsu{+\pounds\$}) and, in the case of superiors, a selection of lowercase letters (\textsu{abcdefghijklmnopqrstuvwxyz}).

The \verb|ts1-| encodings complement the corresponding \verb|t1-| encodings as usual. \path{ts1-yes.etx} simply adapts the names appropriately for Electrum. \verb|ts1-dotoldstyle-yes.etx| also replaces standard symbols with oldstyle variants where these are available. This means that
	\begin{verbatim}
		\oldstylenums{0} \oldstylenums{1} \oldstylenums{2} \oldstylenums{3} 
		\oldstylenums{4} \oldstylenums{5} \oldstylenums{6} \oldstylenums{7} 
		\oldstylenums{8} \oldstylenums{9} \textdollaroldstyle\ \textcentoldstyle
	\end{verbatim}
	should produce \oldstylenums{0} \oldstylenums{1} \oldstylenums{2} \oldstylenums{3} \oldstylenums{4} \oldstylenums{5} \oldstylenums{6} \oldstylenums{7} \oldstylenums{8} \oldstylenums{9} \textdollaroldstyle\ \textcentoldstyle\ for both encodings, but
	\begin{verbatim}
		\textdollar\ \textcent\ \textsterling\ \texteuro\ \textyen\ \textperthousand
	\end{verbatim}
	for example, will produce \textl{\textdollar\ \textcent\ \textsterling\ \texteuro\ \textyen\ \textperthousand\ }if lining figures are active but \textdollar\ \textcent\ \textsterling\ \texteuro\ \textyen\ \textperthousand\ when oldstyle digits are in use. Similarly, \path{ts1-dotinferior.etx} and \path{ts1-dotsuperior.etx} contain subscript and superscript symbols where available (\textin{\texteuro \textsterling \textyen \textdollar \textcent}/\textsu{\texteuro \textsterling \textyen \textdollar \textcent}). Unlike \path{ts1-dotoldstyle-yes.etx}, however, the ``standard'' symbols make no sense here so when inferiors or superiors are in use \emph{only} those symbols available in subscript or superscript form are provided.

\subsection{\LaTeX\ package}

To use the fonts in a \LaTeX\ document, add \verb|\usepackage{electrum}| to your document preamble. This will set the default serif/roman family to \fname{yes} (\fgroup{electrum}) and enable access to the various alternates, styles and ligatures.. Three optional arguments are available to tailor the behaviour of the package: \verb|lf|; \verb|osf|, and \verb|lig|. By default, oldstyle figures are used as standard and lining digits are available using the commands explained in \fref{sec:commands}. To make lining figures the default instead, use one of the following when loading the package:
\begin{verbatim}
	\usepackage[lf]{electrum}
	\usepackage[lf=true]{electrum}
	\usepackage[osf=flase]{electrum}	
\end{verbatim}
Similarly, to explicitly request oldstyle figures:
\begin{verbatim}
	\usepackage[osf]{electrum}
	\usepackage[osf=true]{electrum}
	\usepackage[lf=flase]{electrum}	
\end{verbatim}

Loading \lpack{electrum} with \verb|lig| or \verb|lig=true| will select the versions which enable the additional ligatures and the alternate Q as default. \textbf{\emph{This option is not recommended unless you are \emph{certain} you do not wish to access any of the characters described in \fref{sec:encs}.}} You should also note that this option will mean all of the additional ligatures will be active, which may not be what you want. Again, passing \verb|lig=false| will explicitly request the default --- and strongly recommended --- behaviour which is to \emph{not} enable the additional ligatures by default.
	
	\begin{longtable}{lllllll}
		\toprule
		\textbf{weights}		&	 \textbf{shapes}	&	\textbf{ligatures}	&	\textbf{\texttt{Q}}	&	\textbf{\texttt{Q*}}	& \textbf{figures}	&	\textbf{family}\tabularnewline\midrule\endhead
		\bottomrule\endfoot
		\multirow{7}{.15\textwidth}{light,\\regular,\\semi-bold,\\bold}	%\multirow{nrows}[bigstruts]{width}[fixup]{text}; or * for {width}
			&	\multirow{5}{.26\textwidth}{upright,\\oblique,\\small-caps,\\oblique small-caps}%
				&	\multirow{2}{.15\textwidth}{standard, tt}%
					&	\multirow{2}*{Q}								&	\multirow{2}*{Q*}							&	lining			&	yes\tabularnewline\cmidrule{6-7}
			&	&	&																&																& oldstyle		&	yesj\tabularnewline\cmidrule{3-7}
			&	&	\multirow{2}{.15\textwidth}{standard, ct, it, sp, st, it}%
					&	\multirow{2}*{\textswash{Q}}	&	\multirow{2}*{\textswash{Q*}}	&	lining			&	yesw\tabularnewline\cmidrule{6-7}
			&	&	&																&																&	oldstyle		&	yesjw\tabularnewline\cmidrule{2-7}
			&	\multirow{2}{.26\textwidth}{upright (very incomplete)}%
				&	\multirow{2}*{---}%
					&	\multirow{2}*{---}							&	\multirow{2}*{---}							&	inferior		&	\textl{yes0}\tabularnewline\cmidrule{6-7}
			&	& &																&																& superior		&	\textl{yes1}\tabularnewline
	\end{longtable}

Note that loading \path{electrum.sty} will not affect the default sans-serif or typewriter families.
	
\section{Additional font selection commands}\label{sec:commands}

	The \LaTeX\ package \lpack{electrum}\ loads \lpack{nfssext-cfr}\ which is an extension of the package \lpack{nfssext}\ supplied by Philipp Lehman as part of The Font Installation Guide. The file extends the font selection commands to facilitate access to various font features. Both the original and the extension are designed for use with a wide range of fonts. For this reason, only a subset of the additional commands are relevant to any particular font support package. Those relevant to \lpack{electrumadf}\ are described below.
	
	I consider my additions to \lpack{nfssext-cfr}\ to be \emph{highly experimental}. If things don't work as advertised, apart from letting me know about the problem, you may be able to access the features you need by issuing a \verb|\normalfont| and then selecting features from there. This command will return you to the default document text font --- typically the relevant serif in regular weight, standard width and upright shape with oldstyle or lining figures etc.\ as determined by the packages and options loaded or your distribution's setup.
	
\subsection{nfssext-cfr}

These commands are available when \lpack{electrum} is loaded. If for some reason you wish to make them available when no relevant package is loaded, use \verb|\usepackage{nfssext-cfr}| in your document preamble.

\subsubsection{Weights}

	\begin{longtable}{lll}
		\toprule
		\textbf{weight}		&	\textbf{weight command}	&	\textbf{text command}\\\midrule\endhead
		\bottomrule\endfoot
		light							&	\verb|\lgweight|						&	\verb|\textlg{}|\\
		semibold					&	\verb|\sbweight|					&	\verb|\textsb{}|\\
	\end{longtable}
	
	These work in the same way as the standard \LaTeX\ commands for switching to bold text, \verb|\bfseries| and \verb|\textbf{}| except that since these commands affect only the weight and not the width, \verb|weight| replaces \verb|series|.
	\begin{verbatim}
		\textlg{From Light} through regular	\textsb{and semibold} \textbf{to bold.}
	\end{verbatim}
	produces:
	\begin{center}
		\textlg{From Light} through regular	\textsb{and semibold} \textbf{to bold.}
	\end{center}

\subsubsection{Shapes}

	\begin{longtable}{lll}
		\toprule
		\textbf{shape}			&	\textbf{shape command}	&	\textbf{text command}\\\midrule\endhead
		\bottomrule\endfoot
		oblique small-caps		&	\verb|\sishape|					&	\verb|\textsi{}|\\
	\end{longtable}
		
	For example, \verb|\textsi{Lewis Carroll wrote, ``I always avoid a kangaroo''.}| produces:
		\begin{center}
			\textsi{Lewis Carroll wrote, ``I always avoid a kangaroo''.}
		\end{center}		
	
\subsubsection{Styles}

	\begin{longtable}{llll}
		\toprule
		\textbf{style}			&	\textbf{style command}	&	\textbf{text command}	&	\textbf{effect}\\\midrule\endhead
		\bottomrule\endfoot
		ligature/swash			&	\verb|\swashstyle|				&	\verb|\textswash{}|			&	italic, regular script\\
	\end{longtable}
	
	\verb|\swashstyle| and \verb|\textswash{}| switch to the ``ligature'' families (\fgroup{yesw}/\fgroup{yesjw}). Within the scope of these commands:
	\begin{itemize}
		\item \verb|Q| will typeset the alternate `Q' (\textswash{Q});
		\item \verb|Q*| will typeset the default `Q' (\textswash{Q*});
		\item in upright and oblique text, \verb|ct|, \verb|it|, \verb|sp| and \verb|st|  will typeset the corresponding ligature (\textswash{ct}/\textswash{it}/\textswash{sp}/\textswash{st});
		\item attempting to typeset certain standard characters will produce unexpected results (see \fref{sec:encs}).
	\end{itemize}
	
	Outside the scope of these commands:
	\begin{itemize}
		\item \verb|Q| will typeset the default `Q' (Q);
		\item \verb|Q*| will typeset the alternate `Q' (Q*);
		\item \verb|ct|, \verb|it|, \verb|sp| and \verb|st|  will not produce ligatures (ct/it/sp/st);
		\item except as explained in \fref{sec:encs}, typesetting standard characters should produce the expected results.
	\end{itemize}	
	
	For example, suppose that \lpack{electrum}\ was loaded and the following commands set up:
		\begin{verbatim}
			\newcommand{\fytext}{%
			Q*ueenie, actor-spy and Queen of AQ*UA as Acting Erector Aesthete,\\
			deactivated the sporadically impacted TORQUE despite aspirating stridently\\
			amidst the hysteria of wispy, wasted wasps wistfully whistling.}
			\newcommand{\fytest}{%
			\fytext\\[1em]
			\textswash{\fytext}\\[1em]
			\textsl{\fytext}\\[1em]
			\textswash{\textsl{\fytext}}\\[1em]
			\textsc{\fytext}\\[1em]
			\textsc{\textswash{\fytext}}\\[1em]			
			\textsi{\fytext}\\[1em]
			\textswash{\textsi{\fytext}}}
		\end{verbatim}
		\newcommand{\fytext}{%
			Q*ueenie, actor-spy and Queen of AQ*UA as Acting Erector Aesthete,\\
			deactivated the sporadically impacted TORQUE despite aspirating stridently\\
			amidst the hysteria of wispy, wasted wasps wistfully whistling.}
			\newcommand{\fytest}{%
			\fytext\\[1em]
			\textswash{\fytext}\\[1em]
			\textsl{\fytext}\\[1em]
			\textswash{\textsl{\fytext}}\\[1em]
			\textsc{\fytext}\\[1em]
			\textsc{\textswash{\fytext}}\\[1em]			
			\textsi{\fytext}\\[1em]
			\textswash{\textsi{\fytext}}}
		Then:
		\begin{verbatim}
			\fytext
		\end{verbatim}
		produces:
		\begin{center}
			\fytest
		\end{center}

\subsubsection{Figures}

	\begin{longtable}{lll}
		\toprule
		\textbf{figure style}			&	\textbf{style command}	&	\textbf{text command}\\\midrule\endhead
		\bottomrule\endfoot
		lining										&	\verb|\lstyle|						&	\verb|\textl{}|\\
		oldstyle									&	\verb|\ostyle|						&	\verb|\texto{}|\\
		inferior/subscript				&	\verb|\instyle|						&	\verb|\textin{}|\\
		superior/superscript			&	\verb|\sustyle|					&	\verb|\textsu{}|\\
	\end{longtable}
	
	In this document, oldstyle figures are used when available by default:
		\begin{center}
			0123456789
		\end{center}
	but lining figures are also accessible. For example, \verb|\textl{0123456789}| produces:
		\begin{center}
			\textl{0123456789}
		\end{center}
	In addition to modifying the figure style, these commands affect the style of certain complementary characters in the \textsc{t1} and \textsc{ts1} encodings as explained in \fref{sec:encs}. This means that:
		\begin{verbatim}
			50\%\ off! That's just \texteuro 2.95, \pounds 3.41, \textyen 5.28 
			\& \$8.67\textcent\ \textsl{or} less than \textdollar 1 \& \textsterling 0.99!!
		\end{verbatim}
	 produces:
		\begin{center}
			50\%\ off! That's just \texteuro 2.95, \pounds 3.41, \textyen 5.28
			\& \$8.67\textcent\ \textsl{or} less than \textdollar 1 \& \textsterling 0.99!!
		\end{center}
		when oldstyle digits are in use, but:
		\begin{center}
			\textl{50\%\ off! That's just \texteuro 2.95, \pounds 3.41, \textyen 5.28 
			\& \$8.67\textcent\ \textsl{or} less than \textdollar 1 \& \textsterling 0.99!!}
		\end{center}
		after switching to lining figures.
		
	Note that the commands for inferior and superior figures make further changes. \textbf{\emph{Normal text cannot be typeset within the scope of the commands for inferiors or superiors.}} The commands for subscript activate basic symbols and punctuation to complement the digits. So \verb|Llundain\textin{(1,4+\$5)}| produces Llundain\textin{(1,4+\$5)}. The commands for superscript activate a partial lowercase in as well. For example, \verb|postbox\textsu{9(iii)}| produces postbox\textsu{9(iii)}.
		
%\subsubsection{The slashed zero (0*)}
%
%Both of the modified \textsc{t1} encodings used by \lpack{electrumadf} include a non-standard ligature to accommodate the slashed zero. Provided \lpack{electrum} is loaded and the default serif/roman family is active, \verb|0*| will produce the slashed zero (0*). 
	
\end{document}  