%% dictsym.tex
%% Copyright 2004 Walter Schmidt
%
% This work may be distributed and/or modified under the
% conditions of the LaTeX Project Public License, either version 1.3
% of this license or (at your option) any later version.
% The latest version of this license is in
%   http://www.latex-project.org/lppl.txt
% and version 1.3 or later is part of all distributions of LaTeX
% version 2003/12/01 or later.
%
% This work has the LPPL maintenance status "maintained".
%
% This Current Maintainer of this work is Walter Schmidt <w.a.schmidt@gmx.net>
%
% This work consists of the files dictsym.sty and dictsym.tex
% and the compiled file dictsym.pdf.
%
\documentclass[11pt]{ltxguide}[1995/11/28]
\usepackage[scaled=1.5]{dictsym}

% a bit of logical markup:
\newcommand{\Lpack}[1]{\textsf{#1}}

\title{The \Lpack{dictsym} package for \LaTeX}
\author{Walter Schmidt}
\date{version 2.0\,$\beta$ -- 2004-07-26}

\begin{document}
\maketitle\thispagestyle{empty}

The PostScript font `DictSym' (designed by Georg Verweyen) provides a 
number of symbols commonly found in dictionaries.  
The macro package 
\Lpack{dictsym} serves to use of this font with \LaTeX{}.  It provides
the following commands to print the symbols of the font:

\renewcommand{\arraystretch}{1.5}
\begin{tabular}{ll@{\qquad}ll}
 \dsarchitectural &|\dsarchitectural| &   \dsmathematical  &|\dsmathematical|  \\
 \dsbiological    &|\dsbiological|    &   \dsrailways      &|\dsrailways|      \\
 \dschemical      &|\dschemical|      &   \dstechnical     &|\dstechnical|     \\
 \dsagricultural  &|\dsagricultural|  &   \dsmilitary      &|\dsmilitary|      \\
 \dsheraldical    &|\dsheraldical|    &   \dsaeronautical  &|\dsaeronautical|  \\
 \dsjuridical     &|\dsjuridical|     &   \dscommercial    &|\dscommercial|    \\
 \dsliterary      &|\dsliterary|      &   \dsmedical       &|\dsmedical|       \\
\end{tabular}

By default, the symbols are typeset at the same nominal font size 
as the surrounding text.  When the package is loaded with the option
\texttt{[scaled=}\m{scale}\texttt{]},
the DictSym font will be scaled by the indicated factor.  For instance,
\begin{quote}
  |\usepackage[scaled=1.1]{dictsym}|
\end{quote}
will enlarge the symbols by 10\%.
Thus, you can make the `dictionary symbols' match exactly the optical 
appearance of the surrounding typeface.  The above table shows the
symbols at 150\% of their natural size.


\end{document}


