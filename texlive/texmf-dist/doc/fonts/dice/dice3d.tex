%  DICE3D.TEX
% 
% a short demo on the usage of the DICE3D font
%
% Thomas A Heim (1998)
%
% LICENSE: LPPL
%
\documentclass{article}
\addtolength{\textwidth}{1cm}
\addtolength{\oddsidemargin}{-5mm}
\addtolength{\evensidemargin}{-5mm}
 \newfont\dice{dice3d}
\pagestyle{empty}
\begin{document}
\section*{Dicey font}
Plain 2D-like faces with the appropriate number of dots are obtained
with the characters `1' through `6' in the font \verb+dice3d+.
For an authentic 3D-look, we use the fact that for `good' dice the
point total on opposing faces is always 7. Therefore, if face $n$ is
up, there are only four possible values $m$ on the front. Furthermore,
assuming identical dice throughout, the value on the right side is
fixed once the values on the top and front are specified. Thus, we end up
with 24 different 3D arrangement of the dots / faces. In the font,
these are mapped to the letters `a' through `x'. In order to simplify
the usage, the 3D dice are accessible using \emph{ligatures}, based on the
following principle: A number from `1' to `6' determines the value on 
the top face, and a letter (appended without a space) from `a' to `d' 
determines the value on the front face, ordered according to increasing
value. Alternatively, the value on the front face could be indicated
directly with a letter from `a' to `f'. This can be achieved by changing
the ligature tables appropriately (at the end of file \verb+dice3d.mf+).

\subsection*{Example (at 30pt size)} 
\begin{flushleft}
{\dice 123456} \\
produced with \verb+{\dice 123456}+ \\[2ex]
{\dice 1a 1b 1c 1d 2a 2b 2c 2d} \\
produced with \verb+{\dice 1a 1b 1c 1d 2a 2b 2c 2d}+ \\[2ex]
{\dice 3a 3b 3c 3d 4a 4b 4c 4d} \\
produced with \verb+{\dice 3a 3b 3c 3d 4a 4b 4c 4d}+ \\[2ex]
{\dice 5a 5b 5c 5d 6a 6b 6c 6d} \\
produced with \verb+{\dice 5a 5b 5c 5d 6a 6b 6c 6d}+ 
\end{flushleft}
\end{document}

