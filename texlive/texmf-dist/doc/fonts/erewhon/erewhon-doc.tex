% !TEX TS-program = pdflatexmk
\documentclass[11pt]{amsart}
\usepackage[margin=1.5in]{geometry} 
\usepackage[parfill]{parskip}% Begin paragraphs with an empty line rather than an indent
\usepackage{graphicx}
\pdfmapfile{=erewhon.map}
%SetFonts
% erewhon+newtxmath
\usepackage[sups,p,osf,scaled=.98]{erewhon}
\usepackage[T1]{fontenc}
\usepackage{textcomp}
\usepackage{cabin}
\usepackage[varqu,varl]{zi4}% inconsolata
\usepackage[utopia,vvarbb,bigdelims]{newtxmath}
%SetFonts
\title{LaTeX Support for Erewhon}
\author{Michael Sharpe}
\date{\today}  % Activate to display a given date or no date
\begin{document}
\maketitle
\emph{Erewhon} is a font package based largely on Andrey V. Panov's \emph{Heuristica}, but with so many changes that it is no longer strictly compatible with that package, and is offered instead as an enhanced alternative. (\emph{Heuristica}  extended the \emph{Utopia} font family made available by the \TeX\ Users' Group, adding many accented glyphs, Cyrillic glyphs, ligatures, superior and oldstyle fixed-width figures in all styles, and Small Caps in Regular style only. It is widely distributed as a free font collection in OpenType, TrueType and Type$1$ formats.) \emph{Erewhon} is provided in OpenType and Type$1$ formats with complete \LaTeX\ support files in encodings T$1$, TS$1$, LY$1$, T$2$A, T$2$B and T$2$C.  Changes made in the transition from \emph{Heuristica} to \emph{Erewhon} include:
\begin{itemize}
\item
\textsl{slanted} as well as \textit{Italic} shapes;
\item
\textsc{Small Caps} in \textsc{\textbf{bold}} as well as \textsc{regular} upright shapes, with \textsc{\textit{italic}} and \textsc{\textsl{slanted  Small Caps}} shapes from the slanted variants;
\item expanded lookup tables in the {\tt.otf} files for users of XeLaTeX and LuaLaTeX;
\item a number of f-ligatures have been modified, and a \verb|T_h| ligature added;
\item proportionally spaced figures (lining and oldstyle), adding to the existing taboldstyle figures;
\item full collections of superior lowercase letters (including \`e as \textsu{\`e} and \'e as \textsu{\'e}), mainly for the benefit of languages in which those are in common use---e.g., French, Spanish;
\item size reduced by 6\% from Heuristica, which matched the old version of Utopia---the new size matches that of Adobe's commercial UtopiaStd;
\item shapes of some oldstyle figures modified to have more of an oldstyle appearance;
\item fraction macros based on the new numerator and denominator figures;
\item the bold upright face has been made less cramped.
\end{itemize}

The {\tt newtx} package has been modified, as of version $1.26$,  to offer a new option {\tt utopia} (or, equivalently, {\tt heuristica} or {\tt erewhon}) that uses math italic glyphs taken from Utopia and oldstyle figures from \emph{Erewhon}. Its slanted Greek alphabets are constructed from the {\tt txfonts} slanted Greek letters by reducing their italic angle from $15.5$\textdegree\ to $13$\textdegree, matching Utopia's italic angle. So, for Erewhon text and matching math, you can use\footnote{There is most likely also a way to use {\tt MathDesign} or {\tt fourier} with at least partial compatibility.}:
\begin{verbatim}
\usepackage[p,osf,scaled=.98]{erewhon}
\usepackage[varqu,varl]{inconsolata} % typewriter
\usepackage[type1,scaled=.95]{cabin} % sans serif like Gill Sans
\usepackage[utopia,vvarbb,bigdelims]{newtxmath}
\end{verbatim}
The effect of the options {\tt p,osf} is to force the default figure style in {\tt erewhon} text to be proportional oldstyle 0123456789 while using lining figures $0123456789$ in math mode. If no options are specified, tabular lining figures will be used throughout.

\textsc{Options available:}
\begin{itemize}
\item The option {\tt scaled} allows you to change the scale. E.g., if you want \emph{Erewhon} to render at the same size as the original \emph{Utopia} or \emph{Heuristica}, use {\tt scaled=1.064}.
\item
The option {\tt proportional}, or, equivalently, {\tt p}, specifies the use of proportional rather than the default tabular figures.
\item
 The {\tt space} option allows you to specify a factor by which to increase the interword spacing, which is, IMO, a bit tight.
\item
The option {\tt oldstyle}, or, equivalently, {\tt osf}, specifies oldstyle figures in text mode---math mode always uses tabular lining figures. By itself, {\tt osf} results in tabular oldstyle figures unless you also specify the option {\tt p}, or {\tt proportional}.
\item The option {\tt scosf} changes the figure style to {\tt osf} only within small caps. 
\item
 The option {\tt sups}  changes the footnote marker style to use the superior figures from \emph{Erewhon} rather than the default  superscripts based on reduced lining figures, which usually appear too light. (The {\tt superiors} package offers further options.)
\end{itemize}

\emph{Erewhon} is so austere for a text font and \emph{Inconsolata} is so fancy for a typewriter font that you may find they blend together all too well. For more of a distinction replace the {\tt inconsolata} line above with
\begin{verbatim}
\usepackage{zlmtt} % serifed typewriter font extending cmtt
\end{verbatim}

As Utopia text is a bit cramped, you might try  applying a small amount of letterspacing (tracking) and increasing the interword spacing by means of the {\tt microtype} package, or use the {\tt space} option.

\textsc{Macros:}
\begin{itemize}
\item
\verb|\textlf| and \verb|\texttlf| render their arguments in proportional and tabular lining figures respectively, no matter what the default figure style. E.g., \verb|\textlf{345}| produces \textlf{345}.
\item
\verb|\textosf| and \verb|\texttosf| render their arguments in proportional and tabular oldstyle figures respectively, no matter what the default figure style. For example, \verb|\textosf{345}| produces~\textosf{345}.
\item
\verb|\textsu|  renders its argument in superior figures, no matter what the default figure style. E.g., \verb|\textsu{345}| produces \textsu{345}.
\item
\verb|\textin|  renders its argument in inferior figures, no matter what the default figure style. E.g., \verb|\textin{345}| produces \textin{345}.
\item
\verb|\textnu|  renders its argument in numerator figures, no matter what the default figure style. E.g., \verb|\textnu{345}| produces \textnu{345}.
\item
\verb|\textde|  renders its argument in denominator figures, no matter what the default figure style. E.g., \verb|\textde{345}| produces \textde{345}.
\item
\verb|\textfrac|  renders its two arguments as a vulgar fraction, using \verb|\textnu| for the numerator and \verb|\textde| for the denominator. E.g., \verb|\textfrac{31}{64}| produces~\textfrac{31}{64}.
\end{itemize}
\textsc{Very Brief, Nonsensical Math Example:}\\
Let $B(X)$ be the set of blocks of $\Lambda_{X}$
and let $b(X) \coloneq |{B(X)}|$ so that $\hat\phi=\sum_{Y\subset X}(-1)^{b(Y)}b(Y)$. 
\end{document}   