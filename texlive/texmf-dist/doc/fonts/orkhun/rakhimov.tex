\documentclass[11pt]{article}

\makeatletter
\newlength{\myFootnoteWidth}
\newlength{\myFootnoteLabel}
\setlength{\myFootnoteLabel}{0.6em}%  <-- can be changed to any valid value
\renewcommand{\@makefntext}[1]{%
  \setlength{\myFootnoteWidth}{\columnwidth}%
  \addtolength{\myFootnoteWidth}{-\myFootnoteLabel}%
  \noindent\makebox[\myFootnoteLabel][r]{\@makefnmark \ }%
  \parbox[t]{\myFootnoteWidth}{#1}%
}
\makeatother

\makeatletter
\renewcommand\footnoterule{%
  \vspace{1.5em}%   <-- one line space between text and footnoterule
  \kern-3\p@\hrule\@width.4\columnwidth%
  \kern2.6\p@}
\makeatother


\usepackage[dvips]{graphicx}
\usepackage{amssymb}

\usepackage[usenames]{color}
\usepackage{colortbl}

\usepackage{verbatim}
\usepackage{multicol}

\def\4tex{\raisebox{-0.7mm}{\sf 4}$\!$\TeX{}}
\def\latex{(L\hspace*{-1.35mm}\raisebox{1.0mm}{\footnotesize\sc a})\TeX{}}


\textwidth 6.2in
\textheight 8.0in
\oddsidemargin 0mm
\evensidemargin 0mm

\font\mf=logo10

\font\mfsl=logosl10
\hyphenchar\mf=-1
\hyphenchar\mfsl=-1
% A way to get a hyphen, courtesy of Karl Berry.
\newcommand{\MF}{{\mf META}\-{\mf FONT\/}}
\newcommand{\MFSL}{{\mfsl META}\-{\mfsl FONT\/}}

\newcommand{\MP}{{\mf META}\-{\mf POST\/}}
\newcommand{\MPSL}{{\mfsl META}\-{\mfsl POST\/}}

\newfont{\acayipfont}{cmff10}

\newfont{\wasytwenty}{wasy10 at 20pt}
\newcommand{\bigtelephone}{\hbox{\wasytwenty\symbol{7}}}

\newfont{\buyukacafont}{cmff10 at 20pt}
\newfont{\buyukfont}{cmff10 scaled 2500}
\newfont{\kucukfont}{cmff10 scaled 700}
\newfont{\initial}{cmr10 at 48pt}
\usepackage{wrapfig}

\usepackage{makeidx}
\makeindex


\begin{document}

\thispagestyle{empty}

\font\letteror=orhant
\font\letteroor=orhant14
\font\letterooor=orhant16
\font\letteroooor=orhant20
\font\letterooooor=orhant25
%\font\letteroooooor=orhant11



\newbox{\literaT}
\savebox{\literaT}{\hbox{\initial T}}

%{\bf \huge Old Turkic script or Orkhon-Yenisey script.}\\
\begin{center}
{\bf \huge The Old Turkic Script}\\[6mm]

{\bf Abdugafur A. Rakhimov}\\[10mm]

\end{center}

\noindent {\sf \large In brief about history of Script.}\\[- 2mm]

.\vspace*{-0.25cm}
\begin{wrapfigure}{l}{0.75\wd\literaT}
\vbox to 0.4\ht\literaT{%
   \vss \usebox{\literaT}%
   \vspace*{-0.2\ht\literaT}%
}
\end{wrapfigure}
he ORHONO-YENISEI INSCRIPTIONS, the most ancient writings, monuments of the Turkic people.
In 1696-1722 years these inscriptions are opened by Russian scientists S.Remezov, F.Stralenberg,
D.Messershmid in top of the current of the Yenisei. In 1889, on the rivers Orkhons (Mongolia)
by �.M.Jadrintsev is opened. In 1893, these inscriptions are decoded by the Danish linguist
V.Tomsen. And for the first time are read by Russian linguist V.V.Radlov (1894).
The Orhono-yenisei inscriptions concern by 7-11 centuries. Seven groups the Orhono-Yenisei
inscriptions are known: Baikal, Yenisei, Mongolian, Altai, East Turkistan, Central Asian,
and East European. Accordingly they belong to the breeding the union of the Kurgan,
the empire of the Kirghiz, the empire of the East Turkic, the empire of the West Turkic,
the empire of the Uigur (in Mongolia), the state of the Uigur (in East Turkistan), Khazars
(Chazars) and Pechenegs. On a genre accessory are allocated: historically-biographic
stone-letters texts of Mongolia; lyrics of texts of Yenisei; legal documents,
magic and religious texts (on a paper) from East Turkistan; memorable inscriptions on
rocks, stones and structures; labels on household subjects. The inscriptions of Mongolia
stating history 2nd East Turkic and the empire of the Uigur have the greatest historical value.
\\[2mm]

\noindent {\sf \large The Font Base.}\\[-1mm]

It is known, that \latex{} use only the fonts which have specially been written by Donald Knut
in the language of the \MFSL{} Program. Here, we also used \MFSL{} what to create the Old Turkic script
(Gokturk script or Orkhon script or Orkhon-Yenisey script). This font can be used only freely.\\[-1mm]

In the ZIP-file there are .gf, .pk, .tfm files which it is necessary to copy corresponding folders in your
\latex{}  program:\\[-2mm]

\hspace{3.0cm} \verb!...\TEXMF\fonts\pk!

\hspace{3.0cm} \verb!...\TEXMF\fonts\tfm!\\[-2mm]

It is easy see, that these files make such sizes of fonts as 10pt, 11pt,
14pt, 16pt, 20pt and 25pt. But, using the .mf file and setting the necessary values variable "u" (in the .mf file) you can create
the necessary sizes of fonts (i.e. .gf, .pk, .tfm files you can to create, also, using  .mf file).
In this case, of course, you should compile your .mf file (with new parametre of "u") on the \MFSL{} Program.\\[-3mm]

{\footnotesize
\begin{verbatim}
     ..\BIN\win32\mf \mode=ljfour; mode_setup; input orhun.mf
     ..\BIN\win32\gftodvi orhun.600gf
     ..\BIN\win32\gftopk orhun.600gf
     ..\BIN\win32\dvips orhun.dvi
\end{verbatim}
}

\vspace{0.3cm}

\noindent Then, the necessary font to appear in expansions .gf, .pk, .tfm. Moreover, these files can be copied in
the catalogue (directory) in which you work.\\[2mm]

\noindent {\sf \large To use the Orkhon-Yenisey script in the text.}\\[-2mm]

To write the text in old Turkic language you should use the \TeX{}-command: \verb!\font!.
For example,

\vspace{0.3cm}

{\small
\noindent
\begin{minipage}{4.3in}
\begin{verbatim}
\font\letteror=orhun

{\letteror ABGdDOUoZt}

{\letteror TYylLMnNsS}

{\letteror PCkKrRvmcQ}

{\letteror zVeawgibI:}
\end{verbatim}
\end{minipage}
\begin{minipage}{2.5in}
\begin{tabular}{|l|}
\hline\\
{\letterooooor ABGdDOUoZt}\\[3mm]
{\letterooooor TYylLMnNsS}\\[3mm]
{\letterooooor PCkKrRvmcQ}\\[3mm]
{\letterooooor zVeawgibI:}\\[3mm]
\hline
\end{tabular}
\end{minipage}
}

\vspace{0.5cm}

\noindent So the following historical text is written

\vspace{0.5cm}

\noindent {\bf The monument of the TONYUKUK } ({\small The second stone, northern a part})\\[0.3cm]

\hfill {\letterooor VOTAGLeBMZoNB:rsretrkY:RSMngZa:ngasertLe}\\[-2mm]

\hfill {\letterooor rsrMtrkYNB:RSMngZa:kk}\\[-2mm]

\hfill {\letterooor AMynDOb:AMyDOb:AtNery:nDObresKrot:ngangPa}\\[-2mm]

\hfill {\letterooor :etreCtrkYede:AMyesei:}\\[-2mm]

\hfill {\letterooor ot:ngangPa:NCokQgZa:kkVOTAGLeB:ngasrtLe}\\[-2mm]

\hfill {\letterooor ObekDeROY:nDObresKr}\\[-2mm]

\hfill {\letterooor odGe:gnDObZgO:gnDObresKrot:nga:AGLeBKrot}\\[-2mm]

\hfill {\letterooor RURlU}

\vspace{0.5cm}

\noindent To write the text in Orkhon script, certainly, we should knowd Old Turkic alphabet.
In the following table this alphabet is given

\vspace{0.75cm}

\hspace*{-1.0cm}
\begin{tabular}{ll|ll|ll}
\hline
{\scriptsize O-Y} & {\small sound} & {\scriptsize O-Y}  & {\small sound} & {\scriptsize O-Y} & {\small sound}\\[1mm]
\hline
& & & \\[-3mm]
{\letterooor A}&\verb!a!, \verb!e!                                                          &{\letterooor a}&\verb!k! ({\scriptsize with} {\small \verb!a!} )                  & {\letterooor R} & \verb!r! {\scriptsize (with thick vowels)}\\[2mm]

{\letterooor B}&\verb!b! {\scriptsize (with thin vowels)}                             &{\letterooor w}&\verb!k! ({\scriptsize with hard} {\small \verb!i!} )                   & {\letterooor r} & \verb!r! {\scriptsize (with thin vowels)}\\[2mm]

{\letterooor b}&\verb!b! {\scriptsize (with thick vowels)}                                &{\letterooor l}&\verb!L! {\scriptsize (with thick vowels)}  & {\letterooor S} & \verb!s! {\scriptsize (with thick vowels)}\\[2mm]

{\letterooor C}&\verb!-ch (j)! {\scriptsize (in the end a word)}                                   &{\letterooor L}&\verb!L! {\scriptsize (with thin vowels)}   & {\letterooor s} & \verb!s! {\scriptsize (with thin vowels)}\\[2mm]

{\letterooor I}&\verb!ich!,\verb!  ic!                                                      &{\letterooor m}&\verb!-it! {\scriptsize (in the end a word)} & {\letterooor v} & \verb!-sh! {\scriptsize (in the end a word)}\\[2mm]

{\letterooor D}&\verb!d! {\scriptsize (with thick vowels)}                                &{\letterooor M}&\verb!-m! {\scriptsize (in the end a word)}     & {\letterooor T} & \verb!t! {\scriptsize (with thick vowels)}\\[2mm]

{\letterooor d}&\verb!d! {\scriptsize (with thin vowels)}                             &{\letterooor n}&\verb!n! {\scriptsize (with thick vowels)}  & {\letterooor t} & \verb!t! {\scriptsize (with thin vowels)}\\[2mm]

{\letterooor G}&\verb!g! {\scriptsize (with thin vowels)}                             &{\letterooor N}&\verb!n! {\scriptsize (with thin vowels)}   & {\letterooor O} & \verb!o!, \verb!u!\\[2mm]

{\letterooor g}&\verb!g! {\scriptsize (with thick vowels)}                                &{\letterooor z}&\verb!-ng! {\scriptsize (in the end a word)}    & {\letterooor U} & \verb!o!, \verb!u!\\[2mm]

{\letterooor e}& i, \ {\scriptsize the closed} {\small \verb!e! }                                &{\letterooor V}&\verb!-ny! {\scriptsize (in the end a word)}    & {\letterooor o} & \verb!o'!, {\scriptsize the soft} {\small \verb!u!} \\[2mm]

{\letterooor i}&\verb!k! ({\small \verb!e!,} {\scriptsize with \verb!i!} )                  &{\letterooor c}&\verb!-nch(j)! {\scriptsize (in the end a word)} & {\letterooor Y} & \verb!y! {\scriptsize (with thick vowels)}\\[2mm]

{\letterooor K}&\verb!k! (\verb!o'!, {\scriptsize with the soft} {\small \verb!u!} )  &{\letterooor Q}&\verb!-nt(d)! {\scriptsize (in the end a word)} & {\letterooor y} & \verb!y! {\scriptsize (with thin vowels)}\\[2mm]

{\letterooor k}&\verb!k! ({\scriptsize with} {\small \verb!o!, \verb!u! } )                 &{\letterooor P}&\verb!-p! {\scriptsize (in the end a word)}     & {\letterooor Z} & \verb!-z! {\scriptsize (in the end a word)}\\[2mm]
\hline
\end{tabular}

\vspace{1.0cm}

\noindent In the keyboard the Orkhon script are typed as follows

\vspace{0.45cm}

{
%\small
\scriptsize

\vspace*{0.5cm}

\hspace*{-2.3cm}
\fcolorbox[rgb]{.1,.1,.1}{.9,.9,.9}{
\begin{tabular}{lr}
Q&{\letteror Q}\\[2mm]
q&\phantom{{\letteror w} }
\end{tabular} }
\fcolorbox[rgb]{.1,.1,.1}{.9,.9,.9}{
\begin{tabular}{lr}
W&\phantom{{\letteror w} }\\[2mm]
w&{\letteror w}
\end{tabular} }
\fcolorbox[rgb]{.1,.1,.1}{.9,.9,.9}{
\begin{tabular}{lr}
E&\phantom{{\letteror w} }\\[2mm]
e&{\letteror e}
\end{tabular} }
\fcolorbox[rgb]{.1,.1,.1}{.9,.9,.9}{
\begin{tabular}{lr}
R&{\letteror R}\\[2mm]
r&{\letteror r}
\end{tabular} }
\fcolorbox[rgb]{.1,.1,.1}{.9,.9,.9}{
\begin{tabular}{lr}
T&{\letteror T}\\[2mm]
t&{\letteror t}
\end{tabular} }
\fcolorbox[rgb]{.1,.1,.1}{.9,.9,.9}{
\begin{tabular}{lr}
Y&{\letteror Y}\\[2mm]
y&{\letteror y}
\end{tabular} }
\fcolorbox[rgb]{.1,.1,.1}{.9,.9,.9}{
\begin{tabular}{lr}
U&{\letteror U}\\[2mm]
u&\phantom{{\letteror w} }
\end{tabular} }
\fcolorbox[rgb]{.1,.1,.1}{.9,.9,.9}{
\begin{tabular}{lr}
I&{\letteror I}\\[2mm]
i&{\letteror i}
\end{tabular} }
\fcolorbox[rgb]{.1,.1,.1}{.9,.9,.9}{
\begin{tabular}{lr}
O&{\letteror O}\\[2mm]
o&{\letteror o}
\end{tabular} }
\fcolorbox[rgb]{.1,.1,.1}{.9,.9,.9}{
\begin{tabular}{lr}
P&{\letteror P}\\[2mm]
p&\phantom{{\letteror w} }
\end{tabular} }

\vspace*{0.2cm}

\hspace*{-1.0cm}
\fcolorbox[rgb]{.1,.1,.1}{.9,.9,.9}{
\begin{tabular}{lr}
A&{\letteror A}\\[2mm]
a&{\letteror a}
\end{tabular} }
\fcolorbox[rgb]{.1,.1,.1}{.9,.9,.9}{
\begin{tabular}{lr}
S&{\letteror S}\\[2mm]
s&{\letteror s}
\end{tabular} }
\fcolorbox[rgb]{.1,.1,.1}{.9,.9,.9}{
\begin{tabular}{lr}
D&{\letteror D}\\[2mm]
d&{\letteror d}
\end{tabular} }
\fcolorbox[rgb]{.1,.1,.1}{.9,.9,.9}{
\begin{tabular}{lr}
F&\phantom{{\letteror K}}\\[2mm]
f&\phantom{{\letteror K}}
\end{tabular} }
\fcolorbox[rgb]{.1,.1,.1}{.9,.9,.9}{
\begin{tabular}{lr}
G&{\letteror G}\\[2mm]
g&{\letteror g}
\end{tabular} }
\fcolorbox[rgb]{.1,.1,.1}{.9,.9,.9}{
\begin{tabular}{lr}
H&\phantom{{\letteror K}}\\[2mm]
h&\phantom{{\letteror K}}
\end{tabular} }
\fcolorbox[rgb]{.1,.1,.1}{.9,.9,.9}{
\begin{tabular}{lr}
J&\phantom{{\letteror K}}\\[2mm]
j&\phantom{{\letteror K}}
\end{tabular} }
\fcolorbox[rgb]{.1,.1,.1}{.9,.9,.9}{
\begin{tabular}{lr}
K&{\letteror K}\\[2mm]
k&{\letteror k}
\end{tabular} }
\fcolorbox[rgb]{.1,.1,.1}{.9,.9,.9}{
\begin{tabular}{lr}
L&{\letteror L}\\[2mm]
l&{\letteror l}
\end{tabular} }

\vspace*{0.25cm}

\hspace*{-0.25cm}
\hspace*{0.75cm} \fcolorbox[rgb]{.1,.1,.1}{.9,.9,.9}{
\begin{tabular}{lr}
Z&{\letteror Z}\\[2mm]
z&{\letteror z}
\end{tabular} }
\fcolorbox[rgb]{.1,.1,.1}{.9,.9,.9}{
\begin{tabular}{lr}
X&\phantom{{\letteror K}}\\[2mm]
x&\phantom{{\letteror K}}
\end{tabular} }
\fcolorbox[rgb]{.1,.1,.1}{.9,.9,.9}{
\begin{tabular}{lr}
C&{\letteror C}\\[2mm]
c&{\letteror c}
\end{tabular} }
\fcolorbox[rgb]{.1,.1,.1}{.9,.9,.9}{
\begin{tabular}{lr}
V&{\letteror V}\\[2mm]
v&{\letteror v}
\end{tabular} }
\fcolorbox[rgb]{.1,.1,.1}{.9,.9,.9}{
\begin{tabular}{lr}
B&{\letteror B}\\[2mm]
b&{\letteror b}
\end{tabular} }
\fcolorbox[rgb]{.1,.1,.1}{.9,.9,.9}{
\begin{tabular}{lr}
N&{\letteror N}\\[2mm]
n&{\letteror n}
\end{tabular} }
\fcolorbox[rgb]{.1,.1,.1}{.9,.9,.9}{
\begin{tabular}{lr}
M&{\letteror M}\\[2mm]
m&{\letteror m}
\end{tabular} }

}

%\vspace*{0.5cm}

\newpage

\noindent {\sf \large The conclusion.}\\[-3mm]

\noindent Further we plan to write a orkhun-package which will allow to work more conveniently with these fonts and will expand possibilities use of fonts.
On all questions and wishes, we ask you to address to the address written more low. We will be very glad to have your valuable remarks and wishes.\\[1mm]

\noindent {\sf \large Acknowledgements.}\\[-3mm]

\noindent I would like to thank the members of the Mathematical Department of the Karadeniz Technical University, for hospitality while this work was being done.
In particular I am grateful to my friend to professor {\bf Vam\i k Kadimo\u glu} for helpful conversations.\\[-1mm]




\medskip \smallskip Prof.Dr. {\bf Abdugafur Rakhimov} \newline
The Tashkent institute of railways and engineering. Tashkent, Uzbekistan\newline
The Karadeniz Technical University, Trabzon, Turkey\newline
E-mail: rakhimov@ktu.edu.tr ; \ rakhimov2002@yahoo.com  ; \ gafur\_rakhimov@yahoo.com



\end{document}
