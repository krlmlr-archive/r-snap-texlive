% trylinearb.tex    Linear B font
\documentclass{article}
%%\documentclass[12pt]{article}
\usepackage{linearb}

\title{The Linear B Font}
\author{}
\date{}

\renewcommand{\baselinestretch}{1.2}
\begin{document}
%%\maketitle

    This provides short examples of the Linear B font.

    The vowels (a, e, i, o, u) are: \textlinb{a e i o u}.

    The d syllables (da, de, di, do, du) are: \textlinb{d D f g x}

    The j syllables (ja, je, jo, ju) are: \textlinb{j J b L}.

    The k syllables (ka, ke, ki, ko, ku) are: \textlinb{k K c h v}

    The m syllables (ma, me, mi, mo, mu) are: \textlinb{m M y A B}

    The n syllables (na, ne, ni, no, nu) are: \textlinb{n N C E F}

    The p syllables (pa, pe, pi, po, pu) are: \textlinb{p P G H I}.

    The q syllables (qa, qe, qi, qo) are: \textlinb{q Q X 8}.

    The r syllables (ra, re, ri, ro, ru) are: \textlinb{r R O U V}.

    The s syllables (sa, se, si, so, su) are: \textlinb{s S Y 1 2}

    The t syllables (ta, te, ti, to, tu) are: \textlinb{t T 3 4 5}

    The w syllables (wa, we, wi, wo) are: \textlinb{w W 6 7}.

    The z syllables (za, ze, zo) are: \textlinb{z Z 9}.
  
    The word divider (coded as , and : and /) are: \textlinb{, : /}

    The optional a signs (a2, a3, au) are: \textlinb{\Baii{} \Baiii{} \Bau}

    The optional d signs (dwe, dwo) are: \textlinb{\Bdwe{} \Bdwo}

    The optional n signs (nwa) are: \textlinb{\Bnwa}

    The optional p signs (pa3, pu2, pte) are: \textlinb{\Bpaiii{} \Bpuii{} \Bpte}

    The optional r signs (ra2, ra3, ro2) are: \textlinb{\Braii{} \Braiii{} \Broii}

    The optional s signs (swa, swi) are: \textlinb{\Bswa{} \Bswi}

    The optional t signs (ta2, two) are: \textlinb{\Btaii{} \Btwo}

%%\clearpage
    
    The numbers 1 to 9: \textlinb{\BNi{} \BNii{} \BNiii{} \BNiv{} \BNv{} \BNvi{} \BNvii{} \BNviii{} \BNix}

    The numbers 10 to 90: \textlinb{\BNx{} \BNxx{} \BNxxx{} \BNxl{} \BNl{} \BNlx{} \BNlxx{} \BNlxxx{} \BNxc}

    The numbers 100 to 900: \textlinb{\BNc{} \BNcc{} \BNccc{} \BNcd{} \BNd{} \BNdc{} \BNdcc{} \BNdccc{} \BNcm}

    The number 1000: \textlinb{\BNm}

    The unidentified signs: \textlinb{\BUi\ \BUii\ \BUiii\ \BUiv\ \BUv\ \BUvi\
                                 \BUvii\ \BUviii\ \BUix\ \BUx\ \BUxi} \\
    Note that, depending on the source, the sign \textlinb{\BUxii} is classed 
either as an unidentified sign or as an optional \textit{twe} sign. 

%%\clearpage

\newcommand{\egtext}{\Bti\Bme/\Bto/\Bre\Bti\Bre}
\newcommand{\egnum}{\BNm\BNcm\BNlxxx\BNiv\ \BUiii}
The Linear B text \textlinb{\egtext} transliterates to
\translitlinb{\egtext} and \textlinb{\egnum} transliterates
to \translitlinb{\egnum}.
The previous sentence was produced by:
\begin{verbatim}
\newcommand{\egtext}{\Bti\Bme/\Bto/\Bre\Bti\Bre}
\newcommand{\egnum}{\BNm\BNcm\BNlxxx\BNiv\ \BUiii}
The Linear B text \textlinb{\egtext} transliterates to
\translitlinb{\egtext} and \textlinb{\egnum} transliterates
to \translitlinb{\egnum}.
\end{verbatim}

\newcommand{\weights}{\BPwta\ \BPwtb\ \BPwtc\ \BPwtd\ \BPtalent}
The system of weight measures: \textlinb{\weights} which transliterates as:
                       \translitlinb{\weights}

\newcommand{\fluids}{\BPvola\ \BPvolb\ \BPvolcd\ \BPvolcf}
The system of fluid measures: \textlinb{\fluids} which transliterates as:
                       \translitlinb{\fluids}

\newcommand{\commodities}{\BPcloth\ \BPwool\ \BPwheat\ \BPbarley\
                         \BPwine\ \BPolive\ \BPbronze\ \BPgold}
Some commodities: \textlinb{\commodities} which transliterates as:
                       \translitlinb{\commodities}

\newcommand{\menhorses}{\BPman\ \BPwoman\ \BPhorse\ \BPfoal}
Some people and animals: \textlinb{\menhorses}  which transliterates as:
                         \translitlinb{\menhorses}

\newcommand{\livestock}{\BPpig\ \BPboar\ \BPsow\
                        \BPox\ \BPbull\ \BPcow\
                        \BPsheep\ \BPram\ \BPewe\
                        \BPgoat\ \BPbilly\ \BPnanny}
Some livestock: \textlinb{\livestock}  which transliterates as:
                         \translitlinb{\livestock}

\newcommand{\weapons}{\BPchariot\ \BPchassis\ \BPwheel\ 
                      \BPsword \BParrow\ \BPspear}
Some weapons: \textlinb{\weapons}  which transliterates as:
                         \translitlinb{\weapons}




\end{document}

Just checking that transliteration does no permanent damage: 
First \textlinb{\egtext} 
then as \translitlinb{\egtext} 
and back again \textlinb{\egtext} 

\end{document}

A familiar English saying: 
\textlinb{Now Is The Time For All Good Men And Women To Come To The Aid Of the
Party While the Quick Brown Fox Jumps Over The Lazy Dog}

\end{document}
