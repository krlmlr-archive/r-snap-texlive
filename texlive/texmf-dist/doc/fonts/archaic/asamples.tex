% asamples.tex    Samples of archaic fonts

\documentclass{article}
%%\usepackage[T1]{fontenc}
\usepackage{protosem}
\usepackage{phoenician}
\usepackage{greek6cbc}
\usepackage{greek4cbc}
\usepackage{etruscan}
\usepackage{runic}
\usepackage{linearb}
\usepackage{cypriot}
\usepackage{hieroglf}
\usepackage{ugarite}
\usepackage{oldprsn}
\usepackage{aramaic}
\usepackage{nabatean}
\usepackage{sarabian}
\usepackage{oands}

%\newcommand{\ABC}{ABCDEFGHIJKL MNOPQRSTUVWXYZ}
\newcommand{\abc}{abcdefghijklmnopqrstuvwxyz}

\newcommand{\BC}{\textsc{bc}}
\newcommand{\AD}{\textsc{ad}}

\title{Archaic Font Sampler}
\author{Peter Wilson\thanks{\texttt{herries dot press at earthlink dot net}}\\
        Herries Press}
\date{6 February 2006}
\begin{document}
\maketitle
\tableofcontents

\clearpage
\section{Introduction}

    This document provides some samples of archaic fonts. They are
available from CTAN in the \texttt{fonts/archaic} directory. The fonts
form a set that display how the Latin alphabet and script evolved from the
initial Proto-Semitic script until Roman times.

    The fonts tend to consist of letters only --- punctuation had not 
been invented during this period except for a word-divider in some cases.
Some of the scripts had signs for numbers but in others either letters
doubled as numbers or the numbers were spelt out. The fonts are all
single-cased. Upper- and lower-case letters were again only invented after
the end of this period.

    Other fonts are also available for some scripts that were not on the
main alphabetic tree. The period covered by the scripts is from about 
3000~\BC{} to the Middle Ages.

    For some of the scripts transliterations into the Latin alphabet can
be automatically generated by the accompanying LaTeX packages.

\clearpage


\section{Proto-Semitic}

    Around 1600~\BC{} there were alphabetic scripts in use in the Middle East
that are variously called Proto-Siniatic, Proto-Canaanite, etc. I
have lumped these together into a Proto-Semitic font. Several
of the signs in this alphabet are obviously derived from Egyptian
Hieroglyphs, and it may have been a precursor to the Phoenician script.

    The alphabet consisted of 23 letters, some of which had alternate
forms. Writing was generally from left to right, but could be vertical
or in other directions.

    The font and package bundle is in the \texttt{protosem} subdirectory.

\begin{center}
\protofamily
%abgdzewhiyklmnopuvqrsxt
%abgdzhwHTyklmnospxqrSvt
\Aaleph \Abeth \Agimel \Adaleth \Azayin \Ahe \Avav \Aheth \Ateth
\Ayod \Akaph \Alamed \Amem \Anun \Aayin \Asamekh \Ape \Asade \Aqoph 
\Aresh \Ashin \Ahelmet \Atav \\
abgdzwHTyklmnospxqrSvt \\
\AAaleph \AAbeth \AAdaleth \AAhe \AAheth 
\AAyod \AAkaph \AAlamed \AAayin \AApe \AAsade \AAqoph 
\AAresh \AAhelmet \\
ABDeEYKLOPXQRV
\end{center}


\section{Phoenician}

    The Phoenician alphabetic script dates from about 1000~\BC. 
The Phoenecians wrote right to left. The script is provided in both a
left to right and a right to left form. The alphabet consisted of
22 letters. The font provided actually includes a 23rd letter as
the \textit{vau} character had two forms: \textphnc{V} and \textphnc{F}.

    The font and package bundle is in the \texttt{phoenician} subdirectory.

\begin{center}
\phncfamily
%%ABGDEFVZH\TTheta IKLMN\TXi OP\Tsade QRST \\
%%tsrq\tsade po\tXi nmlki\tTheta hzvfedgba
\Aaleph \Abeth \Agimel \Adaleth \Ahe \Avaf \Avav \Azayin \Aheth \Ateth
\Ayod \Akaph \Alamed \Amem \Anun \Asamekh \Aayin \Ape \Asade \Aqoph 
\Aresh \Ashin \Atav \\
abgdhfwzHTyklmnsopxqrSt \\
\ARaleph \ARbeth \ARgimel \ARdaleth \ARhe \ARvaf \ARvav \ARzayin \ARheth \ARteth
\ARyod \ARkaph \ARlamed \ARmem \ARnun \ARsamekh \ARayin \ARpe \ARsade \ARqoph 
\ARresh \ARshin \ARtav \\
aBGdeFwzETyKLMNsoPXqRSt \\
\end{center}

\section{Greek}

    The Greeks used the Phoenician script and alphabet as a basis for their
own writing. In the earliest times the Greeks used to alternate the writing
direction for successive lines, but later settled down to writing left to 
right. Two forms of the Greek script are provided. 

The earlier one is representative of about the sixth century~\BC, when the
Greek alphabet consisted of 26 letters.
    The font and package bundle is in the \texttt{greek6cbc} subdirectory.

\begin{center}
\gvibcfamily
%%ABGDEFZH\TTheta IKLMN\TXi OPQRSTUX\TPhi\TPsi\TOmega
\Aalpha \Abeta \Agamma \Adelta \Aepsilon \Adigamma \Azeta \Aeta
\Atheta \Aiota \Akappa \Alambda \Amu \Anu \Axi \Aomicron \Api \Akoppa
\Arho \Asigma \Atau \Aupsilon \Achi \Aphi \Apsi \Aomega \\
abgdeFzhTiklmnxopqrstyXfPO \\
\end{center}

   The later script is representative of the fourth century~\BC, when the
alphabet had been reduced to the 24 letters in use today.
    The font and package bundle is in the \texttt{greek4cbc} subdirectory.

\begin{center}
\givbcfamily
%%ABGDEZH\TTheta IKLMN\TXi OPRSTUX\TPhi\TPsi\TOmega 
%%abgdezh\tTheta iklmn\tXi oprstux\tPhi\tPsi\tOmega
\Aalpha \Abeta \Agamma \Adelta \Aepsilon \Azeta \Aeta
\Atheta \Aiota \Akappa \Alambda \Amu \Anu \Axi \Aomicron \Api
\Arho \Asigma \Atau \Aupsilon \Achi \Aphi \Apsi \Aomega \\
abgdezhTiklmnxoprstyXfPO \\
\end{center}

\section{Etruscan}

    The Etruscans, who were the precursors of the Romans in Italy, based their
alphabet and script on the Phoenician and Greek models. Like the Phoenicians,
they wrote right to left. In their turn, the Romans based their alphabet on
the Etruscan one, but they wrote left to right and provided the forms of
our capital letters.

    The script is provided in both a left to right and a right to left form
and is representative of about the eighth century~\BC.
    The font and package bundle is in the \texttt{etruscan} subdirectory.

\begin{center}
\etrfamily
%%ABGDEFZH\TTheta IKLMN\TXi OP\Tsade QRSTUX\TPhi\TPsi 8\\
%%8\tPsi\tPhi xutsrq\tsade po\tXi nmlki\tTheta hzfedgba
\Aalpha \Abeta \Agamma \Adelta \Aepsilon \Adigamma \Azeta \Aeta
\Atheta \Aiota \Akappa \Alambda \Amu \Anu \Axi \Aomicron \Api
\Aesade \Aqoph \Arho \Asigma \Atau \Aupsilon \Achi \Aphi \Apsi \Avau \\
abgdeFzhTiklmnxopSqrstyXfPv \\
\ARalpha \ARbeta \ARgamma \ARdelta \ARepsilon \ARdigamma \ARzeta \AReta
\ARtheta \ARiota \ARkappa \ARlambda \ARmu \ARnu \ARxi \ARomicron \ARpi
\AResade \ARqoph \ARrho \ARsigma \ARtau \ARupsilon \ARchi \ARphi \ARpsi \ARvau \\
aBGDEUzHTiKLMNxoQSqRZJyXfPv \\
\end{center}
    

\section{Runic}

    The \textit{futharc} alphabetic script 
(named, like ours, after the first few letters)
was a script used in Northern Europe until after the Middle Ages, when it
died out after printing was invented. It has glyphs that are reminiscent of
some of those in the other scripts shown here, but some are wildly different.
I have included it because it has the \textit{wen} character, \textfut{W},
which is the first appearance of a character for the `W' sound. Also, it
has the \textit{ger} character, \textfut{J}, which corresponds to the modern
`J' sound --- `J' did not appear in the Latin alphabet until about the mid
1500's.

    The Anglo-Saxon version of the font and package bundle is in the \texttt{runic} 
subdirectory.

\begin{center}
\futfamily
FU\Fthorn ARKGWHNIJYPXSTBEML\Fng DO
\end{center}

\section{Hieroglyphics}

    Hieroglyphics were used by the Egyptians from about 3000~\BC{} to
400~\AD. The script is partly a consonantal alphabet, a syllabary,
and logograms. There are approximately 6000 known different
hieroglyphs, but less than 1000 were in use at any one time.

    The hieroglf package provides about 60 hieroglyphs, enough to
write a few names, like Cleopatra or Ptolemy. Serge Rosmorduc's
hieroglyphic package is for serious Egyptologists and provides about
600 hieroglyphs.

    The hieroglf bundle, which is in the \texttt{hieroglf} subdirectory and
which requires the oands package, includes the following hieroglyphs:

\begin{center}
\renewcommand{\baselinestretch}{1.2}
\pmhgfamily
a b c d e f g h i j k l m n o p q r s t u v w x y z
A B C D E F G H I J K L M N O P Q R S T U V W X Y Z
+ ? / | \Hms\ Hibp\ \Hibw\ \Hibs\ \Hibl\ \Hsv
\end{center}

    The package can also be used for hieroglyphs within either
horizontal or vertical cartouches. The three below are the
names Cleopatra, Caesar and Alexander.

\begin{center}
\renewcommand{\baselinestretch}{1.2}
\Cartouche{\pmglyph{K:l-i-o-p-a-d:r-a}}
\Vertouche{\pmvglyph{k:y:S:l:S}}
\Cartouche{\pmglyph{a-l:k-s-i-n:d-r:S}}
\end{center}

\section{Linear B}

    The \textit{Linear B} script was a syllabary that was used during the 
approximate period between 1600 and 1200~\BC{} for writing the Mycenaean Greek
dialect. Most examples of the script come from Crete, and in particular from
Knossos. The script was deciphered in 1953 by Michael Ventris with assistance
from John Chadwick.

    The script consists of some 60 basic signs, 16 optional signs, and about
11 signs that have yet to be deciphered. The script also had signs for numbers
(1--1000), and signs for various kinds of weights and measures. There were
also sets of signs for different kinds of trade goods, such as pots or wool.
Although Linear B was used for writing Greek, there is no other relationship
between this ancient script and the Greek alphabet.

    The script as supplied includes only the basic, optional, unidentified,
and number signs. The font and package bundle is in the \texttt{linearb}
subdirectory. The basic signs are illustrated below.

\begin{center}
\renewcommand{\baselinestretch}{1.2}
\linbfamily
a e i o u 
d D f g x 
j J b L
k K c h v
m M y A B
n N C E F
p P G H I
q Q X 8
r R O U V
s S Y 1 2
t T 3 4 5
w W 6 7
z Z 9
\end{center}

\section{Cypriot}

    The \textit{Cypriot} script was a syllabary used in Cyprus during the
approximate period between 1000 and 200~\BC{} for writing Greek. It has a 
relationship to Linear~B as it includes some of the same signs. Towards
the end of its life few people could read the script, so inscriptions
were written using both the syllabary and the Greek alphabetic characters.
These bilinguals made it relatively easy to decipher the script, a task that
was essentially completed by the last quarter of the nineteenth century.

    Like Linear B, the Cypriot script has no relationship with the Greek
alphabet apart from the fact that both can be used for writing the same
language.

    The script consisted of 55 signs. The font and package bundle is
in the \texttt{cypriot} subdirectory.

\begin{center}
\renewcommand{\baselinestretch}{1.2}
\cyprfamily
a e i o u
g
j b
k K c h v
l L d f q
m M y A B
n N C E F
p P G H I
r R O U V
s S Y 1 2
t T 3 4 5
w W 6 7
x X z
9

\end{center}

\section{Ugaritic Cuneiform}

    The first cuneiform scripts appeared around 2800~\BC{} and the last 
recorded use of cuneiform was in 75~\AD.

    The Ugaritic cuneiform script dates from about 1300~\BC{} and
was alphabetical. The script consisted of 30 letters and a ideographic word
divider (a short vertical wedge). The font and package bundle
is in the \texttt{ugarite} subdirectory.

\begin{center}
\cugarfamily
\Aaleph\ \Abeth\ \Agimel\ \Ahu\ \Adaleth\ 
\Ahe\ \Avav\ \Azayin\ \Aheth\ \Ateth\ 
\Ayod\ \Akaph\ \Asa\ \Alamed\ \Amem\ 
\Adb\ \Anun\ \Azd\ \Asamekh\ \Aayin\ 
\Ape\ \Asade\ \Aqoph\ \Aresh\ \Atb\ 
\Agd\ \Atav\ \Ai\ \Au\ \Asg\ \Awd \\
a b g I d h w z H T y k X l m D n Z s o p x q r J G t i u V :
\end{center}

    Modern Western transliterations of these glyphs are:
\begin{center}
\translitcugar{%
\Aaleph\ \Abeth\ \Agimel\ \Ahu\ \Adaleth\ 
\Ahe\ \Avav\ \Azayin\ \Aheth\ \Ateth\ 
\Ayod\ \Akaph\ \Asa\ \Alamed\ \Amem\ 
\Adb\ \Anun\ \Azd\ \Asamekh\ \Aayin\ 
\Ape\ \Asade\ \Aqoph\ \Aresh\ \Atb\ 
\Agd\ \Atav\ \Ai\ \Au\ \Asg\ \Awd 
}
\end{center}

\section{Old Persian}

    It is believed that the Old Persian cuneiform script was invented on
the order of the Persian king Darius~I for use on royal monuments.
The script was only in use between about 500 and 350~\BC.

   Old Persian was a syllabary with 36 glyphs. There were also 5 ideographs,
some with multiple forms, 
for the words \textit{king}, \textit{country}, \textit{earth}, \textit{god}
and \textit{Ahuramazda} (the Persian god), together with a word divider.
Numerals were also represented. 
The font and package bundle is in the \texttt{oldprsn} subdirectory.


\begin{center}
\copsnfamily
\Oa\ \Oi\ \Ou\ \Oka\ \Oku\ 
\Oxa\ \Oga\ \Ogu\ \Oca\ \Oja\ 
\Oji\ \Ota\ \Otu\ \Otha\ \Occa\ 
\Oda\ \Odi\ \Odu\ \Ona\ \Onu\ 
\Opa\ \Ofa\ \Oba\ \Oma\ \Omi\ 
\Omu\ \Oya\ \Ora\ \Oru\ \Ola\ 
\Ova\ \Ovi\ \Osa\ \Osva\ \Oza\ 
\Oha\ 
\Oking\ \Ocountrya\ \Ocountryb\ \Oearth\ \Ogod\ 
\OAura\ \OAurb\ \OAurc\ 
\Owd
\end{center}

\section{Aramaic}

    The Aramaic script is an early offshoot from the Phoenician and was used
between about the tenth and second centuries~\BC{} in the Middle East.
The Aramaic script also branched and both modern Arabic and Square Hebrew
scripts.

    The script is alphabetical and consists of 22 letters.

The font and package bundle is in the \texttt{aramaic} subdirectory.

\begin{center}
\aramfamily
\Aaleph \Abeth \Agimel \Adaleth \Ahe \Avav \Azayin \Aheth \Ateth
\Ayod \Akaph \Alamed \Amem \Anun \Asamekh \Aayin \Ape \Asade \Aqoph 
\Aresh \Ashin \Atav \\
abgdhwzHTyklmnsopxqrSt
\end{center}

\section{Nabatean}

    The Nabatean script is an offshoot of the Aramaic script and was in use 
roughly during the period between the fourth century~\BC{} and the fourth 
century~AD. It is a direct ancestor of the modern Arabic script.

    The script is alphabetical and consists of 22 letters.

The font and package bundle is in the \texttt{nabatean} subdirectory.

\begin{center}
\nabfamily
\Aaleph \Abeth \Agimel \Adaleth \Ahe \Avav \Azayin \Aheth \Ateth
\Ayod \Akaph \Alamed \Amem \Anun \Asamekh \Aayin \Ape \Asade \Aqoph 
\Aresh \Ashin \Atav \\
abgdhwzHTyklmnsopxqrSt
\end{center}

\section{South Arabian}

    The South Arabian script was in use in southern Arabia between about the 
fifth century~\BC{} and the sixth century~\AD. It was used, for example,
in the ancient South Arabian kingdoms of the Sabaeans and the Minaeans. 
Different scripts were used in north Arabia.

The font and package bundle is in the \texttt{sarabian} subdirectory.

\begin{center}
\sarabfamily
\SAh \SAl \SAhd \SAm \SAq \SAw \SAsv \SAr \SAb \SAt 
\SAs \SAk \SAn \SAhu \SAsa \SAf \SAa \SAo \SAdd 
\SAg \SAd \SAga \SAtd \SAz \SAdb \SAy \SAtb \SAsd \SAzd \\ 
hlHmqwSrbtsknIXfaoBgdGTzDyJxZ
\end{center}




\clearpage
\section{Transliteration}

    There are generally accepted transliterations of ancient scripts
into modern alphabetic characters, often with diacritics. The
OandS font, in subdirectory \texttt{oands}, provides a few odd 
characters needed for these.

    At present, the font consists of only two characters: \textoands{` z}.

    For example, the following hieroglyphs:
\begin{center}
\renewcommand{\baselinestretch}{1.2}
\pmglyph{\HU\Hw\HL\HJ\Hf-\Hq:\Hvbar-\HR:{\Hr:\Hr}-\Hy-\Ht:\HN-\Hf-\Hn:\Ht-\Hc\Hm\Hv-\HG:\Hvbar-\Hf\HZ\Hw\HV}
\end{center}
are transliterated as:
\begin{center}
\translitpmhg{\HU\ \HJ.\Hf\ \Hq\ \HR\Hr\Hy\Ht.\Hf\ \Hn\Ht\ \Hc\ \HG.\Hf\ \HZ}
\end{center}
and can be translated as:
\begin{center}
His Majesty departed upon his chariot of electrum, his heart joyful.
\end{center}

\newcommand{\xerxes}{%
 \Oxa\Osva\Oya\Oa\Ora\Osva\Oa\Owd\-%                  1
 \Oxa\Osva\Oa\Oya\Otha\Oi\Oya\Owd\-%                  2
 \Ova\Oza\Ora\Oka\Owd\-%                              3
 \Oxa\Osva\Oa\Oya\Otha\Oi\Oya\Owd\-%                  4
 \Oxa\Osva\Oa\Oya\Otha\Oi\Oya\Oa\Ona\Oa\Oma\Owd\-%    5
 \Oda\Oa\Ora\Oya\Ova\Oha\Ou\Osva\Owd\-%               6
 \Oxa\Osva\Oa\Oya\Otha\Oi\Oya\Oha\Oya\Oa\Owd\-%       7
 \Opa\Ou\Occa\Owd\-%                                  8
 \Oha\Oxa\Oa\Oma\Ona\Oi\Osva\Oi\Oya\Owd}
 

    Xerxes had this inscription carved above the doorways of his palace
at Persepolis:
\begin{center}
\textcopsn{\xerxes}
\end{center}
which transliterates as: 
\begin{center}
\translitcopsn{\xerxes} 
\end{center}
and which, when translated, means: 
\begin{center}
Xerxes, the great king, the king of kings, the son of Darius the king,
an Achaemenian.
\end{center}


\end{document}

