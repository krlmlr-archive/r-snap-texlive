\documentclass[a4paper]{article}

\title{``Old Latin''\\
--- Computer Modern like font with ``long s'' ---}
\author{Taro \textsc{Terashita}
\thanks{Ehime University, Japan.
Email: \texttt{tarotera @ agr.ehime-u.ac.jp}
}
}
\date{Version 1.00 (18. Apr. 2010)}
\begin{document}
\maketitle
\tableofcontents

\section{Introduction}

As hobby and research,
I copy from old German text on real paper into digital \LaTeX\ file.
There are already fine fonts for the purpose:
\texttt{yfrak}, \texttt{ygoth} and \texttt{yswab}.
These fonts have not only fine graphic but also academic correctness.
You learn how to typeset in ancient days if you master them.
There is a package \texttt{oldgerm} for such old German fonts.
I used be with it when the idea of ``Old Latin Font'' occurred,
while I use the package \texttt{yfonts} with \texttt{german} today.

Even in such happy time with these old German fonts,
I find also some words with Latin font in original old book.
In such case, normal Latin font is available, of course.
Computer Modern is default of \TeX\ and suitable enough,
but there is a problem with modern Latin fonts, i.~e.\ ``long s''
(in German, ``langes s''):
It looks like ``f'' but lacks right half of side bar.

There is also Latin font with ``long s''
in \LaTeX\ Font Catalogue, like ``Day Roman S''.
But I love Computer Modern
because of its shape and its philosophy.
That is more than a design, that is a system.
And it looked not so difficult to modify long s from ``f'';
just bite off a little bit short side bar, wouldn't it?
Additionally,
I found some ligatures with ``t'' in old text,
for example ``(long)st'' in ``Forst'' or ``ct'' in ``Inspector'',
and tried to design them, too.

Actually, it was not as easy as I've thought.
The problems occur not direct by designing new fonts but
mainly by setting new rules of ligature.
However, I did somehow.

You see not only practical ``long s'' in Roman as result,
but also some fun fonts
in Bold, Dunhil, Slanted, Sans Serif, Typewriter Text,
which are generated through one common body, \texttt{roman.mf}.
(You see then why the font in Italic
was not redesigned.
The font is generated through \texttt{textit.mf}.)

I wish you enjoying this variation named ``Old Latin''.
Your corrections and comments are always welcome.
Especially, I let all combinations of ligatures allowed
\uppercase{without} any knowledge in Germanistik or in history of Typesetting.
So, some ligatures may be not allowed or not possible in real text.
I am happy to hear about such information.

\section{Lists of ``Old Latin''}

\subsection{Parameter files}

\texttt{%
olb10.mf
olbx5.mf olbx6.mf olbx7.mf olbx8.mf olbx9.mf olbx10.mf olbx12.mf
olbxsl10.mf
oldunh10.mf
olff10.mf
olfib8.mf
olr5.mf olr6.mf olr7.mf olr8.mf olr9.mf olr10.mf olr10s.mf olr12.mf olr17.mf
olsl8.mf olsl9.mf olsl10.mf olsl12.mf
olsltt10.mf
olss8.mf olss9.mf olss10.mf olss12.mf olss17.mf
olssbx10.mf
olssdc10.mf
olssi8.mf olssi9.mf olssi10.mf olssi12.mf olssi17.mf
olssq8.mf olssqi8.mf
oltt8.mf oltt9.mf oltt10.mf oltt12.mf
olvtt10.mf
}

\subsection{Common body files}

\texttt{%
oroman.mf
oromanl.mf
oromlig.mf
oromligs.mf
}

\subsection{\LaTeX\ files}
Your \LaTeX\ and dvi-viewer generates \texttt{tfm} and \texttt{pk} files
with these files:\\
\texttt{%
test\_alphabet.tex
test\_ol\_all.tex
test\_ol\_bf.tex
test\_ol\_rm.tex
test\_ol\_sl.tex
test\_ol\_ss.tex
test\_ol\_tt.tex
}

And also sample files in same names with suffix \texttt{.pdf}

\subsection{Required files of Computer Modern}

These are not inclusive in this contribution.
There must be somewhere in your \TeX\ system:\\
\texttt{%
romanu.mf
greeku.mf
romand.mf
romanp.mf
romspl.mf
romspu.mf
punct.mf
accent.mf
comlig.mf
romsub.mf
}

\section{My Environment}

You have all files above and it depends on you how cook them.
As a example, I will show you how I did.
I worked with following softwares:
\begin{itemize}
\item Microsoft Windows XP, Version 5.1.2600
\item pdfTeX, Version 3.1415926-1.40.10 (Web2C 2009)
\item dviout for Windows, Version 3.18.1
\item dvipdfmx, Version 20090919
\end{itemize}
I hope you understand or infer what the following explanation means,
even if you use Mac OS or UNIX.

\section{How to install ``Old Latin''}

\begin{itemize}
\item
Check if all above listed files
(\texttt{ol*.mf, orom*.mf} and \texttt{test\_*.tex})
are in your current work directory.
\item
Tip the command \texttt{latex test\_ol\_all},
then \texttt{latex} stops because there is no \texttt{tfm} files.
Just tip \texttt{r} for run further,
then \texttt{latex} generates \texttt{tfm} files automatically.
Try \texttt{latex} again after that,
then it generates \texttt{dvi} file smoothly
with fresh generated \texttt{tfm} files.

\item
Now you have two ways to display:
 \begin{itemize}
 \item
 \texttt{dviout} for windows
 $\rightarrow$ open file
 $\rightarrow$ sellect \texttt{test\_ol\_all.dvi}.
 At first, \texttt{dviout} will stop because there is no \texttt{pk} files.
 Just click the icon ``Retry'' (it can be several times),
 then \texttt{dviout} generates \texttt{pk} files automatically.
 After that you have to close \texttt{dviout} once and
 copy the fresh generated \texttt{pk} files
 from current work directory
 into the directory which your \texttt{dviout} can refer.
 Then let \texttt{dviout} with \texttt{test\_ol\_all} again and
 it shows you the result.
 \item
 Tip command \texttt{dvipdfmx test\_ol\_all}
 and it generates \texttt{pk} files automatically.
 See the result \texttt{pdf} file with your viewer,
 for example Adobe Reader.
 \end{itemize}
\item
If everything goes well,
try other \texttt{test\_ol\_*.tex} files
to generate all rest \texttt{tfm} and \texttt{pk} files.
Then move \texttt{mf}, \texttt{tfm} and \texttt{pk} files
into each correct directories.
For example, I created new directories
(A hint to decide the place:
Near by the directory named \texttt{gothic}):\\
\texttt{/texmf/fonts/source/public/oldlatin} for \texttt{mf} files,\\
\texttt{/texmf/fonts/tfm/oldlatin} for \texttt{tfm} files and\\
\texttt{/texmf/fonts/pk/cx/public/oldlatin} for \texttt{pk} files.

\end{itemize}

\section{How to use ``Old Latin''}

See the inside of \texttt{test\_ol\_*.tex} files.
That is shortest way to use the fonts.
You declare:\\
\texttt{{$\backslash$}font{$\backslash$}olr=olr10 scaled 1000}\\
and write:\\
\texttt{{$\backslash$}olr Forstwissenschaft}\\
then the word ``Forstwissenschaft'' will be written with ``Old Latin'' font.

You cannot change the size or shape with the commands
like \texttt{{$\backslash$}large} or \texttt{{$\backslash$}textsl}.
You have to declare every fonts for each size and shape.
The rough comparison is:
5 point is for \texttt{$\backslash$tiny},
7 point for \texttt{$\backslash$scriptsize},
8 point for \texttt{$\backslash$footnotesize},
9 point for \texttt{$\backslash$small},
10 point for \texttt{$\backslash$normalsize},
12 point for \texttt{$\backslash$large} and
17 point for \texttt{$\backslash$LARGE}.
And \texttt{olr} is for Roman,
\texttt{olbf} for Boldface,
\texttt{olsl} for Slanted (differs from ``italic''),
\texttt{olss} for Sans Serif,
\texttt{oltt} for Typewriting Text.

If you know NFSS2 well, then you can solve better
(regrettably, I couldn't).
See the \texttt{fntguide.pdf} (or \texttt{.tex}) in your \TeX\ system,
or the book ``The \LaTeX\ Companion''.

For ``Sperrsatz'' (this is a German word),
which has larger space between letters and
was used in order to emphasize,
you can make new font
with changing parameter \texttt{letter\_fit\#}
in your favorite \texttt{ol*.mf}.
I add \texttt{olr10s.mf} as an example.
The result shows you an easygoing atmosphere
especially by ligatured letters,
but I do not recommend you this way.
Use package \texttt{soul.sty}, that is much better.

If you want to call a letter direct with code number,
then get the code number at first.
Each program for letter in \texttt{mf} file
begins with ``\texttt{cmchar}'' and short explanation.
In the next line you will find ``\texttt{beginchar(oct"213"\ldots}''
for example.
Here the number ``213'' is what you want.
Write \verb|\symbol{'213}| in \texttt{tex} file,
and \LaTeX\ generates the letter.

\end{document}
