% bsample.tex    Samples of book-hand fonts
\documentclass{article}
\usepackage{sqrcaps}
\usepackage{rustic}
\usepackage{uncial}
\usepackage{auncial}
\usepackage{huncial}
\usepackage{inslrmaj}
\usepackage{inslrmin}
\usepackage{carolmin}
\usepackage{egothic}
\usepackage{tgothic}
\usepackage{pgothic}
\usepackage{rotunda}
\usepackage{humanist}

%\newcommand{\ABC}{ABCDEFGHIJKL MNOPQRSTUVWXYZ}
\newcommand{\abc}{a b c d e f gh i j k l m n o p q r s t u v w x y z}
\newcommand{\punct}{. , ; : ! ? ` ' \& ( ) [ ]}
\newcommand{\figs}{0123456789}
\newcommand{\dashes}{- -- ---}
\newcommand{\ligs}{``the brown \& white dog --- but fast acting hare''}
\newcommand{\sentence}{%
    now is the time for all good
men, and women, to come to the aid of the party while the quick brown fox
jumps over the lazy dog.}
\newcommand{\latin}{%
    Te canit adcelebratque polus rex gazifer hymnis.
Trans zephyrique globum scandunt tua facta per axem.
Explicit secunda pars summe fratris thome de aquino
ordinis fratrum predicatorium, longissima, prolixissima,
\& tediossima scribent; Deo gratias, Deo gratias, et iterumm
Deo gratias.
}

\renewcommand{\sentence}{%
Here is is an example of the font. Now is the time for all good
men, and women, to come to the aid of the party. Boys and girls come
out to play while the quick brown fox jumps over the lazy dog.}

\hyphenation{glo-b-um tra-ns ad-cel-e-brat-que zeph-yr-ique}

\title{Bookhands Font Sampler}
\author{Peter Wilson\\ \texttt{peter.r.wilson@boeing.com}}
\date{December 2002}
\begin{document}
\maketitle
\tableofcontents
\clearpage

\section{Introduction}

    This document provides some samples of the book-hand fonts. They are
available from CTAN in the \texttt{fonts/bookhands} directory. The 
characters in each font are displayed first, followed by an example
of some text including ligatures. Then some more text is
shown in the font. The first part of the text are sentences containing some
well known English phrases. The second part consists of Latin
sentences. The first two of these are Latin
abecedarian sentences dating from about 790~\textsc{ad}. The last
sentence also dates from around that time.

    As an example, the Computer Modern Roman font would be sampled like this:
\begin{center}
\abc \\
\punct{} \dashes \\
\figs 

\ligs\par
\end{center}

\begin{quotation}
\sentence{} \latin \par
\end{quotation}

    It should be noted that not all the fonts have the complete range of
punctuation marks and ligatures as shown in the CMR example. Punctuation
was virtually non-existant in the first century while by Gutenberg's time
all of our modern punctuation marks were in use.

\clearpage

\section{Square Capitals}

    Square Capitals were used from the first century onwards, principally
in the form of capital letters with other hands. 

    The font and packages are in the \texttt{sqrcaps} subdirectory.

\begin{center}
\sqrcfamily
\abc \\
\punct{} \dashes \\
\figs 

\ligs\par
\end{center}

\begin{quotation}
\renewcommand{\baselinestretch}{1.7}
\sqrcfamily 
\sentence{} \latin \par
\end{quotation}


\clearpage
\section{Roman Rustic}

    The Roman Rustic font was used between the first and sixth centuries.
Capital letters, if used at all, were just larger versions of the normal letters.

    The font and packages are in the \texttt{rustic} subdirectory.

\begin{center}
\rustfamily
\abc \\
\punct{} \dashes \\
\figs 

\ligs\par
\end{center}

\begin{quotation}
\renewcommand{\baselinestretch}{1.9}
\rustfamily 
\sentence{} \latin \par
\end{quotation}


\clearpage
\section{Uncial}

    The Uncial font was used between the third and sixth centuries. 
Capital letters, if used at all, were either just larger versions of the 
normal letters, or Roman Rustic letters.

    The font and packages are in the \texttt{uncial} subdirectory.

\begin{center}
\unclfamily
\abc \\
\punct{} \dashes \\
\figs 

\ligs\par
\end{center}

\begin{quotation}
\renewcommand{\baselinestretch}{1.37}
\unclfamily
\sentence{} \latin \par
\end{quotation}

\clearpage
\section{Half Uncial}

    The Half Uncial font was used between the third and ninth centuries. 
Capital letters were only used at the
start of a paragraph and were always partially, or completely, set in the margin.
They were either larger versions of the normal letters, or letters from earlier
fonts.

    The font and packages are in the \texttt{huncial} subdirectory.

\begin{center}
\hunclfamily
\abc \\
\punct{} \dashes \\
\figs 

\ligs\par
\end{center}

\begin{quotation}
\renewcommand{\baselinestretch}{1.4}
\hunclfamily
\sentence{} \latin \par
\end{quotation}

\clearpage
\section{Artificial Uncial}

    The Artificial Uncial font was used between the sixth and tenth centuries. 
Capital letters were only used at the
start of a paragraph and were always partially, or completely, set in the margin.
They were either larger versions of the normal letters, or letters from earlier
fonts.

    The font and packages are in the \texttt{auncial} subdirectory.

\begin{center}
\aunclfamily
\abc \\
\punct{} \dashes \\
\figs 

\ligs\par
\end{center}

\begin{quotation}
\renewcommand{\baselinestretch}{1.4}
\aunclfamily
\sentence{} \latin \par
\end{quotation}


\clearpage
\section{Insular Majuscule}

    The Insular Majuscule font was used between the sixth and ninth centuries,
firstly in Ireland and then later in England.
Capital letters were only used at the start of sentences. 
They were either larger versions of the normal letters, or highly decorated
very large letters.


    The font and packages are in the \texttt{inslrmaj} subdirectory.

\begin{center}
\imajfamily
\abc \\
\punct{} \dashes \\
\figs 

\ligs\par
\end{center}

\begin{quotation}
\renewcommand{\baselinestretch}{1.5}
\imajfamily
\sentence{} \latin \par
\end{quotation}


\clearpage
\section{Insular Minuscule}

    The Insular Minuscule hand was an informal version of Insular Majuscule and has been
used from the sixth century onwards. 
Capital letters were only used at the start of sentences. 
They were either larger versions of the normal letters, or highly decorated
very large letters.


    The font and packages are in the \texttt{inslrmin} subdirectory.

\begin{center}
\iminfamily
\abc \\
\punct{} \dashes \\
\figs 

\ligs\par
\end{center}

\begin{quotation}
\renewcommand{\baselinestretch}{1.5}
\iminfamily
\sentence{} \latin \par
\end{quotation}


\clearpage
\section{Carolingian Minuscule}

    The Carolingian Minuscule script was used between the 8th and 12th centuries,
throughout the Western Christian world.
Capital letters were only used at the start of sentences. 
They were either larger versions of the normal letters, or taken from
earlier hands.

    The font and packages are in the \texttt{carolmin} subdirectory.

\begin{center}
\cminfamily
\abc \\
\punct{} \dashes \\
\figs 

\ligs\par
\end{center}

\begin{quotation}
\renewcommand{\baselinestretch}{1.1}
\cminfamily
\sentence{} \latin \par
\end{quotation}


\clearpage
\section{Early Gothic}

    The Early Gothic hand was used between the 11th and 12th centuries, as a short lived
intermediary between the Carolingian script and the full-blown Gothic hands.


    The font and packages are in the \texttt{egothic} subdirectory.

\begin{center}
\egothfamily
\abc \\
\punct{} \dashes \\
\figs 

\ligs\par
\end{center}

\begin{quotation}
\renewcommand{\baselinestretch}{1.3}
\egothfamily
\sentence{} \latin \par
\end{quotation}


\clearpage
\section{Gothic Textura Quadrata}

    Gothic Textura Quadrata was the principal Gothic hand on the continent between about
the 13th and 15th centuries. Guthenberg based his types on this hand.

    The font and packages are in the \texttt{tgothic} subdirectory.

\begin{center}
\tgothfamily
\abc \\
\punct{} \dashes \\
\figs 

\ligs\par
\end{center}

\begin{quotation}
\renewcommand{\baselinestretch}{1.3}
\tgothfamily
\sentence{} \latin \par
\end{quotation}


\clearpage
\section{Gothic Textura Prescius}

    Gothic Textura Prescius, which is slightly simpler than Quadrata, was  used in
England from the 13th century. Caxton based his types on this hand.


    The font and packages are in the \texttt{pgothic} subdirectory.

\begin{center}
\pgothfamily
\abc \\
\punct{} \dashes \\
\figs 

\ligs\par
\end{center}

\begin{quotation}
\renewcommand{\baselinestretch}{1.1}
\pgothfamily
\sentence{} \latin \par
\end{quotation}


\clearpage
\section{Rotunda}

    The Italians did not go for the excesses of the Gothic Textura scripts and
they transformed Early Gothic into the Rotunda hand, in use between the 13th and
15th centuries.

    The font and packages are in the \texttt{rotunda} subdirectory.

\begin{center}
\rtndfamily
\abc \\
\punct{} \dashes \\
\figs 

\ligs\par
\end{center}

\begin{quotation}
\renewcommand{\baselinestretch}{1.1}
\rtndfamily
\sentence{} \latin \par
\end{quotation}


\clearpage
\section{Humanist Minuscule}

    Our modern roman types are based on the Humanist Minuscule script that
was used from the 14th century onward in Italy. It was the successor to Rotunda.


    The font and packages are in the \texttt{humanist} subdirectory.

\begin{center}
\hminfamily
\abc \\
\punct{} \dashes \\
\figs 

\ligs\par
\end{center}

\begin{quotation}
\renewcommand{\baselinestretch}{1.0}
\hminfamily
\sentence{} \latin \par
\end{quotation}



\end{document}


