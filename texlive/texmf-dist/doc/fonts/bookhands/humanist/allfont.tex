% allfont.tex    Test Humanist fonts
% Author: Peter Wilson (CUA) now at peter.r.wilson@boeing.com
%                            (or pandgwilson@earthlink.net) 
% Copyright 2002, 2003 Peter R. Wilson
%
% This work may be distributed and/or modified under the
% conditions of the LaTeX Project Public License, either
% version 1.3 of this license or (at your option) any 
% later version.
% The latest version of the license is in
%    http://www.latex-project.org/lppl.txt
% and version 1.3 or later is part of all distributions of
% LaTeX version 2003/06/01 or later.
%
% This work has the LPPL maintenance status "author-maintained".
%
% This work consists of the files listed in the README file.
%

\documentclass{article}
%\documentclass[12pt]{article}
\usepackage{allhmin}

\newcommand{\romannum}[1]{\romannumeral #1}
\newcommand{\Romannum}[1]{\uppercase\expandafter{\romannumeral #1}}
\newcommand{\ABC}{ABCDEFGHIJKL MNOPQRSTUVWXYZ}
\newcommand{\abc}{abcdefghijkl mnopqrstuvwxyz}
\newcommand{\punct}{.,;:!?`' \&{} () []}
\newcommand{\dashes}{- -- ---}
\newcommand{\figs}{0123456789}
\newcommand{\sentence}{%
this is an example of the humanist font. now is the time for all good
men, and women, to come to the aid of the party while the quick brown fox
jumps over the lazy dog.}

\newcommand{\Sentence}{%
This is an example of the Humanist font. Now is the time for all good
men, and women, to come to the aid of the party while the quick brown fox
jumps over the lazy dog.}

\newcommand{\latin}{Te canit adcelebratque polus rex gazifer hymnis.
  Trans zephyrique globum scandunt tua facta per axem.
  Explicit secunda pars summe fratris thome de aquino ordinis fratrum 
  predicatorium, longissima, prolixissima, \& tediosissima scribent;
  Deo gratias, Deo gratias, et iterumm Deo gratias. }

\title{Try Humanist Fonts}
\author{}
\date{}
%%\pagenumbering{roman}
\begin{document}
\maketitle

\tableofcontents

\section{The character set}

    This provides a short test of the characters in the Humanist fonts
--- the \verb|hmin| font family.



\begin{center}
The Humanist Huge normal font. \\ \par
{\Huge \ABC\\ \abc\\ \punct{}\dashes\\ \figs\\ \par }
\end{center}


\begin{center}
The Humanist font in its normal size \\
{\ABC{} \abc{} \figs} \\
\end{center}

\begin{center}
The bold normal font, the normal font, and the bold Computer Modern
Roman, all in the normal size \\
{\textbf{\abc{} \figs}} \\
{\abc{} \figs} \\
\textcmr{\textbf{\abc{} \figs}} \\
\end{center}

\begin{center}
The bold versions, in Huge and tiny sizes. \par
\bfseries
\Huge \abc{} \figs \par
\tiny \abc{} \figs \par
\end{center}

\begin{center}
The font in the tiny size \\ \par
{\tiny \ABC{} \\ \abc\\ \figs\\ \par } 
\end{center}

\begin{center}
    Some ligatures in the normal font \\
{``the lazy dog --- but quick fox?'' \ae{} \AE{} ct st }
\end{center}

\section{Example texts}

    First some well known English phrases in an abcedarian sentence.

\Sentence{}

    After this there are two Latin abecedarian sentences dating from about 
\Romannum{790}, and another sentence from roughly the same period.

\latin{}

    
\textcmss{This is the end of the test file, with this sentence being typeset
using the Computer Modern Sans font in the point size as specified for this
document.}

\end{document}