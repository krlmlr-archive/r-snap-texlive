\documentclass[11pt]{article}
\usepackage[margin=1.25in]{geometry}
\usepackage[parfill]{parskip}    % Activate to begin paragraphs with an empty line rather than an indent
\usepackage{graphicx}
\DeclareFontFamily{OT1}{pzc}{}
\DeclareFontShape{OT1}{pzc}{m}{it}%
              {<-> s * [1.15] pzcmi7t}{}
\DeclareMathAlphabet{\mathpzc}{OT1}{pzc}{m}{it}
\usepackage[scaled=1.15]{urwchancal}
\title{A virtual font for URW Chancery Math}
\author{Michael Sharpe\\msharpe at ucsd dot edu}
%\date{}                                           % Activate to display a given date or no date

\begin{document}
\maketitle
The URW clone of the universally available PostScript font Zapf Chancery may be used as a math calligraphic font by means of the following lines in the preamble:
\begin{verbatim}
\DeclareFontFamily{OT1}{pzc}{}
\DeclareFontShape{OT1}{pzc}{m}{it}%
              {<-> s * [1.15] pzcmi7t}{}
\DeclareMathAlphabet{\mathpzc}{OT1}{pzc}{m}{it}
\end{verbatim}

The results are not always good if you attach math accents or sub-/super-scripts as the metrics are designed for a text font rather than a math font.

This small package provides a virtual math font linked to \verb|uzcmi8r| with side-bearings, subscript position and accent position adjusted manually for each glyph. 
Following 
\begin{verbatim}
\usepackage[scaled=1.15]{urwchancal}
\end{verbatim}
 which redefines \verb|\mathcal| to output URW Chancery glyphs, the input line 
\begin{verbatim}
\[A=\bar{\mathcal{M}}^2_k+\hat{\mathcal{Z}}.\]
\end{verbatim}
yields
\[A=\bar{\mathcal{M}}^2_k+\hat{\mathcal{Z}}.\]
In addition to \verb|scaled|, there is on option \verb|mathscr|, which enables \verb|\mathscr| rather than \verb|\mathcal| to point to URW Chancery.

The long tails on some of the glyphs make this font problematic as a math font, but it has some virtues, not the least being its universal availability. For a font of similar appearance without the long-tail problems, consider the script font in Mathematica5.
\end{document}  