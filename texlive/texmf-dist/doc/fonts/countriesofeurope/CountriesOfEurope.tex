
%---------------------------
\listfiles

%\pdfmapfile{+CountriesOfEurope.map} % only needed, if the map is not enabled with updmap

\documentclass{article}

\usepackage[utf8]{inputenc}
\usepackage[T1]{fontenc}
\usepackage{lmodern,array,longtable,graphicx,ifthen,ragged2e}
\usepackage[scaled=7.5]{CountriesOfEurope}

\makeatletter
\newcommand\Country[1]{%
  \tabular{|>{\Centering}p{2.5cm}|}\hline
    \strut\footnotesize\texttt{\textbackslash#1}\\\hline
    \parbox[c][3.2cm]{2cm}{\CountriesOfEuropeFamily\@nameuse{#1}}\\\hline
  \endtabular}

\makeatother

\newcounter{N}

\begin{document}
\author{Rolf Niepraschk \and Herbert Voß}
\title{Package \texttt{CountriesOfEurope}}
\maketitle


\begin{sloppypar}
This package defines the two macros \verb|\CountriesOfEuropeFamily| and \verb|\EUCountry{no}|,
which allow to print one of the european countries as a single character with the given scaling. 
The first one switches
to the font encoding \verb|U| and loads the font and the second one does the same but also with
printing the character which the given number (128--166), eg for Finland: \verb|\EUCountry{139}|$\rightarrow$%
\scalebox{0.1}{\EUCountry{139}}.
\end{sloppypar}

The font can be loaded
with an optional argument for the scaling factor, which is preset to 1:

\begin{verbatim}
\usepackage[scaled]{CountriesOfEurope}%   scaled to 10
\usepackage[scaled=15]{CountriesOfEurope}
\end{verbatim}

The countris itself are available by a macro from the following list. The characters are
at the position 128--166 in the Type~1 font \verb|CountriesOfEurope.pfb| and also available
with the \verb|\char| primitive. The following table shows the countries in their original size
with a scaling of 7.5:

\begin{longtable}{cccc}
\Country{Albania} & 
\Country{Andorra} & 
\Country{Austria} & 
\Country{Belarus} \\ 
\Country{Belgium} &
\Country{Bosnia} & 
\Country{Bulgaria} & 
\Country{Croatia} \\ 
\Country{Czechia} & 
\Country{Denmark} &
\Country{Estonia} & 
\Country{Finland} \\ 
\Country{France} & 
\Country{Germany} & 
\Country{GreatBritain} &
\Country{Greece} \\
\Country{Hungary} & 
\Country{Iceland} & 
\Country{Ireland} & 
\Country{Italy} \\
\Country{Latvia} & 
\Country{Liechtenstein} & 
\Country{Lithuania} & 
\Country{Luxembourg} \\ 
\Country{Macedonia} &
\Country{Malta} & 
\Country{Moldova} & 
\Country{Montenegro} \\ 
\Country{Netherlands} & 
\Country{Norway} &
\Country{Poland} & 
\Country{Portugal} \\ 
\Country{Romania} & 
\Country{Serbia} & 
\Country{Slovakia} &
\Country{Slovenia} \\ 
\Country{Spain} & 
\Country{Sweden} & 
\Country{Switzerland}

\end{longtable}


\DeclareFontShape{U}{CountriesOfEurope}{m}{n}{ <-> s*[1] CountriesOfEurope}{}
\noindent\rule{\textwidth}{.5mm}

The countries in the original size with the given bounding box and the text command \verb|huge|:

\medskip

\begingroup
\fboxsep=0pt
\setcounter{N}{128}
\huge\noindent%
\CountriesOfEuropeFamily%
\whiledo{\value{N} < 167}{%
  \fbox{\symbol{\value{N}}}%
  \stepcounter{N}}
\endgroup

\end{document}
