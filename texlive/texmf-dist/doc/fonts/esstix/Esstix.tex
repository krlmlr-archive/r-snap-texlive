\documentclass[11pt]{amsart}
\usepackage[margin=1in]{geometry} 
\usepackage[parfill]{parskip}
\pdfmapfile{+esstixOther}
\usepackage{multido}
\title{ESSTIX FONTS}
\author{}
%\date{}                                           % Activate to display a given date or no date

\begin{document}
\maketitle
The ESSTIX fonts were an original production of Elsevier Science BV, Amsterdam, The Netherlands, Copyright \copyright\ 2000. The original documentation stated: ``These fonts may be used and distributed freely by and to everyone. Changes are not permitted without consent of Elsevier Science BV. There is no warranty of any kind.'' The font collection remained an unfinished work, and was donated to the STIX Consortium as a precursor to the STIX font project. The collection was never officially distributed by either Elsevier or STIX, but with the change of ownership, its license terms changed to the SIL Open Font License, version 1.1.

This distribution contains the PostScript versions (in {\tt.pfb} format) of the original ESSTIX fonts in Adobe Standard Encoding, together with the associated {\tt.afm} files. The fonts have not been modified in any way.

According to the original documentation:\\
``The characters in these fonts are those as contained in the Documentation of the Elsevier Science Article DTD, Version 3.0.0, Grid Table B: Symbols. The signs $\cdot$ $ -$ $+$ $ \cdots$  are situated in the middle of a height of 1000 PostScript units, i.e., approximately in the center of the lower case characters without ascenders and descenders. Put next to a horizontal fraction line they will form a uniform line therewith. All other signs, as far as their forms permit, are symmetrical to this center line, e.g., $\cdot$ $=$ $ -$ $<$ $+$ $\times$ $\infty$. The signs are mainly on a width of 1000 PostScript units, with a right- and left-hand bearing of 170 units. The horizontal arrows in font ESSTIX-One are on a width of 1200 units.''

In each font, the useful glyphs are in the ranges 33--126 and 161--184.

The descriptions below are copied with slight modifications from the original descriptions, though parts of the original that do not pertain to \TeX\ have been omitted.

Font  \emph{ESSTIX-One} contains the following characters (shown at 12pt):\\[3pt]
\font\ess=ESSTIX1_ at 12pt
{\ess\multido{\i=33+1}{36}{\char\i}}\newline
{\ess\multido{\i=69+1}{36}{\char\i}}\newline
{\ess\multido{\i=105+1}{22}{\char\i}}{\ess\multido{\i=161+1}{24}{\char\i}}

Font  \emph{ESSTIX-Two} contains:\\[3pt]\font\ess=ESSTIX2_ at 12pt
{\ess\multido{\i=33+1}{36}{\char\i}}\newline
{\ess\multido{\i=69+1}{36}{\char\i}}\newline
{\ess\multido{\i=105+1}{22}{\char\i}}{\ess\multido{\i=161+1}{24}{\char\i}}

Font  \emph{ESSTIX-Three} contains:\\[3pt]
\font\ess=ESSTIX3_ at 12pt
{\ess\multido{\i=33+1}{36}{\char\i}}\newline
{\ess\multido{\i=69+1}{36}{\char\i}}\newline
{\ess\multido{\i=105+1}{22}{\char\i}}
%{\ess\multido{\i=161+1}{24}{\char\i}}

Font  \emph{ESSTIX-Four} contains:\\[3pt]
\font\ess=ESSTIX4_ at 12pt
{\ess\multido{\i=33+1}{36}{\char\i}}\newline
{\ess\multido{\i=69+1}{36}{\char\i}}\newline
{\ess\multido{\i=105+1}{22}{\char\i}}{\ess\multido{\i=161+1}{24}{\char\i}}

\newpage
Font  \emph{ESSTIX-Five} contains:\\[3pt]
\font\ess=ESSTIX5_ at 12pt
{\ess\multido{\i=33+1}{36}{\char\i}}\newline
{\ess\multido{\i=69+1}{36}{\char\i}}\newline
{\ess\multido{\i=105+1}{22}{\char\i}}\newline
{\ess\multido{\i=161+1}{24}{\char\i}}\newline
Note that the dashes in decimal positions 75 and 76 are a minus and a bold minus sign respectively. The glyphs in positions 81 and 82 are EN-dash and EM-dash. The characters in positions 93--116 are chemical bonds. Insofar as these bonds resemble operation signs, the difference is that the bonds have smaller right- and left-hand bearings, e.g., a{\ess\char75}b (minus sign), H{\ess\char101}O (single bond).

Font  \emph{ESSTIX-Six} contains  a number of `oversized' characters:\\[3pt]\font\ess=ESSTIX6_ at 12pt
{\ess\multido{\i=33+1}{36}{\char\i}}\newline
{\ess\multido{\i=69+1}{36}{\char\i}}
%{\ess\multido{\i=105+1}{22}{\char\i}}
%{\ess\multido{\i=161+1}{24}{\char\i}}

Font  \emph{ESSTIX-Seven} contains `oversized' fences:\\[3pt]\font\ess=ESSTIX7_ at 12pt
{\ess\multido{\i=33+1}{36}{\char\i}}\newline
{\ess\multido{\i=69+1}{36}{\char\i}}
%{\ess\multido{\i=105+1}{22}{\char\i}}\newline
%{\ess\multido{\i=161+1}{24}{\char\i}}

Font  \emph{ESSTIX-Eight} contains pieces for assembling horizontally or vertically:\\[3pt]\font\ess=ESSTIX8_ at 12pt
{\ess\multido{\i=33+1}{36}{\char\i}}\newline
{\ess\multido{\i=69+1}{36}{\char\i}}
%{\ess\multido{\i=105+1}{22}{\char\i}}\newline
%{\ess\multido{\i=161+1}{24}{\char\i}}

Font  \emph{ESSTIX-Nine} contains an inclined mathematical Greek:\\[3pt]
\font\ess=ESSTIX9_ at 12pt
{\ess\multido{\i=33+1}{36}{\char\i}}\newline
{\ess\multido{\i=69+1}{36}{\char\i}}
{\ess\multido{\i=105+1}{22}{\char\i}}
%{\ess\multido{\i=161+1}{24}{\char\i}}

Font  \emph{ESSTIX-Ten} contains an upright mathematical Greek:\\[3pt]\font\ess=ESSTIX10 at 12pt
{\ess\multido{\i=33+1}{36}{\char\i}}\newline
{\ess\multido{\i=69+1}{36}{\char\i}}
{\ess\multido{\i=105+1}{22}{\char\i}}
%{\ess\multido{\i=161+1}{24}{\char\i}}

Font  \emph{ESSTIX-Eleven} contains  a bold inclined mathematical Greek:\\[3pt]
\font\ess=ESSTIX11 at 12pt
{\ess\multido{\i=33+1}{36}{\char\i}}\newline
{\ess\multido{\i=69+1}{36}{\char\i}}
{\ess\multido{\i=105+1}{22}{\char\i}}
%{\ess\multido{\i=161+1}{24}{\char\i}}

Font  \emph{ESSTIX-Twelve} contains  a bold mathematical Greek:\\[3pt]
\font\ess=ESSTIX12 at 12pt
{\ess\multido{\i=33+1}{36}{\char\i}}\newline
{\ess\multido{\i=69+1}{36}{\char\i}}
{\ess\multido{\i=105+1}{22}{\char\i}}
%{\ess\multido{\i=161+1}{24}{\char\i}}

Font  \emph{ESSTIX-Thirteen} contains a capital and lower case script:\\[3pt]
\font\ess=ESSTIX13 at 12pt
{\ess\multido{\i=33+1}{60}{\char\i}}\newline
{\ess\multido{\i=97+1}{26}{\char\i}}
%{\ess\multido{\i=105+1}{22}{\char\i}}
%{\ess\multido{\i=161+1}{24}{\char\i}}

Font  \emph{ESSTIX-Fourteen} contains open caps and figures, four Hebrew characters, and an italic and bold italic `round-bottom vee':\\[3pt]\font\ess=ESSTIX14 at 12pt
{\ess\multido{\i=33+1}{36}{\char\i}}\newline
{\ess\multido{\i=69+1}{36}{\char\i}}\newline
{\ess\multido{\i=105+1}{22}{\char\i}}
%{\ess\multido{\i=161+1}{24}{\char\i}}

Font  \emph{ESSTIX-Fifteen} contains capital and lower case fraktur (German) characters:\\[3pt]\font\ess=ESSTIX15 at 12pt
{\ess\multido{\i=33+1}{60}{\char\i}}\newline
{\ess\multido{\i=97+1}{26}{\char\i}}
%{\ess\multido{\i=105+1}{22}{\char\i}}\newline
%{\ess\multido{\i=161+1}{24}{\char\i}}

Font  \emph{ESSTIX-Sixteen} contains phonetic characters:\\[3pt]\font\ess=ESSTIX16 at 12pt
{\ess\multido{\i=33+1}{36}{\char\i}}\newline
{\ess\multido{\i=69+1}{36}{\char\i}}\newline
{\ess\multido{\i=105+1}{22}{\char\i}}\newline
{\ess\multido{\i=161+1}{24}{\char\i}}

Font  \emph{ESSTIX-Seventeen}  also contains phonetic characters, as well as non-standard accents like the triple and quadruple dot accents and a bold dot accent (pos. 97--99). There are a number of signs to form accents over more than one character (pos. 84, 108, 109). These signs are suitable for `stretching'. Pieces to form accents over more than one character are in pos. 100--105.\\[3pt]\font\ess=ESSTIX17 at 12pt
{\ess\multido{\i=33+1}{36}{\char\i}}\newline
{\ess\multido{\i=69+1}{36}{\char\i}}\newline
{\ess\multido{\i=105+1}{22}{\char\i}}\newline
{\ess\multido{\i=161+1}{24}{\char\i}}




\end{document}  