% !TEX TS-program = pdflatexmk
\documentclass{article}
\usepackage[margin=1in]{geometry} 
\usepackage[parfill]{parskip}% Begin paragraphs with an empty line, no indent
\usepackage{enumitem}
\setlist[description]{style=sameline,font=\mdseries\scshape}
\setlength\unitlength{1pt}% for picture
\usepackage{booktabs}
\usepackage{graphicx}
\usepackage{xcolor}
\usepackage{upquote}
\usepackage{fancyvrb}
\def\yellow#1{\setlength{\fboxrule}{0pt}%
\setlength{\fboxsep}{0pt}%
\colorbox{yellow}{#1}}
%SetFonts
%fbb plus newtxmath
\usepackage[full]{textcomp} % to get the right copyright, etc.
\usepackage[sups]{fbb}
\usepackage[scaled=.95,type1]{cabin}
\usepackage[varqu,varl]{zi4}% typewriter
\usepackage[libertine,bigdelims]{newtxmath}
\usepackage[bb=boondox,frak=boondox]{mathalfa}
%\renewcommand*{\rmdefault}{zgmj}
\usepackage[T1]{fontenc}
\useosf
\usepackage{fonttable}
\title{The \textbf{fbb} package---a Bembo--like font}
\author{Michael Sharpe}
\date{}
\begin{document}
\maketitle
\section{The Package}
The {\tt fbb} package offers a family of Bembo--like fonts derived from Cardo in the usual four styles. Text figures may be selected from four types:
\begin{itemize}
\item
Proportional lining (LF), selected by option {\tt lining};
\item
Tabular lining (TLF), selected by options {\tt lining, tabular};
\item
Proportional oldstyle (OsF), selected by option {\tt oldstyle};
\item
Tabular oldstyle (TOsF), selected by options {\tt oldstyle, tabular}.
\end{itemize}
The package also defines five macros that allow you use alternate figure styles locally:
\begin{verbatim}
\textlf{97} % print 97 in proportional lining figures
\texttlf{97} % print 97 in tabular lining figures
\textosf{97} % print 97 in proportional oldstyle figures
\texttosf{97} % print 97 in tabular oldstyle figures
\textsu{97} % print 97 in superior figures
\end{verbatim}
Option {\tt sups} changes the form of footnote markers to use {\tt fbb}'s superior figures, unless you have redefined the meaning of \verb|\thefootnote| prior to loading {\tt fbb}. For more control over size, spacing and position of footnote markers, use the \textsf{superiors} package: Eg,
\begin{verbatim}
\usepackage[supstfm=fbb-Regular-sup-t1]{superiors}
\end{verbatim}


There is a {\tt scaled} option (\emph{e.g.}, {\tt scaled=.97}) that allow you to adjust the text size against, say, a math package. This text package works well with {\tt newtxmath} with the {\tt libertine} option, because the latter has italics of the same italic angle as {\tt fbb} and of very similar xheight and weight. The suggested invocation is:
\begin{verbatim}
\usepackage[full]{textcomp} % to get the right copyright, etc.
\usepackage[lining,tabular]{fbb} % so math uses tabular lining figures
\usepackage[scaled=.95,type1]{cabin} % sans serif in style of Gill Sans
\usepackage[varqu,varl]{zi4}% inconsolata typewriter
\usepackage[T1]{fontenc} % LY1 also works
\usepackage[libertine,bigdelims]{newtxmath}
\usepackage[cal=boondoxo,bb=boondox,frak=boondox]{mathalfa}
\useosf % change normal text to use proportional oldstyle figures
%\usetosf would provide tabular oldstyle figures in text
\end{verbatim}
Here is a short sample based on this preamble:\\[4pt]
\def\Pr{\ensuremath{\mathbb{P}}}
\def\rmd{\mathrm{d}}
The typeset math below follows the ISO recommendations that only variables
be set in italic. Note the use of upright shapes for $\rmd$, $\mathrm{e}$
and $\uppi$. (The first two are entered as \verb|\mathrm{d}| and
\verb|\mathrm{e}|, and in fonts derived from {\tt newtxmath} or {\tt mtpro2},
 the latter is entered as \verb|\uppi|.)

\textbf{Simplest form of the \textit{Central Limit Theorem}:} \textit{Let
$X_1$, $X_2,\cdots$ be a sequence of iid random variables with mean $0$ 
and variance $1$ on a probability space $(\Omega,\mathcal{F},\Pr)$. Then}
\[\Pr\left(\frac{X_1+\cdots+X_n}{\sqrt{n}}\le y\right)\to\mathfrak{N}(y)\coloneq
\int_{-\infty}^y \frac{\mathrm{e}^{-t^2/2}}{\sqrt{2\uppi}}\,
\mathrm{d}t\quad\mbox{as $n\to\infty$,}\]
\textit{or, equivalently, letting} $S_n\coloneq\sum_1^n X_k$,
\[\mathbb{E} f(S_n/\sqrt{n})\to \int_{-\infty}^\infty f(t)
\frac{\mathrm{e}^{-t^2/2}}{\sqrt{2\uppi}}\,\mathrm{d}t
\quad\mbox{as $n\to\infty$, for every $f\in\mathrm{b}
\mathcal{C}(\mathbb{R})$.}\]

\section{Text effects under \texttt{fontaxes}}
This package loads the {\tt fontaxes} package in order to access italic small caps. You should pay attention to the fact that {\tt fontaxes} modifies the behavior of some basic \LaTeX\ text macros such as \verb|\textsc| and \verb|\textup|. Under normal \LaTeX, some text effects are combined, so that, for example, \verb|\textbf{\textit{a}}| produces bold italic {\tt a}, while other effects are not, eg, \verb|\textsc{\textup{a}}| has the same effect as \verb|\textup{a}|, producing the letter {\tt a} in upright, not small cap, style. With {\tt fontaxes}, \verb|\textsc{\textup{a}}| produces instead upright small cap {\tt a}. It offers a macro \verb|\textulc| that undoes small caps, so that, eg, \verb|\textsc{\textulc{a}}| produces {\tt a} in non-small cap mode, with whatever other style choices were in force, such as bold or italics.

%\section{Superior figures}
%The TrueType versions of GaramondNo8 have a full set of superior figures, unlike their PostScript counterparts. The superior figure glyphs in regular weight only have been copied to \texttt{NewG8-sups.pfb} and \texttt{NewG8-sups.afm} and provided with a tfm named \texttt{NewG8-sups.tfm} that can be used by the \textsf{superiors} package to provide adjustable footnote markers. See \textsf{superiors-doc.pdf} (you can find it in \TeX Live by typing \texttt{texdoc superiors} in a Terminal window.) The simplest invocation is
%\begin{verbatim}
%\usepackage[supstfm=NewG8-sups]{superiors}
%\end{verbatim}
\section{Glyphs in TS\textlf{1} encoding}
The layout of the TS\textlf{1} encoded Text Companion font, which is fully rendered \emph{in regular style only}, is as follows. See below for the macros that invoke these glyphs. Though shown in regular weight, upright shape only, a reduced set of  glyphs are available in all other weights and shapes.

\fonttable{fbb-Regular-tosf-ts1}

\textsc{List of macros to access the TS\textlf{1} symbols in text mode:}\\
(Note that slots 0--12 and 26--29 are accents, used like \verb|\t{a}| for a tie accent over the letter a. Slots 23 and 31 do not contain visible glyphs, but have heights indicated by their names.)
\begin{verbatim}
  0 \capitalgrave
  1 \capitalacute
  2 \capitalcircumflex
  3 \capitaltilde
  4 \capitaldieresis
  5 \capitalhungarumlaut
  6 \capitalring
  7 \capitalcaron
  8 \capitalbreve
  9 \capitalmacron
 10 \capitaldotaccent
 11 \capitalcedilla
 12 \capitalogonek
 13 \textquotestraightbase
 18 \textquotestraightdblbase
 21 \texttwelveudash
 22 \textthreequartersemdash
 23 \textcapitalcompwordmark
 24 \textleftarrow
 25 \textrightarrow
 26 \t % tie accent, skewed right
 27 \capitaltie % skewed right
 28 \newtie % tie accent centered
 29 \capitalnewtie % ditto
 31 \textascendercompwordmark
 32 \textblank
 36 \textdollar
 39 \textquotesingle
 42 \textasteriskcentered
 45 \textdblhyphen
 47 \textfractionsolidus
 48 \textzerooldstyle
 49 \textoneoldstyle
 50 \texttwooldstyle
 49 \textthreeoldstyle
 50 \textfouroldstyle
 51 \textfiveoldstyle
 52 \textsixoldstyle
 53 \textsevenoldstyle
 54 \texteightoldstyle
 55 \textnineoldstyle
 60 \textlangle
 61 \textminus
 62 \textrangle
 77 \textmho
 79 \textbigcircle
 87 \textohm
 91 \textlbrackdbl
 93 \textrbrackdbl
 94 \textuparrow
 95 \textdownarrow
 96 \textasciigrave
 98 \textborn
 99 \textdivorced
100 \textdied
108 \textleaf
109 \textmarried
110 \textmusicalnote
126 \texttildelow
127 \textdblhyphenchar
128 \textasciibreve
129 \textasciicaron
130 \textacutedbl
131 \textgravedbl
132 \textdagger
133 \textdaggerdbl
134 \textbardbl
135 \textperthousand
136 \textbullet
137 \textcelsius
138 \textdollaroldstyle
139 \textcentoldstyle
140 \textflorin
141 \textcolonmonetary
142 \textwon
143 \textnaira
144 \textguarani
145 \textpeso
146 \textlira
147 \textrecipe
148 \textinterrobang
149 \textinterrobangdown
150 \textdong
151 \texttrademark
152 \textpertenthousand
153 \textpilcrow
154 \textbaht
155 \textnumero
156 \textdiscount
157 \textestimated
158 \textopenbullet
159 \textservicemark
160 \textlquill
161 \textrquill
162 \textcent
163 \textsterling
164 \textcurrency
165 \textyen
166 \textbrokenbar
167 \textsection
168 \textasciidieresis
169 \textcopyright
170 \textordfeminine
171 \textcopyleft
172 \textlnot
173 \textcircledP
174 \textregistered
175 \textasciimacron
176 \textdegree
177 \textpm
178 \texttwosuperior
179 \textthreesuperior
180 \textasciiacute
181 \textmu
182 \textparagraph
183 \textperiodcentered
184 \textreferencemark
185 \textonesuperior
186 \textordmasculine
187 \textsurd
188 \textonequarter
189 \textonehalf
190 \textthreequarters
191 \texteuro
214 \texttimes
246 \textdiv
\end{verbatim}
There is a macro \verb|\textcircled| that may be used to construct a circled version of a single letter using \verb|\textbigcircle|. The letter is always constructed from the small cap version, so, in effect, you can only construct circled uppercase letters: \verb|\textcircled{M}| and \verb|\textcircled{m}| have the same effect, namely \textcircled{M}.

\section{Historical Background}
Humanist scholar Pietro Bembo, a seminal figure in literature and music of the Italian Renaissance, who later became Cardinal Bembo, wrote an essay in the last decade of the 15th century about his travels to Mt.\ Aetna, which work was published by the Venetian printer Aldus Manutius (whose name gave us \emph{Aldine}) using a new Roman font designed by his punch-cutter, Francesco Griffo that improved on the earlier efforts of Jenson, another printer in Venice. That font seems to have played a similarly seminal r\^ole in  typography. It was the direct progenitor of the many Garamond fonts, and has seen numerous modern revivals whose names make use of every known historical connection to the figures named above, such as those of Bembo's  lover for a time, Lucrezia Borgia.

The metal form of the Bembo font developed by Stanley Morison for English Monotype in the 1920's was widely used in book printing due to its handsome appearance and readability. Commercial digital versions have not had much love from critics until recently.  Adobe's MinionPro and WarnockPro arguably deserve the prizes for the best modern revivals of oldstyle fonts not too distant from Bembo. (Both lack Bembo's tall ascenders and its characteristic overarching f.)

To my knowledge, there is currently only one free source for a Bembo--like font family, that being David Perry's \emph{Cardo} (a contraction of \emph{Cardinal Bembo}), which is not readily accessible to 
\LaTeX\ users and which lacks Bold Italic as well as a full range of Small Caps and figure styles.

This package is named for its Berry form {\tt fbb}, with {\tt f} denoting free (\emph{i.e.}, public) and {\tt bb} the Berry abbreviation for Bembo. It is derived from Cardo, with significant  modifications. Where Cardo is intended primarily  for scholars of ancient languages, those features are removed from {\tt fbb} and issues of more modern concern are added. The package contains the usual four styles (regular, italic, bold, bold italic), each with small caps and figures in tabular lining, proportional lining, tabular oldstyle and proportional oldstyle, as well as superior and inferior figures. The f-ligatures have been  revised/added  so as to function better with \LaTeX, and other glyphs have been changed as necessary to suit the demands of \emph{FontForge}. A kerning table was added to Regular upright weight---a serious omission in the original.  The Bold Italic weight was created algorithmically from Italic, but the result required much intervention by human hand. Small Caps were created for all styles other than regular, which was already present in Cardo. 

On screen and paper, {\tt fbb} appears close in weight to Libertine, though of larger xheight, a bit softer and slightly less plain.  The following two sentences are written in {\tt fbb} and Libertine respectively. The third example sentence is written using {\tt garamondx}, whose natural xheight is comparable to Libertine, but which should normally be scaled down to resemble more familiar Garamonds. Perhaps {\tt fbb} will be prove to be more suitable for older eyes.

\textit{\textsc{Comparison between fbb and Libertine}}: 

Both fbb and Libertine are highly readable fonts in their standard Roman forms, each has a wide range of figures and small caps, but Libertine has the advantage in the number of supported scripts and the variety of weights.

{\fontfamily{LinuxLibertineT-LF}\selectfont Both fbb and Libertine are highly readable fonts in their standard Roman forms, each has a wide range of figures and small caps, but Libertine has the advantage in the number of supported scripts and the variety of weights.}

\textit{\textsc{Same sentence in garamondx}}: 

{\fontfamily{zgmx}\selectfont Both fbb and Libertine are highly readable fonts in their standard Roman forms, each has a wide range of figures and small caps, but Libertine has the advantage in the number of supported scripts and the variety of weights.}

\end{document}