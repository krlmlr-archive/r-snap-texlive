\documentclass[a4paper,twocolumn]{article}
\usepackage{osf-txfonts,ctable,mkpic,moreverb}
\usepackage[a4paper,margin=25mm]{geometry}

\def\Mfpic{\textsl{mfpic}}
\def\Mkpic{\textsl{mkpic}}
\def\Perl{\textsl{Perl}}
\newcommand{\makefigure}[1]{
  \vspace{2ex}
  \begin{figure}[htb]
    \begin{center}
      \mbox{\quad\csname Fig#1\endcsname}
      \caption{#1}
      \label{fig:#1}
    \end{center}
  \end{figure}
}
\pagestyle{myheadings}
\markright{Mkpic: how Perl can help \TeX}

\begin{document}

\title{Mkpic: how Perl can help \TeX}

\author{Wybo Dekker\\wybo@dekkerdocumenten.nl}
\maketitle
\begin{abstract}
  Perl may be an easy interface to \TeX{} when it comes to repetitive
  tasks, like writing letters, creating reports from databases, and
  many more. This article shows how Perl can be used to generate many
  similar pictures \emph{via} the \Mfpic\ style
  \\[2ex]
  \noindent Keywords: \textit{perl, mfpic, mkpic}
\end{abstract}

\section{Introduction}
I recently had to produce about 40 pictures for insertion into 
a book on elementary mathematics.  I decided that the \Mfpic\ would
suite most of my needs. But writing \Mfpic\ commands is not easy.
Figure~\ref{fig:parabola}, for example, can be constructed using the
following \Mfpic\ commands:

\footnotesize
\begin{verbatim}
 1 \mftitle{parabola}
   \setlength{\mfpicunit}{1mm}
   \begin{mfpic}[16][5.45]{0}{4}{-6}{5}
   \axes
 5 \hatchwd{2}
   \tlabel[bc](0,5.54){$y$}
   \tlabel[cl](4.21,0){$x$}
   \tlabel[tc](2,-0.18){\strut 2}
   \tlabel[bc](3,0.18){\strut 3}
10 \tlabel[cr](-0.07,-5){\strut -5}
   \tlabel[cr](-0.07,0){\strut 0}
   \tlabel[cr](-0.07,4){\strut 4}
   \rhatch\lclosed\connect
   \lines{(0,0),(0,4)}
15 \function{0,3,.05}{4-x*x}
   \lines{(3,-5),(3,0)}
   \endconnect
   \function{0,3.2,.05}{4-x*x}
   \dotted\arrow\lines{(3,-5),(0,-5)}
20 \dotted\arrow\lines{(3,-5),(3,0)}
   \tlabel[bc](3,4){\parbox[b]{60mm}{%
     \center $f(x)=4-x^2$}}
   \arrow\lines{(3,3.46),(1.7,1.1)}
   \tlabel[bc](2,5){\parbox[b]{60mm}{%
25   \center Area $O_1$}}
   \arrow\lines{(2,4.46),(1,2)}
   \tlabel[bc](4,2){\parbox[b]{60mm}{%
     \center Area $O_2$}}
   \arrow\lines{(4,1.46),(2.8,-2)}
30 \end{mfpic}
\end{verbatim}
\normalsize

\makefigure{parabola}
%\ctable[figure,label=fig:parabola,caption=parabola]{@{}c@{}}{}{\Figparabola}

As you can see, this implies a lot of typing and one has to type many
nested [], \{\}, and () pairs. Also, several floating point numbers,
such as those in lines 6--12, depend on the scaling factors defined in
line 3. They have to be calculated manually, and changing the scale will
imply recalculation of those values. The scale itself is set in line 3:
I wanted the picture to be 64 mm wide, so I had to calculate
$64/(4-0)=16$ for the scaling factor in the x-direction. It would be
much easier if one could type something like:

\footnotesize
\begin{verbatim}
 1 begin parabola 64 64 0 -6 4 5 $x$ $y$
   xmark 2
   Xmark 3
   ymark -5 0 4
 5 bhat
   lines 0 0 0 4
   func 0 3 .05 4-x*x
   lines 3 -5 3 0
   ehat
10 func 0 3.2 .05 4-x*x
   xydrop 3 -5
   arrow 3 4 1.7 1.1 $f(x)=4-x^2$
   arrow 2 5 1 2 Area $O_1$
   arrow 4 2 2.8 -2 Area $O_2$
15 end
   \end{verbatim}
\normalsize

Here we see no brackets, braces or parentheses anymore, width and
height are set straightforwardly to 64 mm and the labels along the
axes are redefined as xmarks and ymarks, for which nothing has to be
given but the x- and y-values, respectively. The corresponding y- and
x-values are supposed to be calculated automatically. 

Another construction that frequently occurs in my pictures is a label
with an arrow starting from the center of its baseline, such as the
one in lines 21--23 in the long listing. This is replaced in the short
listing with line 12, where the starting position of the arrow is
supposed to be calculated automatically. As a result, if I want to
move the label, the arrow is moved with it automatically.

All this is possible by using a \Perl\ interface that converts the
short command file into an \Mfpic\ source file.

\section{The Perl interface}

In the \Perl{} script \Mkpic{}, the available commands are all defined
in the subroutine \verb|parse_input|. My initial version defined only a
few commands, and while using the script, new commands were inserted
when they were needed. It's easy to insert your own new commands here,
just look at what's already there and create new commands by analogy.
The \verb|__DATA__| section of the script contains the picture needed
for this documentation (and some more).

The first was the \textsl{begin} command, of course, which has also the
most complex definition, as it defines many scale-dependent variables
and \TeX{} commands that might be useful for any command defined later.

\subsection{How to use mkpic}
First of all, read the manpage of the \Perl-script, generated from the
script using pod2latex, which is shown in section~\ref{manpage}.

The easiest way to use the script is to append your own commands to
the \verb|__DATA__| section of the script, maybe after removing what's
already there, and run it. This will produce a file mkpic.sty, which
provides \LaTeX-commands named \verb|\Fig<name>|, where \verb|<name>|
stands for every name you use in the \textsl{begin} command. Finally,
you can use those \verb|\Fig<name>| commands in a \LaTeX{} document.

\section{Some more examples}
Here are a few more examples illustrating some features of the \Mkpic{} 
script:

The following commands will produce figure~\ref{fig:droparrows}:

\footnotesize
\begin{verbatim}
 1 begin droparrows 64 64 0 3 12 8 $x$ $y$
   xmark $a$ 2 $x_1$ 4 $x_2$ 8 $b$ 10
   ydrop 2 4.414
   ydrop 4 5
 5 ydrop 8 5.828
   ydrop 10 6.162
   label cc 4 4 $f(x_1)$
   label cc 8 4 $f(x_2)$
   label cc 7 8 $f(x)$ increases on $[a,b]$
10 label cc 7 7.5 $x_1<x_2\Rightarrow f(x_1)<F(x_2)$
   func 1 11 .1 x**(.5)+3
   end
\end{verbatim}
\normalsize

\makefigure{droparrows}

These commands illustrate how valid \Mfpic{} commands can be
interspersed between \Mkpic{} commands (see figure~\ref{fig:asymptotes}:

\footnotesize
\begin{verbatim}
 1 begin asymptotes 64 64 0 0 10 10 $x$ $y$
   curve 1 1 2 3 4 5.7 7 8.1 9 9
   \shift{(-.05,.05)}
   point 2 3 4 5.7 7 8.1
 5 \shift{(-.05,.05)}
   func 1.4 2.6 .1 1.65*x-.3
   func 3.2 4.8 .1 1.025*x+1.6
   func 6.1 7.9 .1 .62*x+3.76
   label cl 9.5 9 $f(x)$
10 label tl 5 5 $f^\prime(x)>0$
   label tl 5 4 $f^\prime(x)$ decreasing
   end
\end{verbatim}
\normalsize

\makefigure{asymptotes}

Figure~\ref{fig:inflections} is produced by:

\footnotesize
\begin{verbatim}
 1 begin inflections 64 64 -.85 -1.5 1.5 5 $x$ $y$
   func -.6 1.5 .05 6*(x**4)-8*(x**3)+1
   lines -.2 1 .2 1
   label cr -.25 1 horizontal
 5 arrow .5 2 0 1 inflection point
   arrow .8 1 .65 0 inflection point
   arrow 1 5 1.5 4.375 $f(x)=6x^4-8x^3+1$
   Xmark 1
   ymark \raisebox{-3.5mm}{0} 0 -1
10 xydrop 1 -1
   end
\end{verbatim}
\normalsize

\makefigure{inflections}

And here is an elaborate quasi 3D picture.
It shows how comments can be inserted.
Standard axes are suppressed because they need special treatment (see figure~\ref{fig:paraboloid}):

\footnotesize
\begin{verbatim}
 1 begin paraboloid 64 64 -4 -4 4 4 - -
   \dashed
   lines -4 0 0 0 0 0 0 -4 # neg z and neg y
   \dashed
 5 lines 0 0 3 1.73        # neg x
   \arrow
   lines 0 0 0 4           # pos y
   \arrow[5]
   lines 0 0 4 0           # pos z
10 \arrow[5]
   lines 0 0 -3 -1.73      # pos x
   \dotted
   lines -1 4 -1 0 -4 -1.73  # intersections y=-1 plane
   \dotted
15 lines -1 -.577 -4 -.577 # intersection x=2 plane with xy-plane
   % extra helplines
   \dotted
   lines -2 .423 -2 -.577 0 0 0 1 -2 .423
   Ymark 3
20 % end of extras
   \dotted
   lines 0 3 -1 3 -4 1.27
   \dashed\sclosed
   curve -3 2.42 -1.5 2.711 -1 2.42 -2.5 2.134
25 label bc 0 \yhi $z$
   label cl \xhi 0 $y$
   label tr -3.1 -1.8 $x$
   label cl -.85 -.577 2
   label tc 0 5.5 $f(x,y)=x^2-4x+2y^2+4y+7$
30 xmark -1
   Ymark 1
   \shift{(-2,.42)}
   \dashed
   func 0 .5 .1 9*x*x
35 func -.5 0 .1 7*x*x
   \dashed
   func -1 0 .1 2*x*x
   func 0 1 .1 2*x*x
   end
\end{verbatim}
\normalsize
\vspace*{3ex}
\makefigure{paraboloid}

\setcounter{secnumdepth}{1}
\onecolumn
\newpage
\section{The mkpic manpage\label{manpage}}
\small

\subsection{mkpic - interface for making pictures with mfpic}

\subsection{Synopsis\label{SYNOPSIS}\index{SYNOPSIS}}

mkpic [options] [picfile]

options:

\begin{verbatim}
       --clean               remove all but input file and die
       --pdfsample           create pdf file with sample images
       --font=<font command> set default font for labels
       --[no]box             produce framed boxes
       --version             report version number and die
       --help                display help info and die
       --[no]debug           display debugging information
       --log=<logfile>       file for warning messages
\end{verbatim}

commands:

\begin{verbatim}
    begin    name xl yl xmin ymin xmax ymax xlabel ylabel
    end    
    stop    
    <var>=<value>
    #        comment
\end{verbatim}
\begin{verbatim}
    arccst   xcenter ycenter xstart ystart theta
    arcset   xstart ystart xend yend theta
    arccrtt  xcenter ycenter radius theta1 theta2
    arc3     x1 y1 x2 y2 x3 y3
\end{verbatim}
\begin{verbatim}
    xmark    [label1] x1 [label2] x2 ...
    Xmark    [label1] x1 [label2] x2 ...
    ymark    [label1] y1 [label2] y2 ...
    Ymark    [label1] y1 [label2] y2 ...
\end{verbatim}
\begin{verbatim}
    xdrop    x y
    ydrop    x y
    xydrop   x y
\end{verbatim}
\begin{verbatim}
    arrow    x1 y1 x2 y2 label
    label    YX x y label
    xlabels  YX x y dx label ...
    ylabels  YX x y dy label ...
\end{verbatim}
\begin{verbatim}
    point    x1 y1 x2 y2 ...
    dpoint   x1 y1 dx1 dy1 ...
    lines    x1 y1 x2 y2 ...
    dlines   x1 y1 dx1 dy1 ...
    curve    x1 y1 x2 y2 ...
    dcurve   x1 y1 dx1 dy1 ...
\end{verbatim}
\begin{verbatim}
    rect     x1 y1 x2 y2
    drect    x y dx dy
    dcrect   x y dx dy
    crect    x1 y1 x2 y2
    arect    xc yc width height theta
    bar      x xdev height
\end{verbatim}
\begin{verbatim}
    func     xmin xmax step expression-in-x
\end{verbatim}
\begin{verbatim}
    grid     dx dy xgap ygap
    hatch    
    bhat    
    ehat
\end{verbatim}
\subsection{Description\label{DESCRIPTION}\index{DESCRIPTION}}

\textbf{mkpic} provides an easy interface for making small pictures with
mfpic. To this end you create an input file has to be created consisting
of commands, one per line, with space separated parameters (or you modify the
\texttt{\_\_DATA\_\_} section of the \textbf{mkpic} script, which is used if you run it
without an input file). \textbf{mkpic} produces two files. Assuming an input
file named \emph{picfile} defaulting to \emph{mkpic} these are:

\begin{description}

\item[{\emph{picfile.mac}}] 

a macro file which will contain \textbf{TeX} commands for every picture

\item[{\emph{picfile.sty}}] 

a style file for latex, defining the same \textbf{TeX} commands for every picture.

\end{description}

With the \textbf{--pdfsample} option, two other files are produced:

\begin{description}

\item[{\emph{picfile.pdf}}] 

a PDF file containing all pictures. 
This lets you easily check the results of your designs.

\item[{\emph{picfile.tex}}] 

the \textbf{TeX} source used for creating this PDF file.

\end{description}

In \textbf{LaTeX}, you have to include \texttt{$\backslash$usepackage\{picfile\}} and to include
commands like \texttt{$\backslash$Fig$<$name$>$} in your source, where \texttt{name} is the
name you gave one of your pictures in an \textbf{mkpic} \texttt{begin} command.

In \textbf{TeX} and \textbf{ConTeXt}, you have to \texttt{$\backslash$input} \emph{picfile.mac} and to
include commands like \texttt{$\backslash$Fig$<$name$>$} in your source, where \texttt{name} is
the name you gave one of your pictures in an \textbf{mkpic} \texttt{begin} command.

In \textbf{TeX}, you must use the \texttt{$\backslash$bye} command (\textbf{not} \texttt{$\backslash$end} to finish your \textbf{TeX} source

See the RUNNING section for how to run \textbf{mkpic} and \textbf{TeX}/\textbf{LaTeX}/\textbf{ConTeXt}.

\subsection{COMMANDS\label{COMMANDS}\index{COMMANDS}}

The source is set up so that it is easy to add your own commands,

Currently the following commands have been implemented:

\begin{description}

\item[{\textbf{begin end}}] 

Every picture begins with the \textbf{begin} command and ends with the \textbf{end}
command.  The \textbf{begin} command defines a name for the picture and
defines a tex command with that name, prefixed with 'Fig'. The
resulting command is written to a \emph{.mac} file. Thus the command

\begin{verbatim}
        begin aa ...
\end{verbatim}

starts writing \texttt{$\backslash$def$\backslash$Figaa\{...} to the \emph{.mac} file, and the
picture can be reproduced in a \textbf{TeX} document by importing the \emph{.mac} file
and using the \texttt{$\backslash$Figaa} command.

xl and yl are the lengths of the x- and y-axes. xlabel and ylabel are
the label that are placed at the ends of those axes. Use a space to
suppress labeling, or "-" to suppress drawing the axes at all.

\item[{\textbf{stop}}] 

stops further reading of the input. Useful if you have many pictures,
but want to see only the first few for testing purposes.

\item[{\textbf{var=value}}] 

sets the variable var to value. This variable, or an expression containing it,
can be used instead of any numerical parameter. Variable names may contain 
lower and uppercase letters, digits or underscores, with the restriction that 
they must start with a letter and may not end in an underscore.

\item[{\#}] 

denotes a comment

\item[{\textbf{xmark ymark Xmark Ymark}}] 

These commands place one or more labels along the x- or y-axes, either
below (\textbf{xmark} and \textbf{ymark}) of above (\textbf{Xmark} and \textbf{Ymark}) the axis.

For the \textbf{[xXyY]mark} commands a parameter containing any character
other than [-.0-9] is interpreted as the label (this implies that you
cannot use expressions here!) to be placed and its position is expected
in the next parameter. If a parameter is just a number, it is placed at
that x-position. If you want a number to be interpreted as a label, put
it in braces: \textit{\{1950\}}.

\item[{\textbf{arccst}}] 

(Mnemonic: \textbf{c}enter \textbf{s}tart \textbf{t}heta.)
Draws an arc with its center in \textit{xcenter,ycenter}, starting in \textit{xstart,ystart}
and with an arc length of \textit{theta} degrees.

\item[{\textbf{arcset}}] 

(Mnemonic: \textbf{s}tart \textbf{e}nd \textbf{t}heta.)
Draws an arc starting in \textit{xstart,ystart}end ending in \textit{xend,yend}
and with an arc length of \textit{theta} degrees.

\item[{\textbf{arccrtt}}] 

(Mnemonic: \textbf{c}enter \textbf{r}adius \textbf{t}heta1 \textbf{t}heta2.)
Draws an arc with its center in \textit{xcenter,ycenter}, a radius \textit{radius}
starting at \textit{theta1} degrees ænd ending at \textit{theta2} degrees.

\item[{\textbf{arc3}}] 

(Mnemonic: \textbf{3} points.)
Draws an arc starting at \textit{(x1,y1), through (x2,y2) and ending in (x3,y3)}.

\item[{\textbf{xdrop ydrop xydrop}}] 

These commands draw dotted arrows perpendicularly to the x-axis, the
y-axis and both axes, respectively, ending on the axes with the
arrow head.

\item[{\textbf{arrow}}] 

draws an arrow from \textit{(x1,y1)} to \textit{(x2,y2)} labeled on its tail with
\textit{label}

\item[{\textbf{label}}] 

draws a label at \textit{(x,y)}. \textit{YX} tells how it will be adjusted: for
Y=t,b,c \textit{(x,y)} will be, in the y-direction, on top, bottom or center
of the label respectively, for X=l,r,c it will be, in the x-direction,
left, right or center adjusted on \textit{(x,y)}. Thus

\begin{verbatim}
        label tl 0 0 Hello World!
\end{verbatim}

will draw the string "Hello World"  with its lower left corner at (0,0)

\item[{\textbf{xlabels}}] 

draws many labels, starting at \textit{(x,y)}, and incrementing \textit{x} with \textit{dx}
after every label. \textit{YX}: see \textbf{label}. Labels may not contain spaces;
if you need spaces, use \texttt{\~{}} instead.

\item[{\textbf{ylabels}}] 

Same as \textbf{xlabels}, but incrementing \textit{y} with \textit{dy} instead.

\item[{\textbf{point}}] 

draws points (dots) at \textit{(x1,y1)}, \textit{(x2,y2)} et cetera.

\item[{\textbf{dpoint}}] 

draws points (dots) starting at \textit{(x1,y1)} and then moving by 
\textit{(dx1,dy1)}, \textit{(dx2,dy2)} et cetera.

\item[{\textbf{lines}}] 

draws line segments from \textit{(x1,y1)} to \textit{(x2,y2)}, \textit{(x3,y3)} et cetera.

\item[{\textbf{dlines}}] 

draws line segments starting at \textit{(x1,y1)} and then moving by 
\textit{(dx1,dy1)}, \textit{(dx2,dy2)} et cetera.

\item[{\textbf{curve}}] 

draws a bezier curve from \textit{(x1,y1)} to \textit{(x2,y2)}, \textit{(x3,y3)} et cetera.

\item[{\textbf{dcurve}}] 

draws a bezier curve starting at \textit{(x1,y1)} and then moving by 
\textit{(dx1,dy1)}, \textit{(dx2,dy2)} et cetera.

\item[{\textbf{rect}}] 

draws a rectangle with diagonal points at \textit{(x1,y1)} and \textit{(x2,y2)}.

\item[{\textbf{drect}}] 

draws a rectangle with diagonal points at \textit{(x,y)} and \textit{(x+dx,y+dy)}.

\item[{\textbf{crect}}] 

clears a rectangle with diagonal points at \textit{(x1,y1)} and \textit{(x2,y2)}.

\item[{\textbf{dcrect}}] 

clears a rectangle with diagonal points at \textit{(x,y)} and \textit{(x+dx,y+dy)}.

\item[{\textbf{arect}}] 

draws a rectangle with a widdh \textit{width} and æ height \textit{height};
the middle of the bottom is at \textit{(xc,yc)} and the centerline through 
\textit{(xc,yc)} makes an angle \textit{theta} with the x-axis.

\item[{\textbf{bar}}] 

draws a equivalent with \textit{rect x-xdev 0 x+xdev height}

\item[{\textbf{func}}] 

draws the function given by \textit{expression-in-x} between \textit{xmin} and
\textit{xmax}, stepping with \textit{step} units in the x-direction.
Note that the \textit{expression-in-x} will be evaluated by \textbf{metafont}, so 
you will have to use metafont syntax.

\item[{\textbf{grid}}] 

draw dotted grid lines at distances dx and dy in the x- and y
directions; the gaps between the dots are set to \textit{xgap} an \textit{ygap}
respectively. For an esthetic appearance, be sure to use integer
\textit{dx/xgap} ænd \textit{dy/ygap} ratios.

\item[{\textbf{hatch}}] 

hatch the closed curve that follows.

\item[{\textbf{bhat}}] 

starts a path that will eventually be closed, and then hatched.

\item[{\textbf{ehat}}] 

ends a path started with \textbf{bhat}, closes it and then hatches it.

\item[{\textbf{anything else}}] 

will be inserted as is in the macro file, and therefore should be a
valid \textit{mfpic} statement. You use this when you need such a statement
only once, or a few times and therefore see no need to define a proper
command for it.

\end{description}
\subsection{Running mkpic/TeX\label{RUNNING_mkpic_TeX}\index{RUNNING mkpic/TeX}}
\subsubsection*{The difficult way\label{The_difficult_way}\index{The difficult way}}

The effect of running \textbf{mkpic picfile} is the creation of
\emph{picfile.mac}, which you can \texttt{$\backslash$input} into a \textbf{TeX} or \textbf{ConTeXt} source,
and \emph{picfile.sty} which can be input into a \textbf{LaTeX} source using the
\texttt{$\backslash$usepackage} command.

After running \textbf{TeX} (or \textbf{LaTeX} or \textbf{ConTeXt}), you will find a file
\emph{picfile.mf} and you will have to run \textbf{metafont} on it, which
(assuming you configured \textbf{TeX} for 600 dpi) produces picfile.600gf. This
file will have to be converted to a pk file with \textbf{gftopk}. Finally, you
need to run \textbf{TeX} again. So the sequence is:

\begin{description}

\item[{\$}] 

mkpic picfile

\item[{\$}] 

tex file.tex

\item[{\$}] 

mf picfile

\item[{\$}] 

gftopk picfile.600gf

\item[{\$}] 

tex file

\end{description}
\subsubsection*{The easy way\label{The_easy_way}\index{The easy way}}

You can also include this line into your \textbf{TeX} or \textbf{ConTeXt} source (before
\texttt{$\backslash$input}ing picfile.mac) or into your \textbf{LaTeX} source (before
\texttt{$\backslash$usepackage\{picfile\}}:

\texttt{$\backslash$immediate$\backslash$write18\{mkpic picfile\}}

and \textbf{TeX} (\textbf{LaTeX}, \textbf{ConTeXt}) will do everything for you, except that you
will have to run \textbf{TeX} at least twice. You need, however, to 1) finish
your texjob with \texttt{$\backslash$bye}, \textbf{not} \texttt{$\backslash$end}, and 2) enable the \texttt{$\backslash$write18}
command by setting, in \emph{texmf.cnf}, the \texttt{shell\_escape} variable to
true (\texttt{t}) (or ask your system administrator to do so).

\subsection{Bug\label{BUG}\index{BUG}}
Currently only up to 256 pictures can be generated. In the future this
problem will probably be solved by introducing more than one font and
generating tex-command names that have the font name in front.

\subsection{Author\label{AUTHOR}\index{AUTHOR}}

Wybo Dekker (wybo@dekkerdocumenten.nl)

\end{document}
