%% BEGIN poster.doc
%%
%% Documentation for poster.tex/poster.sty.


\documentclass[12pt]{article}
\usepackage [T1]{fontenc}
\usepackage{mathpazo,url}
\usepackage{poster}
\let\Pfv\fileversion
\begin{filecontents}{poster1.tex}
%% BEGIN poster1.tex
%%
%% Sample for poster.tex/poster.sty.
%% Run with LaTeX, with or without the NFSS.
%% You might have problems with missing fonts.
%%
%% See below if using A4 paper.

\documentclass[dvips,a4paper]{article}

\usepackage{poster,wasysym,geometry} 

\mag\magstep5  % Magnification of 1.2^5 (roughly 2.5)
               % Use "true" dimensions below for magnified values.

\begin{document}

\begin{Poster}[vcenter=true,hcenter=true]
\setlength{\fboxsep}{.8truein}%
\setlength{\fboxrule}{.1truein}%
\fbox{\begin{minipage}{11.1truein}

\begin{center}
  \bf ON SOME \boldmath$\Pi$-HEDRAL SURFACES IN QUASI-QUASI SPACE
\end{center}
\begin{center}
  CLAUDE HOPPER, Omnius University
\end{center}

There is at present a school of mathematicians which holds that the
explosive growth of jargon within mathematics is a deplorable trend.  It
is our purpose in this note to continue the work of
Redheffer~\cite{redheffer} in showing how terminology itself can lead to
results of great elegance.

I first consolidate some results of Baker~\cite{baker} and
McLelland~\cite{mclelland}.  We define a class of connected snarfs as
follows: $S_\alpha=\Omega(\gamma_\beta)$.  Then if
$B=(\otimes,\rightarrow,\theta)$ is a Boolean left subideal, we have:
$$
\nabla S_\alpha=\int\int\int_{E(\Omega)}
B(\gamma_{\beta_0},\gamma_{\beta_0})\,d\sigma d\phi d\rho
-\frac{19}{51}\Omega.
$$
Rearranging, transposing, and collecting terms, we have:
$\Omega=\Omega_0$.

The significance of this is obvious, for if $\{S_\alpha\}$ be a class of
connected snarfs, our result shows that its union is an utterly
disjoint subset of a $\pi$-hedral surface in quasi-quasi space.

We next use a result of Spyrpt~\cite{spyrpt} to derive a property of
wild cells in door topologies.  Let $\xi$ be the null operator on a door
topology, $\Box$, which is a super-linear space.  Let $\{P_\gamma\}$ be
the collection of all nonvoid, closed, convex, bounded, compact,
circled, symmetric, connected, central, $Z$-directed, meager sets in
$\Box$.  Then $P=\cup P_\gamma$ is perfect.  Moreover, if $P\neq\phi$,
then $P$ is superb.

\smallskip
{\it Proof.}  The proof uses a lemma due to
Sriniswamiramanathan~\cite{srinis}.  This states that any unbounded
fantastic set it closed.  Hence we have
$$
\Rightarrow P\sim\xi(P_\gamma)-\textstyle\frac{1}{3}.
$$

After some manipulation we obtain
$$
\textstyle\frac{1}{3}=\frac{1}{3}
$$
I have reason to believe~\cite{russell} that this implies $P$ is perfect.
If $P\neq\phi$, $P$ is superb.  Moreover, if $\Box$ is a $T_2$ space, $P$
is simply superb.  This completes the proof.

Our final result is a generalization of a theorem of Tz, and
encompasses some comments on the work of Beaman~\cite{beaman} on the
Jolly function.

Let $\Omega$ be any $\pi$-hedral surface in a semi-quasi space.  Define
a nonnegative, nonnegatively homogeneous subadditive linear functional
$f$ on $X\supset\Omega$ such that $f$ violently suppresses $\Omega$.
Then $f$ is the Jolly function.

\smallskip
{\it Proof.}  Suppose $f$ is not the Jolly function.  Then
$\{\Lambda,\mbox{@},\xi\}\cap\{\Delta,\Omega,\Rightarrow\}$ is void.  Hence
$f$ is morbid.  This is a contradiction, of course.  Therefore, $f$ is
the Jolly function.  Moreover, if $\Omega$ is a circled husk, and
$\Delta$ is a pointed spear, then $f$ is uproarious.

\small
\begin{center}
\bf References
\end{center}
\def\thebibliography#1{%
  \list
 {\bf\arabic{enumi}.}{\settowidth\labelwidth{\bf #1.}\leftmargin\labelwidth
 \advance\leftmargin\labelsep
 \usecounter{enumi}}
 \def\newblock{\hskip .11em plus .33em minus .07em}
 \sloppy\clubpenalty4000\widowpenalty4000
 \sfcode`\.=1000\relax}
\begin{thebibliography}{9}
\bibitem{redheffer}
R. M. Redheffer, A real-life application of mathematical symbolism,
this {\it Magazine}, 38 (1965) 103--4.
\bibitem{baker}
J. A. Baker, Locally pulsating manifolds, East Overshoe Math. J., 19
(1962) 5280--1.
\bibitem{mclelland}
J. McLelland, De-ringed pistons in cylindric algebras,
Vereinigtermathematischerzeitung f\"ur Zilch, 10 (1962) 333--7.
\bibitem{spyrpt}
Mrowclaw Spyrpt, A matrix is a matrix is a matrix, Mat. Zburp., 91
(1959) 28--35.
\bibitem{srinis}
Rajagopalachari Sriniswamiramanathan, Some expansions on the Flausgloten
Theorem on locally congested lutches, J. Math. Soc., North Bombay, 13
(1964) 72--6.
\bibitem{russell}
A. N. Whitehead and B. Russell, Principia Mathematica, Cambridge
University Press, 1925.
\bibitem{beaman}
J. Beaman, Morbidity of the Jolly function, Mathematica Absurdica, 117
(1965) 338--9.
\end{thebibliography}
\end{minipage}}%
\end{Poster}

\end{document}
%% END poster1.tex%%%%%%%%%%%%%%%%%%%%%%%%%%%%%%%%%%%%%%%
\end{filecontents}
\begin{filecontents}{poster2.tex}
%% BEGIN poster2.tex
%%
%% A sample file for poster.tex/poster.sty. Makes a banner.
%% Use Plain TeX (or add preamble and use LaTeX)

\input poster

\special{landscape}  % This works with Rokicki's dvips

% ptmr should be name of the PostScript Times-Roman font:
% 8in is a good size for this font, but might not work with other fonts.
% You can use a Computer Modern font, if you are prepared to make a big
% font bitmap.

\font\bigroman=ptmr at 8in
\bigroman

\poster[vcenter=true,landscape=true]{Happy}

\end
%% END poster2.tex%%%%%%%%%%%%%%%%%%%%%%%%%%%%%%%%%%%%%%%%
\end{filecontents}

% Now adjust for different paper size:
\newdimen\mydim
\mydim=\paperwidth
\advance\mydim-8.5in
\divide\mydim 2
\advance\oddsidemargin \mydim
\advance\evensidemargin \mydim
\mydim=\paperheight
\advance\mydim-11in
\divide\mydim 2
\advance\topmargin \mydim

%% OTHER

% Short meta (works in verbatim. Can't use < for other purposes.
\catcode`\<=13 \def<#1>{{\rm\it #1\/}}    % <meta> (works in verbatim)

% Short verbatim.
\catcode`\"=13
\def"{\verb"}

\catcode`\@=12  % In case I'm using AmS-LaTeX

\title{Documentation\thanks{Documentation revised by Herbert Vo\ss}  for poster.tex:\\[5pt]
  \small Posters and banners with generic \TeX, v. \Pfv}
\author{Timothy Van Zandt\\ \url{tvz@Princeton.EDU}}
\date{\today}

\begin{document}

\maketitle
\begin{abstract}
"poster.tex/poster.sty" contains the macro
\begin{verbatim}
    \poster{<stuff>}
\end{verbatim}
for making posters and banners. <stuff> is processed in restricted horizontal
mode (i.e., "\hbox" or ``LR-mode'') and is then printed on as many sheets of
paper as are needed. You can then construct the poster or banner by trimming
and piecing together the sheets of paper.
\end{abstract}
\clearpage
\tableofcontents
\clearpage

\section{Introduction}
You can also write
\begin{verbatim}
    \Poster <stuff>\endPoster
\end{verbatim}
and \LaTeX\ users can write
\begin{verbatim}
    \begin{Poster} <stuff> \end{Poster}
\end{verbatim}

\section{Details}
\begin{itemize}
\item Use a "\vbox" or \LaTeX's "minipage" or "\parbox" in <stuff> if you want
to include vertical mode material. For more help with LR-boxes, see
"fancybox.sty", available from archives everywhere.

\item Don't worry about margins, headers or footers; "\poster" ignores output
routines entirely.

\item <stuff> can contain "\catcode" changes, such as verbatim environments.
\end{itemize}

If you want to use your regular output routines, and have "poster.tex" print
out each page of your document as a poster, then instead put the command
\begin{verbatim}
    \PosterPage
\end{verbatim}
towards the beginning of your document, or in the \LaTeX\ preamble. Each page
is printed without its margins, but with the headers and footers, if any. (You
can print out your whole dissertation on $8\times 10$-feet pages.) With
"\PosterPage", you do not have to worry about LR-boxes.

Whichever method you use, it is up to you to make everything big. The easiest
way is to set \TeX's "\mag" parameters (to an integer equal to 1000 times the
magnification factor, or to "\magstephalf", or to "\magstep"<n>, where <n> is
1,$\ldots$,5). You may need to generate big bitmaps if using bitmapped fonts
(e.g., \TeX's usual Metafont fonts), rather than scalable outline fonts (e.g.,
PostScript fonts). Using "\magstep"<n> minimizes the need for extra bitmaps.

\section{Parameter}
 "\poster", "\Poster" and "\PosterPage" use the following parameters:
\begin{center}
  \def\arraystretch{1.1}
  \begin{tabular}{lll}
    {\em Parameter} & {\em Value} & {\em Default}\\[2pt]
    "paperwidth"    & <dim>       & "8.5in"\\
    "paperheight"   & <dim>       & "11in"\\
    "imagewidth"    & <dim>       & "7.5in"\\
    "imageheight"   & <dim>       & "10in"\\
    "landscape"     & "true"/"false" & "false"\\
    "hcenter"       & "true"/"false" & "false"\\
    "vcenter"       & "true"/"false" & "false"\\
    "crop"          & "none"/"corners"/"full" & "corners"\\
    "clip"          & "none"/"pstricks"  & "none"
  \end{tabular}
\end{center}

You can include parameter changes as a list of "<key>=<value>" pairs in an
optional argument to "\poster", "\Poster" or "\PosterPage", enclosed in square
brackets. E.g.,
\begin{verbatim}
  \poster[clip=pstricks,hcenter=true]{foo}
\end{verbatim}
No extraneous spaces, please.

You can also redefine the parameters using "\def" or "\LaTeX"'s
"\renewcommand". For parameter "foo", you should redefine "\POSTERfoo". E.g.,
the next example is like the last one:
\begin{verbatim}
  \def\POSTERclip{pstricks}
  \def\POSTERhcenter{true}
  \poster{foo}
\end{verbatim}
This is mainly of interest when developing your own custom "\poster" command
or "poster.tex" file.

Here are some comments on the parameters:
\begin{itemize}
\item
Don't adjust the "paper" and "image" dimensions for your document's
magnification (and don't use \TeX's "true" dimensions). "poster.tex" does this
for you. (That is, set the "page" and "image" dimensions to the actual values
you want for the output.)

\item
Most printers cannot print right up to the edge of the paper. That is why the
"imagewidth" and "imageheight" should be smaller than the "paperwidth" and
"paperheight". The default values are good for printing on 8.5in by 11in paper
in portrait mode with one-half inch margins.

\item
Setting "landscape" to "true" is just a convenient way to switch the "height"
and "width" parameters. You still have to take care of printing your document
in landscape mode. E.g., with Rokicki's "dvips", use
\begin{verbatim}
    \special{landscape}
\end{verbatim}
For other dvi driver's, consult the documentation.

\item
Setting "hcenter" and "vcenter" to "true" causes the image to be centered
horizontally and vertically, respectively, in the total number of pages that
are printed. E.g., when "vcenter" is "true", extra space is added to the top
of the first row of pages and to the bottom of the last row of pages.

\item
The "crop" parameter controls the crop marks, to help you trim each page to
size. When "crop" equals "full", you get crop marks along the full length of
all four sides.

\item
Setting "clip" to "pstricks" causes each page to be clipped to the size of the
image (rather than having the image overlap in the margins on each page), but
this only works if you have loaded the PSTricks package.
\end{itemize}

\section{Examples}
The file "poster1.tex" contains the following sample of a framed poster with a
whole page of text:
\begin{verbatim}
    \documentclass{article}
    \usepackage{poster}
    \mag\magstep5  % Magnification of 1.2^5 (roughly 2.5)
    % Use `true' dimensions below for magnified values.

    \begin{document}

    \begin{Poster}[vcenter=true,hcenter=true]
      \setlength{\fboxsep}{.8truein}%
      \setlength{\fboxrule}{.1truein}%
      \fbox{\begin{minipage}{11.1truein}
        <stuff>
      \end{minipage}}%
    \end{Poster}

    \end{document}
\end{verbatim}

The file "poster2.tex" contain the following sample of a banner in landscape
mode. "ptmr" is meant to be the name of the Times-Roman PostScript font, if
your dvi driver supports such a thing.
\begin{verbatim}
    \font\bigroman=ptmr at 7.5in
    \bigroman
    \poster[vcenter=true,landscape=true]{Animals}
\end{verbatim}


\end{document}
%%
%% END poster.doc

