% USER MANUAL in Plain TeX

\input formlett.sty

% FONTS
\font\hbold=cmbx10 at 17.28pt
\font\lbold=cmbx10 at 14.4pt
\font\bbold=cmbx10 at 12pt
\font\small=cmr8
\font\tiny=cmr5
\font\stt=cmtt9
\catcode`\@=11\relax
%
%
%
% OTHER MACROS
%
\catcode`\~=11\relax \def\t{{\tt ~}}
\catcode`\~=\active\relax
\catcode`\!=0\relax !catcode`!\=11!relax
!def!esc{\}!def!escs{\! }
!catcode`!\=0!relax \catcode`\!=12\relax
\def\rqs{\rq\ }
\def\LaTeX{{\rm L\kern-.36em\raise.3ex\hbox{\font\sc=cmcsc10\sc a}\kern-.15em
    T\kern-.1667em\lower.7ex\hbox{E}\kern-.125emX}}

% makeshift macro
\catcode`\^^M=13\relax%
\def\setraw{\bgroup\rm@ligature\set@space\set@tab%
\set@cr\stt\raw@chars[11][13][13][13][13]\def^^M{\strut\par}\setraw@@}%
\edef\temp@macro{%
   \noexpand\long\noexpand\def\noexpand\setraw@@\noexpand##1^^M##2%
     \esc@  unsetraw{##1##2\noexpand\egroup}}\temp@macro %
\catcode`\^^M=5\relax%


\font\ninerm=cmr9  \font\eightrm=cmr8
\font\ninei=cmmi9  \font\eighti=cmmi8
\font\ninesy=cmsy9 \font\eightsy=cmsy8
\font\ninebf=cmbx9 \font\eightbf=cmbx8
\font\ninett=cmtt9 \font\eighttt=cmtt8
\font\nineit=cmti9 \font\eightit=cmti8
\font\ninesl=cmsl9 \font\eightsl=cmsl8

\font\sixrm=cmr6
\font\sixi=cmmi6
\font\sixsy=cmsy6
\font\sixbf=cmbx6

\skewchar\ninei='177  \skewchar\eighti='177  \skewchar\sixi='177
\skewchar\ninesy='60  \skewchar\eightsy='60  \skewchar\sixsy='60
\hyphenchar\ninett=-1 \hyphenchar\eighttt=-1 \hyphenchar\tentt=-1

\font\tentex=cmtex10
\font\inchhigh=cminch
\font\titlefont=cmssdc10 at 40pt   % titles in chapter openings
\font\eightss=cmssq8               % quotations in chapter closings
\font\eightssi=cmssqi8             % ditto, slanted
\font\tenu=cmu10                   % unslanted te~t italic
\font\manual=manfnt                % METAFONT logo and special symbols
\font\magnifiedfiverm=cmr5 at 10pt % to demonstrate magnification

\newskip\ttglue
\def\eightpoint{\def\rm{\fam0\eightrm}% switch to 8-point type
  \textfont0=\eightrm \scriptfont0=\sixrm \scriptscriptfont0=\fiverm
  \textfont1=\eighti  \scriptfont1=\sixi  \scriptscriptfont1=\fivei
  \textfont2=\eightsy \scriptfont2=\sixsy \scriptscriptfont2=\fivesy
  \textfont3=\tenex   \scriptfont3=\tenex \scriptscriptfont3=\tenex
  \textfont\itfam=\eightit  \def\it{\fam\itfam\eightit}%
  \textfont\slfam=\eightsl  \def\sl{\fam\slfam\eightsl}%
  \textfont\ttfam=\eighttt  \def\tt{\fam\ttfam\eighttt}%
  \textfont\bffam=\eightbf  \scriptfont\bffam=\sixbf
   \scriptscriptfont\bffam=\fivebf  \def\bf{\fam\bffam\eightbf}%,
  \tt \ttglue=.5em plus.25em minus.15em
  \normalbaselineskip=9pt
  \setbox\strutbox=\hbox{\vrule height7pt depth2pt width0pt}%
  \let\sc=\sixrm  \let\big=\eightbig  \normalbaselines\rm}


%%%%%%%%%%%%%%%%
%%%%%%%%%%%%%%%%
%
% END OF EXTRA MACROS AND FONTS
%
%%%%%%%%%%%%%%%%
%%%%%%%%%%%%%%%%




\catcode`\@=11\relax
\PAGENO=1
\voffset=-0.3in


\ifx\pageno\undefined
\errmessage{Please run TeX if you want this article on how to use
<formlett.sty> to be processed as well!}
\fi


\parindent=0pt
\hsize=469.75499pt  % the default
\vsize=9.6in %8.83in

\def\makefootline{\baselineskip=24pt \line{\the\footline}}
\footline={\lower5pt
\hbox to \hsize
 {{\tenrm Jiang Z ~~~{\tensl Mail merge with the use of formlett.sty}
\hss \folio}}}

\leftline{\hbold Mail merge with the use of formlett.sty}
\bigskip


{\advance\leftskip by13mm \overfullrule=0pt
\advance\rightskip by 20mm %28mm
\sl
\noindent Zhuhan JIANG\par
\noindent School of Computing and Information Technology,
 University of Western Sydney, Victoria Road, Parramatta NSW 2150, Australia.
{\sl ~~Email}:~z.jiang@uws.edu.au\par
\medskip
\rm\smooth
\noindent (Last updated on 26 May 2003)\par
\medskip

\noindent
In this article, the author explains how to use a form-letter or mail
merge style {\tt formlett.sty}, designed for plain \TeX\space and
\LaTeX\space or \LaTeX2e. {\tt formlett.sty} supports different
parameter input methods, parameter naming and defaulting mechanism, as
well as facilities for previewing parameter positions and printing
labels. It is written for the purpose of being powerful, robust and
above all easy to use.\par }


\begincolumns

{\lbold Introduction}

\smallskip

Our purpose here is to describe a comprehensive implementation of a
macro system, handling form letters or mail merge under plain \TeX\space
or \LaTeX\space2.09 or \LaTeX2e. The main objective is to provide an
easy way to output many form letters with their own parameters, with or
without the use of multiple files. There will be a coherent and simple
format for putting parameters inside a form letter, with a number of
helping facilities for such as naming parameters and previewing their
positions. A minimum support for printing mailing labels is also
provided.


\medskip

The concept of macros [1] for form letters is not new: there already
exist macros in this connection such as {\tt merge}, {\tt textmerg} and
{\tt address} to name a few, see [2,3] for further details. Our stress
here is therefore laid on the ease to use, along with the power and the
robustness of the macros.

\bigskip


{\lbold Basic format}


%%%%%%%%%%%%
% letter top
%%%%%%%%%%%%

  \beginletter
\sl
\hsize=3.1in\relax
\medskip
  \paranames             % optional
    \tt<<FULL NAME>>;%
    \tt<<ADDRESS-01>>;%
    \tt<<ADDRESS-02>>;%
    \tt<<GIVEN NAME>>;%
    \tt<<MISSING ITEM>>;%
    \tt<<PHONE NUMBER>>!
  \loaddefaultparas      % optional

  \NOPAGENUMBERS\parindent=0pt\overfullrule=0pt
  \paras[1]\par\paras[2]\par\paras[3]\par\medskip

  Dear \paras[4],

  \smallskip
  We have been looking for
  \paras[5] for quite a while
  without any luck, could you help
  us out? If so, please ring
  \paras[6]. \par\medskip
  Cheers!\hfill Michael\par %\vfill\eject
\medskip
  \endletter

%%%%%%%%%%%%
% letter end
%%%%%%%%%%%%


Suppose we would like to write a form letter for mail merge in the
following form

\showparas

where the text inside the double arrow brackets are to be replaced by
the parameters specific to each individual letter. If we number these
parameters one by one in a single group, then the letter template will
be organised as

\preview

and we need just 6 parameters to personalise an individual letter. A
simplest example of supplying a set or a {\sl cluster} of these
parameters to output a merged form letter is to use

\setraw
\moreletter
  Mrs L Stenson;\#1-20 Sunset Street;%
  Hills, Norway;Louise;a Bible;220-8888!
\unsetraw

For more letters to be merged this way, just use again the
command {\tt\string\moreletter} in the above format, in which each
parameter is separated by a semicolon `{\tt ;}' and the whole parameter
cluster is ended by an exclamation mark `{\tt !}'. The actual code in
\LaTeX\space for making the above form letter, along with an extra
merged letter, may read as follows


\edef\setrawcount{\stt\noexpand
  \beginrawlist[][][4][][][3.3in]{\esc@ unsetraw}}

\linecount=1\relax
\setrawcount
  \documentstyle[formlett]{article}
  \begin{document}
  \beginletter
    \NOPAGENUMBERS\parindent=0pt
    \paras[1]\par\paras[2]\par
    \paras[3]\par\medskip
    Dear \paras[4], \par\medskip
    We have been looking for
    \paras[5] for quite a while
    without any luck, could you help
    us out? If so, please ring
    \paras[6]. \par\medskip
    Cheers!\hfill Michael\vfill\eject
  \endletter

  \moreletter % letter one
    Mrs L Stenson;\#1-20 Sunset Street;%
    Hills, Norway;Louise;a Bible;220-8888!
  \moreletter % letter two
    S Wales;UMIST;Manchester;Sir or Madam;%
    a \TeX\ package{+}manual;225-9905!
  \end{document}
\unsetraw
\rm

The line numbers in the above and later on are for quick reference only:
they are not a part of the program code. The letter template in lines
4-13 is defined by the {\tt\string\beginletter} and {\tt\string\endletter}
pair in line 3 and line 14, with each use of {\tt\string\moreletter}
generating a new merged letter. We note that command {\tt \string\paras[{\it
m}]} represents the $m^{th}$ merged parameter (of the first group, see
later), and the command {\tt\string\NOPAGENUMBERS}
will remove the page numbering for both \LaTeX\space and plain
\TeX. We also note that if one replaces lines 1-2 by
{\tt\string\input\space formlett.sty}, then the program will be
compilable under plain \TeX\space as well.


\medskip

Although the mail merge mechanism in the above will be sufficient for
most simple applications, {\tt formlett} provides in fact many more
features. We shall thus in the followings introduce the main features
one by one, from the most useful to the less likely encountered.

\medskip

To start with, we notice that in the above example there are exactly two
parameters to be used for the address. This is inconvenient to say the
least, even if we allocate say three more extra parameters for the
purpose. This naturally calls for the separation of the whole {\sl
cluster} of parameters into several {\sl groups}, so that the numbering
of the parameters in every group will be resynchronised. We thus use
{\tt\string\paras[{\it m}][{\it n}]} to denote the $m^{th}$ parameter of
the $n^{th}$ group of a whole cluster of merging parameters. In fact we
may use {\tt\string\blockparas[{\it m}][{\it n}][{\it pre}][{\it post}]}
to represent parameters in the $n^{th}$ group, from the $m^{th}$ to the
last parameter of that group, where the tokens denoted by {\it pre} and
{\it post} are those to be added in front and behind respectively each
of the legitimate parameters. The defaults for {\it pre} and {\it post}
are {\tt\string\noindent} and {\tt\string\par} respectively. It is
therefore more sensible to specify the merging parameters of the earlier
example via


%%%%%%%%%%%%
% letter top
%%%%%%%%%%%%

  \beginletter
\sl
\hsize=3.1in\relax
\medskip
  \paranames             % optional
    \tt<<FULL NAME>>\hfill
    \tiny\fillzeros[4]\linecount\global\advance\linecount1\relax\par;%
    \tt<<ADDRESS-etc>>\hfill
    \tiny\fillzeros[4]\linecount\global\advance\linecount1\relax\par;%
   +\tt<<GIVEN NAME>>;%
    \tt<<MISSING ITEM>>;%
    \tt<<PHONE NUMBER>>!
  \loaddefaultparas      % optional

  \NOPAGENUMBERS\parindent=0pt\overfullrule=0pt
  \noindent{\it\paras[1]}\par
  \blockparas[2]\par\medskip

  Dear \paras[1][2],

  \smallskip
  We have been looking for
  \paras[2][2] for quite a while
  without any luck, could you help
  us out? If so, please ring
  \paras[3][2]. \par\medskip
  Cheers!\hfill Michael\par %\vfill\eject
\medskip
  \endletter

%%%%%%%%%%%%
% letter end
%%%%%%%%%%%%

\showparas

where {\tt <<ADDRESS-etc>>} will be produced by
{\tt\string\blockparas[2][1]}, i.e. all the parameters of the first
group, starting from the 2nd parameter. We note that the missing
square-bracketed macro parameters for {\tt\string\blockparas} are
automatically provided: in fact macro parameters in {\tt formlett} are
systematically defaulted. Hence for example, {\tt \string\paras[1][1]}
is equivalent to any of the following commands

\setraw
  \paras, \paras[1], \paras[][1],
  \paras[1][], \paras[][]
\unsetraw

As for the more precise position of the merging parameters, they can
always be viewed via the command {\tt\string\preview}, which in this
case gives the following positioning

\preview

This way, the mail merge for the above letter template can also be done
(in plain \TeX) in the following more flexible way

\setrawcount
  \input formlett.sty
  \beginletter
  \NOPAGENUMBERS\parindent=0pt
  \noindent{\it\paras[1]}\par
  \blockparas[2]\par\bigskip
  Dear \paras[1][2],\par\medskip
  We have been looking for
  \paras[2][2] for quite a while
  without any luck, could you help
  us out? If so, please ring
  \paras[3][2]. \par\medskip
  Cheers!\hfill Michael\vfill\eject
  \endletter \preview

  \beginpilemode
    Mrs L Stenson;\#1-20 Sunset Street
    Hills, Norway+Louise;a Bible;220-8888!

    S Wales;UMIST;Manchester+%
    Sir or Madam;a \TeX\space
    package{+}manual;225-9905!
  \endpilemode
  \end
\unsetraw
\rm


We note that by default we used plus sign `{\tt +}' to separate
different groups inside a cluster of parameters, and each cluster inside
a {\tt\string\beginpilemode} and {\tt\string\endpilemode} pair will
output a merged letter, as if it were produced by command
{\tt\string\moreletter} followed by that cluster of parameters.
The output of the first merged letter in fact reads

\moreletter
    Mrs L Stenson;\#1-20 Sunset Street
    Hills, Norway+Louise;a Bible;220-8888!


\medskip

It is often desirable to supply parameters line by line, i.e. one line
as one parameter. Hence the two merged letters produced by lines 25-47
can also be made by replacing the content included in the
{\tt\string\beginpilemode} and {\tt\string\endpilemode} pair (lines
39-46) by the following new code

\setrawcount
\blockmarks
\beginblockmode
Mrs L Stenson
\#1-20 Sunset Street
Hills, Norway
-----
Louise
a Bible
220-8888
======

S Wales
UMIST
Manchester
----
Sir or Madam
a \TeX\ package+manual
225-9905
====
\endblockmode
\defaultmarks
\unsetraw
\rm

where obviously we have used lines with `{\tt----}' to separate groups
and used lines with `{\tt====}' to end a cluster of parameters. This was
done by the use of {\tt\string\blockmarks} which changes the group
delimiter `{\tt +}' and cluster delimiter `{\tt !}' into the above two
new ones, with the {\tt\string\defaultmarks} at the end setting them
back. If the whole cluster of parameters are categorised into a single
(first) group, then we may even use a block of consecutive nonempty
lines to provide the merging parameters. This is done by the use of
command pair {\tt\string\beginlinemode} and {\tt\string\endlinemode}.
Hence we may also conduct the above mail merge in for instance \LaTeX2e
by


\setrawcount
\documentclass{article}
\usepackage{formlett}
\begin{document}
  \beginletter
    \NOPAGENUMBERS\parindent=0pt
    \paras[4]\par\blockparas[5]
    \par\medskip
    Dear \paras[1], \par\medskip
    We have been looking for
    \paras[2] for quite a while
    without any luck, could you help
    us out? If so, please ring
    \paras[3]. \par\medskip
    Cheers!\hfill Michael\vfill\eject
  \endletter

\beginlinemode
Louise
a Bible
220-8888
Mrs L Stenson
\#1-20 Sunset Street
Hills, Norway

Sir or Madam
a \TeX\ package+manual
225-9905
S Wales
UMIST
Manchester
\endlinemode
\end{document}
\unsetraw
\rm

Notice that we have changed the numbering order of the parameters so
that the number of address lines output by {\tt\string\blockparas} is
more flexible.


\bigskip
{\lbold Advanced features}

The advanced features to be discussed below are mainly for the purpose
of (i) writing {\sl long} form letter; (ii) making {\sl many} different
form letters; (iii) reading mail merge parameters that are in {\sl
crude} ASCII form; and (iv) generating mail labels.

\medskip

It is in general better to keep the letter template, i.e. the content
between {\tt\string\beginletter} and {\tt\string\endletter} like those
in lines 4-13, in a separate file, say {\tt myletter.let}, and use
{\tt\string\inputletter$\{$myletter.let$\}$} to load in the letter
template. The advantage of this over the previously used method is that
the letter template can now be arbitrarily large without any risks of
causing computer memory problem.

\medskip

Suppose we have a collection of letter templates, we may wish to remind
ourselves what each parameter represents. For this purpose, we may add
at the top of the letter template a command {\tt\string\paranames} {\it
cluster}{\tt !}, where {\it cluster} is a cluster of parameters each
giving a name to the corresponding parameter. Command
{\tt\string\paranames} takes its macro parameters in the same way
{\tt\string\moreletter} does. After a letter template is read in,
command {\tt\string\showparas} will output a form letter with its
parameters replaced by the corresponding names. The mainly italic
passage containing lines 23-24, for instance, is the output of
{\tt\string\showparas} for the letter template in lines 102-109 given
explicitly later on.

\medskip

We may also use {\tt\string\paradefaults} {\it cluster}{\tt !} to
provide default parameters, i.e. the parameters to replace the empty or
not entered ones, and use {\tt\string \loaddefaultparas} inside letter
template to activate the parameter defaulting. We note that in general
commands {\tt\string\paranames} and {\tt\string\loaddefaultparas}, if
present, should be at the top of the letter template, with
{\tt\string\loaddefaultparas} below the command {\tt\string\paranames}.
Hence for our previous mail merge, we may re-do it via


\setrawcount
  \input formlett.sty
  \beginletter
  \paranames             % optional
    \tt<<FULL NAME>>;\tt<<ADDRESS-etc>>;%
   +\tt<<GIVEN NAME>>;\tt<<MISSING ITEM>>;%
    \tt<<PHONE NUMBER>>!
  \loaddefaultparas      % optional
\unsetraw

\ \ \ {\it $\{$letter details given in lines 27-36$\}$}
\hfill{\tiny\fillzeros[4]\linecount\global\advance\linecount1\relax}

\setrawcount
  \endletter
  \preview \showparas

  \paradefaults       % optional
    To whom this may concern
   +Sir or Madam;something;%
    061-225-9905!

\blockmarks
\beginrawblockmode{}
Mrs L Stenson
#1-20 Sunset Street
Hills, Norway
------
Louise
a Bible
220-8888
=========

......

Above empty line active
\endrawblockmode
\defaultmarks
\end{document}
\unsetraw
\rm

We note that in {\sl blockmode} with {\tt\string\blockmarks}, a line
with `{\tt ....}' marks the immediate start of a parameter block,
whether or not the lines immediately following it are empty or not. Also
that in {\it rawblockmode} enclosed by {\tt\string \beginrawblockmode}
and {\tt\string\endrawblockmode}, all characters are input in their
original ASCII form. In other words, the special characters for
\TeX\space will be ignored here.

\medskip

If the merging parameters are given by exactly $m$ lines per cluster, as
is often the case when they are produced by a database utility, then we
may use the {\tt\string\begindatamode} and {\tt\string\enddatamode} pair
to mark the beginning and the end of the merging parameters. For more
details, see the Appendix at the end.

\medskip

We may keep the letter template and the merging parameters in two
separate files. For instance, if we save lines 103-108 to file {\tt
myletter.let}, lines 39-46 or lines 48-68 to file {\tt myletter.adr},
then we can produce via \LaTeX\ the mail merge by

\setrawcount
  \documentstyle[formlett]{article}
  \begin{document}
  \inputletter{myletter.let}
  \showparas \preview  % utility demo
  \paradefaults To whom it may concern!
  \inputfile{myletter.adr}
  \beginlabels % 1st group as address
  \inputfile{myletter.adr} % for labels
  \endlabels
  \end{document}
\unsetraw
\rm

where the lines 140-142 will generate the address labels using the first
group (as is intended here) of parameters. More precise format of
{\tt\string\beginlabels} and etc can be found in the Appendix.

\medskip

Another way of making labels, if no separate files for the parameters
are present, is to re-read the document itself and make labels instead
of merged form letters in the second reading. We may thus use {\tt
\string\input \ formlett.sty \string\initstyle\ [{\it styles}]$\{${\tt
article}$\}\{${\it preamble}$\}$} to replace {\tt \string\documentstyle\
[formlett, {\it styles}]$\{$article$\}$ {\it preamble}
\string\begin$\{$document$\}$}, which will be valid for \LaTeX\ but
ignored for \TeX\ , and will enable one to use {\tt \string\labelsquit}
at the end to read in the current document again with all the letters
there converted into the corresponding labels. Though
{\tt\string\initstyle} is valid for plain \TeX, \LaTeX2.09 as well as
\LaTeX2e, {\tt\string\initclass} will generate
{\tt\string\documentclass} instead of {\tt\string\documentstyle} when
\LaTeX2e is the processing environment. If you only want to execute
certain commands the first time round (i.e. before {\tt
\string\labelsquit} re-reads the file again), use {\tt
\string\firstread$\{${\it commands}$\}$} for this purpose. For more
detailed use and examples, read the {\sl programmer's document} attached
to the style file {\tt formlett.sty} and the example files it generates.
In fact, many details and extra features that are not contained in this
article may be found there.

\medskip

We already noted that if a form letter is large, we have to use
{\tt\string\inputletter} to load in the template. However, it is often
still desirable to keep everything inside a single file, while allowing
it to make new (scratch) files when it is being compiled. We may thus
use for example

\setraw
  \beginfile{myletter.let}
\unsetraw

\ \ \ {\it $\{$letter template of lines 103-108$\}$}

\setraw
  \endfile
\unsetraw
\rm

to create a letter template file {\tt myletter.let}. Likewise, the
merging parameters can be saved to another run-time created file {\tt
myletter.adr}.


\medskip

We have so far only used the default delimiters `{\tt ;}', `{\tt +}' and
`{\tt !}' to separate single, group and cluster of parameters, with
temporary change via {\tt\string\blockmarks} to `{\tt ....}', `{\tt
----}' and `{\tt ====}' respectively. However we may change the
delimiters to any characters by {\tt
\string\delimiters$\{S\}\{G\}\{C\}$}, where $S$, $G$ and $C$ are the
tokens to delimit single, group and cluster of parameters respectively,
and change the delimiters back to the default by
{\tt\string\defaultmarks} or equivalently by
{\tt\string\delimiters$\{$;$\}\{$+$\}\{$!$\}$}.

\medskip


To conclude this section, let us finally look below at an `ideal' and
`finished' mail merge application code,. It is currently in a form
compilable under \LaTeX2e, and is also valid for plain \TeX\space if the
first three lines (lines 144-146) are removed.

\setrawcount
\documentclass{article}
\usepackage{formlett}  % LaTeX2e
\begin{document}

%%%%  MAKE file scr@tch@.let
\beginfile{scr@tch@.let} % letter content
\paranames <<name>>;<<address>>+<<items>>!
\loaddefaultparas
\parindent=0pt\blockparas\bigskip\bigskip
Dear \paras, \par\bigskip
This part is typically the letter content.
It now displays all the items in the 2nd
parameter group. They are \par
\blockparas[][2] \vfill\eject
\endfile

%%%%  MAKE file scr@tch@.adr
\beginfile{scr@tch@.adr} % address file
\blockmarks
\beginblockmode
T Teng
UMIST
Manchester M60 1QD
=====

Z Jiang
UNE
Armidale NSW 2351
Australia
------
Email: zhuhan@neumann.une.edu.au
=====
\endblockmode
\defaultmarks
\endfile

% MAIN BODY
\paradefaults To Whom It May Concern
  +No further info available{!}!
\inputletter{scr@tch@.let}
\inputfile{scr@tch@.adr}% for letters
\beginlabels
\inputfile{scr@tch@.adr}% for labels
\endlabels
\end{document}
\unsetraw
\rm

We note that if we replace lines 144-146 by {\tt\string\input\space
formlett.sty \string\initstyle$\{\}\{\}$}, then the mail merge can be
processed by either plain \TeX\space or \LaTeX\space (including
\LaTeX2e). It is likewise if {\tt\string\initstyle} in the replaced
line is replaced furthermore by {\tt\string\initclass}, except that now
the native \LaTeX2e environment is preserved, when applicable, rather
than turning it into the compatibility mode of \LaTeX2.09.

\bigskip
{\lbold Programmer's tips}

\smallskip

First we note that all the macro parameters that are to appear inside
square brackets of {\tt formlett}'s commands are defaulted, just like
how {\tt\string\paras[$m$][$n$]} is defaulted earlier on. The actual
default values can be found in the Appendix.

\medskip

For those wizard users who want to do everything their own way, we point
out that if for instance the 3rd letter parameter of the 2nd group of a
cluster is given as {\tt<A>}, then {\tt \string\LET2*3\t} will contain
{\tt \string\b@group \string\relax {\tt<A>}\string\e@group} right after
a cluster is read in. {\tt \string\DEF2*3\string~}, on the other hand,
contains the corresponding default parameter in the same fashion.
Furthermore, the command {\tt\string\checkparas[$m$][$n$]$\{$LET$\}$}
will copy the content of {\tt\string\paras[$m$][$n$]}, minus the
`wrapping' extra tokens {\tt\string\b@group\string\relax} and
{\tt\string\e@group}, to {\tt\string\cachedata} and set
{\tt\string\ifemptyparas} to true or false depending on whether the
content is empty or not. This way, a user may even change the
characteristics of his letter template by first testing the content of
the supplied individual parameters. However, we note that if {\tt\string
\loaddefaultparas} is executed, then the {\tt LET} array, when some of
its elements are not supplied, will contain the corresponding elements
of the {\tt DEF} array. Hence care must be exercised under such
circumstances, when interpreting the {\tt\string\cachedata} generated by
{\tt\string\checkparas[$m$][$n$]$\{$LET$\}$}. If necessary, we may use
{\tt\string\delparadefaults} to delete current default parameter array
{\tt DEF} so as to conduct {\tt\string\checkparas$\{$LET$\}$} more
precisely.

\medskip

There are also a number of generic macro utilities in {\tt formlett},
including a user stack and a multi-dimensional array mechanism.

\medskip

{\tt formlett} was written for all \TeX\space dialects, its format is
thus more close to that of plain \TeX. In fact, we deliberately avoided
mixing up {\tt formlett} with the standard letter environment of \LaTeX.
This is largely due to the fact that the limited facilities of
\LaTeX\space letter environment are easy to come by anyway, and are not
really worth writing buckled code to make {\tt formlett} a type of
extension of the environment. Nevertheless, one can still use the letter
environment of \LaTeX\space inside the form letter template. Besides,
the applicability of {\tt formlett} is not restricted for making form
{\sl letters}; it can also be used to merge other type of articles or
passages.




\bigskip
{\lbold References}

\smallskip
{\parindent=2em
\item{[1].} Knuth D E, {\sl The \TeX book}, Reading, Mass.,
Addison-Wesley, 1992. \par
\item{[2].} Piff M, Text merges in \TeX\ and \LaTeX, {\sl TUGboad},
13(4):518, 1993. \par
\item{[3].} Damrau J and Wester M, Form letters with 3-across
labels capacity, {\sl TUGboat}, 13(4):510, 1991.
\par
}




\penalty -50\bigskip

{\lbold Appendix}



\smallskip


In the followings, we give a brief summary of the main commands given by
{\tt formlett} version 2.1.

\medskip

Let $m$ and $n$ be numbers, $p$, $q$ and $r$ be dimensions, $A$,
$B$, $S$, $G$, $C$ and $T$ be tokens, and $X$ be a box.
Furthermore, we shall denote by $R$ a full set of letter
parameters ended by \lq{\tt !}\rq, with \lq{\tt;}\rqs separating
parameters inside a same group and \lq{\tt+}\rqs separating
different parameter groups. We moreover denote $R$ by $F$, when
\lq{\tt ; + !}\rqs there can be replaced by \lq$S$ $G$ $C$\rqs
respectively if {\tt \string\delimiters$\{S\}\{G\}\{C\}$} is
issued. In the commands tabulated below, the macro parameters
contained in squared brackets support default. In particular, the
defaults are $m$={\tt 1}, $n$={\tt 1}, $p$={\tt 8truecm},
$q$={\tt 1.5em}, $r$={\tt 3pt}, $A$={\tt\string\noindent} and
$B$={\tt\string\par}.



\newdimen\tempdimone
\newdimen\tempdimtwo
\tempdimone=\hsize
\advance\tempdimone by-0.2pt
\tempdimtwo=\hsize
\advance\tempdimtwo by-1.8in
\def\hruler{\hrule width \tempdimone}
\long\def\entry#1#2{%
\par\hbox to \tempdimone{\noindent
       \vrule~\textbox[1.7in]{#1}~\hss
       \vrule~\textbox[\tempdimtwo]{\raggedright#2}~\vrule}%
       \par}

\long\def\longentry#1#2#3{%
\par\hbox to \tempdimone{\noindent
       \vrule~\textbox[3in]{#1}~\hss~\vrule}\par
        {\kern-8pt \advance\tempdimtwo0.10in
         \vrule width1.7in height0pt \vrule width\tempdimtwo height 0.4pt \par
         \kern-3pt}
        \entry{#2}{#3}}

\long\def\Longentry#1#2#3{%
\par\hbox to \tempdimone{\noindent
       \vrule~\textbox[3in]{#1}~\hss~\vrule}\par
       {\advance\tempdimtwo0.08in\kern-1pt
        \vrule~\hfill\vrule width \tempdimtwo height 0.4pt \vrule\par
        \kern-2.5pt}%
        \entry{#2}{#3}}



\bigskip %\bigskip
\penalty-1000  \vskip 0pt plus4in

{\eightpoint
\nointerlineskip\parskip=0pt

\hruler
\entry{{\tt\string\paras[$m$][$n$]}}%
  {$m^{th}$ parameter of $n^{th}$ group }%

\entry{{\tt\string\blockparas[$m$][$n$][$A$][$B$]}}%
  {$m^{th}$ to the last parameter of $n^{th}$ group, each
preceded by $A$ and followed by $B$, wrapped by $\{$\penalty
1000$\}$ if
$B$={\tt\string\relax}}

\entry{{\tt\string\addressparas[$m$][$n$][$p$][$q$]}}%
  {$m^{th}$ to the last parameter of $n^{th}$ group, each put
into a box of width $p$ with indent $q$ for wrapped portions}


\entry{{\tt\string\loaddefaultparas}}%
  {fill empty parameters with defaults}

%\hruler \penalty-1000
\hruler


\entry{{\tt\string\checkparas[$m$][$n$]$\{T\}$}}%
  {$m^{th}$ parameter of $n^{th}$ group copied to {\tt\string\cachedata};
   {\tt\string\ifemptyparas} is true if element is empty; $T$ is often
   {\tt LET} or {\tt DEF}}%


\hruler


\entry{{\tt\string\moreletter\ $F$}}%
  {use parameters $F$ to output a new letter}


\entry{{\tt\string\paranames\ $R$}}%
  {use $R$ as parameter names}

\entry{{\tt\string\paradefaults\ $R$}}%
  {use $R$ as default parameters}

\entry{{\tt\string\delparadefaults}}%
  {delete default parameters}

\hruler


\entry{{\tt\string\delimiters$\{S\}\{G\}\{C\}$}}%
  {use $S$, $G$, $C$ as delimiters}


\entry{{\tt\string\defaultmarks}}%
  {use \lq{\tt ; + !}\rqs as delimiters}


\entry{{\tt\string\blockmarks}}%
  {use \lq{\tt ....}\rq, \lq{\tt ----}\rq, \lq{\tt ====}\rqs as delimiters}

%\hruler \penalty -1000
\hruler

\entry{{\tt\string\preview}}%
  {highlight parameter positions}

\entry{{\tt\string\showparas}}%
  {display parameter names, if any}

\hruler


\entry{%
$\left\{\matrix{ \hbox{\tt\string\beginfile[$T$]$\{{\it
file.ext}\}$}\cr
\hbox{\tt\string\endfile\ \ \ \ \ \ \ \ \ \ \ \ \ }\cr}\right.$\par
 \ \ \ {\tiny }}%
  {write text verbatim to file {\it file.ext} (empty implies {\tt
scr@tch@.tex}), nonempty $T$ replaces {\tt
\string\endfile} to mark last full line}



\entry{{\tt\string\inputletter$\{file.ext\}$}}%
  {input letter content}

\entry{{\tt\string\inputfile$\{file.ext\}$}}%
  {input $file.ext$}

\hruler \penalty -5000 %\vskip0pt plus 1cm
\hruler


\entry{%
$\left\{\matrix{ \hbox{\tt\string\beginletter}\cr
\hbox{\tt\string\endletter~~}\cr}\right.$ }%
  {delimiters for letter content (template)}

\entry{%
$\left\{\matrix{ \hbox{\tt\string\beginpilemode}\cr
\hbox{\tt\string\endpilemode~~}\cr}\right.$ }%
  {normal letter parameters cluster-wise}


\entry{%
$\left\{\matrix{ \hbox{\tt\string\beginblockmode}\cr
\hbox{\tt\string\endblockmode~~}\cr}\right.$ }%
  {for line-by-line blocks of parameters, empty lines active
within each cluster}


\entry{%
$\left\{\matrix{ \hbox{\tt\string\beginlinemode}\cr
\hbox{\tt\string\endlinemode~~}\cr}\right.$ }%
  {for line-by-line parameters, empty lines delimit clusters}

\hruler \penalty-1000
%\hruler

\entry{%
$\left\{\matrix{ \hbox{\tt\string\beginrawblockmode$\{T\}$}\cr
\hbox{\tt\string\endrawblockmode\ \ \ \ \ }\cr}\right.$ }%
  {raw text mode; nonempty $T$ replaces {\tt\string\endrawblockmode}
   to mark end}


\entry{%
$\left\{\matrix{ \hbox{\tt\string\beginrawlinemode$\{T\}$}\cr
\hbox{\tt\string\endrawlinemode\ \ \ \ \ \ }\cr}\right.$ }%
  {raw text parameters and active spaces etc}

\entry{%
$\left\{\matrix{ \hbox{\tt\string\begindatamode[$T$]$\{m\}$}\cr
\hbox{\tt\string\enddatamode\ \ \ \ \ \ \ \ }\cr}\right.$ }%
  {$m$ raw text lines for one form letter}


%\hruler \penalty-2000
\hruler


\entry{{\tt\string\PAGENO=1}}%
  {page number reset to 1}


\entry{{\tt\string\NOPAGENUMBERS}}%
  {no page numbers}


\entry{{\tt\string\textbox[$p$]$\{text\}$}}%
  {$text$ into box of width $p$}


\entry{{\tt\string\boxmore[$r$]$\{X\}$}}%
  {add borderline to box $X$ at a distance $r$}


\entry{{\tt\string\addressbox[$p$][$q$]$\{text\}$}}%
  {$text$ into box of width $p$, with wrapped options indented
by $q$}


\hruler

\Longentry{%
$\left\{\matrix{ \hbox{\tt\string\beginlabels
       [$a$][$b$][$c$][$d$][$e$][$f$]}\cr
 \hbox{\tt\string\endlabels\hskip3cm}\cr}\right.$ }%
 {\strut\par{\sl defaults:}\everypar={\hbox{\ \ \ }}\par\par
  $a$={\tt 20pt}, \par $b$={\tt\string\tt\string\raggedright},
 \par $c$={\tt1}, $d$={\tt1}, \par
  $e$={\tt2.6in},  $f$={\tt2em}}%
 {form letters become labels: address taken from $c^{th}$ to
last parameter of $d^{th}$ group, with width $e$, indent
$f$, borderspace $a$ and font toks $b$}

\hruler


\entry{{\tt\string\firstread$\{T\}$}}%
  {toks $T$ will not be read if the file is re-read via {\tt
   \string\labelsquit}}


\entry{{\tt\string\initstyle[$a$]$\{b\}\{c\}$}}%
  {initiation for {\tt\string\labelsquit},with styles $a$,
   documentstyle $b$ and preamble $c$}


\entry{{\tt\string\initclass[$a$][$o$]$\{b\}\{c\}$}}%
  {similar to {\tt\string\initstyle} ($o$ is \LaTeX2e options),
   but retains  native \LaTeX2e when applicable}


\hruler \vskip0.5pt
\Longentry{%
$\matrix{\hbox{\tt \string\labelsquit
       [$a$][$b$][$c$][$d$][$e$][$f$]$\{$\it file.ext$\}$}\cr
 \hbox{}\cr}$}%
 {\strut\par{\sl defaults:}\everypar={\hbox{\ \ \ }}\par\par
  see that for {\tt\string\beginlabels}}
  {quit after converting letters to labels by reading the current
   document or {\it file.ext}}

\hruler

}

\footline={\lower5pt
\hbox to \hsize
 {{\rm Jiang Z ~~~{\sl Mail merge with the use of formlett.sty}
  \hss {\sl $\bullet$End of Document$\bullet$}
\hss \folio}}}


\vrule height2.9in width0pt depth0pt
\endcolumns
\vfill\eject



\end{document}




