% $Id: manDCPiCpt.tex 13 2013-04-21 15:28:45Z pedro $
\documentclass[a4paper,11pt]{article}
\usepackage{a4wide}
\usepackage[portuguese]{babel}
\usepackage{ae} % Virtual fonts for PDF-files with T1 encoded CMR-fonts.
\usepackage[T1]{fontenc}
\usepackage[utf8x]{inputenc}
\usepackage{dcpic,pictex}
\usepackage{alltt}
\usepackage{verbatim}
\usepackage{longtable} 
\usepackage{svn-multi}
\usepackage{listings}
\usepackage{url}

\svnidlong
{$HeadURL: svn+ssh://gentzen.mat.uc.pt/var/lib/svn/DCPiC/CTAN5.0/manDCPiCpt.tex $}
{$LastChangedDate: 2013-05-01 19:49:49 +0100 (Qua, 01 Mai 2013) $}
{$LastChangedRevision: 15 $}
{$LastChangedBy: pedro $}

\svnid{$Id: manDCPiCpt.tex 13 2013-04-21 15:28:45Z pedro $}

\newcommand{\docversion}{\svnyear/\svnmonth/\svnday\ (v\svnrev)}

\def\dcpicversion{v5.0.0}

\lstset{language=TeX,
  frame = single,
  morekeywords={begindc,enddc,cmor,pup,commdiag,undigraph,digraph,cdigraph,cundigraph,obj,mor,pleft,pup,pdown,pright,north,northeast,east,southeast,south,southwest,west,northwest,atright,atleft,solidarrow,dashArrow,dotArrow,solidline,dashline,dotline,injectionarrow,aplicationarrow,surjectivearrow,equalline,doublearrow,doubleopposite,nullarrow},
  basicstyle=\scriptsize}


\title{DCpic (5.0) --- Manual de Utilização\\
\docversion}

\author{Pedro Quaresma\\ CISUC/Departamento de Matemática,
  Universidade de Coimbra\\ 3001-454 COIMBRA, PORTUGAL\\
    \url{pedro@mat.uc.pt}\quad phone: +351-239~791~137\quad fax:
  +351-239~832~568} 

\date{2013/05/01}

\begin{document}

\maketitle

\begin{abstract}
 O {\em DCpic\/} é um conjunto de comandos para a escrita de grafos,
 para tal desenvolveu-se um conjunto de comandos, com uma sintaxe
 simples, que permite a construção de quase todo o tipo de grafos.

 Originalmente o {\em DCpic\/} ({\bf D}iagramas {\bf C}omutativos
 utilizando o {\bf PiC}TeX) foi concebido para a construção de
 diagramas comutativos tal como são usados em Teoria das
 Categorias~\cite{Herrlich73,Pierce98}, temos então grafos etiquetados
 e com elementos nos nós. A partir da versão 4.0 o conjunto de
 comandos foi alterada de forma a considerar-se também a construção de
 grafos dirigidos, e grafos não dirigidos. A forma de os especificar
 recorre à colocação dos diferentes objectos (nós e arestas) num dado
 referencial ortonormado,

 O {\em DCpic} está baseado no \PiCTeX\ necessitando deste para poder
 ser usado.
\end{abstract}

\vfill
\begin{quotation}
This work may be distributed and/or modified under the
conditions of the LaTeX Project Public License, either version 1.3
of this license or (at your option) any later version.
The latest version of this license is in
  http://www.latex-project.org/lppl.txt
and version 1.3 or later is part of all distributions of LaTeX
version 2005/12/01 or later.
This work has the LPPL maintenance status `maintained'.

The Current Maintainer of this work is Pedro Quaresma (\url{pedro@mat.uc.pt}).

This work consists of the files dcpic.sty.
\end{quotation}

\vspace*{2cm}
\noindent Coimbra, 2013/04/21\\
Pedro Quaresma

\pagebreak
\section{Hist\'oria}

\begin{description}
\item[11/1990 - versão 1.0]
\item[10/1991 - versão 1.1]
\item[9/1993 - versão 1.2:] argumento ``distância entre as
  extremidades da seta e os objectos'' passou a ser opcional; uma nova
  opção para as ``setas'' (opção 3).
\item[2/3/1995 - versão 1.3:] foi acrescentado o tipo de seta de
  aplicação (opção 4) a distância da etiqueta
  à seta respectiva passou a ser fixa (10 unidades de medida).

\item[15/7/1996 - vers\~ao 2.1:] o comando {\tt mor} passou a ter uma
  sintaxe distinta. Os parâmetros 5 e 6 passaram a ser a
  distância entre os objectos e os extremos da seta o
  parâmetro 7 é o nome do morfismo e os parâmetros 8 e 9,
  colocação do morfismo e tipo de morfismo passaram a ser
  opcionais.

\item[5/2001 - versão 3.0:] implementação do comando
  {\tt cmor} baseado no comando de desenho de curvas quadráticas pelo
  \PiCTeX.

\item[11/2001 - versão 3.1:] modificação das pontas das
  setas de forma a estas ficarem semelhantes às setas
  (símbolos) dos TeX.

\item[1/2002 - versão 3.2:] modificação dos comandos {\tt obj}
  e {\tt mor} de forma a introduzir a especificação lógica
  dos morfismos, isto é, passa-se a dizer qual é o objecto de
  partida e/ou o objecto de chegada em vez de ter de especificar o
  morfismo em termos de coordenadas.  Por outro lado o tamanho das
  setas passa a ser ajustado automaticamente em relação ao
  tamanho dos objectos.

\item[5/2002 - vers\~ao 4.0:] {\bf vers\~ao incompatível com as
    anteriores}. Modificação dos comandos {\tt begindc} e {\tt obj}. O
  primeiro passou a ter um argumento (obrigat\'orio) que nos
  permite especificar o tipo de grafo que estamos a querer
  especificar:
  \begin{itemize}
  \item {\tt commdiag} (0), para diagramas comutativos;
  \item {\tt digraph} (1), para grafos orientados;
  \item {\tt undigraph} (2), para grafos n\~ao orientados.
  \end{itemize}

  O comando {\tt obj} modificou a sua sintaxe passou a ter um (após a
  especifica\c c\~ao das coordenadas, um argumento opcional, um argumento
  obrigat\'orio, e um argumento opcional. O primeiro argumento opcional
  d\'a-nos a etiqueta que serve como refer\^encia para a especifica\c c\~ao dos
  morfismos, na sua aus\^encia usa-se o argumento obrigat\'orio para esse
  efeito, o argumento obrigat\'orio d\'a-nos o ``conte\'udo'' do objecto, nos
  diagramas comutativos \'e centrado no ponto dado pelas coordenadas
  sendo o argumento seguinte simplesmente ignorado, nos grafos o
  ``conte\'udo'' \'e colocado numa posi\c c\~ao a norte, a noroeste, a este, \ldots,
  sendo que a posi\c c\~ao concreta \'e especificada pelo \'ultimo dos
  argumentos deste comando, o valor por omiss\~ao \'e o {\tt norte}.

\item[3/2003 - vers\~ao 4.1:] a pedido de Jon Barker \url{<jeb1@soton.ac.uk>}
  criei um novo tipo de seta, a seta de sobrejec\c c\~ao.  Para j\'a a dupla
  seta s\'o fica bem nas setas horizontais ou verticais.

\item[12/2004 - vers\~ao 4.1.1:] nova vers\~ao das setas de sobrejec\c c\~ao que
  corrigue completamente os problemas da solu\c c\~ao anterior.

\item[3/2007 - vers\~ao 4.2:] acrescenta a directiva
  ``providespackage''. Acrescenta linhas a ponteado e a tracejado.

\item[5/2008 - vers\~ao 4.2.1:] apaga alguns contadores para tentar
  diminuir o excessivo uso dos mesmos por parte do PiCTeX.

\item[8/2008 - vers\~ao 4.3:] gra\c cas a Ruben Debeerst
  \url{<debeerst@mathematik.uni-kassel.de>}, acrescentei uma nova
  ``seta'' a ``equalline''. Ap\'os isso decidi tamb\'em acrescentar
  setas duplas, com o mesmo ou diferentes sentidos. Acrescentou-se
  também a seta nula, isto é, sem representação gráfica, a qual pode
  ser usada para acrescentar etiquetas a outras ``setas''.

\item[12/2008 - version 4.3.1:] para evitar conflitos com outros
  pacotes o comando ``id'' \'e internalizado. O comando ``dasharrow'' \'e
  modificado para ``dashArrow'' para evitar um conflito com o AMSTeX.

\item[12/2009 - version 4.3.2:] para evitar um conflito com o pacote
  ``hyperref'' mudou-se o contador ``d'' para ``deuc'', aproveitei e 
  mudei os contadores ``x'' e ``y'' para ``xO'' e ``yO'' 

\item[4/2013 - version 4.4.0:] graças a Xingliang Liang
  \url{jkl9543@gmail.com>} acrescentou-se uma nova seta ``dotarrow''.

\item[4/2013 - version 5.0:] {\bf uma nova unidade para o sistema de
    coordenadas}, 1/10 da anterior. Esta nova unidade permite
  corriguir um problema com a construção das setas duplas, além de
  permitir uma especificação mais fina dos diagramas.
\end{description}

\section{Introdução} 

O conjunto de comandos {\em DCpic} é um conjunto de comandos
\TeX~\cite{Knuth86} dedicado à escrita de diagramas tal como são
usados em Teoria das Categorias~\cite{Herrlich73,Pierce98}, assim como
de grafos dirigidos e não dirigidos~\cite{Harary72}.

Pretendeu-se com a sua escrita ter uma forma simples de especificar
grafos, fazendo-o através da especificação de um conjunto de
``objectos'' (nós do grafo) colocados num dado referencial
ortonormado, e através de um conjuntos de morfismos (arestas) que os
são posicionados explicitamente no referido referencial, ou então,
a são posição é dada especificando qual é o seu nó de
partida e qual é o seu nó de chegada.

O gráfico em si é construído recorrendo aos comandos gráficos
do \PiCTeX.

\section{Utilização}

Antes de mais é necessário carregar os dois conjuntos de comandos
acima referidos, no caso de um documento \LaTeX~\cite{Lamport94} isso
pode ser feito com o seguinte comando (no preâmbulo).

\begin{verbatim}
\usepackage{dcpic,pictex}
\end{verbatim}

Nos outros formatos ter-se-á de usar um comando equivalente. Após
isso os diagramas podem ser escritos através dos comandos
disponibilizados pelo {\em DCpic}.  Por exemplo, os comandos:


\begin{lstlisting}
\begindc{\commdiag}[200]
 \obj(1,4){$A^B$}
 \obj(1,1){$C$}
 \obj(3,4){$A$}
 \obj(3,1){$C\times{}B$}
 \obj(6,4){$A^B\times{}B$}
 \mor{$C$}{$A^B$}{$f$}
 \mor{$C\times{}B$}{$A$}{$\bar f$}[\atleft,\dashArrow]
 \mor{$A^B\times{}B$}{$A$}{$ev$}[\atright,\solidarrow]
 \mor{$C\times{}B$}{$A^B\times{}B$}{$f\times{}id$}[\atright,\solidarrow]
\enddc
\end{lstlisting}

produzem o seguinte diagrama:

$$
\begindc{\commdiag}[200]
 \obj(1,4){$A^B$}
 \obj(1,1){$C$}
 \obj(3,4){$A$}
 \obj(3,1){$C\times{}B$}
 \obj(6,4){$A^B\times{}B$}
 \mor{$C$}{$A^B$}{$f$}
 \mor{$C\times{}B$}{$A$}{$\bar f$}[\atleft,\dashArrow]
 \mor{$A^B\times{}B$}{$A$}{$ev$}[\atright,\solidarrow]
 \mor{$C\times{}B$}{$A^B\times{}B$}{$f\times{}id$}[\atright,\solidarrow]
\enddc
$$

O meio ambiente {\tt begindc}, {\tt enddc} permite-nos construir um
grafo por colocação dos objectos num referencial ortonormado tendo
a origem em (0,0). As arestas (morfismos) vão ligar pares de nós
(objectos) entre si.

\section{Comandos Disponíveis}

De seguida apresenta-se a descrição dos comandos, a sua sintaxe e
a sua funcionalidade. Os argumentos entre parêntesis rectos são
opcionais.

\begin{description}
\item[\tt $\backslash$begindc\{\#1\}[\#2\mbox{]}] -- entrada no ambiente de
  escrita de grafos:

  \begin{tabular}{r@{ -- }l}
    {\tt \#1} & tipo de grafo\\
    \multicolumn{2}{l}{\quad $0\equiv\backslash$commdiag, diagrama comutativo;}\\
    \multicolumn{2}{l}{\quad $1\equiv\backslash$digraph, grafo orientado;}\\
    \multicolumn{2}{l}{\quad $2\equiv\backslash$undigraph, grafo não orientado;}\\
    \multicolumn{2}{l}{\quad $3\equiv\backslash$cdigraph, grafo orientado, com
      objectos circunscritos;}\\
    \multicolumn{2}{l}{\quad $4\equiv\backslash$cundigraph, grafo não orientado,
      com objectos circunscritos.}\\
    {\tt \#2} & factor de escala (opcional)\\
    \multicolumn{2}{l}{\quad valor por omissão: 300}
  \end{tabular}
  
\item[\tt $\backslash$enddc] -- saída do meio ambiente para a
  escrita de grafos.  

\item[{\tt $\backslash$obj(\#1,\#2)[\#3]\{\#4\}[\#5]}:] comando de colocação
  dos nós (objectos).  

  \begin{tabular}{r@{ -- }l}
    {\tt \#1} e  {\tt \#2}& coordenadas do centro da caixa que vai
    conter o texto\\
    {\tt \#3} & etiqueta para identificar o objecto (opcional)\\
    {\tt \#4} & texto (conte{\'u}do do nó)\\
    {\tt \#5} & colocação relativa do objecto (opcional)\\
    \multicolumn{2}{l}{\qquad\quad $0\doteq\backslash$pcent, centrado}\\
    \multicolumn{2}{l}{\qquad\quad $1\doteq\backslash$north, norte}\\
    \multicolumn{2}{l}{\qquad\quad $2\doteq\backslash$northeast, nordeste}\\
    \multicolumn{2}{l}{\qquad\quad $3\doteq\backslash$east, este}\\
    \multicolumn{2}{l}{\qquad\quad $4\doteq\backslash$southeast, sudeste}\\
    \multicolumn{2}{l}{\qquad\quad $5\doteq\backslash$south, sul}\\
    \multicolumn{2}{l}{\qquad\quad $6\doteq\backslash$southwest, sudoeste}\\
    \multicolumn{2}{l}{\qquad\quad $7\doteq\backslash$west, oeste}\\
    \multicolumn{2}{l}{\qquad\quad $8\doteq\backslash$northwest, noroeste}
  \end{tabular}

A etiqueta explícita-se quando não é possível usar o objecto como
forma de identificação do nó, por exemplo num dado grafo não orientado
os nós podem não ter conteúdo e como tal serem todos iguais em termos
de identificação:

Em alguns casos, por exemplo comandos dos \LaTeX\ complexos, pode ser
necessário explicitar o argumento {\tt \#3} mesmo que seja através da
etiqueta vazia {\tt []}. Esse especificar da etiqueta vazia torna-se
necessário para que o mecanismo interno do DCpic de comunicação entre
comandos (pilhas) não se baralhe e entre num ciclo infinito.


$$ 
\begindc{\undigraph}[200]
\obj(1,1)[1]{}
\obj(3,2)[2]{}
\obj(5,1)[3]{}
\obj(3,4)[4]{}
\mor{1}{2}{}
\mor{1}{3}{}
\mor{2}{3}{}
\mor{4}{1}{}
\mor{4}{3}{}
\mor{2}{4}{}
\enddc
$$
foi produzido por:
\begin{lstlisting}
\begindc{\undigraph}[200]
\obj(1,1)[1]{}
\obj(3,2)[2]{}
\obj(5,1)[3]{}
\obj(3,4)[4]{}
\mor{1}{2}{}
\mor{1}{3}{}
\mor{2}{3}{}
\mor{4}{1}{}
\mor{4}{3}{}
\mor{2}{4}{}
\enddc
\end{lstlisting}


O parâmetro referente à colocação do objecto só é relevante quando se
pensa na identificação dos nós num dado grafo orientado (ou não), por
exemplo o grafo ``Around the Word''~\cite{Harary72}:

$$
\begindc{\undigraph}[70]
\obj(6,4){18}[\south]
\obj(18,4){17}[\south]
\obj(8,7){11}[\west]
\obj(12,8){12}[\south]
\obj(16,7){13}[\east]
\obj(8,11){10}[\west]
\obj(10,12){6}[\northwest]
\obj(12,10){5}
\obj(14,12){4}[\northeast]
\obj(16,11){14}[\east]
\obj(2,16){19}
\obj(6,15){9}
\obj(9,16){8}
\obj(11,14){7}
\obj(13,14){3}
\obj(15,16){2}
\obj(18,15){15}
\obj(22,16){16}
\obj(12,19){1}[\northeast]
\obj(12,22){20}
\mor{18}{17}{}\mor{18}{11}{}\mor{18}{19}{}
\mor{11}{12}{}\mor{11}{10}{}\mor{12}{13}{}
\mor{12}{5}{}\mor{10}{6}{}\mor{10}{9}{}
\mor{5}{6}{}\mor{5}{4}{}\mor{13}{17}{}
\mor{13}{14}{}\mor{9}{19}{}\mor{9}{8}{}
\mor{6}{7}{}\mor{4}{3}{}\mor{4}{14}{}
\mor{19}{20}{}\mor{8}{1}{}\mor{8}{7}{}
\mor{7}{3}{}\mor{3}{2}{}\mor{2}{1}{}
\mor{2}{15}{}\mor{14}{15}{}\mor{17}{16}{}
\mor{16}{20}{}\mor{1}{20}{}\mor{15}{16}{}
\enddc
$$
foi produzido por 
\begin{lstlisting}
\begindc{\undigraph}[70]
\obj(6,4){18}[\south]
\obj(18,4){17}[\south]
\obj(8,7){11}[\west]
\obj(12,8){12}[\south]
\obj(16,7){13}[\east]
\obj(8,11){10}[\west]
\obj(10,12){6}[\northwest]
\obj(12,10){5}
\obj(14,12){4}[\northeast]
\obj(16,11){14}[\east]
\obj(2,16){19}
\obj(6,15){9}
\obj(9,16){8}
\obj(11,14){7}
\obj(13,14){3}
\obj(15,16){2}
\obj(18,15){15}
\obj(22,16){16}
\obj(12,19){1}[\northeast]
\obj(12,22){20}
\mor{18}{17}{}\mor{18}{11}{}\mor{18}{19}{}
\mor{11}{12}{}\mor{11}{10}{}\mor{12}{13}{}
\mor{12}{5}{}\mor{10}{6}{}\mor{10}{9}{}
\mor{5}{6}{}\mor{5}{4}{}\mor{13}{17}{}
\mor{13}{14}{}\mor{9}{19}{}\mor{9}{8}{}
\mor{6}{7}{}\mor{4}{3}{}\mor{4}{14}{}
\mor{19}{20}{}\mor{8}{1}{}\mor{8}{7}{}
\mor{7}{3}{}\mor{3}{2}{}\mor{2}{1}{}
\mor{2}{15}{}\mor{14}{15}{}\mor{17}{16}{}
\mor{16}{20}{}\mor{1}{20}{}\mor{15}{16}{}
\enddc
\end{lstlisting}



\item[{\tt $\backslash$mor\{\#1\}\{\#2\}[\#5,\#6]\{\#7\}[\#8,\#9]}:]
  Comando de colocação da seta (morfismo) de ligação de dois objectos
  -- Primeira variante.

  A numeração errada dos argumentos é aqui feita propositadamente,
  aquando da explicação da segunda variante deste comando
  compreender-se-á o porquê desta opção de escrita.

  \begin{longtable}{r@{ -- }p{32em}}
    {\tt \#1} & referência do nó de partida\\
    {\tt \#2} & referência do nó de chegada\\
    {\tt \#5} e  {\tt \#6} & distância do centro dos objectos às
    extremidades inicial e final respectivamente da seta. Valores por
    omissão: 10, 10 (para diagramas) 2, 2 (para os grafos)\\
    {\tt \#7} & texto, ``nome'' do morfismo\\
    {\tt \#8} & colocação do nome do morfismo em relação à seta. Valor
    por omissão, $\backslash$atleft. \\
    \multicolumn{2}{l}{\hspace*{4.5em} 1 $\doteq\backslash$atright, à direita}\\ 
    \multicolumn{2}{l}{\hspace*{4.5em} -1 $\doteq\backslash$atleft, à esquerda}\\
    {\tt \#9} & tipo da seta. Valor por omissão, $\backslash$solidarrow.\\
    \multicolumn{2}{l}{\hspace*{4.5em} 0 $\doteq\backslash$solidarrow, seta sólida} \\
    \multicolumn{2}{l}{\hspace*{4.5em} 1 $\doteq\backslash$dashArrow, seta tracejada} \\
    \multicolumn{2}{l}{\hspace*{4.5em} 2 $\doteq\backslash$dotArrow, seta ponteada} \\
    \multicolumn{2}{l}{\hspace*{4.5em} 3 $\doteq\backslash$solidline, linha sólida} \\
    \multicolumn{2}{l}{\hspace*{4.5em} 4 $\doteq\backslash$dashline, linha a tracejado} \\
    \multicolumn{2}{l}{\hspace*{4.5em} 5 $\doteq\backslash$dotline, linha a ponteado} \\
    \multicolumn{2}{l}{\hspace*{4.5em} 6 $\doteq\backslash$injectionarrow, seta de
      injecção. Valor anterior 3 (vers\~ao $<$ 4.2) } \\
    \multicolumn{2}{l}{\hspace*{4.5em} 7 $\doteq\backslash$aplicationarrow, seta de
      aplicação. Valor anterior 4 (vers\~ao $<$ 4.2) } \\ 
    \multicolumn{2}{l}{\hspace*{4.5em} 8 $\doteq\backslash$surjectivearrow, seta de
      função sobrejectiva. Valor anterior 5 (vers\~ao $<$ 4.2) }\\ 
    \multicolumn{2}{l}{\hspace*{4.5em} 9 $\doteq\backslash$equalline, linha dupla}\\ 
    \multicolumn{2}{l}{\hspace*{4.5em} 10
      $\doteq\backslash$doublearrow, seta dupla}\\ 
    \multicolumn{2}{l}{\hspace*{4.5em} 11
      $\doteq\backslash$doubleopposite, seta dupla em sentidos opostos}\\ 
    \multicolumn{2}{l}{\hspace*{4.5em} 12
      $\doteq\backslash$nullarrow, seta nula, serve o propósito de
      acrescentar etiquetas as outras ``setas''.}\\ 
    \end{longtable}

\item[{\tt $\backslash$mor(\#1,\#2)(\#3,\#4)[\#5,\#6]\{\#7\}[\#8,\#9]}:]
  Comando de colocação da seta (morfismo) de ligação de dois objectos
  -- Segunda variante.

  \begin{longtable}{r@{ -- }p{32em}}
    {\tt \#1} e  {\tt \#2} & coordenadas do nó de partida\\
    {\tt \#3} e  {\tt \#4} & coordenadas do nó de chegada\\
  \end{longtable}

  Todos os outros argumentos têm o significado já explicado (por
  isso a numeração errada). É de notar que para a primeira variante é
  feito o cálculo das coordenadas dos nós de forma automática e depois
  são passados esses valores para a segunda variante do comando.


\item[{\tt $\backslash$cmor(\#1) \#2(\#3,\#4)\{\#5\}[\#6]}] comando para a
  especificação de setas curvas. O algoritmo de construção das setas é
  o do \PiCTeX\ o que implica que se está a especificar uma linha
  quadrática através de um número ímpar de pontos. 

  \begin{tabular}{c@{---}l}
    {\tt \#1} & lista de pontos, em número ímpar \\
    {\tt \#2} & direccionamento da seta \\
    \multicolumn{2}{l}{\qquad 0 $\doteq\backslash$pup, apontar para cima}\\ 
    \multicolumn{2}{l}{\qquad 1 $\doteq \backslash$pdown, apontar para
      baixo}\\
    \multicolumn{2}{l}{\qquad 2 $\doteq \backslash$pright, apontar para a
      direita}\\
    \multicolumn{2}{l}{\qquad 3 $\doteq \backslash$pleft, apontar para a
      esquerda}\\ 
    {\tt \#3} & abcissa do morfismo\\
    {\tt \#4} & ordenada do morfismo\\
    {\tt \#5} & morfismo\\
    {\tt \#6} & tipo de ``seta'', valor por omissão: 0, seta
    sólida.\\
    \multicolumn{2}{l}{\hspace*{4.5em} Os restantes valores poss\'\i veis
      s\~ao os descritos na variante anterior.}\\  

 \end{tabular}

  O comando {\tt cmor} no caso em que não tem o último parâmetro opcional
  tem de ser seguido por um espaço. O espaço antes do direccionamento
  da seta é obrigatório.

  No caso de se ter o valor 2 (``$\backslash$solidline'') o valor para o
  direccionamento da seta não é tipo em conta, no entanto dado se
  tratar de um do parâmetro obrigatório é necessário dar-lhe um valor
\end{description}


\section{Alguns Exemplos}


\subsection{Setas Duplas, Transformações Naturais, \ldots}

É de notar que alguns casos aparentemente omissos na actual versão
podem perfeitamente ser construídos através de uma utilização
imaginativa dos actuais comandos. Por exemplo os seguintes diagramas: 

$$
\begindc{\commdiag}[30]
\obj(5,5){$A$}
\obj(20,5){$B$}
\mor{$A$}{$B$}{$f$}[\atright,\doublearrow]
\mor{$A$}{$B$}{$g$}[\atleft,\nullarrow]
\enddc
\qquad\qquad\qquad
\begindc{\commdiag}[140]
\obj(5,5){$A$}
\obj(9,5){$B$}
\mor(5,6)(9,6){$\downarrow\sigma$}[\atright,\solidarrow]
\mor{$A$}{$B$}{}
\mor(5,4)(9,4){$\downarrow\tau$}
\enddc
$$

Podem ser construídos com a actual versão. Eis como:


\begin{lstlisting}
\begindc{\commdiag}[30]
\obj(5,5){$A$}
\obj(20,5){$B$}
\mor{$A$}{$B$}{$f$}[\atright,\doublearrow]
\mor{$A$}{$B$}{$g$}[\atleft,\nullarrow]

\begindc{\commdiag}[14]
\obj(5,5){$A$}
\obj(9,5){$B$}
\mor(5,6)(9,6){$\downarrow\sigma$}[\atright,\solidarrow]
\mor{$A$}{$B$}{}
\mor(5,4)(9,4){$\downarrow\tau$}
\enddc
\end{lstlisting}


\subsection{Grafos Orientados com Objectos Circunscritos}


$$
\begindc{\cdigraph}[250]
\obj(1,5){A}
\obj(1,4){B}
\obj(1,1){C}
\obj(5,5){E}
\obj(5,3){F}
\obj(5,1){G}
\mor{A}{E}[80,80]{5}
\mor{A}{F}[80,80]{3}
\mor{B}{F}[80,80]{6}[\atright,\solidarrow]
\mor{B}{E}[80,80]{1}
\mor{C}{F}[80,80]{5}
\mor{C}{G}[80,80]{7}
\enddc
$$

Foi produzido através dos seguintes comandos:


\begin{lstlisting}
\begindc{\commdiag}[250]
\obj(1,5){A}
\obj(1,4){B}
\obj(1,1){C}
\obj(5,5){E}
\obj(5,3){F}
\obj(5,1){G}
\mor{A}{E}[80,80]{5}
\mor{A}{F}[80,80]{3}
\mor{B}{F}[80,80]{6}[\atright,\solidarrow]
\mor{B}{E}[80,80]{1}
\mor{C}{F}[80,80]{5}
\mor{C}{G}[80,80]{7}
\enddc
\end{lstlisting}

\subsection{Diferentes Tipos de Setas/Linhas}


\begin{lstlisting}
\begindc{\commdiag}[250]
\obj(10,10)[A]{$OOOOOO$}
\obj(15,10)[A0]{$A_0$}
\obj(14,11)[A1]{$A_1$}
\obj(13,12)[A2]{$A_2$}
\obj(12,13)[A3]{$A_3$}
\obj(10,14)[A4]{$A_4$}
\obj(9,13)[A5]{$A_5$}
\obj(8,12)[A6]{$A_6$}
\obj(7,11)[A7]{$A_7$}
\obj(6,10)[A8]{$A_8$}
\obj(7,9)[A9]{$A_9$}
\obj(9,8)[A10]{$A_{10}$}
\obj(12,8)[A11]{$A_{11}$}
\mor{A}{A0}{$a_0$}[\atright,\solidarrow]
\mor{A}{A1}{$a_1$}[\atright,\dashArrow]
\mor{A}{A2}{$a_2$}[\atright,\dotArrow]
\mor{A}{A3}{$a_3$}[\atright,\solidline]
\mor{A}{A4}{$a_4$}[\atright,\dashline]
\mor{A}{A5}{$a_5$}[\atleft,\dotline]
\mor{A}{A6}{$a_6$}[\atleft,\injectionarrow]
\mor{A}{A7}{$a_7$}[\atleft,\aplicationarrow]
\mor{A}{A8}{$a_8$}[\atleft,\surjectivearrow]
\mor{A}{A9}{$a_9$}[\atleft,\equalline]
\mor{A}{A10}{$a_{10}$}[\atleft,\doublearrow]
\mor{A}{A11}{$a_{11}$}[\atleft,\doubleopposite]
\mor{A}{A11}{$a_{12}$}[\atright,\nullarrow]
\enddc
\end{lstlisting}

$$
\begindc{\commdiag}[250]
\obj(10,10)[A]{$OOOOOO$}
\obj(15,10)[A0]{$A_0$}
\obj(14,11)[A1]{$A_1$}
\obj(13,12)[A2]{$A_2$}
\obj(12,13)[A3]{$A_3$}
\obj(10,14)[A4]{$A_4$}
\obj(9,13)[A5]{$A_5$}
\obj(8,12)[A6]{$A_6$}
\obj(7,11)[A7]{$A_7$}
\obj(6,10)[A8]{$A_8$}
\obj(7,9)[A9]{$A_9$}
\obj(9,8)[A10]{$A_{10}$}
\obj(12,8)[A11]{$A_{11}$}
\mor{A}{A0}{$a_0$}[\atright,\solidarrow]
\mor{A}{A1}{$a_1$}[\atright,\dashArrow]
\mor{A}{A2}{$a_2$}[\atright,\dotArrow]
\mor{A}{A3}{$a_3$}[\atright,\solidline]
\mor{A}{A4}{$a_4$}[\atright,\dashline]
\mor{A}{A5}{$a_5$}[\atleft,\dotline]
\mor{A}{A6}{$a_6$}[\atleft,\injectionarrow]
\mor{A}{A7}{$a_7$}[\atleft,\aplicationarrow]
\mor{A}{A8}{$a_8$}[\atleft,\surjectivearrow]
\mor{A}{A9}{$a_9$}[\atleft,\equalline]
\mor{A}{A10}{$a_{10}$}[\atleft,\doublearrow]
\mor{A}{A11}{$a_{11}$}[\atleft,\doubleopposite]
\mor{A}{A11}{$a_{12}$}[\atright,\nullarrow]
\enddc
$$

\subsection{Diagramas com Setas Curvas}



\begin{lstlisting}
\begindc{\commdiag}[30]
\obj(14,11){$A$}
\obj(39,11){$B$}
\mor(14,12)(39,12){$f$}
\mor(39,10)(14,10){$g$}
\cmor((10,10)(6,11)(5,15)(6,19)(10,20)(14,19)(15,15)) 
  \pdown(2,20){$id_A$}
\cmor((40,7)(41,3)(45,2)(49,3)(50,7)(49,11)(45,12)) 
  \pleft(54,3){$id_B$}
\enddc

\begindc{\commdiag}[30]
\obj(10,15)[A]{$A$}
\obj(40,15)[Aa]{$A$}
\obj(25,15)[B]{$B$}
\mor{A}{B}{$f$}
\mor{B}{Aa}{$g$}
\cmor((10,11)(11,7)(15,6)(25,6)(35,6)(39,7)(40,11)) 
  \pup(25,3){$id_A$}
\enddc
\end{lstlisting}

$$
\begindc{\commdiag}[30]
\obj(14,11){$A$}
\obj(39,11){$B$}
\mor(14,12)(39,12){$f$}%[\atright,\solidarrow]
\mor(39,10)(14,10){$g$}%[\atright,\solidarrow]
\cmor((10,10)(6,11)(5,15)(6,19)(10,20)(14,19)(15,15)) 
  \pdown(2,20){$id_A$}
\cmor((40,7)(41,3)(45,2)(49,3)(50,7)(49,11)(45,12)) 
  \pleft(54,3){$id_B$}
\enddc
\qquad
\begindc{\commdiag}[30]
\obj(10,15)[A]{$A$}
\obj(40,15)[Aa]{$A$}
\obj(25,15)[B]{$B$}
\mor{A}{B}{$f$}%[\atright,\solidarrow]
\mor{B}{Aa}{$g$}%[\atright,\solidarrow]
\cmor((10,11)(11,7)(15,6)(25,6)(35,6)(39,7)(40,11)) 
  \pup(25,3){$id_A$}
\enddc
$$

\subsection{Um Exemplo Complexo}

O diagrama seguinte foi proposto por Feruglio~\cite{Feruglio94} como
um caso de teste. Como \'e poss\'\i vel ver o DCpic produz o diagrama
correctamente a partir de uma especifica\c c\~ao simples.


\begin{lstlisting}
\newcommand{\barraA}{\vrule height2em width0em depth0em}
\newcommand{\barraB}{\vrule height1.6em width0em depth0em}
\begindc{\commdiag}[350]
\obj(1,1)[Gr]{$G$}
\obj(3,1)[Grstar]{$G_{r^*}$}
\obj(5,1)[H]{$H$}
\obj(2,2)[SigmaG]{$\Sigma^G$}
\obj(6,2)[SigmaH]{$\Sigma^H$}
\obj(1,3)[Lm]{$L_m$}
\obj(3,3)[Krm]{$K_{r,m}$}
\obj(5,3)[Rmstar]{$R_{m^*}$}
\obj(1,5)[L]{$L$}
\obj(3,5)[Lr]{$L_r$}
\obj(5,5)[R]{$R$}
\obj(2,6)[SigmaL]{$\Sigma^L$}
\obj(6,6)[SigmaR]{$\Sigma^R$}
\mor{Gr}{SigmaG}{$\lambda^G$}
\mor{Grstar}{Gr}{$i_5$}[\atleft,\aplicationarrow]
\mor{Grstar}{H}{$r^*$}[\atright,\solidarrow]
\mor{H}{SigmaH}{$\lambda^H$}[\atright,\dashArrow]
\mor{SigmaG}{SigmaH}{$\varphi^{r^*}$}[\atleft,\solidarrow]
\mor{Lm}{Gr}{$m$}[\atright,\solidarrow]
\mor{Lm}{L}{$i_2$}[\atleft,\aplicationarrow]
\mor{Krm}{Lm}{$i_3\quad$}[\atright,\aplicationarrow]
\mor{Krm}{Rmstar}{$r$}
\mor{Krm}{Lr}{$i_4$}[\atright,\aplicationarrow]
\mor{Krm}{Grstar}{\barraA$m$}
\mor{Rmstar}{R}{$i_6$}[\atright,\aplicationarrow]
\mor{Rmstar}{H}{\barraB$m^*$}
\mor{L}{SigmaL}{$\lambda^L$}
\mor{Lr}{L}{$i_1\quad$}[\atright,\aplicationarrow]
\mor{Lr}{R}{$r$}
\mor{R}{SigmaR}{$\lambda^R$}[\atright,\solidarrow]
\mor{SigmaL}{SigmaG}{$\varphi^m$}[\atright,\solidarrow]
\mor{SigmaL}{SigmaR}{$\varphi^r$}
\mor{SigmaR}{SigmaH}{$\varphi^{m^*}$}
\enddc
\end{lstlisting}

\newcommand{\barraA}{\vrule height2em width0em depth0em}
\newcommand{\barraB}{\vrule height1.6em width0em depth0em}
$$\begindc{\commdiag}[350]
\obj(1,1)[Gr]{$G$}
\obj(3,1)[Grstar]{$G_{r^*}$}
\obj(5,1)[H]{$H$}
\obj(2,2)[SigmaG]{$\Sigma^G$}
\obj(6,2)[SigmaH]{$\Sigma^H$}
\obj(1,3)[Lm]{$L_m$}
\obj(3,3)[Krm]{$K_{r,m}$}
\obj(5,3)[Rmstar]{$R_{m^*}$}
\obj(1,5)[L]{$L$}
\obj(3,5)[Lr]{$L_r$}
\obj(5,5)[R]{$R$}
\obj(2,6)[SigmaL]{$\Sigma^L$}
\obj(6,6)[SigmaR]{$\Sigma^R$}
\mor{Gr}{SigmaG}{$\lambda^G$}
\mor{Grstar}{Gr}{$i_5$}[\atleft,\aplicationarrow]
\mor{Grstar}{H}{$r^*$}[\atright,\solidarrow]
\mor{H}{SigmaH}{$\lambda^H$}[\atright,\dashArrow]
\mor{SigmaG}{SigmaH}{$\varphi^{r^*}$}[\atleft,\solidarrow]
\mor{Lm}{Gr}{$m$}[\atright,\solidarrow]
\mor{Lm}{L}{$i_2$}[\atleft,\aplicationarrow]
\mor{Krm}{Lm}{$i_3\quad$}[\atright,\aplicationarrow]
\mor{Krm}{Rmstar}{$r$}
\mor{Krm}{Lr}{$i_4$}[\atright,\aplicationarrow]
\mor{Krm}{Grstar}{\barraA$m$}
\mor{Rmstar}{R}{$i_6$}[\atright,\aplicationarrow]
\mor{Rmstar}{H}{\barraB$m^*$}
\mor{L}{SigmaL}{$\lambda^L$}
\mor{Lr}{L}{$i_1\quad$}[\atright,\aplicationarrow]
\mor{Lr}{R}{$r$}
\mor{R}{SigmaR}{$\lambda^R$}[\atright,\solidarrow]
\mor{SigmaL}{SigmaG}{$\varphi^m$}[\atright,\solidarrow]
\mor{SigmaL}{SigmaR}{$\varphi^r$}
\mor{SigmaR}{SigmaH}{$\varphi^{m^*}$}
\enddc
$$


%\bibliographystyle{plain}
%\bibliography{pedro}

\newcommand{\noopsort}[1]{} \newcommand{\singleletter}[1]{#1}
\begin{thebibliography}{1}

\bibitem{Feruglio94}
{Gabriel Valiente} Feruglio.
\newblock Typesetting commutative diagrams.
\newblock {\em TUGboat}, 15(4):466--484, 1994.

\bibitem{Harary72}
Frank Harary.
\newblock {\em Graph Theory}.
\newblock Addison-Wesley, Reading, Massachusetts, 1972.

\bibitem{Herrlich73}
Horst Herrlich and George Strecker.
\newblock {\em Category Theory}.
\newblock Allyn and Bacon Inc., 1973.

\bibitem{Knuth86}
Donald~E. Knuth.
\newblock {\em The {\TeX}book}.
\newblock Addison-Wesley Publishing Company, Reading, Massachusetts, 1986.

\bibitem{Lamport94}
Leslie Lamport.
\newblock {\em {\LaTeX}: A Document Preparation System}.
\newblock Addison-Wesley Publishing Company, Reading,Massachusetts, 2nd
  edition, 1994.

\bibitem{Pierce98}
Benjamin Pierce.
\newblock {\em Basic Category Theory for Computer Scientists}.
\newblock Foundations of Computing. The MIT Press, London, England, 1998.

\end{thebibliography}

\appendix
\section{O Código}

{\scriptsize
\verbatiminput{dcpic.sty}}

\end{document}

%%% Local Variables: 
%%% mode: latex
%%% TeX-master: t
%%% End: 
