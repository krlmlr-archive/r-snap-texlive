% This file is part of the Petri-nets packages. See file README for
% copyright notice.

\documentclass[11pt]{article}

\usepackage{pnets}

\catcode`\|=\active
\def|{\verb|}

%%
%%

\begin{document}

\title{Petri-nets packages}
\author{Franck Pommereau (pommereau@univ-paris12.fr)}
\date{Last update: \pnversion}
\maketitle

\begin{abstract}\noindent
This paper describes Petri-nets, a set of \TeX/\LaTeX{} packages about
Petri nets and related models. One package allows to draw Petri-nets
in PostScript or PDF documents. One other defines macros related to
PBC, M-nets and \bpn{} models. A last package just gathers together
the two previous.
\end{abstract}

\begin{small}
\tableofcontents
\end{small}

%%
%%

\section{Introduction}

Petri-nets is a set of packages which I use to write my papers about
Petri nets and related models (essentially: PBC, M-nets and \bpn). It
features three packages: one for drawing net, one other which defines
additional macros for textual purpose (like the name ``\bpn'') and a
last one for both purposes. The first package has been designed with a
precise goal: saving me a lot of work when I have to draw nets. This
goal explains some choices I did: typesetting labels in math mode by
default, ending commands with the line, etc. I feel this results in an
intuitive and high-level way to define Petri nets, but if you have
ideas to improve this, don't hesitate to contact me. The text command
package is more classical and if it saves also a lot of typing, its
main goal is to ensure uniform notations.

All these packages may evolve and grow up and if you (or I) feel that
a feature is missing, you can send me an e-mail explaining your needs
(\TeX{} code is welcome too). Notice that I don't pretend being a
specialist in Petri nets and I'm only used to work on three or four
models; so I may ignore particular things in particular models. This
may explain some missing features.

%%
%%

\subsection{Installation}

The last version of the package may be downloaded on the web at the
URL {\it http:/\slash www.univ-paris12.fr\slash lacl\slash
pommereau\slash petrinets.tar.gz\/}, a copy should be available on
{\it CTAN\slash macros\slash generic\slash petri-nets}; the
distribution contains the following files:

\begin{itemize}
\item |pnets.tex| is the main \TeX{} package;
\item |pnets.sty| is its \LaTeX{} counterpart;
\item |pndraw.tex| and |pndraw.sty| are the \TeX{} and \LaTeX{}
      sub-packages for drawing nets;
\item |pntext.tex| and |pntext.sty| are the \TeX{} and \LaTeX{}
      sub-packages for text commands;
\item |pnversion.tex| defines macro |\pnversion|;
\item |pndoc.ps|, |pndoc.pdf| and |pndoc.tex| are the present paper
      and its source;
\item |COPYING| is the text of the GNU GPL;
\item |ChangeLog| collects all changes done to Petri-nets;
\item |README| is a short introduction text;
\item |pn2pdf| is a Perl script used to produce PDF files with
      pdf\LaTeX.
\end{itemize}

In order to have Petri-nets working, you need to copy files |pnets.*|,
|pndraw.*|, |pntext.*| and |pnversion.tex| in a place where \TeX{}
will be able to find them. This can be somewhere in the default search
path (see your local \TeX{} documentation) or in a directory included
in your |TEXINPUT| environment variable.

Packages from Petri-nets have been developed and tested with
te\TeX-1.0.2 for Linux. \TeX{} packages should work with any \TeX{}
version greater than or equal to 3 (my version is 3.14159, Web2C
7.3.7). \LaTeX{} packages have been designed for \LaTeX$2_\varepsilon$
so you may experiment troubles with an older version.

In order to typeset symbols for classical sets of numbers (\setN,
\setR, etc.) Petri-nets uses fonts from AMS. So you must have package
|amsfonts| installed for \LaTeX{} or fonts |bbm| for \TeX{} (I think
these conditions are more or less equivalent).

The drawing basis are provided by package PSTricks, my version is
PST97 but I used only macros described in the manual of version 0.93a
so I hope it should work with it.

Using PSTricks has an important consequence: actual drawings are done
in PostScript, so, you may not see them directly from the DVI viewer
(at least not correctly); additionally, you must use a PostScript
driver in order to produce your final document (for example, dvips
works well). Some support to produce PDF documents with pdf{\LaTeX} is
provided (see section~\ref{sec:pdf}).

PSTricks conflicts with packages |color|, |graphics| and |graphicx|,
Petri-nets thus inherits this conflict. If you wish to use them
together, you should first load package |pstcol|, then Petri-nets and
at last, |color|, |graphics| or |graphicx|. The following is quoted
from PSTricks' |README|:

\begin{quote}
To use the standard `|color|' package (which is available both for
plain \TeX{} and \LaTeX) with PSTricks, you must load the `|pstcol|'
extra package written by David Carlisle, which interface the two
packages, loading them in the right order, and overriding some small
parts of PSTricks to allow it to use the `|color|' package system for
specifying color.  We {\it strongly\/} recommend that you use this way
today.

\LaTeX{} users must also take care that the `|pstcol|' package is
required in place of the `|pstricks|' one if the `|graphics|' or
`|graphicx|' package is also loaded.
\end{quote}

If you wish to produce PDF files using pdf{\LaTeX}, you also need to
copy the script |pn2pdf| in a directory from which it can be executed.
This script requires a working Perl with the packages |Digest::MD5|
and |Getopt::Long| installed. Moreover, the script calls the following
programs: |latex|, |dvips| and |epstopdf| which must be installed on
your system (there should be not problem for the first two). In order
to include the PDF pictures, the package |graphicx| is used (in this
case, it does not lead to any conflict since PSTricks is not loaded
when running pdf\LaTeX). Finally, the package |ifpdf| is used to
detect whether a document is compiled using \LaTeX{} or pdf\LaTeX.
This package should be thus installed.

You can test your installation by recompiling the present manual which
is written in \LaTeX. If it compiles successfully, this means that
Petri-nets and PSTricks have been found by \TeX. If one file is not
found, \TeX{} will complain, giving the file name. Then, running dvips
will ensure that the PostScript headers of PSTricks can be found. You
may also test the PDF support by producing the PDF version of this
manual, see section~\ref{sec:pdf} for the way to proceed.

If you successfully use Petri-nets under another configuration, feel
free to send me an e-mail at {\it pommereau@univ-paris12.fr}. You may
also send bug reports or comments, they are welcome. Of course, bug
fixes are welcome too.

%%
%%

\subsection{Loading the packages}

From \TeX, you can load the package with |\input pnets|, from \LaTeX,
you should put |\usepackage{pnets}| in the preamble of your document.
Since you may want to use only the drawing macros, you can use
|\input pndraw| or from \LaTeX: |\usepackage{pndraw}|. Similarly, in
order to load only text commands, you can use |\input pntext| or
|\usepackage{pntext}|.

Both packages define a macro |\pnversion| which gives the date of the
last update of the package. The date is given as a triple of numbers
of the form |year-month-day|.

%%
%%

\subsection{Things to do, known bugs and problems}

In order to draw additional labels at the right position, |pndraw| has
to perform some time consuming floating point computing. I plane in
the future to perform this directly in PostScript, in order to speed-up
\TeX{} compilation stage.

Only transitions are allowed to change their size according to their
inner label, it may be nice to have this feature for the other
nodes. But it's may be a little bit difficult or quite ugly for some
nodes.

I also could add looping arcs. This is useless for Petri nets but it
would allow one to use the packages for other purposes, like drawing
automata.

The PDf pictures are clipped to the bounding box of the corresponding
net. I've seen on the newsgroups that there exists a solution, but it
was written in German which I can't read.

When one looks at a PDF document using Acrobat Reader, small blur
points appear at the top-right and bottom-left corners of the Petri
nets. This comes from two white points added in order to force dvips
to produce the right bounding box when a net is generated alone in an
EPS file. So, this is a bug of Acrobat Reader rather than one of
Petri-nets.

In a late future, I'd like to make the package independent of
PSTricks, so it would be more portable (but maybe less powerful).

%%
%%

\subsection{Legal stuff}

Petri-nets is {\copyright} 2002 Franck Pommereau ({\it
pommereau@univ-paris12.fr\/}).

This program is free software; you can redistribute it and/or modify
it under the terms of the {\it GNU General Public License\/} as
published by the Free Software Foundation; either version 2 of the
License, or any later version (see file |COPYING|).

This program is distributed in the hope that it will be useful, but
{\it without any warranty\/}; without even the implied warranty of
{\it merchantability\/} or {\it fitness for a particular purpose}.
See the {\it GNU General Public License\/} for more details.

You should have received a copy of the {\it GNU General Public
License\/} along with this program; if not, write to the Free Software
Foundation, Inc., 59 Temple Place, Suite 330, Boston, MA 02111-1307
USA

%%
%%

\subsection{Contributors}

I am very grateful to all the persons who contributed to my work, in
particular:

\begin{quote}
Denis Girou,
Hanna Klaudel,
Andr\'e Steenveld.
\end{quote}

%%
%%

\section{Text commands: package {\tt pntext}}

This section lists the commands defined in package |pntext|, they are
all available from \TeX{} as well as from \LaTeX.

\paragraph{Math sets.}
%
Command |\mathset{A}| produces symbol \mathset{A} (and of course it
works for any text). Shortcuts are defined: |\setN| for |\mathset{N}|
and similarly for |\setZ|, |\setQ|, |\setR| and |\setC|.

\paragraph{Status of places.}
%
Commands |\iplace|, |\eplace| and |\xplace| produce respectively
characters \iplace, \eplace{} and \xplace. Command |\placestatus{n}|
typesets character \placestatus{n} (in the case you would like a new
place type).

\paragraph{Operators for communication.}
%
Synchronization operator (\sy) is available through command |\sy|, in
math mode, this command is surrounded by additional white space as
usual for binary operators. Similar operators are |\rs| for
restriction and |\tie| for asynchronous links. Scoping is available
with commands |\lscope| and |\rscope| which produce respectively a
left and a right double bracket, you may also use |\scope{A}{N}| to
produce \scope{A}{N}.

If you use |\lscope| (resp. |\rscope|) without the corresponding
|\rscope| (resp. |\lscope|) in the same equation or array cell, \TeX{}
will complain about some missing |\right| (resp. |\left|). In order to
close (resp. open) a scope without drawing the bracket, you may use
macro |\Rscope| (resp. |\Lscope|).

\paragraph{Choice.}
%
Choice operator \choice{} is produced with command |\choice|. In math
mode, it behaves like any binary operator.

\paragraph{\bpn.}
%
The logo of \bpn{} is typeset using command |\bpn|. A \bpn{} keyword
(say \bpnkw{program}) is typeset with |\bpnkw{program}| and for a
non-terminal in the syntax (say \bpnnt{scope}) you may use
|\bpnnt{scope}|. Function \mnet{} is defined as |\mnet|.

\paragraph{Sets of values and variables.}
%
\Var{} and \Val{} are typeset using commands |\Var| and |\Val|. They
are better to use than directly the text they typeset because they
adjust their spacing in math mode: compare ``$Var$'' with ``$\Var$''.

\paragraph{To be continued\dots}

%%
%%

\section{Drawing nets: package {\tt pndraw}}

%%
%%

\subsection{Some words about PSTricks}

Reading PSTricks manual could be a good idea, but if you don't want
to, you should know a few things about it in order to use Petri-nets
successfully.

In PSTricks, any point is designated by two coordinates in a grid,
centered on a reference point which has coordinates $(0,0)$. The
figure~\ref{axis} shows an example.

\begin{figure}[t]
\begin{center}
\begin{petrinet}(-1.5,-1.5)(5.5,3.5)
\pst\psgrid[subgriddiv=1,griddots=10](-1,-1)(5,3)
\pst\psline{->}(0,-1.5)(0,3.5)
\pst\psline{->}(-1.5,0)(5.5,0)
\pst\psdot(4,2)
\pst\rput[br](4,2){$(4,2)$}
\pst\psdot(0,0)
\pst\rput[br](0,0){$(0,0)$}
\pst\psdot(2.5,-.5)
\pst\rput[l](2.5,-.5){$\,(2.5,-.5)$}
\end{petrinet}
\end{center}
\caption{The Cartesian coordinate system.}
\label{axis}
\end{figure}

The distance from the origin $(0,0)$ to a point $(x,y)$ depends on
three parameters. |units| is a global scale, if |unit=1.2cm|, $(4,2)$
becomes an abbreviation for $(4 \times 1.2\hbox{ cm}, 2 \times
1.2\hbox{ cm})$. There's also |xunit| and |yunit| which act
respectively only on horizontal or vertical scale; for example, if
|xunit=20pt| and |yunit=1in|, $(4,2)$ stands for $(4 \times 20\hbox{
pt}, 2 \times 1\hbox{ in})$. As you can see on these short examples,
|unit|, |xunit| and |yunit| can be affected any value which is a valid
\TeX{} |dimen| (or \LaTeX{} length).

The default value for all these three parameters is 1~cm; to change
it, you can use the |\psset| command as in |\psset{unit=.5cm}| or in a
combined way as in |\psset{xunit=1cm,yunit=2cm}|.

Since coordinates are formed as a comma separated couple of numbers,
you {\it must not\/} use comma as decimal separator here: |(4.5,2.3)|
is correct but |(4,5,2,3)| is not.

Sometime you'll be asked to give an angle value as a macro parameter.
In Petri-nets, angles are measured in degrees; they can be given as a
numerical value or with a one/two letter(s) code as shown in the
figure~\ref{angles}. Of course, when you specify an angle by its
numerical value, you can use any number, even negative, and not only
one of the eight values shown on figure~\ref{angles}.

\begin{figure}[t]
\begin{center}
\begin{petrinet}[unit=2cm](-2,-1.3)(2,1.3)
\pst\pscircle(0,0){1}%
{\count0=0 %
 \loop %
 \ifnum\count0<360 %
   \uput{.95}[\the\count0]{\the\count0}(0,0){\bf--}%
   \advance\count0 by 45 %
 \repeat}
\pst\uput{1.1}[0](0,0){$0=r$}
\pst\uput{1.1}[45](0,0){$45=tr/rt$}
\pst\uput{1.1}[90](0,0){$90=t$}
\pst\uput{1.1}[135](0,0){$135=tl/lt$}
\pst\uput{1.1}[180](0,0){$180=l$}
\pst\uput{1.1}[225](0,0){$225=bl/lb$}
\pst\uput{1.1}[270](0,0){$270=b$}
\pst\uput{1.1}[315](0,0){$315=br/rb$}
\end{petrinet}
\caption{Translation from angle values to letter code. $r$ is for {\it
right}, $t$ for {\it top}, $l$ for {\it left\/} and $b$ for {\it
bottom}.}
\label{angles}
\end{center}
\end{figure}

The macro |\psset| is used to change PSTricks default parameters, it
takes one argument which is a comma-separated list of |name=value|
pairs. For example, |pndraw.tex| itself uses:
%
\begin{verbatim}
\psset{linewidth=.5pt,
       doublesep=.5pt,
       labelsep=2pt}
\end{verbatim}

As you can see, blank spaces before a |name| are ignored. You already
know how to change units, some additional |name|s should be useful for
Petri-nets:

\begin{itemize}
\item |linewidth| is the default width for {\it any\/} line drawn by
      PSTricks;
\item |doubleline| can be |true| or |false|, when set to |true|, any
      line drawn is doubled;
\item |doublesep| is the distance between the two lines of a doubled
      line;
\item |labelsep| is the distance between an arc and its label(s);
\item |linecolor| is the color used to draw lines, it can be, for
      example, |black|, |darkgray|, |gray|, |lightgray|, |white|,
      |red|, |blue|, |green|, |cyan|, |magenta| or |yellow|;
\item |fillcolor| is the color used to fill shapes; you can use for it
      the same colors as for |linecolor|.
\end{itemize}

PSTricks recognizes many other parameters and you should refer to its
manual for all the details.

%%
%%

\subsection{Beginning and ending nets}

To begin a net, just type |\beginnet| for \TeX{} or |\begin{petrinet}|
under \LaTeX; both this commands are expecting two pairs of
coordinates to give the bottom-left and up-right extrema of the net.

For example, |\begin{petrinet}(-1,-2)(5,3)| starts a net which should
extend in the rectangular area, which we call the {\it bounding box},
delimited by $(-1,-2)$ at the lower left corner and $(5,3)$ at the
upper right. These coordinates are sensitive to current values of
|unit|, |xunit| and |yunit|.

The real effect of these two pairs is to fix the size of the bounding
box which carry the net drawn (it is actually a |\hbox| set in
horizontal mode but who cares?). If some parts of the net extends
outside of the declared bounding box, it will be drawn outside and
that's all. In other words, the bounding box announced just reserves
room for the net and the drawing itself is invisible to \TeX. This
allows you to lies about the real dimension of your drawings: you can
declare a bigger or a smaller bounding box, if you want more or less
white space around your nets. This behaviour is different when running
pdf\LaTeX, see the section~\ref{sec:pdf} for details.

Before to give the coordinates of the bounding box for a net, you can
give optional parameters inside square brackets. These parameters are
interpreted as PSTricks options which are applied to the following net
and stay local to it. For example, if we have |unit=1cm| before
starting a net, command |\begin{petrinet}[unit=2cm](0,0)(1,1)| starts
a net with |unit=2cm| for its bounding box and any coordinate given
inside the net. But as the net ends, the previous value of |unit| is
restored.

In order to have bounding boxes drawn in your nets: you may use
command |\showbb|, optionally followed by bracketed options. For the
figure~\ref{fig:nodes example}, I added |\showbb| at the beginning of
the net. The starred version of |\showbb| fills the background and so,
command |\showbb*[fillcolor=red]| draws a filled red rectangle over
the bounding box.

To end the drawing, you should use |\endnet| or |\end{petrinet}| (did
you guess?). All macros between these two ones are interpreted as
PSTricks or Petri-nets commands. In the following, we call a {\it
net\/} all the stuff inclosed between |\begin{petrinet}| and
|\end{petrinet}| (if you use \LaTeX). One point has to be remembered:
inside a net, the end of line has a special meaning because it is used
to end most Petri-nets drawing commands. So if you want to use
commands spread over several lines, you have to end every line but the
last with a comment (|%|). Also notice that if you call the |\psset|
command inside a net, its effect will remain local to this particular
net.

If you wish something to be done every time a net begins, you may set
token list |\everynet| to what you want to be inserted between command
|\begin{petrinet}| and your drawing commands. For example, with
|\everynet={\small}|, all the following nets will be set in small
types.

Except for beginning and ending nets, all the macros are the same
under \TeX{} or \LaTeX.

%%
%%

\subsection{Drawing nodes}

The available node shapes are places, transitions, modules, sources
and stores, which are drawn in figure~\ref{fig:nodes}. They are
produced with the macros |\place|, |\trans|, |\module|, |\source| and
|\store| respectively. Their size is controlled with the dimensions
|\placesize|, |\transsize|, |\modulesize|, |\sourcesize| and
|\storesize| respectively. For instance, |\transsize=8mm| sets the
size of transitions to |8mm|; |\nodessize{5mm}| sets the size of {\it
all\/} the nodes to |5mm|.

\begin{figure}
\begin{center}
\begin{petrinet}[xunit=2cm](-.5,-.5)(4.5,1)
\place{place}(0,0) p
\label"{t}place\strut
\trans{trans}(1,0) t
\label"{t}transition\strut
\module{module}(2,0) m
\label"{t}module\strut
\source{source}(3,0) s
\label"{t}source\strut
\store{store}(4,0) labels are ignored for stores
\label"{t}store\strut
\end{petrinet}
\end{center}
\caption{The nodes available in Petri-nets.}
\label{fig:nodes}
\end{figure}

Creating a node is possible using the command
``node\{name\}$(x,y)$label'' where:
%
\begin{itemize}
\item ``node'' is one of the above node commands;
\item ``name'' is the name of this node which should be unique for a
      given net (if not, the new node overwrites the old one). This
      name is case sensitive and should not contain any special
      characters (as |$|, |\|, etc.); %$
\item ``$(x,y)$'' are the coordinates of the node's center;
\item ``label'' is the text to typeset inside the node, it
       extends until the end of the line. This label is silently
       discarded for stores.
\end{itemize}

For instance, the nodes in figure~\ref{fig:nodes} where drawn with the
commands:
%
\begin{verbatim}
\place{place}(0,0) p
\trans{trans}(1,0) t
\module{module}(2,0) m
\source{source}(3,0) s
\store{store}(4,0) labels are ignored for stores
\end{verbatim}

By default, labels are typeset in math mode, if you wish a label
typeset in text mode, add option |"| after the node command. Notice
that any space after the coordinates of the node is part of the label,
it has no importance in math mode but in text mode it leads to a label
which starts with a white space. Option |=| sets double-line mode and
so the node is drawn with doubled lines. Option |!| sets the line
width to 2pt so the node appears thicker than usually. PSTricks
options may be given between square brackets, for instance adding
|[linecolor=red,linewidth=1pt]| leads to a red node with 1pt lines.

For transitions, an additional option |*| may be used in order to have
the boundaries of the transition adjusted to the label typeset inside.
By default, the size of a node is fixed and the label overlaps outside
if too long, for transitions with option |*|, the size is just fine.

\subsubsection{Additional labels}

You can give additional outer labels to a node. Each label is typeset
using the macro%
%
\footnote{For \LaTeX, this means that the well known cross-referencing
macro {\tt\string\label} is not available inside a {\tt petrinet}
environment. But outside, it works as usual.}
%
|\label| which has two arguments: the first is an angle indication and
the second, which extends until the end of the line, is the text to
typeset. The angle indication is a numeric value or a code as
explained earlier. Option |"| explained above is also available for
additional labels (and should be given just after |\label|).

Before I give an example, let me explain a useful parameter for
typesetting label: the token list |\everylabel| is expanded before any
label is typeset. For example, command |\everylabel={\scriptstyle}|
leads to typeset all the following labels in script style. This works
for really any label, not only for nodes labels. If |\everylabel| is
set inside a net, its effect will remain local to this net.

The figure~\ref{fig:nodes example} has been typeset using the
following commands:
%
\begin{verbatim}
\begin{petrinet}[xunit=2cm](-.5,-.5)(2.5,.5)
\showbb
\place{p1}(0,0) \bullet
\trans*="{t}(1,0)~transition~
  \label"{70}more text
\place!{p2}(2,0)
  \label{r} \left\{     % I can type multi-line
    {\textstyle 1, 2, 3 \atop   % labels, thanks to comments.
     \textstyle 4, 5, 6} \right\}
\end{petrinet}
\end{verbatim}

\begin{figure}
\begin{center}
\begin{petrinet}[xunit=2cm](-.5,-.5)(2.5,.5)
\showbb
\place{p1}(0,0) \bullet
\trans*="{t}(1,0)~transition~
  \label"{70}more text
\place!{p2}(2,0)
  \label{r} \left\{     % I can type multi-line
    {\textstyle 1, 2, 3 \atop   % labels, thanks to comments.
     \textstyle 4, 5, 6} \right\}
\end{petrinet}
\caption{Example of nodes. In the PDF, nets is clipped to its bounding
         box, see section~\ref{sec:pdf}. }
\label{fig:nodes example}
\end{center}
\end{figure}

You may notice that the net is not well centered, it extends more on
the right. This is because I declared a bounding box of $(-.5,-.5)$ to
$(2.5,.5)$ while in fact, the label of the right-most place extends on
the right, outside of the bounding box (which is made visible by the
macro |\showbb|).

%%
%%

\subsubsection{Free text}

Another kind of node is unboxed text. The macro |\text| is similar to
|\trans*| but without border. Like for other nodes, the text is
typeset in math mode and it can be added outer labels with the macro
|\label|.

This macro knows only the option |"| which has the usual meaning.

%%
%%

\subsection{Linking nodes}

You can draw links between two arbitrary nodes. The package won't
check if requested links are valid from Petri nets point of view, so
you can draw an arc between two places for instance. Basically, macro
|\link| has three parameters: the name of the starting node, that of
the ending node, and the text to typeset on the link (which extends
until the end of the line). You can add optional parameters, before
the first node name:
%
\begin{itemize}
\item options inside square brackets are PSTricks options, as usual;
\item a real number, between 0 and 1, enclosed in angle brackets ({\it
      e.g.}, |<.3>|) can be used to specify the position of the label,
      between starting and ending nodes. Value 0 stands for ``on
      the starting node'' while 1 means ``on the ending node''; the
      default value is $.5$ ({\it i.e.}, on the middle of the link);
\item the characters |^|, |_| or |*| may be used to specify the side of
      the link where the label is typeset. For an link which extends
      from the left to the right, |^| means ``above'', |_| means
      ``below'' and |*| means ``over''. If nothing is specified, the
      label is placed above the link;
\item the options |"|, |=| and |!| have their usual meanings;
\item arrowheads and such can be specified between two signs |/|.
      The available values are shown in the figure~\ref{fig:arrows},
      default is no arrow.
\end{itemize}

\begin{figure}
\begin{center}
\catcode`\|=12
\begin{petrinet}[yunit=6mm](-.6,-.3)(8,15.3)
\pst\catcode`\|=12
\text{caption}(3,15)\rlap{none (default)}
  \text"{t}(0,15){\tt -}
  \text{l}(1,15)
  \text{r}(2,15)
  \link/-/{l}{r}
\text{caption}(3,14)\rlap{arrowheads}
  \text"{t}(0,14){\tt <->}
  \text{l}(1,14)
  \text{r}(2,14)
  \link/<->/{l}{r}
\text{caption}(3,13)\rlap{reverse arrowheads}
  \text"{t}(0,13){\tt >-<}
  \text{l}(1,13)
  \text{r}(2,13)
  \link/>-</{l}{r}
\text{caption}(3,12)\rlap{double arrowheads}
  \text"{t}(0,12){\tt <{}<->{}>}
  \text{l}(1,12)
  \text{r}(2,12)
  \link/<<->>/{l}{r}
\text{caption}(3,11)\rlap{reverse double arrowheads}
  \text"{t}(0,11){\tt >{}>-<{}<}
  \text{l}(1,11)
  \text{r}(2,11)
  \link/>>-<</{l}{r}
\text{caption}(3,10)\rlap{T-bar flush to endpoints}
  \text"{t}(0,10){\tt |-|}
  \text{l}(1,10)
  \text{r}(2,10)
  \link/|-|/{l}{r}
\text{caption}(3,9)\rlap{T-bar centred on endpoints}
  \text"{t}(0,9){\tt |*-|*}
  \text{l}(1,9)
  \text{r}(2,9)
  \link/|*-|*/{l}{r}
\text{caption}(3,8)\rlap{square brackets}
  \text"{t}(0,8){\tt [-]}
  \text{l}(1,8)
  \text{r}(2,8)
  \link/[-]/{l}{r}
\text{caption}(3,7)\rlap{rounded brackets}
  \text"{t}(0,7){\tt (-)}
  \text{l}(1,7)
  \text{r}(2,7)
  \link/(-)/{l}{r}
\text{caption}(3,6)\rlap{circles centred on endpoints}
  \text"{t}(0,6){\tt o-o}
  \text{l}(1,6)
  \text{r}(2,6)
  \link/o-o/{l}{r}
\text{caption}(3,5)\rlap{circles flush to endpoints}
  \text"{t}(0,5){\tt oo-oo}
  \text{l}(1,5)
  \text{r}(2,5)
  \link/oo-oo/{l}{r}
\text{caption}(3,4)\rlap{disks centred on endpoints}
  \text"{t}(0,4){\tt *-*}
  \text{l}(1,4)
  \text{r}(2,4)
  \link/*-*/{l}{r}
\text{caption}(3,3)\rlap{disks flush to endpoints}
  \text"{t}(0,3){\tt **-**}
  \text{l}(1,3)
  \text{r}(2,3)
  \link/**-**/{l}{r}
\text{caption}(3,2)\rlap{extended rounded ends}
  \text"{t}(0,2){\tt c-c}
  \text{l}(1,2)
  \text{r}(2,2)
  \link/c-c/{l}{r}
\text{caption}(3,1)\rlap{flush rounded ends}
  \text"{t}(0,1){\tt cc-cc}
  \text{l}(1,1)
  \text{r}(2,1)
  \link/cc-cc/{l}{r}
\text{caption}(3,0)\rlap{extended square ends}
  \text"{t}(0,0){\tt C-C}
  \text{l}(1,0)
  \text{r}(2,0)
  \link/C-C/{l}{r}
\end{petrinet}
\end{center}
\caption{The different types of links termination. They can be freely
mixed.}
\label{fig:arrows}
\end{figure}

The order in which you use these options has no importance, provided
that all options are given before the starting node. Additionally, if
you use contradicting options, only the last used is taken into
account. For instance, the two following lines are equivalent:

|\link_[linecolor=gray,linewidth=5pt]*<.2>^[linewidth=1pt]|

|\link^<.2>[linecolor=gray,linewidth=1pt]|.

To draw curved links, you should use PSTricks' |arcangle| option which
is an angle in degrees (you cannot use a letter code here), measuring
the deviation from a straight line between nodes, at the starting and
at the ending of the link. Try, it's easy.

To change the appearance of the arrowhead, you must change PSTricks'
parameters |arrowsize|, |arrowlength| and |arrowinset|. Petri-nets
uses the following:
%
\begin{verbatim}
\psset{arrowlength=1.4,
       arrowinset=0,
       arrowsize=2pt 2}
\end{verbatim}
%
I wont explain how it works, just try, it's easy too. (Or read the
manual of PSTricks where all the details are given.)

%%
%%

\subsubsection{Additional labels}

Like for nodes, links can be added more labels. This is also made with
the macro |\label|, but here, its syntax is like for the macro
|\link|, except that you must not specify the starting and ending
nodes (but the options of |\link| are available).

%%
%%

\subsubsection{Arcs}

Since links are almost always draw with a single arrowhead at the end
(and then called {\it arcs\/}), a shortcut is provided: the macro
|\arc| may be used instead of |\link/->/|.

%%
%%

\section{Tips, tricks and troubleshooting}

\paragraph{Nets as macro arguments.}
%
If you try to define a net inside the argument of a macro, you will
have an error. For instance, the code
%
\begin{verbatim}
\centerline{\begin{petrinet}(0,0)(1,1)
\place{p}(0,0) p
\trans{t}(1,1) t
\arc{p}{t}
\arc{t}{p}
\end{petrinet}}
\end{verbatim}
%
produces an error such as:
%
\begin{verbatim}
! Argument of \net:place:draw has an extra }.
<inserted text> 
                \par 
l.11 \end{petrinet}}
\end{verbatim}

The reason is that the macro |\centerline| reads all the net and treat
the end of lines as spaces. Similar things occur with the other
characters which are considered in a special way inside nets. The
right way to produce the desired effect is to use the macro
|\savenet|:
%
\begin{verbatim}
\savenet
\begin{petrinet}(0,0)(1,1)
\place{p}(0,0) p
\trans{t}(1,1) t
\arc{p}{t}
\arc{t}{p}
\end{petrinet}
\centerline{\shownet}
\end{verbatim}
%
the form |\savenet\begin{petrinet}| works as well and from plain
{\TeX} you should use the form |\savenet\beginnet|. After one
|\savenet| you may even use |\shownet| several times in order to
duplicate the saved net.

\paragraph{Error messages.}
%
Sometimes, you'll get an error message such as:

\begin{verbatim}
! Argument of \net:place:draw has an extra }.
<inserted text> 
                \par 
l.785 \place{i3}(2,0) i_3
                         
? 
\end{verbatim}

This usually means that the error was just before the line for which
\TeX{} complains (here line 785 which is correct). And usually, this
error is an omission of one of the arguments of a drawing macro.

Another kind of message occurs quite often:

\begin{verbatim}
PSTricks error.  See User's Guide for further information.
                 Type  H <return>  for immediate help.
! Graphics parameter `rcangle' not defined..
\@pstrickserr ... immediate help.}\errmessage {#1}
                                                  \endgroup 
l.794 \arc[rcangle=30]
                      {ab2}{i3} \bullet
? 
\end{verbatim}

This is because on line 794, you misspelled the word ``arcangle'',
giving ``rcangle'' instead. This error is detected by PSTricks since
bracketed arguments are sent verbatim to it.

In general, since the package |pndraw| uses a lot of tricks in order
to read the arguments of the commands, you should not trust too much
the error messages: just look at the line indicated in the message and
seek for a mistake in it or in the lines before. (Actually, this
advice can be used for almost any error message issued by \LaTeX\dots)

\paragraph{Drawing commands on multiple lines.}
%
Inside a net, label arguments are delimited by the end of the line. To
type an argument which extends on more than one line, you should end
each line but the last with a comment~|%|.

\paragraph{Empty node.}
%
If you want to draw ``floating'' arcs, {\it i.e.}, arcs wich are not
attached to a place or a transition, you just can anchor them on empty
texts: the following net is depicted in the figure~\ref{empty text}.

\begin{verbatim}
\begin{petrinet}(0,0)(2,0)
\text{from}(0,0)
\text{to}(2,0)
\arc{from}{to}
\end{petrinet}
\end{verbatim}

\begin{figure}[t]
\begin{center}
\begin{petrinet}(-.2,-.2)(2.2,.2)
\text{from}(0,0)
\text{to}(2,0)
\arc{from}{to}
\end{petrinet}
\end{center}
\caption{An arc anchored on empty texts}
\label{empty text}
\end{figure}

\paragraph{Text mode labels.}
%
Take care of the interaction between |\everylabel| and the option |"|
which typesets a label in text mode. For example, setting
|\everylabel{\scriptstyle}| leads to an error every time a text mode
label is typeset because |\scriptstyle| is a math command. So you
should prefer the longer but safer form: |\everylabel{\ifmmode|\hskip
0pt|\scriptstyle|\hskip 0pt|\else|\hskip 0pt|\scriptsize|\hskip
0pt|\fi}|.

\paragraph{Labels at the wrong position.}
%
If the additional labels of a node appear centered on it instead of
outside, this may come from the way you visualize your document. DVI
viewers usually do not interpret correctly all of the PostScript
commands used by PSTricks. Trying a PostScript viewer (GhostView is
certainly a good choice) may solve your problem.

\paragraph{To be continued\dots}

%%
%%

\section{Producing PDF documents}
\label{sec:pdf}

The {\LaTeX} version of |pndraw| provides some support for pdf{\LaTeX}
through the Perl script |pn2pdf| included in the archive. The usage is
quite automated, even completely if you run |pdflatex| with the option
|--shell|. If you don't want to enable this option, run first
|pdflatex|, then run |pn2pdf| |-b| |document| (assuming your file is
called |document.tex|) and finally run |pdflatex| again.\footnote{This
process was different in the previous version of the package!} Using
the option |--shell| simply allows |pdflatex| to run |pn2pdf| for you.

Several files are produced by these different runs. You may never see
most of them since they are deleted when the become useless. {\it Take
care that none of your files uses one of these names otherwise it will
be overwrote or deleted!}
%
\begin{itemize}
\item |document.sum| is used to remember some information in order
      to avoid recreating pictures if not necessary.
\item |document.|$d$ is created and deleted for test purpose when the
      option |--shell| has been used, $d$ is the date when |pdflatex|
      was started, as produced by the macros
      |\the\year\the\month\the\day\the\time| (in the current document,
      it leads to \the\year\the\month\the\day\the\time).
\item |document.bpn| is created when |pdflatex| is run without the
      option |--shell|, it contains the directives for producing the
      figures with |pn2pdf -b|.
\item |document.pre| is used by |pn2pdf| in order to remember the
      commands (like |\psset|) issued outside of |petrinets|
      environments.
\item |document-fig|$n$|.pn|, |document-fig|$n$|.tex|,
      |document-fig|$n$|.aux|,\\ |document-fig|$n$|.dvi|,
      |document-fig|$n$|.log|, |document-fig|$n$|.eps| are created
      during the generation of |document-fig|$n$|.pdf| which contains
      the rendering of the $n$th net of the document.
\end{itemize}
%
The files |document-fig|$n$|.pdf| should not be deleted before the
final document is produced since they are included from |pdflatex|. If
they are not found, a warning is issued during the compilation.

There is several important things you should notice:
%
\begin{enumerate}
\item Everything outside of the bounding box of a net is
      cropped during the creation of the PDF picture. So, take care to
      check how your picture is rendered (using |\showbb| for
      instance).
\item Only one pass of {\LaTeX} is made on each figure. If you need to
      several passes, you may run |pn2pdf| manually, with the options
      |-f| and |-k| (see below).
\item {\LaTeX} and {pdf\LaTeX} do not always produce identical
      rendering of the same document. However, Petri-nets pictures
      should be the same since they are always typeset by \LaTeX.
\end{enumerate}

To conclude, here is a summary of the ways to run |pn2pdf|:
%
\begin{itemize}
\item |pn2pdf document-fig|$n$|.pn|\\
      generate |pn2pdf document-fig|$n$|.pdf|
\item the option |-f| or |--force| may be used to force the creation
      of PDF files even if their source did not change.
\item |pn2pdf -t FILES| or |pn2pdf --test FILES|\\
      used by {pdf\LaTeX} in order to check is the option |--shell|
      was used or not.
\item |pn2pdf -b document| or |pn2pdf --batch document|\\
      reads the file |document.bpn| in order to produce the figures.
\item |pn2pdf -d FILES| or |pn2pdf --delete FILES|\\
      deletes |FILES| (which may be patterns like |*.aux|), this is
      use by |pdflatex| in order to remove some temporary files.
\item the option |-k| or |--keep| prevents |pn2pdf| from deleting the
      temporary files.
\item the option |-h| or |--help| prints a short help and exit.
\end{itemize}

%%
%%

\newpage
\appendix

\section{Reference pages}

\subsection{Text commands: package {\tt pntext}}

\begin{center}
\begin{tabular}{l@{\qquad}l}
|\bpn| & \bpn \\
|\bpnkw{key-word}| & \bpnkw{key-word} \\
|\bpnnt{non-terminal}| & \bpnnt{non-terminal} \\
|\choice| & \choice \\
|\eplace| & \eplace \\
|\iplace| & \iplace \\
|\lscope| & $\lscope\Rscope$ \\
|\Lscope| & invisible version of |\lscope| \\
|\mathset{A}| & \mathset{A} \\
|\mnet| & \mnet \\
|\placestatus{d}| & \placestatus{d} \\
|\pnversion| & \pnversion{} (current version) \\
|\rs| & \rs \\
|\rscope| & $\Lscope\rscope$ \\
|\Rscope| & invisible version of |\rscope| \\
|\scope{a}{N}| & $\scope{a}{N}$ \\
|\setC| & \setC \\
|\setN| & \setN \\
|\setQ| & \setQ \\
|\setR| & \setR \\
|\setZ| & \setZ \\
|\sy| & \sy \\
|\tie| & \tie \\
|\Val| & \Val \\
|\Var| & \Var \\
|\xplace| & \xplace \\
\end{tabular}
\end{center}

\subsection{Drawing command: package {\tt pndraw}}

\def\n{$\hookleftarrow$}
\def\<{$\displaystyle\big\langle$}
\def\>{$\displaystyle\big\rangle$}
\newenvironment{options}{\\[5pt]\begin{tabular}{@{\qquad}ll}}{\end{tabular}}

The parts placed between angle brackets \<$\cdots$\> are the optional
ones, the others are mandatory. Symbol \n{} denotes the end of the
line.

\begin{itemize}
\item |\arc|\<|^_*"=![|options|]<|pos|>|\>|{|node1|}{|node2|}|\<label\>\n\\
Draws a labelled arc between node1 and node2.
\begin{options}
|^| & label above the arc (default)\\
|_| & label below the arc \\
|*| & label over the arc \\
|"| & label in text mode \\
|=| & double line \\
|!| & thick line \\
|[|options|]| & PSTricks options \\
|<|pos|>| & position of the label ($0 \leq {}$pos${}\leq 1$) \\
\end{options}
%
\item |\beginnet|\<|[|options|]|\>$(x_1,y_1)(x_2,y_2)$\n\\
      |\begin{petrinet}|\<|[|options|]|\>$(x_1,y_1)(x_2,y_2)$\n\\
Begins a net whose bounding box is defined by $(x_1,y_1)$ as
bottom left corner and $(x_2,y_2)$ at top right corner.
\begin{options}
|[|options|]| & PSTricks options \\
\end{options}
%
\item |\endnet|\\
      |\end{petrinet}|\\
Ends a net.
%
\item |\everylabel={|\<tokens\>|}|\\
Expands tokens each time a label is typeset.
%
\item |\everynet={|\<tokens\>|}|\\
Expands tokens each time a net is started.
%
\item |\label|\<|"{|pos|}|label\>\n\\
      |\label|\<|"<|pos|>|label\>\n\\
Draws an additional label for a node (first line) or a link (second
line).
\begin{options}
|"| & label in text mode \\
|{|pos|}| & position of the label, an angle (in degrees) \\
          & or a corner code (t,l,b,r,tl,tr,bl,br) \\
|<|pos|>| & position of the label ($0 \leq$ pos $\leq 1$)\\
\end{options}
%
\item |\link|\<|^_*"=![|options|]<|pos|>/|arrow|/|\>|{|node1|}{|node2|}|\<label\>\n\\
Draws a labelled link between node1 and node2.
\begin{options}
|^| & label above the arc (default) \\
|_| & label below the arc \\
|*| & label over the arc \\
|"| & label in text mode \\
|=| & double line \\
|!| & thick line \\
|[|options|]| & PSTricks options \\
|<|pos|>| & position of the label ($0 \leq {}$pos${}\leq 1$) \\
|/|arrow|/| & arrows specifications (see the figure~\ref{fig:arrows} \\
\end{options}
%
\item |\module|\<|"=![|options|]|\>|{|name|}|$(x,y)$\<label\>\n\\
Draws a labelled module centered on $(x,y)$.
\begin{options}
|"| & label in text mode \\
|=| & double line \\
|!| & thick line \\
|[|options|]| & PSTricks options \\
\end{options}
%
\item |\modulesize=|dimen\\
Sets the size of the modules.
%
\item |\nodessize{|dimen|}|\\
Sets the size of all nodes.
%
\item |\place|\<|"=![|options|]|\>|{|name|}|$(x,y)$\<label\>\n\\
Draws a labelled place centered on $(x,y)$.
\begin{options}
|"| & label in text mode \\
|=| & double line \\
|!| & thick line \\
|[|options|]| & PSTricks options \\
\end{options}
%
\item |\placesize=|dimen\\
Sets the size of the places.
%
\item |\psset{|name|=|value\<|,...|\>|}|\\
Sets PSTricks options.
\begin{options}
|arcangle=|angle & angle at the ends of links \\
|arrowinset=|real & size of arrowheads insets \\
|arrowlength=|real & length of arrowheads \\
|arrowsize=|dim integer & size of arrowheads \\
|doubleline=|boolean & double lines on/off \\
|doublesep=|dim & distance between double lines \\
|fillcolor=|color & background color \\
|labelsep=|dim & distance between labels and nodes \\
|linecolor=|color & lines color \\
|linewidth=|dim & lines thickness \\
|unit=|dim & global scale \\
|xunit=|dim & horizontal scale \\
|yunit=|dim & vertical scale\\
\end{options}
%
\item |\savenet|\\
Just before the beginning of a net: saves it without displaying it.
%
\item |\showbb|\<|*[|options|]|\>\n\\
Draws the bounding box of a net.
\begin{options}
|*| & background filled \\
|[|options|]| & PSTricks options \\
\end{options}
%
\item |\shownet|\\
Displays the last net saved by |\savenet|.
%
\item |\source|\<|"=![|options|]|\>|{|name|}|$(x,y)$\<label\>\n\\
Draws a labelled source centered on $(x,y)$.
\begin{options}
|"| & label in text mode \\
|=| & double line \\
|!| & thick line \\
|[|options|]| & PSTricks options \\
\end{options}
%
\item |\sourcesize=|dimen\\
Sets the size of sources.
%
\item |\store|\<|"=![|options|]|\>|{|name|}|$(x,y)$\n\\
Draws a store centered on $(x,y)$.
\begin{options}
|"| & label in text mode \\
|=| & double line \\
|!| & thick line \\
|[|options|]| & PSTricks options \\
\end{options}
%
\item |\storesize=|dimen\\
Sets the size of stores.
%
\item |\text|\<|"|\>|{|name|}|$(x,y)$\<label\>\n\\
Draws an unboxed labelled node centered on $(x,y)$.
\begin{options}
|"| & label in text mode \\
\end{options}
%
\item |\trans|\<|*"=![|options|]|\>|{|name|}|$(x,y)$\<label\>\n\\
Draws a labelled transition centered on $(x,y)$.
\begin{options}
|*| & automatic size \\
|"| & label in text mode \\
|=| & double line \\
|!| & thick line \\
|[|options|]| & PSTricks options \\
\end{options}
%
\item |\transsize=|dimen\\
Sets the size of transitions.
\end{itemize}

\end{document}
