%% $Id: xcomment-doc.tex 316 2010-05-14 21:32:34Z herbert $
\documentclass[12pt]{article}
\usepackage[T1]{fontenc}
\usepackage{mathpazo,url}
\usepackage{pstricks,xcomment,fancyvrb,shortvrb,dtklogos}
\MakeShortVerb{|}

\let\XCfv\fileversion
\let\meta\texttt

\title{Documentation for xcomment.sty%
   \thanks{Documentation revised by Herbert Vo\ss. 
   This file borrows much from verbatim.sty, v.1.4c (90/10/18)
  Copyright (C) 1989, 1990 by Rainer Sch\"opf.}}

\author{Timothy Van Zandt \url{Timothy.VAN-ZANDT@insead.edu}}

\date{\today}

\begin{document}

\maketitle
\clearpage
\tableofcontents

\clearpage
\fvset{numbers=left,frame=single,fontsize=\small,fontfamily=tt}

\begin{abstract}
The \LaTeX{} style option verbatim.sty, by Rainer Sch\"opf, allows one to
redefine an environment to be a comment, and thereby selectively omit
certain environments when typesetting a document.
Suppose instead that one wants to typeset only selected environments? For
example, one might want to print only a document's tables and figures,
without having to enclose all the text outside these environments within
comments. This style option allows such selection.
\end{abstract}

\section{Usage notes}

This style file defines a new environment, |xcomment|, which permits one to
typeset only selected environments, without having to enclose all the text
outside these environments within comments. The main interest in such a
feature is that it allows document styles to have great control over what
parts of a document are typeset, thus extending the modularity of \LaTeX.
For example, this option was originally written for use in a document style
for seminar notes and slides. The main text of the input file consists of
the notes for a seminar, and each slide goes in a |slide| environment. A
simple modification of the style options allows one to typeset only the
slides, only the notes, or both together, in a variety of styles.


The |xcomment| environment has as a single mandatory argument a list
(possibly empty) of environments, separated by commas and with no spaces.
Within the |xcomment| environment, only text within each of the specified
environments is typeset.
The |\xcomment| command can also be used directly, with the same argument,
and it would typically go in the preamble. Invoking the |\xcomment| command
is equivalent to putting |\begin{xcomment}| at the invocation of the command
or immediately after |\begin{document}|, whichever occurs later in the
document, and |\end{xcomment}| just before the end of the document.

For example, if |\xcomment{table,figure}| is put in the preamble, only
tables and figures are typeset (but see |\nofloat| below). If the list of
environments is empty, as in |\begin{xcomment}{}|, the |xcomment|
environment is essentially like the |comment| environment in verbatim.sty.

|xcomment| environments can be nested, but if the nested environments
have the same name, the inner environments must be inside text that is
typeset as specified by the |xcomment| environment the next level up.

Here is an example of such nesting. Suppose we want to include only figures,
but we also want to be able to comment out some of the figures so that they
are not included. This is achieved in the following example:
\begin{Verbatim}
\newxcomment[]{mycomment}
\begin{xcomment}{figure,mycomment}
  This is stuff that is not included.
  \begin{figure}
    This figure is included.
  \end{figure}
  More stuff that is not included.
  \begin{mycomment}               Out and back into comment mode
    \begin{figure}                Ignored by mycomment envir.
      This figure is NOT included.
    \end{figure}                  Also ignored by mycomment envir.
  \end{mycomment}                 In and back out of comment mode.
  More stuff that is not included.
\end{xcomment}
\end{Verbatim}

|xcomment| will follow |\input| and |\include| commands. You must use the
\LaTeX{} syntax |\input{file}| (as opposed to \verb*+\input file +), and the
inputted file must end with |\endinput|.
You should be in |xcomment| mode when the |\endinput| command occurs
if and only if you were in |\xcomment| mode when the |\input| or |\include|
command was found.

  |xcomment| is searching literally for |\begin{foo}| and |\end{foo}| when
determining whether to switch in or out of comment mode. It will not find,
e.g., |!begin[foo]|, even if |!|, |[| or |]| have category codes 0, 1 and 2,
respectively. (If you don't understand this, you can safely ignore it.)

The comment character, |%|, is still a comment character in |xcomment|
environments (even if used with |\%|). You can change the command character by
redefining |\xcommentchar|. For example, if I want to use |"| as a comment
character:
\begin{Verbatim}
    \renewcommand{\xcommentchar}{\"}
\end{Verbatim}
If you define |\xcommentchar| to be empty, then no comment character is used.

In the |xcomment| environment, text is processed a line at a time and
discarded until |\end{xcomment}| or |\begin{|\meta{environment}|}| is
encountered, where {\em environment} is to be included. The remaining text
on the line is not thrown away, as it is in verbatim.sty. Instead, it is
rescanned, and the only restrictions are that it have balanced braces and
that it not contain commands that again shift into {\em and} out of
``comment'' mode.

This is an important feature, because the included environments may have
arguments that are best placed on the same line as the
|\begin{|\meta{environment}|}| command. However, the rescanning creates a
temporary file. By default, the file is |\jobname.tmp|. The command
|\rescanfile{|\meta{file}|}| causes \meta{file} to be created instead.
|\rescanfile{}| suppresses the creation of a temporary file; the leftover
text is simply discarded. |\norescanfile| also suppresses the creation
of a temporary file, but in this case the text is simply inserted
without being rescanned (i.e., with category codes 0 (escape
|\|), 1 (begin group |{|), 2 (end group |}|) and 6 (parameter |#|) switched
to 12 (other).

The command |\envirsep| is executed between the environments that are
typeset. Its default definition is |\par|.

The command
\begin{Verbatim}
    \newxcomment[|\meta{environment list}|]{|\meta{name}|}
\end{Verbatim}
defines |\name| and the |name| environment to work like |\xcomment| and the
|xcomment| environment. A list of environments to be included can be given
as an optional first argument to |\newxcomment|, in which case the new
command and environment do not take an argument. For example, if you put
|\newxcomment[]{mycomment}| in the preamble, the |mycomment| environment
works like the |comment| environment in verbatim.sty, except for the
rescanning.

The command |\nofloat{|\meta{environment list}|}| is provided to disable the
floating of environments in the list (also separated by commas and without
spaces), since if there are only floats and no text, the floats will
accumulate to the end of the document and \TeX{} may run out of memory.

\paragraph{Caveats:}
\begin{itemize}
\item If |\xcomment| is invoked in the preamble, |\document| should not be
subsequently redefined.
\item Be careful what argument you give to |\rescanfile|, since any existing
file with that name will be destroyed, and the extension |tex| is added if
no extension is given.
\end{itemize}

\paragraph{Changes to v1.2:}
\begin{itemize}
\item |\include| now allowed.
\item Comment characters not obeyed, as determined by |\xcommentchar|.
\end{itemize}

\paragraph{Changes to v1.1:}
\begin{itemize}
\item Fixed bug in |\@xcomment| that caused problems when invoking |\xcomment|
in the preamble.
\end{itemize}

\paragraph{Changes to v1.0:}
\begin{itemize}
\item Main loop rewritten.
\item |\rescanfile{trash}| changed to |\rescanfile{}|.
\item |\rescanfile{bounce}| no longer supported. Use |\norescanfile|.
\item |\input| now allowed.
\end{itemize}




\section{The implementation}
The code is an adaption of code in verbatim.sty, and in fact the latter's
structure has been preserved when possible. On the one hand, the code is
simpler than in verbatim.sty because we throw away text from the input file
rather than typeset it verbatim. On the other hand, it is more involved
because we have to check for more possible endings to the |xcomment| than in
verbatim.sty and we have to preserve the input that follows the |\begin|ning
of an included environment on the same line.

As in verbatim.sty and the |\comment\endcomment| commands in \AmSTeX, the
basic strategy is to change the category codes so that control sequences and
other troublesome special characters are neutralized, and then to scan for
the strings that mark the end of the comment. This would be a 3-liner in
|awk|, but this is \TeX, and the amount of code required to do the scanning
increases rapidly as we allow for a greater variety of ending strings.

We begin by ensuring that file is not read in twice, and then we identify
the file on the VDU and the transcript file.

\fvset{fontsize=\footnotesize}

\begin{Verbatim}
\@ifundefined{xcomment@@@}{}{\endinput}
\typeout{Style-Option: `xcomment' v\fileversion \space <\filedate> (tvz) }
\end{Verbatim}


\subsection{User commands}

The main toplevel macro is |\@xcomment|, which is defined in the next
section. It works like |\xcomment| as described above, except that it has as
a first argument an {\em environment} so that it stops when it encounters
|\end{|{\em environment}|}|. |\newxcomment| then has a simple definition,
and |\xcomment| is defined using the |\newxcomment| command.
\begin{Verbatim}
\def\newxcomment{\@ifnextchar [{\@newxcommentwitharg}%
  {\@newxcomment}}
\def\@newxcomment#1{%
  \expandafter\def\csname #1\endcsname##1{\@xcomment{#1}{##1}}}%
\def\@newxcommentwitharg[#1]#2{%
  \expandafter\def\csname #2\endcsname{\@xcomment{#2}{#1}}}
\newxcomment{xcomment}
\end{Verbatim}
%
|\envirsep| is inserted between the environments that are typeset, and is
set to |\par| by default. |\rescanfile| is the name of the temporary file
used for rescanning, and is set to |\jobname.tmp| by default.
\begin{Verbatim}
\def\envirsep{\par}
\def\rescanfile#1{\def\@rescanfile{#1}}
\rescanfile{\jobname.tmp}
\def\norescanfile{\let\@rescanfile\relax}
\end{Verbatim}
%
|\nofloat| disables floating, and |\vfill| is inserted forcefully on each
side of each float. The macro is just a hack, and is mainly for printing
only floats. It may not work well even for that. |\envirsep| is still
inserted between floats, and if it is set to |\vspace{}|, the added space is
inserted only between floats on the same page, given the user some control
over spacing.
\begin{Verbatim}
\def\@nofloat#1{\hrule height\z@\nobreak\vfill\vbox\bgroup\def\@captype{#1}}
\def\end@nofloat{\egroup\nobreak\vfill\nobreak\hrule height\z@\medbreak}
\def\nofloat#1{\@for\@tempa:=#1\do{\@namedef{#1}{\@nofloat{#1}}%
  \@namedef{end#1}{\end@nofloat}}}
\end{Verbatim}


\subsection{Preliminaries}
%
|\xc@makeother| takes as argument a character and changes
its category code to $12$ (other) if its category code is originally 0, 1, 2
or 6. We do not have to change as many catcodes as in Verbatim.sty, because
the input is either thrown away or rescanned. The fewer codes we change the
better.
\begin{Verbatim}
\def\xc@makeother#1{%
  \ifnum\the\catcode`#1=0\catcode`#112%
  \else \ifnum\the\catcode`#1=1\catcode`#112%
    \else \ifnum\the\catcode`#1=2\catcode`#112%
      \else \ifnum\the\catcode`#1=6\catcode`#112%
  \fi\fi\fi\fi\relax}
   \end{Verbatim}

This macro changes the category codes of a token list that is already in
\TeX's stomach. The code is a modification of a |\retokenize| macro by
Raymond Chen.
\begin{Verbatim}
\newwrite\tokout
\newread\tokin
\end{Verbatim}
The argument of |\rescan| is a token list register, say, |\mytoks|.
|\mytoks| presumably contains a string from the input file, sent to |\TeX|'s
stomach as a stream of tokens. |\rescan| gives |\mytoks| the token list that
would have arisen if |\TeX| had had the current catcodes in effect when it
read the input string, with the following exceptions:

\begin{itemize}
\item The token list must have balanced braces under the new catcodes.
\item Parameter tokens in the original list are written twice, and so mess
things up.
\item Some characters with catcode 10 (space) under the old catcodes are
lost, and all that remain are treated as character 32 (\verb*+ +) under the
new catcodes, whatever this character's new catcode is.
\item Escape characters under the old catcodes are treated as escape
characters even if their catcodes have changed. (This can be fixed by
setting |\escapchar| during the write to the escape character under the old
catcodes, assuming there was only one and it is known)
\item Leading spaces under the new catcodes are not ignored, which is wrong
if and only if the string started at the beginning of a line and the space
characters were not spaces under the old catcodes.
\end{itemize}

Only the balanced braces exception is a problem in this application, but the
other exceptions are pointed out since such a rescanning macro has other
applications.
\begin{Verbatim}
\def\rescan#1{%
\end{Verbatim}
First we check if |\rescanfile|, which contains the name of the temporary
file to be used, has a special meaning. If |\relax|, do nothing. If empty,
empty the token register.
\begin{Verbatim}
  \ifx\@rescanfile\relax\else
    \ifx\@rescanfile\@empty #1{}\else
\end{Verbatim}
Put the list of tokens in braces and write them to the temporary file.%
\begin{Verbatim}
      \immediate\openout\tokout=\@rescanfile
      \immediate\write\tokout{{\the#1}\relax}%
      \immediate\closeout\tokout
\end{Verbatim}
Read the contents of the file |\@tempd|.
\begin{Verbatim}
      \openin\tokin=\@rescanfile
      \read\tokin to\@tempd
      \closein\tokin
\end{Verbatim}
Suppose |#1| is |\mytok|, and \meta{token list} is the list of tokens with
the current catcodes. Then the next line expands to |\mytok{|\meta{token
list}|}\relax|.
\begin{Verbatim}
      \expandafter#1\@tempd%
  \fi\fi}
\end{Verbatim}
%
%
\subsection{Toplevel macros}
%
|\@xcomment| checks whether it was invoked before |\begin{document}| and
outside of an environment. If so, it modifies |\begin{document}| to invoke
|\@xcomment| with the same arguments. Otherwise, it processes its arguments
and goes into ``comment'' mode by invoking |\xc@begin|.

\begin{Verbatim}
\def\@xcomment#1#2{%
  \ifx\@preamblecmds\@notprerr
    \def\xc@csname{#1}%
    \edef\xc@envirlist{#2}%
    \ifx\xc@envirlist\@empty \@bsphack \else
      \begingroup
\end{Verbatim}
|\@envirsep| is what is actually placed before each environment. Initially,
it is empty because nothing should precede the first environment that is
typeset.
\begin{Verbatim}
      \def\@envirsep{}%
\end{Verbatim}
|\do@end| is appended to the meaning of |\end|, and it is initially set to
|\xc@begin|. This means that the |\end| of an included environment switches
us back into ``comment'' mode, as desired. The |\end| of nested environments
should have their usual meaning, however. Therefore, |\do@begin| is
prepended to the definition of |\begin|, and it sets |\do@end| to |\relax|.
Included environment are begun with |\normal@begin| because we only want to
reset |\do@end| for nested environements. The group begun above keeps the
|\end| of the |xcomment| environment from be being affected by these
changes.

\begin{Verbatim}
      \@ifundefined{normal@begin}{\let\normal@begin\begin}{}%
      \@ifundefined{normal@end}{\let\normal@end\end}{}%
      \def\begin##1{\do@begin{##1}\normal@begin{##1}}%
      \def\end##1{\normal@end{##1}\do@end}%
      \def\do@begin##1{\@ifundefined{##1}{}{\def\do@end{}}}%
      \let\do@end\xc@begin
    \fi
    \let\next\xc@begin
  \else
    \expandafter\@temptokena\expandafter{\document\@xcomment{@@@}{#2}}%
    \edef\document{\the\@temptokena}%
    \let\next\relax
  \fi
  \next}
\def\end@xcomment{\ifx\xc@envirlist\@empty \@esphack \else \endgroup \fi}%
\end{Verbatim}
%
|\xc@begin| starts up ``comment'' mode by changing the category codes and
invoking |\xcomment@|. The |\begingroup| keeps these catcode changes local.
\begin{Verbatim}
\def\xc@begin
\end{Verbatim}


\end{document}

\subsection{The main loop}

Since the code in the main loop uses literal |\|, |{|, |}|, |%| and newline
characters, we represent them with |!|, |[|, |]| |"| and |~| by setting the
corresponding lccodes to the correponding character but with catcode 12, and
then changing the code to lowercase before executing the macro definitions.
These catcode changes are kept local, and the macros must therefore be
defined globally.
\begin{Verbatim}
\begingroup
\catcode`\!=12
\catcode`\[=12
\catcode`\]=12
\catcode`\"=12
\lccode`\!=`\\
\lccode`\[=`\{
\lccode`\]=`\}
\lccode`\"=`\%
\catcode`\~=\active
\lccode`\~=`\^^M
\lowercase{
\end{Verbatim}

The three macros |\xcomment@|, |\xcomment@@|, and |\xcomment@@@| form the
main loop when in ``comment'' mode. |\xcomment@| reads a line of input.
|\xcomment@@| searches for the first occurence of a backslash.
|\xcomment@@@| checks what the next character is, and depending an the
answer various macros are invoked to see if one of the termination strings
has been found.
%
|\xcomment@| takes the line it has read, puts |\@nnil| |\| and |\@nil| as
delimiters at the end, and then calls |\xcomment@@|.
\begin{Verbatim}
  \gdef\xcomment@#1~{\xcomment@@#1\@nnil!\@nil}
\end{Verbatim}
%
|\xcomment@@| takes everything up to the next occurrence of |\| as its
argument and throws it away together with the |\|. Then |\xcomment@@@| reads
the next token on the line, including the delimiters added by |\xcomment@|.
\begin{Verbatim}
  \gdef\xcomment@@#1!{\xcomment@@@}
\end{Verbatim}

|\xcomment@@@| checks for key strings.
\begin{Verbatim}
  \gdef\xcomment@@@#1\@nil{%
    \def\@tempa{#1}%
    \ifx\@tempa\@empty
      \let\next\xcomment@
    \else
      \def\next{\xcomment@@#1\@nil}%
      \@testtrue
      \xc@checkbegin#1\relax begin[]\relax\relax
      \if@test
        \xc@checkend#1\relax end[]\relax\relax
        \if@test
          \xc@checkinput#1\relax input[]\relax\relax
          \if@test
            \xc@checkinclude#1\relax include[]\relax\relax
            \if@test
              \xc@checkendinput#1\relax endinputss\relax\relax
    \fi\fi\fi\fi\fi
    \next}
\end{Verbatim}
%
Check for |\begin| and |\end|.
\begin{Verbatim}
  \gdef\xc@checkbegin#1begin[#2]#3\relax#4\relax{%
    \def\@tempa{#1}%
    \ifx\@tempa\@empty
      \def\@tempa{#2}%
      \@for\@tempb:=\xc@envirlist\do{%
        \ifx\@tempa\@tempb\def\next{\xc@end{#2}#3\@nil}\fi}%
    \fi}
  \gdef\xc@checkend#1end[#2]#3\relax#4\relax{%
    \def\@tempa{#1}%
    \ifx\@tempa\@empty
      \@testfalse
      \def\@tempa{#2}\def\@tempb{document}%
      \ifx\@tempa\@tempb
        \def\next{\endgroup\end@xcomment\end{document}}%
      \else
        \ifx\@tempa\xc@csname
          \def\next{\end@@xcomment{#2}#3\@nil}%
    \fi\fi\fi}
\end{Verbatim}

Check for |\input| and |\endinput|.
\begin{Verbatim}
  \gdef\xc@checkinput#1input[#2]#3\relax#4\relax{%
    \def\@tempa{#1}%
    \ifx\@tempa\@empty
      \def\next{\expandafter\xcomment@\@@input #2 \xcomment@@#3\@nil}%
      \@testfalse
    \fi}
  \gdef\xc@checkendinput#1endinput#2#3\relax#4\relax{%
    \def\@tempa{#1}%
    \ifx\@tempa\@empty\ifcat\noexpand#2a\relax\else
      \let\next\endinput
    \fi\fi}
\end{Verbatim}

Check for |\input| and |\endinput|.
\begin{Verbatim}
  \gdef\xc@checkinclude#1include[#2]#3\relax#4\relax{%
    \def\@tempa{#1}%
    \ifx\@tempa\@empty
      \def\next{%
        \global\let\xc@savedinput\@@input
        \def\@@input{%
          \global\let\@@input\xc@savedinput
          \expandafter\xcomment@\@@input}%
        \include{#2}%
        \global\let\@@input\xc@savedinput
        \xcomment@@#3\@nil}%
      \@testfalse
    \fi}
   \end{Verbatim}


When exiting the |xcomment| environment, we only have to be careful to
rescan the line. Note that if get this far, all three delimiters appended to
the current line are intact. We append a space before rescanning, because of
a \meta{return} that was stripped by |\xcomment@|.
\begin{Verbatim}
  \gdef\end@@xcomment#1#2\@nnil!\@nil{%
    \endgroup
    \toks@{#2 }\rescan{\toks@}%
    \edef\next{\noexpand\end@xcomment\noexpand\end{#1}\the\toks@}%
    \next}%
\end{Verbatim}
%
%
When exiting ``comment'' mode to typeset an included environment, we have to
get rid of the delimiters added to the current line. The |\endgroup| ends
the scope of the catcodes changes. The first time an environment is typeset,
|\@envirsep| is null. Thereafter,
it is set to |\envirsep|.
\begin{Verbatim}
  \gdef\xc@end#1#2\@nnil!\@nil{%
    \endgroup
    \@envirsep \let\@envirsep\envirsep
    \toks@{#2 }\rescan{\toks@}%
    \edef\next{\noexpand\normal@begin{#1}\the\toks@}%
    \next}%
\end{Verbatim}


Now we end the lowercase command and the group that were used to insert
literal |\|, |{| and |}| characters in the code.
\begin{Verbatim}
  } END OF LOWERCASE
\endgroup
\end{Verbatim}
%
\end{document}
