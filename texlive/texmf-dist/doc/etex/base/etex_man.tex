%&latex % english hyphenation

\documentclass{article}

\usepackage{etex_man}

\begin{document}

\begin{center}
%% {\LARGE Draft, \today}\\[6pt]
{\huge\bf The \eTeX\ manual}\\[6pt]
{\sl Version 2, February 1998\/}\\[18pt]
by The \NTS\ Team\\[6pt]
Peter Breitenlohner, Max-Planck-Institut f\"ur Physik, M\"unchen\\[18pt]
\small\noindent
The preparation of this report was supported in part by {\sc Dante},\\
Deutschsprachige Anwendervereinigung \TeX\ e.V.\\
`\TeX' is a trademark of the American Mathematical Society.
\end{center}

\section{Introduction}

The \NTS\ project intends to develop an `New Typesetting System' (\NTS)
that will eventually replace today's \TeX3.  The \NTS\ program will
include many features missing in \TeX, but there will also exist a mode
of operation that is 100\% compatible with \TeX3.  It will, necessarily,
require quite some time to develop \NTS\ to maturity and make it widely
available.

Meanwhile \eTeX\ intends to fill the gap between \TeX3 and the future
\NTS. It consists of a series of features extending the capabilities of
\TeX3.%
\footnote{The \TeX3 program; for the moment there are no plans to extend
the software related to \TeX.}

Since compatibility between \eTeX\ and \TeX3 has been a main concern,
\eTeX\ has two modes of operation:\\
(1)~In \TeX\ compatibility mode it fully deserves the name \TeX\ and
there are neither extended features nor additional primitive commands.
That means in particular that \eTeX\ passes the \|TRIP| test
\cite{tripman} without any restriction.  There are, however, a few
minor modifications that would be legitimate in any implementation of
\TeX.\\
(2)~In extended mode there are additional primitive commands and the
extended features of \eTeX\ are available.

We have tried to make \eTeX\ as compatible with \TeX\ as possible
even in extended mode.  In a few cases there are, however, some subtle
differences described in detail later on.  Therefore the \eTeX\ features
available in extended mode are grouped into two categories:\\
(1)~Most of them have no semantic effect as long as none of the
additional primitives are executed; these `extensions' are permanently
enabled.\\
(2)~The remaining optional \eTeX\ features (`enhancements') can be
individually enabled and disabled; initially they are all disabled.
For each enhancement there is a state variable \|\...state|; an
enhancement is enabled or disabled by assigning a positive or
non-positive value respectively to that state variable.

For \eTeX\ Versions~1 and~2 there is just one enhancement:  mixed
direction typesetting (\TeXXeT) with the state variable \|\TeXXeTstate|.

Version~1.1 of \eTeX\ was released in November 1996, Version~2.0 in
February 1998.
It is expected that there will be about one \eTeX\ version per year,
where each later version adds new features.  It would be desirable if
these \eTeX\ versions were incorporated into many of the existing
implementations of \TeX3 without much delay.

With each \eTeX\ version there will be an \|e-TRIP| test \cite{etripman}
in order to help to verify that a particular implementation deserves the
name \eTeX\ in the same way as the \|TRIP| test \cite{tripman} helps to
verify that an implementation deserves the name \TeX.

\section{Generating \eTeX}

\subsection{Generating the \eTeX\ Program}

An implementation of \TeX\ consists of a WEB change file \|tex.ch|
containing all system-dependent changes for a particular system.  The
WEB system program \|TANGLE| applies this change file to the
system-independent file \|tex.web| defining the \TeX\ program in order to
generate a \TeX\ Pascal file for that system \cite{webman}.  Similarly
an implementation of \eTeX\ consists of a system-dependent change file
\|etex.sys| to be applied to the system-independent file \|e-tex.web|
defining the \eTeX\ program.  Since \eTeX\ differs from \TeX\ by a
relatively small fraction of its code \|e-tex.web| does, however, not
exist as a physical file; it is instead defined in terms of a
system-independent change file \|e-tex.ch| to be applied to \|tex.web|.
Similarly it should be possible to define the system-dependent change
file \|etex.sys| for a particular system in terms of its deviations
from the corresponding file \|tex.ch| \cite{etexgen}.

\subsection{Generating Format Files for \eTeX}

When (the INITEX or VIRTEX version of) the \TeX\ program is started, it
analyzes the first non-blank input line from the command line or (with
the \|**| prompt) from the terminal:  The first non-blank character of
that input line may be an \|&| followed immediately by the name of the
format to
be loaded; otherwise VIRTEX uses a default format whereas INITEX starts
without loading a format file.

For eINITEX (the INITEX version of \eTeX) there is an additional
possibility:  If the first non-blank input character is an \|*|
(immediately followed what would be the first non-blank input character
for INITEX), the program starts in extended mode without loading a
format file.  If the first non-blank character is neither \|&| nor \|*|
then eINITEX starts without loading a format but in compatibility mode.
Whenever a format file is loaded by eINITEX or eVIRTEX the mode
(compatibility or extended) is inherited from the format.

It is recommended that the input file \|etex.src| be used instead of
\|plain.tex| when generating an \eTeX\ format in extended mode.  That
file will first read \|plain.tex| (without reading \|hyphen.tex|) and
will then supply macro definitions supporting \eTeX\ features.

\section{\eTeX\ Extensions}

\subsection{Compatibility and Extended Mode}

Once \eTeX\ has entered compatibility mode it behaves as any other
implementation of \TeX.  All of \eTeX's additional commands are absent;
it is therefore impossible to access any of the extensions or
enhancements.  The ability of eINITEX to initially choose between
compatibility and extended mode is, however, by itself a feature not
present in any \TeX\ implementation.

The remainder of this document is devoted to a detailed and mostly
technical description of all aspects where \eTeX\ (in extended mode)
behaves differently from \TeX. It will be assumed that the reader is
familiar with \TeXbook\ \cite{texbook} describing \TeX's behaviour in
quite some detail.

All of \eTeX's extensions and enhancements available in extended mode are
activated by either executing some new primitive command or by assigning
a nonzero value to some new integer parameter or state variable. Since
all these new variables are initially zero,%
\footnote{To be precise all state variables are zero when eINITEX or eVIRTEX
is started; integer parameters that are not state variables are zero when
eINITEX is started without loading a format file or inherited from the
format file otherwise.}
\eTeX\ behaves as \TeX\ as long as none of \eTeX's new control sequences
are used, with the following exceptions which should, however, have
no effect on the typesetting of error-free \TeX\ documents (produced with
error-free formats):\\
(1) When \|\tracingcommands| has a value of~3 or more, or\\
when \|\tracinglostchars| has a value of~2 or more, \eTeX\ will display
additional information not available in \TeX.\\
(2) When using a count, dimen, skip, muskip, box, or token register number
in the range 256--32767, \eTeX\ will access one of its additional registers
whereas \TeX\ will produce an error and use register number zero.

\subsection{Optimization}

When a value is assigned to an \<internal quantity> within a save group,
the former value is restored when the group ends, provided the
assignment was not global. This is achieved by saving the former value
on \TeX's `save stack'. \eTeX\ refrains from creating such save stack
entries when the old and new value are the same (`reassignments').

\|\aftergroup| tokens are also kept on \TeX's save stack.  When the
current group ends, \TeX\ converts each \|\aftergroup| token into a
token list and inserts this list as new `input level' into the input stack.
\eTeX\ collects all \|\aftergroup| tokens from one group into one token
list and thus conserves input levels.

When a completed page is written to the DVI file (shipped out), \TeX\
multiplies the relevant stretch or shrink components of glue nodes in a
box by the glue expansion factor of that box and converts the product to
DVI units.  In order to avoid overflow each resulting value $x$ is
artificially limited to the range $|x|\le10^9$.
Consider the example:
\begin{verbatim}
   \shipout\vbox to100pt{
     \hrule width10pt
     \vskip 0pt plus1000fil
     \vskip 0pt plus1000fil
     \vskip 0pt plus-2000fil
     \hrule
     \vskip 0pt plus0.00005fil
     }
\end{verbatim}
Here the three glues between the two rules add up to zero; when \TeX\
converts each stretch component individually they will, however, add up
to $10^9$ DVI units due to the truncation mentioned above. \eTeX, however,
accumulates the relevant stretch or shrink components of consecutive
glue nodes (possibly separated by insert, mark, adjust, kern, and
penalty nodes) before converting them to DVI units.  During this process
glue nodes may be converted into equivalent kern nodes and some glue
specifications may be recycled; this may affect the memory usage
statistics displayed after the page has been shipped out.

\subsection{Tracing and Diagnostics}

When \|\tracingcommands| has a value of~3 or more, the commands
following a prefix (\|\global|, etc.) are shown as well, e.g.:
\begin{verbatim}
   \global\count0=0    =>    {\global}
                             {\count}
\end{verbatim}

When \|\tracinglostchars| has a value of~2 or more, missing characters
are displayed on the terminal even if the value of \|\tracingonline| is
0~or less.

When \|\tracingscantokens| has a value of~1 or more, the opening and
closing of pseudo-files (generated by \|\scantokens|) is recorded as for
any other file, with `\verb*| |' as filename.

When the program is compiled with the code for collecting statistics and
\|\tracingassigns| has a value of~1 or more, all assignments subject to
\TeX's grouping mechanism are traced, e.g.:
\begin{verbatim}
   \def\foo{\relax}    =>    {changing \foo=undefined}
                             {into \foo=macro:->\relax }
   \global\count17=7   =>    {globally changing \count17=0}
                             {into \count17=7}
   \count17=7          =>    {reassigning \count17=7}
\end{verbatim}

When \|\tracingifs| has a value of~1 or more, all conditionals
(including \|\unless|, \|\or|, \|\else|, and \|\fi|) are traced, together
with the starting line and nesting level; the \|\showifs| command displays
the state of all currently active conditionals. Thus the input
\begin{verbatim}
   \unless\iffalse
      \iffalse
      \else
         \showifs
      \fi
   \fi
\end{verbatim}
might yield
\begin{verbatim}
   {\unless\iffalse: (level 1) entered on line 1}
   {\iffalse: (level 2) entered on line 2}
   {\else: \iffalse (level 2) entered on line 2}
   ### level 2: \iffalse\else entered on line 2
   ### level 1: \unless\iffalse entered on line 1
   {\fi: \iffalse (level 2) entered on line 2}
   {\fi: \unless\iffalse (level 1) entered on line 1}
\end{verbatim}

When \|\tracinggroups| has a value of~1 or more, the start and end of
each save group is traced, together with the starting line and grouping
level; the \|\showgroups| command displays the state of all currently
active save groups. Thus the input
\begin{verbatim}
   \begingroup
      {
         \showgroups
      }
   \endgroup
\end{verbatim}
might yield
\begin{verbatim}
  {entering semi simple group (level 1) at line 1}
  {entering simple group (level 2) at line 2}
  ### simple group (level 2) entered at line 1 ({)
  ### semi simple group (level 1) entered at line 1 (\begingroup)
  ### bottom level
  {leaving simple group (level 2) entered at line 2}
  {leaving semi simple group (level 1) entered at line 1}
\end{verbatim}

Occasionally conditionals and/or save groups are not properly nested
with respect to \|\input| files.  Although this might be perfectly
legitimate, such anomalies are mostly unintentional and may cause quite
obscure errors.  When \|\tracingnesting| has a value of~1 or more,
these anomalies are shown; when \|\tracingnesting| has a value of~2 or more,
the current context (traceback) is shown as well. Thus the input
\begin{verbatim}
   \newlinechar=`\^^J
   \begingroup
      \iftrue
         \scantokens{%
      \endgroup
   ^^J\fi
   ^^J\bgroup
      ^^\tracingnesting=2
      ^^J\iffalse
      ^^J\else
        }%
     \egroup
   \fi
\end{verbatim}
might yield%
\footnote{The \cs{scantokens} command will be discussed later.}
\begin{verbatim}
Warning: end of semi simple group (level 1) entered at line 2 of
 a different file
Warning: end of \iftrue entered on line 3 of a different file
Warning: end of file when simple group (level 1) entered at line
 3 is incomplete
Warning: end of file when \iffalse\else entered on line 5 is inc
omplete
l.7 \else

l.11      }
           %
\end{verbatim}

The command \|\showtokens{|\<token list>\|}| displays the token list, and
allows the display of quantities that cannot be displayed by \|\show| or
\|\showthe|, e.g.:
\begin{verbatim}
  \showtokens\expandafter{\jobname}
  \showtokens\expandafter{\topmarks 27}
\end{verbatim}

\subsection{Status Enquiries}

A number of \TeX's internal quantities can be assigned values but
these values cannot be retrieved in \TeX. \eTeX\ introduces several new
primitives that allow the retrieval of information about its internal state.

\noindent
\|\eTeXversion| returns \eTeX's (major) version number;\\
\|\eTeXrevision| expands into a list of character tokens representing
the revision (minor version) number.  Thus
\begin{verbatim}
   \message{\number\eTeXversion\eTeXrevision}
\end{verbatim}
should write the complete version as shown when \eTeX\ is started.

\noindent
When used as number, \|\interactionmode| returns one of the
values 0~(batchmode), 1~(nonstopmode), 2~(scrollmode),
or~3~(errorstopmode).  Assigning one of these values to
\|\interactionmode| changes the current interaction mode accordingly;
such assignments are always global.

\noindent
\|\currentgrouplevel| returns the current save group level;\\
\|\currentgrouptype| returns a number representing the type of the
innermost group:
\begin{multilist}{2}{\hfil\qquad#:&\quad#\qquad\hfil}
\item 0&bottom level (no group)\cr
\item 1&simple group\cr
\item 2&hbox group\cr
\item 3&adjusted hbox group\cr
\item 4&vbox group\cr
\item 5&vtop group\cr
\item 6&align group\cr
\item 7&no align group\cr
\item 8&output group\cr
\item 9&math group\cr
\item 10&disc group\cr
\item 11&insert group\cr
\item 12&vcenter group\cr
\item 13&math choice group\cr
\item 14&semi simple group\cr
\item 15&math shift group\cr
\item 16&math left group\cr
\end{multilist}

\noindent
\|\currentiflevel| returns the number of currently active
conditionals;\\
\|\currentifbranch| indicates which branch of the innermost conditional
is taken: 1~`then branch', $-1$~`else branch', or 0~not yet decided;\\
\|\currentiftype| returns 0~if there are no active conditionals, a
positive number indicating the type of the innermost active conditional,
or the negative of that number when the conditional was prefixed by
\|\unless|:
\begin{multilist}{3}{\hfil\qquad#:&\quad#\qquad\hfil}
\item 1&\cs{if}\cr
\item 2&\cs{ifcat}\cr
\item 3&\cs{ifnum}\cr
\item 4&\cs{ifdim}\cr
\item 5&\cs{ifodd}\cr
\item 6&\cs{ifvmode}\cr
\item 7&\cs{ifhmode}\cr
\item 8&\cs{ifmmode}\cr
\item 9&\cs{ifinner}\cr
\item 10&\cs{ifvoid}\cr
\item 11&\cs{ifhbox}\cr
\item 12&\cs{ifvbox}\cr
\item 13&\cs{ifx}\cr
\item 14&\cs{ifeof}\cr
\item 15&\cs{iftrue}\cr
\item 16&\cs{iffalse}\cr
\item 17&\cs{ifcase}\cr
\item 18&\cs{ifdefined}\cr
\item 19&\cs{ifcsname}\cr
\item 20&\cs{iffontchar}\cr
\end{multilist}

\noindent
\|\lastnodetype| returns a number indicating the type of the last node,
if any, on the current (vertical, horizontal, or math) list:
\begin{multilist}{2}{\hfil\qquad#:&\quad#\qquad\hfil}
\item -1&none (empty list)\cr
\item 0&char node\cr
\item 1&hlist node\cr
\item 2&vlist node\cr
\item 3&rule node\cr
\item 4&ins node\cr
\item 5&mark node\cr
\item 6&adjust node\cr
\item 7&ligature node\cr
\item 8&disc node\cr
\item 9&whatsit node\cr
\item 10&math node\cr
\item 11&glue node\cr
\item 12&kern node\cr
\item 13&penalty node\cr
\item 14&unset node\cr
\item 15&math mode nodes\cr
\end{multilist}

\noindent
The commands \|\fontcharht|, \|\fontcharwd|, \|\fontchardp|, and
\|\fontcharic| followed by a font specification and a character code,
return a dimension: the height, width, depth, or italic correction of the
character in the font, or \[0pt] if no such character exists;
the conditional \|\iffontchar| tests the existence of that character.

\noindent
When used as number, \|\parshape| returns the number of lines of the current
parshape specification (or zero).\\
\eTeX's \|\parshapeindent|, \|\parshapelength|, and \|\parshapedimen|,
followed by a number $n$ return the dimensions of the parshape
specification:\\
\[0pt] for $n\le0$ or when no parshape is currently active, otherwise\\
\|\parshapeindent|$\,n$ and \|\parshapedimen|$\,2n-1$ both return the
indentation of line $n$ (explicitly specified or implied by repeating the
last specification),\\
\|\parshapelength|$\,n$ and \|\parshapedimen|$\,2n$ both return the length
of line $n$.

\subsection{Expressions}

\eTeX\ introduces the notion of expressions of type number, dimen, glue, or
muglue, that can be used whenever a quantity of that type is needed. Such
expressions are evaluated by \eTeX's scanning mechanism; they are initiated
by one of the commands \|\numexpr|, \|\dimexpr|, \|\glueexpr|, or \|\muexpr|
(determining the type~$t$) and optionally terminated by one \|\relax| (that
will be absorbed by the scanning mechanism). An expression consists of one
or more terms of the same type to be added or subtracted; a term of type~$t$
consists of a factor of that type, optionally multiplied and\slash or
divided by numeric factors; finally a factor of type~$t$ is either a
parenthesized subexpression or a quantity (number, etc.) of that type.
Thus, the conditional
\begin{verbatim}
  \ifdim\dimexpr (2pt-5pt)*\numexpr 3-3*13/5\relax + 34pt/2<\wd20
\end{verbatim}
is true if and only if the width of box~20 exceeds 32\[pt]. Note the use of
\|\relax| to terminate the inner (numeric) expression, the outer (dimen)
expression is terminated automatically by the token \|<|$_{12}$ that does
not fit into the expression syntax.

The arithmetic performed by \eTeX's expressions does not do much that could
not be done by \TeX's arithmetic operations \|\advance|, \|\multiply|, and
\|\divide|, although there are some notable differences: Each factor is
checked to be in the allowed range, numbers must be less than $2^{31}$ in
absolute value, dimensions or glue components must be less than
$2^{14}$\[pt], \[mu], \[fil], etc.\ respectively. The arithmetic operations
are performed individually, except for `scaling' operations (a
multiplication immediately followed by a division) which are performed as
one combined operation with a 64-bit product as intermediate value. The
result of each operation is again checked to be in the allowed range.
Finally the results of divisions and scalings are rounded, whereas \TeX's
\|\divide| truncates.

The important new feature is, however, that the evaluation of expressions
does not involve assignments and can therefore be performed in
circumstances where assignments are not allowed, e.g., inside an \|\edef| or
\|\write|. This also allows the definition of purely expandable loop constructions:
\begin{verbatim}
  \def\foo#1#2{\number#1
    \ifnum#1<#2,
      \expandafter\foo
      \expandafter{\number\numexpr#1+1\expandafter}%
      \expandafter{\number#2\expandafter}%
    \fi}
\end{verbatim}
such that, e.g., `\|\foo{7}{13}|' expands into `\|7, 8, 9, 10, 11, 12, 13|'.

The commands \|\gluestretch| and \|\glueshrink| are to be followed by a glue
specification and return the stretch or shrink component of that glue as
dimensions (with \[fil] etc.\ replaced by \[pt]), the commands
\|\gluestretchorder| and \|\glueshrinkorder| return the order of infinity:
0~for \[pt], 1~for \[fil], 2~for \[fill], and 3~for \[filll].

The commands \|\gluetomu| and \|\mutoglue| convert glue into muglue
and vice versa by simply equating 1\[pt] with 1\[mu], precisely what \TeX\
does (in addition to an error message) when the wrong kind of glue is used.

\subsection{Additional Registers and Marks}

\eTeX\ increases the number of \TeX's count, dimen, skip, muskip, box, and
token registers from 256 to 32768. The additional registers, numbered
256--32767, can be used exactly as the first 256, except that they can
not be used for insertion classes.

As in \TeX, the first 256 registers of each kind are realized as static
arrays that are part of the `table of equivalents'; values to be restored
when a save group ends are kept on the save stack. The additional registers
are realized as sparse arrays built from \TeX's main memory and are
therefore less efficient. They use a four-level index structure and
individual registers are present only when needed. Values to be restored
when a particular save group ends are kept in a linked list (again built
from main memory) with one save stack entry pointing to that list.%
\footnote{With the effect that the order of restoring (or discarding) saved
values may be somewhat surprising.}

\medskip
\eTeX\ generalizes \TeX's mark concept to mark classes 0--32767, with mark
class~0 used for \TeX's marks.\\
The command \|\marks| followed by a mark class~$n$ and a mark text appends a
mark node to the current list; \|\marks0| is synonymous with \|\mark|. The
page builder and the \|\vsplit| command record information about the mark nodes
found on the page or box produced, separately for each mark class. The
information for mark class~0 is kept in a small static array as in \TeX, the
information for the additional mark classes is again kept in a sparse array
with entries present only when needed.\\
The command \|\firstmarks|$\,n$ expands to the mark text for mark class~$n$
first encountered on the most recent page, etc., and again \|\firstmarks0|
is synonymous with \|\firstmark|.

\subsection{Input Handling}

The command \|\readline|\<number>\[to]\<control sequence> defines the
control sequence as parameterless macro whose replacement text is the
contents of the next line read from the designated file, as for \|\read|.
The difference is that the current category codes are ignored and all
characters on that line (including an endline character) are converted to
character tokens with category 12 (`other'), except that the character
code~32 gets category 10 (`space').

The command \|\scantokens{...}| absorbs a list of unexpanded tokens,
converts it into a character string that is treated as if it were an
external file, and starts to read from this `pseudo-file'. A rather
similar effect can be achieved by the commands
\begin{verbatim}
   \toks0={...}
   \immediate\openout0=file
   \immediate\write0{\the\toks0}
   \immediate\closeout0
   \input file
\end{verbatim}
In particular every occurrence of the current newline character is
interpreted as start of a new line, and input characters will be converted
into tokens as usual.
The \|\scantokens| command is, however, expandable and does not use token
registers, write streams, or external files. Furthermore the conversion from
\TeX's internal ASCII codes to external characters and back to ASCII codes
is skipped. Finally the current context (traceback) shown, e.g., as part
of an error message continues beyond an input line from a pseudo-file until
an input line from a real file (or the terminal) is found.

When \eTeX's input mechanism attempts to read beyond the end of an \|\input|
file or \|\scantokens| pseudo-file, and before checking for `runaway'
conditions and closing the file, it will first read a list of tokens that
has been predefined by the command \|\everyeof={|\<token list>\|}|.

\subsection{Breaking Paragraphs into Lines}

Traditional typesetting with lead type used to adjust (stretch or shrink)
the interword spaces in the last line of a paragraph by the same amount as
those in the preceding line. With \TeX\ the last line is, however, usually
typeset at its natural width due to infinitely stretchable parfillskip glue.
\eTeX\ allows interpolation between these two extremes by specifying a
suitable value for \|\lastlinefit|. For a value of~0 or less, \eTeX\
behaves as \TeX, values from~1 to 1000 indicate a glue adjustment fraction
$f$ times 1000, values above 1000 are interpreted as $f=1$.

The new algorithm is used only if\\
1. \|\lastlinefit| is positive;\\
2. \|\parfillskip| has infinite stretchability; and\\
3. the stretchability of \|\leftskip| plus \|\rightskip| is finite.%
\footnote{As usual for parameters influencing \TeX's line-breaking algorithm,
the values current at the end of the (partial) paragraph are used.}\\
Thus the last line of a paragraph would normally be typeset at its
natural width and the stretchability of parfillskip glue would be used to
achieve the desired line width. The algorithm proceeds as usual, considering
all possible sequences of feasible break points and accumulating demerits for
the stretching or shrinking of lines as well as for visually incompatible
lines. When a candidate for the last line has been reached, the following
conditions are tested:\\
4. the previous line was not `infinitely bad' and was stretched with positive
finite stretchability or was shrunk with positive shrinkability;\\
5. the last line has infinite stretchability entirely due to parfillskip
glue;\\
6. if the previous line was stretched or shrunk the last line has
positive finite stretchability or shrinkability respectively.\\
If all three conditions are satisfied, a glue adjustment factor of $f$ times
that of the preceding line will be applied to the relevant stretch or
shrink components of all glue nodes in the last line, and the corresponding
demerits are computed. (The last line will, however, not be stretched beyond
the desired line width.)

When all possible candidates for the last line of the paragraph have been
examined, the one having fewest accumulated demerits is chosen. If \eTeX's
modified algorithm was applied to that last line, the actual stretching or
shrinking is achieved by suitably modifying the parfillskip glue node.

All computations described so far are performed with machine-independent
integer arithmetic. Note, however, that the actual stretching requires
machine-dependent floating point arithmetic. Therefore, when a paragraph is
interrupted by a displayed equation and the line preceding the display is
subject to the adjustment just described, the display will in general be
preceded by abovedisplayskip and not by abovedisplayshortskip glue.

\medskip

After breaking a paragraph into lines, \TeX\ computes the interline
penalties by adding the values of:\\
\|\interlinepenalty| between any two lines,\\
\|\clubpenalty| after the first line of a (partial) paragraph,\\
\|\widowpenalty| before the last line of the paragraph,\\
\|\displaywidowpenalty| before the line immediately preceding a displayed
equation, and\\
\|\brokenpenalty| after lines ending with a discretionary break.\\
\eTeX\ generalizes the concept of interline, club, widow, and display widow
penalty by allowing their replacement by arrays of penalty values with the
commands\\
\|\interlinepenalties|,\\
\|\clubpenalties|,\\
\|\widowpenalties|, and\\
\|\displaywidowpenalties|.\\
Each of these commands is to be followed by an optional equal sign and a
number $n$.  If $n\le0$ the respective array is reset and \TeX's
corresponding single value is used as usual; a positive value $n$
declares an array of length $n$ and must be followed by $n$ penalty
values.  When one of these arrays has been set, its values are used
instead of \TeX's corresponding single values as follows (repeating the
last value when necessary):\\
the $i^{\rm th}$ interline penalty value is used after line $i$ of the
paragraph;\\
the $i^{\rm th}$ club penalty value is used after line $i$ of a partial
paragraph;\\
the $i^{\rm th}$ widow penalty value is used after line $m-i$ of a
paragraph without displayed equations or the last partial paragraph of
length $m$;\\
the $i^{\rm th}$ display widow penalty value is used after line $m-i$ of a
partial paragraph of length $m$ that is followed by a displayed equation.

When used after \|\the| or in situations where \TeX\ expects to see a
number, the same four commands serve to retrieve the arrays of penalties.
Specifying, e.g., \|\clubpenalties|\<number> with a number $n$, returns~0
for $n<0$ or when the club penalty array has been reset, the length of the
declared club penalty array for $n=0$, or the $n^{\rm th}$ club penalty
value for $n>0$ (again repeating the last  value when necessary).

\subsection{Math Formulas}

\TeX's \|\left|\<delimiter>\|...\right|\<delimiter> produces two delimiters
with a common size adjusted to the height and depth of the enclosed material.
In \eTeX\ this can be generalized by occurrences of \|\middle|\<delimiter>
dividing the enclosed material into segments resulting in a sequence
of delimiters with a common size adjusted to the maximal height and depth of
all enclosed segments. The spacing between a segment and the delimiter to
its left or right is as for \TeX's left or right delimiter respectively.

\subsection{Hyphenation}

\TeX\ uses the \|\lccode| values for two quite unrelated purposes:\\
(1) when \|\lowercase| converts character tokens to their lower-case
equivalents (in the same way as \|\uppercase| uses the \|\uccode| values);
and\\
(2) when hyphenation patterns or exceptions are read, and when words are
hyphenated during the line-breaking algorithm.

\eTeX\ introduces the concept of (language-dependent) hyphenation codes that
are used instead of the \|\lccode| values for hyphenation purposes. In order
to explain the details of \eTeX's behaviour, we need some technical aspects
of hyphenation patterns. When INITEX starts without reading a format file,
the (initially empty) hyphenation patterns are in a form suitable for
inserting new patterns specified by \|\patterns| commands; when INITEX
attemps hyphenation or prepares to write a format file, they are compressed
into a more compact form suitable for finding hyphens. Only these compressed
patterns can be read from a format file (by INITEX or VIRTEX).

In \eTeX\ the hyphenation patterns are supplemented by hyphenation codes.
When eINITEX starts without reading a format file both are initially empty;
when a \|\patterns| command is executed and \|\savinghyphcodes| has a positive
value, the current \|\lccode| values are saved as hyphenation codes for the
current language. These saved hyphenation codes are later compressed together
with the patterns and written to or read from a format file. When the
patterns have been compressed (always true for eVIRTEX) and hyphenation
codes have been saved for the current language, they are used
instead of the \|\lccode| values for hyphenation purposes (reading
hyphenation exceptions and hyphenating words).

\subsection{Discarded Items}

When \TeX's page builder transfers (vertical mode) material from the `recent
contributions' to the `page so far', it discards glue, kern, and penalty
nodes (discardable items) preceding the first box or rule on the page under
construction and inserts a topskip glue node immediately before that box or
rule. Note, however, that this topskip glue need not be the first node on
the page, it may be preceded by insertion, mark, and whatsit nodes.
Similarly when the \|\vsplit| command has split the first part off a vbox,
discardable items are discarded from the top of the remaining vbox and a
splittopskip glue node is inserted immediately before the first box or rule.

When \eTeX's parameter \|\savingvdiscards| has been assigned a positive
value, these `discarded items' are saved in two lists and can
be recovered by the commands \|\pagediscards| and \|\splitdiscards| that
act like `unvboxing' hypothetical box registers containing a vbox with the
discarded items.

The list of items discarded by the page builder is emptied at the end of
the output routine and by the \|\pagediscards| command; new items may
be added as long as the new `page so far' contains no box or rule.

The list of items discarded by the \|\vsplit| command is emptied at the
start of a vsplit operation and by the \|\splitdiscards| command; new items
are added at the end of a vsplit operation.

\subsection{Expandable Commands}

Chapter~20 of \TeXbook\ gives complete lists of all expandable \TeX\
commands and of all cases where expandable tokens are not expanded.
For \eTeX\ there are these additional conditionals:

\begin{itemize}
\item
\|\ifdefined|\<token>\quad(test if token is defined)
\end{itemize}
\noindent
True if \<token> is defined; creates no new hash table entry.

\begin{itemize}
\item
\|\ifcsname...\endcsname|\quad(test if control sequence is defined)
\end{itemize}
\noindent
True if the control sequence \|\csname...\endcsname| would be defined;
creates no new hash table entry.

\begin{itemize}
\item
\|\iffontchar|\<font>\<8-bit number>\quad(test if char exists)
\end{itemize}
\noindent
True if \|\char|\<8-bit number> in \|\font|\<font> exists.

These are \eTeX's additional expandable commands:

\begin{itemize}
\item\|\unless|.\\
The next (unexpanded) token must be a boolean conditional
(i.e., not \|\ifcase|); the truth value of that conditional is reversed.

\item\|\eTeXrevision|.\\
The expansion is a list of character tokens of category 12 (`other')
representing \eTeX's revision (minor version) number, e.g., `.0' or
`.1'.

\item\|\topmarks|\<15-bit number>,
\|\firstmarks|\<15-bit number>,\\
\|\botmarks|\<15-bit number>,
\|\splitfirstmarks|\<15-bit number>, and\\
\|\splitbotmarks|\<15-bit number>.\\
These commands generalize \TeX's \|\topmark| etc.\ to 32768 distinct
mark classes; the special case \|\topmarks0| is synonymous with
\|\topmark| etc.

\item\|\unexpanded|\<general text>.\\
The expansion is the token list \<balanced text>.

\item\|\detokenize|\<general text>.\\
The expansion is a list of character tokens representing the token list
\<balanced text>. As with the lists of character tokens produced by \TeX's
\|\the| and \eTeX's \|\readline|, these tokens have category 12 (`other'),
except that the character code~32 gets category 10 (`space').

\item\|\scantokens|\<general text>.\\
The expansion is null; but \eTeX\ creates a pseudo-file containing the
characters representing the token list \<balanced text> and prepares to
read from this pseudo-file before looking at any more tokens from its
current source.

\end{itemize}

These are the additional \eTeX\ cases when expandable tokens are not
expanded:

\begin{itemize}
\item
When \eTeX\ is reading the argument token for \|\ifdefined|.

\item
When \eTeX\ is absorbing the token list for \|\unexpanded|,
\|\detokenize|, \|\scantokens|, or \|\showtokens|.

\item
Protected macros (defined with the \|\protected| prefix) are not
expanded when building an expanded token list (for \|\edef|, \|\xdef|,
\|\message|,\\
\|\errmessage|, \|\special|, \|\mark|, \|\marks| or when writing the
token list for \|\write| to a file) or when looking ahead in an
alignment for \|\noalign| or \|\omit|.%
\footnote{Whereas protected macros were introduced with \eTeX\ Version~1,
suppression of their expansion in alignments was introduced with Version~2.}

\item
When building an expanded token list, the tokens resulting from the
expansion of \|\unexpanded| are not expanded further (this is the same
behaviour as is exhibited by the tokens resulting from the expansion of
\|\the|\<token variable> in both \TeX\ and \eTeX).

\end{itemize}

\section{\eTeX\ Enhancements}

The execution of most new primitives related to enhancements is
disallowed when the corresponding enhancement is currently disabled and
will lead to an `\|Improper...|' error message.  The offending command
may nevertheless already have had some effect such as, e.g., bringing
\eTeX\ into horizontal mode.

\subsection{Mixed-Direction Typesetting}

This feature supports mixed left-to-right and right-to-left typesetting
and introduces the four text-direction primitives \|\beginL|, \|\endL|,
\|\beginR|, and \|\endR|.  The code is inspired by but different from
\TeXeT\ \cite{texet}.

In order to avoid confusion with \TeXeT\ the present implementation of
mixed-direction typesetting is called \TeXXeT.  It uses the same text-direction
primitives, but differs from \TeXeT\ in several important aspects:\\
(1)~Right-to-left text is reversed explicitly by \eTeX\ and is written
to a normal DVI file without any \|begin_reflect| or \|end_reflect|
commands;\\
(2)~a math node is (ab)used instead of a whatsit node to record the text-direction
primitives in order to minimize the influence on the line-breaking
algorithm for pure left-to-right text;\\
(3)~right-to-left text interrupted by a displayed equation is
automatically resumed after that equation;\\
(4)~display math material is always printed left-to-right, even in
constructions such as:
\begin{verbatim}
   \hbox{\beginR\vbox{\noindent$$abc\eqno(123)$$}\endR}
\end{verbatim}

\TeXXeT\ is enabled or disabled by assigning a positive or non-positive
value respectively to the \|\TeXXeTstate| state variable.  As long as
\TeXXeT\ is disabled, \eTeX\ and \TeX3 build horizontal lists and
paragraphs in exactly the same way.  Even \TeXXeT\ will, in general,
produce the same results as \TeX3 for pure left-to-right text.  There
are, however, circumstances where some differences may arise.  This is
best illustrated by an example:
\begin{verbatim}
   \vbox{\noindent
      $\hfil\break
      \null\hfil\break
      \null$\par
\end{verbatim}
Here \TeX\ will produce three lines containing the following nodes:\\
1. mathon, hfil glue, break penalty, and rightskip glue;\\
2. empty hbox, hfil glue, break penalty, and rightskip glue;\\
3. empty hbox, mathoff, nobreak penalty, parfillskip glue, and rightskip
   glue.\\
These lines can be retrieved via:
\begin{verbatim}
      \setbox3=\lastbox
      \unskip\unpenalty
      \setbox2=\lastbox
      \unskip\unpenalty
      \setbox1=\lastbox
\end{verbatim}
Later on these lines can be `unhboxed' as part of a new paragraph and
possibly their contents analyzed.  As a consequence in \TeX\ (and \eTeX\
in compatibility mode) there may be horizontal lists where mathon
and mathoff nodes are not properly paired.  Therefore \TeX\ might
attempt hyphenation of `words' originating from math mode or prevent
hyphenation of words originating from horizontal mode.

Math-mode material is always typeset left-to-right by \TeXXeT, even when
it is contained inside right-to-left text.  Therefore \TeXXeT\ will
insert additional \|beginM| and \|endM| math nodes such that
material originating from math mode is always enclosed between properly
paired math nodes.  Consequently \TeXXeT\ will never attempt hyphenation
of `words' originating from math mode nor prevent hyphenation of words
originating from horizontal mode.

The additional math nodes introduced by \TeXXeT\ are, however,
transparent to operations such as \|\lastpenalty| that inspect or remove
the last node of a horizontal list.%
\footnote{This was not the case for some earlier \TeXXeT\ implementations.}

When \TeXXeT\ is enabled or disabled during the construction of a box,
that box may contain text-direction directives or math nodes that are
not properly paired.  Such unpaired nodes may cause warning messages
when the box is shipped out.  It is, therefore, advisable that \TeXXeT\
be enabled or disabled only in vertical mode.

\section{Syntax Extensions for \eTeX}

\subsection{Mode-Independent Commands}

The syntax for \TeX's mode-independent commands, as described in the
first part of Chapter~24 of \TeXbook, is extended by modifications of
existing commands as well as by new commands.

First, \eTeX\ has 32768 \|\count|, \|\dimen|, \|\skip|, \|\muskip|,
\|\box|, and \|\toks| registers instead of \TeX's 256.  Thus it allows
a \<15-bit number> instead of an \<8-bit number> in almost all syntax
constructions referring to these registers; the only exception to this is
the \|\insert| command:  insertion classes are restricted to the range
0--254 in \eTeX\ as they are in \TeX.

Next, \eTeX\ extends the list of \TeX's internal quantities:
\begin{syntax}
<internal integer>\is\more  \alt|\eTeXversion|
  \alt|\interactionmode|\alt<penalties><number>
  \alt|\lastnodetype|\alt|\currentgrouplevel|\alt|\currentgrouptype|
  \alt|\currentiflevel|\alt|\currentiftype|\alt|\currentifbranch|
  \alt|\gluestretchorder|<glue>\alt|\glueshrinkorder|<glue>
  \alt|\numexpr|<integer expr><optional spaces and |\relax|>
<penalties>\is|\interlinepenalties|\alt|\clubpenalties|
  \alt|\widowpenalties|\alt|\displaywidowpenalties|
<internal dimen>\is\more
  \alt|\parshapeindent|<number>\alt|\parshapelength|<number>
  \alt|\parshapedimen|<number>
  \alt|\gluestretch|<glue>\alt|\glueshrink|<glue>
  \alt|\fontcharht|<font><8-bit number>%
  \alt|\fontcharwd|<font><8-bit number>
  \alt|\fontchardp|<font><8-bit number>%
  \alt|\fontcharic|<font><8-bit number>
  \alt|\dimexpr|<dimen expr><optional spaces and |\relax|>
<internal glue>\is\more  \alt|\mutoglue|<muglue>
  \alt|\glueexpr|<glue expr><optional spaces and |\relax|>
<internal muglue>\is\more  \alt|\gluetomu|<glue>
  \alt|\muexpr|<muglue expr><optional spaces and |\relax|>
\end{syntax}

The additional possibilities for \<integer parameter> are:
\begin{paramlist}
\|\TeXXeTstate|\quad(positive if mixed-direction typesetting is enabled)

\|\tracingassigns|\quad(positive if showing assignments)

\|\tracinggroups|\quad(positive if showing save groups)

\|\tracingifs|\quad(positive if showing conditionals)

\|\tracingscantokens|\quad(positive
   if showing the opening and closing of \|\scantokens| pseudo-files)

\|\tracingnesting|\quad(positive
   if showing improper nesting of groups and conditionals within files)

\|\predisplaydirection|\quad(text direction preceding a display)

\|\lastlinefit|\quad(adjustment
   ratio for last line of paragraph, times 1000)

\|\savingvdiscards|\quad(positive
   if saving items discarded from vertical lists)

\|\savinghyphcodes|\quad(positive
   if \|\patterns| saves \|\lccode| values as hyphenation codes)
\end{paramlist}
\noindent
Note that the \eTeX\ state variable \|\TeXXeTstate| (the only one so
far) is an \<integer parameter>.  That need not be the case for all
future state variables; it might turn out that some future enhancements
can be enabled and disabled only globally, not subject to grouping.

The additional possibilities for \<token parameter> are:
\begin{paramlist}
\|\everyeof|\quad(tokens to insert when an \|\input| file ends)
\end{paramlist}

Here is the syntax for \eTeX's expressions:
\begin{syntax}
<integer expr>\is<integer term>
  \alt<integer expr><add or sub><integer term>
<integer term>\is<integer factor>
  \alt<integer term><mul or div><integer factor>
<integer factor>\is<number>
  \alt<left paren><integer expr><right paren>
<dimen expr>\is<dimen term>
  \alt<dimen expr><add or sub><dimen term>
<dimen term>\is<dimen factor>
  \alt<dimen term><mul or div><integer factor>
<dimen factor>\is<dimen>
  \alt<left paren><dimen expr><right paren>
<glue expr>\is<glue term>
  \alt<glue expr><add or sub><glue term>
<glue term>\is<glue factor>
  \alt<glue term><mul or div><integer factor>
<glue factor>\is<glue>
  \alt<left paren><glue expr><right paren>
<muglue expr>\is<muglue term>
  \alt<muglue expr><add or sub><muglue term>
<muglue term>\is<muglue factor>
  \alt<muglue term><mul or div><integer factor>
<muglue factor>\is<muglue>
  \alt<left paren><muglue expr><right paren>
<optional spaces and |\relax|>\is<optional spaces>
  \alt<optional spaces>|\relax|
<add or sub>\is<optional spaces>\ot+\alt<optional spaces>\ot-
<div or mul>\is<optional spaces>\ot*\alt<optional spaces>\ot/
<left paren>\is<optional spaces>\ot(
<right paren>\is<optional spaces>\ot)
\end{syntax}

Next, \eTeX\ extends the syntax for assignments:
\begin{syntax}
<prefix>\is\more\alt|\protected|
<simple assignment>\is\more
  \alt<penalties assignment>
  \alt|\readline|<number>[to]<control sequence>
<penalties assignment>\is%
  <penalties><equals><number><penalty values>
<interaction mode assignment>\is\more
  \alt|\interactionmode|<equals><2-bit number>
\end{syntax}
\noindent
In a \<penalties assignment> for which the \<number> is $n$, the
\<penalty values> are \<empty> if $n\le0$, otherwise they consist of $n$
consecutive occurrences of \<number>.

Finally, the remaining mode-independent \eTeX\ commands:

\begin{itemize}
\item
\|\showgroups|, \|\showifs|, \|\showtokens|\<general text>.
These commands are intended to help you figure out what \eTeX\ thinks it
is doing.
The \|\showtokens| command displays the token list \<balanced text>.

\item
\|\marks|\<15-bit number>\<general text>.  This command generalizes
\TeX's \|\mark| command to 32768 distinct mark classes; the special case
\|\marks0| is synonymous with \|\mark|.

\end{itemize}

\subsection{Vertical-Mode Commands}

The syntax for \TeX's vertical-mode commands, as described in the second
part of Chapter~24 of \TeXbook, is extended by \eTeX\ as follows:

\begin{itemize}
\item
\|\pagediscards|, \|\splitdiscards|.
These two commands are similar to \|\unvbox|.
When \|\savingvdiscards| is positive, items discarded by the page
builder and by the \|\vsplit| command are collected in two special
lists.  One of these special lists is appended to the current vertical
list (in the same way as \|\unvbox| appends the vertical list inside a
vbox) and becomes empty.

\item
Here are the additional possibilities for \<horizontal command>:
\begin{syntax}
<horizontal command>\is\more
  \alt|\beginL|\alt|\endL|\alt|\beginR|\alt|\endR|
\end{syntax}

\end{itemize}

\subsection{Horizontal-Mode Commands}

The syntax for \TeX's horizontal-mode commands, as described in
Chapter~25 of \TeXbook, is extended by \eTeX\ as follows:

\begin{itemize}
\item
Here are the additional possibilities for \<vertical command>:
\begin{syntax}
<vertical command>\is\more
  \alt|\pagediscards|\alt|\splitdiscards|
\end{syntax}

\item
\|\beginL|, \|\endL|, \|\beginR|, \|\endR| (text-direction commands).\\
The use of these commands is illegal when the \TeXXeT\ enhancement is
currently disabled; otherwise a \|beginL|, etc.\ text-direction node (a
new kind of math node) is appended to the current horizontal list.
These nodes delimit the beginning and end of hlist segments containing
left-to-right~(L) or right-to-left~(R) text.  Before a paragraph is
broken into lines, \|endL| and \|endR| nodes are added to terminate any
unfinished L~or R~segments; when a paragraph is continued after display
math mode, any such unfinished segments are automatically resumed,
starting the new hlist with \|beginL| and \|beginR| nodes as necessary.

\item
\|\marks|\<15-bit number>\<general text>.  This command generalizes
\TeX's \|\mark| command to 32768 distinct mark classes; the special case
\|\marks0| is synonymous with \|\mark|.

\end{itemize}

\subsection{Math-Mode Commands}

The syntax for \TeX's math-mode commands, as described in Chapter~26 of
\TeXbook, is extended by \eTeX\ as follows:

\begin{itemize}
\item
\|\left|\<delim>\<math mode material>\\
\|\middle|\<delim>\<math mode material>\|...|\|\right|\<delim>\\
(generalizing \TeX's
\|\left|\<delim>\<math mode material>\|\right|\<delim>).\\
For each \<math mode material> \eTeX\ begins a new group, starting out
with a new math list (always in the same style) that begins with a left
boundary item containing everything processed so far.  This group must
be terminated with either `\|\middle|' or `\|right|', at which time the
internal math list is completed with a new boundary item containing the
new delimiter.  In the case of `\|\middle|', a new group is started
again, in the case of `\|\right|', \eTeX\ appends an Inner atom to the
current list; the nucleus of this atom contains the internal math list
just completed.

\end{itemize}

\begin{thebibliography}{9}

\bibitem{tripman}
{\sl A torture test for \TeX\/},
by Donald E.~Knuth, Stanford Computer Science Report~1027.

\bibitem{etripman}
{\sl A torture test for \eTeX\/},
by The \NTS\ Team (Peter Breitenlohner and Bernd Raichle).
Version~2, January 1998.

\bibitem{webman}
{\sl The WEB system of structured documentation\/},
by Donald E.~Knuth,\hfil\break Stanford Computer Science Report~980.

\bibitem{etexgen}
{\sl How to generate \eTeX\/},
by The \NTS\ Team (Peter Breitenlohner and Phil Taylor).
Version~2, January 1998.

\bibitem{texbook}
\TeXbook\ (Computers and Typesetting, Vol.~A),
by Donald E.~Knuth,
Addison Wesley, Reading, Massachusetts, 1986.

\bibitem{texet}

{\sl Mixing right-to-left texts with left-to-right texts\/},
by Donald~E. Knuth and Pierre MacKay,
{\sl TUGboat\/} {\bf 8}, 14--25, 1987.

\end{thebibliography}

\end{document}

