%!TEX root = /Users/ego/Boulot/Alterqcm/doc/doc_aq-main.tex   

\section{Installation.}

\subsection{Avec TeXlive sous Linux ou OS X}\NameDist{TeXLive}\NameSys{Linux}
\NameDist{MacTeX}\NameSys{OS X}
\tkzname{alterqcm}  est présent sur les serveurs du \tkzname{CTAN} et fait   partie de \tkzname{TeXLive} alors  \tkzname{tlmgr} vous permettra de l'installer.  Si  \tkzname{alterqcm} ne fait pas encore partie de votre distribution, cette section vous montre comment l'installer, elle est aussi nécessaire si vous avez envie d'installer une version beta  ou personnalisée de \tkzname{alterqcm}. 

Le plus simple est de créer un dossier \tikz[remember picture,baseline=(n1.base)]\node [fill=blue!30,draw] (n1) {prof};\footnote{ou bien un autre nom}  avec comme chemin : \colorbox{blue!20}{ texmf/tex/latex/prof}. Voici les chemins de ce dossier sur mes deux ordinateurs:

\medskip
\begin{itemize}\setlength{\itemsep}{5pt}

\item   sous OS X \colorbox{blue!30}{\textbf{/Users/ego/Library/texmf}}; 

\item   sous Ubuntu \colorbox{blue!30}{\textbf{/home/ego/texmf}}.
\end{itemize}

Je suppose que si vous mettez vos packages ailleurs, vous savez pourquoi !

L'installation que je propose n'est valable que pour un utilisateur. 

\medskip
\begin{enumerate}
\item Téléchargez le fichier \tikz[remember picture,baseline=(n2.base)]\node [fill=blue!20,draw] (n2) {alterqcm.sty}; sur l'un des serveurs  du \tkzname{CTAN}.

\item Placez le fichier \tikz[remember picture,baseline=(n2.base)]\node [fill=blue!20,draw] (n2) {alterqcm.sty}; dans le dossier \tkzname{latex}  ou bien  dans un dossier personnel \tikz[baseline=(tk.base)]\node [fill=blue!30,draw] (tk) {prof};.
\begin{itemize}\setlength{\itemsep}{5pt}

\item  \colorbox{blue!30}{\textbf{\texttildelow/Library/texmf/latex}}; 

\item   \colorbox{blue!30}{\textbf{\texttildelow/Library/texmf/latex/prof}}.
\end{itemize}   


\item Ouvrir un terminal, puis faire \colorbox{red!20}{|sudo texhash|} si nécessaire.

\end{enumerate} 


\subsection{Avec MikTeX sous Windows XP}\NameDist{MikTeX}\NameSys{Windows XP}


Je ne connais pas grand-chose à ce système, mais un utilisateur de mes packages \tkzimp{Wolfgang Buechel} a eu la gentillesse de me faire parvenir ce qui suit~:

Pour ajouter \tkzname{alterqcm.sty} à MiKTeX\footnote{Essai réalisé avec la version \tkzname{2.7}}:

\begin{itemize}\setlength{\itemsep}{10pt}
  \item ajouter un dossier \tkzname{prof} dans le dossier
       \textcolor{blue!60!black}{\texttt{[MiKTeX-dir]/tex/latex}}
  \item copier  le fichier  \tkzname{alterqcm.sty} dans le dossier \tkzname{prof},
  \item mettre à jour  MiKTeX, pour cela dans shell DOS lancer la commande   \textbf{\textcolor{red}{|mktexlsr -u|}} 
  
   ou bien encore, choisir \textcolor{red!50}{|Start/Programs/Miktex/Settings/General|}
   
    puis appuyer sur le bouton  \textbf{\textcolor{red}{|Refresh FNDB|}}.
\end{itemize}      

\endinput

