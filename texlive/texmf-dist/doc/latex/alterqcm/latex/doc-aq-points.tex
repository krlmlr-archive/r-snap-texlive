\section{Points attibués à un QCM}

Il est possible d'attribuer des points à un QCM à l'aide de la macro rudimentaire suivante \tkzcname{AQpoints}

\begin{tkzltxexample}[small]
 \newcommand\AQpoints[1]{%
 \marginpar{\hspace*{1em}    
 \begin{tabular}{|c|}
   \hline  
   \textbf{#1}\\ 
   \hline\\ 
   \hline 
 \end{tabular}}} 
\end{tkzltxexample}
  
\subsection{Exemple} 

\begin{tkzltxexample}[]
    \AQpoints{10}
  \begin{alterqcm}[symb = \dingsquare, lq=7cm]
  \AQquestion{Si \numprint{3,24} est la troncature de $x$ au centième, alors on est sûr que :}
  {{\begin{minipage}[t]{\linewidth-1cm}$3,235\leqslant x <3,245$\\
    \end{minipage}} ,
   {\begin{minipage}[t]{\linewidth-1cm} $3,24\leqslant x <3,25$\\
    \end{minipage}} ,
   {\begin{minipage}[t]{\linewidth-1cm}
       $x$ est plus près de \numprint{3,24} que de \numprint{3,25}
    \end{minipage}}}
  \end{alterqcm}
\end{tkzltxexample}

\medskip
\AQpoints{10}
  \begin{alterqcm}[symb = \dingsquare, lq=7cm]
  \AQquestion{Si \numprint{3,24} est la troncature de $x$ au centième, alors on est sûr que :}
  {{\begin{minipage}[t]{\linewidth-1cm}$3,235\leqslant x <3,245$\\
    \end{minipage}} ,
   {\begin{minipage}[t]{\linewidth-1cm} $3,24\leqslant x <3,25$\\
    \end{minipage}} ,
   {\begin{minipage}[t]{\linewidth-1cm}
       $x$ est plus près de \numprint{3,24} que de \numprint{3,25}
    \end{minipage}}}
\end{alterqcm}   
\endinput  