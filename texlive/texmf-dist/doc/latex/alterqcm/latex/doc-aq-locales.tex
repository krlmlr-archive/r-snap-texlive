%!TEX root = /Users/ego/Boulot/Alterqcm/doc/doc_aq-main.tex 
\section{Options locales de la macro  \tkzcname{AQquestion}}

\subsection{Utilisation locale de \tkzname{pq}}
\Iopt{AQquestion}{pq} 
 Le tableau suivant est obtenu avec comme options |lq=85mm| et  |size=\large|. Les questions sont mal positionnées. L'option locale \tkzname{pq} résout ce problème, le texte peut être déplacé de 1mm vers le haut avec  \tkzcname{AQquestion[pq=1mm]}. 
  et de |6mm| pour la seconde.

\medskip 


 \begin{alterqcm}[lq=55mm,size=\large]

\AQquestion{Si la fonction $f$ est strictement croissante sur $\mathbf{R}$ alors l'équation $f(x) = 0$ admet :}
{{Au moins une solution},
{Au plus une solution},
{Exactement une solution}
}
\AQquestion{Si la fonction $f$ est continue et positive sur $[a~ ;~ b]$ et $\mathcal{C}_{f}$ sa courbe représentative dans un repère orthogonal. En unités d'aire, l'aire $\mathcal{A}$ du domaine délimité par $\mathcal{C}_{f}$, l'axe des abscisses et les droites d'équations $x =  a$ 5 et $x = b$ est donnée par la formule : }
{%
{$\mathcal{A}= \displaystyle \int_{b}^a f(x)\ \text{d}x$},
{$\mathcal{A}= \displaystyle \int_{a}^b f(x)\ \text{d}x$},
{$\mathcal{A} = f(b) - f(a)$}}
\end{alterqcm}

\medskip 
\tkzname{Voici la version corrigée}

\begin{alterqcm}[lq=55mm,size=\large]
\AQquestion[pq=1mm]{Si la fonction $f$ est strictement croissante sur 
$\mathbf{R}$ alors l'équation $f(x) = 0$ admet :}
{{Au moins une solution},
{Au plus une solution},
{Exactement une solution}
}
\AQquestion[pq=6mm]{Si la fonction $f$ est continue et positive sur $[a~ ;~ b]$
 et $\mathcal{C}_{f}$ sa courbe représentative dans un repère orthogonal.
  En unités d'aire, l'aire $\mathcal{A}$ du domaine délimité par $\mathcal{C}_{f}$,
   l'axe des abscisses et les droites d'équations $x =  a$ et $x = b$ est donnée
    par la formule : }
{{$\mathcal{A}= \displaystyle \int_{b}^a f(x)\ \text{d}x$},
{$\mathcal{A}= \displaystyle \int_{a}^b f(x)\ \text{d}x$},
{$\mathcal{A} = f(b) - f(a)$}
}
\end{alterqcm}

\medskip  
\begin{tkzexample}[code only, small]  
 \begin{alterqcm}[lq=55mm,size=\large]
 \AQquestion[pq=1mm]{Si la fonction $f$ est strictement croissante sur 
 $\mathbf{R}$ alors l'équation $f(x) = 0$ admet :}
 {{Au moins une solution},
 {Au plus une solution},
 {Exactement une solution}}
\end{tkzexample}

\medskip
\begin{tkzexample}[code only, small]  
 \AQquestion[pq=6mm]{Si la fonction $f$ est continue et positive sur $[a~ ;~ b]$
  et $\mathcal{C}_{f}$ sa courbe représentative dans un repère orthogonal.
   En unités d'aire, l'aire $\mathcal{A}$ du domaine délimité par $\mathcal{C}_{f}$,
    l'axe des abscisses et les droites d'équations $x =  a$ et $x = b$ est donnée
     par la formule : }
 {{$\mathcal{A}= \displaystyle \int_{b}^a f(x)\ \text{d}x$},
 {$\mathcal{A}= \displaystyle \int_{a}^b f(x)\ \text{d}x$},
 {$\mathcal{A} = f(b) - f(a)$}}
 \end{alterqcm} 
\end{tkzexample}

\subsection{Utilisation globale et locale de \tkzname{pq}}\
 \Iopt{AQquestion}{pq} \IoptEnv{alterqcm}{pq}
Cette fois, il est nécessaire de déplacer plusieurs questions, j'ai placé un |pq=2mm| globalement c'est à dire comme ceci :\tkzcname{begin\{alterqcm\}[lq=85mm,pq=2mm]}. \textbf{Toutes} les questions sont affectées par cette option mais certaines questions  étaient bien placées et doivent le rester, aussi localement je leur redonne un |pq=0mm|.

\medskip
\begin{alterqcm}[lq=85mm,pq=2mm]
\AQquestion{Soit une série statistique à deux variables. Les valeurs de $x$ sont 1, 2, 5, 7, 11, 13 et une équation de la droite de régression de $y$ en $x$ par la moindres carrés est $y = 1,35x +22,8$. Les coordonnées du point moyen sont :}
{{$(6,5;30,575)$},
{$(32,575 ; 6,5)$},
{$(6,5 ; 31,575)$}}

\AQquestion[pq=0mm]{$(u_{n})$ est une suite arithmétique de raison $-5$.\\ Laquelle de ces affirmations est exacte ? }
{{Pour tout entier $n,~  u_{n+1} - u_{n} = 5$},
{$u_{10}= u_{2}+ 40$},
{$u_{3} = u_{7} + 20$}
}
\AQquestion[pq=0mm]{L'égalité $\ln (x^2 - 1) = \ln (x - 1) + \ln (x+1)$ est vraie}
{{Pour tout $x$ de  $]- \infty~;~-1[ \cup]1~;~+ \infty[$},
{Pour tout $x$ de $\mathbf{R} - \{-1~ ;~ 1\}$.},
{Pour tout $x$ de $]1~ ;~+\infty[$}
}
\AQquestion{Pour tout réel $x$, le nombre \[\dfrac{\text{e}^x - 1}{\text{e}^x + 2}\hskip12pt \text{égal à :} \] }
{{$-\dfrac{1}{2}$},
{$\dfrac{\text{e}^{-x} - 1}{\text{e}^{-x} + 2}$},
{$\dfrac{1 - \text{e}^{-x}}{1 + 2\text{e}^{-x}}$}
}
\AQquestion{On pose I $= \displaystyle\int_{\ln 2}^{\ln 3} \dfrac{1}{\text{e}^x - 1}\,\text{d}x$ et J $ = \displaystyle\int_{\ln 2}^{\ln 3} \dfrac{\text{e}^x}{\text{e}^x - 1}\,\text{d}x$ \\ alors le nombre  I $-$ J est égal à}
{{$\ln \dfrac{2}{3}$},
{$\ln \dfrac{3}{2}$},
{$\dfrac{3}{2}$}
}
\end{alterqcm}

\medskip
\begin{tkzexample}[code only,vbox,small]
 \begin{alterqcm}[lq=85mm,pq=2mm]
  \AQquestion[pq=0mm]{L'égalité $\ln (x^2 - 1) = \ln (x - 1) + \ln (x+1)$
   est vraie}
  {{Pour tout $x$ de  $]- \infty~;~-1[ \cup]1~;~+ \infty[$},
  {Pour tout $x$ de $\mathbf{R} - \{-1~ ;~ 1\}$.},
  {Pour tout $x$ de $]1~ ;~+\infty[$}}
  \AQquestion{Pour tout réel $x$, le nombre \[\dfrac{\text{e}^x - 1}
  {\text{e}^x + 2}\hskip12pt \text{égal à :} \] }
  {{$-\dfrac{1}{2}$},
  {$\dfrac{\text{e}^{-x} - 1}{\text{e}^{-x} + 2}$},
  {$\dfrac{1 - \text{e}^{-x}}{1 + 2\text{e}^{-x}}$}}
  \end{alterqcm}
 \end{tkzexample}


\subsection{\tkzname{correction} et \tkzname{br} : rang de la bonne réponse} 
\Iopt{AQquestion}{br}  \Iopt{AQquestion}{correction} 
Tout d'abord, il est nécessaire de demander un corrigé. Pour cela, il suffit d'inclure l'option   \tkzname{correction} qui est un booléen, ainsi positionné sur \tkzname{true}. Ensuite dans chaque question, il est nécessaire de donner la liste des bonnes réponses. Par exemple, avec   \tkzname{br=1} ou bien encore \tkzname{br=\{1,3\}}. 

Voici le corrigé de l'exercice précédent :

\medskip
\begin{tkzexample}[vbox,small]
\begin{alterqcm}[VF,correction,lq=125mm]
 \AQquestion[br=1]{Pour tout $x \in ]-3~;~2],~f'(x) \geqslant 0$.}
 \AQquestion[br=2]{La fonction $F$ présente un maximum en $2$}
 \AQquestion[br=2]{$\displaystyle\int_{0}^2 f'(x)\:\text{d}x = - 2$}
\end{alterqcm}
\end{tkzexample}


\endinput