\documentclass{article} % test of LaTeX package shadethm.sty
\usepackage{shadethm}
\newshadetheorem{theorem}{Theorem}[section]
\newshadetheorem{cor}[theorem]{Corollary}
\newtheorem{remark}[theorem]{Remark}
\newtheorem{unshadedtheorem}[theorem]{Theorem}
\newtheorem{comment}{Comment}[section]

\begin{document}
Document begins with some initial text.

\section{First Section}
Here is some initial text before the shaded box.

\begin{theorem}
This is the text to be shaded.
\end{theorem}

This is some text after the first shade.
It is separated from the environments by a blank line; that is 
all handled as usual in \LaTeX.
\begin{cor}
This is more text to be shaded.
It is numbered with the same counter as the environment above.

Multiple paragraphs will be handled with the usual paragraph indentation,
unless of course the user asks for a different paragraph indentation inside
the environment.
\end{cor}
This is not separated from the prior environment by a blank line and
so I don't expect a new paragraph.
This shows that the default is for the shaded areas have the usual text size 
so that shading sticks out into the margin.
This is the shadethm option \emph{bodymargin}.
The behavior of having the shading end at the margin
(so the words do not take up the full text width)
comes from the option \emph{shademargin}, invoked with 
\verb!\usepackage[shademargin]{shadethm}!. 


\begin{unshadedtheorem}
More text.
This time not set in shade, but it is still numbered in the same sequence.
\end{unshadedtheorem}

A bit more.

\begin{comment}
More text.
This time neither set in shade, nor numbered in the same sequence.
\end{comment}


\section{A New Section}

Section preamble.

\begin{theorem}
This theorem is shaded and the number has been reset by the section.
\end{theorem}

\begin{remark}
  And a closing remark
\end{remark}
Finishing text.

\clearpage
\section{A Test of Colors}
This section shows how you can fool around to see some of you own colors.
% Set up some lengths so that you can see the border.
\setlength{\shadeboxsep}{2pt}  
\setlength{\shadeboxrule}{.4pt}  
\setlength{\shadedtextwidth}{\textwidth}
\addtolength{\shadedtextwidth}{-2\shadeboxsep}
\addtolength{\shadedtextwidth}{-2\shadeboxrule}
\setlength{\shadeleftshift}{0pt} 
\setlength{\shaderightshift}{0pt}
% Macro to test colors
\newcommand{\thmwithcmykcolor}[2]{%
\par Material before the theorem.
\definecolor{shadethmcolor}{cmyk}{#1}%  
\definecolor{shaderulecolor}{cmyk}{#2}%
\begin{theorem}
This is a theorem set in the cmyk color \textbf{#1}, surrounded with a 
borderline set in the cmyk color \textbf{#2}.
Note, for instance, that shades darken on overheads, so test your colors
where you will use them. 
\end{theorem}}
% Some colors
\thmwithcmykcolor{.10,.10,0,0}{.75,.75,0,.5}
\thmwithcmykcolor{.04,.04,0,.10}{.99,0,0.52,0.70}
\thmwithcmykcolor{0,.13,.11,0}{0,0.88,0.85,0.35}
The prior theorem illustrates what happens if you use the option 
\emph{colored}, as supplied in the file \verb!colored.sth!,
except that the supplied file has the color stick out into the margin.

One more supplied option is \emph{shadein} where each shaded theorem
is indented, like a \LaTeX\ quotation.
% ---> Uncomment if you want to test dvips's named colors
% \newcommand{\thmwithnamedcolor}[1]{%
% \par Material before the theorem.
% \definecolor{shadethmcolor}{named}{#1}%  
% \definecolor{shaderulecolor}{named}{Black}%
% \begin{theorem}
% This is a theorem set in the named color \textbf{#1}, surrounded with a blac
% borderline.
% Note, for instance, that colors darken on overheads, so test your colors
% where you will use them. 
% \end{theorem}}
%
% These are all the DVIPS named colors from color.pro (as of 1999-Aug-11).
% \thmwithnamedcolor{GreenYellow}
% \thmwithnamedcolor{Yellow}
% \thmwithnamedcolor{Goldenrod}
% \thmwithnamedcolor{Dandelion}
% \thmwithnamedcolor{Apricot}
% \thmwithnamedcolor{Peach}
% \thmwithnamedcolor{Melon}
% \thmwithnamedcolor{YellowOrange}
% \thmwithnamedcolor{Orange}
% \thmwithnamedcolor{BurntOrange}
% \thmwithnamedcolor{Bittersweet}
% \thmwithnamedcolor{RedOrange}
% \thmwithnamedcolor{Mahogany}
% \thmwithnamedcolor{Maroon}
% \thmwithnamedcolor{BrickRed}
% \thmwithnamedcolor{Red}
% \thmwithnamedcolor{OrangeRed}
% \thmwithnamedcolor{RubineRed}
% \thmwithnamedcolor{WildStrawberry}
% \thmwithnamedcolor{Salmon}
% \thmwithnamedcolor{CarnationPink}
% \thmwithnamedcolor{Magenta}
% \thmwithnamedcolor{RedViolet}
% \thmwithnamedcolor{Fuchsia}
% \thmwithnamedcolor{Lavender}
% \thmwithnamedcolor{Thistle}
% \thmwithnamedcolor{Orchid}
% \thmwithnamedcolor{DarkOrchid}
% \thmwithnamedcolor{Purple}
% \thmwithnamedcolor{Plum}
% \thmwithnamedcolor{Violet}
% \thmwithnamedcolor{RoyalPurple}
% \thmwithnamedcolor{BlueViolet}
% \thmwithnamedcolor{Periwinkle}
% \thmwithnamedcolor{CadetBlue}
% \thmwithnamedcolor{CornflowerBlue}
% \thmwithnamedcolor{MidnightBlue}
% \thmwithnamedcolor{NavyBlue}
% \thmwithnamedcolor{RoyalBlue}
% \thmwithnamedcolor{Blue}
% \thmwithnamedcolor{Cerulean}
% \thmwithnamedcolor{Cyan}
% \thmwithnamedcolor{ProcessBlue}
% \thmwithnamedcolor{SkyBlue}
% \thmwithnamedcolor{Turquoise}
% \thmwithnamedcolor{TealBlue}
% \thmwithnamedcolor{Aquamarine}
% \thmwithnamedcolor{BlueGreen}
% \thmwithnamedcolor{Emerald}
% \thmwithnamedcolor{JungleGreen}
% \thmwithnamedcolor{SeaGreen}
% \thmwithnamedcolor{Green}
% \thmwithnamedcolor{ForestGreen}
% \thmwithnamedcolor{PineGreen}
% \thmwithnamedcolor{LimeGreen}
% \thmwithnamedcolor{YellowGreen}
% \thmwithnamedcolor{SpringGreen}
% \thmwithnamedcolor{OliveGreen}
% \thmwithnamedcolor{RawSienna}
% \thmwithnamedcolor{Sepia}
% \thmwithnamedcolor{Brown}
% \thmwithnamedcolor{Gray}

Enjoy!
\textit{--Jim Hefferon}
\end{document}
