\documentclass{article}
%\usepackage{caption}
%\usepackage{caption2}
%\usepackage{ccaption}
\usepackage{captcont}
%\usepackage{caption}
%\usepackage{caption2}
%\usepackage{ccaption}
\usepackage[TABTOPCAP]{subfigure}
%\usepackage{caption}
%\usepackage{caption2}
%\usepackage{ccaption}

\newcommand{\figbox}[1]{%
  \fbox{%
    \vbox to .6in{%
    \vfil
    \hbox to .8in{%
      \hfil
      #1%
      \hfil}%
    \vfil}}}
\newcommand{\bigfigbox}[1]{%
  \fbox{%
    \vbox to 6in{%
    \vfil
    \hbox to 4in{%
      \hfil
      #1%
      \hfil}%
    \vfil}}}
\makeatletter
\def\subfigtopskip{4pt}
\def\subfigbottomskip{4pt}
\def\subfigcapskip{2pt}
\subtabletopcaptrue

\begin{document}

\setcounter{lofdepth}{2}
\listoffigures
\clearpage
\setcounter{lotdepth}{2}
\listoftables
\clearpage

First we test the regular figures.  The first figure is
Figure~\ref{fig:First} on page \pageref{fig:First}. The second is
Figure~\ref{fig:SecondA} and appears on pages~\pageref{fig:SecondA},
\pageref{fig:SecondB}, \pageref{fig:SecondC}, and \pageref{fig:SecondD}.

\begin{figure}%
  \begin{center}%
    \bigfigbox{Regular figure 1}%
  \end{center}%
  \vspace{-10pt}%
  \caption{Regular figure 1}%
  \label{fig:First}%
\end{figure}

\begin{figure}%
  \begin{center}%
    \bigfigbox{Continued figure 2A}%
  \end{center}%
  \vspace{-10pt}%
  \captcont{Continued figure 2A}%
  \label{fig:SecondA}%
\end{figure}

\begin{figure}%
  \begin{center}%
    \bigfigbox{Continued figure 2B}%
  \end{center}%
  \vspace{-10pt}%
  \captcont*{Continued figure 2B}%
  \label{fig:SecondB}%
\end{figure}

\begin{figure}%
  \begin{center}%
    \bigfigbox{Continued figure 2C}%
  \end{center}%
  \vspace{-10pt}%
  \captcont*{Continued figure 2C}%
  \label{fig:SecondC}%
\end{figure}

\begin{figure}%
  \begin{center}%
    \bigfigbox{Continued figure 2D}
  \end{center}%
  \vspace{-10pt}%
  \caption*{Continued figure 2D}
  \label{fig:SecondD}%
\end{figure}


Then we test the regular tables.  The first table is
Table~\ref{tab:First} on page \pageref{tab:First}. The second is
Table~\ref{tab:SecondA} and appears on pages~\pageref{tab:SecondA},
\pageref{tab:SecondB}, \pageref{tab:SecondC}, and \pageref{tab:SecondD}.

\begin{table}%
  \caption{Regular table 1}%
  \label{tab:First}%
  \begin{center}%
    \bigfigbox{Regular table 1}%
  \end{center}%
\end{table}

\begin{table}%
  \caption{Continued table 2A}%
  \label{tab:SecondA}%
  \begin{center}%
    \bigfigbox{Continued table 2A}%
  \end{center}%
\end{table}

\begin{table}%
  \captcont*{Continued table 2B}%
  \label{tab:SecondB}%
  \begin{center}%
    \bigfigbox{Continued table 2B}%
  \end{center}%
\end{table}

\begin{table}%
  \captcont*{Continued table 2C}%
  \label{tab:SecondC}%
  \begin{center}%
    \bigfigbox{Continued table 2C}%
  \end{center}%
\end{table}

\begin{table}%
  \captcont*{Continued table 2D}
  \label{tab:SecondD}%
  \begin{center}%
    \bigfigbox{Continued table 2D}
  \end{center}%
\end{table}

Next we look at the subfigures in a simple figure.  They are
Figures~\ref{fig:sub1A}, \ref{fig:sub1B}, \ref{fig:sub1C}, and
\ref{fig:sub1D}, which are part of Figure~\ref{fig:sub1} on 
page~\pageref{fig:sub1}.

The continued subfigures are contained in three parts.  The first is
Figure~\ref{fig:sub21} on page~\pageref{fig:sub21} which contains the
Subfigures~\ref{fig:sub2A}, \ref{fig:sub2B}, \ref{fig:sub2C}, and
\ref{fig:sub2D}.
The second is
Figure~\ref{fig:sub22} on page~\pageref{fig:sub22} which contains the
Subfigures~\ref{fig:sub2E}, \ref{fig:sub2F}, \ref{fig:sub2G}, and
\ref{fig:sub2H}.
The third is
Figure~\ref{fig:sub23} on page~\pageref{fig:sub23} which contains the
Subfigures~\ref{fig:sub2I}, \ref{fig:sub2J}, \ref{fig:sub2K}, and
\ref{fig:sub2L}.

\begin{figure}%
  \begin{center}%
    \subfigure[\label{fig:sub1A}]{\figbox{Subfigure 1A}}%
    \hspace{10pt}%
    \subfigure[\label{fig:sub1B}]{\figbox{Subfigure 1B}}\\
    \subfigure[\label{fig:sub1C}]{\figbox{Subfigure 1C}}%
    \hspace{10pt}%
    \subfigure[\label{fig:sub1D}]{\figbox{Subfigure 1D}}%
  \end{center}%
  \vspace{-10pt}%
  \caption{This is a simple figure.}%
  \label{fig:sub1}%
\end{figure}

\begin{figure}%
  \begin{center}%
    \subfigure[\label{fig:sub2A}]{\figbox{Subfigure 2A}}%
    \hspace{10pt}%
    \subfigure[\label{fig:sub2B}]{\figbox{Subfigure 2B}}\\
    \subfigure[\label{fig:sub2C}]{\figbox{Subfigure 2C}}%
    \hspace{10pt}%
    \subfigure[\label{fig:sub2D}]{\figbox{Subfigure 2D}}%
  \end{center}%
  \vspace{-10pt}%
  \captcont{This is a continued figure.}%
  \label{fig:sub21}%
\end{figure}
  
\begin{figure}%
  \begin{center}%
    \subfigure[\label{fig:sub2E}]{\figbox{Subfigure 2E}}%
    \hspace{10pt}%
    \subfigure[\label{fig:sub2F}]{\figbox{Subfigure 2F}}\\
    \subfigure[\label{fig:sub2G}]{\figbox{Subfigure 2G}}%
    \hspace{10pt}%
    \subfigure[\label{fig:sub2H}]{\figbox{Subfigure 2H}}%
  \end{center}%
  \vspace{-10pt}%
  \captcont*{This is a continued figure (cont.)}%
  \label{fig:sub22}%
\end{figure}

\begin{figure}%
  \begin{center}%
    \subfigure[\label{fig:sub2I}]{\figbox{Subfigure 2I}}%
    \hspace{10pt}%
    \subfigure[\label{fig:sub2J}]{\figbox{Subfigure 2J}}\\
    \subfigure[\label{fig:sub2K}]{\figbox{Subfigure 2K}}%
    \hspace{10pt}%
    \subfigure[\label{fig:sub2L}]{\figbox{Subfigure 2L}}%
  \end{center}%
  \vspace{-10pt}%
  \caption*{This is a continued figure (cont.)}%
  \label{fig:sub23}%
\end{figure}

Next we look at the subtables in a simple table environment.  They are
Tables~\ref{tab:sub1A}, \ref{tab:sub1B}, \ref{tab:sub1C}, and
\ref{tab:sub1D}, which are part of Table~\ref{tab:sub1} on 
page~\pageref{tab:sub1}.

The continued subtables are contained in three parts.  The first is
Table~\ref{tab:sub21} on page~\pageref{tab:sub21} which contains the
Subtables~\ref{tab:sub2A}, \ref{tab:sub2B}, \ref{tab:sub2C}, and
\ref{tab:sub2D}.
The second is
Table~\ref{tab:sub22} on page~\pageref{tab:sub22} which contains the
Subtables~\ref{tab:sub2E}, \ref{tab:sub2F}, \ref{tab:sub2G}, and
\ref{tab:sub2H}.
The third is
Table~\ref{tab:sub23} on page~\pageref{tab:sub23} which contains the
Subtables~\ref{tab:sub2I}, \ref{tab:sub2J}, \ref{tab:sub2K}, and
\ref{tab:sub2L}.

\begin{table}%
  \caption{This is a simple table.}%
  \label{tab:sub1}%
  \begin{center}%
    \subtable[\label{tab:sub1A}]{\figbox{Subtable 1A}}%
    \hspace{10pt}%
    \subtable[\label{tab:sub1B}]{\figbox{Subtable 1B}}\\
    \subtable[\label{tab:sub1C}]{\figbox{Subtable 1C}}%
    \hspace{10pt}%
    \subtable[\label{tab:sub1D}]{\figbox{Subtable 1D}}%
  \end{center}%
\end{table}

\begin{table}%
  \caption{This is a continued table.}%
  \label{tab:sub21}%
  \begin{center}%
    \subtable[\label{tab:sub2A}]{\figbox{Subtable 2A}}%
    \hspace{10pt}%
    \subtable[\label{tab:sub2B}]{\figbox{Subtable 2B}}\\
    \subtable[\label{tab:sub2C}]{\figbox{Subtable 2C}}%
    \hspace{10pt}%
    \subtable[\label{tab:sub2D}]{\figbox{Subtable 2D}}%
  \end{center}
\end{table}

\begin{table}%
  \captcont*{This is a continued table (cont.)}%
  \label{tab:sub22}%
  \begin{center}%
    \subtable[\label{tab:sub2E}]{\figbox{Subtable 2E}}%
    \hspace{10pt}%
    \subtable[\label{tab:sub2F}]{\figbox{Subtable 2F}}\\
    \subtable[\label{tab:sub2G}]{\figbox{Subtable 2G}}%
    \hspace{10pt}%
    \subtable[\label{tab:sub2H}]{\figbox{Subtable 2H}}%
  \end{center}%
\end{table}

\begin{table}%
  \captcont*{This is a continued table (cont.)}%
  \label{tab:sub23}%
  \begin{center}%
    \subtable[\label{tab:sub2I}]{\figbox{Subtable 2I}}%
    \hspace{10pt}%
    \subtable[\label{tab:sub2J}]{\figbox{Subtable 2J}}\\
    \subtable[\label{tab:sub2K}]{\figbox{Subtable 2K}}%
    \hspace{10pt}%
    \subtable[\label{tab:sub2L}]{\figbox{Subtable 2L}}%
    \listsubcaptions
  \end{center}%
\end{table}

\end{document}
