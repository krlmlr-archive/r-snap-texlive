\def\minfileversion{V1.8d}     %^^Aof minutes.sty
\def\minfiledate{2009/12/04}   %^^Aof minutes.sty
%%
%% \CharacterTable
%%  {Upper-case    \A\B\C\D\E\F\G\H\I\J\K\L\M\N\O\P\Q\R\S\T\U\V\W\X\Y\Z
%%   Lower-case    \a\b\c\d\e\f\g\h\i\j\k\l\m\n\o\p\q\r\s\t\u\v\w\x\y\z
%%   Digits        \0\1\2\3\4\5\6\7\8\9
%%   Exclamation   \!     Double quote  \"     Hash (number) \#
%%   Dollar        \$     Percent       \%     Ampersand     \&
%%   Acute accent  \'     Left paren    \(     Right paren   \)
%%   Asterisk      \*     Plus          \+     Comma         \,
%%   Minus         \-     Point         \.     Solidus       \/
%%   Colon         \:     Semicolon     \;     Less than     \<
%%   Equals        \=     Greater than  \>     Question mark \?
%%   Commercial at \@     Left bracket  \[     Backslash     \\
%%   Right bracket \]     Circumflex    \^     Underscore    \_
%%   Grave accent  \`     Left brace    \{     Vertical bar  \|
%%   Right brace   \}     Tilde         \~}
%%
\begin{Protokoll}{Titel des deutschen Protokolls}
\untertitel{Untertitel}
\moderation{Moderator, Sitzungsleiter}
\protokollant{Protokollant}
\teilnehmer{Teilnehmer}
\gaeste{G\"aste}
\sitzungsdatum{25.\ Dezember 1999}
\sitzungsbeginn{10:00}
\sitzungsende{20:00}
\sitzungsort{Esslingen}
\verteiler{Vereinsmitglieder}
\fehlend[entschuldigt]{abwesend}
%%\fehlendEntschuldigt{}
%%\fehlendUnentschuldigt{}
\protokollKopf

\topic{Top eins}%<-- hier Tagesordnungspunkt einfuegen
\subtopic{Unterpunkt zu Top eins}%<-- Unterpunkt
Text zum Tagesordnungspunkt.
%%
\topic{Termine und Aufgaben}
\subtopic{Termine}
\termin{2000/12/24}[10:00]{Heiliger Vormittag}
\termin{2000/12/24}{Heiligabend}
\termin{2000/12/24}[20:00]{Bescherung}[Ob meine W\"unsche erf\"ullt werden?]
\termin*{2000/12/25}{Weihnachten (ohne Kalendereintrag)}

\topic{Aufgaben}
\aufgabe{Wer}{Was}
\aufgabe*{Jemand macht was}
\aufgabe*[Heute]{Jemand soll heute was machen}
\aufgabe[Erledigt]{Zust{\"a}ndig}[Gestern]{Jemand macht was}

\newcols[][1]%notwendig wegen einem Fehler
\begin{Geheim}
\topic{Geheimer Punkt}
Dieser Text ist geheim und kann mit der Option \texttt{Secret}
ausgegeben werden.
\end{Geheim}

\subtopic{Nach Geheimen Text}
Wenn dieser Abschnitt nicht im Hauptabschnitt "`Geheim"' liegt,
dann wurde die Ausgabe des geheimen Textes unterdr{\"u}ckt.
\geheim{Jetzt noch ein kleines Geheimnis}
\newcols

\zusatztopic{Au{\ss}erordentlicher Tagesordnungspunkt}
\subtopic{Unterpunkt}

\topic{Anh{\"a}nge}
\anhang{Anhang mit zwei Seiten}{2}

\anhang[att:de]{Anhang mit Referenz}{2}

\topic{Abstimmungen und Entscheidungen}
\subtopic{Meinungen und Argumentationen}
\meinung{Herder}{Abweichende Meinung zum Protokoll}

Eine Diskussionsfolge:
\begin{Meinungen}
\item[Goethe] Eine Meinung
\item[Schiller] Eine andere Meinung
\end{Meinungen}

Argumente k"onnen mit Pro und Contra gef�hrt werden:
\begin{Argumentation}
\pro Grund daf"ur
\Pro wichtiger Grund daf"ur
\contra Grund dagegen
\Contra wichtiger Grund dagegen
\item Kommentar dazu
\ergebnis Ergebnis
\end{Argumentation}

\subtopic{Einzelne Abstimmungen}
\abstimmung{Kurze Abstimmung}{1}{2}{3}

Und noch eine Abstimmung mit Ergebnis:\par
\abstimmung{Kurze Abstimmung}{1}{2}{3}[Ergebnis]

\subtopic{Mehrere Abstimmungen in Folge}
\begin{Abstimmung}
\abstimmung{Abstimmung eins}{1}{2}{3}
\abstimmung{Abstimmung zwei}{1}{2}{3}[Entscheidung]
\abstimmung{Abstimmung drei}{1}{2}{3}
\end{Abstimmung}

\subtopic{Beschl{\"u}sse}
\beschlussthema{Thema}{Titel des Themas}
\beschluss{Thema}{Entscheidung gefallen}
\beschluss*{Entscheidung ohne Thema}[Langtext zur Entscheidung]
\end{Protokoll}
\endinput
%%
%% End of file `SampleDE.tex'.
