%%
%% This is file `mtc-bk.tex',
%% generated with the docstrip utility.
%%
%% The original source files were:
%%
%% minitoc.dtx  (with options: `mtc-bk')
%% This is a generated file.
%% 
%% 
%% This file may be distributed and/or modified under the conditions of
%% the LaTeX Project Public License, either version 1.3 of this license
%% or (at your option) any later version.  The latest version of this
%% license is in:
%% 
%%    http://www.latex-project.org/lppl.txt
%% 
%% and version 1.3 or later is part of all distributions of LaTeX version
%% 2003/12/01 or later.
%% 
%% This work has the LPPL maintenance status "author-maintained".
%% 
%% This Current Maintainer of this work is Jean-Pierre F. Drucbert (JPFD).
%% 
%% This work consists of all the files listed in the file `minitoc.l'
%% distributed with it.
%% Copyright 1993 1994 1995 1996 1997 1998 1999 2000
%%           2001 2002 2003 2004 2005 2006 2007 2008
%% Jean-Pierre F. Drucbert
%% <jean-pierre.drucbert@onera.fr>
%%%%%%%%%%%%%%%%%%%%% A example file (differs from previous versions)
%% mtc-bk.tex
%% This file contains a set of tests for the minitoc.sty version #60 package file.
%% You can alter most of parameters to test.
%% Class: book/report (\chapter must be defined).
%% You can use a copy of this file to play with minitoc commands and parameters.
\documentclass[12pt,a4paper]{report} % the report class uses less pages
%% \documentclass[12pt,a4paper]{book}
\ProvidesFile{mtc-bk.tex}%
  [2007/06/06]
\usepackage{lipsum} % provides filling text
%% \usepackage{hyperref} % if used, load it BEFORE minitoc
%% \usepackage{mtcoff}
\usepackage[tight]{minitoc} % tight option make shorter mini-tables
\setcounter{secnumdepth}{5} % depth of numbering of sectionning commands
\setcounter{tocdepth}{3}               % depth of table of contents
\setlength{\mtcindent}{24pt}           % indentation of minitocs, default
\renewcommand{\mtcfont}{\small\rm}     % font for minitocs, default
\renewcommand{\mtcSfont}{\small\bf}    % font for minitocs, sections, default
%% \renewcommand{\mtcSSfont}{\small\sf} % font for minitocs, subsections
%% you can make experiments with \mtcSSSfont, \mtcPfont and \mtcSPfont
%% but it is ``fontomania''...
\raggedbottom                   % or \flushbottom, at your choice
%%% TEST: uncomment the next line to test resetting chapter number in each part
%% \makeatletter \@addtoreset{chapter}{part} \makeatother
%%% END TEST
\begin{document}
\mtcpagenumbers
\noptcrule
%% \nomtcrule                    % suppresses minitoc rules
%% \nomtcpagenumbers             % suppresses minitoc page numbers
%% \nomlfpagenumbers             % ---------- minilof ---- -------
%% \nomltpagenumbers             % ---------- minilot ---- -------
\dominitoc
\dominilof[c]                   % centers title of minilof's
\dominilot
\doparttoc                      % test of parttoc/partlof stuff
\dopartlof                      % added in version #15
\dopartlot                      % added in version #15
\tableofcontents            % or \faketableofcontents
\listoffigures              % or \fakelistoffigures
\fakelistoftables           % or \listoftables
%% \addtocounter{chapter}{-1} % to begin with Chapter 0
\part{First Part}
\parttoc \partlof[r] \partlot[r]
\twocolumn\sloppy               % the minitoc in twocolumn layout is ugly,
\chapter{AAAAA}                 % a chapter with a lot of sections
\minitoc[r]                     % minitoc title on the right
\lipsum[1]
\section{S1} \lipsum[2]
\section{S2} \lipsum[3]
\section{S3} \lipsum[4]
\section*{S4}
\addcontentsline{toc}{section}{\protect\numberline{}{S4}}
\lipsum[5]
\section{S5} \lipsum[6]
\section{S6} \lipsum[6]
\section{S7} \lipsum[7]
\section{S8} \lipsum[9]
\section{S9} \lipsum[10]
\section{S10} \lipsum[11]
\section{S11} \lipsum[12]
\section{S12} \lipsum[13]
\section{S13} \lipsum[14]
\section{S14} \lipsum[15]
\section{S15} \lipsum[16]
\section{S16} \lipsum[17]
\section{S17} \lipsum[18]
\section{S18} \lipsum[19]
\section{S19} \lipsum[20]
\section{S20} \lipsum[21]
\section{S21} \lipsum[22]
\section{S22} \lipsum[23]
\section{S23} \lipsum[24]
\section{S24} \lipsum[25]
\section{S25} \lipsum[26]
\section{S26} \lipsum[27]
\section{S27} \lipsum[28]
\section{S28} \lipsum[29]
\section{S29} \lipsum[30]
\section{S30} \lipsum[31]
\subsection{SS1} \lipsum[32]
\section{S31} \lipsum[33]
\onecolumn\fussy        % back to one column
\chapter{BBBBB}
\minitoc
\mtcskip                % put some skip here
\minilof                % a minilof
\mtcskip                % put some skip here
\minilot                % a minilot
\lipsum[34]
\section{T1} \lipsum[35]
\begin{figure}[t]       % tests compatibility with floating bodies
\setlength{\unitlength}{1mm}
\begin{picture}(100,50)
\end{picture}
\caption{F1}            % (tables are similar)
\end{figure}
\begin{table}[b]        % tests compatibility with floating bodies
\setlength{\unitlength}{1mm}
\begin{picture}(100,50)
\end{picture}
\caption{T1}            % (tables are similar)
\end{table}
\clearpage
\subsection[tt1]{TT1}   % tests optional arg. of a sectionning command
\lipsum[36]
\subsubsection{TTT1} \lipsum[37]
\paragraph{TTTT1} \lipsum[38]
\begin{figure}
\setlength{\unitlength}{1mm}
\begin{picture}(100,50)
\end{picture}
\caption[f2]{F2}        % tests optional arg. of a caption
\end{figure}
\section{T2} \lipsum[39]
\chapter*{CCCCC}        % tests a pseudo-chapter; could have a minitoc.
\addstarredchapter{CCCCC}
\lipsum[40]
\section*{U1}
\addcontentsline{toc}{section}{U1}
\lipsum[41]
\subsection*{UU1}
\addcontentsline{toc}{subsection}{UU1}
\lipsum[42]
\subsubsection*{UUU1}
\addcontentsline{toc}{subsubsection}{UUU1}
\lipsum[43]
\paragraph*{UUUU1}
\addcontentsline{toc}{paragraph}{UUUU1}
\lipsum[44]
\section*{U2}
\addcontentsline{toc}{section}{U2}
\lipsum[45]
\part{Second Part}
\parttoc
\partlof[c]
%%                       % the following chapter should have no minitoc,
\chapter{DDDDD}          % but if you uncomment \minitoc,
%% \minitoc              % the minitoc appears
\lipsum[46]
\section{V1} \lipsum[47]
\subsection{VV1} \lipsum[48]
\subsubsection{VVV1} \lipsum[49]
\paragraph{VVVV1} \lipsum[50]
\begin{figure}[t]        % tests compatibility with floating bodies
\setlength{\unitlength}{1mm}
\begin{picture}(100,50)
\end{picture}
\caption{F3}             % (I have not tested tables, but it is similar)
\end{figure}
\lipsum[51]
\section{V2} \lipsum[52]
\chapter{EEEEE}                 % this chapter should have a minitoc
{%                              % left brace, see below
\setcounter{minitocdepth}{3}    % depth of mini table of contents;
                                % try with different values.
\minitoc
}                               % right brace
%% this pair of braces is used to keep local the change
%% on minitocdepth.
\lipsum[53]
\section{W1}                    % with the given depth
\lipsum[54]
\subsection{WW1} \lipsum[55]
\subsubsection{WWW1} \lipsum[56]
\begin{figure}[t]        % tests compatibility with floating bodies
\setlength{\unitlength}{1mm}
\begin{picture}(100,50)
\end{picture}
\caption{F4}             % (I have not tested tables here, but it is similar)
\end{figure}
\lipsum[57]
\paragraph{WWWW1} \lipsum[58]
\subparagraph{WWWWW1} \lipsum[59]
\section{W2} \lipsum[60]
\appendix
\part{Appendices}
\parttoc
\begin{mtchideinmaintoc}[-1]
\chapter{Comments}
\minitoc
\section{C1} \lipsum[61]
\section{C2} \lipsum[62]
\section{C3} \lipsum[63]
\begin{figure}[t]        % tests compatibility with floating bodies
\setlength{\unitlength}{1mm}
\begin{picture}(100,50)
\end{picture}
\caption{F5}             % (I have not tested tables, but it is similar)
\end{figure}
\section{C4}
\chapter{Evolution}
\minitoc
\minilof %Empty => invisible
\minilot %Empty => invisible
\section{D1} \lipsum[64]
\section{D2} \lipsum[65]
\section{D3} \lipsum[66]
\section{D4} \lipsum[67]
%% this line closes the omitted part
\addtocontents{toc}{\protect\partbegin}
%% this line restore the depth in the main TOC
\end{mtchideinmaintoc}
\lipsum[68]
\end{document}
%%%%%%%%%%%%%%%%%%%%%%%%%%%%%%%%%%%%%%%%%%%%%%%%%%%%%%%%%%%%%%%%%%%%%%%%%%%%%%%%%%%%%%%%%%%
%%
\endinput
%%
%% End of file `mtc-bk.tex'.
