%%
%% This is file `ktv-data.tex',
%% generated with the docstrip utility.
%%
%% The original source files were:
%%
%% ktv-texdata.dtx  (with options: `datafile')
%% 
%% IMPORTANT NOTICE:
%% 
%% For the copyright see the source file.
%% 
%% Any modified versions of this file must be renamed
%% with new filenames distinct from ktv-data.tex.
%% 
%% For distribution of the original source see the terms
%% for copying and modification in the file ktv-texdata.dtx.
%% 
%% This generated file may be distributed as long as the
%% original source files, as listed above, are part of the
%% same distribution. (The sources need not necessarily be
%% in the same archive or directory.)
\bxx(yyy)1;
This is {\it yyyyy}.one.
\exx
\bxx(yyy)2;
This is {\it yyyyy}.two with hint.
\exx
\begin{hint}
This is HINT of {\it yyyyy}.2
\end{hint}
\bxx(yyy)3;
This is {\it yyyyy}.three.
\exx
\bxx(yyy)4;
This is {\it yyyyy}.four.
\exx
\bxx(yyy)[TT]5;
This is {\it yyyyy}.six with header TT
\exx

^^M % ------------------------------------------------------------------

\bxx1;
This is one.
\\In the data file, this `bxx' doesnot specify the enrvironment.
\\So this used default environment.
\exx
\bxx2;
This is two with hint.
\\In the data file, this `bxx' doesnot specify the enrvironment.
\\So this used default environment.
\exx
\begin{hint}
This is hint of two.
\\In the data file, this `bxx' doesnot specify the enrvironment.
\\So this used default environment.
\end{hint}
\bxx3;
This is three.
\\In the data file, this `bxx' doesnot specify the enrvironment.
\\So this used default environment.
\exx
\bxx4;
This is four.
\\In the data file, this `bxx' doesnot specify the enrvironment.
\\So this used default environment.
\exx
\bxx5;
This is six.
\\In the data file, this `bxx' doesnot specify the enrvironment.
\\So this used default environment.
\exx

^^M % ------------------------------------------------------------------

\bxx(zzz)1;
This is ZZZZZZZ.one.
\exx
\bxx(zzz)2;
This is ZZZZZZZ.two.
\\Containning an equation:
\begin{equation}
\int_{0}^{\infty}f(x)\,dx = \pi
\end{equation}
\exx
\bxx(zzz)3;
This is ZZZZZZZ.three.
\exx
\bxx(zzz)4;
This is ZZZZZZZ.four.
\exx
\bxx(zzz)[TT]5;
This is ZZZZZZZ.six with header TT.
\exx
\endinput
%%
%% End of file `ktv-data.tex'.
