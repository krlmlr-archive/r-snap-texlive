\documentclass[11pt,a4paper]{article}
\usepackage[a4paper,margin=3cm]{geometry}
\setlength{\parskip}{2pt}
\makeatletter

\newcommand\cs[1]{\texttt{\char`\\#1}}
\renewcommand\arg[1]{\nolinebreak[2]\mbox{\texttt{\string{#1\string}}}}
\newcommand\oarg[1]{\nolinebreak[2]\mbox{\texttt{[#1]}}}
\newcommand\sarg{\texttt{*}}

\newcommand\meta[1]{\mbox{\ensuremath{\langle}\textit{#1}\ensuremath{\rangle}}}
\newcommand\prototype[1]{\par\pagebreak[1]\bigskip\begingroup
   \noindent\samepage #1\par \nopagebreak \smallskip \endgroup \@afterheading}

% meaningful names, as in faq.sty
\let\environment\texttt
\let\Package\textsf
\makeatother

\begin{document}
\title{The \Package{tabto} package}
\author{Donald Arseneau\thanks{Documentation file assembled by Robin
    Fairbairns}}
\date{Mar 2013, version 1.3}
\maketitle

\section*{Tabbing to fixed positions in a paragraph}

Two new text positioning commands are defined: \cs{tabto} and \cs{tab}.

\prototype{\cs{tabto}\arg{\meta{length}}}

Tab to a position relative to the left margin in a paragraph
(any indentation due to a list is part of the `margin' in this
context).  If the text on the line already goes past the desired
position, the tab starts a new line and moves to the requested
horizontal position.


\prototype{\cs{tabto}\sarg\arg{\meta{length}}}

Similar to \cs{tabto}, except it will perform backspacing, and 
over-print previous text on the line whenever that text is already
longer than the specified length (i.e., no linebreak is produced).
Line-breaks are suppressed immediately after \cs{tabto} or \cs{tabto*}.

The length register \cs{CurrentLineWidth} will report the width
of the existing text on the line, and it may be used in the
\meta{length} argument (using calc.sty, for example). Also, there
is \cs{TabPrevPos} which stores the \cs{CurrentLineWidth} from the
previous tab command, and can be used to return to that position
if no line breaks have occurred in between.

\prototype{\cs{tab}}

Tab to the next tab-stop chosen from a list of tab positions, in
the traditional style of typewriters.  A \cs{tab} will always move
to the next tab stop (or the next line), even if it is already
exactly at a tab stop. Thus, ``\cs{tab}\cs{tab}'' skips a position.  A
linebreak is permitted immediately following a \cs{tab}, in case the
ensuing text does not fit well in the remaining space.

The tab-stop positions are declared using either \cs{TabPositions}
or \cs{NumTabs}:

\prototype{\cs{TabPositions}\arg{\meta{length}, \meta{length},
             \textrm{\dots}\meta{length}}}

Declares the tab stops as a comma-separated list of positions 
relative to the left margin. A tab-stop at \texttt{0pt} is implicit, and 
need not be listed.

\prototype{\cs{NumTabs}\arg{\meta{number}}}

Declares a list of \meta{number} equally-spaced tabs, starting at the
left margin and spanning \cs{linewidth}.  For example
\cs{NumTabs}\arg{2} declares tab-stops at \texttt{0pt} and
\texttt{0.5}\cs{linewidth}, the same as
\cs{TabPositions}\arg{0pt, 0.5\cs{linewidth}}
or \cs{TabPositions}\arg{0.5\cs{linewidth}}.


\end{document}
