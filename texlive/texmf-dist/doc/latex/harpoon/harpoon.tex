\documentclass[a4paper]{article}

\usepackage[dvips]{harpoon}

\title{A First Look at {\tt harpoon.sty}}
\author{Tobias Kuipers\\ {\tt kuipers@fwi.uva.nl}}
\date{November 2, 1994}

\begin{document}

\maketitle

\section{Introduction}

This packages provides a number of harpoons to be set over or under
arbitrary text. The commands are invoked as \verb+\someharp{text}+,
which will put some harpoon over {\tt text}. {\tt text} is typeset in LR
mode. If you want text to be typeset in math mode you should type
\verb+\someharp{$text$}+. 

\section{The Commands}

The package is invoked with 1 option: \verb+\usepackage[xxx]{harpoon}+,
where {\tt xxx} is your favourite graphics device driver. On most Unix
machines, this will be {\tt dvips} and on a Macintosh this could be {\tt
oztex}. Refer to the graphics package for more information.

The commands are
\begin{itemize}
	\item \verb+\overleftharp+, which looks like \overleftharp{this}
	\item \verb+\overrightharp+, which looks like \overrightharp{this}
	\item \verb+\overleftharpdown+, which looks like \overleftharpdown{this}
	\item \verb+\overrightharpdown+, which looks like \overrightharpdown{this}
	\item \verb+\underleftharp+, which looks like \underleftharp{this}
	\item \verb+\underrightharp+, which looks like \underrightharp{this}
	\item \verb+\underleftharpdown+, which looks like \underleftharpdown{this}
	\item \verb+\underrightharpdown+, which looks like \underrightharpdown{this}
\end{itemize}

\end{document}
