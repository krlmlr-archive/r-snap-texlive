%%
%% This is file `engpron-ex-en.tex',
%% generated with the docstrip utility.
%%
%% The original source files were:
%%
%% engpron.dtx  (with options: `exemple,en')
%% This is a generated file.
%% 
%% Copyright (C) 2003-2008 Y. Henel, le TeXnicien de surface
%% <le.texnicien.de.surface@wanadoo.fr>
%% 
%% This file may be distributed and/or modified under the conditions of
%% the LaTeX Project Public License, either version 1.2 of this license
%% or (at your option) any later version.  The latest version of this
%% license is in:
%% 
%%    http://www.latex-project.org/lppl.txt
%% 
%% and version 1.2 or later is part of all distributions of LaTeX version
%% 1999/12/01 or later.
\documentclass[a4paper]{article}
\usepackage{array,xspace}
\usepackage[latin1]{inputenc}
\usepackage[tame,hyphenable]{engpron}[2008/08/15]
\usepackage[T1]{fontenc}
\usepackage{lmodern}
\usepackage[frenchb, english]{babel}
\setlength{\parindent}{0pt}
\newcommand\BOP{\discretionary{}{}{}}
\newcommand{\TO}{\textemdash\ \ignorespaces}
\newcommand{\TF}{\unskip\ \textemdash\xspace}
\newcommand\Pkg[1]{\textsf{#1}}
\newcommand\Option[1]{\textsc{#1}}
\begin{document}
\begin{center}\LARGE
 Usage of the \Pkg{engpron} package
\end{center}
Here are some examples of the usage of the \texttt{^^a3}-macros. The
\verb+engpron+ package is loaded whith the options \Option{jones},
\Option{monstress}, \Option{unhyphenable}, \Option{visible},
\Option{nice}, and \Option{final} \TO default options\TF and the
explicit option \Option{tame} which replaces the default option
\Option{wild}.
\paragraph{The \texttt{^^a3X} macro} This macro \emph{must} be
followed by a single letter. It notes the optional sounds the way
Jones~\cite{jones15} does if option \Option{jones} has been chosen,
or,  the Harraps~\cite{harraps} does if option \Option{harraps} has
been chosen.

One would type for example
\verb!\PRON{!\texttt{^^a3e^^a3h^^a3Xt^^a3s^^a3U^^a3kli^^a3k^^a3en}\verb!}!
to obtain \PRON{^^a3e^^a3h^^a3Xt^^a3s^^a3U^^a3kli^^a3k^^a3en}.

Followed by the letter `e' it gives \Pron{^^a3Xe}
(\verb+\Pron{^^a3Xe}+) as in \PRON{^^a3B^^a3af^^a3kek^^a3Ht^^a3q^^a3k^^a3s^^a3Xen} one can
obtain with \verb+\PRON{^^a3B^^a3af^^a3kek^^a3Ht^^a3q^^a3k^^a3s^^a3Xen}+.
\paragraph{\texttt{^^a3M} and \texttt{^^a3m} macros} They give the French
nasalised vowels, French way for the 1rst one and English way for the
2nd. See the 2nd table. They must be followed by one of the following
letters: a, i, o, u. One may think to the French
`restaur\textbf{ant}', `v\textbf{in}', `b\textbf{on} march^^e9',
`Verd\textbf{un}'.
\paragraph{Macro \texttt{^^a3Z}} It notes the final `r' which is only
pronunced when followed by a vowel. One has e.g. \Pron{^^a3H^^a3akt^^a3e^^a3Z} with
\verb+\Pron{^^a3H^^a3akt^^a3e^^a3Z}+.
\paragraph{Macro \texttt{^^a3k}} Its beheviour is controlled by the
options \Option{hyphenable} \TO which is the contrary of
\Option{unhyphenable}\TF and \Option{visible} \TO contrary of
\Option{invisible}. It is used to mark the syllable limits.

With the default option \Option{visible} it produces a glyph which is
by default [.] and defined by the macro
\verb+\EPSyllabeMarque+.  This macro can be redefined with a
\verb+\renewcommand+ if necessary. With option \Option{invisible} \verb+^^a3k+
doesn't produce any glyph.

With option \Option{hyphenable} \verb+^^a3k+ allows a break but prevent it
with \Option{unhyphenable} which is the default.
\paragraph{Macro \texttt{^^a3K}} Its behaviour is determined by the
following letter. \texttt{X} gives an invisible breakable point,
\texttt{k} a visible unbreakable mark of syllable limit, \texttt{K} a
visible \emph{breakable} mark of syllable limit, \texttt{b} an
unbreakable mark of secondary stress, \texttt{B} a \emph{breakable}
mark of secondary stress, \texttt{h} an unbreakable mark of primary
stress, and, avant last, \texttt{H} a \emph{breakable} mark of primary
stress. Its behaviour \textbf{doesn't depend} on the enforced options.
\vspace*{\stretch{1}}

\begin{thebibliography}{99}
\bibitem{jones15} Daniel~\textsc{Jones} \dag.
\emph{English Pronouncing Dictionary}.
15th Edition. This major new edition edited by Peter Roach \& James
Hartman.
Cambridge University Press, 1997.
\textsc{isbn} : \texttt{0-521-45903-6}
\bibitem{harraps} J. E. \textsc{Manson}, M.A.
\emph{Harrap's New Shorter French And English Dictionary.}
Revised edition \copyright{} George G. Harrap \&
Co. Ltd. \oldstylenums{1967}.
Bordas diffuseur, \oldstylenums{1980}.
\textsc{isbn} : \texttt{0 245 59062 5}
\end{thebibliography}
\newpage

\vspace*{\stretch{1}}

\begin{LivreActive}\Large
\hspace*{\stretch{1}}
\begin{tabular}[c]{|>{\ttfamily }c|c||>{\ttfamily }c|c||>{\ttfamily
    }c|c||>{\ttfamily }c|c||>{\ttfamily }c|c|}\hline
\multicolumn{10}{|c|}{
The macros in alphabetic order
}\\
\multicolumn{10}{|c|}{
The star refers to the preceding explanations
}\\ \hline
\^^a3a & ^^a3a & \^^a3b & ^^a3b & \^^a3c & ^^a3c & \^^a3d & ^^a3d & \^^a3e  & ^^a3e \\ \hline
\^^a3f & ^^a3f & \^^a3g & ^^a3g & \^^a3h & ^^a3h & \^^a3i & ^^a3i & \^^a3j  & ^^a3j \\ \hline
\^^a3k & $\ast$ & \^^a3l & ^^a3l & \^^a3m & $\ast$ & \^^a3n & ^^a3n & \^^a3o  & ^^a3o \\ \hline
\^^a3p & ^^a3p & \^^a3q & ^^a3q & \^^a3r & ^^a3r & \^^a3s & ^^a3s & \^^a3t  & ^^a3t \\ \hline
\^^a3u & ^^a3u & \^^a3v & ^^a3v & \^^a3w & ^^a3w & \^^a3x & ^^a3x & \^^a3y  & ^^a3y \\ \hline
\^^a3z & ^^a3z &     &    &     &    &     &    &      &    \\ \hline
\^^a3A & ^^a3A & \^^a3B & ^^a3B & \^^a3C & ^^a3C & \^^a3D & ^^a3D & \^^a3E  & ^^a3E \\ \hline
\^^a3F & ^^a3F & \^^a3G & ^^a3G & \^^a3H & ^^a3H & \^^a3I & ^^a3I & \^^a3J  & ^^a3J \\ \hline
\^^a3K & $\ast$ & \^^a3L & ^^a3L & \^^a3M & $\ast$ & \^^a3N & ^^a3N & \^^a3O  & ^^a3O \\ \hline
\^^a3P & ^^a3P & \^^a3Q & ^^a3Q & \^^a3R & ^^a3R & \^^a3S & ^^a3S & \^^a3T  & ^^a3T \\ \hline
\^^a3U & ^^a3U & \^^a3V & ^^a3V & \^^a3W & ^^a3W & \^^a3X & $\ast$ & \^^a3Y  & ^^a3Y \\ \hline
\^^a3Z & ^^a3Z &     &    &     &    &     &    &      &    \\ \hline
\end{tabular}
\hspace*{\stretch{1}}
\end{LivreActive}

\vspace*{\stretch{1}}
\newpage
\thispagestyle{empty}

\vspace*{\stretch{1}}
\begin{LivreActive}
\hspace*{\stretch{1}}
{\Large
\begin{tabular}[c]{|>{\ttfamily }c|c||>{\ttfamily }c|c||>{\ttfamily
    }c|c||>{\ttfamily }c|c|}\hline
\multicolumn{8}{|c|}{
Vowels
}\\ \hline
\^^a3a & ^^a3a & \^^a3A & ^^a3A & \^^a3e & ^^a3e & \^^a3E & ^^a3E \\ \hline
\^^a3i & ^^a3i & \^^a3I & ^^a3I & \^^a3o & ^^a3o & \^^a3O & ^^a3O \\ \hline
\^^a3u & ^^a3u & \^^a3U & ^^a3U & \^^a3v & ^^a3v & \^^a3x & ^^a3x \\ \hline
\^^a3c & ^^a3c & \^^a3C & ^^a3C & \^^a3y & ^^a3y &     &    \\ \hline
\multicolumn{8}{|c|}{
Diphtongs
}\\ \hline
\^^a3p & ^^a3p & \^^a3q & ^^a3q & \^^a3r & ^^a3r & \^^a3P & ^^a3P\\ \hline
\^^a3Q & ^^a3Q & \^^a3R & ^^a3R & \^^a3w & ^^a3w & \^^a3W & ^^a3W\\ \hline
\^^a3V & ^^a3V &     &    &     &    &     &   \\ \hline
\multicolumn{8}{|c|}{
Consonants
}\\ \hline
\^^a3d & ^^a3d & \^^a3f & ^^a3f & \^^a3j & ^^a3j & \^^a3l & ^^a3l \\ \hline
\^^a3n & ^^a3n & \^^a3s & ^^a3s & \^^a3t & ^^a3t & \^^a3z & ^^a3z \\ \hline
\^^a3T & ^^a3T & \^^a3L & ^^a3L &     &    &     &    \\ \hline
\multicolumn{8}{|c|}{
Stress
}\\ \hline
\^^a3b & ^^a3b & \^^a3B & ^^a3B & \^^a3h & ^^a3h & \^^a3H & ^^a3H \\ \hline
\^^a3Kb & ^^a3Kb & \^^a3KB & ^^a3KB & \^^a3Kh & ^^a3Kh & \^^a3KH & ^^a3KH \\ \hline
\multicolumn{8}{|c|}{
Nasalised vowels
}\\
\multicolumn{8}{|c|}{
French pronunciation
}\\ \hline
\^^a3Ma & ^^a3Ma & \^^a3Mi & ^^a3Mi & \^^a3Mo & ^^a3Mo & \^^a3Mu & ^^a3Mu \\ \hline
\multicolumn{8}{|c|}{
English pronunciation
}\\ \hline
\^^a3ma & ^^a3ma & \^^a3mi & ^^a3mi & \^^a3mo & ^^a3mo & \^^a3mu & ^^a3mu \\ \hline
\multicolumn{8}{|c|}{
Syllables marking
}\\ \hline
\^^a3k & ^^a3k & \^^a3Kk & ^^a3Kk & \^^a3KK & ^^a3KK & \^^a3KX& ^^a3KX\\ \hline
\end{tabular}}
\hspace*{\stretch{1}}
\vspace*{\stretch{4}}

{\footnotesize
The preceding table is written in a \verb+LivreActive+-environment and
one must use the \verb+\+\texttt{\^^a3}-macro to obtain the character
\^^a3. But one can type `^^a3a' with a straightforward `\texttt{\^^a3a}'.

Inside a \verb+LivreActive+-environment, one obtains \pron{^^a3H^^a3akt^^a3e^^a3Z}
with \verb+\pron{+\texttt{\^^a3H}\BOP\texttt{\^^a3akt}\BOP\texttt{\^^a3e}\BOP%
\texttt{\^^a3Z}\verb+}+ but one will notice by reading the present
\texttt{engpron-ex-en.tex} file what is required to obtain a semblance of
\texttt{verbatim}.
}

\vspace{2\baselineskip}
\newlength{\montruc}\settowidth{\montruc}{Yvon \textsc{Henel}, \TeX
  nicien de surface.}
\hspace*{\stretch{1}}\makebox[\montruc][c]{Yvon \textsc{Henel}, \TeX
  nicien de surface.}\\[.5\baselineskip]
\hspace*{\stretch{1}}\makebox[\montruc][c]{2008-07-15}

\vspace*{\stretch{1}}
\end{LivreActive}
\end{document}
\endinput
%%
%% End of file `engpron-ex-en.tex'.
