\documentclass[pagesize=auto, fontsize=12pt, DIV=9, parskip=half]{scrartcl}

\usepackage{fixltx2e}
\usepackage{etex}
\usepackage{lmodern}
\usepackage[T1]{fontenc}
\usepackage{textcomp}
\usepackage{microtype}
\usepackage{hyperref}

\newcommand*{\mail}[1]{\href{mailto:#1}{\texttt{#1}}}
\newcommand*{\pkg}[1]{\textsf{#1}}
\newcommand*{\cs}[1]{\texttt{\textbackslash#1}}
\makeatletter
\newcommand*{\cmd}[1]{\cs{\expandafter\@gobble\string#1}}
\makeatother
\newcommand*{\opt}[1]{\texttt{#1}}

\addtokomafont{title}{\rmfamily}

\title{The \pkg{placeins} package\thanks{This manual corresponds to \pkg{placeins.sty}~ver~2.2, dated~April 18, 2005.}}
\author{Donald Arseneau\thanks{\mail{asnd@triumf.ca}}}
\date{April 18, 2005}


\begin{document}

\maketitle

\begin{sloppypar}
  \pkg{Placeins.sty} keeps floats `in their place', preventing them from floating
  past a ``\cmd{\FloatBarrier}'' command into another section.  To use it, declare
  ``\verb+\usepackage{placeins}+'' and insert ``\cmd{\FloatBarrier}'' at places that floats 
  should not move past, perhaps at every ``\cmd{\section}''.
\end{sloppypar}


\minisec{Option: \opt{[section]}}

\begin{sloppypar}
  A more convenient way to stop floats at section boundaries is to change 
  the definition of ``\cmd{\section}'' to include ``\cmd{\FloatBarrier}'', either at the
  beginning, before ``\cmd{\@startsection}'', or in the `style' specification (see 
  The \LaTeX\ Companion, section~2.2.2; or~2.3 in the 1st ed).  If you specify 
  ``\verb+\usepackage[section]{placeins}+'', then the ``\cmd{\section}'' command will be 
  redefined with ``\cmd{\FloatBarrier}'' inserted at the beginning.
\end{sloppypar}


\minisec{Options: \opt{[above]} \opt{[below]}}

Something you may not like is that, by default, ``\cmd{\FloatBarrier}'' is very 
strict, and will (try to) prevent a float from appearing above the start 
of the current section or below the start of the next section, even 
though the float is still on the same page as its intended section.  
Each restriction can be relaxed separately by using the ``\opt{[above]}'' and 
``\opt{[below]}'' package options: ``\opt{[above]}'' allows floats to appear above their 
section, if on the same page; ``\opt{[below]}'' allows below.

NOTE!  The original version of \pkg{placeins.sty} acted like it was loaded
with the option ``\opt{[above]}'' specified.

There is a problem with \LaTeX's ``\cmd{\suppressfloats}'' being out of step with 
the page breaking (see usenet msg \texttt{<yfi656pbsn0.fsf@triumf.ca>} and thread)
which sometimes allows a float to go above a ``\cmd{\FloatBarrier}'' placed near
the top of a page. Maybe \pkg{placeins} will fix it sometime later.


\minisec{Option: \opt{[verbose]}}

There is a package option ``\opt{[verbose]}'' that causes many messages to be
written in the log file.  It might be used to answer the question:
`How did \emph{that} get \emph{there}?!?'

\end{document}
