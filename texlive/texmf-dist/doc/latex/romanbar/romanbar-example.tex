%%
%% This is file `romanbar-example.tex',
%% generated with the docstrip utility.
%%
%% The original source files were:
%%
%% romanbar.dtx  (with options: `example')
%% 
%% This is a generated file.
%% 
%% Project: romanbar
%% Version: 2012/01/01 v1.0f
%% 
%% Copyright (C) 2011 by
%%     H.-Martin M"unch <Martin dot Muench at Uni-Bonn dot de>
%% 
%% The usual disclaimer applies:
%% If it doesn't work right that's your problem.
%% (Nevertheless, send an e-mail to the maintainer
%%  when you find an error in this package.)
%% 
%% This work may be distributed and/or modified under the
%% conditions of the LaTeX Project Public License, either
%% version 1.3c of this license or (at your option) any later
%% version. This version of this license is in
%%    http://www.latex-project.org/lppl/lppl-1-3c.txt
%% and the latest version of this license is in
%%    http://www.latex-project.org/lppl.txt
%% and version 1.3c or later is part of all distributions of
%% LaTeX version 2005/12/01 or later.
%% 
%% This work has the LPPL maintenance status "maintained".
%% 
%% The Current Maintainer of this work is H.-Martin Muench.
%% 
%% This work consists of the main source file romanbar.dtx,
%% the README, and the derived files
%%    romanbar.sty, romanbar.pdf,
%%    romanbar.ins, romanbar.drv,
%%    romanbar-example.tex, romanbar-example.pdf.
%% 
\documentclass[british]{article}[2007/10/19]% v1.4h
%%%%%%%%%%%%%%%%%%%%%%%%%%%%%%%%%%%%%%%%%%%%%%%%%%%%%%%%%%%%%%%%%%%%%
\usepackage[extension=pdf,%
 plainpages=false,%
 pdfpagelabels=true,%
 hyperindex=false,%
 pdflang={en},%
 pdftitle={romanbar package example},%
 pdfauthor={H.-Martin Muench},%
 pdfsubject={Example for the romanbar package},%
 pdfkeywords={LaTeX, romanbar, roman, Roman, bars, H.-Martin Muench},%
 pdfview={XYZ null null 1},%
 pdfstartview={XYZ null null 1},%
 pdfpagelayout=SinglePage]{hyperref}[2011/12/04]% v6.82m
\usepackage{romanbar}[2012/01/01]% v1.0f
\gdef\unit#1{\mathord{\thinspace\mathrm{#1}}}%
\listfiles
\begin{document}
\pagenumbering{arabic}
\section*{Example for romanbar}

This example demonstrates the use of package\newline
\textsf{romanbar}, v1.0f as of 2012/01/01 (HMM).\newline
There are no options to be used.\newline

\noindent For more details please see the documentation!\newline

\noindent Save per page about $200\unit{ml}$ water,
$2\unit{g}$ CO$_{2}$ and $2\unit{g}$ wood:\newline
Therefore please print only if this is really necessary.\newline

\noindent This package provides the command \verb|\Romanbar|
to print bars below and over the following:

\begin{description}
\item[-] Roman numbers: \verb|\Romanbar{MMXII}| prints \Romanbar{MMXII}

\item[-] Arabic numbers turned into upper-case Roman numbers:\newline
           \verb|\Romanbar{2012}| prints \Romanbar{2012}

\item[-] negative Arabic numbers turned into upper-case Roman numbers
           \newline
           (although historically there were no negative Roman numbers):
           \newline
           \verb|\Romanbar{-12}| prints \Romanbar{-12}

\item[-] zero Arabic number ($0$; although historically
           there was no Roman zero):\newline
           \verb|\Romanbar{0}| prints \Romanbar{0}

\item[-] some arbitrary text:
           \verb|\Romanbar{Caesar}| prints \Romanbar{Caesar}\newline
           (with descenders: \Romanbar{AgjpqyW})

\newcounter{example}
\setcounter{example}{21}

\item[-] some counter's value:
           \verb|\Romanbar{\theexample}| prints \Romanbar{\theexample}
           \newline
           (where the value of \texttt{example} is \theexample)

\item[-] Arabic numbers, without turning them into upper-case Roman
           numbers:\newline
           \verb|\Romanbar{\relax 2012}| prints \Romanbar{\relax  2012}
\end{description}

Special care was taken for "L" (50), e.\,g. in 555/DLV: \Romanbar{555}.\\

\verb|\romannum{...}| converts an Arabic number into a lower-case Roman one,
e.\,g. \verb|\romannum{2012}| prints \romannum{2012}, and
\verb|\Romannum{...}| converts an Arabic number into an upper-case Roman one,
e.\,g. \verb|\Romannum{2012}| prints \Romannum{2012}.

\end{document}
\endinput
%%
%% End of file `romanbar-example.tex'.
