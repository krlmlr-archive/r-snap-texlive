%%
%% This is file `dinbrief.tex',
%% generated with the docstrip utility.
%%
%% The original source files were:
%%
%% dinbrief.dtx  (with options: `documentation')
%% 
%% =======================================================================
%% 
%% Copyright (C) 1993, 96, 97 by University of Karlsruhe (Computing Center).
%% Copyright (C) 1998, 2000   by University of Karlsruhe (Computing Center)
%%                            and Richard Gussmann.
%% All rights reserved.
%% For additional copyright information see further down in this file.
%% 
%% This file is part of the DINBRIEF package
%% -----------------------------------------------------------------------
%% 
%% It may be distributed under the terms of the LaTeX Project Public
%% License (LPPL), as described in lppl.txt in the base LaTeX distribution.
%% Either version 1.1 or, at your option, any later version.
%% 
%% The latest version of this license is in
%% 
%%         http://www.latex-project.org/lppl.txt
%% 
%% LPPL Version 1.1 or later is part of all distributions of LaTeX
%% version 1999/06/01 or later.
%% 
%% 
%% For error reports in case of UNCHANGED versions see readme files.
%% 
%% Please do not request updates from us directly.  Distribution is
%% done through Mail-Servers, TeX organizations and others.
%% 
%% If you receive only some of these files from someone, complain!
%% 
%%
%% \CharacterTable
%%  {Upper-case    \A\B\C\D\E\F\G\H\I\J\K\L\M\N\O\P\Q\R\S\T\U\V\W\X\Y\Z
%%   Lower-case    \a\b\c\d\e\f\g\h\i\j\k\l\m\n\o\p\q\r\s\t\u\v\w\x\y\z
%%   Digits        \0\1\2\3\4\5\6\7\8\9
%%   Exclamation   \!     Double quote  \"     Hash (number) \#
%%   Dollar        \$     Percent       \%     Ampersand     \&
%%   Acute accent  \'     Left paren    \(     Right paren   \)
%%   Asterisk      \*     Plus          \+     Comma         \,
%%   Minus         \-     Point         \.     Solidus       \/
%%   Colon         \:     Semicolon     \;     Less than     \<
%%   Equals        \=     Greater than  \>     Question mark \?
%%   Commercial at \@     Left bracket  \[     Backslash     \\
%%   Right bracket \]     Circumflex    \^     Underscore    \_
%%   Grave accent  \`     Left brace    \{     Vertical bar  \|
%%   Right brace   \}     Tilde         \~}
%%
\expandafter\ifx\csname documentclass\endcsname\relax
    \documentstyle[german]{article}
    \typeout{Using the command \string\documentstyle.}
    \newcommand\LaTeXe{\LaTeX\kern.15em2$_\varepsilon$}
  \else
    \documentclass[10pt]{article}
    \usepackage{german}
    \typeout{Using the command \string\documentclass.}
  \fi

\newcommand\Dopt[1]{{\tt #1\/}}
\newcommand\file[1]{{\tt #1\/}}

\title{Standard Document Class `dinbrief'\\ for \LaTeX{} version 2e\\
       Standard Document Style `dinbrief'\\ for \LaTeX{} version 2.09}

\author{%
Copyright \copyright\ 1993,\ 96,\ 98\\ by Klaus Dieter Braune, Richard Gussmann
}
\newenvironment{decl}%
    {\par\small\addvspace{4.5ex plus 1ex}%
     \vskip -\parskip
     \noindent\hspace{-\leftmargini}%
     \begin{tabular}{|l|}\hline\ignorespaces}%
    {\\\hline\end{tabular}\par\nopagebreak\addvspace{2.3ex}%
     \vskip -\parskip}
\newcommand{\declline}[1]{\\\multicolumn1{|r|}{\small#1}}
\newcommand{\m}[1]{\mbox{$\langle$\emph{#1}$\rangle$}}
\renewcommand{\arg}[1]{{\tt\string{}\m{#1}{\tt\string}}}
\expandafter\ifx\csname oarg\endcsname\relax
  \newcommand{\oarg}[1]{{\tt[}\m{#1}{\tt]}}
\fi
\makeatletter
\expandafter\ifx\csname cmd\endcsname\relax
  \def\cmd#1{\cs{\expandafter\cmd@to@cs\string#1}}
\fi
\expandafter\ifx\csname cmd@to@cs\endcsname\relax
\def\cmd@to@cs#1#2{\char\number`#2\relax}
\fi
\makeatother
\expandafter\ifx\csname cs\endcsname\relax
\def\cs#1{{\tt\char`\\#1}}
\fi
\newcommand{\env}[2]{\cmd{#1}{\protect\tt\char`\{#2\char`\}}}
\newcommand{\envname}[1]{{\protect\tt#1}}
\germanTeX
\expandafter\ifx\csname emph\endcsname\relax
  \newcommand\emph[1]{{\em#1\/}}% This is \emph{not} the LaTeX2e
                                %  definition!
\fi

\begin{document}
\maketitle
\renewcommand{\textfraction}{0.10}
\renewcommand{\topfraction}{0.65}
\renewcommand{\bottomfraction}{0.85}
\expandafter\ifx\csname sect\endcsname\relax
  \let\sect=\section
\fi
\expandafter\ifx\csname ssect\endcsname\relax
  \let\ssect=\subsection
\fi
\expandafter\ifx\csname sssect\endcsname\relax
  \let\sssect=\subsubsection
\fi
\sect{Benutzerhandbuch (User's Guide)}

Mit \LaTeX\ k"onnen (nat"urlich) auch Briefe geschrieben werden.
F"ur englische Briefe gibt es die Dokumentklasse \envname{letter}.
Deutsche Briefe k"onnen mit der Klasse \envname{dinbrief}
geschrieben werden.

Die vorliegende Klasse \envname{dinbrief} basiert auf dem
\envname{dinbrief}-Stil der Universit"at Karlsruhe. Dieser
Stil wurde aus \envname{dletter.sty} entwickelt, welcher
von D.~Heinrich abgefa"st wurde. Seinerseits geht dieser
zur"uck auf den Stil \envname{a4letter} von H.~Partl, welcher
seinen Ursprung im urspr"unglichen \envname{letter}-Stil
von L.~Lamport hatte. Zus"atzlich wurden aus den \envname{dinbrief.sty}
von R.~Sengerling der Befehlsvorrat und die Option \Dopt{norm}
"ubernommen. Ferner enth"alt diese Klasse Tips und Anregungen von
B.~Raichle.

In den Briefen k"onnen u.a.\ Formeln,
Tabellen und beliebige Listen verwendet werden.
In einem Dokument k"onnen mehrere Briefe geschrieben werden.
Die Gliederung in Abs"atze erfolgt durch Einf"ugen einer
Leerzeile (wie in \LaTeX\ "ublich).

\sect{Zus"atzliche Optionen der {\protect\tt dinbrief}-Klasse}%
\index{Briefe>Optionen}\index{Optionen der dinbrief-Klasse=Optionen der \envname{dinbrief}-Klasse}

In der DIN Norm~676 werden zwei Formen von Briefen festgelegt.
In "alteren Versionen dieses Paketes wurde nur diejenige Form unterst"utzt,
in der oben ein Rand von 45mm f"ur den Briefkopf freigelassen wird.
In der zweiten Version ist dieser Rand auf 27mm verk"urzt; damit steht f"ur
den Brieftext mehr Platz (18mm) zur Verf"ugung.

Die Wahl der gew"unschten Form kann durch Angabe der Optionen \Dopt{addresshigh}
f"ur einen oberen Rand von 27mm oder \Dopt{addressstd} f"ur einen oberen Rand
von 45mm getroffen werden.
Die Angabe der Option \Dopt{addressstd} kann unterbleiben, da sie ohnehin
voreingestellt ist.

\sect{Befehle in der {\protect\tt dinbrief}-Klasse}%
\index{Briefe>DIN 676}\index{DIN-Brief}\index{Briefe>dinbrief=\envname{dinbrief}}

Bereits vor \env\begin{document} kann man Angaben machen, die f"ur
alle Briefe g"ultig sind, z.B.\ {\bf Absender\/} (\cs{address}
bzw.\ \cs{backaddress}), {\bf Absendeort\/} (\cs{place}),
{\bf Telefon\/} (\cs{phone}) und {\bf Unterschrift\/} (\cs{signature}).

\ssect{Aus der {\protect\tt letter}-Klasse "ubernommene Befehle}

Jeder Brief steht in einer eigenen \envname{letter}-Umgebung.
Der Empf"anger wird als Argument des
\env\begin{letter}-Befehls angegeben
(\env\begin{letter}\arg{Anschrift}).

Eine entscheidende Bedeutung beim Schreiben von Briefen
hat der \cs{opening}-Befehl. Nur dieser Befehl setzt
den Briefkopf, die Absenderangaben und die Adresse des
Empf"angers. Die {\bf Anrede des Empf"angers\/} wird als
Argument angegeben (\cmd\opening\arg{Anrede}).

Danach folgt der eigentliche Brieftext. Die abschlie"sende
{\bf Gru"sformel\/} wird mit dem Befehl \cmd\closing\arg{Gru"sformel}
gesetzt. Dieser Befehl f"ugt auch die maschinenschriftliche
Wiederholung der Unterschrift an, wie sie mit dem
\cs{signature}-Befehl festgelegt wurde.
Die erweiterte Form \cmd\closing\oarg{Unterschrift}\arg{Gru"sformel}
erm"oglicht es, die Unterschrift unter den Brief als Text oder Graphik
einzubinden.

Im Anschlu"s an die Gru"sformel werden {\bf Anlagen-\/}
(\cmd\encl\arg{Anlage}), {\bf Verteilvermerke\/}
(\cmd\cc\arg{Verteiler}) und das {\bf Postscriptum\/}
(\cmd\ps\arg{Postscriptum}) an den Brief angef"ugt.

Mit dem Befehl \cs{makelabels} (vor \env\begin{document})
werden zus"atzlich {\bf Adre"s-Etiketten\/} erzeugt.

Dar"uberhinaus wurden die oben erw"ahnten Befehle
\cs{address}, \cs{place} und \cs{signature} aus
der \envname{letter}-Klasse "ubernommen.

\begin{figure}[p]
\begin{center}
{\small
\begin{verbatim}
\documentclass[12pt]{dinbrief}
\usepackage{german}

\address{R"udiger Kurz\\
         Am See 1\\
         76133 Karlsruhe}
\backaddress{R. Kurz, Am See 1, 76133 Karslruhe}

\signature{R"udiger Kurz}
\place{Karlsruhe}

\begin{document}
\phone{0721}{222222}
\begin{letter}{Deutsche Bundespost\\
               Fernmeldeamt Karlsruhe\\
               Postfach 7300\\[\medskipamount]
               {\bf 76131 Karlsruhe}}

\yourmail{01.04.93}
\sign{123456}
\subject{Betrieb eines Mikrowellensenders}

\opening{Sehr geehrte Damen und Herren,}

anbei sende ich Ihnen eine Kopie der bisherigen Genehmigung f"ur
unseren Mikrowellenherd...

... Ihre Bem"uhungen im voraus vielen Dank.

\closing{Mit freundlichen Gr"u"sen,}

\ps{Wir bitten um schnelle Erledigung.}
\cc{Deutsche Bundespost\\
    Karlsruher Privatfunk \\
    S"uddeutscher Rundfunk}

\encl{Abschrift der Urkunde}

\end{letter}
\end{document}
\end{verbatim}}
\caption{Brief mit \LaTeX.}\label{brief}
\end{center}
\end{figure}

\ssect{Zus"atzliche Befehle im DIN-Brief}

Der Befehl \cmd\phone\arg{Vorwahl}\arg{Rufnummer/Durchwahl}
legt die {\bf Telefonnummer\/} des Absenders fest.
Sie wird in der Bezugszeichenzeile ausgegeben.

Der {\bf Bezug\/} auf einen empfangenen Brief ist m"oglich
mit Hilfe des Befehls
\cmd\yourmail\arg{Ihre Zeichen, Ihre Nachricht vom}.

Mit dem Befehl \cmd\sign\arg{Unsere Zeichen, unsere Nachricht vom}
kann eine {\bf eigene Kennzeichnung\/} des Briefes angegeben werden.

Mit dem Befehl \cmd\writer\arg{Sachbearbeiter} kann
der {\bf Name des Sachbearbeiters\/} festgelegt werden.

Die Bezugszeichenzeile wird nur gesetzt, falls einer der
Befehle \cs{yourmail}, \cs{sign} oder \cs{writer} verwendet
wird. Der Befehl \cs{writer} schaltet zus"atzlich auf das
in der DIN Norm~676 (Entwurf Mai~1991) festgelegte Layout um.

Der {\bf Betreff (die stichwortartige Inhaltsangabe)\/} des
Briefes wird durch den Befehl \cs{subject}\arg{Betreff}
angegeben. Mit \cs{backaddress} wird die Adresse festgelegt,
die als {\bf Absenderadresse im Brief\kern0pt fenster}
eingeblendet wird.

Abbildung~\ref{brief} enth"alt ein Beispiel f"ur einen Brief.
Die Anwendung der Befehle und ihre Reihenfolge in der Quelldatei
kann dem Beispiel entnommen werden.

F"ur alle, denen das "`Fenster"' um die Adresse nicht gef"allt,
besteht die M"oglichkeit, durch Angabe von
\cmd\nowindowrules\index{nowindowrules=\verb+\nowindowrules+}
vor dem Befehl \cmd\opening\ dieses abzuschalten.
Durch \cmd\windowrules\index{windowrules=\verb+\windowrules+} l"a"st es
sich wieder aktivieren.

Der Trennstrich zwischen der R"ucksendeadresse und der Anschrift im
Anschriftenfeld kann mit dem Befehl
\cmd\nobackaddressrule\index{nobackaddressrule=\verb+\nobackaddressrule+}
unterdr"uckt und bei folgenden Briefen mit
\cmd\backaddressrule\index{backaddressrule=\verb+\backaddressrule+} wieder
aktiviert werden.

Die Faltmarkierung am linken Blattrand wird durch den Befehl
\cmd\nowindowtics\index{nowindowtics=\verb+\nowindowtics+} ab- und mit
\cmd\windowtics\index{windowtics=\verb+\windowtics+} wieder angeschaltet.
\medskip

Die Ver"anderung des Layouts der Briefe ist mit Hilfe von insgesamt 13 Befehlen
m"oglich.
Um f"ur einen Brief wirksam zu werden, m"ussen diese Befehle {\em vor\/}
dem Befehl \cmd\opening\ angegeben werden.
Die ersten 9 dieser Befehle legen Gr"o"se und Position des Fensters f"ur die
Anschrift und der Adresse des Absenders im Anschriften-Fenster fest.
Die "ubrigen Befehle legen den Abstand der Oberkante der Bezugszeichenzeile,
den Abstand der Oberkante des Bereiches f"ur Angaben zu Kapitalgesellschaften
und die Abst"ande der Faltmarkierungen vom oberen Papierrand fest.
Ein Befehl zur Festlegung der Lochmarkierung ist "ubrigens nicht vorgesehen,
da die Lochmarkierung in die Mitte der Seite -- abh"angig vom Wert der L"ange
\cmd\paperheight\ -- gesetzt wird.

Der Befehl \cmd\setaddressllcorner\arg{L"angenangabe}\arg{L"angenangabe}\index{setaddressllcorner=\verb+\setaddressllcorner+}
legt mit seinen beiden Argumenten die Position der linken {\em unteren\/} Ecke
des Fensters bezogen auf die linke {\em obere\/} Ecke des Blattes fest: das
erste den Abstand vom linken und das zweite den Anstand vom oberen Blattrand.
Der Abstand des Fensters vom linken Rand -- ohne "Anderung des Abstandes vom
oberen Rand -- kann mit dem Befehl
\cmd\setaddressllhpos\arg{L"angenangabe}\index{setaddressllhpos=\verb+\setaddressllhpos+}
festgelegt werden.
Entsprechend legt der Befehl
\cmd\setaddressllvpos\arg{L"angenangabe}\index{setaddressllvpos=\verb+\setaddressllvpos+}
den Abstand des unteren Randes des Fensters vom oberen Blattrand fest.

Zur Auswahl der zu verwendenden Variante der Briefform -- hochgestellte oder
tiefer gestelltes Feld f"ur die Anschrift -- stehen die beiden Befehle
\cmd\addresshigh\ (hochgestellte Anschrift) und \cmd\addressstd\
(voreingestellte tiefer gestellte Anschrift) zur Verf"ugung.
Beide Befehle sind nur wirksam, wenn sie {\em vor\/} dem \cmd\opening-Befehl
angegeben werden.
Gegebenenfalls k"onnen sie auch -- wie viele anderen Befehle -- in der
Konfigurationsdatei verwendet werden.

Auch zur Festlegung der H"ohe und Breite des Fensters f"ur die Anschrift sind
Befehle vorgesehen.
Der Befehl \cmd\setaddresswidth\arg{L"angenangabe}\index{setaddresswidth=\verb+\setaddresswidth+}
setzt die Breite des Fensters auf die angegebene L"ange.
Entsprechend legt der Befehl
\cmd\setaddressheight\arg{L"angenangabe}\index{setaddressheight=\verb+\setaddressheight+}
die H"ohe des Fensters fest (ohne das direkt dar"uber angeordnete Feld f"ur
den Absender).

In der Regel m"ochte man den Text der Anschrift nicht direkt am linken Rand
beginnen lassen -- wie auch in diesem Paket voreingstellt.
Mit dem Befehl \cmd\setaddressoffset\arg{L"angenangabe}\index{setaddressoffset=\verb+\setaddressoffset+}
kann man diese Einr"uckung selbst festlegen.
Der Text der Anschrift wird "ubrigens nicht abgeschnitten und kann daher
gegebenenfalls "uber den rechten Rand des Fensters hinausragen -- bitte
brechen Sie in diesem Fall zu lange Zeilen an einer geigneten Stelle selbst
um.

Die H"ohe des Feldes f"ur den Absender direkt oben anschlie"send an das
Feld f"ur die Anschrift kann mit dem Befehl
\cmd\setbackaddressheight\arg{L"angenangabe}\index{setbackaddressheight=\verb+\setbackaddressheight+}
festgelegt werden.

Werden der DIN Norm entsprechende Fensterumschl"age verwendet, sollten die
Voreinstellungen f"ur Gr"o"se und Position des Fensters f"ur die Anschrift
und den Absender nicht ver"andert werden, damit sie im Fenster sichtbar sind.

Der Befehl \cmd\setreflinetop\arg{L"angenangabe}\index{setreflinetop=\verb+\setreflinetop+}
legt den Abstand der Oberkante der Bezugszeichenzeile vom oberen Papierrand
fest; bei den beiden Formen der DIN Norm liegt diese Oberkante 8.5mm unterhalb
des Fensters f"ur die Anschrift.

Am Ende der ersten Seite von Briefen ist ein Feld f"ur Angaben zu
Kapitalgesellschaften vorgesehen, das man bei eigenen Briefen gegebenenfalls
zur Angabe des eigenen Kontos verwenden kann.
Die Formbl"atter der Norm sehen f"ur dieses Feld nur die letzten 16mm des
Papiers vor.
Ben"otigt man mehr Platz f"ur Angaben in diesem Feld, kann man mit dem Befehl
\cmd\setbottomtexttop\arg{L"angenangabe}\index{setbottomtexttop=\verb+\setbottomtexttop+}
die Oberkante des Feldes geeignet festlegen.
Der f"ur den Brieftext zur Verf"ugung stehende Platz wird automatisch
angepa"st.

Im Zusammenhang mit einer Verschiebung des Fensters f"ur die Anschrift
m"ussen gegebenenfalls die Faltmarkierungen angepa"st werden.
Den Abstand der oberen Faltmarkierung vom oberen Blattrand legt man mit dem
Befehl \cmd\setupperfoldmarkvpos\arg{L"angenangabe}\index{setupperfoldmarkvpos=\verb+\setupperfoldmarkvpos+}
fest, den der unteren Faltmarkierung entsprechend mit dem Befehl
\cmd\setlowerfoldmarkvpos\arg{L"angenangabe}\index{setlowerfoldmarkvpos=\verb+\setlowerfoldmarkvpos+}.
Die H"ohe der Markierung zum Lochen (in der Blattmitte) ergibt sich automatisch
aus der Blatth"ohe und kann nicht ver"andert werden.
\medskip

Ausf"uhrungen und Erl"auterungen zum Erstellen von Adre"s-Etiketten
finden Sie in einem eigenen Abschnitt weiter hinten in diesem Dokument.
Die Befehlsreferenz enth"alt ebenfalls kurze Beschreibungen der
Befehle zu den Adre"s-Etiketten.

Eine einfache Absenderanschrift in der auf Schreibmaschinen "ublichen
Form (DIN~5008) wird durch den Befehl \cmd\stdaddress\arg{Absenderanschrift}
anstelle von \cmd\address{} erreicht. Die einzelnen Zeilen sind durch
\verb.\\. zu trennen. Das Datum wird ebenfalls oben ausgegeben.
Die Bezugszeichenzeile entf"allt.

Mit dem Befehl \cmd\postremark\ kann ein Postvermerk gesetzt werden.
Dieser Vermerk erscheint im Anschriftenfeld oben, von den "ubrigen
im Argument des Befehls \env\begin{letter}\arg{Anschrift} durch eine
Leerzeile getrennt. Der Behandlungsvermerk wird durch den
Befehl \cmd\handling\arg{Text}\ vereinbart.

Die Anlagen- und Verteilvermerke k"onnen mit dem Befehl \cmd\enclright\
rechts auf Grad~50 anstelle nach der Gru"sformel ausgegeben werden.
Die Befehle \cmd\encl\ und \cmd\cc\ sind dazu vor dem \cmd\closing-Befehl
anzugeben.

\ssect{Befehlsreferenz}

\begin{description}
  \item[\env\begin{letter}\arg{{Anschrift}} \dots\ \env\end{letter}] \hfil\break
        Diese Befehle rahmen jeden einzelnen Brief ein. Die
        Anschrift des Em\-pf"an\-gers wird als Argument des Befehls
        \env\begin{letter}\arg{Anschrift} angegeben.
        Die einzelnen Zeilen in der Anschrift werden durch
        \cmd\\ getrennt.
        Es d"urfen weitere Briefe folgen.

  \item[\cmd\signature\arg{Unterschrift des Absenders}] \hfil\break
        Dieser Befehl legt die maschinenschriftliche
        Wiederholung der Unterschrift fest.
        Der Befehl gilt solange, bis ein weiterer
        \cmd\signature-Befehl eine neue "`Unterschrift"'
        festlegt.

  \item[\cmd\address\arg{{Name und Adresse des Absenders}}] \hfil\break
        Die Adresse des Absenders wird vereinbart.
        Dieser Befehl gilt f"ur den laufenden und alle
        weiteren Briefe; er gilt solange, bis ein
        weiterer \cmd\address-Befehl angegeben wird.

  \item[\cmd\backaddress\arg{{Absenderadresse im Brief\kern 0pt fenster}}] \hfil\break
        Der Befehl legt die Anschrift des Absenders oben im
        Anschriftenfeld des Briefs fest.

  \item[\cmd\place\arg{{Ortsangabe im Brief}}] \hfil\break
        Mit diesem Befehl wird der Absendeort angegeben, der
        zusammen mit dem Datum im Briefkopf ausgegeben wird.

  \item[\cmd\date\arg{{Briefdatum}}] \hfil\break
        Soll als Absendedatum {\sl nicht\/} das aktuelle
        Tagesdatum (des Rechners) eingesetzt werden,
        kann mit diesem Befehl das Datum
        explizit angegeben werden.

        Ohne diesen Befehl wird das aktuelle Tagesdatum
        im Brief verwendet.

  \item[\cmd\yourmail\arg{{Ihre Zeichen, Ihre Nachricht vom}}] \hfil\break
        Der Befehl legt den Inhalt des Feldes {\bf Ihre
        Zeichen, Ihre Nachricht vom\/} in der
        Bezugszeichenzeile fest.

  \item[\cmd\sign\arg{{Unsere Zeichen (, unsere Nachricht vom)}}] \hfil\break
        Dieser Befehl legt den Inhalt des Feldes
        {\bf Unsere Zeichen \dots\/} fest.

  \item[\cmd\phone\arg{{Vorwahl}}\arg{{Rufnummer/Durchwahl}}] \hfil\break
        Die Telefonnummer aufgeteilt nach Vorwahl und Rufnummer oder
        Durchwahl wird mit dem Befehl \cmd\phone{} vereinbart. Diese
        Angaben werden in der Bezugszeichenzeile ausgegeben.

  \item[\cmd\writer\arg{{Sachbearbeiter}}] \hfil\break
        Die Neufassung der Norm DIN~676 vom Mai 1991 sieht in
        der Bezugszeichenzeile ein weiteres Feld f"ur den
        Sachbearbeiter vor. Mit dem Befehl \cmd\writer{}
        kann ein solcher Sachbearbeiter angegeben werden.

        {\sl Die Verwendung dieses Befehls gestaltet den Brief
        entsprechend der Norm DIN~676 vom Mai 1991.\/}

  \item[\cmd\subject\arg{{Betreff}}] \hfil\break
        Mit diesem Befehl wird der Betreff gesetzt, der den
        Empf"anger "uber den Gegenstand des Briefes
        informiert.

        \cmd\concern{} steht aus historischen Gr"unden
        als Synonym bis auf weiteres zur Verf"ugung.

  \item[\cmd\centeraddress] \hfil\break
        Die Empf"angeranschrift wird im Brief\kern 0pt fenster
        vertikal zentriert.

  \item[\cmd\normaladdress] \hfil\break
        Die Empf"angeranschrift wird im Anschriftenfeld unten gesetzt.

  \item[\cmd\opening\arg{{Anrede}}] \hfil\break
        Dieser Befehl vereinbart die Anrede des Empf"angers und setzt
        den Briefkopf, die Empf"angerangaben, eine eventuell vorhandene
        Bezugszeichenzeile, den Betreff und die Anrede des Empf"angers.

        {\bf Dieser Befehl darf nicht fehlen!\/}

  \item[\cmd\closing\oarg{Unterschrift}\arg{{Gru"sformel}}] \hfil\break
        Der Befehl \cmd\closing{} setzt die Gru"sformel und nach
        drei Leerzeilen die maschinenschrifliche Wiederholung
        der Unterschrift.

        Die erweiterte Form \cmd\closing\oarg{Unterschrift}\arg{Gru"sformel}
        setzt zus"atzlich das optionale Argument -- beispielsweise eine mit
        \cmd\includegraphics\ eingebundene PostScript-Datei -- als
        Unterschrift in die freigelassenen Zeilen.
        "Uberschreitet die H"ohe des optionalen Argumentes zwei Zeilen,
        rutscht die maschinenschrifliche Wiederholung der Unterschrift
        entsprechend nach unten.
        Bitte beachten Sie, da"s ein optionales Argument eckige Klammern nur
        enthalten darf, wenn das optionale Argument selbst in geschweifte
        Klammern eingeschlossen ist.

  \item[\cmd\encl\arg{{Anlagen}}] \hfil\break
        Der Vermerk "uber dem Brief beigef"ugte Anlagen
        wird mit dem Befehl \cmd\encl{} an den Brief
        angeh"angt. Die einzelnen Eintragungen k"onnen
        durch \cmd\\{} getrennt werden.

        Die Reihenfolge der Verwendung der Befehle
        \cmd\encl, \cmd\cc{} und \cmd\ps{} ist beliebig,
        falls die Befehle nach dem \cmd\closing-Befehl
        angegeben werden. Die Norm empfiehlt allerdings
        den Anlagenvermerk vor dem Verteilvermerk
        anzubringen.

  \item[\cmd\cc\arg{{Verteiler}}] \hfil\break
        Der Vermerk "uber weitere Empf"anger dieses Briefes wird
        mit dem Befehl \cmd\cc{} gesetzt. Die einzelnen Eintragungen
        k"onnen durch \cmd\\ getrennt werden.

        Die Befehle \cmd\encl{} und \cmd\cc{} k"onnen sowohl
        vor wie auch nach dem \cmd\closing-Befehl stehen. Wird
        der Befehl \cmd\enclright{} verwendet, m"ussen die beiden
        Befehle vor dem \cmd\closing-Befehl stehen.

  \item[\cmd\ps\arg{{Postscriptum}}] \hfil\break
        Gesch"aftsbriefe enthalten kein Postskriptum.
        Es wurde trotzdem die M"og\-lich\-keit geschaffen,
        ein solches zu verwenden. Mit dem Befehl
        \cmd\ps{} wird ein Postskriptum gesetzt.

  \item[\cmd\makelabels] \hfil\break
        Dieser Befehl mu"s in der Pr"aambel stehen; also
        zwischen \cmd\documentstyle{} oder \cmd\documentclass{}
        und dem \env\begin{document}-Befehl.
        Er aktiviert das Ausdrucken von Adress-Etiketten.

  \item[\cmd\labelstyle\arg{{Stil der Label}}] \hfil\break
        Dieser Befehl vereinbart das Layout der
        Adress-Etiketten. Es gibt Drucker, die in der
        Lage sind, Briefumschl"age zu bedrucken. Mit diesem
        Befehl legt man die Form der Briefumschl"age fest.

        {\sl Zur Zeit steht nur das Layout \env\labelstyle{plain}
        zur Verf"ugung.\/}

  \item[\cmd\bottomtext\arg{{Feld f\"ur Kapitalgesellschaften}}] \hfil\break
        Am Fu"s der ersten Briefseite werden Gesch"aftsangaben und
        zus"atzlich bei Kapitalgesellschaften gesellschaftsrechtliche
        Angaben angegeben. Der Befehl \cmd\bottomtext{} vereinbart
        diese Angaben.


  \item[\cmd\windowrules] \hfil\break
        Das Anschriftenfeld im Brief wird durch Linien ober-
        und unterhalb vom "ubrigen Brief abgegrenzt.
        Die Hervorhebung wird aktiviert.

  \item[\cmd\nowindowrules] \hfil\break
        Der Befehl schaltet den Rahmen ab.

  \item[\cmd\backaddressrule] \hfil\break
        Der Absender und die Anschrift im Anschriftenfeld des Briefes werden
durch eine Linie getrennt (Voreinstellung).

  \item[\cmd\nobackaddressrule] \hfil\break
Die Trennungslinie zwischen dem Absender und der Anschrift im
Anschriftenfeld des Briefes wird unterdr"uckt.

  \item[\cmd\windowtics] \hfil\break
        Es werden Faltmarkierungen am linken Briefrand
        geruckt.

  \item[\cmd\nowindowtics] \hfil\break
        Es werden keine Faltmarkierungen am linken Briefrand
        ausgedruckt.

  \item[\cmd\disabledraftstandard] \hfil\break
        Der Brief wird entsprechend den
        Vorschriften der Norm DIN~676 vom Dezember 1976 auf
        dem Briefbogen ausgegeben.

  \item[\cmd\enabledraftstandard] \hfil\break
        Der Brief wird entsprechend den
        Vorschriften des Entwurfs der Norm DIN~676 vom
        Mai~1991 auf dem Briefbogen ausgegeben.

  \item[\cmd\setaddressllcorner\arg{Abstand vom linken Rand}%
        \arg{Abstand vom oberen  Rand}] \hfil\break
        Mit diesem Befehl legt man die Position der linken unteren Ecke
        des Fensters f"ur die Anschrift des Empf"angers fest.
        Beide Argumente sind L"angen, wobei die erste den (horizontalen)
        Abstand zum linken Blattrand und die zweite den (vertikalen)
        Abstand zum oberen Blattrand angibt.

  \item[\cmd\setaddressllhpos\arg{Abstand vom linken Rand}] \hfil\break
        Der Befehle legt den (horizontalen) Abstand des Fensters f"ur die
        Anschrift vom linken Blattrand fest.
        Soll gleichzeitig die H"ohe des Fensters ge"andert werden, kann der
        Befehle \cmd\setaddressllcorner\ verwendet werden.

        In der DIN Norm ist der Abstand vom linken Rand fest als 20mm
        vorgegeben (dieser Wert ist voreingestellt).

  \item[\cmd\setaddressllvpos\arg{Abstand vom oberen Rand}] \hfil\break
        Der Befehl legt den (vertikalen) Abstand des Fensters f"ur die
        Anschrift vom oberen Blattrand fest.
        Soll das Fenster gleichzeitig horizontal verschoben werden, kann der
        Befehle \cmd\setaddressllcorner\ verwendet werden.

        Die DIN Norm sieht zwei Varianten mit einer hochgestellten und einer
        tiefer gestellten Anschrift vor.
        Voreingestellt ist die tiefer gestellte Variante mit einem Wert von
        90mm.
        F"ur die hochgestellte Variante ist der Wert 72mm vorgeschrieben.

        Zur einfachen Auswahl der Varianten stehen die beiden Optionen
        \Dopt{addresshigh} (hochgestellte Anschrift) und \Dopt{addressstd}
(voreingestellte tiefer gestellte Anschrift) vorgesehen.

  \item[\cmd\addresshigh] \hfil\break
        Dieser Befehl setzt die L"angen \verb|\addrvpos|, \verb|\reflinevpos|,
        \verb|\ufldmarkvpos| und \verb|\lfldmarkvpos| entsprechend der Variante
        mit hochgestelltem Feld f"ur die Anschrift entsprechend der Form~A der
        DIN Norm.
        Der Befehl ist nur wirksam, wenn er {\em vor\/} dem
        \cmd\opening-Befehl steht (auch in der Konfigurationsdatei).

  \item[\cmd\addressstd] \hfil\break
        Der Befehl setzt die L"angen \verb|\addrvpos|, \verb|\reflinevpos|,
        \verb|\ufldmarkvpos| und \verb|\lfldmarkvpos| entsprechend der Variante mit
        tiefer gestelltem Feld f"ur die Anschrift entsprechend der Form~B der
        DIN Norm und entspricht der Voreinstellung.
        Der Befehl ist ebenfalls nur {\em vor\/} dem \cmd\opening-Befehl
        wirksam (auch in der Konfigurationsdatei).

  \item[\cmd\setaddresswidth\arg{Breite des Fensters f"ur die Anschrift}]
        \hfil\break
        Mit diesem Befehl wird die Breite des Fensters f"ur die Anschrift
        festgelegt (voreingestellt: 85mm).

  \item[\cmd\setaddressheight\arg{H"ohe des Fensters f"ur die Anschrift}]
        \hfil\break
        Die H"ohe des Fensters f"ur die Anschrift kann man mit diesem Befehl
        festlegen.  Voreingestellt sind 40mm.

  \item[\cmd\setaddressoffset\arg{Abstand vom linken Fensterrand}] \hfil\break
        In der Regel m"ochte man, da"s die Anschrift nicht direkt am linken
        Rand des Fensters f"ur die Anschrift beginnt, sondern einige
        Millimeter einger"uckt ist.
        Mit diesem Befehl kann man die Einr"uckung festlegen.
        Voreingestellt sind 4mm.
        Die Anschrift beginnt bei dieser Einstellung b"undig mit dem
        Text des Briefes.

  \item[\cmd\setbackaddressheight\arg{H"ohe des Feldes f"ur den Absender}]
        \hfil\break
        Das Feld f"ur den Absender schlie"st oben an das Fensters f"ur die
        Anschrift an.  Die H"ohe dieses Feldes wird mit dem Befehl
        \cmd\setbackaddressheight\ festgelegt.  Voreingestellt sind 5mm.

  \item[\cmd\setreflinetop\arg{Abstand vom oberen Rand}] \hfil\break
        Mit diesem Befehl wird der Abstand der Oberkante der
        Bezugszeichenzeile vom oberen Blattrand festgelegt.
        Voreingestellt sind 98.5mm; bei Angabe der Option f"ur eine
        hochgestellte Anschrift wird der Wert auf 80.5mm gesetzt.

  \item[\cmd\setbottomtexttop\arg{Abstand vom oberen Rand}] \hfil\break
        Mit diesem Befehl legt man den den Abstand der Oberkante des Feldes
        f"ur Gesch"aftsangaben und (nur bei Kapitalgesellschaften)
        gesellschaftsrechtliche Angaben auf der ersten Seite und dem oberen
        Blattrand fest.
        Voreingestellt sind 281mm; bei mehrzeiligen Angaben in diesem Fekd
        mu"s man den Wert entsprechend verringern.
        Der Abgleich mit der H"ohe der letzten verf"ugbaren Zeile auf der
        Seite erfolgt automatisch.

  \item[\cmd\setupperfoldmarkvpos\arg{Abstand vom oberen Rand}] \hfil\break
        Mit diesem Befehl legt man den Abstand der \emph{oberen}
        Faltmarkierung vom oberen Blattrand fest.
        Voreingestellt sind 105mm und bei Angabe der Option f"ur eine
        hochgestellte Anschrift 87mm.

  \item[\cmd\setlowerfoldmarkvpos\arg{Abstand vom oberen Rand}] \hfil\break
        Den Abstand der \emph{unteren} Faltmarkierung vom oberen Blattrand
        legt man mit diesem Befehl fest.
        Voreingestellt sind 210mm und bei Angabe der Option f"ur eine
        hochgestellte Anschrift 192mm.

  \item[\cmd\setlabelwidth\arg{Breite eines Labels}] \hfil\break
        Dieser Befehl legt die Breite eines Labels fest.
        Als Argument erwartet dieser Befehl eine L"angenangabe.

        {\sl Die Labelbreite ist in der vorliegenden Version
        auf 105~$mm$ festgelegt worden und sollte nicht
        ge"andert werden. Dieser Befehl ist f"ur sp"atere
        Erweiterungen reserviert.\/}

  \item[\cmd\setlabelheight\arg{H"ohe eines Labels}] \hfil\break
        Der Befehl \cmd\setlabelheight{} vereinbart die im
        Argument angegebene L"ange als Labelh"ohe.

  \item[\cmd\setlabeltopmargin\arg{oberer Rand}] \hfil\break
        Bevor die beiden obersten Label (bei beiden Spalten)
        ausgegeben werden, wird oben ein Rand gelassen, dessen
        H"ohe mit \cmd\setlabeltopmargin{} angegeben wird.
        Es wird eine L"angenangabe erwartet.

  \item[\cmd\setlabelnumber\arg{Labelanzahl pro Spalte}] \hfil\break
        Dieser Befehl bestimmt die Zahl der Labels in einer Spalte.

  \item[\cmd\spare\arg{Anzahl leerer Labels}] \hfil\break
        Es werden die im Argument des Befehls angegebene
        Anzahl von Label freigelassen, bevor das erste
        Adre"s--Etikett ausgegeben wird. Die Label werden
        spaltenweise durchgez"ahlt.

  \item[\cmd\offlabels] \hfil\break
  \item[\cmd\onlabels] \hfil\break
  \item[\cmd\nolabels] \hfil\break
  \item[\cmd\stdaddress\arg{Adresse des Absenders}] \hfil\break
        Dieser Befehl setzt die Absenderanschrift in der
        auf Schreibmaschinen "ublichen Form (DIN 5008).
        Mit diesem Befehl kann der Briefkopf nicht frei gestaltet
        werden.

  \item[\env\begin{dinquote} \dots\ \env\end{dinquote}] \hfil\break
        Diese Umgebung r"uckt den Text auf der linken Seite um
        ein Inch ein. Rechts folgt keine Einr"uckung.

  \item[\cmd\enclright] \hfil\break
        Die Anlagen- und Verteilvermerke beginnen, falls sie vor
        den \cmd\closing-Befehl angegeben wurden, rechts
        neben der Gru"sformel.

  \item[\cmd\postremark\arg{Postvermerk}] \hfil\break
        Der Postvermerk wird mit dem Befehl \cmd\postremark{}
        vereinbart. Der Postvermerk geht der Empf"angeranschrift
        mit einer Leerzeile voraus. Dieser Befehl mu"s zwischen
        \env\begin{letter} und dem Befehl \cmd\opening\ stehen.

  \item[\cmd\handling\arg{Behandlungsvermerk}] \hfil\break
        Der Behandlungsvermerk wird rechts neben der
        Empf"angeranschrift auf Grad 50 (bei einer 10er Teilung)
        in H"ohe der letzten Zeile der Empf"angeranschrift
        ausgegeben. Dieser Befehl mu"s zwischen
        \env\begin{letter} und dem Befehl \cmd\opening\ stehen.
\end{description}


\begin{table}[htp]
\caption{Zusammenfassung der Dinbrief-Befehle (Teil 1):}\index{dinbrief>Befehle}

\begin{center}
\begin{tabular}{l}
  \hline
    \verb|\begin{document}|                                          \\
    \verb|\end{document}|                                            \\
  \hline
    \verb|\begin{letter}|\arg{{Anschrift}}                           \\
    \verb|\end{letter}|                                              \\
  \hline
    \verb|\signature|\arg{Unterschrift des Absenders}           \\
    \verb|\address|\arg{{Name und Adresse des Absenders}}       \\
    \verb|\backaddress|\arg{{Absenderadresse im Brieffenster}}  \\
  \hline
    \verb|\place|\arg{{Ortsangabe im Brief}}                    \\
    \verb|\date|\arg{{Briefdatum}}                              \\
    \verb|\yourmail|\arg{{Ihre Zeichen, Ihre Nachricht vom}}    \\
    \verb|\sign|\arg{{Unsere Zeichen (, unsere Nachricht vom)}} \\
    \verb|\phone|\arg{{Vorwahl}}\arg{{Rufnummer/Durchwahl}}     \\
    \verb|\writer|\arg{{Sachbearbeiter}}                        \\
  \hline
    \verb|\subject|\arg{{Betreff}}                              \\
    \verb|\concern|\arg{{Betreff}}                              \\
    \verb|\opening|\arg{{Anrede}}                               \\
    \verb|\closing|\oarg{Unterschrift}\arg{{Gru"sformel}}       \\
  \hline
    \verb|\centeraddress|                                       \\
    \verb|\normaladdress|                                       \\
  \hline
    \verb|\encl|\arg{{Anlagen}}                               \\
    \verb|\ps|\arg{{Postscriptum}}                            \\
    \verb|\cc|\arg{{Verteiler}}                               \\
  \hline
    \verb|\makelabels|                                        \\
    \verb|\labelstyle|\arg{{Stil der Label}}                  \\
  \hline
    \verb|\bottomtext|\arg{{Feld f\"ur Kapitalgesellschaften}}\\
  \hline
    \verb|\nowindowrules|                                     \\
    \verb|\windowrules|                                       \\
    \verb|\nobackaddressrule|                                 \\
    \verb|\backaddressrule|                                   \\
    \verb|\nowindowtics|                                      \\
    \verb|\windowtics|                                        \\
  \hline
    \verb|\disabledraftstandard|                              \\
    \verb|\enabledraftstandard|                               \\
  \hline
    \verb|\setaddressllcorner|\arg{Abstand vom linken Rand}   \\
    \verb|\setaddressllhpos|\arg{Abstand vom linken Rand}     \\
    \verb|\setaddressllvpos|\arg{Abstand vom oberen Rand}     \\
    \verb|\addresshigh|                                       \\
    \verb|\addressstd|                                        \\
    \verb|\setaddresswidth|\arg{Breite des Anschriften-Fensters}\\
    \verb|\setaddressheight|\arg{H"ohe des Anschriften-Fensters}\\
    \verb|\setaddressoffset|\arg{Abstand vom linken Fensterrand}\\
    \verb|\setbackaddressheight|\arg{H"ohe des Anschriften-Fensters}\\
  \hline
    \verb|\setreflinetop|\arg{Abstand vom oberen Rand}     \\
    \verb|\setbottomtexttop|\arg{Abstand vom oberen Rand}   \\
    \verb|\setupperfoldmarkvpos|\arg{Abstand vom oberen Rand} \\
    \verb|\setlowerfoldmarkvpos|\arg{Abstand vom oberen Rand}  \\
  \hline
\end{tabular}
\end{center}
\end{table}

\begin{table}[htp]
\caption{Zusammenfassung der Dinbrief-Befehle (Teil 2):}\index{dinbrief>Befehle}

\begin{center}
\begin{tabular}{l}
  \hline
    \verb|\setlabelwidth|\arg{Breite eines Labels}            \\
    \verb|\setlabelheight|\arg{H"ohe eines Labels}            \\
    \verb|\setlabeltopmargin|\arg{oberer Rand}                \\
    \verb|\setlabelnumber|\arg{Labelanzahl pro Spalte}        \\
    \verb|\spare|\arg{Anzahl leerer Labels}                   \\
  \hline
    \verb|\stdaddress|\arg{Adresse des Absenders}             \\
    \verb|\begin{dinquote}|                                   \\
    \verb|\end{dinquote}|                                     \\
    \verb|\enclright|                                         \\
    \verb|\postremark|\arg{Postvermerk}                       \\
    \verb|\handling|\arg{Behandlungsvermerk}                  \\
  \hline
\end{tabular}
\end{center}
\end{table}


\begin{table}[ht]
\caption{"Uberschriftvariablen und deren Inhalt}%
\index{dinbrief>"Uberschriftvariablen}

\begin{center}
(Voreinstellung entspricht DIN)

\begin{tabular}{l}
 \hline
  \verb|\ccname|\{{\tt Verteiler}\}        \\
  \verb|\enclname|\{{\tt Anlage(n)}\}      \\
  \verb|\psname|\{{\tt PS}\}               \\
 \hline
  \verb|\phonemsg|\{{\tt Telefon}\}        \\
  \verb|\signmsgold|\{{\tt Unsere Zeichen}\}  \\
  \verb|\signmsgnew|\{{\tt Unsere Zeichen, unsere Nachricht vom}\}  \\
  \verb|\yourmailmsg|\{{\tt Ihre Zeichen, Ihre Nachricht vom}\}  \\
 \hline
\end{tabular}
\end{center}

\end{table}

\ssect{Bezugszeichenzeile}\index{Bezugszeichenzeile}
Die vorliegende Version des \envname{dinbrief}s enth"alt
zwei verschiedene Formen von Bezugszeichenzeilen.
Die beiden Formen sind in der Norm 676 vom Dezember 1976 und
im Entwurf zur Norm 676 vom Mai 1991 definiert. Mit den
Befehlen \cmd\enabledraftstandard\ und \cmd\disabledraftstandard\
schaltet man auf die Form des Entwurfs oder der geltenden Norm
um. Die beiden Formen sind zur besseren Unterscheidung in
Abbildung~\ref{fig:referlines} wiedergegeben.

\begin{figure}[htb]
\begin{center}
\unitlength0.65mm
\begin{picture}(190, 60)
  \linethickness{0.4pt}
  \thinlines
  %
    \put( 10.0, 49){\parbox[t]{50.8\unitlength}{{\sf\tiny Ihre Zeichen, Ihre Nachricht vom}\\{\cmd\yourmail}}}
    \put( 60.8, 49){\parbox[t]{50.8\unitlength}{{\sf\tiny Unsere Zeichen}\\{\cmd\sign}}}
    \put(111.6, 49){\parbox[t]{25.4\unitlength}{{\sf\tiny Telefon}\\{\cmd\phone}}}
    \put(137.0, 49){\parbox[t]{25.4\unitlength}{{\sf\tiny Ortsname\\[-2ex] (Datum)}\\{\cmd\place}\\{\cmd\date}}}
    \put(  0.0, 56){\makebox(190, 0){Bezugszeichenzeile nach DIN 676 vom Dezember 1976}}
    \put(  0.0, 30){\framebox(190, 30){}}
  %
    \put( 10.0, 16){\parbox[t]{50.8\unitlength}{{\sf\tiny Ihr Zeichen, Ihre Nachricht vom}\\{\cmd\yourmail}}}
    \put( 60.8, 16){\parbox[t]{50.8\unitlength}{{\sf\tiny Unser Zeichen, unsere Nachricht vom}\\{\cmd\sign}}}
    \put(111.6, 16){\parbox[t]{50.8\unitlength}{{\sf\tiny Telefon, Bearbeiter}\\{\cmd\phone}\\{\cmd\writer}}}
    \put(162.4, 16){\parbox[t]{15.0\unitlength}{{\sf\tiny Datum}\\{\cmd\place}\\{\cmd\date}}}
    \put(  0.0, 23){\makebox(190, 0){Bezugszeichenzeile nach DIN 676 vom Mai 1991 (Entwurf)}}
    \put(  0.0,  1){\framebox(190, 27){}}
\end{picture}%
\caption{Formen von Bezugszeichenzeilen}\label{fig:referlines}
\end{center}
\end{figure}

\ssect{Standardkonfiguration mit einer Konfigurationsdatei}\index{Konfigurationsdatei}
Am Ende der Bearbeitung der Dokumentklasse \envname{dinbrief} wird die
Konfigurationsdatei \file{dinbrief.cfg} eingelesen, falls eine Datei dieses
Namens im Suchpfad f"ur \TeX-Eingabedateien gefunden wird.
Die Suche wird in der gleichen Weise durchgef"uhrt wie bei anderen
\TeX-Eingabedateien.

In der Konfigurationsdatei k"onnen alle Befehle verwendet werden, die man vor
\cmd\begin\arg{document} angeben darf.
Beispielsweise kann die Datei mit den daf"ur vorgesehenen Befehlen das Layout
des Briefes abweichend von der DIN Norm festlegen, einen Briefkopf definieren
oder das Feld f"ur Angaben bei Kapitalgesellschaften vergr"o"ern und auch
die Angaben selbst mit dem Befehl \cmd\bottomtext\ festlegen.

Die Angabe der Optionen \Dopt{addresshigh} oder \Dopt{addressstd}
"uberschreibt Angaben vertikalen Positionierung des  Feldes f"ur die Anschrift,
der Bezugszeichenzeile und der Faltmarkierungen.
Explizite Festlegungen vor dem Befehl \cmd\opening\ "uberschreiben sowohl
Einstellungen durch Angabe von Optionen in der \cmd\documentclass-Anweisung
als auch in der Konfigurationsdatei.

Beim Auspacken der \envname{dinbrief}-Verteilung wird auch eine
Konfigurationsdatei erstellt, die nur Kommentare enth"alt.
Einzelne Befehle k"onnen durch Entfernen des Kommentarzeichens aktiviert
werden.
Die verwendeten Befehle sind in dieser Dokumentation beschrieben.

\ssect{Briefkopf}\index{Briefkopf}
Bei h"aufigem Briefeschreiben kommt sicher bald der
Wunsch nach einem eigenen Briefkopf auf; auch dies ist
mit \LaTeX\ zu verwirklichen.
Am besten erstellt man eine Datei mit den Einstellungen
f"ur eigene Briefe, wie im Beispiel die Datei {\tt brfkopf.tex}.
Diese Datei kann z.B.\ den Briefkopf aus
Abb.~\ref{briefkopf} enthalten.

\renewcommand{\textfraction}{0.35}

\begin{figure}[p]
\begin{center}
\begin{verbatim}
\newlength{\UKAwd}
\newlength{\ADDRwd}
\font\fa=cmcsc10  scaled 1440
\font\fb=cmss12   scaled 1095
\font\fc=cmss10   scaled 1000
\def\briefkopf{
 \settowidth{\UKAwd}{\fa Institut f"ur Verpackungen}
 \settowidth{\ADDRwd}{\fc EARN/BITNET: yx99 at dkauni2}
 \vspace*{7truemm}
 \raisebox{-11.3mm}{%
   \setlength{\unitlength}{1truemm}
   \begin{picture}(15,15)(0,0)
     \thicklines
     \put(7.5,7.5){\circle{15}}
     \put(7.5,7.5){\circle{10}}
     \put(7.5,7.5){\circle{ 5}}
   \end{picture}%
 }
 {\fc\hspace{.7em}}
 \parbox[t]{\UKAwd}{
        \centering{\fa Universit\"at Gralsruhe} \\
        \centering{\fa Institut f"ur Verpackungen} \\[.5ex]
        \centering{\fb Prof.\ Dr.\ Fritz Schreiber}
        }
 \hfill
 \parbox[t]{\ADDRwd}{
        \fc Engesserstr.\ 9 $\cdot$ Postfach 6980 \\
        \fc 76128 Karlsruhe\\
        \fc Telefon: (0721) 608-9790 \\
        }  }
\signature{Prof.\ Dr.\ Fritz Schreiber}
\place{Karlsruhe}
\address{\briefkopf}
\phone{(0721)}{608-9790}
\def\FS{Prof.\,F.\,Schreiber, Univ.\,Karlsruhe,
        Postf.\,6980, 76128\,Karlsruhe\rule[-1ex]{0pt}{0pt}}
\end{verbatim}
\caption{Definition eines Briefkopfs}\label{briefkopf}
\end{center}
\end{figure}

Am Anfang des Briefes sollte nun der Befehl \verb+%%
%% This is file `brfkopf.tex',
%% generated with the docstrip utility.
%%
%% The original source files were:
%%
%% dinbrief.dtx  (with options: `brfkopf')
%% 
%% =======================================================================
%% 
%% Copyright (C) 1993, 96, 97 by University of Karlsruhe (Computing Center).
%% Copyright (C) 1998, 2000   by University of Karlsruhe (Computing Center)
%%                            and Richard Gussmann.
%% All rights reserved.
%% For additional copyright information see further down in this file.
%% 
%% This file is part of the DINBRIEF package
%% -----------------------------------------------------------------------
%% 
%% It may be distributed under the terms of the LaTeX Project Public
%% License (LPPL), as described in lppl.txt in the base LaTeX distribution.
%% Either version 1.1 or, at your option, any later version.
%% 
%% The latest version of this license is in
%% 
%%         http://www.latex-project.org/lppl.txt
%% 
%% LPPL Version 1.1 or later is part of all distributions of LaTeX
%% version 1999/06/01 or later.
%% 
%% 
%% For error reports in case of UNCHANGED versions see readme files.
%% 
%% Please do not request updates from us directly.  Distribution is
%% done through Mail-Servers, TeX organizations and others.
%% 
%% If you receive only some of these files from someone, complain!
%% 
%%
%% \CharacterTable
%%  {Upper-case    \A\B\C\D\E\F\G\H\I\J\K\L\M\N\O\P\Q\R\S\T\U\V\W\X\Y\Z
%%   Lower-case    \a\b\c\d\e\f\g\h\i\j\k\l\m\n\o\p\q\r\s\t\u\v\w\x\y\z
%%   Digits        \0\1\2\3\4\5\6\7\8\9
%%   Exclamation   \!     Double quote  \"     Hash (number) \#
%%   Dollar        \$     Percent       \%     Ampersand     \&
%%   Acute accent  \'     Left paren    \(     Right paren   \)
%%   Asterisk      \*     Plus          \+     Comma         \,
%%   Minus         \-     Point         \.     Solidus       \/
%%   Colon         \:     Semicolon     \;     Less than     \<
%%   Equals        \=     Greater than  \>     Question mark \?
%%   Commercial at \@     Left bracket  \[     Backslash     \\
%%   Right bracket \]     Circumflex    \^     Underscore    \_
%%   Grave accent  \`     Left brace    \{     Vertical bar  \|
%%   Right brace   \}     Tilde         \~}
%%
\newlength{\UKAwd}
\newlength{\ADDRwd}
\font\fa=cmcsc10  scaled 1440
\font\fb=cmss12   scaled 1095
\font\fc=cmss10   scaled 1000
\def\briefkopf{
 \settowidth{\UKAwd}{\fa Institut f"ur Verpackungen}
 \settowidth{\ADDRwd}{\fc EARN/BITNET: yx99 at dkauni2}
 \expandafter\ifx\csname fontsize\endcsname\relax\else
   \fontsize{12}{14.4pt}\selectfont
 \fi
 \vspace*{7truemm}
 \raisebox{-11.3mm}{%
   \setlength{\unitlength}{1truemm}
   \begin{picture}(15,15)(0,0)
     \thicklines
     \put(7.5,7.5){\circle{15}}
     \put(7.5,7.5){\circle{10}}
     \put(7.5,7.5){\circle{ 5}}
   \end{picture}%
 }
 {\fc\hspace{.2em}}
 \parbox[t]{\UKAwd}{\centering{\fa Universit\"at Gralsruhe} \\
                    \centering{\fa Institut f"ur Verpackungen} \\[.5ex]
                    \centering{\fb Prof.\ Dr.\ Fritz Schreiber} }
 \hfill
 \parbox[t]{\ADDRwd}{\fc Im Hinterhof 2 $\cdot$ Postfach 8960 \\
                     \fc D--76821 Gralsruhe \\
                     \fc Telefon: (0127) 806-0815 \\
                     \fc Electronic Mail: \\
                     \fc EARN/BITNET: yx99 at error2 }
 }
\signature{Prof.\ Dr.\ Fritz Schreiber}
\place{Gralsruhe}
\address{\briefkopf}
\phone{(0127)}{806-0815}
\def\FS{Prof.\,F.\,Schreiber, Uni.\,Gralsruhe,
        Postf.\,8960, 76821\,Gralsruhe\rule[-1ex]{0pt}{0pt}}

\endinput
%%
%% End of file `brfkopf.tex'.
+
aufgenommen werden gefolgt von \verb+\address{\myaddress}+%
\index{myaddress=\verb+\myaddress+}. Dies sorgt
f"ur die gew"unschte Ausgabe des Briefkopfes am Beginn des Briefes.
Nat"urlich lassen sich auch andere als die hier verwendeten Schriftarten
verwenden.

\ssect{Kopfzeilen}\index{Kopfzeilen}

Es stehen verschiedene Kopfzeilen zur Verf"ugung die "uber die
Option \linebreak[4]\verb+\pagestyle{...}+\index{pagestyle=\verb+\pagestyle+}
ausgew"ahlt werden k"onnen.
Bei \verb+plain+\index{plain=\verb+plain+} wird eine Seitennumerierung
bei mehrseitigen Briefen in der Fu"szeile eingeblendet, die Kopfzeile
bleibt leer. Durch \verb+headings+\index{headings=\verb+headings+} wird
die Kopfzeile mit einer Anrede und der Seitenzahl bei mehrseitigen
Briefen gesetzt.

\ssect{Einblenden von Unterschriften}
\cmd\closing\oarg{Unterschrift}\arg{Gru"sformel} erlaubt es,
die Unterschrift als Graphik einzubinden. F"ur unser Beispiel
nehmen wir an, da"s die Graphik als encapuslated Postscript-Datei
im Verzeichnis und mit Namen \file{graph/sig.eps} vorliegt.
Die Befehle zum Einbinden von Graphiken stellt das Paket
{\sc graphics} zur Verf�gung. In der Pr"aambel des Briefes wird
ein Befehl zum Setzen der Unterschrift definiert:\\[\medskipamount]
\hspace*{2em}\verb|\newcommand{\setsignature}{\includegraphics{graph/sig.eps}}|\\[\medskipamount]
Die Gru"sformel wird dann in der erweiterten Form gesetzt:\\[\medskipamount]
\hspace*{2em}\verb|\closing[\protect\setsignature]{Mit freundlichen Gr"u"sen}|\\[\medskipamount]
Gegebenenfalls ist es notwendig, die Unterschrift mit dem
Befehl \verb+\raisebox+ in die notwendige Position zu schieben.
Dazu ist die Definition des Macros zum Einbinden der Unterschrift
wie folgt anzupassen:\\[\medskipamount]
\hspace*{2em}\verb|\newcommand{\setsignature}{\raisebox{-3mm}{\includegraphics{graph/sig.eps}}}|\\[\medskipamount]
Die L"angenangabe (hier -3mm) ist entsprechend anzupassen.

\ssect{Briefe in englischer oder franz"osischer Sprache}

Wer Briefe in anderen Sprachen schreiben m"ochte, kann f"ur Englisch und
Franz"osisch die Trennung (abh"angig von der Installation) und Befehle f"ur
Buchstaben mit Akzenten mit dem Befehl
\cmd\selectlanguage\arg{Sprache}
\index{selectlanguage=\verb+\selectlanguage+} umschalten.
Das Umsetzen von Bezeichnungen z.B.\ f"ur Anlage \dots{} mu"s explizit durch
Befehle erfolgen, die in der Dokumentation zum {\tt dinbrief} beschrieben sind.

\ssect{Serienbriefe}\index{Serienbriefe}

Mit \LaTeX\ lassen sich auch Serienbriefe schreiben. Man ben"otigt dazu
nur ein kleines Makro wie z.B. das folgende:

\begin{center}
\begin{minipage}{0.75\textwidth}
\begin{verbatim}
\def\mailto#1{
  \begin{letter}{#1}
  \input{brftext}
  \end{letter}}
\end{verbatim}
\end{minipage}
\end{center}

Mit dem Befehl \verb+\input{brftext}+ wird die Datei geladen, die den Text f"ur
den Serienbrief enth"alt. In einer weiteren Datei stehen unsere Adressaten im
folgenden Format:

\begin{center}
\begin{minipage}{0.75\textwidth}
\begin{verbatim}
\mailto{Karle Huber\\
        Lichtensteinstr. 45\\[\medskipamount]
        77777 Hintertupfingen}
\mailto{Anna H"aberle\\
        Wallstra"se 7\\[\medskipamount]
        88888 L"andle}
\end{verbatim}
\end{minipage}
\end{center}

Die Briefe k"onnen nun mit einer Umgebung wie der in Abbildung~\ref{serie}
ausgedruckt werden. In der Zeile \verb|\input{#address}| ist der
Platzhalter \verb|#address| durch den Dateinamen zu ersetzen.

\begin{figure}[p]
\begin{center}
\begin{verbatim}
\documentclass[12pt]{dinbrief}
\usepackage{german}

%%
%% This is file `brfkopf.tex',
%% generated with the docstrip utility.
%%
%% The original source files were:
%%
%% dinbrief.dtx  (with options: `brfkopf')
%% 
%% =======================================================================
%% 
%% Copyright (C) 1993, 96, 97 by University of Karlsruhe (Computing Center).
%% Copyright (C) 1998, 2000   by University of Karlsruhe (Computing Center)
%%                            and Richard Gussmann.
%% All rights reserved.
%% For additional copyright information see further down in this file.
%% 
%% This file is part of the DINBRIEF package
%% -----------------------------------------------------------------------
%% 
%% It may be distributed under the terms of the LaTeX Project Public
%% License (LPPL), as described in lppl.txt in the base LaTeX distribution.
%% Either version 1.1 or, at your option, any later version.
%% 
%% The latest version of this license is in
%% 
%%         http://www.latex-project.org/lppl.txt
%% 
%% LPPL Version 1.1 or later is part of all distributions of LaTeX
%% version 1999/06/01 or later.
%% 
%% 
%% For error reports in case of UNCHANGED versions see readme files.
%% 
%% Please do not request updates from us directly.  Distribution is
%% done through Mail-Servers, TeX organizations and others.
%% 
%% If you receive only some of these files from someone, complain!
%% 
%%
%% \CharacterTable
%%  {Upper-case    \A\B\C\D\E\F\G\H\I\J\K\L\M\N\O\P\Q\R\S\T\U\V\W\X\Y\Z
%%   Lower-case    \a\b\c\d\e\f\g\h\i\j\k\l\m\n\o\p\q\r\s\t\u\v\w\x\y\z
%%   Digits        \0\1\2\3\4\5\6\7\8\9
%%   Exclamation   \!     Double quote  \"     Hash (number) \#
%%   Dollar        \$     Percent       \%     Ampersand     \&
%%   Acute accent  \'     Left paren    \(     Right paren   \)
%%   Asterisk      \*     Plus          \+     Comma         \,
%%   Minus         \-     Point         \.     Solidus       \/
%%   Colon         \:     Semicolon     \;     Less than     \<
%%   Equals        \=     Greater than  \>     Question mark \?
%%   Commercial at \@     Left bracket  \[     Backslash     \\
%%   Right bracket \]     Circumflex    \^     Underscore    \_
%%   Grave accent  \`     Left brace    \{     Vertical bar  \|
%%   Right brace   \}     Tilde         \~}
%%
\newlength{\UKAwd}
\newlength{\ADDRwd}
\font\fa=cmcsc10  scaled 1440
\font\fb=cmss12   scaled 1095
\font\fc=cmss10   scaled 1000
\def\briefkopf{
 \settowidth{\UKAwd}{\fa Institut f"ur Verpackungen}
 \settowidth{\ADDRwd}{\fc EARN/BITNET: yx99 at dkauni2}
 \expandafter\ifx\csname fontsize\endcsname\relax\else
   \fontsize{12}{14.4pt}\selectfont
 \fi
 \vspace*{7truemm}
 \raisebox{-11.3mm}{%
   \setlength{\unitlength}{1truemm}
   \begin{picture}(15,15)(0,0)
     \thicklines
     \put(7.5,7.5){\circle{15}}
     \put(7.5,7.5){\circle{10}}
     \put(7.5,7.5){\circle{ 5}}
   \end{picture}%
 }
 {\fc\hspace{.2em}}
 \parbox[t]{\UKAwd}{\centering{\fa Universit\"at Gralsruhe} \\
                    \centering{\fa Institut f"ur Verpackungen} \\[.5ex]
                    \centering{\fb Prof.\ Dr.\ Fritz Schreiber} }
 \hfill
 \parbox[t]{\ADDRwd}{\fc Im Hinterhof 2 $\cdot$ Postfach 8960 \\
                     \fc D--76821 Gralsruhe \\
                     \fc Telefon: (0127) 806-0815 \\
                     \fc Electronic Mail: \\
                     \fc EARN/BITNET: yx99 at error2 }
 }
\signature{Prof.\ Dr.\ Fritz Schreiber}
\place{Gralsruhe}
\address{\briefkopf}
\phone{(0127)}{806-0815}
\def\FS{Prof.\,F.\,Schreiber, Uni.\,Gralsruhe,
        Postf.\,8960, 76821\,Gralsruhe\rule[-1ex]{0pt}{0pt}}

\endinput
%%
%% End of file `brfkopf.tex'.

\address{\myaddress}
\backaddress{R. Kurz, Am See 1, 76139 Karlsruhe}

\signature{R. Kurz}
\place{76139 Karlsruhe}

\def\mailto#1{                % zum ausdrucken von
                              % Serienbriefen
  \begin{letter}{#1}
  \input{brftext}             % Datei, die den Text enthaelt
  \end{letter}}

\begin{document}

\input{#address}              % Adress-Datei

\end{document}
\end{verbatim}
\caption{Erstellen von Serienbriefen}
\label{serie}
\end{center}
\end{figure}

Beachten sollte man, da"s dann der Text in der Datei {\tt brftext.tex}
direkt mit
\verb+\opening{...}+ beginnt (also kein \verb+\begin{letter}+ und
\verb+\end{letter}+ enth"alt) und mit \verb+\closing{...}+ bzw.
\verb+\ps{...}+ abschlie"st.

Ein Beispiel f"ur die Datei {\tt brftext.tex} finden Sie in
Abbildung~\ref{brftext}.

\begin{figure}[p]
\begin{center}
\begin{verbatim}
\opening{Betrieb eines Mikrowellensenders}

Sehr geehrte Damen und Herren,

anbei sende ich Ihnen eine Kopie der bisherigen Genehmigung f"ur
unseren Mikrowellenherd...

... Ihre Bem"uhungen im voraus vielen Dank.

\closing{Mit freundlichen Gr"u"sen,}

\ps{Wir bitten um schnelle Erledigung.}
\cc{Deutsche Bundespost\\
    Karlsruher Privatfunk\\
    S"uddeutscher Rundfunk}

\encl{Abschrift der Urkunde}
\end{verbatim}
\caption{Rumpf eines Serienbriefes}
\label{brftext}
\end{center}
\end{figure}

\ssect{Einige Regeln f"ur das Briefeschreiben}

Dieser Abschnitt enth"alt Passagen aus den
Normen DIN~5008 (Regeln f"ur das Maschinenschreiben) und
DIN~676 (Gesch"aftsbrief), erg"anzt um einige
zus"atzliche Hinweise und Tips.
Der Abschnitt erhebt keinen Anspruch auf
Vollst"andigkeit. Er soll Anf"angern wie auch
Ge"ubten einen "Uberblick "uber die wichtigsten
Regeln geben. Ferner werden die Grenzen der
vorliegenden Version aufgezeigt und es wird
auf bekannte Fehler hingewiesen.

\begin{enumerate}
  \item {\bf Zeilenabstand}

        Es wird mit einfachem Grundzeilenabstand geschrieben.

  \item {\bf Anschriftenfeld}

        Die Angaben im Anschriftenfeld werden auf
        folgende Weise gegliedert:
        \begin{enumerate}
          \item Sendungsart, Versendungsform, Vorausverf"ugung
          \item Leerzeile
          \item Empf"angerbezeichnung
          \item Postfach oder Stra"se und Hausnummer
          \item Leerzeile
          \item Postleitzahl und Bestimmungsort
          \item Leerzeile
          \item Bestimmungsland
        \end{enumerate}

        Bei Auslandsanschriften ist die Leerzeile zwischen
        der Zeile mit Postfach oder Stra"se und Hausnummer
        und der Zeile mit dem Bestimmungsort entbehrlich,
        wenn das Bestimmungsland unter der entsprechenden
        Zeile angegeben werden mu"s.

        Im Verkehr mit bestimmten L"andern kann auf die
        Angabe des Bestimmungslandes verzichtet werden,
        wenn das Unterscheidungskennzeichen f"ur den
        grenz"uberschreitenden Kraftfahrzeugverkehr
        der Postleitzahl --- durch einen Bindestrich
        getrennt --- vorangestellt wird.

        Nach dem ersten Eintrag im Anschriftenfeld
        darf nur ein \verb|\\| stehen. Direkte
        L"angenangaben (z.B.\ \verb|\\[\medskipamount]|)
        sind nicht zul"assig und verursachen einen Fehler.
        Der Fehler kann umgangen werden, indem eine
        Konstruktion
        \m{Versendungsform}\verb|\\~\\|%
        \m{Empf"angerbezeichnung} usw.\
        verwendet wird.

  \item {\bf Bezugszeichen und Tagangabe}

        Die Eintragungen in dieser Zeile werden automatisch
        an der richtigen Stelle plaziert.

  \item {\bf Betreff und Teilbetreff}

        Betreff und Teilbetreff sind stichwortartige
        Inhaltsangaben. Der Betreff bezieht sich auf den
        ganzen Brief, Teilbetreffe beziehen sich auf
        Briefteile.

        Der {\em Wortlauf des Betreffs\/} wird ohne
        Schlu"spunkt geschrieben.
        Der {\em Teilbetreff\/} beginnt an der Fluchtlinie
        (linker Rand), schlie"st mit einem Punkt und wird
        unterstrichen. Wir empfehlen diesen Text besser
        durch eine andere Schriftart (z.B.\ fett)
        hervorzuheben. Der Text wird unmittelbar angef"ugt.
  \item {\bf Behandlungsvermerke}

        Behandlungsvermerke (z.B.\ eilt) werden
        neben das Anschriftenfeld, beginnend auf
        Grad 50 (bei einer 10er Teilung), oder im
        Anschlu"s an die Betreffangabe geschrieben;
        sie k"onnen hervorgehoben werden.

  \item {\bf Anlagen- und Verteilvermerke}

        Anlagen- und Verteilvermerke beginnen an der
        Fluchtlinie oder auf Grad 50 (60 oder 75).
        Die vorliegende Version des `dinbriefs'
        unterst"utzt nur Anlagen- und Verteilvermerke
        auf der Fluchtlinie.

        Der Anlagenvermerk geht dem Verteilvermerk
        voraus.

  \item {\bf Postscriptum}

        Die DIN Norm 5008 sieht kein Postscriptum vor.
        Die vorliegende Version des `dinbriefs'
        unterst"utzt trotzdem ein Postscriptum.
        Wir empfehlen das Postscriptum unmittelbar
        nach der Gru"sformel \verb|\closing| oder
        nach Anlagen- und Verteilvermerken zu setzen.

  \item {\bf Seitennumerierung}

        Die Seiten eines Schriftst"ucks sind von der
        2.~Seite an oben fortlaufend zu benummern.
        Die Pagestyles \verb|headings| und
        \verb|contheadings| unterst"utzen diese
        Forderung. Das Verfahren ist jedoch noch nicht
        befriedigend.

  \item {\bf Hinweis auf Folgeseiten}

        Wenn der Text eines Schriftst"ucks eine n"achste
        Seite beansprucht, wird empfohlen
        \begin{itemize}
          \item am Fu"s der bereits beschrifteten Seite,
          \item nach der letzten Textzeile,
          \item mit mindestens einer Leerzeile Abstand,
          \item auf Grad 60 (72 oder 90) beginnend,
        \end{itemize}
        als Hinweis auf die folgende Seite drei Punkte
        zu schreiben.

        Dieses Vorgehen wird zur Zeit nicht unterst"utzt.
        Der Pagestyle \verb|contheadings| schreibt jedoch
        an das Ende der laufenden Seite die Seitenzahl
        der Folgeseite und auf Folgeseiten die
        aktuelle Seite in der Kopf der Seite.

        Die Kombination des Befehls \verb|\bottomtext|,
        zum Einblenden einer weiteren Kommunikationszeile
        am unteren Blattende der ersten Seite sowie
        von gesellschaftsrechtlichen Angaben,
        mit Seitenstilen, die die Fu"szeile unten mit
        der Seitennummer oder Folgeseitennummer
        beschriften, hat unter Umst"anden zur Folge,
        da"s die Seiten- oder Folgeseitennummer
        von diesen Feldern "uberschrieben wird.

  \item {\bf Kommunikationszeile am Blattende und
        gesellschaftsrechtliche Angaben}

        Eine Kommunikationszeile am Blattende kann die folgenden
        Angaben enthalten: Gesch"aftsr"aume, Telefon, Telefax,
        Teletex, Telex, Btx und Kontoverbindungen.

        Bei Kapitalgesellschaften sind die Angaben "uber
        \begin{itemize}
          \item die Rechtsform und den Sitz der Gesellschaft,
          \item das Registergericht des Sitzes der Gesellschaft
                und die Nummer, unter der die Gesellschaft in
                das Handelsregister eingetragen ist,
          \item den Namen des Vorsitzenden des Aufsichtsrates
                (sofern die Gesellschaft nach gesetzlicher
                Vorschrift einen Aufsichtsrat zu bilden hat),
          \item die Namen des Vorsitzenden und aller Mitglieder
                des Vorstandes (bei Gesellschaften mit
                beschr"ankter Haftung die Namen der
                Gesch"afts\-f"uhrer),
        \end{itemize}
        am Fu"s des Vordrucks aufzuf"uhren.\hfil\break
        Die Rechtsform kann auch im Briefkopf als Bestandteil
        der Firma angegeben werden.
\end{enumerate}

\ssect{Adre"s--Etiketten}
Das Ausdrucken von Adre"s-Etiketten ist w"unschenswert, wenn
keine Fensterbriefh"ullen verwendet werden. Damit das Bedrucken
von verschiedenen Etiketts"atzen m"oglich wird, kann die Breite
und H"ohe der einzelnen Etiketten mit den Befehlen
\verb|\setlabelwidth|\arg{Breite} und
\verb|\setlabelheight|\arg{H"ohe} eingestellt
werden. Der obere Rand kann mit dem Befehl
\verb|\setlabeltopmargin|\arg{oberer Rand} festgelegt werden.
Die Anzahl der Labels in einer Spalte wird durch den Befehl
\verb|\setlabelnumber|\arg{Anzahl} angegeben. Die mehrfache
Verwendung eines Etikettenblatts wird durch den Befehl
\verb|\spare|\arg{Anzahl} m"oglich. Dieser Befehl r"uckt
den Druckbeginn um {\em Anzahl\/} Positionen vor.
Die Ausgabe der Adre"s--Etiketten erfolgt spaltenweise.

In vielen Druckern bleiben die Etikettenbl"atter h"angen,
wenn die Tr"agerfolie in beiden Spalten freiliegt. Drucker,
die alternativ einen ebenen Papiertransport besitzen, sollten
zur Ausgabe von Etiketten auf diesen umgeschaltet werden.

\ssect{Kompatibilit"at zu Rainer Sengerlings `dinbrief'}
Im Jahr 1992 wurde an der Universit"at Karlsruhe ein Briefstil
entwickelt, dem der Namen `dinbrief' gegeben wurde. Im gleichen
Jahr ver"offentlichte Rainer Sengerling einen Briefstil unter
dem gleichen Namen.

Rainer Sengerling hat darauf verzichtet seinen Briefstil
an \LaTeXe\ anzupassen. Daraufhin haben wir den Briefstil
`dinbrief' im Dezember 1994 als \LaTeXe-Klasse ver"offentlicht.
Die beiden Briefstile haben unterschiedliche Befehlss"atze,
was bei vielen Benutzern zu Verwirrung und Irritationen
gef"uhrt hat. Wegen der unterschiedlichen Befehle der beiden
Stile haben uns viele Anfragen erreicht. Die jetzt vorliegende
Fassung stellt beide Befehlss"atze zur Verf"ugung und f"uhrt die
bisher unabh"anigen Briefstile zusammen.

Gegenw"artig wird die Klassenoption \verb|german| nicht ausgewertet
und f"uhrt zu einer Warnung. Alternativ mu"s mit dem Befehl
\begin{quote}
  \verb|\usepackage{german}|\\
\end{quote}
der Stil "`{\tt german}"' geladen werden.

\begin{description}

  \item[\cmd\Retouradresse\arg{{Absenderadresse im Brief\kern 0pt fenster}}]
        Der Befehl legt die Anschrift des Absenders oben im
        Anschriftenfeld des Briefs fest.

  \item[\cmd\Retourlabel] \hfil\break
        Dieser Befehl erzeugt Absenderadre"s-Etiketten, falls
        die Erzeugung der Etiketten aktiviert wurde.

        Ferner wird die Ausgabe der \cmd\Retouradresse\ unterdr"uckt.

        {\bf Dieser Befehl wird gegenw"artig nicht unterst"utzt und
        erzeugt eine Warnung.\/}

  \item[\cmd\Fenster] \hfil\break
        Bei der Verwendung von Fensterbriefh"ullen wird die mit
        \cmd\Retouradresse\ vereinbarte einzeilige R"ucksendeadresse
        (Absenderadresse) oben im Anschriftenfeld eingeblendet,
        falls der Befehl \cmd\Fenster\ angegeben wurde. Ferner
        werden die Faltmarken am linken Rand ausgegeben.

        Daf"ur wird die Ausgabe der Etiketten unterdr"uckt.

        {\bf Dieser Befehl wird gegenw"artig nicht unterst"utzt und
        erzeugt eine Warnung.\/}

  \item[\cmd\Postvermerk\arg{Postvermerk}] \hfil\break
        Dieser Befehl vereinbart Vermerke f"ur den
        Postversand wie z.B.\ Einschreiben.

  \item[\cmd\Behandlungsvermerk\arg{Behandlungsvermerk}] \hfil\break
        Der Befehl \cmd\Behandlungsvermerk\ dient zur
        Angabe von (man wirds kaum glauben) Behandlungsvermerken
        wie z.B.\ Eilt, pers"onlich oder "`F"ur die unterste
        Schublade"'.

  \item[\cmd\Absender\arg{Name und Adresse des Absenders}] \hfil\break
        Die Adresse des Absenders wird vereinbart.
        Dieser Befehl gilt f"ur den laufenden und alle
        weiteren Briefe; er gilt solange, bis ein
        weiterer \cmd\Absender- oder \cmd\address-Befehl
        angegeben wird.

  \item[\cmd\Absender\arg{Teil 1::Teil 2}] \hfil\break
        Dies ist eine Sonderform des \cmd\Absender-Befehls.
        Der durch \verb|::| abgetrennte {\em Teil2\/}
        erscheint im Briefkopf, aber nicht im
        Absenderetikett, falls ein solches mit dem
        Befehl \cmd\Retourlabel\ angefordert
        wurde.

        Dieser Befehl wird korrekt unterst"utzt. Da aber
        der Befehl \cmd\Retourlabel\ nicht implementiert
        wurde, hat die Aufteilung im Moment praktisch keine
        Bedeutung; sie kann aber in sp"ateren Versionen
        wieder notwendig sein.

  \item[\cmd\Datum\arg{Briefdatum}] \hfil\break
        Soll als Absendedatum {\sl nicht\/} das aktuelle
        Tagesdatum (des Rechners) eingesetzt werden, kann
        mit diesem Befehl das Datum explizit angegeben werden.

        Der Befehl \cmd\heute\ wird nicht unterst"utzt.

  \item[\cmd\Betreff\arg{{Betreff}}] \hfil\break
        Mit diesem Befehl wird der Betreff gesetzt, der den
        Empf"anger "uber den Gegenstand des Briefes
        informiert. Der Betreff kann ein oder mehrzeilig sein.

  \item[\cmd\Einrueckung\arg{Text}] \hfil\break
        Der Text wird um $1 in$ einger"uckt und geht bis zum
        rechten Rand. Der Text darf Abs"atze enthalten.

  \item[\cmd\anlagenrechts] \hfil\break
        Die Anlagen- und Verteilvermerke beginnen {\sl rechts\/}
        neben der Gru"sformel auf Grad 50. Mit diesem Befehl
        kann Platz gespart werden, falls die Seite fast voll
        ist, und kein neues Blatt begonnen werden soll.
        Dieser Befehl mu"s vor \cmd\closing\ stehen.
        Die Anlagen- und Verteilvermerke m"ussen in diesem
        Fall ebenfalls vor dem \cmd\closing-Befehl stehen.

  \item[\cmd\Anlagen\arg{Anlagenvermerk}] \hfil\break
        Mit diesem Befehl werden eventuelle Anlagenvermerke
        vereinbart oder gesetzt. Dieser Befehl darf sowohl
        vor dem \cmd\closing-Befehl als auch hinter diesem
        stehen. Der Anlagenvermerk wird nur vereinbart falls
        der Befehl vor dem \cmd\closing-Befehl angegeben wird.
        Der Anlagenvermerk wird sofort ausgegeben und gesetzt,
        falls der Befehl nach dem \cmd\closing-Befehl angegeben
        wird.

  \item[\cmd\Verteiler\arg{Verteilvermerk}] \hfil\break
        Dieser Befehl vereinbart oder setzt Verteilvermerke.
        Die Ausf"uhrungen zum \cmd\Anlagen-Befehl gelten
        auch f"ur diesen Befehl.

  \item[\cmd\Etihoehe\arg{H"ohe eines Adressetiketts}] \hfil\break
        Verschiedene Herstelle vertreiben Adressetiketten,
        die alle ein bischen anders sind. Um den Briefstil
        an unterschiedlich hohe Adre"setiketten anzupassen,
        kann mit dem Befehl \cmd\Etihoehe\ die H"ohe eines
        Etiketts angegeben werden.

  \item[\cmd\Etirand\arg{oberer Rand}] \hfil\break
        Die B"ogen mit den Adre"setiketten haben oft einen
        Versatz oder oberen Rand. Mit dem Befehl \cmd\Etirand\
        kann dieser Versatz eingestellt werden.

  \item[\cmd\Etizahl\arg{Zahl der Etiketten}] \hfil\break
        Auf ein Blatt mit Adre"setiketten passen leider
        nicht beliebig viele sondern nur eine begrenzte
        Anzahl von Etiketten, die sich dann auch nicht kunterbunt
        auf dem Blatt tummeln, sondern Spaltenweise angeordnet
        sind. Die Zahl der Etiketten, die in eine Spalte
        passen werden durch den Befehl \cmd\Etizahl\
        angegeben.

  \item[\cmd\spare\arg{{n}}] \hfil\break
        Das Bedrucken der Adre"setiketten beginnt nach {\em n\/}
        leeren Adre"slabeln. Die Label werden spaltenweise
        durchgez"ahlt.

\end{description}

\clearpage
\end{document}
\endinput
%%
%% End of file `dinbrief.tex'.
