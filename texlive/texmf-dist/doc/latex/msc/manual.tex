% file: manual.tex
% Copyright 2008 V. Bos, T. van Deursen, and S. Mauw
% This file is part of the MSC Macro Package.
%
\documentclass[12pt,a4paper]{article}
\usepackage{a4wide}
\usepackage{url}
\usepackage{moreverb}
\usepackage{multicol}
\usepackage{msc}

% we allow a ragged right
\setlength{\rightskip}{0pt plus 0.05\linewidth minus 0pt}





\newlength{\rpwidth}
\setlength{\rpwidth}{.5cm}
\newlength{\rpheight}
\setlength{\rpheight}{0.5\levelheight}
\newcommand{\rpN}{%
  \psframe(-0.5\rpwidth,-\rpheight)(0.5\rpwidth,0\rpheight)%
  \rput[B](0\rpwidth,-0.8\rpheight){\tiny \textsc{n}}%
  \pscircle[fillstyle=solid,fillcolor=black](0\rpwidth,0\rpheight){.5\labeldist}%
}
\newcommand{\rpNE}{%
  \psframe(-\rpwidth,-\rpheight)(0\rpwidth,0\rpheight)%
  \rput[B](-.5\rpwidth,-0.8\rpheight){\tiny \textsc{ne}}%
  \pscircle[fillstyle=solid,fillcolor=black](0\rpwidth,0\rpheight){.5\labeldist}%
}
\newcommand{\rpE}{%
  \psframe(-\rpwidth,-.5\rpheight)(0\rpwidth,.5\rpheight)%
  \rput[B](-.5\rpwidth,-0.3\rpheight){\tiny \textsc{e}}%
  \pscircle[fillstyle=solid,fillcolor=black](0\rpwidth,0\rpheight){.5\labeldist}%
}
\newcommand{\rpSE}{%
  \psframe(-\rpwidth,0\rpheight)(0\rpwidth,\rpheight)%
  \rput[B](-.5\rpwidth,0.2\rpheight){\tiny \textsc{se}}%
  \pscircle[fillstyle=solid,fillcolor=black](0\rpwidth,0\rpheight){.5\labeldist}%
}
\newcommand{\rpS}{%
  \psframe(-.5\rpwidth,\rpheight)(.5\rpwidth,0\rpheight)%
  \rput[t](0\rpwidth,0.8\rpheight){\tiny \textsc{s}}%
  \pscircle[fillstyle=solid,fillcolor=black](0\rpwidth,0\rpheight){.5\labeldist}%
}
\newcommand{\rpSW}{%
  \psframe(0\rpwidth,0\rpheight)(\rpwidth,\rpheight)%
  \rput[B](.5\rpwidth,0.2\rpheight){\tiny \textsc{sw}}%
  \pscircle[fillstyle=solid,fillcolor=black](0\rpwidth,0\rpheight){.5\labeldist}%
}
\newcommand{\rpW}{%
  \psframe(0\rpwidth,-.5\rpheight)(\rpwidth,.5\rpheight)%
  \rput[B](.5\rpwidth,-0.3\rpheight){\tiny \textsc{w}}%
  \pscircle[fillstyle=solid,fillcolor=black](0\rpwidth,0\rpheight){.5\labeldist}%
}
\newcommand{\rpNW}{%
  \psframe(0\rpwidth,-\rpheight)(\rpwidth,0\rpheight)%
  \rput[B](.5\rpwidth,-0.8\rpheight){\tiny \textsc{nw}}%
  \pscircle[fillstyle=solid,fillcolor=black](0\rpwidth,0\rpheight){.5\labeldist}%
}





% The following code is taken from the doc package. It defines a global 
% macro \bslash that produces a bslash (if present in the current font). 
\makeatletter
{\catcode`\|=\z@ \catcode`\\=12 |gdef|bslash{\}}
\makeatother
\newcommand{\cmd}[1]{\texttt{\bslash #1}}

\newcommand{\acro}[1]{{#1}}

\newcommand{\MSC}{\acro{MSC}}
\newcommand{\HMSC}{\acro{HMSC}}
\newcommand{\MSCdoc}{\MSC{}doc}
\newcommand{\mscpack}{\MSC{} macro package}

\newcommand{\env}[1]{\texttt{#1}}
\newcommand{\opt}[1]{[#1]}
\newcommand{\cmdarg}[1]{\{\emph{#1}\}}
\newcommand{\coordarg}[1]{\emph{#1}}
\newcommand{\coordargs}[2]{(\coordarg{#1},\coordarg{#2})}
\newcommand{\lnsvalue}[3]{large/normal/small value #1/#2/#3}

\newenvironment{defs}{%
  \begin{list}{}%
              {\setlength{\labelwidth}{0pt}%
               \setlength{\labelsep}{1em}%
               \setlength{\leftmargin}{1em}%
               \setlength{\parsep}{1ex}%
               \setlength{\listparindent}{0pt}%
               \setlength{\rightmargin}{0pt}%
               \renewcommand{\makelabel}[1]{##1}%
               \raggedright%
              }%
  }{%
  \end{list}}



\title{
  A \LaTeX\ macro package for Message Sequence Charts\\{\large User Manual}
}

\author{
 \begin{tabular}{c}
  \begin{tabular}{ccc}
   Victor Bos &
   Ton van Deursen &
   Sjouke Mauw \\
   &
   \scriptsize Universit\'e du Luxembourg &
   \scriptsize Universit\'e du Luxembourg \\[-0.8ex]
   \scriptsize \texttt{victor.bos@ssf.fi} &
   \scriptsize \texttt{ton.vandeursen@uni.lu} & 
   \scriptsize \texttt{sjouke.mauw@uni.lu}
  \end{tabular}\\
 \end{tabular}
}

\date{\small Version \mscversion, last update \today\\
      Describing \mscpack{} version \mscversion}

\begin{document}


\maketitle

\begin{abstract}
\noindent
The \mscpack{} facilitates the \LaTeX\ user to easily include
Message Sequence Charts in his texts. This document describes the
design and use of the \mscpack.
\end{abstract}

\tableofcontents
\section{New}
\label{new}

\paragraph{Version~1.16} solves a bug that was due to a change in
syntax of the \verb+scalebox+ macro in the \verb+PSTricks+ package.

The \verb+action+ and \verb+condition+ macros are extended with a
starred option. The starred version of the macros automatically
adjusts the size of the rectangle and hexagon based on the size of the
contents. 

\paragraph{Version~1.13} has a reimplementation of message commands in MSC
diagrams. The affected commands are: \verb+\create+, \verb+\found+,
\verb+\lost+, \verb+\mess+, and \verb+\order+. The new implementation
provides more control over the placement of message labels.

The command \verb+\selfmesslabelpos+ has been removed.

The bounding box bug has been partly solved. Now, a white \verb+\fbox+
is drawn around every (msc, mscdoc, hmsc) diagram. This makes it
possible for \texttt{dvips -E} to compute the correct bounding box for
the diagrams. Due to the \verb+\fbox+, each diagram is extended with
0.3pt on each side (left, top, right, and bottom). This bugfix fails
if the background is not white. The xdvi program shows the white
\verb+\fbox+ in black with the result that diagrams have two visible
frames. This seems to be a bug of xdvi.

The lines around comments (\verb+\msccomment+, Section~\ref{comments})
are changed from gray into black. The reason for this is that the gray
lines became invisible after converting the document to HTML.

\paragraph{Version~1.12} is a non-public version. It features a preliminary
implementation of message label position control.

\paragraph{Version~1.11} is a mainly a bugfix of version~1.10a, see~\cite{BM02b}.
The next list shows the new features of version~1.10a compared to
version 1.4.

\begin{itemize}
\item Support for method replies (dashed message arrows):\cmd{mess*} (Section~\ref{messages})
\item Fat instances, i.e., double line instances (Section~\ref{instances})
\item Comments (Section~\ref{comments})
\item Coregions are now special cases of regions (which includes also
activation and suspension regions). The \cmd{coregionstart} and
\cmd{coregionend} commands are obsolete and the \cmd{coregionbarwidth}
is replaced by \cmd{regionbarwidth}.
\item Gates (Section~\ref{gates})
\item High-level \MSC{}'s (Section~\ref{hmsc})
\item \cmd{inlinestart} has additional optional argument to allow for different left and right overlap (Section~\ref{inlines})
\item \MSCdoc{}s (Section~\ref{mscdoc})
\item Reference manual~\cite{BM02}
\item \cmd{referencestart} has additional optional argument to allow for different left and right overlap (Section~\ref{references})
\item Regions: generalization of coregions supporting activations, suspension, and coregions (Section~\ref{regions})
\item Time measurements (Section~\ref{measurements})
\end{itemize}


\section{Introduction}
\label{introduction}

The \MSC{} language is a visual language for the description of the
interaction between different components of a system.
This language is standardized by the ITU (International
Telecommunication Union) in Recommendation Z.120~\cite{z120}. An
introductory text on \MSC{} can be found
in~\cite{RudolphGrabowskiGraubmann96}. \MSC{}s have a
wide application domain, ranging from requirements specification to
testing and documentation.
An example of a Message Sequence Chart is in Figure~\ref{quick}.

In order to support easy drawing of \MSC{}s in \LaTeX\ documents, we have
developed the \mscpack.  The current version of the \mscpack{}
supports the following \MSC{} constructs: \MSC{} frame, instances (both
single line width and double line width), messages (including
self-messages and messages to the environment), actions, singular and
combined timer events (set, timeout, reset, set-timeout, set-stop),
lost and found messages, generalized order, conditions, coregions,
activation regions, suspension regions, gates, instance creation,
instance stop, time measurements, references, and inline
expressions. In addition, there is support for \HMSC{}'s (high-level
\MSC{}s) and \MSCdoc{}s.

In this manual we explain the design and the use of the \mscpack. For
a complete overview of all features, we refer to the reference
manual~\cite{BM02}, which is included in the distribution under the
name \verb+refman.ps+. Another way to learn how to use the \mscpack{}
is to have a look at the \LaTeX{} source code of the manual and the
source code of the reference manual. They are included in the
distribution under the names \verb+manual.tex+ and \verb+refman.tex+,
respectively.  The \MSC{} constructs are simply introduced as
syntactic constructs. This paper is not meant to describe their use or
meaning.

We list the backgrounds of the package and some design decisions in
Section~\ref{background}.
Section~\ref{install} contains notes on installing the
package. Section~\ref{quickstart} contains an example of using the
package. It allows the impatient reader to quickly start using the
package. The details of using the package are explained in
Section~\ref{use}. In Section~\ref{parameters} the parameters are
explained which determine detailed layout of the various symbols.
A large but meaningless example is given in
Section~\ref{example}.


\section{Background and motivation}
\label{background}
Several commercial and non-commercial tools are available, which
support drawing or generating Message Sequence Charts. However, these
tools are in general not freely available and often not flexible
enough to satisfy all user's wishes with respect to the layout and
graphical appearance of an \MSC{}.
Therefore, people often use general drawing tools, such as {\em xfig}
to draw \MSC{}s. However flexible this approach is, it takes quite some
effort to produce nice \MSC{} drawings in a tool which is not dedicated
to \MSC{}s. Furthermore, when drawing a number of \MSC{}s it requires some
preciseness in order to obtain a consistent set of \MSC{}s.

For these reasons, we have started the design of a set of
\LaTeX\ macros which support the drawing of \MSC{}s. In this way, an \MSC{} can be
represented in \LaTeX\ in a textual format and compiled into
e.g.\ PostScript.

We aimed at satisfying the following requirements and design
decisions.
\begin{itemize}
\item
The package should follow the ITU standard with respect to shape and
placement of the symbols. (The current version supports
the \MSC2000 standard.) 
\item
Static and dynamic semantics are not considered. The user is allowed
to violate all semantical restrictions and draw inconsistent \MSC{}s.
The package only supports elementary syntactical requirements.
\item
The package should offer functionality at the right level of
abstraction. Rather than supplying coordinates of pixels, the
user should be able to express the placement of symbols in terms of
{\em levels}.
Nevertheless, the textual representation of \MSC{}s as defined by the ITU
standard has a level of abstraction which is too high for our purposes.
It lacks information about the actual positioning of the \MSC{} symbols,
while we think that in our package this should be under user control.
\item
There should be only minimal automatic restructuring and layout of the
\MSC{} (e.g.\ the relative positioning of two messages should be as
defined by the user, even if the messages are not causally ordered).
\item
The user can customize the appearance of the \MSC{}s by manipulating
an appropriate set of parameters.
\end{itemize}

\section{Installation, copyright and system requirements}
\label{install}

The \mscpack{} is still under development. The authors
appreciate any comments and suggestions for improvements. The most
recent version of the package can be downloaded from
\url{http://satoss.uni.lu/mscpackage/}.


The \mscpack{} has a \emph{LaTeX Project Public License}
(LPPL), see~\url{http://www.latex-project.org/lppl.txt}:
{\small
\verbatiminput{COPYRIGHT}
} As such, it is free of charge and can be freely
distributed. Furthermore, it is allowed to make modifications to the
package, provided that modified versions get different names.  The
authors accept no liability with respect to the functioning of the
package.

The \mscpack{} runs with \LaTeXe. It has been tested with \LaTeXe\
version dated 1998/06/01 using \TeX\ version 3.14159.  The following
additional packages are required: \textsf{pstricks}, \textsf{calc},
\textsf{ifthen}, and \textsf{color}.  These packages are in general
part of the standard \LaTeXe\ distribution. These additional packages
can be obtained from the {\em ctan} database for \LaTeX\, e.g.\ via
the following URL: \url{http://www.tex.ac.uk/}. The \textsf{pstricks}
package is described in~\cite[Chapter~4]{GSM97}.  The generated output
can only be previewed with recent previewing software (e.g.\ xdvi
version 20c).  It may be needed to update all \LaTeX\ related software
to a more recent version in order to smoothly work with the \mscpack.

The \mscpack{} can be installed easily. Just put the file
\verb+msc.sty+ in a directory which is searched by \LaTeX{} for style
files. The set of directories actually searched depends on the \TeX\
installation, but often the {\em current directory} is included. UNIX
users may have to set the environment variable \texttt{\$TEXINPUTS} to
an appropriate value.  For more details on this topic consult
documentation of your \TeX\ installation.

%The following text is currently not applicable -sm
%
%A problem may occur when using more packages than the \MSC{} macro
%package in your \LaTeX\ document. This happens, e.g.\ when using the
%{\em epsfig} package. These problems are due to the combination of
%several packages which are required (but not provided) by the \MSC{}
%macro package. The solution is to change the order of the
%\verb+usepackage+ clauses.

\section{Quick start}
\label{quickstart}
The \mscpack{} is easy to use.
Below is an example of the use of the package and
Figure~\ref{quick} shows the generated \MSC{}.

{\small
\begin{verbatim}
\documentclass{article}
\usepackage{msc}
\begin{document}

\begin{msc}{Example}

\declinst{usr}{User}{}
\declinst{m1}{Machine 1}{control}
\declinst{m2}{Machine 2}{drill}
\declinst{m3}{Machine 3}{test}

\mess{startm1}{usr}{m1}
\nextlevel
\mess{startm2}{m1}{m2}
\nextlevel
\mess{continue}{m2}{m3}
\mess{log}{m1}{envleft}
\nextlevel
\mess{output}{m3}{usr}[2]
\nextlevel
\mess{free}{m1}{usr}
\nextlevel

\end{msc}

\end{document}

\end{verbatim}
}

\begin{figure}[htb]
\begin{center}

\begin{msc}{Example}

\declinst{usr}{User}{}
\declinst{m1}{Machine 1}{control}
\declinst{m2}{Machine 2}{drill}
\declinst{m3}{Machine 3}{test}

\mess{startm1}{usr}{m1}
\nextlevel
\mess{startm2}{m1}{m2}
\nextlevel
\mess{continue}{m2}{m3}
\mess{log}{m1}{envleft}
\nextlevel
\mess{output}{m3}{usr}[2]
\nextlevel
\mess{free}{m1}{usr}
\nextlevel

\end{msc}

\end{center}
\caption{The generated \MSC{}}
\label{quick}
\end{figure}

The \mscpack{} is activated by the clause
\verb+\usepackage{msc}+.
This package contains, among others, a
definition of the environment \verb+msc+. 
This environment is used to draw \MSC{}s. The \MSC{} definition is surrounded by
the clauses \verb+\begin{msc}{Example}+ and \verb+\end{msc}+. The name of
the \MSC{}, {\tt Example}, is displayed in the upper-left corner of the
\MSC{}.

The next four lines define the \emph{instances}:
\verb+\declinst{m1}{Machine 1}{control}+ defines an instance with
\emph{nickname} \verb+m1+ and a description consisting of two parts,
namely, \verb+Machine 1+ and \verb+control+.  The nickname is used in
all subsequent references to this instance.  The first part of the
description is drawn above the rectangular instance head symbol, and
the second part of the description is drawn inside the instance head
symbol.

The following lines contain the definitions of the messages. Every
message has a source and a destination instance. The clause
\verb+\mess{startm1}{usr}{m1}+ defines a message with name {\tt
startm1}, which goes from instance {\tt usr} to instance {\tt m1}.

In order to control the vertical placement of the messages, the \MSC{} is
divided into levels. At every level, any number of messages may start.
The vertical position of the end point of a message is determined 
by the optional fourth argument of the message definition, as in the
clause \verb+\mess{output}{m2}{user}[2]+.
This argument is the vertical offset (in number of levels) between the
start point of the message (i.e.\ the current level) and its end point.
If the value is 0 the message is drawn horizontally. A negative offset
means that the arrow has an upward slope.

The clause \verb+\nextlevel+ is used to advance to the next level.

\section{Use of the \mscpack}
\label{use}

\subsection{The \MSC{} frame}
\label{mscframe}
The \verb+msc+ environment is used for making \MSC{} definitions. Thus, such a
definition looks as follows.

\begin{verbatim}
\begin{msc}[headerpos]{mscname}
...definition of the MSC...
\end{msc}
\end{verbatim}

This draws the frame and the header of the \MSC{}.
The argument \verb+mscname+ is the name of the \MSC{}.
The header of an \MSC{} is formed from the keyword \verb+msc+, followed
by the \verb+mscname+. Positioning of this header is controlled by
the optional argument \verb+headerpos+. This argument can have values
\verb+l+ (for a left aligned header), \verb+c+ (for a centered header)
and \verb+r+ (for a right aligned header).
The default value of \verb+headerpos+ is \verb+l+.

The size of the \MSC{} frame is determined vertically by the number of levels 
occurring in the \MSC{} (see Section~\ref{levels}) and horizontally by 
the number of instances (see Section~\ref{instances}).

The parameter \verb+\topnamedist+ controls the distance between the
top of the \MSC{} frame and the header (see Section~\ref{parameters}).
The parameter \verb+\leftnamedist+ controls the distance between the
left of the \MSC{} frame and the header if the \verb+headerpos+ is
\verb+l+ and it controls the distance between the right of the \MSC{}
frame and the header if \verb+headerpos+ is \verb+r+ (see
Section~\ref{parameters}).

\subsection{Levels}
\label{levels}
An \MSC{} is vertically divided in {\em levels}. All events in an \MSC{} are
attached to a certain level, or stretch out over several levels.
Any number of events can be drawn at a certain level.
An event will always be drawn (or started) at the current level, unless a level
offset is specified (see e.g.\ the \verb+\mess+ command in
Section~\ref{messages}). The level offset is an integer number, which
denotes at which level, relative to the current level, an event should
be drawn.
Drawing starts at level 0. 
The following command is used to advance to the next level.
\begin{verbatim}
\nextlevel[leveloffset]
\end{verbatim}

The \verb+leveloffset+ is an integer value which denotes the number of
levels to advance. By default, the value of \verb+leveloffset+ is 1,
which means drawing continues at the next level.  Setting
\verb+leveloffset+ to a negative value may result in unexpected
drawings, however, see the \emph{Tricks} section in the reference
manual~\cite{BM02}.

There are three parameters which control the size of the levels (see
Section~\ref{parameters}). These are \verb+\firstlevelheight+ (the
distance between the instance start symbol and the first level),
\verb+\levelheight+ (the distance between two consecutive levels), and
\verb+\lastlevelheight+ (the distance between the last level and the
instance end symbol). Figure~\ref{parametersfig} on
page~\pageref{parametersfig} shows all lengths of the \mscpack.

\subsection{Instances}
\label{instances}
All instances have to be declared before they can be used. An instance
consists of an instance head symbol with an associated name, an
instance axis and an instance end symbol. Normal instances have a
single line axis. Fat instances have a double line axis.  The order of
the instance declarations determines the order in which the instances
occur in the drawing.

An instance is declared with the following command.

\begin{verbatim}
\declinst(*){nickname}{instancenameabove}{instancenamewithin}
\end{verbatim}

The starred version produces a fat instance.
The \verb+nickname+ is used for referring to this instance in the rest of the
\MSC{} definition.
The \verb+instancenameabove+ is put above the instance head symbol. 
The \verb+instancenamewithin+ is put inside the instance head symbol. 

Several parameters allow the user to customize the shape and
positioning of instances (see Section~\ref{parameters}). These are
\verb+\topheaddist+ (the distance between the top of the \MSC{} and the
instance head symbol), \verb+\instheadheight+ (the height of the
instance head symbol), \verb+\instfootheight+ (the height of the
instance foot symbol), \verb+\bottomfootdist+ (the distance between
the instance foot symbol and the bottom of the \MSC{} frame),
\verb+\instwidth+ (the width of the instance head and foot symbols),
\verb+\instdist+ (the distance between two instance axes),
\verb+\envinstdist+ (the distance between the edge of the \MSC{} and the
first/last instance axis), and \verb+\labeldist+ (the distance between
the instance head symbol and the part of the instance name drawn above
the head symbol).  The command \verb+\setfootcolor{color}+ sets the
fill color of the footer symbol. Valid color values are, e.g.,
\texttt{black} (default), \texttt{white}, and \texttt{gray}.

The following \MSC{} shows the declaration of an \MSC{} with three
instances. The first and the last are normal instances (one line axis)
whereas the second is a fat instance (double line axis). The second
line, \verb+\setmscvalues{small}+, indicates that the small drawing
style should be used (see Section~\ref{parameters}).

\medskip

\begin{minipage}[c]{0.4\linewidth}
\begin{msc}[l]{instances}
\setmscvalues{small}

\declinst{i}{above}{within}
\declinst*{j}{}{j}
\declinst{k}{k}{}

\end{msc}
\end{minipage}
%
\begin{minipage}[c]{0.5\linewidth}
{\small
\begin{verbatim}
\begin{msc}{instances}
\setmscvalues{small}

\declinst{i}{above}{within}
\declinst*{j}{}{j}
\declinst{k}{k}{}

\end{msc}
\end{verbatim}
}
\end{minipage}


\subsection{Messages}
\label{messages}
A message is denoted by an arrow from the sending instance to the
receiving instance. The instances are referred to by their nicknames.
A message is defined with the following command.

\begin{verbatim}
\mess(*)[pos]{name}[labelpos]{sender}[placement]{receiver}[leveloffset]
\end{verbatim}
The \verb+name+ of the message may be any string. The \mscpack{}
processes the \verb+name+ argument in LR-mode,
see~\cite[page~36]{Lam94}. This means that the string will consist of
one line. To generate multi-line message names, use the standard
\cmd{parbox} command, see the \emph{Tricks} section in the reference
manual~\cite{BM02}.  By default, the name of a message label is drawn
above the center of the arrow, but the optional parameters \verb+pos+,
\verb+labelpos+, and \verb+placement+ influence the actual location,
as described below.  The arrow starts at the current level at the
sending instance. The arrow ends at the current level plus the
\verb+leveloffset+, at the receiving instance.  The \verb+leveloffset+
is an optional integer argument with default value 0.  The
\verb+sender+ and \verb+receiver+ should be the nicknames of declared
instances.

In case the sending and the receiving instance are the same, the
message is a \emph{self message}. A self message is drawn as a
polyline connecting the instance axis to itself.

The starred version of the command, \verb+\mess*+, produces the same
result as \verb+\mess+, except that the arrow is drawn with a dashed
line. This can be used to draw a {\em method reply} (see~\cite{z120}).

As mentioned above, placement of the message and its label is
controlled by the optional parameters \verb+pos+, \verb+labelpos+, and
\verb+placement+. In case of a self message, \verb+pos+ denotes the
position of the arrow with respect to the instance axis. Valid values
are \verb+l+ (left) and \verb+r+ (right). The default value is
\verb+l+.  In case of a non-self message, the \verb+pos+ parameter is
ignored.

Whereas the \verb+pos+ parameter defines the position of the arrow
symbol with respect to the instance axis, the \verb+labelpos+
parameter defines the position of the message label with
respect to the message arrow. In case of a self message, valid values
for \verb+labelpos+ are \verb+l+ (left) and~\verb+r+ (right). The
default value is equal to the, possibly user defined, value of the
\verb+pos+ parameter. In case of a non-self message, valid values for
\verb+labelpos+ are \verb+t+ (on top) and~\verb+b+ (below). The
default value is~\verb+t+.

Finally, the \verb+placement+ parameter defines the relative distance
of the message label to the beginning of the message. Valid values are
real numbers in the closed interval $[0,1]$. The default value
is~$0.5$. While drawing a message, the \mscpack{} computes the
coordinates of the message label using \verb+placement+ and the length
and coordinates of the arrow. It then computes a \emph{reference
point} for the message label and places it on the coordinates just
computed. Figures \ref{fig:refpoints} (page~\pageref{fig:refpoints})
and~\ref{fig:refpoints:B} (page~\pageref{fig:refpoints:B})
schematically show the reference points for message labels. In the
first figure, the labels are located at the default position. In the
second figure, all labels are shifted along the arrow by setting
$\verb+placement+ = 0.9$. The \verb+\lost+ and \verb+\found+ commands
(Section~\ref{lostfound}) and the \verb+\create+
(Section~\ref{createstop}) command use the same method to determine
reference points and message label locations. Note that the boxes with
the location of the reference points are not generated by the \LaTeX{}
code given in Figures \ref{fig:refpoints} and~\ref{fig:refpoints:B};
we enriched the \LaTeX{} code with some extra \textsf{pstricks} code
(see \LaTeX{} source code of this document).

In addition to label position control, there are three ways to control
the shape of messages (see Section~\ref{parameters}).  These are:
\verb+\selfmesswidth+ (a parameter to specify the width of the
polyline used for drawing self-messages),  \verb+\labeldist+ (a
parameter to specify the distance between the label of a message and
the message arrow), and \verb+\messarrowscale{size}+ (a command to set
the size of the arrow head of a message). \verb+size+ should be
positive real number.

Messages to or from the environment (i.e.\ the left or the right side
of the \MSC{} frame) can be
specified by setting the sender or the receiver argument to one of the values
\verb+envleft+ or \verb+envright+. (Note: Since instances and
environments are treated equally in the implementation, at every position
where the nickname of an instance is required, also
\verb+envleft+ and \verb+envright+ are allowed.)

The following \MSC{} shows an example of the use of messages.
In this sample \MSC{} and the following \MSC{}s in this section we will
not list the complete textual representations of the \MSC{}s. For brevity
we omit the environment call and the declarations of the instances.
Note the final \verb+\nextlevel+ command which is needed
to make the instance axis long enough to receive message
\verb+a+.

\medskip

\begin{minipage}[c]{0.4\linewidth}
\begin{msc}{messages}
\setmscvalues{small}
\setlength{\topheaddist}{0.8cm}

\declinst{i}{}{i}
\declinst{j}{}{j}
\declinst{k}{}{k}

\mess{a}{j}{i}[3]
\mess{self}{i}{i}
\nextlevel
\mess*{b}{j}{k}
\mess[b]{c}{k}{envright}
\nextlevel
\mess{d}{k}[.6]{i}
\nextlevel

\end{msc}
\end{minipage}
%
\begin{minipage}[c]{0.5\linewidth}
{\small
\begin{verbatim}
\mess{a}{j}{i}[3]
\mess{self}{i}{i}
\nextlevel
\mess*{b}{j}{k}
\mess[b]{c}{k}{envright}
\nextlevel
\mess{d}{k}[.6]{i}
\nextlevel
\end{verbatim}
}
\end{minipage}

\begin{figure}[htb]
\begin{minipage}{\linewidth}
\setmscvalues{small}
\begin{multicols}{2}
\begin{msc}{Label reference points}
\declinst{m0}{I0}{}
\declinst{m1}{I1}{}
\declinst{m2}{I2}{}
\nextlevel

\mess{\rpS}{m0}{m1}
\nextlevel
\mess{\rpN}[b]{m1}{m2}
\nextlevel[2]

\mess{\rpS}{m1}{m0}
\nextlevel
\mess{\rpN}[b]{m2}{m1}
\nextlevel[2]

\mess{\rpE}{m0}{m0}[2]
\mess[r]{\rpW}{m2}{m2}[2]
\nextlevel[4]
\mess{\rpW}[r]{m0}{m0}[2]
\mess[r]{\rpE}[l]{m2}{m2}[2]
\nextlevel[6]

\mess{\rpE}{m0}{m0}[-2]
\mess[r]{\rpW}{m2}{m2}[-2]
\nextlevel[4]
\mess{\rpW}[r]{m0}{m0}[-2]
\mess[r]{\rpE}[l]{m2}{m2}[-2]
\nextlevel[2]

\mess{\rpSW}{m0}{m1}[2]
\mess{\rpNE}[b]{m1}{m2}[2]
\nextlevel[6]

\mess{\rpSW}{m1}{m0}[-2]
\mess{\rpNE}[b]{m2}{m1}[-2]
\nextlevel[2]

\mess{\rpSE}{m1}{m0}[2]
\mess{\rpNW}[b]{m2}{m1}[2]
\nextlevel[6]

\mess{\rpSE}{m0}{m1}[-2]
\mess{\rpNW}[b]{m1}{m2}[-2]
\nextlevel[2]
\end{msc}
\bigskip

\scriptsize
\begin{verbatim}
\begin{msc}{Label reference points}
\declinst{m0}{I0}{}
\declinst{m1}{I1}{}
\declinst{m2}{I2}{}
\nextlevel

\mess{S}{m0}{m1}
\nextlevel
\mess{N}[b]{m1}{m2}
\nextlevel[2]

\mess{S}{m1}{m0}
\nextlevel
\mess{N}[b]{m2}{m1}
\nextlevel[2]

\mess{E}{m0}{m0}[2]
\mess[r]{W}{m2}{m2}[2]
\nextlevel[4]
\mess{W}[r]{m0}{m0}[2]
\mess[r]{E}[l]{m2}{m2}[2]
\nextlevel[6]

\mess{E}{m0}{m0}[-2]
\mess[r]{W}{m2}{m2}[-2]
\nextlevel[4]
\mess{W}[r]{m0}{m0}[-2]
\mess[r]{E}[l]{m2}{m2}[-2]
\nextlevel[2]

\mess{SW}{m0}{m1}[2]
\mess{NE}[b]{m1}{m2}[2]
\nextlevel[6]

\mess{SW}{m1}{m0}[-2]
\mess{NE}[b]{m2}{m1}[-2]
\nextlevel[2]

\mess{SE}{m1}{m0}[2]
\mess{NW}[b]{m2}{m1}[2]
\nextlevel[6]

\mess{SE}{m0}{m1}[-2]
\mess{NW}[b]{m1}{m2}[-2]
\nextlevel[2]
\end{msc}
\end{verbatim}
\end{multicols}
\end{minipage}
\caption{Reference points of message labels}
\label{fig:refpoints}
\end{figure}


\begin{figure}[htb]
\begin{minipage}{\linewidth}
\setmscvalues{small}
\begin{multicols}{2}
\begin{msc}{Label reference points (2)}
\declinst{m0}{I0}{}
\declinst{m1}{I1}{}
\declinst{m2}{I2}{}
\nextlevel

\mess{\rpS}{m0}[.9]{m1}
\nextlevel
\mess{\rpN}[b]{m1}[.9]{m2}
\nextlevel[2]

\mess{\rpS}{m1}[.9]{m0}
\nextlevel
\mess{\rpN}[b]{m2}{m1}
\nextlevel[2]

\mess{\rpE}{m0}[.9]{m0}[2]
\mess[r]{\rpW}{m2}[.9]{m2}[2]
\nextlevel[4]
\mess{\rpW}[r]{m0}[.9]{m0}[2]
\mess[r]{\rpE}[l]{m2}[.9]{m2}[2]
\nextlevel[6]

\mess{\rpE}{m0}[.9]{m0}[-2]
\mess[r]{\rpW}{m2}[.9]{m2}[-2]
\nextlevel[4]
\mess{\rpW}[r]{m0}[.9]{m0}[-2]
\mess[r]{\rpE}[l]{m2}[.9]{m2}[-2]
\nextlevel[2]

\mess{\rpSW}{m0}[.9]{m1}[2]
\mess{\rpNE}[b]{m1}[.9]{m2}[2]
\nextlevel[6]

\mess{\rpSW}{m1}[.9]{m0}[-2]
\mess{\rpNE}[b]{m2}[.9]{m1}[-2]
\nextlevel[2]

\mess{\rpSE}{m1}[.9]{m0}[2]
\mess{\rpNW}[b]{m2}[.9]{m1}[2]
\nextlevel[6]

\mess{\rpSE}{m0}[.9]{m1}[-2]
\mess{\rpNW}[b]{m1}[.9]{m2}[-2]
\nextlevel[2]
\end{msc}
\bigskip

\scriptsize
\begin{verbatim}
\begin{msc}{Label reference points (2)}
\declinst{m0}{I0}{}
\declinst{m1}{I1}{}
\declinst{m2}{I2}{}
\nextlevel

\mess{S}{m0}[.9]{m1}
\nextlevel
\mess{N}[b]{m1}[.9]{m2}
\nextlevel[2]

\mess{S}{m1}[.9]{m0}
\nextlevel
\mess{N}[b]{m2}{m1}
\nextlevel[2]

\mess{E}{m0}[.9]{m0}[2]
\mess[r]{W}{m2}[.9]{m2}[2]
\nextlevel[4]
\mess{W}[r]{m0}[.9]{m0}[2]
\mess[r]{E}[l]{m2}[.9]{m2}[2]
\nextlevel[6]

\mess{E}{m0}[.9]{m0}[-2]
\mess[r]{W}{m2}[.9]{m2}[-2]
\nextlevel[4]
\mess{W}[r]{m0}[.9]{m0}[-2]
\mess[r]{E}[l]{m2}[.9]{m2}[-2]
\nextlevel[2]

\mess{SW}{m0}[.9]{m1}[2]
\mess{NE}[b]{m1}[.9]{m2}[2]
\nextlevel[6]

\mess{SW}{m1}[.9]{m0}[-2]
\mess{NE}[b]{m2}[.9]{m1}[-2]
\nextlevel[2]

\mess{SE}{m1}[.9]{m0}[2]
\mess{NW}[b]{m2}[.9]{m1}[2]
\nextlevel[6]

\mess{SE}{m0}[.9]{m1}[-2]
\mess{NW}[b]{m1}[.9]{m2}[-2]
\nextlevel[2]
\end{msc}
\end{verbatim}
\end{multicols}
\end{minipage}
\caption{Reference points of shifted message labels}
\label{fig:refpoints:B}
\end{figure}


\subsection{Comments}
\label{comments}

Comments are additional texts to clarify (events on) an instance. The
following command can be used to add comments to an \MSC{} diagram.

\begin{verbatim}
\msccomment[position]{text}{instname}
\end{verbatim}

The \verb|instname| parameter defines the instance to which the
comment is attached. The text of the comment is specified by the
\verb|text| parameter and is processed in LR-mode. The \verb|position|
parameter defines the horizontal position of the comment relative to
its instance. Valid values of \verb|position| are \verb|l| (left),
\verb|r| (right), or any \LaTeX{} length. Its default value
is~\verb|l|. If the value of \verb|position| is \verb|l|
(or~\verb|r|), the comment will be placed \verb|\msccommentdist| units
to the left (or right) of the \verb|instname| instance. If
\verb|position| is a \LaTeX{} length, the comment will be placed
\verb|position| units from the \verb|instname| instance. A negative
length puts the comment to the left and a positive length puts the
comment to the right of the instance.

The following diagram shows how to use comments. In this diagram, the
distance between the frame and the instances (\verb|\envinstdist|) is
doubled in order to fit the comments inside the frame.

\medskip
\begin{minipage}[c]{0.4\linewidth}
\setmscvalues{small}
\begin{msc}{comments}
\setlength{\envinstdist}{2\envinstdist}
\declinst{i}{}{i}
\declinst{j}{}{j}
\mess{a}{i}{j}[2]
\msccomment{start}{i}
\nextlevel[2]
\msccomment[r]{end}{j}
\nextlevel
\end{msc}
\end{minipage}
%
\begin{minipage}[c]{0.54\linewidth}
{\small
\begin{verbatim}
\setlength{\envinstdist}{2\envinstdist}
\declinst{i}{}{i}
\declinst{j}{}{j}
\mess{a}{i}{j}[2]
\msccomment{start}{i}
\nextlevel[2]
\msccomment[r]{end}{j}
\nextlevel
\end{verbatim}
}
\end{minipage}




\subsection{Actions}
\label{actions}
An instance can perform an action, which is denoted by a rectangle.

\begin{verbatim}
\action(*){name}{instance}
\end{verbatim}

The action is attached at the current level to the \verb+instance+.
The \verb+name+ is centered inside the action symbol and is processed
in LR-mode.

The following parameters determine the detailed drawing of the action
symbol (see Section~\ref{parameters}):
\verb+\actionwidth+ (the width of the action symbol), and
\verb+\actionheight+ (the height of the action symbol).

The starred version of the command, \verb+\action*+, produces the same
result as \verb+\action+, except that the height and width of the
action symbol are adjusted to fit the contents of the rectangle.

The next example shows that after an action often a multiple level
increment is needed to obtain nice pictures.

\medskip

\begin{minipage}[c]{0.4\linewidth}
\begin{msc}{action}
\setmscvalues{small}
%\setlength{\topheaddist}{0.8cm}

\declinst{i}{}{i}
\declinst{j}{}{j}

\mess{a}{j}{i}
\nextlevel
\action{doit}{i}
\nextlevel[2]
\mess{b}{i}{j}

\end{msc}
\end{minipage}
%
\begin{minipage}[c]{0.5\linewidth}
{\small
\begin{verbatim}
\mess{a}{j}{i}
\nextlevel
\action{doit}{i}
\nextlevel[2]
\mess{b}{i}{j}
\end{verbatim}
}
\end{minipage}


\subsection{Timers}
\label{timers}
There are five commands to draw timer events.
\begin{verbatim}
\settimer[placement]{name}{instance}
\timeout[placement]{name}{instance}
\stoptimer[placement]{name}{instance}
\settimeout[placement]{name}{instance}[offset]
\setstoptimer[placement]{name}{instance}[offset]
\end{verbatim}

Setting of a timer is drawn as a line connecting the \verb+instance+
to the {\em hour glass} symbol. The \verb+name+ is put near this
symbol. A time-out is represented by an arrow from an
{\em hour glass} symbol to the \verb+instance+. Stopping a timer is
drawn as a line connecting the \verb+instance+ with the timer stop
symbol (denoted by a cross).
The command \verb+\settimeout+ is a combination of the setting of a
timer and a time out. The \verb+offset+ denotes the number of levels
between the two events. The default value for \verb+offset+ is 2.
Likewise, \verb+\setstoptimer+ is a combination
of the setting of a timer and stopping a timer.

The optional argument \verb+placement+ can have values
\verb+l+ (meaning that the timer is drawn left of the instance axis)
and \verb+r+ (meaning that the timer is drawn right of the instance
axis). By default it is drawn at the left side of the instance.

Several parameters can be used to control the detailed layout of timer
symbols (see Section~\ref{parameters}):
\verb+\timerwidth+ (the width of the hour glass and time out symbols),
\verb+\selfmesswidth+ (the length of the arm between the symbol
and the instance axis), and
\verb+\labeldist+ (the distance between the name and the timersymbol).
Furthermore, the size of the arrow head can be controlled with the
command \verb+\messarrowscale{size}+.

The various timer symbols are shown in the following example.

\medskip

\begin{minipage}[c]{0.4\linewidth}
\begin{msc}{timers}
\setmscvalues{small}
\setlength{\topheaddist}{0.8cm}

\declinst{i}{}{i}
\declinst{j}{}{j}
\declinst{k}{}{k}

\settimer{T,50}{j}
\setstoptimer[r]{V}{k}[6]
\nextlevel[2]
\timeout{T}{j}
\settimeout{U}{i}
\nextlevel[2]
\settimer[r]{T,20}{j}
\nextlevel[2]
\stoptimer[r]{T}{j}

\end{msc}
\end{minipage}
%
\begin{minipage}[c]{0.5\linewidth}
{\small
\begin{verbatim}
\settimer{T,50}{j}
\setstoptimer[r]{V}{k}[6]
\nextlevel[2]
\timeout{T}{j}
\settimeout{U}{i}
\nextlevel[2]
\settimer[r]{T,20}{j}
\nextlevel[2]
\stoptimer[r]{T}{j}
\end{verbatim}
}
\end{minipage}

\subsection{Time measurements}
\label{measurements}
There are several commands to add time measurements to an \MSC{}.

\begin{verbatim}
\mscmark[placement]{name}{instance}
\measure(*)[placement]{name}{instance1}{instance2}[offset]
\measurestart(*)[placement]{name}{instance}{gate}
\measureend(*)[placement]{name}{instance}{gate}
\end{verbatim}

An absolute time stamp is attached to an event on an \verb+instance+
with the command
\verb+mscmark+. The \verb+name+ is the text attached to the mark symbol,
which is a dashed polyline.
The position of the mark symbol relative to the marked event is
determined by the \verb+placement+.
(\verb+tl+ means top-left, \verb+tr+ means top-right, \verb+bl+ means
bottom-left, and \verb+br+ means bottom-right).

A \verb+\measure+ connects two events from \verb+instance1+ and
\verb+instance2+. The first event is at the current level. The second
event is \verb+offset+ levels lower than the first event. The
\verb+name+ is attached to the measure symbol.  The measure symbol can
be placed at the left or at the right of \verb+instance1+. This is
controlled by the optional argument \verb+placement+, which can have
values \verb+l+ and \verb+r+.

In case the two events are far apart, the measure may be split in two
parts. The first part is drawn with the \verb+\measurestart+ command
and the second part with the \verb+\measureend+ command. The points
where these two parts should be connected are drawn by a small circle,
to which the text \verb+gate+ is attached.

There are two equivalent forms of the measurement symbol. The first
form, where the arrow heads are at the inside of the measured
interval, is the default form. The second form, where the arrow heads
are at the outside of the measured interval, is obtained by the
commands \verb+\measure*+, \verb+\measurestart*+, and
\verb+\measureend*+.


Several parameters can be used to control the detailed layout of the
time measurement symbols (see Section~\ref{parameters}):
\verb+\labeldist+ (a parameter to specify the distance between the
label of a measurement and the measurement symbol),
\verb+\messarrowscale{size}+ (a command to set the size of the arrow
head),
\verb+\selfmesswidth+ (specifies the width of the measurement
symbols).

The following example illustrates marks and measurements in an \MSC{}. In
order to include the marks inside the frame of the diagram, the
distance between the frame and the instances (called
\verb+\envinstdist+) is increased (before the instances are declared).

\medskip

\begin{minipage}[c]{0.5\linewidth}
\begin{msc}{Time measurements}
\setmscvalues{small}
\setlength{\envinstdist}
          {2\envinstdist}

\declinst{i}{}{i}
\declinst{j}{}{j}
\declinst{k}{}{k}

\mess{m1}{i}{j}[1]
\mscmark{t=0.0}{i}
\measure{0.6}{i}{j}[4]
\nextlevel
\mscmark[tr]{t=0.3}{j}
\nextlevel[3]
\mess{m2}{j}{k}
\measurestart*[r]{0.2}{k}{g}
\nextlevel[6]
\mess{m3}{k}{i}
\mscmark[bl]{t=1.0}{i}
\measureend*[r]{0.2}{k}{g}
\nextlevel
\end{msc}
\end{minipage}
%
\begin{minipage}[c]{0.44\linewidth}
{\small
\begin{verbatim}
\setlength{\envinstdist}
          {2\envinstdist}
\declinst{i}{}{i}
\declinst{j}{}{j}
\declinst{k}{}{k}
\mess{m1}{i}{j}[1]
\mscmark{t=0.0}{i}
\measure{0.6}{i}{j}[4]
\nextlevel
\mscmark[tr]{t=0.3}{j}
\nextlevel[3]
\mess{m2}{j}{k}
\measurestart*[r]{0.2}{k}{g}
\nextlevel[6]
\mess{m3}{k}{i}
\mscmark[bl]{t=1.0}{i}
\measureend*[r]{0.2}{k}{g}
\nextlevel
\end{verbatim}
}
\end{minipage}


\subsection{Lost and found messages}
\label{lostfound}
A lost message is denoted by an arrow starting at an instance and
ending at a filled circle. A found message is denoted by an arrow
starting at an open circle and ending at an instance.

The following commands are used to define lost and found messages.
\begin{verbatim}
\lost[pos]{name}[labelpos]{gate}{instance}[placement]
\found[pos]{name}[labelpos]{gate}{instance}[placement]
\end{verbatim}

The argument \verb+instance+ determines the instance to which the
arrow is attached.  The \verb+name+ of the message is put above the
message arrow. The \verb+gate+ is a text associated to the circle.
The optional arguments \verb+pos+, \verb+labelpos+, and
\verb+placement+ have the same function as in the \verb+\mess+ command
(Section~\ref{messages}).  That is, \verb+pos+ controls the placement
of the lost or or found message with respect to the instance
axis. Valid values are \verb+l+ (left) and~\verb+r+ (right). The
default value is~\verb+l+.  The optional parameters \verb+labelpos+
and \verb+placement+ control the placement of \verb+name+ with respect
to the message arrow. Valid values for \verb+labelpos+ are \verb+t+
(on top) and~\verb+b+ (below). The default value is~\verb+t+. Valid
values for \verb+placement+ are real numbers in the closed interval
$[0,1]$ and denote the relative distance of the message label
\verb+name+ to the beginning of the arrow.  The default value for
\verb+placement+ is~$0.5$.

Several parameters can be used to control the detailed layout of lost
and found messages (see Section~\ref{parameters}):
\verb+\lostsymbolradius+ (the radius of the circle),
\verb+\selfmesswidth+ (the length of the arrow), and
\verb+\labeldist+ (the distance between the name and the arrow).
Furthermore, the size of the arrow head can be controlled with the
command \verb+\messarrowscale{size}+.

The following example shows a found and a lost message.

\medskip

\begin{minipage}[c]{0.4\linewidth}
\begin{msc}{lost and found}
\setmscvalues{small}
\setlength{\topheaddist}{0.8cm}

\declinst{i}{}{i}
\declinst{j}{}{j}

\found{m}{g}{i}
\nextlevel
\mess{p}{i}{j}
\nextlevel
\lost[r]{n}{}{j}

\end{msc}
\end{minipage}
%
\begin{minipage}[c]{0.5\linewidth}
{\small
\begin{verbatim}
\found{m}{g}{i}
\nextlevel
\mess{p}{i}{j}
\nextlevel
\lost[r]{n}{}{j}
\end{verbatim}
}
\end{minipage}



\subsection{Conditions}
\label{conditions}
A condition is denoted by a hexagon. It is used to express that the
system has entered a certain state. A condition relates to a number of
instances. All conditions which take part in the condition are covered
by the condition symbol. The other instances are drawn through the
condition symbol.
The following command is used to draw a condition.

\begin{verbatim}
\condition{text}{instancelist}
\end{verbatim}

The \verb+text+ is placed in the center of the condition. The
\verb+instancelist+ expresses which instances take part in the
condition. It is a list of nicknames of instances, separated by
commas. Take care not to add extra white space around the nicknames,
since this is considered part of the nickname in \LaTeX.
The order in which the instances are listed is immaterial.

There are two parameters which control the shape of the condition
symbol (see Section~\ref{parameters}):
\verb+\conditionheight+ (the height of the condition symbol), and
\verb+\conditionoverlap+ (the width of the part of the condition
symbol which extends over the rightmost/leftmost contained instance
axis).

The starred version of the command, \verb+\condition*+, produces the same
result as \verb+\condition+, except that the height and width of the
condition symbol are adjusted to fit the contents of the hexagon.

The following example contains some conditions.

\medskip

\begin{minipage}[c]{0.4\linewidth}
\begin{msc}{conditions}
\setmscvalues{small}
\setlength{\topheaddist}{0.8cm}

\declinst{i}{}{i}
\declinst{j}{}{j}
\declinst{k}{}{k}

\condition{some condition}{i,k}
\nextlevel[3]
\mess{m}{i}{j}
\action{a}{k}
\nextlevel
\condition{C}{i}
\nextlevel[2]
\condition{A, B, C}{i,j,k}
\nextlevel[2]

\end{msc}
\end{minipage}
%
\begin{minipage}[c]{0.5\linewidth}
{\small
\begin{verbatim}
\condition{some condition}{i,k}
\nextlevel[3]
\mess{m}{i}{j}
\action{a}{k}
\nextlevel
\condition{C}{i}
\nextlevel[2]
\condition{A, B, C}{i,j,k}
\nextlevel[2]
\end{verbatim}
}
\end{minipage}

\subsection{Generalized ordering}
\label{ordering}
A generalized order is treated much like a regular message (see
Section~\ref{messages}). There
are three differences: a generalized order is drawn with a
dotted line, it has no label, and the arrow head is in the middle of
the line. A generalized order is defined with the following command.
\begin{verbatim}
\order[pos]{sender}{receiver}[leveloffset]
\end{verbatim}
The \verb+sender+ and \verb+receiver+ are the nicknames of the
instances which are connected by the generalized ordering symbol.
At the \verb+receiver+ instance, the generalized ordering symbol ends
at the current level plus the \verb+leveloffset+. The
\verb+leveloffset+ is an optional integer value, with default 0.

In case \verb+sender+ and \verb+receiver+ denote the same instance,
the order is a \emph{self order}.  The placement of the order arrow of
a self order is controlled by the optional argument \verb+pos+.  It
can have values \verb+l+ (meaning that the ordering symbol is drawn
left of the instance axis) and \verb+r+ (meaning that the ordering
symbol is drawn right of the instance axis). By default it is drawn at
the left side of the instance. For non-self orders, the \verb+pos+
argument is ignored.

Additionally, there are two ways to control the shape
of the generalized ordering symbol (see Section~\ref{parameters}).
These are:
\verb+\selfmesswidth+ (a parameter to specify the width of the polyline used for
drawing orderings on a single instance axis).
and
\verb+\messarrowscale{size}+ (a command to set the size of the arrow
head in the ordering symbol).

Orderings to or from the environment (i.e.\ the left or the right side
of the \MSC{} frame) can be
specified by setting the sender or the receiver argument to the value
\verb+envleft+ or \verb+envright+.

An example of a generalized order is given in the following diagram.



\medskip

\begin{minipage}[c]{0.4\linewidth}
\begin{msc}{generalized order}
\setmscvalues{normal}
\setlength{\topheaddist}{0.8cm}

\declinst{i}{}{i}
\declinst{j}{}{j}

\mess{m}{envleft}{i}
\order{i}{j}[1]
\nextlevel
\mess{k}{j}{envright}
\nextlevel
\end{msc}
\end{minipage}
%
\begin{minipage}[c]{0.5\linewidth}
{\small
\begin{verbatim}
\mess{m}{envleft}{i}
\order{i}{j}[1]
\nextlevel
\mess{k}{j}{envright}
\end{verbatim}
}
\end{minipage}


\subsection{Instance regions}
\label{regions}
A part of the instance axis can be drawn in a different style. Such a
part is called an {\em instance region}. The following
regions are supported:
{\em coregion} (the instance axis is dashed, which means that the order of the
attached instances is immaterial),
{\em suspension region} (a small rectangle with dashed left and right
sides, which denotes that the instance is suspended),
{\em activation region} (a small filled rectangle, which denotes that
the instance has control).

The following commands are used to draw an instance region.

\begin{verbatim}
\regionstart{type}{instname}
\regionend{instname}
\end{verbatim}

If the \verb+\regionstart+ command is used the instance axis of
\verb+instname+ is drawn in the shape determined by \verb+type+,
starting from the current level.  The type can have values
\verb+coregion+, \verb+suspension+, and \verb+activation+.

The shape of an instance region can be controlled with the following
parameters (see Section~\ref{parameters}):
\verb+\regionbarwidth+ (the width of the coregion start and end
symbol),
\verb+\instwidth+ (the width of the fat instance rectangle),
\verb+\regionwidth+ (the width of the activation and suspension rectangle).

In the following example, several instance regions are demonstrated.
Notice the relative order of \verb|\mess|, \verb|\regionstart|, and
\verb|\regionend| commands. Interchanging lines
\verb|\regionstart{activation}{i}| and \verb|\mess{q}{j}{i}| makes the
arrow head of the~`q' message partly invisible.  Also note that the
second activation region on~$i$ ends with a method reply
(which is produced by the command \cmd{mess*}).

\medskip

\begin{minipage}[c]{0.4\linewidth}
\begin{msc}{regions}
\setmscvalues{small}
\setlength{\topheaddist}{0.8cm}

\declinst{i}{}{i}
\declinst{j}{}{j}
\declinst{k}{}{k}

\regionstart{activation}{i}
\nextlevel[3]
\regionend{i}
\mess{m}{i}{j}
\nextlevel
\regionstart{coregion}{j}
\nextlevel
\regionstart{suspension}{k}
\mess{p}{j}{k}
\nextlevel
\regionstart{activation}{i}
\mess{q}{j}{i}
\nextlevel
\regionend{j}
\nextlevel[3]
\regionend{i}
\mess*{r}{i}{j}
\nextlevel
\mess{s}{j}{k}
\regionend{k}
\nextlevel
\end{msc}
\end{minipage}
%
\begin{minipage}[c]{0.5\linewidth}
{\scriptsize
\begin{verbatim}
\regionstart{activation}{i}
\nextlevel[3]
\regionend{i}
\mess{m}{i}{j}
\nextlevel
\regionstart{coregion}{j}
\nextlevel
\regionstart{suspension}{k}
\mess{p}{j}{k}
\nextlevel
\regionstart{activation}{i}
\mess{q}{j}{i}
\nextlevel
\regionend{j}
\nextlevel[3]
\regionend{i}
\mess*{r}{i}{j}
\nextlevel
\mess{s}{j}{k}
\regionend{k}
\nextlevel
\end{verbatim}
}
\end{minipage}

In some situations, the gray-colored activation regions can hide
message-labels. \emph{Level back-up} can help in these situations, see
the \emph{Tricks} section in the reference manual~\cite{BM02}.

\subsection{Instance creation and instance stop}
\label{createstop}
The \MSC{} language offers constructs for dynamic instance creation and
instance destruction.
An instance can dynamically create another instance by issuing a
create command. An instance creation is drawn as a dashed message
arrow. At the side of the arrow head, the instance head symbol for the
created instance is drawn.
An instance end symbol does not denote the end of the specified
process, but merely the end of its current description. Therefore, a
different symbol is needed which denotes that an instance stops before
the end of the \MSC{} in which it is contained.
The instance stop symbol is a cross.

The following commands are used for instance creation and instance
stop.

\begin{verbatim}
\dummyinst{createdinst}
\create{name}
       [labelpos]
       {creator}
       [placement]
       {createdinst}
       {instancenameabove}
       {instancenamewithin}
\stop{instance}
\end{verbatim}

In order to reserve space for an instance which will be created
dynamically, the command \verb+dummyinst+ must be used. This command
is mixed with the declarations of normal instances (see the
\verb+declinst+ command, Section~\ref{instances}).
The argument \verb+createdinst+ is the nickname of the instance that
will be created later.

An instance can be created with the \verb+\create+ command.
This command results in a dashed horizontal message arrow labeled with
\verb+name+. The arrow starts at the current level at the instance
with nickname \verb+creator+ and ends at the current level at the
instance head of the instance with nickname \verb+createdinst+. This
instance must have been declared first with the \verb+dummyinst+
command.
The name of the created instance consists of two parts. The part
called \verb+instancenameabove+ is placed above the instance head and the 
\verb+instancenamewithin+ is centered within the instance head.

As with normal messages, placement of the message label is controlled
by the optional parameters \verb+labelpos+ and \verb+placement+. That
is, \verb+labelpos+ denotes the relative position of the message label
with respect to the message arrow. Valid values are \verb+t+ (on top)
and \verb+b+ (below). The default value is \verb+t+. The optional
parameter \verb+placement+ denotes the relative distance of the
message label with respect to the beginning of the arrow. Valid values
are real numbers in the closed interval $[0,1]$. The default value
is~$0.5$. See Figure~\ref{fig:refpoints}
(page~\pageref{fig:refpoints}) and its description in
Section~\ref{messages} for more information on the placement of
message labels.

An instance is stopped with the \verb+\stop+ command.
The \verb+instance+ is the nickname of the stopped instance. The
instance axis is not drawn any further below the level at which the
\verb+\stop+ command is issued. Also, the instance foot symbol is not
drawn.

The size of the stop symbol is determined by the parameter
\verb+\stopwidth+ (see Section~\ref{parameters}).
The following parameters apply to the instance head symbol:
\verb+\instheadheight+ (the height of the instance head symbol),
\verb+\instwidth+ (the width of the instance head symbol),
and
\verb+\labeldist+ (the distance between the instance head symbol and
the part of the instance name drawn above the head symbol).

Take care not to specify any events on an instance which has not yet
been created or which has already
been stopped. This may lead to unexpected drawings.
However, it is possible to create an instance after it has stopped,
as showed in the next example.

\medskip

\begin{minipage}[c]{0.4\linewidth}
\begin{msc}{dynamic instances}
\setmscvalues{small}
\setlength{\topheaddist}{0.8cm}

\declinst{i}{}{i}
\dummyinst{j}
\declinst{k}{}{k}

\mess{p}{i}{k}
\nextlevel[3]
\create{kick}{k}{j}{}{j}
\nextlevel
\stop{k}
\nextlevel
\mess{ok}{j}{i}
\nextlevel
\stop{j}
\nextlevel[2]
\create{start}[b]{i}[.75]{k}{}{again}
\nextlevel[2]

\end{msc}
\end{minipage}
%
\begin{minipage}[c]{0.5\linewidth}
{\small
\begin{verbatim}
\declinst{i}{}{i}
\dummyinst{j}
\declinst{k}{}{k}

\mess{p}{i}{k}
\nextlevel[3]
\create{kick}{k}{j}{}{j}
\nextlevel
\stop{k}
\nextlevel
\mess{ok}{j}{i}
\nextlevel
\stop{j}
\nextlevel[2]
\create{start}[b]{i}[.75]{k}{}{again}
\nextlevel[2]
\end{verbatim}
}
\end{minipage}

\subsection{\MSC{} references}
\label{references}
Within an \MSC{} a reference to other \MSC{}s can be included. Such a
reference is drawn as a rectangle with rounded corners, covering part
of the \MSC{}. The following commands are used to draw \MSC{} references.

\begin{verbatim}
\referencestart[lo][ro]{nickname}{text}{leftinstance}{rightinstance}
\referenceend{nickname}
\end{verbatim}

The reference symbol starts at the level where the
\verb+\referencestart+ command is used, and ends at the level where
the corresponding \verb+\referenceend+ command occurs. These commands
correspond if they have the same \verb+nickname+.  The \verb+text+ is
placed in the center of the reference symbol. The reference covers all
instances from \verb+leftinstance+ to \verb+rightinstance+.  The
distance between the left (right) edge of the reference symbol and the
leftmost (rightmost) covered instance axis is defined by the optional
overlap value \verb+lo+. The default value for \verb+lo+ in large/normal/small
\MSC{}s is 1.5/1/0.75 (see Section~\ref{parameters} for selecting
large, normal or small drawing mode). If the second optional value
\verb+ro+ is present too, the optional value \verb+lo+ defines the left
overlap and the optional value \verb+ro+ defines the right overlap.

The left and right edge of the reference symbol are
\verb+nicknameleft+ and \verb+nicknameright+, where \verb+nickname+ is
the nickname of the \MSC{} reference as defined in the
\verb+\referencestart+ command. 
These names can be used at every place where the nickname of an
instance is required, e.g.\ as the sender or receiver of a message.
This is shown in the following example.

\medskip

\begin{minipage}[c]{0.4\linewidth}
\begin{msc}{references}
\setmscvalues{small}
\setlength{\topheaddist}{0.8cm}

\declinst{i}{}{i}
\declinst{j}{}{j}
\declinst{k}{}{k}

\mess{a}{j}{k}
\nextlevel
\referencestart{r}{a reference}{i}{j}
\nextlevel
\mess{b}{rright}{k}
\nextlevel
\referenceend{r}
\nextlevel
\referencestart{p}{another reference}{i}{k}
\nextlevel
\mess{c}{envright}{pright}
\nextlevel[3]
\referenceend{p}

\end{msc}
\end{minipage}
%
\begin{minipage}[c]{0.55\linewidth}
{\small
\begin{verbatim}
\mess{a}{j}{k}
\nextlevel
\referencestart{r}{a reference}{i}{j}
\nextlevel
\mess{b}{rright}{k}
\nextlevel
\referenceend{r}
\nextlevel
\referencestart{p}{another reference}{i}{k}
\nextlevel
\mess{c}{envright}{pright}
\nextlevel[3]
\referenceend{p}
\end{verbatim}
}
\end{minipage}

\subsection{Inline expressions}
\label{inlines}
An inline expression is a part of an \MSC{} on which an operation is
defined. A rectangle surrounds the part of the \MSC{} containing the
operands. The operands are separated by horizontal dashed lines.
The operator is placed in the upper left corner of the inline
expression symbol.
The following commands are used to draw inline expressions.

\begin{verbatim}
\inlinestart[lo][ro]{nickname}{operator}{leftinstance}{rightinstance}
\inlineseparator{nickname}
\inlineend(*){nickname}
\end{verbatim}

The inline expression starts at the level where the \verb+\inlinestart+
command occurs. The inline expression spans over the instances from
\verb+leftinstance+ to \verb+rightinstance+. The \verb+operand+ is
placed in the upper left corner of the rectangle. At every level where
a corresponding \verb+\inlineseparator+ command occurs, a dashed line
is drawn. The inline expression ends at the level where the
\verb+\inlineend+ command occurs. The \verb+nickname+ is used to match
corresponding commands.
The command \verb+\inlineend*+ does the same as the command
\verb+\inlineend+, except that the bottom line of the rectangle is
dashed. This is used to indicate that the operand is optional.

The left and right edge of the inline expression symbol are named
\verb+nicknameleft+ and \verb+nicknameright+, where \verb+nickname+ is
the nickname of the inline expression as defined in the
\verb+\inlinestart+ command. 
These names can be used at every place where the nickname of an
instance is required, e.g.\ as the sender or receiver of a message.

The distance between the left (right) edge of the inline expression
symbol and
the leftmost (rightmost) included instance axis is defined by
the optional overlap value \verb+lo+. The default value for
large/normal/small \MSC{}s is
1.5/1/0.75 (see Section~\ref{parameters} for selecting large, normal or small
drawing mode).
If the second optional value
\verb+ro+ is present too, the optional value \verb+lo+ defines the left
overlap and the optional value \verb+ro+ defines the right overlap.
These options make it possible to easily draw nested inline expressions
as in the following example.

\medskip

\begin{minipage}[c]{0.4\linewidth}
\begin{msc}{inline expressions}
\setmscvalues{small}
\setlength{\topheaddist}{0.8cm}


\declinst{i}{}{i}
\declinst{j}{}{j}
\declinst{k}{}{k}


\inlinestart{exp1}{par}{i}{k}
\nextlevel
\mess{a}{k}{j}
\nextlevel
\inlinestart[0.6cm]{exp2}{alt}{i}{j}
\nextlevel[2]
\mess{b}{j}{i}
\nextlevel
\inlineseparator{exp2}
\nextlevel
\mess{c}{j}{i}
\nextlevel
\inlineend{exp2}
\nextlevel
\inlineseparator{exp1}
\nextlevel
\mess{d}{j}{k}
\nextlevel
\inlineend{exp1}
\nextlevel
\inlinestart{exp3}{opt}{i}{j}
\nextlevel
\mess{e}{envright}{exp3right}
\mess{e}{exp3right}{j}
\nextlevel
\mess{f}{j}{i}
\nextlevel
\inlineend*{exp3}


\end{msc}
\end{minipage}
%
\begin{minipage}[c]{0.5\linewidth}
{\scriptsize
\begin{verbatim}
\inlinestart{exp1}{par}{i}{k}
\nextlevel
\mess{a}{k}{j}
\nextlevel
\inlinestart[0.6cm]{exp2}{alt}{i}{j}
\nextlevel[2]
\mess{b}{j}{i}
\nextlevel
\inlineseparator{exp2}
\nextlevel
\mess{c}{j}{i}
\nextlevel
\inlineend{exp2}
\nextlevel
\inlineseparator{exp1}
\nextlevel
\mess{d}{j}{k}
\nextlevel
\inlineend{exp1}
\nextlevel
\inlinestart{exp3}{opt}{i}{j}
\nextlevel
\mess{e}{envright}{exp3right}
\mess{e}{exp3right}{j}
\nextlevel
\mess{f}{j}{i}
\nextlevel
\inlineend*{exp3}
\end{verbatim}
}
\end{minipage}

\subsection{Gates}
\label{gates}

A gate determines a connection point for messages.

The following command can be used to draw gates.

\begin{verbatim}
\gate(*)[hpos][vpos]{gatename}{instname}
\end{verbatim}

The unstarred version produces a normal (invisible) gate. The starred
version produces a visible gate (a small dot). The gate is drawn at the
current level at the instance \verb+instname+ (which can also be the
left and right edge of e.g.\ an \MSC{} reference).  The
\verb+gatename+ is attached to the gate.  The positioning of the
\verb+gatename+ relative to the gate is determined by the values of
\verb+hpos+ (horizontal position) and \verb+vpos+ (vertical position).
Possible values for \verb+hpos+ are \verb+l+ (left, default) and
\verb+right+ (right). Possible values for \verb+vpos+ are \verb+t+
(top, default), \verb+c+ (center), and \verb+b+ (bottom).

There are several parameters to control the size and shape of the
gate symbol (see Section~\ref{parameters}). These are
\verb+\gatesymbolradius+ (sets the radius of the gate symbol),
\verb+\labeldist+ (the vertical distance between the gate name and
the gate).

The next example shows a number of gates.

\medskip

\begin{minipage}[c]{0.4\linewidth}
\begin{msc}{gates}
\setmscvalues{small}

\declinst{i}{}{i}
\declinst{j}{}{j}
\declinst{k}{}{k}
\referencestart{r}{ref}{i}{j}
\nextlevel
\gate{$g$}{rright}
\mess{b}{rright}{envright}
\gate[r][c]{$h$}{envright}
\nextlevel
\nextlevel
\mess{c}{rright}{k}
\gate*[l][b]{$h'$}{rright}
\nextlevel[2]
\referenceend{r}
\end{msc}
\end{minipage}
%
\begin{minipage}[c]{0.5\linewidth}
{\footnotesize
\begin{verbatim}
\declinst{i}{}{i}
\declinst{j}{}{j}
\declinst{k}{}{k}
\referencestart{r}{ref}{i}{j}
\nextlevel
\gate{$g$}{rright}
\mess{b}{rright}{envright}
\gate[r][c]{$h$}{envright}
\nextlevel
\nextlevel
\mess{c}{rright}{k}
\gate*[l][b]{$h'$}{rright}
\nextlevel[2]
\referenceend{r}
\end{verbatim}
}
\end{minipage}


\subsection{High-level \MSC{}s}
\label{hmsc}
A High-level \MSC{} (\HMSC{}) is a drawing which defines the relation between 
a number of \MSC{}s. It is composed of a start symbol (an upside down
triangle), a number of end symbols (represented by triangles), a
number of \MSC{} references (these are rectangles with rounded corners),
a number of conditions (hexagons) and possibly several connection
points (circles). These symbols are connected by arrows.

The following commands can be used to draw \HMSC{}s.

\begin{verbatim}
\begin{hmsc}[headerpos]{hmscname}(llx,lly)(urx,ury)
\end{hmsc}
\hmscstartsymbol{nickname}(x,y)
\hmscendsymbol{nickname}(x,y)
\hmscreference{nickname}{text}(x,y)
\hmsccondition{nickname}{text}(x,y)
\hmscconnection{nickname}(x,y)
\arrow{from-nickname}[coord-list]{to-nickname}
\end{verbatim}

In order to draw \HMSC{}s, a new environment is defined, which is called
\verb+hmsc+. The command to begin this environment has several
arguments. The argument \verb+headerpos+ is optional. It controls
positioning of the header of the \HMSC{}.  This argument can have values
\verb+l+ (for a left aligned header), \verb+c+ (for a centered header)
and \verb+r+ (for a right aligned header).  The header of an \HMSC{} is
formed from the keyword \verb+msc+, followed by the
\verb+hmscname+. The size of the \HMSC{} frame is determined by the
coordinates of the lower-left corner, \verb+(llx,lly)+, and the
coordinates of the upper-right corner, \verb+(urx,ury)+.

The \HMSC{} grid is not drawn, but used to control the positioning of the
\HMSC{} symbols (\verb+startsymbol+, \verb+endsymbol+, \verb+reference+,
\verb+condition+, and \verb+connection+).
The center of each symbol is drawn on the grid point
with coordinates \verb+{x,y}+. Each symbol also has a \verb+nickname+
for later reference.

\HMSC{} symbols can be connected by means of the \verb+arrow+ command.
This draws an arrow from the symbol with nickname \verb+from-nickname+
to the symbol with nickname \verb+to-nickname+. The optional argument
\verb+coord-list+ can be used if the line connecting the source and
the destination should not be straight. The \verb+coord-list+ has
the following syntax: \verb+(x1,y1)(x2,y2)...(xk,yk)+. This means
that the connecting line goes through the points with coordinates
(x1,y1), (x2,y2), \ldots, (xk,yk).

Arrows always leave the start symbol at the bottom. They enter the end symbol
at the top. 
Arrows start and end either at the middle of the top or at the middle
of the bottom of a reference and condition symbol. The incoming
(outgoing) direction of the arrow determines whether it will start 
(end) at the top or at the bottom.

There are several parameters to control the size and shape of the
symbols (see Section~\ref{parameters}). These are
\verb+\hmscconditionheight+ (the height of the condition symbol),
\verb+\hmscconditionwidth+ (the width of the condition symbol,
excluding the left and right angular parts),
\verb+\hmscreferenceheight+ (the height of the reference symbol),
\verb+\hmscreferencewidth+ (the width of the reference symbol),
\verb+\messarrowscale{size}+ (a command to set the size of the arrow
head of a connection line);
\verb+setconnectiontype(type)+ (set the shape of the polyline
connection the symbols; \verb+type+ can be \verb+straight+,
\verb+rounded+, and \verb+curved+),
\verb+\startsymbolwidth+ (the width of the start and end symbol),
\verb+\topnamedist+ (sets the distance between the top of the \HMSC{} frame
and the \HMSC{} header).

An example of an \HMSC{} is in the following diagram. Notice that the
width and height of reference symbols are changed locally (i.e.,
between \verb+{+ and \verb+}+ braces) just before the big
reference~\verb+b+ is defined.

\medskip

\noindent
\begin{minipage}[c]{0.375\linewidth}

\begin{hmsc}{High level}(-3,0)(3,12)
\hmscstartsymbol{s}(0,10)
\hmscreference{a}{A}(0,9)
\hmscconnection{c}(0,7.5)
{
\setlength{\hmscreferencewidth}{2cm}
\setlength{\hmscreferenceheight}{3\baselineskip}
\hmscreference{b}{\parbox{1.9cm}
  {\centering 
   this is a big 
   hmsc reference}
  }(0,6)
}
\hmsccondition{t}{Again}(1,4.25)
\hmsccondition{ok}{Ok}(0,3.5)
\hmsccondition{q}{\textit{Break}}(-2,3.5)
\hmscendsymbol{e1}(-2,2)
\hmscreference{do}{Do it!}(0,2)
\hmscendsymbol{e2}(0,1)

\arrow{s}{a}
\arrow{a}{c}
\arrow{c}{b}
\arrow{c}[(-2,7.5)]{q}
\arrow{b}{q}
\arrow{q}{e1}
\arrow{b}{ok}
\arrow{ok}{do}
\arrow{do}{e2}
\arrow{b}{t}
\arrow{t}[(1,3.5)(2.5,3.5)(2.5,7.5)]{c}
\end{hmsc}
\end{minipage}\hfill
\begin{minipage}[c]{0.6\linewidth}
{\footnotesize
\begin{verbatim}
\begin{hmsc}{High level}(-3,0)(3,12)
\hmscstartsymbol{s}(0,10)
\hmscreference{a}{A}(0,9)
\hmscconnection{c}(0,7.5)
{\setlength{\hmscreferencewidth}{2cm}
 \setlength{\hmscreferenceheight}
           {3\baselineskip}
 \hmscreference{b}{\parbox{1.9cm}
   {\centering this is a big hmsc reference}
   }(0,6)}
\hmsccondition{t}{Again}(1,4.5)
\hmsccondition{ok}{Ok}(0,3.5)
\hmsccondition{q}{\textit{Break}}(-2,3.5)
\hmscendsymbol{e1}(-2,2)
\hmscreference{do}{Do it!}(0,2)
\hmscendsymbol{e2}(0,1)
\arrow{s}{a}
\arrow{a}{c}
\arrow{c}{b}
\arrow{c}[(-2,7.5)]{q}
\arrow{b}{q}
\arrow{q}{e1}
\arrow{b}{ok}
\arrow{ok}{do}
\arrow{do}{e2}
\arrow{b}{t}
\arrow{t}[(1,3.5)(2.5,3.5)(2.5,7.5)]{c}
\end{hmsc}
\end{verbatim}
}
\end{minipage}


\subsection{\MSC{} documents}
\label{mscdoc}
An \MSCdoc{} is a drawing which contains various declarations
of objects used in the \MSC{} description. For drawing \MSCdoc{}s
he following commands are provided.

\begin{verbatim}
\begin{mscdoc}[headerpos]{mscdocname}{text}(llx,lly)(urx,ury)
\end{mscdoc}
\reference{text}(x,y)
\separator{y}
\end{verbatim}

As for \MSC{} and \HMSC{}, a new environment is defined, which is named
\verb+mscdoc+. The command to begin an \MSCdoc{} has several
arguments. The argument \verb+headerpos+ is optional. It controls
positioning of the header of the \MSCdoc{}.
This argument can have values
\verb+l+ (for a left aligned header), \verb+c+ (for a centered header)
and \verb+r+ (for a right aligned header).
The header of an \MSCdoc{} is formed from the keyword \verb+mscdocument+,
followed by the \verb+mscdocname+.
The \verb+text+ is placed left-aligned below the \MSCdoc{} header.
The size of the \MSCdoc{} frame is determined by
coordinates of the lower-left corner, \verb+(llx,lly)+, and the
coordinates of the upper-right corner, \verb+(urx,ury)+.

The \MSCdoc{} grid is not drawn, but used to control the positioning of the
\MSC{} references.
The center of such a reference is drawn on the grid point
with coordinates \verb+{x,y}+.

The \verb+separator+ command draws a dashed horizontal line. The \MSC{}
references above the separator are the exported, while the ones below
the separator are local.

There are several parameters to control the size and shape of the
symbols (see Section~\ref{parameters}).
\verb+\mscdocreferenceheight+ (the height of the reference symbol),
\verb+\mscdocreferencewidth+ (the width of the reference symbol),
\verb+\topnamedist+ (sets the distance between the top of the \MSCdoc{}
frame and the \MSCdoc{} header).

An example of an \MSCdoc{} is in the following diagram. Notice that
the size of references in an \MSCdoc{} had to be changed for the last
reference.

\medskip


\begin{minipage}{0.4\linewidth}
\begin{mscdoc}{My declarations}%
              (0,0)(6,8)
\reference{a}(1,6)
\reference{b}(3,6)
\reference{c}(1,4)
\reference{d}(3,4)
\separator{3}
\setlength{\mscdocreferencewidth}{4.5\mscunit}
\setlength{\mscdocreferenceheight}{2\baselineskip}
\reference{%
 \parbox{4\mscunit}%
 {\raggedright This is a
  two-line reference}}
 (3,1.5)
\end{mscdoc}
\end{minipage}
%
\begin{minipage}{0.5\linewidth}
{\small
\begin{verbatim}
\begin{mscdoc}{My declarations}%
              (0,0)(6,8)
\reference{a}(1,6)
\reference{b}(3,6)
\reference{c}(1,4)
\reference{d}(3,4)
\separator{3}
\setlength{\mscdocreferencewidth}
          {4.5\mscunit}
\setlength{\mscdocreferenceheight}
          {2\baselineskip}
\reference{%
 \parbox{4\mscunit}%
 {\raggedright This is a
  two-line reference}}
 (3,1.5)
\end{mscdoc}
\end{verbatim}
}
\end{minipage}



\section{Style parameters}
\label{parameters}

By means of a collection of parameters, the graphical appearance of
an \MSC{} can be fine tuned to the user's taste.
The general parameters
are displayed in Figure~\ref{parametersfig} on page~\pageref{parametersfig}.

%========================
\begin{figure}[htb]
\vspace{2ex} %Needed to get enough white space before the top label of
             %the figure.

\newcommand{\printlength}[1]{\mbox{\footnotesize\ttfamily \char`\\#1}}
\begin{center}
\setmscvalues{normal}
\begin{msc}{Lengths}
\psset{linewidth=.4pt,dotsep=1pt}
\newlength{\lta}
\newlength{\ltb}
% \topnamedist
\psline[linestyle=dotted](1.5cm,-\topnamedist)(2.5cm,-\topnamedist)
\psline[arrowscale=1.3]{->}(2.5cm,.2cm)(2.5cm,0cm)
\setlength{\lta}{\topnamedist+.2cm}
\psline[arrowscale=1.3]{->}(2.5cm,-\lta)(2.5cm,-\topnamedist)
\rput[tl](2.5cm,.4cm){\printlength{topnamedist}}

% \leftnamedist
\setlength{\lta}{\leftnamedist+10pt}
\setlength{\ltb}{\topnamedist}
\psline[linestyle=dotted,arrowscale=1.0]{->}(-10pt,-\ltb)(0,-\ltb)
\psline[linestyle=dotted,arrowscale=1.0]{->}(\lta,-\ltb)(\leftnamedist,-\ltb)
\psline[linestyle=dotted](\leftnamedist,10pt)(\leftnamedist,-\baselineskip)
\rput[br](-10pt,-\ltb){\printlength{leftnamedist}}

% \topheaddist
\setlength{\lta}{\envinstdist+\instdist-.5\instwidth}
\psline[arrowscale=1.3,linestyle=dotted]{<->}(\lta,0cm)(\lta,-\topheaddist)
\rput[l](\lta,-.5\topheaddist){ \printlength{topheaddist}}

% \bottomfootdist
\setlength{\ltb}{\topheaddist+\instheadheight+\firstlevelheight+\lastlevelheight+\instfootheight+\levelheight*12}
\setlength{\lta}{\envinstdist+2\instdist-.5\instwidth}
\rput[tl](\lta,-\ltb){\psline[arrowscale=1.3,linestyle=dotted]{<->}(0,0)(0,-\bottomfootdist)}
\setlength{\ltb}{\ltb+.5\bottomfootdist}
\rput[l](\lta,-\ltb){\printlength{bottomfootdist}}

% \instheadheight
\setlength{\lta}{\envinstdist+2\instdist+.5\instwidth+.2cm}
\setlength{\ltb}{\topheaddist}
\rput[t](\lta,-\ltb){\psline[arrowscale=1.3,linestyle=dotted]{<->}(0,0)(0,-\instheadheight)}
\setlength{\ltb}{\ltb+.5\instheadheight}
\rput[l](\lta,-\ltb){\printlength{instheadheight}}

% \firstlevelheight
\setlength{\lta}{\envinstdist+\instdist+.2cm}
\setlength{\ltb}{\topheaddist+\instheadheight}
\rput(\lta,-\ltb){\psline[arrowscale=1.3,linestyle=dotted]{<->}(0,-\firstlevelheight)}
\setlength{\ltb}{\ltb+.5\firstlevelheight}
\rput[l](\lta,-\ltb){\printlength{firstlevelheight}}

% \levelheight
\setlength{\lta}{\envinstdist+\instdist+.2cm}
\setlength{\ltb}{\topheaddist+\instheadheight+\firstlevelheight}
\rput[t](\lta,-\ltb){\psline[arrowscale=1.3,linestyle=dotted]{<->}(0,0)(0,-\levelheight)}
\setlength{\ltb}{\ltb+.5\levelheight}
\rput[l](\lta,-\ltb){\printlength{levelheight}}

% \actionheight
\setlength{\lta}{\envinstdist+2\instdist+0.5\actionwidth+.2cm}
\setlength{\ltb}{\topheaddist+\instheadheight+\firstlevelheight+\levelheight*7}
\rput[t](\lta,-\ltb){\psline[arrowscale=1.3,linestyle=dotted]{<->}(0,0)(0,-\actionheight)}
\setlength{\ltb}{\ltb+.7\actionheight}
\rput[bl](\lta,-\ltb){\printlength{actionheight}}

% \actionwidth
\setlength{\lta}{\envinstdist+2\instdist-0.5\actionwidth}
\setlength{\ltb}{\topheaddist+\instheadheight+\firstlevelheight+\levelheight*7-.2cm}
\rput(\lta,-\ltb){\psline[arrowscale=1.3,linestyle=dotted]{<->}(0,0)(\actionwidth,0)}
\setlength{\lta}{\lta+.5\actionwidth}
\setlength{\ltb}{\ltb-.1cm}
\rput[bl](\lta,-\ltb){\rule{0pt}{2ex}\printlength{actionwidth}}

% \conditionheight
\setlength{\lta}{\envinstdist+2\instdist+\conditionoverlap+0.8\conditionheight}
\setlength{\ltb}{\topheaddist+\instheadheight+\firstlevelheight+\levelheight*10}
\rput[t](\lta,-\ltb){\psline[arrowscale=1.3,linestyle=dotted]{<->}(0,0)(0,-\conditionheight)}
\setlength{\ltb}{\ltb+.7\conditionheight}
\rput[bl](\lta,-\ltb){\printlength{conditionheight}}

% \inlineoverlap
\setlength{\lta}{\envinstdist}
\setlength{\ltb}{\topheaddist+\instheadheight+\firstlevelheight+\levelheight*5-.2cm}
\rput(\lta,-\ltb){\psline[arrowscale=1.3,linestyle=dotted]{<->}(0,0)(\inlineoverlap,0)}
%\setlength{\lta}{\lta+\inlineoverlap}
\setlength{\ltb}{\ltb-.1cm}
\rput[bl](\lta,-\ltb){\rule{0pt}{2ex}\printlength{inlineoverlap}}

% \referenceoverlap
\setlength{\lta}{\envinstdist}
\setlength{\ltb}{\topheaddist+\instheadheight+\firstlevelheight+\levelheight*7-.2cm}
\rput(\lta,-\ltb){\psline[arrowscale=1.3,linestyle=dotted]{<->}(0,0)(\referenceoverlap,0)}
%\setlength{\lta}{\lta+\referenceoverlap}
\setlength{\ltb}{\ltb-.1cm}
\rput[bl](\lta,-\ltb){\rule{0pt}{2ex}\printlength{referenceoverlap}}

% \conditionoverlap
\setlength{\lta}{\envinstdist+2\instdist}
\setlength{\ltb}{\topheaddist+\instheadheight+\firstlevelheight+\levelheight*10-.2cm}
\rput(\lta,-\ltb){\psline[arrowscale=1.3,linestyle=dotted]{<->}(0,0)(\conditionoverlap,0)}
%\setlength{\lta}{\lta+\conditionoverlap}
\setlength{\ltb}{\ltb-.1cm}
\rput[bl](\lta,-\ltb){\rule{0pt}{2ex}\printlength{conditionoverlap}}

% \lastlevelheight
\setlength{\lta}{\envinstdist+2\instdist+.2cm}
\setlength{\ltb}{\topheaddist+\instheadheight+\firstlevelheight+\levelheight*12}
\rput[t](\lta,-\ltb){\psline[arrowscale=1.3,linestyle=dotted]{<->}(0,0)(0,-\lastlevelheight)}
%\setlength{\ltb}{\ltb+.5\lastlevelheight}
\rput[bl](\lta,-\ltb){\printlength{lastlevelheight}}

% \instfootheight
\setlength{\lta}{\envinstdist+2\instdist+.5\instwidth+.1cm}
\setlength{\ltb}{\topheaddist+\instheadheight+\firstlevelheight+\levelheight*12+\lastlevelheight}
\rput[t](\lta,-\ltb){\psline[arrowscale=1.3]{->}(0,.3cm)(0,0)}
\setlength{\ltb}{\ltb+\instfootheight}
\rput[t](\lta,-\ltb){\psline[arrowscale=1.3]{->}(0,-.3cm)(0,0)}
\rput[tl](\lta,-\ltb){ \printlength{instfootheight}}

% \selfmesswidth
\setlength{\lta}{\envinstdist-\selfmesswidth}
\setlength{\ltb}{\topheaddist+\instheadheight+\firstlevelheight+\levelheight*2-.2cm}
\rput(\lta,-\ltb){\psline[arrowscale=1.3,linestyle=dotted]{<->}(0,0)(\selfmesswidth,0)}
%\setlength{\lta}{\lta+\selfmesswidth}
\rput[r](\lta,-\ltb){\printlength{selfmesswidth}}

% \regionbarwidth
\setlength{\lta}{\envinstdist+2\instdist-.5\regionbarwidth}
\setlength{\ltb}{\topheaddist+\instheadheight+\firstlevelheight+\levelheight*2-.2cm}
\rput(\lta,-\ltb){\psline[arrowscale=1.3,linestyle=dotted]{<->}(0,0)(\regionbarwidth,0)}
\setlength{\lta}{\lta+\regionbarwidth}
\rput[l](\lta,-\ltb){\printlength{regionbarwidth}}

% \instwidth
\setlength{\ltb}{\topheaddist+\instheadheight+\firstlevelheight+\lastlevelheight+\instfootheight+\levelheight*12+.2cm}
\setlength{\lta}{\envinstdist-0.5\instwidth}
\rput(\lta,-\ltb){\psline[arrowscale=1.3,linestyle=dotted]{<->}(0,0)(\instwidth,0)}
\rput[tl](\lta,-\ltb){\rule{0pt}{2ex}\printlength{instwidth}}

% \instdist
\setlength{\lta}{\envinstdist}
\setlength{\ltb}{\topheaddist+\instheadheight+\firstlevelheight+2.5\levelheight}
\rput[l](\lta,-\ltb){\psline[arrowscale=1.3,linestyle=dotted]{<->}(0,0)(\instdist,0)}
\setlength{\lta}{\lta+.5\instdist}
\setlength{\ltb}{\ltb-.1cm}
\rput[b](\lta,-\ltb){\printlength{instdist}}

% \envinstwidth
\setlength{\lta}{0cm}
\setlength{\ltb}{\topheaddist+\instheadheight+\firstlevelheight+.5\levelheight}
\rput[l](\lta,-\ltb){\psline[arrowscale=1.3,linestyle=dotted]{<->}(0,0)(\envinstdist,0)}
\rput[tl](\lta,-\ltb){ \printlength{envinstdist}}


% \labeldist
\setlength{\lta}{\envinstdist-.5\instwidth}
\setlength{\ltb}{\topheaddist-\labeldist}
\rput[l](\lta,-\ltb){\psline[arrowscale=1.3,linestyle=dotted](-.2cm,0)(.5cm,0)}
\rput[b](\lta,-\ltb){\psline[arrowscale=1.3]{->}(-.1cm,.3cm)(-.1cm,0)}
\setlength{\ltb}{\topheaddist}
\rput[l](\lta,-\ltb){\psline[arrowscale=1.3,linestyle=dotted](-.2cm,0)(.5cm,0)}
\rput[b](\lta,-\ltb){\psline[arrowscale=1.3]{->}(-.1cm,-.3cm)(-.1cm,0)}
\rput[tr](\lta,-\ltb){\printlength{labeldist} }

% \timerwidth
\setlength{\lta}{\envinstdist-\selfmesswidth-0.5\timerwidth}
\setlength{\ltb}{\topheaddist+\instheadheight+\firstlevelheight+\levelheight*4+.4cm}
\rput(\lta,-\ltb){\psline[arrowscale=1.3,linestyle=dotted]{<->}(0,0)(\timerwidth,0)}
%\setlength{\lta}{\lta+\selfmesswidth}
\rput[r](\lta,-\ltb){\printlength{timerwidth}}

% \stopwidth
\setlength{\lta}{\envinstdist+\instdist-0.5\stopwidth}
\setlength{\ltb}{\topheaddist+\instheadheight+\firstlevelheight+\levelheight*4+0.5\stopwidth+.1cm}
\rput(\lta,-\ltb){\psline[arrowscale=1.3,linestyle=dotted]{<->}(0,0)(\stopwidth,0)}
\setlength{\lta}{\lta+\stopwidth}
\rput[l](\lta,-\ltb){\printlength{stopwidth}}

% \lostsymbolradius
\setlength{\lta}{\envinstdist+2\instdist+\selfmesswidth+2\lostsymbolradius+0.1cm}
\setlength{\ltb}{\topheaddist+\instheadheight+\firstlevelheight+\levelheight*4}
\rput[t](\lta,-\ltb){\psline[arrowscale=1.3]{->}(0,.3cm)(0,0)}
\setlength{\ltb}{\ltb+\lostsymbolradius}
\rput[t](\lta,-\ltb){\psline[arrowscale=1.3]{->}(0,-.3cm)(0,0)}
\rput[tl](\lta,-\ltb){ \printlength{lostsymbolradius}}

\psset{linewidth=.8pt,dotsep=3pt}

\declinst{m1}{aname}{iname}
\declinst{st}{}{}
\declinst{m2}{}{}

\mess{msg1}{m1}{st}
\nextlevel
\mess{msg2}{m1}{st}
\nextlevel
\mess{self}{m1}{m1}
\coregionstart{m2}
\nextlevel
\coregionend{m2}
\nextlevel
\lost[r]{x}{}{m2}
\stop{st}
\settimer{T}{m1}
\nextlevel
\inlinestart{alt}{alt}{m1}{m1}
\nextlevel[2]
\inlineend{alt}
\action{a}{m2}
\nextlevel
\referencestart{ref}{reference}{m1}{m1}
\nextlevel
\referenceend{ref}
\nextlevel
\condition{cond1}{m1,m2}
\nextlevel
\nextlevel
\mess{msg3}{m2}{m1}
\end{msc}
\end{center}


\caption{User controllable parameters}
\label{parametersfig}
\end{figure}

The value of a parameter can be changed using standard
\LaTeX\ commands, e.g.
\begin{verbatim}
\setlength{\levelheight}{1cm}
\end{verbatim}

The following list describes all parameters.
The default values for drawing \MSC{}s at large, normal and small size are
included. See the command \verb+\setmscvalues{size}+ below for
restoring the parameters to their original values.

%========================
\begin{defs}

\item[\cmd{actionheight}]
Height of action symbols.\\
(\lnsvalue{0.75}{0.6}{0.5} cm.)

\item[\cmd{actionwidth}]
Width of action symbol.\\
(\lnsvalue{1.25}{1.25}{1.2} cm.)

\item[\cmd{bottomfootdist}]
Distance between bottom of foot symbol and frame.\\
(\lnsvalue{1.0}{0.7}{0.5} cm.)

\item[\cmd{commentdist}]
Distance between a comment and its instance.\\
(\lnsvalue{0.5}{0.5}{0.5} cm.)

\item[\cmd{conditionheight}]
Height of condition symbols.\\
(\lnsvalue{0.75}{0.6}{0.5} cm.)

\item[\cmd{conditionoverlap}]
Overlap of condition symbol.\\
(\lnsvalue{0.6}{0.5}{0.4} cm.)

\item[\cmd{envinstdist}]
Distance between environments and nearest instance line.\\
(\lnsvalue{2.5}{2.0}{1.2} cm.)

\item[\cmd{firstlevelheight}] Height of level just below head
symbols. Should not be changed inside the \MSC{} environment.\\
(\lnsvalue{0.75}{0.6}{0.4} cm.)

\item[\cmd{hmscconditionheight}]
Height of \HMSC{} condition symbol.\\
(\lnsvalue{0.375}{0.3}{0.25} cm.)

\item[\cmd{hmscconditionwidth}]
Width of \HMSC{} condition symbol.\\
(\lnsvalue{1.0}{0.8}{0.7} cm.)

\item[\cmd{hmscconnectionradius}]
Radius of \HMSC{} connection symbol.\\
(\lnsvalue{0.06}{0.05}{0.04} cm.)

\item[\cmd{hmscreferenceheight}]
Height of \HMSC{} reference symbol.\\
(\lnsvalue{0.375}{0.3}{0.25} cm.)

\item[\cmd{hmscreferencewidth}]
Width of \HMSC{} reference symbol.\\
(\lnsvalue{1.0}{0.8}{0.7} cm.)

\item[\cmd{hmscstartsymbolwidth}]
Width of \HMSC{} start symbol.\\
(\lnsvalue{0.75}{0.6}{0.3} cm.)

\item[\cmd{inlineoverlap}]
Overlap of inline symbol.\\
(\lnsvalue{1.5}{1.0}{0.75} cm.)

\item[\cmd{instbarwidth}]
Default width of vertical instance bars (applies to fat instances only).\\
(\lnsvalue{0.0}{0.0}{0.0} cm.)

\item[\cmd{instdist}]
Distance between instance axes.\\
(\lnsvalue{3.0}{2.2}{1.5} cm.)

\item[\cmd{instfootheight}] Height of foot symbols. Should not be
changed inside the \MSC{} environment.\\
(\lnsvalue{0.25}{0.2}{0.15} cm.)

\item[\cmd{instheadheight}] Height of head symbols. Should not be
changed inside the \MSC{} environment.\\
(\lnsvalue{0.6}{0.55}{0.5} cm.)

\item[\cmd{instwidth}]
Width of header and foot symbols.\\
(\lnsvalue{1.75}{1.6}{1.2} cm.)

\item[\cmd{labeldist}]
Distance between labels and the symbols to which they belong (for instance, message labels and arrows).\\
(\lnsvalue{1.0}{1.0}{1.0} ex.)

\item[\cmd{lastlevelheight}] Height of level just above foot
symbols. Should not be changed inside the \MSC{} environment.\\
(\lnsvalue{0.5}{0.4}{0.3} cm.)

\item[\cmd{leftnamedist}] Distance between left of the frame and
(left of) \MSC, \HMSC, or \MSCdoc{} title.\\
(\lnsvalue{0.3}{0.2}{0.1} cm.)

\item[\cmd{levelheight}]
Height of a level.\\
(\lnsvalue{0.75}{0.5}{0.4} cm.)

\item[\cmd{lostsymbolradius}]
Radius of the lost and found symbols.\\
(\lnsvalue{0.15}{0.12}{0.08} cm.)

\item[\cmd{markdist}]
Horizontal distance from a mark to its instance.\\
(\lnsvalue{1.0}{1.0}{1.0} cm.)

\item[\cmd{measuredist}]
Horizontal distance from a measure to its (closest) instance.\\
(\lnsvalue{1.0}{1.0}{1.0} cm.)

\item[\cmd{measuresymbolwidth}]
Width of a measure symbol.\\
(\lnsvalue{0.75}{0.6}{0.4} cm.)

\item[\cmd{mscdocreferenceheight}]
Height of reference symbol in an \MSCdoc.\\
(\lnsvalue{0.375}{0.3}{0.25} cm.)

\item[\cmd{mscdocreferencewidth}]
Width of reference symbol in an \MSCdoc.\\
(\lnsvalue{1.0}{0.8}{0.7} cm.)

\item[\cmd{referenceoverlap}]
Overlap of reference symbol.\\
(\lnsvalue{1.5}{1.0}{0.75} cm.)

\item[\cmd{regionbarwidth}]
Width of region bars.\\
(\lnsvalue{0.5}{0.4}{0.2} cm.)

\item[\cmd{selfmesswidth}]
Length of horizontal arms of self messages.\\
(\lnsvalue{0.75}{0.6}{0.4} cm.)

\item[\cmd{stopwidth}]
Width of the stop symbol.\\
(\lnsvalue{0.6}{0.5}{0.3} cm.)

\item[\cmd{timerwidth}]
Width of the \emph{timer} symbols.\\
(\lnsvalue{0.4}{0.3}{0.2} cm.)

\item[\cmd{topheaddist}]
Distance between top of head symbols and frame.\\
(\lnsvalue{1.5}{1.3}{1.2} cm.)

\item[\cmd{topnamedist}] Distance between top of the frame and
(top of) \MSC, \HMSC, or \MSCdoc{} title.\\
(\lnsvalue{0.3}{0.2}{0.2} cm.)

\end{defs}

In addition there are several commands which allow the user to adjust
the \MSC{} drawing to his own taste.

\begin{defs}

\item[\cmd{messarrowscale}\{\emph{scalefactor}\}] Sets the scale
factor (a positive real number) of message arrow
heads. (\lnsvalue{2}{1.5}{1.2})

\item[\cmd{setmscscale}\{\emph{scalefactor}\}] Sets the scale factor
of the \MSC{} environment to \emph{scalefactor}. the scale factor is
supposed to be a real number. Scaling is done when the \MSC{}
environment ends (\verb|\end{msc}|). The default of \emph{scalefactor}
is~1. A more consistent way for
varying the size of the \MSC{} can be obtained by using the
\verb+\setmesvalues+ command as described below.
(default value 1.)

\item[\cmd{psset}\{\texttt{linewidth=D}\}] This command sets the width
of all lines in \MSC{}s, \HMSC{}s, and \MSCdoc{}s to length \verb+D+.  If
this command is issued outside the msc environment, then the value is
set for the complete document. If it is used directly after the start
of the msc environment it only holds for this \MSC{}.\\
(large/normal/small value 0.8/0.7/0.6 pt.)

\item[\cmd{setfootcolor}\{\emph{color}\}] Sets the color of the foot
symbols of \MSC{} instances. Possible values are \emph{black},
\emph{white}, \emph{gray}, or \emph{lightgray}. For more color values,
see the documentation of the \LaTeXe{} \textsf{color} package.

\end{defs}

The following command can be used to set the above mentioned
style parameters to suitable values.

\begin{defs}
\item[\cmd{setmscvalues}\{\emph{size}\}] Sets all parameters of the \mscpack{} to 
 predefined values. Valid values for \emph{size} are:
\verb|small|, \verb|normal|, and \verb|large|. 
(The default value of \verb+size+ is \verb+normal+.
This can be used for drawings with at maximum six instances on a
sheet of A4 paper.
For sizes \verb+large+ and \verb+small+, a maximum of four and nine
instances respectively fit on a sheet of A4 paper.)
\end{defs}

Caution has to be taken when changing the value of a parameter within
the \MSC{} definition. The following parameters can be changed within an
\MSC{} definition without unexpected side effects:

\begin{flushleft}
\verb+\actionheight+,
\verb+\actionwidth+,
\verb+\bottomfootdist+,
\verb+\conditionheight+,
\verb+\conditionoverlap+,
\verb+\inlineoverlap+,
\verb+\instfootheight+,
\verb+\instwidth+ (however, this may cause different sizes of
corresponding instance header and footer symbols),
\verb+\labeldist+,
\verb+\lastlevelheight+,
\verb+\levelheight+,
\verb+\lostsymbolradius+,
\verb+\referenceoverlap+,
\verb+\regionbarwidth+,
\verb+\selfmesswidth+,
\verb+\stopwidth+,
\verb+\timerwidth+, and
\verb+\topnamedist+.
\end{flushleft}


In addition to the parameters specific to the \mscpack,
standard \LaTeX\ commands can be used to change the type style and
other details.
For example, if the command \verb+\sffamily+ is included directly after 
the start of the msc environment, the text in the \MSC{}
is drawn using a {\sf sans serif} font. Likewise, the text size can be
changed by inserting, e.g., the command \verb+\small+.
The \verb+\raisebox+ and \verb+\parbox+ commands
can also be used to position and format names.


\section{Example}
\label{example}
Figure~\ref{ex} on page~\pageref{ex} shows the \MSC{} defined in the
following \LaTeX\ fragment.

{\small
\begin{verbatim}
\begin{msc}{Example}
\declinst{usr}{The user}{User}
\declinst{m1}{Control}{M1}
\dummyinst{m2}
\declinst{m3}{Another Machine}{M3}

\create{start}{m1}{m2}{Processing}{M2}
\mess{msg 0}{envleft}{usr}
\mess{msg 1}{envright}{m2}[1]
\nextlevel

\mess{msg 2}{usr}{m1}
\order{m1}{m2}[4]
\action{a}{m3}

\nextlevel
\found{msg x}{}{usr}
\nextlevel

\mess{msg 3}{usr}{m2}[-1]
\coregionstart{m1}
\settimeout{S}{m3}[2]
\nextlevel


\mess{msg 4}{m1}{usr}
\coregionstart{m2}
\settimer[r]{T}{m3}
\nextlevel

\mess[r]{msg 5}{m2}{m2}[3]
\mess{msg 6}{usr}{usr}[2]
\nextlevel

\mess{msg 7}{m2}{usr}
\timeout[r]{T}{m3}
\nextlevel

\coregionend{m2}
\nextlevel

\coregionend{m1}
\stoptimer[r]{T'}{m3}
\nextlevel

\lost[r]{msg y}{Mach 1}{usr}
\mess{msg 8}{m1}{envright}
\nextlevel

\condition{condition 1}{usr,m2}
\setstoptimer[r]{U}{m3}
\nextlevel[2]
\stop{usr}

\end{msc}
\end{verbatim}
}

\begin{figure}[htb]
\begin{center}
\begin{msc}{Example}
\declinst{usr}{The user}{User}
\declinst{m1}{Control}{M1}
\dummyinst{m2}
\declinst{m3}{Another Machine}{M3}

\create{start}{m1}{m2}{Processing}{M2}
\mess{msg 0}{envleft}{usr}
\mess{msg 1}{envright}{m2}[1]
\nextlevel

\mess{msg 2}{usr}{m1}
\order{m1}{m2}[4]
\action{a}{m3}

\nextlevel
\found{msg x}{}{usr}
\nextlevel

\mess{msg 3}{usr}{m2}[-1]
\coregionstart{m1}
\settimeout{S}{m3}[2]
\nextlevel


\mess{msg 4}{m1}{usr}
\coregionstart{m2}
\settimer[r]{T}{m3}
\nextlevel

\mess[r]{msg 5}{m2}{m2}[3]
\mess{msg 6}{usr}{usr}[2]
\nextlevel

\mess{msg 7}{m2}{usr}
\timeout[r]{T}{m3}
\nextlevel

\coregionend{m2}
\nextlevel

\coregionend{m1}
\stoptimer[r]{T'}{m3}
\nextlevel

\lost[r]{msg y}{Mach 1}{usr}
\mess{msg 8}{m1}{envright}
\nextlevel

\condition{condition 1}{usr,m2}
\setstoptimer[r]{U}{m3}
\nextlevel[2]
\stop{usr}

\end{msc}

\caption{A menagerie of \MSC{} symbols}
\label{ex}
\end{center}
\end{figure}


\section{Acknowledgments}
Thanks are due to the following people for providing us with useful
input: Peter Peters, Michel Reniers.

\bibliographystyle{plain}
\bibliography{biblio}

\end{document}
