%% directory.tex
%% Copyright 1998-2004 Christophe Geuzaine <geuz@geuz.org>
%
% $Id: directory.tex,v 1.32 2004/09/22 21:05:18 geuzaine Exp $
%
% This program can be redistributed and/or modified under the terms
% of the LaTeX Project Public License distributed from CTAN
% archives in directory macros/latex/base/lppl.txt; either
% version 1 of the License, or (at your option) any later version.
%
% 'directory' user's guide and example file
% 
% The parts you may want to customize are labeled with "CUSTOM:". The lines
% beginning with "%%tth:" are used by TTH (a LaTeX to HTML translator).

\documentclass[10pt]{article}

\usepackage[break,longdates]{directory} 

% CUSTOM: uncomment one of the following:
% \directorystyle{email}
% \directorystyle{phone}
% \directorystyle{birthday}
% \directorystyle{letter}
% \directorystyle{address-html}
% \directorystyle{email-html}
% \directorystyle{address-vcard}
% \directorystyle{address-ldif}
\directorystyle{address}

% CUSTOM: here is how I customize the directory for this user's guide:
\renewcommand{\dirsymbol}{\raisebox{1ex}{\tiny{[see \S\ref{sec:output}]}}}
\renewcommand{\Dirlabel}[1]{{#1}}
\renewcommand{\Dirheader}[1]{\item\hspace{-\dirindent}\textbf{\MakeUppercase{#1}}}

\ifx\pdfoutput\undefined\else
  \usepackage[colorlinks=true,urlcolor=blue]{hyperref}
  \newcommand\MyURL{\begingroup\Url}
  \renewcommand{\Diremail}[1]{\href{mailto:#1}{\MyURL{#1}}}
  \renewcommand{\Dirurl}[1]{\href{#1}{\MyURL{#1}}}
\fi
\def\baselinestretch{0.98}

\newcommand{\BibTeX}{\textsc{Bib}\TeX}
\newcommand{\Quote}[1]{`{#1}'}
\newcommand{\DbleQuote}[1]{``{#1}''}
\newcommand{\Download}{\footnote{\texttt{directory} is available at
  \url{http://www.geuz.org/directory/} or through the CTAN in the
  \url{biblio/bibtex/contrib/directory/} subdirectory. \texttt{directory}
  is distributed under the \LaTeX\ Project Public License (LPPL) since
  version 1.11.}}

%%tth: \def\Quote#1{'{#1}'}
%%tth: \def\DbleQuote#1{''{#1}''}
%%tth: \def\Download{\footnote{\texttt{directory} is distributed under the
%%tth: \LaTeX\ Project Public License (LPPL) since version 1.11.}}
%%tth: \def\LaTeX{LaTeX}
%%tth: \def\BibTeX{BibTeX}

\pagestyle{headings}

\begin{document}

\title{\texttt{directory}: address books with \LaTeX\ and \BibTeX}

% CUSTOM: try e.g. \author{\wdir{c.geuzaine}} with \directorystyle{letter}
\author{\dir{c.geuzaine}}

%%tth: \special{html:<H3 align=center>Christophe Geuzaine</H3>}

\date{Version 1.20, 22 September 2004}

\maketitle

%%tth: \section*{Download}
%%tth: The current distribution of \texttt{directory} is 
%%tth: \href{http://www.geuz.org/directory/src/directory-1.20.tgz}{directory-1.20.tgz}.
%%tth: This distribution contains both a
%%tth: \href{http://www.geuz.org/directory/directory.ps}{PostScript}
%%tth: and a \href{http://www.geuz.org/directory/directory.pdf}{PDF}
%%tth: version of the documentation. The separate files are also available
%%tth: through the \href{http://www.ctan.org/}{CTAN} sites in the 
%%tth: \href{http://www.ctan.org/tex-archive/biblio/bibtex/contrib/directory/}
%%tth: {biblio/bibtex/contrib/directory/} subdirectory. Older versions are
%%tth: still available \href{http://www.geuz.org/directory/src/}{here}.
%%tth: Any questions, comments or suggestions? Send me an 
%%tth: \href{mailto:geuz@geuz.org}{e-mail}. 

\tableofcontents

\section{Introduction}

\texttt{directory}\Download\ is a macro package for \LaTeX\ and \BibTeX\ that
facilitates the construction, the maintenance and the exploitation of an
address book like database. It consists of five \BibTeX\ style files
(\texttt{address.bst}, \texttt{phone.bst}, \texttt{email.bst},
\texttt{birthday.bst} and \texttt{letter.bst}) designed to be used in
conjunction with the \LaTeX\ style file \texttt{directory.sty}. Depending on
the bibliographical style used, the package has two main applications:

\begin{enumerate}
\item
the construction of a list of information (address, phone number, etc.)
about selected persons, companies or places;
\item
the inclusion of a selected piece of information concerning a person, a
place or a company at a desired location in a document.
\end{enumerate}

In the first case, \texttt{directory} behaves just like standard
bibliographical styles: while standard bibliographical styles handle data
concerning books, articles, proceedings, etc., \texttt{directory} handles
data relative to people, companies or places. For example, the name in the
title of this guide refers to the corresponding entry in the directory
listed in section~\ref{sec:output}. The first four \BibTeX\ files provide
several ways to handle this data.

In the second case, the package enables bits of the database to be put in your
document. It can for example be used to put the address of your correspondent
in the address field of a letter.  

\sloppypar
Four special \BibTeX\ style files (\texttt{address-html.bst},
\texttt{email-html.bst}, \texttt{address-vcard.bst} and
\texttt{address-ldif.bst}) are also provided for an easy generation of HTML,
vCard and LDIF versions of your directories.

\section{General description}

\subsection{Package inclusion and options}

The package is included by the usual \verb'\usepackage{directory}' command
at the top of the document. Four options are available:
\begin{description}
\item[\textmd{\texttt{break}:}] allows the directory fields to be broken
  across pages;
\item[\textmd{\texttt{german}:}] creates directories in German;
\item[\textmd{\texttt{french}:}] creates directories in French;
\item[\textmd{\texttt{longdates}:}] prints birthday dates using month names
  instead of numbers.
\end{description}
The old (before version 1.10) formatting options are now handled by the same
command mechanism as all other customization options (see
section~\ref{sec:custom}).

The directory is produced by the
\verb'\directory['\emph{extension}\verb']{'\emph{filename}\verb'}' command,
where \emph{filename} stands for the name of the \texttt{bib} file (without
the \texttt{bib} extension) and where the optional argument \emph{extension}
gives, if necessary, the extension of the file output by \BibTeX\ (see
section~\ref{sec:usingboth}). As usual, multiple \texttt{bib} files can be
included, a comma separating the different file names.

\subsection{Making a citation}

An entry is cited in the text by a \verb'\dir{'\emph{key}\verb'}',  
\verb'\pdir{'\emph{key}\verb'}', \verb'\rdir{'\emph{key}\verb'}' or
\verb'\wdir{'\emph{key}\verb'}' command, equivalent to the standard
\verb'\cite{'\emph{key}\verb'}' command, where \emph{key} is used in the same
way as ever (referring to an entry in a \texttt{bib} file). 
The differences between the four citation commands will be explained in the next
section.
A \verb'\nodir{'\emph{key}\verb'}' command exists and acts exactly like
\verb'\nocite{'\emph{key}\verb'}' for standard bibliographies.

\subsection{The \texttt{bst} styles}

The style of the directory is chosen by a \verb'\directorystyle{'\emph{style}\verb'}'  
command, where \emph{style} is one of the following:
\begin{description}
\item[\textmd{\texttt{address}:}] 
full listing in the directory of all fields corresponding to the
\emph{key} entry. The \verb'\dir{'\emph{key}\verb'}' command also prints the
\texttt{name} field of the entry in the document;  
\item[\textmd{\texttt{phone}:}] 
only the phone, cellular and fax fields corresponding to the cited entry are
displayed in the directory. The \verb'\dir{'\emph{key}\verb'}' command acts in
the same way as with the \texttt{address} style, except that the names are
abbreviated; 
\item[\textmd{\texttt{email}:}] 
only e-mail addresses are displayed in the directory. The
\verb'\dir{'\emph{key}\verb'}' command acts in the same way as with the
\texttt{address} style;
\item[\textmd{\texttt{birthday}:}] 
only the birthdays are displayed, sorting the entries in chronological
order. The \verb'\dir{'\emph{key}\verb'}' command acts in the same way as with
the \texttt{address} style;
\item[\textmd{\texttt{letter}:}] 
acts in a slightly different way than the four preceding styles. No directory
is produced with the \verb|\directory| command. The
\verb'\dir{'\emph{key}\verb'}' command results in the \texttt{name} field of
the \emph{key} entry to be printed in the document. The
\verb'\pdir{'\emph{key}\verb'}' (respectively \verb'\rdir{'\emph{key}\verb'}'
or \verb'\wdir{'\emph{key}\verb'}') command prints the name and the private
(respectively residence or work) address in the document in a tabulated way.
\end{description}


\subsection{The \texttt{bib} file fields}

\texttt{directory} defines three entry types: \verb'@person{}',
\verb'@company{}' and \verb'@place{}'. In all these types, \texttt{name} is
the only mandatory field, since it serves as a key for sorting the entries.
Here are all the available fields that can be defined for each entry:

\begin{verbatim}
@person{key,
  name = "Full name(s), in standard BibTeX format",
  nickname = "Nickname(s)",
  birthday = "Birthday date(s), in numeric 'day month' format",
  birthyear = "Birth year(s)",

  p.street = "Street of private residence",
  p.city = "City of private residence",
  p.zip = "ZIP code of private residence",
  p.state = "State of private residence",
  p.country = "Country of private residence",
  p.phone = "Private phone number",
  p.cellular = "Private mobile phone number",
  p.fax = "Private fax number",
  p.email = "Private e-mail address",
  p.url = "Private home page",
  p.account = "Private bank account",

  r.street = "Street of alternative residence",
  r.city = "City of alternative residence",
  r.zip = "ZIP code of alternative residence",
  r.state = "State of alternative residence",
  r.country = "Country of alternative residence",
  r.phone = "Alternative phone number",
  r.cellular = "Alternative mobile phone number",
  r.fax = "Alternative fax number",
  r.email = "Alternative e-mail address",
  r.url = "Alternative home page",
  r.account = "Alternative bank account",

  w.name = "Work organization name",
  w.title = "Job title",
  w.street = "Street of work organization",
  w.city = "City of work organization",
  w.zip = "ZIP code of work organization",
  w.state = "State of work organization",
  w.country = "Country of work organization",
  w.phone = "Work phone number",
  w.cellular = "Work mobile phone number",
  w.fax = "Work fax number",
  w.email = "Work e-mail address",
  w.url = "Work home page",
  w.account = "Work bank account",

  note = "Additional notes about the person",
}
\end{verbatim}

\begin{verbatim}
@company{key,
  name = "Company name",
  street = "Company street",
  city = "Company city",
  zip = "Company ZIP code",
  state = "Company state",
  country = "Company country",
  phone = "Company phone number",
  cellular = "Company mobile phone number",
  fax = "Company fax number",
  email = "Company e-mail address",
  url = "Company home page",
  account = "Company bank account",
  note = "Additional notes about the company",
}
\end{verbatim}

\begin{verbatim}
@place{key,
  name = "Place name",
  street = "Place street",
  city = "Place city",
  zip = "Place ZIP code",
  state = "Place state",
  country = "Place country",
  phone = "Place phone number",
  cellular = "Place mobile phone number",
  fax = "Place fax number",
  note = "Additional notes about the place",
}
\end{verbatim}

Multiple names, nicknames or birthday dates should be be separated with
``\texttt{and}''. For backward compatibility, or if an unconventional
address formatting is needed, the \texttt{street}, \texttt{city},
\texttt{state}, \texttt{zip} and \texttt{country} fields (and their
\texttt{p.}, \texttt{r.} and \texttt{w.}  equivalents) can be replaced by
generic \texttt{address}, \texttt{p.address}, \texttt{r.address} or
\texttt{w.address} fields. As soon as an \texttt{address} field is defined,
any \texttt{street}, \texttt{city}, \texttt{state}, \texttt{zip} or
\texttt{country} field definition is ignored, and the formatting in the
\texttt{address} field is kept as is.

\section{Customization}
\label{sec:custom}

\subsection{Dimensions}

Three new dimensions defining the indentation of the fields (\verb'\dirindent'
and \verb'\dirparindent') and the amount of space between two entries
(\verb'\dirsep') have been introduced. The default values are:
\begin{quote}
\begin{verbatim}
\setlength{\dirindent}{3em}
\setlength{\dirparindent}{0em}
\setlength{\dirsep}{3ex}
\end{verbatim}
\end{quote}

If you want to explicitly introduce a new paragraph in a field, you
should use the \verb'\dirbreak' command.

A fourth dimension (\verb'\dirtablewidth') sets the width of the table used
to display fields in when the \texttt{letter} style is selected. The default
value is:
\begin{quote}
\begin{verbatim}
\setlength{\dirtablewidth}{0.5\textwidth}
\end{verbatim}
\end{quote}


\subsection{Flags and formats}
\label{sec:flagsformats}

Each field of a directory is easily customizable by redefining one of the
commands summarized in table~\ref{tab:commands} at the end of this user's
guide (page~\pageref{tab:commands}).

For example, to produce nicely formatted address booklets, you could
redefine the \verb'\Dirheader' command as
\begin{quote}
\begin{verbatim}
\pagestyle{headings}
\renewcommand{\Dirheader}[1]
 {\newpage\markboth{\MakeUppercase{#1}}{\MakeUppercase{#1}}}
\end{verbatim}
\end{quote}
(which will split the directory across pages, with the first letter used in
the sorting algorithm in the header of each page) or
\begin{quote}
\begin{verbatim}
\renewcommand{\Dirheader}[1]
  {\item\hspace{-\dirindent}\textbf{\MakeUppercase{#1}}}
\end{verbatim}
\end{quote}
(which will produce inline headings).

When a field type appears multiple times in an entry, the default settings
assume the same formatting for each one. For example, there are three
instances of a \texttt{phone} type field in a full \texttt{person} entry,
i.e.\ \texttt{p.phone}, \texttt{r.phone} and \texttt{w.phone}, and the
\verb'\dirphone' and \verb'\Dirphone' customization commands apply to these
three instances in the same way.

To particularize the formatting of one of these instances, you can use
special versions of the customization commands, constructed by inserting
\texttt{p}, \texttt{r} or \texttt{w} after the \verb'\dir' or \verb'\Dir'
prefix of the original commands. For example, to customize only the
\texttt{phone} field in the work part, you should use \verb'\dirwphone' and
\verb'\Dirwphone'. 

To change the formatting of names, you have to edit the \BibTeX\ style
files. For example, the default name format ``Christophe von Geuzaine, Jr.''
can be changed into ``von Geuzaine C., Jr.'' in your address books by replacing the line
\begin{quote}
\begin{verbatim}
s nameptr "{ff }{vv }{ll}{, jj}" format.name$ 't :=
\end{verbatim}
\end{quote}
in the file \texttt{address.bst} by the line
\begin{quote}
\begin{verbatim}
s nameptr "{vv }{ll}{ f.}{, jj}" format.name$ 't :=
\end{verbatim}
\end{quote}


\section{Using both \texttt{directory} and \texttt{bibliography}}
\label{sec:usingboth}

Since \BibTeX\ always produces an output file of the form
\Quote{\emph{filename}\texttt{.bbl}}, it is necessary---in order to use \emph{both} 
directory and bibliography entries---, after generating the \texttt{bbl} file
corresponding to the directory, to rename it with a new extension (for example
\texttt{dir}), and to give this new extension as an optional argument to the
\verb'\directory' command. The normal procedure can then be followed during the
rest of the bibliography processing. Remember that changing the directory
(adding an entry or suppressing one) forces you to restart from the beginning. 

\section{Generating directories with hypertext links}

You can use the \texttt{hyperref} package along with \texttt{directory}. For
example, adding the following lines in the preamble of your document and
using \texttt{pdflatex} will produce a PDF version of your directory, with
working links for the \verb'email' and \verb'url' fields.

\begin{quote}
\begin{verbatim}
\ifx\pdfoutput\undefined\else
  \usepackage{hyperref}
  \newcommand\MyURL{\begingroup\Url}
  \renewcommand{\Diremail}[1]{\href{mailto:#1}{\MyURL{#1}}}
  \renewcommand{\Dirurl}[1]{\href{#1}{\MyURL{#1}}}
\fi
\end{verbatim}
\end{quote}

\section{Generating HTML, vCard or LDIF directories}

Four special \BibTeX\ style files (\texttt{address-html},
\texttt{email-html}, \texttt{address-vcard} and \texttt{address-ldif}) allow
the easy generation of HTML, vCard and LDIF versions of your directories:

\begin{description}
\item[\textmd{\texttt{address-html}:}] 
full listing in the HTML directory of all fields corresponding to the
\emph{key} entry. The output formatting is similar to that produced by \LaTeX\
with the \texttt{address} style;
\item[\textmd{\texttt{email-html}:}] 
only e-mail addresses are displayed in the HTML directory;
\item[\textmd{\texttt{address-vcard}:}] 
full listing in the vCard directory of all fields corresponding to the
\emph{key} entry.
\item[\textmd{\texttt{address-ldif}:}] 
full listing in the LDIF directory of all fields corresponding to the
\emph{key} entry.
\end{description}

Since \BibTeX\ directly outputs a \texttt{bbl} file in HTML, vCard or LDIF
format, no additional program is needed to make the HTML/vCard/LDIF
conversion. The \texttt{bbl} file directly contains the HTML/vCard/LDIF
code, ready to be included in a HTML document or to be imported in a
vCard/LDIF-aware application (Apple Address Book, Microsoft Outlook, Mozilla
Mail, etc.).

This method presents nevertheless a little drawback: after \BibTeX'ing your
\LaTeX\ file, running \LaTeX\ on the same file (even with another
\verb|\directorystyle|) will produce errors, since the \texttt{bbl} file is
not understandable by \LaTeX. You have to either delete the \texttt{bbl}
file or to override the error messages (and to change the
\verb|\directorystyle|) before any subsequent successful \LaTeX\ run.

The handling of special characters in the HTML/vCard/LDIF directories is
also somewhat problematic: any special \LaTeX\ character sequence is output
the way it is in the \texttt{bib} file. This implies for example that
\verb|{\'e}| is printed in the HTML document as \verb|{\'e}|, and not as
\verb|&eacute;|. The vCard style assumes an ISO Latin 1 encoding of the
directory. If a special encoding is used in the \texttt{bib} file, the LDIF
output will need to be converted to UTF8. See the comments in the \BibTeX\
style files for more information.

\section{Example}
\label{sec:example}

Despite the option described in section~\ref{sec:usingboth}, one of the most
interesting way of using \texttt{directory} is to build a separate address
book, including several \texttt{bib} files referring to several categories of
people, companies or places, as in the example shown in this document:  
\begin{quote}  
\verb|\nodir{*}| \\
\verb|\directory{family,business}|
\end{quote}

A second interesting way of using \texttt{directory} is to use it in your faxes
or letters. Using the standard \LaTeX\ class \texttt{letter.cls} with the
directory style \texttt{letter}, you may for example begin a letter by the
following command (\verb'\wdir' must be \texttt{protect}'ed since the argument
of the \texttt{letter} environment is a moving argument):
\begin{quote}  
\verb'\begin{letter}{\protect\wdir{c.geuzaine}}'
\end{quote}

Take a look at the example \texttt{tex} and \texttt{bib}
files (\texttt{directory.tex}, \texttt{family.bib} and
\texttt{business.bib}) and try the options out. The source files are
commented and easy to customize. I would be very happy to get your suggestions
to improve this package. 

\subsection{Source file}

Here are four \texttt{bib} entries (taken from \texttt{family.bib} and 
\texttt{business.bib}):

\begin{verbatim}
@Person{c.geuzaine,
  name       = "Christophe Geuzaine",
  birthday   = "06 02",
  birthyear  = "1973",
  p.email    = "geuz@geuz.org",
  p.url      = "http://www.geuz.org",
  w.title    = "Postdoctoral Scholar",
  w.name     = "Caltech, Applied and Computational Mathematics",
  w.url      = "http://www.acm.caltech.edu",
  w.street   = "1200 E California Blvd",
  w.city     = "Pasadena",
  w.state    = "CA",
  w.zip      = 91125,
  w.country  = "USA",
  w.phone    = "1 626 395 4552",
}

@Person{d.d.knu,
  name       = "Knudson, Daffy Duck and Bunny, Bugs and Mr. Pluto",
  nickname   = "gnat and gnu and pluto",
  birthday   = "10 02 and 05 11 and 01 01",
  p.phone    = "+01-(0)2-765.43.21",
  p.cellular = "+01-(0)5-555.55.55",
  p.account  = "010-1234567-05",
  r.street   = "Haight Street 512",
  r.zip      = 80214,
  r.city     = "Novosibirsk",
  r.country  = "Gnuland",
  r.phone    = "+01-(0)2-876.54.32",
  w.name     = "University of Novosibirsk, 
                Department of Octopus Parthenogenesis",
}

@Company{knudsoft,
  name   = "The Knudsoft Company",
  email  = "knud@knudsoft.com", 
  url    = "http://knudsoft.com/hole/gates.htm", 
}

@Place{knudsoft:rs.2,
  name   = "Knudsoft (RS.2 Computer Room)",
  phone  = "+01-(0)2-434.23.23",
}
\end{verbatim}

\subsection{Output}
\label{sec:output}

The output resulting from the \verb'\directory{family,business}' command is 
shown below (all entries are listed, thanks to the \verb'\nodir{*}' command):

% CUSTOM: try e.g. citing just some of the entries
\nodir{*}

% CUSTOM: this has the same effect as \directory[bbl]{family,business}
\directory{family,business} 

%%tth: \par directory generated with \href{directory-address.html}{address-html.bst},
%%tth: \href{directory-email.html}{email-html.bst}, 
%%tth: \href{directory-address.vcf}{address-vcard.bst} and
%%tth: \href{directory-address.ldif}{address-ldif.bst}.

\section{Contributors}

Many thanks to Bernd Schandl, Robert Walker Sumner, Thomas Ruedas and
J\"urgen G\"obel for their suggestions and corrections.

\section{Versions}

\begin{description}
\item[0.95] (Jan 8, 1998)
First distributed version. 
\item[0.96] (Jan 9, 1998)
New documentation. Introduction of customization commands. New alignment
mechanism in the \texttt{addressbook} and \texttt{phonebook} environments.  
\item[0.97] (Jan 26, 1998)
Entries \texttt{ccp} and \texttt{p.ccp} changed to \texttt{account} and
\texttt{p.account}.
\item[0.98] (Feb 9, 1998)
New style \texttt{letter.bst}. New commands \verb|\pdir|, \verb|\rdir| and
\verb|\wdir| to produce in-text addresses when used with the
\texttt{letter.bst} style. New internal key generation. 
\item[0.99] (Feb 12, 1998)
Name change of old customization flags (\verb'\'\emph{name}\verb'flag' becomes
\verb'\dir'\emph{name}). New flags introduced: \verb|\dirnickname|, \verb|\dirphone|,
\verb|\dirfax|, \verb|\diremail|, \verb|\dirurl|, \verb|\diraccount| and
\verb|\dirand|.
\item[1.00] (Mar 26, 1998)
New HTML styles (\texttt{address-html.bst} and \texttt{email-html.bst}). 
\item[1.01] (Oct 26, 1998)
Minor corrections.
\item[1.10] (May 6, 1999)
Major rewriting of \texttt{bst} files (suppression of direct \LaTeX\
formatting). Definition of new customization commands. New package global
options to split directories across pages and allow page breaks inside
directory fields. The \texttt{url.sty} package is now required. 
\item[1.11] (May 7, 1999)
Introduction of \verb'\dirparindent'. 
\item[1.12] (May 11, 1999)
Formatting commands can be particularized to each subfield by adding
\texttt{p}, \texttt{r} or \texttt{w} after the \verb'\dir' or \verb'\Dir'
prefix of the original customization command. Many simplifications and small
corrections in the page breaking mechanism and in the list environments.
\item[1.13] (Jun 21, 1999)
Fixed bug for long entries without blank spaces (e.g.\ in \texttt{url}
fields).
\item[1.14] (Jun 21, 2000)
More flexible definition of \verb|\Dirheader|.
\item[1.15] (Aug 28, 2000)
Added fields for cellular phones (suggested by Stefano Ferrari). Added
section explaining how to use \texttt{hyperref} to generate PDF documents
with hyperlinks. Updated web site address.
\item[1.16] (Feb 5, 2002) 
Added \verb'\dirtablewidth' to set the width of the fields when the
\texttt{letter} style is selected. Suppressed the \texttt{split} option
(redefining the \verb'\Dirheader' command makes it possible to achieve the
same result: see section~\ref{sec:flagsformats}).
\item[1.17] (Dec 15, 2002)
Revised documentation.
\item[1.18] (Sep 13, 2003)
Added vCard and LDIF support.
\item[1.19] (Sep 15, 2003)
Split \texttt{address} into \texttt{street}, \texttt{city}, \texttt{state},
\texttt{zip} and \texttt{country}.
\item[1.20] (Sep 22, 2004)
Fixed small vcard export bug when using the old \texttt{address} fields;
new \texttt{german}, \texttt{french} and \texttt{longdates} options.
\end{description}


\begin{table}[p]
\caption{Summary of customization commands\label{tab:commands}}
\begin{center}\small
\begin{tabular}{@{}l@{\hspace{5pt}}c@{\hspace{5pt}}p{3.8cm}l@{}}
\hline
Command & Arg. & Explanation & Default \\
\hline

\verb'\dirsymbol' & 0 & 
\raggedright In-text symbol produced after a directory citation & 
\verb*'' \\

\verb'\dirand' & 0 &
\DbleQuote{anding} string &
\verb*'\normalfont{and}' \\

\verb'\dirbirthday' & 0 &
Birthday field flag &
\verb*'$\star$~' \\

\verb'\dirprivate' & 0 &
Private field flag &
\verb*'\emph{p}~' \\

\verb'\dirresidence' & 0 &
Residence field flag &
\verb*'\emph{r}~' \\

\verb'\dirwork' & 0 &
Work field flag &
\verb*'\emph{w}~' \\

\verb'\dirnote' & 0 &
Note field flag &
\verb*'$\triangleright$~' \\

\verb'\dirnickname' & 0 &
Nickname field flag &
\verb*'' \\

\verb'\diraddress' & 0 &
Address fields flag &
\verb*'' \\

\verb'\dirphone'$^*$ & 0 &
Phone fields flag &
\verb*'tel: ' \\

\verb'\dircellular'$^*$ & 0 &
Cellular phone fields flag &
\verb*'mobile: ' \\

\verb'\dirfax'$^*$ & 0 &
Fax fields flag &
\verb*'fax: ' \\

\verb'\diremail'$^*$ & 0 &
E-mail fields flag &
\verb*'' \\

\verb'\dirurl'$^*$ & 0 &
Url fields flag &
\verb*'' \\

\verb'\diraccount'$^*$ & 0 &
Account fields flag &
\verb*'acc: ' \\

\verb'\dirtitle' & 0 &
Title field flag &
\verb*'' \\

\verb'\dirname' & 0 &
Name field flag &
\verb*'' \\

\verb'\Dirlabel' & 1 &
Label format &
\verb*'{\textbf{#1}}' \\

\verb'\Dirheader' & 1 & 
Command issued for each new starting letter in the directory (the arg.\ is
the first letter used in the sorting algorithm) & 
\verb*'{}' \\

\verb'\Dirbirthday' & 2 &
\raggedright Birthday format (the first arg.\ is the day, the second
is the month) & 
\verb*'{\number#2}/{\number#1}' \\

\verb'\Dirbirthyear' & 1 &
\raggedright Birth year format when a \texttt{birthday} field exists &
\verb*'/{#1}' \\

\verb'\DirbirthyearAlone' & 1 &
\raggedright Birth year format when no \texttt{birthday} field exists &
\verb*'{#1}' \\

\verb'\Dirnickname' & 1 &
Nickname format &
\verb*'(aka \emph{#1})' \\

\verb'\Diraddress'$^*$ & 1 &
Address format &
\verb*'{#1}' \\

\verb'\Dirphone'$^*$ & 1 &
Phone format &
\verb*'{#1}' \\

\verb'\Dircellular'$^*$ & 1 &
Cellular phone format &
\verb*'{#1}' \\

\verb'\Dirfax'$^*$ & 1 &
Fax format &
\verb*'{#1}' \\

\verb'\Diremail'$^*$ & 1 &
E-mail format &
\verb*'\url{#1}' \\

\verb'\Dirurl'$^*$ & 1 &
Url format &
\verb*'\url{#1}' \\

\verb'\Diraccount'$^*$ & 1 &
Account format &
\verb*'\url{#1}' \\

\verb'\Dirtitle' & 1 &
Title format &
\verb*'{#1}' \\

\verb'\Dirname' & 1 &
Name format &
\verb*'{#1}' \\

\verb'\Dirnote' & 1 &
Note format &
\verb*'{#1}' \\

\hline
\end{tabular}
\end{center}
\vspace{-5pt}\small
$^*$\,The commands marked with an asterisk also exist
in three other versions, controlling independently the private, residence and
work parts (e.g.\ \verb'\dirphone' can be particularized to \verb'\dirpphone',
\verb'\dirrphone' and \verb'\dirwphone').
\end{table}

\end{document}
