%%
%% This is file `tablefootnote-example.tex',
%% generated with the docstrip utility.
%%
%% The original source files were:
%%
%% tablefootnote.dtx  (with options: `example')
%% 
%% This is a generated file.
%% 
%% Project: tablefootnote
%% Version: 2014/01/26 v1.1c
%% 
%% Copyright (C) 2011 - 2014 by
%%     H.-Martin M"unch <Martin dot Muench at Uni-Bonn dot de>
%% 
%% The usual disclaimer applies:
%% If it doesn't work right that's your problem.
%% (Nevertheless, send an e-mail to the maintainer
%%  when you find an error in this package.)
%% 
%% This work may be distributed and/or modified under the
%% conditions of the LaTeX Project Public License, either
%% version 1.3c of this license or (at your option) any later
%% version. This version of this license is in
%%    http://www.latex-project.org/lppl/lppl-1-3c.txt
%% and the latest version of this license is in
%%    http://www.latex-project.org/lppl.txt
%% and version 1.3c or later is part of all distributions of
%% LaTeX version 2005/12/01 or later.
%% 
%% This work has the LPPL maintenance status "maintained".
%% 
%% The Current Maintainer of this work is H.-Martin Muench.
%% 
%% This work consists of the main source file tablefootnote.dtx,
%% the README, and the derived files
%%    tablefootnote.sty, tablefootnote.pdf,
%%    tablefootnote.ins, tablefootnote.drv,
%%    tablefootnote-example.tex, tablefootnote-example.pdf.
%% 
%% In memoriam Tommy Muench + 2014/01/02.
%% 
\documentclass[british]{article}[2007/10/19]% v1.4h
%%%%%%%%%%%%%%%%%%%%%%%%%%%%%%%%%%%%%%%%%%%%%%%%%%%%%%%%%%%%%%%%%%%%%
\usepackage{float}[2001/11/08]%    v1.3d
\usepackage{placeins}[2005/04/18]% v2.2  ; for \FloatBarrier
\usepackage{rotating}[2009/03/28]% v2.16a; for sidewaystable-environment
\usepackage[%
 hyperfootnotes=true,%
 extension=pdf,%
 plainpages=false,%
 pdfpagelabels=true,%
 hyperindex=false,%
 pdflang={en},%
 pdftitle={tablefootnote package example},%
 pdfauthor={H.-Martin Muench},%
 pdfsubject={Example for the tablefootnote package},%
 pdfkeywords={LaTeX, tablefootnote, footnote, table, H.-Martin Muench},%
 % pdfview=FitH and FitBH do not work: hyperlinks in sidewaystables
 % do no lead to the footnotes, due to a bug in pdfTeX,
 % computing wrong anchor coordinates (Heiko Oberdiek, 29. October 2011)
 % pdfview=Fit, FitV, FitR, FitB, FitBV work
 % print is OK for all those options
 pdfstartview=FitH,%
 pdfpagelayout=OneColumn%
]{hyperref}[2012/11/06]% v6.83m
 % Due to the urls used in the example, either the hyperref or the url
 % package are needed (or the urls must be removed before compiling).

\usepackage{footnotebackref}[2012/07/01]% v1.0
\usepackage{tablefootnote}[2014/01/26]%   v1.1c

\gdef\unit#1{\mathord{\thinspace\mathrm{#1}}}%
\listfiles
\begin{document}
\pagenumbering{arabic}
\section*{Example for tablefootnote}

This example demonstrates the use of package\newline
\textsf{tablefootnote}, v1.1c as of 2014/01/26 (HMM).\newline
There were no options used. (The package provides no options.)\newline

\textbf{The \texttt{tablefootnote-example.tex} needs to be compiled
at least twice to get the references right!}\newline

If the etoolbox-package is found, it is automatically used.\newline

For more details please see the documentation!\newline

\noindent Save per page about $200\unit{ml}$ water,
$2\unit{g}$ CO$_{2}$ and $2\unit{g}$ wood:\newline
Therefore please print only if this is really necessary.\newline

Here is some text.\footnote{And this is a text footnote.}\newline

Tables \ref{tab.symbol}, \ref{tab.normal}, \ref{tab.another} and
\ref{tab.floatH} show normal tables, table~\ref{tab.sideways} depicts
a sidewaystable. Table~\ref{tab.floatH} uses the float
specifier~\texttt{H} from the float package.\newline

\texttt{Hyperref} option \verb|pdfview=FitH| and \verb|FitBH| do
not work due to a bug in pdf\TeX{}, computing wrong
anchor coordinates (\textsc{Heiko Oberdiek}, 29.~October 2011).
Depending on used pdf-viewer, hyperlinks in sidewaystables lead
e.\,g.~at the end of the document, not at the footnote.
\verb|pdfview=Fit|, \verb|FitV|, \verb|FitR|, \verb|FitB|,
\verb|FitBV| work, print is OK for all those options.

\bigskip

\listoftables

\pagebreak

\renewcommand{\thefootnote}{\fnsymbol{footnote}}
\verb|\renewcommand{\thefootnote}{\fnsymbol{footnote}}|
causes footnote{-}symbol{-}footnotes, which are possible
(see Table~\ref{tab.symbol}).

\begin{table}
\centering
\begin{tabular}{ccc}
Another\tablefootnote{A table footnote.} & %
text\tablefootnote{Another table footnote.} & %
in a table\tablefootnote{A third table footnote.}
\end{tabular}
\caption[A footnotesymbol table]{%
A table with footnote-symbol-footnotes.\label{tab.symbol}}
\end{table}

Some text.\footnote{A text footnote.}

\renewcommand{\thefootnote}{\arabic{footnote}}
\verb|\renewcommand{\thefootnote}{\arabic{footnote}}|
switches back to normal footnote numbers again.

\pagebreak

\begin{table}
\centering
\begin{tabular}{ccc}
Some\tablefootnote{A table footnote.} & %
text\tablefootnote[99]{A table footnote with custom footnote number.} & %
in a table\tablefootnote{A third table footnote.}
\end{tabular}
\caption[A table]{A normal table.\label{tab.normal}}
\end{table}

Some text.\footnote{Another text footnote.}

\pagebreak

More text.\footnote{And yet another text footnote.}

\begin{table}[t]
\centering
\begin{tabular}{|c|c|c|}
\hline
Another\tablefootnote{A $3^{rd}$ table footnote.} & %
text\tablefootnote{Another $3^{rd}$ table footnote.} & %
in a table\tablefootnote{A $3^{rd}$ third table footnote.}\\ \hline
\end{tabular}
\caption[Another table]{Another table (third one)\label{tab.another}}
\end{table}

Please note that Table~\ref{tab.another} floated to the top of the page.
While the footnotes are set and (when hyperref is used) hyperlinked,
they are not automatically adapted. Thus either do not use a footnote
at the same page before the table, or place the table in
\textquotedblleft here\textquotedblright\ or
\textquotedblleft bottom\textquotedblright\ position.
\verb|\clear(double)page|, \verb|h(!)|, \verb|H|~from the
\texttt{float} package, or \verb|\FloatBarrier| from the
\texttt{picins} package might help, too. (Or move the table in the
source code near the position where it floats to
or use the optional footnote marks.)

Table~\ref{tab.floatH} (page~\pageref{tab.floatH}) uses float specifier
\texttt{H} from the float package and does not float.

Some text.\footnote{This is just another text footnote.}

\pagebreak

\FloatBarrier

\begin{sidewaystable}
\centering%
\begin{tabular}{ccc}
Text\tablefootnote{Please rotate the view for testing the %
hyperlinks.} & %
in a\tablefootnote[98]{Another sidewaystable footnote %
with optional footnote mark.} & %
sidewaystable%
\tablefootnote{Sidewaystable-environment provided by %
\url{http://www.ctan.org/pkg/rotating} package.}
\end{tabular}
\caption[A sideways table]{A table in the \texttt{sideways} %
environment\label{tab.sideways}}%
\end{sidewaystable}%

\FloatBarrier

\pagebreak

A last table, here with float specifier \texttt{H} from the
float\footnote{\url{http://www.ctan.org/pkg/float}} package.

\begin{table}[H]
\centering
\begin{tabular}{ccc}
Another\tablefootnote{A $5^{th}$ table footnote.} & %
text\tablefootnote{Another $5^{th}$ table footnote.} & %
in a table\tablefootnote{A $5^{th}$ third table footnote.}
\end{tabular}
\caption[A last table]{A very last table\label{tab.floatH}}
\end{table}

Some text.\footnote{This is just another text footnote.}

\pagebreak

End of the example for the
tablefootnote\footnote{\url{http://www.ctan.org/pkg/tablefootnote}}
package.

\end{document}
\endinput
%%
%% End of file `tablefootnote-example.tex'.
