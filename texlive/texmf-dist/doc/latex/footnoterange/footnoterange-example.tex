%%
%% This is file `footnoterange-example.tex',
%% generated with the docstrip utility.
%%
%% The original source files were:
%%
%% footnoterange.dtx  (with options: `example')
%% 
%% This is a generated file.
%% 
%% Project: footnoterange
%% Version: 2012/02/17 v1.0a
%% 
%% Copyright (C) 2012 by
%%     H.-Martin M"unch <Martin dot Muench at Uni-Bonn dot de>
%% 
%% The usual disclaimer applies:
%% If it doesn't work right that's your problem.
%% (Nevertheless, send an e-mail to the maintainer
%%  when you find an error in this package.)
%% 
%% This work may be distributed and/or modified under the
%% conditions of the LaTeX Project Public License, either
%% version 1.3c of this license or (at your option) any later
%% version. This version of this license is in
%%    http://www.latex-project.org/lppl/lppl-1-3c.txt
%% and the latest version of this license is in
%%    http://www.latex-project.org/lppl.txt
%% and version 1.3c or later is part of all distributions of
%% LaTeX version 2005/12/01 or later.
%% 
%% This work has the LPPL maintenance status "maintained".
%% 
%% The Current Maintainer of this work is H.-Martin Muench.
%% 
%% This work consists of the main source file footnoterange.dtx,
%% the README, and the derived files
%%    footnoterange.sty, footnoterange.pdf,
%%    footnoterange.ins, footnoterange.drv,
%%    footnoterange-example.tex, footnoterange-example.pdf.
%% 
\documentclass[british]{article}[2007/10/19]% v1.4h
%%%%%%%%%%%%%%%%%%%%%%%%%%%%%%%%%%%%%%%%%%%%%%%%%%%%%%%%%%%%%%%%%%%%%
\usepackage[%
 extension=pdf,%
 plainpages=false,%
 pdfpagelabels=true,%
 hyperindex=false,%
 pdflang={en},%
 pdftitle={footnoterange package example},%
 pdfauthor={H.-Martin Muench},%
 pdfsubject={Example for the footnoterange package},%
 pdfkeywords={LaTeX, footnoterange, H.-Martin Muench},%
 pdfview=FitH,%
 pdfstartview=FitH,%
 pdfpagelayout=OneColumn,%
 hyperfootnotes=true%
]{hyperref}[2012/02/06]% v6.82o

\usepackage{footnoterange}[2012/02/17]% v1.0a

\gdef\unit#1{\mathord{\thinspace\mathrm{#1}}}%
\listfiles
\begin{document}
\pagenumbering{arabic}
\section*{Example for footnoterange}

This example demonstrates the use of package\newline
\textsf{footnoterange}, v1.0a as of 2012/02/17 (HMM).\newline
The package does not use options.\newline

\noindent For more details please see the documentation!\newline

\noindent Save per page about $200\unit{ml}$ water,
$2\unit{g}$ CO$_{2}$ and $2\unit{g}$ wood:\newline
Therefore please print only if this is really necessary.\newline

This\footnote{Lorem} text bears a multiplicity of footnotes.
Because the \texttt{hyperref} package is used with option
\texttt{hyperfootnotes=true}, the footnotes%
\begin{footnoterange}%
\footnote{ipsum}%
\footnote{dolor}%
\footnote{sit}%
\footnote{amet,}%
\footnote{consectetur}%
\footnote{adipisicing}%
\footnote{elit,}%
\footnote{sed}%
\end{footnoterange}%
 are hyperlinked.\footnote{do eiusmod\ldots} Using
\texttt{hyperfootnotes=false} or not using \texttt{hyperref}
will remove the hyperlinks to the footnotes.\newline

This text also bears a multiplicity of footnotes,
but due to the use of the starred%
\begin{footnoterange*}%
\footnote{Ut purus elit,}%
\footnote{vestibulum ut,}%
\footnote{placerat ac,}%
\footnote{adipiscing vitae,}%
\footnote{felis.}%
\footnote{Curabitur}%
\footnote{dictum}%
\footnote{gravida}%
\end{footnoterange*}%
 form of the environment they are not
hyperlinked.\footnote{mauris\ldots}\newline

The use of the optional parameter for footnotes%
\begin{footnoterange}%
\footnote[101]{Nam arcu libero,}%
\footnote[102]{nonummy eget,}%
\footnote[103]{consectetuer id,}%
\footnote[104]{vulputate a,}%
\footnote[105]{magna.}
\end{footnoterange}%
 is also possible, but should be used consistently.

\pagebreak

When there is only one footnote%
\begin{footnoterange}%
\footnote{a single footnote}%
\end{footnoterange}%
 in the \texttt{footnoterange} environment, only one
footnotemark is used and an info is written into the
log-file.\newline

The usage of other footnote-number-representations
(e.\,g.~\verb|\Roman|, \verb|\roman|, \verb|\Alph|, \verb|\alph|,
\verb|\fnsymbol|) is also possible, but two things must be taken
into account: The number of footnote references is restricted (for
example with \verb|\Alph| only references A to Z are possible), which
can be fixed e.\,g. with the \texttt{alphalph} package, and references
to footnote-symbol-ranges (\verb|\fnsymbol|) are probably
not very clear.

\end{document}
\endinput
%%
%% End of file `footnoterange-example.tex'.
