% rotman.tex   User Manual for isorot package
\documentclass[11pt]{article}
\usepackage{isorot}

%\makeatletter
\ifx\pdfoutput\undefined\else
  \pdfoutput=1
\fi
%\makeatother

\setlength{\textheight}{8.0in}
\setlength{\textwidth}{6.0in}
\setlength{\oddsidemargin}{0.25in}
\setlength{\evensidemargin}{0.25in}
\setlength{\marginparwidth}{0.6in}
\setcounter{secnumdepth}{4}
\setcounter{tocdepth}{4}

%
%    remainder of preamble is some special macro definitions
\makeatletter
%   the \meta{} command
%
\begingroup
\obeyspaces%
\catcode`\^^M\active%
\gdef\meta{\begingroup\obeyspaces\catcode`\^^M\active%
\let^^M\do@space\let \do@space%
\def\-{\egroup\discretionary{-}{}{}\hbox\bgroup\it}%
\m@ta}%
\endgroup
\def\m@ta#1{\leavevmode\hbox\bgroup\texttt{<}\textit{#1}\/\texttt{>}\egroup
    \endgroup}
\def\do@space{\egroup\space
    \hbox\bgroup\it\futurelet\next\sp@ce}
\def\sp@ce{\ifx\next\do@space\expandafter\sp@@ce\fi}
\def\sp@@ce#1{\futurelet\next\sp@ce}
%
\makeatother

%
\makeatletter
%
% the \latex command
\newcommand{\latex}{\LaTeX}
\newcommand{\tex}{\TeX}
%
%   the \file{} command
%
\newcommand{\file}[1]{\textsf{#1}}
%
% example and note environments
\newenvironment{example}{\begin{quotation}\small\textbf{Example: }}{\par\end{quotation}}
\newenvironment{note}{\begin{quotation}\small\textbf{Note: }}{\par\end{quotation}}

% internal refs
\newcommand{\tref}[1]{Table~\ref{#1}}
\newcommand{\fref}[1]{Figure~\ref{#1}}

%   index a command
\newcommand{\bs}{\symbol{'134}}
\newcommand{\ixcom}[1]{\index{#1/ @{\tt \protect\bs #1}}}
%   index an environment
\newcommand{\ixenv}[1]{\index{#1 @{\tt #1} (environment)}}
%   index a starred environment
\newcommand{\ixenvs}[1]{\index{#1s @{\tt #1*} (environment)}}
%   index an option
\newcommand{\ixopt}[1]{\index{#1 @{\tt #1} (option)}}
%   index a package
\newcommand{\ixpack}[1]{\index{#1 @\file{#1} (package)}}
%   index a class
\newcommand{\ixclass}[1]{\index{#1 @\file{#1} (class)}}
%   index in typewriter font
\newcommand{\ixtt}[1]{\index{#1@{\tt #1}}}
%   index LaTeX
\newcommand{\ixltx}{\index{latex@\latex}}
%   index LaTeX 2e
\newcommand{\ixltxe}{\index{latex2e@\latex 2e}}
%   index LaTeX v2.09
\newcommand{\ixltxv}{\index{latex209@\latex{} v2.09}}
\makeatother
%
%   end of preamble
%

\title{The \file{isorot} Package User Manual}
\author{Peter Wilson\\ \texttt{peter.r.wilson@boeing.com}}
\date{15 February 2000}

\begin{document}
\pagenumbering{roman}
\maketitle

\begin{abstract}
The facilities in the \file{isorot} package are described. The package
was initially designed for use with the \file{iso} class but can be used
with the `normal' classes as well. The package enables the rotation of 
document elements, like text or tables of figures.
\end{abstract}
\tableofcontents
\listoffigures
\listoftables
\clearpage
\pagestyle{headings}
\pagenumbering{arabic}


\section{Introduction}

   The \file{isorot} package enables the rotation of document elements on
a page. It uses the \latex{} \verb|\special| command to perform its
effects, and thus can only be used
with a limited number of \TeX{} to print routes. The facilities
available are summarized in \tref{tab4}.

    \file{isorot} is a modification of the \file{rotation.sty} file
created by Rahtz and Barroca~\cite{rahtz}. Further examples of the usage
of their style are given in Goosens \emph{et al}~\cite{goosens}. The
package also uses David Carlisle's \file{graphicx} and \file{lscape}
packages.

\begin{note}Several examples of the effects of the commands described herein
are shown. In many cases the results are not pretty. This should act as
a warning that using rotational elements requires more care than
most other document elements.\end{note} %end note

\begin{sidewaystable}
\ixcom{rotdriver} \ixcom{clockwise} \ixcom{counterclockwise}
\ixcom{figuresright} \ixcom{figuresleft} \ixcom{rotcaption}
\ixcom{controtcaption}
\ixenv{sideways} \ixenv{turn} \ixenv{rotate} \ixenv{sidewaystable}
\ixenv{sidewaysfigure} \ixenv{landscape}
\centering
\caption{The rotation facilities} \label{tab4}
\begin{tabular}{|l|l|} \hline
\textbf{Facility} & \textbf{Effect} \\ \hline
\multicolumn{2}{|c|}{\textbf{Commands}} \\ \hline
\verb|\rotdriver{<driver>}| & 
declare the name of the dvi to Postscript translator (default {\tt dvips}) \\
\verb|\clockwise| & 
sets rotation direction clockwise for positive angles (the default) \\
\verb|\counterclockwise| &
sets rotation direction counterclockwise for positive angles \\
\verb|\figuresright| &
sets rotation direction for sideways floats counterclockwise (the default) \\
\verb|\figuresleft| &
sets rotation direction for sideways floats clockwise \\
\verb|\rotcaption| &
like the \verb|caption| command, but rotates the caption through 90 degrees \\
\verb|\controtcaption| &
like the \verb|contcaption| command, but rotates the caption through 90 degrees \\ \hline
\multicolumn{2}{|c|}{\textbf{Environments}} \\ \hline
\verb|sideways| &
rotates the contents through 90 degrees counterclockwise \\
\verb|turn| &
rotates the contents through the given angle \\
\verb|rotate| &
rotates the contents through the given angle, but no space allowed for the result\\
\verb|sidewaystable| &
like the \verb|table| environment, but rotated 90 degrees \\
\verb|sidewaystable*| &
twocolumn version of \verb|sidewaystable| \\
\verb|sidewaysfigure| &
like the \verb|figure| environment, but rotated 90 degrees \\
\verb|sidewaysfigure*| &
twocolumn version of \verb|sidewaysfigure| \\
\verb|landscape| &
prints all enclosed pages in landscape mode \\ \hline
\end{tabular}
\end{sidewaystable}

\section{Options}

    The \file{isorot} facility has one option, 
namely \verb|debugshow|\ixopt{debugshow}. Calling this option produces
messages on the screen and in the \file{log} file regarding the actions 
being taken.

\begin{note} This option is principally of interest to the maintainer
of the facility. \end{note}

    The font used for the captions of rotated figures or tables is controlled
by the \verb|rotcapfont| command. Under normal circumstances this is a null
command but when used with the \file{iso} class it is defined as: \\
\verb|\newcommand{\rotcapfont}{\captionsisize\bf}| \\
 where \verb|\captionsize| is defined in the class. You can renew 
\verb|\rotcapfont| to change the caption font to your liking. 

\section{DVI drivers}

    The \file{isorot} facility supports only a limited number of
dvi to postscript translators. The default translator is \emph{dvips}.
 The following command must be put in
the preamble of the document if \emph{dvips} is not being used: 
\verb|\rotdriver{<drivername>}|,\ixcom{rotdriver} where
\verb|<drivername>| is one of the following:\footnote{I have been able to 
try the {\tt dvips} driver
but not the others. If anyone has experience with the other drivers, or has
extended the range of drivers, I would like to be given the results.} %end footnote

\begin{enumerate}
\item \verb|dvipdf| for the \emph{dvipdf} 
      translator;\ixtt{dvipdf}
\item \verb|dvips| for Tom Rockicki's \emph{dvips} 
      translator;\ixtt{dvips}
\item \verb|dvipsone| for Y\&Y's \emph{dvipsone} 
      translator;\ixtt{dvipsone}
\item \verb|dvitops| for James Clark's \emph{dvitops} 
      translator;\ixtt{dvitops}
\item \verb|dviwindo| for Y\&Y's \emph{dviwindo} 
      translator;\ixtt{dviwindo}
\item \verb|pctex32| for Personal TeX's PC TeX for 32 bit Windows 
      (\emph{pctex32})
      translator;\ixtt{pctex32}
\item \verb|pctexps| for Personal TeX's PC PTI Laser/PS (\emph{pctexps})
      translator;\ixtt{pctexps}
\item \verb|pubps| for the Arbortext's \emph{pubps} 
      translator.\ixtt{pubps}
\item \verb|textures| for Blue Sky's \emph{Textures} 
      translator;\ixtt{textures}

\end{enumerate}

    The \file{isorot} package can also be used in documents processed
by pdfLaTeX.

\section{Rotational directions}

    \file{isorot} enables the textual and other elements of a document
to be rotated from their normal horizontal layout. In some cases elements
can be rotated through arbitrary angles, whereas in others only 90 degree
rotation is possible. 

    By default, a rotation through a positive number of
degrees corresponds to a clockwise rotation. The command 
\verb|\counterclockwise|\ixcom{counterclockwise}
sets the following rotations to be counterclockwise for positive angles.
The command \verb|\clockwise|\ixcom{clockwise}
sets the following rotations to be clockwise for positive angles.
These commands can be used to toggle the rotational behavior.
    

    Rotated floating environments are normally rotated so that they are
printed with a counterclockwise rotation (i.e. the original bottom of the float
is placed at the right hand side of the paper), 
which is what is normally required.
This behavior can be altered by the command
\verb|\figuresleft|,\ixcom{figuresleft}
which will give the reverse effect. The command
\verb|\figuresright|\ixcom{figuresright}
will set the behavior to the default.
These commands can be used to toggle the rotational behavior of
floats.


\section{Rotation of text}

    The \verb|sideways|\ixenv{sideways}
environment rotates the contents of the environment
by 90 degrees counterclockwise, and leaves space for the result.

    The \verb|\begin{turn}{|\meta{angle}\verb|}|\ixenv{turn}
environment rotates the contents by the given number
of degrees in the direction specified by the most recent of the
\verb|\clockwise|\ixcom{clockwise} or
\verb|\counterclockwise|\ixcom{counterclockwise}
commands, leaving space for the result.

    The \verb|\begin{rotate}{|\meta{angle}\verb|}|\ixenv{rotate}
environment rotates the contents by the given number
of degrees in the direction specified by the most recent of the
\verb|\clockwise|\ixcom{clockwise} or
\verb|\counterclockwise|\ixcom{counterclockwise}
commands, but no arrangements are made for leaving space for the result.


\begin{example}Some simple rotations: \label{ex:1}

This code
\begin{verbatim}
Default rotation direction: \\
A 
\begin{sideways}%
B C
\end{sideways}
D E F G H I J K L M
\begin{turn}{-90}%
Minus 90 turn
\end{turn}
N O P
\begin{rotate}{90}%
Plus 90 rotate
\end{rotate}
Q \\
and continue on with another line after rotations.
\end{verbatim}
produces the following (note how space is allowed for the \verb|turn|ed 
text, whereas the \verb|rotate|d text runs into the text below).

Default rotation direction: \\
A 
\begin{sideways}%
B C
\end{sideways}
D E F G H I J K L M
\begin{turn}{-90}%
Minus 90 turn
\end{turn}
N O P
\begin{rotate}{90}%
Plus 90 rotate
\end{rotate}
Q \\
and continue on with another line after rotations.
\end{example} % end example
\begin{example}This example shows the effect of using the 
\verb|\counterclockwise|\ixcom{counterclockwise}
command.

This code
\begin{verbatim}
Flip rotation direction: \\
\counterclockwise
A 
\begin{sideways}%
B C
\end{sideways}
D E F G H I J K L M
\begin{turn}{-90}%
Minus 90 turn
\end{turn}
N O P
\begin{rotate}{90}%
Plus 90 rotate
\end{rotate}
Q \\
Set rotation direction back to default value.
\clockwise
\end{verbatim}

produces the following, which should be compared with example~\ref{ex:1}.

Flip rotation direction: \\
\counterclockwise
A 
\begin{sideways}%
B C
\end{sideways}
D E F G H I J K L M
\begin{turn}{-90}%
Minus 90 turn
\end{turn}
N O P
\begin{rotate}{90}%
Plus 90 rotate
\end{rotate}
Q \\
Set rotation direction back to default value.
\clockwise
\end{example} % end example


    Although the examples so far have only shown the rotation of text, boxes
can also be rotated.

\begin{example}Rotating a box.

This code
\begin{verbatim}
\newsavebox{\foo}
\newlength{\fool}
\settowidth{\fool}{Hurrah for ISO.}
\savebox{\foo}{\parbox{\fool}{Hurrah for ISO. Hurrah for ISO.
                              Hurrah for ISO. Hurrah for ISO.}}
Start
\usebox{\foo}
\&
\begin{turn}{-45}\usebox{\foo}\end{turn}
\&
\begin{turn}{45}\usebox{\foo}\end{turn}
End
\end{verbatim}
produces:

\newsavebox{\foo}
\newlength{\fool}
\settowidth{\fool}{Hurrah for ISO.}
\savebox{\foo}{\parbox{\fool}{Hurrah for ISO. Hurrah for ISO.
                              Hurrah for ISO. Hurrah for ISO.}}
Start
\usebox{\foo}
\&
\begin{turn}{-45}\usebox{\foo}\end{turn}
\&
\begin{turn}{45}\usebox{\foo}\end{turn}
End

\end{example} %end example

    Elements can be rotated through arbitrary angles, and also rotated
elements can be nested inside other rotated elements.


\begin{example}Repeated rotation:

    The following example code shows that text can be rotated through any angle.
The result is shown in \fref{fig:wheel}.
\begin{verbatim}
\newcount\prwc
\newsavebox{\prwtext}
\newdimen\prwspace
\def\wheel#1#2{%
  \savebox{\prwtext}{#1\begin{sideways}#2\end{sideways}}%
  \prwspace\wd\prwtext%
  \advance\prwspace by 1cm%
  \centerline{%
  \rule{0pt}{\prwspace}%
  \rule[-\prwspace]{0pt}{\prwspace}%
  \prwc=-180\loop\ifnum\prwc<180
  \rlap{\begin{rotate}{\the\prwc}%
  \rule{1cm}{0pt}\usebox{\prwtext}\end{rotate}}%
  \advance\prwc by 20\repeat}}
\begin{figure}
\wheel{Express yourself ---}{Hooray for STEP!}
\caption{Example rotation through multiple angles}
\label{fig:wheel}
\end{figure}
\end{verbatim}

\newcount\prwc
\newsavebox{\prwtext}
\newdimen\prwspace
\def\wheel#1#2{%
  \savebox{\prwtext}{#1\begin{sideways}#2\end{sideways}}%
  \prwspace\wd\prwtext%
  \advance\prwspace by 1cm%
  \centerline{%
  \rule{0pt}{\prwspace}%
  \rule[-\prwspace]{0pt}{\prwspace}%
  \prwc=-180\loop\ifnum\prwc<180
  \rlap{\begin{rotate}{\the\prwc}%
  \rule{1cm}{0pt}\usebox{\prwtext}\end{rotate}}%
  \advance\prwc by 20\repeat}}
\begin{figure}
\vspace*{1cm}
\wheel{Express yourself ---}{Hooray for STEP!}
\vspace*{1cm}
\caption{Example rotation through multiple angles}
\label{fig:wheel}
\end{figure}
\end{example} %end example

    Later in the manual, Figures~\ref{fig:angles1} and~\ref{fig:angles2} 
also show rotations through a
range of angles, both positive and negative.

\begin{example}Nested rotations. \label{ex:sidetabular}

This code
\begin{verbatim}
    Here is some text before a \verb|sideways| environment. 
And some more, and more and more garble gobble cluck
click clack clock cluck and so on and on and on.
\begin{center}
\begin{sideways}
\rule{1in}{0pt}
\begin{tabular}{|lr|}
\begin{rotate}{-45}\emph{Word}\end{rotate} & \begin{rotate}{-90}%
Occurrences\end{rotate}
\\
\hline
hello & 33 \\
goodbye & 34 \\
\hline
\end{tabular}
\end{sideways}
\end{center}
    Here is some text after a \verb|sideways| environment.
 And some more, and more and more garble gobble cluck
click clack clock cluck and so on and on and on.
\end{verbatim}
produces:

    Here is some text before a \verb|sideways| environment. 
And some more, and more and more garble gobble cluck
click clack clock cluck and so on and on and on.
\begin{center}
\begin{sideways}
%\rule{1in}{0pt}
\begin{tabular}{|lr|}
\begin{rotate}{-45}\emph{Word}\end{rotate} & \begin{rotate}{-90}%
Occurrences\end{rotate} \\ \hline
hello & 33 \\
goodbye & 34 \\ \hline
\end{tabular}
\end{sideways}
\end{center}
    Here is some text after a \verb|sideways| environment.
 And some more, and more and more garble gobble cluck
click clack clock cluck and so on and on and on.
\end{example} %end example



\section{Rotations of tables and figures}

    The previous examples have demonstrated the rotation of textual elements.
For instance, the last one showed that tabular material can be rotated using
the \verb|sideways|\ixenv{sideways}
environment. (Actually, any of the previously
mentioned environments could have been used instead.)
Two further environments are provided which rotate a \latex{} float through
90 degrees. These are:
\begin{itemize}
\item \verb|sidewaystable|\ixenv{sidewaystable}, which
  corresponds to the standard \latex{} \verb|table|\ixenv{table}
  environment; and
\item \verb|sidewaysfigure|\ixenv{sidewaysfigure}, which
  corresponds to the standard \latex{} \verb|figure|\ixenv{figure}
  environment.
\end{itemize}
There are also starred versions of these, namely 
\verb|sidewaystable*|\ixenvs{sidewaystable} and
\verb|sidewaysfigure*|\ixenvs{sidewaysfigure}, for use in twocolumn mode.
However, the correspondence with the standard environments is not strictly
complete as a sideways float is alway placed on a page by itself.

    The direction of rotation may be controlled by the 
\verb|\figuresright|\ixcom{figuresright} and
\verb|\figuresleft|\ixcom{figuresleft} commands.

\begin{example}Table~\ref{tab4} was produced by the code below: \label{ex:4}

\begin{verbatim}
\begin{sidewaystable}
\centering
\caption{The rotation facilities} \label{tab4}
\begin{tabular}{|l|l|} \hline
\textbf{Facility} & \textbf{Effect} \\ \hline
\multicolumn{2}{|c|}{\textbf{Commands}} \\ \hline
\verb|\rotdriver{<driver>}| & 
declare the name of the dvi to Postscript translator (default {\tt dvips}) \\
......
\verb|sidewaysfigure| &
like the \verb|figure| environment, but rotated 90 degrees \\ \hline
\end{tabular}
\end{sidewaystable}
\end{verbatim}
\end{example} % end example



\section{Rotation of float captions and bodies}

    Sometimes it may be useful to rotate a caption independently of the
rotation of a figure or table. The command 
\verb|\rotcaption|\ixcom{rotcaption} is analogous
to the normal \verb|\caption|\ixcom{caption} command, 
and inserts the caption rotated
by 90~degrees. There is also the companion command 
\verb|\controtcaption|\ixcom{controtcaption}, analagous to the
\verb|\contcaption|\ixcom{contcaption} command,
for continuation captions.

%\newsavebox{\picbox}

\begin{figure}
\centering
\caption{Example figure with a standard caption.} \label{fig:nocrot}
\setlength{\unitlength}{0.2in}
\footnotesize
\begin{picture}(17,2)
\thicklines
\put(0,0){\begin{picture}(4,1)
  \put(1.5,0.5){\oval(3,1)}
  \put(1.5,0.5){\makebox(0,0){2,5 (1)}}
  \put(3,0.5){\line(1,0){1.0}}
  \put(4.25,0.5){\circle{0.5}}
  \end{picture}}

\put(4.5,0){\begin{picture}(8,1)
  \put(0,0){\dashbox{0.25}(4,1){date}}
  \put(4,0.5){\line(1,0){3.5}}
  \put(7.75,0.5){\circle{0.5}}
  \put(6,1){\makebox(0,0){A[1:3]}}
  \end{picture}}

\put(12.5,0){\begin{picture}(4,1)
  \put(0,0){\framebox(4,1){INTEGER}}
  \put(3.75,0){\line(0,1){1}}
  \end{picture}}
\end{picture}
\normalsize
\setlength{\unitlength}{1pt}
\end{figure}


\begin{example}Float with a regular caption.

Figure~\ref{fig:nocrot} is produced by the code below:
\begin{verbatim}
\begin{figure}
\centering
\caption{Example figure with a standard caption.} \label{fig:nocrot}
\setlength{\unitlength}{0.2in}
\footnotesize
\begin{picture}(17,2)
\thicklines
\put(0,0){\begin{picture}(4,1)
  \put(1.5,0.5){\oval(3,1)}
  \put(1.5,0.5){\makebox(0,0){2,5 (1)}}
  \put(3,0.5){\line(1,0){1.0}}
  \put(4.25,0.5){\circle{0.5}}
  \end{picture}}
\put(4.5,0){\begin{picture}(8,1)
  \put(0,0){\dashbox{0.25}(4,1){date}}
  \put(4,0.5){\line(1,0){3.5}}
  \put(7.75,0.5){\circle{0.5}}
  \put(6,1){\makebox(0,0){A[1:3]}}
  \end{picture}}
\put(12.5,0){\begin{picture}(4,1)
  \put(0,0){\framebox(4,1){INTEGER}}
  \put(3.75,0){\line(0,1){1}}
  \end{picture}}
\end{picture}
\normalsize
\setlength{\unitlength}{1pt}
\end{figure}
\end{verbatim}
\end{example} % end example

\begin{example}Float with a rotated caption.

Figure~\ref{fig:crot} is produced by the code below:
\begin{verbatim}
\begin{figure}
\centering
\rotcaption{Figure~\protect\ref{fig:nocrot} with a rotated caption.}
 \label{fig:crot}
\setlength{\unitlength}{0.2in}
\footnotesize
\begin{picture}(17,2)
...
\end{picture}
\normalsize
\setlength{\unitlength}{1pt}
\end{figure}
\end{verbatim}
\end{example} % end example


\begin{figure}
\centering
\rotcaption{Figure~\protect\ref{fig:nocrot} with a rotated caption.}
 \label{fig:crot}
\setlength{\unitlength}{0.2in}
\footnotesize
\begin{picture}(17,2)
\thicklines
\put(0,0){\begin{picture}(4,1)
  \put(1.5,0.5){\oval(3,1)}
  \put(1.5,0.5){\makebox(0,0){2,5 (1)}}
  \put(3,0.5){\line(1,0){1.0}}
  \put(4.25,0.5){\circle{0.5}}
  \end{picture}}
\put(4.5,0){\begin{picture}(8,1)
  \put(0,0){\dashbox{0.25}(4,1){date}}
  \put(4,0.5){\line(1,0){3.5}}
  \put(7.75,0.5){\circle{0.5}}
  \put(6,1){\makebox(0,0){A[1:3]}}
  \end{picture}}
\put(12.5,0){\begin{picture}(4,1)
  \put(0,0){\framebox(4,1){INTEGER}}
  \put(3.75,0){\line(0,1){1}}
  \end{picture}}
\end{picture}
\normalsize
\setlength{\unitlength}{1pt}
\end{figure}

    As can be seen from \fref{fig:crot} the advisability of rotating a caption
depends on the size of the body of the float. It may be better in certain
cases to leave the caption in its regular position and rotate the body of
the float instead.

\def\prwrot#1{%
\settowidth{\fool}{ISOROT}
\savebox{\foo}{\parbox{\fool}{ISOROT ISOROT ISOROT ISOROT}}%
\framebox{---\begin{turn}{#1}\framebox{\usebox{\foo}}\end{turn}---}}%
\def\degrees{{\small$^{o}$}}

\begin{figure}
\centering
\begin{tabular}{|c|c|c|} \hline
\prwrot{0} &\prwrot{-40}&\prwrot{-80}\\
0\degrees & -40\degrees & -80\degrees \\ \hline
\prwrot{-120}&\prwrot{-160}&\prwrot{-200}\\
-120\degrees & -160\degrees & -200\degrees \\ \hline
\prwrot{-240}&\prwrot{-280}&\prwrot{-320}\\
-240\degrees & -280\degrees & -320\degrees \\ \hline
\end{tabular}
\caption{Rotation of paragraphs between 0 and -320 degrees} \label{fig:angles1}
\end{figure}


\begin{example}Regular caption and float.

Figure~\ref{fig:angles1} is a regular figure and caption. It is produced by
the following code:
\begin{verbatim}
\def\prwrot#1{%
\settowidth{\fool}{ISOROT}
\savebox{\foo}{\parbox{\fool}{ISOROT ISOROT ISOROT ISOROT}}%
\framebox{---\begin{turn}{#1}\framebox{\usebox{\foo}}\end{turn}---}}%
\def\degrees{{\small$^{o}$}}
\end{verbatim}

\begin{verbatim}
\begin{figure}
\centering
\begin{tabular}{|c|c|c|} \hline
\prwrot{0} &\prwrot{-40}&\prwrot{-80}\\
0\degrees & -40\degrees & -80\degrees \\ \hline
\prwrot{-120}&\prwrot{-160}&\prwrot{-200}\\
-120\degrees & -160\degrees & -200\degrees \\ \hline
\prwrot{-240}&\prwrot{-280}&\prwrot{-320}\\
-240\degrees & -280\degrees & -320\degrees \\ \hline
\end{tabular}
\caption{Rotation of paragraphs between 0 and -320 degrees} \label{fig:angles1}
\end{figure}
\end{verbatim}
\end{example} % end example


\begin{figure}
\centering
\begin{sideways}
\begin{tabular}{|c|c|c|} \hline
\prwrot{0} &\prwrot{40}&\prwrot{80}\\
0\degrees & 40\degrees & 80\degrees \\ \hline
\prwrot{120}&\prwrot{160}&\prwrot{200}\\
120\degrees & 160\degrees & 200\degrees \\ \hline
\prwrot{240}&\prwrot{280}&\prwrot{320}\\
240\degrees & 280\degrees & 320\degrees \\ \hline
\end{tabular}
\end{sideways}
\caption[Rotation of paragraphs between 0 and 320 degrees]%
        {Rotation of paragraphs between 0 and 320 degrees (with figure
         body turned sideways)}\label{fig:angles2}
\end{figure}

\begin{example}Regular caption and rotated float body.

Figure~\ref{fig:angles2} is a regular figure and caption where the figure
contents have been rotated. It was produced by the following code.
\begin{verbatim}
\begin{figure}
\centering
\begin{sideways}
\begin{tabular}{|c|c|c|} \hline
\prwrot{0} &\prwrot{40}&\prwrot{80}\\
0\degrees & 40\degrees & 80\degrees \\ \hline
\prwrot{120}&\prwrot{160}&\prwrot{200}\\
120\degrees & 160\degrees & 200\degrees \\ \hline
\prwrot{240}&\prwrot{280}&\prwrot{320}\\
240\degrees & 280\degrees & 320\degrees \\ \hline
\end{tabular}
\end{sideways}
\caption[Rotation of paragraphs between 0 and 320 degrees]%
        {Rotation of paragraphs between 0 and 320 degrees (with figure
         body turned sideways)}\label{fig:angles2}
\end{figure}
\end{verbatim}
\end{example} % end example

\begin{landscape}
\section{Landscaping}

    \latex{} normally prints in portrait mode. 
The \verb|landscape|\ixenv{landscape} environment
prints all the enclosed stuff in landscape mode, except for headers
and footers which are not rotated.

\begin{example} Landscaping

The source for this part of the document is:
\begin{verbatim}
\begin{landscape}
\section{Landscaping}

    \latex{} normally prints in portrait mode. The ...
...
... long, wide tables.
\end{landscape}
\end{verbatim}
\end{example}

    The environment starts by clearing the current page and then switches
to portrait mode. At the end of the environment the current page is cleared
and the next page is back to normal portrait mode. 

    All the other rotation commands and environments produce boxes and
\latex{} will not break a box across a page. The \verb|landscape| environemt
does not produce a box and so many pages can be printed in landscape mode
with \latex{} taking care of the page breaking for you. 

    Landscape mode is not particularly useful for normal text as the
lines are far too long for comfortable reading. Where it can be useful
is where you have a table that is too wide to fit on a portrait page, so
needs to be rotated, yet is also too long to fit on the page when it is
rotated. The \file{supertabular}\ixpack{supertabular},
the \file{longtable}\ixpack{longtable},
and the \file{xtab}\ixpack{xtab}
packages provide facilities for automatically breaking long tables across 
pages. Any of these can be used in conjunction with landscaping to both 
rotate and automatically page break long, wide tables.
\end{landscape}


%%%%%%%%%%%%%%
%\end{document}
%%%%%%%%%%%%%

\bibliographystyle{alpha}
\begin{thebibliography}{GMS94}

\bibitem[GMS94]{goosens}
Michel Goossens, Frank Mittelbach and Alexander Samarin.
\newblock \textit{The LaTeX Companion}.
\newblock Addison-Wesley Publishing Co. 1994.

\bibitem[RB92]{rahtz}
Sebastian Rahtz and Leonor Barroca.
\newblock \textit{A style option for rotated objects in LaTeX}.
\newblock TUGBoat, volume 13, number 2, pp 156--180, July 1992.

\end{thebibliography}

\end{document}

\begin{references}
\reference{LAMPORT, L.,}{LaTeX --- A Document Preparation System,}
            {Addison-Wesley Publishing Co., 2nd edition, 1994.} \label{lamport}
\reference{WILSON, P.R.,}{LaTeX files for typesetting ISO standards:
           Source code,}
           {NISTIR,  National Institute of Standards and Technology,
           Gaithersburg, MD 20899. June 1996.} \label{isoe}
\reference{WILSON, P.R.,}{LaTeX package files for ISO~10303: User manual,}
           {NISTIR,  National Institute of Standards and Technology,
           Gaithersburg, MD 20899. June 1996.} \label{stepsty}
\reference{RAHTZ, S., and BARROCA, L.,}{A style option for rotated
           objects in \latex,}{ TUGBoat, volume 13, number 2, pp 156--180, 
           July 1992.} \label{rahtz}
\reference{GOOSSENS, M., MITTELBACH, F. and SAMARIN, A.,}{%
           The LaTeX Companion,}
           {Addison-Wesley Publishing Co., 1994.} \label{goosens}
\reference{GOOSSENS, M., and RAHTZ, S.,}{%
           The LaTeX Web Companion --- Integrating TeX, HTML and XML,}
           {Addison-Wesley Publishing Co., 1999.} \label{lwebcom}
\reference{CHEN, P. and HARRISON, M.A.,}{Index preparation and
           processing,}{Software--Practice and Experience, 19(9):897--915,
           September 1988.} \label{chen}
\reference{KOPKA, H. and DALY, P.W.,}{A Guide to LaTeX,}
           {Addison-Wesley Publishing Co., 1993.} \label{kopka}
\reference{WALSH, N.,}{Making TeX Work,}{O'Reilly \& Associates, Inc.,
           103 Morris Street, Suite A, Sebastopol, CA 95472. 1994. } \label{walsh}
\reference{ISO 8879:1986,}{Information processing --- 
                                Text and office systems ---
           Standard Generalized Markup Language (SGML).}{} \label{sgml}
\reference{GOLDFARB, C.F.,}{The SGML Handbook,}
           {Oxford University Press, 1990.} \label{goldfarb}
\reference{BRYAN, M.,}{SGML --- An Author's Guide to the Standard Generalized
           Markup Language,}{Addison-Wesley Publishing Co., 1988. } \label{bryan}
\reference{PHILLIPS, L. and LUBELL, J.,}{An SGML Environment for STEP,}%
          {NISTIR 5515, National Institute of Standards and Technology,
           Gaithersburg, MD 20899. November 1994.} \label{pandl}
\reference{WILSON, P. R.,}{LTX2X: A LaTeX to X Auto-tagger,}%
          {NISTIR, National Institute of Standards and Technology,
           Gaithersburg, MD 20899. June 1996.} \label{ltx2x}
\reference{RESSLER, S.,}{The National PDES Testbed Mail Server User's Guide,}
           {NSTIR 4508, National Institute of Standards and Technology,
           Gaithersburg, MD 20899. January 1991.} \label{ressler}
\reference{RINAUDOT, G.R.,}{STEP On Line Information Service (SOLIS),}
          {NISTIR 5511, National Institute of Standards and Technology,
          Gaithersburg, MD 20899. October 1994. } \label{rinaudot}
\reference{KROL, E.,}{The Whole Internet --- User's Guide \& Catalog,}
           {O'Reilly \& Associates, Inc.,
           103 Morris Street, Suite A, Sebastopol, CA 95472. 1993. } \label{krol}
\reference{WILSON, P.R.,}{The hyphenat package,}%
          {1999. (Available from CTAN)} \label{bib:hyphenat}
\reference{WILSON, P.R.,}{The xtab package,}%
          {1998. (Available from CTAN)} \label{bib:xtab}
\end{references}



\end{document}
