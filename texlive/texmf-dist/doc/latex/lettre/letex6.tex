% This file is part of the "lettre" package.
%
% This work may be distributed and/or modified under the
% conditions of the LaTeX Project Public License, either version 1.3
% of this license or (at your option) any later version.
% The latest version of this license is in
%   http://www.latex-project.org/lppl.txt
% and version 1.3 or later is part of all distributions of LaTeX
% version 2005/12/01 or later.
%
% This work has the LPPL maintenance status `maintained'.
%
% The Current Maintainer of this work is Vincent Bela�che.
%
% This work consists of all files listed in doc/latex/lettre/readme/LICENSE.
\documentclass[12pt,origdate]{lettre}
\usepackage[german,francais]{babel}
\usepackage[OT1]{fontenc}
\usepackage{mltex}
\begin{document}
\newcommand{\dhfam}{\fontencoding{OT1}\fontfamily{cmdh}\fontseries{m}%
                    \fontshape{n}}
\newcommand{\cmd}{\dhfam\fontsize{10}{12pt}\selectfont}
\newcommand{\Cmd}{\dhfam\fontsize{12}{14pt}\selectfont}
\newcommand{\CMD}{\dhfam\fontsize{14}{17pt}\selectfont}
%
% Adresse precisee,
% =================
% langue allemande, format headings,
% ==================================
% precision de departement, lieu, telephone, fax, E-Mail.
% =======================================================
% Champs: annexes.
% ================
% Lignes auxiliaires de reference et d'E-Mail.
% ============================================
%
\begin{letter}{ Herrn Professor H.F.~Hess \\
                Astronomisches Institut   \\
                Universit\"at Mainz   \\
                Jupitergasse 4 \\
                D-4102 \underline{B\"onningen}   }

\allemand
\pagestyle{headings}

\name{Dr~T.G. Kurwezger}
\address{\centering
         \CMD Centre des D\'es Stochastiques \\ 
         \cmd CH-1291 Prairie du Gr\"utli }
\location{Dr~Terry G. Kurwezger \\
          D\'epartement des Tas}
\telephone{+41(1) 671 27 12}
\lieu{La Prairie}
\signature{Terry}
\email{tkur@cds.unigr.ch}

\nref{ TGK/dm }
\fax{+41(1) 671 27 45}
\username{tkur}
\ccitt{OU=cds;O=unigr;C=ch}
\internet{cds.unigr.ch}

\opening{ Lieber Heinz, }

Vielen Dank f\"ur deine Anruf und die Einladung, einen Vortrag in Mainz
zu halten. Ich schlage den folgenden Titel vor:

\begin{center}
{\large KOSMOS, was, wo, wann ?}
\end{center}
\medskip

\noindent{\large\it Zusammenfassung}

Kosmos \"uberalles. Was sind die lichtst\"arksten Objekte die wir kennen ?
Wo sind die Grenzen des Universums ? Warum befinden sich Quasaren
im Zentrum von Galaxien, am Rande des beobachtbaren Universums ? 
Wann anf\"angt die moderne Astronomie ?

Falls du eine l\"angere Zusammenfassung w\"unschst, habe ich eine mit diesem Brief beigef\"ugt.

\closing{Mit freundlichen Gr\"ussen}
\encl{1 Zusammenfassung}
\end{letter}
%
\end{document}
