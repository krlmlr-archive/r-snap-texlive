% This file is part of the "lettre" package.
%
% This work may be distributed and/or modified under the
% conditions of the LaTeX Project Public License, either version 1.3
% of this license or (at your option) any later version.
% The latest version of this license is in
%   http://www.latex-project.org/lppl.txt
% and version 1.3 or later is part of all distributions of LaTeX
% version 2005/12/01 or later.
%
% This work has the LPPL maintenance status `maintained'.
%
% The Current Maintainer of this work is Vincent Bela�che.
%
% This work consists of all files listed in doc/latex/lettre/readme/LICENSE.
\documentclass[12pt,origdate]{lettre}
\usepackage{epic,eepic}
\usepackage[francais]{babel}
\usepackage[OT1]{fontenc}
\usepackage{mltex}
\begin{document}
%
% Fichier de defaut de l'Observatoire
% ===================================
\institut{obs}
%
% Entete officielle, triple signature, format empty, date.
% ========================================================
% Lignes auxiliaires de reference et d'E-Mail. 
% ============================================
% Tests: quote, quotation, verbatim, minipage, 
% ============================================
%        picture, tabbing. 
% ========================
%
\begin{letter}{	Pr.~E.N.~Photon \\ 
                D\'epartement d'Astrotopographie \\ 
                Universit\'e de Saint Pozium \\
                3945, Quai du G\'eneral Gisant \\
                CH-6800 Motte-au-Rolla }

\pagestyle{empty}

\psobs
\name{Dr~D.P.~Dnavegem}
\location{Pr.~J.~Su\"{\i}jy-Rest\\Groupe des Forces Statiques}
\lieu{Sauverni}
\date{au 25 Joviet 2091}
\signature{Pr.~J.~Su\"{\i}jy-Rest\\ Doyen et\\ Chef de D\'epartement}
\secondsignature{Dr~D.P.~Dnavegem\\ Collaborateur Scientifique}
\thirdsignature{A.~Jout\'e\\ Assistant}

\vref{EP/mjs}
\nref{jsr002}
\username{jsr}
\bitnet{cgeuge54}
\decnet{chgate::20159}

\opening{Cher Professeur Photon,}

Nous vous remercions d'avoir donn\'e suite \`a notre requ\^ete, et vous
confirmons notre participation au symposium en tant que sp\'ecialistes
des affaires \'etranges. Veuillez trouver ici un r\'esum\'e de notre 
communication commune:

\begin{quote}\bf\sl
L'influence n\'efaste des extra-terrestres pendulaires et frontaliers
sur les communications t\'el\'evisuelles intercontinentales.
\end{quote}

\begin{quotation}
L'\'emergence de courants plasmuriques forts dans la r\'egion d'atterrissage 
des  v\'ehicules de liaison plan\'etaires (VLP) est \`a l'origine des champs 
gravito--organiques \`a bolomisations al\'eatoires connus depuis la fin du 
si\`ecle pass\'e. Ces modifications de l'\'equilibre physico-chimique de 
l'atmosph\`ere donnent lieu \`a toute une panoplie de ph\'enom\`enes plus 
ou moins inqui\'etants et spectaculaires, tels que les trous dans la couche 
d'ozone ou les aurores bor\'eales ou australes que l'on attribuait par le 
pass\'e \`a des regains d'activit\'e solaire.

On a d\'ecouvert r\'ecemment qu'aux heures de pointe, le flux des VLP, 
anciennement acronym\'es OVNIS,  pouvait provoquer des battements et des 
ph\'enom\`enes de r\'esonances dans certaines configurations de terrain, 
et sous certaines conditions, telles que les meilleurs blindages 
gravito--organiques ne pouvaient y \^etre totalement opaques. 

Les communications t\'el\'evisuelles intracontinentales, bas\'ees sur les 
technologies les plus r\'ecentes de fibres auditiques en 
Corduron$^{\mbox{\copyright}}$ de chez Dubond de Velours sont compl\'etement 
insensibles \`a de telles perturbations, contrairement aux anciennes lignes 
intercontinentales en cablage traditionnel (polygraphite impr\'egn\'e).  
\end{quotation}

Nous avons d\'evelopp\'e un logiciel d'analyse permettant de traiter 
l'information statistique fournies par les sondes FVLP, pour fournir 
\`a nos clients l'information sur les endroits les plus touch\'es du globe.
Veuillez trouver ci-apr\`es le pseudo-code du protocole de communication, 
une illustration des sondes se transmettant l'information de mani\`ere 
autonome, ainsi qu'une table des param\`etres de celles-ci.

\begin{verbatim}
BEGIN
   if(alive(S1) && alive(S3) && alive(S5)) then
      BEGIN
         contact{s1,S1};
         contact{S1,S3};
         contact{S3,S5};
         contact{S5,s3};
      END         
   endif
   if(alive(S2) && alive(S4) && alive(S6)) then
      BEGIN
         contact{s2,S2};
         contact{S2,S4};
         contact{S4,S6};
         contact{S6,s4};
      END         
   endif
END         
\end{verbatim}

\setlength{\unitlength}{0.0063in}
%
\begingroup\makeatletter\ifx\SetFigFont\undefined
% extract first six characters in \fmtname
\def\x#1#2#3#4#5#6#7\relax{\def\x{#1#2#3#4#5#6}}%
\expandafter\x\fmtname xxxxxx\relax \def\y{splain}%
\ifx\x\y   % LaTeX or SliTeX?
\gdef\SetFigFont#1#2#3{%
  \ifnum #1<17\tiny\else \ifnum #1<20\small\else
  \ifnum #1<24\normalsize\else \ifnum #1<29\large\else
  \ifnum #1<34\Large\else \ifnum #1<41\LARGE\else
     \huge\fi\fi\fi\fi\fi\fi
  \csname #3\endcsname}%
\else
\gdef\SetFigFont#1#2#3{\begingroup
  \count@#1\relax \ifnum 25<\count@\count@25\fi
  \def\x{\endgroup\@setsize\SetFigFont{#2pt}}%
  \expandafter\x
    \csname \romannumeral\the\count@ pt\expandafter\endcsname
    \csname @\romannumeral\the\count@ pt\endcsname
  \csname #3\endcsname}%
\fi
\fi\endgroup
\begin{picture}(673,385)(0,-10)
\put(339,184){\ellipse{82}{82}}
\put(159,319){\ellipse{10}{10}}
\put(249,19){\ellipse{10}{10}}
\put(24,124){\ellipse{10}{10}}
\put(394,344){\ellipse{10}{10}}
\put(649,259){\ellipse{10}{10}}
\put(595,73){\ellipse{10}{10}}
\path(319,174)(319,179)(314,179)
	(314,174)(319,174)
\path(364,194)(364,199)(359,199)
	(359,194)(364,194)
\path(159,319)(649,259)
\path(640.816,257.987)(649.000,259.000)(641.302,261.958)
\path(394,344)(595,73)
\path(588.628,78.234)(595.000,73.000)(591.841,80.617)
\path(354,159)(354,164)(349,164)
	(349,159)(354,159)
\path(649,259)(362,196)
\path(369.385,199.669)(362.000,196.000)(370.243,195.762)
\path(595,73)(351,162)
\path(359.201,161.138)(351.000,162.000)(357.830,157.380)
\path(316,177)(249,19)
\path(250.282,27.146)(249.000,19.000)(253.964,25.584)
\path(325,207)(325,212)(320,212)
	(320,207)(325,207)
\path(249,19)(159,319)
\path(163.214,311.912)(159.000,319.000)(159.383,310.763)
\dottedline{5}(162,314)(298,188)
\dottedline{5}(164,316)(328,223)
\drawline(394,339)(394,339)
\dottedline{5}(394,339)(373,205)
\dottedline{5}(644,258)(368,213)
\dottedline{5}(644,256)(380,180)
\dottedline{5}(591,75)(380,186)
\dottedline{5}(590,73)(343,143)
\dottedline{5}(253,22)(351,145)
\dottedline{5}(250,23)(304,162)
\dottedline{5}(29,123)(307,158)
\dottedline{5}(30,125)(299,193)
\path(323,209)(24,124)
\path(31.148,128.111)(24.000,124.000)(32.242,124.264)
\path(24,124)(394,344)
\path(388.146,338.192)(394.000,344.000)(386.102,341.630)
\dottedline{5}(392,339)(320,220)
\put(139,323){\makebox(0,0)[lb]{\smash{{{\SetFigFont{6}{7.2}{rm}S3}}}}}
\put(0,115){\makebox(0,0)[lb]{\smash{{{\SetFigFont{6}{7.2}{rm}S2}}}}}
\put(391,354){\makebox(0,0)[lb]{\smash{{{\SetFigFont{6}{7.2}{rm}S4}}}}}
\put(317,195){\makebox(0,0)[lb]{\smash{{{\SetFigFont{6}{7.2}{rm}s2}}}}}
\put(314,162){\makebox(0,0)[lb]{\smash{{{\SetFigFont{6}{7.2}{rm}s1}}}}}
\put(352,203){\makebox(0,0)[lb]{\smash{{{\SetFigFont{6}{7.2}{rm}s3}}}}}
\put(342,168){\makebox(0,0)[lb]{\smash{{{\SetFigFont{6}{7.2}{rm}s4}}}}}
\put(660,256){\makebox(0,0)[lb]{\smash{{{\SetFigFont{6}{7.2}{rm}S5}}}}}
\put(600,58){\makebox(0,0)[lb]{\smash{{{\SetFigFont{6}{7.2}{rm}S6}}}}}
\put(238,0){\makebox(0,0)[lb]{\smash{{{\SetFigFont{6}{7.2}{rm}S1}}}}}
\end{picture}
 % graphique en mode picture avec eepic

\begin{minipage}{7cm}
Les valeurs param\'etriques des satellites sont donn\'ees ci-contre, par ordre 
de date de lancement. Les unit\'es sont MKSA, dans la mesure du possible, 
l'excentricit\'e des orbites est donn\'e comme le rapport grand/petit axe, 
et le taux de transmission en TB/s.
\end{minipage}\hfill
\begin{minipage}{7cm}
\begin{tabbing}
n$^{\textrm o} $ \=masse \=g.a/p.a \=puissance \=t$_{\textrm tr}$\\
S1\>247\>1.16\>53.5\>1.3\\
S2\>211\>1.40\>49.3\>1.1\\
S3\>233\>1.27\>51.0\>1.2\\
S4\>199\>1.91\>48.8\>1.0\\
S5\>270\>1.33\>65.2\>1.5\\
S6\>270\>1.33\>65.2\>1.5\\
\end{tabbing}
\end{minipage}

\closing{Veuillez agr\'eer, Monsieur le professeur, l'expression 
         de nos condol\'eances distingu\'ees.} 

\end{letter}
%
\end{document}
