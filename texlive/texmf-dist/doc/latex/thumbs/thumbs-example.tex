%%
%% This is file `thumbs-example.tex',
%% generated with the docstrip utility.
%%
%% The original source files were:
%%
%% thumbs.dtx  (with options: `example')
%% 
%% This is a generated file.
%% 
%% Project: thumbs
%% Version: 2014/03/09 v1.0q
%% 
%% Copyright (C) 2010 - 2014 by
%%     H.-Martin M"unch <Martin dot Muench at Uni-Bonn dot de>
%% 
%% The usual disclaimer applies:
%% If it doesn't work right that's your problem.
%% (Nevertheless, send an e-mail to the maintainer
%%  when you find an error in this package.)
%% 
%% This work may be distributed and/or modified under the
%% conditions of the LaTeX Project Public License, either
%% version 1.3c of this license or (at your option) any later
%% version. This version of this license is in
%%    http://www.latex-project.org/lppl/lppl-1-3c.txt
%% and the latest version of this license is in
%%    http://www.latex-project.org/lppl.txt
%% and version 1.3c or later is part of all distributions of
%% LaTeX version 2005/12/01 or later.
%% 
%% This work has the LPPL maintenance status "maintained".
%% 
%% The Current Maintainer of this work is H.-Martin Muench.
%% 
%% This work consists of the main source file thumbs.dtx,
%% the README, and the derived files
%%    thumbs.sty, thumbs.pdf,
%%    thumbs.ins, thumbs.drv,
%%    thumbs-example.tex, thumbs-example.pdf.
%% 
%% In memoriam Tommy Muench + 2014/01/02.
%% 
\documentclass[twoside,british]{article}[2007/10/19]% v1.4h
\usepackage{lipsum}[2011/04/14]%  v1.2
\usepackage{eurosym}[1998/08/06]% v1.1
\usepackage[extension=pdf,%
 pdfpagelayout=TwoPageRight,pdfpagemode=UseThumbs,%
 plainpages=false,pdfpagelabels=true,%
 hyperindex=false,%
 pdflang={en},%
 pdftitle={thumbs package example},%
 pdfauthor={H.-Martin Muench},%
 pdfsubject={Example for the thumbs package},%
 pdfkeywords={LaTeX, thumbs, thumb marks, H.-Martin Muench},%
 pdfview=Fit,pdfstartview=Fit,%
 linktoc=all]{hyperref}[2012/11/06]% v6.83m
\usepackage[thumblink=rule,linefill=dots,height={auto},minheight={47pt},%
            width={auto},distance={2mm},topthumbmargin={auto},bottomthumbmargin={auto},%
            eventxtindent={5pt},oddtxtexdent={5pt},%
            evenmarkindent={0pt},oddmarkexdent={0pt},evenprintvoffset={0pt},%
            nophantomsection=false,ignorehoffset=true,ignorevoffset=true,final=true,%
            hidethumbs=false,verbose=true]{thumbs}[2014/03/09]% v1.0q
\nopagecolor% use \pagecolor{white} if \nopagecolor does not work
\gdef\unit#1{\mathord{\thinspace\mathrm{#1}}}
\makeatletter
\ltx@ifpackageloaded{hyperref}{% hyperref loaded
}{% hyperref not loaded
 \usepackage{url}% otherwise the "\url"s in this example must be removed manually
}
\makeatother
\listfiles
\begin{document}
\pagenumbering{arabic}
\section*{Example for thumbs}
\addcontentsline{toc}{section}{Example for thumbs}
\markboth{Example for thumbs}{Example for thumbs}

This example demonstrates the most common uses of package
\textsf{thumbs}, v1.0q as of 2014/03/09 (HMM).
The used options were \texttt{thumblink=rule}, \texttt{linefill=dots},
\texttt{height=auto}, \texttt{minheight=\{47pt\}}, \texttt{width={auto}},
\texttt{distance=\{2mm\}}, \newline
\texttt{topthumbmargin=\{auto\}}, \texttt{bottomthumbmargin=\{auto\}}, \newline
\texttt{eventxtindent=\{5pt\}}, \texttt{oddtxtexdent=\{5pt\}},
\texttt{evenmarkindent=\{0pt\}}, \texttt{oddmarkexdent=\{0pt\}},
\texttt{evenprintvoffset=\{0pt\}},
\texttt{nophantomsection=false},
\texttt{ignorehoffset=true}, \texttt{ignorevoffset=true}, \newline
\texttt{final=true},\texttt{hidethumbs=false}, and \texttt{verbose=true}.

\noindent These are the default options, except \texttt{verbose=true}.
For more details please see the documentation!\newline

\textbf{Hyperlinks or not:} If the \textsf{hyperref} package is loaded,
the references in the overview page for the thumb marks are also hyperlinked
(except when option \texttt{thumblink=none} is used).\newline

\bigskip

{\color{teal} Save per page about $200\unit{ml}$ water, $2\unit{g}$ CO$_{2}$
and $2\unit{g}$ wood:\newline
Therefore please print only if this is really necessary.}\newline

\bigskip

\textbf{%
For testing purpose page \pageref{greenpage} has been completely coloured!
\newline
Better exclude it from printing\ldots \newline}

\bigskip

Some thumb mark texts are too large for the thumb mark by intention
(especially when the paper size and therefore also the thumb mark size
is decreased). When option \texttt{width=\{autoauto\}} would be used,
the thumb mark width would be automatically increased.
Please see page~\pageref{HugeText} for details!

\bigskip

For printing this example to another format of paper (e.\,g. A4)
it is necessary to add the according option (e.\,g. \verb|a4paper|)
to the document class and recompile it! (In that case the
thumb marks column change will occur at another point, of course.)
With paper format equal to document format the document can be printed
without adapting the size, like e.\,g.
\textquotedblleft shrink to page\textquotedblright .
That would add a white border around the document
and by moving the thumb marks from the edge of the paper they no longer appear
at the side of the stack of paper. It is also necessary to use a printer
capable of printing up to the border of the sheet of paper. Alternatively
it is possible after the printing to cut the paper to the right size.
While performing this manually is probably quite cumbersome,
printing houses use paper, which is slightly larger than the
desired format, and afterwards cut it to format.

\newpage

\addtitlethumb{Frontmatter}{0}{white}{gray}{pagesLTS.0}

At the first page no thumb mark was used, but we want to begin with thumb marks
at the first page, therefore a
\begin{verbatim}
\addtitlethumb{Frontmatter}{0}{white}{gray}{pagesLTS.0}
\end{verbatim}
was used at the beginning of this page.

\newpage

\tableofcontents

\newpage

To include an overview page for the thumb marks,
\begin{verbatim}
\addthumbsoverviewtocontents{section}{Thumb marks overview}%
\thumbsoverview{Table of Thumbs}
\end{verbatim}
is used, where \verb|\addthumbsoverviewtocontents| adds the thumb
marks overview page to the table of contents.

\smallskip

Generally it is desirable to have a hyperlink from the thumbs overview page
to lead to the thumb mark and not to some earlier place. Therefore automatically
a \verb|\phantomsection| is placed before each thumb mark. But for example when
using the thumb mark after a \verb|\chapter{...}| command, it is probably nicer
to have the link point at the top of that chapter's title (instead of the line
below it). The automatical placing of the \verb|\phantomsection| can be disabled
either globally by using option \texttt{nophantomsection}, or locally for the next
thumb mark by the command \verb|\thumbsnophantom|. (When disabled globally,
still manual use of \verb|\phantomsection| is possible.)

\addthumbsoverviewtocontents{section}{Thumb marks overview}%
\thumbsoverview{Table of Thumbs}

That were the overview pages for the thumb marks.

\newpage

\section{The first section}
\addthumb{First section}{\space\Huge{\textbf{$1^ \textrm{st}$}}}{yellow}{green}

\begin{verbatim}
\addthumb{First section}{\space\Huge{\textbf{$1^ \textrm{st}$}}}{yellow}{green}
\end{verbatim}

A thumb mark is added for this section. The parameters are: title for the thumb mark,
the text to be displayed in the thumb mark (choose your own format),
the colour of the text in the thumb mark,
and the background colour of the thumb mark (parameters in this order).\newline

Now for some pages of \textquotedblleft content\textquotedblright\ldots

\newpage
\lipsum[1]
\newpage
\lipsum[1]
\newpage
\lipsum[1]
\newpage

\section{The second section}
\addthumb{Second section}{\Huge{\textbf{\arabic{section}}}}{green}{yellow}

For this section, the text to be displayed in the thumb mark was set to
\begin{verbatim}
\Huge{\textbf{\arabic{section}}}
\end{verbatim}
i.\,e. the number of the section will be displayed (huge \& bold).\newline

Let us change the thumb mark on a page with an even number:

\newpage

\section{The third section}
\addthumb{Third section}{\Huge{\textbf{\arabic{section}}}}{blue}{red}

No problem!

And you do not need to have a section to add a thumb:

\newpage

\addthumb{Still third section}{\Huge{\textbf{\arabic{section}b}}}{red}{blue}

This is still the third section, but there is a new thumb mark.

On the other hand, you can even get rid of the thumb marks
for some page(s):

\newpage

\stopthumb

The command
\begin{verbatim}
\stopthumb
\end{verbatim}
was used here. Until another \verb|\addthumb| (with parameters) or
\begin{verbatim}
\continuethumb
\end{verbatim}
is used, there will be no more thumb marks.

\newpage

Still no thumb marks.

\newpage

Still no thumb marks.

\newpage

Still no thumb marks.

\newpage

\continuethumb

Thumb mark continued (unchanged).

\newpage

Thumb mark continued (unchanged).

\newpage

Time for another thumb,

\addthumb{Another heading}{Small text}{white}{black}

and another.

\addthumb{Huge Text paragraph}{\Huge{Huge\\ \ Text}}{yellow}{green}

\bigskip

\textquotedblleft {\Huge{Huge Text}}\textquotedblright\ is too large for
the thumb mark. When option \texttt{width=\{autoauto\}} would be used,
the thumb mark width would be automatically increased. Now the text is
either split over two lines (try \verb|Huge\\ \ Text| for another format)
or (in case \verb|Huge~Text| is used) is written over the border of the
thumb mark. When the text is too wide for the thumb mark and cannot
be split, \LaTeX{} might nevertheless place the text into the next line.
By this the text is placed too low. Adding a
\hbox{\verb|\protect\vspace*{-| some lenght \verb|}|} to the text could help,
for example\\
\verb|\addthumb{Huge Text}{\protect\vspace*{-3pt}\Huge{Huge~Text}}...|.
\label{HugeText}

\addthumb{Huge Text}{\Huge{Huge~Text}}{red}{blue}

\addthumb{Huge Bold Text}{\Huge{\textbf{HBT}}}{black}{yellow}

\bigskip

When there is more than one thumb mark at one page, this is also no problem.

\newpage

Some text

\newpage

Some text

\newpage

Some text

\newpage

\section{xcolor}
\addthumb{xcolor}{\Huge{\textbf{xcolor}}}{magenta}{cyan}

It is probably a good idea to have a look at the \textsf{xcolor} package
and use other colours than used in this example.

(About automatically increasing the thumb mark width to the thumb mark text
width please see the note at page~\pageref{HugeText}.)

\newpage

\addthumb{A mark}{\Huge{\textbf{A}}}{lime}{darkgray}

I just need to add further thumb marks to get them reaching the bottom of the page.

Generally the vertical size of the thumb marks is set to the value given in the
height option. If it is \texttt{auto}, the size of the thumb marks is decreased,
so that they fit all on one page. But when they get smaller than \texttt{minheight},
instead of decreasing their size further, a~new thumbs column is started
(which will happen here).

\newpage

\addthumb{B mark}{\Huge{\textbf{B}}}{brown}{pink}

There! A new thumb column was started automatically!

\newpage

\addthumb{C mark}{\Huge{\textbf{C}}}{brown}{pink}

You can, of course, keep the colour for more than one thumb mark.

\newpage

\addthumb{$1/1.\,955\,83$\, EUR}{\Huge{\textbf{D}}}{orange}{violet}

I am just adding further thumb marks.

If you are curious why the thumb mark between
\textquotedblleft C mark\textquotedblright\ and \textquotedblleft E mark\textquotedblright\ has
not been named \textquotedblleft D mark\textquotedblright\ but
\textquotedblleft $1/1.\,955\,83$\, EUR\textquotedblright :

$1\unit{DM}=1\unit{D\ Mark}=1\unit{Deutsche\ Mark}$\newline
$=\frac{1}{1.\,955\,83}\,$\euro $\,=1/1.\,955\,83\unit{Euro}=1/1.\,955\,83\unit{EUR}$.

\newpage

Let us have a look at \verb|\thumbsoverviewverso|:

\addthumbsoverviewtocontents{section}{Table of Thumbs, verso mode}%
\thumbsoverviewverso{Table of Thumbs, verso mode}

\newpage

And, of course, also at \verb|\thumbsoverviewdouble|:

\addthumbsoverviewtocontents{section}{Table of Thumbs, double mode}%
\thumbsoverviewdouble{Table of Thumbs, double mode}

\newpage

\addthumb{E mark}{\Huge{\textbf{E}}}{lightgray}{black}

I am just adding further thumb marks.

\newpage

\addthumb{F mark}{\Huge{\textbf{F}}}{magenta}{black}

Some text.

\newpage
\thumbnewcolumn
\addthumb{New thumb marks column}{\Huge{\textit{NC}}}{magenta}{black}

There! A new thumb column was started manually!

\newpage

Some text.

\newpage

\addthumb{G mark}{\Huge{\textbf{G}}}{orange}{violet}

I just added another thumb mark.

\newpage

\pagecolor{green}

\makeatletter
\ltx@ifpackageloaded{hyperref}{% hyperref loaded
\phantomsection%
}{% hyperref not loaded
}%
\makeatother

\label{greenpage}

Some faulty pdf-viewer sometimes (for the same document!)
adds a white line at the bottom and right side of the document
when presenting it. This does not change the printed version.
To test for this problem, this page has been completely coloured.
(Probably better exclude this page from printing!)

\textsc{Heiko Oberdiek} wrote at Tue, 26 Apr 2011 14:13:29 +0200
in the \newline
comp.text.tex newsgroup (see e.\,g. \newline
\url{http://groups.google.com/group/de.comp.text.tex/msg/b3aea4a60e1c3737}):\newline
\textquotedblleft Der Ursprung ist 0 0,
da gibt es nicht viel zu runden; bei den anderen Seiten
werden pt als bp in die PDF-Datei geschrieben, d.h.
der Balken ist um 72.27/72 zu gro\ss{}, das
sollte auch Rundungsfehler abdecken.\textquotedblright

(The origin is 0 0, there is not much to be rounded;
for the other sides the $\unit{pt}$ are written as $\unit{bp}$ into
the pdf-file, i.\,e. the rule is too large by $72.27/72$, which
should cover also rounding errors.)

The thumb marks are also too large - on purpose! This has been done
to assure, that they cover the page up to its (paper) border,
therefore they are placed a little bit over the paper margin.

Now I red somewhere in the net (should have remembered to note the url),
that white margins are presented, whenever there is some object outside
of the page. Thus, it is a feature, not a bug?!
What I do not understand: The same document sometimes is presented
with white lines and sometimes without (same viewer, same PC).\newline
But at least it does not influence the printed version.

\newpage

\pagecolor{white}

It is possible to use the Table of Thumbs more than once (for example
at the beginning and the end of the document) and to refer to them via e.\,g.
\verb|\pageref{TableOfThumbs1}, \pageref{TableOfThumbs2}|,... ,
here: page \pageref{TableOfThumbs1}, page \pageref{TableOfThumbs2},
and via e.\,g. \verb|\pageref{TableOfThumbs}| it is referred to the last used
Table of Thumbs (for compatibility with older package versions).
If there is only one Table of Thumbs, this one is also the last one, of course.
Here it is at page \pageref{TableOfThumbs}.\newline

Now let us have a look at \verb|\thumbsoverviewback|:

\addthumbsoverviewtocontents{section}{Table of Thumbs, back mode}%
\thumbsoverviewback{Table of Thumbs, back mode}

\newpage

Text can be placed after any of the Tables of Thumbs, of course.

\end{document}
\endinput
%%
%% End of file `thumbs-example.tex'.
