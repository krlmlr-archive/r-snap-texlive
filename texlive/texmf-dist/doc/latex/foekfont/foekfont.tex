\documentclass[a4paper,11pt,danish,DIVcalc]{scrartcl}
\usepackage[T1]{fontenc}
\usepackage[utf8]{inputenc}
\usepackage{babel,listings,foekfont,microtype}
\usepackage[sc,osf]{mathpazo}
\usepackage[scaled]{luximono}
\renewcommand*\sfdefault{iwonac}
\linespread{1.05}         % Palatino needs more leading (space between
                          % lines)
\author{Palle Jørgensen}
%\date{28. September 2006}
\title{The \texttt{foekfont} package}
\newcommand*\sourcefile[1]{\subsection{#1}
  \lstinputlisting{#1}}
\lstset{language=[latex]tex,breaklines=true}
\pdfoutput=1

\begin{document}
\maketitle


\section{Introduction}
\label{sec:introduction}

The \texttt{foekfont} package is for typesetting the title of the
\madsfoek\footnote{Mads Føk, http://www.madsfoek.dk} magazine; The
students magazine at Faculty of
Science\footnote{http://www.nat.au.dk}, University of
Aarhus\footnote{http://www.au.dk}.


\section{Using the \texttt{foekfont} package}
\label{sec:using-madsf-pack}

The package offers three commands; the \verb+\foekfamily+ command,
which changes the font family to the FoekFont. A command with an argument,
\verb+\foek+ for typesetting small text pieces in FoekFont, e.g.\
\verb+\foek{cheese}+ produces \foek{cheese}. Finally a fast command
for the typesetting of the magazine title, \verb+\madsfoek+.

\section{License}
\label{sec:license}

The license of the FoekFont is GNU General Public License, GPL.
\clearpage
\appendix

\section{Source files of the foekfont bundle}
\label{sec:source-files-madsf}

\sourcefile{foekfont.sty}
\sourcefile{t1foekfont.fd}
\sourcefile{ot1foekfont.fd}
\sourcefile{foekfont.map}

\end{document}

%%% Local Variables: 
%%% mode: latex
%%% TeX-master: t
%%% End: 
