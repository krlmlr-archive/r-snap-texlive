\documentclass[10pt, onekey]{examdesign}
\Fullpages
\NumberOfVersions{4}
\def\eV{\mbox{eV}}

\begin{document}

\begin{frontmatter}
\vspace*{3in}
\begin{center}
  \huge Modern Physics \\[6pt]
  \Large Final Exam
\end{center}
\vfill
\begin{flushright}
  Name: \rule{3in}{.4pt} \\[10pt]
  Student I.D.: \rule{3in}{.4pt}
\end{flushright}
\end{frontmatter}

\begin{examtop}
\end{examtop}

\begin{shortanswer}[title={\Large Short Answer (10 pts each)},suppressprefix]
Answer all problems in as thorough detail as possible. Be sure to include all
your work. Partial credit will be given even if the answer is not fully correct.

\begin{question}
  Determine the charge in the rest mass of a system consisting of a proton
  and an electron as the two particles combine to form a hydrogen atom. The
  ionization energy of hydrogen is 13.6 eV.
  \examvspace*{1.5in}
  \begin{answer}
    The rest mass of the proton is $1.672649 \times 10^{-27}$kg; that of the
    electron is $9.109534 \times 10^{-31}$kg. When the electron and proton
    combine to form a hydrogen atom, the ionization energy of 13.6 eV is
    released as ultraviolet radiation. The rest mass of the hydrogen atom is
    therefore smaller than the sum of the electron and proton rest masses by
    the amount
    \[
    \Delta M = {(13.6 \eV)(1.6 \times 10^{-19} J/\eV) \over (3 \times 10^8 m/s)^2}
             = 2.42 \times 10^{-35}kg
    \]
  \end{answer}
\end{question}

\begin{question}
  The Eiffel Tower in Paris is 300m tall. What is the fractional gravitational
  red shift due to this elevation difference?
  \examvspace*{1in}
  \begin{answer}
    \[
    {\delta\nu \over \nu} = {gH \over c^2} = {(9.8m/s^2)(300m) \over
                            (3 \times 10^8 m/s)^2} = 3.27\times 10^{-14}
    \]
  \end{answer}
\end{question}

\begin{question}
  Derive Stefan's law from Planck's law.  Using that result, obtain an
  expression for the Stefan-Boltzmann constant $\sigma$ in terms of known
  physical constants and determine its value.
  \examvspace*{1.5in}
  \begin{answer}
    The radiance
  \end{answer}
\end{question}

\begin{question}
  Calculate the density of an object of one solar mass whose radius is the
  critical Schwarzschild radius.  Compare this density with the nuclear
  density of approximately $2.3 \times 10^{17} kg/m^3$.
  \examvspace*{1.5in}
  \begin{answer}
    The mass of the sun is $2 \times 10^{30} kg$. The universal gravitational
    constant is $6.67 \times 10^{-11} N\cdot m^2/kg^2$.

    The Schwarzschild radius of one solar mass is
    \begin{eqnarray*}
    R_c & = & {2(2 \times 10^{30}lg)(6.67 \times 10^{-11} N\cdot m62/kg^2)
              \over (3 \times 10^8 m/s)^2} \\
        & = & 2.96 \times 10^3 m \approx 3 km
    \end{eqnarray*}
    If the sun collapsed to a sphere of 3 km radius without loss of mass, it
    would then be a black hole. The mass density would be
    \[
    \rho = {M \over V} = {2 \times 10^{30} kg \over ({4\pi \over 3})
                         (2.96 \times 10^3 m)^3} = 1.84 \times 10^{19} kg/m^3
    \]
  \end{answer}
\end{question}
\end{shortanswer}

\begin{endmatter}
  \vspace*{1.5in}
  \centerline{\Large A Partial List of Fundamental Constants}
  \bigskip
  \begin{center}
  \begin{tabular}{lcl}
    Constant & Symbol & Approximate Value \\ \hline
    Speed of light in vacuum & $c$ & $3.00 \times 10^8$m/s \\
    Permeability of vacuum & $\mu_0$ & $12.6 \times 10^{-7}$H/m \\
    Permittivity of vacuum & $\epsilon_0$ & $8.85 \times 10^{-12}$F/m \\
    Magnetic flux quantum & $\phi_0 = {h \over 2e}$ & $2.07 \times 10^{-15}$Wb \\
    Electron mass & $m_e$ & $9.11 \times 10^{-31}$kg \\
    Proton mass & $m_p$ & $1.673 \times 10^{-27}$kg \\
    Neutron mass & $m_n$ & $1.675 \times 10^{-27}$kg \\
    Proton-electron mass ratio & $m_p \over m_e$ & 1836
  \end{tabular}
  \end{center}
\end{endmatter}

\end{document}
