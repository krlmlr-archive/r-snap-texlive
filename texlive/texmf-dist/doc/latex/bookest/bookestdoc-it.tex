\documentclass[a4paper,oneside,centered,noparindent,noparskip]{bookest}

\usepackage[utf8x]{inputenc}
\usepackage[italian]{babel}
\usepackage{palatino}
\usepackage{guit}

\hypersetup{
pdftitle={The bookest class},
pdfsubject={Un'estensione per la classe book},
pdfauthor={Riccardo Bresciani},
pdfkeywords={TeX, LaTeX, pdfLaTeX, book, bookest},
pdfstartview=FitV,
%colorlinks
}

\setoddheadtext{{\colorA The \texttt{bookest} class --- Versione 1.0.4\hfill Riccardo Bresciani}}
\setoddfoot{\hfill{\colorA\thepage}\hfill}

\makeatletter
\renewcommand \thesection{\@arabic\c@section.}
\renewcommand\thesubsection{\thesection\@arabic\c@subsection}
\makeatother

\newcommand{\tA}[1]{\texttt{\colorA #1}}

\begin{document}
 \chapter*{The \texttt{bookest} class\\{\Large Versione 1.1 --- \today}\\{\Large Riccardo Bresciani}}
La classe \texttt{bookest} è un'estensione della classe standard \texttt{book}, classe alla quale si appoggia e che viene caricata con le opzioni di default.

\ppar
Le estensioni fornite dalla classe riguardano:
\begin{enumerate}
 \item colori;
 \item layout del documento;
 \item testatine e pie' di pagina;
 \item layout della copertina;
 \item \dots
\end{enumerate}

La pagina web di \texttt{bookest} è \url{http://tex.boris-web.net/bookest}.

\section{Colori}
\texttt{bookest} richiede il pacchetto \texttt{color} per fornire il supporto del colore al documento\footnote{Per l'uso con pdf\TeX\ fino alla versione 1.30, l'uso del pacchetto \texttt{pdfcolmk} --- segnalato da Massimiliano Dominici (\GuIT) --- permette di ovviare al mancato supporto di \emph{colorstack} da parte delle versioni più vecchie di pdf\TeX.}.

\ppar
Vengono definiti in particolare i colori \texttt{A} e \texttt{B}, che vengono utilizzati nella definizione dei comandi di sezionamento e nella colorazione di alcuni elementi del testo --- per esempio i righelli di \texttt{footnote} o le \emph{label} degli elenchi \texttt{itemize}, \texttt{enumerate} e \texttt{description}.

\ppar
Di default il documento è in bianco e nero, l'utente può però definire i colori a proprio piacimento utilizzando i comandi elencati di seguito in \ref{coloricomandi}.

\ppar
\texttt{bookest} fornisce delle combinazioni predefinite di colori, attivabili con le relative opzioni della classe elencate in \ref{coloriopzioni}. L'utilizzo di queste opzioni richiede il pacchetto \texttt{hyperref} dal momento che vengono settati i colori per i link ed i riferimenti (colore \texttt{A}) e per le citazioni (colore \texttt{B}).

\subsection{Comandi}\label{coloricomandi}
\begin{description}
 \item[\texttt{\textbackslash colorA}, \texttt{\textbackslash colorB}, \texttt{\textbackslash black}] dichiarano i colori utilizzati dalla classe (\texttt{A} e \texttt{B} vuoti di default e definibili dall'utente, più il nero);
 \item[\texttt{\textbackslash setcolorA\{\textit{<R> <G> <B>}\}}, \texttt{\textbackslash setcolorB\{\textit{<R> <G> <B>}\}}] specificano i colori \texttt{A} e \texttt{B} in formato RGB;
 \item[\texttt{\textbackslash SETcolorA\{\textit{<comando>}\}}, \texttt{\textbackslash SETcolorB\{\textit{<comando>}\}}] ridefiniscono i comandi \texttt{\textbackslash colorA} e \texttt{\textbackslash colorB} in \texttt{\textit{<comando>}};
\end{description}

\subsection{Opzioni}\label{coloriopzioni}
\begin{description}
 \item[\texttt{noitemcolor}] annulla la colorazione degli elenchi \texttt{itemize}, \texttt{enumerate} e \texttt{description};
\end{description}
\ppar
\begin{description}
 \item[\texttt{blue}, \texttt{green}, \texttt{red}] sono temi di colore generici;
 \item[\texttt{guitgreen}] è un tema di colore che riprende i colori del Gruppo Utilizzatori Italiani di \TeX\ (\GuIT).
 \item[\texttt{sssupcolor1}, \texttt{sssupcolor2}, \texttt{sssupcolor3}] sono temi di colore che riprendono i colori del logo della Scuola Superiore Sant'Anna;
 \item[\texttt{enscblue}] è un tema di colore che riprende il colore del logo dell'École Normale Supérieure di Cachan.
\end{description}

\section{Layout del documento}
\texttt{bookest} fornisce opzioni per controllare:
\begin{itemize}
 \item il fronte-retro (eredita quelle di \texttt{book});
 \item i margini;
 \item l'interlinea;
 \item l'indentazione nei paragrafi e la loro spaziatura reciproca;
 \item la creazione di diverse versioni del documento (libro o articolo).
\end{itemize}

\texttt{bookest} fornisce inoltre due comandi per inserire un'immagine o una scritta nel documento come filigrana.

\subsection{Opzioni}
\begin{description}
 \item[\texttt{oneside}, \texttt{twoside}] abilita o disabilita il fronte-retro (default: \texttt{twoside});
\end{description}
\ppar
\begin{description}
 \item[\texttt{centered}] imposta i margini della pagina (2.5 cm ai lati, 3 cm sopra e sotto) --- richiede il pacchetto \texttt{geometry};
 \item[\texttt{left5mm}] imposta i margini della pagina considerando 5 mm per la rilegatura (3 cm a sinistra, 2 cm a destra e 3 cm sopra e sotto) --- richiede il pacchetto \texttt{geometry};
 \item[\texttt{left8mm}] imposta i margini della pagina considerando 8 mm per la rilegatura (3.3 cm a sinistra, 1.7 cm a destra e 3 cm sopra e sotto) --- richiede il pacchetto \texttt{geometry};
\end{description}
\ppar
\begin{description}
 \item[\texttt{onehalfspacing}] imposta l'interlinea a 1.5 --- richiede il pacchetto \texttt{setspace};
 \item[\texttt{doublespacing}] imposta l'interlinea a 2 --- richiede il pacchetto \texttt{setspace};
\end{description}
\ppar
\begin{description}
 \item[\texttt{noparindent}] annulla l'indentazione dei paragrafi;
 \item[\texttt{noparskip}] annulla lo spazio verticale tra i paragrafi.
\end{description}
\ppar
\begin{description}
 \item[\texttt{article}] crea un output ``article-like'';
 \item[\texttt{nomatter}] annulla le distinzioni tra \textit{front matter}, \textit{main matter} e \textit{back matter}.
\end{description}

\subsection{Comandi}
\begin{description}
 \item [\texttt{\textbackslash shipouttext\{\textit{<rot>}\}\{\textit{<sc>}\}\{\textit{<testo>}\}}] inserisce in ogni pagina il testo \texttt{\textit{<testo>}} come filigrana, ruotandolo di \texttt{\textit{<rot>}} gradi in senso antiorario e applicando un fattore di scala \texttt{\textit{<sc>}}. Il colore di default è grigio al 5\% --- richiede i pacchetti \texttt{everyshi} e \texttt{color};
 \item [\texttt{\textbackslash shipoutimage\{\textit{<opzioni>}\}\{\textit{<file>}\}}] inserisce in ogni pagina l'immagine \texttt{\textit{<file>}} come filigrana, utilizzando \texttt{\textit{<opzioni>}} come opzioni \texttt{\textbackslash includegraphics} --- richiede i pacchetti \texttt{everyshi} e \texttt{color}.
\end{description}

\ppar
\begin{description}
 \item [\texttt{\textbackslash notinarticle\{\textit{<codice>}\}}] il codice \LaTeX\ \texttt{\textit{<codice>}} non compare nel documento compilato con l'opzione \texttt{article}.
 \item [\texttt{\textbackslash nomatter}] annulla le distinzioni tra \textit{front matter}, \textit{main matter} e \textit{back matter}.
\end{description}

\section{Testatine e pie' di pagina}
\texttt{bookest} fornisce dei comandi per permettere all'utente di impostare agevolmente testatine e pie' di pagina, nonché un'impostazione predefinita diversa da quella di \texttt{book}\footnote{Per utilizzare gli stili predefiniti di \texttt{book} è sufficiente utilizzare il comando \texttt{\textbackslash pagestyle\{\textit{<stile>}\}}.}.

\ppar
Ridefinisce inoltre l'intestazione dei capitoli e lo stile \texttt{plain} per integrarvi i colori.

\subsection{Comandi}
\begin{description}
 \item[\texttt{\textbackslash setoddhead}, \texttt{\textbackslash setevenhead}] definiscono le testatine delle pagine dispari e pari;
 \item[\texttt{\textbackslash oddheadtext}] è il testo che verrà utilizzato nelle testatine delle pagine dispari (default: \texttt{\{\textbackslash colorA\{ \textbackslash slshape\textbackslash rightmark\}\textbackslash hfill\textbackslash thepage\}});
 \item[\texttt{\textbackslash evenheadtext}] è il testo che verrà utilizzato nelle testatine delle pagine pari (default nel caso \texttt{oneside}: \texttt{\textbackslash oddheadtext}; nel caso \texttt{twoside}: \texttt{\{\textbackslash colorA\textbackslash thepage\textbackslash hfill\textbackslash slshape\textbackslash leftmark\}});
 \item[\texttt{\textbackslash setoddheadtext}, \texttt{\textbackslash setevenheadtext}] settano il testo in \texttt{\textbackslash oddheadtext} e \texttt{\textbackslash evenheadtext};
 \item[\texttt{\textbackslash setoddfoot}, \texttt{\textbackslash setevenfoot}] definiscono i pie' di pagina delle pagine dispari e pari;
 \item[\texttt{\textbackslash oddfoottext}, \texttt{\textbackslash evenfoottext}] sono i testi che verranno utilizzati nei pie' di pagina dispari e pari (default: vuoto);
 \item[\texttt{\textbackslash setoddfoottext}, \texttt{\textbackslash setevenfoottext}] settano il testo in \texttt{\textbackslash oddfoottext} e \texttt{\textbackslash evenfoottext};
 \item[\texttt{\textbackslash setleftmark}, \texttt{\textbackslash setrightmark}] settano il testo in \texttt{\textbackslash leftmark} e \texttt{\textbackslash rightmark};
 \item[\texttt{\textbackslash makeheadrule}] definisce la linea orizzontale utilizzata nelle testatine (default: \texttt{\{\textbackslash colorB\textbackslash hrule \textbackslash @width \textbackslash textwidth \textbackslash @height 0.4pt \textbackslash vskip-0.4pt\}});
 \item[\texttt{\textbackslash makefootrule}] definisce la linea orizzontale utilizzata nei pie' di pagina (default: \texttt{\textbackslash makeheadrule});
\end{description}

\section{Layout della copertina}
\texttt{bookest} fornisce comandi per permettere all'utente di personalizzare con facilità la copertina del proprio documento, in particolare a partire da layout predefiniti che possono essere attivati con le opzioni in \ref{copertinaopzioni}.

\ppar
Il layout predefinito ha autore e titolo in alto al centro e a pie' di pagina viene posizionato un \emph{footer} composto dal contenuto di \texttt{\textbackslash titlingpageprefooter} e quello di \texttt{\textbackslash titlingpagefooter} separati da una linea orizzontale. Tra il titolo ed il \emph{footer} viene posizionato il contenuto di \texttt{\textbackslash titlingpagemiddle}.

Le varie opzioni permettono di variare il posizionamento del logo; per ciascuna opzione \texttt{\textit{<optlogo>}} in \ref{copertinaopzioni} esiste una variante \texttt{\textit{<optlogo>}-nofooter} in cui il \emph{footer} non è presente.

\subsection{Comandi}
\begin{description}
 \item[\texttt{\textbackslash inslogo\{\textit{<file>}\}}] inserisce l'immagine \texttt{\textit{<file>}} con opzioni \texttt{\textbackslash includegraphics} definite in precedenza e usate per il logo (default: \texttt{width=0.6\textbackslash paperwidth});
 \item[\texttt{\textbackslash setlogooptions\{\textit{<opzioni>}\}}] definisce \texttt{\textit{<opzioni>}} come le opzioni di \texttt{\textbackslash includegraphics} che verranno usate da \texttt{\textbackslash inslogo};
 \item[\texttt{\textbackslash logo}, \texttt{\textbackslash leftlogo}, \texttt{\textbackslash rightlogo}] sono il percorso (relativo o assoluto) dell'immagine che verrà utilizzata come logo a seconda delle opzioni scelte (default per \texttt{\textbackslash logo} è il percorso relativo \texttt{logo}, default per gli altri è \texttt{\textbackslash logo});
 \item[\texttt{\textbackslash setlogo\{\textit{<percorso>}\}}, \texttt{\textbackslash setleftlogo\{\textit{<percorso>}\}}, \texttt{\textbackslash setrightlogo\{\textit{<percorso>}\}}] settano a \texttt{\textit{<percorso>}} il contenuto di \texttt{\textbackslash logo}, \texttt{\textbackslash leftlogo}, \texttt{\textbackslash rightlogo};
 \item[\texttt{\textbackslash titlingpagemiddle}] è il testo che verrà posizionato a metà della \emph{titling page};
 \item[\texttt{\textbackslash settitlingpagemiddle}] setta il testo in \texttt{\textbackslash titlingpagemiddle};
 \item[\texttt{\textbackslash titlingpageprefooter}] è il testo che verrà posizionato prima del \emph{footer} nella \emph{titling page};
 \item[\texttt{\textbackslash settitlingpageprefooter}] setta il testo in \texttt{\textbackslash titlingpageprefooter};
 \item[\texttt{\textbackslash titlingpagefooter}] è il testo che verrà utilizzato come \emph{footer} nella \emph{titling page} (default: \texttt{\textbackslash today});
 \item[\texttt{\textbackslash settitlingpagefooter}] setta il testo in \texttt{\textbackslash titlingpagefooter};
 \item[\texttt{\textbackslash settitlingpagetitle}] definisce il formato del titolo all'interno della \emph{titling page};
 \item[\texttt{\textbackslash titling}] inserisce la \emph{titling page}.
\end{description}

\subsection{Opzioni}\label{copertinaopzioni}
\begin{description}
 \item[\texttt{nofooter}] variante senza \emph{footer} del layout predefinito;
 \item[\texttt{logo}, \texttt{logo-nofooter}] aggiunge al layout predefinito il logo in \texttt{\textbackslash logo} sotto il titolo (e relativa variante \texttt{nofooter}) --- richiede il pacchetto \texttt{graphicx};
 \item[\texttt{logo-bg}, \texttt{logo-bg-nofooter}] aggiunge al layout predefinito il logo in \texttt{\textbackslash logo} in background (e relativa variante \texttt{nofooter}) --- richiede i pacchetti \texttt{graphicx} ed \texttt{eso-pic};
 \item[\texttt{logo-topl}, \texttt{logo-topl-nofooter}] aggiunge al layout predefinito il logo in \texttt{\textbackslash leftlogo} prima del titolo in alto a sinistra (e relativa variante \texttt{nofooter}) --- richiede il pacchetto \texttt{graphicx};
 \item[\texttt{logo-topc}, \texttt{logo-topc-nofooter}] aggiunge al layout predefinito il logo in \texttt{\textbackslash logo} prima del titolo in alto al centro (e relativa variante \texttt{nofooter}) --- richiede il pacchetto \texttt{graphicx};
 \item[\texttt{logo-topr}, \texttt{logo-topr-nofooter}] aggiunge al layout predefinito il logo in \texttt{\textbackslash rightlogo} prima del titolo in alto a destra (e relativa variante \texttt{nofooter}) --- richiede il pacchetto \texttt{graphicx};
 \item[\texttt{logo-toplr}, \texttt{logo-toplr-nofooter}] aggiunge al layout predefinito il logo in \texttt{\textbackslash leftlogo} in alto a sinistra ed il logo in \texttt{\textbackslash rightlogo} in alto a destra prima del titolo (e relativa variante \texttt{nofooter}) --- richiede il pacchetto \texttt{graphicx}.
 \item[\texttt{logo-toplcr}, \texttt{logo-toplcr-nofooter}] aggiunge al layout predefinito il logo in \texttt{\textbackslash leftlogo} in alto a sinistra, il logo in \texttt{\textbackslash logo} in alto al centro ed il logo in \texttt{\textbackslash rightlogo} in alto a destra prima del titolo (e relativa variante \texttt{nofooter}) --- richiede il pacchetto \texttt{graphicx}.
\end{description}

\section{Miscellanea}
\texttt{bookest} fornisce anche altre piccole scorciatoie, che possono essere utili nell'utilizzo della classe e che vengono elencate qui di seguito:

\subsection{Opzioni}
\begin{description}
 \item[\texttt{noepigraph}] elimina le epigrafi.
\end{description}

\subsection{Comandi}
\begin{description}
 \item[\texttt{\textbackslash setbibname\{\textit{<nome>}\}}] rinomina il titolo della bibliografia in \texttt{\textit{<nome>}};
 \item[\texttt{\textbackslash setcontentsname\{\textit{<nome>}\}}] rinomina il titolo dell'indice in \texttt{\textit{<nome>}};
 \item[\texttt{\textbackslash ppar}] inserisce uno spazio verticale di \texttt{1.5ex} --- utile per esempio con l'opzione \texttt{noparskip};
 \item[\texttt{\textbackslash dimstleftskip}] setta \texttt{\textbackslash leftskip} a \texttt{1cm};
 \item[\texttt{\textbackslash UCase}] fornisce il comando \texttt{\textbackslash MakeUppercase}, che viene invece ridefinito come comando vuoto per dare maggiore flessibilità ai comandi per testatine e pie' di pagina;
 \item[\texttt{\textbackslash epigraph\{\textit{<testo1>}\}\{\textit{<testo2>}\}\{\textit{<ambiente>}\}\{\textit{<l>}\}}] crea un epigrafe, in cui \texttt{\textit{<testo1>}} viene separato da \texttt{\textit{<testo2>}} da una linea orizzontale di colore \texttt{B}. Il tutto ha larghezza \texttt{\textit{<l>}} ed è contenuto nell'ambiente \texttt{\textit{<ambiente>}}.
  \item[\texttt{\textbackslash noepigraph}] elimina le epigrafi;
\end{description}

\subsection{Ambienti}
\begin{description}
 \item[\texttt{abstract}] è un ambiente di larghezza \texttt{0.9\textbackslash textwidth}, con un parametro \texttt{\textit{<titolo>}} che verrà scritto in grassetto prima del testo contenuto nell'ambiente;
\item[\texttt{dimst}] è un ambiente in cui il testo è in \texttt{slshape} e con il margine sinistro maggiorato di 1 cm.
\end{description}

\section{Contatti}
Per commenti, suggerimenti o segnalazione bugs potete contattarmi all'indirizzo \href{mailto:bresciani@sssup.it}{\textit{bresciani@sssup.it}}.

\end{document}
