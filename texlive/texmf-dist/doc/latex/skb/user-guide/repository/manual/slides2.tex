This provides the following combinations for \cmd{\skbslide}
\begin{skbnotelist}
 \item Slide only: argument 1 has the name for the \skbacft{ISO:PDF}, argument 2 is empty
 \item Annotation only: argument 1 is empty, argument 2 has the name for the \TeX~file, option \skbem[code]{annotate} used
 \item Slide with Annotation 1: argument 1 has the name for the \skbacft{ISO:PDF}, argument 2 has the name for the \TeX~file, option \skbem[code]{annotate} used
 \item Slide with Annotation 2: argument 1 has the name for the \skbacft{ISO:PDF}, argument is empty, option \skbem[code]{annotate} used
 \item do nothing: leave both arguments empty
\end{skbnotelist}

\opt{not}{This slide show a few examples.}%
\opt{text}{\noindent Some examples on how to use \cmd{\skbslide:} \lstinputlisting[style=genericLN,language=TeX]{\skbfileroot{examples/skbslide}}}%
In line 1 and 2 we load \skbem[code]{myslides/slide1.pdf} and \skbem[code]{myslides/slide2.pdf}
from the default directory without any annotations and clear the page after that. In line 3 we load \skbem[code]{myslides/slide2.pdf} and request this slide to 
be annotated without giving a specific file name, thus loading \skbem[code]{myslides/slide3.tex}, both files from the default slides directory.
In lines 4\&5 we change the directory for the notes and request a particular file to be loaded, resulting in the slide loaded as \skbem[code]{myslides/theme1.pdf} from the 
slides directory and the annotations loaded as \skbem[code]{text/theme1.tex} from the repository. Finally, in lines 6\&7 we change both folders to the repository, this loading 
\skbem[code]{text/theme2.pdf} and \skbem[code]{text/theme2.tex} from the repository.


\opt{note}{\skbheading{Slides and Citations}}

\DescribeMacro{\skbslidecite}
The macro \cmd{\skbslidecite} provides some simple means to add citations to annotated slides.
It takes two arguments, the first one for the type of citation and the second one for the actual citation.
\opt{text}{Here a simple example:}\opt{note}{This second block on this slide shows a simple example.}
\opt{text}{\lstinputlisting[style=genericLN,language=TeX]{\skbfileroot{examples/skbslidecite}}}%
The first line states that the slide contains material from a book of Tannenbaum and the second line 
states that the annotation contains material from an \skbacft{organisation:IETF} \skbacft{IETF:RFC}
standard documents (\cite{standard:IETF:RFC:1155}). Since this
macro is very simple, any content can be given for the two arguments.