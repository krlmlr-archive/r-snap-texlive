\skbheading{User Manual}

The \skbacft{A3DS:SKB} provides macros that simplify file handling and hide some \LaTeX~
code (i.e. for figures) from the user, thus helping everyone to focus on
the actual document one wants to write. There are a few macros, and they
can be catagorised as follows.
\begin{skbnotelist}
  \item Installation, rebuilt and configuration: this part deals with the 
        installation of the package with your local \LaTeX~distribution,
        the rebuilt of the styles, classes and documentation (all of them are
        provided, but you never know, it might become useful) and the configuration
        of the \skbacft{A3DS:SKB} using configuration files and the macro \cmd{\skbconfig}.
  \item Files, figures and slides: the combination of \cmd{\skbheading} and
        \cmd{\skbinput} will process files and the document level of headings.
        The macro \cmd{\skbfigure} makes it easy to include figures in your document 
        and the macro \cmd{\skbslide} helps with \skbacft{ISO:PDF} slides and annotations (if 
        you are not using a classic \LaTeX solution such as the beamer package).
  \item Filenames, acronyms and references: here we deal with macros that provide 
        access to the path and filenames the \skbacft{A3DS:SKB} maintains, plus loading acronym and 
        reference databases.
  \item Other useful macros: there are some more macros that complete the \skbacft{A3DS:SKB}. There 
        are little helpers for emphasising text, limiting the space between items in some
        list environments, putting acronyms into footnotes, filling meta information for 
        \skbacft{ISO:PDF} files and last not least macros that help dealing with optional and conditional text.
\end{skbnotelist}

For the impatient, we start with a few examples. The first one shows how to use the \skbacft{A3DS:SKB} to
produce a simple article. The second one exmplains how the documentation for the \skbacft{A3DS:SKB} is 
created using most of the \skbacft{A3DS:SKB} macros. Then we detail the usage of all the macros, following 
the above introduced categorisation.