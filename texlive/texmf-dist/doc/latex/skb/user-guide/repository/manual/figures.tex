\opt{text}{\skbheading{Figures}}

\DescribeMacro{\skbfigure}
The classic way to add figures to your document is to have a \skbacft{ISO:PDF} or \skbacft{ISO:PNG} or \skbacft{ISO:JPG} 
file ready, include it using \cmd{\includegraphics} while putting it into a box
to resize it (i.e. to the width of the text in your document), putting this
very box into a figure environment so that \LaTeX~can process list of figures etc.
and of course adding lable and caption to it.%
\opt{text}{%
  Here is some \LaTeX~example, which also uses the center environment:
  \lstinputlisting[style=generic,language=TeX]{\skbfileroot{examples/figure-classic}}
}%
\opt{note}{The first block of this slide show a \LaTeX~example, which also uses the
center environment.}%

With the \skbacft{A3DS:SKB} macro \cmd{\skbfigure} things become a little bit simplier.
takes a number of options and one argument.%
\opt{text}{The following code}\opt{note}{The second block on this slide}
shows a number of examples for using this macro.

\lstinputlisting[style=genericLN,language=TeX]{\skbfileroot{examples/skbfigure}}

Let's start with the easy usage, simply using the one argument to load a figure,
as shown in line 1.
This call will simply use \cmd{\includegraphics} and \cmd{\resizebox} to load the figure
\skbem[code]{myfig} from the figure directory, so we do not need to say
\skbem[code]{../figures} anymore. To use the figure and the center environment, we simply
add two options requesting exactly that, as shown in line 2.
In other words, using the option \skbem[code]{figure} will put the \skbem[code]{myfig} in
a figure environment and using the option \skbem[code]{center} will center the figure.

Similar for width and height information. Say the figure should be rescaled to the width 
of the text in your document you simply add width to the options, as shown in line 3
Use \skbem[code]{height} for height or both options if required. Note that the width and 
the hight are automatically applied to the \cmd{\resizebox} and \cmd{\includegraphics}.
You can also add caption and label information using the respective options (lines 4 and 5).
Now we will have the same result as the classi \LaTeX~example. You can also add the 
required position for your figure, if using the figure environment applying the option
\skbem[code]{position} with the usual paramters, including \skbem[code]{H} from the float
environment.

\opt{text}{
  \begin{table}[ht!]
    \caption{Options for skbfigure}
    \label{tab:skbfigure:options}
    \begin{tabular*}{\textwidth}{ >{\small}p{.2\textwidth} >{\small}p{.75\textwidth}}
      \toprule
      \textbf{Option} & \textbf{Description} \\
      \midrule
      \skbinput[from=rep]{manual/skbfigure-opt-table}
      \bottomrule
    \end{tabular*}
  \end{table}
}

\opt{text}{\skbinput[from=rep]{manual/figures2}}