\documentclass[11pt]{report}
\usepackage[polutonikogreek,english]{babel}
\usepackage{metre}
\textwidth 5in
\topsep 0pt
\partopsep 0pt
\pagestyle{empty}
\Magnitudo{+1}
\newcommand{\Greek}[1]{\foreignlanguage{polutonikogreek}{#1}}
%
%   shorthands for the `metrike' environment
%
\renewcommand{\(}{\begin{metrike}}
\renewcommand{\)}{\end{metrike}}
% 
%   some convenient shorhands for metres
%
\newcommand{\dactyl}{\metra{\m\b\b}}
\newcommand{\spondee}{\metra{\m\m}}
\newcommand{\enoplion}{\metra{\m\m\b\b\m\b\m\m}}
%
\begin{document}
%
\hrule height 1.5pt \vspace{1ex}
{\large From D.L. Page's commentary to \textit{Medea} (1st ed.), pp.~182-183.}\par 
\vspace{1ex}
Copyright Oxford University Press.\par
Reproduced by permission of Oxford University Press.\par 
\vspace{1ex} \hrule height 1.5pt
\vspace{5ex}
{\noindent\textit{The Parodos.}}\par
\vspace{1ex}
As in Aeschylus' \textit{Septem} and Euripides' \textit{Bacchae}, the
Chorus begins with a short lyric to which there is no antistrophic 
reply, \oldstylenums{131} sqq.\,:\par
%
%
\begin{tabbing}
%
%   defining the tab stops (note \kill at the end)
%
\hspace*{2em}\oldstylenums{131}\hspace{.25em}%
\=\metra{\b\b\m}\hspace{.25em}%
\=\metra{\m\m}\hspace{.25em}%
\=\metra{\b\b\m}\hspace{.25em}%
\=\metra{\b\bm}\hspace{2em}%
\=Anapaests\kill%
%
\oldstylenums{131}%
\>\metra{\b\b\m}%
\>\metra{\m\m}%
\>\metra{\b\b\m}%
\>\metra{\b\b\m}%
\>Anapaests\\%
%
\>\metra{\m\m}%
\>\metra{\m\m}%
\>\metra{\m\b\b}%
\>\metra{\m\b\b}%
\>Anapaests\\%
%
\>\metra{\m\b\b}%
\>\metra{\m\m}%
\>\metra{\m\m}%
\>\metra{\m}%
\>Paroemiac\\%
%
%   redefining the tab stops (note \kill at the end)
%
\oldstylenums{131}\hspace{.25em}%
\=\metra{\m\b\b}\hspace{.25em}%
\=\metra{\m\b\b}\hspace{.25em}%
\=\metra{\m\b\b}\hspace{.25em}%
\=\metra{\m\b\b}\hspace{.25em}%
\=\metra{\m\b\m}\hspace{2em}%
\=Lyric dactyl\kill%
%
\>\metra{\m\b\b}%
\>\metra{\m\b\b}%
\>\metra{\m\b\b}%
\>\metra{\m\b\b}%
\>\metra{\m\b\m}%
\>Lyric dactyl\\%
%
\>\metra{\m\b\b}%
\>\metra{\m\b\b}%
\>\metra{\m\b\b}%
\>\metra{\m\b\b}%
\>\metra{\m\b\m}%
\>Lyric dactyl\\%
%
\>%
\>\metra{\m\b\b}%
\>\metra{\b\m\c\m\b\m\c\b\m\m}%
\>%
\>%
\>Iamb.\\%
\end{tabbing}\par
At \oldstylenums{148} the Chorus begins a strophe which opens with
anapaests:\par
\vspace{1ex}
{%    here we use the TeX primitive \halign
\tabskip .2em %
\halign {%
\hspace*{2em}#\hspace*{4em}&&#\hfill\cr% the pattern
\oldstylenums{148}&\dactyl &\spondee &\spondee &\spondee \cr
&\spondee &\spondee &\spondee &\spondee\cr
&\spondee &\spondee \cr
}% \halign
}

Three enoplia follow:{\obeylines
\hspace{6.25em}\enoplion
\hspace{6.25em}\enoplion
\hspace{6.25em}\enoplion
}
Then a reizianum, which is similar in scansion to the enoplion
for the first six syllables:{\obeylines
\hspace{6.25em}\metra{\m\m\b\b\m\m}
}

Then comes a trochaic metron:{\obeylines
\hspace{6.25em}\metra{\m\b\m\b}
}\noindent followed by three more enoplia{\obeylines
\hspace{6.25em}\enoplion
\hspace{6.25em}\enoplion
\hspace{6.25em}\enoplion
}\noindent then a molossus:{\obeylines
\hspace{6.25em}\metra{\m\m\m}
}

And the strophe is closed by a line which is similar in scansion to
the enoplia, being in fact a glyconic followed by a spondee:{\obeylines
\hspace{6.25em}\metra{\m\m\m\b\b\m\b\m\m\m}
}

The Parodos is ended by a short choral lyric to which there is no
antistrophic reply, \oldstylenums{204} sqq.\,:\par 
\vspace{1ex}
%\Elevatio{.5}% 
%
{%   here we use the TeX primitive \halign
\tabskip 1em plus 2em minus .5em%
\halign to \hsize{%
#\hfill&#\hfill&\quad\quad#\hfill\cr% the pattern
\oldstylenums{204}&\(\=aq\=an \=a-\)\-&molossus\cr
&\-\(-\-i\=on p\-ol\=u\C st\-on\=on g\-o\=wn\)&iambic\cr
&\(l\-ig\-ur\-a d \-aq\-e\-a\c m\-og\-er\-a b\-o\={a|}\)&iambic\cr
% 
\noalign{Then another iambic metron leads to a reizianum, similar in
scansion to a dactylo-epitrite:}
%
&\(t\-on \=en l\-eqe\=i\c pr\-od\-ot\=an k\-ak\-on\=umf\=on\)\cr
%
\noalign{Then follow two more iambic metra:}
%
&\(j\-e\=okl\-ute\=i\c d \-ad\-ik\-a p\-ajo\=usv\C\-a\)\cr
%
\noalign{Then another quasi-dactylo-epitrite:}
%
&\(t\=an Z\=hn\-os \=or\C k\-i\=an J\-em\-in \=a n\-in \-eb\=asv\-en\)\cr
%
\noalign{\noindent followed by a hemiepes, repeating the scansion of
%
\-\-\ \(-\=an J\-em\-in \=a n\-in \-eb\=a-\)\ \-\-\,:}
&\(\=Ell\-ad \-es \=ant\-ip\-or\=on\)\cr
%
\noalign{Then a return to the opening iambic rhythm:}
%
&\(d\-i \-al\-a n\-uq\-i\-on\c \-ef \=alm\-ur\=an\)\cr
%
\noalign{\noindent and finally a pherecratean clausula:}
%
&\(P\=onto\=u kl\={h|}d> \-ap\-er\=ant\=on\)\cr
}% halign
}%
\vspace{5ex}
\hrule height 1.5pt \vspace{1ex}
{\large From E.R. Dodds' commentary to \textit{Bacchae} (2nd ed.), pp.~72-73.}\par 
\vspace{1ex}
Copyright Oxford University Press 1960.\par
Reproduced by permission of Oxford University Press.\par 
\vspace{1ex} \hrule height 1.5pt
\vspace{5ex}
\Greek{pros'imion} and \Greek{str}+\Greek{>ant.\,a$^\prime$}: ionics
\textit{a minore}.  This is the characteristic metre of the
\textit{Bacchae}.  Its use was doubtless traditional for Dionysiac
plays, since it is proper to Dionysiac cult-hymns; it appears
in the refrain of Philodamos' Delfic hymn to Dion. and in the
Iacchos-hymn of the \textit{Frogs} (\oldstylenums{324} ff.).
The standard line consists of two \Greek{m'etra} (\oldstylenums{65}
and \oldstylenums{69} have three), either

\begin{list}{}{}
%
\item[(\textit{a})] `straight' form, \metra{\b\b\m\m \c \b\b\m\m}
(Hor., \textit{Odes} \oldstylenums{3}, \oldstylenums{12} is composed
entirely in this form),
%
\item[or (\textit{b})] `anaclastic' or broken form (anacreontics),
\metra{\b\b\m\b\c\m\b\m\m}.
\end{list}
%
(\textit{b}) yields a swifter and more emotional rhythm than
(\textit{a}). Both forms are illustrated by Kipling's line:\par
\begin{metrica}
\-And th\-e s\='unsh\=ine\c \-and th\-e p\='almtre\=es \-and th\-e
t\=inkl\-y t\=empl\-e b\=ells\ \metra[\m{-2}\ss{1.225}\ms{st}]{\K\m}%
\end{metrica}\par
Variations include (\oldstylenums{1}) `catal\=exis', suppressing the
last syllable of the second \Greek{m'etron} and occasionally of the
first (\oldstylenums{64} = \metra{\b\b\m\k\m \c \b\b\m\k\m});
(\oldstylenums{2}) `syncopation', suppressing a short syllable within
the \Greek{m'etron} (\oldstylenums{71} = \metra{\b\b\m\b\c\m\p\m\m});
(\oldstylenums{3}) `resolution' of either long into two shorts
(\oldstylenums{79} = \metra{\b\b\bb\m\c\b\b\m\m}); (\oldstylenums{4})
the reverse substitution of a long for two shorts
(\oldstylenums{81} = \metra{\m\m\m\c\b\b\m\m}).\ [\ldots]

\Greek{str.}+\Greek{>ant.\,b$^\prime$}. In \oldstylenums{105}%
-\oldstylenums{10} = \oldstylenums{120}-\oldstylenums{5} the first element
is a choriambus \metra{\m\b\b\m} (resolved in
\oldstylenums{107}-\oldstylenums{122} into \metra{\bb\b\b\bb}); the
second is either an iambic \Greek{m'etron}
(\oldstylenums{109}-\oldstylenums{124}, with the last long resolved)
or a bacchius \metra{\b\m\m} (perhaps equivalent to a syncopated
iambic \Greek{m'etron} \metra{\b\m\p\m}). 

In the remainder, ionics (\oldstylenums{113}-\oldstylenums{14})
are mixed with glyconics (see on stasimon \oldstylenums{1},
\oldstylenums{402} ff., which for the beginner forms a
better introduction to the metre) and dactyls. The last
named describe the rush of the dancers to the mountains
(\oldstylenums{116}-\oldstylenums{17}); and in general the excited and
swiftly changing rhythms seem to reflect, as Deichgr\"aber observes,
the Dionysiac unrest.
\strut\strut\par
Scheme:
\begin{tabbing}
%
%    defining the tab stops (note \kill at the end)
%
\oldstylenums{115}-\oldstylenums{130}\hspace{2em}
\=\metra{\m\b\b\bbm}%
\=\metra{\b\m\s\b\b\m}\hspace{3em}%
\=? glyc. with trisillabic close. \kill
%
\oldstylenums{105}-\oldstylenums{120}%
\>\>\metra{\m\b\b\m}\'\metra{\b\m\m}%
\>chor{.}+ba.\\
%
\>\>\metra{\m\b\b\m}\'\metra{\b\m\m}%
\>chor{.}+ba.\\
%
\>\>\metra{\bb\b\b\bb}\'\metra{\b\m\m}%
\>chor{.}+ba.\\
%
\>\>\metra{\m\b\b\bbm}\'\metra{\b\m\m}%
\>chor{.}+ba.\\
%
\>\>\metra{\m\b\b\m}\'\metra{\b\m\b\bb}%
\>chor{.}+iamb.\\
%
\oldstylenums{110}-\oldstylenums{125}%
\>\>\metra{\m\b\b\m}\'\metra{\b\m\m}%
\>chor{.}+ba.\\
%
\>\metra{\m\m\s\m\b\b\m\s\b\m}%
\>\>glyc.\\
%
\>\metra{\bb\b\s\m\b\b\m\s\b\b\m}%
\>\>? glyc. with trisillabic close. \\
%
\>\metra{\m\m\m\s\b\b\m\m\s\b\b\m\m}%
\>\>\oldstylenums{3} ion.\\
%
\>\metra{\bb\m\m\s\b\b\m\m\s\b\b\m\m}%
\>\>\oldstylenums{3} ion.\\
%
\oldstylenums{115}-\oldstylenums{130}%
\>\metra{\bb\b\s\m\b\b\m\s\b\b\m}%
\>\>? glyc. with trisillabic close. \\
%
\>\metra{\m\b\b\s\m\b\b\s\m\b\b\s\m\K\m}%
\>\>\oldstylenums{4} dact.\\
%
\>\metra{\m\b\b\s\m\b\b}%
\>\>\oldstylenums{2} dact.\\
%
\>\metra{\b\m\s\m\b\b\m\s\b\m}%
\>\>glyc.\\
%
\>\metra{\m\m\s\m\b\b\m\s\m}%
\>\>pher.\\
\end{tabbing}
\end{document}
