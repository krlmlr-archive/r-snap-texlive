\documentclass{article}
\usepackage[T1]{fontenc}
\usepackage[utf8]{inputenc}
\usepackage{duerer,listings,multicol}
\title{The \texttt{duerer} package\medskip\\
  \large\LaTeX\ support for the duerer fonts}
\author{version 1.1 \today}
\lstset{language=[latex]tex,breaklines=true}
\date{Palle J\o rgensen}

\begin{document}
\maketitle

\section{Introduction}
\label{sec:introduction}

The \texttt{duerer} package provides support for the duerer (Dürer)
fonts. The duerer fonts are already installed on many systems, this is
only support for using the duerer fonts with \LaTeX.

Please note that the duerer fonts only provides uppercase characters.

The license of the duerer pcakage and the related files is GNU General
Public License.

\section{Using the duerer package}
\label{sec:using-duerer-fonts}

If you want some text typeset with the duerer fonts for a short text
you can use one of the commands

\begin{lstlisting}
  \textdurm{...}, \textdubf{...}, \textdusl{...}
\end{lstlisting}
which typesets the text with Dürer Roman, Dürer Roman Bold and Dürer
Roman Slanted respektively.

Furthermore there are three other duerer font families available
\begin{lstlisting}
  \textdutt{...}, \textdusf{...}, \textduin{...}
\end{lstlisting}
which typesets the text with Dürer Typewriter Type, Dürer Sans Serif,
Dürer Informal respectively.

If you want to typeset longer passages of text with the duerer fonts,
you can use the environments

\begin{lstlisting}
  durmfamily, dusffamily, duttfamily, duinfamily
\end{lstlisting}
Inside durmfamily the normal \LaTeX\ font switches \verb+\slshape+ and
\verb+\bfseries+ works. Furthermore \verb+\emph+ works too.

It is possible to use the commands
\begin{lstlisting}
  \durmfamily, \dusffamily, \duttfamily, \duinfamily
\end{lstlisting}
but these commands also changes the current fontencoding; use with
caution\dots \clearpage
\appendix

\section{Source of the files in the duerer bundle}
\label{sec:source}

\subsection{duerer.sty}
\label{sec:duerer.sty}
\lstinputlisting{duerer.sty}

\subsection{ot1cdr.fd}
\label{sec:ot1cdr.fd}
\lstinputlisting{ot1cdr.fd}

\subsection{ot1cdss.fd}
\label{sec:ot1cdss.fd}
\lstinputlisting{ot1cdss.fd}

\subsection{ot1cdtt.fd}
\label{sec:ot1cdtt.fd}
\lstinputlisting{ot1cdtt.fd}

\subsection{ot1cdin.fd}
\label{sec:ot1cdin.fd}
\lstinputlisting{ot1cdin.fd}
\end{document}

%%% Local Variables: 
%%% mode: latex
%%% TeX-master: t
%%% End: 
