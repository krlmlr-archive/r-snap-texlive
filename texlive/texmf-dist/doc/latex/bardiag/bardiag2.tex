\documentclass[12pt]{article}
%%%%%%%%%%%%%%%%%%%%%%%%%%%%%%%%%%%%%%%%
\usepackage{barddoc}

\title{The \bard package: documentation part 3
       \thanks{This file has version number \barvers and describes \bard ver.\barvers}
}

\usepackage{xr}
\externaldocument{bardiag}


%%%%%%%%%%%%%%%%%%%%%%%%%%%%%%%%%%%%%%%%
\begin{document}
\maketitle

%%%%%%%%%%%%%%%%%%%%%%%%%%%%%%%%%%%%%%%%%%


%#########################################
\appendix \section*{\appendixname}

\mysection{\label{app1}Source file for the example \figref{fig:9} from \texttt{bardiag.tex|.ps}}
\lstinputlisting{figs/examp5.tex}

%\section{\label{app2}Source of the \bard}
%{\small
%\verbatiminput{bardiag.sty}
%}

\clearpage
\mysection{\label{app2}Examples of alternative shapes, 
         defined in \texttt{bardiag.cfg}}
As mentioned above, file \texttt{bardiag.cfg} is automatically loaded.
One can define colors or parameters there, or, say, some new shapes.
To demonstrate a couple of ``alternative'' shapes, we redraw the
\figref{fig:8a} as shown below [\figref{fig:a2:1}].
Use, e.g.,\\ 
\verb+\renewcommand{\ActiveBarPrimitive}{\barArrowRect}+\\
and the provided \texttt{bardiag.cfg} to draw \figref{fig:a2:1:a}.

\renewcommand{\shownumbers}{0}
\begin{figure}[!ht]
\renewcommand{\betweenticks}{50}
\renewcommand{\tdx}{0.4}
\renewcommand{\tdy}{15}
\subfigure[Using \texttt{barArrowRect}]{\label{fig:a2:1:a}
  \centering
  \renewcommand{\frameTD}{0}
  \renewcommand{\ActiveBarPrimitive}{\barArrowRect}
  \renewcommand{\arrowheight}{5\dbar}
  \renewcommand{\placenumber}{\bottom}
    \bardiagrambegin{7.8}{210}{2cm}{1}{1.5}{0.7cm}{0.02cm}
      \drawlevellines
      \baritem{1998}{120}{green}
      \baritem{1999}{123}{red}
      \baritem{2000}{147}{yellow}
      \baritem{2001}{176}{green}
      \baritem{2002}{132}{red} 
      %
      \diagLegendbegin{0}{200}{3.8}
        \diagLegenditem{\it Legend}{red}
      \diagLegendend
  \bardiagramend{{\large Year}}{{\large Income}}
}
\subfigure[Using \texttt{barGradRect}]{\label{fig:a2:1:b}
 \renewcommand{\frameTD}{0}
 \renewcommand{\ActiveBarPrimitive}{\barGradRect}
   \bardiagrambegin{7.8}{210}{2cm}{1}{1.5}{0.7cm}{0.02cm}
      \drawlevellines
      \baritem{1998}{120}{green}
      \baritem{1999}{123}{red}
      \baritem{2000}{147}{yellow}
      \baritem{2001}{176}{green}
      \baritem{2002}{132}{red} 
      %
      \diagLegendbegin{0}{200}{3.8}
        \diagLegenditem{\it Legend}{red}
      \diagLegendend
  \bardiagramend{{\large Year}}{{\large Income}}
}\\
\subfigure[Using \texttt{barGradMidRect}]{\label{fig:a2:1:c}
  \centering
  \renewcommand{\frameTD}{0}
  \renewcommand{\ActiveBarPrimitive}{\barGradMidRect}
    \bardiagrambegin{7.8}{210}{2cm}{1}{1.5}{0.7cm}{0.02cm}
      \drawlevellines
      \baritem{1998}{120}{green}
      \baritem{1999}{123}{red}
      \baritem{2000}{147}{yellow}
      \baritem{2001}{176}{green}
      \baritem{2002}{132}{red} 
      %
      \diagLegendbegin{0}{200}{3.8}
        \diagLegenditem{\it Legend}{red}
      \diagLegendend
  \bardiagramend{{\large Year}}{{\large Income}}
}
\subfigure[Using \texttt{barGradCylinder}]{\label{fig:a2:1:d}
 \renewcommand{\frameTD}{0}
 \renewcommand{\ActiveBarPrimitive}{\barGradCylinder}
 \renewcommand{\tdx}{0.5} % depth of 3d
 \renewcommand{\tdy}{15}
   \bardiagrambegin{7.8}{210}{2cm}{1}{1.5}{0.7cm}{0.02cm}
      \drawlevellines
      \baritem{1998}{120}{green}
      \baritem{1999}{123}{red}
      \baritem{2000}{147}{yellow}
      \baritem{2001}{176}{green}
      \baritem{2002}{132}{red} 
      %
      \diagLegendbegin{0}{200}{3.8}
        \diagLegenditem{\it Legend}{red}
      \diagLegendend
  \bardiagramend{{\large Year}}{{\large Income}}
}
\caption{\label{fig:a2:1}Alternative shapes}
\end{figure}
%----------------------------------------
\printindex

\end{document}
