\documentclass[pagesize=auto]{scrartcl}

\usepackage{fixltx2e}
\usepackage{etex}
\usepackage{lmodern}
\usepackage[T1]{fontenc}
\usepackage{textcomp}
\usepackage[utf8]{inputenc}
\usepackage{microtype}
\usepackage{hyperref}

\newcommand*{\mail}[1]{\href{mailto:#1}{\texttt{#1}}}
\newcommand*{\pkg}[1]{\textsf{#1}}
\newcommand*{\cs}[1]{\texttt{\textbackslash#1}}
\makeatletter
\newcommand*{\cmd}[1]{\cs{\expandafter\@gobble\string#1}}
\makeatother
\newcommand*{\env}[1]{\texttt{#1}}
\newcommand*{\meta}[1]{\textlangle\textsl{#1}\textrangle}
\newcommand*{\marg}[1]{\texttt{\{}\meta{#1}\texttt{\}}}
\newcommand*{\oarg}[1]{\texttt{[}\meta{#1}\texttt{]}}

\addtokomafont{title}{\rmfamily}

\title{The \pkg{endnotes} package}
\author{John Lavagnino\thanks{Centre for Computing in the Humanities, King's College London}~~(\mail{John.Lavagnino@kcl.ac.uk})}
\date{15 January 2003}


\begin{document}

\maketitle

\noindent
Based on the \textsc{footnotes} section of
\texttt{LATEX.TEX} (version~2.09 -- release of 19 April 1986), with
``\texttt{footnote}'' changed to ``\texttt{endnote}'' and ``\texttt{fn}'' changed to ``\texttt{en}'' (where
appropriate), with all the \env{minipage} stuff pulled out, and with
some small changes for the different operation of endnotes.
Subsequently updated to follow the code for
\LaTeXe\ \textlangle 2000/06/01\textrangle.

Uses an extra external file, with \texttt{.ent} extension, to hold the
text of the endnotes.  This may be deleted after the run; a new
version is generated each time---it doesn't require information
collected from the previous run.

This code does not obey \cmd{\nofiles}.  Perhaps it should.

\bigskip

To turn all the footnotes in your documents into endnotes, say
%
\begin{verbatim}
\let\footnote=\endnote
\end{verbatim}
%
in your preamble, and then add something like
%
\begin{verbatim}
\newpage
\begingroup
\parindent 0pt
\parskip 2ex
\def\enotesize{\normalsize}
\theendnotes
\endgroup
\end{verbatim}
%
as the last thing in your document.  (But \cmd{\theendnotes} all
by itself will work.)


\section{Change log}

\renewcommand*{\labelenumi}{\theenumi)}

\begin{labeling}[~--]{DW}
\item[JL] Modified to include \cmd{\addtoendnotes}.  JL, 10/22/89.

\item[JK] Modification by Jörg Knappen 25.\,2.\,1991:

  Introduced \cmd{\notesname} in the spirit of international \LaTeX.
  \cmd{\notesname} is set per default to be \verb+{Notes}+, but can easily
  be redifined, e.\,g.\ for german language\\
  \verb+\renewcommand{\notesname}{Anmerkungen}+

\item[DW] Modification by Dominik Wujastyk, London, 19 September 1991:
  
  Moved the line\\
  \verb+\edef\@currentlabel{\csname p@endnote\endcsname\@theenmark}+\\
  out of the definition of \cmd{\@endnotetext} and into the definition
  of \cmd{\@doanenote} so that \cmd{\label} and \cmd{\ref} commands work correctly in
  endnotes.  Otherwise, the \cmd{\label} just pointed to the last section
  heading (or whatever) preceding the \cmd{\theendnotes} command.

\item[JL] Revised documentation and macros.  24 Sept 1991.

\item[\null]
  \begingroup
  \setlength{\leftmarginii}{25mm}%
  \begin{enumerate}
  \item[\rlap{\hspace{-\labelwidth}modified by \texttt{-\null-bg} (B.\,Gaulle) 09/14/94 for:}]
  \item replace \verb+»+ (why a 8bit char here?) by \verb+^+ as a default.
  \item force \cmd{\catcode} of \verb+>+ to be 12 (implied by \cmd{\@doanenote}).
  \item[\rlap{\hspace{-\labelwidth}\phantom{modified} by \texttt{-\null-bg} again 03/22/95 for:}]
  \item reseting appropriate \texttt{catcode} of \verb+>+ in case it were
    used as an active char before \cmd{\@endanenote} (was
    pointed by Ch.\ Pallier).
  \end{enumerate}
  \endgroup
  
\item[JL] John Lavagnino, 12 January 2003: a number of small updates:
  
  Incorporate change suggested by Frank Mittelbach to
  \cmd{\enoteheading}, so that first note has paragraph indentation.
  Frank's note:
  %
  \begin{quote}
    the idea of this code is to fix the problem that without it
    the first endnote after the heading will not be indented thus looking
    somewhat strange. Problem however is that since there is no
    indentation \verb+\leavemode\par+ will make an absolutely empty pargraph so
    that no baseline calculation is done. therefore \verb+\vskip-\baselineskip+
    will put the first endnote directly below the heading without the
    usual spaccing. using \cmd{\mbox} insead will cure this defect.
  \end{quote}
  
  Also incorporated Frank's suggestion to define
  \cmd{\makeenmark} and \cmd{\theenmark}, so that users can change more of the
  layout without using \cmd{\makeatletter}.  \cmd{\makeenmark} defaults to
  \cmd{\@makeenmark}, so old code is still supported; and \cmd{\theenmark} is
  just syntactic sugar for \cmd{\@theenmark}, which is still the real
  value (and shouldn't be directly modified by user code).

  Definition of \cmd{\ETC}.\ also dropped: surely nobody is still
  using \TeX~2.992.  (If you are, you need to upgrade it or
  endnotes longer than 1000~characters will be truncated.)
  
  Update much of the code to track the current \LaTeXe\ code more
  closely. Clean up \cmd{\theendnotes}.

\item[JL] John Lavagnino, 15 January 2003: fix my garbled version of
  Frank's updates.

\end{labeling}


\section{Endnote commands}

\addtokomafont{labelinglabel}{\small}

\begin{labeling}[~:]{\cmd{\endnotetext}\oarg{num}\marg{text}}
  \item[\cmd{\endnote}\marg{note}] User command to insert a endnote.

  \item[\cmd{\endnote}\oarg{num}\marg{note}] User command to insert a endnote numbered
    \meta{num}, where \meta{num} is a number -- 1, 2,
    etc.  For example, if endnotes are numbered
    *, **, etc. within pages, then  \verb+\endnote[2]{...}+
    produces endnote `**'.  This command does  not
    step the endnote counter.

  \item[\cmd{\endnotemark}\oarg{num}] Command to produce just the endnote mark in
    the text, but no endnote.  With no argument,
    it steps the endnote counter before  generating
    the mark.

  \item[\cmd{\endnotetext}\oarg{num}\marg{text}] Command to produce the endnote but no
    mark.  \cmd{\endnote} is equivalent to
    \cmd{\endnotemark}\cmd{\endnotetext}.

  \item[\cmd{\addtoendnotes}\marg{text}] Command to add text or commands to current
    endnotes file: for inserting headings,
    pagebreaks, and the like into endnotes
    sections.  \meta{text} a moving argument:
    \cmd{\protect} required for fragile commands.
    
\end{labeling}


\section{Endnote user commands}

Endnotes use the following parameters, similar to those relating
to footnotes:
%
\begin{labeling}[~:]{\cmd{\@makeentext}\marg{note}}
  \item[\cmd{\enotesize}] Size-changing command for endnotes.

  \item[\cmd{\theendnote}] In usual \LaTeX\ style, produces the endnote number.

  \item[\cmd{\theenmark}] Holds the current endnote's mark---e.\,g., \dag\ or `1' or `a'.
    (You don't want to set this yourself, as it comes
    either from the autonumbering of notes or from
    the optional argument to \cmd{\endnote}. But you'll need
    to use it if you define your own \cmd{\makeenmark}.)

  \item[\cmd{\makeenmark}] A macro to generate the endnote marker from  \cmd{\theenmark}.
    The default definition is \verb+\hbox{$^\theenmark$}+.

  \item[\cmd{\@makeentext}\marg{note}]
    Must produce the actual endnote, using \cmd{\theenmark} as the mark
    of the endnote and \meta{note} as the text.  It is called when effectively
    inside a \cmd{\parbox}, with $\cmd{\hsize} = \cmd{\columnwidth}$.  For example, it might
    be as simple as
    \verb*+$^{\theenmark}$ +\meta{note}
    
\end{labeling}


\pagebreak[2]

\section{Endnote pseudocode}

\begin{verbatim}
\endnote{NOTE}  ==
 BEGIN
   \stepcounter{endnote}
   \@theenmark :=G eval (\theendnote)
   \@endnotemark
   \@endnotetext{NOTE}
 END

\endnote[NUM]{NOTE} ==
 BEGIN
   begingroup
      counter endnote :=L NUM
      \@theenmark :=G eval (\theendnote)
   endgroup
   \@endnotemark
   \@endnotetext{NOTE}
 END

\@endnotetext{NOTE} ==
 BEGIN
   write to \@enotes file: "\@doanenote{ENDNOTE MARK}"
   begingroup
      \next := NOTE
      set \newlinechar for \write to \space
      write to \@enotes file: \meaning\next
              (that is, "macro:->NOTE)
   endgroup
 END

\addtoendnotes{TEXT} ==
 BEGIN
   open endnotes file if not already open
   begingroup
      let \protect to \string
      set \newlinechar for \write to \space
      write TEXT to \@enotes file
   endgroup
 END

\endnotemark      ==
 BEGIN \stepcounter{endnote}
       \@theenmark :=G eval(\theendnote)
       \@endnotemark
 END

\endnotemark[NUM] ==
  BEGIN
      begingroup
        endnote counter :=L NUM
       \@theenmark :=G eval(\theendnote)
      endgroup
      \@endnotemark
  END

\@endnotemark ==
  BEGIN
   \leavevmode
   IF hmode THEN \@x@sf := \the\spacefactor FI
   \makeenmark          % put number in main text
   IF hmode THEN \spacefactor := \@x@sf FI
  END

\endnotetext      ==
   BEGIN \@theenmark :=G eval (\theendnote)
         \@endnotetext
   END

\endnotetext[NUM] ==
   BEGIN begingroup  counter endnote :=L NUM
                     \@theenmark :=G eval (\theendnote)
         endgroup
         \@endnotetext
   END
\end{verbatim}

\end{document}
