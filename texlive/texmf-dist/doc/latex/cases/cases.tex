\documentclass[DIV=9, pagesize=auto]{scrartcl}

\usepackage{fixltx2e}
\usepackage{lmodern}
\usepackage[T1]{fontenc}
\usepackage{textcomp}
\usepackage{amsmath}
\usepackage{cases}
\usepackage{microtype}
\usepackage{hyperref}

\newcommand*{\fancybreak}{%
  \par
  \nopagebreak\medskip\nopagebreak
  \noindent\null\hfill$*\quad*\quad*\quad$\hfill\null\par
  \nopagebreak\medskip\pagebreak[0]%
}
\newcommand*{\mail}[1]{\href{mailto:#1}{\texttt{#1}}}
\newcommand*{\pkg}[1]{\textsf{#1}}
\newcommand*{\cmd}[1]{\texttt{\string#1}}
\newcommand*{\env}[1]{\texttt{#1}}
\newcommand*{\meta}[1]{\textlangle\textsl{#1}\textrangle}

\addtokomafont{title}{\rmfamily}

\title{The \pkg{cases} package\thanks{This manual corresponds to \pkg{cases}~v2.5, dated~May 2002.}}
\author{Donald Arseneau\\\mail{asnd@triumf.ca}}
\date{May 2002}


\begin{document}

\maketitle

\begingroup
\small
\noindent
Copyright \textcopyright~1993, 1994, 1995, 2000, 2002 by Donald Arseneau, \mail{asnd@triumf.ca}.
These macros may be freely transmitted, reproduced, or modified 
provided that this notice is left intact.  Sub-equation numbering
is based on \pkg{subeqn.sty} by Stephen Gildea; most of the rest is based
on \LaTeX's \cmd{\eqnarray} by Leslie Lamport and the \LaTeX3 team.
\par
\endgroup

\fancybreak

This provides a \LaTeX\ environment \verb+{numcases}+ to produce multi-case 
equations with a separate equation number for each case.  There is
also \verb+{subnumcases}+ which numbers each case with the overall equation
number plus a letter [8a, 8b, etc.].   The syntax is
%
\begin{verbatim}
\begin{numcases}{left_side}
  case_1 & explanation_1 \\
  case_2 & explanation_2 \\
  ...
  case_n & explanation_n
\end{numcases}
\end{verbatim}
%
Each \meta{case} is a math formula, and each \meta{explanation} is a piece of lr mode
text (which may contain math mode in \verb+\(...\)+ or \verb+$...$+).  The explanations
are optional.  Equation numbers are inserted automatically, just as for
the \env{eqnarray} environment.  In particular, the \cmd{\nonumber} command suppresses
an equation number and the \cmd{\label} command allows reference to a particular 
case.  In a \env{subnumcases} environment, a \cmd{\label} in the \meta{left\_side} of the 
equation gives the overall equation number, without any letter. 

To use this package, 
include  ``\verb+\usepackage{cases}+''  after ``\cmd{\documentclass}''.  You may also
specify  ``\verb+\usepackage[subnum]{cases}+''  to force \emph{all} \env{numcases}
environments to be treated as \env{subnumcases}.

\emph{Question:} Is there a \verb+{numcases*}+ environment for
unnumbered cases?\\
\emph{Answer:} There is a \verb+{cases}+
environment in \AmS-\LaTeX, but it is just as convenient to stick
with the canonical \LaTeX\ array:
% 
\begin{verbatim}
\[ left side = \left\{ \begin{array}...\end{array} \right. \]
\end{verbatim}
% 
Speaking of \pkg{\AmS-math}, they use an entirely different system
of equation numbering, and this package uses ordinary \LaTeX\ %
numbering.

\fancybreak

\noindent
A simple example is:
%
\begin{verbatim}
\begin{numcases}{|x|=}
  x, & for $x \geq 0$\\
  -x, & for $x < 0$
\end{numcases}
\end{verbatim}
%
Giving:
%
\begin{numcases}{|x|=}
  x, & for $x \geq 0$\\
  -x, & for $x < 0$
\end{numcases}

\fancybreak

\noindent
Another example is calculating the square root of $c+id$. First compute
\phantomsection
\begin{subnumcases}{\label{w} w\equiv}
  0    & $c = d = 0$\label{wzero}\\
  \sqrt{|c|}\,\sqrt{\frac{1 + \sqrt{1+(d/c)^2}}{2}}   & $|c| \geq |d|$ \\
  \sqrt{|d|}\,\sqrt{\frac{|c/d| + \sqrt{1+(c/d)^2}}{2}}   & $|c| < |d|$
\end{subnumcases}
Then, using $w$ from eq.~(\ref{w}), the square root is
\begin{subnumcases}{\sqrt{c+id}=}
  0                    & $w=0$ (case \ref{wzero})\\
  w+i\frac{d}{2w}      & $w \neq 0$, $c \geq 0$ \\
  \frac{|d|}{2w} + iw  & $w \neq 0$, $c < 0$, $d \geq 0$ \\
  \frac{|d|}{2w} - iw  & $w \neq 0$, $c < 0$, $d < 0$ 
\end{subnumcases}

\pagebreak[2]

\noindent
This was produced by:
%
\small
\begin{verbatim}
Another example is calculating the square root of $c+id$. First compute
\begin{subnumcases}{\label{w} w\equiv}
 0    & $c = d = 0$\label{wzero}\\
\sqrt{|c|}\,\sqrt{\frac{1 + \sqrt{1+(d/c)^2}}{2}}   & $|c| \geq |d|$ \\
\sqrt{|d|}\,\sqrt{\frac{|c/d| + \sqrt{1+(c/d)^2}}{2}}   & $|c| < |d|$
\end{subnumcases}
Then, using $w$ from eq.~(\ref{w}), the square root is
\begin{subnumcases}{\sqrt{c+id}=}
0                    & $w=0$ (case \ref{wzero})\\
w+i\frac{d}{2w}      & $w \neq 0$, $c \geq 0$ \\
\frac{|d|}{2w} + iw  & $w \neq 0$, $c < 0$, $d \geq 0$ \\
\frac{|d|}{2w} - iw  & $w \neq 0$, $c < 0$, $d < 0$ 
\end{subnumcases}
\end{verbatim}

\end{document}
