\documentclass[DIV=8, fontsize=12, pagesize=auto]{scrartcl}

\usepackage{fixltx2e}
\usepackage{lmodern}
\usepackage{mflogo}
\usepackage[T1]{fontenc}
\usepackage{microtype}

\addtokomafont{title}{\rmfamily}

\title{The \textsf{calrsfs} package}
\author{Vadim V. Zhytnikov, \texttt{vvzhy@phy.ncu.edu.tw}}
\date{2008/11/18}


\begin{document}

\maketitle

\noindent
\TeX/\LaTeX\ is the standard in typesetting research 
papers in physics. But many my colleges (and myself) were 
a little unhappy about the appearance of \verb+\cal+ letters 
(\verb+\mathcal+ in \LaTeXe). The shape of these characters is 
very different from typefaces normally used in physical 
journals and books.

With the appearance of Ralph Smith's \textsf{rsfs} fonts and
NFSS this problem can be easily resolved. Just put files
`\texttt{OMSrsfs.fd}' and `\texttt{calrsfs.sty}' in \LaTeXe\ input directory.
The command
%
\begin{verbatim}
\usepackage{calrsfs}
\end{verbatim}
%
in the preamble of \LaTeX\ file replaces all \verb+\mathcal+
letters by nice letters of \textsf{rsfs}'s font and you can enjoy
`standard' shape of Lagrangian\ldots

The \MF\ sources of \textsf{rsfs} fonts are available in CTAN
in  the directory \texttt{/tex-archive/fonts/rsfs}

\end{document}
