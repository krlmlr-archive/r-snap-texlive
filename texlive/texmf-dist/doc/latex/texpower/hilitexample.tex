%%
%% This is file `hilitexample.tex',
%% generated with the docstrip utility.
%%
%% The original source files were:
%%
%% texpower-doc.dtx  (with options: `hilitexample,hilitexample-src,end')
%% 
%% --------------------------------------------------------------
%% TeXPower bundle - dynamic online presentations with LaTeX
%% Copyright (C) 1999-2004 Stephan Lehmke
%% Copyright (C) 2003-2005 Hans Fredrik Nordhaug
%% 
%% This program is free software; you can redistribute it and/or
%% modify it under the terms of the GNU General Public License
%% as published by the Free Software Foundation; either version 2
%% of the License, or (at your option) any later version.
%% 
%% This program is distributed in the hope that it will be useful,
%% but WITHOUT ANY WARRANTY; without even the implied warranty of
%% MERCHANTABILITY or FITNESS FOR A PARTICULAR PURPOSE.  See the
%% GNU General Public License for more details.
%% --------------------------------------------------------------
%% 
%% The list of all files belonging to the TeXPower bundle is
%% given in the file `00readme.txt'.
%% 
\ProvidesFile{hilitexample.tex}%
      [2005/04/07 TeXPower example file]
%-----------------------------------------------------------------------------------------------------------------
%
% Highlighting example for the package texpower.sty.
%
%-----------------------------------------------------------------------------------------------------------------
% Use slifonts.

\RequirePackage{tpslifonts}

% Input the generic preamble.

%%
%% This is file `__TPpreamble.tex',
%% generated with the docstrip utility.
%%
%% The original source files were:
%%
%% texpower-doc.dtx  (with options: `preamble')
%% 
%% --------------------------------------------------------------
%% TeXPower bundle - dynamic online presentations with LaTeX
%% Copyright (C) 1999-2004 Stephan Lehmke
%% Copyright (C) 2003-2005 Hans Fredrik Nordhaug
%% 
%% This program is free software; you can redistribute it and/or
%% modify it under the terms of the GNU General Public License
%% as published by the Free Software Foundation; either version 2
%% of the License, or (at your option) any later version.
%% 
%% This program is distributed in the hope that it will be useful,
%% but WITHOUT ANY WARRANTY; without even the implied warranty of
%% MERCHANTABILITY or FITNESS FOR A PARTICULAR PURPOSE.  See the
%% GNU General Public License for more details.
%% --------------------------------------------------------------
%% 
%% The list of all files belonging to the TeXPower bundle is
%% given in the file `00readme.txt'.
%% 
%
\documentclass
[%
%-----------------------------------------------------------------------------------------------------------------
% Document class options:
% -----------------------
%
% Landscape slides formatted for letter paper fit most screen resolutions (more or less).
%
  letterpaper,%
  landscape,%
%
% The KOMA option makes powersem load scrartcl.cls instead of article.cls.
%
  KOMA,%
% KOMA document class options are accepted.
  smallheadings,%
%
% The calcdimensions option makes powersem calculate the slide dimensions automatically from paper size and margins.
  calcdimensions,%
%
% The display option sets everything up for producing slides to be displayed interactively.
% This option is also recognized by the texpower package.
%
  display%
%-----------------------------------------------------------------------------------------------------------------
]
%-----------------------------------------------------------------------------------------------------------------
% Document class powersem, based on seminar.cls for simulating ppower via latex+distiller (instead of pdflatex).
%
{powersem}

%-----------------------------------------------------------------------------------------------------------------
% Set slide margins rather small for maximum use of space. This is a demo, remember.
%
\renewcommand{\slidetopmargin}{5mm}
\renewcommand{\slidebottommargin}{5mm}

\renewcommand{\slideleftmargin}{5mm}
\renewcommand{\sliderightmargin}{5mm}


%-----------------------------------------------------------------------------------------------------------------
% Some setup for more reasonable spacing.
%

\makeatletter

\renewcommand\section{\@startsection{section}{1}{\z@}%
  {-1.5ex\@plus -1ex \@minus -.5ex}%
  {.5ex \@plus .2ex}%
  {\raggedsection\normalfont\size@section\sectfont}}

\renewcommand\subsection{\@startsection{subsection}{2}{\z@}%
  {-1.25ex\@plus -1ex \@minus -.2ex}%
  {.5ex \@plus .2ex}%
  {\raggedsection\normalfont\size@subsection\sectfont}}

\renewcommand\subsubsection{\@startsection{subsubsection}{3}{\z@}%
  {-1.25ex\@plus -1ex \@minus -.2ex}%
  {.5ex \@plus .2ex}%
  {\raggedsection\normalfont\size@subsubsection\sectfont}}

\renewcommand\paragraph{\@startsection{paragraph}{4}{\z@}%
  {1.25ex \@plus1ex \@minus.2ex}%
  {-1em}%
  {\raggedsection\normalfont\size@paragraph\sectfont}}

\def\slideitemsep{.5ex plus .3ex minus .2ex}

\makeatother

%-----------------------------------------------------------------------------------------------------------------
% We need some more packages...
%

\usepackage{url}

\usepackage[latin1]{inputenc}

% One more Text emphasis command...

\let\name=\textsc

%-----------------------------------------------------------------------------------------------------------------
% We load hyperref and fixseminar which fixes some problems with seminar.
%
\usepackage[ps2pdf,plainpages=false,bookmarksopen,colorlinks,urlcolor=red,pdfpagemode=FullScreen]{hyperref}
\usepackage{fixseminar}

%-----------------------------------------------------------------------------------------------------------------
% Finally, the texpower package is loaded.
%
\usepackage{texpower}

%% The configuration file allows user-specific settings.

%%
%% This is file `__TPcfg.tex',
%% generated with the docstrip utility.
%%
%% The original source files were:
%%
%% texpower-doc.dtx  (with options: `config')
%% 
%% --------------------------------------------------------------
%% TeXPower bundle - dynamic online presentations with LaTeX
%% Copyright (C) 1999-2004 Stephan Lehmke
%% Copyright (C) 2003-2005 Hans Fredrik Nordhaug
%% 
%% This program is free software; you can redistribute it and/or
%% modify it under the terms of the GNU General Public License
%% as published by the Free Software Foundation; either version 2
%% of the License, or (at your option) any later version.
%% 
%% This program is distributed in the hope that it will be useful,
%% but WITHOUT ANY WARRANTY; without even the implied warranty of
%% MERCHANTABILITY or FITNESS FOR A PARTICULAR PURPOSE.  See the
%% GNU General Public License for more details.
%% --------------------------------------------------------------
%% 
%% The list of all files belonging to the TeXPower bundle is
%% given in the file `00readme.txt'.
%% 
%-----------------------------------------------------------------------------------------------------------------
%
% Code for user-specific configuration of TeXPower documentation files.
%
% This file is input by others. Don't compile it separately.
%
\hypersetup{baseurl={http://texpower.sourceforge.net/doc/}}
\hypersetup{pdfsubject={Documentation and Examples for the texpower package}}
\hypersetup{pdfauthor={Stephan Lehmke}}
\endinput
%%
%% End of file `__TPcfg.tex'.


%-----------------------------------------------------------------------------------------------------------------
% Some more parameters...
%
\slidesmag{5}
\slideframe{none}
\pagestyle{empty}
\setcounter{tocdepth}{2}
\renewcommand{\currentpagevalue}{\value{slide}}

%-----------------------------------------------------------------------------------------------------------------
% The following command produces a title page for every example and documentation file.

\newcommand{\makeslidetitle}[1]
{%
  \title{The \TeX Power bundle\\[2ex]{\normalfont #1}}
  \author
  {%
    Stephan Lehmke\\
    \mdseries
    University of Dortmund\\
    \mdseries
    Department of Computer Science I\\
    \url{mailto:Stephan.Lehmke@udo.edu}%
  }
  {\centerslidestrue
  \maketitle
  \newslide}
  \setcounter{firststep}{1}% This way, the first step of all examples is displayed.
}
\endinput
%%
%% End of file `__TPpreamble.tex'.

\hypersetup{pdftitle={texpower highlighting example}}

% The package soul is needed for \highlighttext to work.

\usepackage{soul}

%-----------------------------------------------------------------------------------------------------------------
% Finally, everything is set up. Here we go...
%
\begin{document}
\begin{slide}
%-----------------------------------------------------------------------------------------------------------------
%
\makeslidetitle{\macroname{stepwise} Example: Highlighting Text}\label{Sec:Exhl}
% The default for \step's which are not yet `active' is to be `invisible'. In preceding examples, we have redefined
% internal macros like \hidestepcontents or \activatestep to achieve special `highlighting' effects.
% Here, we demonstrate how make text `stand out' from the background incrementally without having it appear from thin
% air.

% The first example demonstrates how to make \step its argument `stand out' instead of making it appear `out of
% nowhere'.

\makeatletter
\ifthenelse{\boolean{TPcolor}}% Can we use colors?
{% Yes. In this case \step should change colors from `dimmed' to `active'.

  % The command \hidedimmed switches colors to `dimmed', the command \highlightenhanced switches colors to `enhanced'.
  \liststepwise[\let\hidestepcontents=\hidedimmed\let\activatestep=\highlightenhanced]
  {%
    \vspace*{\fill}
    \begin{quote}
      \Large%
      \step{Instead of making things appear out of `thin air',} \step{we can also make them appear `out of the
        background'} \step{by incrementally changing color from \concept{inactive}} \step{to \concept{active}.}
      \step{This also works with \emph{color emphasis}} \step{and \concept{math coloring}: $a=b^2$.}
    \end{quote}
    \vspace*{\fill}\vspace*{\fill}
    }%
  }
{% No. In this case we have to `handwave' using the soul package.
  % Instead of hiding its argument, the new command \myhide just changes the font size (maintaining the overall text
  % length though).
  \def\myhide
  {%
    \leavevmode%
    \SOUL@setup
    \def\SOUL@everytoken{\makebox[\widthof{\the\SOUL@token}][s]{\rule[\depthof{\the\SOUL@token}*-1]{0pt}{\depthof{\the\SOUL@token}+\heightof{\the\SOUL@token}}\hrulefill\tiny\the\SOUL@token\hrulefill}\SOUL@setkern\SOUL@charkern}%
    \SOUL@%
    }

  \liststepwise[\let\hidestepcontents=\myhide\setcounter{firststep}{0}]
  {%
    \vspace*{\fill}
    \begin{quote}
      \LARGE%
      \step{Instead of making things appear out of `thin air',} \step{we can also make them appear `out of the
        background'} \step{incrementally.}
    \end{quote}
    \vspace*{\fill}\vspace*{\fill}
    }%
  }
\makeatother

\newslide

% Next, it is demonstrated how we can `flip through' an itemize environment by just highlighting the items in turn
% instead of making them appear one by one.
%
% Define a macro \mystep which implements the highlighting effect.
\ifthenelse{\boolean{TPcolor}}% Can we use colors?
{% Yes. In this case highlighting is implemented by switching color.
  \def\mystep% Note that \mystep takes no argument. It just changes the way the next item is displayed.
  {%
    \usecolorset{stwcolors}%             Restore the undimmed colors valid at the beginning of \stepwise.
    \dstep[][\boolean{firstactivation}]% \dstep switches colors. The optional argument makes it appear only once.
    }%
}
{% No. In this case highlighting is implemented by redefining \labelitemi.
  \def\mystep
  {%
    \switch[][\boolean{firstactivation}]% The optional arguments make \switch act only once.
    {\def\labelitemi{\origmath{\gg}}}{\def\labelitemi{\origmath{\cdot}}}%
    }%
  }

% We define a custom itemize environment which automatically adds calls to \mystep:
\newenvironment{stepitemize}
{%
  \huge
  \begin{itemize}
    \let\origitem=\item
    % Here, the \mystep command is hidden inside \item
    \renewcommand{\item}{\mystep\origitem}%
    }
  {%
  \end{itemize}
  }

Instead of displaying incrementally, we can just `flip through' some items by highlighting them:

% Note that we use the starred version of \liststepwise so that the first item is highlighted on the first slide
% produced by \liststepwise.
%
\liststepwise*
{%
  \begin{stepitemize}
  \item Item 1
  \item Item 2
  \item Item 3
  \end{stepitemize}
  }

\pause

% The following example demonstrates highlighting inside a paragraph using \highlighttext. By redefining \activatestep
% to \highlighttext, the argument of every \step will be highlighted when the \step is activated for the first
% time. Note that highlighting doesn't influence line breaks because \highlighttext is implemented using the soul
% package.
%
% Again, we define \hidestepcontents to display its argument, so that the complete text is visible from the outset.

\stepwise[\let\activatestep=\highlighttext\let\hidestepcontents=\displayidentical]
{%
  \vspace*{\fill}
  \begin{minipage}{\linewidth}
    \LARGE
    \step{Inside} a paragraph, we can \step{highlight} text \step{without influencing} \step{line breaks}.
  \end{minipage}
  \vspace*{\fill}
  }
\newslide
\end{slide}
\end{document}
\endinput
%%
%% End of file `hilitexample.tex'.
