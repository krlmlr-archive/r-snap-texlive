\address{Herbert}{Vo\ss}{Wasgenstra\ss e 21\\
    14129 Berlin\\
    \protect\url{herbert@dante.de}}
%\address{Herbert}{Vo\ss}{Wasgenstraße 21\\14129 Berlin\\\protect\url{herbert@dante.de}}

Liebe Leserinnen und Leser,

Wunder geschehen ja eigentlich nicht in regelmäßigen Abständen, jedenfalls wären es dann definitionsgemäß
keine mehr. So bleibt das Problem, wie man es bezeichnen soll, was Sie in den Händen halten. Dass bekanntermaßen nach einer Ausgabe 
der Zeitschrift "`\DTK"'
natürlich immer auch vor einer Ausgabe ist, wird keinen erstaunen. Wenn aber nach einer Ausgabe der
"`Eingabespeicher"' der eingereichten Manuskripte immer vollständig geleert ist, so bleibt es ein Wunder,
dass bis zur nächsten Ausgabe dieser "`Speicher"' so aufgefüllt wird, dass es wieder für eine
neue Ausgabe in gewohntem Umfang reicht. Der Dank gebührt natürlich den Autoren, denn ohne sie könnte es tatsächlich zu einem echten
Wunder werden \ldots

Dass \TeX, als reines Textsatzsystem konzipiert, durch derartig viele verschiedene Pakete
zur Grafikausgabe unterstützt wird, ist ebenfalls ein kleines Wunder, zumal die
Qualität mit der vom Textsatz gewohnten durchaus konkurrieren kann. In diesem Heft
finden Sie wieder eine Variante beschrieben. Weiterhin lesen Sie den Bericht von
der letzten Herbsttagung in Esslingen und dem Linuxtag in Kiel, auf dem \dante\ erstmalig
vertreten war. Weiterhin finden Sie Texte, die eher den praktischen Bereich von \LaTeX\
beleuchten und natürlich, wie gewohnt, die Liste der neuen Pakete. Der Umfang dieser Liste ist das letzte Wunder
von dem man berichten kann, denn nach über 30 Jahren \TeX\ und 20 Jahren \LaTeX\ erscheint
die innovative Kraft der \TeX-Gemeinde ungebrochen.


\bigskip
Ich wünsche Ihnen  wie immer viel Spaß beim Lesen und verbleibe

%\medskip
%\enlargethispage{1.5ex}
mit \TeX nischen Grüßen, 

Ihr Herbert Voß

\endinput
