\begingroup\small
"`Die \TeX{}nische Komödie"' ist die Mitgliedszeitschrift von
\dante{} Der Bezugspreis ist im Mitgliedsbeitrag enthalten.
Namentlich gekennzeichnete Beiträge geben die Meinung der
Schreibenden wieder.  Reproduktion oder Nutzung der erschienenen
Beiträge durch konventionelle, elektronische oder beliebige andere
Verfahren ist nur im nicht-kommerziellen Rahmen gestattet.
Verwendungen in größerem Umfang bitte zur Information bei \dante{}
melden.

Beiträge sollten in Standard-\LaTeX-Quellcode unter Verwendung der
Dokumentenklasse \texttt{dtk} erstellt und per \mbox{E-Mail} oder 
Datenträger (CD) an untenstehende Adresse
der Redaktion geschickt werden.  Sind
spezielle Makros, \LaTeX-Pakete oder Schriften dafür nötig, so
m"ussen auch diese komplett mitgeliefert werden.  Außerdem m"ussen sie auf
Anfrage Interessierten zugänglich gemacht werden.  \smallskip


Diese Ausgabe wurde mit 
\texttt{pdfTeX 3.1415926-1.40.10-2.2} %\TeX Live2007 
erstellt.
%\texttt{LaTeX2e (2001/06/01)},
%{\tt GSview\,1.3} (f"ur die Bildschirmdarstellung)
%{\tt ghostview 1.3} (f"ur die Bildschirmdarstellung)
%{\tt dviscr 1.4s} (f"ur die Bildschirmdarstellung),
%{\tt dvipm 1.5a} (f"ur die Bildschirmdarstellung),
%\texttt{Ac"-ro"-bat Read"-er 5.0.8}  
%  und \texttt{windvi 0.67}  
%und 
%\texttt{xdvi(k)} \texttt{22.40k}
%f"ur die Bild"-schirm"-dar"-stellung.
%und \texttt{dvips(k)~5.86e}
%(f"ur die endg"ultige Belichtung)
%f"ur Korrektur und Belichtung.
%Die Schriften zur Belichtung wurden mit
%dem \MF-Modus \texttt{linoone} (1270\,dpi) berechnet.
Als Stan"-dard-Schriften kamen die Type-1-Fonts Latin-Modern und LuxiMono zum Einsatz.

\smallskip
\vfill
\noindent
\begin{tabular}{@{}l@{ }l@{}}
  Erscheinungsweise: & viertelj"ahrlich\\
  Erscheinungsort:   & Heidelberg\\
  Auf\/lage:         & 2700\\
  Herausgeber: & \Dante\\
               & Postfach 10\,18\,40\\
               & 69008 Heidelberg\\[3pt]
               & \begin{tabular}[b]{@{}ll@{}}
%                Tel.: & 0\,62\,21/2\,97\,66\\
%                Fax:  & 0\,62\,21/16\,79\,06\\
                   E-Mail: & \texttt{dante@dante.de}\\
                           & \texttt{dtkred@dante.de} (Redaktion)
                 \end{tabular}\\[4pt]
  Druck:       & Konrad Triltsch Print und digitale Medien GmbH\\
               & Johannes-Gutenberg-Str. 1--3,
                 97199 Ochsenfurt-Hohestadt\\[4pt]
  Redaktion:    &  Herbert Vo\ss\ (verantwortlicher Redakteur)\\
  Mitarbeit   : & %\raisebox{-\height}{\begin{minipage}{.75\textwidth}
%                  \begin{multicols}3 \raggedright
%                   \makebox[0.25\linewidth][l]{Luzia Dietsche}		%\\
%                   \makebox[0.25\linewidth][l]{Rudolf Herrmann}		%\\
%                   Moriz Hoffmann-""Axthelm	\\
%                   \makebox[0.25\linewidth][l]{Klaus H"oppner}		%	\\
%                   Lutz Ihlenburg		\\
                   \makebox[0.25\linewidth][l]{Gert Ingold}			
%                  \makebox[0.25\linewidth][l]{Manfred Lotz}			
%                   Uwe M"unch			\\
%                &   \makebox[0.25\linewidth][l]{Gerd Neugebauer}		%\\
                  \makebox[0.25\linewidth][l]{Rolf Niepraschk}		
%                   \makebox[0.25\linewidth][l]{G"unter Partosch}		\\
% &                   \makebox[0.25\linewidth][l]{Bernd Raichle}		%\\
                \makebox[0.25\linewidth][l]{Christine Römer}		
%                   \makebox[0.25\linewidth][l]{Volker RW Schaa}		%\\
%                   \makebox[0.25\linewidth][l]{Hilmar Schlegel}		
%                   \makebox[0.25\linewidth][l]{Martin Schröder}		\\
%                   \makebox[0.25\linewidth][l]{Uwe Ziegenhagen}		
%                 \end{multicols}
               %\end{minipage}}

\end{tabular}

\bigskip

\noindent
\makebox[\linewidth]{%
  Redaktionsschluss für Heft 1/2010: 15. Januar 2010
  \hfill
  {ISSN 1434-5897}}

\par\endgroup
\endinput
