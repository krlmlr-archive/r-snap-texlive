\documentclass[compose]{exam-n}
\begin{document}

\begin{question}{30} \comment{by Graham Woan}
Two variables, $A$ and $B$, have a joint Gaussian probability
distribution function (pdf) with a negative correlation
coefficient.  Sketch the form of this function as a contour plot
in the $AB$  plane, and use it to distinguish between the most
probable joint values of $(A,B)$  and the most probable value of
$A$ given (a different) $B$. \partmarks{5}

Explain what is meant by \emph{marginalisation} in Bayesian
inference and how it can be interpreted in terms the above plot.
\partmarks{5}

Doppler observations of stars with extrasolar planets give us data
on $m\sin i$  of the planet, where $m$ is the planet's mass and
$i$ the angle between the normal to the planetary orbit and the
line of sight to Earth (i.e. the orbital inclination), which can
take a value between 0 and $\pi/2$ .

Assuming that planets can orbit stars in any plane, show that the
probability distribution for $i$ is $p(i) = \sin i$.
\partmarks{5}

A paper reports a value for $m\sin i$  of $x$, subject to a Gaussian
error of variance $\sigma^2$.  Assuming the mass has a uniform
prior, show that the posterior probability distribution for the mass
of the planet is
\begin{equation*}
p(m|x)\propto\int_0^1\exp\left[-\frac{\left(x-m\sqrt{1-\mu^2}\right)^2}{2\sigma^2}\right]
\ddd \mu,
\end{equation*}
where $\mu=\cos i$.
\partmarks{9}

Determine the corresponding expression for the posterior pdf of
$\mu$, and explain how both are normalised.
\partmarks{6}
\end{question}
\end{document}
