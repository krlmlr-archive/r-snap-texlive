%%
%% This is file `exam-n-example.tex',
%% generated with the docstrip utility.
%%
%% The original source files were:
%%
%% exam-n.dtx  (with options: `example')
%% exam-n: format exam questions
%% Release version 1.1, 2014 May 03.
%%
%%%% File: exam-n.dtx
%%%% Copyright 2005--2014, Norman Gray
%%
%% This work may be distributed and/or modified under the
%% conditions of the LaTeX Project Public License, either version 1.3
%% of this license or (at your option) any later version.
%% The latest version of this license is in
%%   http://www.latex-project.org/lppl.txt
%% and version 1.3 or later is part of all distributions of LaTeX
%% version 2005/12/01 or later.
%%
%% This work has the LPPL maintenance status `maintained'.
%%
%% The Current Maintainer of this work is Norman Gray <http://nxg.me.uk>
%%
%% This work consists of the files exam-n.dtx and exam-n.ins,
%% the derived file exam-n.cls,
%% and the associated *.clo files.

%%%% Source: Mercurial revision 6f818b549982, 2014-05-03 13:42 +0100, tag 1.1 + 0
%%
 
 
%%%START example (Makefile strips out this block)
\documentclass{exam-n}  % standard final version
%%\documentclass[draft,showsolutions]{exam-n}  % draft style, showing solutions
%%\documentclass[compose]{exam-n}  % compose (author's) style

\examdate{Wednesday, 18 May 2005}
\examtime{9.30am -- 12 noon\\(or) 9.30am -- 1.45am}

\exambanner{Examination for the Degrees of \BSc(Science) and
  \MSci\ on the Honours Standard}
\schoolcoursecode{P304D and P304H}
\universitycoursecode{PHYS3031 and PHYS4025}
\coursetitle{Quantum Mechanics}
\degreedescriptions{Physics 3\\Chemical Physics 3\\Physics with
  Astrophysics 3\\Theoretical Physics 3M\\Joint Physics 3}
\paperident{GR/P304}

\rubric{Candidates for examination in \emph{Quantum Mechanics} should
  answer question 1 (16 marks) and \emph{either 2A or 2B} (24 marks each)}

\numquestions{3}

\begin{document}
\maketitle

\section{I}

\begin{question}{20}
\part At various points in the development of the mathematical theory of
General Relativity, we pick a coordinate system in which
differentiation is simple, and do a calculation using non-covariant
differentiation, indicated by a comma.  We then immediately deduce the
covariant result, replacing this comma with a semicolon.

Separately, the strong equivalence principle is sometimes
referred to as the `comma goes to semicolon' rule.

Explain the logic of each of these replacements of a comma with a
semicolon, putting particular stress on the distinction between
them.\partmarks{10}

\part The radial and angular coordinates, $r$ and $\phi$ respectively,
of a test particle moving in the Schwartzschild metric exterior to a
star of mass $M\ll r$, are related by the equation
\[
  r = \frac{h^2}M \left(
        1 + e\cos\phi + \frac{3M^2}{h^2}e\phi\sin\phi
        \right)^{-1},
\]
where $h$ and $e$ are constants.  Show that this equation takes the
form of a precessing ellipse, of semi-latus rectum $l=h^2/M$, in which
the pericentre line advances each orbit by an amount
$\Delta=6\pi M^2/h^2$, stating clearly any assumptions that
you make.\partmarks{6}

The solar-mass star HD83443 has a 0.35 Jupiter-mass planet that
follows a circular orbit of period 2.986 days and radius 0.038\units\au.
Calculate the rate of precession, in arcseconds per year, of the
pericentre line of the planet's orbit.\partmarks{4}

[Schwartzschild radius of the Sun${}= 3.0\times10^3\units{m}$,
$1\units\au=1.5\times10^{11}\units m$].

\begin{solution}
  In the first type of calculation, we do a calculation in the LIF, in
which~$\Gamma^i_{jk}=0$, so that single partial differentiation is the
same as covariant differentiation.  If this process produces a
geometrical object such as a scalar or a tensor, then we know that the
result is frame-invariant.  If the result involves only single partial
differentiation -- that is, no second derivatives -- then since
partial differentiation is the same as covariant differentiation in
these coordinates, we cannot distinguish partial and covariant
derivatives, and can replace the commas by semicolons.  Since these
are now manifestly covariant derivatives, so that the result is a
tensor, and thus frame-invariant, the same expression would be true in
any frame.

The second situation is the statement that the expressions of physical
laws in SR, such as the conservation equation
$T^{\mu\nu}_{,\nu}=0$, must take the same \emph{form} when written
as a covariant equation in GR, crucially without any curvature
coupling.  The slogan `comma goes to semicolon' is just a mnemonic for
this.

The distinction is that the first is a mathematical trick, of sorts,
whereas the second is a version of the equivalence principle, and thus
a statement with deep physical content.

They don't have to explain things at this length or with this
coherence (?) to get quite a few marks.  They just have to show they
have a clue.
\end{solution}
\end{question}

\end{document}
%%%END example
 
\endinput
%%
%% End of file `exam-n-example.tex'.
