% mslapa is based on previous APA formats. The changes made to produce the
% mslapa format were originally made by Michael S.  Landy in 1995 and are
% public domain. The copyright and licensing of the file as a whole also
% depends on the copyright and licensing of the original code.  
% This statement added 2008/11/25 by Clea F. Rees in an attempt to make the
% situation a little clearer following communication with Michael S. Landy.

\documentclass[draft]{article}

\usepackage{mslapa}

\begin{document}

\centerline{\Large Example of the Use of MSLAPA Bibliography Style}
\vspace{.3in}
\centerline{\large Michael S. Landy}
\vspace{.3in}

\section{Introduction}

blah blah blah \cite[note1]{ALOI89}.
blah blah blah.

Another paragraph of blah blah blah \cite(note2,){BRAU68,ALOI89}.
%\tracingmacros=1
A nocite-ation of Bergen and Landy '91.\nocite{BERG91}

\subsection{Stuff I understand}

blah blah blah.

\subsection{Stuff I made up}

blah blah \cite(note3,)[note4]{BERG91} blah.
blah blah \cite{DOSH89A,DOSH89B,DOSH89C}.
blah blah in the work of Dosher and colleagues \citeyear{DOSH89A}.
blah blah in the work of Dosher and colleagues \citeyear{DOSH89B,DOSH89C}.
blah blah in the work of Dosher and colleagues \citeyear{DOSH89B,DOSH90}.
blah blah \shortcite{DOSH89A}.
blah blah \shortcite{ALOI89,DOSH89A,DOSH89B}.

\bibliographystyle{mslapa}
\bibliography{bibfile}

\end{document}
