\listfiles
\documentclass[ngerman]{article}
\usepackage[T1]{fontenc}
\usepackage[utf8]{inputenc}
\usepackage[osf]{mathpazo}
\usepackage[scaled]{beramono}
\usepackage{babel}
\usepackage{longtable}
\usepackage[makeidx]{hvindex}
\usepackage[colorlinks,linktocpage]{hyperref}
\providecommand\phantomsection{}
\title{\texttt{hvindex}, v. \fileversion\\ simplifying indexing \ldots}
\author{Herbert Voß}
\date{\filedate}
\begin{document}
\maketitle

\tableofcontents

\section{Package options}
With the \Index{option} \ttIndex{makeidx} the also called package is 
loaded and the macro \texttt{\textbackslash{}makeindex}
is executed.

The package should be loaded \emph{after} package \ttIndex{babel}.

\section{Introduction}
The \Index{package} \ttIndex{hvindex} defines some \Index{macros}
as a \Index{macro!replacement} for 
\Index{macros!index@\texttt{\textbackslash index}} 
which makes \Index{live} easier\index{easy}. You have to type every
\Index{word} only \Index{once} to get it into the \Index{index} 
and into the \Index{text}.


\small
\begin{verbatim}
The \Index{package} \ttIndex{hvindex} defines some \Index{macros}
as a \Index{macro!replacement} for 
\Index{macros!index@\texttt{\textbackslash index}} 
which makes \Index{live} easier\index{easy}. You have to type every
\Index{word} only \Index{once} to get it into the \Index{index} 
and into the \Index{text}.
\end{verbatim}
\normalsize


\clearpage
\section{The macros}

\begin{longtable}{@{} l  l l @{}}\\\hline
\emph{code} & \emph{Text} & \emph{index}\\\hline
\endfirsthead
\multicolumn{3}{@{}l}{\ldots\ continued from previous page}\\\hline
\emph{code} & \emph{Text} & \emph{index}\\\hline
\endhead
\hline
\multicolumn{3}{r@{}}{continued on next page \ldots}\\
\hline
\endfoot
\hline
\endlastfoot
\verb+\Index{foo}+                      &$\Rightarrow$ \Index{foo}                   & foo\\
\verb+\Index{foo!bar}+                  &$\Rightarrow$ \Index{foo!bar}               & foo\\
                                        &                                            & \kern1em bar\\
\verb+\Index{foo!bar!baz}+              &$\Rightarrow$ \Index{foo!bar!baz}           & foo\\
                                        &                                            & \kern1em bar\\
                                        &                                            & \kern2em baz\\
\verb+\Index{foo@\textsc{foo}}+         &$\Rightarrow$ \Index{foo@\textsc{foo}}      & \textsc{foo}\\
\verb+\Index{foo!bar@\textsf{baz}}+     &$\Rightarrow$ \Index{foo!bar@\textsf{bar}}  & foo\\
                                        &                                            & \kern1em\textsf{bar}\\
\verb+\Index{foo!bar!baz@\texttt{baz}}+ &$\Rightarrow$ \Index{foo!bar!baz@\texttt{baz}}&foo\\
                                        &                                            & \kern1em bar\\
                                        &                                            & \kern2em \texttt{baz}\\
\verb+\Index{foo@\textsc{foo}!bar}+     &$\Rightarrow$ \Index{foo@\textsc{foo}!bar}  & \textsc{foo}\\
                                        &                                            & \kern1em bar\\
\verb+\Index{foo!bar@\textsc{bar}!baz}+ &$\Rightarrow$ \Index{foo!bar@\textsc{bar}!baz}& foo\\
                                        &                                            & \kern1em\textsc{bar}\\
                                        &                                            & \kern2em baz\\
\verb+\Index{foo@\textsc{foo}}+         &$\Rightarrow$ \Index{foo@\textsc{foo}}      & \textsc{foo}\\ 
\verb+\Index{foo!bar@\textsf{baz}}+     &$\Rightarrow$ \Index{foo!bar@\textsf{bar}}  & foo\\
                                        &                                            & \kern1em\textsf{bar}\\
\verb+\Index{foo!bar!baz@\texttt{baz}}+ &$\Rightarrow$ \Index{foo!bar!baz@\texttt{baz}}& foo\\
                                        &                                            & \kern1em bar\\
                                        &                                            & \kern2em\texttt{baz}\\
\verb+\ttIndex{foo}+                    &$\Rightarrow$ \ttIndex{foo} & \texttt{foo}\\
\verb+\bfIndex{foo}+                    &$\Rightarrow$ \bfIndex{foo} & \textbf{foo}\\
\verb+\sfIndex{foo}+                    &$\Rightarrow$ \sfIndex{foo} & \textsf{foo}\\
\verb+\scIndex{foo}+                    &$\Rightarrow$ \scIndex{foo} & \textsc{foo}\\
\verb+\itIndex{foo}+                    &$\Rightarrow$ \itIndex{foo} & \textit{foo}\\
\verb+\ttIndex{foo!bar}+                &$\Rightarrow$ \ttIndex{foo!bar} & foo\\
                                        &                                            & \kern1em\texttt{bar}\\
\verb+\bfIndex{foo!bar}+                &$\Rightarrow$ \bfIndex{foo!bar} & foo\\
                                        &                                            & \kern1em\textbf{bar}\\
\verb+\sfIndex{foo!bar}+                &$\Rightarrow$ \sfIndex{foo!bar} & foo\\
                                        &                                            & \kern1em\textsf{bar}\\
\verb+\scIndex{foo!bar}+                &$\Rightarrow$ \scIndex{foo!bar} & foo\\
                                        &                                            & \kern1em\textsc{bar}\\
\verb+\itIndex{foo!bar}+                &$\Rightarrow$ \itIndex{foo!bar} & foo\\
                                        &                                            & \kern1em\textit{bar}\\
\clearpage
\verb+\ttIndex{foo!bar!baz}+            &$\Rightarrow$ \ttIndex{foo!bar!baz} & foo\\
                                        &                                            & \kern1em bar\\
                                        &                                            & \kern2em\texttt{baz}\\
\verb+\bfIndex{foo!bar!baz}+            &$\Rightarrow$ \bfIndex{foo!bar!baz} & foo\\
                                        &                                            & \kern1em bar\\
                                        &                                            & \kern2em\textbf{baz}\\
\verb+\sfIndex{foo!bar!baz}+            &$\Rightarrow$ \sfIndex{foo!bar!baz} & foo\\
                                        &                                            & \kern1em bar\\
                                        &                                            & \kern2em\textsf{baz}\\
\verb+\scIndex{foo!bar!baz}+            &$\Rightarrow$ \scIndex{foo!bar!baz} & foo\\
                                        &                                            & \kern1em bar\\
                                        &                                            & \kern2em\textsc{baz}\\
\verb+\itIndex{foo!bar!baz}+            &$\Rightarrow$ \itIndex{foo!bar!baz} & foo\\
                                        &                                            & \kern1em bar\\
                                        &                                            & \kern2em\textit{baz}\\\hline
\verb+\sIndex{foo|bar}+                 &$\Rightarrow$ \sIndex{foo|bar}    & foo, see bar \\
\verb+\saIndex{baz|bar}+                &$\Rightarrow$ \saIndex{baz|bar}& baz, see also bar\\\hline

\end{longtable}

It is not easy to get the braces into an index, because \verb+\index+ reads its argument \Index{verbatim}.
The package defines the follwong macros:

\verb+\iBraceL+, which writes the \IBraceL{} into the \Index{index}, same for \verb+\iBraceR+ for the right \IBraceR.
With the upper case variant it is writen into the index and also printed into the text,
like this one: \IBraceL\ and \IBraceR\ (\verb|\IBraceL\ and \IBraceR|). The advantage of these macros is, that
the \Index{braces} are sorted correct.
%, the character \Index{"|} must appear between them and it does \ldots
%BTW: with \texttt{hyperref} we have to write \verb+\index{{"|}}+, to get it as a correct \Index{link} into
%the index. \texttt{hvindex} does it for you if you use \verb+\Index{"|}+.
\index{bar}\index{{"|}}

\clearpage
\phantomsection
\addcontentsline{toc}{section}{Index}
\printindex
\end{document}


