% Fichier comprenant les macros construisant les solides

% \prismereg : dessine un prisme à base régulière
% Synthaxe : \prismreg[	n = nombre de côtés de la base (>2) - par défaut : 3,
%						bordercolor = ... (par défaut : noir),
%						incolor = ... (par défaut : blanc),
%						coefopaq = ... (par défaut : 0.5),
%						axe (booléen : si oui, axe tracé),
%						axecolor = couleur de l'axe (par défaut : rouge),
%						name (booléen : si oui, nom du sommet et du centre de la base mentionnés),
%						sommet = nom du sommet (par défaut : S),
%						posommet = ...,
%						centre = nom du centre de la base (par défaut : O),
%						scalecentre = coef. d'agrandissement du point du centre de la base,
%						poscentre = position du nom du centre de la base (par défaut : below),
%						hauteur = ... (par défaut, 5 cm),
%						rayon = ... (par défaut, 2 cm),
%						legende : booléen,
%						incl = coefficient d'inclinaison de la base (par défaut : 1),
%						rotat = angle de rotation (par défaut : 0°)]

\define@cmdkey [PAS] {prismereg} {n}{}
\define@cmdkey [PAS] {prismereg} {bordercolor}{}
\define@cmdkey [PAS] {prismereg} {incolor}{}
\define@cmdkey [PAS] {prismereg} {coefopaq}{}
\define@boolkey[PAS] {prismereg} {axe}[true]{}
\define@cmdkey [PAS] {prismereg} {axecolor}{}
\define@boolkey[PAS] {prismereg} {name}[true]{}
\define@boolkey[PAS] {prismereg} {legende}[true]{}
\define@cmdkey [PAS] {prismereg} {hauteur}{}
\define@cmdkey [PAS] {prismereg} {rayon}{}
\define@cmdkey [PAS] {prismereg} {incl}{}
\define@cmdkey [PAS] {prismereg} {rotat}{}

\presetkeys    [PAS] {prismereg} {
								n = 3,
								bordercolor = black,
 							 	incolor = white,
 							 	coefopaq = 0.5,
 							 	axe = false,
 							 	axecolor = red,
 							 	legende = false,	
								name = false,
 							 	hauteur = 5,
 							 	rayon = 2,
 							 	rotat = 0.1,
 							 	incl = 1}{}

\newcommand*{\prismereg}[1][]{\pasPrismereg[#1]}

\def\pasPrismereg[#1]{
	\setkeys[PAS]{prismereg}{#1}
		\pgfmathparse{\cmdPAS@prismereg@n==3}
		\IfStrEq{\pgfmathresult}{1.0}{\def\rotat{10}}{\def\rotat{\cmdPAS@prismereg@rotat}}
		\def\cotes{\cmdPAS@prismereg@n}
		\pgfmathparse{\cotes-1}\let\cote\pgfmathresult
		\ifPAS@prismereg@name
			\foreach \i in {1,2,...,\cotes}
			{
				\pgfmathparse{(360/\cotes)*(\i-1)+\cmdPAS@prismereg@rotat}\let\angle\pgfmathresult
				\pgfmathparse{\cmdPAS@prismereg@rayon*cos(\angle)}\let\Mx\pgfmathresult
				\pgfmathparse{\cmdPAS@prismereg@incl*sin(\angle)}\let\My\pgfmathresult
				\IfStrEq{\angle}{0.1}
				{
					\def\posname{right}
				}
				{
					\IfStrEq{\angle}{180.0}
					{
						\def\posname{left}
					}
					{
						\IfStrEq{\angle}{90.0}
						{
							\def\posname{above}
						}
						{
							\IfStrEq{\angle}{270.0}
							{
								\def\posname{below}
							}
							{
								\pgfmathparse{\angle<180}
								\IfStrEq{\pgfmathresult}{1.0}
								{
									\pgfmathparse{\angle<90}
									\IfStrEq{\pgfmathresult}{1.0}
									{
										\def\posname{above right}
									}
									{
										\def\posname{above left}
									}
								}
								{
									\pgfmathparse{\angle<270}
									\IfStrEq{\pgfmathresult}{1.0}
									{
										\def\posname{below left}
									}
									{
										\def\posname{below right}
									}
								}
							}
						}
					}
				}
				\node[\posname,\cmdPAS@prismereg@bordercolor] at (\Mx,\My) {$A_{\i}$};
			}
		\fi
		\pgfmathparse{\cmdPAS@prismereg@rayon*cos((360/\cotes)*\cote+\rotat)}\let\Mx\pgfmathresult
		\pgfmathparse{\cmdPAS@prismereg@incl*sin((360/\cotes)*\cote+\rotat)}\let\My\pgfmathresult
		\pgfmathparse{\cmdPAS@prismereg@rayon*cos(\rotat)}\let\Nx\pgfmathresult
		\pgfmathparse{\cmdPAS@prismereg@incl*sin(\rotat)}\let\Ny\pgfmathresult
		\pgfmathparse{\cmdPAS@prismereg@rayon*cos((360/\cotes)+\rotat)}\let\Px\pgfmathresult
		\pgfmathparse{\cmdPAS@prismereg@incl*sin((360/\cotes)+\rotat)}\let\Py\pgfmathresult
		\shade[draw=\cmdPAS@prismereg@bordercolor,shading=ball, ball color=\cmdPAS@prismereg@incolor,opacity=\cmdPAS@prismereg@coefopaq] (\Nx,\Ny) -- (\Px,\Py) -- (\Px,\Py+\cmdPAS@prismereg@hauteur) -- (\Nx,\Ny+\cmdPAS@prismereg@hauteur) -- cycle;
		\draw[\cmdPAS@prismereg@bordercolor] (\Mx,\My) -- (\Nx,\Ny) -- (\Nx,\Ny+\cmdPAS@prismereg@hauteur);
		\foreach \i in {1,...,\cote}
		{
			\pgfmathparse{\cmdPAS@prismereg@rayon*cos((360/\cotes)*\i+\rotat)}\let\Mx\pgfmathresult
			\pgfmathparse{\cmdPAS@prismereg@incl*sin((360/\cotes)*\i+\rotat)}\let\My\pgfmathresult
			\pgfmathparse{\cmdPAS@prismereg@hauteur+\cmdPAS@prismereg@incl*sin((360/\cotes)*\i+\rotat)}\let\MMy\pgfmathresult
			\path[coordinate] (\Mx,\MMy) coordinate (T\i);
			\IfStrEq{\i}{1}{\global\def\xi{\Mx}\global\def\yi{\My}}{}
			\pgfmathparse{\cmdPAS@prismereg@rayon*cos((360/\cotes)*(\i-1)+\rotat)}\let\Nx\pgfmathresult
			\pgfmathparse{\cmdPAS@prismereg@incl*sin((360/\cotes)*(\i-1)+\rotat)}\let\Ny\pgfmathresult
			\pgfmathparse{\cmdPAS@prismereg@rayon*cos((360/\cotes)*(\i+1)+\rotat)}\let\Px\pgfmathresult
			\pgfmathparse{\cmdPAS@prismereg@incl*sin((360/\cotes)*(\i+1)+\rotat)}\let\Py\pgfmathresult
			\IfStrEq{\cotes}{3}{\def\ang{120}}{\def\ang{140}}
			\pgfmathparse{(360/\cotes)*\i<181}
			\IfStrEq{\pgfmathresult}{1.0}
				{
					\draw[dotted,\cmdPAS@prismereg@bordercolor] (\Nx,\Ny) -- (\Mx,\My);
				}
				{
					\draw[\cmdPAS@prismereg@bordercolor] (\Nx,\Ny) -- (\Mx,\My);
				}
			\pgfmathparse{(360/\cotes)*\i<\ang}
			\IfStrEq{\pgfmathresult}{1.0}
				{
					\shade[shading=ball, ball color=\cmdPAS@prismereg@incolor,opacity=\cmdPAS@prismereg@coefopaq] (\Mx,\My+\cmdPAS@prismereg@hauteur) -- (\Mx,\My) -- (\Px,\Py) -- (\Px,\Py+\cmdPAS@prismereg@hauteur) -- cycle;
					\draw[dotted,\cmdPAS@prismereg@bordercolor] (\Mx,\My) -- (\Mx,\My+\cmdPAS@prismereg@hauteur);
				}
				{
					\shade[shading=ball, ball color=\cmdPAS@prismereg@incolor,opacity=\cmdPAS@prismereg@coefopaq] (\Mx,\My+\cmdPAS@prismereg@hauteur) -- (\Mx,\My) -- (\Px,\Py) -- (\Px,\Py+\cmdPAS@prismereg@hauteur) -- cycle;
					\draw[\cmdPAS@prismereg@bordercolor] (\Mx,\My+\cmdPAS@prismereg@hauteur) -- (\Mx,\My) -- (\Px,\Py) -- (\Px,\Py+\cmdPAS@prismereg@hauteur);
				}
		}
		\pgfmathparse{\cmdPAS@prismereg@rayon*cos(\rotat)}\let\Mx\pgfmathresult
		\pgfmathparse{\cmdPAS@prismereg@hauteur+\cmdPAS@prismereg@incl*sin(\rotat)}\let\MMy\pgfmathresult
		\path[coordinate] (\Mx,\MMy) coordinate (T\cotes);
		\shade[shading=ball, ball color=\cmdPAS@prismereg@incolor,opacity=\cmdPAS@prismereg@coefopaq] (T1)
		\foreach \i in {2,...,\cotes}{--(T\i)}--cycle;
		\draw[\cmdPAS@prismereg@bordercolor] (T1)
		\foreach \i in {2,...,\cotes}{--(T\i)}--cycle;
		\ifPAS@prismereg@legende
			\IfStrEq{\cmdPAS@prismereg@incolor}{white}{\def\couleur{black}}{\def\couleur{\cmdPAS@prismereg@incolor}}
			% Bases
			\pgfmathparse{\cmdPAS@prismereg@rayon*cos(-85)+.5}\let\x\pgfmathresult
			\pgfmathparse{\cmdPAS@prismereg@incl*sin(-85)-1}\let\y\pgfmathresult
			\draw[<-,>=stealth,dotted,\couleur] (.5,-.5) to [bend right=30] (\x,\y) node[right,\couleur] {\texttt{base (polygonale)}};
			\draw[<-,>=stealth,dotted,\couleur] (.5,.5+\cmdPAS@prismereg@hauteur) to [bend left=30] (\x,\y+\cmdPAS@prismereg@hauteur+4) node[right,\couleur] {\texttt{base (polygonale)}};
			% Faces latérales
			\pgfmathparse{\cmdPAS@prismereg@rayon*cos(\rotat)}\let\Mx\pgfmathresult
			\pgfmathparse{\cmdPAS@prismereg@incl*sin(\rotat)}\let\My\pgfmathresult
			\pgfmathparse{\cmdPAS@prismereg@rayon*cos(\rotat-360/\cotes)}\let\Nx\pgfmathresult
			\pgfmathparse{\cmdPAS@prismereg@incl*sin(\rotat-360/\cotes)}\let\Ny\pgfmathresult
			\coordinate (M) at (\Mx,\My);
			\coordinate (N) at (\Nx,\Ny);
			\coordinate (T) at (\Nx,\Ny+\cmdPAS@prismereg@hauteur);
			\coordinate (G) at (barycentric cs:M=1,N=1,T=1,T\cotes=1);
			\draw[<-,>=stealth,\couleur] (G) to [bend left=30] ($(G)+(1.5,.5)$) node[right,\couleur] {\begin{minipage}{3cm}\texttt{face lat\'erale (rectangulaire)}\end{minipage}};
			% Hauteur
			\draw[\cmdPAS@prismereg@axecolor,dotted] (T\cotes) -- ($(T\cotes)+(5,0)$);
			\draw[\cmdPAS@prismereg@axecolor,dotted] (M) -- ($(M)+(5,0)$);
			\draw[<->,\cmdPAS@prismereg@axecolor,>=stealth'] ($(T\cotes)+(5,0)$) -- ($(M)+(5,0)$) node[midway,right,\cmdPAS@prismereg@axecolor] {\texttt{hauteur}};
		\fi
		\ifPAS@prismereg@axe
			\draw[dotted,\cmdPAS@prismereg@axecolor] (0,\cmdPAS@prismereg@hauteur) -- (0,-\cmdPAS@prismereg@incl);
			\draw[\cmdPAS@prismereg@axecolor] (0,\cmdPAS@prismereg@hauteur) -- (0,\cmdPAS@prismereg@hauteur+1);
			\draw[\cmdPAS@prismereg@axecolor] (0,-\cmdPAS@prismereg@incl) -- (0,-\cmdPAS@prismereg@incl-1);
			\ifPAS@prismereg@legende
				\draw[<-,>=stealth',\cmdPAS@prismereg@axecolor] (0,\cmdPAS@prismereg@hauteur+.5) to [bend right=20] (-\cmdPAS@prismereg@rayon/2,\cmdPAS@prismereg@hauteur+1) node[left,\cmdPAS@prismereg@axecolor] {\texttt{axe de rotation}};
			\fi
		\fi
		\ifPAS@prismereg@name
			\foreach \i in {1,2,...,\cotes}
			{
				\pgfmathparse{(360/\cotes)*(\i-1)+\cmdPAS@prismereg@rotat}\let\angle\pgfmathresult
				\pgfmathparse{\cmdPAS@prismereg@rayon*cos(\angle)}\let\Mx\pgfmathresult
				\pgfmathparse{\cmdPAS@prismereg@incl*sin(\angle)}\let\My\pgfmathresult
				\IfStrEq{\angle}{0.1}
				{
					\def\posname{right}
				}
				{
					\IfStrEq{\angle}{180.0}
					{
						\def\posname{left}
					}
					{
						\IfStrEq{\angle}{90.0}
						{
							\def\posname{above}
						}
						{
							\IfStrEq{\angle}{270.0}
							{
								\def\posname{below}
							}
							{
								\pgfmathparse{\angle<180}
								\IfStrEq{\pgfmathresult}{1.0}
								{
									\pgfmathparse{\angle<90}
									\IfStrEq{\pgfmathresult}{1.0}
									{
										\def\posname{above right}
									}
									{
										\def\posname{above left}
									}
								}
								{
									\pgfmathparse{\angle<270}
									\IfStrEq{\pgfmathresult}{1.0}
									{
										\def\posname{below left}
									}
									{
										\def\posname{below right}
									}
								}
							}
						}
					}
				}
				\node[\posname,\cmdPAS@prismereg@bordercolor] at (\Mx,\My+\cmdPAS@prismereg@hauteur) {$B_{\i}$};
			}
		\fi
}


% \pyramreg : dessine une pyramide à base régulière
% Synthaxe : \pyramreg[	n = nombre de côtés de la base (>2) - par défaut : 3,
%						bordercolor = ... (par défaut : noir),
%						incolor = ... (par défaut : blanc),
%						coefopaq = ... (par défaut : 0.5),
%						axe (booléen : si oui, axe tracé),
%						axecolor = couleur de l'axe (par défaut : rouge),
%						name (booléen : si oui, nom du sommet et du centre de la base mentionnés),
%						sommet = nom du sommet (par défaut : S),
%						posommet = ...,
%						centre = nom du centre de la base (par défaut : O),
%						scalecentre = coef. d'agrandissement du point du centre de la base,
%						poscentre = position du nom du centre de la base (par défaut : below),
%						hauteur = ... (par défaut, 5 cm),
%						rayon = ... (par défaut, 2 cm),
%						legende : booléen,
%						incl = coefficient d'inclinaison de la base (par défaut : 1),
%						rotat = angle de rotation]

\define@cmdkey [PAS] {pyramreg} {n}{}
\define@cmdkey [PAS] {pyramreg} {bordercolor}{}
\define@cmdkey [PAS] {pyramreg} {incolor}{}
\define@cmdkey [PAS] {pyramreg} {coefopaq}{}
\define@boolkey[PAS] {pyramreg} {axe}[true]{}
\define@cmdkey [PAS] {pyramreg} {axecolor}{}
\define@boolkey[PAS] {pyramreg} {name}[true]{}
\define@boolkey[PAS] {pyramreg} {legende}[true]{}
\define@cmdkey [PAS] {pyramreg} {sommet}{}
\define@cmdkey [PAS] {pyramreg} {posommet}{}
\define@cmdkey [PAS] {pyramreg} {centre}{}
\define@cmdkey [PAS] {pyramreg} {poscentre}{}
\define@cmdkey [PAS] {pyramreg} {scalecentre}{}
\define@cmdkey [PAS] {pyramreg} {hauteur}{}
\define@cmdkey [PAS] {pyramreg} {rayon}{}
\define@cmdkey [PAS] {pyramreg} {rotat}{}
\define@cmdkey [PAS] {pyramreg} {incl}{}

\presetkeys    [PAS] {pyramreg} {
								n = 3,
								bordercolor = black,
 							 	incolor = white,
 							 	coefopaq = 0.5,
 							 	axe = false,
 							 	axecolor = red,
 							 	name = false,
 							 	legende = false,
 							 	sommet = S, 
 							 	posommet = above,							 	
 							 	centre = O,
 							 	poscentre = below,
 							 	scalecentre = 1,
 							 	hauteur = 5,
 							 	rayon = 2,
								rotat = 0.1,
 							 	incl = 1}{}

\newcommand*{\pyramreg}[1][]{\pasPyramreg[#1]}

\def\pasPyramreg[#1]{
	\setkeys[PAS]{pyramreg}{#1}
		\def\cotes{\cmdPAS@pyramreg@n}
		\pgfmathparse{\cotes-1}\let\cote\pgfmathresult
		\ifPAS@pyramreg@legende
			\pgfmathparse{\cmdPAS@pyramreg@rayon*cos(-85)+.5}\let\x\pgfmathresult
			\pgfmathparse{\cmdPAS@pyramreg@incl*sin(-85)-1}\let\y\pgfmathresult
			\draw[<-,>=stealth,dotted,\cmdPAS@pyramreg@incolor] (.5,-.5) to [bend right=30] (\x,\y) node[right,\cmdPAS@pyramreg@incolor] {\texttt{base (polygonale)}};
		\fi
		\ifPAS@pyramreg@name
			\fill[\cmdPAS@pyramreg@bordercolor] (0,0) circle (\cmdPAS@pyramreg@scalecentre*0.01cm) node[\cmdPAS@pyramreg@bordercolor,\cmdPAS@pyramreg@poscentre] {\cmdPAS@pyramreg@centre};
			\node[\cmdPAS@pyramreg@posommet,\cmdPAS@pyramreg@bordercolor] at (0,\cmdPAS@pyramreg@hauteur) {\cmdPAS@pyramreg@sommet};
			\foreach \i in {1,2,...,\cotes}
			{
				\pgfmathparse{(360/\cotes)*(\i-1)+\cmdPAS@pyramreg@rotat}\let\angle\pgfmathresult
				\pgfmathparse{\cmdPAS@pyramreg@rayon*cos(\angle)}\let\Mx\pgfmathresult
				\pgfmathparse{\cmdPAS@pyramreg@incl*sin(\angle)}\let\My\pgfmathresult
				\IfStrEq{\angle}{0.1}
				{
					\def\posname{right}
				}
				{
					\IfStrEq{\angle}{180.0}
					{
						\def\posname{left}
					}
					{
						\IfStrEq{\angle}{90.0}
						{
							\def\posname{above}
						}
						{
							\IfStrEq{\angle}{270.0}
							{
								\def\posname{below}
							}
							{
								\pgfmathparse{\angle<180}
								\IfStrEq{\pgfmathresult}{1.0}
								{
									\pgfmathparse{\angle<90}
									\IfStrEq{\pgfmathresult}{1.0}
									{
										\def\posname{above right}
									}
									{
										\def\posname{above left}
									}
								}
								{
									\pgfmathparse{\angle<270}
									\IfStrEq{\pgfmathresult}{1.0}
									{
										\def\posname{below left}
									}
									{
										\def\posname{below right}
									}
								}
							}
						}
					}
				}
				\node[\posname,\cmdPAS@pyramreg@bordercolor] at (\Mx,\My) {$A_{\i}$};
			}
		\fi
		\foreach \i in {1,...,\cote}
		{
			\pgfmathparse{\cmdPAS@pyramreg@rayon*cos((360/\cotes)*\i+\cmdPAS@pyramreg@rotat)}\let\Mx\pgfmathresult
			\pgfmathparse{\cmdPAS@pyramreg@incl*sin((360/\cotes)*\i+\cmdPAS@pyramreg@rotat)}\let\My\pgfmathresult
			\pgfmathparse{\cmdPAS@pyramreg@rayon*cos((360/\cotes)*(\i-1)+\cmdPAS@pyramreg@rotat)}\let\Nx\pgfmathresult
			\pgfmathparse{\cmdPAS@pyramreg@incl*sin((360/\cotes)*(\i-1)+\cmdPAS@pyramreg@rotat)}\let\Ny\pgfmathresult
			\pgfmathparse{\cmdPAS@pyramreg@rayon*cos((360/\cotes)*(\i+1)+\cmdPAS@pyramreg@rotat)}\let\Px\pgfmathresult
			\pgfmathparse{\cmdPAS@pyramreg@incl*sin((360/\cotes)*(\i+1)+\cmdPAS@pyramreg@rotat)}\let\Py\pgfmathresult
			\IfStrEq{\cotes}{3}{\def\ang{120}}{\def\ang{140}}
			\pgfmathparse{(360/\cotes)*\i<181}
			\IfStrEq{\pgfmathresult}{1.0}{\draw[dotted,\cmdPAS@pyramreg@bordercolor] (\Nx,\Ny) -- (\Mx,\My);}{\draw[\cmdPAS@pyramreg@bordercolor] (\Nx,\Ny) -- (\Mx,\My);}
			\pgfmathparse{(360/\cotes)*\i+\cmdPAS@pyramreg@rotat<\ang}
			\IfStrEq{\pgfmathresult}{1.0}{\draw[dotted,\cmdPAS@pyramreg@bordercolor] (\Mx,\My) -- (0,\cmdPAS@pyramreg@hauteur);}
			{
			\shade[draw=\cmdPAS@pyramreg@bordercolor,shading=ball, ball color=\cmdPAS@pyramreg@incolor,opacity=\cmdPAS@pyramreg@coefopaq] (0,\cmdPAS@pyramreg@hauteur) -- (\Mx,\My) -- (\Px,\Py) -- cycle;
		\draw[\cmdPAS@pyramreg@bordercolor] (0,\cmdPAS@pyramreg@hauteur) -- (\Mx,\My) -- (\Px,\Py);
			}
		}
		\pgfmathparse{\cmdPAS@pyramreg@rayon*cos((360/\cotes)*\cote+\cmdPAS@pyramreg@rotat)}\let\Mx\pgfmathresult
		\pgfmathparse{\cmdPAS@pyramreg@incl*sin((360/\cotes)*\cote+\cmdPAS@pyramreg@rotat)}\let\My\pgfmathresult
		\pgfmathparse{\cmdPAS@pyramreg@rayon*cos(\cmdPAS@pyramreg@rotat)}\let\Nx\pgfmathresult
		\pgfmathparse{+\cmdPAS@pyramreg@incl*sin(\cmdPAS@pyramreg@rotat)}\let\Ny\pgfmathresult
		\draw[\cmdPAS@pyramreg@bordercolor] (\Mx,\My) -- (\Nx,\Ny) -- (0,\cmdPAS@pyramreg@hauteur);
		\ifPAS@pyramreg@axe
			\draw[dotted,\cmdPAS@pyramreg@axecolor] (0,\cmdPAS@pyramreg@hauteur) -- (0,-\cmdPAS@pyramreg@incl);
			\draw[\cmdPAS@pyramreg@axecolor] (0,\cmdPAS@pyramreg@hauteur) -- (0,\cmdPAS@pyramreg@hauteur+1);
			\draw[\cmdPAS@pyramreg@axecolor] (0,-\cmdPAS@pyramreg@incl) -- (0,-\cmdPAS@pyramreg@incl-1);
			\ifPAS@pyramreg@legende
				\draw[<-,>=stealth',\cmdPAS@pyramreg@axecolor] (0,\cmdPAS@pyramreg@hauteur+.5) to [bend left=20] (-\cmdPAS@pyramreg@rayon/2,\cmdPAS@pyramreg@hauteur) node[left,\cmdPAS@pyramreg@axecolor] {\texttt{axe de rotation}};
			\fi
		\fi
		\ifPAS@pyramreg@legende
			\pgfmathparse{(\cmdPAS@pyramreg@hauteur-\Ny)/2}\let\y\pgfmathresult
			\pgfmathparse{1-\cmdPAS@pyramreg@coefopaq}\let\op\pgfmathresult
			\draw[<-,>=stealth,\cmdPAS@pyramreg@bordercolor] (\Nx/2,\y) to [bend left=30] (\cmdPAS@pyramreg@rayon+.5,\y+1) node[right,\cmdPAS@pyramreg@bordercolor] {\texttt{arête lat\'erale}};
			\draw[<-,>=stealth,\cmdPAS@pyramreg@bordercolor] (0,\cmdPAS@pyramreg@hauteur) to [bend left=10] (\cmdPAS@pyramreg@rayon+.5,\cmdPAS@pyramreg@hauteur+0.5) node[right,\cmdPAS@pyramreg@bordercolor] {\texttt{sommet}};
			\draw[gray,dotted] (0,\cmdPAS@pyramreg@hauteur) -- (\cmdPAS@pyramreg@rayon+4,\cmdPAS@pyramreg@hauteur);
			\draw[gray,dotted] (\cmdPAS@pyramreg@rayon,0) -- (\cmdPAS@pyramreg@rayon+4,0);
			\draw[gray,dotted,opacity=\op] (0,0) -- (\cmdPAS@pyramreg@rayon,0);
			\draw[<->,>=stealth',\cmdPAS@pyramreg@bordercolor] (\cmdPAS@pyramreg@rayon+4,\cmdPAS@pyramreg@hauteur) -- (\cmdPAS@pyramreg@rayon+4,0) node[midway,right,\cmdPAS@pyramreg@bordercolor] {\texttt{hauteur}};
			\draw[<-,>=stealth,\cmdPAS@pyramreg@incolor] (\Nx/2,.5) to [bend right=30] (\cmdPAS@pyramreg@rayon+.5,-1) node[right,\cmdPAS@pyramreg@incolor] {\texttt{face lat\'erale (triangulaire)}};
		\fi
}


% \boule : dessine une boule
% Synthaxe : \boule[border, (booléen : si oui, un bord est dessiné
%					bordercolor = ... (par défaut : noir),
%					name, (booléen : si oui, le centre est dessiné),
%					centre = nom du centre,
%					poscentre = position du nom du centre,
%					incolor = couleur de remplissage (par défaut : blanc),
%					coefopaq = coefficient d'opacité,
%					grandcercle (booléen : si oui, le grand cercle est dessiné),
%					legende (booléen),
%					scale = coefficient d'agrandissement]

\define@boolkey[PAS] {boule} {border}[true]{}
\define@cmdkey [PAS] {boule} {bordercolor}{}
\define@boolkey[PAS] {boule} {name}[true]{}
\define@cmdkey [PAS] {boule} {centre}{}
\define@cmdkey [PAS] {boule} {poscentre}{}
\define@cmdkey [PAS] {boule} {incolor}{}
\define@cmdkey [PAS] {boule} {coefopaq}{}
\define@boolkey[PAS] {boule} {grandcercle}[true]{}
\define@boolkey[PAS] {boule} {legende}[true]{}
\define@cmdkey [PAS] {boule} {scale}{}

\presetkeys    [PAS] {boule} {	border = false,
								bordercolor = black,
 							 	incolor = white,
 							 	coefopaq = 0.5,
 							 	centre = O,
 							 	poscentre = below,
 							 	scale = 1,
 							 	grandcercle = false,
 							 	legende = false,
 							 	name = false}{}

\newcommand*{\boule}[1][]{\pasBoule[#1]}

\def\pasBoule[#1]{
	\setkeys[PAS]{boule}{#1}
	\begin{scope}[scale=\cmdPAS@boule@scale]
		\ifPAS@boule@grandcercle		
			\draw[\cmdPAS@boule@bordercolor,dotted] (2,0) arc (0:180:2cm and .5cm);
		\fi
		\ifPAS@boule@name
			\fill[\cmdPAS@boule@bordercolor] (0,0) circle (0.01cm) node[\cmdPAS@boule@poscentre,scale=\cmdPAS@boule@scale] {\cmdPAS@boule@centre};
		\fi
		\shade[shading=ball, ball color=\cmdPAS@boule@incolor,opacity=\cmdPAS@boule@coefopaq] (0,0) circle (2cm);
		\ifPAS@boule@grandcercle	
			\draw[\cmdPAS@boule@bordercolor] (2,0) arc (0:-180:2cm and .5cm);
		\fi
		\ifPAS@boule@border
			\draw[\cmdPAS@boule@bordercolor] (0,0) circle (2cm);
		\fi
		\ifPAS@boule@legende
			\pgfmathparse{2*cos(-60)}\let\x\pgfmathresult
			\pgfmathparse{.5*sin(-60)}\let\y\pgfmathresult	
			\draw[<-,>=stealth',\cmdPAS@boule@bordercolor] (\x,\y) to [bend right=30] (3,-0.4) node[right,\cmdPAS@boule@bordercolor] {\texttt{un grand cercle}};
			\begin{scope}[rotate=50]
				\draw[\cmdPAS@boule@bordercolor!50!black,dotted] (2,0) arc (0:180:2cm and .5cm);
				\draw[\cmdPAS@boule@bordercolor!50!black] (2,0) arc (0:-180:2cm and .5cm);				
			\end{scope}
			\pgfmathparse{1.5*cos(38)}\let\x\pgfmathresult
			\pgfmathparse{1.5*sin(38)}\let\y\pgfmathresult	
			\draw[<-,>=stealth',\cmdPAS@boule@bordercolor!50!black] (\x,\y) to [bend right=30] (3,\y) node[right,\cmdPAS@boule@bordercolor!50!black] {\texttt{un autre grand cercle}};

		\fi
	\end{scope}
}

% \cone : dessine un cône de révolution
% Synthaxe :  \cone[bordercolor=couleur des bords, (black par défaut)
%					incolor = couleur de la face, (white par défaut)
%					coefopaq = coefficient d'opacité (0.5 par défaut),
%					rayon = rayon du disque de base, (1 cm par défaut),
%				    hauteur = hauteur du sommet (3 cm par défaut),
%					sommet = nom du sommet, (par défaut : S)
%					posommet = position du nom du sommet (par défaut : above),
%					centre = nom du centre du disque de base, (par défaut : O),
%					incl = coefficient d'inclinaison du disque de base,
%					poscentre = position du centre du disque de base (par défaut : below),
%					scalecentre = coefficient d'agrendissement du point du centre du disque de base,
%					axe : booléen,
%					axecolor = couleur de l'axe de révolution,
%					legende : booléen,
%					name : booléen]

\define@cmdkey [PAS] {cone} {bordercolor}{}
\define@cmdkey [PAS] {cone} {incolor}{}
\define@cmdkey [PAS] {cone} {hauteur}{}
\define@cmdkey [PAS] {cone} {coefopaq}{}
\define@cmdkey [PAS] {cone} {rayon}{}
\define@cmdkey [PAS] {cone} {sommet}{}
\define@cmdkey [PAS] {cone} {posommet}{}
\define@cmdkey [PAS] {cone} {centre}{}
\define@cmdkey [PAS] {cone} {poscentre}{}
\define@cmdkey [PAS] {cone} {scalecentre}{}
\define@cmdkey [PAS] {cone} {incl}{}
\define@boolkey[PAS] {cone} {name}[true]{}
\define@boolkey[PAS] {cone} {axe}[true]{}
\define@boolkey[PAS] {cone} {legende}[true]{}
\define@cmdkey [PAS] {cone} {axecolor}{}
\presetkeys    [PAS] {cone} {bordercolor = black,
 							 incolor = white,
 							 coefopaq = 0.5,
 							 rayon = 1,
 							 hauteur = 3,
 							 sommet = S,
 							 posommet = above,
 							 centre = O,
 							 poscentre = below,
 							 scalecentre = 1,
 							 incl = 0.33,
 							 name = false,
 							 axe = false,
 							 legende = false,
 							 axecolor = red}{}

\newcommand*{\cone}[1][]{\pasCone[#1]}

\def\pasCone[#1]{
	\setkeys[PAS]{cone}{#1}
		\pgfmathparse{\cmdPAS@cone@incl*\cmdPAS@cone@rayon}\let\petitaxe\pgfmathresult
		\pgfmathparse{2*\cmdPAS@cone@rayon}\let\grandaxe\pgfmathresult		
		\pgfmathparse{0.01*\cmdPAS@cone@scalecentre}\let\centre\pgfmathresult			
		\draw[dashed,\cmdPAS@cone@bordercolor] (0,0) arc (180:0:\cmdPAS@cone@rayon cm and \petitaxe cm);
		\ifPAS@cone@axe
			\draw[dotted,\cmdPAS@cone@axecolor] (\cmdPAS@cone@rayon,\cmdPAS@cone@hauteur) -- (\cmdPAS@cone@rayon,-\petitaxe);
			\draw[\cmdPAS@cone@axecolor] (\cmdPAS@cone@rayon,\cmdPAS@cone@hauteur) -- (\cmdPAS@cone@rayon,\cmdPAS@cone@hauteur+1);
			\draw[\cmdPAS@cone@axecolor] (\cmdPAS@cone@rayon,-\petitaxe) -- (\cmdPAS@cone@rayon,-\petitaxe-1);
			\ifPAS@cone@legende
				\draw[\cmdPAS@cone@axecolor,<-,>=stealth'] (\cmdPAS@cone@rayon,\cmdPAS@cone@hauteur+.5) to [bend left=30] (\cmdPAS@cone@rayon+1,\cmdPAS@cone@hauteur+1) node[right,\cmdPAS@cone@axecolor] {\texttt{axe de r\'evolution}};
			\fi		
		\fi
		\shade[draw=\cmdPAS@cone@bordercolor,shading=ball, ball color=\cmdPAS@cone@incolor,opacity=\cmdPAS@cone@coefopaq] (0,0) -- (\cmdPAS@cone@rayon,\cmdPAS@cone@hauteur) -- (\grandaxe,0) arc (0:-180:\cmdPAS@cone@rayon cm and \petitaxe cm);
		\ifPAS@cone@name
			\node[\cmdPAS@cone@posommet,\cmdPAS@cone@bordercolor] at (\cmdPAS@cone@rayon,\cmdPAS@cone@hauteur) {\cmdPAS@cone@sommet};
			\node[\cmdPAS@cone@poscentre,\cmdPAS@cone@bordercolor] at (\cmdPAS@cone@rayon,0) {\cmdPAS@cone@centre};
			\fill[\cmdPAS@cone@bordercolor]	(\cmdPAS@cone@rayon,0) circle (\centre cm);
		\fi
		\ifPAS@cone@legende
			\pgfmathparse{\petitaxe*sin(-40)}\let\y\pgfmathresult
			\pgfmathparse{\cmdPAS@cone@rayon*(1+cos(-40))}\let\x\pgfmathresult			
			\draw[\cmdPAS@cone@bordercolor,thick] (\cmdPAS@cone@rayon,\cmdPAS@cone@hauteur) -- (\x,\y);
			\draw[<-,>=stealth',\cmdPAS@cone@bordercolor] (\x,\y) to [bend right=30] (2*\cmdPAS@cone@rayon+1,0) node[\cmdPAS@cone@bordercolor,right] {\texttt{g\'en\'eratrice}};
			\draw[black,<-,>=stealth'] (\cmdPAS@cone@rayon,\cmdPAS@cone@hauteur) to [bend right=45] (\cmdPAS@cone@rayon/2,\cmdPAS@cone@hauteur) node[black,below left] {\texttt{sommet}};
			\draw[black,<-,>=stealth'] (\cmdPAS@cone@rayon,0) to [bend right=30] (\cmdPAS@cone@rayon/2,-\petitaxe-.5) node[black,below] {\texttt{centre du disque de base}};
		\fi
}


% \cylindre : dessine un cône de révolution
% Synthaxe :  \cylindre[bordercolor=couleur des bords, (black par défaut)
%					incolor = couleur de la face, (white par défaut)
%					coefopaq = coefficient d'opacité (0.5 par défaut),
%					rayon = rayon du disque de base, (1 cm par défaut),
%				    hauteur = hauteur du sommet (3 cm par défaut),
%					incl = coefficient d'inclinaison du disque de base,
%					centrebas = nom du centre du disque de base, (par défaut : B),
%					centrehaut = nom du centre du disque de base, (par défaut : H),
%					poscentrebas = position du centre du disque de base (par défaut : below),
%					poscentrehaut = position du centre du disque du haut (par défaut : below),
%					scalecentre = coefficient d'agrendissement du point du centre du disque de base,
%					axe : booléen,
%					axecolor = couleur de l'axe de révolution,
%					legende : booléen,
%					name : booléen]

\define@cmdkey [PAS] {cylindre} {bordercolor}{}
\define@cmdkey [PAS] {cylindre} {incolor}{}
\define@cmdkey [PAS] {cylindre} {hauteur}{}
\define@cmdkey [PAS] {cylindre} {coefopaq}{}
\define@cmdkey [PAS] {cylindre} {rayon}{}
\define@cmdkey [PAS] {cylindre} {centrebas}{}
\define@cmdkey [PAS] {cylindre} {poscentrebas}{}
\define@cmdkey [PAS] {cylindre} {centrehaut}{}
\define@cmdkey [PAS] {cylindre} {poscentrehaut}{}
\define@cmdkey [PAS] {cylindre} {scalecentre}{}
\define@cmdkey [PAS] {cylindre} {incl}{}
\define@boolkey[PAS] {cylindre} {name}[true]{}
\define@boolkey[PAS] {cylindre} {axe}[true]{}
\define@boolkey[PAS] {cylindre} {legende}[true]{}
\define@boolkey[PAS] {cylindre} {rectgener}[true]{}
\define@cmdkey [PAS] {cylindre} {axecolor}{}
\presetkeys    [PAS] {cylindre} {bordercolor = black,
 							 incolor = white,
 							 coefopaq = 0.5,
 							 rayon = 1,
 							 hauteur = 3,
 							 centrebas = B,
 							 poscentrebas = above,
 							 centrehaut = H,
 							 poscentrehaut = below,
 							 scalecentre = 1,
 							 incl = 0.33,
 							 name = false,
 							 axe = false,
 							 legende = false,
							 rectgener = false,
 							 axecolor = red}{}

\newcommand*{\cylindre}[1][]{\pasCylindre[#1]}

\def\pasCylindre[#1]{
	\setkeys[PAS]{cylindre}{#1}
		\pgfmathparse{\cmdPAS@cylindre@incl*\cmdPAS@cylindre@rayon}\let\petitaxe\pgfmathresult
		\pgfmathparse{2*\cmdPAS@cylindre@rayon}\let\grandaxe\pgfmathresult		
		\pgfmathparse{0.01*\cmdPAS@cylindre@scalecentre}\let\centre\pgfmathresult	
		\ifPAS@cylindre@rectgener
			\pgfmathparse{\petitaxe*sin(-40)}\let\y\pgfmathresult
			\pgfmathparse{\cmdPAS@cylindre@rayon*(cos(-40))}\let\x\pgfmathresult	
			\filldraw[opacity=2*\cmdPAS@cylindre@coefopaq,fill=\cmdPAS@cylindre@incolor,draw=\cmdPAS@cylindre@bordercolor] (0,0) -- (\x,\y) -- (\x,\y+\cmdPAS@cylindre@hauteur) -- (0,\cmdPAS@cylindre@hauteur);
			\ifPAS@cylindre@legende
				\draw[\cmdPAS@cylindre@bordercolor,<-,>=stealth'] (.75*\x,.75*\y+.3) to [bend right=45] (1.3*\x,2*\y) node[right,\cmdPAS@cylindre@bordercolor] {\texttt{rectangle g\'en\'erateur}};
			\fi
		\fi			
		\draw[dashed,\cmdPAS@cylindre@bordercolor] (\cmdPAS@cylindre@rayon,0) arc (0:180:\cmdPAS@cylindre@rayon cm and \petitaxe cm);
		\ifPAS@cylindre@axe
			\draw[dotted,\cmdPAS@cylindre@axecolor] (0,\cmdPAS@cylindre@hauteur) -- (0,-\petitaxe);
			\draw[\cmdPAS@cylindre@axecolor] (0,\cmdPAS@cylindre@hauteur) -- (0,\cmdPAS@cylindre@hauteur+1);
			\draw[\cmdPAS@cylindre@axecolor] (0,-\petitaxe) -- (0,-\petitaxe-1);
			\ifPAS@cylindre@legende
				\draw[\cmdPAS@cylindre@axecolor,<-,>=stealth'] (0,\cmdPAS@cylindre@hauteur+.5) to [bend left=30] (1,\cmdPAS@cylindre@hauteur+1) node[right,\cmdPAS@cylindre@axecolor] {\texttt{axe de r\'evolution}};
			\fi		
		\fi
		\shade[draw=\cmdPAS@cylindre@bordercolor,shading=ball, ball color=\cmdPAS@cylindre@incolor,opacity=\cmdPAS@cylindre@coefopaq] (-\cmdPAS@cylindre@rayon,0) arc (180:360:\cmdPAS@cylindre@rayon cm and \petitaxe cm) -- (\cmdPAS@cylindre@rayon,\cmdPAS@cylindre@hauteur) arc (0:180:\cmdPAS@cylindre@rayon cm and \petitaxe cm) -- cycle;
		\draw[draw=\cmdPAS@cylindre@bordercolor] (-\cmdPAS@cylindre@rayon,0) arc (180:360:\cmdPAS@cylindre@rayon cm and \petitaxe cm) -- (\cmdPAS@cylindre@rayon,\cmdPAS@cylindre@hauteur) arc (0:180:\cmdPAS@cylindre@rayon cm and \petitaxe cm) -- cycle;
		\draw[draw=\cmdPAS@cylindre@bordercolor] (-\cmdPAS@cylindre@rayon,\cmdPAS@cylindre@hauteur) arc (180:360:\cmdPAS@cylindre@rayon cm and \petitaxe cm);
		\ifPAS@cylindre@name
			\node[\cmdPAS@cylindre@poscentrehaut,\cmdPAS@cylindre@bordercolor] at (0,\cmdPAS@cylindre@hauteur) {\cmdPAS@cylindre@centrehaut};
			\node[\cmdPAS@cylindre@poscentrebas,\cmdPAS@cylindre@bordercolor] at (0,0) {\cmdPAS@cylindre@centrebas};
			\fill[\cmdPAS@cylindre@bordercolor]	(0,0) circle (\centre cm);
			\fill[\cmdPAS@cylindre@bordercolor]	(0,\cmdPAS@cylindre@hauteur) circle (\centre cm);
		\fi
		\ifPAS@cylindre@legende		
			\draw[dotted,\cmdPAS@cylindre@bordercolor] (\cmdPAS@cylindre@rayon,0) -- (\cmdPAS@cylindre@rayon+1,0);
			\draw[dotted,\cmdPAS@cylindre@bordercolor] (\cmdPAS@cylindre@rayon,\cmdPAS@cylindre@hauteur) -- (\cmdPAS@cylindre@rayon+1,\cmdPAS@cylindre@hauteur);
			\draw[<->,>=stealth',\cmdPAS@cylindre@bordercolor] (\cmdPAS@cylindre@rayon+1,0) -- (\cmdPAS@cylindre@rayon+1,\cmdPAS@cylindre@hauteur) node[midway,right,\cmdPAS@cylindre@bordercolor] {\texttt{hauteur}};
			\draw[black,<-,>=stealth',\cmdPAS@cylindre@bordercolor] (-0.5*\cmdPAS@cylindre@rayon,0.5*\cmdPAS@cylindre@hauteur) to [bend right=45] (-1.2*\cmdPAS@cylindre@rayon,0.5*\cmdPAS@cylindre@hauteur+.3) node[\cmdPAS@cylindre@bordercolor,left] {\texttt{face lat\'erale}};
		\fi	
}


% \cube : dessine un cube
% Synthaxe : \cube[bordercolor=couleur voulue pour les bords, - par défaut : black (noir)
%				   incolor = couleur d'intérieur, - par défaut : white (blanc)
%				   angle = angle de la perspective, - par défaut : 45°
%				   coefopaq = coefficient d'opacité - par défaut : 0.5,
%				   name (booléen : si mentionné, noms des sommets apparents,
%				   prof = profondeur pour un parallélépipède rectangle,
%				   scale = agrandissement,
%				   legende : booléen pour mettre les légendes ou pas]

\define@cmdkey [PAS] {cube} {bordercolor}{}
\define@cmdkey [PAS] {cube} {incolor}{}
\define@cmdkey [PAS] {cube} {angle}{}
\define@cmdkey [PAS] {cube} {coefopaq}{}
\define@cmdkey [PAS] {cube} {scale}{}
\define@cmdkey [PAS] {cube} {prof}{}
\define@boolkey[PAS] {cube} {name}[true]{}
\define@boolkey[PAS] {cube} {legende}[true]{}
\presetkeys    [PAS] {cube} {angle = 45,
 							 bordercolor = black,
 							 incolor = white,
 							 coefopaq = 0.5,
 							 prof = 1,
 							 scale = 1,
 							 legende = false,
 							 name = false}{}

\newcommand*{\cube}[1][]{\pasCube[#1]}

\def\pasCube[#1]{
	\setkeys[PAS]{cube}{#1}
	\begin{scope}[scale=\cmdPAS@cube@scale]
		\shade[draw=\cmdPAS@cube@bordercolor,shading=ball, ball color=\cmdPAS@cube@incolor,opacity=\cmdPAS@cube@coefopaq] (0,0) -- (1,0) -- (1,1) -- (0,1) -- cycle;
		\pgfmathparse{\cmdPAS@cube@prof*cos(\cmdPAS@cube@angle)/2}\let\x\pgfmathresult
		\pgfmathparse{\cmdPAS@cube@prof*sin(\cmdPAS@cube@angle)/2}\let\y\pgfmathresult
		\pgfmathparse{1+\x}\let\xx\pgfmathresult
		\pgfmathparse{1+\y}\let\yy\pgfmathresult
		\draw[dotted,\cmdPAS@cube@bordercolor] (0,0) -- (\x,\y) -- (\xx,\y);
		\draw[dotted,\cmdPAS@cube@bordercolor] (\x,\y) -- (\x,\yy);
		\shade[draw=\cmdPAS@cube@bordercolor,shading=ball, ball color=\cmdPAS@cube@incolor,opacity=\cmdPAS@cube@coefopaq] (0,1) -- (\x,\yy) -- (\xx,\yy) -- (1,1) -- cycle;
		\filldraw[draw=\cmdPAS@cube@bordercolor,shading=ball, ball color=\cmdPAS@cube@incolor,opacity=\cmdPAS@cube@coefopaq] (1,0) -- (1,1) -- (\xx,\yy) -- (\xx,\y) -- cycle;
		\ifPAS@cube@name
			\node[below left,\cmdPAS@cube@bordercolor] at (0,0) {A};
			\node[below right,\cmdPAS@cube@bordercolor] at (1,0) {B};
			\node[below right,\cmdPAS@cube@bordercolor] at (\xx,\y) {C};
			\node[below,\cmdPAS@cube@bordercolor] at (\x,\y) {D};
			\node[left,\cmdPAS@cube@bordercolor] at (0,1) {E};
			\node[below right,\cmdPAS@cube@bordercolor] at (1,1) {F};
			\node[right,\cmdPAS@cube@bordercolor] at (\xx,\yy) {G};
			\node[above,\cmdPAS@cube@bordercolor] at (\x,\yy) {H};
		\fi
		\ifPAS@cube@legende
			\pgfmathparse{(\xx-1)/2+1}\let\xxx\pgfmathresult
			\pgfmathparse{(\yy-1)/2+1}\let\yyy\pgfmathresult			
			\draw[<-,>=stealth',\cmdPAS@cube@bordercolor] (.5,0) to [bend right=30] (2,0) node[right,\cmdPAS@cube@bordercolor] {\texttt{arête}};
			\IfStrEq{\cmdPAS@cube@incolor}{white}{\def\c{black}}{\def\c{\cmdPAS@cube@incolor}}
			\draw[<-,>=stealth',\c] (\xxx,.5) to [bend left=30] ($(\xxx,.5)+(1,0.25)$) node[right,\c] {\texttt{face}};
			\fill[\cmdPAS@cube@bordercolor] (1,1) circle (0.01cm);
			\draw[<-,>=stealth',\cmdPAS@cube@bordercolor] (1,1) to [bend left=45] ($(\xx,\yy)+(0.25,0.25)$) node[right,\cmdPAS@cube@bordercolor] {\texttt{sommet}};
		\fi
	\end{scope}
}
