\documentclass{article}
\usepackage{casyltex}
\title{\CASylTeX: Macros for Cree/Inuktitut\\ Version 2.00}
\author{Ivan A Derzhanski}
\date{25 October 2008}
\def\zdemo#1{$\frac{\hbox{\sylla #1}}{\hbox{\Ztwo\sylla #1}}$}

\begin{document}
\maketitle

This version of package \CASylTeX\ (Canadian Aboriginal Syllabics)
enables you to typeset Cree and Inuktitut text%
\footnote{Support for other languages will be added in later versions.}
%
in James Evans' syllabic script.
It consists of the style sheet \verb!casyltex.sty!
and the fount \verb!casyll10!.

The typesetting of Cree/Inuktitut text
is done by the environment \verb!syllab!.
For short quotations (no longer than one paragraph)
the macro \verb!\sylla! is also available.

The input of \CASylTeX\ is romanised text in lowercase (except as specified below)
and with no punctuation (other than full stops, which are rendered as~\sylla{..}).
The following syllables are recognised:

\medskip
\hspace{-.45in}\begin{tabular}{r|cccccccccccccccc}
& --- & \texttt c, \texttt g & \texttt j, \texttt y & \texttt k
& \texttt l & \texttt L & \texttt m & \texttt n
& \texttt p & \texttt q & \texttt r & \texttt s
& \texttt S & \texttt t & \texttt T & \texttt v, \texttt f
\\\hline
\texttt a & \sylla{a} & \sylla{ca} & \sylla{ja} & \sylla{ka}
& \sylla{la} & \sylla{La} & \sylla{ma} & \sylla{na}
& \sylla{pa} & \sylla{qa} & \sylla{ra}/\Rtwo\sylla{ra} & \sylla{sa}
& \sylla{Sa} & \sylla{ta} & \sylla{Ta}/\Ttwo\sylla{Ta} & \sylla{va}
\\
\texttt e & \sylla{e} & \sylla{ce} & \sylla{je} & \sylla{ke}
& \sylla{le} & \sylla{Le} & \sylla{me} & \sylla{ne}
& \sylla{pe} & \sylla{qe} & \sylla{re}/\Rtwo\sylla{re} & \sylla{se}
& \sylla{Se} & \sylla{te} & \sylla{Te}/\Ttwo\sylla{Te} & \sylla{ve}
\\
\texttt i & \sylla{i} & \sylla{ci} & \sylla{ji} & \sylla{ki}
& \sylla{li} & \sylla{Li} & \sylla{mi} & \sylla{ni}
& \sylla{pi} & \sylla{qi} & \sylla{ri}/\Rtwo\sylla{ri} & \sylla{si}
& \sylla{Si} & \sylla{ti} & \sylla{Ti}/\Ttwo\sylla{Ti} & \sylla{vi}
\\
\texttt o, \texttt u & \sylla{o} & \sylla{co} & \sylla{jo} & \sylla{ko}
& \sylla{lo} & \sylla{Lo} & \sylla{mo} & \sylla{no}
& \sylla{po} & \sylla{qo} & \sylla{ro}/\Rtwo\sylla{ro} & \sylla{so}
& \sylla{So} & \sylla{to} & \sylla{Ti}/\Ttwo\sylla{Ti} & \sylla{vo}
\\\hline
--- & --- & \zdemo c & \zdemo j & \zdemo k
& \zdemo l & \sylla L & \zdemo m & \zdemo n
& \zdemo p & \sylla q & \zdemo r/\Rtwo\sylla r & \zdemo s
& \zdemo S & \zdemo t & \sylla T/\Ttwo \sylla T & \sylla v
\\
\end{tabular}
\medskip\\
%
And the following non-syllables:

\medskip
\begin{tabular}{cccc}
\texttt h & \texttt K & \texttt M & \texttt x \\\hline
\sylla h & \sylla K & \sylla M & \sylla x \\
\end{tabular}
\medskip\\
%
The input \texttt N generates \sylla N (resp.~\sylla{\Ztwo N}, see below;
but there is no \texttt{Na},
so you have to write \texttt{Nga} for \sylla{Nga}, etc.).

The input \texttt w generates a dot next to the syllabic character
if a vowel follows and \sylla w otherwise.
By default the dot appears after the character (as for West Cree), but can be made to
appear before it (as for East Cree) by \verb!\wfronttrue! (back by \verb!\wfrontfalse!).

The letters \texttt c and~\texttt g, \texttt j and~\texttt y,
\texttt o and~\texttt u, \texttt v and~\texttt f have the same effect.
Inuktitut~\textit\& (voiceless fricative \textit l)
and regional Cree \textit{sh} and \textit{th}
are input as \texttt L,~\texttt S and~\texttt T, respectively.
Vowel length is marked by capitalising the vowel letter
or by \verb!'! after~it; \verb!hwe'! and~\verb!hwE!
both generate~\sylla{hwE}.

There are two shapes available for \texttt r. You can choose
\sylla r (the default) by~\verb!\Rone! and \sylla {\Rtwo r}
by~\verb!\Rtwo!, and whichever shape is chosen, \texttt R
generates the other one.

There are also two options for most syllable-final consonants:
they either look like superscript syllables with~\textit a
(the default) or have independent shapes.  You can indicate
your preference by~\verb!\Zone! or~\verb!\Ztwo!.

\end{document}
