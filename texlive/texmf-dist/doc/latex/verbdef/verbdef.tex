\documentclass[a4paper]{article}
\usepackage[dvips,a4paper]{geometry}
\usepackage{verbdef}
\verbdef\verbdefcmd|\verbdef|
\providecommand\cs[1]{\texttt{\char`\\#1}}
\begin{document}
\title{The \textsf{verbdef} package\thanks{This documents version 0.2, of
    2000-10-06}}
\author{Robin Fairbairns}
\maketitle

\section{What is it for?}

Verbatim text is notoriously difficult in \TeX{}; in \LaTeX{} it adds
confusion by being ``more fragile than fragile''.

This package provides one way out of the impasse: the command
\verbdefcmd, which defines a robust command that expands to verbatim
text.

\section{Using \verbdefcmd}

The \verbdefcmd{} command can take an its verbatim string between
matching characters, as \cs{verb} does:
\begin{quote}
\verbdefcmd\cs{demo}\texttt{|demonstration text|}
\end{quote}
or you may give it a more normal-looking argument:
\begin{quote}
  \verbdefcmd\cs{demo}\texttt{\char`\{demonstration text\char`\}}
\end{quote}

The command also respects ``\texttt{*}'', as it's used in \cs{verb}
and the \texttt{verbatim} environment:
\begin{quote}
  \verbdefcmd\texttt{*}\cs{stardemo}\texttt{\char`\{with visible spaces\char`\}}
\end{quote}
%
% another self-reference...
\verbdef*\stardemo{with visible spaces}%
which will define a verbatim text \stardemo{} (just like that).

\section{Caveat}

If the command you're defining with \verbdefcmd{} is to be used in a
\cs{section}-type command, or a \cs{caption}, it's going to appear in
the table of contents (or list of whatever else).  In this case you
\emph{must} define the command before the \cs{tableofcontents} command
(or whatever).  I recommend defining such commands in the preamble of
the document\footnote{In fact, the source of the present document is
  so structured, since \verbdefcmd{} is used to define a command for
  \verbdefcmd\dots}.
\end{document}
