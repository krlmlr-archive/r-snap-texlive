% Copyright 2010 Thomas Koenig, Alexander Michel
%
% This file is part of NumericPlots.
%
% NumericPlots is free software: you can redistribute it and/or modify
% it under the terms of the GNU General Public License as published by
% the Free Software Foundation, either version 3 of the License, or
% any later version.
%
% NumericPlots is distributed in the hope that it will be useful,
% but WITHOUT ANY WARRANTY; without even the implied warranty of
% MERCHANTABILITY or FITNESS FOR A PARTICULAR PURPOSE.  See the
% GNU General Public License for more details.
%
% You should have received a copy of the GNU General Public License
% along with NumericPlots.  If not, see <http://www.gnu.org/licenses/>.

\part{Using the package}

\section{Basic Functionality}

The package NumericPlots\\
\verb+\usepackage{NumericPlots}+
\\
is intended to be used to plot numeric data which
may, e.g., be exported from Matlab by export2latex.m. The data must be defined
in the form
\lstinputlisting[lastline=9]{examples/DefineData}
where the first column contains the x, the second column the y-data.

\emph{Please note that the package relies on the \texttt{listplot}-command from
pstricks!}

\subsection{plots}


The easiest plot may be done by

\begin{minipage}[T]{0.45\linewidth}
	\lstinputlisting{examples/basic_EasyPlot}
\end{minipage}
\hspace{0.05\linewidth}
\begin{minipage}[T]{0.45\linewidth}
	\begin{NumericDataPlot}{\textwidth}{6.5cm}
	\setxAxis{xMin=-1, xMax=2, Dx=0.5, xO=0}
	\setyAxis{yMin=-50, yMax=150, Dy=25, yO=0}
	
	\plotxAxis{x-axis label}
	\plotyAxis{y-axis label}
	
	\listplot[style=StdLineStyA]{\IdentI}
\end{NumericDataPlot}
\end{minipage}

if you want to add a legend, you simply call

\begin{minipage}[T]{0.45\linewidth}
	\lstinputlisting{examples/basic_Legend}
\end{minipage}
\hspace{0.05\linewidth}
\begin{minipage}[T]{0.45\linewidth}
	\LegendDefinition{
	\LegLine{style=StdLineStyA} & IdentI
}
\end{minipage}

To plot multiple data in one plot call

\begin{minipage}[T]{0.45\linewidth}
	\lstinputlisting{examples/basic_MultipleData}
\end{minipage}
\hspace{0.05\linewidth}
\begin{minipage}[T]{0.45\linewidth}
	\centering
	\begin{NumericDataPlot}
   {\textwidth}{6.5cm}
	\setxAxis
	   {xMin=1, xMax=2, Dx=0.2}
	\setyAxis
	   {yMin=50, yMax=150, Dy=25}
	
	\plotxAxis{x-axis label}
	\plotyAxis{y-axis label}
	
	\listplot[style=StdLineStyA]
	   {\IdentI}
	\listplot[style=StdLineStyB]
	   {\IdentII}
	
	\putSE{\LegendDefinition{
		\LegLine{style=StdLineStyA} & IdentI \\
		\LegLine{style=StdLineStyB} & IdentII
	}}
\end{NumericDataPlot}
\end{minipage}


\subsection{Label and TickLabels}

The commands \texttt{plotxAxis} and \texttt{plotyAxis} take the options
\texttt{NoLabel}, \texttt{NoTicks}, \texttt{NoTickLabel} as well as
\texttt{LabelOption} and \texttt{TickLabelOption} which may be used to eliminate
or change the look of the labels.

Standard values for \texttt{LabelOption} and \texttt{TickLabelOption} may be set\\
by \verb|\newcommand{\StdLabelOption}{\color{blue}|\\
and \verb|\newcommand{\StdTickLabelOption}{\small}|.

The option \texttt{LabelSep} may be used for \verb|\plotxAxis| and
\verb|\plotyAxis| to set the seperation between the axis and the label. Standard
value is \verb|\baselineskip+1ex| for the x-label and \verb|7ex| for the
y-label.

\begin{minipage}[T]{0.45\linewidth}
	\lstinputlisting{examples/basic_Labels}
\end{minipage}
\hspace{0.05\linewidth}
\begin{minipage}[T]{0.45\linewidth}
	\begin{NumericDataPlot}{\textwidth}{6.5cm}
	\setxAxis{xMin=1, xMax=1.6, Dx=0.1, DDx=0.2}
	\setyAxis{yMin=75, yMax=130, Dy=12.5}
	
	\plotxAxis
	[LabelOption=\LARGE,%
	TickLabelOption=\color{red},%
	LabelSep=40pt]
	{x-axis label}
	\plotyAxis
	[NoLabel, NoTicks, NoTickLabel]
	{y-axis label}
	
	\listplot[style=StdLineStyA]{\IdentI}
	\listplot[style=StdLineStyB]{\IdentII}
	
\end{NumericDataPlot}
\end{minipage}

It is furthermore possible to change the position and the rotation of the labels
and tick labels, see the example in section \ref{sec:Details:Labels}.

It is also possible to put single tick labels, see the example in section
\ref{sec:Details:Labels}.

\subsection{Place ``Objects'' in the plot.}\label{sec:PlaceObjects}

There are basically two different options to place objects in the plot. To
understand the difference one has to keep in mind that the axis have two
different coordinate systems. One is the system defined by xMin, xMax, yMin and
yMax (refered to as ``DataCoordinateSystem''), the other ist the system defined
by xCoordMin, xCoordMax, yCoordMin and yCoordMax (refered to as
``PictureCoordinateSystem''), see section \ref{sec:MultiplePlots}.

It is now possible to place stuff in the graph with the DataCoordinates with the
command NDPput, see the following example.

\begin{minipage}[T]{0.45\linewidth}
	\lstinputlisting{examples/basic_PlaceObjects}
\end{minipage}
\hspace{0.05\linewidth}
\begin{minipage}[T]{0.45\linewidth}
	\input{examples/basic_PlaceObjects}
\end{minipage}


For convenience the commands \verb|\putXX{object}| where
$XX\in\left(N,S,E,W,NW,NE,SW,SE\right)$ are defined to place something in the
North, South,\ldots, SouthEast corner of the plot. Also, the command
\verb|\putExpY{xx}| and \verb|\putExpX{xx}| may be used to place exponents at
the axes.

\begin{minipage}[T]{0.45\linewidth}
	\lstinputlisting[firstline=14,lastline=24]{examples/basic_PlaceObjectsII}
\end{minipage}
\hspace{0.05\linewidth}
\begin{minipage}[T]{0.45\linewidth}
		\begin{NumericDataPlot}
	   {\textwidth}{6.5cm}
		\setxAxis
		   {xMin=1, xMax=1.6, Dx=0.2}
		\setyAxis
		   {yMin=50, yMax=150, Dy=25}
		
		\plotxAxis{x-axis label}
		\plotyAxis[NoLabel]{}
		
		\listplot[style=StdLineStyA]
		   {\IdentI}
		
		\putExpY{$\times 10^{-3}$}
		\putExpX{$\times 10^{-6}$}
		
		\putN{N}
		\putS{S}
		\putW{W}
		\putE{E}
		\putNW{NW}
		\putNE{NE}
		\putSW{SW}
		\putSE{SE}
	\end{NumericDataPlot}
\end{minipage}

Alternatively, stuff can be placed
within the plot with \verb|\rput|.


\begin{minipage}[T]{0.5\linewidth}
	\lstinputlisting[firstline=14,lastline=28]{examples/basic_UseRput}
\end{minipage}
\begin{minipage}[T]{0.5\linewidth}
	\begin{NumericDataPlot}
   {\textwidth}{6.5cm}
	\setxAxis
	   {xMin=1, xMax=1.6, Dx=0.2}
	\setyAxis
	   {yMin=50, yMax=150, Dy=25}
	
	\plotxAxis{x-axis label}
	\plotyAxis[NoLabel]{}
	
	\listplot[style=StdLineStyA]
	   {\IdentI}
	
	% put text in the middle of the plot
	\rput{45}(500,500){text}
	% put a formula in the lower left corner
	\rput[bl](0,0){$a^2+b^2=c^2$}
	
	% or put nodes...
	\NDPput[x=1.2, y=125]{\pnode{A}}
	\rput(750,900){\Rnode{B}{peak}}
	% ...and draw a line between them
	\ncline{<-}{A}{B}
	
	% or put the legend at a specific position
	\rput{-45}(750,250){\LegendDefinition{
		\LegLine{style=StdLineStyA} & IdentI
	}}
\end{NumericDataPlot}
\end{minipage}


% =================================
% |								  |
% |     Linestyles and colors     |
% |								  |
% =================================

\subsection{Linestyles and colors}

While using the package, there are predefined linestyles which may be used:

\begin{minipage}{0.55\linewidth}
	\centering
	\small
	
	\begin{NumericDataPlot}[lly=1cm,llx=1cm,urx=0.25cm,ury=0.25cm]{\linewidth}{0.85\linewidth}
		\setxAxis{xMin=2, xMax=17, xO=5, Dx=4}
		\setyAxis{yMin=20, yMax=70, yO=20, Dy=20}
			
        \plotxAxis[NoLabel, AxisStyle=Boxed]{}
        \plotyAxis[NoLabel, AxisStyle=Boxed]{}

		\listplot[style=StdLineStyA]{\DataA}
		\listplot[style=StdLineStyB]{\DataB}
		\listplot[style=StdLineStyC]{\DataC}
		\listplot[style=StdLineStyD]{\DataD}
		\listplot[style=StdLineStyE]{\DataE}
		\listplot[style=StdLineStyF]{\DataF}
		\listplot[style=StdLineStyG]{\DataG}
	\end{NumericDataPlot}
\end{minipage}
\begin{minipage}{0.4\linewidth}
	\centering
	\LegendDefinition{
		\LegLine{style=StdLineStyA} & StdLineStyA \\
		\LegLine{style=StdLineStyB} & StdLineStyB \\
		\LegLine{style=StdLineStyC} & StdLineStyC \\
		\LegLine{style=StdLineStyD} & StdLineStyD \\
		\LegLine{style=StdLineStyE} & StdLineStyE \\
		\LegLine{style=StdLineStyF} & StdLineStyF \\
		\LegLine{style=StdLineStyG} & StdLineStyG \\
	}
\end{minipage}

When using the package option \texttt{BW} the standard line styles will be
replaced by their black and white counterparts:

\begin{minipage}{0.6\linewidth}
	\centering
	\small
	
	\begin{NumericDataPlot}[lly=1cm,llx=1cm,urx=0.25cm,ury=0.25cm]{\linewidth}{0.85\linewidth}
		\setxAxis{xMin=2, xMax=17, xO=5, Dx=4}
		\setyAxis{yMin=20, yMax=70, yO=20, Dy=20}
			
        \plotxAxis[NoLabel, AxisStyle=Boxed]{}
        \plotyAxis[NoLabel, AxisStyle=Boxed]{}

		\listplot[style=BWStdLineStyA]{\DataA}
		\listplot[style=BWStdLineStyB]{\DataB}
		\listplot[style=BWStdLineStyC]{\DataC}
		\listplot[style=BWStdLineStyD]{\DataD}
		\listplot[style=BWStdLineStyE]{\DataE}
		\listplot[style=BWStdLineStyF]{\DataF}
		\listplot[style=BWStdLineStyG]{\DataG}
	\end{NumericDataPlot}
\end{minipage}
\begin{minipage}{0.4\linewidth}
	\centering
	\LegendDefinition{
		\LegLine{style=BWStdLineStyA} & BWStdLineStyA \\
		\LegLine{style=BWStdLineStyB} & BWStdLineStyB \\
		\LegLine{style=BWStdLineStyC} & BWStdLineStyC \\
		\LegLine{style=BWStdLineStyD} & BWStdLineStyD \\
		\LegLine{style=BWStdLineStyE} & BWStdLineStyE \\
		\LegLine{style=BWStdLineStyF} & BWStdLineStyF \\
		\LegLine{style=BWStdLineStyG} & BWStdLineStyG \\
	}
\end{minipage}

For values which are nearly the same (reference and measurement, e.g.) the
following line styles may be used:

\begin{minipage}{0.58\linewidth}
	\centering
	\small
	
	\begin{NumericDataPlot}[lly=1cm,llx=1cm,urx=0.25cm,ury=0.25cm]{\linewidth}{0.85\linewidth}
		\setxAxis{xMin=2, xMax=17, xO=5, Dx=4}
		\setyAxis{yMin=20, yMax=70, yO=20, Dy=20}
			
        \plotxAxis[NoLabel, AxisStyle=Boxed]{}
        \plotyAxis[NoLabel, AxisStyle=Boxed]{}

		\listplot[style=StdLineStyX]{\DataC}
		\listplot[style=StdLineStyY]{\DataC}
		
		\listplot[style=BWStdLineStyX]{\DataD}
		\listplot[style=BWStdLineStyY]{\DataD}
	\end{NumericDataPlot}
\end{minipage}
\begin{minipage}{0.38\linewidth}
	\centering
	\LegendDefinition{
		\LegLine{style=StdLineStyX} & StdLineStyX \\
		\LegLine{style=StdLineStyY} & StdLineStyY \\
		\LegLine{style=BWStdLineStyX} & BWStdLineStyX \\
		\LegLine{style=BWStdLineStyY} & BWStdLineStyY \\
	}
\end{minipage}

It is, of course, possible to redefine the available linestyles or to define new
linestyles.

\begin{minipage}[T]{0.48\linewidth}
	\lstinputlisting[firstline=4,lastline=26]{examples/basic_UserLinestyles}
\end{minipage}
\begin{minipage}[T]{0.48\linewidth}
	\begin{NumericDataPlot}
   {\textwidth}{6.5cm}

   \definecolor{MyColor}{cmyk}{0.6 0.21 1.0 0.2}
   \newpsstyle{MyLine}{linecolor=MyColor, linewidth=2pt, linestyle=dashed,
   dash=1pt 1pt 4pt 1pt 1pt 3pt, dotstyle=*, showpoints=true, dotsize=5pt}
   \newpsstyle{MyLineA}{linecolor=blue, linestyle=dotted, dotstyle=asterisk,
   showpoints=true}

	\setxAxis
	   {xMin=1, xMax=1.6, Dx=0.2}
	\setyAxis
	   {yMin=50, yMax=150, Dy=25}
	
	\plotxAxis[NoLabel]{}
	\plotyAxis[NoLabel]{}
	
	\listplot[style=MyLine]
	   {\IdentI}
	\listplot[style=MyLineA]
	   {\IdentII}
	
	\putSE{\LegendDefinition{
		\LegLine{style=MyLine} & IdentI\\
		\LegLine{style=MyLineA} & IdentII
	}}
\end{NumericDataPlot}
\end{minipage}



% =================================
% |								  |
% |            Legend             |
% |								  |
% =================================

\subsection{Legend}

The legend may be created with \verb|\LegendDefinition|. The command takes the
two optional arguments \texttt{nrCols} and \texttt{LabelOrientation=[l|c|r]}.
The mandatory argument is the definition of a table as demonstrated in the
follwing examples.

\begin{minipage}[T]{0.5\linewidth}
	\lstinputlisting{examples/basic_LegendI}
\end{minipage}
\hspace{1ex}
\begin{minipage}[T]{0.4\linewidth}
	\LegendDefinition{
	\LegLine{style=StdLineStyA} & IdentI\\
	\LegLine{style=StdLineStyB, linewidth=3pt} & second legend
}
\end{minipage}

\begin{minipage}[T]{0.5\linewidth}
	\lstinputlisting{examples/basic_LegendII}
\end{minipage}
\hspace{1ex}
\begin{minipage}[T]{0.4\linewidth}
		\newpsstyle{LegendBoxStyle}%
		{framearc=0.2, fillstyle=solid, fillcolor=yellow, opacity=0.2}
	\LegendDefinition[nrCols=2]{
		\LegLine{style=StdLineStyA} & IdentI &
		\LegLine{style=StdLineStyB, linewidth=3pt} & legend 2
	}
	\newpsstyle{LegendBoxStyle}%
		{fillstyle=solid, fillcolor=white}
\end{minipage}

\begin{minipage}[T]{0.5\linewidth}
	\lstinputlisting{examples/basic_LegendIII}
\end{minipage}
\hspace{1ex}
\begin{minipage}[T]{0.4\linewidth}
	\LegendDefinition[LabelOrientation=c]{
	\LegLine{style=StdLineStyA} & IdentI\\
	\LegLine{style=StdLineStyB, linewidth=3pt} & second legend\\
	\LegLine{style=StdLineStyC} & whatever this data is\ldots\\
	\LegLine{style=StdLineStyD} & and more data\\
	\LegDot{style=StdLineStyC, plotstyle=dots} & plotstyle=dots
}
\end{minipage}


\subsection{Add Lines to the Plot}

Horizontal and vertical lines may be added to the plot with the commands
\verb|\NDPhline{coord}|, \verb|\NDPvline{coord}| and \verb|\NDPline{coord}|. It is also possible to put
nodes and draw lines between them, see section \ref{sec:PlaceObjects}.

\begin{minipage}[T]{0.5\linewidth}
	\lstinputlisting[firstline=14,lastline=17]{examples/basic_Lines}
\end{minipage}
\hspace{1ex}
\begin{minipage}[T]{0.4\linewidth}
	\begin{NumericDataPlot}
   {\textwidth}{6cm}
	\setxAxis
	   {xMin=1, xMax=1.6, Dx=0.2}
	\setyAxis
	   {yMin=50, yMax=150, Dy=25}
	
	\plotxAxis[NoLabel]{}
	\plotyAxis[NoLabel]{}
	
	\listplot[style=StdLineStyA]
	   {\IdentI}
	
	\NDPhline[linecolor=LineColorD]{73}
	\NDPvline[linecolor=LineColorE, %
		linestyle=dashed]{1.5}
	\NDPline[linecolor=red]{1.1}{75}{1.3}{125}
\end{NumericDataPlot}
\end{minipage}


\subsection{Add Boxes to the Plot}

Horizontal and vertical boxes may be added to the plot with the commands
\verb|\NDPhbox{coord}|, \verb|\NDPvbox{coord}| and \verb|\NDPbox{coord}|. It is also possible to put
nodes and draw lines between them, see section \ref{sec:PlaceObjects}.

\begin{minipage}[T]{0.5\linewidth}
	\lstinputlisting[firstline=11,lastline=21]{examples/basic_Boxes}
\end{minipage}
\hspace{1ex}
\begin{minipage}[T]{0.4\linewidth}
	\begin{NumericDataPlot}
   {\textwidth}{6cm}
	\setxAxis
	   {xMin=1, xMax=1.6, Dx=0.2}
	\setyAxis
	   {yMin=50, yMax=150, Dy=25}
	
	\plotxAxis[NoLabel]{}
	\plotyAxis[NoLabel]{}
			
	\NDPhbox[fillstyle=solid,fillcolor = green]%
            {75.0}{100.0}%

	\listplot[style=StdLineStyA] {\IdentI}

	\NDPvbox[fillstyle=solid,fillcolor=red]
            {1.4}{1.5}

    \NDPbox[fillstyle=solid,%
        fillcolor=orange, opacity=0.2, linestyle=none]%
		{1.1}{60}{1.15}{130}
\end{NumericDataPlot}
\end{minipage}


\subsection{Grid}

As shwon in the example below, it is possible to plot a fine grid but not have a
tick label at each of the grid lines. While the options \texttt{Dx} and
\texttt{Dy} define the distance between the grid lines, the options \texttt{DDx}
and \texttt{DDy} define the distance between the tick labels. If \texttt{DDx} or
\texttt{DDy} are not set, they take the values of \texttt{Dx} and \texttt{Dy}. 

One may choose not to plot the grid with
the option \texttt{NoGrid} for the commands \verb|\plotxAxis| and \verb|\plotyAxis|.

If the grid is plottet with the axis it may happen that the grid is plottet over
the axis. To avoid this, plot the grid first and then plot the axis as shown.

\begin{minipage}[T]{0.5\linewidth}
	\lstinputlisting{examples/basic_Grid}
\end{minipage}
\hspace{1ex}
\begin{minipage}[T]{0.4\linewidth}
	\begin{NumericDataPlot}
   {\textwidth}{6.5cm}
	\setxAxis{xMin=1, xMax=1.6, Dx=0.1, DDx=0.2}
	\setyAxis{yMin=50, yMax=150, Dy=12.5, DDy=25}
	
	\plotxGrid
	\plotyGrid
	\plotxAxis[NoLabel, NoGrid, AxisStyle=Boxed]{}
	\plotyAxis[NoLabel, NoGrid, AxisStyle=Boxed]{}
	
	\listplot[style=StdLineStyA]{\IdentI}
	
\end{NumericDataPlot}
\end{minipage}



\subsection{Logarithmic axes}


\begin{minipage}[T]{0.5\linewidth}
	\lstinputlisting{examples/basic_LogarithmicI}
\end{minipage}
\hspace{1ex}
\begin{minipage}[T]{0.4\linewidth}
	\begin{NumericDataPlot}{\textwidth}{6.5cm}
	\setxAxis{xMin=6, xMax=2500, xLog}
	\setyAxis{yMin=0, yMax=2500, Dy=500}
	\plotxAxis{}
	\plotyAxis{}
	
	\listplot{\LogData}
\end{NumericDataPlot}
\end{minipage}


\begin{minipage}[T]{0.5\linewidth}
	\lstinputlisting{examples/basic_LogarithmicII}
\end{minipage}
\hspace{1ex}
\begin{minipage}[T]{0.4\linewidth}
	\begin{NumericDataPlot}{\textwidth}{6.5cm}
	\setxAxis{xMin=1, xMax=1500, xLog}
	\setyAxis{yMin=1, yMax=1500, yLog}
	\plotxAxis{}
	\plotyAxis{}
	
	\listplot{\LogData}
\end{NumericDataPlot}
\end{minipage}


\begin{minipage}[T]{0.5\linewidth}
	\lstinputlisting{examples/basic_LogarithmicIII}
\end{minipage}
\hspace{1ex}
\begin{minipage}[T]{0.4\linewidth}
	\begin{NumericDataPlot}{\textwidth}{6.5cm}
	\setxAxis{xMin=1, xMax=1500, xLog, xLogNoSubGrid}
	\setyAxis{yMin=1, yMax=1500, yLog, yLogNoSubGrid}
	\plotxAxis{}
	\plotyAxis{}
	
	\listplot{\LogData}
\end{NumericDataPlot}
\end{minipage}


\subsection{plots with holes}

If plotting data, one might not want to plot some part of this data, e.g. if
only the data in a certain y-range is interesting. If one chooses to eliminate the
invalid data either by using \verb+yStart+ or \verb+yEnd+ or by eliminating the
data before the export, pstricks will always connect the last valid point of an
interval with the first valid point of the next interval. In Matlab e.g.,
invalid data (NaN) is just ignored. This behavior is also possible with NumericPlots and
export2latex. If export2latex is called with data which contains NaN and with
the option.NaNsplit=true, it will split the data in several intervals and append
a consecutive number to the identifier for each interval. The exported yMax,
yMin values etc. will still be valid for the whole data-vector. The exported
data might look like:

\begin{verbatim}
\expandafter\def\csname DataIdentNrRanges\endcsname{3}
\expandafter\def\csname DataIdent1\endcsname{
560.00 5.40
...
574.72 6.25
}

\expandafter\def\csname DataIdent2\endcsname{
588.73 4.78
...
609.53 12.09
}


\expandafter\def\csname DataIdent3\endcsname{
620.56 27.81
...
649.56 27.32
}
\end{verbatim}


\expandafter\def\csname DataIdentNrRanges\endcsname{3}
\expandafter\def\csname DataIdent1\endcsname{
560.00 5.40
560.30 6.25
560.59 4.74
560.90 6.25
561.60 5.18
561.65 6.25
561.70 4.73
562.60 4.75
562.63 6.06
563.75 6.12
563.75 5.00
564.50 6.25
564.76 4.71
564.90 4.88
565.51 6.25
565.63 5.00
566.13 5.69
566.61 6.78
567.10 4.88
567.60 4.89
567.66 6.25
568.30 4.88
568.31 6.25
569.10 4.77
569.51 6.25
569.60 6.13
570.35 4.85
570.63 6.25
570.71 5.00
571.30 4.64
571.31 6.25
572.30 4.66
572.62 6.06
572.90 5.00
572.93 5.61
573.68 5.00
574.11 6.25
574.66 4.76
574.72 6.25
}

\expandafter\def\csname DataIdent2\endcsname{
588.73 4.78
588.90 6.15
589.50 4.76
589.67 4.90
589.74 6.25
590.66 5.96
590.73 4.55
591.25 6.25
591.69 4.68
592.10 4.91
592.31 6.25
593.20 6.25
593.55 5.36
593.64 4.79
593.74 6.25
594.64 4.74
594.75 6.11
595.25 4.67
595.31 6.12
596.00 5.84
596.30 11.90
597.10 10.69
597.20 12.00
597.69 12.00
598.15 10.00
598.50 11.97
598.67 10.00
599.31 10.00
599.69 11.91
600.13 12.22
600.79 10.79
600.83 10.31
600.86 11.72
601.74 10.56
601.80 11.25
602.78 11.98
602.94 10.00
603.72 12.08
603.94 10.00
604.54 10.00
604.70 11.96
605.14 11.25
605.24 10.83
605.69 11.92
605.80 10.00
606.70 11.72
607.03 10.00
607.49 11.15
607.54 10.00
608.23 12.15
608.72 10.00
609.30 10.87
609.53 12.09
}

\expandafter\def\csname DataIdent3\endcsname{
620.56 27.81
620.78 26.25
620.85 27.80
621.56 27.32
621.74 27.78
622.00 27.32
622.74 26.36
623.08 28.75
623.31 27.79
623.71 26.31
624.05 27.32
624.65 27.79
625.18 27.83
625.45 27.35
625.95 27.74
626.28 27.30
626.75 26.34
626.88 27.81
627.54 28.75
627.72 26.38
628.23 27.29
628.36 27.74
628.88 26.25
629.28 27.82
629.78 28.84
629.93 27.32
631.13 26.27
631.22 27.76
631.94 28.75
631.98 27.33
632.54 26.25
632.66 27.78
632.89 27.29
633.33 28.88
633.76 26.45
633.94 27.85
634.76 26.33
635.14 27.74
635.68 27.82
635.84 26.35
636.08 27.84
636.46 27.34
637.04 27.74
637.19 27.32
637.70 27.34
638.31 27.84
638.50 27.78
638.65 27.33
639.45 27.79
639.83 26.34
640.03 27.38
640.68 27.83
641.45 27.31
641.50 27.77
642.08 27.80
642.18 27.32
642.76 27.93
642.86 26.25
643.75 27.32
643.91 27.76
644.38 27.30
644.74 27.83
645.06 27.37
645.08 27.80
645.80 28.80
646.13 26.47
646.76 26.57
647.09 27.81
647.26 27.83
647.98 26.25
648.20 27.33
648.34 28.75
648.93 28.91
649.56 27.32
}

This data can then easily be plotted using the command\\
\verb+\multilistplot{<options>}{<Identifier>}+\footnote{Please notice, that the
first argument of multilistplot is a mandatory argument. Also notice, that the
second argument ist only the name of the data including ``Data'' but not the
command as it would be for the listplot command}.
The command will call\\
\verb+\listplot[<options>]{\<Identifier>}+ for each of
the intervals. The first argument is directly passed to the listplot command which is part of the
pstricks package. The second argument is the identifier of the data that should
be plotted.

\begin{minipage}[T]{0.5\linewidth}
	\lstinputlisting{examples/basic_PlotWHoles}
\end{minipage}
\hspace{1ex}
\begin{minipage}[T]{0.4\linewidth}
	\begin{NumericDataPlot}{\textwidth}{6.5cm}
	\setxAxis{xMin=560, xMax=650, Dx=20}
	\setyAxis{yMin=4, yMax=30, Dy=5, yO=5}
	\plotxAxis{}
	\plotyAxis{}
	
	\multilistplot{style=StdLineStyA}{DataIdent}
\end{NumericDataPlot}
\end{minipage}

