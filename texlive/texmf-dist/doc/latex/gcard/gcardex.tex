% greeting card document using gcard package, showing basic techniques of size 
% selection and horizontal and vertical placement of text

% For landscape orientation, include ``landscape'' as a global
% option, and call the geometry package explicitly to specify your
% output driver.  This example assumes dvips.  (The explicit \usepackage{geometry}
% is unnecessary with pdftex, and in fact pdftex will even ignore other driver
% specifications.)

% G. C. McBane  19 May 2007

\documentclass[12 pt,landscape]{article}
\usepackage{palatino}
\usepackage[dvips]{geometry}
\usepackage{gcard}

% user should set next four lengths to taste; defined for a single panel
% after folding.  All four panels use the same margins.
% Here, edge and gutter are set the same, as are top and bottom; that's
% not required, and it's okay to use 4 different values.
\setlength{\gcguttermargin}{8 mm} % inside edge of textblock  
\setlength{\gcedgemargin}{\gcguttermargin}  % outside edge
\setlength{\gctopmargin}{6 mm}        % top
\setlength{\gcbottommargin}{\gctopmargin}  % bottom


\begin{document}

\begin{frontcover}
\centering 
\huge
\textit{We hope you're\\
having a\ldots}
\end{frontcover}


% back cover; pushed down to 0.5 cm above bottom margin
\begin{backcover}
\centering
\vfill
McB\"{u}chau Family Productions
\vspace{.5 cm}
\end{backcover}
                                
% inside left
\begin{insideleft}
\centering
\small
THIS SPACE INTENTIONALLY LEFT BLANK
\end{insideleft}

% inside right
\begin{insideright}
 \begin{center} % same effect as \centering used for front cover
\Huge
\textit{Happy\\
Mother's Day!}
 \end{center}
\end{insideright}

\end{document}
