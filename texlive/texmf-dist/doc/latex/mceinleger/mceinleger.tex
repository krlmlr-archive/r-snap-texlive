%% Vorlage f�r Musikkassetten-Einleger
%% Autor: Florian Benischke, Wien 2002
%% Copyright (C) 2002 Florian Benischke. All
%% rights reserved. Permission is granted to to customize the
%% declarations in this file to serve the needs of your installation.
%% However, no permission is granted to distribute a modified version of
%% this file under its original name.


\documentclass[12pt,a4paper]{article}
\usepackage[latin1]{inputenc}
\usepackage{ngerman}


\begin{document}

\title{Musikkassetten--Einleger}
\author{Florian Benischke}
\maketitle

\tableofcontents \setlength{\unitlength}{0.5cm}

\section{Hintergrund}

Mit Hilfe dieses Pakets lassen sich H�llen und Einleger f�r
Musikkassetten gestalten, wobei die R�ck- und Vorderseite sowie
die Seitenseite beschriftet werden k�nnen. Unter \glqq H�lle''
wird in diesem Zusammenhang eine Einlage f�r Musikkassetten
verstanden, die aus einer Vorder-, Schmal- und R�ckseite besteht,
ein \glqq Einleger'' umfasst zus�tzlich noch eine einklappbare
Seite (die an die Vorderseite anschlie�t), die zus�tzliche
Informationen wie z.B. ein Inhaltsverzeichnis enthalten kann.
\\*[2ex]

\huge{\textbf{WICHTIG:}} \normalsize{\textbf{Dieses Paket braucht
zus�tzlich das Paket \emph{rotating}, damit es funktionieren kann
!!}}

\clearpage

\section{Anwendung}

2 M�glichkeiten stehen zur Auswahl: Der Befehl \texttt{huelle}
erzeugt folgende Anordnung:


\begin{picture}(15.0,15.0)
  \put(1.0,1.0){\framebox(3.5,10.2)}
  \put(4.5,1.0){\framebox(1.3,10.2)}
  \put(5.8,1.0){\framebox(6.4,10.2)}
\end{picture}

\noindent{Der Befehl \texttt{einleger} erlaubt zus�tzlich noch die
Gestaltung einer weiteren Seite, die durch Einklappen einen
MC-Einleger erzeugt:}

\begin{picture}(15.0,15.0)
  \put(1.0,1.0){\framebox(3.5,10.2)}
  \put(4.5,1.0){\framebox(1.3,10.2)}
  \put(5.8,1.0){\framebox(6.4,10.2)}
  \put(12.2,1.0){\framebox(6.4,10.2)}
  \end{picture}

\clearpage

\subsection{MC-H�llen}

\noindent{\texttt{$\backslash$usepackage $\{${mceinleger}$\}$}}\\
...\\
\texttt{$\backslash$huelle$\{$\emph{A}$\}$$\{$\emph{B}$\}$$\{$\emph{C}$\}$$\{$\emph{D}$\}$}\\

wobei gilt:\\
A \ldots Text der R�ckseite\\
B \ldots Text der seitlich sichtbaren Seite (= Schmalseite)\\
C \ldots Text der Vorderseite\\
D \ldots Anordnung des Textes auf der Vorderseite; folgende
M�glichkeiten stehen zur Verf�gung:

\begin{description}

\item[t] top
\item[b] bottom
\item[c] centered
\item[s] stretched
\item[l] left
\item[r] right

\end{description}

\noindent{Dabei sind auch Kombinationen wie z.B. \texttt{tl} f�r
top-left zul�ssig.}

\subsection{MC-Einleger}

\noindent{\texttt{$\backslash$usepackage $\{${mceinleger}$\}$}}\\
...\\
\texttt{$\backslash$einleger$\{$\emph{A}$\}$$\{$\emph{B}$\}$$\{$\emph{C}$\}$$\{$\emph{D}$\}$$\{$\emph{E}$\}$$\{$\emph{F}$\}$}\\

wobei gilt:\\
A \ldots Text der R�ckseite\\
B \ldots Text der seitlich sichtbaren Seite (= Schmalseite)\\
C \ldots Text der Vorderseite\\
D \ldots Anordnung des Textes auf der Vorderseite (M�glichkeiten
hiezu s. MC-H�llen)\\
E \ldots Text auf der einklappbaren Seite\\
F \ldots Anordnung des Textes auf der einklappbaren Seite

\section{English version}

\textbf{IMPORTANT: In order to work properly, this package needs
the package \emph{rotating} !!}\\*[2ex]

This package offers you 2 different options: the command
\texttt{huelle} produces the following:


\begin{picture}(15.0,15.0)
  \put(1.0,1.0){\framebox(2.5,10.2)}
  \put(3.5,1.0){\framebox(1.3,10.2)}
  \put(4.8,1.0){\framebox(6.4,10.2)}
\end{picture}

\noindent{The command \texttt{einleger} allows you to create an
additional page:}

\begin{picture}(15.0,15.0)
  \put(1.0,1.0){\framebox(2.5,10.2)}
  \put(3.5,1.0){\framebox(1.3,10.2)}
  \put(4.8,1.0){\framebox(6.4,10.2)}
  \put(11.2,1.0){\framebox(6.4,10.2)}
  \end{picture}

\clearpage

\subsection{Command \texttt{huelle}}

\noindent{\texttt{$\backslash$usepackage $\{${mceinleger}$\}$}}\\
...\\
\texttt{$\backslash$huelle$\{$\emph{A}$\}$$\{$\emph{B}$\}$$\{$\emph{C}$\}$$\{$\emph{D}$\}$$\{$\emph{E}$\}$$\{$\emph{F}$\}$}\\

where:\\
A \ldots text on the backcover\\
B \ldots text seen from the narrow side\\
C \ldots text on the frontcover\\
D \ldots arrangement of the frontcover-text with the following
possibilities:

\begin{description}

\item[t] top
\item[b] bottom
\item[c] centered
\item[s] stretched
\item[l] left
\item[r] right

\end{description}

\noindent{Combinations such as \texttt{tl} are allowed.}

\subsection{Command \texttt{einleger}}

\noindent{\texttt{$\backslash$usepackage $\{${mceinleger}$\}$}}\\
...\\
\texttt{$\backslash$huelle$\{$\emph{A}$\}$$\{$\emph{B}$\}$$\{$\emph{C}$\}$$\{$\emph{D}$\}$}\\

where:\\
A \ldots text on the backcover\\
B \ldots text seen by the narrow side\\
C \ldots text on the frontcover\\
D \ldots arrangement of the frontcover-text\\
E \ldots text on the additional page\\
F \ldots arrangement of text on additional page


\end{document}
