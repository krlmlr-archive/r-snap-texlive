%%
%% This is file `example.tex',
%% generated with the docstrip utility.
%%
%% The original source files were:
%%
%% logpap.dtx  (with options: `example')
%% 
%% Copyright (c) 2004 Mike Kaufmann, all rights reserved
%% 
%% This program is provided under the terms of the
%% LaTeX Project Public License distributed from CTAN
%% archives in directory macros/latex/base/lppl.txt.
%% 
%% Author: Mike Kaufmann
%%         Mike.Kaufmann@ei.fh-giessen.de
%% 
%% \CharacterTable
%%  {Upper-case    \A\B\C\D\E\F\G\H\I\J\K\L\M\N\O\P\Q\R\S\T\U\V\W\X\Y\Z
%%   Lower-case    \a\b\c\d\e\f\g\h\i\j\k\l\m\n\o\p\q\r\s\t\u\v\w\x\y\z
%%   Digits        \0\1\2\3\4\5\6\7\8\9
%%   Exclamation   \!     Double quote  \"     Hash (number) \#
%%   Dollar        \$     Percent       \%     Ampersand     \&
%%   Acute accent  \'     Left paren    \(     Right paren   \)
%%   Asterisk      \*     Plus          \+     Comma         \,
%%   Minus         \-     Point         \.     Solidus       \/
%%   Colon         \:     Semicolon     \;     Less than     \<
%%   Equals        \=     Greater than  \>     Question mark \?
%%   Commercial at \@     Left bracket  \[     Backslash     \\
%%   Right bracket \]     Circumflex    \^     Underscore    \_
%%   Grave accent  \`     Left brace    \{     Vertical bar  \|
%%   Right brace   \}     Tilde         \~}
%%
\documentclass[a4paper]{article}
\usepackage[latin1]{inputenc}
\usepackage{logpap}

\voffset-2cm
\hoffset-0.5cm
\textheight25cm
\textwidth14cm

\parindent0pt
\parskip1ex plus.3ex minus.2ex
\pagestyle{empty}
\unitlength1mm

\newcommand*{\lp}{\textsf{logpap}}
\newcommand*{\bs}{\char '134 }
\newcommand*{\lb}{\char '173 }
\newcommand*{\rb}{\char '175 }
\newcommand*{\param}[1]{\texttt{\textit{#1}}}

\begin{document}
\centerline{\textbf{\LARGE Some Examples for the \lp\ package.\footnote{The
source of this example file is part of \texttt{logpap.dtx}.}}}

\vspace{4mm}
The \lp\ package provides four macros for drawing logarithmic-logarithmic,
logarithmic-linear, linear-logarithmic and (because it was easy to implement)
linear-linear graph paper with \LaTeX.

\vspace{2mm}
\begin{picture}(140,50)
\LPSet{nofsnx,notwoniney}
\put( 0, 5){\linlogpap(60mm,12.5mm)(6,1){0,10}{1}[X][Y]}
\put(75, 5){\linlinpap(60mm,12.5mm)(6,1){0,10}{0,10}[X][Y]}
\put( 0,32.5){\loglogpap(60mm,12.5mm)(2,1){100m}{1}[X][Y]}
\put(75,32.5){\loglinpap(60mm,12.5mm)(2,1){100m}{0,10}[X][Y]}
\end{picture}

\vspace{2mm}
Altough the \lp\ package only draws logarithmic graph paper, there are
various things that can be changed. Figure \ref{fig:ov} will give an
overwiev. Here the first graph paper part where made with the default
settings and the command
\verb|\put(0,91){\loglinpap(80mm,10mm)(1,1){1}| \verb|{0}[X-name][Y-name]}|.
On the right side you can find the commands used to change the appearance.

Not all commands possible are shown here. Every option can be used with
\verb|\LPSet{|\param{option1}\verb|,| \param{option2}\verb|,...}| within the
document, every \param{option} has a counterpart \verb|no|\param{option} and
for every \param{option}\verb|x| also an \param{option}\verb|y| exists. Also
commands to set the three line widths, colors and the appearance of the $\mu$
are not shown.

\begin{figure}[ht]
\centering
\begin{picture}(140,106)
\put( 0,91){\loglinpap(80mm,10mm)(1,1){1}{0}[X-name][Y-name]}
\put(85,96){\makebox(0,0)[cl]{with default settings}}

\LPSet{nodimensions,noticksupright,tenlower}
\put( 0,69){\loglinpap(80mm,10mm)(1,1){1}{0}[optional][optional]}
\put(85,78){\makebox(0,0)[cl]{\texttt{\bs LPSet\lb nodimensions,}}}
\put(85,74){\makebox(0,0)[cl]{\texttt{\ \ \ \ \ \ \ noticksupright,}}}
\put(85,70){\makebox(0,0)[cl]{\texttt{\ \ \ \ \ \ \ tenlower\rb}}}
\LPSet{dimensions,ticksupright,notenlower}

\LPSet{notext,nofsnx}\DefineLPMinLineDist{1.9999mm}
\put( 0,47){\loglinpap(80mm,10mm)(1,1){1}{0}}
\put(85,56){\makebox(0,0)[cl]{\texttt{\bs LPSet\lb notext,}}}
\put(85,52){\makebox(0,0)[cl]{\texttt{\ \ \ \ \ \ \ nofsnx\rb}}}
\put(85,48){\makebox(0,0)[cl]{\texttt{\bs DefineLPMinLineDist\lb 1.9999mm\rb}}}
\LPSet{text,fsnx}\DefineLPMinLineDist{0.9999mm}

\DefineLPText{new text\quad}\DefineLPLabelFont{\sffamily\tiny}\LPSet{notwoninex}
\put( 0,25){\loglinpap(80mm,10mm)(1,1){1}{0}}
\put(85,36){\makebox(0,0)[cl]{\texttt{\bs DefineLPText\lb new text\bs quad\rb}}}
\put(85,32){\makebox(0,0)[cl]{\texttt{\bs DefineLPLabelFont}}}
\put(85,28){\makebox(0,0)[cl]{\texttt{\ \ \ \ \lb\bs sffamily\bs tiny\rb}}}
\put(85,24){\makebox(0,0)[cl]{\texttt{\bs LPSet\lb notwoninex\rb}}}
\DefineLPText{made with \LaTeX\ and \textsf{logpap}\quad}\DefineLPLabelFont{\scriptsize}\LPSet{twoninex}

\DefineLPThickTickLen{1mm}\DefineLPMedTickLen{0.5mm}\DefineLPLabelDist{0.5mm}
\put( 0, 3){\loglinpap(80mm,10mm)(1,1){1}{0}}
\put(85,12){\makebox(0,0)[cl]{\texttt{\bs DefineLPThickTickLen\lb 1mm\rb}}}
\put(85, 8){\makebox(0,0)[cl]{\texttt{\bs DefineLPMedTickLen\lb 0.5mm\rb}}}
\put(85, 4){\makebox(0,0)[cl]{\texttt{\bs DefineLPLabelDist\lb 0.5mm\rb}}}
\DefineLPThickTickLen{2mm}\DefineLPMedTickLen{1mm}\DefineLPLabelDist{1mm}
\end{picture}
\caption{Overview}\label{fig:ov}
\end{figure}
\end{document}
\endinput
%%
%% End of file `example.tex'.
