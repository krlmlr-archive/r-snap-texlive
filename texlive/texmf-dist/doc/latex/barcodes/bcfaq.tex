%%
%% This is file `bcfaq.tex',
%% generated with the docstrip utility.
%%
%% The original source files were:
%%
%% barcodes.dtx  (with options: `bcfaq')
%% As this is a generated file, you may perhaps not want to edit it.
%% This file belongs to the barcode package.
%% It may be of no great use without the rest of the barcode package.
%% \CharacterTable
%%  {Upper-case    \A\B\C\D\E\F\G\H\I\J\K\L\M\N\O\P\Q\R\S\T\U\V\W\X\Y\Z
%%   Lower-case    \a\b\c\d\e\f\g\h\i\j\k\l\m\n\o\p\q\r\s\t\u\v\w\x\y\z
%%   Digits        \0\1\2\3\4\5\6\7\8\9
%%   Exclamation   \!     Double quote  \"     Hash (number) \#
%%   Dollar        \$     Percent       \%     Ampersand     \&
%%   Acute accent  \'     Left paren    \(     Right paren   \)
%%   Asterisk      \*     Plus          \+     Comma         \,
%%   Minus         \-     Point         \.     Solidus       \/
%%   Colon         \:     Semicolon     \;     Less than     \<
%%   Equals        \=     Greater than  \>     Question mark \?
%%   Commercial at \@     Left bracket  \[     Backslash     \\
%%   Right bracket \]     Circumflex    \^     Underscore    \_
%%   Grave accent  \`     Left brace    \{     Vertical bar  \|
%%   Right brace   \}     Tilde         \~}
%%
\documentclass[a4paper]{article}
\bibliographystyle{alpha}


\font\itf=wlitf scaled 1200

\def\bcbox#1{\lower3pt\hbox{\itf +#1-}}

\newbox\bsavebox
\newdimen\bcboxdepth
\bcboxdepth=4pt
\def\bdbox#1{\setbox\bsavebox\hbox{\itf +#1-}
\vbox{\hsize=\wd\bsavebox\offinterlineskip%
\copy\bsavebox%
\vskip-\ht\bsavebox\vskip\bcboxdepth%
\box\bsavebox\vskip-\bcboxdepth}}
\def\bebox#1{\setbox\bsavebox\hbox{\itf +#1-}
\vbox{\hsize=\wd\bsavebox\offinterlineskip%
\copy\bsavebox%
\copy\bsavebox%
\vskip-\ht\bsavebox\vskip\bcboxdepth%
\box\bsavebox\vskip-\bcboxdepth}}

\def\tbs{{\tt\char92}}
\begin{document}
\title{Barcodes-FAQ}
\author{Peter Willadt}
\date{2003-05-24}
\maketitle
\begin{abstract}

This file deals with questions about barcode fonts created for
\TeX{}. Its purpose is not to replace the regular documents, but to add
informations that may be of no interest for the casual reader.

\end{abstract}

\section{Changelog}

\subsection{Changes of the barcode package in general}
\begin{tabular}{ll}
\em Date & \em Change\\
2003-05-24 & new release with some corrections to codean.pl\\
           & Thanks to Jacek Ruzycka (uv\_centcom@yahoo.com)\\
1999-05-29 & added license note to README\\
1999-05-29 & added install.bat\\
1998-11-28 & included wlcr39, wlcf39, and wlc93\\
1998-04-21 & fixed wlitf bug\\
1998-04-10 & switched partially to docstrip archive\\
   & added support for code 11\\
1998-01-24 & added code 128 in MF format, changed codean.pl\\
1997-11-08 & changed metrics of {\sc itf} start/stop chars\\
1997-10-09 & released first version of the barcode package
\end{tabular}

\subsection{Changes within this document}

\begin{tabular}{ll}
\em Date & \em Change\\
1999-05-29 & added short info about install.bat\\
1998-11-28 & added docs about code 93\\
1998-04-10 & added docs about code 11\\
1998-01-24 & added information about code 128\\
1997-12-08 & added contributions to {\sc ean} coding by Kalvis M. Jansons\\
 & changed section on plessey\\
1997-11-09 & added several items, rearranged sections\\
1997-11-08 & started writing first version
\end{tabular}

\section{Introduction}

The barcode package itself can be found on {\tt ctan} in the
subdirectory\\ {\tt fonts/barcodes/willadt}. It contains---among
others---several fonts in {\sc vpl} format, so your {\sc dvi} driver
should be able to handle virtual fonts. If it doesn't, you may perhaps
use {\em dvicopy} to resolve the references to virtual fonts, or you
may perhaps upgrade to a more modern \TeX{} package.

\section{What about docs?}

\subsection{Docs about barcodes}

With the barcode package, there comes documentation about {\sc ean}
fonts. In the README file, there are short examples of the use of
the other fonts, too. Also the files {\tt examples.tex} shows the look
and basic usage of the barcodes.

In {\sl TUGboat}, barcoding has been covered in several
articles \cite{bc:ean,bc:sauter,bc:vulis}. The barcode package itself
is covered in an article that has appeared in the december issue,
1997, of {\sl TUGboat}. As TUGboat is making the elder articles available
online, I recommend looking for it.

\subsection{Some information about code 128}

Code 128 is able to print all 128 ascii chars. There are a little more
than a hundred glyphs that are interpreted in three different
ways. Some of the glyphs are used as shift chars to determine which
interpretation to use. A checksum is mandatory. To use it, you must
install {\tt wlc128.mf}.

The preprocessing---switching among the different character sets,
coding efficiently, calculating the checksum---is handled by the
newer versions of codean.pl. This is perhaps not as pleasing as to do
it with \TeX{} alone, but, alas, it works.

The way to run the preprocessor is described in {\tt eandoc.tex}. To
use code 128, you have to write a line starting with {\tt\tbs cxxviii\{}
followed by the characters you want to code and followed by a closing
brace. Non-printing seven-bit ascii characters may be specified in hex
in C or \TeX{} style, like {\tt\tbs x3f} or \verb|^^3f|. Anything up to
the rightmost closing brace in this line will be coded---there is no
brace matching. But you will get a warning message if your text
includes a right brace. So you may want to code braces in hex form to
avoid this message. Please note that {\tt codean.pl} does almost no
checking for errors, so if you intend to produce garbage, there are
lots of ways. If you need code 128 escape characters, they may be
included as hex characters with codes starting at 0x80.

\begin{tabular}{lllll}
\em Input Code & \em C128 number &\multicolumn{3}{c}{\em meaning}\\
\em(hex) & \em(dec.) & \em Set A & \em Set B & \em Set C\\
0x80  & 96  & FNC3 & FNC3 & 96\\
0x81 & 97 & FNC2 & FNC2 & 97\\
0x82 & 98 & SHIFT & SHIFT & 98\\
0x83 & 99 & CODE C& CODE C& 99\\
0x84 & 100 & CODE B& FNC4 & CODE B\\
0x85 & 101 & FNC4 & CODE A& CODE A\\
0x86 & 102 & FNC1 & FNC1 & FNC1\\
0x87 & 103 & START A& START A& START A\\
0x88 & 104 & START B& START B& START B\\
0x89 & 105 & START C& START C& START C\\
0x8a & 106 & STOP & STOP & STOP
\end{tabular}

But please keep in mind that {\tt codean.pl} does almost anything for you,
you should not have the need to insert start/stop codes and the
like. Only if you use EAN128, you will need {\tt FNC1}. Just to make
it clear: For your barcode to include {\tt FNC1} you have to write
\verb|^^86| before preprocessing.

{\tt codean.pl} will insert a sequence of hex digits; the original
text will be appended to the line as a comment.
Also, your \TeX{} file has to define some macros. They look like
this:

\begin{verbatim}
\font\fntcxx=wlc128 scaled \magstep3
\def\CXXVIII{\bgroup\fntcxx\let\next\hexchar\next}
\def\cxxviii{\message{OOPS, use codean.pl}}
\def\hexchar#1#2{\if#1@
            \global\let\next\egroup
            \else\char"#1#2\fi\next}
\end{verbatim}

If you do not like end recursion, you might use other ones. If
you dislike the hex format produced by {\tt codean.pl}, you are free to
change it---as long as you don't redistribute.

\subsection{A little bit about code 11}

Code 11 is a numeric-only barcode. It is almost as space-efficient as
{\sc itf}, and it comes with a checksum. The checksum should be
swallowed by the reading device. The checksum is a weighted mod-11
checksum. There are apparently two kinds of checksum in use: such with
one checkdigit and such with two checkdigits.

To use code 11, you should include {\tt barcode.sty}. \TeX{} will
calculate the checksum on its own. If you need two check digits, you
should say\\
{\tt\tbs codexichecksumktrue}.\\
How many check digits you
actually need should be made clear in the documentation to your
reading device\footnote{You may also want to check the documentation
to find out if your reading device supports code 11 at all; code 11 is
not that common.}. You may code the digits 0--9, and the minus
sign. The start and stop sign is mapped to the @ character. Here is a
full example. {Please note: If you do not use \LaTeX{}, you may
have to supply a definition for {\tt\tbs makeatletter}
and {\tt\tbs makeatother}}:

\begin{verbatim}
\input barcodes.sty
%% with two check digits
\codexichecksumktrue
\codxi{12-1234}

%% with only one check digit.
\codexichecksumkfalse
\codxi{12-1234}
\bye
\end{verbatim}

\subsection{Which Code 39 should I use, there are different versions?}

Well, you may \verb|\input code39| in your \TeX\ document to get the
macro version. It is absolutely flexible in repect of barcode size and
so on, but it is made of \TeX\ macros and there are probably
situations where you may prefer a {\em real} font. So there is {\tt
wlc39.mf} as a real font, {\tt wlcr39.mf} as a font running sideways,
and{\tt wlcf39.vpl} as a full ascii code 39 in vpl format. The
ltter two fonts are experimental, that means, they are not that much
tested as the others.

I recommend using either the \TeX\ macro or {\tt wlc39.mf}. The others
are too special for normal use, rotation by the graohics driver is
mor robust than building towers of letters, and full ascii code~39 is
less than optimal.

\subsection{About code 93}

Code 93 is a little bit more compact than code 39 (as it uses also the
gap between the characters), but in other aspects it is quite
similiar. A checksum with two digits is mandatory. As with code 11,
checksumming works by weighing the characters different depending on
their position within the barcode.

The version of code 93 that is included within the package resorts to
uppercase ascii, digits, and to some other signs. Start and stop sign
are mapped to {\tt<} and {\tt>}, respectively.

Full ascii works similiar to full ascii with code 39 (that means with
escape characters), but fortunately the escape characters are
different from ordinary characters, so that you can hardly be
misunderstood. Unfortunately the escape characters are named (\$),
(\%), (/), and (+), so I decided to map them to opening and closing
parentheses and brackets, respectively. This may not seem very clever
when it comes to full ascii, but these characters may also appear as
checksum characters, so we need them when doing uppercase ascii,
too. For full ascii, it would be clever to redo the font in a
completely different way and to map start sign, stop sign and the four
escape characters to positions above 128, so that any 7 bit ascii
character can be input directly.

Code 93 is supported by codean.pl in the same way as ean or code 128
are. You have to write {\tt\tbs xciii\{YOUR DATA 123\}}, run the file
through {\tt codean.pl}, then you'll get\\
{\tt\tbs XCIII\{YOUR DATA 123NN\}\%Code93(YOUR DATA 123)}

The NN make up the two check characters. Normally, they are different,
but in this example they are equal, by incident.

\subsection{Docs about VPL Files}

VPL files are documented in the files {\tt VPtoVF.WEB} and {\tt
VFtoVP.WEB}. Both files can be found on {\sc ctan} under {\tt
SYSTEMS/KNUTH}. {\em WEAVE} may be used to produce \TeX{} files. The
documentation about the VPL/VF file formats is covered within the
first few sections.

\section{I want other metrics etc.}

\subsection{My barcodes should have depth}

You have got several possibilities to obtain depth. The most simple
approach may be to set the barcodes with the regular barcode fonts into an
{\tt\tbs hbox} and then to {\tt\tbs lower} this hbox.

Let's look at an example:

\begin{verbatim}
\font\itf=wlitf scaled 1200
\def\bcbox#1{\lower3pt\hbox{\itf +#1-}}
 Example:
{\sc itf} barcode looks like \quad\bcbox{1009}\quad
for 1009, e.g.
\end{verbatim}

This code yields the following result:

{\sc itf} barcode looks like \quad\bcbox{1009}\quad for 1009, e.g.

Another approach is to edit the font sources. In case of {\sc vpl}
files, this is quite painful. Rules in {\sc vpl} files do not have
depth, so you have to change the characters mapping in a way that you
add a {\em movedown} before you draw, and a {\em moveup} afterwards.
You may see the changes for an arbitrary character from {\tt
WLITF.VPL}:

\begin{verbatim}
(CHARACTER D48
   (COMMENT from WLITF.VPL}
   (CHARWD D 14)     (CHARHT R 10)
   (CHARDP R  0)     (CHARIC R  0.0)
   (MAP
      (SETRULE R 14 R 1)(MOVERIGHT R 1)
      (SETRULE R 14 R 1)(MOVERIGHT R 1)
      (SETRULE R 14 R 2)(MOVERIGHT R 2)
      (SETRULE R 14 R 2)(MOVERIGHT R 2)
      (SETRULE R 14 R 1)(MOVERIGHT R 1)
      )
   )
\end{verbatim}

This character may be lowered by four units by editing in a way that the
following lines result. Please note that there were not only the {\em
moveup} and {\em movedown} commands added, but also the height an
depth have been changed.

\begin{verbatim}
(CHARACTER D48
   (COMMENT Lowered to give some depth)
   (CHARWD D 14)     (CHARHT R 10)
   (CHARDP R  4)     (CHARIC R  0.0)
   (MAP
      (MOVEDOWN R 4)
      (SETRULE R 14 R 1)(MOVERIGHT R 1)
      (SETRULE R 14 R 1)(MOVERIGHT R 1)
      (SETRULE R 14 R 2)(MOVERIGHT R 2)
      (SETRULE R 14 R 2)(MOVERIGHT R 2)
      (SETRULE R 14 R 1)(MOVERIGHT R 1)
      (MOVEUP R 4)
      )
   )
\end{verbatim}

\subsection{My barcodes are too tall}

For the sake of the person who has to scan the barcodes, barcodes just
can't be tall enough. If the complete code is as tall as wide, the
reading device can be twisted against the code at an angle of 45
degrees. If they are half as high, the angle reduces to 26 degrees.

If you really want short codes, you have to edit the sources. In the
{\sc vpl} file, you have to change the {\em charht} and the first parameter
to the {\em setrule} command at well. You also should change the {\em
designsize} parameter.

For the EAN font, search for {\tt bheight\#} and change {\tt22.85
mm\#} to the height you need. To get an EAN font without the digits
(bars only), you also have to edit the character definitions by
leaving out the last lines (saying {\tt nuller;}, e.g. and you have to
change the formula for calculating {\tt totheight} to not include the
height of the digits any more, also you should remove any references
to {\tt klotz}.

\subsection{My barcodes are not tall enough}

You may also change the sources, like described in the previous
subsection. But there is another solution that is perhaps more
practical. The idea is simple that double black is still black and
that you may overlay several printouts of the same barcode. Here is
the example code (it may be made taller by adding more copies of the
barcode box, of course). There is an extra dimension {\em bcboxdepth}
added to yield depth for the barcodes.

\begin{verbatim}
\newbox\bsavebox
\newdimen\bcboxdepth
\bcboxdepth=4pt
\def\bdbox#1{\setbox\bsavebox\hbox{\itf +#1-}
\vbox{\hsize=\wd\bsavebox\copy\bsavebox%
\vskip-\ht\bsavebox\vskip\bcboxdepth%
\box\bsavebox\vskip-\bcboxdepth}}
\def\bebox#1{\setbox\bsavebox\hbox{\itf +#1-}
\vbox{\hsize=\wd\bsavebox\copy\bsavebox%
\vskip0pt\copy\bsavebox%
\vskip-\ht\bsavebox\vskip\bcboxdepth%
\box\bsavebox\vskip-\bcboxdepth}}
 Example:
Tall {\sc itf} barcode looks like \quad\bdbox{1009}\quad
or (still taller) \quad\bebox{1009}\quad for 1009, e.g.
\end{verbatim}

And here is the result, you may compare it with the output of the
previous example:

Tall {\sc itf} barcode looks like \quad\bdbox{1009}\quad or (still
taller) \quad\bebox{1009}\quad for 1009, e.g.

\subsection{My barcodes are too wide/too narrow}

Perhaps you should change the barcode size (by changing
{\tt\tbs magnification}) and then read the subsections above about too
tall bars and bars that are not tall enough.

\subsection{My barcodes should run vertically}

Vertically oriented barcode is a good idea for cans, bottles and
similiar things.

It just depends on your {\sc dvi} driver. Perhaps there is a
possibility to handle rotation via {\tt\tbs special}, as with {\em
dvips} and the {\em rotate} macros. For experimantal purposes I have
included a code 39 version running down instead right ({\tt
wlcr39.mf}). You may perhaps use it with {\tt\tbs shortstack} or
similiar. Here is some code I used (successfully) with plain \TeX:

\begin{verbatim}
\font\testfont=wlcr39 scaled 1200
\def\bcbax{\let\next\bcbox\vbox\bgroup
    \offinterlineskip\testfont\setbox0\hbox{@}\hsize=\wd0
    \noindent{}@\hfil\break\next}
\def\fertig{@\hfil\break\egroup}
\def\bcbox#1{\if @#1\let\next\fertig\else#1\hfil\break\fi\next}

Look buddy, this is \bcbax1406632@ code 39
stacked and running downwards.
\end{verbatim}

\subsection{I heard something about full ascii code 39}

Well, umh, now that you ask\dots The truth is: You may perhaps be able to
configure your reading device to accept full ascii code~39. Then you may
code any 7~bit character you may think of. The bad news is that most of
the symbols have to be coded by two
characters -- and as code 39 has already low density, you should not
expect to get much use of it. Perhaps you might want to try code 128
instead---it features full 7-bit ascii without any compromise.

Another anti-feature is that switching to reading full-ascii code~39
may lead to some bad reads of `normal' code~39. In full ascii, the lower
case letters, e.g., are made by prepending the uppercase letters with a
plus sign, so {\tt+A} is read as {\tt a}. So, if you have got regular
code~39 that contains such character sequences, and you have switched
your reading device to full ascii, you may get bad results.

But, now that you want it: With the barcode package there comes a font
{\tt wlcf39.vpl}. You may install it like the other vpl files. It has got
132 characters and has already taken all the coding for full ascii, so
when you typ {\tt e,\$4+A}, it will silently be mapped to what you
would have typed as {\tt+E/L/D/KA}, if you would use the plain code~39
format\footnote{Please take note: 6 characters are blown up to 9,
think again about using another code}. In my version of full ascii
code~39, the start/stop sign is mapped to @@ (or \verb|\char128|). So
you actually had to add @@ to the front and the end of the example
string. If you can not exclude that the character sequence @@ appears
within the text you want to code, you have to avoid ligature
processing. You may do that like this:

\begin{verbatim}
\font\fullas=wlcf39 scaled 1728
\def\alphanolig{\char64\kern0pt}
\def\printacode#1{{\fullas@@{\catcode`\@=\active%
\let @=\alphanolig#1}@@}}
\end{verbatim}

You also have to escape or circumscribe special characters to make
them really printable, of course, so that when you wanted a backspace,
a tab, a space, a dollar sign, a percent sign, and a bell character to
be coded, you would type something like

\begin{verbatim}
{\fullas @@\char8\char9\char32\$\%\char7}
\end{verbatim}

\section{Troubleshooting}

\subsection{My barcodes can't be decoded}

At the very, very first, check if you really have added start and stop
codes. If you haven't, you're lost. Then check if you have enabled
decoding of your special barcodes within your reading device. With
modern scanners, you can disable almost any coding scheme.

At first, try to make them larger. If that does't help, make them
larger, again. If that will not help, obtain fonts in \TeX{} or
Metafont format and make them brighter. Or, perhaps better, obtain a
better printer.

You also should check the contrast between bars and background,
especially when using a dot matrix printer or coloured background
or---still worse---red bars.

\subsection{My decoder reads extra digits, sometimes}

I guess you use {\sc itf} barcode with a number of digits that is
sometimes even and sometimes odd. As {\sc itf} always codes two digits
at once, you should take care just to code an even number of digits at
any time. If an excess zero at the beginning of your number is not
acceptable, you might try another kind of barcode instead of {\sc
itf}.

\subsection{Can't run pdf\TeX\ with barcode fonts}

Get a later version of pdf\TeX, and you're done.

\section{Missing items}

\subsection{I am missing two-of-five}

I got specs for two-of-five (three bars). If you want it, let me know;
I might implement it. For other kinds of two-of-five, I haven't got
complete specs.

\subsection{I am missing plessey}

I got the specs but I haven't yet got the time to implement them. With
plessey, there also rise several questions due to the fact that
plessey has never been officialy standardized. The only thing that is
absolutely sure about plessey is how binary ones and zeros are
encoded. There are codes that are more wide-spread. Several modern
barcode readers can't cope with plessey code.

\subsection{I am missing \dots{} some other barcode}

I have checked out several two-dimensional barcodes. But for most of
them, support by \TeX{} seems to be rather pointless. Perhaps Metafont
could be called any time you have to code something to draw the
symbols, but you might as well use a custom drawing program that comes
from the barcode vendor. Also, most two-dimensional barcodes (and
other barcodes not mentioned here) are proprietary. Last not least the
reading devices I have access to can deal only with a finite range of
barcodes.

I am sorry, but I haven't got the specs for further barcodes
not mentioned here. If you could be so kind as to send them to me, I
might perhaps implement them.

\section{Contributions}

\subsection{EAN without preprocessing}

The following code is due to Kalvis M. Jansons.

It handles printing of {\sc ean} without having to run a
preprocessor. Not only are the {\sc ean} bars drawn, also the checksum
checksum is checked. It can be easily
adapted to non-\LaTeX{} use by omitting anything before \verb|\font|

\begin{verse}
Kalvis M. Jansons

eMail: \verb|<kalvis@math.ucl.ac.uk>|

\end{verse}

An easy way to use the barcode fonts in \LaTeX{}:

\begin{verbatim}
\documentclass[a4paper]{article}

\pagestyle{empty}

\setlength{\oddsidemargin}{0pt}
\setlength{\textwidth}{\paperwidth}
\addtolength{\textwidth}{-2in}
\setlength{\marginparwidth}{0pt}

\setlength{\textheight}{\paperheight}
\addtolength{\textheight}{-2.5in}
\setlength{\topmargin}{0pt}

\font\eanfont=WLEAN scaled 2000
\def\ean#1{\vbox{\vskip20pt\eanfont#1\vskip20pt}}
\newcount\num
\def\a#1{\num=#1 \advance\num by `A \char\num}
\def\b#1{\num=#1 \advance\num by `a \char\num}
\def\c#1{\num=#1 \advance\num by `K \char\num}
\def\C#1#2#3#4#5#6{\c#1\c#2\c#3\c#4\c#5\c#6}

\def\A#1#2#3#4#5#6#7{\ifcase #7
{\a#1\a#2\a#3\a#4\a#5\a#6}%
\or {\a#1\a#2\b#3\a#4\b#5\b#6}%
\or {\a#1\a#2\b#3\b#4\a#5\b#6}%
\or {\a#1\a#2\b#3\b#4\b#5\a#6}%
\or {\a#1\b#2\a#3\a#4\b#5\b#6}%
\or {\a#1\b#2\b#3\a#4\a#5\b#6}%
\or {\a#1\b#2\b#3\b#4\a#5\a#6}%
\or {\a#1\b#2\a#3\b#4\a#5\b#6}%
\or {\a#1\b#2\a#3\b#4\b#5\a#6}%
\or {\a#1\b#2\b#3\a#4\b#5\a#6}%
\fi}

\newcount\cha
\newcount\chb
\makeatletter
\long\def\for{\@for}
\makeatother
\gdef\mysix#1#2#3#4#5#6{,#1#2,#3#4,#5#6}
\gdef\mywork#1#2{\advance\cha by #1 \advance\chb by #2}
\gdef\barch#1.#2.#3.{
\xdef\mylist{0#1\mysix#2\mysix#3}
\cha=0
\chb=0
\for \x:=\mylist\do{\expandafter\mywork\x}
\multiply\cha by 3
\advance\chb by \cha
\cha=\chb
\divide\cha by 10
\multiply\cha by -10
\advance\chb by \cha}
\def\bar#1.#2.#3.{\barch#1.#2.#3.
\ifnum\chb>0 #1#2#3 has a bad check sum!\\[20pt]
\else \ean{#1 +\A#2#1-\C#3+}\fi}
\begin{document}

 Examples

\bar3.034325.106199.

\bar4.074400.410000.

\bar5.449000.055521.

\bar5.010027.522336.

\bar8.410005.421052.

\bar9.780192.828941.

\end{document}
\end{verbatim}

\section{Installation}

Installation is described within the README file, in short.
Here I assume that you have a TDS compliant \TeX{} installation
already running.

Installation consists of several steps:

\begin{enumerate}
 \item unpack barcodes.dtx by running barcodes.ins through \TeX (I
guess you have already done that, lest you would not be able to read
this text), like {\tt tex barcodes.ins}
\item run {\tt vptovf} on all the vpl files, e.g. type something like
 {\tt vptovf wlitf.vpl wlitf.vf wlitf.tfm}
\item if your \TeX\ system does not support automatic generation of
{\tt tfm} files, run mf on all the .mf files.
\item move all the files to appropriate locations
\item install codean.pl for use with Perl
\item run the documentation through \LaTeX{}
\item clean up
\end{enumerate}

Items 1,2,4,6, and part of 7 can be automatically executed on ms-dos or
ms windows systems by running {\tt install.bat}. If you do not supply
any parameters, {\tt install} will install to {\tt C:\char92TeXMF}. If
you do supply a parameter, it should be the path to your TEXMF tree
(no trailing backslash, please).

Here are the appropriate locations mentioned above -- all of them
in the TEXMF tree:

\begin{tabular}{ll}
\em move & \em to\\
\tt *.mf & \tt fonts/source/public/misc\\
\tt *.tfm & \tt fonts/tfm/public/misc\\
\tt *.vf & \tt fonts/vf/public\\
\tt code39.tex &\tt tex/generic/misc\\
\tt barcodes.sty &\tt tex/latex/misc\\
\end{tabular}

Of course other locations might also work, but to me it seems fine
this way.

Running plain \TeX{} on {\tt examples.tex} is the ultimate test: This
will only work when all the installation is well done.

The installation of Perl programs is beyond the scope of this
documentation. If you do not need code 128, you can go without
{\tt codean.pl}, especially if you use the macros described above.

\bibliography{bcfaq}

\end{document}
\endinput
%%
%% End of file `bcfaq.tex'.
