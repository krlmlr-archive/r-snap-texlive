%%
%% This is file `msizetst.tex',
%% generated with the docstrip utility.
%%
%% The original source files were:
%%
%% moresize.dtx  (with options: `testfile')
%% 
%%  Version: v1.9 [1999/07/26]
%%  Author : Christian Cornelssen <cornsub1@flintstone.ukbf.fu-berlin.de>
%% 
%%  Copyright 1997,1998,1999 Christian Cornelssen
%%  This program can be redistributed and/or modified under the terms
%%  of the LaTeX Project Public License Distributed from CTAN
%%  archives in directory macros/latex/base/lppl.txt; either
%%  version 1 of the License, or any later version.
%% 
%% This is a test file for text and math size settings.
%% You might want to test several configurations.  Therefore,
%% it might be a good idea to save different versions of this file
%% under different names.
\documentclass[11pt,a4paper]{article}
\usepackage{exscale}    %% for scaled \sum, \int etc.

%% Available options for the `moresize' package:
%%   10pt, 11pt, 12pt   (in \documentclass) determines font sizes
%%   normalscripts      avoid script size explosion in huge writings
%%   smallscripts       smaller, more appropriate script sizes
%%   ecpatch            tell NFSS that huge EC fonts are available
\usepackage[smallscripts]{moresize}

%% Use the `t1enc' package for testing with DC/EC fonts
\usepackage{t1enc}

%% Consider a paragraph with a text line of depth d (below the baseline),
%% followed by a line with an inline formula,
%% followed by a text line with height h (above the baseline).
%% Then the formula's height may not exceed \baselineskip - d,
%% and its depth may not exceed \baselineskip - h,
%% or the baseline skips will be stretched.
%% The roman-text brackets `[' and `]' are used for estimating d and h.
%% Note that the space for the formula seems to get larger if NFSS
%% substitutes a smaller font, because \baselineskip does not shrink
%% accordingly!
\newlength\mathht       %% height available for an inline formula
\newlength\mathdp       %% depth available for an inline formula
\newcommand\calcmathspace{%
        \settodepth\mathht{\textrm{[}}%
        \setlength\mathht{-\mathht}%
        \addtolength\mathht{\baselineskip}%
        \settoheight\mathdp{\textrm{]}}%
        \setlength\mathdp{-\mathdp}%
        \addtolength\mathdp{\baselineskip}}

%% The following command shows its argument, typeset in text style,
%% between two horizontal lines showing the available space in a text line.
%% If the argument overprints the lines, it would stretch
%% the baseline skip in a paragraph.
\newcommand\linebox[1]{\calcmathspace\begin{tabular}{@{}l@{}}\hline
        \raisebox{0pt}[\mathht][\mathdp]{#1}\\\hline\end{tabular}}

%% A macro for sample math expressions.
%% The original definition gives expressions which should fit into
%% paragraphs without enlarging the baseline skip.
%% For variable-sized \sum or \int symbols, \usepackage{exscale} is needed.
%% Caution: Main (i.e. non-script) symbols are 24.88pt maximum
%% with CM math fonts. You may try using text fonts instead,
%% but some vertical positionings and spacings still go wrong then.
\newcommand\msample{\ensuremath{
    \frac{X}{Y}
\,  g^{i_1\ldots i_n}_{j_1\ldots j_n}
\,  \frac{t^2}{2!}
\,  e^{-\frac{t}{\tau}}
}}

%% None of the size options 10pt, 11pt, and 12pt offers the entire spectrum
%% of available font sizes.
%% For example, the 10pt option offers no command for selecting 10.95pt,
%% and the 12pt option does not provide a command for 9pt.
%% The 11pt option fills both gaps, but misses the extremes 35.83pt and 5pt.
%% Therefore, a shorthand for lower-level size switches is provided.
%% E.g. \Size{20.74}{25} selects a 20.74pt font size with 25pt baselineskip.
\newcommand\Size[2]{\fontsize{#1}{#2}\selectfont}

\begin{document}
\parindent0pt

%% The test output should fit onto one page
\pagestyle{empty}
\enlargethispage{1in}

%% Don't forget to list the moresize options in the test output!
\textsf{moresize} options:
%%[10pt]
[11pt]
%%[12pt]
%%[normalscripts]
[smallscripts]
%%[ecpatch]

%% Are you using HUGE fonts?
%%Using CM fonts only.
%%Using DC fonts for text, CM fonts for math.
Using EC fonts for text, CM fonts for math.

\begin{center}
\newcommand\n{\\\\[-1ex]}       %% for separating the tabular lines
\begin{tabular}{@{}ll@{}}
        \Size{35.83}{43}\linebox{36pt}
                                & \Size{35.83}{43}\linebox{\msample}
\n      \HUGE\linebox{HUGE}     & \HUGE\linebox{\msample}
\n      \Huge\linebox{Huge}     & \Huge\linebox{\msample}
\n      \huge\linebox{huge}     & \huge\linebox{\msample}
\n      \LARGE\linebox{LARGE}   & \LARGE\linebox{\msample}
\n      \Large\linebox{Large}   & \Large\linebox{\msample}
\n      \large\linebox{large}   & \large\linebox{\msample}
\n      \normalsize\linebox{normalsize}
                                & \normalsize\linebox{\msample}
\n      \small\linebox{small}   & \small\linebox{\msample}
\n      \footnotesize\linebox{footnotesize}
                                & \footnotesize\linebox{\msample}
\n      \scriptsize\linebox{scriptsize}
                                & \scriptsize\linebox{\msample}
\n      \ssmall\linebox{ssmall} & \ssmall\linebox{\msample}
\n      \tiny\linebox{tiny}     & \tiny\linebox{\msample}
\n      \Size{5}{6}\linebox{5pt}
                                & \Size{5}{6}\linebox{\msample}
\end{tabular}
\end{center}
\end{document}
%% 
%%
%% End of file `msizetst.tex'.
