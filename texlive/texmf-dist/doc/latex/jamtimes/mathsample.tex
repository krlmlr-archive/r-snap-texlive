\documentclass{article}
\usepackage{ifpdf}
\ifpdf\pdfmapfile{+jtm.map}\fi
%\usepackage[T1]{fontenc}
\usepackage{jamtimes}
\usepackage{lipsum,textcomp,amsmath,url,amsfonts,longtable}
\DeclareMathSymbol{\dit}{\mathord}{letters}{`d}
\DeclareMathSymbol{\dup}{\mathord}{operators}{`d}
\def\test#1{#1}

\def\testnums{%
  \test 0 \test 1 \test 2 \test 3 \test 4 \test 5 \test 6 \test 7
  \test 8 \test 9 }
\def\testupperi{%
  \test A \test B \test C \test D \test E \test F \test G \test H
  \test I \test J \test K \test L \test M }
\def\testupperii{%
  \test N \test O \test P \test Q \test R \test S \test T \test U
  \test V \test W \test X \test Y \test Z }
\def\testupper{%
  \testupperi\testupperii}

\def\testloweri{%
  \test a \test b \test c \test d \test e \test f \test g \test h
  \test i \test j \test k \test l \test m }
\def\testlowerii{%
  \test n \test o \test p \test q \test r \test s \test t \test u
  \test v \test w \test x \test y \test z 
  \test\imath \test\jmath }
\def\testlower{%
  \testloweri\testlowerii}

\def\testupgreeki{%
  \test A \test B \test\Gamma \test\Delta \test E \test Z \test H
  \test\Theta \test I \test K \test\Lambda \test M }
\def\testupgreekii{%
  \test N \test\Xi \test O \test\Pi \test P \test\Sigma \test T
  \test\Upsilon \test\Phi \test X \test\Psi \test\Omega 
}
\def\testupgreek{%
  \testupgreeki\testupgreekii}

\def\testlowgreeki{%
  \test\alpha \test\beta \test\gamma \test\delta \test\epsilon
  \test\zeta \test\eta \test\theta \test\iota \test\kappa \test\lambda
  \test\mu }
\def\testlowgreekii{%
  \test\nu \test\xi \test o \test\pi \test\rho \test\sigma \test\tau
  \test\upsilon \test\phi \test\chi \test\psi \test\omega }
\def\testlowgreekiii{%
  \test\varepsilon \test\vartheta \test\varpi \test\varrho
  \test\varsigma \test\varphi \test\ell \test\wp}
\def\testlowgreek{%
  \testlowgreeki\testlowgreekii\testlowgreekiii}
\begin{document}

\section{Sebastian's math test}



The default math mode font is $Math\ Italic$. This should not be
confused with ordinary \emph{Text Italic} -- notice the different spacing\,!
\verb|\mathbf| produces bold roman letters: $ \mathbf{abcABC} $.
If you wish to embolden complete formulas,
use the \verb|\boldmath| command \emph{before} going into math mode.  
This changes the default math fonts to bold. 
 
\begin{tabular}{ll}
\texttt{normal}   & $ x = 2\pi \Rightarrow x \simeq 6.28 $\\
\texttt{mathbf}   & $\mathbf{x} = 2\pi \Rightarrow \mathbf{x} \simeq 6.28 $\\
\texttt{boldmath} & {\boldmath $x = \mathbf{2}\pi \Rightarrow x 
                    \simeq{\mathbf{6.28}}              $}\\
\end{tabular}
\smallskip

Greek is available in upper and lower case:
$\alpha,\beta \dots \Omega$, and there are special
symbols such as $ \hbar$ (compare to $h$).
Digits in formulas $1, 2, 3\dots$ may differ from those in text: 4, 5,
6\dots

There is Sans Serif alphabet $\mathsf{abcdeABCD}$ selected by
\verb|\mathsf| and Typewriter math $\mathtt{abcdeABCD}$ selected by
\verb|\mathtt|. 

There is a calligraphic alphabet \verb|\mathcal| for upper case letters
$ \mathcal{ABCDE}\dots $, and there are letters for number sets: $\mathbb{A\dots Z} $,
which are produced using \verb|\mathbb|.  There are Fraktur letters
$\mathfrak{abcdeABCDE}$ produced using \verb|\mathfrak|
 
\begin{equation}
  \sigma(t)=\frac{1}{\sqrt{2\pi}}
  \int^t_0 e^{-x^2/2} dx 
\end{equation}

\begin{equation}
  \prod_{j\geq 0}
  \left(\sum_{k\geq 0}a_{jk} z^k\right) 
= \sum_{k\geq 0} z^n
  \left( \sum_{{k_0,k_1,\ldots\geq 0}
          \atop{k_0+k_1+\ldots=n}    }
        a{_0k_0}a_{1k_1}\ldots  \right) 
\end{equation}

\begin{equation}
\pi(n) = \sum_{m=2}^{n}
  \left\lfloor \left(\sum_{k=1}^{m-1}
       \lfloor(m/k)/\lceil m/k\rceil 
       \rfloor \right)^{-1}
  \right\rfloor
\end{equation}

\begin{equation}
\{\underbrace{%
    \overbrace{\mathstrut a,\ldots,a}^{k\ a's},
    \overbrace{\mathstrut b,\ldots,b}^{l\ b's}}
  _{k+l\ \mathrm{elements}}                   \}
\end{equation}

\[
\mbox{W}^+\
\begin{array}{l}
\nearrow\raise5pt\hbox{$\mu^+ + \nu_{\mu}$}\\
\rightarrow         \pi^+ +\pi^0         \\[5pt]
\rightarrow \kappa^+ +\pi^0              \\
\searrow\lower5pt\hbox{$\mathrm{e}^+ 
          +\nu_{\scriptstyle\mathrm{e}}$}
\end{array}
\]

\[
\frac{\pm
\left|\begin{array}{ccc}
x_1-x_2  & y_1-y_2 & z_1-z_2 \\
l_1      & m_1     & n_1     \\
l_2      & m_2     & n_2
\end{array}\right|}{
\sqrt{\left|\begin{array}{cc}l_1&m_1\\
l_2&m_2\end{array}\right|^2
+     \left|\begin{array}{cc}m_1&n_1\\
n_1&l_1\end{array}\right|^2
+     \left|\begin{array}{cc}m_2&n_2\\
n_2&l_2\end{array}\right|^2}}
\]




\section{Math Tests}
\label{sec:mthtests}



Math test are taken from\cite{Schmidt04:PSNFSS9.2}.

\parindent 0pt
%\mathindent 1em


\subsection{Math Alphabets}

Math Italic (\texttt{\string\mathnormal})
\def\test#1{\mathnormal{#1},}
\begin{eqnarray*}
%  && {\testnums}\\
  && {\testupper}\\
  && {\testlower}\\ 
  && {\testupgreek}\\
  && {\testlowgreek}
\end{eqnarray*}%

Math Roman (\texttt{\string\mathrm})
\def\test#1{\mathrm{#1},}
\begin{eqnarray*}
  && {\testnums}\\
  && {\testupper}\\
  && {\testlower}\\ 
  && {\testupgreek}\\
  && {\testlowgreek}
\end{eqnarray*}%

%Math Italic Bold
%\def\test#1{\mathbm{#1},}
%\begin{eqnarray*}
%  && {\testnums}\\
%  && {\testupper}\\
%  && {\testlower}\\ 
%  && {\testupgreek}\\
%  && {\testlowgreek}
%\end{eqnarray*}%

Math Bold (\texttt{\string\mathbf})
\def\test#1{\mathbf{#1},}
\begin{eqnarray*}
  && {\testnums}\\
  && {\testupper}\\
  && {\testlower}\\ 
%  && {\testupgreek}
\end{eqnarray*}%

Math Sans Serif (\texttt{\string\mathsf})
\def\test#1{\mathsf{#1},}
\begin{eqnarray*}
  && {\testnums}\\
  && {\testupper}\\
  && {\testlower}\\ 
%  && {\testupgreek}
\end{eqnarray*}%



Caligraphic (\texttt{\string\mathcal})
\def\test#1{\mathcal{#1},}
\begin{eqnarray*}
  && {\testupper}
\end{eqnarray*}%

%Script (\texttt{\string\mathscr})
%\def\test#1{\mathscr{#1},}
%\begin{eqnarray*}
%  && {\testupper}
%\end{eqnarray*}%

Fraktur (\texttt{\string\mathfrak})
\def\test#1{\mathfrak{#1},}
\begin{eqnarray*}
  && {\testupper}\\
  && {\testlower}
\end{eqnarray*}%

Blackboard Bold (\texttt{\string\mathbb})
\def\test#1{\mathbb{#1},}
\begin{eqnarray*}
  && {\testupper}
\end{eqnarray*}%


\clearpage
\subsection{Character Sidebearings}

\def\test#1{|#1|+}
\begin{eqnarray*}
  && {\testupperi}\\
  && {\testupperii}\\
  && {\testloweri}\\ 
  && {\testlowerii}\\ 
  && {\testupgreeki}\\
  && {\testupgreekii}\\
  && {\testlowgreeki}\\
  && {\testlowgreekii}\\
  && {\testlowgreekiii}
\end{eqnarray*}%
%
\def\test#1{|\mathrm{#1}|+}%
\begin{eqnarray*}
  && {\testupperi}\\
  && {\testupperii}\\
  && {\testloweri}\\ 
  && {\testlowerii}\\ 
  && {\testupgreeki}\\
  && {\testupgreekii}
\end{eqnarray*}%
%
%\def\test#1{|\mathbm{#1}|+}%
%\begin{eqnarray*}
%  && {\testupperi}\\
%  && {\testupperii}\\
%  && {\testloweri}\\ 
%  && {\testlowerii}\\ 
%  && {\testupgreeki}\\
%  && {\testupgreekii}\\
%  && {\testlowgreeki}\\
%  && {\testlowgreekii}\\
%  && {\testlowgreekiii}
%\end{eqnarray*}%
%%
%\def\test#1{|\mathbf{#1}|+}%
%\begin{eqnarray*}
%  && {\testupperi}\\
%  && {\testupperii}\\
%  && {\testloweri}\\ 
%  && {\testlowerii}\\ 
%  && {\testupgreeki}\\
%  && {\testupgreekii}
%\end{eqnarray*}%
%
\def\test#1{|\mathcal{#1}|+}%
\begin{eqnarray*}
  && {\testupperi}\\
  && {\testupperii}
\end{eqnarray*}%


\clearpage
\subsection{Superscript positioning}

\def\test#1{#1^{2}+}%
\begin{eqnarray*}
  && {\testupperi}\\
  && {\testupperii}\\
  && {\testloweri}\\ 
  && {\testlowerii}\\ 
  && {\testupgreeki}\\
  && {\testupgreekii}\\
  && {\testlowgreeki}\\
  && {\testlowgreekii}\\
  && {\testlowgreekiii}
\end{eqnarray*}%
%
\def\test#1{\mathrm{#1}^{2}+}%
\begin{eqnarray*}
  && {\testupperi}\\
  && {\testupperii}\\
  && {\testloweri}\\ 
  && {\testlowerii}\\ 
  && {\testupgreeki}\\
  && {\testupgreekii}
\end{eqnarray*}%
%
%\def\test#1{\mathbm{#1}^{2}+}%
%\begin{eqnarray*}
%  && {\testupperi}\\
%  && {\testupperii}\\
%  && {\testloweri}\\ 
%  && {\testlowerii}\\ 
%  && {\testupgreeki}\\
%  && {\testupgreekii}\\
%  && {\testlowgreeki}\\
%  && {\testlowgreekii}\\
%  && {\testlowgreekiii}
%\end{eqnarray*}%
%
%\def\test#1{\mathbf{#1}^{2}+}%
%\begin{eqnarray*}
%  && {\testupperi}\\
%  && {\testupperii}\\
%  && {\testloweri}\\ 
%  && {\testlowerii}\\ 
%  && {\testupgreeki}\\
%  && {\testupgreekii}
%\end{eqnarray*}
%
\def\test#1{\mathcal{#1}^{2}+}%
\begin{eqnarray*}
  && {\testupperi}\\
  && {\testupperii}
\end{eqnarray*}%


\clearpage
\subsection{Subscript positioning}

\def\test#1{\mathnormal{#1}_{i}+}%
\begin{eqnarray*}
  && {\testupperi}\\
  && {\testupperii}\\
  && {\testloweri}\\ 
  && {\testlowerii}\\ 
  && {\testupgreeki}\\
  && {\testupgreekii}\\
  && {\testlowgreeki}\\
  && {\testlowgreekii}\\
  && {\testlowgreekiii}
\end{eqnarray*}%
%
\def\test#1{\mathrm{#1}_{i}+}%
\begin{eqnarray*}
  && {\testupperi}\\
  && {\testupperii}\\
  && {\testloweri}\\ 
  && {\testlowerii}\\ 
  && {\testupgreeki}\\
  && {\testupgreekii}
\end{eqnarray*}%
%
%\def\test#1{\mathbm{#1}_{i}+}%
%\begin{eqnarray*}
%  && {\testupperi}\\
%  && {\testupperii}\\
%  && {\testloweri}\\ 
%  && {\testlowerii}\\ 
%  && {\testupgreeki}\\
%  && {\testupgreekii}\\
%  && {\testlowgreeki}\\
%  && {\testlowgreekii}\\
%  && {\testlowgreekiii}
%\end{eqnarray*}
%%
%\def\test#1{\mathbf{#1}_{i}+}%
%\begin{eqnarray*}
%  && {\testupperi}\\
%  && {\testupperii}\\
%  && {\testloweri}\\ 
%  && {\testlowerii}\\ 
%  && {\testupgreeki}\\
%  && {\testupgreekii}
%\end{eqnarray*}%
%
\def\test#1{\mathcal{#1}_{i}+}%
\begin{eqnarray*}
  && {\testupperi}\\
  && {\testupperii}
\end{eqnarray*}%


\clearpage
\subsection{Accent positioning}

\def\test#1{\hat{#1}+}%
\begin{eqnarray*}
  && {\testupperi}\\
  && {\testupperii}\\
  && {\testloweri}\\ 
  && {\testlowerii}\\ 
  && {\testupgreeki}\\
  && {\testupgreekii}\\
  && {\testlowgreeki}\\
  && {\testlowgreekii}\\
  && {\testlowgreekiii}
\end{eqnarray*}%
%
\def\test#1{\hat{\mathrm{#1}}+}%
\begin{eqnarray*}
  && {\testupperi}\\
  && {\testupperii}\\
  && {\testloweri}\\ 
  && {\testlowerii}\\ 
%  && {\testupgreeki}\\
%  && {\testupgreekii}
\end{eqnarray*}%
%
%\def\test#1{\hat{\mathbm{#1}}+}%
%\begin{eqnarray*}
%  && {\testupperi}\\
%  && {\testupperii}\\
%  && {\testloweri}\\ 
%  && {\testlowerii}\\ 
%  && {\testupgreeki}\\
%  && {\testupgreekii}\\
%  && {\testlowgreeki}\\
%  && {\testlowgreekii}\\
%  && {\testlowgreekiii}
%\end{eqnarray*}%
%%
%\def\test#1{\hat{\mathbf{#1}}+}%
%\begin{eqnarray*}
%  && {\testupperi}\\
%  && {\testupperii}\\
%  && {\testloweri}\\ 
%  && {\testlowerii}\\ 
%  && {\testupgreeki}\\
%  && {\testupgreekii}
%\end{eqnarray*}
%
\def\test#1{\hat{\mathcal{#1}}+}%
\begin{eqnarray*}
  && {\testupperi}\\
  && {\testupperii}
\end{eqnarray*}%


\clearpage
\subsection{Differentials}

\begin{eqnarray*}
\gdef\test#1{\dit #1+}%
  && {\testupperi}\\
  && {\testupperii}\\
  && {\testloweri}\\ 
  && {\testlowerii}\\ 
  && {\testupgreeki}\\
  && {\testupgreekii}\\
  && {\testlowgreeki}\\
  && {\testlowgreekii}\\
  && {\testlowgreekiii}\\
\gdef\test#1{\dit \mathrm{#1}+}%
  && {\testupgreeki}\\
  && {\testupgreekii}
\end{eqnarray*}%
%
\begin{eqnarray*}
\gdef\test#1{\dup #1+}%
  && {\testupperi}\\
  && {\testupperii}\\
  && {\testloweri}\\ 
  && {\testlowerii}\\ 
  && {\testupgreeki}\\
  && {\testupgreekii}\\
  && {\testlowgreeki}\\
  && {\testlowgreekii}\\
  && {\testlowgreekiii}\\
\gdef\test#1{\dup \mathrm{#1}+}%
  && {\testupgreeki}\\
  && {\testupgreekii}
\end{eqnarray*}%
%
\begin{eqnarray*}
\gdef\test#1{\partial #1+}%
  && {\testupperi}\\
  && {\testupperii}\\
  && {\testloweri}\\ 
  && {\testlowerii}\\ 
  && {\testupgreeki}\\
  && {\testupgreekii}\\
  && {\testlowgreeki}\\
  && {\testlowgreekii}\\
  && {\testlowgreekiii}\\
\gdef\test#1{\partial \mathrm{#1}+}%
  && {\testupgreeki}\\
  && {\testupgreekii}
\end{eqnarray*}%


\clearpage
\subsection{Slash kerning}

\def\test#1{1/#1+}
\begin{eqnarray*}
  && {\testupperi}\\
  && {\testupperii}\\
  && {\testloweri}\\ 
  && {\testlowerii}\\ 
  && {\testupgreeki}\\
  && {\testupgreekii}\\
  && {\testlowgreeki}\\
  && {\testlowgreekii}\\
  && {\testlowgreekiii}
\end{eqnarray*}

\def\test#1{#1/2+}
\begin{eqnarray*}
  && {\testupperi}\\
  && {\testupperii}\\
  && {\testloweri}\\ 
  && {\testlowerii}\\ 
  && {\testupgreeki}\\
  && {\testupgreekii}\\
  && {\testlowgreeki}\\
  && {\testlowgreekii}\\
  && {\testlowgreekiii}
\end{eqnarray*}


\clearpage
\subsection{Big operators}

\def\testop#1{#1_{i=1}^{n} x^{n} \quad}
\begin{displaymath}
  \testop\sum 
  \testop\prod 
  \testop\coprod 
  \testop\int 
  \testop\oint
\end{displaymath}
\begin{displaymath}
  \testop\bigotimes 
  \testop\bigoplus
  \testop\bigodot
  \testop\bigwedge 
  \testop\bigvee 
  \testop\biguplus 
  \testop\bigcup 
  \testop\bigcap 
  \testop\bigsqcup 
% \testop\bigsqcap
\end{displaymath}


\subsection{Radicals}

\begin{displaymath}
  \sqrt{x+y} \qquad \sqrt{x^{2}+y^{2}} \qquad 
  \sqrt{x_{i}^{2}+y_{j}^{2}} \qquad
  \sqrt{\left(\frac{\cos x}{2}\right)} \qquad 
  \sqrt{\left(\frac{\sin x}{2}\right)}
\end{displaymath}
  
\begingroup
\delimitershortfall-1pt
\begin{displaymath}
  \sqrt{\sqrt{\sqrt{\sqrt{\sqrt{\sqrt{\sqrt{x+y}}}}}}}
\end{displaymath}
\endgroup % \delimitershortfall


\subsection{Over- and underbraces}

\begin{displaymath}
  \overbrace{x} \quad
  \overbrace{x+y} \quad
  \overbrace{x^{2}+y^{2}} \quad
  \overbrace{x_{i}^{2}+y_{j}^{2}} \quad
  \underbrace{x} \quad
  \underbrace{x+y} \quad
  \underbrace{x_{i}+y_{j}} \quad
  \underbrace{x_{i}^{2}+y_{j}^{2}} \quad
\end{displaymath}


\subsection{Normal and wide accents}

\begin{displaymath}
  \dot{x} \quad 
  \ddot{x} \quad 
  \vec{x} \quad 
  \bar{x} \quad
  \overline{x} \quad
  \overline{xx} \quad
  \tilde{x} \quad
  \widetilde{x} \quad
  \widetilde{xx} \quad
  \widetilde{xxx} \quad
  \hat{x} \quad 
  \widehat{x} \quad 
  \widehat{xx} \quad 
  \widehat{xxx} \quad
\end{displaymath}


\subsection{Long arrows}

\begin{displaymath}
  \leftarrow \mathrel{-} \rightarrow \quad
  \leftrightarrow \quad
  \longleftarrow  \quad
  \longrightarrow \quad
  \longleftrightarrow \quad
  \Leftarrow = \Rightarrow \quad
  \Leftrightarrow \quad
  \Longleftarrow  \quad
  \Longrightarrow \quad
  \Longleftrightarrow \quad
\end{displaymath}


\subsection{Left and right delimters}

\def\testdelim#1#2{ - #1 f #2 - }
\begin{displaymath}
  \testdelim() 
  \testdelim[] 
  \testdelim\lfloor\rfloor 
  \testdelim\lceil\rceil 
  \testdelim\langle\rangle 
  \testdelim\{\} 
\end{displaymath}

\def\testdelim#1#2{ - \left#1 f \right#2 - }
\begin{displaymath}
  \testdelim() 
  \testdelim[] 
  \testdelim\lfloor\rfloor 
  \testdelim\lceil\rceil 
  \testdelim\langle\rangle 
  \testdelim\{\} 
% \testdelim\lgroup\rgroup
% \testdelim\lmoustache\rmoustache
\end{displaymath}
\begin{displaymath}
  \testdelim)(
  \testdelim][
  \testdelim// 
  \testdelim\backslash\backslash
  \testdelim/\backslash 
  \testdelim\backslash/
\end{displaymath}


\clearpage
\subsection{Big-g-g delimters}

\def\testdelim#1#2{%
  - \left#1\left#1\left#1\left#1\left#1\left#1\left#1\left#1 - 
  \right#2\right#2\right#2\right#2\right#2\right#2\right#2\right#2 -}

\begingroup
\delimitershortfall-1pt
\begin{displaymath}
  \testdelim\lfloor\rfloor 
  \qquad 
  \testdelim()
\end{displaymath}
\begin{displaymath}
  \testdelim\lceil\rceil 
  \qquad 
  \testdelim\{\} 
\end{displaymath}
\begin{displaymath}
  \testdelim[] 
  \qquad 
  \testdelim\lgroup\rgroup
\end{displaymath}
\begin{displaymath}
  \testdelim\langle\rangle
  \qquad 
  \testdelim\lmoustache\rmoustache
\end{displaymath}
\begin{displaymath}
  \testdelim\uparrow\downarrow \quad
  \testdelim\Uparrow\Downarrow \quad
\end{displaymath}
\endgroup % \delimitershortfall

\subsection{Symbols}
\label{sec:symbols}

This is from~\cite{Eijkhout07:TeXbyTopic}

\begin{longtable}{lllll}
Symbol & Control Sequence & mathcode & Family & Hex Position \\
$\partial$&partial&           "0140&1&40\\
$\flat$&flat&              "015B&1&5B\\
$\natural$&natural&           "015C&1&5C\\
$\sharp$&sharp&             "015D&1&5D\\
$\ell$&ell&               "0160&1&60\\
$\imath$&imath&             "017B&1&7B\\
$\jmath$&jmath&             "017C&1&7C\\
$\wp$&wp&                "017D&1&7D\\
$\prime$&prime&             "0230&2&30\\
$\infty$&infty&             "0231&2&31\\
$\triangle$&triangle&          "0234&2&34\\
$\forall$&forall&            "0238&2&38\\
$\exists$&exists&            "0239&2&39\\
$\neg$&neg&               "023A&2&3A\\
$\emptyset$&emptyset&          "023B&2&3B\\
$\Re$&Re&                "023C&2&3C\\
$\Im$&Im&                "023D&2&3D\\
$\top$&top&               "023E&2&3E\\
$\bot$&bot&               "023F&2&3F\\
$\aleph$&aleph&             "0240&2&40\\
$\nabla$&nabla&             "0272&2&72\\
$\clubsuit$&clubsuit&          "027C&2&7C\\
$\diamondsuit$&diamondsuit&       "027D&2&7D\\
$\heartsuit$&heartsuit&         "027E&2&7E\\
$\spadesuit$&spadesuit&         "027F&2&7F\\  
$\smallint \displaystyle\smallint$&
    smallint&          "1273&2&73\cr
$\bigsqcup \displaystyle\bigsqcup$&
    bigsqcup&          "1346&3&46\cr
$\ointop \displaystyle\ointop$&
    ointop&            "1348&3&48\cr
$\bigodot \displaystyle\bigodot$&
    bigodot&           "134A&3&4A\cr
$\bigoplus \displaystyle\bigoplus$&
    bigoplus&          "134C&3&4C\cr
$\bigotimes \displaystyle\bigotimes$&
    bigotimes&         "134E&3&4E\cr
$\sum \displaystyle\sum$&
    sum&               "1350&3&50\cr
$\prod \displaystyle\prod$&
    prod&              "1351&3&51\cr
$\intop \displaystyle\intop$&
    intop&             "1352&3&52\cr
$\bigcup \displaystyle\bigcup$&
    bigcup&            "1353&3&53\cr
$\bigcap \displaystyle\bigcap$&
    bigcap&            "1354&3&54\cr
$\biguplus \displaystyle\biguplus$&
    biguplus&          "1355&3&55\cr
$\bigwedge \displaystyle\bigwedge$&
    bigwedge&          "1356&3&56\cr
$\bigvee \displaystyle\bigvee$&
    bigvee&            "1357&3&57\cr
$\coprod \displaystyle\coprod$&
    coprod&            "1360&3&60\cr
$\triangleright$&triangleright&     "212E&1&2E\cr
$\triangleleft$&triangleleft&      "212F&1&2F\cr
$\star$&star&              "213F&1&3F\cr
$\cdot$&cdot&              "2201&2&01\cr
$\times$&times&             "2202&2&02\cr
$\ast$&ast&               "2203&2&03\cr
$\div$&div&               "2204&2&04\cr
$\diamond$&diamond&           "2205&2&05\cr
$\pm$&pm&                "2206&2&06\cr
$\mp$&mp&                "2207&2&07\cr
$\oplus$&oplus&             "2208&2&08\cr
$\ominus$&ominus&            "2209&2&09\cr
$\otimes$&otimes&            "220A&2&0A\cr
$\oslash$&oslash&            "220B&2&0B\cr
$\odot$&odot&              "220C&2&0C\cr
$\bigcirc$&bigcirc&           "220D&2&0D\cr
$\circ$&circ&              "220E&2&0E\cr
$\bullet$&bullet&            "220F&2&0F\cr
$\bigtriangleup$&bigtriangleup&     "2234&2&34\cr
$\bigtriangledown$&bigtriangledown&   "2235&2&35\cr
$\cup$&cup&               "225B&2&5B\cr
$\cap$&cap&               "225C&2&5C\cr
$\uplus$&uplus&             "225D&2&5D\cr
$\wedge$&wedge&             "225E&2&5E\cr
$\vee$&vee&               "225F&2&5F\cr
$\setminus$&setminus&          "226E&2&6E\cr
$\wr$&wr&                "226F&2&6F\cr
$\amalg$&amalg&             "2271&2&71\cr
$\sqcup$&sqcup&             "2274&2&74\cr
$\sqcap$&sqcap&             "2275&2&75\cr
$\dagger$&dagger&            "2279&2&79\cr
$\ddagger$&ddagger&           "227A&2&7A\cr
$\leftharpoonup$&leftharpoonup&     "3128&1&28\cr
$\leftharpoondown$&leftharpoondown&   "3129&1&29\cr
$\rightharpoonup$&rightharpoonup&    "312A&1&2A\cr
$\rightharpoondown$&rightharpoondown&  "312B&1&2B\cr
$\smile$&smile&             "315E&1&5E\cr
$\frown$&frown&             "315F&1&5F\cr
$\asymp$&asymp&             "3210&2&10\cr
$\equiv$&equiv&             "3211&2&11\cr
$\subseteq$&subseteq&          "3212&2&12\cr
$\supseteq$&supseteq&          "3213&2&13\cr
$\leq$&leq&               "3214&2&14\cr
$\geq$&geq&               "3215&2&15\cr
$\preceq$&preceq&            "3216&2&16\cr
$\succeq$&succeq&            "3217&2&17\cr
$\sim$&sim&               "3218&2&18\cr
$\approx$&approx&            "3219&2&19\cr
$\subset$&subset&            "321A&2&1A\cr
$\supset$&supset&            "321B&2&1B\cr
$\ll$&ll&                "321C&2&1C\cr
$\gg$&gg&                "321D&2&1D\cr
$\prec$&prec&              "321E&2&1E\cr
$\succ$&succ&              "321F&2&1F\cr
$\leftarrow$&leftarrow&         "3220&2&20\cr
$\rightarrow$&rightarrow&        "3221&2&21\cr
$\leftrightarrow$&leftrightarrow&    "3224&2&24\cr
$\nearrow$&nearrow&           "3225&2&25\cr
$\searrow$&searrow&           "3226&2&26\cr
$\simeq$&simeq&             "3227&2&27\cr
$\Leftarrow$&Leftarrow&         "3228&2&28\cr
$\Rightarrow$&Rightarrow&        "3229&2&29\cr
$\Leftrightarrow$&Leftrightarrow&    "322C&2&2C\cr
$\nwarrow$&nwarrow&           "322D&2&2D\cr
$\swarrow$&swarrow&           "322E&2&2E\cr
$\propto$&propto&            "322F&2&2F\cr
$\in$&in&                "3232&2&32\cr
$\ni$&ni&                "3233&2&33\cr
$\not$&not&               "3236&2&36\cr
$\mapstochar$&mapstochar&        "3237&2&37\cr
$\perp$&perp&              "323F&2&3F\cr
$\vdash$&vdash&             "3260&2&60\cr
$\dashv$&dashv&             "3261&2&61\cr
$\mid$&mid&               "326A&2&6A\cr
$\parallel$&parallel&          "326B&2&6B\cr
$\sqsubseteq$&sqsubseteq&        "3276&2&76\cr
$\sqsupseteq$&sqsupseteq&        "3277&2&77\cr
\end{longtable}


\subsection{Miscellanneous formulae}
\label{sec:misc}

Taken from~\cite{Downes04:amsart}
\begin{displaymath}
  \hbar\nu=E,\qquad \hbar\ne\pi, \qquad \partial j, \qquad x^j, \qquad x^l
\end{displaymath}

Some other other equations:  $\sum^J a'$,  $r^a$ and $D^k$.


Let $\mathbf{A}=(a_{ij})$ be the adjacency matrix of graph $G$. The
corresponding Kirchhoff matrix $\mathbf{K}=(k_{ij})$ is obtained from
$\mathbf{A}$ by replacing in $-\mathbf{A}$ each diagonal entry by the
degree of its corresponding vertex; i.e., the $i$th diagonal entry is
identified with the degree of the $i$th vertex. It is well known that
\begin{equation}
\det\mathbf{K}(i|i)=\text{ the number of spanning trees of $G$},
\quad i=1,\dots,n
\end{equation}
where $\mathbf{K}(i|i)$ is the $i$th principal submatrix of
$\mathbf{K}$.

\newcommand{\abs}[1]{\left\lvert#1\right\rvert}
\newcommand{\wh}{\widehat}
Let $C_{i(j)}$ be the set of graphs obtained from $G$ by attaching edge
$(v_iv_j)$ to each spanning tree of $G$. Denote by $C_i=\bigcup_j
C_{i(j)}$. It is obvious that the collection of Hamiltonian cycles is a
subset of $C_i$. Note that the cardinality of $C_i$ is $k_{ii}\det
\mathbf{K}(i|i)$. Let $\wh X=\{\hat x_1,\dots,\hat x_n\}$.  Define multiplication for the elements of $\wh X$ by
\begin{equation}\label{multdef}
\hat x_i\hat x_j=\hat x_j\hat x_i,\quad \hat x^2_i=0,\quad
i,j=1,\dots,n.
\end{equation}
Let $\hat k_{ij}=k_{ij}\hat x_j$ and $\hat k_{ij}=-\sum_{j\not=i} \hat
k_{ij}$. Then the number of Hamiltonian cycles $H_c$ is given by the
relation
\begin{equation}\label{H-cycles}
\biggl(\prod^n_{\,j=1}\hat x_j\biggr)H_c=\frac{1}{2}\hat k_{ij}\det
\wh{\mathbf{K}}(i|i),\qquad i=1,\dots,n.
\end{equation}
The task here is to express \eqref{H-cycles}
in a form free of any $\hat x_i$,
$i=1,\dots,n$. The result also leads to the resolution of enumeration of
Hamiltonian paths in a graph.

It is well known that the enumeration of Hamiltonian cycles and paths
in a complete graph $K_n$ and in a complete bipartite graph
$K_{n_1n_2}$ can only be found from \textit{first combinatorial
  principles}. One wonders if there exists a formula which can be used
very efficiently to produce $K_n$ and $K_{n_1n_2}$. Recently, using
Lagrangian methods, Goulden and Jackson have shown that $H_c$ can be
expressed in terms of the determinant and permanent of the adjacency
matrix. However, the formula of Goulden and
Jackson determines neither $K_n$ nor $K_{n_1n_2}$ effectively. In this
paper, using an algebraic method, we parametrize the adjacency matrix.
The resulting formula also involves the determinant and permanent, but
it can easily be applied to $K_n$ and $K_{n_1n_2}$. In addition, we
eliminate the permanent from $H_c$ and show that $H_c$ can be
represented by a determinantal function of multivariables, each
variable with domain $\{0,1\}$. Furthermore, we show that $H_c$ can be
written by number of spanning trees of subgraphs. Finally, we apply
the formulas to a complete multigraph $K_{n_1\dots n_p}$.

The conditions $a_{ij}=a_{ji}$, $i,j=1,\dots,n$, are not required in
this paper. All formulas can be extended to a digraph simply by
multiplying $H_c$ by 2.

The boundedness, property of $\Phi_ 0$, then yields
\[\int_{\mathcal{D}}\abs{\overline\partial u}^2e^{\alpha\abs{z}^2}\geq c_6\alpha
\int_{\mathcal{D}}\abs{u}^2e^{\alpha\abs{z}^2}
+c_7\delta^{-2}\int_ A\abs{u}^2e^{\alpha\abs{z}^2}.\]

Let $B(X)$ be the set of blocks of $\Lambda_{X}$
and let $b(X) = \abs{B(X)}$. If $\phi \in Q_{X}$ then
$\phi$ is constant on the blocks of $\Lambda_{X}$.
\begin{equation}\label{far-d}
 P_{X} = \{ \phi \in M \mid \Lambda_{\phi} = \Lambda_{X} \},
\qquad
Q_{X} = \{\phi \in M \mid \Lambda_{\phi} \geq \Lambda_{X} \}.
\end{equation}
If $\Lambda_{\phi} \geq \Lambda_{X}$ then
$\Lambda_{\phi} = \Lambda_{Y}$ for some $Y \geq X$ so that
\[ Q_{X} = \bigcup_{Y \geq X} P_{Y}. \]
Thus by M\"obius inversion
\[ \abs{P_{Y}}= \sum_{X\geq Y} \mu (Y,X)\abs{Q_{X}}.\]
Thus there is a bijection from $Q_{X}$ to $W^{B(X)}$.
In particular $\abs{Q_{X}} = w^{b(X)}$.


\renewcommand{\arraystretch}{2.2}
\[W(\Phi)= \begin{Vmatrix}
\dfrac\varphi{(\varphi_1,\varepsilon_1)}&0&\dots&0\\
\dfrac{\varphi k_{n2}}{(\varphi_2,\varepsilon_1)}&
\dfrac\varphi{(\varphi_2,\varepsilon_2)}&\dots&0\\
\hdotsfor{5}\\
\dfrac{\varphi k_{n1}}{(\varphi_n,\varepsilon_1)}&
\dfrac{\varphi k_{n2}}{(\varphi_n,\varepsilon_2)}&\dots&
\dfrac{\varphi k_{n\,n-1}}{(\varphi_n,\varepsilon_{n-1})}&
\dfrac{\varphi}{(\varphi_n,\varepsilon_n)}
\end{Vmatrix}\]



\bibliography{jamtimes}
\bibliographystyle{unsrt}



\end{document}
