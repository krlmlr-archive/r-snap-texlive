% STOLEN VERSION of ``zguide.tex''  11th December 1990
% Copyright (C) J. M. Spivey 1990 

% The zed style option and its documentation may be freely copied,
% distributed and used for any purpose except direct commercial gain,
% provided that they are copied and distributed as a whole and without
% modification. The author accepts no liablility for their accuracy or
% fitness for any purpose.

\documentclass{article}

\usepackage{zed-csp}

\def\fuzz{{\large\it f\kern0.1em}{\normalsize\sc uzz}}
\let\Fuzz=\fuzz
\def\ZRM{{\rm ZRM}}

\makeatletter % Oh-ho: here comes some more lark's vomit ...

\def\demo{\par\vbox\bgroup\begingroup\quote}
\def\gives{\endquote\endgroup\egroup}
\def\enddemo{\global\@ignoretrue}

% Try doing this in pure LaTeX without `visual formatting'!
\def\symtab{\setbox0=\vbox\bgroup \def\\{\cr}
        \halign\bgroup\strut$##$\hfil&\quad##\hfil\cr}
\def\endsymtab{\crcr\egroup\egroup
	\dimen0=\ht0 \divide\dimen0 by2 \advance\dimen0 by\ht\strutbox
	\splittopskip=\ht\strutbox \vbadness=10000
 	\predisplaypenalty=0
	$$\halign{##\cr\hbox to\linewidth{% 
		\valign{##\vfil\cr
			\setbox1=\vsplit0 to\dimen0 \unvbox1\cr 
			\noalign{\hfil}\unvbox0\cr
			\noalign{\hfil}}}\cr 
		\noalign{\prevdepth=\dp\strutbox}}$$
	\global\@ignoretrue}

\makeatother % Sanity is restored

\begin{document}

\title{A guide to the {\tt zed} style option}
\author{Mike Spivey}
\date{December 1990}
\maketitle

\section{Introduction}

This document is a guide to the version of the {\tt zed} style option
for \LaTeX\ dated 11th December 1990.  This new version of the style
option is fully compatible with the {\tt fuzz} style option
distributed with the my \fuzz\ type-checker for Z, but uses two fonts
from the AMS in place of the special Z font distributed with
\fuzz.  Some of the symbols have been cobbled together by combining
two or more characters, but the results are good enough for rough
drafts. The style option requires the `old' AMS fonts, and will not at
present work with Sli\TeX\ or the Sch\"opf-Mittelbach font selection
scheme.

The rest of this guide is mostly extracted from the manual for \fuzz,
and it assumes a basic knowledge of \LaTeX.  I have not removed some
information about how the \fuzz\ type-checker treats various
constructs, in case you later want to type-check a document you have
formatted with the {\tt zed} style option. For information about the
\fuzz\ package and how to order it, see the end of this guide.

This guide and the {\tt zed} style option itself may be freely copied,
distributed and used for any purpose except direct commercial gain,
provided that they are copied and distributed as a whole and without
modification. The author accepts no liablility for their accuracy or
fitness for any purpose.

\section{Loading the {\tt zed} style option}

Every \LaTeX\ document should begin with a \verb/\documentstyle/
command. If the document contains a Z specification, this command
should include the style option \verb/zed/.  For example:
\begin{quote}
	\verb/\documentstyle[12pt,zed]{article}/
\end{quote}
Including \verb/zed/ as a style option loads macros from the file
\verb/zed.sty/ and also loads four fonts of extra mathematical symbols
called \verb/msxm9/, \verb/msym9/, \verb/msxm10/, and \verb/msym10/.
Your \LaTeX\ installation must have these fonts for the \verb/zed/
style option to work; if it doesn't, or they are in the wrong place to
be found by \TeX, then you will get an error message like this:
\begin{quote}
	\verb/! Font \ninsxm=msxm9 not loadable .../
\end{quote}
The \verb/zed/ style option can be used with any of the standard
\LaTeX\ styles, and it can appear either before or after the type-size
option if one is used. It can be combined with most of the standard
style options, but it should not be combined with \verb/fleqn/,
because {\tt zed} already makes provision for setting mathematics
flush left. At present, \verb/zed/ does not work with Sli\TeX.

\section{Making boxes}

To print a schema, use the \verb/schema/ environment. Here is an
example, showing first the input, then the output from \LaTeX:
\begin{demo}
\begin{verbatim}
\begin{schema}{PhoneDB}
    known: \power NAME \\ phone: NAME \pfun PHONE
\where
    known = \dom phone
\end{schema}
\end{verbatim}
\gives
\begin{schema}{PhoneDB}
    known: \power NAME \\ phone: NAME \pfun PHONE
\where
    known = \dom phone
\end{schema}
\end{demo}
The name of the schema appears as an argument to the environment, and
the horizontal dividing line between declarations and predicates is
indicated by \verb/\where/. Successive lines in the declaration and
predicate parts are separated by the command \verb/\\/. In this
example, the Z symbols `$\power$', `$\pfun$' and `$\dom$' have been
entered as the commands \verb/\power/, \verb/\pfun/ and \verb/\dom/:
for a complete list of these commands, see Section~\ref{s:symtabs}
below.

Like the \verb/displaymath/ environment of \LaTeX, the \verb/schema/
environment (and the others we shall come to in a moment) can appear
in the middle of a paragraph, and ordinarily should have no blank
lines either before or after it. Blank lines before the environment
are ignored, but blank lines afterwards cause the following text to
begin a new paragraph.

For a schema without a predicate part, the command \verb/\where/ is
simply omitted, as in the following example:
\begin{demo}
\begin{verbatim}
\begin{schema}{Document[CHAR]}
    left, right: \seq CHAR
\end{schema}
\end{verbatim}
\gives
\begin{schema}{Document[CHAR]}
    left, right: \seq CHAR
\end{schema}
\end{demo}
This example also shows how to set schemas with generic parameters.

For axiomatic descriptions, the \verb/axdef/ environment is used.
Here is an example:
\begin{demo}
\begin{verbatim}
\begin{axdef}
    limit: \nat
\where
    limit \leq 65535
\end{axdef}
\end{verbatim}
\gives
\begin{axdef}
    limit: \nat
\where
    limit \leq 65535
\end{axdef}
\end{demo}
In both kinds of box, predicates and declarations can be split between
lines before or after infixed symbols, as shown in the following
example:
\begin{demo}
\begin{verbatim}
\begin{axdef}
    policy: \power_1 RESOURCE \fun RESOURCE
\where
    \forall S: \power_1 RESOURCE @ \\
\t1         policy(S) \in S
\end{axdef}
\end{verbatim}
\gives
\begin{axdef}
    policy: \power_1 RESOURCE \fun RESOURCE
\where
    \forall S: \power_1 RESOURCE @ \\
\t1         policy(S) \in S
\end{axdef}
\end{demo}
The strange hint \verb/\t1/ in this example makes the corresponding
line in the output have one helping of indentation.  As things get
more nested, you can say \verb/\t2/, \verb/\t3/, and so on.  But if
you should ever get beyond \verb/t9/, you'll need to use braces around
the argument: \verb/\t{10}/, and you'd better look for some way to
simplify your specification!

This system of tab stops is a little crude, but it is easier to use
than the alternatives, and usually gives acceptable results. The
\verb/\t/$n$ commands are completely ignored by the type-checker, 
so you are free to use them as you like to improve the look of your
specification.  The size of `helping' you get with
\verb/\t/$n$ is a style parameter \verb/\zedindent/, and the default
is 2em.

For generic definitions, there's the \verb/gendef/ environment: for
example,
\begin{demo}
\begin{verbatim}
\begin{gendef}[X,Y]
    first: X \cross Y \fun X
\where
    \forall x: X; y: Y @ \\
\t1     first(x,y) = x
\end{gendef}
\end{verbatim}
\gives
\begin{gendef}[X,Y]
    first: X \cross Y \fun X
\where
    \forall x: X; y: Y @ \\
\t1     first(x,y) = x
\end{gendef}
\end{demo}
In this environment, the formal generic parmeters are an optional
argument. Omitting this argument results in a box with a solid double
bar at the top, which can be used for uniquely defining non-generic
constants.

If a schema or other box contains more than one predicate below the
line, it often looks better to add a small vertical space between
them.  This can be done with the command \verb/\also/:
\begin{demo}
\begin{verbatim}
\begin{schema}{AddPhone}
    \Delta PhoneDB \\ name?: NAME \\ number?: PHONE
\where
    name? \notin known
\also
    phone' = phone \oplus \{name? \mapsto number?\}
\end{schema}
\end{verbatim}
\gives
\begin{schema}{AddPhone}
    \Delta PhoneDB \\ name?: NAME \\ number?: PHONE
\where
    name? \notin known
\also
    phone' = phone \oplus \{name? \mapsto number?\}
\end{schema}
\end{demo}

Some Z paragraphs do not appear in boxes, and for these the \verb/zed/
environment is used:
\begin{demo}
\begin{verbatim}
\begin{zed}
    [NAME, DATE]
\also
    REPORT ::= ok | unknown \ldata NAME \rdata
\also
    \exists n: NAME @ \\
\t1     birthday(n) \in December.
\end{zed}
\end{verbatim}
\gives
\begin{zed}
    [NAME, DATE]
\also
    REPORT ::= ok | unknown \ldata NAME \rdata
\also
    \exists n: NAME @ \\
\t1     birthday(n) \in December.
\end{zed}
\end{demo}
This environment should be used for basic type definitions,
constraints, abbreviation definitions, free type definitions, and the
horizontal form of schema definitions. As the example illustrates, a
full stop or comma is allowed just before the closing \verb/\end/
command of any of the Z environments, if that suits your taste (or is
forced on you by a publisher's house rules).  This punctuation is
ignored by the type-checker.

For large free type definitions, the \verb/syntax/ environment
provides a useful alternative to the \verb/zed/ environment, as the
following example suggests:
\begin{demo}
\begin{verbatim}
\begin{syntax}
    OP & ::= & plus | minus | times | divide
\also
    EXP & ::= & const \ldata \nat \rdata \\ 
        & | & binop \ldata OP \cross EXP \cross EXP \rdata
\end{syntax}
\end{verbatim}
\gives
\begin{syntax}
    OP & ::= & plus | minus | times | divide
\also
    EXP & ::= & const \ldata \nat \rdata \\ & | & binop \ldata OP
\cross EXP \cross EXP \rdata
\end{syntax}
\end{demo}
Just as in the \verb/eqnarray/ environment of \LaTeX, the fields are
separated by \verb/&/ characters, and these are ignored by the
type-checker.

\section{Inside the boxes}\label{s:symtabs}

The first thing to notice about the text inside the boxes is that
multi-character identifiers look better than they do with ordinary
\LaTeX: instead of $\mit specifications$, you get $specifications$.
The letters are not spread apart, and ligatures like $fi$ are used.
This is achieved by an adjustment to the way \TeX\ treats letters in
mathematical formulas, and no special commands are needed in the input
file. Embedded underline characters can be set with the \verb/\_/
command, which is also used for dummy arguments of operators:
\verb/not\_known/ gives $not\_known$, and \verb/\_ + \_/ gives 
$\_ + \_$~.

The various special symbols of the Z language and library have
mnemonic names. Some of these names are the same as in ordinary
\LaTeX, and some symbols have new names more suggestive of their
meaning in Z. The spaces inserted around the symbols have been
adjusted to make them look better in Z specifications.

A few symbols have two names, reflecting two different uses for the
symbol in Z. The symbol $\semi$ is called \verb/\semi/ when it is used
as an operation on schemas, and \verb/\comp/ when it is used for
composition of relations. The symbol $\hide$ is called \verb/\hide/ as
the hiding operator of the schema calculus, and \verb/\setminus/ for
the set difference operator. The symbol $\project$ is called
\verb/project/ as the schema projection operator, and \verb/\filter/
for filtering of sequences.  The spaces around the schema operations
are a little larger, and the type-checker recognizes each name only in
the appropriate context.

For most symbols, two attributes are of interest: the syntactic class
({\sf In-Fun}, \dots) assigned to it by the type-checker, and the kind
of symbol \LaTeX\ generates from it. The first of these affects the
parsing of an expression containing the symbol, and the second affects
the way spaces will be inserted when the expression is printed. In the
description below, `thin', `medium' and `thick' spaces are the same as
those produced by the \LaTeX\ commands \verb/\,/ and \verb/\:/ and
\verb/\;/ respectively.

Here are the mnemonics for the basic elements of the Z language:
\begin{symtab}
        \power & \verb/\power/ \\
	\cross & \verb/\cross/ \\
	\in & \verb/\in/ \\
	| & \verb/|/ or \verb/\mid/ \\
	@ & \verb/@/ or \verb/\spot/ \\
	\theta & \verb/\theta/ \\
	\lambda & \verb/\lambda/ \\
	\mu & \verb/\mu/ \\
	\Delta & \verb/\Delta/ \\
	\Xi & \verb/\Xi/ \\
	\defs & \verb/\defs/
\end{symtab}
The operators of propositional logic and the schema calculus are as
follows. Many of these names are already defined by \LaTeX, but the
spacing is often adjusted to make them look better in Z
specifications.
\begin{symtab}
        \lnot & \verb/\lnot/ \\
	\land & \verb/\land/ \\
	\lor & \verb/\lor/ \\
	\implies & \verb/\implies/ \\
	\iff & \verb/\iff/ \\
	\forall & \verb/\forall/ \\
	\exists & \verb/\exists/ \\
	\exists_1 & \verb/\exists_1/ \\
	\hide & \verb/\hide/ \\
	\project & \verb/\project/ \\
	\pre & \verb/\pre/ \\
	\semi & \verb/\semi/
\end{symtab}
Here are the various sorts of fancy brackets:
\begin{symtab}
        \{\ldots\} & \verb/\{ ... \}/ \\
	\langle\ldots\rangle & \verb/\langle ... \rangle/ \\
	\lbag\ldots\rbag & \verb/\lbag ... \rbag/ \\
	\ldata\ldots\rdata & \verb/\ldata ... \rdata/ \\
	\ldots\limg\ldots\rimg & \verb/... \limg ... \rimg/
\end{symtab}
Those are all the symbols `built-in' to the Z language; now for the
symbols defined as part of the mathematical tool-kit. First come the
symbols which are not defined as infix operators, etc.:
\begin{symtab}
        \empty & \verb/\empty/ \\
	\bigcup & \verb/\bigcup/ \\
	\bigcap & \verb/\bigcap/ \\
	\dom & \verb/\dom/ \\
	\ran & \verb/\ran/ \\
	\nat & \verb/\nat/ \\
	\num & \verb/\num/ \\
	\nat_1 & \verb/\nat_1/ \\
	 \# & \verb/\#/ \\
	\dcat & \verb/\dcat/
\end{symtab}
Here are the infix function symbols; they are defined in \LaTeX\ as
binary operators, so medium spaces are inserted automatically. The
type-checker recognizes them as of class {\sf In-Fun}. Each symbol is
shown with its priority:
\begin{symtab}
        \mapsto & \verb/\mapsto/\quad\hfill 1 \\
	\upto & \verb/\upto/\quad\hfill 2 \\
	+ & \verb/+/\quad\hfill 3 \\
	- & \verb/-/\quad\hfill 3 \\
	\cup & \verb/\cup/\quad\hfill 3 \\
	\setminus & \verb/\setminus/\quad\hfill 3 \\
	\cat & \verb/\cat/\quad\hfill 3 \\
	\uplus & \verb/\uplus/\quad\hfill 3 \\
	* & \verb/*/\quad\hfill 4 \\
	\div & \verb/\div/\quad\hfill 4 \\
	\mod & \verb/\mod/\quad\hfill 4 \\
	\cap & \verb/\cap/\quad\hfill 4 \\
	\comp & \verb/\comp/\quad\hfill 4 \\
	\circ & \verb/\circ/\quad\hfill 4 \\
	\filter & \verb/\filter/\quad\hfill 4 \\
	\oplus & \verb/\oplus/\quad\hfill 5 \\
	\dres & \verb/\dres/\quad\hfill 6 \\
	\rres & \verb/\rres/\quad\hfill 6 \\
	\ndres & \verb/\ndres/\quad\hfill 6 \\
	\nrres & \verb/\nrres/\quad\hfill 6
\end{symtab}
The postfix function symbols (class {\sf Post-Fun}) all produce
superscripts:
\begin{symtab}
        \inv & \verb/\inv/ \\
	\plus & \verb/\plus/ \\
	\star & \verb/\star/ \\
	\bsup n \esup & \verb/\bsup n \esup/
\end{symtab}
As an example, \verb/R \star/ is printed as $R \star$. For iteration,
the commands \verb/\bsup ... \esup/ should be used: for example,
\verb/R \bsup n \esup/ is printed as $R \bsup n \esup$. The
type-checker regards this formula as equivalent to $iter~n~R$, as
explained on page 112 of the \ZRM.

The infix relation symbols (class {\sf In-Rel}) are defined in
\LaTeX\ as relations, so thick spaces are inserted around them 
automatically:
\begin{symtab}
        \neq & \verb/\neq/ \\
	\notin & \verb/\notin/ \\
	\subseteq & \verb/\subseteq/ \\
	\subset & \verb/\subset/ \\
	< & \verb/</ \\
	> & \verb/>/ \\
	\leq & \verb/\leq/ \\
	\geq & \verb/\geq/ \\
	\partition & \verb/\partition/ \\
	\inbag & \verb/\inbag/ \\
\end{symtab}
There is only one prefix relation symbol (class {\sf Pre-Rel}). It
separates itself from an argument with a thick space:
\begin{symtab}
        \disjoint & \verb/\disjoint/
\end{symtab}
The infix generic symbols are seen by \LaTeX\ as relation symbols, so
they are surrounded by thick spaces. Of course, the type-checker
itself assigns them class {\sf In-Gen}:
\begin{symtab}
        \rel & \verb/\rel/ \\
	\pfun & \verb/\pfun/ \\
	\fun & \verb/\fun/ \\
	\pinj & \verb/\pinj/ \\
	\inj & \verb/\inj/ \\
	\psurj & \verb/\psurj/ \\
	\surj & \verb/\surj/ \\
	\bij & \verb/\bij/ \\
	\ffun & \verb/\ffun/ \\
	\finj & \verb/\finj/
\end{symtab}
Prefix generic symbols are assigned class {\sf Pre-Gen} by the
type-checker; in \LaTeX, they are defined as operator symbols, so that
a thin space is inserted between the symbol and a following generic
parameter:
\begin{symtab}
        \power_1 & \verb/\power_1/ \\
	\id & \verb/\id/ \\
	\finset & \verb/\finset/ \\
	\finset_1 & \verb/\finset_1/ \\
	\seq & \verb/\seq/ \\
	\seq_1 & \verb/\seq_1/ \\
	\iseq & \verb/\iseq/ \\
	\bag &
\verb/\bag/
\end{symtab}

\section{Fine points}

In math mode, which is used for type-setting the contents of Z boxes,
\TeX\ ignores all space characters in the input file. The spaces which
appear between elements of a mathematical formula are determined by
\TeX\ itself, working from information about the symbols in the
formula. Although this information has been adjusted in the {\tt zed}
style option to make Z texts look as balanced as possible, there are
one or two situations in which \TeX\ needs a little help.

Special care is needed when function application is indicated by
juxtaposing two identifiers, as in the expression $rev~words$. This
expression should be typed as %
\verb/rev~words/. Typing just %
\verb/rev words/ results in the output $rev words$, since \TeX\ ignores the
space separating the two identifiers.  In a formula, the character
%
\verb/~/ inserts the same amount of space as the \LaTeX\ %
\verb/\,/
command, but it looks better in the input file.  The type-checker
completely ignores both the %
\verb/~/ character and the \LaTeX\ spacing
commands, except that it issues a warning if it finds that a needed
one is missing, for example, between two identifiers.  It is not
necessary to separate symbols like %
\verb/\dom/ and %
\verb/\ran/ from
their arguments with a %
\verb/~/, because \TeX\ inserts the right
amount of space automatically. For example, the input %
\verb/\dom f/
produces $\dom f$.

It is good style also to insert small spaces inside the braces of a
set comprehension, as in this example:
\begin{demo}
%
\verb/\{~x: \nat | x \leq 10 @ x * x~\}/
\gives
\[\{~x: \nat | x \leq 10 @ x * x~\}\]
\end{demo}
This helps to distinguish it visually from a set display, which should
not have the space:
\begin{demo}
%
\verb/\{1, 2, 3\}/
\gives
\[\{1, 2, 3\}\]
\end{demo}
The space symbol %
\verb/~/ is ignored by the type-checker, so this is
purely a matter of appearance.  It also looks better if you add small
spaces inside the square brackets of `horizontal' schema texts.

\TeX\ also needs help when a binary operator appears at the end of a
line, as in the following example:
\begin{demo}
\begin{verbatim}
\begin{zed}
    directory' = directory \cup {} \\
\t3                 \{new\_name? \mapsto new\_number?\}
\end{zed}
\end{verbatim}
\gives
\begin{zed}
    directory' = directory \cup {} \\
\t3                 \{new\_name? \mapsto new\_number?\}
\end{zed}
\end{demo}
\TeX\ will not recognize %
\verb/\cup/ as a binary operator and insert
the correct space unless it is surrounded by two operands, so the
dummy operand %
\verb/{}/ has been inserted: this is ignored by the
type-checker. This problem affects only binary operators; relation
signs do not need to be surrounded by arguments to be recognized by
\TeX.

\section{Bits and pieces}

Specification documents often contain mathematical text which does not
form part of the specification proper. This section describes some
environments for setting various kinds of mathematics; they are
provided for convenience, and they are all ignored by the
type-checker. Besides these environments for making displays, run-in
mathematics can be set with the usual %
\verb/math/ environment, or with
the commands %
\verb/$ ... $/ or \hbox{%
\verb/\( ... \)/}. All the Z
symbols listed in Section~\ref{s:symtabs} can be used with these
commands.

The simplest display environment is provided by the commands
%
\verb/\[ ... \]/.  This form acts just like %
\verb/\begin{zed} ... \end{zed}/, except that the contents are ignored by the
type-checker. Here is an example:
\begin{demo}
\begin{verbatim}
\[
    \exists PhoneDB @ \\
\t1     known = \empty
\]       
\end{verbatim}
\gives
\[
    \exists PhoneDB @ \\
\t1     known = \empty
\]       
\end{demo}
These commands generalize the standard \LaTeX\ commands with the same
name, because the displayed material can be several lines. Note,
however, that the contents are set as text style rather than display
style mathematics.

A schema box with no name is generated by the %
\verb/schema*/
environment:
\begin{demo}
\begin{verbatim}
\begin{schema*}
        x, y: \nat
\where
        x > y
\end{schema*}
\end{verbatim}
\gives
\begin{schema*}
        x, y: \nat
\where
        x > y
\end{schema*}
\end{demo}
This form is often useful for showing the result of expanding a
complex schema-expression.

Another kind of mathematical display is provided by the %
\verb/argue/
environment.  This is like the %
\verb/zed/ environment, but the
separation between lines is increased a little, and page breaks may
occur between lines.  The intended use is for arguments like this:
\begin{demo}
\begin{verbatim}
\begin{argue}
    S \dres (T \dres R) \\
\t1     = \id S \comp \id T \comp R \\
\t1     = \id (S \cap T) \comp R & law about $\id$ \\
\t1     = (S \cap T) \dres R.
\end{argue}
\end{verbatim}
\gives
\begin{argue}
    S \dres (T \dres R) \\
\t1     = \id S \comp \id T \comp R \\
\t1     = \id (S \cap T) \comp R & law about $\id$ \\
\t1	= (S \cap T) \dres R.
\end{argue}
\end{demo}
When the left-hand side is long, I find this style better than the
\LaTeX\ %
\verb/eqnarray/ style, which wastes a lot of space. The second
field on each line is optional. Again, the %
\verb/argue/ environment is
ignored by the type-checker.

Finally, there is the %
\verb/infrule/ environment, used for inference
rules:
\begin{demo}
\begin{verbatim}
\begin{infrule}
    \Gamma \shows P
\derive[x \notin freevars(\Gamma)]
    \Gamma \shows \forall x @ P
\end{infrule}
\end{verbatim}
\gives
\begin{infrule}
    \Gamma \shows P
\derive[x \notin freevars(\Gamma)]
    \Gamma \shows \forall x \spot P
\end{infrule}
\end{demo}
The horizontal line is generated by %
\verb/\derive/; the optional
argument is a side-condition of the rule.

\section{Style parameters}

A few style parameters affect the way Z text is set out; they can be
changed at any time if your taste doesn't match mine.

\begin{description}
\item[\tt\string\zedindent] The indentation for mathematical text.
        By default, this is the same as %
\verb|\leftmargini|, the indentation
        used for list environments.
\item[\tt\string\zedleftsep] The space between the vertical line on the left
        of schemas, etc., and the maths inside. The default is 1em.
\item[\tt\string\zedtab] The unit of indentation used by %
\verb|\t|. The default
        is 2em.
\item[\tt\string\zedbar] The length of the horizontal bar in the middle of a
        schema. The default is 6em.
\item[\tt\string\zedskip] The vertical space inserted by
        %
\verb/\also/. By default, this is the same as that inserted by
        %
\verb/\medskip/.
\end{description}

\newpage
\section{The fuzz package}

The \fuzz\ package consists of two parts -- a style option compatible
with the {\tt zed} style option described here, and an analysis and
checking program.  Using \fuzz\ together with \LaTeX, you can:
\begin{itemize}
\item   Input Z specifications as ordinary ASCII files.
\item   Process them for laser printing or photo-typesetting.
\item   Check them for conformance with the Z language rules.
\item   Produce a listing showing the schemas in the specification with
        components and their types.
\end{itemize}
The \fuzz\ analysis program works on the same ASCII file as \LaTeX; it
extracts the formal text and checks it for conformance with the rules
of the Z language, producing clear error messages.  Analysis of a
1300-line specification takes about 7 seconds on a SUN 3/75.

The \fuzz\ distribution contains the \LaTeX\ style option, a special
font of Z symbols, object code for the analysis program, a library
containing the standard mathematical tool-kit, and some example
specifications. To use \fuzz, you will need to have \LaTeX\ installed
on your machine, but everything else you need is included. \fuzz\ is
currently available under `no-nonsense' licence conditions for the IBM
PC and other DOS machines, and for the SUN 3 and SUN 4 under SUN UNIX.
The PC version can also be used on the PS/2.  We are willing to
produce versions for other machines according to demand.  

\newpage
\thispagestyle{empty}
\section*{Ordering information}

You can order the \fuzz\ package either by cutting out the coupon
below and sending it with your payment, or by sending an official
order -- we will send an invoice.  Please send all orders to the
address below.  Technical enquiries can be sent to Mike Spivey at the
same address, or by E-mail to {\tt mike@uk.ac.oxford.prg} .

\vfill
{\center\large\bf \Fuzz\ package: order form\par}
\bigskip
\noindent To: Mrs.\ A. Spivey, 34, Westlands Grove, Stockton Lane, 
\hfil\break\hphantom{To: }York, YO3 0EF, England.
\par\bigskip
\newdimen\x \setbox0=\hbox{Telephone:} \x=\wd0
\begingroup
\baselineskip=24pt
\def\addrline#1{\hbox to\linewidth{\hbox to\x{#1\hfil}\qquad\hrulefill}}
\addrline{Name:}
\addrline{Address:}
\addrline{}
\addrline{}
\addrline{Telephone:}
\endgroup
\bigskip
\noindent Please send [\qquad] copies of the \fuzz\ package for the 
following machines:
\par\bigskip
\halign{\hskip\x\qquad#\hfil\qquad&#\qquad\hfil&#\hfil\cr
        [\quad]& SUN 3 version: Cartridge tape& \pounds 300\cr
        \noalign{\medskip}
        [\quad]& SUN 4 version: 3.5in disk&     \pounds 275\cr
        \noalign{\medskip}
        [\quad]& SUN 4 version: Cartridge tape& \pounds 300\cr
        \noalign{\medskip}
        [\quad]& IBM PC version: 5.25in disk&   \pounds 200\cr
        \noalign{\medskip}
        [\quad]& IBM PC version: 3.5in disk&    \pounds 200\cr}
\bigskip
\noindent I enclose a cheque for \pounds [\qquad], payable to 
Dr.\ J. M.~Spivey.
\par\bigskip\noindent
Signed:
\par\bigskip
\end{document}
