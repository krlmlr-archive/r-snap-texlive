\documentclass[a4paper]{article}
\usepackage[dvips,a4paper]{geometry}
\providecommand\cs[1]{\texttt{\char`\\#1}}
\begin{document}
\title{The \texttt{fixfoot} package\thanks{This is version 0.3a, of
    2007-12-12}}
\author{Robin Fairbairns}
\maketitle

\section{What is it for?}

When editing an edition of my father's letters, after his death, I
found I was making the same comment, in footnotes, rather a lot.  (For
example, he had a way of mis-spelling, or jumbling words for comic
effect; to point out that these \emph{weren't} transcription errors, I
marked them all with a footnoted ``\emph{Sic}''.)

Since these footnotes made tedious reading at the best of times, and
since sometimes those ``sic'' footnotes appeared several times on one
page, I conceived the idea of this package: it ensures that a
particular (pre-declared) footnote will only appear once on any page.

\section{Using the thing}

The package defines only one command, \cs{DeclareFixedFootnote}; this
declares a footnote to be repeated.  Apart from an optional
\texttt{*}, it takes two arguments: the command you'll use to call up
the footnote, and the text of each footnote generated.
If the optional \texttt{*} is present, the command you've created will
end with \cs{xspace}, so as to save you from the onerous business of
escaping the space after the footnote invocation.

A couple of examples:
\begin{quote}
\begin{verbatim}
\DeclareFixedFootnote*{\sic}{\emph{sic}}
\DeclareFixedFootnote{\prooflater}{This theorem will be
                                   proved later}
\end{verbatim}
\end{quote}

\section{How does it do it?}

The package works by exchanging information with itself, through the
medium of the \texttt{.aux} file (like labels, and per-page footnotes
in the \textsf{footmisc} package).  I never got around to making it
tell you to re-run the document when you need to (as labels and those
per-page footnotes do).
\end{document}
