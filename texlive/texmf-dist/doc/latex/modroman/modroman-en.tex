%%
%% This is file `modroman-en.tex',
%% generated with the docstrip utility.
%%
%% The original source files were:
%%
%% modroman.dtx  (with options: `doc,en')
%% 
%% This is a generated file.
%% 
%% Copyright (C) 2003, 2004, 2008, 2010 by Yvon Henel,
%% dit �le TeXnicien de surface� <le.texnicien.de.surface@wanadoo.fr>
%% 
%% This file may be distributed and/or modified under the conditions of
%% the LaTeX Project Public License, either version 1.2 of this license
%% or (at your option) any later version.  The latest version of this
%% license is in:
%% 
%%    http://www.latex-project.org/lppl.txt
%% 
%% and version 1.2 or later is part of all distributions of LaTeX version
%% 1999/12/01 or later.
%% 
\documentclass[a4paper]{ltxdoc}
\usepackage[latin1]{inputenc}
\usepackage[T1]{fontenc}
\usepackage{modroman}[2010/04/09]
\usepackage{xspace}
\usepackage{array}
\usepackage{ifthen}
\usepackage{MODRdoctools}
\IfFileExists{lmodern.sty}{\usepackage{lmodern}}{}
\usepackage[frenchb,english]{babel}
\usepackage[colorlinks=true,
            linkcolor=blue,
            urlcolor=blue,
            citecolor=blue]{hyperref}

\begin{document}
\renewcommand\thepage{\textit{\modroman{page}}}
\title{The package \textsf{modroman}\thanks{
    This document corresponds to the file
    \textsf{modroman}~\fileversion, dated \filedate.}}
\author{\href{mailto:le.texnicien.de.surface@wanadoo.fr}{Le \TeX
    nicien de surface}}

\maketitle

\begin{abstract}
\begin{otherlanguage}
{french}
\bfseries{}
  Documentation anglaise pour l'utilisateur final de l'extension
  \Pkg{modroman}.
  La documentation fran�aise est disponible sous le nom de
  \texttt{modroman-fr}.
\end{otherlanguage}

\vspace{6pt}
\noindent\hrulefill
\vspace{6pt}

This is the English documentation of \Pkg{modroman} for the final
user.

\MODRdescroen
\end{abstract}

\tableofcontents{}

\section{Introduction}

\subsection{Motivation}

The ways of writing numbers with roman numerals are more diverse than
it could be thought when one considers the \TeX{} \cs{romannumeral}
and \LaTeX{} \cs{roman} and \cs{Roman}. Many other forms were used at
a time or another through history. I saw, many years ago, the form
\nbshortroman{8} in a manuscript. The first versions \TO from 0.1 to
0.4\TF of this package provided just that form, with, user willing,
the use of a~|u| to denote a group of~5 as in `xuij' for~17.

The interested reader could cast a glance on the page wikipedia
devotes to roman numerals to see that the world is not always as
simple as one would like it to be.

\subsection{
Technical Remarks
}

The code of the first versions enable one to number the pages with
\emph{modified} roman numerals but one could not then use the
reference tools such as \cs{label} and \cs{ref} or even hope to see
the page numbers correctly written in the table of contents.

As the years go by I don't really become more clever but, for I read
not a few documentations of packages, I end with knowing a bit more
and I happen to stumble upon the solution to a problem I had just
caught a glimpse of.

I don't forget what I owe to T.~\textsc{Lachand}-\textsc{Robert}
in~\cite{tlachand} -- numerous ideas, detailed examples, clear
explanations -- but I now use other sources --
\texttt{source2e}~\cite{source2e} to be accurate. That is where I have
found the \emph{trick} which enables me to provide this new version of
\Pkg{modroman}. The reader who would like to know more should have a
look at the definition of the macro \cs{Roman} and its auxiliary
macros.

While rewriting the code I happen to understand that I was able to go
a bit farther than I have gone previously without a tremendous extra
cost. That's why one will now find more macros and more presentations
of the roman numerals and an additional macro.

This version~1 keeps the compatibility with the previous version but
the code has been completely rewritten and the package now provides to
the user in addition to \cs{modroman} and \cs{modromannumeral} about
fifteen other macros.

However, the main change introduced by this version~1 is the fact that
from now on \cs{modroman} \TO and its pals \TF is purely expandable
\TO see page~\pageref{puredev} for more details. One can then use it
to number pages and obtain the correct thing in the table of content
-- which was not the case until now. Caution, I do \emph{not} say that
\cs{modromannumeral} is purely expandable -- it is not.

\subsection{
Purely Expandable Macros
}

One could, if one understands French, read the thread
`\foreignlanguage{frenchb}{test de d�veloppabilit� pure ?}' on the
news group \texttt{fr.comp.text.tex} to see that that notion is not as
simple as one could think at n-th sight \texttt{:-)} however, here,
when I will say that a macro is `purely expandable' I will understand
what follows.

Let's assume that the macro \cs{thing} is such that
\cs{thing}\marg{12} gives `xij' \TO does it ring a bell?\TF
then\label{puredev}
\begin{enumerate}
\item the macro \cs{THING} defined by
  |\edef|\BOP|\TRUC|\BOP|{|\BOP|\truc|\BOP|{|\BOP|12|\BOP|}|\BOP|}|
  is such that:
  \begin{enumerate}
  \item \cs{THING} gives `xij' and
  \item |\meaning\THING| gives `\texttt{macro:->xij}'
  \end{enumerate}
\item moreover, if one defines \cs{Axij} then the construct
  |\csname| |A|\BOP|\truc|\BOP|{|\BOP|12|\BOP|}|\BOP|\endcsname|
  truly calls the macro \cs{Axij}.
\end{enumerate}

\section{
Usage
}

\subsection{
The Macros
}

Macros the name of which ends with \texttt{numeral} are to be used as
\TeX{} \cs{romannumeral}. They must be followed by a number and eat
the spaces which are after it, e.~g.
|\longromannumeral| \verb*+368  AND+ gives `\longromannumeral
368   AND'.

Macros the name of which begins with \cs{nb} take a number as argument
such as |\nbLongRoman{127}| which gives `\nbLongRoman{127}'.

Macros the name of which doesn't begin with \cs{nb} but ends with
\texttt{roman} are used as \LaTeX{} \cs{roman}: their only argument is
the name of a counter. With |\newcounter{machin}|,
|\setcounter{machin}{124}|, |\shortroman{machin}| one obtains
`\nbshortroman{124}'.

In what follows \meta{nbr} denotes a number, \meta{ctr} denotes the
\LaTeX{} name of a counter such as \texttt{page} or \texttt{chapter}.

Here comes a presentation of all the macros available with this
package. They are grouped by family where a family is defined with
respect to the obtained presentation of roman numerals.

After the macro's syntax, there will be \PD to mean that the macro is
purely expandable \TO see page~\pageref{puredev}\TF, \LT to mean that
it is used the \LaTeX{} way, \TX to mean that it is used as the \TeX{}
\cs{romannumeral}.

The examples are governed by the default options: \Option{jfinal},
\Option{vpourv}, \Option{court}, \Option{min}.

\subsubsection{
shortroman
Family
}

\begin{macro}{\shortroman}
  \cs{shortroman}\marg{ctr} \PD \LT
\end{macro}

\begin{macro}{\shortromannumeral}
  \cs{shortromannumeral} \meta{nbr} \TX
\end{macro}

\begin{macro}{\nbshortroman}
    \cs{nbshortroman}\marg{nbr} \PD \LT
\end{macro}

\Exemples{shortroman}

\subsubsection{
longroman
Family
}

\begin{macro}{\longroman}
  \cs{longroman}\marg{ctr} \PD \LT
\end{macro}

\begin{macro}{\longromannumeral}
  \cs{longromannumeral} \meta{nbr} \TX
\end{macro}

\begin{macro}{\nblongroman}
    \cs{nblongroman}\marg{nbr} \PD \LT
\end{macro}

\Exemples{longroman}

\subsubsection{
LongRoman
Family
}

\begin{macro}{\LongRoman}
  \cs{LongRoman}\marg{ctr} \PD \LT
\end{macro}

\begin{macro}{\LongRomannumeral}
  \cs{LongRomannumeral} \meta{nbr} \TX
\end{macro}

\begin{macro}{\nbLongRoman}
    \cs{nbLongRoman}\marg{nbr} \PD \LT
\end{macro}

\Exemples{LongRoman}

\subsubsection{
roman
Family
}

\TeX{} provides \cs{romannumeral} and \LaTeX{} \cs{roman}. I
complete the family with \cs{nbroman}.

\begin{macro}{\nbroman}
    \cs{nbroman}\marg{nbr} \PD \LT
\end{macro}

\Exemples{roman}

\subsubsection{
Roman
Family
}

\LaTeX{} provides \cs{Roman}. I complete the family with \cs{nbRoman} and
\cs{Romannumeral}.

\begin{macro}{\Romannumeral}
  \cs{Romannumeral} \meta{nbr} \TX
\end{macro}

\begin{macro}{\nbRoman}
    \cs{nbRoman}\marg{nbr} \PD \LT
\end{macro}

\Exemples{Roman}

\subsubsection{
modroman
Family
}

The output of the macros \cs{modroman}, \cs{modromannumeral}, and
\cs{nbmodroman} is determined by the chosen options. By default:

\begin{macro}{\modroman}
  \cs{modroman}\marg{ctr} $=$ \cs{shortroman}\marg{ctr} \PD \LT
\end{macro}

\begin{macro}{\modromannumeral}
  \cs{modromannumeral}\meta{nbr} $=$ \cs{shortromannumeral}\meta{nbr} \TX
\end{macro}

\begin{macro}{\nbmodroman}
    \cs{nbmodroman}\marg{nbr} $=$ \cs{nbshortroman}\marg{nbr} \PD \LT
\end{macro}

\Exemples{modroman}

\subsubsection{
Other macros
}

One can redefine the behaviour of families \cs{shortroman} and
\cs{longroman} with the macro \cs{RedefineMRmdclxvij}.

\begin{macro}{\RedefineMRmdclxvij}
  \cs{RedefineMRmdclxvij}\oarg{ISOL}\BOP\marg{M}\BOP\marg{D}\BOP\marg{C}%
  \BOP\marg{L}\BOP\marg{X}\BOP\marg{V}\BOP\marg{I}\BOP\marg{J}
\end{macro}

The above arguments determine the look of the roman numerals produced
after \TO one should take care of the side effects and one would have
to limit the scope of the redefinition ot a group if necessary\TF by
the families \cs{shortroman}, \cs{longroman}, and, if it is linked to
one of the two preceding, \cs{modroman}.

The optional argument \meta{ISOL} gives the look of the isolated |i| \CAD the
number~1. If one doesn't give the argument the look of the isolated
|i| is the look of the non-final |i| determined by \meta{I}

The look of the final~|i| is given by the argument \meta{J}. All other
arguments give the look of the corresponding (lowercase) digit,
e.~g. \meta{M} gives the look of~|m|.

\begin{macro}{\printntimes}
  \cs{printntimes}\marg{nbr}%
\marg{text}
\end{macro}

Here are two examples using the macro.

|$\ast$ \texttt{\printntimes{15}{*-*}} $\ast$|

gives

$\ast$ \texttt{\printntimes{15}{*-*}} $\ast$

\bigskip

|\newcommand\truc{\par\centering ***\par}|\par
|\newcommand\saut{\par\noindent\hrulefill\par}|\par
|\saut\texttt{\printntimes{5}{\truc}}\saut|

gives

\newcommand\truc{\par\centering ***\par}
\newcommand\saut{\par\noindent\hrulefill\par}
\saut\texttt{\printntimes{5}{\truc}}\saut

\subsection{
The Options
}

There exist -- since the version~0.2 -- options \Option{vpourv} and
\Option{upourv}. The default option is \Option{vpourv} with which
\cs{modromannumeral}|5| is written as `v'. With the option
\Option{upourv} the same \cs{modromannumeral}|5| is written as `u'. It
was a special requirement from \emph{one} person posting on
\url{fr.comp.text.tex}. \emph{The (almost) French `vpourv' stands for
  `v for v'.}

For sake of symetry, I add, with this version~1, two
antithetical options \Option{jfinal} -- final j -- the default, and
\Option{ifinal} -- final i -- by which one can choose if the last~i of
the number will be written as a~|j| or not.

I add also two pairs of antithetic options. First \Option{min} \TO
\CAD \emph{minuscule} lowercase, default option \TF and \Option{maj}
\TO \CAD \emph{majuscule} uppercase\TF then \Option{court} \TO short,
default option \TF and \Option{long}.

Last, with this version~1, I add the option \Option{sansmod} --
without modification -- which makes the macros of the \cs{modroman}
family aliases of those of \cs{roman} family.

The last five options determine the behaviour of the macros of the
family \cs{modroman}.

\DescribeOption{upourv}
With that option, the roman numeral `v' is turn into an~`u' and one
obtains, e.g., `xuij' for~17.

\DescribeOption{vpourv}
That option, enforced by default, is the opposite of the previous
one. With it, one obtains \Exemple{17} for~17.

The next three options appear with the version~1 of the package.

\DescribeOption{jfinal}
With that default option, if the processed number is greater than~1
and if the last roman \emph{digit} is an~i then it is turned into
a~j. See the examples above.

\DescribeOption{ifinal}
That option is the opposite of the previous one. When enforced, one
obtains \romannumeral 17{} for~17.

\DescribeOption{sansmod}
With that option \cs{modroman}, \cs{modromannumeral} and
\cs{nbmodroman} are just aliases -- obtained with \cs{let} -- of
\cs{roman}, \cs{romannumeral}, and \cs{nbroman} respectively.

If one choses options \Option{vpourv}, , \Option{court}, \Option{min},
and \Option{ifinal} together, one enforces the option
\Option{sansmod}.

\bigskip

The following table shows which family is linked to the \cs{modroman}
family according to the chosen options when \Option{sansmod} is not
enforced.

\begin{center}
  \begin{tabular}{|*{3}{c|}}\hline
    & \Option{court} & \Option{long}\\ \hline
    \Option{min} & \cs{shortroman} & \cs{longroman} \\ \hline
    \Option{maj} & \cs{Roman} & \cs{LongRoman} \\ \hline
  \end{tabular}
\end{center}

\input{MODRbiblio}

\iffalse
\newpage
\section{
Numbers from \nbmodroman{1} to \nbmodroman{1000}
}

\newcounter{machin}\setcounter{machin}{1}
\noindent
\whiledo{\value{machin}<1001}{%
  \mbox{\arabic{machin} -- \shortroman{machin} -- \longroman{machin}
    -- \LongRoman{machin}}\stepcounter{machin}\par\noindent}
\newpage
\fi
\vspace*{\stretch{4}}
\noindent\hrulefill
\begin{center}
In the preambule of this document, there is
\\
|\renewcommand\thepage{\textit{\modroman{page}}}|
\\
hence the page numbering.
\end{center}
\noindent\hrulefill

\vspace*{\stretch{2}}

\begin{center}
\framebox{
 Here ends the documentation of \Pkg{modroman}.
}
\end{center}
\vspace*{\stretch{4}}
\end{document}
\endinput
%%
%% End of file `modroman-en.tex'.
