\documentclass[12pt]{article}
\usepackage[backslant]{aurical}
\usepackage[a4paper]{geometry}
\usepackage[bookmarksopen=false]{hyperref}
\usepackage[ngerman]{babel}
\usepackage{multicol}
\usepackage[T1]{fontenc}
\usepackage[latin1]{inputenc}
\newcommand{\longs}{s\rule{0pt}{0pt}}
\begin{document}
\setlength{\parindent}{0pt}
\setlength{\parskip}{\baselineskip}
\begin{center}
\Fontlukas\fontsize{40pt}{40pt}\selectfont
Aurical \\
\fontsize{30pt}{30pt}\selectfont %
\textemdash{} a set of three decorative fonts in T1 encoding \textemdash
\end{center}


This package implements three calligraphic fonts I created for fun in 2004 and
2005, which cover almost all glyphs from T1. The fonts have been created using
fontforge\footnote{\url{fontforge.sourceforge.net}} and
potrace\footnote{\url{potrace.sourceforge.net}}.

Installation: Unpack the provided archive file ``\verb!aurical_texmf.zip!'' to
your local TEXMF directory. Then update your filename database and add the
mapfile ``aurical.map'' to your dvips and pdftex configuration. For example, in
te\TeX 3 you have to issue the commands \texttt{texhash} and \texttt{updmap-sys
--enable aurical.map}. For other systems consult the corresponding
documentation.


Usage: To use the fonts, just include \verb!\usepackage{aurical}! in the
preamble of your document. Then you can use the commands \verb!\Fontauri!,
\verb!\Fontskrivan!, \verb!\Fontlukas! and \verb!\Fontamici! to select one of
the calligraphic fonts. Low-quality machine generated boldface and slanted 
versions of each font are also included and integrated into NFSS. Thus they 
are selected by the usual font commands like \verb!\bfseries! and \verb!\slshape!.
Since script-like fonts are already slanted to the right, additional slanting
does not always look good. Therefore, backward slanted fonts are provided also,  
which can be activated by \verb!\usepackage[backslant]{aurical}!.
\newpage

\centering
\fbox{\parbox{0.8\linewidth}{\fontsize{14pt}{17pt}\selectfont\Fontauri
\textbf{Auriocus Kalligraphicus} is the first of my calligraphic fonts done in April 2004. It contains all glyphs
from T1 except perthousandzero, but many of the non-letter glyphs look odd: \# \$ \%. 
Its name is a pseudo--latin combination of my
nickname auriocus and the word calligraphic. Like the other fonts in this package
it provides only oldstyle figures: 0123456789 }}

\fbox{\parbox{0.8\linewidth}{\fontsize{14pt}{17pt}\selectfont\Fontlukas
\textbf{Luk\'a\v s Svatba} has been originally invented as AmiciLogo for the design of the cover of a
mediaeval music CD in 2004.
In May 2005, a friend asked me to add czech diacritics so he
could use the font for his wedding (>>svatba<< means wedding in czech). I removed
the long s together with its ligatures, because they are not suitable for
writing modern czech and renamed the font Luk\'a\v s Svatba as a dedication to his wedding. The font
currently covers the whole T1 encoding. 

\Fontamici The original variant with the \textbf{long s*} is still available, as shown in
this paragraph. It is selected by the command {\Fontskrivan \char92 Fontamici}. 
%
It contains a few extra ligatures like Ch, ch, ss, ssi, sk, sl, and a special
swash character to write the logo of the band: \\ \hspace*{5em}Amici Musicae @ntiquae\\ To
make room for these additional glyphs, some characters had to be withdrawn.
Besides the standard ligatures fi, fl, ffi and ffl, which are faked by negative
kerning, \Fontlukas \v s \Fontamici could be removed, because it is written like
\v s in ancient czech texts. Similiarly, the uppercase german double~s* SS,
perthousandzero and compwordmark are withdrawn to make space for additional
ligatures. The long s* is automatically replaced by s before a space or punctuation symbol. 
If it is necessary to typeset s inside a word like in some  compounds, e.g. 
german ,,Aus+flug``, use \Fontskrivan s+\Fontamici. To force an s* where it is normally replaced,
e.g. in the german shortening u.s*.w. or as the single letter s*, the corresponding input is 
\Fontskrivan s*\Fontamici. This is the same input convention as used by \Fontskrivan fraktur.sty \Fontamici 
by Matthias M\"uhlich.}}

\fbox{\parbox{0.8\linewidth}{\fontsize{14pt}{16pt}\selectfont\Fontskrivan
The last of the three fonts designed by me, \textbf{Jana Sk\v rivana}, is my cursive handwriting drawn with a
copperplate calligraphy pen. It's dedicated to a girl, who can sing like a lark
(>>sk\v rivan<< means lark in czech) and has been finished in December 2005. Sadly, a printout made with this
font never looks equally attractive like a real hand-written sample. 
Jana Sk\v rivana can be combined with Luk\'a\v s Svatba to typeset an URL or computer input, 
as shown in the above paragraph.}}

And now enjoy the fonts!

\hfill{\Fontamici\fontsize{20pt}{20pt}\selectfont Christian Gollwitzer
\raisebox{2pt}{(}\kern4pt
@uriocus\/\kern3pt)} % some magic numbers to make the swash capital look good



\end{document}
