%%
%% This is file `powerdot-styletest.tex',
%% generated with the docstrip utility.
%%
%% The original source files were:
%%
%% powerdot.dtx  (with options: `pdstyletest')
%% 
%% ---------------------------------------------------------------
%% Copyright (C) 2005-2014 Hendri Adriaens
%% ---------------------------------------------------------------
%%
%% This work may be distributed and/or modified under the
%% conditions of the LaTeX Project Public License, either version 1.3
%% of this license or (at your option) any later version.
%% The latest version of this license is in
%%   http://www.latex-project.org/lppl.txt
%% and version 1.3 or later is part of all distributions of LaTeX
%% version 2003/12/01 or later.
%%
%% This work has the LPPL maintenance status "maintained".
%%
%% This Current Maintainer of this work is Hendri Adriaens.
%%
%% This work consists of all files listed in manifest.txt.
%%
%%
%% For testing, enter your style below.
%% Switch on only one paper/orient option at a time.
\documentclass[
  style=your style,
  paper=screen,
%%  paper=a4paper,
%%  paper=letterpaper,
  orient=landscape,
%%  orient=portrait,
  size=11pt,
%%  hlsections,
  clock
]{powerdot}

\pdsetup{
  lf=left footer,
  cf=center footer,
  rf=right footer,
  randomdots,dprop={dotstyle=ocircle}
}

%% For testing text height.
\makeatletter
\def\textheightrule{%
  \raisebox\baselineskip{\rule{1cm}\pd@@textheight}%
}
\makeatother

\title{This is a test file to test new styles with --
this title is very long on purpose.\thanks{Adjust textheight
to position this footnote.}}
\author{Hendri Adriaens \and Christopher Ellison}
\date{August 16, 2005}

\begin{document}

\maketitle

\begin{slide}{Test normal slide}
  Lorem ipsum dolor sit amet, consectetuer adipiscing elit. Ut purus
  elit, vestibulum ut, placerat ac, adipiscing vitae, felis. Curabitur
  dictum gravida mauris. Nam arcu libero, nonummy eget, consectetuer
  id, vulputate a, magna. Donec vehicula augue eu neque. Pellentesque
  habitant morbi tristique senectus et netus et malesuada fames ac
  turpis egestas. Mauris ut leo. Cras viverra metus rhoncus sem.

  Notice the color of the equation number!
  \begin{equation}
    (a+b)^n=\sum_{k=0}^n{n\choose k}a^{n-k}b^k
  \end{equation}
\end{slide}

\begin{slide}{Test itemize}
  Some text.\pause
  \begin{itemize}
    \item level 1\pause
    \begin{itemize}
      \item level 2\pause
      \begin{itemize}
        \item level 3\pause
        \begin{itemize}
          \item level 4
        \end{itemize}
      \end{itemize}
    \end{itemize}
  \end{itemize}
  Some text.\footnote{Adjust textheight
  to position this footnote.}
\end{slide}

\section{Normal section}

\begin{slide}{Test enumerate and inactive color}
  Some text.\pause
  \begin{enumerate}[type=1]
    \item level 1\pause
    \begin{enumerate}
      \item level 2\pause
      \begin{enumerate}
        \item level 3\pause
        \begin{enumerate}
          \item level 4
        \end{enumerate}
      \end{enumerate}
    \end{enumerate}
  \end{enumerate}
  Some text.
\end{slide}

\begin{slide}{The rule has height \texttt{textheight}}
  \textheightrule
\end{slide}

\section[template=wideslide]{Wide slide section}

\begin{wideslide}{Test wideslide}
  Lorem ipsum dolor sit amet, consectetuer adipiscing elit. Ut purus
  elit, vestibulum ut, placerat ac, adipiscing vitae, felis. Curabitur
  dictum gravida mauris. Nam arcu libero, nonummy eget, consectetuer
  id, vulputate a, magna. Donec vehicula augue eu neque. Pellentesque
  habitant morbi tristique senectus et netus et malesuada fames ac
  turpis egestas. Mauris ut leo. Cras viverra metus rhoncus
  sem.\footnote{Adjust textheight to position this footnote.}

  Notice the color of the equation number!
  \begin{equation}
    (a+b)^n=\sum_{k=0}^n{n\choose k}a^{n-k}b^k
  \end{equation}
\end{wideslide}

\begin{wideslide}{The rule has height \texttt{textheight}}
  \textheightrule
\end{wideslide}

\end{document}
\endinput
%%
%% End of file `powerdot-styletest.tex'.
