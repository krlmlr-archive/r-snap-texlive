\documentclass{article}
\usepackage[fleqn]{amsmath}
\usepackage[
    web={centertitlepage,designv,forcolorpaper,
         usesf,latextoc,pro}, %tight,
    eforms,aebxmp
]{aeb_pro}
\usepackage{ran_toks}
\useThisSeed{1441984427}
%\useLastAsSeed
%\rtdebugtrue
%\ranToksOn
%\ranToksOff


%\usepackage{myriadpro}
\usepackage[altbullet]{lucidbry}

%\usepackage{makeidx}
%\makeindex
\usepackage{acroman}

\makeatletter
\def\eq@fititin#1{\noindent\unskip\nobreak\hfill\penalty50
    \hskip2em\hbox{}\nobreak\hfill#1}
\def\fitit{\eq@fititin{\exrtnlabelformat}}
\@mparswitchfalse\reversemarginpar
%\def\meta#1{$\langle\textit{\texttt{#1}}\rangle$}
\def\meta#1{\textit{\texttt{#1}}}

\makeatother
%\usepackage[active]{srcltx}

\urlstyle{rm}

\DeclareDocInfo
{
    university={\AcroTeX.Net},
    title={\texorpdfstring{The \textsf{ran\_toks}}{The manual for the ran\_toks}
        Package\texorpdfstring{\\[6pt]\large}{: }
        Randomizing the order of tokens},
    author={D. P. Story},
    email={dpstory@acrotex.net},
    subject=Documentation for the ran\_toks package,
    talksite={\url{www.acrotex.net}},
    version={1.0},
    Keywords={LaTeX,PDF,random, tokens, JavaScript,Adobe Acrobat},
    copyrightStatus=True,
    copyrightNotice={Copyright (C) \the\year, D. P. Story},
    copyrightInfoURL={http://www.acrotex.net}
}
\DeclareInitView{windowoptions={showtitle}}


\def\dps{$\hbox{$\mathfrak D$\kern-.3em\hbox{$\mathfrak P$}%
       \kern-.6em \hbox{$\mathcal S$}}$}

\universityLayout{fontsize=Large}
\titleLayout{fontsize=LARGE}
\authorLayout{fontsize=Large}
\tocLayout{fontsize=Large,color=aeb}
\sectionLayout{indent=-62.5pt,fontsize=large,color=aeb}
\subsectionLayout{indent=-31.25pt,color=aeb}
\subsubsectionLayout{indent=0pt,color=aeb}
\subsubDefaultDing{\texorpdfstring{$\bullet$}{\textrm\textbullet}}

\def\exSrc{\makebox[0pt][r]{\large{\Pisymbol{webd}{157}}\enspace}}

%\pagestyle{empty}
\parindent0pt
\parskip\medskipamount


\definePath\bgPath{"C:/Users/Public/Documents/%
    ManualBGs/Manual_BG_Print_AeB.pdf"}
\begin{docassembly}
\addWatermarkFromFile({%
    bOnTop: false,
    cDIPath: \bgPath
})
\executeSave()
\end{docassembly}

\begin{document}

\maketitle

\selectColors{linkColor=black}
\tableofcontents
\selectColors{linkColor=webgreen}

\section{Introduction}

This is a short package for randomizing the order of tokens. The package
is long overdue; users of \textbf{AeB} and of \textsf{eqexam} have long asked for a way to
randomize the order of the problems in a test or quiz, or anything for
that matter.

\exSrc The example file for this package is \texttt{ran\_toks.tex} found in the
\texttt{examples} folder. The file reproduces the sample code of this
manual.

\section{The Preamble and Package Options}

The preamble for this package is
\begin{Verbatim}[xleftmargin=20pt,commandchars=!()]
!fbox(\usepackage{ran_toks})
\end{Verbatim}
The package itself has no options.

The requirements for \textsf{ran\_toks} are the \textsf{verbatim} package
(part of the standard {\LaTeX} distribution, and the macro file
\texttt{random.tex}, by Donald Arseneau


\section{The main commands and environments}\label{rtmain}

There are two styles for defining a series of tokens to be randomized,
using either the \cs{ranToks} command or the \cs{bRTVToks}/\cs{eRTVToks}
pair. Each of these is discussed in the next two subsections.

\subsection{The \texorpdfstring{\protect\cs{ranToks}
command}{\CMD{ranToks}} command}

The \cs{ranToks} command was the original concept; declare a series of
tokens to be randomized.
\begin{Verbatim}[xleftmargin=20pt,commandchars=!()]
!fbox(!parbox(1.5in)(\ranToks{!meta(name)}{%
    {!meta(token_1)}
    {!meta(token_2)}
    ...
    {!meta(token_n)}
}))
\end{Verbatim}
were \meta{token\_k} is any non-verbatim content;\footnote{Any token that
can be in the argument of a command.} each token is enclosed in braces
(\verb!{}!), this is required. The \meta{name} parameter is required, and
must be unique for the document; it is used to build the names of internal
macros. Of course several such \cs{ranToks} can be used in the document,
either in the preamble or in the body of the document. Multiple
\cs{ranToks} commands must have a different \meta{name} parameter.

\emph{After} a \cs{ranToks} command has been executed, the number of tokens
counted is accessible through the \cs{nToksFor} command,
\begin{Verbatim}[xleftmargin=20pt,commandchars=!()]
!fbox(\nToksFor{!meta(name)})
\end{Verbatim}
The one argument is \meta{name}, and will expand to the total number
of tokens listing as argument in the \cs{ranToks} command by the same
name.

The \cs{ranToks} command does not display the randomized tokens, for that
the command \cs{useRanTok} is used.
\begin{Verbatim}[xleftmargin=20pt,commandchars=!()]
!fbox(!parbox(1.3in)(\useRTName{!meta(name)}
\useRanTok{1}
\useRanTok{2}
...
\useRanTok{n}))
\end{Verbatim}
The argument of \cs{useRanTok} is a positive integer between 1 and
\verb!\nToksFor{!\meta{name}\verb!}!, the number of tokens declared by \cs{ranToks}, inclusive.
There is no space created following the \cs{useRanTok} command, so if
these are to be used ``inline'', enclose them in braces (\verb~{}~), eg,
\verb!{\useRanTok{1}}!. The use of \cs{useRTName} is optional unless the
listing of the \cs{useRanTok} commands is separated from the \cs{ranToks}
command that defined them by another \cs{ranToks} command of a different
name. That should be clear!

Consider this example.

\begin{Verbatim}[xleftmargin=20pt,commandchars=!()]
\ranToks{myPals}{%
    {Jim}{Richard}{Don}
    {Alex}{Tom}{J\"{u}rgen}
}
\end{Verbatim}
\ranToks{myPals}{%
    {Jim}{Richard}{Don}
    {Alex}{Tom}{J\"{u}rgen}
}
I have {\nToksFor{myPals}} pals, they are \useRanTok{1}, \useRanTok{2},
\useRanTok{3}, \useRanTok{4}, {\useRanTok{5}} and \useRanTok{6}. (Listed
in the order of best friend to least best friend.)
The verbatim listing is,
\begin{Verbatim}[xleftmargin=20pt,commandchars=!()]
I have {\nToksFor{myPals}} pals, they are \useRanTok{1},
\useRanTok{2}, \useRanTok{3}, \useRanTok{4}, {\useRanTok{5}}
and \useRanTok{6}.
\end{Verbatim}
Notice that \cs{useRanToks} are not enclosed in braces for 1--4 because
they are each followed by a comma; the fifth token, \verb!{useRanTok{5}}!, is
enclosed in braces to generate a space following the insertion of the text.

Repeating the sentence yields, ``I have {\nToksFor{myPals}} pals, they are
\useRanTok{1}, \useRanTok{2}, \useRanTok{3}, \useRanTok{4},
{\useRanTok{5}} and \useRanTok{6}'' the exact same random order. To obtain
a different order, re-execute the \cs{ranToks} command with the same
arguments. Doing just that, \ranToks{myPals}{{Jim}{Richard}{Don}
{Alex}{Tom}{J\"{u}rgen}}we obtain, ``I have {\nToksFor{myPals}} pals, they
are \useRanTok{1}, \useRanTok{2}, \useRanTok{3}, \useRanTok{4},
{\useRanTok{5}} and \useRanTok{6}.'' A new order? For most applications,
re-randomizing the same token list in the same document is not very likely
something you need to do.

My original application for this, the one that motivated writing this
package at long last, was the need to arrange several form buttons
randomly on the page. My point is that the listing given in the argument
of \cs{ranToks} can pretty much be anything that is allowed to be an
argument of a macro; this would exclude verbatim text created by \cs{verb}
and verbatim environments.

\subsection{The
\texorpdfstring{\protect\cs{bRTVToks}/\protect\cs{eRTVToks}}
    {\CMD{bRTVToks}/\CMD{eRTVToks}} pair of commands}

Sometimes the content to be randomized is quite large and contain verbatim
text. For this, it may be more convenient to use the
\cs{bRTVToks}/\cs{eRTVToks} command pair. The syntax is
\begin{Verbatim}[xleftmargin=20pt,commandchars=!()]
!fbox(!parbox(1.6in)(\bRTVToks{!meta(name)}
\begin{rtVW}
    !meta(content)
\end{rtVW}
...
...
\begin{rtVW}
    !meta(content)
\end{rtVW}
\eRTVToks))
\end{Verbatim}
The \cs{bRTVToks}\texttt{\{\meta{name}\}} command begins the (pseudo)
environment and is ended by \cs{eRTVToks}. Between these two are a series
of \texttt{rtVW} (random toks verbatim write) environments. When the document
is compiled, the contents of each of these environments is written to the
disk drive and saved under a different name (based on the parameter
\meta{name}). Later, using the \cs{useRanTok} commands, they are input
back into the document in a random order.

The use of \cs{useRTName} and \cs{useRanTok} were explained and illustrated
in the previous section. Let's go to the examples,

\begin{Verbatim}[xleftmargin=20pt,commandchars=!()]
\bRTVToks{myThoughts}
\begin{rtVW}
\begin{minipage}[t]{.67\linewidth}
Roses are red and violets are blue,
I've forgotten the rest, have you too?
\end{minipage}
\end{rtVW}
\begin{rtVW}
\begin{minipage}[t]{.67\linewidth}
I gave up saying bad things like
\verb!$#%%%^*%^&#$@#! when I was just a teenager
\end{minipage}
\end{rtVW}
\begin{rtVW}
\begin{minipage}[t]{.67\linewidth}
I am a good guy, pass it on! The code for this last sentence is,
\begin{verbatim}
%#$% I am a good guy, pass it on! ^&*&^*
\end{verbatim}
How did that other stuff get in there?
\end{minipage}
\end{rtVW}
\eRTVToks
\end{Verbatim}
OK, now, let's display these three in random order. Here we place them in
an \texttt{enumerate} environment.

\bRTVToks{myThoughts}%
\begin{rtVW}
\begin{minipage}[t]{.67\linewidth}
Roses are red and violets are blue,
I've forgotten the rest, have you too?
\end{minipage}
\end{rtVW}
\begin{rtVW}
\begin{minipage}[t]{.67\linewidth}
I gave up saying bad things like
\verb!$#%%%^*%^&#$@#! when I was just a teenager
\end{minipage}
\end{rtVW}
\begin{rtVW}
\begin{minipage}[t]{.67\linewidth}
I am a good guy, pass it on! The code for this last sentence is,
\begin{verbatim}
%#$% I am a good guy, pass it on! ^&*&^*
\end{verbatim}
How did that other stuff get in there?
\end{minipage}
\end{rtVW}
\eRTVToks
\begin{enumerate}
    \item \useRanTok{1}
    \item \useRanTok{2}
    \item \useRanTok{3}
\end{enumerate}
The verbatim listing of the example above is
\begin{Verbatim}[xleftmargin=20pt,commandchars=!()]
\begin{enumerate}
    \item \useRanTok{1}
    \item \useRanTok{2}
    \item \useRanTok{3}
\end{enumerate}
\end{Verbatim}

\section{Additional arguments and commands}\label{AddCmds}

The syntax given earlier for \cs{useRanTok} was not completely specified.
It is
\begin{Verbatim}[xleftmargin=20pt,commandchars=!()]
!fbox(\useRanTok[!meta(name)]{!meta(number)})
\end{Verbatim}
The optional first parameter specifies the \meta{name} of the list from
which to draw a random token; the \meta{number} is the number of the
token in the range of 1 and \verb!\nToksFor{!\meta{name}\verb!}!,
inclusive. The optional argument is useful in special circumstances when
you want to mix two random lists together.

To illustrate: \useRanTok[myPals]{1}, \useRanTok[myThoughts]{1}

The verbatim listing is
\begin{Verbatim}[xleftmargin=20pt,commandchars=!()]
To illustrate: \useRanTok[myPals]{1}, \useRanTok[myThoughts]{1}
\end{Verbatim}
This is a rather bad illustration. Every time this document is compiled
there is a different page break on page~\pageref*{AddCmds}. This must be used
wisely.

The original order of the list of tokens is not lost, you can retrieve
them using the command \cs{rtTokByNum},
\begin{Verbatim}[xleftmargin=20pt,commandchars=!()]
!fbox(\rtTokByNum[!meta(name)]{!meta(number)})
\end{Verbatim}
This command expands to the token declared in the list named \meta{name}
that appears at the \meta{number} place in the list. (Rather awkwardly written.)
For example, my really best pals are {\rtTokByNum[myPals]{3}} and
\rtTokByNum[myPals]{4}, but don't tell them. The listing is,
\begin{Verbatim}[xleftmargin=20pt,commandchars=!()]
For example, my really best pals are {\rtTokByNum[myPals]{3}}
and \rtTokByNum[myPals]{4}, but don't tell them.
\end{Verbatim}
In some sense, \cs{rtTokByNum[\meta{name}]} acts like a simple array, the
length of which is \cs{nToksFor\{\meta{name}\}}, and whose $k^{\text{th}}$
element is \cs{rtTokByNum[\meta{name}]\{\meta{k}\}}.

The randomization may be turned off using \cs{ranToksOff} or turned back
on with \cs{ranToksOn}.
\begin{Verbatim}[xleftmargin=20pt,commandchars=!()]
!fbox(!parbox(.9in)(\ranToksOff
\ranToksOn))
\end{Verbatim}
This can be done globally in the preamble for the whole of the document,
or in the body of the document just prior to either \cs{ranToks} or
\cs{bRTVToks}. For example,
\begin{Verbatim}[xleftmargin=20pt,commandchars=!()]
\ranToksOff
\ranToks{integers}{ {1}{2}{3}{4} }
\ranToksOff
\end{Verbatim}
This should make $\verb!\useRanTok{3}! =  \verb!\rtTokByNum{3}! = 3 $;
executing this code, we get, \ranToksOff\ranToks{integers}{ {1}{2}{3}{4} }\ranToksOff
$\useRanTok{3} = \rtTokByNum{3} = 3 $? As anticipated.

The command \cs{ranToksOff} is probably best in the preamble to turn off
all randomization while the rest of the document is being composed.

The \textsf{ran\_toks} package writes an auxiliary file named
\verb!\jobname\_rt.sav!, below represents two typical lines in this file.
\begin{Verbatim}[xleftmargin=20pt,commandchars=!()]
1604051353 % initializing seed value
5747283528 % last random number used
\end{Verbatim}
The first line is the initializing seed value used for the last
compilation of the document; the second line is the last value of the
pseudo-random number generator used in the document.

Normally, the pseudo-random number generator provided by
\texttt{random.tex} produces a new initial seed value every minute. So if
you recompile again before another minute, you'll get the same initial
seed value.

To obtain a new initial seed value each time you compile, place
\cs{useLastAsSeed} in the preamble.
\begin{Verbatim}[xleftmargin=20pt,commandchars=!()]
!fbox(\useLastAsSeed)
\end{Verbatim}
When the document is compiled, the initial seed value taken as the second
line in the \verb!\jobname\_rt.sav! file, as seen in the above example.
With this command in the preamble, a new set of random numbers is
generated on each compile. If the file \cs{jobname\_rt.sav} does not
exist, the generator will be initialized by its usual method, using the time and date.

The command \cs{useThisSeed} allows you to reproduce a previous
pseudo-random sequence.
\begin{Verbatim}[xleftmargin=20pt,commandchars=!()]
!fbox(\useThisSeed{!meta(init_seed_value)})
\end{Verbatim}
This command needs to be placed in the preamble. The value of
\meta{init\_seed\_value} is an integer, normally taken from the
first line of the \verb!\jobname\_rt.sav! file.

When creating tests (possibly using \textsf{eqexam}), the problems, or
contiguous collections of problems, can be randomly ordered using the
\cs{bRTVToks}/\cs{eRTVToks} paradigm. For example, suppose there are two
classes and you want a random order (some of) the problems for each of the
two classes. Proceed as follows:
\begin{enumerate}
\item Compile the document, open \cs{jobname\_rt.sav}, and copy the
    first line (in the above example, that would be
    \texttt{1604051353}).
\item Place \verb!\useThisSeed{1604051353}! in the preamble. Compiling
    will bring back the same pseudo-random sequence very time.
\item Comment this line out, and repeat the process (use
    \cs{useLastAsSeed} to generate new random sequences at each
    compile) until you get another distinct randomization, open
    \cs{jobname\_rt.sav}, and copy the first line again, say its \texttt{735794511}.
\item Place \verb!\useThisSeed{735794511}! in the preamble.
\item Label each
\begin{Verbatim}
%\useThisSeed{1604051353} % 11:00 class
%\useThisSeed{735794511}  % 12:30 class
\end{Verbatim}
To reproduce the random sequence for the class, just uncomment the random
seed used for that class.
\end{enumerate}
If you are using \textsf{eqexam}, the process can be automated as follows:
\begin{Verbatim}[xleftmargin=20pt,commandchars=!()]
\vA{\useThisSeed{1604051353}} % 11:00 class
\vB{\useThisSeed{735794511}}  % 12:30 class
\end{Verbatim}
Again, this goes in the preamble.

Now, I simply must get back to my retirement. \dps

\end{document}
