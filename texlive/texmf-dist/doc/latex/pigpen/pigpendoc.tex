\documentclass[a4paper]{article}
\usepackage{pigpen}

\newcommand\exa{\nopagebreak \begin{flushleft}\smallskip \nopagebreak
                \begin{minipage}[t]{6cm}\sloppy}
\newcommand\exb{\end{minipage}\kern 1cm\begin{minipage}[t]{8cm}\sloppy }
\newcommand\exc{\end{minipage}\kern -3cm \smallskip\end{flushleft}}

\title{{\pigpenfont PIG PEN FOR TEX}\\[0.5cm]
	The Pigpen Cipher for \TeX\ 0.2}
\author{Oliver Corff}
\date{December 07, 2008}
\begin{document}
\maketitle


\section{The Pigpen cipher}

The \TeX\ Pigpen cipher font is a small hack\footnote{%
	The \TeX\ Pigpen cipher was inspired by a discussion on
	\texttt{comp.text.tex} in December 2008.} 
to type the so-called masonic cipher, or Freemason's cipher,
which is a simple substitution cipher based on a grid. 
A 3x3 grid is filled with the letters A to I and J to R.
\begin{center}
\begin{minipage}[t]{3cm}%
\begin{tabular}{c|c|c}
	A&B&C\\
	\hline
	D&E&F\\
	\hline
	G&H&I\\
\end{tabular}\end{minipage}%
\begin{minipage}[t]{3cm}%
\begin{tabular}{c|c|c}
	J.&K.&L.\\
	\hline
	M.&N.&O.\\
	\hline
	P.&Q.&R.\\
\end{tabular}\end{minipage}
\begin{minipage}[t]{3cm}%
\begin{tabular}{ccc}
	 &S& \\
	T&$\times$&U\\
	 &V& \\
\end{tabular}\end{minipage}%
\begin{minipage}[t]{3cm}%
\begin{tabular}{ccc}
	 &W.& \\
	X.&$\times$&Y.\\
	 &Z.&\\
\end{tabular}\end{minipage}
\end{center}

\begin{center}
{\pigpenfont
\begin{minipage}[t]{3cm}%
\begin{tabular}{ccc}
	A&B&C\\
	D&E&F\\
	G&H&I\\
\end{tabular}\end{minipage}%
\begin{minipage}[t]{3cm}%
\begin{tabular}{ccc}
	J&K&L\\
	M&N&O\\
	P&Q&R\\
\end{tabular}\end{minipage}
\begin{minipage}[t]{3cm}%
\begin{tabular}{ccc}
	 &S& \\
	T&$\times$&U\\
	 &V& \\
\end{tabular}\end{minipage}%
\begin{minipage}[t]{3cm}%
\begin{tabular}{ccc}
	  &W&\\
	X&$\times$&Y\\
	  &Z&\\
\end{tabular}\end{minipage}}
\end{center}

The letters in this grid are simply replaced by the inner edges 
of their box within the grid. The subset of letters J to R is marked
by a dot.

The letters S, T, U and V are arranged in the four quadrants of an X,
the letters W, X, Y and Z follow this arragement but are marked with
a dot. Again, the letters are substituted by the inner edges of their
respective corner.

\section{Usage}

In the preamble of a \LaTeXe\ document, the package is declared by
saying \verb|\usepackage{pigpen}|. Options are not available.
There is only one user command, \verb|\pigpenfont|:
\exa
	{\pigpenfont LUCAS}
\exb
	\begin{verbatim}
	{\pigpenfont LUCAS}
	\end{verbatim}
\exc



\end{document}
