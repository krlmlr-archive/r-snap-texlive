\documentclass[a4paper]{nlctdoc}

\usepackage[utf8]{inputenc}
\usepackage[T1]{fontenc}
\usepackage{lmodern}
\usepackage{color}
\usepackage{creatdtx}
\usepackage{probsoln}

\usepackage[colorlinks,
            bookmarks,
            hyperindex=false,
            pdfauthor={Nicola L.C. Talbot},
            pdftitle={probsoln: creating problem sheets optionally with solutions}]{hyperref}
\doxitem{Option}{option}{package options}

\RecordChanges
\PageIndex
\CheckSum{1822}

\newcommand*{\dq}[1]{``#1''}

\begin{document}
\MakeShortVerb{"}
\DeleteShortVerb{\|}

 \title{probsoln v3.04: 
creating problem sheets optionally with solutions}
 \author{Nicola L.C. Talbot\\[10pt]
School of Computing Sciences\\
University of East Anglia\\
Norwich. Norfolk\\
NR4 7TJ. United Kingdom.\\
\url{http://theoval.cmp.uea.ac.uk/~nlct/}}

 \date{2012-08-23}
 \maketitle
\tableofcontents

 \section{Introduction}
The \styfmt{probsoln} package is designed for teachers or lecturers
who want to create problem sheets for their students. This package
was designed with mathematics problems in mind, but can be used for
other subjects as well. The idea is to create a file containing a
large number of problems with their solutions which can be read in
by \LaTeX, and then select a number of problems to typeset. This
means that once the database has been set up, each year you can
easily create a new problem sheet that is sufficiently different
from the previous year, thus preventing the temptation of current
students seeking out the previous year's students, and checking out
their answers. There is also an option that can be passed to the
package to determine whether or not the solutions should be printed.
In this way, one file can either produce the student's version or
the teacher's version.

\section{Package Options}\label{sec:pkgopt}
The following options may be passed to this package:
\begin{description}
\item[\pkgopt{answers}]  Show the answers
\item[\pkgopt{noanswers}] Don't show the answers (default)
\item[\pkgopt{draft}] Display the label and dataset name when a problem is used
\item[\pkgopt{final}] Don't display label and dataset name when a problem is used
\item[\pkgopt{usedefaultargs}] Make \ics{thisproblem} use the
default arguments supplied in the problem definition.
\item[\pkgopt{nousedefaultargs}] Make \ics{thisproblem} prompt for
problem arguments (default).
\end{description}

\section{Verbatim}\label{sec:verbatim}

As from version 3.02, problems and solutions may contain verbatim
text, but you must use the \iterm{fragile}\texttt{fragile} (or
\texttt{fragile=true}) option for the associated environments.

Alternatively, if most of your problems contain verbatim, you can
globally set this option using:
\begin{verbatim}
\setkeys{probsoln}{fragile}
\end{verbatim}
You can switch off this option using \texttt{fragile=false}.

The \texttt{fragile} option writes information to a temporary file.
This defaults to "\jobname.vrb" but the name may be changed. The
extension (".vrb") is given by:
\begin{definition}[\DescribeMacro{\ProbSolnFragileExt}]
\cs{ProbSolnFragileExt}
\end{definition}
The base name (\cs{jobname}) is given by:
\begin{definition}[\DescribeMacro{\ProbSolnFragileFile}]
\cs{ProbSolnFragileFile}
\end{definition}

\section{Showing and Hiding Solutions}\label{sec:showanswers}

In addition to the \pkgopt{answers} and \pkgopt{noanswers} package
options, it is also possible to show or suppress the solutions
using
\begin{definition}[\DescribeMacro{\showanswers}]
\cs{showanswers}
\end{definition}
and
\begin{definition}[\DescribeMacro{\hideanswers}]
\cs{hideanswers}
\end{definition}
respectively.

The boolean variable \bool{showanswers} determines whether the
answers should be displayed. You can use this value with the
\sty{ifthen} package to specify different text depending on 
whether the solutions should be displayed. For example:
\begin{verbatim}
Assignment 1\ifthenelse{\boolean{showanswers}}{ (Solution Sheet)}{}
\end{verbatim}
Alternatively you can use \ics{ifshowanswers}\ldots\cs{else}\ldots
\cs{fi}:
\begin{verbatim}
Assignment 1\ifshowanswers\space (Solution Sheet)\fi
\end{verbatim}

For longer passages, you can use the environments
\begin{definition}[\DescribeEnv{onlyproblem}]
\cs{begin}\marg{onlyproblem}\oarg{option}
\end{definition}
and 
\begin{definition}[\DescribeEnv{onlysolution}]
\cs{begin}\marg{onlysolution}\oarg{option}
\end{definition}
For example:
\begin{verbatim}
\begin{onlyproblem}%
What is the derivative of $f(x) = x^2$?
\end{onlyproblem}%
\begin{onlysolution}%
$f'(x) = 2x$
\end{onlysolution}
\end{verbatim}
The above will only display the question if \bool{showanswers}
is false and will only display the solution if \bool{showanswers}
is true. If you want the question to appear in the answer
sheet as well as the solution, then don't put the question in
the \env{onlyproblem} environment:
\begin{verbatim}
What is the derivative of $f(x) = x^2$?
\begin{onlysolution}%
Solution: $f'(x) = 2x$
\end{onlysolution}
\end{verbatim}

\begin{important}
If you want to include verbatim text in the body of
\env{onlyproblem} or \env{onlysolution}, you need to specify
\texttt{fragile} in the optional argument of the environment.
(See \sectionref{sec:verbatim} for further details.)
\end{important}

If you use \envfmt{onlysolution} within the \env{defproblem}
environment, the problem will be tagged as having a solution
and will be added to the list used by \ics{foreachsolution}.
The optional argument of \envfmt{onlysolution} (and \env{onlyproblem})
is inherited from the parent \env{defproblem} setting.

\section{General Formatting Commands}\label{sec:formatting}

The commands and environments described in this section are
provided to assist formatting problems and their solutions.
\begin{definition}[\DescribeEnv{solution}]
\verb|\begin{solution}|\meta{text}\verb|\end{solution}|
\end{definition}
By default, this is equivalent to 
\begin{display}
\verb|\par\noindent\textbf{\solutionname}: |\meta{text}
\end{display}
where \DescribeMacro{\solutionname}\cs{solutionname} defaults
to \dq{\solutionname}. Note that you must place the \env{solution}
environment inside the \envfmt{onlysolution} environment or
between \ics{ifshowanswers}\ldots\cs{fi} to ensure that it
is suppressed when the solutions are not wanted. (See
\sectionref{sec:showanswers}.) 


Note that the \styfmt{probsoln} package will only define the 
\env{solution} environment if it is not already defined.

\begin{definition}[\DescribeEnv{textenum}]
\verb|\begin{textenum}|\ldots\verb|\end{textenum}|
\end{definition}
The \envfmt{textenum} environment is like the \env{enumerate}
environment but is in-line. It uses the same counter that the
\envfmt{enumerate} environment would use at that level so the
question can be compact but the answer can use \envfmt{enumerate}
instead. For example:
\begin{verbatim}
\begin{onlyproblem}%
  Differentiate the following:
  \begin{textenum}
    \item $f(x)=2^x$; \item $f(x)=\cot(x)$
  \end{textenum}
\end{onlyproblem}
\begin{onlysolution}
  \begin{enumerate}
  \item
    \begin{align*}
    f(x) &= 2^x = \exp(\ln(x^2)) =\exp(2\ln(x))\\
    f'(x) &= \exp(2\ln(x))\times \frac{2}{x}\\
      &= f(x)\frac{2}{x}
    \end{align*}
  \item
    \begin{align*}
    f(x) &= \cot(x) = (\tan(x))^{-2}\\
    f'(x) &= -(\tan(x))^{-2}\times\sec^2(x)\\
    &=-\csc^2x
    \end{align*}
  \end{enumerate}
\end{onlysolution}
\end{verbatim}
In this example, the items in the question are brief, so an
\env{enumerate} environment would result in a lot of unnecessary
white space, but the answers require more space, so an
\envfmt{enumerate} environment is more appropriate. Since the
\envfmt{textenum} environment uses the same counters as the
\envfmt{enumerate} environment, the question and answer sheets use
consistent labelling. Note that there are other packages available
on CTAN that you can use to create in-line lists. Check the
\urlfootref{http://www.tex.ac.uk/tex-archive/help/Catalogue/bytopic.html\#enumeration}{TeX
Catalogue} for further details.

\DescribeMacro{\correctitem}\DescribeMacro{\incorrectitem}
\begin{definition}
\cs{correctitem}\\
\cs{incorrectitem}
\end{definition}
You can use the commands \cs{correctitem} and \cs{incorrectitem} 
in place of \ics{item}. If the solutions are suppressed, these
commands behave in the same way as \cs{item}, otherwise they
format the item label using one of the commands:
\DescribeMacro{\correctitemformat}\DescribeMacro{\incorrectitemformat}
\begin{definition}
\cs{correctitemformat}\marg{label}\\
\cs{incorrectitemformat}\marg{label}
\end{definition}
For example:
\begin{verbatim}
Under which of the following functions does $S=\{a_1,a_2\}$
become a probability space?
\begin{enumerate}
\incorrectitem $P(a_1)=\frac{1}{3}$, $P(a_2)=\frac{1}{2}$
\correctitem $P(a_1)=\frac{3}{4}$, $P(a_2)=\frac{1}{4}$
\correctitem $P(a_1)=1$, $P(a_2)=0$
\incorrectitem $P(a_1)=\frac{5}{4}$, $P(a_2)=-\frac{1}{4}$
\end{enumerate}
\end{verbatim}
The default definition of \cs{correctitemformat} puts a frame around
the label.

\section{Defining a Problem}\label{sec:defproblem}

It is possible to construct a problem sheet with solutions using the
commands described in the previous sections, however it is also
possible to define a set of problems for later use. In this way you
can create an external file containing many problems some or all of
which can be loaded and used in a document. The \styfmt{probsoln}
package has a default data set labelled \dq{default} in which you
can store problems. Alternatively, you can create multiple data
sets. You can then iterate through each problem in a problem set.
You can use a previously defined problem more than once, which means
that by judicious use of \env{onlyproblem}, \env{onlysolution} or
the \bool{showanswers} boolean variable in conjunction with
\ics{showanswers} and \ics{hideanswers}, you can print the solutions
in a different location to the questions (for example in an
appendix).

\begin{definition}[\DescribeEnv{defproblem}]
\verb|\begin{defproblem}|\oarg{n}\oarg{default args}\marg{label}\oarg{option}\newline
\meta{definition}\newline
\verb|\end{defproblem}|
\end{definition}
This defines the problem whose label is given by \meta{label}. The
label must be unique for a given data set and should not contain
active characters or a comma. (Active characters include the special characters
such as \$ and \&, but some packages may make other symbols active,
such as the colon (:) character. For example, the \sty{ngerman} and
\sty{babel} packages make certain punctuation active. Check the
relevant package documentation for details.)

\begin{important}
The final optional argument \meta{option} may be \texttt{fragile} to
indicate that the problem contains verbatim text. Any occurrences of
\env{onlyproblem} or \env{onlysolution} contained within
\envfmt{defproblem} are inherited from \envfmt{defproblem}. (See
\sectionref{sec:verbatim} for further details.)
\end{important}

If \env{defproblem} occurs in the document or is included via
\ics{input} or \ics{include}, then the problem will be added to
the default data set. If \envfmt{defproblem} occurs in an external
file that is loaded using one of the commands defined in
\sectionref{sec:load} then the problem will be added to
the specified data set.

The contents of the \env{defproblem} environment should be the text
that defines the problem. This may include any of the commands
defined in \sectionref{sec:showanswers} and
\sectionref{sec:formatting}.

The problem may optionally take \meta{n} arguments (where 
\meta{n} is from 0 to 9). The arguments can be referenced
in the definition via \texttt{\#1},\ldots,\texttt{\#9}.
If \meta{n} is omitted then the problem doesn't take any
arguments.
The following example defines a problem with one argument:
\begin{verbatim}
\begin{defproblem}[1]{diffsin}
Differentiate $f(x)=\sin(#1x)$.
\begin{onlysolution}%
  \begin{solution}
    $f'(x) = #1\cos(#1x)$
  \end{solution}
\end{onlysolution}
\end{defproblem}
\end{verbatim}

The second optional argument \meta{default args} supplies 
default problem arguments that will automatically be used within
\ics{thisproblem} when used in \ics{foreachproblem} in conjunction
with the package option \pkgopt{usedefaultargs}. (See \sectionref{sec:foreach}.)
For example:
\begin{verbatim}
\begin{defproblem}[1][{2}]{diffsin}
Differentiate $f(x)=\sin(#1x)$.
\begin{onlysolution}%
  \begin{solution}
    $f'(x) = #1\cos(#1x)$
  \end{solution}
\end{onlysolution}
\end{defproblem}
\end{verbatim}

\begin{definition}[\DescribeMacro{\newproblem}]
\cs{newproblem}\oarg{n}\oarg{default args}\marg{label}\marg{problem}\marg{solution}
\end{definition}
This is a shortcut command for:
\begin{ttfamily}\obeylines
\cs{begin}\{defproblem\}\oarg{n}\oarg{default args}\marg{label}\%
\meta{problem}\%
\cs{begin}\{onlysolution\}\%
\cs{begin}\{solution\}\%
\meta{solution}\%
\cs{end}\{solution\}\%
\cs{end}\{onlysolution\}\%
\cs{end}\{defproblem\}
\end{ttfamily}
For example:
\begin{verbatim}
\newproblem[1]{diffsin}{%
  \(f(x) = \sin(#1x)\)
}%
{%
  \(f'(x) = #1\cos(#1x)\)
}
\end{verbatim}
is equivalent to
\begin{verbatim}
\begin{defproblem}[1]{diffcos}%
  \(f(x) = \cos(#1x)\)
\begin{onlysolution}%
  \begin{solution}%
    \(f'(x) = -#1\sin(#1x)\)
  \end{solution}%
\end{onlysolution}%
\end{defproblem}
\end{verbatim}
(In this example, the argument will need to be a positive number
to avoid a double minus in the answer. If you want to perform
floating point arithmetic on the arguments, then try the
\sty{fp} or \sty{pgfmath} packages.)

Alternatively, if you want to supply default arguments to use when
iterating through problems with \ics{foreachproblem}:
\begin{verbatim}
\newproblem[1][{3}]{diffsin}{%
  \(f(x) = \sin(#1x)\)
}%
{%
  \(f'(x) = #1\cos(#1x)\)
}
\end{verbatim}


\begin{definition}[\DescribeMacro{\newproblem*}]
\cs{newproblem*}\oarg{n}\oarg{default args}\marg{label}\marg{definition}
\end{definition}
This is a shortcut for:
\begin{ttfamily}\obeylines
\cs{begin}\{defproblem\}\oarg{n}\oarg{default args}\marg{label}\%
\meta{definition}\%
\cs{end}\{defproblem\}
\end{ttfamily}

\begin{important}
Note that you can't use verbatim text with \cs{newproblem} or
\cs{newproblem*}. Use the \env{defproblem} environment instead with
the \texttt{fragile option}.
\end{important}

\section{Using a Problem}\label{sec:useproblem}

Once you have defined a problem using \env{defproblem} or
\ics{newproblem} (see \sectionref{sec:defproblem}), you can 
later display the problem using:
\begin{definition}[\DescribeMacro{\useproblem}]
\cs{useproblem}\oarg{data set}\marg{label}\marg{arg$_1$}\ldots
\marg{arg$_N$}
\end{definition}
where \meta{data set} is the name of the data set that contains
the problem (the default data set is used if omitted), 
\meta{label} is the label identifying the required problem and
\meta{arg$_1$}, \ldots, \meta{arg$_N$} 
are the arguments to pass to the problem, if the problem was 
defined to have arguments (where $N$ is the number 
of arguments specified when the problem was defined).

For example, in the previous section the problem \texttt{diffcos} 
was defined to have one argument, so it can be used as follows:
\begin{verbatim}
\useproblem{diffcos}{3}
\end{verbatim}
This will be equivalent to:
\begin{verbatim}
\(f(x) = \cos(3x)\)
\begin{onlysolution}%
\begin{solution}%
\(f'(x) = -3\sin(3x)\)
\end{solution}%
\end{onlysolution}%
\end{verbatim}

\section{Loading Problems From External Files}\label{sec:load}

You can store all your problem definitions (see
\sectionref{sec:defproblem}) in an external file. 
These problems can all be appended to the default data set by
including the file via \ics{input} or they can be appended
to other data sets using one of the commands described below.
Once you have loaded all the required problems, you can
iterate through the data sets using the commands described
in \sectionref{sec:foreach}. Note that the commands below
will create a new data set, if the named data set doesn't
exist.

\begin{definition}[\DescribeMacro{\loadallproblems}]
\cs{loadallproblems}\oarg{data set}\marg{filename}
\end{definition}
This will load all problems defined in \meta{filename} and
append them to the specified data set, in the order in which
they are defined in the file. If \meta{data set} is
omitted, the default data set will be used. If \meta{data set}
doesn't exist, it will be created.

\begin{definition}[\DescribeMacro{\loadselectedproblems}]
\cs{loadselectedproblems}\oarg{data set}\marg{labels}\marg{filename}
\end{definition}
This is like \cs{loadallproblems}, but only those problems whose
label is listed in the comma-separated list \meta{labels} are
loaded. For example, if I have some problems defined in the
file \texttt{derivatives.tex}, then
\begin{verbatim}
\loadselectedproblems{diffsin,diffcos}{derivatives}
\end{verbatim}
will only load the problems whose labels are \texttt{diffsin}
and \texttt{diffcos}, respectively. All the other problems in 
the file will remain undefined.

\begin{definition}[\DescribeMacro{\loadexceptproblems}]
\cs{loadexceptproblems}\oarg{data set}\marg{exception list}\marg{filename}
\end{definition}
This is the reverse of \cs{loadselectedproblems}. This loads all
problems except those whose labels are listed in \meta{exception
list}.

\begin{definition}[\DescribeMacro{\loadrandomproblems}]
\cs{loadrandomproblems}\oarg{data set}\marg{n}\marg{filename}
\end{definition}
This randomly loads \meta{n} problems from \meta{filename} and
adds them to the given data set. If \meta{data set} is omitted,
the default data set is assumed. Note that the problems will be
added to the data set in a random order, not in the order in
which they were defined. There must be at least \meta{n} problems
defined in \meta{filename}.

\begin{definition}[\DescribeMacro{\loadrandomexcept}]
\cs{loadrandomexcept}\oarg{data
set}\marg{n}\marg{filename}\marg{exception list}
\end{definition}
This is similar to \cs{loadrandomproblems} except that it won't load
those problems whose labels are listed in \meta{exception list}.
\textbf{If you want to automatically exclude problems included in
previous documents, see \sectionref{sec:exprev}.}

Note that the random number generator has been modified in version
3.01 in order to fix a bug. If you want to ensure that your random
numbers are compatible with earlier versions, you can switch to the
old generator using
\begin{definition}[\DescribeMacro{\PSNuseoldrandom}]
\cs{PSNuseoldrandom}
\end{definition}

\begin{important}
It is generally not a good idea to place anything in 
\meta{filename} that is not inside the body of \env{defproblem} 
or in the arguments to \ics{newproblem} or \ics{newproblem*}.
All the commands in this section input the external file within
a local scope, so command definitions would need to be made
global to have any effect. In addition, \cs{loadrandomproblems}
has to load each file twice, which means that anything outside
a problem definition will be parsed twice.
\end{important}

\subsection{Randomly Selecting Problems Not Selected in Previous
Documents}
\label{sec:exprev}

Suppose you have a large set of questions that you want to randomly
select for assignments and exams. The chances are, you don't want to
include questions that have been previously set for, say, the last
three years. That is, you don't want to select questions the
students may already have seen. As from version 3.03, you can now do
this.

The \sty{probsoln} package defaults to the UK academic year, which
starts in September. If this isn't appropriate, you can change it
using:
\begin{definition}[\DescribeMacro{\SetStartMonth}]
\cs{SetStartMonth}\marg{n}
\end{definition}
where \meta{n} is the number of the month. (1 = January, 2 =
February, etc.)

The \emph{start year} is the calender year in effect when the
academic year started. For example, if this is the academic year
2011/12, then the start year is 2011. This is automatically set to
the start of the current academic year. It is also updated when
\cs{SetStartMonth} is used.\footnote{So don't use \cs{SetStartMonth}
after \cs{SetStartYear}.} If you want to set it to a specific year,
you can use:
\begin{definition}[\DescribeMacro{\SetStartYear}]
\cs{SetStartYear}\marg{year}
\end{definition}
For example: \verb|\SetStartYear{2008}| indicates the academic year
2008/9.

There are two files concerned with previously used labels. They are:
\begin{description}

  \item[The previously used labels file] This keeps track of all
    problems used in previous years, as well as problems used by
    other documents that have this as their previously used labels
    file, and it contains the problem labels from the last run of
    the current document.

  \item[The current used labels file] This defaults to
\cs{jobname}\texttt{.prb}, but the name can be changed using:
  \begin{definition}[\DescribeMacro{\SetUsedFileName}]
  \cs{SetUsedFileName}\marg{name}
  \end{definition}
  This file keeps track of all the labels used in the current
  document from the previous \LaTeX\ run. Note that if you want to
  delete this file, first clear it using
  \begin{definition}[\DescribeMacro{\ClearUsedFile}]
  \cs{ClearUsedFile}\marg{file}
  \end{definition}
  in place of \cs{ExcludePreviousFile}\marg{file}, described below.
  The argument \meta{file} is the previously used labels file
  described above. \cs{ClearUsedFile} will remove all labels in
  the current used labels file from the previously used labels file
  and clear the current used labels file. Once this file is empty,
  it may then be deleted.

\end{description}

Before loading randomly selected problems, first specify the
previously used labels file with the command:
\begin{definition}[\DescribeMacro{\ExcludePreviousFile}]
\cs{ExcludePreviousFile}\oarg{number of years}\marg{file name}
\end{definition}
where \meta{file name} is the name of the previously used file. The
optional argument \meta{number of years} specifies the year cut-off.
This defaults to 3, which means that only those labels used this
year or the previous 2 years will be kept. Any problems used before
then may be reused.

Suppose I'm lecturing a first year undergraduate mathematics course
(designated, say, mth101). I want to set assignments on each topic
and an exam at the end of the year (as well as a resit or second
sitting paper). I've got databases with problems for each topic, but
the first and second sitting exams mustn't include any of the
problems used in the assignments or any problems used in assignments
or exams for the previous two academic years. I'm going to arrange
my directory structure as follows:
\begin{itemize}
\item \texttt{mth101/}
  \begin{itemize}
   \item \texttt{assignment1/} (differentiation)
     \begin{itemize}
       \item \texttt{assignment1.tex}
     \end{itemize}
   \item \texttt{assignment2/} (probability spaces)
     \begin{itemize}
       \item \texttt{assignment2.tex}
     \end{itemize}
   \item \texttt{assignment3/} (linear algebra)
     \begin{itemize}
       \item \texttt{assignment3.tex}
     \end{itemize}
   \item \texttt{exams/}
     \begin{itemize}
       \item \texttt{exam.tex} (first sitting)
       \item \texttt{resit.tex} (second sitting)
     \end{itemize}
   \item \texttt{databases/}
     \begin{itemize}
       \item \texttt{differentiation.tex}
       \item \texttt{probabilityspaces.tex}
       \item \texttt{linearalgebra.tex}
     \end{itemize}
   \item \texttt{previouslabels.tex} (created by \sty{probsoln})
  \end{itemize}
\end{itemize}

\section{Iterating Through Datasets}\label{sec:foreach}

Once you have defined all your problems for a given data set, you
can use an individual problem with \ics{useproblem} (see
\sectionref{sec:useproblem}) but it is more likely that you will
want to iterate through all the problems so that you don't need to
remember the labels of all the problems you have defined.

\begin{definition}[\DescribeMacro{\foreachproblem}]
\cs{foreachproblem}\oarg{data set}\marg{body}
\end{definition}
This does \meta{body} for each problem in the given data set.
If \meta{data set} is omitted, the default data set is used.
Within \meta{body} you can use 
\begin{definition}[\DescribeMacro{\thisproblem}]
\cs{thisproblem}
\end{definition}
to use the current problem and
\begin{definition}[\DescribeMacro{\thisproblemlabel}]
\cs{thisproblemlabel} 
\end{definition}
to access the current label. If the problem requires arguments,
and no default arguments were supplied in the problem definition or
the package option \pkgopt{usedefaultargs} was not used, then
you will be prompted for arguments, so if you want to use this
approach you will need to use \LaTeX\ in interactive mode. If
you do provide arguments, they will be stored in the event that
you need to iterate through the data set again. The
arguments will be included in \cs{thisproblem}, so you only
need to use \cs{thisproblem} without having to specify
\ics{useproblem}.

For example, to iterate through all problems
in the default data set:
\begin{verbatim}
\begin{enumerate}
\foreachproblem{\item\thisproblem}
\end{enumerate}
\end{verbatim}

\begin{definition}[\DescribeMacro{\foreachsolution}]
\cs{foreachsolution}\oarg{data set}\marg{body}
\end{definition}
This is equivalent to \cs{foreachsolution}, but only iterates
through problems that contain the \env{onlysolution} environment.
Note that you still need to use \ics{showanswers} or the
\pkgopt{answers} package option for the contents of the
\env{onlysolution} environment to appear.

\begin{definition}[\DescribeMacro{\foreachdataset}]
\cs{foreachdataset}\marg{cmd}\marg{body}
\end{definition}
This does \meta{body} for each of the defined data sets. Within
\meta{body}, \meta{cmd} will be set to the name of the current
data set. For example, to display all problems in all data sets:
\begin{verbatim}
\begin{enumerate}
\foreachdataset{\thisdataset}{%
\foreachproblem[\thisdataset]{\item\thisproblem}}
\end{enumerate}
\end{verbatim}

Suppose I have two external files called
\texttt{derivatives.tex} and \texttt{probspaces.tex} which
define problems using both \env{onlyproblem} and 
\env{onlysolution} for example:
\begin{verbatim}
\begin{defproblem}{cosxsqsinx}%
\begin{onlyproblem}%
$y = \cos(x^2)\sin x$.%
\end{onlyproblem}%
\begin{onlysolution}%
\[\frac{dy}{dx} = -\sin(x^2)2x\sin x + \cos(x^2)\cos x\]
\end{onlysolution}%
\end{defproblem}
\end{verbatim}
I can write a document that creates two data sets, one for
the derivative problems and one for the problems about
probability spaces. I can then use \ics{hideanswers} and
iterate through the require data set to produce the problems.
Later, I can use \ics{showanswers} and iterate over all problems defined in both data
sets to produce the chapter containing all the answers. When
displaying the questions, I have taken advantage of the fact that
I can cross-reference items within an \env{enumerate} environment,
and redefined \ics{theenumi} to label the questions according to
the chapter. The cross-reference label is constructed from
the problem label and is referenced in the answer section to
ensure that the answers have the same label as the questions.
\begin{verbatim}
\documentclass{report}
\usepackage{probsoln}
\begin{document}
\hideanswers
\chapter{Differentiation}
% randomly select 25 problems from derivatives.tex and add to
% the data set called 'deriv'
\loadrandomproblems[deriv]{25}{derivatives}

% Display the problems
\renewcommand{\theenumi}{\thechapter.\arabic{enumi}}
\begin{enumerate}
\foreachproblem[deriv]{\item\label{prob:\thisproblemlabel}\thisproblem}
\end{enumerate}
% You may need to change \theenumi back here

\chapter{Probability Spaces}
% randomly select 25 problems from probspaces.tex and add to
% the data set called 'spaces'
\loadrandomproblems[spaces]{25}{probspaces}

% Display the problems
\renewcommand{\theenumi}{\thechapter.\arabic{enumi}}
\begin{enumerate}
\foreachproblem[spaces]{\item\label{prob:\thisproblemlabel}\thisproblem}
\end{enumerate}
% You may need to change \theenumi back here

\appendix

\chapter{Solutions}
\showanswers
\begin{itemize}
\foreachdataset{\thisdataset}{%
\foreachproblem[\thisdataset]{\item[\ref{prob:\thisproblemlabel}]\thisproblem}
}
\end{itemize}

\end{document}
\end{verbatim}

\section{Random Number Generator}\label{sec:random}

This package provides a pseudo-random number generator that is used
by \ics{loadrandomproblems}. As noted earlier the random number
generator has been modified in version 3.01 in order to fix a bug.
If you want to ensure that your random numbers are compatible with
earlier versions, you can switch to the old generator using
\begin{definition}[\DescribeMacro{\PSNuseoldrandom}]
\cs{PSNuseoldrandom}
\end{definition}

\begin{definition}[\DescribeMacro{\PSNrandseed}]
\cs{PSNrandseed}\marg{n}
\end{definition}
This sets the seed to \meta{n} which must be a non-zero integer.
For example, to generate a different set of random numbers
every time you \LaTeX\ your document,\footnote{assuming you
leave at least a minute between runs.} put the following in your
preamble:
\begin{verbatim}
\PSNrandseed{\time}
\end{verbatim}
or to generate a different set of random numbers every year you
\LaTeX\ your document:
\begin{verbatim}
\PSNrandseed{\year}
\end{verbatim}

\begin{definition}[\DescribeMacro{\PSNgetrandseed}]
\cs{PSNgetrandseed}\marg{register}
\end{definition}
This stores the current seed in the count register specified by 
\meta{register}.
For example:
\begin{verbatim}
\newcount\myseed
\PSNgetrandseed{\myseed}
\end{verbatim}

\begin{definition}[\DescribeMacro{\PSNrandom}]
\cs{PSNrandom}\marg{register}\marg{n}
\end{definition}
Generates a random integer from 1 to \meta{n} and stores in 
the count register specified by \meta{register}. For example,
the following generates an integer from 1 to 10 and stores it
in the register \cs{myreg}:
\begin{verbatim}
\newcount\myreg
\PSNrandom{\myreg}{10}
\end{verbatim}

\begin{definition}[\DescribeMacro{\random}]
\cs{random}\marg{counter}\marg{min}\marg{max}
\end{definition}
Generates a random integer from \meta{min} to \meta{max} and
stores in the given counter. For example, the following generates
a random number between 3 and 8 (inclusive) and stores it in
the counter \texttt{myrand}.
\begin{verbatim}
\newcounter{myrand}
\random{myrand}{3}{8}
\end{verbatim}

\begin{definition}[\DescribeMacro{\doforrandN}]
\cs{doforrandN}\marg{n}\marg{cmd}\marg{list}\marg{text}
\end{definition}
Randomly selects \meta{n} values from the comma-separated list
given by \meta{list} and iterates through this subset. On
each iteration it sets \meta{cmd} to the current value and
does \meta{text}. For example, the following will load a
randomly selected problem from two of the listed files (where
\texttt{file1.tex}, \texttt{file2.tex} and \texttt{file3.tex}
are files containing at least one problem):
\begin{verbatim}
\doforrandN{2}{\thisfile}{file1,file2,file3}{%
\loadrandomproblems{1}{\thisfile}}
\end{verbatim}

\section{Compatibility With Versions Prior to 3.0}

Version 3.0 of the \sty{probsoln} package completely changed the
structure of the package, but the commands described in this
section have been provided to maintain compatibility with 
earlier versions. The only problems that are likely to occur are
those where commands are contained within groups. This will 
effect any commands that are contained in external files that are
outside of the arguments to \ics{newproblem} and \ics{newproblem*}.
However, since the external files had to be parsed twice in
order to load the problems, this shouldn't be an issue as adding
anything other than problem definitions in those files would
be problematic anyway.

The other likely difference is where the random generator is
used in a group. This includes commands such as 
\ics{selectrandomly}. For example, if your document contained
something like:
\begin{verbatim}
\begin{enumerate}
\selectrandomly{file1}{8}

\item Solve the following:
\begin{enumerate}
\selectrandomly{file2}{4}
\end{enumerate}

\selectrandomly{file3}{2}
\end{enumerate}
\end{verbatim}
Then using versions prior to v3.0 will produce a different
set of random numbers since the second \cs{selectrandomly}
is in a different level of grouping. If you want to ensure
that the document produces exactly the same random set with
the new version as with the old version, you will need to
get and set the random number seed. For example, the above
would need to be modified so that it becomes:
\begin{verbatim}
\begin{enumerate}
\selectrandomly{file1}{8}

\item Solve the following:
\newcount\oldseed
\PSNgetrandseed{\oldseed}
\begin{enumerate}
\selectrandomly{file2}{4}
\end{enumerate}
\PSNrandseed{\oldseed}

\selectrandomly{file3}{2}
\end{enumerate}
\end{verbatim}

\begin{definition}[\DescribeMacro{\selectrandomly}]
\cs{selectrandomly}\marg{filename}\marg{n}
\end{definition}
This is now equivalent to:
\begin{ttfamily}\obeylines
\{\cs{loadrandomproblems}\oarg{filename}\marg{n}\marg{filename}\}\%
\cs{foreachproblem}\oarg{filename}\{\cs{PSNitem}\cs{thisproblem}\cs{endPSNitem}\}
\end{ttfamily}

\begin{definition}[\DescribeMacro{\selectallproblems}]
\cs{selectallproblems}\marg{filename}
\end{definition}
This is now equivalent to:
\begin{ttfamily}\obeylines
\{\cs{loadallproblems}\oarg{filename}\marg{filename}\}\%
\cs{foreachproblem}\oarg{filename}\{\cs{PSNitem}\cs{thisproblem}\cs{endPSNitem}\}
\end{ttfamily}

Note that in both the above cases, a new data set is created
with the same name as the file name.

 \StopEventually{\clearpage\phantomsection\addcontentsline{toc}{section}{Index}\PrintIndex}

\end{document}
